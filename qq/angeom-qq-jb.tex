\documentclass{article}
\usepackage[fontsize=12pt]{fontsize}
\usepackage[utf8]{inputenc}
\usepackage[T2A]{fontenc}
% \usepackage{unicode-math}

\usepackage{array}
\usepackage[a4paper,
left=7mm,
right=5mm,
top=7mm,]{geometry}
\usepackage{amsmath}
\usepackage{amsfonts}
\usepackage{setspace}



\renewcommand{\baselinestretch}{1} 

\everymath{\displaystyle}
\everydisplay{\displaystyle}
% \linespread{1.25}

\DeclareMathOperator{\sign}{sign}


\begin{document}

\pagenumbering{gobble}


\begin{tabular}{m{17cm}}
\textbf{1-variant}
\newline

T1. 
Vektorlardıń skalyar kóbeymesi.
 \\
T2. 
Tegislikte tuwrınıń teńlemeleri.
 \\
A1. 
$M(2;-1)$, $N(-1;4)$ hám $P(-2;2)$ noqatları
úshmúyeshliktiń tárepleriniń ortaları. Tóbeleriniń koordinataların
anıqlań.
 \\
A2. 
$5x-y+3=0$ tuwrısınıń $k$ múyeshlik
koefficientin hám $Oy$ kósherinen kesip alǵan kesindiniń algebralıq
mánisi $b$ nı anıqlań.
 \\
A3. 
Berilgeni: $\overrightarrow{a}| = 10,|\overrightarrow{b}| = 2$ hám
$\left(\overrightarrow{a},\overrightarrow{b} \right) = 12$. Esaplań
$\left| \left\lbrack \overrightarrow{a},\overrightarrow{b} \right\rbrack \right|$.
 \\
B1. 
Tóbeleri $A_1(1; 1), A_2(2; 3)$ hám $A(5;-1)$
noqatlarında jaylasqan úshmúyeshliktiń tuwrımúyeshli ekenligin dálilleń.
 \\
B2. 
Berilgen eki noqattan ótetuǵın tuwrınıń múyeshlik
koefficienti $k$ nı esaplań: $A(-4;3)$, $B(1;8)$.
 \\
B3. 
$\vec{a}$ hám $\vec{b}$ vektorlar óz ara perpendikulyar; $\vec{c}$ vektor olar menen $\pi/3$ ge teń bolǵan múyeshler payda etedi; $|\vec{a}| = 3$, $|\vec{b}| = 5,\ |\vec{c}| = 8$ ekenligi belgili, tómendegilerdi esaplań: 
$\left(3\vec{a} - 2\vec{b},\vec{b} + 3\vec{c} \right)$.
 \\
C1. 
Úshmúyeshliktiń tóbeleri
\(A( - 1; - 1),\ B(3;5),\ C( - 4;1)\) berilgen. $A$ tóbesi sırtqı
múyeshi bssektrisasınıń, $BC$ tárepiniń dawamı menen kesilisiw
noqatın tabıń.
 \\
C2. 
\(\alpha_{1}(\ 5x + 3y - 2) + \beta_{1}(3x - y - 4) = 0\) ,
\(\alpha_{2}(x - y + 1) + \beta_{2}(2x - y - 2) = 0\) eki tuwrılar
dástesi teńlemeleri berilgen. Usı tuwrılar dásteleriniń orayın
anıqlamay, olardıń ekewinede tiyisli bolǵan tuwrınıń teńlemesin dúziń.
 \\
C3. 
\(\lbrack\vec{a},\vec{b}\rbrack^{2} <  {\vec{a}}^{2}{\vec{b}}^{2}\) ekenligin dálilleń; qanday jaǵdayda bul jerde teńlik belgisi boladı?
 \\

\end{tabular}
\vspace{1cm}


\begin{tabular}{m{17cm}}
\textbf{2-variant}
\newline

T1. 
Vektorlardıń vektorlıq kóbeymesi hám aralas kóbeyme.
 \\
T2. 
Tegisliktegi tuwrılardıń ózara jaylasıwı.
 \\
A1. 
$A(2;2)$, $B(-1;6)$, $C(-5;3)$ hám $D(-2;-1)$
noqatları kvadrat tóbeleri ekenligin dálilleń.
 \\
A2. 
$5x+3y+2=0$ tuwrısınıń $k$ múyeshlik
koefficientin hám $Oy$ kósherinen kesip alǵan kesindiniń algebralıq
mánisi $b$ nı anıqlań.
 \\
A3. 
Eger \(a = \{ 3; - 2;1\},\ \ \ \ b = \{ 2;1;2\},\ \ \ \ c = \{ 3; - 1; - 2\}\) bolsa, $\overrightarrow{a}, \overrightarrow{b}, \overrightarrow{c}$ vektorlar komplanar bolıwın tekseriń.
 \\
B1. 
\(P(2;2)\) hám \(Q(1;5)\) noqatları menen teńdey úsh
bólekke bólingen kesindiniń úshları $A$ hám $B$ noqatlarınıń
koordinataların anıqlań.
 \\
B2. 
$ABC$ úshmúyeshliginiń tárepleri: 
\(AB:4x+3y-5=0,\ BC:x-3y+10=0,\ AC:x-2=0
\) teńlemeleri menen berilgen. Tóbeleriniń koordinataların anıqlań.
 \\
B3. 
$\vec{a}$ hám $\vec{b}$ vektorlar $\varphi = 2\pi/3$ múyesh payda etedi. $|\vec{a}| = 3,|\vec{b}| = 4$ ekenligi belgili. Esaplań: 
$(\vec{a} + \vec{b}) ^{2}$.
 \\
C1. 
Úshmúyeshliktiń tóbeleri
\(A(3; - 5),\ B(1; - 3),\ C(2; - 2)\) berilgen. $B$ tóbesi sırtqı
múyeshi bessektrisa uzınlıǵın anıqlań.
 \\
C2. 
\(x - 4y - 5 = 0,\ x - 4y + 3 = 0\) tuwrıları
arasındaǵı kesindi, berilgen \(P(1;1)\) noqatta teń ekige bólinetuǵın
tuwrınıń teńlemesin dúziń.
 \\
C3. 
Birdeylikti dálilleń: \((\lbrack\vec{a},\vec{b}\rbrack,\vec{c} + \lambda\vec{a} + \mu\vec{b}) = (\lbrack\vec{a},\vec{b}\rbrack,\vec{c})\), bunda \(\lambda\) hám \(\mu\)-qálegen sanlar.
 \\

\end{tabular}
\vspace{1cm}


\begin{tabular}{m{17cm}}
\textbf{3-variant}
\newline

T1. 
Vektordıń koordinataları.
 \\
T2. 
Tegisliktiń teńlemeleri. Tegisliklerdiń ózara jaylasıwı.
 \\
A1. 
$ABCD$-parallelogrammınıń úsh tóbesi
$A(2;3)$, $B(4;-1)$ hám $C(0;5)$ berilgen. Tórtinshi $D$
tóbesin tabıń.
 \\
A2. 
$M(4;-5)$ noqatı kvadrattıń bir tóbesi. 
Kvadrattıń bir tárepi $5x-4y+1=0$ tuwrısında jatadı. 
Kvadrattıń maydanın esaplań.
 \\
A3. 
Vektor koordinata kósherleri menen tómendegi múyeshlerdi payda etiwi mumkin be:
$\alpha = 45^{{^\circ}},\ \ \ \ \beta = 135^{{^\circ}},\ \gamma = 60^{{^\circ}}$.
 \\
B1. 
Eki tóbesi \(A(3;1)\) hám \(B(1;-3)\) noqatlarında, hám
awırlıq orayı $Ox$ kósherine tiyisli úshmúyeshliktiń maydanı
\(S=3\) ke teń. Úshinshi $C$ tóbesiniń koordinataların anıqlań. \\
B2. 
\(P(2;3)\) hám \(Q(5;-1)\) noqatları, berilgen eki
tuwrınıń: $12x-y-7=0,\ 13x+4y-5=0$.
kesilisiwinen payda bolǵan birdey múyeshte me, qońsılas múyeshlerde me yáki vertikal
múyeshlerde jatama?.
 \\
B3. 
$\vec{a}$ hám $\vec{b}$ vektorlar $\varphi = 2\pi/3$ múyesh payda etedi. $|\vec{a}| = 3,|\vec{b}| = 4$ ekenligi belgili. Esaplań: 
$\left(\vec{a},\vec{b} \right)$.
 \\
C1. 
Eki tóbesi \(A(2; - 3)\) hám \(B( - 5;1)\) noqatlarında,
úshinshi tóbesi $C$ ordinata kósherine tiyisli úshmúyeshliktiń
medianalarınıń kesilisiw noqatı $M$ abscissa kósherinde jatadı.
$M$ hám $C$ noqatlarınıń koordinataların anıqlań.
 \\
C2. 
\(\alpha (\ 2x - 3y + 20) + \beta(3\ x + 5y - 27) = 0\) tuwrılar
dástesiniń orayı, diagonalı \(x + 7y - 16 = 0\) tuwrısında jatatuǵın
kvadrattıń bir tóbesi. Usı kvadrattıń tárepleriniń hám ekinshi diagonali
teńlemelerin dúziń.
 \\
C3. 
\(\vec{a},\ \vec{b},\ \vec{c}\) vektorlar \(\lbrack\vec{a},\vec{b}\rbrack + \lbrack\vec{b},\vec{c}\rbrack + \lbrack\vec{c},\vec{a}\rbrack = 0\) shártti qanaatlandırıwsh
 \\

\end{tabular}
\vspace{1cm}


\begin{tabular}{m{17cm}}
\textbf{4-variant}
\newline

T1. 
Sızıqlı baylanıslı hám sızıqlı baylanıssız vektorlar.
 \\
T2. 
Tegislik hám tuwrılardıń ózara jaylasıwı.
 \\
A1. 
Birtekli besmúyeshli plastinkanıń tóbeleri berilgen:
$A(2;3), \ B(0;6), \ C(-1;5), \ D(0;1)$ hám $E(1;1)$. Onıń awırlıq
orayı koordinataların anıqlań.
 \\
A2. 
$\alpha(2x+3y-1)+\beta(x-2y-4)=0$ teńlemesi
menen berilgen tuwrılar dástesiniń orayınıń koordinataların anıqlań.
 \\
A3. 
Eger \(a = \{ 2;3; - 1\},\ \ \ \ b = \{ 1; - 1;3\},\ \ \ \ c = \{ 1;9; - 11\}\) bolsa, $\overrightarrow{a}, \overrightarrow{b}, \overrightarrow{c}$ vektorlar komplanar bolıwın tekseriń.
 \\
B1. 
Tórtmúyeshliktiń tóbeleri
\(A(-2;14),\ B(4;-2),\ C(6;-2)\) hám \(D(6;10)\) berilgen. Usı
tórtmúyeshliktiń $AC$ hám $BD$ dioganallarınıń kesilisiw
noqatın tabıń.
 \\
B2. 
Berilgen \(8x-15y-25=0\) tuwrısınan awısıwı -2 ge
teń bolǵan noqatlardıń geometriyalıq ornı teńlemesin dúziń.
 \\
B3. 
$\vec{a} = \{ 2;1; - 1\}$ vektorǵa kollinear bolǵan hám $\left(\vec{x},\vec{a} \right) = 3$ shártti qanaatlandırıwshı $\vec{x}$ vektordı tabıń.
 \\
C1. 
Úshmúyeshliktiń tóbeleri \(M_{1}( - 3;6),\ M_{2}(9; - 10)\) 
hám \(M_{3}( - 5;4)\) berilgen. Usı úshmúyeshlikke sırtlay sızılǵan
sheńber orayı $C$ nı hám radiusı $R$ di anıqlań.
 \\
C2. 
Úshmúyeshliktiń tárepleriniń teńlemeleri berilgen:
\(x - 4y + 11 = 0,\ 5x + 4y - 17 = 0,\ x + 2y - 1 = 0.\) 
Úshmúyeshliktiń tóbeleriniń koordinataların anıqlamay, onıń
biyiklikleriniń teńlemelerin dúziń.
 \\
C3. 
\(\vec{p} = \vec{b} - \frac{\vec{a} (\vec{a},\vec{b}) }{{\vec{a}}^{2}}\) vektor \(\vec{a}\) vektorǵa perpendikulyar ekenligin dálilleń.
 \\

\end{tabular}
\vspace{1cm}


\begin{tabular}{m{17cm}}
\textbf{5-variant}
\newline

T1. 
Vektor túsinigi. Vektorlar ústinde sızıqlı ámeller.
 \\
T2. 
Keńisliktegi tuwrınıń teńlemeleri. Tuwrılardıń ózara jaylasıwı.
 \\
A1. 
Tóbeleri $M(3;-4)$, $N(-2;3)$ hám $P(4;5)$ 
noqatlarında jaylasqan úshmúyeshliklerdiń maydanın esaplań.
 \\
A2. 
$m$ parametriniń qanday mánislerinde 
$mx+(2m+3)y+m+6=0$, $(2m+1)x+(m-1)y+m-2=0$ tuwrıları ordinata
kósherinde jatıwshı noqatta kesilisedi.
 \\
A3. 
Tegislikte eki vektor
$\overrightarrow{p} = \{ 2; - 3\}$, $\overrightarrow{q} = \{ 1;2\}$.
$\overrightarrow{a} = \{9;4\}$ vektordıń
$\overrightarrow{p},\ \overrightarrow{q}$ bazis boyınsha jayılması tabılsın.
 \\
B1. 
Tóbeleri \(M_{1}(1;1), M_{2}(0,2)\) hám
\(M_{3}(2;-1)\) noqatlarında jaylasqan úshmúyeshliktiń ishki
múyeshleri arasında doǵal múyesh bar yáki joqlıǵın anıqlań.
 \\
B2. 
\(4x+3y-1=0\) hám \(3x-2y+5=0\)
tuwrılarınıń kesilisiw noqatınan ótip (bul noqattı anıqlamay), ordinata
kósherinen \(b=4\) kesindi kesip alatuǵın tuwrınıń teńlemesin dúziń.
 \\
B3. 
$\vec{a} = \{ 3; - 1; - 2\}$ hám $\vec{b} = \{ 1;2; - 1\}$ vektorları berilgen. Tómendegi vektor kóbeymelerdiń koordinataların tabıń: 
$\left\lbrack 2\vec{a} + \vec{b},\vec{b} \right\rbrack$.
 \\
C1. 
Eki tóbesi \(A(2;1)\) hám \(B(5; - 3)\) noqatlarında, hám
diagonallarınıń kesilisiw noqatı ordinata kósherine tiyisli
parallelogrammnıń maydanı \(S = 17\) ke teń. Qalǵan eki tóbesiniń
koordinataların anıqlań. \\
C2. 
\(P(4; - 5)\) noqatınan ótip,
\(A(5; - 2)\) hám \(B(3;9)\) noqatlarınan teńdey aralıqta jaylasqan
tuwrınıń teńlemesin dúziń.
 \\
C3. 
Birdeylikti dálilleń: \(\lbrack\vec{a},\vec{b}\rbrack^{2} + (\vec{a},\vec{b}) ^{2} = {\vec{a}}^{2}{\vec{b}}^{2}\).
 \\

\end{tabular}
\vspace{1cm}


\begin{tabular}{m{17cm}}
\textbf{6-variant}
\newline

T1. Analitikalıq geometriya pániniń predmeti hám usılları. 
 \\
T2. Tegislikte hám keńislikte dekart koordinatalar sistemasın almastırıw. 
 \\
A1. 
Birtekli tórtmúyeshli plastinkanıń tóbeleri berilgen:
$A(2;1), \ B(5;3), \ C(-1;7)$ hám $D(-7;5)$. Onıń awırlıq orayı
koordinataların anıqlań.
 \\
A2. 
$\alpha(x+2y-5)+\beta(3x-2y+1)=0$ tuwrılar
dástesi arasınan, tómendegi tuwrılardıń teńlemesin tabıń:
$Ox$ kósherine parallel.
 \\
A3. 
Eger \(a = \{ 2; - 1;2\},\ \ \ \ b = \{ 1;2; - 3\},\ \ \ \ c = \{ 3; - 4;7\}\) bolsa, $\overrightarrow{a}, \overrightarrow{b}, \overrightarrow{c}$ vektorlar komplanar bolıwın tekseriń. \\
B1. 
Úshmúyeshliktiń tóbeleri
\(A(3;-5),\ B(-3;3),\ C(-1;-2)\) berilgen. $A$ tóbesiniń ishki
múyeshi bessektrisanıń uzınlıǵın anıqlań.
 \\
B2. 
Berilgen \(3x-4y-10=0\) tuwrısına parallel hám onnan
$d=3$ qashıqlıqta jatatuǵın tuwrılardıń teńlemesin dúziń.
 \\
B3. 
$\vec{a}$ hám $\vec{b}$ vektorlar $\varphi = 2\pi/3$ múyesh payda etedi. $|\vec{a}| = 1,|\vec{b}| = 2$ ekenligin bilip, tómendegilerdi esaplań: 
$\lbrack 2\overrightarrow{a} + \overrightarrow{b},\overrightarrow{a} + 2\overrightarrow{b}\rbrack^{2}$.
 \\
C1. \(A(4;2)\) noqatı arqalı, eki koordinata kósherlerine
urınıwshı sheńber ótkerildi. Onıń orayı $C$ nı hám radiusı
$R$ di tabıń.
 \\
C2. 
\(P(3;5)\) noqatınan ótip, \(4x + 6y - 7 = 0\) tuwrısı
menen \(45^{0}\) múyesh jasap kesilisetuǵın tuwrı teńlemesin dúziń.
 \\
C3. \(\vec{a} + \vec{b} + \vec{c} = 0\) shártti qanaatlandırıwshı birlik \(\vec{a},\ \vec{b}\) hám \(\vec{c}\) vektorlar berilgen. Esaplań: \(\left(\vec{a},\vec{b} \right) + \left(\vec{b},\vec{c} \right) + \left(\vec{c},\vec{a} \right) \).
 \\

\end{tabular}
\vspace{1cm}


\begin{tabular}{m{17cm}}
\textbf{7-variant}
\newline

T1. 
Koordinataları menen berilgen vektrolardıń skalyar, vektorlıq hám aralas kóbeymeleri. \\
T2. 
Noqattan tegislikke shekem, keńislikte noqattan tuwrıǵa shekemgi hám ayqas tuwrılar arasındaǵı aralıq. \\
A1. 
$A(1;-3)$ hám $B(4;3)$ noqatların tutastırıwshı
kesindi teńdey úsh bólekke bólindi. Bóliwshi noqatlardıń koordinataların
anıqlań.
 \\
A2. 
Ulıwma teńlemesi menen berilgen tuwrılardıń      
óz-ara jaylasıwın anıqlań, eger kesilisetuǵın bolsa kesilisiw noqatın 
tabıń: $6x+10y+9=0, 3x+5y-6=0$.
 \\
A3. 
Vektor koordinata kósherleri menen tómendegi múyeshlerdi payda etiwi
mumkin be: $\alpha = 90^{{^\circ}},\ \beta = 150^{{^\circ}}$,
$\gamma = 60^{{^\circ}}?$ 
 \\
B1. 
Tóbeleri \(M(-1;3),\ N(1,2)\ \)hám \(P(0;4)\)
noqatlarında jaylasqan úshmúyeshliktiń ishki múyeshleri súyir múyesh
ekenligin dálilleń.
 \\
B2. 
$ABCD$ parallelogrammınıń eki qońsı tóbeleri
\(A(3,3),\ B(-1;7)\) hám diagonallarınıń kesilisiw noqatı
\(E(2;-4)\) berilgen. Usı parallelogram tárepleriniń teńlemelerin
dúziń.
 \\
B3. 
$\vec{a}$ hám $\vec{b}$ vektorlar $\varphi = 2\pi/3$ múyesh payda etedi. $|\vec{a}| = 3,|\vec{b}| = 4$ ekenligi belgili. Esaplań: 
$\left(3\vec{a} - 2\vec{b},\vec{a} + 2\vec{b} \right)$.
 \\
C1. 
\(M_{1}(1; - 2)\) noqatı arqalı, padiusı 5 ke teń,
$Ox$ kósherine urınıwshı sheńber ótkerildi. Usı sheńberdiń orayı
$С$ nı anıqlań.
 \\
C2. 
Kvadrattıń eki tárepiniń teńlemeleri berilgen:
\(5x + 12y - 15 = 0,\ 5x + 12y + 25 = 0.\) \(M( - 3;4)\) noqatı
kvadrattıń tárepine tiyisli ekenligin bilgen jaǵdayda, qalǵan
tárepleriniń teńlemelerin dúziń.
 \\
C3. 
Birdeylikti dálilleń: \((\lbrack\vec{a} + \vec{b},\vec{b} + \vec{c}\rbrack,\vec{c} + \vec{a}) = 2 (\lbrack\vec{a},\vec{b}\rbrack,\vec{c}) \).
 \\

\end{tabular}
\vspace{1cm}


\begin{tabular}{m{17cm}}
\textbf{8-variant}
\newline

T1. 
Sızıqlı baylanıslı hám sızıqlı baylanıssız vektorlar.
 \\
T2. 
Noqattan tuwrıǵa shekemgi aralıq. Tuwrılar dástesi.
 \\
A1. $M_1(1; -2)$, $M_2(2; 1)$ noqatları berilgen. 
Tómendegi kesindilerdiń koordinata kósherlerine proekciyaların tabıń: $\overline{M_1M_2}$ \\
 \\
A2. 
$\alpha(x+2y-5)+\beta(3x-2y+1)=0$ tuwrılar
dástesi arasınan, tómendegi tuwrılardıń teńlemesin tabıń:
$M(4;-1)$ noqatınan ótetuǵın.
 \\
A3. 
$\alpha$
nıń qanday mánisinde
$\overrightarrow{a} = \alpha\overrightarrow{i} - 3\overrightarrow{j} + 2\overrightarrow{k}$
hám
$\overrightarrow{b} = \overrightarrow{i} + 2\overrightarrow{j} - \alpha\overrightarrow{k}$
vektorlar óz ara perpendikulyar bolıwın anıqlań. 
 \\
B1. 
Eki qarama-qarsı tóbeleri $P(3; -4)$ hám $Q(l;2)$ noqatlarında jaylasqan rombanıń tárepi uzınlıǵı \(5\sqrt{2}\). Usı romba biyikliginiń uzınlıǵın esaplań.
 \\
B2. 
Eki tuwrı aqrasındaǵı múyeshti tabıń: $2x+y-9=0,\ 3x-y+11=0$.
 \\
B3. 
$\vec{a}$ hám $\vec{b}$ vektorlar óz ara perpendikulyar; $\vec{c}$ vektor olar menen $\pi/3$ ge teń bolǵan múyeshler payda etedi; $|\vec{a}| = 3$, $|\vec{b}| = 5,\ |\vec{c}| = 8$ ekenligi belgili, tómendegilerdi esaplań: 
$(\vec{a} + 2\vec{b} - 3\vec{c}) ^{2}$.
 \\
C1. 
Úshmúyeshliktiń tóbeleri \(M_{1}( - 3;6),\ M_{2}(9; - 10)\) 
hám \(M_{3}( - 5;4)\) berilgen. Usı úshmúyeshlikke sırtlay sızılǵan
sheńber orayı $C$ nı hám radiusı $R$ di anıqlań.
 \\
C2. 
$ABC$ úshmúyeshliginiń bir tóbesin \(C(4;3)\) , hám de
basqa-basqa tóbelerinen júrgizilgen medianasınıń:
\(6x + 10y - 59 = 0\) , hám bissektrisasınıń: \(x - 4y + 10 = 0\) 
teńlemelerin bilgen jaǵdayda, tárepleriniń teńlemelerin dúziń.
 \\
C3. 
\(\vec{a} + \vec{b}\) vektor \(\vec{a} - \vec{b}\) vektorǵa perpendikulyar bolıwı ushın \(\vec{a}\) hám \(\vec{b}\) vektorlar qanday shártlerdi qanaatlandırıwı kerek?
 \\

\end{tabular}
\vspace{1cm}


\begin{tabular}{m{17cm}}
\textbf{9-variant}
\newline

T1. 
Koordinataları menen berilgen vektrolardıń skalyar, vektorlıq hám aralas kóbeymeleri. \\
T2. 
Tegislik hám tuwrılardıń ózara jaylasıwı.
 \\
A1. 
Úshmúyeshliktiń tóbeleri $A(1;4)$, $B(3;-9)$, $C(-5;2)$
berilgen. $B$ tóbesinen júrgizilgen mediana uzınlıǵın anıqlań.
 \\
A2. 
Ulıwma teńlemesi menen berilgen tuwrılardıń
óz-ara jaylasıwın anıqlań, eger kesilisetuǵın bolsa kesilisiw noqatın
tabıń: $3x+y\sqrt{3}=0, x\sqrt{3}+3y-6=0$.
 \\
A3. 
Ushları $A (1;2;1), B (3;-1;7)$ hám $C(7;4;-2)$ bolǵan úshmúyeshliktiń
ishki múyeshlerin esaplap tabıń. Bul úshmúyeshliktiń teń qaptallı ekenligin dálilleń. 
 \\
B1. 
Parallelogrammnıń úsh tóbesi \(A(3;7),\ B(2;-3)\) hám
\(C(-1;4)\) noqatlarında jaylasqan. $B$ tóbesinen $AC$
tárepine túsirilgen biyikliktiń uzınlıǵın esaplań.
 \\
B2. 
\(\alpha(3x+y-1)+\beta(2x-y-9)=0\) tuwrılar dástesi
berilgen. \(x+3y+13=0\) tuwrınıń usı tuwrılar dástesine tiyisli
yamasa tiyisli emesligin anıqlań.
 \\
B3. 
$\vec{a}$ hám $\vec{b}$ vektorlar $\varphi = 2\pi/3$ múyesh payda etedi. $|\vec{a}| = 3,|\vec{b}| = 4$ ekenligi belgili. Esaplań: 
${\vec{a}}^{2}$. 
 \\
C1. \(A(4;2)\) noqatı arqalı, eki koordinata kósherlerine
urınıwshı sheńber ótkerildi. Onıń orayı $C$ nı hám radiusı
$R$ di tabıń.
 \\
C2. 
Berilgen tuwrılardıń: \(3x - y - 10 = 0\) hám
\(2x - 6y - 1 = 0\) kesilisiwinde payda bolǵan doǵal múyesh
bissektrisasınıń teńlemesin dúziń.
 \\
C3. 
\(\vec{a}+\vec{b}\) hám \(\vec{a} - \vec{b}\) vektorlar kollinear bolıwı ushın \(\vec{a},\vec{b}\) vektorlar qanday shártti qanaatlandırıwı kerek?
 \\

\end{tabular}
\vspace{1cm}


\begin{tabular}{m{17cm}}
\textbf{10-variant}
\newline

T1. 
Vektor túsinigi. Vektorlar ústinde sızıqlı ámeller.
 \\
T2. 
Tegisliktegi tuwrılardıń ózara jaylasıwı.
 \\
A1. 
Kvadrattıń eki qarama-qarsı tóbeleri $P(3; 5)$ hám
$Q(1; -3)$ berilgen. Onıń maydanın esaplań.
 \\
A2. 
$a$ hám $b$ parametrleriniń qanday mánislerinde
$ax-2y-1=0$, $6x-4y-b=0$ tuwrıları betlesedi?
 \\
A3. 
$\overrightarrow{a}
= \{ 1; - 1;3\}, \ \ \ \ \overrightarrow{b} = \{ - 2;2;1\}$, $\overrightarrow{c} = \{3; -2;5\}$ vektorları berilgen. Esaplań: 
$(\lbrack\overrightarrow{a},\overrightarrow{b}\rbrack,\overrightarrow{c})$.
 \\
B1. 
Úshmúyeshliktiń tóbeleri \(A(5;0),\ B(0;1)\) hám \(C(3;3)\)
noqatlarında. Onıń ishki múyeshlerin tabıń.
 \\
B2. 
\(N(4;-5)\) noqatınan ótip, $2x+5y-7=0$ 
tuwrılarına parallel tuwrılardıń teńlemesin dúziń. Máseleni múyeshlik
koefficientti esaplamay sheshiń.
 \\
B3. 
$\vec{a}$ hám $\vec{b}$ vektorlar $\varphi = 2\pi/3$ múyesh payda etedi. $|\vec{a}| = 1,|\vec{b}| = 2$ ekenligin bilip, tómendegilerdi esaplań: 
$\lbrack\vec{a},\vec{b}\rbrack^{2}$.
 \\
C1. 
Úshmúyeshliktiń tóbeleri
\(A( - 1; - 1),\ B(3;5),\ C( - 4;1)\) berilgen. $A$ tóbesi sırtqı
múyeshi bssektrisasınıń, $BC$ tárepiniń dawamı menen kesilisiw
noqatın tabıń.
 \\
C2. 
\(\alpha (\ 2x - y - 4) + \beta(\ x - y - 4) = 0\) 
tuwrılar dástesi berilgen. Usı tuwrılar dástesinen, berilgen
\(Q(3; - 1)\) noqatınan aralıǵı \(d = 3\) -ke teń tuwrılar teńlemesin
tabıń.
 \\
C3. 
\(\vec{a},\ \vec{b}\) hám \(\vec{c}\) vektorlar \(\vec{a} + \vec{b} + \vec{c} = 0\) shártti qanaatlandıradı. \(\lbrack\vec{a},\vec{b}\rbrack = \lbrack\vec{b},\vec{c}\rbrack = \lbrack\vec{c},\vec{a}\rbrack\) ekenligin dálilleń.
 \\

\end{tabular}
\vspace{1cm}


\begin{tabular}{m{17cm}}
\textbf{11-variant}
\newline

T1. Analitikalıq geometriya pániniń predmeti hám usılları. 
 \\
T2. 
Keńisliktegi tuwrınıń teńlemeleri. Tuwrılardıń ózara jaylasıwı.
 \\
A1. 
$A(4;2)$, $B(7;-2)$ hám $C(1;6)$ noqatları birtekli
sımnan islengen úshmúyeshlik tóbeleri. Usı úshmúyeshliktiń awırlıq
 \\
A2. 
$\alpha(x+2y-5)+\beta(3x-2y+1)=0$ tuwrılar
dástesi arasınan, tómendegi tuwrılardıń teńlemesin tabıń:
$2x+3y+7=0$ tuwrısına perpendikuliyar.
 \\
A3. 
Berilgeni: $\overrightarrow{a}| = 3,|\overrightarrow{b}| = 26$ hám
$\lbrack\overrightarrow{a},\overrightarrow{b}\rbrack| = 72$. Esaplań
$\left(\overrightarrow{a},\overrightarrow{b} \right) $.
 \\
B1. 
Abcsissa kósherinde sonday $M$ noqatın tabıń,
\(N(2;-3)\) noqatınan qashıqlıǵı 5 ke teń bolatuǵın.
 \\
B2. 
Kvadrattıń eki tárepi
\(5x-12y+65=0,\ 5x-12y-26=0\) tuwrılarında
jatatuǵının bilgen jaǵdayda, maydanın esaplań.
 \\
B3. 
$a$ hám $b$ vektorlar $\varphi = \pi/6$ múyesh payda etedi; $|a| = \sqrt{3},|b| = 1$ ekenligi belgili. $p = a + b$ hám $q = a - b$ vektorlar arasındaǵi $\alpha$ múyeshti esaplań.
 \\
C1. 
Úshmúyeshliktiń tóbeleri
\(A(3; - 5),\ B(1; - 3),\ C(2; - 2)\) berilgen. $B$ tóbesi sırtqı
múyeshi bessektrisa uzınlıǵın anıqlań.
 \\
C2. 
\(\alpha (\ 5x + 2y + 4) + \beta(x + 9y - 25) = 0\) 
tuwrılar dástesi berilgen. Usı tuwrılar dástesinen,
\(12x + 8y - 7 = 0,\ 2x - 3y + 5 = 0\) tuwrıları birge, teń
qaptallı úshmúyeshlikler jasawshı tuwrılar teńlemesin tabıń.
 \\
C3. 
\(\vec{p} = \vec{b} (\vec{a},\vec{c}) - \vec{c}(\vec{a},\vec{b})\) vektor \(\vec{a}\) vektorǵa perpendikulyar ekenligin dálilleń.
 \\

\end{tabular}
\vspace{1cm}


\begin{tabular}{m{17cm}}
\textbf{12-variant}
\newline

T1. 
Vektorlardıń skalyar kóbeymesi.
 \\
T2. 
Tegislikte tuwrınıń teńlemeleri.
 \\
A1. 
Úsh tóbesi $A(-2;3), \ B(4;-5)$ hám
$C(-3;1)$ noqatlarda jaylasqan parallelogrammnıń maydanın anıqlań.
 \\
A2. 
$a$ hám $b$ parametrleriniń qanday mánislerinde
$ax-2y-1=0$, $6x-4y-b=0$ tuwrıları ulıwma noqatqa iye boladı?
 \\
A3. 
$\overrightarrow{a}$ hám $\overrightarrow{b}$ vektorlar
$\varphi = \pi/6$ múyesh payda etedi.
$|\overrightarrow{a}| = 6,|\overrightarrow{b}| = 5$ ekenligin bilip,
$\left| \left\lbrack \overrightarrow{a},\overrightarrow{b} \right\rbrack \right|$ shamasın esaplań. 
 \\
B1. 
Eki noqat berilgen \(M(2;2)\) hám \(N(5;-2)\); abscissa kósherinde sonday $P$ noqatın tabıń, $MPN$ múyeshi tuwrı múyesh bolsın.
 \\
B2. 
Dóńes tórtmúyeshliktiń tóbeleri
\(A(-2;-6),\ B(7;6),\ C(3;9)\) hám \(D(-3;1)\) noqatlarda
jaylasqan. Diagonallarınıń kesilisiw noqatı tabılsın.
 \\
B3. 
$|\vec{a}| = 3,|\vec{b}| = 5$ berilgen. $\alpha$ niń qanday mánisinde $\vec{a} + \alpha\vec{b}$, $\vec{a} - \alpha\vec{b}$ vektorlar óz ara perpendikulyar bolatuǵının anıqlań.
 \\
C1. 
Eki tóbesi \(A(2; - 3)\) hám \(B( - 5;1)\) noqatlarında,
úshinshi tóbesi $C$ ordinata kósherine tiyisli úshmúyeshliktiń
medianalarınıń kesilisiw noqatı $M$ abscissa kósherinde jatadı.
$M$ hám $C$ noqatlarınıń koordinataların anıqlań.
 \\
C2. 
Berilgen tuwrılardıń:
\(3x + 4y - 10 = 0\) hám \(12x - 5y - 13 = 0\) kesilisiwinde payda
bolǵan súyir múyesh bissektrisasınıń teńlemesin dúziń.
 \\
C3. 
\(\vec{a},\ \vec{b},\ \vec{c}\) vektorları ushın \(\alpha\vec{a} + \beta\vec{b} + \gamma\vec{c} = 0\) birdeyligi komplanarlıqtıń zárúr hám jeterli shárti bolıwın dálilleń, bunda \(\alpha,\beta,\gamma\) sanlarınan keminde birewi nolge teń emes. \\

\end{tabular}
\vspace{1cm}


\begin{tabular}{m{17cm}}
\textbf{13-variant}
\newline

T1. 
Vektorlardıń vektorlıq kóbeymesi hám aralas kóbeyme.
 \\
T2. 
Noqattan tegislikke shekem, keńislikte noqattan tuwrıǵa shekemgi hám ayqas tuwrılar arasındaǵı aralıq. \\
A1. 
Tóbeleri $M_1(-3;2)$, $M_2(5;-2)$ hám $M_3(1;3)$ 
noqatlarında jaylasqan úshmúyeshliklerdiń maydanın esaplań.
 \\
A2. 
$\alpha(x+2y-5)+\beta(3x-2y+1)=0$ tuwrılar
dástesi arasınan, tómendegi tuwrılardıń teńlemesin tabıń:
$Oy$ kósherine parallel.
 \\
A3. 
Úshmúyeshliktiń tóbeleri
$A (3;2; - 3) $, $B (5;1; - 1)$ hám $C (1; - 2;1) $. Onıń $A$ tóbesindegi sırtqı múyeshi anıqlansın. 
 \\
B1. 
Úshmúyeshliktiń tóbeleri \(A(2;-5),\ B(1;-2),\ C(4;7)\)
berilgen. $AC$ tárepi menen $B$ tóbesiniń ishki múyeshi
bissektrisasınıń kesilisiw noqatın tabıń.
 \\
B2. 
Ulıwma teńlemesi \(2x-5y+4=0\) bolǵan tuwrı
berilgen. \(M(-3;5)\) noqatınan ótip, berilgen tuwrıǵa: a) parallel;
b) perpendikuliyar bolǵan tuwrılar teńlemesin dúziń.
 \\
B3. 
$\vec{a}$ hám $\vec{b}$ vektorlar $\varphi = 2\pi/3$ múyesh payda etedi. $|\vec{a}| = 3,|\vec{b}| = 4$ ekenligi belgili. Esaplań: 
${\vec{b}}^{2}$.
 \\
C1. 
\(M_{1}(1; - 2)\) noqatı arqalı, padiusı 5 ke teń,
$Ox$ kósherine urınıwshı sheńber ótkerildi. Usı sheńberdiń orayı
$С$ nı anıqlań.
 \\
C2. 
\(\alpha (\ 2x + y + 4) + \beta(\ x - 2y - 3) = 0\) 
tuwrılar dástesi berilgen. Usı tuwrılar dástesinen, berilgen
\(P(2; - 3)\) noqatınan aralıǵı \(d = \sqrt{10}\) -ǵa teń tuwrılar
teńlemesin tabıń.
 \\
C3. 
\(ABC\) úshmúyeshliktiń tárepleri menen sáykes keliwshi \(\vec{AB} = \vec{b}\) hám \(\vec{AC} = \vec{c}\) vektorlar berilgen. Bul úshmúyeshliktiń \(B\) tóbesinen túsirilgen \(BD\) biyikliginiń \(\vec{b},\ \vec{c}\) bazis boyınsha jayılmasın tabıń.
 \\

\end{tabular}
\vspace{1cm}


\begin{tabular}{m{17cm}}
\textbf{14-variant}
\newline

T1. 
Vektordıń koordinataları.
 \\
T2. 
Noqattan tuwrıǵa shekemgi aralıq. Tuwrılar dástesi.
 \\
A1. 
Birtekli elementten islengen saptıń awırlıq orayı
$M(1;4)$ noqatında, bir ushı $P(-2;2)$noqatında jaylasqan. Usı
saptıń ekinshi ushı $Q$-dıń koordinataların anıqlań.
 \\
A2. 
$m$ hám $n$ parametrleriniń qanday mánislerinde
$mx+8y+n=0$, $2x+my-1=0$ tuwrıları parallel boladı?
\textbf{246 (293*)} $m$ parametriniń qanday mánislerinde 
$(m-1)x+my-5=0$, $mx+(2m-1)y+7=0$ tuwrıları abscissa
kósherinde jatıwshı noqatta kesilisedi.
 \\
A3. 
Úshmúyeshliktiń tóbeleri
$A (- 1; - 2;4) $, $B (- 4; - 2;0) $ hám $C (3; - 2;1) $. Onıń $B$ tóbesindegi
ishki múyeshi anıqlań. 
 \\
B1. 
\(M_{1}(1;2)\) noqatına, \(A(1;0)\) hám \(B(-1;-2)\)
noqatlarınan ótetuǵın tuwrıǵa qarata simmetriyalı bolǵan \(M_{2}\) noqatınıń koordinataların tabıń.
 \\
B2. 
\(A(4;-5)\) noqatınan ótip, \(B(-2;3)\) noqatına
shekemgi qashıqlıǵı 12 ge teń bolǵan tuwrılardıń teńlemesin dúziń.
 \\
B3. 
$\vec{a}$ hám $\vec{b}$ vektorlar óz ara perpendikulyar. $|\vec{a}| = 3,|\vec{b}| = 4$ ekenligin belgili, tómendegilerdi esaplań: 
$|\lbrack\vec{a} + \vec{b},\vec{a} - \vec{b}\rbrack|$. 
 \\
C1. 
Eki tóbesi \(A(2;1)\) hám \(B(5; - 3)\) noqatlarında, hám
diagonallarınıń kesilisiw noqatı ordinata kósherine tiyisli
parallelogrammnıń maydanı \(S = 17\) ke teń. Qalǵan eki tóbesiniń
koordinataların anıqlań. \\
C2. 
\(x - 3y - 4 = 0,\ x - 3y + 4 = 0\) tuwrıları
arasındaǵı kesindi, berilgen \(P(6;2)\) noqatta teń ekige bólinetuǵın
tuwrınıń teńlemesin dúziń.
 \\
C3. 
\(\vec{a} + \vec{b}\) vektor \(\vec{a} - \vec{b}\) vektorǵa perpendikulyar bolıwı ushın \(\vec{a}\) hám \(\vec{b}\) vektorlar qanday shártlerdi qanaatlandırıwı kerek?
 \\

\end{tabular}
\vspace{1cm}


\begin{tabular}{m{17cm}}
\textbf{15-variant}
\newline

T1. Analitikalıq geometriya pániniń predmeti hám usılları. 
 \\
T2. Tegislikte hám keńislikte dekart koordinatalar sistemasın almastırıw. 
 \\
A1. 
Parallelogrammnıń úsh tóbesi
$A(3;-5)$, $B(5;-3)$, $C(-1;3)$ berilgen. $B$ tóbesine
qaraqma-qarsı jaylasqan $D$ tóbesin anıqlań.
 \\
A2. 
$5x-3y+15=0$ tuwrısınıń koordinata múyeshinen
kesip alǵan úshmúyeshliktiń maydanın esaplań.
 \\
A3. Vektor koordinata kósherleri menen tómendegi múyeshlerdi payda etiwi mumkin be:
$\alpha = 45^{{^\circ}},\beta = 60^{{^\circ}},\gamma = 120^{{^\circ}}$. 
 \\
B1. 
Tuwrı \(A(5;2)\) hám \(B( -4; -7)\) noqatlarınan ótedi.
Usı tuwrınıń ordinata kósheri menen kesilisiw noqatın tabıń.
 \\
B2. 
\(M(7;-2)\) noqatınan ótip, \(N(4;-6)\) noqatına
shekemgi qashıqlıǵı 5 ke teń bolǵan tuwrılardıń teńlemesin dúziń.
 \\
B3. 
$\vec{a}$ hám $\vec{b}$ vektorlar $\varphi = 2\pi/3$ múyesh payda etedi. $|\vec{a}| = 3,|\vec{b}| = 4$ ekenligi belgili. Esaplań: 
$(\vec{a} - \vec{b}) ^{2};$ 7) $(3\vec{a} + 2\vec{b}) ^{2}$.
 \\
C1. 
Úshmúyeshliktiń tóbeleri \(M_{1}( - 3;6),\ M_{2}(9; - 10)\) 
hám \(M_{3}( - 5;4)\) berilgen. Usı úshmúyeshlikke sırtlay sızılǵan
sheńber orayı $C$ nı hám radiusı $R$ di anıqlań.
 \\
C2. 
\(\alpha (\ 2x + y + 1) + \beta(\ x - 3y - 10) = 0\) 
tuwrılar dástesi berilgen. Usı tuwrılar dástesinen, koordinata
kósherlerinen nolge teń emes, teńdey ólshemdegi (koordinata basınan
baslap) kesindilerdi kesip alıwshı tuwrılar teńlemesin tabıń.
 \\
C3. 
\(\vec{p} = \vec{b} - \frac{\vec{a} (\vec{a},\vec{b}) }{{\vec{a}}^{2}}\) vektor \(\vec{a}\) vektorǵa perpendikulyar ekenligin dálilleń.
 \\

\end{tabular}
\vspace{1cm}


\begin{tabular}{m{17cm}}
\textbf{16-variant}
\newline

T1. 
Koordinataları menen berilgen vektrolardıń skalyar, vektorlıq hám aralas kóbeymeleri. \\
T2. 
Tegisliktiń teńlemeleri. Tegisliklerdiń ózara jaylasıwı.
 \\
A1. 
Eki tóbesi $A(-3; 2)$ hám $B(1; 6)$ noqatlarında
jaylasqan durıs úshmúyeshliktiń maydanın esaplań.
 \\
A2. 
$P(2;2)$ noqatınan ótip, koordinata múyeshinen 
maydanı 1 ge teń úshmúyeshlik kesip alatuǵın tuwrılardıń 
teńlemesin dúziń.
 \\
A3. 
Tórtmúyeshliktiń tóbeleri berilgen:
$A (1; - 2;2) $, $B (1;4;0),C (- 4;1;1) $ hám $D (- 5; -5;3) $. Onıń diagonalları $AC$ hám $BD$ óz ara 
perpendikulyar ekenligin dálilleń.
 \\
B1. 
Tuwrı \(M(2;-3)\) hám \(N(-6;5)\) noqatlarınan ótedi.
Usı tuwrıda ordinatası $-5$ ke teń noqattı tabıń.
 \\
B2. 
\(N(5;8)\) noqatınıń, \(5x-11y-43=0\) tuwrısındaǵı
proekciyasın tabıń.
 \\
B3. Tegislikte úsh vektor $\vec{a} = \{ 3; - 2\}$, $\vec{b} = \{ - 2;1\}$ hám $\vec{c} = \{ 7; - 4\}$ berilgen. Bul úsh vektordıń hár biriniń qalǵan ekewin bazis sıpatında qabıl etip jayılmasın tabıń.
 \\
C1. 
Úshmúyeshliktiń tóbeleri
\(A( - 1; - 1),\ B(3;5),\ C( - 4;1)\) berilgen. $A$ tóbesi sırtqı
múyeshi bssektrisasınıń, $BC$ tárepiniń dawamı menen kesilisiw
noqatın tabıń.
 \\
C2. 
$ABC$ úshmúyeshliginiń bir tóbesi \(B( - 4; - 5)\) ,
hám eki biyikliginiń teńlemeri:
\(3x + 8y + 13 = 0\ ,\ 5x + 3y - 4 = 0\) berilgen. Tárepleriniń
teńlemelerin dúziń.
 \\
C3. 
Birdeylikti dálilleń: \(\lbrack\vec{a},\vec{b}\rbrack^{2} + (\vec{a},\vec{b}) ^{2} = {\vec{a}}^{2}{\vec{b}}^{2}\).
 \\

\end{tabular}
\vspace{1cm}


\begin{tabular}{m{17cm}}
\textbf{17-variant}
\newline

T1. 
Vektorlardıń skalyar kóbeymesi.
 \\
T2. 
Noqattan tuwrıǵa shekemgi aralıq. Tuwrılar dástesi.
 \\
A1. 
Eki tóbesi $A(3;1)$ hám $B(1;-3)$ noqatlarında, al
úshinshi $C$ tóbesi $Oy$ kósherine tiyisli bolǵan úshmúyeshliktiń
maydanı $S=3$ ke teń. $C$ tóbesiniń koordinataların anıqlań.
 \\
A2. 
$B(-1;5)$ noqatınan $5x+12y-26=0$ tuwrısına 
shekemgi awısıwdı hám aralıqtı esaplań.
 \\
A3. 
$\overrightarrow{a} = \{ 2; - 4;4\}$ hám $\overrightarrow{b} = \{ - 3;2;6\}$
vektorlar payda etken múyesh kosinusın esaplań.. 
 \\
B1. 
Tuwrı \(A(7;-3)\) hám \(B(23;-6)\) noqatlarınan ótedi.
Usı tuwrınıń abscissa kósheri menen kesilisiw noqatın tabıń.
 \\
B2. 
Berilgen tuwrılar arasındaǵı múyeshti anıqlań: $3x+2y+4=0,\ 5x-y+1=0$.
 \\
B3. 
$\vec{a} = \{ 6; - 8; - 7,5\}$ vektorǵa kollinear bolǵan $\vec{x}$ vektor $Oz$ kósheri menen súyir múyesh payda etedi. $|\vec{x}| = 50$ ekenligin bilgen halda onıń koordinataların tabıń.
 \\
C1. 
Eki tóbesi \(A(2; - 3)\) hám \(B( - 5;1)\) noqatlarında,
úshinshi tóbesi $C$ ordinata kósherine tiyisli úshmúyeshliktiń
medianalarınıń kesilisiw noqatı $M$ abscissa kósherinde jatadı.
$M$ hám $C$ noqatlarınıń koordinataların anıqlań.
 \\
C2. 
\(P(2;5)\) hám \(Q( - 3;2)\) noqatlardan aralıqlarınıń
ayırması eń úlken bolǵan, ordinata kósherinde jaylasqan noqattı tabıń.
 \\
C3. \(\vec{a} + \vec{b} + \vec{c} = 0\) shártti qanaatlandırıwshı birlik \(\vec{a},\ \vec{b}\) hám \(\vec{c}\) vektorlar berilgen. Esaplań: \(\left(\vec{a},\vec{b} \right) + \left(\vec{b},\vec{c} \right) + \left(\vec{c},\vec{a} \right) \).
 \\

\end{tabular}
\vspace{1cm}


\begin{tabular}{m{17cm}}
\textbf{18-variant}
\newline

T1. 
Sızıqlı baylanıslı hám sızıqlı baylanıssız vektorlar.
 \\
T2. Tegislikte hám keńislikte dekart koordinatalar sistemasın almastırıw. 
 \\
A1. 
$ABCD$ parallelogrammınıń úsh tóbesi $A(3; -7)$, 
$B(5; -7)$, $C(-2; 5)$ berilgen, tórtinshi tóbesi $D$, 
$B$ tóbesine qarama-qarsı. Usı parallelogrammnıń diagonalları
uzınlıqların anıqlań.
 \\
A2. 
Ulıwma teńlemesi menen berilgen tuwrılardıń
óz-ara jaylasıwın anıqlań, eger kesilisetuǵın bolsa kesilisiw noqatın 
tabıń: $12x+15y-39=0, 16x-9y-23=0$.
 \\
A3. Vektor koordinata kósherleri menen tómendegi múyeshlerdi payda etiwi mumkin be:
$\alpha = 45^{{^\circ}},\beta = 60^{{^\circ}},\gamma = 120^{{^\circ}}$. 
 \\
B1. 
Úshmúyeshliktiń tóbeleri \(A(3;6),\ B(-1;3)\) hám
\(C(2:-1)\) noqatlarında jaylasqan. $C$ tóbesinen túsirilgen biyikliktiń uzınlıǵın esaplań.
 \\
B2. 
Tóbeleri \(A(4;-4),\ B(6;-1)\) hám \(C(-1;2)\)
noqatlarında jaylasqan bir tekli plastinkadan jasalǵan úshmúyeshliktiń
awırlıq orayınan ótip, tómende berilgen
\(\alpha(2x+3y-1)+\beta(3x-4y-3)=0\) tuwrılar dástesine
tiyisli tuwrınıń teńlemesin dúziń.
 \\
B3. 
$A (1;2; - 1),B (0;1;5) $, $C (- 1;2;1),D (2;1;3) $ bir tegislikte jatıwın dálilleń.
 \\
C1. 
\(M_{1}(1; - 2)\) noqatı arqalı, padiusı 5 ke teń,
$Ox$ kósherine urınıwshı sheńber ótkerildi. Usı sheńberdiń orayı
$С$ nı anıqlań.
 \\
C2. 
Úshmúyeshliklerdiń tóbeleri
\(A(2; - 2),\ B(3; - 5),\ C(5;7)\) noqatlarında jaylasqan. $C$
tóbesinen ótip, $A$ tóbesinen júrgizilgen bissektrisaǵa
perpendikuliyar tuwrınıń teńlemesin dúziń.
 \\
C3. 
\(\lbrack\vec{a},\vec{b}\rbrack^{2} <  {\vec{a}}^{2}{\vec{b}}^{2}\) ekenligin dálilleń; qanday jaǵdayda bul jerde teńlik belgisi boladı?
 \\

\end{tabular}
\vspace{1cm}


\begin{tabular}{m{17cm}}
\textbf{19-variant}
\newline

T1. 
Vektor túsinigi. Vektorlar ústinde sızıqlı ámeller.
 \\
T2. 
Tegisliktiń teńlemeleri. Tegisliklerdiń ózara jaylasıwı.
 \\
A1. 
Tóbeleri $A(2;-3)$, $B(3;2)$ hám $C(-2;5)$ 
noqatlarında jaylasqan úshmúyeshliklerdiń maydanın esaplań.
 \\
A2. 
$P(12;6)$ noqatınan ótip, koordinata múyeshinen 
maydanı 150 ge teń úshmúyeshlik kesip alatuǵın tuwrılardıń 
teńlemesin dúziń.
 \\
A3. 
$\overrightarrow{a}$ hám $\overrightarrow{b}$ vektorlar
$\varphi = \pi/6$ múyesh payda etedi.
$|\overrightarrow{a}| = 6,|\overrightarrow{b}| = 5$ ekenligin bilip,
$\left| \left\lbrack \overrightarrow{a},\overrightarrow{b} \right\rbrack \right|$ shamasın esaplań. 
 \\
B1. 
Tórtmúyeshliktiń tóbeleri
\(A(-3;12),\ B(3;-4),\ C(5;-4)\) hám \(D(5;8)\) berilgen. Usı
tórtmúyeshliktiń $AC$ diagonalı $BD$ dioganalın qanday
qatnasta bóliwin anıqlań.
 \\
B2. 
\(P(2;7)\) noqatınan ótip, \(Q(1;2)\) noqatına shekemgi
qashıqlıǵı 5 ke teń bolǵan tuwrılardıń teńlemesin dúziń.
 \\
B3. 
$\vec{a}$ hám $\vec{b}$ vektorlar $\varphi = 2\pi/3$ múyesh payda etedi. $|\vec{a}| = 1,|\vec{b}| = 2$ ekenligin bilip, tómendegilerdi esaplań: 
$\lbrack\overrightarrow{a} + 3\overrightarrow{b},3\overrightarrow{a} - \overrightarrow{b}\rbrack^{2}$
 \\
C1. 
Eki tóbesi \(A(2;1)\) hám \(B(5; - 3)\) noqatlarında, hám
diagonallarınıń kesilisiw noqatı ordinata kósherine tiyisli
parallelogrammnıń maydanı \(S = 17\) ke teń. Qalǵan eki tóbesiniń
koordinataların anıqlań. \\
C2. 
Úshmúyeshliktiń \(A( - 3; - 2),\ B(5; - 4),\ C( - 1;3)\) 
tóbelerinen ótip, qarama-qarsı tárepke parallel tuwrılardıń teńlemelerin
dúziń.
 \\
C3. 
\(\vec{a}+\vec{b}\) hám \(\vec{a} - \vec{b}\) vektorlar kollinear bolıwı ushın \(\vec{a},\vec{b}\) vektorlar qanday shártti qanaatlandırıwı kerek?
 \\

\end{tabular}
\vspace{1cm}


\begin{tabular}{m{17cm}}
\textbf{20-variant}
\newline

T1. 
Vektorlardıń vektorlıq kóbeymesi hám aralas kóbeyme.
 \\
T2. 
Tegisliktegi tuwrılardıń ózara jaylasıwı.
 \\
A1. 
Kvadrattıń eki qońsı tóbeleri $A(3; -7)$ hám
$B(-1;4)$ berilgen. Onıń maydanın esaplań.
 \\
A2. 
$2x+3y-6=0$ tuwrısınıń $k$ múyeshlik
koefficientin hám $Oy$ kósherinen kesip alǵan kesindiniń algebralıq
mánisi $b$ nı anıqlań.
 \\
A3. 
Eger \(a = \{ 2; - 1;2\},\ \ \ \ b = \{ 1;2; - 3\},\ \ \ \ c = \{ 3; - 4;7\}\) bolsa, $\overrightarrow{a}, \overrightarrow{b}, \overrightarrow{c}$ vektorlar komplanar bolıwın tekseriń. \\
B1. 
Ordinata kósherinde sonday $M$ noqatın tabıń,
\(N(-8;13)\) noqatınan qashıqlıǵı 17 ge teń bolatuǵın.
 \\
B2. 
Tómendegi hár-bir tuwrılar jubı ushın, olarǵa parallel
bolıp, dál ortasınan ótetuǵın tuwrı teńlemesin dúziń: $3x-2y-3=0$, $3x-2y-17=0$.
 \\
B3. 
Tóbeleri $A (2;-1;1)$, $B (5;5;4)$,$C (3;2;-1)$ hám $D (4;1;3)$ noqatlarda jaylasqan tetraedrdiń kólemi esaplań. \\
C1. \(A(4;2)\) noqatı arqalı, eki koordinata kósherlerine
urınıwshı sheńber ótkerildi. Onıń orayı $C$ nı hám radiusı
$R$ di tabıń.
 \\
C2. Eki tóbesi \(A(1; - 2),\ B(2;3)\) noqatlarda jaylasqan,
maydanı \(S = 8\) ge teń bolǵan úshmúyeshliktiń úshinshi tóbesi
$C$ \(2x + y - 2 = 0\) tuwrısına tiyisli. Usı $C$ tóbesiniń
koordinatasın anıqlań.
 \\
C3. 
\(\vec{a},\ \vec{b},\ \vec{c}\) vektorları ushın \(\alpha\vec{a} + \beta\vec{b} + \gamma\vec{c} = 0\) birdeyligi komplanarlıqtıń zárúr hám jeterli shárti bolıwın dálilleń, bunda \(\alpha,\beta,\gamma\) sanlarınan keminde birewi nolge teń emes. \\

\end{tabular}
\vspace{1cm}


\begin{tabular}{m{17cm}}
\textbf{21-variant}
\newline

T1. 
Vektordıń koordinataları.
 \\
T2. 
Tegislikte tuwrınıń teńlemeleri.
 \\
A1. 
Úshmúyeshliktiń tóbeleriniń koordinataları berilgen
$A(1;-3)$, $B(3;-5)$ hám $C(-5;7)$. Tárepleriniń ortaların
anıqlań.
 \\
A2. 
$P(8;6)$ noqatınan ótip, koordinata múyeshinen 
maydanı 12 ge teń úshmúyeshlik kesip alatuǵın tuwrılardıń teńlemesin 
dúziń.
 \\
A3. 
$\overrightarrow{a}
= \{ 1; - 1;3\}, \ \ \ \ \overrightarrow{b} = \{ - 2;2;1\}$, $\overrightarrow{c} = \{3; -2;5\}$ vektorları berilgen. Esaplań: 
$(\lbrack\overrightarrow{a},\overrightarrow{b}\rbrack,\overrightarrow{c})$.
 \\
B1. 
Úshmúyeshliktiń tóbeleri
\(A\left(-\sqrt{3};1 \right),\ B(0;2)\) hám
\(C\left(-2\sqrt{3};2 \right)\) noqatlarında. Onıń $A$
tóbesindegi sırtqı múyeshin tabıń.
 \\
B2. 
\(\alpha(3x-2y-1)+\beta(4x-5y+8)=0\) tuwrılar
dástesi berilgen. Usı tuwrılar dástesine tiyisli hám \(x+2y+4=0\)
tuwrınıń \(2x+3y+5=0\) hám \(x+7y-1=0\) tuwrıları menen
kesilisiwinde payda bolǵan kesindi ortasınan ótken tuwrınıń teńlemesin
dúziń.
 \\
B3. 
$\vec{a}$ hám $\vec{b}$ vektorlar óz ara perpendikulyar; $\vec{c}$ vektor olar menen $\pi/3$ ge teń bolǵan múyeshler payda etedi; $|\vec{a}| = 3$, $|\vec{b}| = 5,\ |\vec{c}| = 8$ ekenligi belgili, tómendegilerdi esaplań: 
$(\vec{a} + \vec{b} + \vec{c}) ^{2}$.
 \\
C1. 
Úshmúyeshliktiń tóbeleri
\(A(3; - 5),\ B(1; - 3),\ C(2; - 2)\) berilgen. $B$ tóbesi sırtqı
múyeshi bessektrisa uzınlıǵın anıqlań.
 \\
C2. 
Kesiliwshi tuwrılar arasındaǵı múyesh bissektrisalarınıń
teńlemesin dúziń: $x + 4y + 9 = 0$, $4x - y + 10 = 0$.
 \\
C3. 
Birdeylikti dálilleń: \((\lbrack\vec{a},\vec{b}\rbrack,\vec{c} + \lambda\vec{a} + \mu\vec{b}) = (\lbrack\vec{a},\vec{b}\rbrack,\vec{c})\), bunda \(\lambda\) hám \(\mu\)-qálegen sanlar.
 \\

\end{tabular}
\vspace{1cm}


\begin{tabular}{m{17cm}}
\textbf{22-variant}
\newline

T1. 
Vektordıń koordinataları.
 \\
T2. 
Noqattan tegislikke shekem, keńislikte noqattan tuwrıǵa shekemgi hám ayqas tuwrılar arasındaǵı aralıq. \\
A1. 
Parallelogrammnıń eki qońsı tóbeleri $A(-3;5)$, $B(1;7)$
hám dioganallarınıń kesilisiw noqatı $M(1;1)$ berilgen. Qalǵan eki
tóbesin anıqlań.
 \\
A2. 
Ulıwma teńlemesi menen berilgen tuwrılardıń      
óz-ara jaylasıwın anıqlań, eger kesilisetuǵın bolsa kesilisiw noqatın
tabıń: $4x-7=0, 3x+8=0$.
 \\
A3. 
Eger \(a = \{ 3; - 2;1\},\ \ \ \ b = \{ 2;1;2\},\ \ \ \ c = \{ 3; - 1; - 2\}\) bolsa, $\overrightarrow{a}, \overrightarrow{b}, \overrightarrow{c}$ vektorlar komplanar bolıwın tekseriń.
 \\
B1. 
Bir tuwrıǵa tiyisli \(A(1;-1),\ B(3;3)\) hám
\(C(4;5)\) noqatları berilgen. Hár-bir noqattıń, qalǵan eki noqat arqalı anıqlanıwshı kesindini bóliw qatnası $\lambda$ nı anıqlań.
 \\
B2. 
\(\alpha(5x+3y+6)+\beta(3x-4y-37)=0\) tuwrılar
dástesi berilgen. \(7x+2y-15=0\) tuwrınıń usı tuwrılar dástesine
tiyisli yamasa tiyisli emesligin anıqlań.
 \\
B3. 
$\vec{a} + \vec{b} + \vec{c} = 0$ shártti qanaatlandırıwshı $\vec{a},\ \vec{b}$ hám $\vec{c}$ vektorlar berilgen. $|\vec{a}| = 3,\ |\vec{b}| = 1$ hám $|\vec{c}| = 4$ ekenligi belgili, $\left(\vec{a},\vec{b} \right) + \left(\vec{b},\vec{c} \right) + (\vec{c})$ ańlatpasın esaplań.
 \\
C1. 
Úshmúyeshliktiń tóbeleri \(M_{1}( - 3;6),\ M_{2}(9; - 10)\) 
hám \(M_{3}( - 5;4)\) berilgen. Usı úshmúyeshlikke sırtlay sızılǵan
sheńber orayı $C$ nı hám radiusı $R$ di anıqlań.
 \\
C2. 
\(A(0;5)\) hám \(B(5;2)\) noqatlardan aralıqlarınıń
ayırması eń úlken bolǵan, \(3x - y - 2 = 0\) tuwrısında jaylasqan
noqattı tabıń.
 \\
C3. 
Birdeylikti dálilleń: \((\lbrack\vec{a} + \vec{b},\vec{b} + \vec{c}\rbrack,\vec{c} + \vec{a}) = 2 (\lbrack\vec{a},\vec{b}\rbrack,\vec{c}) \).
 \\

\end{tabular}
\vspace{1cm}


\begin{tabular}{m{17cm}}
\textbf{23-variant}
\newline

T1. 
Vektor túsinigi. Vektorlar ústinde sızıqlı ámeller.
 \\
T2. 
Tegislik hám tuwrılardıń ózara jaylasıwı.
 \\
A1. 
Birtekli elementten islengen saptıń ushları
$A(3;-5)$hám $B(-1;1)$ noqatlarında jaylasqan. Onıń awırlıq
orayı koordinatasın anıqlań.
 \\
A2. 
$\alpha(x+2y-5)+\beta(3x-2y+1)=0$ tuwrılar
dástesi arasınan, tómendegi tuwrılardıń teńlemesin tabıń:
koordinata basınan ótetuǵın.
 \\
A3. 
Eger \(a = \{ 2;3; - 1\},\ \ \ \ b = \{ 1; - 1;3\},\ \ \ \ c = \{ 1;9; - 11\}\) bolsa, $\overrightarrow{a}, \overrightarrow{b}, \overrightarrow{c}$ vektorlar komplanar bolıwın tekseriń.
 \\
B1. 
Tuwrı \(M_{1}(-12;-13)\) hám \(M_{2}(-2;-5)\)
noqatlarınan ótedi. Usı tuwrıda abscissası 3 ke teń noqattı tabıń.
 \\
B2. 
\(P(3;8)\) hám \(Q(-1;-6)\) noqatlarınan ótken
tuwrınıń koordinatalıq kósherler menen kesilisiw noqatların tabıń.
 \\
B3. 
$\vec{a} = \{ 3; - 1; - 2\}$ hám $\vec{b} = \{ 1;2; - 1\}$ vektorları berilgen. Tómendegi vektor kóbeymelerdiń koordinataların tabıń: 
$\left\lbrack \vec{a},\vec{b} \right\rbrack$.
 \\
C1. 
\(M_{1}(1; - 2)\) noqatı arqalı, padiusı 5 ke teń,
$Ox$ kósherine urınıwshı sheńber ótkerildi. Usı sheńberdiń orayı
$С$ nı anıqlań.
 \\
C2. 
$ABC$ úshmúyeshliginiń bir tóbesin \(B(2;6)\) , hám
bir tóbesinen júrgizilgen biyikliginiń: \(x - 7y + 15 = 0\) , hám
bissektrisasınıń: \(7x + y + 5 = 0\) teńlemelerin bilgen jaǵdayda,
tárepleriniń teńlemelerin dúziń.
 \\
C3. 
\(ABC\) úshmúyeshliktiń tárepleri menen sáykes keliwshi \(\vec{AB} = \vec{b}\) hám \(\vec{AC} = \vec{c}\) vektorlar berilgen. Bul úshmúyeshliktiń \(B\) tóbesinen túsirilgen \(BD\) biyikliginiń \(\vec{b},\ \vec{c}\) bazis boyınsha jayılmasın tabıń.
 \\

\end{tabular}
\vspace{1cm}


\begin{tabular}{m{17cm}}
\textbf{24-variant}
\newline

T1. 
Koordinataları menen berilgen vektrolardıń skalyar, vektorlıq hám aralas kóbeymeleri. \\
T2. 
Keńisliktegi tuwrınıń teńlemeleri. Tuwrılardıń ózara jaylasıwı.
 \\
A1. 
Berilgen $A(3; -5)$, $B(-2; -7)$ hám
$C(18; 1)$ noqatları bir tuwrıda jatatuǵınlıǵın dálilleń.
 \\
A2. 
$C(0;7)$ noqatınan $2x+3y-13=0$ tuwrısına 
shekemgi awısıwdı hám aralıqtı esaplań.
 \\
A3. 
Berilgeni: $\overrightarrow{a}| = 10,|\overrightarrow{b}| = 2$ hám
$\left(\overrightarrow{a},\overrightarrow{b} \right) = 12$. Esaplań
$\left| \left\lbrack \overrightarrow{a},\overrightarrow{b} \right\rbrack \right|$.
 \\
B1. Eki qarama-qarsı tóbeleri \(P(4;9)\) hám \(Q(-2; 1)\) noqatlarında jaylasqan rombanıń tárepi uzınlıǵı \(5\sqrt{10}\). Usı
romba maydanın esaplań.
 \\
B2. 
Parallellogramnıń eki tárepiniń teńlemeleri
\(8x+3y+1=0,\ 2x+y-1=0\) hám bir diagonalı teńlemesi
\(3x+2y+3=0\) berilgen. Parallellogram tóbeleri koordinataların
anıqlań
 \\
B3. 
$A (2; -1;2),B (1;2; - 1) $ hám $C (3;2;1) $ noqatlar berilgen. Tómendegi vektor kóbeymelerdiń koordinataların tabıń: 
$\lbrack\overline{AB},\overline{BC}\rbrack$.
 \\
C1. 
Eki tóbesi \(A(2;1)\) hám \(B(5; - 3)\) noqatlarında, hám
diagonallarınıń kesilisiw noqatı ordinata kósherine tiyisli
parallelogrammnıń maydanı \(S = 17\) ke teń. Qalǵan eki tóbesiniń
koordinataların anıqlań. \\
C2. 
Úshmúyeshliktiń tóbeleri
\(A(3;2),\ B( - 4;4),\ C( - 2; - 5)\) koordinataları menen berilgen.
Biyiklikleriniń teńlemesin dúziń.
 \\
C3. 
\(\vec{a},\ \vec{b},\ \vec{c}\) vektorlar \(\lbrack\vec{a},\vec{b}\rbrack + \lbrack\vec{b},\vec{c}\rbrack + \lbrack\vec{c},\vec{a}\rbrack = 0\) shártti qanaatlandırıwsh
 \\

\end{tabular}
\vspace{1cm}


\begin{tabular}{m{17cm}}
\textbf{25-variant}
\newline

T1. 
Vektorlardıń skalyar kóbeymesi.
 \\
T2. 
Noqattan tegislikke shekem, keńislikte noqattan tuwrıǵa shekemgi hám ayqas tuwrılar arasındaǵı aralıq. \\
A1. 
Eki tóbesi $A(2;1)$ hám $B(3;-2)$ noqatlarında, al
úshinshi $C$ tóbesi $Ox$ kósherine tiyisli bolǵan úshmúyeshliktiń
maydanı $S=4$ ke teń. $C$ tóbesiniń koordinataların anıqlań. \\
A2. 
$M(4;3)$ noqatınan, koordinata múyeshinen 
maydanı 3 ke teń úshmúyeshlik kesip alatuǵın tuwrı júrgizildi. 
Usı tuwrınıń koordinata kósherleri menen kesilisiw noqatları 
koordinataların anıqlań.
 \\
A3. 
Úshmúyeshliktiń tóbeleri
$A (3;2; - 3) $, $B (5;1; - 1)$ hám $C (1; - 2;1) $. Onıń $A$ tóbesindegi sırtqı múyeshi anıqlansın. 
 \\
B1. 
Bir tuwrıǵa tiyisli \(A(1;-1),\ B(3;3)\) hám
\(C(4;5)\) noqatları berilgen. Hár-bir noqattıń, qalǵan eki noqat arqalı anıqlanıwshı kesindini bóliw qatnası $\lambda$ nı anıqlań.
 \\
B2. 
Berilgen parallel tuwrılardan teńdey aralıqta jatatuǵın
noqatlardıń geometriyalıq ornı teńlemesin dúziń: $2x+y+7=0,\ 2x+y-3=0$.
 \\
B3. 
$\vec{a}$ hám $\vec{b}$ vektorlar óz ara perpendikulyar. $|\vec{a}| = 3,|\vec{b}| = 4$ ekenligin belgili, tómendegilerdi esaplań: 
$|\lbrack 3\vec{a} - \vec{b},\vec{a}-2\vec{b}\rbrack|$.
 \\
C1. 
Úshmúyeshliktiń tóbeleri
\(A(3; - 5),\ B(1; - 3),\ C(2; - 2)\) berilgen. $B$ tóbesi sırtqı
múyeshi bessektrisa uzınlıǵın anıqlań.
 \\
C2. 
Berilgen tuwrılardıń: \(4x + 4y + 1 = 0\) hám
\(2x - 2y - 7 = 0\) kesilisiwinde payda bolǵan, koordinata bası
jatatuǵın múyeshke qońsı múyesh bissektrisasınıń teńlemesin dúziń.
 \\
C3. 
\(\vec{a},\ \vec{b}\) hám \(\vec{c}\) vektorlar \(\vec{a} + \vec{b} + \vec{c} = 0\) shártti qanaatlandıradı. \(\lbrack\vec{a},\vec{b}\rbrack = \lbrack\vec{b},\vec{c}\rbrack = \lbrack\vec{c},\vec{a}\rbrack\) ekenligin dálilleń.
 \\

\end{tabular}
\vspace{1cm}


\begin{tabular}{m{17cm}}
\textbf{26-variant}
\newline

T1. Analitikalıq geometriya pániniń predmeti hám usılları. 
 \\
T2. 
Tegisliktegi tuwrılardıń ózara jaylasıwı.
 \\
A1. 
$A(2;2)$, $B(-1;6)$, $C(-5;3)$ hám $D(-2;-1)$
noqatları kvadrat tóbeleri ekenligin dálilleń.
 \\
A2. 
Ulıwma teńlemesi menen berilgen tuwrılardıń
óz-ara jaylasıwın anıqlań, eger kesilisetuǵın bolsa kesilisiw noqatın
tabıń: $x-5=0, y+12=0$.
 \\
A3. 
Vektor koordinata kósherleri menen tómendegi múyeshlerdi payda etiwi mumkin be:
$\alpha = 45^{{^\circ}},\ \ \ \ \beta = 135^{{^\circ}},\ \gamma = 60^{{^\circ}}$.
 \\
B1. 
Tuwrı \(A(5;2)\) hám \(B( -4; -7)\) noqatlarınan ótedi.
Usı tuwrınıń ordinata kósheri menen kesilisiw noqatın tabıń.
 \\
B2. Berilgen tuwrılardıń kesilisiw noqatın tabıń: 
\(3x-4y-29=0, 2x+5y+19=0\).
 \\
B3. 
$A (2; -1;2),B (1;2; - 1) $ hám $C (3;2;1) $ noqatlar berilgen. Tómendegi vektor kóbeymelerdiń koordinataların tabıń: 
$\lbrack\overline{BC} - 2\overline{CA},\overline{CB}\rbrack$.
 \\
C1. 
Eki tóbesi \(A(2; - 3)\) hám \(B( - 5;1)\) noqatlarında,
úshinshi tóbesi $C$ ordinata kósherine tiyisli úshmúyeshliktiń
medianalarınıń kesilisiw noqatı $M$ abscissa kósherinde jatadı.
$M$ hám $C$ noqatlarınıń koordinataların anıqlań.
 \\
C2. 
Koordinata basınan ótip,
\(2x + y + 9 = 0,\ x - y + 12 = 0\) tuwrıları menen birge, maydanı
1,5 kv.birlikke teń úshmúyeshlik jasaytuǵın tuwrınıń teńlemesin dúziń.
 \\
C3. 
\(\vec{p} = \vec{b} (\vec{a},\vec{c}) - \vec{c}(\vec{a},\vec{b})\) vektor \(\vec{a}\) vektorǵa perpendikulyar ekenligin dálilleń.
 \\

\end{tabular}
\vspace{1cm}


\begin{tabular}{m{17cm}}
\textbf{27-variant}
\newline

T1. 
Vektorlardıń vektorlıq kóbeymesi hám aralas kóbeyme.
 \\
T2. 
Tegislik hám tuwrılardıń ózara jaylasıwı.
 \\
A1. 
Úshmúyeshliktiń tóbeleri $A(1;4)$, $B(3;-9)$, $C(-5;2)$
berilgen. $B$ tóbesinen júrgizilgen mediana uzınlıǵın anıqlań.
 \\
A2. 
$Q_1$, $Q_2$, $Q_3$, $Q_4$, $Q_5$ noqatları 
$x-3y+2=0$ tuwrısına tiyisli hám ordinataları sáykes túrde 
1, 0, 2, -1, 3 ke teń. Olardıń abscissaların tabıń.
 \\
A3. 
Ushları $A (1;2;1), B (3;-1;7)$ hám $C(7;4;-2)$ bolǵan úshmúyeshliktiń
ishki múyeshlerin esaplap tabıń. Bul úshmúyeshliktiń teń qaptallı ekenligin dálilleń. 
 \\
B1. Eki qarama-qarsı tóbeleri \(P(4;9)\) hám \(Q(-2; 1)\) noqatlarında jaylasqan rombanıń tárepi uzınlıǵı \(5\sqrt{10}\). Usı
romba maydanın esaplań.
 \\
B2. 
Tuwrımúyeshliktiń bir tóbesi \(A(2;-3)\), hám eki
tárepiniń teńlemeleri \(2x+3y+9=0,\ 3x-2y-7=0\)
berilgen. Qalǵan eki tárepiniń teńlemelerin dúziń.
 \\
B3. 
$\vec{a} = \{ 3; - 1; - 2\}$ hám $\vec{b} = \{ 1;2; - 1\}$ vektorları berilgen. Tómendegi vektor kóbeymelerdiń koordinataların tabıń: 
$\left\lbrack 2\vec{a} - \vec{b},2\vec{a} + \vec{b} \right\rbrack$.
 \\
C1. \(A(4;2)\) noqatı arqalı, eki koordinata kósherlerine
urınıwshı sheńber ótkerildi. Onıń orayı $C$ nı hám radiusı
$R$ di tabıń.
 \\
C2. 
Berilgen tuwrılardıń:
\(3x + y + 10 = 0\) hám \(2x - 6y - 5 = 0\) kesilisiwinde payda
bolǵan, koordinata bası jatatuǵın múyesh bissektrisasınıń teńlemesin
dúziń.
 \\
C3. 
\(\vec{p} = \vec{b} - \frac{\vec{a} (\vec{a},\vec{b}) }{{\vec{a}}^{2}}\) vektor \(\vec{a}\) vektorǵa perpendikulyar ekenligin dálilleń.
 \\

\end{tabular}
\vspace{1cm}


\begin{tabular}{m{17cm}}
\textbf{28-variant}
\newline

T1. 
Sızıqlı baylanıslı hám sızıqlı baylanıssız vektorlar.
 \\
T2. 
Noqattan tuwrıǵa shekemgi aralıq. Tuwrılar dástesi.
 \\
A1. 
$ABCD$-parallelogrammınıń úsh tóbesi
$A(2;3)$, $B(4;-1)$ hám $C(0;5)$ berilgen. Tórtinshi $D$
tóbesin tabıń.
 \\
A2. 
$M(-3;8)$ noqatınan ótip, koordinata kósherlerinen
teńdey kesindilerdi kesip alatuǵın tuwrılardıń teńlemesin dúziń.
 \\
A3. 
Úshmúyeshliktiń tóbeleri
$A (- 1; - 2;4) $, $B (- 4; - 2;0) $ hám $C (3; - 2;1) $. Onıń $B$ tóbesindegi
ishki múyeshi anıqlań. 
 \\
B1. 
Eki tóbesi \(A(3;1)\) hám \(B(1;-3)\) noqatlarında, hám
awırlıq orayı $Ox$ kósherine tiyisli úshmúyeshliktiń maydanı
\(S=3\) ke teń. Úshinshi $C$ tóbesiniń koordinataların anıqlań. \\
B2. 
\(\alpha(5x+3y-7)+\beta(3x+10y+4)=0\) tuwrılar
dástesi berilgen. $a$ nıń qanday mánisinde \(ax+5y+9=0\)
tuwrı usı tuwrılar dástesine tiyisli bolmaydı. \\
B3. 
$\vec{a} = \{ 3; - 1; - 2\}$ hám $\vec{b} = \{ 1;2; - 1\}$ vektorları berilgen. Tómendegi vektor kóbeymelerdiń koordinataların tabıń: 
$\left\lbrack 2\vec{a} - \vec{b},2\vec{a} + \vec{b} \right\rbrack$.
 \\
C1. 
Úshmúyeshliktiń tóbeleri
\(A( - 1; - 1),\ B(3;5),\ C( - 4;1)\) berilgen. $A$ tóbesi sırtqı
múyeshi bssektrisasınıń, $BC$ tárepiniń dawamı menen kesilisiw
noqatın tabıń.
 \\
C2. 
\(Q(5; - 6)\) noqatınıń, \(A(3;8)\) hám \(B(7;5)\) 
noqatlardan ótken tuwrıdaǵı proekciyasın tabıń.
 \\
C3. 
Birdeylikti dálilleń: \((\lbrack\vec{a},\vec{b}\rbrack,\vec{c} + \lambda\vec{a} + \mu\vec{b}) = (\lbrack\vec{a},\vec{b}\rbrack,\vec{c})\), bunda \(\lambda\) hám \(\mu\)-qálegen sanlar.
 \\

\end{tabular}
\vspace{1cm}


\begin{tabular}{m{17cm}}
\textbf{29-variant}
\newline

T1. 
Koordinataları menen berilgen vektrolardıń skalyar, vektorlıq hám aralas kóbeymeleri. \\
T2. 
Tegislikte tuwrınıń teńlemeleri.
 \\
A1. 
Tóbeleri $A(2;-3)$, $B(3;2)$ hám $C(-2;5)$ 
noqatlarında jaylasqan úshmúyeshliklerdiń maydanın esaplań.
 \\
A2. 
$2x-y+2=0$, $4x-2y+4=0$, $6x-3y+6=0$ 
tuwrıları bir noqatta kesilise me?
 \\
A3. 
Berilgeni: $\overrightarrow{a}| = 3,|\overrightarrow{b}| = 26$ hám
$\lbrack\overrightarrow{a},\overrightarrow{b}\rbrack| = 72$. Esaplań
$\left(\overrightarrow{a},\overrightarrow{b} \right) $.
 \\
B1. 
Úshmúyeshliktiń tóbeleri \(A(3;6),\ B(-1;3)\) hám
\(C(2:-1)\) noqatlarında jaylasqan. $C$ tóbesinen túsirilgen biyikliktiń uzınlıǵın esaplań.
 \\
B2. 
Koordinata bası, berilgen tuwrılardıń:
\(3x+y-4=0\) hám \(3x-2y+6=0\) kesilisiwinde payda
bolǵan súyir yamasa doǵal múyeshke tiyisli bolıwın anıqlań.
 \\
B3. 
$\vec{a}$ hám $\vec{b}$ vektorlar $\varphi = 2\pi/3$ múyesh payda etedi. $|\vec{a}| = 3,|\vec{b}| = 4$ ekenligi belgili. Esaplań: 
$\left(\vec{a},\vec{b} \right)$.
 \\
C1. 
\(M_{1}(1; - 2)\) noqatı arqalı, padiusı 5 ke teń,
$Ox$ kósherine urınıwshı sheńber ótkerildi. Usı sheńberdiń orayı
$С$ nı anıqlań.
 \\
C2. 
$M$ noqatınıń \(12x - 5y + 49 = 0\) hám
\(\ 3x + 4y - 20 = 0\) tuwrılarınan awısıwları sáykes $-4$ hám
$-7$. $M$ noqatınıń koordinataların tabıń.
 \\
C3. 
Birdeylikti dálilleń: \((\lbrack\vec{a} + \vec{b},\vec{b} + \vec{c}\rbrack,\vec{c} + \vec{a}) = 2 (\lbrack\vec{a},\vec{b}\rbrack,\vec{c}) \).
 \\

\end{tabular}
\vspace{1cm}


\begin{tabular}{m{17cm}}
\textbf{30-variant}
\newline

T1. Analitikalıq geometriya pániniń predmeti hám usılları. 
 \\
T2. 
Keńisliktegi tuwrınıń teńlemeleri. Tuwrılardıń ózara jaylasıwı.
 \\
A1. 
Tóbeleri $M_1(-3;2)$, $M_2(5;-2)$ hám $M_3(1;3)$ 
noqatlarında jaylasqan úshmúyeshliklerdiń maydanın esaplań.
 \\
A2. 
$P_1$, $P_2$, $P_3$, $P_4$, $P_5$  noqatları 
$3x-2y-6=0$ tuwrısına tiyisli hám abscissaları sáykes túrde 
4, 0, 2, -2, -6 ǵa teń. Olardıń ordinataların tabıń.
 \\
A3. 
Tórtmúyeshliktiń tóbeleri berilgen:
$A (1; - 2;2) $, $B (1;4;0),C (- 4;1;1) $ hám $D (- 5; -5;3) $. Onıń diagonalları $AC$ hám $BD$ óz ara 
perpendikulyar ekenligin dálilleń.
 \\
B1. 
Úshmúyeshliktiń tóbeleri
\(A\left(-\sqrt{3};1 \right),\ B(0;2)\) hám
\(C\left(-2\sqrt{3};2 \right)\) noqatlarında. Onıń $A$
tóbesindegi sırtqı múyeshin tabıń.
 \\
B2. 
Tuwrımúyeshliktiń eki tárepi
\(5x+2y-7=0,\ 5x+2y-36=0\) hám diagonalı
\(3x+7y-10=0\) teńlemeleri menen berilgen. Qalǵan eki tárepi
teńlemelerin dúziń.
 \\
B3. 
$\vec{a}$ hám $\vec{b}$ vektorlar $\varphi = 2\pi/3$ múyesh payda etedi. $|\vec{a}| = 1,|\vec{b}| = 2$ ekenligin bilip, tómendegilerdi esaplań: 
$\lbrack\overrightarrow{a} + 3\overrightarrow{b},3\overrightarrow{a} - \overrightarrow{b}\rbrack^{2}$
 \\
C1. 
Úshmúyeshliktiń tóbeleri \(M_{1}( - 3;6),\ M_{2}(9; - 10)\) 
hám \(M_{3}( - 5;4)\) berilgen. Usı úshmúyeshlikke sırtlay sızılǵan
sheńber orayı $C$ nı hám radiusı $R$ di anıqlań.
 \\
C2. 
Berilgen tuwrılardıń:
\(2x + 3y - 8 = 0\) hám \(3x - 2y - 5 = 0\) kesilisiwinde payda
bolǵan, \(M(2; - 3)\) noqatı tiyisli múyeshke qońsı múyesh
bissektrisasınıń teńlemesin dúziń.
 \\
C3. \(\vec{a} + \vec{b} + \vec{c} = 0\) shártti qanaatlandırıwshı birlik \(\vec{a},\ \vec{b}\) hám \(\vec{c}\) vektorlar berilgen. Esaplań: \(\left(\vec{a},\vec{b} \right) + \left(\vec{b},\vec{c} \right) + \left(\vec{c},\vec{a} \right) \).
 \\

\end{tabular}
\vspace{1cm}


\begin{tabular}{m{17cm}}
\textbf{31-variant}
\newline

T1. 
Vektor túsinigi. Vektorlar ústinde sızıqlı ámeller.
 \\
T2. 
Tegisliktiń teńlemeleri. Tegisliklerdiń ózara jaylasıwı.
 \\
A1. 
Birtekli tórtmúyeshli plastinkanıń tóbeleri berilgen:
$A(2;1), \ B(5;3), \ C(-1;7)$ hám $D(-7;5)$. Onıń awırlıq orayı
koordinataların anıqlań.
 \\
A2. 
$B(-5;5)$ noqatınan ótip, koordinata múyeshinen
maydanı 50 ge teń úshmúyeshlik kesip alatuǵın tuwrılardıń teńlemesin
dúziń.
 \\
A3. 
Tegislikte eki vektor
$\overrightarrow{p} = \{ 2; - 3\}$, $\overrightarrow{q} = \{ 1;2\}$.
$\overrightarrow{a} = \{9;4\}$ vektordıń
$\overrightarrow{p},\ \overrightarrow{q}$ bazis boyınsha jayılması tabılsın.
 \\
B1. 
Tórtmúyeshliktiń tóbeleri
\(A(-2;14),\ B(4;-2),\ C(6;-2)\) hám \(D(6;10)\) berilgen. Usı
tórtmúyeshliktiń $AC$ hám $BD$ dioganallarınıń kesilisiw
noqatın tabıń.
 \\
B2. 
Parallel tuwrılar arasındaǵı aralıqtı esaplań: $5x-12y+13=0,\ 5x-12y-26=0$.
 \\
B3. 
$\vec{a}$ hám $\vec{b}$ vektorlar $\varphi = 2\pi/3$ múyesh payda etedi. $|\vec{a}| = 3,|\vec{b}| = 4$ ekenligi belgili. Esaplań: 
${\vec{b}}^{2}$.
 \\
C1. 
Eki tóbesi \(A(2;1)\) hám \(B(5; - 3)\) noqatlarında, hám
diagonallarınıń kesilisiw noqatı ordinata kósherine tiyisli
parallelogrammnıń maydanı \(S = 17\) ke teń. Qalǵan eki tóbesiniń
koordinataların anıqlań. \\
C2. 
Úshmúyeshliktiń eki tóbesi \(A(6;4),\ B( - 10;2)\) , hám
biyiklikleriniń kesilisiw noqatı \(N(5;2)\) berilgen. Úshinshi $C$
tóbesiniń koordinataların tabıń.
 \\
C3. 
\(\lbrack\vec{a},\vec{b}\rbrack^{2} <  {\vec{a}}^{2}{\vec{b}}^{2}\) ekenligin dálilleń; qanday jaǵdayda bul jerde teńlik belgisi boladı?
 \\

\end{tabular}
\vspace{1cm}


\begin{tabular}{m{17cm}}
\textbf{32-variant}
\newline

T1. 
Vektordıń koordinataları.
 \\
T2. Tegislikte hám keńislikte dekart koordinatalar sistemasın almastırıw. 
 \\
A1. 
Kvadrattıń eki qońsı tóbeleri $A(3; -7)$ hám
$B(-1;4)$ berilgen. Onıń maydanın esaplań.
 \\
A2. 
$y-3=0$ tuwrısınıń $k$ múyeshlik
koefficientin hám $Oy$ kósherinen kesip alǵan kesindiniń algebralıq
mánisi $b$ nı anıqlań.
 \\
A3. 
$\alpha$
nıń qanday mánisinde
$\overrightarrow{a} = \alpha\overrightarrow{i} - 3\overrightarrow{j} + 2\overrightarrow{k}$
hám
$\overrightarrow{b} = \overrightarrow{i} + 2\overrightarrow{j} - \alpha\overrightarrow{k}$
vektorlar óz ara perpendikulyar bolıwın anıqlań. 
 \\
B1. 
Abcsissa kósherinde sonday $M$ noqatın tabıń,
\(N(2;-3)\) noqatınan qashıqlıǵı 5 ke teń bolatuǵın.
 \\
B2. 
\(P(-3;2)\) noqatı, tárepleriniń teńlemeleri
\(x+y-4=0,\ 3x-7y+8=0,\ 4x-y-31=0\) menen
berilgen úshmúyeshliktiń sırtanda yamasa ishinde jatatuǵınlıǵın anıqlań.
 \\
B3. 
$A (2; -1;2),B (1;2; - 1) $ hám $C (3;2;1) $ noqatlar berilgen. Tómendegi vektor kóbeymelerdiń koordinataların tabıń: 
$\lbrack\overline{BC} - 2\overline{CA},\overline{CB}\rbrack$.
 \\
C1. \(A(4;2)\) noqatı arqalı, eki koordinata kósherlerine
urınıwshı sheńber ótkerildi. Onıń orayı $C$ nı hám radiusı
$R$ di tabıń.
 \\
C2. 
$ABC$ úshmúyeshliginiń bir tóbesin \(A(2; - 1)\) , hám
de basqa-basqa tóbelerinen júrgizilgen biyikliginiń:
\(3x - 4y + 27 = 0\) , hám bissektrisasınıń: \(x + 2y - 5 = 0\) 
teńlemelerin bilgen jaǵdayda, tárepleriniń teńlemelerin dúziń.
 \\
C3. 
\(\vec{a}+\vec{b}\) hám \(\vec{a} - \vec{b}\) vektorlar kollinear bolıwı ushın \(\vec{a},\vec{b}\) vektorlar qanday shártti qanaatlandırıwı kerek?
 \\

\end{tabular}
\vspace{1cm}


\begin{tabular}{m{17cm}}
\textbf{33-variant}
\newline

T1. 
Sızıqlı baylanıslı hám sızıqlı baylanıssız vektorlar.
 \\
T2. 
Tegisliktiń teńlemeleri. Tegisliklerdiń ózara jaylasıwı.
 \\
A1. 
Úsh tóbesi $A(-2;3), \ B(4;-5)$ hám
$C(-3;1)$ noqatlarda jaylasqan parallelogrammnıń maydanın anıqlań.
 \\
A2. 
$3x+2y=0$ tuwrısınıń $k$ múyeshlik
koefficientin hám $Oy$ kósherinen kesip alǵan kesindiniń algebralıq
mánisi $b$ nı anıqlań.
 \\
A3. 
Vektor koordinata kósherleri menen tómendegi múyeshlerdi payda etiwi
mumkin be: $\alpha = 90^{{^\circ}},\ \beta = 150^{{^\circ}}$,
$\gamma = 60^{{^\circ}}?$ 
 \\
B1. 
Tuwrı \(A(7;-3)\) hám \(B(23;-6)\) noqatlarınan ótedi.
Usı tuwrınıń abscissa kósheri menen kesilisiw noqatın tabıń.
 \\
B2. 
Úshmúyeshliktiń tárepleri \(x+5y-7=0\),
\(3x-2y-4=0\), \(7x+y+19=0\) tuwrılarında jatadı. Onıń
maydanın esaplań.
 \\
B3. 
$\vec{a}$ hám $\vec{b}$ vektorlar $\varphi = 2\pi/3$ múyesh payda etedi. $|\vec{a}| = 3,|\vec{b}| = 4$ ekenligi belgili. Esaplań: 
$(\vec{a} - \vec{b}) ^{2};$ 7) $(3\vec{a} + 2\vec{b}) ^{2}$.
 \\
C1. 
Úshmúyeshliktiń tóbeleri
\(A( - 1; - 1),\ B(3;5),\ C( - 4;1)\) berilgen. $A$ tóbesi sırtqı
múyeshi bssektrisasınıń, $BC$ tárepiniń dawamı menen kesilisiw
noqatın tabıń.
 \\
C2. 
\(N(2; - 5)\) noqatınıń \(9x - 7y + 30 = 0\) tuwrısına
qarata simmetriyalı noqatın tabıń.
 \\
C3. 
\(\vec{a},\ \vec{b},\ \vec{c}\) vektorlar \(\lbrack\vec{a},\vec{b}\rbrack + \lbrack\vec{b},\vec{c}\rbrack + \lbrack\vec{c},\vec{a}\rbrack = 0\) shártti qanaatlandırıwsh
 \\

\end{tabular}
\vspace{1cm}


\begin{tabular}{m{17cm}}
\textbf{34-variant}
\newline

T1. 
Vektorlardıń vektorlıq kóbeymesi hám aralas kóbeyme.
 \\
T2. 
Noqattan tegislikke shekem, keńislikte noqattan tuwrıǵa shekemgi hám ayqas tuwrılar arasındaǵı aralıq. \\
A1. 
Birtekli besmúyeshli plastinkanıń tóbeleri berilgen:
$A(2;3), \ B(0;6), \ C(-1;5), \ D(0;1)$ hám $E(1;1)$. Onıń awırlıq
orayı koordinataların anıqlań.
 \\
A2. Berilgen $M_1 (3; 1) $, $M_2 (2; 3) $, $M_3 (6; 3) $,
$M_4 (-3;-3) $. $M_5 (3;-1) $, $M_6 (-2; 1) $ noqatlardıń qaysıları
$2x-3y-3 = 0$ tuwrısına tiyisli hám qaysıları tiyisli
emes.
 \\
A3. 
$\overrightarrow{a} = \{ 2; - 4;4\}$ hám $\overrightarrow{b} = \{ - 3;2;6\}$
vektorlar payda etken múyesh kosinusın esaplań.. 
 \\
B1. 
Úshmúyeshliktiń tóbeleri \(A(2;-5),\ B(1;-2),\ C(4;7)\)
berilgen. $AC$ tárepi menen $B$ tóbesiniń ishki múyeshi
bissektrisasınıń kesilisiw noqatın tabıń.
 \\
B2. 
Koordinata bası, tárepleriniń teńlemeleri
\(8x+3y+31=0,\ x+8y-19=0,\ 7x-5y-11=0\) menen
berilgen úshmúyeshliktiń sırtında yamasa ishinde jatatuǵınlıǵın anıqlań.
 \\
B3. Tegislikte úsh vektor $\vec{a} = \{ 3; - 2\}$, $\vec{b} = \{ - 2;1\}$ hám $\vec{c} = \{ 7; - 4\}$ berilgen. Bul úsh vektordıń hár biriniń qalǵan ekewin bazis sıpatında qabıl etip jayılmasın tabıń.
 \\
C1. 
Úshmúyeshliktiń tóbeleri
\(A(3; - 5),\ B(1; - 3),\ C(2; - 2)\) berilgen. $B$ tóbesi sırtqı
múyeshi bessektrisa uzınlıǵın anıqlań.
 \\
C2. 
\(\alpha (\ 2x + 5y + 4) + \beta(3\ x - 2y + 25) = 0\) tuwrılar
dástesi berilgen. Usı tuwrılar dástesinen, koordinata kósherlerinen
nolge teń emes, teńdey ólshemdegi (koordinata basınan baslap)
kesindilerdi kesip alıwshı tuwrı teńlemesin tabıń.
 \\
C3. 
Birdeylikti dálilleń: \(\lbrack\vec{a},\vec{b}\rbrack^{2} + (\vec{a},\vec{b}) ^{2} = {\vec{a}}^{2}{\vec{b}}^{2}\).
 \\

\end{tabular}
\vspace{1cm}


\begin{tabular}{m{17cm}}
\textbf{35-variant}
\newline

T1. 
Vektorlardıń skalyar kóbeymesi.
 \\
T2. 
Tegisliktegi tuwrılardıń ózara jaylasıwı.
 \\
A1. 
Úshmúyeshliktiń tóbeleriniń koordinataları berilgen
$A(1;-3)$, $B(3;-5)$ hám $C(-5;7)$. Tárepleriniń ortaların
anıqlań.
 \\
A2. 
Ulıwma teńlemesi menen berilgen tuwrılardıń      
óz-ara jaylasıwın anıqlań, eger kesilisetuǵın bolsa kesilisiw noqatın 
tabıń: $12x+59y-19=0, 8x+33y-19=0$.
 \\
A3. 
Berilgeni: $\overrightarrow{a}| = 3,|\overrightarrow{b}| = 26$ hám
$\lbrack\overrightarrow{a},\overrightarrow{b}\rbrack| = 72$. Esaplań
$\left(\overrightarrow{a},\overrightarrow{b} \right) $.
 \\
B1. 
Tóbeleri \(M_{1}(1;1), M_{2}(0,2)\) hám
\(M_{3}(2;-1)\) noqatlarında jaylasqan úshmúyeshliktiń ishki
múyeshleri arasında doǵal múyesh bar yáki joqlıǵın anıqlań.
 \\
B2. 
\(\alpha(3x+2y-9)+\beta(2x+5y+5)=0\)
tuwrılar dástesi berilgen. $K$ nıń qanday mánisinde
\(4x-3y+K=0\) tuwrı usı tuwrılar dástesine tiyisli boladı.
 \\
B3. 
$A (2; -1;2),B (1;2; - 1) $ hám $C (3;2;1) $ noqatlar berilgen. Tómendegi vektor kóbeymelerdiń koordinataların tabıń: 
$\lbrack\overline{AB},\overline{BC}\rbrack$.
 \\
C1. 
Eki tóbesi \(A(2; - 3)\) hám \(B( - 5;1)\) noqatlarında,
úshinshi tóbesi $C$ ordinata kósherine tiyisli úshmúyeshliktiń
medianalarınıń kesilisiw noqatı $M$ abscissa kósherinde jatadı.
$M$ hám $C$ noqatlarınıń koordinataların anıqlań.
 \\
C2. 
\(A(4;5)\) noqatı, diagonalı \(7x - y - 8 = 0\) teńlemesi
menen berilgen kvadrattıń bir tóbesi. Usı kvadrattıń tárepleriniń hám
ekinshi diagonalınıń teńlemesin dúziń.
 \\
C3. 
\(\vec{a},\ \vec{b},\ \vec{c}\) vektorları ushın \(\alpha\vec{a} + \beta\vec{b} + \gamma\vec{c} = 0\) birdeyligi komplanarlıqtıń zárúr hám jeterli shárti bolıwın dálilleń, bunda \(\alpha,\beta,\gamma\) sanlarınan keminde birewi nolge teń emes. \\

\end{tabular}
\vspace{1cm}


\begin{tabular}{m{17cm}}
\textbf{36-variant}
\newline

T1. 
Vektordıń koordinataları.
 \\
T2. 
Noqattan tuwrıǵa shekemgi aralıq. Tuwrılar dástesi.
 \\
A1. 
Eki tóbesi $A(2;1)$ hám $B(3;-2)$ noqatlarında, al
úshinshi $C$ tóbesi $Ox$ kósherine tiyisli bolǵan úshmúyeshliktiń
maydanı $S=4$ ke teń. $C$ tóbesiniń koordinataların anıqlań. \\
A2. 
$5x+3y-7=0$, $x-2y-4=0$, $3x-y+3=0$ 
tuwrıları bir noqatta kesilise me?
 \\
A3. 
Tórtmúyeshliktiń tóbeleri berilgen:
$A (1; - 2;2) $, $B (1;4;0),C (- 4;1;1) $ hám $D (- 5; -5;3) $. Onıń diagonalları $AC$ hám $BD$ óz ara 
perpendikulyar ekenligin dálilleń.
 \\
B1. 
Úshmúyeshliktiń tóbeleri
\(A(3;-5),\ B(-3;3),\ C(-1;-2)\) berilgen. $A$ tóbesiniń ishki
múyeshi bessektrisanıń uzınlıǵın anıqlań.
 \\
B2. 
\(2x+y-2=0\) hám \(x-5y-3=0\)
tuwrılarınıń kesilisiw noqatınan ótip (bul noqattı anıqlamay), ushları
\(A(-1;-4)\) hám \(B(5;-6)\) noqatlarinda jaylasqan kesindiniń
dál ortasınan ótiwshi tuwrınıń teńlemesin dúziń.
 \\
B3. 
$\vec{a} = \{ 3; - 1; - 2\}$ hám $\vec{b} = \{ 1;2; - 1\}$ vektorları berilgen. Tómendegi vektor kóbeymelerdiń koordinataların tabıń: 
$\left\lbrack \vec{a},\vec{b} \right\rbrack$.
 \\
C1. 
Eki tóbesi \(A(2; - 3)\) hám \(B( - 5;1)\) noqatlarında,
úshinshi tóbesi $C$ ordinata kósherine tiyisli úshmúyeshliktiń
medianalarınıń kesilisiw noqatı $M$ abscissa kósherinde jatadı.
$M$ hám $C$ noqatlarınıń koordinataların anıqlań.
 \\
C2. 
\(A(3;7)\) hám \(C(6; - 5)\) noqatları kvadrattıń
qarama-qarsı tóbeleri. Onıń tárepleriniń teńlemesin dúziń.
 \\
C3. 
\(\vec{p} = \vec{b} (\vec{a},\vec{c}) - \vec{c}(\vec{a},\vec{b})\) vektor \(\vec{a}\) vektorǵa perpendikulyar ekenligin dálilleń.
 \\

\end{tabular}
\vspace{1cm}


\begin{tabular}{m{17cm}}
\textbf{37-variant}
\newline

T1. 
Koordinataları menen berilgen vektrolardıń skalyar, vektorlıq hám aralas kóbeymeleri. \\
T2. 
Keńisliktegi tuwrınıń teńlemeleri. Tuwrılardıń ózara jaylasıwı.
 \\
A1. 
$A(4;2)$, $B(7;-2)$ hám $C(1;6)$ noqatları birtekli
sımnan islengen úshmúyeshlik tóbeleri. Usı úshmúyeshliktiń awırlıq
 \\
A2. 
$A(3;-2)$ noqatınan $3x+4y-15=0$ tuwrısına 
shekemgi awısıwdı hám aralıqtı esaplań.
 \\
A3. 
Úshmúyeshliktiń tóbeleri
$A (- 1; - 2;4) $, $B (- 4; - 2;0) $ hám $C (3; - 2;1) $. Onıń $B$ tóbesindegi
ishki múyeshi anıqlań. 
 \\
B1. 
Ordinata kósherinde sonday $M$ noqatın tabıń,
\(N(-8;13)\) noqatınan qashıqlıǵı 17 ge teń bolatuǵın.
 \\
B2. 
Tómende berilgen tuwrılar jubınıń qaysıları
perpendikuliyar ekenligin anıqlań: $4x+y+6=0,\ 2x-8y-13=0$.
 \\
B3. 
$\vec{a}$ hám $\vec{b}$ vektorlar óz ara perpendikulyar; $\vec{c}$ vektor olar menen $\pi/3$ ge teń bolǵan múyeshler payda etedi; $|\vec{a}| = 3$, $|\vec{b}| = 5,\ |\vec{c}| = 8$ ekenligi belgili, tómendegilerdi esaplań: 
$(\vec{a} + 2\vec{b} - 3\vec{c}) ^{2}$.
 \\
C1. \(A(4;2)\) noqatı arqalı, eki koordinata kósherlerine
urınıwshı sheńber ótkerildi. Onıń orayı $C$ nı hám radiusı
$R$ di tabıń.
 \\
C2. 
Kvadrattıń eki tárepiniń teńlemesi:
\(4x + 3y + 7 = 0,\ 4x + 3y - 15 = 0\) , hám bir tóbesi \(A(3;1)\) 
berilgen. Qalǵan eki tárepiniń teńlemelerin dúziń.
 \\
C3. 
\(\vec{a},\ \vec{b}\) hám \(\vec{c}\) vektorlar \(\vec{a} + \vec{b} + \vec{c} = 0\) shártti qanaatlandıradı. \(\lbrack\vec{a},\vec{b}\rbrack = \lbrack\vec{b},\vec{c}\rbrack = \lbrack\vec{c},\vec{a}\rbrack\) ekenligin dálilleń.
 \\

\end{tabular}
\vspace{1cm}


\begin{tabular}{m{17cm}}
\textbf{38-variant}
\newline

T1. 
Vektor túsinigi. Vektorlar ústinde sızıqlı ámeller.
 \\
T2. 
Tegislikte tuwrınıń teńlemeleri.
 \\
A1. 
$M(2;-1)$, $N(-1;4)$ hám $P(-2;2)$ noqatları
úshmúyeshliktiń tárepleriniń ortaları. Tóbeleriniń koordinataların
anıqlań.
 \\
A2. 
Ulıwma teńlemesi menen berilgen tuwrılardıń
óz-ara jaylasıwın anıqlań, eger kesilisetuǵın bolsa kesilisiw noqatın
tabıń: $2x-5y+1=0, 6x-15y+3=0$.
 \\
A3. 
Eger \(a = \{ 3; - 2;1\},\ \ \ \ b = \{ 2;1;2\},\ \ \ \ c = \{ 3; - 1; - 2\}\) bolsa, $\overrightarrow{a}, \overrightarrow{b}, \overrightarrow{c}$ vektorlar komplanar bolıwın tekseriń.
 \\
B1. 
Tóbeleri $A_1(1; 1), A_2(2; 3)$ hám $A(5;-1)$
noqatlarında jaylasqan úshmúyeshliktiń tuwrımúyeshli ekenligin dálilleń.
 \\
B2. 
\(M(2;-5)\)noqatı, berilgen tuwrılardıń:
\(3x+5y-4=0\) hám \(x-2y+3=0\) kesilisiwinde payda
bolǵan súyir yamasa doǵal múyeshke tiyisli bolıwın anıqlań.
 \\
B3. 
$A (1;2; - 1),B (0;1;5) $, $C (- 1;2;1),D (2;1;3) $ bir tegislikte jatıwın dálilleń.
 \\
C1. 
Úshmúyeshliktiń tóbeleri \(M_{1}( - 3;6),\ M_{2}(9; - 10)\) 
hám \(M_{3}( - 5;4)\) berilgen. Usı úshmúyeshlikke sırtlay sızılǵan
sheńber orayı $C$ nı hám radiusı $R$ di anıqlań.
 \\
C2. 
\(3x + 2y + 5 = 0\) hám \(2x + 7y - 8 = 0\) tuwrılarınıń
kesilisiw noqatınan ótip, \(2x + 3y - 7 = 0\) tuwrısı menen
$45^0$ múyesh jasawshı tuwrınıń teńlemesin dúziń.
Máseleni berilgen tuwrılardıń kesilisiw noqatınıń koordinataların anıqlamay
sheshiń.
 \\
C3. 
\(ABC\) úshmúyeshliktiń tárepleri menen sáykes keliwshi \(\vec{AB} = \vec{b}\) hám \(\vec{AC} = \vec{c}\) vektorlar berilgen. Bul úshmúyeshliktiń \(B\) tóbesinen túsirilgen \(BD\) biyikliginiń \(\vec{b},\ \vec{c}\) bazis boyınsha jayılmasın tabıń.
 \\

\end{tabular}
\vspace{1cm}


\begin{tabular}{m{17cm}}
\textbf{39-variant}
\newline

T1. 
Sızıqlı baylanıslı hám sızıqlı baylanıssız vektorlar.
 \\
T2. 
Tegislik hám tuwrılardıń ózara jaylasıwı.
 \\
A1. 
Parallelogrammnıń eki qońsı tóbeleri $A(-3;5)$, $B(1;7)$
hám dioganallarınıń kesilisiw noqatı $M(1;1)$ berilgen. Qalǵan eki
tóbesin anıqlań.
 \\
A2. 
$M(3;3)$ noqatınan ótip, koordinata kósherlerinen teńdey
kesindilerdi kesip alatuǵın tuwrılardıń teńlemesin dúziń.
 \\
A3. Vektor koordinata kósherleri menen tómendegi múyeshlerdi payda etiwi mumkin be:
$\alpha = 45^{{^\circ}},\beta = 60^{{^\circ}},\gamma = 120^{{^\circ}}$. 
 \\
B1. 
Tórtmúyeshliktiń tóbeleri
\(A(-3;12),\ B(3;-4),\ C(5;-4)\) hám \(D(5;8)\) berilgen. Usı
tórtmúyeshliktiń $AC$ diagonalı $BD$ dioganalın qanday
qatnasta bóliwin anıqlań.
 \\
B2. 
\(P(1;-2)\) noqatı hám koordintalar bası, berilgen eki
tuwrınıń: $12x-5y-7=0,\ 3x+4y-8=0$.
kesilisiwinen payda bolǵan birdey múyeshte me, qońsılas múyeshlerde me yáki vertikal
múyeshlerde jatama?.
 \\
B3. 
$\vec{a}$ hám $\vec{b}$ vektorlar óz ara perpendikulyar. $|\vec{a}| = 3,|\vec{b}| = 4$ ekenligin belgili, tómendegilerdi esaplań: 
$|\lbrack\vec{a} + \vec{b},\vec{a} - \vec{b}\rbrack|$. 
 \\
C1. 
\(M_{1}(1; - 2)\) noqatı arqalı, padiusı 5 ke teń,
$Ox$ kósherine urınıwshı sheńber ótkerildi. Usı sheńberdiń orayı
$С$ nı anıqlań.
 \\
C2. 
\(\alpha (\ 2x - 3y + 20) + \beta(3\ x + 5y - 27) = 0\) tuwrılar
dástesiniń orayı, eki biyikliginiń teńlemeleri
\(x - 4y + 1 = 0,\ 2x + y + 1 = 0\) menen berilgen úshmúyeshliktiń bir
tóbesi. Usı úshmúyeshliktiń tárepleriniń hám úshinshi biyikliginiń
teńlemesin dúziń. \\
C3. 
\(\vec{a} + \vec{b}\) vektor \(\vec{a} - \vec{b}\) vektorǵa perpendikulyar bolıwı ushın \(\vec{a}\) hám \(\vec{b}\) vektorlar qanday shártlerdi qanaatlandırıwı kerek?
 \\

\end{tabular}
\vspace{1cm}


\begin{tabular}{m{17cm}}
\textbf{40-variant}
\newline

T1. Analitikalıq geometriya pániniń predmeti hám usılları. 
 \\
T2. Tegislikte hám keńislikte dekart koordinatalar sistemasın almastırıw. 
 \\
A1. 
$A(1;-3)$ hám $B(4;3)$ noqatların tutastırıwshı
kesindi teńdey úsh bólekke bólindi. Bóliwshi noqatlardıń koordinataların
anıqlań.
 \\
A2. 
Ulıwma teńlemesi menen berilgen tuwrılardıń
óz-ara jaylasıwın anıqlań, eger kesilisetuǵın bolsa kesilisiw noqatın
tabıń: $x\sqrt{2}+12=0, 4x+24\sqrt{2}=0$.
 \\
A3. 
$\alpha$
nıń qanday mánisinde
$\overrightarrow{a} = \alpha\overrightarrow{i} - 3\overrightarrow{j} + 2\overrightarrow{k}$
hám
$\overrightarrow{b} = \overrightarrow{i} + 2\overrightarrow{j} - \alpha\overrightarrow{k}$
vektorlar óz ara perpendikulyar bolıwın anıqlań. 
 \\
B1. 
Parallelogrammnıń úsh tóbesi \(A(3;7),\ B(2;-3)\) hám
\(C(-1;4)\) noqatlarında jaylasqan. $B$ tóbesinen $AC$
tárepine túsirilgen biyikliktiń uzınlıǵın esaplań.
 \\
B2. 
Tárepleri
\(7x+y+31=0,\ 3x+4y-1=0,\ x-7y-17=0\) teńlemeleri
menen berilgen úshmúyeshliktiń teń qaptallı ekenligin dálilleń. 
Máseleni úshmúyeshliktiń
múyeshlerin tabıw arqalı sheshiń.
 \\
B3. 
$\vec{a} = \{ 6; - 8; - 7,5\}$ vektorǵa kollinear bolǵan $\vec{x}$ vektor $Oz$ kósheri menen súyir múyesh payda etedi. $|\vec{x}| = 50$ ekenligin bilgen halda onıń koordinataların tabıń.
 \\
C1. 
Úshmúyeshliktiń tóbeleri
\(A( - 1; - 1),\ B(3;5),\ C( - 4;1)\) berilgen. $A$ tóbesi sırtqı
múyeshi bssektrisasınıń, $BC$ tárepiniń dawamı menen kesilisiw
noqatın tabıń.
 \\
C2. 
Eki noqat \(A(3; - 5)\) hám \(B( - 2;3)\) berilgen.
$B$ noqattan ótip, $AB$ kesindige perpendikuliyar tuwrı
teńlemesin dúziń.
 \\
C3. 
\(ABC\) úshmúyeshliktiń tárepleri menen sáykes keliwshi \(\vec{AB} = \vec{b}\) hám \(\vec{AC} = \vec{c}\) vektorlar berilgen. Bul úshmúyeshliktiń \(B\) tóbesinen túsirilgen \(BD\) biyikliginiń \(\vec{b},\ \vec{c}\) bazis boyınsha jayılmasın tabıń.
 \\

\end{tabular}
\vspace{1cm}


\begin{tabular}{m{17cm}}
\textbf{41-variant}
\newline

T1. 
Vektorlardıń skalyar kóbeymesi.
 \\
T2. 
Tegisliktegi tuwrılardıń ózara jaylasıwı.
 \\
A1. 
Birtekli elementten islengen saptıń ushları
$A(3;-5)$hám $B(-1;1)$ noqatlarında jaylasqan. Onıń awırlıq
orayı koordinatasın anıqlań.
 \\
A2. 
$D(-3;-5)$ noqatınan $4x-3y+20=0$ tuwrısına 
shekemgi awısıwdı hám aralıqtı esaplań.
 \\
A3. 
Berilgeni: $\overrightarrow{a}| = 10,|\overrightarrow{b}| = 2$ hám
$\left(\overrightarrow{a},\overrightarrow{b} \right) = 12$. Esaplań
$\left| \left\lbrack \overrightarrow{a},\overrightarrow{b} \right\rbrack \right|$.
 \\
B1. 
\(M_{1}(1;2)\) noqatına, \(A(1;0)\) hám \(B(-1;-2)\)
noqatlarınan ótetuǵın tuwrıǵa qarata simmetriyalı bolǵan \(M_{2}\) noqatınıń koordinataların tabıń.
 \\
B2. 
Úshmúyeshliktiń tóbeleri \(A(1;0),\ B(5;-2),\ C(3;2)\)
koordinataları menen berilgen. Úshmúyeshliklerdiń tárepleriniń hám
medianalarınıń teńlemelerin dúziń.
 \\
B3. 
$|\vec{a}| = 3,|\vec{b}| = 5$ berilgen. $\alpha$ niń qanday mánisinde $\vec{a} + \alpha\vec{b}$, $\vec{a} - \alpha\vec{b}$ vektorlar óz ara perpendikulyar bolatuǵının anıqlań.
 \\
C1. 
Eki tóbesi \(A(2;1)\) hám \(B(5; - 3)\) noqatlarında, hám
diagonallarınıń kesilisiw noqatı ordinata kósherine tiyisli
parallelogrammnıń maydanı \(S = 17\) ke teń. Qalǵan eki tóbesiniń
koordinataların anıqlań. \\
C2. 
$ABC$ úshmúyeshliginiń bir tóbesi koordinataları
\(A(2; - 1)\) hám bir tóbeden túsirilgen biyiklik
\( 7x - 10y + 1 = 0\) , bissektrisa \(3x - 2y + 5 = 0\) 
teńlemeleri berilgen. $B$ hám $C$ tóbeleri koordinataların
anıqlamay, $ABC$ úshmúyeshliginiń tárepleri teńlemelerin dúziń.
 \\
C3. 
\(\vec{a},\ \vec{b},\ \vec{c}\) vektorları ushın \(\alpha\vec{a} + \beta\vec{b} + \gamma\vec{c} = 0\) birdeyligi komplanarlıqtıń zárúr hám jeterli shárti bolıwın dálilleń, bunda \(\alpha,\beta,\gamma\) sanlarınan keminde birewi nolge teń emes. \\

\end{tabular}
\vspace{1cm}


\begin{tabular}{m{17cm}}
\textbf{42-variant}
\newline

T1. 
Vektorlardıń vektorlıq kóbeymesi hám aralas kóbeyme.
 \\
T2. 
Tegislikte tuwrınıń teńlemeleri.
 \\
A1. 
Parallelogrammnıń úsh tóbesi
$A(3;-5)$, $B(5;-3)$, $C(-1;3)$ berilgen. $B$ tóbesine
qaraqma-qarsı jaylasqan $D$ tóbesin anıqlań.
 \\
A2. 
$\alpha(x+2y-5)+\beta(3x-2y+1)=0$ tuwrılar
dástesi arasınan, tómendegi tuwrılardıń teńlemesin tabıń:
$3x+4y-10=0$ tuwrısına parallel.
 \\
A3. 
$\overrightarrow{a}
= \{ 1; - 1;3\}, \ \ \ \ \overrightarrow{b} = \{ - 2;2;1\}$, $\overrightarrow{c} = \{3; -2;5\}$ vektorları berilgen. Esaplań: 
$(\lbrack\overrightarrow{a},\overrightarrow{b}\rbrack,\overrightarrow{c})$.
 \\
B1. 
Tuwrı \(M_{1}(-12;-13)\) hám \(M_{2}(-2;-5)\)
noqatlarınan ótedi. Usı tuwrıda abscissası 3 ke teń noqattı tabıń.
 \\
B2. 
Berilgen tuwrılar arasındaǵı múyeshti anıqlań: $3x+2y+4=0,\ 5x-y+1=0$.
 \\
B3. 
$\vec{a} = \{ 2;1; - 1\}$ vektorǵa kollinear bolǵan hám $\left(\vec{x},\vec{a} \right) = 3$ shártti qanaatlandırıwshı $\vec{x}$ vektordı tabıń.
 \\
C1. 
Úshmúyeshliktiń tóbeleri
\(A(3; - 5),\ B(1; - 3),\ C(2; - 2)\) berilgen. $B$ tóbesi sırtqı
múyeshi bessektrisa uzınlıǵın anıqlań.
 \\
C2. 
\(\alpha (\ 2x + 3y + 5) + \beta(\ x + y + 3) = 0\) 
tuwrılar dástesi berilgen. Usı tuwrılar dástesinen,
\(x - y - 5 = 0\) hám \(x - y - 2 = 0\) tuwrıları arasındaǵı
kesindi uzınlıǵı \(\sqrt{5}\) ke teń bolǵan tuwrılardıń teńlemelerin
tabıń.
 \\
C3. 
Birdeylikti dálilleń: \((\lbrack\vec{a} + \vec{b},\vec{b} + \vec{c}\rbrack,\vec{c} + \vec{a}) = 2 (\lbrack\vec{a},\vec{b}\rbrack,\vec{c}) \).
 \\

\end{tabular}
\vspace{1cm}


\begin{tabular}{m{17cm}}
\textbf{43-variant}
\newline

T1. 
Vektorlardıń vektorlıq kóbeymesi hám aralas kóbeyme.
 \\
T2. 
Noqattan tuwrıǵa shekemgi aralıq. Tuwrılar dástesi.
 \\
A1. 
Eki tóbesi $A(3;1)$ hám $B(1;-3)$ noqatlarında, al
úshinshi $C$ tóbesi $Oy$ kósherine tiyisli bolǵan úshmúyeshliktiń
maydanı $S=3$ ke teń. $C$ tóbesiniń koordinataların anıqlań.
 \\
A2. 
Ulıwma teńlemesi menen berilgen tuwrılardıń      
óz-ara jaylasıwın anıqlań, eger kesilisetuǵın bolsa kesilisiw noqatın 
tabıń: $14x-9y-24=0, 7x-2y-17=0$.
 \\
A3. 
Tegislikte eki vektor
$\overrightarrow{p} = \{ 2; - 3\}$, $\overrightarrow{q} = \{ 1;2\}$.
$\overrightarrow{a} = \{9;4\}$ vektordıń
$\overrightarrow{p},\ \overrightarrow{q}$ bazis boyınsha jayılması tabılsın.
 \\
B1. 
\(P(2;2)\) hám \(Q(1;5)\) noqatları menen teńdey úsh
bólekke bólingen kesindiniń úshları $A$ hám $B$ noqatlarınıń
koordinataların anıqlań.
 \\
B2. 
Berilgen eki noqattan ótetuǵın tuwrınıń múyeshlik
koefficienti $k$ nı esaplań: $A(-4;3)$, $B(1;8)$.
 \\
B3. 
$\vec{a} = \{ 3; - 1; - 2\}$ hám $\vec{b} = \{ 1;2; - 1\}$ vektorları berilgen. Tómendegi vektor kóbeymelerdiń koordinataların tabıń: 
$\left\lbrack 2\vec{a} + \vec{b},\vec{b} \right\rbrack$.
 \\
C1. 
\(M_{1}(1; - 2)\) noqatı arqalı, padiusı 5 ke teń,
$Ox$ kósherine urınıwshı sheńber ótkerildi. Usı sheńberdiń orayı
$С$ nı anıqlań.
 \\
C2. 
Berilgen tuwrılardıń:
\(x + 2y - 10 = 0\) hám \(3x - 6y - 5 = 0\) kesilisiwinde payda
bolǵan, \(M(1; - 3)\) noqatı jatatuǵın múyesh bissektrisasınıń
teńlemesin dúziń.
 \\
C3. 
\(\vec{a} + \vec{b}\) vektor \(\vec{a} - \vec{b}\) vektorǵa perpendikulyar bolıwı ushın \(\vec{a}\) hám \(\vec{b}\) vektorlar qanday shártlerdi qanaatlandırıwı kerek?
 \\

\end{tabular}
\vspace{1cm}


\begin{tabular}{m{17cm}}
\textbf{44-variant}
\newline

T1. 
Vektor túsinigi. Vektorlar ústinde sızıqlı ámeller.
 \\
T2. 
Keńisliktegi tuwrınıń teńlemeleri. Tuwrılardıń ózara jaylasıwı.
 \\
A1. 
Tóbeleri $M(3;-4)$, $N(-2;3)$ hám $P(4;5)$ 
noqatlarında jaylasqan úshmúyeshliklerdiń maydanın esaplań.
 \\
A2. 
$\alpha(x+2y-5)+\beta(3x-2y+1)=0$ tuwrılar
dástesi arasınan, tómendegi tuwrılardıń teńlemesin tabıń:
$Oy$ kósherine perpendikuliyar. \\
A3. 
Ushları $A (1;2;1), B (3;-1;7)$ hám $C(7;4;-2)$ bolǵan úshmúyeshliktiń
ishki múyeshlerin esaplap tabıń. Bul úshmúyeshliktiń teń qaptallı ekenligin dálilleń. 
 \\
B1. 
Eki qarama-qarsı tóbeleri $P(3; -4)$ hám $Q(l;2)$ noqatlarında jaylasqan rombanıń tárepi uzınlıǵı \(5\sqrt{2}\). Usı romba biyikliginiń uzınlıǵın esaplań.
 \\
B2. 
Parallel tuwrılar arasındaǵı aralıqtı esaplań: $5x-12y+13=0,\ 5x-12y-26=0$.
 \\
B3. 
$\vec{a}$ hám $\vec{b}$ vektorlar óz ara perpendikulyar. $|\vec{a}| = 3,|\vec{b}| = 4$ ekenligin belgili, tómendegilerdi esaplań: 
$|\lbrack 3\vec{a} - \vec{b},\vec{a}-2\vec{b}\rbrack|$.
 \\
C1. 
Úshmúyeshliktiń tóbeleri \(M_{1}( - 3;6),\ M_{2}(9; - 10)\) 
hám \(M_{3}( - 5;4)\) berilgen. Usı úshmúyeshlikke sırtlay sızılǵan
sheńber orayı $C$ nı hám radiusı $R$ di anıqlań.
 \\
C2. 
Eki tóbesi \(A(2; - 3),\ B(3; - 2)\) noqatlarda
jaylasqan, maydanı \(S = 1,5\) ke teń bolǵan úshmúyeshliktiń,
awırlıq orayı \(3x - y - 8 = 0\) tuwrısına tiyisli. Úshinshi $C$
tóbesiniń koordinatasın anıqlań.
 \\
C3. \(\vec{a} + \vec{b} + \vec{c} = 0\) shártti qanaatlandırıwshı birlik \(\vec{a},\ \vec{b}\) hám \(\vec{c}\) vektorlar berilgen. Esaplań: \(\left(\vec{a},\vec{b} \right) + \left(\vec{b},\vec{c} \right) + \left(\vec{c},\vec{a} \right) \).
 \\

\end{tabular}
\vspace{1cm}


\begin{tabular}{m{17cm}}
\textbf{45-variant}
\newline

T1. 
Sızıqlı baylanıslı hám sızıqlı baylanıssız vektorlar.
 \\
T2. 
Tegisliktiń teńlemeleri. Tegisliklerdiń ózara jaylasıwı.
 \\
A1. 
$ABCD$ parallelogrammınıń úsh tóbesi $A(3; -7)$, 
$B(5; -7)$, $C(-2; 5)$ berilgen, tórtinshi tóbesi $D$, 
$B$ tóbesine qarama-qarsı. Usı parallelogrammnıń diagonalları
uzınlıqların anıqlań.
 \\
A2. 
$x+2y-17=0$, $2x-y+1=0$, $x+2y-3=0$ 
tuwrıları bir noqatta kesilise me?
 \\
A3. 
Eger \(a = \{ 2; - 1;2\},\ \ \ \ b = \{ 1;2; - 3\},\ \ \ \ c = \{ 3; - 4;7\}\) bolsa, $\overrightarrow{a}, \overrightarrow{b}, \overrightarrow{c}$ vektorlar komplanar bolıwın tekseriń. \\
B1. 
Eki noqat berilgen \(M(2;2)\) hám \(N(5;-2)\); abscissa kósherinde sonday $P$ noqatın tabıń, $MPN$ múyeshi tuwrı múyesh bolsın.
 \\
B2. 
\(\alpha(5x+3y+6)+\beta(3x-4y-37)=0\) tuwrılar
dástesi berilgen. \(7x+2y-15=0\) tuwrınıń usı tuwrılar dástesine
tiyisli yamasa tiyisli emesligin anıqlań.
 \\
B3. 
$\vec{a}$ hám $\vec{b}$ vektorlar $\varphi = 2\pi/3$ múyesh payda etedi. $|\vec{a}| = 3,|\vec{b}| = 4$ ekenligi belgili. Esaplań: 
$(\vec{a} + \vec{b}) ^{2}$.
 \\
C1. 
Eki tóbesi \(A(2;1)\) hám \(B(5; - 3)\) noqatlarında, hám
diagonallarınıń kesilisiw noqatı ordinata kósherine tiyisli
parallelogrammnıń maydanı \(S = 17\) ke teń. Qalǵan eki tóbesiniń
koordinataların anıqlań. \\
C2. 
\(P(2; - 1)\) noqatınan ótip,
\(2x - y + 5 = 0,\ 3x + 6y - 1 = 0\) tuwrıları menen teń qaptallı
úshmúyeshlik payda etetuǵın tuwrınıń teńlemesin dúziń.
 \\
C3. 
\(\vec{p} = \vec{b} - \frac{\vec{a} (\vec{a},\vec{b}) }{{\vec{a}}^{2}}\) vektor \(\vec{a}\) vektorǵa perpendikulyar ekenligin dálilleń.
 \\

\end{tabular}
\vspace{1cm}


\begin{tabular}{m{17cm}}
\textbf{46-variant}
\newline

T1. 
Vektorlardıń skalyar kóbeymesi.
 \\
T2. Tegislikte hám keńislikte dekart koordinatalar sistemasın almastırıw. 
 \\
A1. $M_1(1; -2)$, $M_2(2; 1)$ noqatları berilgen. 
Tómendegi kesindilerdiń koordinata kósherlerine proekciyaların tabıń: $\overline{M_1M_2}$ \\
 \\
A2. 
Ulıwma teńlemesi menen berilgen tuwrılardıń      
óz-ara jaylasıwın anıqlań, eger kesilisetuǵın bolsa kesilisiw noqatın 
tabıń: $3x+2y-27=0, x+5y-35=0$.
 \\
A3. 
Vektor koordinata kósherleri menen tómendegi múyeshlerdi payda etiwi
mumkin be: $\alpha = 90^{{^\circ}},\ \beta = 150^{{^\circ}}$,
$\gamma = 60^{{^\circ}}?$ 
 \\
B1. 
Tóbeleri \(M(-1;3),\ N(1,2)\ \)hám \(P(0;4)\)
noqatlarında jaylasqan úshmúyeshliktiń ishki múyeshleri súyir múyesh
ekenligin dálilleń.
 \\
B2. 
Úshmúyeshliktiń tóbeleri \(A(1;0),\ B(5;-2),\ C(3;2)\)
koordinataları menen berilgen. Úshmúyeshliklerdiń tárepleriniń hám
medianalarınıń teńlemelerin dúziń.
 \\
B3. 
$\vec{a}$ hám $\vec{b}$ vektorlar $\varphi = 2\pi/3$ múyesh payda etedi. $|\vec{a}| = 1,|\vec{b}| = 2$ ekenligin bilip, tómendegilerdi esaplań: 
$\lbrack 2\overrightarrow{a} + \overrightarrow{b},\overrightarrow{a} + 2\overrightarrow{b}\rbrack^{2}$.
 \\
C1. \(A(4;2)\) noqatı arqalı, eki koordinata kósherlerine
urınıwshı sheńber ótkerildi. Onıń orayı $C$ nı hám radiusı
$R$ di tabıń.
 \\
C2. 
$ABC$ úshmúyeshliginiń eki tóbesi
\(A(6; - 2),\ B(10;14)\) , hám biyiklikleriniń kesilisiw noqatı
\(N(4; - 1)\) berilgen. Bul úshmúyeshliktiń tárepleri teńlemesin dúziń.
 \\
C3. 
Birdeylikti dálilleń: \((\lbrack\vec{a},\vec{b}\rbrack,\vec{c} + \lambda\vec{a} + \mu\vec{b}) = (\lbrack\vec{a},\vec{b}\rbrack,\vec{c})\), bunda \(\lambda\) hám \(\mu\)-qálegen sanlar.
 \\

\end{tabular}
\vspace{1cm}


\begin{tabular}{m{17cm}}
\textbf{47-variant}
\newline

T1. 
Vektordıń koordinataları.
 \\
T2. 
Noqattan tegislikke shekem, keńislikte noqattan tuwrıǵa shekemgi hám ayqas tuwrılar arasındaǵı aralıq. \\
A1. 
Berilgen $A(3; -5)$, $B(-2; -7)$ hám
$C(18; 1)$ noqatları bir tuwrıda jatatuǵınlıǵın dálilleń.
 \\
A2. 
$a$ hám $b$ parametrleriniń qanday mánislerinde
$ax-2y-1=0$, $6x-4y-b=0$ tuwrıları parallel boladı?
 \\
A3. 
$\overrightarrow{a} = \{ 2; - 4;4\}$ hám $\overrightarrow{b} = \{ - 3;2;6\}$
vektorlar payda etken múyesh kosinusın esaplań.. 
 \\
B1. 
Tuwrı \(M(2;-3)\) hám \(N(-6;5)\) noqatlarınan ótedi.
Usı tuwrıda ordinatası $-5$ ke teń noqattı tabıń.
 \\
B2. 
Koordinata bası, berilgen tuwrılardıń:
\(3x+y-4=0\) hám \(3x-2y+6=0\) kesilisiwinde payda
bolǵan súyir yamasa doǵal múyeshke tiyisli bolıwın anıqlań.
 \\
B3. 
$\vec{a}$ hám $\vec{b}$ vektorlar $\varphi = 2\pi/3$ múyesh payda etedi. $|\vec{a}| = 3,|\vec{b}| = 4$ ekenligi belgili. Esaplań: 
${\vec{a}}^{2}$. 
 \\
C1. 
Úshmúyeshliktiń tóbeleri
\(A(3; - 5),\ B(1; - 3),\ C(2; - 2)\) berilgen. $B$ tóbesi sırtqı
múyeshi bessektrisa uzınlıǵın anıqlań.
 \\
C2. 
Úshmúyeshliktiń tóbeleri \(A(1;-2),\ B(5; 4)\) hám
\(C(-2;0)\) noqatlarda jaylasqan. $A$ tóbesindegi ishki hám sırtqı
múyeshleri bissektrisalarınıń teńlemelerin dúziń.
 \\
C3. 
\(\vec{a},\ \vec{b}\) hám \(\vec{c}\) vektorlar \(\vec{a} + \vec{b} + \vec{c} = 0\) shártti qanaatlandıradı. \(\lbrack\vec{a},\vec{b}\rbrack = \lbrack\vec{b},\vec{c}\rbrack = \lbrack\vec{c},\vec{a}\rbrack\) ekenligin dálilleń.
 \\

\end{tabular}
\vspace{1cm}


\begin{tabular}{m{17cm}}
\textbf{48-variant}
\newline

T1. 
Koordinataları menen berilgen vektrolardıń skalyar, vektorlıq hám aralas kóbeymeleri. \\
T2. 
Tegislik hám tuwrılardıń ózara jaylasıwı.
 \\
A1. 
Eki tóbesi $A(-3; 2)$ hám $B(1; 6)$ noqatlarında
jaylasqan durıs úshmúyeshliktiń maydanın esaplań.
 \\
A2. 
Ulıwma teńlemesi menen berilgen tuwrılardıń      
óz-ara jaylasıwın anıqlań, eger kesilisetuǵın bolsa kesilisiw noqatın
tabıń: $2x-3y+12=0, 4x-6y-21=0$.
 \\
A3. 
Úshmúyeshliktiń tóbeleri
$A (3;2; - 3) $, $B (5;1; - 1)$ hám $C (1; - 2;1) $. Onıń $A$ tóbesindegi sırtqı múyeshi anıqlansın. 
 \\
B1. 
Úshmúyeshliktiń tóbeleri \(A(5;0),\ B(0;1)\) hám \(C(3;3)\)
noqatlarında. Onıń ishki múyeshlerin tabıń.
 \\
B2. 
Tómende berilgen tuwrılar jubınıń qaysıları
perpendikuliyar ekenligin anıqlań: $4x+y+6=0,\ 2x-8y-13=0$.
 \\
B3. 
$a$ hám $b$ vektorlar $\varphi = \pi/6$ múyesh payda etedi; $|a| = \sqrt{3},|b| = 1$ ekenligi belgili. $p = a + b$ hám $q = a - b$ vektorlar arasındaǵi $\alpha$ múyeshti esaplań.
 \\
C1. 
Úshmúyeshliktiń tóbeleri
\(A( - 1; - 1),\ B(3;5),\ C( - 4;1)\) berilgen. $A$ tóbesi sırtqı
múyeshi bssektrisasınıń, $BC$ tárepiniń dawamı menen kesilisiw
noqatın tabıń.
 \\
C2. 
\(A(2; - 3)\) noqatı, bir tárepi \(4x - 3y + 11 = 0\) 
tuwrısında jatatuǵın kvadrattıń bir tóbesi. Qalǵan tárepleri tiyisli
tuwrılardıń teńlemesin dúziń.
 \\
C3. 
\(\vec{a}+\vec{b}\) hám \(\vec{a} - \vec{b}\) vektorlar kollinear bolıwı ushın \(\vec{a},\vec{b}\) vektorlar qanday shártti qanaatlandırıwı kerek?
 \\

\end{tabular}
\vspace{1cm}


\begin{tabular}{m{17cm}}
\textbf{49-variant}
\newline

T1. Analitikalıq geometriya pániniń predmeti hám usılları. 
 \\
T2. Tegislikte hám keńislikte dekart koordinatalar sistemasın almastırıw. 
 \\
A1. 
Birtekli elementten islengen saptıń awırlıq orayı
$M(1;4)$ noqatında, bir ushı $P(-2;2)$noqatında jaylasqan. Usı
saptıń ekinshi ushı $Q$-dıń koordinataların anıqlań.
 \\
A2. 
Ulıwma teńlemesi menen berilgen tuwrılardıń      
óz-ara jaylasıwın anıqlań, eger kesilisetuǵın bolsa kesilisiw noqatın
tabıń: $2y+9=0, y-5=0$.
 \\
A3. 
Eger \(a = \{ 2;3; - 1\},\ \ \ \ b = \{ 1; - 1;3\},\ \ \ \ c = \{ 1;9; - 11\}\) bolsa, $\overrightarrow{a}, \overrightarrow{b}, \overrightarrow{c}$ vektorlar komplanar bolıwın tekseriń.
 \\
B1. 
Tuwrı \(A(5;2)\) hám \(B( -4; -7)\) noqatlarınan ótedi.
Usı tuwrınıń ordinata kósheri menen kesilisiw noqatın tabıń.
 \\
B2. Berilgen tuwrılardıń kesilisiw noqatın tabıń: 
\(3x-4y-29=0, 2x+5y+19=0\).
 \\
B3. 
$\vec{a}$ hám $\vec{b}$ vektorlar $\varphi = 2\pi/3$ múyesh payda etedi. $|\vec{a}| = 3,|\vec{b}| = 4$ ekenligi belgili. Esaplań: 
$\left(3\vec{a} - 2\vec{b},\vec{a} + 2\vec{b} \right)$.
 \\
C1. 
Eki tóbesi \(A(2; - 3)\) hám \(B( - 5;1)\) noqatlarında,
úshinshi tóbesi $C$ ordinata kósherine tiyisli úshmúyeshliktiń
medianalarınıń kesilisiw noqatı $M$ abscissa kósherinde jatadı.
$M$ hám $C$ noqatlarınıń koordinataların anıqlań.
 \\
C2. 
$ABC$ úshmúyeshliginiń bir tóbesin \(C(4; - 1)\) , hám
bir tóbesinen júrgizilgen biyikliginiń: \(2x - 3y + 12 = 0\) , hám
medianasınıń: \(2x + 3y = 0\) teńlemelerin bilgen jaǵdayda, tárepleriniń
teńlemelerin dúziń.
 \\
C3. 
\(\lbrack\vec{a},\vec{b}\rbrack^{2} <  {\vec{a}}^{2}{\vec{b}}^{2}\) ekenligin dálilleń; qanday jaǵdayda bul jerde teńlik belgisi boladı?
 \\

\end{tabular}
\vspace{1cm}


\begin{tabular}{m{17cm}}
\textbf{50-variant}
\newline

T1. 
Sızıqlı baylanıslı hám sızıqlı baylanıssız vektorlar.
 \\
T2. 
Noqattan tuwrıǵa shekemgi aralıq. Tuwrılar dástesi.
 \\
A1. 
Kvadrattıń eki qarama-qarsı tóbeleri $P(3; 5)$ hám
$Q(1; -3)$ berilgen. Onıń maydanın esaplań.
 \\
A2. 
$\alpha(x+2y-5)+\beta(3x-2y+1)=0$ tuwrılar
dástesi arasınan, tómendegi tuwrılardıń teńlemesin tabıń:
$Ox$ kósherine perpendikuliyar.
 \\
A3. 
$\overrightarrow{a}$ hám $\overrightarrow{b}$ vektorlar
$\varphi = \pi/6$ múyesh payda etedi.
$|\overrightarrow{a}| = 6,|\overrightarrow{b}| = 5$ ekenligin bilip,
$\left| \left\lbrack \overrightarrow{a},\overrightarrow{b} \right\rbrack \right|$ shamasın esaplań. 
 \\
B1. 
Úshmúyeshliktiń tóbeleri \(A(5;0),\ B(0;1)\) hám \(C(3;3)\)
noqatlarında. Onıń ishki múyeshlerin tabıń.
 \\
B2. 
Tuwrımúyeshliktiń eki tárepi
\(5x+2y-7=0,\ 5x+2y-36=0\) hám diagonalı
\(3x+7y-10=0\) teńlemeleri menen berilgen. Qalǵan eki tárepi
teńlemelerin dúziń.
 \\
B3. 
Tóbeleri $A (2;-1;1)$, $B (5;5;4)$,$C (3;2;-1)$ hám $D (4;1;3)$ noqatlarda jaylasqan tetraedrdiń kólemi esaplań. \\
C1. 
Eki tóbesi \(A(2;1)\) hám \(B(5; - 3)\) noqatlarında, hám
diagonallarınıń kesilisiw noqatı ordinata kósherine tiyisli
parallelogrammnıń maydanı \(S = 17\) ke teń. Qalǵan eki tóbesiniń
koordinataların anıqlań. \\
C2. 
$ABC$ úshmúyeshliginiń bir tóbesi \(A(1;3)\) noqatta,
hám eki medianası \(x - 2y + 1 = 0\ ,\ y - 1 = 0\) tuwrılarında
jaylasqan. Tárepleriniń teńlemelerin dúziń.
 \\
C3. 
Birdeylikti dálilleń: \(\lbrack\vec{a},\vec{b}\rbrack^{2} + (\vec{a},\vec{b}) ^{2} = {\vec{a}}^{2}{\vec{b}}^{2}\).
 \\

\end{tabular}
\vspace{1cm}

\begin{tabular}{m{17cm}}
    \textbf{51-variant}
    \newline
    
    T1. 
    Vektorlardıń skalyar kóbeymesi.
     \\
    T2. 
    Tegislikte tuwrınıń teńlemeleri.
     \\
    A1. 
    $M(2;-1)$, $N(-1;4)$ hám $P(-2;2)$ noqatları
    úshmúyeshliktiń tárepleriniń ortaları. Tóbeleriniń koordinataların
    anıqlań.
     \\
    A2. 
    $5x-y+3=0$ tuwrısınıń $k$ múyeshlik
    koefficientin hám $Oy$ kósherinen kesip alǵan kesindiniń algebralıq
    mánisi $b$ nı anıqlań.
     \\
    A3. 
    Berilgeni: $\overrightarrow{a}| = 10,|\overrightarrow{b}| = 2$ hám
    $\left(\overrightarrow{a},\overrightarrow{b} \right) = 12$. Esaplań
    $\left| \left\lbrack \overrightarrow{a},\overrightarrow{b} \right\rbrack \right|$.
     \\
    B1. 
    Tóbeleri $A_1(1; 1), A_2(2; 3)$ hám $A(5;-1)$
    noqatlarında jaylasqan úshmúyeshliktiń tuwrımúyeshli ekenligin dálilleń.
     \\
    B2. 
    Berilgen eki noqattan ótetuǵın tuwrınıń múyeshlik
    koefficienti $k$ nı esaplań: $A(-4;3)$, $B(1;8)$.
     \\
    B3. 
    $\vec{a}$ hám $\vec{b}$ vektorlar óz ara perpendikulyar; $\vec{c}$ vektor olar menen $\pi/3$ ge teń bolǵan múyeshler payda etedi; $|\vec{a}| = 3$, $|\vec{b}| = 5,\ |\vec{c}| = 8$ ekenligi belgili, tómendegilerdi esaplań: 
    $\left(3\vec{a} - 2\vec{b},\vec{b} + 3\vec{c} \right)$.
     \\
    C1. 
    Úshmúyeshliktiń tóbeleri
    \(A( - 1; - 1),\ B(3;5),\ C( - 4;1)\) berilgen. $A$ tóbesi sırtqı
    múyeshi bssektrisasınıń, $BC$ tárepiniń dawamı menen kesilisiw
    noqatın tabıń.
     \\
    C2. 
    \(\alpha_{1}(\ 5x + 3y - 2) + \beta_{1}(3x - y - 4) = 0\) ,
    \(\alpha_{2}(x - y + 1) + \beta_{2}(2x - y - 2) = 0\) eki tuwrılar
    dástesi teńlemeleri berilgen. Usı tuwrılar dásteleriniń orayın
    anıqlamay, olardıń ekewinede tiyisli bolǵan tuwrınıń teńlemesin dúziń.
     \\
    C3. 
    \(\lbrack\vec{a},\vec{b}\rbrack^{2} <  {\vec{a}}^{2}{\vec{b}}^{2}\) ekenligin dálilleń; qanday jaǵdayda bul jerde teńlik belgisi boladı?
     \\
    
    \end{tabular}
    \vspace{1cm}
    
    
    \begin{tabular}{m{17cm}}
    \textbf{52-variant}
    \newline
    
    T1. 
    Vektorlardıń vektorlıq kóbeymesi hám aralas kóbeyme.
     \\
    T2. 
    Tegisliktegi tuwrılardıń ózara jaylasıwı.
     \\
    A1. 
    $A(2;2)$, $B(-1;6)$, $C(-5;3)$ hám $D(-2;-1)$
    noqatları kvadrat tóbeleri ekenligin dálilleń.
     \\
    A2. 
    $5x+3y+2=0$ tuwrısınıń $k$ múyeshlik
    koefficientin hám $Oy$ kósherinen kesip alǵan kesindiniń algebralıq
    mánisi $b$ nı anıqlań.
     \\
    A3. 
    Eger \(a = \{ 3; - 2;1\},\ \ \ \ b = \{ 2;1;2\},\ \ \ \ c = \{ 3; - 1; - 2\}\) bolsa, $\overrightarrow{a}, \overrightarrow{b}, \overrightarrow{c}$ vektorlar komplanar bolıwın tekseriń.
     \\
    B1. 
    \(P(2;2)\) hám \(Q(1;5)\) noqatları menen teńdey úsh
    bólekke bólingen kesindiniń úshları $A$ hám $B$ noqatlarınıń
    koordinataların anıqlań.
     \\
    B2. 
    $ABC$ úshmúyeshliginiń tárepleri: 
    \(AB:4x+3y-5=0,\ BC:x-3y+10=0,\ AC:x-2=0
    \) teńlemeleri menen berilgen. Tóbeleriniń koordinataların anıqlań.
     \\
    B3. 
    $\vec{a}$ hám $\vec{b}$ vektorlar $\varphi = 2\pi/3$ múyesh payda etedi. $|\vec{a}| = 3,|\vec{b}| = 4$ ekenligi belgili. Esaplań: 
    $(\vec{a} + \vec{b}) ^{2}$.
     \\
    C1. 
    Úshmúyeshliktiń tóbeleri
    \(A(3; - 5),\ B(1; - 3),\ C(2; - 2)\) berilgen. $B$ tóbesi sırtqı
    múyeshi bessektrisa uzınlıǵın anıqlań.
     \\
    C2. 
    \(x - 4y - 5 = 0,\ x - 4y + 3 = 0\) tuwrıları
    arasındaǵı kesindi, berilgen \(P(1;1)\) noqatta teń ekige bólinetuǵın
    tuwrınıń teńlemesin dúziń.
     \\
    C3. 
    Birdeylikti dálilleń: \((\lbrack\vec{a},\vec{b}\rbrack,\vec{c} + \lambda\vec{a} + \mu\vec{b}) = (\lbrack\vec{a},\vec{b}\rbrack,\vec{c})\), bunda \(\lambda\) hám \(\mu\)-qálegen sanlar.
     \\
    
    \end{tabular}
    \vspace{1cm}
    
    
    \begin{tabular}{m{17cm}}
    \textbf{53-variant}
    \newline
    
    T1. 
    Vektordıń koordinataları.
     \\
    T2. 
    Tegisliktiń teńlemeleri. Tegisliklerdiń ózara jaylasıwı.
     \\
    A1. 
    $ABCD$-parallelogrammınıń úsh tóbesi
    $A(2;3)$, $B(4;-1)$ hám $C(0;5)$ berilgen. Tórtinshi $D$
    tóbesin tabıń.
     \\
    A2. 
    $M(4;-5)$ noqatı kvadrattıń bir tóbesi. 
    Kvadrattıń bir tárepi $5x-4y+1=0$ tuwrısında jatadı. 
    Kvadrattıń maydanın esaplań.
     \\
    A3. 
    Vektor koordinata kósherleri menen tómendegi múyeshlerdi payda etiwi mumkin be:
    $\alpha = 45^{{^\circ}},\ \ \ \ \beta = 135^{{^\circ}},\ \gamma = 60^{{^\circ}}$.
     \\
    B1. 
    Eki tóbesi \(A(3;1)\) hám \(B(1;-3)\) noqatlarında, hám
    awırlıq orayı $Ox$ kósherine tiyisli úshmúyeshliktiń maydanı
    \(S=3\) ke teń. Úshinshi $C$ tóbesiniń koordinataların anıqlań. \\
    B2. 
    \(P(2;3)\) hám \(Q(5;-1)\) noqatları, berilgen eki
    tuwrınıń: $12x-y-7=0,\ 13x+4y-5=0$.
    kesilisiwinen payda bolǵan birdey múyeshte me, qońsılas múyeshlerde me yáki vertikal
    múyeshlerde jatama?.
     \\
    B3. 
    $\vec{a}$ hám $\vec{b}$ vektorlar $\varphi = 2\pi/3$ múyesh payda etedi. $|\vec{a}| = 3,|\vec{b}| = 4$ ekenligi belgili. Esaplań: 
    $\left(\vec{a},\vec{b} \right)$.
     \\
    C1. 
    Eki tóbesi \(A(2; - 3)\) hám \(B( - 5;1)\) noqatlarında,
    úshinshi tóbesi $C$ ordinata kósherine tiyisli úshmúyeshliktiń
    medianalarınıń kesilisiw noqatı $M$ abscissa kósherinde jatadı.
    $M$ hám $C$ noqatlarınıń koordinataların anıqlań.
     \\
    C2. 
    \(\alpha (\ 2x - 3y + 20) + \beta(3\ x + 5y - 27) = 0\) tuwrılar
    dástesiniń orayı, diagonalı \(x + 7y - 16 = 0\) tuwrısında jatatuǵın
    kvadrattıń bir tóbesi. Usı kvadrattıń tárepleriniń hám ekinshi diagonali
    teńlemelerin dúziń.
     \\
    C3. 
    \(\vec{a},\ \vec{b},\ \vec{c}\) vektorlar \(\lbrack\vec{a},\vec{b}\rbrack + \lbrack\vec{b},\vec{c}\rbrack + \lbrack\vec{c},\vec{a}\rbrack = 0\) shártti qanaatlandırıwsh
     \\
    
    \end{tabular}
    \vspace{1cm}
    
    
    \begin{tabular}{m{17cm}}
    \textbf{54-variant}
    \newline
    
    T1. 
    Sızıqlı baylanıslı hám sızıqlı baylanıssız vektorlar.
     \\
    T2. 
    Tegislik hám tuwrılardıń ózara jaylasıwı.
     \\
    A1. 
    Birtekli besmúyeshli plastinkanıń tóbeleri berilgen:
    $A(2;3), \ B(0;6), \ C(-1;5), \ D(0;1)$ hám $E(1;1)$. Onıń awırlıq
    orayı koordinataların anıqlań.
     \\
    A2. 
    $\alpha(2x+3y-1)+\beta(x-2y-4)=0$ teńlemesi
    menen berilgen tuwrılar dástesiniń orayınıń koordinataların anıqlań.
     \\
    A3. 
    Eger \(a = \{ 2;3; - 1\},\ \ \ \ b = \{ 1; - 1;3\},\ \ \ \ c = \{ 1;9; - 11\}\) bolsa, $\overrightarrow{a}, \overrightarrow{b}, \overrightarrow{c}$ vektorlar komplanar bolıwın tekseriń.
     \\
    B1. 
    Tórtmúyeshliktiń tóbeleri
    \(A(-2;14),\ B(4;-2),\ C(6;-2)\) hám \(D(6;10)\) berilgen. Usı
    tórtmúyeshliktiń $AC$ hám $BD$ dioganallarınıń kesilisiw
    noqatın tabıń.
     \\
    B2. 
    Berilgen \(8x-15y-25=0\) tuwrısınan awısıwı -2 ge
    teń bolǵan noqatlardıń geometriyalıq ornı teńlemesin dúziń.
     \\
    B3. 
    $\vec{a} = \{ 2;1; - 1\}$ vektorǵa kollinear bolǵan hám $\left(\vec{x},\vec{a} \right) = 3$ shártti qanaatlandırıwshı $\vec{x}$ vektordı tabıń.
     \\
    C1. 
    Úshmúyeshliktiń tóbeleri \(M_{1}( - 3;6),\ M_{2}(9; - 10)\) 
    hám \(M_{3}( - 5;4)\) berilgen. Usı úshmúyeshlikke sırtlay sızılǵan
    sheńber orayı $C$ nı hám radiusı $R$ di anıqlań.
     \\
    C2. 
    Úshmúyeshliktiń tárepleriniń teńlemeleri berilgen:
    \(x - 4y + 11 = 0,\ 5x + 4y - 17 = 0,\ x + 2y - 1 = 0.\) 
    Úshmúyeshliktiń tóbeleriniń koordinataların anıqlamay, onıń
    biyiklikleriniń teńlemelerin dúziń.
     \\
    C3. 
    \(\vec{p} = \vec{b} - \frac{\vec{a} (\vec{a},\vec{b}) }{{\vec{a}}^{2}}\) vektor \(\vec{a}\) vektorǵa perpendikulyar ekenligin dálilleń.
     \\
    
    \end{tabular}
    \vspace{1cm}
    
    
    \begin{tabular}{m{17cm}}
    \textbf{55-variant}
    \newline
    
    T1. 
    Vektor túsinigi. Vektorlar ústinde sızıqlı ámeller.
     \\
    T2. 
    Keńisliktegi tuwrınıń teńlemeleri. Tuwrılardıń ózara jaylasıwı.
     \\
    A1. 
    Tóbeleri $M(3;-4)$, $N(-2;3)$ hám $P(4;5)$ 
    noqatlarında jaylasqan úshmúyeshliklerdiń maydanın esaplań.
     \\
    A2. 
    $m$ parametriniń qanday mánislerinde 
    $mx+(2m+3)y+m+6=0$, $(2m+1)x+(m-1)y+m-2=0$ tuwrıları ordinata
    kósherinde jatıwshı noqatta kesilisedi.
     \\
    A3. 
    Tegislikte eki vektor
    $\overrightarrow{p} = \{ 2; - 3\}$, $\overrightarrow{q} = \{ 1;2\}$.
    $\overrightarrow{a} = \{9;4\}$ vektordıń
    $\overrightarrow{p},\ \overrightarrow{q}$ bazis boyınsha jayılması tabılsın.
     \\
    B1. 
    Tóbeleri \(M_{1}(1;1), M_{2}(0,2)\) hám
    \(M_{3}(2;-1)\) noqatlarında jaylasqan úshmúyeshliktiń ishki
    múyeshleri arasında doǵal múyesh bar yáki joqlıǵın anıqlań.
     \\
    B2. 
    \(4x+3y-1=0\) hám \(3x-2y+5=0\)
    tuwrılarınıń kesilisiw noqatınan ótip (bul noqattı anıqlamay), ordinata
    kósherinen \(b=4\) kesindi kesip alatuǵın tuwrınıń teńlemesin dúziń.
     \\
    B3. 
    $\vec{a} = \{ 3; - 1; - 2\}$ hám $\vec{b} = \{ 1;2; - 1\}$ vektorları berilgen. Tómendegi vektor kóbeymelerdiń koordinataların tabıń: 
    $\left\lbrack 2\vec{a} + \vec{b},\vec{b} \right\rbrack$.
     \\
    C1. 
    Eki tóbesi \(A(2;1)\) hám \(B(5; - 3)\) noqatlarında, hám
    diagonallarınıń kesilisiw noqatı ordinata kósherine tiyisli
    parallelogrammnıń maydanı \(S = 17\) ke teń. Qalǵan eki tóbesiniń
    koordinataların anıqlań. \\
    C2. 
    \(P(4; - 5)\) noqatınan ótip,
    \(A(5; - 2)\) hám \(B(3;9)\) noqatlarınan teńdey aralıqta jaylasqan
    tuwrınıń teńlemesin dúziń.
     \\
    C3. 
    Birdeylikti dálilleń: \(\lbrack\vec{a},\vec{b}\rbrack^{2} + (\vec{a},\vec{b}) ^{2} = {\vec{a}}^{2}{\vec{b}}^{2}\).
     \\
    
    \end{tabular}
    \vspace{1cm}
    
    
    \begin{tabular}{m{17cm}}
    \textbf{56-variant}
    \newline
    
    T1. Analitikalıq geometriya pániniń predmeti hám usılları. 
     \\
    T2. Tegislikte hám keńislikte dekart koordinatalar sistemasın almastırıw. 
     \\
    A1. 
    Birtekli tórtmúyeshli plastinkanıń tóbeleri berilgen:
    $A(2;1), \ B(5;3), \ C(-1;7)$ hám $D(-7;5)$. Onıń awırlıq orayı
    koordinataların anıqlań.
     \\
    A2. 
    $\alpha(x+2y-5)+\beta(3x-2y+1)=0$ tuwrılar
    dástesi arasınan, tómendegi tuwrılardıń teńlemesin tabıń:
    $Ox$ kósherine parallel.
     \\
    A3. 
    Eger \(a = \{ 2; - 1;2\},\ \ \ \ b = \{ 1;2; - 3\},\ \ \ \ c = \{ 3; - 4;7\}\) bolsa, $\overrightarrow{a}, \overrightarrow{b}, \overrightarrow{c}$ vektorlar komplanar bolıwın tekseriń. \\
    B1. 
    Úshmúyeshliktiń tóbeleri
    \(A(3;-5),\ B(-3;3),\ C(-1;-2)\) berilgen. $A$ tóbesiniń ishki
    múyeshi bessektrisanıń uzınlıǵın anıqlań.
     \\
    B2. 
    Berilgen \(3x-4y-10=0\) tuwrısına parallel hám onnan
    $d=3$ qashıqlıqta jatatuǵın tuwrılardıń teńlemesin dúziń.
     \\
    B3. 
    $\vec{a}$ hám $\vec{b}$ vektorlar $\varphi = 2\pi/3$ múyesh payda etedi. $|\vec{a}| = 1,|\vec{b}| = 2$ ekenligin bilip, tómendegilerdi esaplań: 
    $\lbrack 2\overrightarrow{a} + \overrightarrow{b},\overrightarrow{a} + 2\overrightarrow{b}\rbrack^{2}$.
     \\
    C1. \(A(4;2)\) noqatı arqalı, eki koordinata kósherlerine
    urınıwshı sheńber ótkerildi. Onıń orayı $C$ nı hám radiusı
    $R$ di tabıń.
     \\
    C2. 
    \(P(3;5)\) noqatınan ótip, \(4x + 6y - 7 = 0\) tuwrısı
    menen \(45^{0}\) múyesh jasap kesilisetuǵın tuwrı teńlemesin dúziń.
     \\
    C3. \(\vec{a} + \vec{b} + \vec{c} = 0\) shártti qanaatlandırıwshı birlik \(\vec{a},\ \vec{b}\) hám \(\vec{c}\) vektorlar berilgen. Esaplań: \(\left(\vec{a},\vec{b} \right) + \left(\vec{b},\vec{c} \right) + \left(\vec{c},\vec{a} \right) \).
     \\
    
    \end{tabular}
    \vspace{1cm}
    
    
    \begin{tabular}{m{17cm}}
    \textbf{57-variant}
    \newline
    
    T1. 
    Koordinataları menen berilgen vektrolardıń skalyar, vektorlıq hám aralas kóbeymeleri. \\
    T2. 
    Noqattan tegislikke shekem, keńislikte noqattan tuwrıǵa shekemgi hám ayqas tuwrılar arasındaǵı aralıq. \\
    A1. 
    $A(1;-3)$ hám $B(4;3)$ noqatların tutastırıwshı
    kesindi teńdey úsh bólekke bólindi. Bóliwshi noqatlardıń koordinataların
    anıqlań.
     \\
    A2. 
    Ulıwma teńlemesi menen berilgen tuwrılardıń      
    óz-ara jaylasıwın anıqlań, eger kesilisetuǵın bolsa kesilisiw noqatın 
    tabıń: $6x+10y+9=0, 3x+5y-6=0$.
     \\
    A3. 
    Vektor koordinata kósherleri menen tómendegi múyeshlerdi payda etiwi
    mumkin be: $\alpha = 90^{{^\circ}},\ \beta = 150^{{^\circ}}$,
    $\gamma = 60^{{^\circ}}?$ 
     \\
    B1. 
    Tóbeleri \(M(-1;3),\ N(1,2)\ \)hám \(P(0;4)\)
    noqatlarında jaylasqan úshmúyeshliktiń ishki múyeshleri súyir múyesh
    ekenligin dálilleń.
     \\
    B2. 
    $ABCD$ parallelogrammınıń eki qońsı tóbeleri
    \(A(3,3),\ B(-1;7)\) hám diagonallarınıń kesilisiw noqatı
    \(E(2;-4)\) berilgen. Usı parallelogram tárepleriniń teńlemelerin
    dúziń.
     \\
    B3. 
    $\vec{a}$ hám $\vec{b}$ vektorlar $\varphi = 2\pi/3$ múyesh payda etedi. $|\vec{a}| = 3,|\vec{b}| = 4$ ekenligi belgili. Esaplań: 
    $\left(3\vec{a} - 2\vec{b},\vec{a} + 2\vec{b} \right)$.
     \\
    C1. 
    \(M_{1}(1; - 2)\) noqatı arqalı, padiusı 5 ke teń,
    $Ox$ kósherine urınıwshı sheńber ótkerildi. Usı sheńberdiń orayı
    $С$ nı anıqlań.
     \\
    C2. 
    Kvadrattıń eki tárepiniń teńlemeleri berilgen:
    \(5x + 12y - 15 = 0,\ 5x + 12y + 25 = 0.\) \(M( - 3;4)\) noqatı
    kvadrattıń tárepine tiyisli ekenligin bilgen jaǵdayda, qalǵan
    tárepleriniń teńlemelerin dúziń.
     \\
    C3. 
    Birdeylikti dálilleń: \((\lbrack\vec{a} + \vec{b},\vec{b} + \vec{c}\rbrack,\vec{c} + \vec{a}) = 2 (\lbrack\vec{a},\vec{b}\rbrack,\vec{c}) \).
     \\
    
    \end{tabular}
    \vspace{1cm}
    
    
    \begin{tabular}{m{17cm}}
    \textbf{58-variant}
    \newline
    
    T1. 
    Sızıqlı baylanıslı hám sızıqlı baylanıssız vektorlar.
     \\
    T2. 
    Noqattan tuwrıǵa shekemgi aralıq. Tuwrılar dástesi.
     \\
    A1. $M_1(1; -2)$, $M_2(2; 1)$ noqatları berilgen. 
    Tómendegi kesindilerdiń koordinata kósherlerine proekciyaların tabıń: $\overline{M_1M_2}$ \\
     \\
    A2. 
    $\alpha(x+2y-5)+\beta(3x-2y+1)=0$ tuwrılar
    dástesi arasınan, tómendegi tuwrılardıń teńlemesin tabıń:
    $M(4;-1)$ noqatınan ótetuǵın.
     \\
    A3. 
    $\alpha$
    nıń qanday mánisinde
    $\overrightarrow{a} = \alpha\overrightarrow{i} - 3\overrightarrow{j} + 2\overrightarrow{k}$
    hám
    $\overrightarrow{b} = \overrightarrow{i} + 2\overrightarrow{j} - \alpha\overrightarrow{k}$
    vektorlar óz ara perpendikulyar bolıwın anıqlań. 
     \\
    B1. 
    Eki qarama-qarsı tóbeleri $P(3; -4)$ hám $Q(l;2)$ noqatlarında jaylasqan rombanıń tárepi uzınlıǵı \(5\sqrt{2}\). Usı romba biyikliginiń uzınlıǵın esaplań.
     \\
    B2. 
    Eki tuwrı aqrasındaǵı múyeshti tabıń: $2x+y-9=0,\ 3x-y+11=0$.
     \\
    B3. 
    $\vec{a}$ hám $\vec{b}$ vektorlar óz ara perpendikulyar; $\vec{c}$ vektor olar menen $\pi/3$ ge teń bolǵan múyeshler payda etedi; $|\vec{a}| = 3$, $|\vec{b}| = 5,\ |\vec{c}| = 8$ ekenligi belgili, tómendegilerdi esaplań: 
    $(\vec{a} + 2\vec{b} - 3\vec{c}) ^{2}$.
     \\
    C1. 
    Úshmúyeshliktiń tóbeleri \(M_{1}( - 3;6),\ M_{2}(9; - 10)\) 
    hám \(M_{3}( - 5;4)\) berilgen. Usı úshmúyeshlikke sırtlay sızılǵan
    sheńber orayı $C$ nı hám radiusı $R$ di anıqlań.
     \\
    C2. 
    $ABC$ úshmúyeshliginiń bir tóbesin \(C(4;3)\) , hám de
    basqa-basqa tóbelerinen júrgizilgen medianasınıń:
    \(6x + 10y - 59 = 0\) , hám bissektrisasınıń: \(x - 4y + 10 = 0\) 
    teńlemelerin bilgen jaǵdayda, tárepleriniń teńlemelerin dúziń.
     \\
    C3. 
    \(\vec{a} + \vec{b}\) vektor \(\vec{a} - \vec{b}\) vektorǵa perpendikulyar bolıwı ushın \(\vec{a}\) hám \(\vec{b}\) vektorlar qanday shártlerdi qanaatlandırıwı kerek?
     \\
    
    \end{tabular}
    \vspace{1cm}
    
    
    \begin{tabular}{m{17cm}}
    \textbf{59-variant}
    \newline
    
    T1. 
    Koordinataları menen berilgen vektrolardıń skalyar, vektorlıq hám aralas kóbeymeleri. \\
    T2. 
    Tegislik hám tuwrılardıń ózara jaylasıwı.
     \\
    A1. 
    Úshmúyeshliktiń tóbeleri $A(1;4)$, $B(3;-9)$, $C(-5;2)$
    berilgen. $B$ tóbesinen júrgizilgen mediana uzınlıǵın anıqlań.
     \\
    A2. 
    Ulıwma teńlemesi menen berilgen tuwrılardıń
    óz-ara jaylasıwın anıqlań, eger kesilisetuǵın bolsa kesilisiw noqatın
    tabıń: $3x+y\sqrt{3}=0, x\sqrt{3}+3y-6=0$.
     \\
    A3. 
    Ushları $A (1;2;1), B (3;-1;7)$ hám $C(7;4;-2)$ bolǵan úshmúyeshliktiń
    ishki múyeshlerin esaplap tabıń. Bul úshmúyeshliktiń teń qaptallı ekenligin dálilleń. 
     \\
    B1. 
    Parallelogrammnıń úsh tóbesi \(A(3;7),\ B(2;-3)\) hám
    \(C(-1;4)\) noqatlarında jaylasqan. $B$ tóbesinen $AC$
    tárepine túsirilgen biyikliktiń uzınlıǵın esaplań.
     \\
    B2. 
    \(\alpha(3x+y-1)+\beta(2x-y-9)=0\) tuwrılar dástesi
    berilgen. \(x+3y+13=0\) tuwrınıń usı tuwrılar dástesine tiyisli
    yamasa tiyisli emesligin anıqlań.
     \\
    B3. 
    $\vec{a}$ hám $\vec{b}$ vektorlar $\varphi = 2\pi/3$ múyesh payda etedi. $|\vec{a}| = 3,|\vec{b}| = 4$ ekenligi belgili. Esaplań: 
    ${\vec{a}}^{2}$. 
     \\
    C1. \(A(4;2)\) noqatı arqalı, eki koordinata kósherlerine
    urınıwshı sheńber ótkerildi. Onıń orayı $C$ nı hám radiusı
    $R$ di tabıń.
     \\
    C2. 
    Berilgen tuwrılardıń: \(3x - y - 10 = 0\) hám
    \(2x - 6y - 1 = 0\) kesilisiwinde payda bolǵan doǵal múyesh
    bissektrisasınıń teńlemesin dúziń.
     \\
    C3. 
    \(\vec{a}+\vec{b}\) hám \(\vec{a} - \vec{b}\) vektorlar kollinear bolıwı ushın \(\vec{a},\vec{b}\) vektorlar qanday shártti qanaatlandırıwı kerek?
     \\
    
    \end{tabular}
    \vspace{1cm}
    
    
    \begin{tabular}{m{17cm}}
    \textbf{60-variant}
    \newline
    
    T1. 
    Vektor túsinigi. Vektorlar ústinde sızıqlı ámeller.
     \\
    T2. 
    Tegisliktegi tuwrılardıń ózara jaylasıwı.
     \\
    A1. 
    Kvadrattıń eki qarama-qarsı tóbeleri $P(3; 5)$ hám
    $Q(1; -3)$ berilgen. Onıń maydanın esaplań.
     \\
    A2. 
    $a$ hám $b$ parametrleriniń qanday mánislerinde
    $ax-2y-1=0$, $6x-4y-b=0$ tuwrıları betlesedi?
     \\
    A3. 
    $\overrightarrow{a}
    = \{ 1; - 1;3\}, \ \ \ \ \overrightarrow{b} = \{ - 2;2;1\}$, $\overrightarrow{c} = \{3; -2;5\}$ vektorları berilgen. Esaplań: 
    $(\lbrack\overrightarrow{a},\overrightarrow{b}\rbrack,\overrightarrow{c})$.
     \\
    B1. 
    Úshmúyeshliktiń tóbeleri \(A(5;0),\ B(0;1)\) hám \(C(3;3)\)
    noqatlarında. Onıń ishki múyeshlerin tabıń.
     \\
    B2. 
    \(N(4;-5)\) noqatınan ótip, $2x+5y-7=0$ 
    tuwrılarına parallel tuwrılardıń teńlemesin dúziń. Máseleni múyeshlik
    koefficientti esaplamay sheshiń.
     \\
    B3. 
    $\vec{a}$ hám $\vec{b}$ vektorlar $\varphi = 2\pi/3$ múyesh payda etedi. $|\vec{a}| = 1,|\vec{b}| = 2$ ekenligin bilip, tómendegilerdi esaplań: 
    $\lbrack\vec{a},\vec{b}\rbrack^{2}$.
     \\
    C1. 
    Úshmúyeshliktiń tóbeleri
    \(A( - 1; - 1),\ B(3;5),\ C( - 4;1)\) berilgen. $A$ tóbesi sırtqı
    múyeshi bssektrisasınıń, $BC$ tárepiniń dawamı menen kesilisiw
    noqatın tabıń.
     \\
    C2. 
    \(\alpha (\ 2x - y - 4) + \beta(\ x - y - 4) = 0\) 
    tuwrılar dástesi berilgen. Usı tuwrılar dástesinen, berilgen
    \(Q(3; - 1)\) noqatınan aralıǵı \(d = 3\) -ke teń tuwrılar teńlemesin
    tabıń.
     \\
    C3. 
    \(\vec{a},\ \vec{b}\) hám \(\vec{c}\) vektorlar \(\vec{a} + \vec{b} + \vec{c} = 0\) shártti qanaatlandıradı. \(\lbrack\vec{a},\vec{b}\rbrack = \lbrack\vec{b},\vec{c}\rbrack = \lbrack\vec{c},\vec{a}\rbrack\) ekenligin dálilleń.
     \\
    
    \end{tabular}
    \vspace{1cm}


\end{document}
