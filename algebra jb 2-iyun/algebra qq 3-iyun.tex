\documentclass{article}
\usepackage[fontsize=12pt]{fontsize}
\usepackage[utf8]{inputenc}
\usepackage[T2A]{fontenc}
% \usepackage{unicode-math}

\usepackage{array}
\usepackage[a4paper,
left=7mm,
right=5mm,
top=7mm,]{geometry}
\usepackage{amsmath}
% \usepackage{amssymbol}
\usepackage{amsfonts}
\usepackage{setspace}
\onehalfspace



\renewcommand{\baselinestretch}{1} 

\everymath{\displaystyle}
\everydisplay{\displaystyle}
% \linespread{1.25}

\DeclareMathOperator{\sign}{sign}


\begin{document}

\pagenumbering{gobble}


\begin{tabular}{m{17cm}}
\textbf{1-variant}
\newline

\textbf{T1.} Evklid keńisligi. (Skalyar kóbeyme, ortogonal vektorlar, ortonormal bazis.) \\
\textbf{T2.} Unitar túrlendiriwler. (Unitar sızıqlı túrlendiriwler túsinigi,  Unitar túrlendiriwge túyinles túrlendiriwlerdiń matricası,   Ortonormal baziste unitar túrlendiriwlerdiń matricası) \\
\textbf{A1.} \(\mathbb{R}^{2}\) keńislikte anıqlangan tómendegi sáwlelendiriw skalyar kóbeyme bolatuģının anıqlań: \((x,y) = x_{1}y_{1} - x_{2}y_{2}\) \\
\textbf{A2.} Tómendegi kvadratlıq forma oń anıqlangan bolatuģın\(\lambda\) nıń barlıq mánislerin tabıń: \(5x_{1}^{2} + x_{2}^{2} + \lambda x_{3}^{2} + 4x_{1}x_{2} - 2x_{1}x_{3} - 2x_{2}x_{3}\); \\
\textbf{A3.} Tómendegi sáwlelendiriwlerden qaysıları keńislikte sızıqlı túrlendiriw boladı: 1. \\
\textbf{B1.} Ortogonallastırıw procesinen paydalanıp, berilgen vektorlar sistemasini ortogonallastirıń: \((1,1, - 1, - 2)\), \((5,8, - 2, - 3)\), (3, 9, 3, 8); \\
\textbf{B2.} Tómendegi kvadratlıq formanıń kanonikalıq kórinisin hám bul túrge keltiriwshi menshikli mánislerin tabıń: \(3x_{1}^{2} - 2x_{2}^{2} + 2x_{3}^{2} + 4x_{1}x_{2} - 3x_{1}x_{3} - x_{2}x_{3}\); \\
\textbf{B3.} Matricası tómendegishe bolǵan sızıqlı túrlendiriwdiń menshikli mánisi hám menshikli vektorların tabıń: \(\begin{pmatrix} 0 & 0 & 1 \\ 1 & 4 & 0 \\  - 2 & 0 & 2 \end{pmatrix}\); \\
\textbf{C1.} Jordan normal formasını tabıń \(\begin{pmatrix} 0 & 0 & 1 \\ 1 & 4 & 0 \\  - 2 & 0 & 2 \end{pmatrix}\); \\
\textbf{C2.} Tómendegi kvadratlıq formalarga sáykes keliwshi ermit bisızıqlı formalardı tabiń:\((5 - i)x_{1}\overline{x_{2}} + (5 + i)\overline{x_{1}}x_{2} + x_{2}\overline{x_{2}}\); \\
\textbf{C3.} Matricası tómendegishe bolǵan sızıqlı túrlendiriwdiń menshikli mánisi hám menshikli vektorların tabıń: \(\begin{pmatrix} 3 & - 1 & 0 & 0 \\ 1 & 1 & 0 & 0 \\ 3 & 0 & 5 & - 3 \\ 4 & - 1 & 3 & - 1 \end{pmatrix}\); \\

\end{tabular}
\vspace{1cm}


\begin{tabular}{m{17cm}}
\textbf{2-variant}
\newline

\textbf{T1.} Sızıqlı keńislikler.   (Vektor,  sızıqlı baylanıs, bazis, ólshem, )  \\
\textbf{T2.} Keri túrlendiriwler. ( Keri túrlendiriw túsinigi,   Keri túrlendiriwdiń sızıqlılıǵı) \\
\textbf{A1.} \(\mathbb{R}^{2}\) keńislikte anıqlangan tómendegi sáwlelendiriw skalyar kóbeyme bolatuģının anıqlań: \((x,y) = x_{1}y_{1} + 2x_{2}y_{1} + 2x_{1}y_{2} + 7x_{2}y_{2}\) \\
\textbf{A2.} Tómendegi vektorlar sisteması óz ara ortogonallıqqa tekseriń hám olardı ortogonallıq baziske shekem toltırın: \((2,1,2),\ (1,2, - 2)\); \\
\textbf{A3.} Tómendegi sáwlelendiriw \(V\) vektor keńislikte sızıqlı túrlendiriw boladı: \(V\) sızıqlı keńislik,\(Ax = a\), bul jerde \(a\)-fiksirlengen vektor; \\
\textbf{B1.} \(\mathbb{R}^{3}\) keńislikte \((x,y) = x_{1}y_{1} + 3x_{2}y_{2} + 2x_{3}y_{3}\) berilgen skalyar kóbeyme ushın \(a = (1, - 3,2)\) va \(b = (2,1, - 1)\) \(b = (0,1)\) vektorlar arasındaǵı múyeshti tabıń . \\
\textbf{B2.} Tómendegi kvadratlıq formanıń kanonikalıq kórinisin hám bul túrge keltiriwshi menshikli mánislerin tabıń: \(x_{1}x_{2} + x_{1}x_{3} + x_{2}x_{3}\); \\
\textbf{B3.} Tómendegi vektorlar sisteması óz ara ortogonallıqqa tekseriń hám olardı ortogonallıq baziske shekem toltırın: \((0,i,1,1),\ \ (1,2,1 + i, - 1 + i)\). \\
\textbf{C1.} Jordan normal formasını tabıń \(\begin{pmatrix} 7 & 0 & 0 \\ 10 & - 19 & 0 \\ 12 & - 24 & 13 \end{pmatrix}\); \\
\textbf{C2.} Tómendegi bisiziqli formalar ekvivalent emes ekenligin dálilleń:\(f_{1}(x,y) = 2x_{1}y_{2} - 3x_{1}y_{3} + x_{2}y_{3} - 2x_{2}y_{1} - x_{3}y_{2} - 3x_{3}y_{1}\),\(f_{2}(x,y) = x_{1}y_{2} - x_{2}y_{1} + 2x_{2}y_{2} + 3x_{1}y_{3} - 3x_{3}y_{1};\) \\
\textbf{C3.} Matricası tómendegishe bolǵan sızıqlı túrlendiriwdiń menshikli mánisi hám menshikli vektorların tabıń: \(\begin{pmatrix} 1 & 1 & 1 & 1 \\ 1 & 1 & - 1 & - 1 \\ 1 & - 1 & 1 & - 1 \\ 1 & - 1 & - 1 & 1 \end{pmatrix}\). \\

\end{tabular}
\vspace{1cm}


\begin{tabular}{m{17cm}}
\textbf{3-variant}
\newline

\textbf{T1.} Ortogonal  tolıqtırıwshı. (Ortogonal tolıqtırıwshı,  ortogonal proekciya) \\
\textbf{T2.} Túyinles túrlendiriw. ( Evklid keńisligindegi sızıqlı túrlendiriwler menen bisızıqlı formalar arasındaǵı baylanıs, Berilgen túrlendiriwge túyinles túrlendiriwler, Óz-ózine túyinles túrlendiriwler) \\
\textbf{A1.} Tómendegi kvadratlıq formanıń rangni anıqlań: \(x_{1}x_{2} + x_{2}x_{3} + x_{3}x_{4} + x_{1}x_{4}\); \\
\textbf{A2.} Tómendegi vektorlar sisteması óz ara ortogonallıqqa tekseriń hám olardı ortogonallıq baziske shekem toltırın: \((1,2, - 1),(3, - 1,1)\); \\
\textbf{A3.} Matricası tómendegishe bolgan sızıqlı túrlendiriwdin menshikli mánisi hám menshikli vektorların tabıń \(\begin{pmatrix} 3 & 4 \\ 5 & 2 \end{pmatrix}\); \\
\textbf{B1.} Ortogonallastırıw procesinen paydalanıp, berilgen vektorlar sistemasini ortogonallastirıń: \((1,1, - 1)\), (1, 1,1 ), \((3,2, - 1)\); \\
\textbf{B2.} Eger \(f\) sızıqlı funkciya\(e_{1},e_{2},e_{3}\) bazisde \(f(x) = 2x_{1} - 3x_{2} + x_{3}\) arqalı anıqlanǵan bolsa, onıń \(e_{1}^{'},e_{2}^{'},e_{3}^{'}\) bazisdegi kórinisin tabıń\(e_{1}^{'} = 4e_{1} - e_{2} - 3e_{3},\ e_{2}^{'} = 2e_{1} + e_{2},\ e_{3}^{'} = 3e_{1} + 2e_{2}\). \\
\textbf{B3.} Tómendegi vektorlar sisteması óz ara ortogonallıqqa tekseriń hám olardı ortogonallıq baziske shekem toltırın. \((i,i,1, - 1),\ \ (1, - 1 + i,0,1),\ \ \); \\
\textbf{C1.} Berilgen \(A\) bisızıqlı formanıń\(e_{1},e_{2},e_{3}\) bazisdegi matritsası hám\(e_{1}^{'},e_{2}^{'},e_{3}^{'}\) baziske ótiw formulaları berilgen bolsa, onda bul bisızıqli formanıń \(e_{1}^{'},e_{2}^{'},e_{3}^{'}\) bazisdegi matritsasini tabıń: \(\begin{pmatrix} 2 & 2 & 3 \\  - 4 & 3 & 1 \\ 3 & 1 & 2 \end{pmatrix},\ \begin{matrix}  & e_{1}^{'} = e_{1} + 3e_{2} - 2e_{3} \\  & e_{2}^{'} = 2e_{1} + e_{2} - e_{3} \\  & e_{3}^{'} = e_{1} + e_{2} - 3e_{3} \end{matrix}\) \\
\textbf{C2.} Tómendegi bisiziqli formalar ekvivalent emes ekenligin dálilleń:\(f_{1}(x,y) = x_{1}y_{1} + 2x_{1}y_{2} + 2x_{2}y_{1} + 5x_{2}y_{2} + 6x_{2}y_{3} + 8x_{3}y_{2} + 10x_{3}y_{3}\), \(f_{2}(x,y) = 2x_{1}y_{1} - x_{1}y_{3} + x_{2}y_{2} - x_{3}y_{1} + 5x_{3}y_{3}\). \\
\textbf{C3.} Matricası tómendegishe bolǵan sızıqlı túrlendiriwdiń menshikli mánisi hám menshikli vektorların tabıń: \(\begin{pmatrix} 2 & - 1 & 2 \\ 5 & - 3 & 3 \\  - 1 & 0 & - 2 \end{pmatrix}\); \\

\end{tabular}
\vspace{1cm}


\begin{tabular}{m{17cm}}
\textbf{4-variant}
\newline

\textbf{T1.} Kompleks evklid keńislikleri.  (Kompleks vektorlı keńislik, Ermit kvadratlıq forma.) \\
\textbf{T2.} Sızıqlı túrlendiriwler.  (Sızıqlı túrlendiriw túsinigi, Sızıqlı túrlendiriwler ústinde ámeller, Sızıqlı túrlendiriwlerdiń obrazı hám yadrosı.) \\
\textbf{A1.} Tómendegi kvadratlıq formanıń rangni anıqlań: \(x_{1}x_{2} + x_{1}x_{3} + x_{2}x_{3}\); \\
\textbf{A2.} Tómendegi vektorlar sisteması óz ara ortogonallıqqa tekseriń hám olardı ortogonallıq baziske shekem toltırın: \((1, - 2,2, - 3),(2, - 3,2,4)\); \\
\textbf{A3.} Tómendegi sáwlelendiriwlerden qaysıları keńislikte sızıqlı túrlendiriw boladı:; \\
\textbf{B1.} Ortogonallastırıw procesinen paydalanıp, berilgen vektorlar sistemasini ortogonallastirıń: \((1,1, - 1,0)\), \((2,0, - 1,0)\), \((1, - 1,1, - 1)\), (2, \(0,1,1\) ). \\
\textbf{B2.} Tómendegi kvadratlıq formanıń kanonikalıq kórinisin hám bul túrge keltiriwshi menshikli mánislerin tabıń: \(x_{1}^{2} - 2x_{2}^{2} - 2x_{3}^{2} - 4x_{1}x_{2} - 4x_{1}x_{3} + 8x_{2}x_{3}\); \\
\textbf{B3.} Matricası tómendegishe bolǵan sızıqlı túrlendiriwdiń menshikli mánisi hám menshikli vektorların tabıń: \(\begin{pmatrix} 4 & - 5 & 2 \\ 0 & - 7 & 3 \\ 0 & 0 & 4 \end{pmatrix}\); \\
\textbf{C1.} Jordan narmal formasini tabıń\(\begin{pmatrix} 4 & - 5 & 2 \\ 0 & - 7 & 3 \\ 0 & 0 & 4 \end{pmatrix}\); \\
\textbf{C2.} Bazi bir ortonormal bazisde berilgen kvadratlıq formani kanonik kóriniske keltiriwshi ortonormal bazisin tabıń: \(x_{1}^{2} + x_{2}^{2} + 5x_{3}^{2} - 6x_{1}x_{2} - 2x_{1}x_{3} + 2x_{2}x_{3}\); \\
\textbf{C3.} Ortogonallastırıw procesinen paydalanıp, berilgen vektorlar sistemasini ortogonallastirıń: \((2,1,3, - 1)\), ( \(7,4,3, - 3\) ), ( \(1,1, - 6,0\) ), (5, 7, 7, 8); \\

\end{tabular}
\vspace{1cm}


\begin{tabular}{m{17cm}}
\textbf{5-variant}
\newline

\textbf{T1.} Kvadratlıq forma. (Lagran usulı, Yakobi usulı kvadratlıq formanı keltiriw.) \\
\textbf{T2.} Óz-ara orın almasıwshı túrlendiriwler. (Óz-ara orın almasıwshı túrlendiriwler,  Ortogonal bazis haqqında teorema,  Normal túrlendiriwlerdiń kanonikalıq kórinisi) \\
\textbf{A1.} Tómendegi kvadratlıq formanıń rangni anıqlań: \(x_{1}^{2} - 2x_{2}^{2} - 2x_{3}^{2} - 4x_{1}x_{2} - 4x_{1}x_{3} + 8x_{2}x_{3}\); \\
\textbf{A2.} Tómendegi funkciyalı haqiqiy sanlar maydanı ústinde anıqlangan \(V\) keńislikte sızıqlı funkciya boladı: \(V = \mathbb{R}^{3},\ \ f(x) = |x|\); \\
\textbf{A3.} Tómendegi sáwlelendiriwlerden qaysıları keńislikte sızıqlı túrlendiriw boladı:; \\
\textbf{B1.} Ortogonallastırıw procesinen paydalanıp, berilgen vektorlar sistemasini ortogonallastirıń: \((1,2,1,3)\), (4, 1, 1, 1), (3, 1, 1, 0); \\
\textbf{B2.} Tómendegi kvadratlıq formanıń kanonikalıq kórinisin hám bul túrge keltiriwshi menshikli mánislerin tabıń: \(7x_{1}^{2} + 5x_{2}^{2} + 3x_{3}^{2} - 8x_{1}x_{2} + 8x_{2}x_{3}\); \\
\textbf{B3.} Tómendegi vektorlar sisteması óz ara ortogonallıqqa tekseriń hám olardı ortogonallıq baziske shekem toltırın: \((0,1,i),\ \ (1 + i,i,1)\); \\
\textbf{C1.} Berilgen \(A\) bisızıqlı formanıń\(e_{1},e_{2},e_{3}\) bazisdegi matritsası hám\(e_{1}^{'},e_{2}^{'},e_{3}^{'}\) baziske ótiw formulaları berilgen bolsa, onda bul bisızıqli formanıń\(e_{1}^{'},e_{2}^{'},e_{3}^{'}\) bazisdegi matritsasini tabıń: \(\ \) \(\begin{pmatrix} 0 & 2 & 1 \\  - 2 & 2 & 0 \\  - 1 & 0 & 3 \end{pmatrix}\), \(e_{1}^{'} = e_{1} + 2e_{2} - e_{3}\), \(e_{2}^{'} = e_{2} - e_{3}\), \(e_{3}^{'} = - e_{1} + e_{2} - 3e_{3}\) \\
\textbf{C2.} Tómendegi kvadratlıq forma oń anıqlangan bolatuģın\(\lambda\) parametrnıń barlıq mánislerin tabıń: \(\lambda x_{1}\overline{x_{1}} - ix_{1}\overline{x_{2}} + ix_{2}\overline{x_{1}} + 3x_{2}\overline{x_{2}}\); \\
\textbf{C3.} Ortogonallastırıw procesinen paydalanıp, berilgen vektorlar sistemasini ortogonallastirıń: \((1,1, - 1,0)\), \((2,0, - 1,0)\), \((1, - 1,1, - 1)\), (2, \(0,1,1\) ). \\

\end{tabular}
\vspace{1cm}


\begin{tabular}{m{17cm}}
\textbf{6-variant}
\newline

\textbf{T1.} İnertsiya nızamı. (invariantlar,  eki kvadratılıq forma arasindaǵı baylanıs ) \\
\textbf{T2.} Sızıqlı túrlendiriwler matritsasınıń Jordan normal kórinisi. (Jordan kletkasınıń xarakteristikalıq matricası, Jordan matricasınıń uqsaslıǵı haqqında teorema,  Matricalardı jordan normal kórinisine keltiriw) \\
\textbf{A1.} Tómendegi kvadratlıq formanıń rangni anıqlań: \(3x_{1}^{2} - 2x_{2}^{2} + 2x_{3}^{2} + 4x_{1}x_{2} - 3x_{1}x_{3} - x_{2}x_{3}\); \\
\textbf{A2.} Tómendegi kvadratlıq forma oń anıqlangan bolatuģın\(\lambda\) nıń barlıq mánislerin tabıń: \(2x_{1}^{2} + 2x_{2}^{2} + x_{3}^{2} + 2\lambda x_{1}x_{2} + 6x_{1}x_{3} + 2x_{2}x_{3}\); \\
\textbf{A3.} Tómendegi sáwlelendiriwlerden qaysıları keńislikte sızıqlı túrlendiriw boladı:; \\
\textbf{B1.} Ortogonallastırıw procesinen paydalanıp, berilgen vektorlar sistemasini ortogonallastirıń: \((1,2,2, - 1)\), ( \(1,1, - 5,3\) ), (3, 2, 8, -7); \\
\textbf{B2.} Tómendegi kvadratlıq formanı kanonikalıq kóriniske keltiriń: \(12x_{1}^{2} + 3x_{2}^{2} + 12x_{3}^{2} - 12x_{1}x_{2} + 24x_{1}x_{3} - 8x_{2}x_{3}\); \\
\textbf{B3.} Tómendegi vektorlar sisteması óz ara ortogonallıqqa tekseriń hám olardı ortogonallıq baziske shekem toltırın: \((1,i, - i),\ \ ( - 2 - i,1 + i,2 - i)\); \\
\textbf{C1.} Jordan normal formasını tabıń \(\begin{pmatrix} 4 & 1 & - 4 \\ 1 & 4 & 0 \\  - 4 & 0 & 4 \end{pmatrix}\); \\
\textbf{C2.} Tómendegi kvadratlıq forma oń anıqlangan bolatuģın\(\lambda\) parametrnıń barlıq mánislerin tabıń: \(x_{1}\overline{x_{1}} + ix_{1}\overline{x_{2}} - ix_{2}\overline{x_{1}} + \lambda x_{2}\overline{x_{2}}\); \\
\textbf{C3.} Matricası tómendegishe bolǵan sızıqlı túrlendiriwdiń menshikli mánisi hám menshikli vektorların tabıń: \(\begin{pmatrix} 1 & - 3 & 4 \\ 4 & - 7 & 8 \\ 6 & - 7 & 7 \end{pmatrix}\); \\

\end{tabular}
\vspace{1cm}


\begin{tabular}{m{17cm}}
\textbf{7-variant}
\newline

\textbf{T1.} Sızıqlı, bisızıqlı hám kvadratlıq formalar. (Bisızıqlı forma,  simmetriyalı bisızıqlı formalar)  \\
\textbf{T2.} Óz-ara orın almasıwshı túrlendiriwler. (Óz-ara orın almasıwshı túrlendiriwler,  Ortogonal bazis haqqında teorema,  Normal túrlendiriwlerdiń kanonikalıq kórinisi) \\
\textbf{A1.} \(\mathbb{R}^{2}\) keńislikte anıqlangan tómendegi sáwlelendiriw skalyar kóbeyme bolatuģının anıqlań: \((x,y) = x_{1}y_{1} + x_{2}y_{1} + 3x_{1}y_{2} + 2x_{2}y_{2}\) \\
\textbf{A2.} Tómendegi kvadratlıq forma oń anıqlangan bolatuģın\(\lambda\) nıń barlıq mánislerin tabıń: \(x_{1}^{2} + \lambda x_{2}^{2} + x_{3}^{2} - 4x_{1}x_{2} - 8x_{2}x_{3}\); \\
\textbf{A3.} Tómendegi sáwlelendiriw\(V = \mathbb{R}^{3}\) keńislikte sızıqlı túrlendiriw boladı: \(A\left( x_{1},x_{2},x_{3} \right) = \left( x_{1} + 3x_{3},x_{2}^{3},x_{1} + x_{3} \right)\). \\
\textbf{B1.} \(\mathbb{R}^{2}\) keńislikte \((x,y) = x_{1}y_{1} + 2x_{2}y_{1} + 2x_{1}y_{2} + 5x_{2}y_{2}\) berilgen skalyar kóbeyme ushın \(a = (1,0)\) hám \(b = (0,1)\) vektorlar arasındaǵı múyeshti tabıń \\
\textbf{B2.} Eger \(f\) sızıqlı funkciya\(e_{1},e_{2},e_{3}\) bazisde \(f(x) = 2x_{1} - 3x_{2} + x_{3}\) arqalı anıqlanǵan bolsa, onıń \(e_{1}^{'},e_{2}^{'},e_{3}^{'}\) bazisdegi kórinisin tabıń\(e_{1}^{'} = e_{1} + e_{2} - 2e_{3},\ e_{2}^{'} = e_{1} + e_{2} + 2e_{3},\ e_{3}^{'} = e_{2} + e_{3}\); \\
\textbf{B3.} Matricası tómendegishe bolǵan sızıqlı túrlendiriwdiń menshikli mánisi hám menshikli vektorların tabıń: \(\begin{pmatrix} 7 & 0 & 0 \\ 10 & - 19 & 0 \\ 12 & - 24 & 13 \end{pmatrix}\); \\
\textbf{C1.} Jordan normal formasını tabıń \(\begin{pmatrix} 1 & - 2 & 1 \\  - 2 & 1 & 4 \\  - 1 & 4 & 1 \end{pmatrix}\). \\
\textbf{C2.} Tómendegi kvadratlıq formalarga sáykes keliwshi ermit bisızıqlı formalardı tabiń. \(x_{1}\overline{x_{1}} - ix_{1}\overline{x_{2}} - ix_{2}\overline{x_{1}} + 2x_{2}\overline{x_{2}}\); \\
\textbf{C3.} Matricası tómendegishe bolǵan sızıqlı túrlendiriwdiń menshikli mánisi hám menshikli vektorların tabıń: \(\begin{pmatrix} 1 & 0 & 0 & 0 \\ 0 & 0 & 0 & 0 \\ 1 & 0 & 0 & 0 \\ 0 & 0 & 0 & 1 \end{pmatrix}\); \\

\end{tabular}
\vspace{1cm}


\begin{tabular}{m{17cm}}
\textbf{8-variant}
\newline

\textbf{T1.} Sızıqlı, bisızıqlı hám kvadratlıq formalar. (Bisızıqlı forma,  simmetriyalı bisızıqlı formalar)  \\
\textbf{T2.} Keri túrlendiriwler. ( Keri túrlendiriw túsinigi,   Keri túrlendiriwdiń sızıqlılıǵı) \\
\textbf{A1.} \(\mathbb{R}^{2}\) keńislikte anıqlangan tómendegi sáwlelendiriw skalyar kóbeyme bolatuģının anıqlań: \((x,y) = x_{1}y_{1} - x_{2}y_{1} - x_{1}y_{2} + x_{2}y_{2}\) \\
\textbf{A2.} Tómendegi vektorlar sisteması óz ara ortogonallıqqa tekseriń hám olardı ortogonallıq baziske shekem toltırın: \((1,1,1,2)\), \((1,2,3, - 3)\). \\
\textbf{A3.} Matricası tómendegishe: \\
\textbf{B1.} Ortogonallastırıw procesinen paydalanıp, berilgen vektorlar sistemasini ortogonallastirıń: \((1,0,0)\), (0, 1, -1), (1, 1, 1); \\
\textbf{B2.} Tómendegi kvadratlıq formanıń kanonikalıq kórinisin hám bul túrge keltiriwshi menshikli mánislerin tabıń: \(5x_{1}^{2} + 6x_{2}^{2} + 4x_{3}^{2} - 4x_{1}x_{2} - 4x_{1}x_{3}\); \\
\textbf{B3.} Tómendegi vektorlar sisteması óz ara ortogonallıqqa tekseriń hám olardı ortogonallıq baziske shekem toltırın: \((1,1,1,1),(1,1, - 1, - 1),(1, - 1,1, - 1)\); \\
\textbf{C1.} Berilgen \(A\) bisızıqlı formanıń\(e_{1},e_{2},e_{3}\) bazisdegi matritsası hám\(e_{1}^{'},e_{2}^{'},e_{3}^{'}\) baziske ótiw formulaları berilgen bolsa, onda bul bisızıqli formanıń\(e_{1}^{'},e_{2}^{'},e_{3}^{'}\) bazisdegi matritsasini tabıń: \(\begin{pmatrix} 1 & 2 & 3 \\ 4 & 5 & 6 \\ 7 & 8 & 9 \end{pmatrix}\), \(e_{1}^{'} = e_{1} - e_{2}\), \(e_{2}^{'} = e_{1} + e_{3}\), \(e_{3}^{'} = e_{1} + e_{2} + e_{3}\) \\
\textbf{C2.} Bazi bir ortonormal bazisde berilgen kvadratlıq formani kanonik kóriniske keltiriwshi ortonormal bazisin tabıń: \(x_{1}^{2} - 5x_{2}^{2} + x_{3}^{2} + 4x_{1}x_{2} + 2x_{1}x_{3} + 4x_{2}x_{3}\); \\
\textbf{C3.} Matricası tómendegishe bolǵan sızıqlı túrlendiriwdiń menshikli mánisi hám menshikli vektorların tabıń: \(\begin{pmatrix} 1 & 0 & 0 & 0 \\ 0 & 0 & 0 & 0 \\ 0 & 0 & 0 & 0 \\ 1 & 0 & 0 & 0 \end{pmatrix}\); \\

\end{tabular}
\vspace{1cm}


\begin{tabular}{m{17cm}}
\textbf{9-variant}
\newline

\textbf{T1.} İnertsiya nızamı. (invariantlar,  eki kvadratılıq forma arasindaǵı baylanıs ) \\
\textbf{T2.} Unitar túrlendiriwler. (Unitar sızıqlı túrlendiriwler túsinigi,  Unitar túrlendiriwge túyinles túrlendiriwlerdiń matricası,   Ortonormal baziste unitar túrlendiriwlerdiń matricası) \\
\textbf{A1.} \(\mathbb{R}^{2}\) keńislikte anıqlangan tómendegi sáwlelendiriw skalyar kóbeyme bolatuģının anıqlań: \((x,y) = x_{1}y_{1} + 2x_{2}y_{2}\) \\
\textbf{A2.} Tómendegi kvadratlıq forma oń anıqlangan bolatuģın\(\lambda\) nıń barlıq mánislerin tabıń: \(x_{1}^{2} + 4x_{2}^{2} + x_{3}^{2} + 2\lambda x_{1}x_{2} + 10x_{1}x_{3} + 6x_{2}x_{3}\); \\
\textbf{A3.} Tómendegi sáwlelendiriwlerden qaysıları keńislikte sızıqlı túrlendiriw boladı:; \\
\textbf{B1.} Ortogonallastırıw procesinen paydalanıp, berilgen vektorlar sistemasini ortogonallastirıń: \((2,0,1,1)\), ( \(1,2,0,1\) ), ( \(0,1, - 2,0\) ); \\
\textbf{B2.} Tómendegi kvadratlıq formanıń kanonikalıq kórinisin hám bul túrge keltiriwshi menshikli mánislerin tabıń: \(2x_{1}^{2} + 3x_{2}^{2} + 4x_{3}^{2} - 2x_{1}x_{2} + 4x_{1}x_{3} - 3x_{2}x_{3}\); \\
\textbf{B3.} Tómendegi vektorlar sisteması óz ara ortogonallıqqa tekseriń hám olardı ortogonallıq baziske shekem toltırın: \((1,2,0, - 1),(3, - 1,1,1),( - 1,2,2,3)\); \\
\textbf{C1.} Berilgen \(A\) bisızıqlı formanıń\(e_{1},e_{2},e_{3}\) bazisdegi matritsası hám\(e_{1}^{'},e_{2}^{'},e_{3}^{'}\) baziske ótiw formulaları berilgen bolsa, onda bul bisızıqli formanıń \(e_{1}^{'},e_{2}^{'},e_{3}^{'}\) bazisdegi matritsasini tabıń: \(\begin{pmatrix} 1 & 1 & 2 \\  - 1 & 2 & 1 \\  - 1 & 1 & - 1 \end{pmatrix},\begin{matrix}  & e_{1}^{'} = e_{1} + e_{2} - 2e_{3} \\  & e_{2}^{'} = e_{1} + e_{2} + 2e_{3} \\  & e_{3}^{'} = e_{2} + e_{3} \end{matrix}\) \\
\textbf{C2.} Tómendegi kvadratlıq formalarga sáykes keliwshi ermit bisızıqlı formalardı tabiń: \(x_{1}\overline{x_{1}} + (2 + i)x_{1}\overline{x_{2}} + (2 - i)x_{2}\overline{x_{1}} + ix_{1}\overline{x_{3}} - ix_{3}\overline{x_{1}} - x_{3}\overline{x_{3}}\); \\
\textbf{C3.} Matricası tómendegishe bolǵan sızıqlı túrlendiriwdiń menshikli mánisi hám menshikli vektorların tabıń: \(\begin{pmatrix} 0 & 2 & 1 \\  - 2 & 0 & 3 \\  - 1 & - 3 & 0 \end{pmatrix}\); \\

\end{tabular}
\vspace{1cm}


\begin{tabular}{m{17cm}}
\textbf{10-variant}
\newline

\textbf{T1.} Sızıqlı keńislikler.   (Vektor,  sızıqlı baylanıs, bazis, ólshem, )  \\
\textbf{T2.} Sızıqlı túrlendiriwler.  (Sızıqlı túrlendiriw túsinigi, Sızıqlı túrlendiriwler ústinde ámeller, Sızıqlı túrlendiriwlerdiń obrazı hám yadrosı.) \\
\textbf{A1.} Tómendegi kvadratlıq formanıń rangni anıqlań: \(x_{1}^{2} + x_{2}^{2} + x_{3}^{2} + x_{4}^{2} + 2x_{1}x_{2} - 2x_{1}x_{4} - 2x_{2}x_{3} + 2x_{3}x_{4}\). \\
\textbf{A2.} Tómendegi kvadratlıq forma oń anıqlangan bolatuģın\(\lambda\) nıń barlıq mánislerin tabıń: \(2x_{1}^{2} + x_{2}^{2} + 3x_{3}^{2} + 2\lambda x_{1}x_{2} + 2x_{1}x_{3}\); \\
\textbf{A3.} Tómendegi sáwlelendiriw\(V = \mathbb{R}^{3}\) keńislikte sızıqlı túrlendiriw boladı: \(A\left( x_{1},x_{2},x_{3} \right) = \left( x_{1},x_{2} + 1,x_{3} + 2 \right)\); \\
\textbf{B1.} Ortogonallastırıw procesinen paydalanıp, berilgen vektorlar sistemasini ortogonallastirıń: \((1,1,0,0)\), (1, 0, 1, 1); \\
\textbf{B2.} Tómendegi kvadratlıq formanı kanonikalıq kóriniske keltiriń: \(x_{1}x_{2} + x_{1}x_{3} + x_{1}x_{4} + x_{2}x_{3} + x_{2}x_{4} + x_{3}x_{4}\); \\
\textbf{B3.} Matricası tómendegishe bolǵan sızıqlı túrlendiriwdiń menshikli mánisi hám menshikli vektorların tabıń: \(\begin{pmatrix} 2 & - 1 & 2 \\ 0 & - 3 & 0 \\ 0 & 0 & 1 \end{pmatrix}\); \\
\textbf{C1.} Jordan normal formasını tabıń \(\begin{pmatrix} 2 & - 1 & 2 \\ 0 & - 3 & 0 \\ 0 & 0 & 1 \end{pmatrix}\); \\
\textbf{C2.} Bazi bir ortonormal bazisde berilgen kvadratlıq formani kanonik kóriniske keltiriwshi ortonormal bazisin tabıń: \(11x_{1}^{2} + 5x_{2}^{2} + 2x_{3}^{2} + 16x_{1}x_{2} + 4x_{1}x_{3} - 20x_{2}x_{3}\); \\
\textbf{C3.} Matricası tómendegishe bolǵan sızıqlı túrlendiriwdiń menshikli mánisi hám menshikli vektorların tabıń: \(\begin{pmatrix} 5 & 6 & - 3 \\  - 1 & 0 & 1 \\ 1 & 2 & - 1 \end{pmatrix}\); \\

\end{tabular}
\vspace{1cm}


\begin{tabular}{m{17cm}}
\textbf{11-variant}
\newline

\textbf{T1.} Ortogonal  tolıqtırıwshı. (Ortogonal tolıqtırıwshı,  ortogonal proekciya) \\
\textbf{T2.} Túyinles túrlendiriw. ( Evklid keńisligindegi sızıqlı túrlendiriwler menen bisızıqlı formalar arasındaǵı baylanıs, Berilgen túrlendiriwge túyinles túrlendiriwler, Óz-ózine túyinles túrlendiriwler) \\
\textbf{A1.} Tómendegi kvadratlıq formanıń rangni anıqlań: \(2x_{1}^{2} + 3x_{2}^{2} + 4x_{3}^{2} - 2x_{1}x_{2} + 4x_{1}x_{3} - 3x_{2}x_{3}\) \\
\textbf{A2.} Tómendegi kvadratlıq forma oń anıqlangan bolatuģın\(\lambda\) nıń barlıq mánislerin tabıń: \(x_{1}^{2} + x_{2}^{2} + 5x_{3}^{2} + 2\lambda x_{1}x_{2} - 2x_{1}x_{3} + 4x_{2}x_{3}\); \\
\textbf{A3.} Tómendegi sáwlelendiriw\(V = \mathbb{R}^{3}\) keńislikte sızıqlı túrlendiriw boladı: \(A\left( x_{1},x_{2},x_{3} \right) = \left( x_{2} + x_{3},2x_{1} + x_{3},3x_{1} - x_{2} + x_{3} \right)\); \\
\textbf{B1.} Ortogonallastırıw procesinen paydalanıp, berilgen vektorlar sistemasini ortogonallastirıń: \((2,0,1,1)\), ( \(1,2,0,1\) ), ( \(0,1, - 2,0\) ); \\
\textbf{B2.} Eger \(f\) sızıqlı funkciya\(e_{1},e_{2},e_{3}\) bazisde \(f(x) = 2x_{1} - 3x_{2} + x_{3}\) arqalı anıqlanǵan bolsa, onıń \(e_{1}^{'},e_{2}^{'},e_{3}^{'}\) bazisdegi kórinisin tabıń\(e_{1}^{'} = e_{1} - e_{2},\ e_{2}^{'} = e_{1} + e_{3},\ \ e_{3}^{'} = e_{1} + e_{2} + e_{3}\); \\
\textbf{B3.} Tómendegi vektorlar sisteması óz ara ortogonallıqqa tekseriń hám olardı ortogonallıq baziske shekem toltırın: \((1,2,0, - 1),(3, - 1,1,1),( - 1,2,2,3)\); \\
\textbf{C1.} Jordan normal formasını tabıń \(\begin{pmatrix} 0 & 0 & 1 \\ 1 & 4 & 0 \\  - 2 & 0 & 2 \end{pmatrix}\); \\
\textbf{C2.} Tómendegi kvadratlıq formalarga sáykes keliwshi ermit bisızıqlı formalardı tabiń: \(ix_{1}\overline{x_{2}} - ix_{2}\overline{x_{1}} + (3 - 2i)x_{1}\overline{x_{3}} + (3 + 2i)x_{3}\overline{x_{1}} + 2x_{2}\overline{x_{3}} + 2x_{3}\overline{x_{2}}\). \\
\textbf{C3.} Ortogonallastırıw procesinen paydalanıp, berilgen vektorlar sistemasini ortogonallastirıń: \((1,1, - 1,0)\), \((2,0, - 1,0)\), \((1, - 1,1, - 1)\), (2, \(0,1,1\) ). \\

\end{tabular}
\vspace{1cm}


\begin{tabular}{m{17cm}}
\textbf{12-variant}
\newline

\textbf{T1.} Kvadratlıq forma. (Lagran usulı, Yakobi usulı kvadratlıq formanı keltiriw.) \\
\textbf{T2.} Sızıqlı túrlendiriwler matritsasınıń Jordan normal kórinisi. (Jordan kletkasınıń xarakteristikalıq matricası, Jordan matricasınıń uqsaslıǵı haqqında teorema,  Matricalardı jordan normal kórinisine keltiriw) \\
\textbf{A1.} \(\mathbb{R}^{2}\) keńislikte anıqlangan tómendegi sáwlelendiriw skalyar kóbeyme bolatuģının anıqlań: \((x,y) = x_{1}y_{1} - 2x_{2}y_{1} - 2x_{1}y_{2} + x_{2}y_{2}\) \\
\textbf{A2.} Tómendegi funkciyalı haqiqiy sanlar maydanı ústinde anıqlangan \(V\) keńislikte sızıqlı funkciya boladı: \(V = M_{n}\left( \mathbb{R} \right),\ \ f(A) = \det(A)\); \\
\textbf{A3.} Tómendegi sáwlelendiriwmos ravishda Berilgen \(V\) vektor keńislikte sızıqlı túrlendiriw boladı: \(V\) sızıqlı keńislik,\(Ax = x + a\), bul jerde \(a\)-fiksirlengen vektor; \\
\textbf{B1.} Ortogonallastırıw procesinen paydalanıp, berilgen vektorlar sistemasini ortogonallastirıń: \((1,2,1,3)\), (4, 1, 1, 1), (3, 1, 1, 0); \\
\textbf{B2.} Tómendegi kvadratlıq formanı kanonikalıq kóriniske keltiriń: \(2x_{1}^{2} + 18x_{2}^{2} + 8x_{3}^{2} - 12x_{1}x_{2} + 8x_{1}x_{3} - 27x_{2}x_{3}\); \\
\textbf{B3.} Tómendegi vektorlar sisteması óz ara ortogonallıqqa tekseriń hám olardı ortogonallıq baziske shekem toltırın. \((i,i,1, - 1),\ \ (1, - 1 + i,0,1),\ \ \); \\
\textbf{C1.} Jordan normal formasını tabıń \(\begin{pmatrix} 1 & - 2 & 1 \\  - 2 & 1 & 4 \\  - 1 & 4 & 1 \end{pmatrix}\). \\
\textbf{C2.} Bazi bir ortonormal bazisde berilgen kvadratlıq formani kanonik kóriniske keltiriwshi ortonormal bazisin tabıń: \(x_{1}^{2} + x_{2}^{2} + x_{3}^{2} + 4x_{1}x_{2} + 4x_{1}x_{3} + 4x_{2}x_{3}\); \\
\textbf{C3.} Matricası tómendegishe bolǵan sızıqlı túrlendiriwdiń menshikli mánisi hám menshikli vektorların tabıń: \(\begin{pmatrix} 1 & 0 & 0 & 0 \\ 0 & 0 & 0 & 0 \\ 0 & 0 & 0 & 0 \\ 1 & 0 & 0 & 0 \end{pmatrix}\); \\

\end{tabular}
\vspace{1cm}


\begin{tabular}{m{17cm}}
\textbf{13-variant}
\newline

\textbf{T1.} Evklid keńisligi. (Skalyar kóbeyme, ortogonal vektorlar, ortonormal bazis.) \\
\textbf{T2.} Sızıqlı túrlendiriwler.  (Sızıqlı túrlendiriw túsinigi, Sızıqlı túrlendiriwler ústinde ámeller, Sızıqlı túrlendiriwlerdiń obrazı hám yadrosı.) \\
\textbf{A1.} Tómendegi kvadratlıq formanıń rangni anıqlań: \(x_{1}x_{2} + x_{1}x_{3} + x_{2}x_{3}\); \\
\textbf{A2.} Tómendegi kvadratlıq forma oń anıqlangan bolatuģın\(\lambda\) nıń barlıq mánislerin tabıń: \(x_{1}^{2} + x_{2}^{2} + 5x_{3}^{2} + 2\lambda x_{1}x_{2} - 2x_{1}x_{3} + 4x_{2}x_{3}\); \\
\textbf{A3.} Tómendegi sáwlelendiriw\(V = \mathbb{R}^{3}\) keńislikte sızıqlı túrlendiriw boladı: \(A\left( x_{1},x_{2},x_{3} \right) = \left( x_{1} + 2,x_{2} + 5,x_{3} \right)\); \\
\textbf{B1.} Ortogonallastırıw procesinen paydalanıp, berilgen vektorlar sistemasini ortogonallastirıń: \((1,1, - 1, - 2)\), \((5,8, - 2, - 3)\), (3, 9, 3, 8); \\
\textbf{B2.} Eger \(f\) sızıqlı funkciya\(e_{1},e_{2},e_{3}\) bazisde \(f(x) = 2x_{1} - 3x_{2} + x_{3}\) arqalı anıqlanǵan bolsa, onıń \(e_{1}^{'},e_{2}^{'},e_{3}^{'}\) bazisdegi kórinisin tabıń\(e_{1}^{'} = e_{1} + 3e_{2} - 2e_{3},\ e_{2}^{'} = 2e_{1} + e_{2} - e_{3},\ e_{3}^{'} = e_{1} + e_{2} - 3e_{3}\); \\
\textbf{B3.} Tómendegi vektorlar sisteması óz ara ortogonallıqqa tekseriń hám olardı ortogonallıq baziske shekem toltırın: \((0,i,1,1),\ \ (1,2,1 + i, - 1 + i)\). \\
\textbf{C1.} Berilgen \(A\) bisızıqlı formanıń\(e_{1},e_{2},e_{3}\) bazisdegi matritsası hám\(e_{1}^{'},e_{2}^{'},e_{3}^{'}\) baziske ótiw formulaları berilgen bolsa, onda bul bisızıqli formanıń\(e_{1}^{'},e_{2}^{'},e_{3}^{'}\) bazisdegi matritsasini tabıń: \(\begin{pmatrix} 1 & 2 & 3 \\ 4 & 5 & 6 \\ 7 & 8 & 9 \end{pmatrix}\), \(e_{1}^{'} = e_{1} - e_{2}\), \(e_{2}^{'} = e_{1} + e_{3}\), \(e_{3}^{'} = e_{1} + e_{2} + e_{3}\) \\
\textbf{C2.} Tómendegi bisiziqli formalar ekvivalent emes ekenligin dálilleń:\(f_{1}(x,y) = x_{1}y_{1} + 2x_{1}y_{2} + 2x_{2}y_{1} + 5x_{2}y_{2} + 6x_{2}y_{3} + 8x_{3}y_{2} + 10x_{3}y_{3}\), \(f_{2}(x,y) = 2x_{1}y_{1} - x_{1}y_{3} + x_{2}y_{2} - x_{3}y_{1} + 5x_{3}y_{3}\). \\
\textbf{C3.} Matricası tómendegishe bolǵan sızıqlı túrlendiriwdiń menshikli mánisi hám menshikli vektorların tabıń: \(\begin{pmatrix} 3 & - 1 & 0 & 0 \\ 1 & 1 & 0 & 0 \\ 3 & 0 & 5 & - 3 \\ 4 & - 1 & 3 & - 1 \end{pmatrix}\); \\

\end{tabular}
\vspace{1cm}


\begin{tabular}{m{17cm}}
\textbf{14-variant}
\newline

\textbf{T1.} Kompleks evklid keńislikleri.  (Kompleks vektorlı keńislik, Ermit kvadratlıq forma.) \\
\textbf{T2.} Unitar túrlendiriwler. (Unitar sızıqlı túrlendiriwler túsinigi,  Unitar túrlendiriwge túyinles túrlendiriwlerdiń matricası,   Ortonormal baziste unitar túrlendiriwlerdiń matricası) \\
\textbf{A1.} \(\mathbb{R}^{2}\) keńislikte anıqlangan tómendegi sáwlelendiriw skalyar kóbeyme bolatuģının anıqlań: \((x,y) = x_{1}y_{1} - 2x_{2}y_{1} - 2x_{1}y_{2} + x_{2}y_{2}\) \\
\textbf{A2.} Tómendegi vektorlar sisteması óz ara ortogonallıqqa tekseriń hám olardı ortogonallıq baziske shekem toltırın: \((1, - 2,2, - 3),(2, - 3,2,4)\); \\
\textbf{A3.} Tómendegi sáwlelendiriw\(V = \mathbb{R}^{3}\) keńislikte sızıqlı túrlendiriw boladı: \(A\left( x_{1},x_{2},x_{3} \right) = \left( x_{1},x_{2},x_{1} + x_{2} + x_{3} \right)\); \\
\textbf{B1.} Ortogonallastırıw procesinen paydalanıp, berilgen vektorlar sistemasini ortogonallastirıń: \((1,1, - 1)\), (1, 1,1 ), \((3,2, - 1)\); \\
\textbf{B2.} Eger \(f\) sızıqlı funkciya\(e_{1},e_{2},e_{3}\) bazisde \(f(x) = 2x_{1} - 3x_{2} + x_{3}\) arqalı anıqlanǵan bolsa, onıń \(e_{1}^{'},e_{2}^{'},e_{3}^{'}\) bazisdegi kórinisin tabıń\(e_{1}^{'} = e_{1} - e_{2},\ e_{2}^{'} = e_{1} + e_{3},\ \ e_{3}^{'} = e_{1} + e_{2} + e_{3}\); \\
\textbf{B3.} Matricası tómendegishe bolǵan sızıqlı túrlendiriwdiń menshikli mánisi hám menshikli vektorların tabıń: \(\begin{pmatrix} 4 & - 5 & 2 \\ 0 & - 7 & 3 \\ 0 & 0 & 4 \end{pmatrix}\); \\
\textbf{C1.} Jordan normal formasını tabıń \(\begin{pmatrix} 7 & 0 & 0 \\ 10 & - 19 & 0 \\ 12 & - 24 & 13 \end{pmatrix}\); \\
\textbf{C2.} Tómendegi kvadratlıq formalarga sáykes keliwshi ermit bisızıqlı formalardı tabiń: \(ix_{1}\overline{x_{2}} - ix_{2}\overline{x_{1}} + (3 - 2i)x_{1}\overline{x_{3}} + (3 + 2i)x_{3}\overline{x_{1}} + 2x_{2}\overline{x_{3}} + 2x_{3}\overline{x_{2}}\). \\
\textbf{C3.} Ortogonallastırıw procesinen paydalanıp, berilgen vektorlar sistemasini ortogonallastirıń: \((2,1,3, - 1)\), ( \(7,4,3, - 3\) ), ( \(1,1, - 6,0\) ), (5, 7, 7, 8); \\

\end{tabular}
\vspace{1cm}


\begin{tabular}{m{17cm}}
\textbf{15-variant}
\newline

\textbf{T1.} İnertsiya nızamı. (invariantlar,  eki kvadratılıq forma arasindaǵı baylanıs ) \\
\textbf{T2.} Sızıqlı túrlendiriwler matritsasınıń Jordan normal kórinisi. (Jordan kletkasınıń xarakteristikalıq matricası, Jordan matricasınıń uqsaslıǵı haqqında teorema,  Matricalardı jordan normal kórinisine keltiriw) \\
\textbf{A1.} \(\mathbb{R}^{2}\) keńislikte anıqlangan tómendegi sáwlelendiriw skalyar kóbeyme bolatuģının anıqlań: \((x,y) = x_{1}y_{1} - x_{2}y_{1} - x_{1}y_{2} + x_{2}y_{2}\) \\
\textbf{A2.} Tómendegi kvadratlıq forma oń anıqlangan bolatuģın\(\lambda\) nıń barlıq mánislerin tabıń: \(2x_{1}^{2} + 2x_{2}^{2} + x_{3}^{2} + 2\lambda x_{1}x_{2} + 6x_{1}x_{3} + 2x_{2}x_{3}\); \\
\textbf{A3.} Tómendegi ańlatpalardan qaysıları sáykes túrde berilgen vektor keńislikte sızıqlı túrlendiriw boladı: sızıqlı keńislik,, bul jerde -fiksirlengen vektor; \\
\textbf{B1.} \(\mathbb{R}^{2}\) keńislikte \((x,y) = x_{1}y_{1} + 2x_{2}y_{1} + 2x_{1}y_{2} + 5x_{2}y_{2}\) berilgen skalyar kóbeyme ushın \(a = (1,0)\) hám \(b = (0,1)\) vektorlar arasındaǵı múyeshti tabıń \\
\textbf{B2.} Tómendegi kvadratlıq formanı kanonikalıq kóriniske keltiriń: \(2x_{1}^{2} + 18x_{2}^{2} + 8x_{3}^{2} - 12x_{1}x_{2} + 8x_{1}x_{3} - 27x_{2}x_{3}\); \\
\textbf{B3.} Matricası tómendegishe bolǵan sızıqlı túrlendiriwdiń menshikli mánisi hám menshikli vektorların tabıń: \(\begin{pmatrix} 7 & 0 & 0 \\ 10 & - 19 & 0 \\ 12 & - 24 & 13 \end{pmatrix}\); \\
\textbf{C1.} Jordan narmal formasini tabıń\(\begin{pmatrix} 4 & - 5 & 2 \\ 0 & - 7 & 3 \\ 0 & 0 & 4 \end{pmatrix}\); \\
\textbf{C2.} Bazi bir ortonormal bazisde berilgen kvadratlıq formani kanonik kóriniske keltiriwshi ortonormal bazisin tabıń: \(x_{1}^{2} + x_{2}^{2} + 5x_{3}^{2} - 6x_{1}x_{2} - 2x_{1}x_{3} + 2x_{2}x_{3}\); \\
\textbf{C3.} Matricası tómendegishe bolǵan sızıqlı túrlendiriwdiń menshikli mánisi hám menshikli vektorların tabıń: \(\begin{pmatrix} 0 & 2 & 1 \\  - 2 & 0 & 3 \\  - 1 & - 3 & 0 \end{pmatrix}\); \\

\end{tabular}
\vspace{1cm}


\begin{tabular}{m{17cm}}
\textbf{16-variant}
\newline

\textbf{T1.} Sızıqlı keńislikler.   (Vektor,  sızıqlı baylanıs, bazis, ólshem, )  \\
\textbf{T2.} Óz-ara orın almasıwshı túrlendiriwler. (Óz-ara orın almasıwshı túrlendiriwler,  Ortogonal bazis haqqında teorema,  Normal túrlendiriwlerdiń kanonikalıq kórinisi) \\
\textbf{A1.} Tómendegi kvadratlıq formanıń rangni anıqlań: \(x_{1}^{2} - 2x_{2}^{2} - 2x_{3}^{2} - 4x_{1}x_{2} - 4x_{1}x_{3} + 8x_{2}x_{3}\); \\
\textbf{A2.} Tómendegi funkciyalı haqiqiy sanlar maydanı ústinde anıqlangan \(V\) keńislikte sızıqlı funkciya boladı: \(V = M_{n}\left( \mathbb{R} \right),\ \ f(A) = \det(A)\); \\
\textbf{A3.} Tómendegi sáykes túrlendiriwlerden qaysıları berilgen vektor keńislikte sızıqlı túrlendiriw boladı: sızıqlı keńislik,, bul jerde -fiksirlengen vektor; \\
\textbf{B1.} \(\mathbb{R}^{3}\) keńislikte \((x,y) = x_{1}y_{1} + 3x_{2}y_{2} + 2x_{3}y_{3}\) berilgen skalyar kóbeyme ushın \(a = (1, - 3,2)\) va \(b = (2,1, - 1)\) \(b = (0,1)\) vektorlar arasındaǵı múyeshti tabıń . \\
\textbf{B2.} Eger \(f\) sızıqlı funkciya\(e_{1},e_{2},e_{3}\) bazisde \(f(x) = 2x_{1} - 3x_{2} + x_{3}\) arqalı anıqlanǵan bolsa, onıń \(e_{1}^{'},e_{2}^{'},e_{3}^{'}\) bazisdegi kórinisin tabıń\(e_{1}^{'} = e_{1} + e_{2} - 2e_{3},\ e_{2}^{'} = e_{1} + e_{2} + 2e_{3},\ e_{3}^{'} = e_{2} + e_{3}\); \\
\textbf{B3.} Matricası tómendegishe bolǵan sızıqlı túrlendiriwdiń menshikli mánisi hám menshikli vektorların tabıń: \(\begin{pmatrix} 2 & - 1 & 2 \\ 0 & - 3 & 0 \\ 0 & 0 & 1 \end{pmatrix}\); \\
\textbf{C1.} Jordan normal formasını tabıń \(\begin{pmatrix} 4 & 1 & - 4 \\ 1 & 4 & 0 \\  - 4 & 0 & 4 \end{pmatrix}\); \\
\textbf{C2.} Tómendegi kvadratlıq formalarga sáykes keliwshi ermit bisızıqlı formalardı tabiń: \(x_{1}\overline{x_{1}} + (2 + i)x_{1}\overline{x_{2}} + (2 - i)x_{2}\overline{x_{1}} + ix_{1}\overline{x_{3}} - ix_{3}\overline{x_{1}} - x_{3}\overline{x_{3}}\); \\
\textbf{C3.} Matricası tómendegishe bolǵan sızıqlı túrlendiriwdiń menshikli mánisi hám menshikli vektorların tabıń: \(\begin{pmatrix} 1 & 1 & 1 & 1 \\ 1 & 1 & - 1 & - 1 \\ 1 & - 1 & 1 & - 1 \\ 1 & - 1 & - 1 & 1 \end{pmatrix}\). \\

\end{tabular}
\vspace{1cm}


\begin{tabular}{m{17cm}}
\textbf{17-variant}
\newline

\textbf{T1.} Kvadratlıq forma. (Lagran usulı, Yakobi usulı kvadratlıq formanı keltiriw.) \\
\textbf{T2.} Túyinles túrlendiriw. ( Evklid keńisligindegi sızıqlı túrlendiriwler menen bisızıqlı formalar arasındaǵı baylanıs, Berilgen túrlendiriwge túyinles túrlendiriwler, Óz-ózine túyinles túrlendiriwler) \\
\textbf{A1.} Tómendegi kvadratlıq formanıń rangni anıqlań: \(x_{1}x_{2} + x_{2}x_{3} + x_{3}x_{4} + x_{1}x_{4}\); \\
\textbf{A2.} Tómendegi kvadratlıq forma oń anıqlangan bolatuģın\(\lambda\) nıń barlıq mánislerin tabıń: \(x_{1}^{2} + 4x_{2}^{2} + x_{3}^{2} + 2\lambda x_{1}x_{2} + 10x_{1}x_{3} + 6x_{2}x_{3}\); \\
\textbf{A3.} Tómendegi sáwlelendiriw\(V = \mathbb{R}^{3}\) keńislikte sızıqlı túrlendiriw boladı: \(A\left( x_{1},x_{2},x_{3} \right) = \left( 2x_{1} + x_{2},x_{1} + x_{3},x_{3}^{2} \right)\); \\
\textbf{B1.} Ortogonallastırıw procesinen paydalanıp, berilgen vektorlar sistemasini ortogonallastirıń: \((1,1,0,0)\), (1, 0, 1, 1); \\
\textbf{B2.} Tómendegi kvadratlıq formanıń kanonikalıq kórinisin hám bul túrge keltiriwshi menshikli mánislerin tabıń: \(7x_{1}^{2} + 5x_{2}^{2} + 3x_{3}^{2} - 8x_{1}x_{2} + 8x_{2}x_{3}\); \\
\textbf{B3.} Tómendegi vektorlar sisteması óz ara ortogonallıqqa tekseriń hám olardı ortogonallıq baziske shekem toltırın: \((1,1,1,1),(1,1, - 1, - 1),(1, - 1,1, - 1)\); \\
\textbf{C1.} Jordan normal formasını tabıń \(\begin{pmatrix} 2 & - 1 & 2 \\ 0 & - 3 & 0 \\ 0 & 0 & 1 \end{pmatrix}\); \\
\textbf{C2.} Bazi bir ortonormal bazisde berilgen kvadratlıq formani kanonik kóriniske keltiriwshi ortonormal bazisin tabıń: \(11x_{1}^{2} + 5x_{2}^{2} + 2x_{3}^{2} + 16x_{1}x_{2} + 4x_{1}x_{3} - 20x_{2}x_{3}\); \\
\textbf{C3.} Matricası tómendegishe bolǵan sızıqlı túrlendiriwdiń menshikli mánisi hám menshikli vektorların tabıń: \(\begin{pmatrix} 1 & 0 & 0 & 0 \\ 0 & 0 & 0 & 0 \\ 1 & 0 & 0 & 0 \\ 0 & 0 & 0 & 1 \end{pmatrix}\); \\

\end{tabular}
\vspace{1cm}


\begin{tabular}{m{17cm}}
\textbf{18-variant}
\newline

\textbf{T1.} Evklid keńisligi. (Skalyar kóbeyme, ortogonal vektorlar, ortonormal bazis.) \\
\textbf{T2.} Keri túrlendiriwler. ( Keri túrlendiriw túsinigi,   Keri túrlendiriwdiń sızıqlılıǵı) \\
\textbf{A1.} \(\mathbb{R}^{2}\) keńislikte anıqlangan tómendegi sáwlelendiriw skalyar kóbeyme bolatuģının anıqlań: \((x,y) = x_{1}y_{1} + 2x_{2}y_{2}\) \\
\textbf{A2.} Tómendegi vektorlar sisteması óz ara ortogonallıqqa tekseriń hám olardı ortogonallıq baziske shekem toltırın: \((2,1,2),\ (1,2, - 2)\); \\
\textbf{A3.} Tómendegi sáwlelendiriwlerden qaysıları sáykes túrde berilgen vektor keńislikte sızıqlı túrlendiriw boladı: sızıqlı keńislik, bul jerde -fiksirlengen san; \\
\textbf{B1.} Ortogonallastırıw procesinen paydalanıp, berilgen vektorlar sistemasini ortogonallastirıń: \((1,0,0)\), (0, 1, -1), (1, 1, 1); \\
\textbf{B2.} Tómendegi kvadratlıq formanıń kanonikalıq kórinisin hám bul túrge keltiriwshi menshikli mánislerin tabıń: \(3x_{1}^{2} - 2x_{2}^{2} + 2x_{3}^{2} + 4x_{1}x_{2} - 3x_{1}x_{3} - x_{2}x_{3}\); \\
\textbf{B3.} Tómendegi vektorlar sisteması óz ara ortogonallıqqa tekseriń hám olardı ortogonallıq baziske shekem toltırın: \((1,i, - i),\ \ ( - 2 - i,1 + i,2 - i)\); \\
\textbf{C1.} Berilgen \(A\) bisızıqlı formanıń\(e_{1},e_{2},e_{3}\) bazisdegi matritsası hám\(e_{1}^{'},e_{2}^{'},e_{3}^{'}\) baziske ótiw formulaları berilgen bolsa, onda bul bisızıqli formanıń\(e_{1}^{'},e_{2}^{'},e_{3}^{'}\) bazisdegi matritsasini tabıń: \(\ \) \(\begin{pmatrix} 0 & 2 & 1 \\  - 2 & 2 & 0 \\  - 1 & 0 & 3 \end{pmatrix}\), \(e_{1}^{'} = e_{1} + 2e_{2} - e_{3}\), \(e_{2}^{'} = e_{2} - e_{3}\), \(e_{3}^{'} = - e_{1} + e_{2} - 3e_{3}\) \\
\textbf{C2.} Tómendegi bisiziqli formalar ekvivalent emes ekenligin dálilleń:\(f_{1}(x,y) = 2x_{1}y_{2} - 3x_{1}y_{3} + x_{2}y_{3} - 2x_{2}y_{1} - x_{3}y_{2} - 3x_{3}y_{1}\),\(f_{2}(x,y) = x_{1}y_{2} - x_{2}y_{1} + 2x_{2}y_{2} + 3x_{1}y_{3} - 3x_{3}y_{1};\) \\
\textbf{C3.} Matricası tómendegishe bolǵan sızıqlı túrlendiriwdiń menshikli mánisi hám menshikli vektorların tabıń: \(\begin{pmatrix} 5 & 6 & - 3 \\  - 1 & 0 & 1 \\ 1 & 2 & - 1 \end{pmatrix}\); \\

\end{tabular}
\vspace{1cm}


\begin{tabular}{m{17cm}}
\textbf{19-variant}
\newline

\textbf{T1.} Ortogonal  tolıqtırıwshı. (Ortogonal tolıqtırıwshı,  ortogonal proekciya) \\
\textbf{T2.} Túyinles túrlendiriw. ( Evklid keńisligindegi sızıqlı túrlendiriwler menen bisızıqlı formalar arasındaǵı baylanıs, Berilgen túrlendiriwge túyinles túrlendiriwler, Óz-ózine túyinles túrlendiriwler) \\
\textbf{A1.} Tómendegi kvadratlıq formanıń rangni anıqlań: \(3x_{1}^{2} - 2x_{2}^{2} + 2x_{3}^{2} + 4x_{1}x_{2} - 3x_{1}x_{3} - x_{2}x_{3}\); \\
\textbf{A2.} Tómendegi funkciyalı haqiqiy sanlar maydanı ústinde anıqlangan \(V\) keńislikte sızıqlı funkciya boladı: \(V = \mathbb{R}^{3},\ \ f(x) = |x|\); \\
\textbf{A3.} Tómendegi sáwlelendiriwlerden qaysıları keńislikte sızıqlı túrlendiriw boladı:. \\
\textbf{B1.} Ortogonallastırıw procesinen paydalanıp, berilgen vektorlar sistemasini ortogonallastirıń: \((1,2,2, - 1)\), ( \(1,1, - 5,3\) ), (3, 2, 8, -7); \\
\textbf{B2.} Tómendegi kvadratlıq formanı kanonikalıq kóriniske keltiriń: \(x_{1}x_{2} + x_{1}x_{3} + x_{1}x_{4} + x_{2}x_{3} + x_{2}x_{4} + x_{3}x_{4}\); \\
\textbf{B3.} Tómendegi vektorlar sisteması óz ara ortogonallıqqa tekseriń hám olardı ortogonallıq baziske shekem toltırın: \((0,1,i),\ \ (1 + i,i,1)\); \\
\textbf{C1.} Berilgen \(A\) bisızıqlı formanıń\(e_{1},e_{2},e_{3}\) bazisdegi matritsası hám\(e_{1}^{'},e_{2}^{'},e_{3}^{'}\) baziske ótiw formulaları berilgen bolsa, onda bul bisızıqli formanıń \(e_{1}^{'},e_{2}^{'},e_{3}^{'}\) bazisdegi matritsasini tabıń: \(\begin{pmatrix} 1 & 1 & 2 \\  - 1 & 2 & 1 \\  - 1 & 1 & - 1 \end{pmatrix},\begin{matrix}  & e_{1}^{'} = e_{1} + e_{2} - 2e_{3} \\  & e_{2}^{'} = e_{1} + e_{2} + 2e_{3} \\  & e_{3}^{'} = e_{2} + e_{3} \end{matrix}\) \\
\textbf{C2.} Bazi bir ortonormal bazisde berilgen kvadratlıq formani kanonik kóriniske keltiriwshi ortonormal bazisin tabıń: \(x_{1}^{2} + x_{2}^{2} + x_{3}^{2} + 4x_{1}x_{2} + 4x_{1}x_{3} + 4x_{2}x_{3}\); \\
\textbf{C3.} Matricası tómendegishe bolǵan sızıqlı túrlendiriwdiń menshikli mánisi hám menshikli vektorların tabıń: \(\begin{pmatrix} 1 & - 3 & 4 \\ 4 & - 7 & 8 \\ 6 & - 7 & 7 \end{pmatrix}\); \\

\end{tabular}
\vspace{1cm}


\begin{tabular}{m{17cm}}
\textbf{20-variant}
\newline

\textbf{T1.} Sızıqlı, bisızıqlı hám kvadratlıq formalar. (Bisızıqlı forma,  simmetriyalı bisızıqlı formalar)  \\
\textbf{T2.} Keri túrlendiriwler. ( Keri túrlendiriw túsinigi,   Keri túrlendiriwdiń sızıqlılıǵı) \\
\textbf{A1.} \(\mathbb{R}^{2}\) keńislikte anıqlangan tómendegi sáwlelendiriw skalyar kóbeyme bolatuģının anıqlań: \((x,y) = x_{1}y_{1} + x_{2}y_{1} + 3x_{1}y_{2} + 2x_{2}y_{2}\) \\
\textbf{A2.} Tómendegi kvadratlıq forma oń anıqlangan bolatuģın\(\lambda\) nıń barlıq mánislerin tabıń: \(2x_{1}^{2} + x_{2}^{2} + 3x_{3}^{2} + 2\lambda x_{1}x_{2} + 2x_{1}x_{3}\); \\
\textbf{A3.} Tómendegi sáwlelendiriwmos ravishda Berilgen \(V\) vektor keńislikte sızıqlı túrlendiriw boladı: \(V\) sızıqlı keńislik,\(Ax = \alpha x\) bul jerde \(\alpha\)-fiksirlangan son; \\
\textbf{B1.} Ortogonallastırıw procesinen paydalanıp, berilgen vektorlar sistemasini ortogonallastirıń: \((1,1, - 1,0)\), \((2,0, - 1,0)\), \((1, - 1,1, - 1)\), (2, \(0,1,1\) ). \\
\textbf{B2.} Tómendegi kvadratlıq formanıń kanonikalıq kórinisin hám bul túrge keltiriwshi menshikli mánislerin tabıń: \(2x_{1}^{2} + 3x_{2}^{2} + 4x_{3}^{2} - 2x_{1}x_{2} + 4x_{1}x_{3} - 3x_{2}x_{3}\); \\
\textbf{B3.} Matricası tómendegishe bolǵan sızıqlı túrlendiriwdiń menshikli mánisi hám menshikli vektorların tabıń: \(\begin{pmatrix} 0 & 0 & 1 \\ 1 & 4 & 0 \\  - 2 & 0 & 2 \end{pmatrix}\); \\
\textbf{C1.} Berilgen \(A\) bisızıqlı formanıń\(e_{1},e_{2},e_{3}\) bazisdegi matritsası hám\(e_{1}^{'},e_{2}^{'},e_{3}^{'}\) baziske ótiw formulaları berilgen bolsa, onda bul bisızıqli formanıń \(e_{1}^{'},e_{2}^{'},e_{3}^{'}\) bazisdegi matritsasini tabıń: \(\begin{pmatrix} 2 & 2 & 3 \\  - 4 & 3 & 1 \\ 3 & 1 & 2 \end{pmatrix},\ \begin{matrix}  & e_{1}^{'} = e_{1} + 3e_{2} - 2e_{3} \\  & e_{2}^{'} = 2e_{1} + e_{2} - e_{3} \\  & e_{3}^{'} = e_{1} + e_{2} - 3e_{3} \end{matrix}\) \\
\textbf{C2.} Tómendegi kvadratlıq forma oń anıqlangan bolatuģın\(\lambda\) parametrnıń barlıq mánislerin tabıń: \(x_{1}\overline{x_{1}} + ix_{1}\overline{x_{2}} - ix_{2}\overline{x_{1}} + \lambda x_{2}\overline{x_{2}}\); \\
\textbf{C3.} Matricası tómendegishe bolǵan sızıqlı túrlendiriwdiń menshikli mánisi hám menshikli vektorların tabıń: \(\begin{pmatrix} 2 & - 1 & 2 \\ 5 & - 3 & 3 \\  - 1 & 0 & - 2 \end{pmatrix}\); \\

\end{tabular}
\vspace{1cm}


\begin{tabular}{m{17cm}}
\textbf{21-variant}
\newline

\textbf{T1.} Kompleks evklid keńislikleri.  (Kompleks vektorlı keńislik, Ermit kvadratlıq forma.) \\
\textbf{T2.} Sızıqlı túrlendiriwler.  (Sızıqlı túrlendiriw túsinigi, Sızıqlı túrlendiriwler ústinde ámeller, Sızıqlı túrlendiriwlerdiń obrazı hám yadrosı.) \\
\textbf{A1.} \(\mathbb{R}^{2}\) keńislikte anıqlangan tómendegi sáwlelendiriw skalyar kóbeyme bolatuģının anıqlań: \((x,y) = x_{1}y_{1} + 2x_{2}y_{1} + 2x_{1}y_{2} + 7x_{2}y_{2}\) \\
\textbf{A2.} Tómendegi vektorlar sisteması óz ara ortogonallıqqa tekseriń hám olardı ortogonallıq baziske shekem toltırın: \((1,2, - 1),(3, - 1,1)\); \\
\textbf{A3.} Matricası tómendegishe bolgan sızıqlı túrlendiriwdin menshikli mánisi hám menshikli vektorların tabıń: \(\begin{pmatrix} 2 & 1 \\ 1 & 2 \end{pmatrix}\); \\
\textbf{B1.} \(\mathbb{R}^{2}\) keńislikte \((x,y) = x_{1}y_{1} + 2x_{2}y_{1} + 2x_{1}y_{2} + 5x_{2}y_{2}\) berilgen skalyar kóbeyme ushın \(a = (1,0)\) hám \(b = (0,1)\) vektorlar arasındaǵı múyeshti tabıń \\
\textbf{B2.} Tómendegi kvadratlıq formanı kanonikalıq kóriniske keltiriń: \(12x_{1}^{2} + 3x_{2}^{2} + 12x_{3}^{2} - 12x_{1}x_{2} + 24x_{1}x_{3} - 8x_{2}x_{3}\); \\
\textbf{B3.} Tómendegi vektorlar sisteması óz ara ortogonallıqqa tekseriń hám olardı ortogonallıq baziske shekem toltırın: \((0,1,i),\ \ (1 + i,i,1)\); \\
\textbf{C1.} Jordan normal formasını tabıń \(\begin{pmatrix} 2 & - 1 & 2 \\ 0 & - 3 & 0 \\ 0 & 0 & 1 \end{pmatrix}\); \\
\textbf{C2.} Tómendegi kvadratlıq formalarga sáykes keliwshi ermit bisızıqlı formalardı tabiń:\((5 - i)x_{1}\overline{x_{2}} + (5 + i)\overline{x_{1}}x_{2} + x_{2}\overline{x_{2}}\); \\
\textbf{C3.} Matricası tómendegishe bolǵan sızıqlı túrlendiriwdiń menshikli mánisi hám menshikli vektorların tabıń: \(\begin{pmatrix} 1 & 0 & 0 & 0 \\ 0 & 0 & 0 & 0 \\ 0 & 0 & 0 & 0 \\ 1 & 0 & 0 & 0 \end{pmatrix}\); \\

\end{tabular}
\vspace{1cm}


\begin{tabular}{m{17cm}}
\textbf{22-variant}
\newline

\textbf{T1.} İnertsiya nızamı. (invariantlar,  eki kvadratılıq forma arasindaǵı baylanıs ) \\
\textbf{T2.} Sızıqlı túrlendiriwler matritsasınıń Jordan normal kórinisi. (Jordan kletkasınıń xarakteristikalıq matricası, Jordan matricasınıń uqsaslıǵı haqqında teorema,  Matricalardı jordan normal kórinisine keltiriw) \\
\textbf{A1.} \(\mathbb{R}^{2}\) keńislikte anıqlangan tómendegi sáwlelendiriw skalyar kóbeyme bolatuģının anıqlań: \((x,y) = x_{1}y_{1} - x_{2}y_{2}\) \\
\textbf{A2.} Tómendegi kvadratlıq forma oń anıqlangan bolatuģın\(\lambda\) nıń barlıq mánislerin tabıń: \(x_{1}^{2} + \lambda x_{2}^{2} + x_{3}^{2} - 4x_{1}x_{2} - 8x_{2}x_{3}\); \\
\textbf{A3.} Tómendegi sáwlelendiriw\(V = \mathbb{R}^{3}\) keńislikte sızıqlı túrlendiriw boladı: \(A\left( x_{1},x_{2},x_{3} \right) = \left( x_{1} + 2,x_{2} + 5,x_{3} \right)\); \\
\textbf{B1.} Ortogonallastırıw procesinen paydalanıp, berilgen vektorlar sistemasini ortogonallastirıń: \((2,0,1,1)\), ( \(1,2,0,1\) ), ( \(0,1, - 2,0\) ); \\
\textbf{B2.} Tómendegi kvadratlıq formanıń kanonikalıq kórinisin hám bul túrge keltiriwshi menshikli mánislerin tabıń: \(5x_{1}^{2} + 6x_{2}^{2} + 4x_{3}^{2} - 4x_{1}x_{2} - 4x_{1}x_{3}\); \\
\textbf{B3.} Matricası tómendegishe bolǵan sızıqlı túrlendiriwdiń menshikli mánisi hám menshikli vektorların tabıń: \(\begin{pmatrix} 4 & - 5 & 2 \\ 0 & - 7 & 3 \\ 0 & 0 & 4 \end{pmatrix}\); \\
\textbf{C1.} Berilgen \(A\) bisızıqlı formanıń\(e_{1},e_{2},e_{3}\) bazisdegi matritsası hám\(e_{1}^{'},e_{2}^{'},e_{3}^{'}\) baziske ótiw formulaları berilgen bolsa, onda bul bisızıqli formanıń \(e_{1}^{'},e_{2}^{'},e_{3}^{'}\) bazisdegi matritsasini tabıń: \(\begin{pmatrix} 1 & 1 & 2 \\  - 1 & 2 & 1 \\  - 1 & 1 & - 1 \end{pmatrix},\begin{matrix}  & e_{1}^{'} = e_{1} + e_{2} - 2e_{3} \\  & e_{2}^{'} = e_{1} + e_{2} + 2e_{3} \\  & e_{3}^{'} = e_{2} + e_{3} \end{matrix}\) \\
\textbf{C2.} Tómendegi kvadratlıq forma oń anıqlangan bolatuģın\(\lambda\) parametrnıń barlıq mánislerin tabıń: \(\lambda x_{1}\overline{x_{1}} - ix_{1}\overline{x_{2}} + ix_{2}\overline{x_{1}} + 3x_{2}\overline{x_{2}}\); \\
\textbf{C3.} Matricası tómendegishe bolǵan sızıqlı túrlendiriwdiń menshikli mánisi hám menshikli vektorların tabıń: \(\begin{pmatrix} 1 & - 3 & 4 \\ 4 & - 7 & 8 \\ 6 & - 7 & 7 \end{pmatrix}\); \\

\end{tabular}
\vspace{1cm}


\begin{tabular}{m{17cm}}
\textbf{23-variant}
\newline

\textbf{T1.} Kompleks evklid keńislikleri.  (Kompleks vektorlı keńislik, Ermit kvadratlıq forma.) \\
\textbf{T2.} Unitar túrlendiriwler. (Unitar sızıqlı túrlendiriwler túsinigi,  Unitar túrlendiriwge túyinles túrlendiriwlerdiń matricası,   Ortonormal baziste unitar túrlendiriwlerdiń matricası) \\
\textbf{A1.} Tómendegi kvadratlıq formanıń rangni anıqlań: \(x_{1}^{2} + x_{2}^{2} + x_{3}^{2} + x_{4}^{2} + 2x_{1}x_{2} - 2x_{1}x_{4} - 2x_{2}x_{3} + 2x_{3}x_{4}\). \\
\textbf{A2.} Tómendegi vektorlar sisteması óz ara ortogonallıqqa tekseriń hám olardı ortogonallıq baziske shekem toltırın: \((1,1,1,2)\), \((1,2,3, - 3)\). \\
\textbf{A3.} Matricası tómendegishe bolgan sızıqlı túrlendiriwdin menshikli mánisi hám menshikli vektorların tabıń: \(\begin{pmatrix} 2 & 1 \\ 1 & 2 \end{pmatrix}\); \\
\textbf{B1.} Ortogonallastırıw procesinen paydalanıp, berilgen vektorlar sistemasini ortogonallastirıń: \((1,1, - 1, - 2)\), \((5,8, - 2, - 3)\), (3, 9, 3, 8); \\
\textbf{B2.} Eger \(f\) sızıqlı funkciya\(e_{1},e_{2},e_{3}\) bazisde \(f(x) = 2x_{1} - 3x_{2} + x_{3}\) arqalı anıqlanǵan bolsa, onıń \(e_{1}^{'},e_{2}^{'},e_{3}^{'}\) bazisdegi kórinisin tabıń\(e_{1}^{'} = 4e_{1} - e_{2} - 3e_{3},\ e_{2}^{'} = 2e_{1} + e_{2},\ e_{3}^{'} = 3e_{1} + 2e_{2}\). \\
\textbf{B3.} Tómendegi vektorlar sisteması óz ara ortogonallıqqa tekseriń hám olardı ortogonallıq baziske shekem toltırın: \((0,i,1,1),\ \ (1,2,1 + i, - 1 + i)\). \\
\textbf{C1.} Berilgen \(A\) bisızıqlı formanıń\(e_{1},e_{2},e_{3}\) bazisdegi matritsası hám\(e_{1}^{'},e_{2}^{'},e_{3}^{'}\) baziske ótiw formulaları berilgen bolsa, onda bul bisızıqli formanıń\(e_{1}^{'},e_{2}^{'},e_{3}^{'}\) bazisdegi matritsasini tabıń: \(\begin{pmatrix} 1 & 2 & 3 \\ 4 & 5 & 6 \\ 7 & 8 & 9 \end{pmatrix}\), \(e_{1}^{'} = e_{1} - e_{2}\), \(e_{2}^{'} = e_{1} + e_{3}\), \(e_{3}^{'} = e_{1} + e_{2} + e_{3}\) \\
\textbf{C2.} Bazi bir ortonormal bazisde berilgen kvadratlıq formani kanonik kóriniske keltiriwshi ortonormal bazisin tabıń: \(x_{1}^{2} - 5x_{2}^{2} + x_{3}^{2} + 4x_{1}x_{2} + 2x_{1}x_{3} + 4x_{2}x_{3}\); \\
\textbf{C3.} Matricası tómendegishe bolǵan sızıqlı túrlendiriwdiń menshikli mánisi hám menshikli vektorların tabıń: \(\begin{pmatrix} 0 & 2 & 1 \\  - 2 & 0 & 3 \\  - 1 & - 3 & 0 \end{pmatrix}\); \\

\end{tabular}
\vspace{1cm}


\begin{tabular}{m{17cm}}
\textbf{24-variant}
\newline

\textbf{T1.} Sızıqlı keńislikler.   (Vektor,  sızıqlı baylanıs, bazis, ólshem, )  \\
\textbf{T2.} Óz-ara orın almasıwshı túrlendiriwler. (Óz-ara orın almasıwshı túrlendiriwler,  Ortogonal bazis haqqında teorema,  Normal túrlendiriwlerdiń kanonikalıq kórinisi) \\
\textbf{A1.} Tómendegi kvadratlıq formanıń rangni anıqlań: \(2x_{1}^{2} + 3x_{2}^{2} + 4x_{3}^{2} - 2x_{1}x_{2} + 4x_{1}x_{3} - 3x_{2}x_{3}\) \\
\textbf{A2.} Tómendegi kvadratlıq forma oń anıqlangan bolatuģın\(\lambda\) nıń barlıq mánislerin tabıń: \(5x_{1}^{2} + x_{2}^{2} + \lambda x_{3}^{2} + 4x_{1}x_{2} - 2x_{1}x_{3} - 2x_{2}x_{3}\); \\
\textbf{A3.} Tómendegi sáwlelendiriw\(V = \mathbb{R}^{3}\) keńislikte sızıqlı túrlendiriw boladı: \(A\left( x_{1},x_{2},x_{3} \right) = \left( x_{1},x_{2} + 1,x_{3} + 2 \right)\); \\
\textbf{B1.} Ortogonallastırıw procesinen paydalanıp, berilgen vektorlar sistemasini ortogonallastirıń: \((1,1, - 1)\), (1, 1,1 ), \((3,2, - 1)\); \\
\textbf{B2.} Eger \(f\) sızıqlı funkciya\(e_{1},e_{2},e_{3}\) bazisde \(f(x) = 2x_{1} - 3x_{2} + x_{3}\) arqalı anıqlanǵan bolsa, onıń \(e_{1}^{'},e_{2}^{'},e_{3}^{'}\) bazisdegi kórinisin tabıń\(e_{1}^{'} = e_{1} + 3e_{2} - 2e_{3},\ e_{2}^{'} = 2e_{1} + e_{2} - e_{3},\ e_{3}^{'} = e_{1} + e_{2} - 3e_{3}\); \\
\textbf{B3.} Tómendegi vektorlar sisteması óz ara ortogonallıqqa tekseriń hám olardı ortogonallıq baziske shekem toltırın: \((1,2,0, - 1),(3, - 1,1,1),( - 1,2,2,3)\); \\
\textbf{C1.} Jordan narmal formasini tabıń\(\begin{pmatrix} 4 & - 5 & 2 \\ 0 & - 7 & 3 \\ 0 & 0 & 4 \end{pmatrix}\); \\
\textbf{C2.} Tómendegi kvadratlıq formalarga sáykes keliwshi ermit bisızıqlı formalardı tabiń. \(x_{1}\overline{x_{1}} - ix_{1}\overline{x_{2}} - ix_{2}\overline{x_{1}} + 2x_{2}\overline{x_{2}}\); \\
\textbf{C3.} Matricası tómendegishe bolǵan sızıqlı túrlendiriwdiń menshikli mánisi hám menshikli vektorların tabıń: \(\begin{pmatrix} 1 & 0 & 0 & 0 \\ 0 & 0 & 0 & 0 \\ 1 & 0 & 0 & 0 \\ 0 & 0 & 0 & 1 \end{pmatrix}\); \\

\end{tabular}
\vspace{1cm}


\begin{tabular}{m{17cm}}
\textbf{25-variant}
\newline

\textbf{T1.} Kvadratlıq forma. (Lagran usulı, Yakobi usulı kvadratlıq formanı keltiriw.) \\
\textbf{T2.} Óz-ara orın almasıwshı túrlendiriwler. (Óz-ara orın almasıwshı túrlendiriwler,  Ortogonal bazis haqqında teorema,  Normal túrlendiriwlerdiń kanonikalıq kórinisi) \\
\textbf{A1.} \(\mathbb{R}^{2}\) keńislikte anıqlangan tómendegi sáwlelendiriw skalyar kóbeyme bolatuģının anıqlań: \((x,y) = x_{1}y_{1} + 2x_{2}y_{1} + 2x_{1}y_{2} + 7x_{2}y_{2}\) \\
\textbf{A2.} Tómendegi kvadratlıq forma oń anıqlangan bolatuģın\(\lambda\) nıń barlıq mánislerin tabıń: \(2x_{1}^{2} + x_{2}^{2} + 3x_{3}^{2} + 2\lambda x_{1}x_{2} + 2x_{1}x_{3}\); \\
\textbf{A3.} Tómendegi sáwlelendiriwmos ravishda Berilgen \(V\) vektor keńislikte sızıqlı túrlendiriw boladı: \(V\) sızıqlı keńislik,\(Ax = \alpha x\) bul jerde \(\alpha\)-fiksirlangan son; \\
\textbf{B1.} \(\mathbb{R}^{3}\) keńislikte \((x,y) = x_{1}y_{1} + 3x_{2}y_{2} + 2x_{3}y_{3}\) berilgen skalyar kóbeyme ushın \(a = (1, - 3,2)\) va \(b = (2,1, - 1)\) \(b = (0,1)\) vektorlar arasındaǵı múyeshti tabıń . \\
\textbf{B2.} Tómendegi kvadratlıq formanıń kanonikalıq kórinisin hám bul túrge keltiriwshi menshikli mánislerin tabıń: \(x_{1}^{2} - 2x_{2}^{2} - 2x_{3}^{2} - 4x_{1}x_{2} - 4x_{1}x_{3} + 8x_{2}x_{3}\); \\
\textbf{B3.} Matricası tómendegishe bolǵan sızıqlı túrlendiriwdiń menshikli mánisi hám menshikli vektorların tabıń: \(\begin{pmatrix} 0 & 0 & 1 \\ 1 & 4 & 0 \\  - 2 & 0 & 2 \end{pmatrix}\); \\
\textbf{C1.} Berilgen \(A\) bisızıqlı formanıń\(e_{1},e_{2},e_{3}\) bazisdegi matritsası hám\(e_{1}^{'},e_{2}^{'},e_{3}^{'}\) baziske ótiw formulaları berilgen bolsa, onda bul bisızıqli formanıń \(e_{1}^{'},e_{2}^{'},e_{3}^{'}\) bazisdegi matritsasini tabıń: \(\begin{pmatrix} 2 & 2 & 3 \\  - 4 & 3 & 1 \\ 3 & 1 & 2 \end{pmatrix},\ \begin{matrix}  & e_{1}^{'} = e_{1} + 3e_{2} - 2e_{3} \\  & e_{2}^{'} = 2e_{1} + e_{2} - e_{3} \\  & e_{3}^{'} = e_{1} + e_{2} - 3e_{3} \end{matrix}\) \\
\textbf{C2.} Bazi bir ortonormal bazisde berilgen kvadratlıq formani kanonik kóriniske keltiriwshi ortonormal bazisin tabıń: \(11x_{1}^{2} + 5x_{2}^{2} + 2x_{3}^{2} + 16x_{1}x_{2} + 4x_{1}x_{3} - 20x_{2}x_{3}\); \\
\textbf{C3.} Matricası tómendegishe bolǵan sızıqlı túrlendiriwdiń menshikli mánisi hám menshikli vektorların tabıń: \(\begin{pmatrix} 3 & - 1 & 0 & 0 \\ 1 & 1 & 0 & 0 \\ 3 & 0 & 5 & - 3 \\ 4 & - 1 & 3 & - 1 \end{pmatrix}\); \\

\end{tabular}
\vspace{1cm}


\begin{tabular}{m{17cm}}
\textbf{26-variant}
\newline

\textbf{T1.} Ortogonal  tolıqtırıwshı. (Ortogonal tolıqtırıwshı,  ortogonal proekciya) \\
\textbf{T2.} Unitar túrlendiriwler. (Unitar sızıqlı túrlendiriwler túsinigi,  Unitar túrlendiriwge túyinles túrlendiriwlerdiń matricası,   Ortonormal baziste unitar túrlendiriwlerdiń matricası) \\
\textbf{A1.} Tómendegi kvadratlıq formanıń rangni anıqlań: \(x_{1}x_{2} + x_{2}x_{3} + x_{3}x_{4} + x_{1}x_{4}\); \\
\textbf{A2.} Tómendegi vektorlar sisteması óz ara ortogonallıqqa tekseriń hám olardı ortogonallıq baziske shekem toltırın: \((1, - 2,2, - 3),(2, - 3,2,4)\); \\
\textbf{A3.} Tómendegi sáwlelendiriwlerden qaysıları sáykes túrde berilgen vektor keńislikte sızıqlı túrlendiriw boladı: sızıqlı keńislik, bul jerde -fiksirlengen san; \\
\textbf{B1.} Ortogonallastırıw procesinen paydalanıp, berilgen vektorlar sistemasini ortogonallastirıń: \((1,1, - 1,0)\), \((2,0, - 1,0)\), \((1, - 1,1, - 1)\), (2, \(0,1,1\) ). \\
\textbf{B2.} Tómendegi kvadratlıq formanıń kanonikalıq kórinisin hám bul túrge keltiriwshi menshikli mánislerin tabıń: \(x_{1}x_{2} + x_{1}x_{3} + x_{2}x_{3}\); \\
\textbf{B3.} Tómendegi vektorlar sisteması óz ara ortogonallıqqa tekseriń hám olardı ortogonallıq baziske shekem toltırın: \((1,1,1,1),(1,1, - 1, - 1),(1, - 1,1, - 1)\); \\
\textbf{C1.} Jordan normal formasını tabıń \(\begin{pmatrix} 1 & - 2 & 1 \\  - 2 & 1 & 4 \\  - 1 & 4 & 1 \end{pmatrix}\). \\
\textbf{C2.} Tómendegi bisiziqli formalar ekvivalent emes ekenligin dálilleń:\(f_{1}(x,y) = x_{1}y_{1} + 2x_{1}y_{2} + 2x_{2}y_{1} + 5x_{2}y_{2} + 6x_{2}y_{3} + 8x_{3}y_{2} + 10x_{3}y_{3}\), \(f_{2}(x,y) = 2x_{1}y_{1} - x_{1}y_{3} + x_{2}y_{2} - x_{3}y_{1} + 5x_{3}y_{3}\). \\
\textbf{C3.} Ortogonallastırıw procesinen paydalanıp, berilgen vektorlar sistemasini ortogonallastirıń: \((2,1,3, - 1)\), ( \(7,4,3, - 3\) ), ( \(1,1, - 6,0\) ), (5, 7, 7, 8); \\

\end{tabular}
\vspace{1cm}


\begin{tabular}{m{17cm}}
\textbf{27-variant}
\newline

\textbf{T1.} Evklid keńisligi. (Skalyar kóbeyme, ortogonal vektorlar, ortonormal bazis.) \\
\textbf{T2.} Sızıqlı túrlendiriwler.  (Sızıqlı túrlendiriw túsinigi, Sızıqlı túrlendiriwler ústinde ámeller, Sızıqlı túrlendiriwlerdiń obrazı hám yadrosı.) \\
\textbf{A1.} Tómendegi kvadratlıq formanıń rangni anıqlań: \(3x_{1}^{2} - 2x_{2}^{2} + 2x_{3}^{2} + 4x_{1}x_{2} - 3x_{1}x_{3} - x_{2}x_{3}\); \\
\textbf{A2.} Tómendegi kvadratlıq forma oń anıqlangan bolatuģın\(\lambda\) nıń barlıq mánislerin tabıń: \(5x_{1}^{2} + x_{2}^{2} + \lambda x_{3}^{2} + 4x_{1}x_{2} - 2x_{1}x_{3} - 2x_{2}x_{3}\); \\
\textbf{A3.} Tómendegi sáwlelendiriwlerden qaysıları keńislikte sızıqlı túrlendiriw boladı:; \\
\textbf{B1.} Ortogonallastırıw procesinen paydalanıp, berilgen vektorlar sistemasini ortogonallastirıń: \((1,1,0,0)\), (1, 0, 1, 1); \\
\textbf{B2.} Tómendegi kvadratlıq formanıń kanonikalıq kórinisin hám bul túrge keltiriwshi menshikli mánislerin tabıń: \(x_{1}^{2} - 2x_{2}^{2} - 2x_{3}^{2} - 4x_{1}x_{2} - 4x_{1}x_{3} + 8x_{2}x_{3}\); \\
\textbf{B3.} Matricası tómendegishe bolǵan sızıqlı túrlendiriwdiń menshikli mánisi hám menshikli vektorların tabıń: \(\begin{pmatrix} 2 & - 1 & 2 \\ 0 & - 3 & 0 \\ 0 & 0 & 1 \end{pmatrix}\); \\
\textbf{C1.} Jordan normal formasını tabıń \(\begin{pmatrix} 7 & 0 & 0 \\ 10 & - 19 & 0 \\ 12 & - 24 & 13 \end{pmatrix}\); \\
\textbf{C2.} Tómendegi kvadratlıq forma oń anıqlangan bolatuģın\(\lambda\) parametrnıń barlıq mánislerin tabıń: \(\lambda x_{1}\overline{x_{1}} - ix_{1}\overline{x_{2}} + ix_{2}\overline{x_{1}} + 3x_{2}\overline{x_{2}}\); \\
\textbf{C3.} Ortogonallastırıw procesinen paydalanıp, berilgen vektorlar sistemasini ortogonallastirıń: \((1,1, - 1,0)\), \((2,0, - 1,0)\), \((1, - 1,1, - 1)\), (2, \(0,1,1\) ). \\

\end{tabular}
\vspace{1cm}


\begin{tabular}{m{17cm}}
\textbf{28-variant}
\newline

\textbf{T1.} Sızıqlı, bisızıqlı hám kvadratlıq formalar. (Bisızıqlı forma,  simmetriyalı bisızıqlı formalar)  \\
\textbf{T2.} Keri túrlendiriwler. ( Keri túrlendiriw túsinigi,   Keri túrlendiriwdiń sızıqlılıǵı) \\
\textbf{A1.} Tómendegi kvadratlıq formanıń rangni anıqlań: \(2x_{1}^{2} + 3x_{2}^{2} + 4x_{3}^{2} - 2x_{1}x_{2} + 4x_{1}x_{3} - 3x_{2}x_{3}\) \\
\textbf{A2.} Tómendegi kvadratlıq forma oń anıqlangan bolatuģın\(\lambda\) nıń barlıq mánislerin tabıń: \(x_{1}^{2} + x_{2}^{2} + 5x_{3}^{2} + 2\lambda x_{1}x_{2} - 2x_{1}x_{3} + 4x_{2}x_{3}\); \\
\textbf{A3.} Tómendegi sáwlelendiriwlerden qaysıları keńislikte sızıqlı túrlendiriw boladı: 1. \\
\textbf{B1.} Ortogonallastırıw procesinen paydalanıp, berilgen vektorlar sistemasini ortogonallastirıń: \((1,2,1,3)\), (4, 1, 1, 1), (3, 1, 1, 0); \\
\textbf{B2.} Tómendegi kvadratlıq formanıń kanonikalıq kórinisin hám bul túrge keltiriwshi menshikli mánislerin tabıń: \(5x_{1}^{2} + 6x_{2}^{2} + 4x_{3}^{2} - 4x_{1}x_{2} - 4x_{1}x_{3}\); \\
\textbf{B3.} Tómendegi vektorlar sisteması óz ara ortogonallıqqa tekseriń hám olardı ortogonallıq baziske shekem toltırın. \((i,i,1, - 1),\ \ (1, - 1 + i,0,1),\ \ \); \\
\textbf{C1.} Berilgen \(A\) bisızıqlı formanıń\(e_{1},e_{2},e_{3}\) bazisdegi matritsası hám\(e_{1}^{'},e_{2}^{'},e_{3}^{'}\) baziske ótiw formulaları berilgen bolsa, onda bul bisızıqli formanıń\(e_{1}^{'},e_{2}^{'},e_{3}^{'}\) bazisdegi matritsasini tabıń: \(\ \) \(\begin{pmatrix} 0 & 2 & 1 \\  - 2 & 2 & 0 \\  - 1 & 0 & 3 \end{pmatrix}\), \(e_{1}^{'} = e_{1} + 2e_{2} - e_{3}\), \(e_{2}^{'} = e_{2} - e_{3}\), \(e_{3}^{'} = - e_{1} + e_{2} - 3e_{3}\) \\
\textbf{C2.} Tómendegi kvadratlıq forma oń anıqlangan bolatuģın\(\lambda\) parametrnıń barlıq mánislerin tabıń: \(x_{1}\overline{x_{1}} + ix_{1}\overline{x_{2}} - ix_{2}\overline{x_{1}} + \lambda x_{2}\overline{x_{2}}\); \\
\textbf{C3.} Matricası tómendegishe bolǵan sızıqlı túrlendiriwdiń menshikli mánisi hám menshikli vektorların tabıń: \(\begin{pmatrix} 1 & 1 & 1 & 1 \\ 1 & 1 & - 1 & - 1 \\ 1 & - 1 & 1 & - 1 \\ 1 & - 1 & - 1 & 1 \end{pmatrix}\). \\

\end{tabular}
\vspace{1cm}


\begin{tabular}{m{17cm}}
\textbf{29-variant}
\newline

\textbf{T1.} İnertsiya nızamı. (invariantlar,  eki kvadratılıq forma arasindaǵı baylanıs ) \\
\textbf{T2.} Sızıqlı túrlendiriwler matritsasınıń Jordan normal kórinisi. (Jordan kletkasınıń xarakteristikalıq matricası, Jordan matricasınıń uqsaslıǵı haqqında teorema,  Matricalardı jordan normal kórinisine keltiriw) \\
\textbf{A1.} Tómendegi kvadratlıq formanıń rangni anıqlań: \(x_{1}x_{2} + x_{1}x_{3} + x_{2}x_{3}\); \\
\textbf{A2.} Tómendegi funkciyalı haqiqiy sanlar maydanı ústinde anıqlangan \(V\) keńislikte sızıqlı funkciya boladı: \(V = \mathbb{R}^{3},\ \ f(x) = |x|\); \\
\textbf{A3.} Tómendegi sáwlelendiriwmos ravishda Berilgen \(V\) vektor keńislikte sızıqlı túrlendiriw boladı: \(V\) sızıqlı keńislik,\(Ax = x + a\), bul jerde \(a\)-fiksirlengen vektor; \\
\textbf{B1.} Ortogonallastırıw procesinen paydalanıp, berilgen vektorlar sistemasini ortogonallastirıń: \((1,0,0)\), (0, 1, -1), (1, 1, 1); \\
\textbf{B2.} Tómendegi kvadratlıq formanı kanonikalıq kóriniske keltiriń: \(2x_{1}^{2} + 18x_{2}^{2} + 8x_{3}^{2} - 12x_{1}x_{2} + 8x_{1}x_{3} - 27x_{2}x_{3}\); \\
\textbf{B3.} Matricası tómendegishe bolǵan sızıqlı túrlendiriwdiń menshikli mánisi hám menshikli vektorların tabıń: \(\begin{pmatrix} 7 & 0 & 0 \\ 10 & - 19 & 0 \\ 12 & - 24 & 13 \end{pmatrix}\); \\
\textbf{C1.} Jordan normal formasını tabıń \(\begin{pmatrix} 0 & 0 & 1 \\ 1 & 4 & 0 \\  - 2 & 0 & 2 \end{pmatrix}\); \\
\textbf{C2.} Bazi bir ortonormal bazisde berilgen kvadratlıq formani kanonik kóriniske keltiriwshi ortonormal bazisin tabıń: \(x_{1}^{2} - 5x_{2}^{2} + x_{3}^{2} + 4x_{1}x_{2} + 2x_{1}x_{3} + 4x_{2}x_{3}\); \\
\textbf{C3.} Matricası tómendegishe bolǵan sızıqlı túrlendiriwdiń menshikli mánisi hám menshikli vektorların tabıń: \(\begin{pmatrix} 5 & 6 & - 3 \\  - 1 & 0 & 1 \\ 1 & 2 & - 1 \end{pmatrix}\); \\

\end{tabular}
\vspace{1cm}


\begin{tabular}{m{17cm}}
\textbf{30-variant}
\newline

\textbf{T1.} Kvadratlıq forma. (Lagran usulı, Yakobi usulı kvadratlıq formanı keltiriw.) \\
\textbf{T2.} Túyinles túrlendiriw. ( Evklid keńisligindegi sızıqlı túrlendiriwler menen bisızıqlı formalar arasındaǵı baylanıs, Berilgen túrlendiriwge túyinles túrlendiriwler, Óz-ózine túyinles túrlendiriwler) \\
\textbf{A1.} \(\mathbb{R}^{2}\) keńislikte anıqlangan tómendegi sáwlelendiriw skalyar kóbeyme bolatuģının anıqlań: \((x,y) = x_{1}y_{1} - 2x_{2}y_{1} - 2x_{1}y_{2} + x_{2}y_{2}\) \\
\textbf{A2.} Tómendegi kvadratlıq forma oń anıqlangan bolatuģın\(\lambda\) nıń barlıq mánislerin tabıń: \(x_{1}^{2} + \lambda x_{2}^{2} + x_{3}^{2} - 4x_{1}x_{2} - 8x_{2}x_{3}\); \\
\textbf{A3.} Tómendegi sáwlelendiriwlerden qaysıları keńislikte sızıqlı túrlendiriw boladı:. \\
\textbf{B1.} Ortogonallastırıw procesinen paydalanıp, berilgen vektorlar sistemasini ortogonallastirıń: \((1,2,2, - 1)\), ( \(1,1, - 5,3\) ), (3, 2, 8, -7); \\
\textbf{B2.} Tómendegi kvadratlıq formanı kanonikalıq kóriniske keltiriń: \(x_{1}x_{2} + x_{1}x_{3} + x_{1}x_{4} + x_{2}x_{3} + x_{2}x_{4} + x_{3}x_{4}\); \\
\textbf{B3.} Tómendegi vektorlar sisteması óz ara ortogonallıqqa tekseriń hám olardı ortogonallıq baziske shekem toltırın: \((1,i, - i),\ \ ( - 2 - i,1 + i,2 - i)\); \\
\textbf{C1.} Jordan normal formasını tabıń \(\begin{pmatrix} 4 & 1 & - 4 \\ 1 & 4 & 0 \\  - 4 & 0 & 4 \end{pmatrix}\); \\
\textbf{C2.} Tómendegi kvadratlıq formalarga sáykes keliwshi ermit bisızıqlı formalardı tabiń. \(x_{1}\overline{x_{1}} - ix_{1}\overline{x_{2}} - ix_{2}\overline{x_{1}} + 2x_{2}\overline{x_{2}}\); \\
\textbf{C3.} Matricası tómendegishe bolǵan sızıqlı túrlendiriwdiń menshikli mánisi hám menshikli vektorların tabıń: \(\begin{pmatrix} 2 & - 1 & 2 \\ 5 & - 3 & 3 \\  - 1 & 0 & - 2 \end{pmatrix}\); \\

\end{tabular}
\vspace{1cm}


\begin{tabular}{m{17cm}}
\textbf{31-variant}
\newline

\textbf{T1.} Sızıqlı, bisızıqlı hám kvadratlıq formalar. (Bisızıqlı forma,  simmetriyalı bisızıqlı formalar)  \\
\textbf{T2.} Keri túrlendiriwler. ( Keri túrlendiriw túsinigi,   Keri túrlendiriwdiń sızıqlılıǵı) \\
\textbf{A1.} \(\mathbb{R}^{2}\) keńislikte anıqlangan tómendegi sáwlelendiriw skalyar kóbeyme bolatuģının anıqlań: \((x,y) = x_{1}y_{1} - x_{2}y_{2}\) \\
\textbf{A2.} Tómendegi kvadratlıq forma oń anıqlangan bolatuģın\(\lambda\) nıń barlıq mánislerin tabıń: \(x_{1}^{2} + 4x_{2}^{2} + x_{3}^{2} + 2\lambda x_{1}x_{2} + 10x_{1}x_{3} + 6x_{2}x_{3}\); \\
\textbf{A3.} Tómendegi sáwlelendiriwlerden qaysıları keńislikte sızıqlı túrlendiriw boladı:; \\
\textbf{B1.} Ortogonallastırıw procesinen paydalanıp, berilgen vektorlar sistemasini ortogonallastirıń: \((1,2,1,3)\), (4, 1, 1, 1), (3, 1, 1, 0); \\
\textbf{B2.} Tómendegi kvadratlıq formanıń kanonikalıq kórinisin hám bul túrge keltiriwshi menshikli mánislerin tabıń: \(x_{1}x_{2} + x_{1}x_{3} + x_{2}x_{3}\); \\
\textbf{B3.} Tómendegi vektorlar sisteması óz ara ortogonallıqqa tekseriń hám olardı ortogonallıq baziske shekem toltırın: \((1,2,0, - 1),(3, - 1,1,1),( - 1,2,2,3)\); \\
\textbf{C1.} Jordan normal formasını tabıń \(\begin{pmatrix} 4 & 1 & - 4 \\ 1 & 4 & 0 \\  - 4 & 0 & 4 \end{pmatrix}\); \\
\textbf{C2.} Tómendegi kvadratlıq formalarga sáykes keliwshi ermit bisızıqlı formalardı tabiń:\((5 - i)x_{1}\overline{x_{2}} + (5 + i)\overline{x_{1}}x_{2} + x_{2}\overline{x_{2}}\); \\
\textbf{C3.} Ortogonallastırıw procesinen paydalanıp, berilgen vektorlar sistemasini ortogonallastirıń: \((1,1, - 1,0)\), \((2,0, - 1,0)\), \((1, - 1,1, - 1)\), (2, \(0,1,1\) ). \\

\end{tabular}
\vspace{1cm}


\begin{tabular}{m{17cm}}
\textbf{32-variant}
\newline

\textbf{T1.} Sızıqlı keńislikler.   (Vektor,  sızıqlı baylanıs, bazis, ólshem, )  \\
\textbf{T2.} Túyinles túrlendiriw. ( Evklid keńisligindegi sızıqlı túrlendiriwler menen bisızıqlı formalar arasındaǵı baylanıs, Berilgen túrlendiriwge túyinles túrlendiriwler, Óz-ózine túyinles túrlendiriwler) \\
\textbf{A1.} \(\mathbb{R}^{2}\) keńislikte anıqlangan tómendegi sáwlelendiriw skalyar kóbeyme bolatuģının anıqlań: \((x,y) = x_{1}y_{1} - x_{2}y_{1} - x_{1}y_{2} + x_{2}y_{2}\) \\
\textbf{A2.} Tómendegi kvadratlıq forma oń anıqlangan bolatuģın\(\lambda\) nıń barlıq mánislerin tabıń: \(2x_{1}^{2} + 2x_{2}^{2} + x_{3}^{2} + 2\lambda x_{1}x_{2} + 6x_{1}x_{3} + 2x_{2}x_{3}\); \\
\textbf{A3.} Tómendegi ańlatpalardan qaysıları sáykes túrde berilgen vektor keńislikte sızıqlı túrlendiriw boladı: sızıqlı keńislik,, bul jerde -fiksirlengen vektor; \\
\textbf{B1.} Ortogonallastırıw procesinen paydalanıp, berilgen vektorlar sistemasini ortogonallastirıń: \((1,1, - 1)\), (1, 1,1 ), \((3,2, - 1)\); \\
\textbf{B2.} Eger \(f\) sızıqlı funkciya\(e_{1},e_{2},e_{3}\) bazisde \(f(x) = 2x_{1} - 3x_{2} + x_{3}\) arqalı anıqlanǵan bolsa, onıń \(e_{1}^{'},e_{2}^{'},e_{3}^{'}\) bazisdegi kórinisin tabıń\(e_{1}^{'} = e_{1} + e_{2} - 2e_{3},\ e_{2}^{'} = e_{1} + e_{2} + 2e_{3},\ e_{3}^{'} = e_{2} + e_{3}\); \\
\textbf{B3.} Matricası tómendegishe bolǵan sızıqlı túrlendiriwdiń menshikli mánisi hám menshikli vektorların tabıń: \(\begin{pmatrix} 0 & 0 & 1 \\ 1 & 4 & 0 \\  - 2 & 0 & 2 \end{pmatrix}\); \\
\textbf{C1.} Berilgen \(A\) bisızıqlı formanıń\(e_{1},e_{2},e_{3}\) bazisdegi matritsası hám\(e_{1}^{'},e_{2}^{'},e_{3}^{'}\) baziske ótiw formulaları berilgen bolsa, onda bul bisızıqli formanıń \(e_{1}^{'},e_{2}^{'},e_{3}^{'}\) bazisdegi matritsasini tabıń: \(\begin{pmatrix} 2 & 2 & 3 \\  - 4 & 3 & 1 \\ 3 & 1 & 2 \end{pmatrix},\ \begin{matrix}  & e_{1}^{'} = e_{1} + 3e_{2} - 2e_{3} \\  & e_{2}^{'} = 2e_{1} + e_{2} - e_{3} \\  & e_{3}^{'} = e_{1} + e_{2} - 3e_{3} \end{matrix}\) \\
\textbf{C2.} Bazi bir ortonormal bazisde berilgen kvadratlıq formani kanonik kóriniske keltiriwshi ortonormal bazisin tabıń: \(x_{1}^{2} + x_{2}^{2} + 5x_{3}^{2} - 6x_{1}x_{2} - 2x_{1}x_{3} + 2x_{2}x_{3}\); \\
\textbf{C3.} Matricası tómendegishe bolǵan sızıqlı túrlendiriwdiń menshikli mánisi hám menshikli vektorların tabıń: \(\begin{pmatrix} 1 & 1 & 1 & 1 \\ 1 & 1 & - 1 & - 1 \\ 1 & - 1 & 1 & - 1 \\ 1 & - 1 & - 1 & 1 \end{pmatrix}\). \\

\end{tabular}
\vspace{1cm}


\begin{tabular}{m{17cm}}
\textbf{33-variant}
\newline

\textbf{T1.} Ortogonal  tolıqtırıwshı. (Ortogonal tolıqtırıwshı,  ortogonal proekciya) \\
\textbf{T2.} Sızıqlı túrlendiriwler matritsasınıń Jordan normal kórinisi. (Jordan kletkasınıń xarakteristikalıq matricası, Jordan matricasınıń uqsaslıǵı haqqında teorema,  Matricalardı jordan normal kórinisine keltiriw) \\
\textbf{A1.} \(\mathbb{R}^{2}\) keńislikte anıqlangan tómendegi sáwlelendiriw skalyar kóbeyme bolatuģının anıqlań: \((x,y) = x_{1}y_{1} + 2x_{2}y_{2}\) \\
\textbf{A2.} Tómendegi vektorlar sisteması óz ara ortogonallıqqa tekseriń hám olardı ortogonallıq baziske shekem toltırın: \((1,2, - 1),(3, - 1,1)\); \\
\textbf{A3.} Tómendegi sáwlelendiriw\(V = \mathbb{R}^{3}\) keńislikte sızıqlı túrlendiriw boladı: \(A\left( x_{1},x_{2},x_{3} \right) = \left( x_{1},x_{2},x_{1} + x_{2} + x_{3} \right)\); \\
\textbf{B1.} Ortogonallastırıw procesinen paydalanıp, berilgen vektorlar sistemasini ortogonallastirıń: \((1,1, - 1, - 2)\), \((5,8, - 2, - 3)\), (3, 9, 3, 8); \\
\textbf{B2.} Tómendegi kvadratlıq formanıń kanonikalıq kórinisin hám bul túrge keltiriwshi menshikli mánislerin tabıń: \(3x_{1}^{2} - 2x_{2}^{2} + 2x_{3}^{2} + 4x_{1}x_{2} - 3x_{1}x_{3} - x_{2}x_{3}\); \\
\textbf{B3.} Tómendegi vektorlar sisteması óz ara ortogonallıqqa tekseriń hám olardı ortogonallıq baziske shekem toltırın: \((1,i, - i),\ \ ( - 2 - i,1 + i,2 - i)\); \\
\textbf{C1.} Jordan normal formasını tabıń \(\begin{pmatrix} 1 & - 2 & 1 \\  - 2 & 1 & 4 \\  - 1 & 4 & 1 \end{pmatrix}\). \\
\textbf{C2.} Bazi bir ortonormal bazisde berilgen kvadratlıq formani kanonik kóriniske keltiriwshi ortonormal bazisin tabıń: \(x_{1}^{2} + x_{2}^{2} + x_{3}^{2} + 4x_{1}x_{2} + 4x_{1}x_{3} + 4x_{2}x_{3}\); \\
\textbf{C3.} Matricası tómendegishe bolǵan sızıqlı túrlendiriwdiń menshikli mánisi hám menshikli vektorların tabıń: \(\begin{pmatrix} 3 & - 1 & 0 & 0 \\ 1 & 1 & 0 & 0 \\ 3 & 0 & 5 & - 3 \\ 4 & - 1 & 3 & - 1 \end{pmatrix}\); \\

\end{tabular}
\vspace{1cm}


\begin{tabular}{m{17cm}}
\textbf{34-variant}
\newline

\textbf{T1.} Evklid keńisligi. (Skalyar kóbeyme, ortogonal vektorlar, ortonormal bazis.) \\
\textbf{T2.} Óz-ara orın almasıwshı túrlendiriwler. (Óz-ara orın almasıwshı túrlendiriwler,  Ortogonal bazis haqqında teorema,  Normal túrlendiriwlerdiń kanonikalıq kórinisi) \\
\textbf{A1.} Tómendegi kvadratlıq formanıń rangni anıqlań: \(x_{1}^{2} + x_{2}^{2} + x_{3}^{2} + x_{4}^{2} + 2x_{1}x_{2} - 2x_{1}x_{4} - 2x_{2}x_{3} + 2x_{3}x_{4}\). \\
\textbf{A2.} Tómendegi vektorlar sisteması óz ara ortogonallıqqa tekseriń hám olardı ortogonallıq baziske shekem toltırın: \((2,1,2),\ (1,2, - 2)\); \\
\textbf{A3.} Matricası tómendegishe bolgan sızıqlı túrlendiriwdin menshikli mánisi hám menshikli vektorların tabıń \(\begin{pmatrix} 3 & 4 \\ 5 & 2 \end{pmatrix}\); \\
\textbf{B1.} Ortogonallastırıw procesinen paydalanıp, berilgen vektorlar sistemasini ortogonallastirıń: \((2,0,1,1)\), ( \(1,2,0,1\) ), ( \(0,1, - 2,0\) ); \\
\textbf{B2.} Eger \(f\) sızıqlı funkciya\(e_{1},e_{2},e_{3}\) bazisde \(f(x) = 2x_{1} - 3x_{2} + x_{3}\) arqalı anıqlanǵan bolsa, onıń \(e_{1}^{'},e_{2}^{'},e_{3}^{'}\) bazisdegi kórinisin tabıń\(e_{1}^{'} = 4e_{1} - e_{2} - 3e_{3},\ e_{2}^{'} = 2e_{1} + e_{2},\ e_{3}^{'} = 3e_{1} + 2e_{2}\). \\
\textbf{B3.} Matricası tómendegishe bolǵan sızıqlı túrlendiriwdiń menshikli mánisi hám menshikli vektorların tabıń: \(\begin{pmatrix} 2 & - 1 & 2 \\ 0 & - 3 & 0 \\ 0 & 0 & 1 \end{pmatrix}\); \\
\textbf{C1.} Berilgen \(A\) bisızıqlı formanıń\(e_{1},e_{2},e_{3}\) bazisdegi matritsası hám\(e_{1}^{'},e_{2}^{'},e_{3}^{'}\) baziske ótiw formulaları berilgen bolsa, onda bul bisızıqli formanıń\(e_{1}^{'},e_{2}^{'},e_{3}^{'}\) bazisdegi matritsasini tabıń: \(\ \) \(\begin{pmatrix} 0 & 2 & 1 \\  - 2 & 2 & 0 \\  - 1 & 0 & 3 \end{pmatrix}\), \(e_{1}^{'} = e_{1} + 2e_{2} - e_{3}\), \(e_{2}^{'} = e_{2} - e_{3}\), \(e_{3}^{'} = - e_{1} + e_{2} - 3e_{3}\) \\
\textbf{C2.} Tómendegi kvadratlıq formalarga sáykes keliwshi ermit bisızıqlı formalardı tabiń: \(x_{1}\overline{x_{1}} + (2 + i)x_{1}\overline{x_{2}} + (2 - i)x_{2}\overline{x_{1}} + ix_{1}\overline{x_{3}} - ix_{3}\overline{x_{1}} - x_{3}\overline{x_{3}}\); \\
\textbf{C3.} Matricası tómendegishe bolǵan sızıqlı túrlendiriwdiń menshikli mánisi hám menshikli vektorların tabıń: \(\begin{pmatrix} 1 & - 3 & 4 \\ 4 & - 7 & 8 \\ 6 & - 7 & 7 \end{pmatrix}\); \\

\end{tabular}
\vspace{1cm}


\begin{tabular}{m{17cm}}
\textbf{35-variant}
\newline

\textbf{T1.} Kompleks evklid keńislikleri.  (Kompleks vektorlı keńislik, Ermit kvadratlıq forma.) \\
\textbf{T2.} Unitar túrlendiriwler. (Unitar sızıqlı túrlendiriwler túsinigi,  Unitar túrlendiriwge túyinles túrlendiriwlerdiń matricası,   Ortonormal baziste unitar túrlendiriwlerdiń matricası) \\
\textbf{A1.} Tómendegi kvadratlıq formanıń rangni anıqlań: \(x_{1}^{2} - 2x_{2}^{2} - 2x_{3}^{2} - 4x_{1}x_{2} - 4x_{1}x_{3} + 8x_{2}x_{3}\); \\
\textbf{A2.} Tómendegi funkciyalı haqiqiy sanlar maydanı ústinde anıqlangan \(V\) keńislikte sızıqlı funkciya boladı: \(V = M_{n}\left( \mathbb{R} \right),\ \ f(A) = \det(A)\); \\
\textbf{A3.} Tómendegi sáwlelendiriwlerden qaysıları keńislikte sızıqlı túrlendiriw boladı:; \\
\textbf{B1.} Ortogonallastırıw procesinen paydalanıp, berilgen vektorlar sistemasini ortogonallastirıń: \((1,0,0)\), (0, 1, -1), (1, 1, 1); \\
\textbf{B2.} Tómendegi kvadratlıq formanıń kanonikalıq kórinisin hám bul túrge keltiriwshi menshikli mánislerin tabıń: \(7x_{1}^{2} + 5x_{2}^{2} + 3x_{3}^{2} - 8x_{1}x_{2} + 8x_{2}x_{3}\); \\
\textbf{B3.} Tómendegi vektorlar sisteması óz ara ortogonallıqqa tekseriń hám olardı ortogonallıq baziske shekem toltırın. \((i,i,1, - 1),\ \ (1, - 1 + i,0,1),\ \ \); \\
\textbf{C1.} Jordan normal formasını tabıń \(\begin{pmatrix} 2 & - 1 & 2 \\ 0 & - 3 & 0 \\ 0 & 0 & 1 \end{pmatrix}\); \\
\textbf{C2.} Tómendegi kvadratlıq formalarga sáykes keliwshi ermit bisızıqlı formalardı tabiń: \(ix_{1}\overline{x_{2}} - ix_{2}\overline{x_{1}} + (3 - 2i)x_{1}\overline{x_{3}} + (3 + 2i)x_{3}\overline{x_{1}} + 2x_{2}\overline{x_{3}} + 2x_{3}\overline{x_{2}}\). \\
\textbf{C3.} Matricası tómendegishe bolǵan sızıqlı túrlendiriwdiń menshikli mánisi hám menshikli vektorların tabıń: \(\begin{pmatrix} 0 & 2 & 1 \\  - 2 & 0 & 3 \\  - 1 & - 3 & 0 \end{pmatrix}\); \\

\end{tabular}
\vspace{1cm}


\begin{tabular}{m{17cm}}
\textbf{36-variant}
\newline

\textbf{T1.} Sızıqlı keńislikler.   (Vektor,  sızıqlı baylanıs, bazis, ólshem, )  \\
\textbf{T2.} Sızıqlı túrlendiriwler.  (Sızıqlı túrlendiriw túsinigi, Sızıqlı túrlendiriwler ústinde ámeller, Sızıqlı túrlendiriwlerdiń obrazı hám yadrosı.) \\
\textbf{A1.} \(\mathbb{R}^{2}\) keńislikte anıqlangan tómendegi sáwlelendiriw skalyar kóbeyme bolatuģının anıqlań: \((x,y) = x_{1}y_{1} + x_{2}y_{1} + 3x_{1}y_{2} + 2x_{2}y_{2}\) \\
\textbf{A2.} Tómendegi vektorlar sisteması óz ara ortogonallıqqa tekseriń hám olardı ortogonallıq baziske shekem toltırın: \((1,1,1,2)\), \((1,2,3, - 3)\). \\
\textbf{A3.} Tómendegi sáwlelendiriw\(V = \mathbb{R}^{3}\) keńislikte sızıqlı túrlendiriw boladı: \(A\left( x_{1},x_{2},x_{3} \right) = \left( x_{2} + x_{3},2x_{1} + x_{3},3x_{1} - x_{2} + x_{3} \right)\); \\
\textbf{B1.} \(\mathbb{R}^{3}\) keńislikte \((x,y) = x_{1}y_{1} + 3x_{2}y_{2} + 2x_{3}y_{3}\) berilgen skalyar kóbeyme ushın \(a = (1, - 3,2)\) va \(b = (2,1, - 1)\) \(b = (0,1)\) vektorlar arasındaǵı múyeshti tabıń . \\
\textbf{B2.} Eger \(f\) sızıqlı funkciya\(e_{1},e_{2},e_{3}\) bazisde \(f(x) = 2x_{1} - 3x_{2} + x_{3}\) arqalı anıqlanǵan bolsa, onıń \(e_{1}^{'},e_{2}^{'},e_{3}^{'}\) bazisdegi kórinisin tabıń\(e_{1}^{'} = e_{1} - e_{2},\ e_{2}^{'} = e_{1} + e_{3},\ \ e_{3}^{'} = e_{1} + e_{2} + e_{3}\); \\
\textbf{B3.} Tómendegi vektorlar sisteması óz ara ortogonallıqqa tekseriń hám olardı ortogonallıq baziske shekem toltırın: \((0,i,1,1),\ \ (1,2,1 + i, - 1 + i)\). \\
\textbf{C1.} Berilgen \(A\) bisızıqlı formanıń\(e_{1},e_{2},e_{3}\) bazisdegi matritsası hám\(e_{1}^{'},e_{2}^{'},e_{3}^{'}\) baziske ótiw formulaları berilgen bolsa, onda bul bisızıqli formanıń \(e_{1}^{'},e_{2}^{'},e_{3}^{'}\) bazisdegi matritsasini tabıń: \(\begin{pmatrix} 1 & 1 & 2 \\  - 1 & 2 & 1 \\  - 1 & 1 & - 1 \end{pmatrix},\begin{matrix}  & e_{1}^{'} = e_{1} + e_{2} - 2e_{3} \\  & e_{2}^{'} = e_{1} + e_{2} + 2e_{3} \\  & e_{3}^{'} = e_{2} + e_{3} \end{matrix}\) \\
\textbf{C2.} Tómendegi bisiziqli formalar ekvivalent emes ekenligin dálilleń:\(f_{1}(x,y) = 2x_{1}y_{2} - 3x_{1}y_{3} + x_{2}y_{3} - 2x_{2}y_{1} - x_{3}y_{2} - 3x_{3}y_{1}\),\(f_{2}(x,y) = x_{1}y_{2} - x_{2}y_{1} + 2x_{2}y_{2} + 3x_{1}y_{3} - 3x_{3}y_{1};\) \\
\textbf{C3.} Matricası tómendegishe bolǵan sızıqlı túrlendiriwdiń menshikli mánisi hám menshikli vektorların tabıń: \(\begin{pmatrix} 5 & 6 & - 3 \\  - 1 & 0 & 1 \\ 1 & 2 & - 1 \end{pmatrix}\); \\

\end{tabular}
\vspace{1cm}


\begin{tabular}{m{17cm}}
\textbf{37-variant}
\newline

\textbf{T1.} Evklid keńisligi. (Skalyar kóbeyme, ortogonal vektorlar, ortonormal bazis.) \\
\textbf{T2.} Unitar túrlendiriwler. (Unitar sızıqlı túrlendiriwler túsinigi,  Unitar túrlendiriwge túyinles túrlendiriwlerdiń matricası,   Ortonormal baziste unitar túrlendiriwlerdiń matricası) \\
\textbf{A1.} \(\mathbb{R}^{2}\) keńislikte anıqlangan tómendegi sáwlelendiriw skalyar kóbeyme bolatuģının anıqlań: \((x,y) = x_{1}y_{1} + 2x_{2}y_{2}\) \\
\textbf{A2.} Tómendegi kvadratlıq forma oń anıqlangan bolatuģın\(\lambda\) nıń barlıq mánislerin tabıń: \(5x_{1}^{2} + x_{2}^{2} + \lambda x_{3}^{2} + 4x_{1}x_{2} - 2x_{1}x_{3} - 2x_{2}x_{3}\); \\
\textbf{A3.} Tómendegi sáwlelendiriw \(V\) vektor keńislikte sızıqlı túrlendiriw boladı: \(V\) sızıqlı keńislik,\(Ax = a\), bul jerde \(a\)-fiksirlengen vektor; \\
\textbf{B1.} Ortogonallastırıw procesinen paydalanıp, berilgen vektorlar sistemasini ortogonallastirıń: \((1,1,0,0)\), (1, 0, 1, 1); \\
\textbf{B2.} Tómendegi kvadratlıq formanıń kanonikalıq kórinisin hám bul túrge keltiriwshi menshikli mánislerin tabıń: \(2x_{1}^{2} + 3x_{2}^{2} + 4x_{3}^{2} - 2x_{1}x_{2} + 4x_{1}x_{3} - 3x_{2}x_{3}\); \\
\textbf{B3.} Matricası tómendegishe bolǵan sızıqlı túrlendiriwdiń menshikli mánisi hám menshikli vektorların tabıń: \(\begin{pmatrix} 7 & 0 & 0 \\ 10 & - 19 & 0 \\ 12 & - 24 & 13 \end{pmatrix}\); \\
\textbf{C1.} Jordan narmal formasini tabıń\(\begin{pmatrix} 4 & - 5 & 2 \\ 0 & - 7 & 3 \\ 0 & 0 & 4 \end{pmatrix}\); \\
\textbf{C2.} Tómendegi kvadratlıq formalarga sáykes keliwshi ermit bisızıqlı formalardı tabiń: \(x_{1}\overline{x_{1}} + (2 + i)x_{1}\overline{x_{2}} + (2 - i)x_{2}\overline{x_{1}} + ix_{1}\overline{x_{3}} - ix_{3}\overline{x_{1}} - x_{3}\overline{x_{3}}\); \\
\textbf{C3.} Matricası tómendegishe bolǵan sızıqlı túrlendiriwdiń menshikli mánisi hám menshikli vektorların tabıń: \(\begin{pmatrix} 1 & 0 & 0 & 0 \\ 0 & 0 & 0 & 0 \\ 1 & 0 & 0 & 0 \\ 0 & 0 & 0 & 1 \end{pmatrix}\); \\

\end{tabular}
\vspace{1cm}


\begin{tabular}{m{17cm}}
\textbf{38-variant}
\newline

\textbf{T1.} İnertsiya nızamı. (invariantlar,  eki kvadratılıq forma arasindaǵı baylanıs ) \\
\textbf{T2.} Sızıqlı túrlendiriwler.  (Sızıqlı túrlendiriw túsinigi, Sızıqlı túrlendiriwler ústinde ámeller, Sızıqlı túrlendiriwlerdiń obrazı hám yadrosı.) \\
\textbf{A1.} Tómendegi kvadratlıq formanıń rangni anıqlań: \(x_{1}x_{2} + x_{1}x_{3} + x_{2}x_{3}\); \\
\textbf{A2.} Tómendegi vektorlar sisteması óz ara ortogonallıqqa tekseriń hám olardı ortogonallıq baziske shekem toltırın: \((1, - 2,2, - 3),(2, - 3,2,4)\); \\
\textbf{A3.} Tómendegi sáwlelendiriwlerden qaysıları keńislikte sızıqlı túrlendiriw boladı:; \\
\textbf{B1.} Ortogonallastırıw procesinen paydalanıp, berilgen vektorlar sistemasini ortogonallastirıń: \((1,1, - 1,0)\), \((2,0, - 1,0)\), \((1, - 1,1, - 1)\), (2, \(0,1,1\) ). \\
\textbf{B2.} Eger \(f\) sızıqlı funkciya\(e_{1},e_{2},e_{3}\) bazisde \(f(x) = 2x_{1} - 3x_{2} + x_{3}\) arqalı anıqlanǵan bolsa, onıń \(e_{1}^{'},e_{2}^{'},e_{3}^{'}\) bazisdegi kórinisin tabıń\(e_{1}^{'} = e_{1} + 3e_{2} - 2e_{3},\ e_{2}^{'} = 2e_{1} + e_{2} - e_{3},\ e_{3}^{'} = e_{1} + e_{2} - 3e_{3}\); \\
\textbf{B3.} Tómendegi vektorlar sisteması óz ara ortogonallıqqa tekseriń hám olardı ortogonallıq baziske shekem toltırın: \((0,1,i),\ \ (1 + i,i,1)\); \\
\textbf{C1.} Berilgen \(A\) bisızıqlı formanıń\(e_{1},e_{2},e_{3}\) bazisdegi matritsası hám\(e_{1}^{'},e_{2}^{'},e_{3}^{'}\) baziske ótiw formulaları berilgen bolsa, onda bul bisızıqli formanıń\(e_{1}^{'},e_{2}^{'},e_{3}^{'}\) bazisdegi matritsasini tabıń: \(\begin{pmatrix} 1 & 2 & 3 \\ 4 & 5 & 6 \\ 7 & 8 & 9 \end{pmatrix}\), \(e_{1}^{'} = e_{1} - e_{2}\), \(e_{2}^{'} = e_{1} + e_{3}\), \(e_{3}^{'} = e_{1} + e_{2} + e_{3}\) \\
\textbf{C2.} Tómendegi bisiziqli formalar ekvivalent emes ekenligin dálilleń:\(f_{1}(x,y) = x_{1}y_{1} + 2x_{1}y_{2} + 2x_{2}y_{1} + 5x_{2}y_{2} + 6x_{2}y_{3} + 8x_{3}y_{2} + 10x_{3}y_{3}\), \(f_{2}(x,y) = 2x_{1}y_{1} - x_{1}y_{3} + x_{2}y_{2} - x_{3}y_{1} + 5x_{3}y_{3}\). \\
\textbf{C3.} Matricası tómendegishe bolǵan sızıqlı túrlendiriwdiń menshikli mánisi hám menshikli vektorların tabıń: \(\begin{pmatrix} 2 & - 1 & 2 \\ 5 & - 3 & 3 \\  - 1 & 0 & - 2 \end{pmatrix}\); \\

\end{tabular}
\vspace{1cm}


\begin{tabular}{m{17cm}}
\textbf{39-variant}
\newline

\textbf{T1.} Sızıqlı, bisızıqlı hám kvadratlıq formalar. (Bisızıqlı forma,  simmetriyalı bisızıqlı formalar)  \\
\textbf{T2.} Túyinles túrlendiriw. ( Evklid keńisligindegi sızıqlı túrlendiriwler menen bisızıqlı formalar arasındaǵı baylanıs, Berilgen túrlendiriwge túyinles túrlendiriwler, Óz-ózine túyinles túrlendiriwler) \\
\textbf{A1.} \(\mathbb{R}^{2}\) keńislikte anıqlangan tómendegi sáwlelendiriw skalyar kóbeyme bolatuģının anıqlań: \((x,y) = x_{1}y_{1} + 2x_{2}y_{1} + 2x_{1}y_{2} + 7x_{2}y_{2}\) \\
\textbf{A2.} Tómendegi funkciyalı haqiqiy sanlar maydanı ústinde anıqlangan \(V\) keńislikte sızıqlı funkciya boladı: \(V = M_{n}\left( \mathbb{R} \right),\ \ f(A) = \det(A)\); \\
\textbf{A3.} Matricası tómendegishe: \\
\textbf{B1.} \(\mathbb{R}^{2}\) keńislikte \((x,y) = x_{1}y_{1} + 2x_{2}y_{1} + 2x_{1}y_{2} + 5x_{2}y_{2}\) berilgen skalyar kóbeyme ushın \(a = (1,0)\) hám \(b = (0,1)\) vektorlar arasındaǵı múyeshti tabıń \\
\textbf{B2.} Tómendegi kvadratlıq formanı kanonikalıq kóriniske keltiriń: \(12x_{1}^{2} + 3x_{2}^{2} + 12x_{3}^{2} - 12x_{1}x_{2} + 24x_{1}x_{3} - 8x_{2}x_{3}\); \\
\textbf{B3.} Matricası tómendegishe bolǵan sızıqlı túrlendiriwdiń menshikli mánisi hám menshikli vektorların tabıń: \(\begin{pmatrix} 4 & - 5 & 2 \\ 0 & - 7 & 3 \\ 0 & 0 & 4 \end{pmatrix}\); \\
\textbf{C1.} Jordan normal formasını tabıń \(\begin{pmatrix} 0 & 0 & 1 \\ 1 & 4 & 0 \\  - 2 & 0 & 2 \end{pmatrix}\); \\
\textbf{C2.} Tómendegi kvadratlıq formalarga sáykes keliwshi ermit bisızıqlı formalardı tabiń:\((5 - i)x_{1}\overline{x_{2}} + (5 + i)\overline{x_{1}}x_{2} + x_{2}\overline{x_{2}}\); \\
\textbf{C3.} Matricası tómendegishe bolǵan sızıqlı túrlendiriwdiń menshikli mánisi hám menshikli vektorların tabıń: \(\begin{pmatrix} 1 & 0 & 0 & 0 \\ 0 & 0 & 0 & 0 \\ 0 & 0 & 0 & 0 \\ 1 & 0 & 0 & 0 \end{pmatrix}\); \\

\end{tabular}
\vspace{1cm}


\begin{tabular}{m{17cm}}
\textbf{40-variant}
\newline

\textbf{T1.} Kvadratlıq forma. (Lagran usulı, Yakobi usulı kvadratlıq formanı keltiriw.) \\
\textbf{T2.} Óz-ara orın almasıwshı túrlendiriwler. (Óz-ara orın almasıwshı túrlendiriwler,  Ortogonal bazis haqqında teorema,  Normal túrlendiriwlerdiń kanonikalıq kórinisi) \\
\textbf{A1.} Tómendegi kvadratlıq formanıń rangni anıqlań: \(3x_{1}^{2} - 2x_{2}^{2} + 2x_{3}^{2} + 4x_{1}x_{2} - 3x_{1}x_{3} - x_{2}x_{3}\); \\
\textbf{A2.} Tómendegi kvadratlıq forma oń anıqlangan bolatuģın\(\lambda\) nıń barlıq mánislerin tabıń: \(x_{1}^{2} + 4x_{2}^{2} + x_{3}^{2} + 2\lambda x_{1}x_{2} + 10x_{1}x_{3} + 6x_{2}x_{3}\); \\
\textbf{A3.} Tómendegi sáwlelendiriw\(V = \mathbb{R}^{3}\) keńislikte sızıqlı túrlendiriw boladı: \(A\left( x_{1},x_{2},x_{3} \right) = \left( x_{1} + 3x_{3},x_{2}^{3},x_{1} + x_{3} \right)\). \\
\textbf{B1.} Ortogonallastırıw procesinen paydalanıp, berilgen vektorlar sistemasini ortogonallastirıń: \((1,2,2, - 1)\), ( \(1,1, - 5,3\) ), (3, 2, 8, -7); \\
\textbf{B2.} Eger \(f\) sızıqlı funkciya\(e_{1},e_{2},e_{3}\) bazisde \(f(x) = 2x_{1} - 3x_{2} + x_{3}\) arqalı anıqlanǵan bolsa, onıń \(e_{1}^{'},e_{2}^{'},e_{3}^{'}\) bazisdegi kórinisin tabıń\(e_{1}^{'} = 4e_{1} - e_{2} - 3e_{3},\ e_{2}^{'} = 2e_{1} + e_{2},\ e_{3}^{'} = 3e_{1} + 2e_{2}\). \\
\textbf{B3.} Tómendegi vektorlar sisteması óz ara ortogonallıqqa tekseriń hám olardı ortogonallıq baziske shekem toltırın: \((1,1,1,1),(1,1, - 1, - 1),(1, - 1,1, - 1)\); \\
\textbf{C1.} Jordan normal formasını tabıń \(\begin{pmatrix} 7 & 0 & 0 \\ 10 & - 19 & 0 \\ 12 & - 24 & 13 \end{pmatrix}\); \\
\textbf{C2.} Bazi bir ortonormal bazisde berilgen kvadratlıq formani kanonik kóriniske keltiriwshi ortonormal bazisin tabıń: \(x_{1}^{2} - 5x_{2}^{2} + x_{3}^{2} + 4x_{1}x_{2} + 2x_{1}x_{3} + 4x_{2}x_{3}\); \\
\textbf{C3.} Ortogonallastırıw procesinen paydalanıp, berilgen vektorlar sistemasini ortogonallastirıń: \((2,1,3, - 1)\), ( \(7,4,3, - 3\) ), ( \(1,1, - 6,0\) ), (5, 7, 7, 8); \\

\end{tabular}
\vspace{1cm}


\begin{tabular}{m{17cm}}
\textbf{41-variant}
\newline

\textbf{T1.} Kompleks evklid keńislikleri.  (Kompleks vektorlı keńislik, Ermit kvadratlıq forma.) \\
\textbf{T2.} Sızıqlı túrlendiriwler matritsasınıń Jordan normal kórinisi. (Jordan kletkasınıń xarakteristikalıq matricası, Jordan matricasınıń uqsaslıǵı haqqında teorema,  Matricalardı jordan normal kórinisine keltiriw) \\
\textbf{A1.} Tómendegi kvadratlıq formanıń rangni anıqlań: \(x_{1}^{2} + x_{2}^{2} + x_{3}^{2} + x_{4}^{2} + 2x_{1}x_{2} - 2x_{1}x_{4} - 2x_{2}x_{3} + 2x_{3}x_{4}\). \\
\textbf{A2.} Tómendegi vektorlar sisteması óz ara ortogonallıqqa tekseriń hám olardı ortogonallıq baziske shekem toltırın: \((1,2, - 1),(3, - 1,1)\); \\
\textbf{A3.} Tómendegi sáwlelendiriw\(V = \mathbb{R}^{3}\) keńislikte sızıqlı túrlendiriw boladı: \(A\left( x_{1},x_{2},x_{3} \right) = \left( 2x_{1} + x_{2},x_{1} + x_{3},x_{3}^{2} \right)\); \\
\textbf{B1.} Ortogonallastırıw procesinen paydalanıp, berilgen vektorlar sistemasini ortogonallastirıń: \((1,1, - 1,0)\), \((2,0, - 1,0)\), \((1, - 1,1, - 1)\), (2, \(0,1,1\) ). \\
\textbf{B2.} Tómendegi kvadratlıq formanı kanonikalıq kóriniske keltiriń: \(12x_{1}^{2} + 3x_{2}^{2} + 12x_{3}^{2} - 12x_{1}x_{2} + 24x_{1}x_{3} - 8x_{2}x_{3}\); \\
\textbf{B3.} Tómendegi vektorlar sisteması óz ara ortogonallıqqa tekseriń hám olardı ortogonallıq baziske shekem toltırın: \((1,1,1,1),(1,1, - 1, - 1),(1, - 1,1, - 1)\); \\
\textbf{C1.} Jordan narmal formasini tabıń\(\begin{pmatrix} 4 & - 5 & 2 \\ 0 & - 7 & 3 \\ 0 & 0 & 4 \end{pmatrix}\); \\
\textbf{C2.} Tómendegi kvadratlıq formalarga sáykes keliwshi ermit bisızıqlı formalardı tabiń. \(x_{1}\overline{x_{1}} - ix_{1}\overline{x_{2}} - ix_{2}\overline{x_{1}} + 2x_{2}\overline{x_{2}}\); \\
\textbf{C3.} Matricası tómendegishe bolǵan sızıqlı túrlendiriwdiń menshikli mánisi hám menshikli vektorların tabıń: \(\begin{pmatrix} 1 & 0 & 0 & 0 \\ 0 & 0 & 0 & 0 \\ 0 & 0 & 0 & 0 \\ 1 & 0 & 0 & 0 \end{pmatrix}\); \\

\end{tabular}
\vspace{1cm}


\begin{tabular}{m{17cm}}
\textbf{42-variant}
\newline

\textbf{T1.} Ortogonal  tolıqtırıwshı. (Ortogonal tolıqtırıwshı,  ortogonal proekciya) \\
\textbf{T2.} Keri túrlendiriwler. ( Keri túrlendiriw túsinigi,   Keri túrlendiriwdiń sızıqlılıǵı) \\
\textbf{A1.} \(\mathbb{R}^{2}\) keńislikte anıqlangan tómendegi sáwlelendiriw skalyar kóbeyme bolatuģının anıqlań: \((x,y) = x_{1}y_{1} + x_{2}y_{1} + 3x_{1}y_{2} + 2x_{2}y_{2}\) \\
\textbf{A2.} Tómendegi funkciyalı haqiqiy sanlar maydanı ústinde anıqlangan \(V\) keńislikte sızıqlı funkciya boladı: \(V = \mathbb{R}^{3},\ \ f(x) = |x|\); \\
\textbf{A3.} Tómendegi sáykes túrlendiriwlerden qaysıları berilgen vektor keńislikte sızıqlı túrlendiriw boladı: sızıqlı keńislik,, bul jerde -fiksirlengen vektor; \\
\textbf{B1.} Ortogonallastırıw procesinen paydalanıp, berilgen vektorlar sistemasini ortogonallastirıń: \((1,1, - 1, - 2)\), \((5,8, - 2, - 3)\), (3, 9, 3, 8); \\
\textbf{B2.} Tómendegi kvadratlıq formanıń kanonikalıq kórinisin hám bul túrge keltiriwshi menshikli mánislerin tabıń: \(5x_{1}^{2} + 6x_{2}^{2} + 4x_{3}^{2} - 4x_{1}x_{2} - 4x_{1}x_{3}\); \\
\textbf{B3.} Matricası tómendegishe bolǵan sızıqlı túrlendiriwdiń menshikli mánisi hám menshikli vektorların tabıń: \(\begin{pmatrix} 2 & - 1 & 2 \\ 0 & - 3 & 0 \\ 0 & 0 & 1 \end{pmatrix}\); \\
\textbf{C1.} Jordan normal formasını tabıń \(\begin{pmatrix} 7 & 0 & 0 \\ 10 & - 19 & 0 \\ 12 & - 24 & 13 \end{pmatrix}\); \\
\textbf{C2.} Bazi bir ortonormal bazisde berilgen kvadratlıq formani kanonik kóriniske keltiriwshi ortonormal bazisin tabıń: \(11x_{1}^{2} + 5x_{2}^{2} + 2x_{3}^{2} + 16x_{1}x_{2} + 4x_{1}x_{3} - 20x_{2}x_{3}\); \\
\textbf{C3.} Ortogonallastırıw procesinen paydalanıp, berilgen vektorlar sistemasini ortogonallastirıń: \((1,1, - 1,0)\), \((2,0, - 1,0)\), \((1, - 1,1, - 1)\), (2, \(0,1,1\) ). \\

\end{tabular}
\vspace{1cm}


\begin{tabular}{m{17cm}}
\textbf{43-variant}
\newline

\textbf{T1.} Ortogonal  tolıqtırıwshı. (Ortogonal tolıqtırıwshı,  ortogonal proekciya) \\
\textbf{T2.} Keri túrlendiriwler. ( Keri túrlendiriw túsinigi,   Keri túrlendiriwdiń sızıqlılıǵı) \\
\textbf{A1.} Tómendegi kvadratlıq formanıń rangni anıqlań: \(2x_{1}^{2} + 3x_{2}^{2} + 4x_{3}^{2} - 2x_{1}x_{2} + 4x_{1}x_{3} - 3x_{2}x_{3}\) \\
\textbf{A2.} Tómendegi vektorlar sisteması óz ara ortogonallıqqa tekseriń hám olardı ortogonallıq baziske shekem toltırın: \((1,1,1,2)\), \((1,2,3, - 3)\). \\
\textbf{A3.} Tómendegi sáwlelendiriwlerden qaysıları keńislikte sızıqlı túrlendiriw boladı:; \\
\textbf{B1.} \(\mathbb{R}^{2}\) keńislikte \((x,y) = x_{1}y_{1} + 2x_{2}y_{1} + 2x_{1}y_{2} + 5x_{2}y_{2}\) berilgen skalyar kóbeyme ushın \(a = (1,0)\) hám \(b = (0,1)\) vektorlar arasındaǵı múyeshti tabıń \\
\textbf{B2.} Tómendegi kvadratlıq formanıń kanonikalıq kórinisin hám bul túrge keltiriwshi menshikli mánislerin tabıń: \(3x_{1}^{2} - 2x_{2}^{2} + 2x_{3}^{2} + 4x_{1}x_{2} - 3x_{1}x_{3} - x_{2}x_{3}\); \\
\textbf{B3.} Tómendegi vektorlar sisteması óz ara ortogonallıqqa tekseriń hám olardı ortogonallıq baziske shekem toltırın: \((1,2,0, - 1),(3, - 1,1,1),( - 1,2,2,3)\); \\
\textbf{C1.} Jordan normal formasını tabıń \(\begin{pmatrix} 0 & 0 & 1 \\ 1 & 4 & 0 \\  - 2 & 0 & 2 \end{pmatrix}\); \\
\textbf{C2.} Bazi bir ortonormal bazisde berilgen kvadratlıq formani kanonik kóriniske keltiriwshi ortonormal bazisin tabıń: \(x_{1}^{2} + x_{2}^{2} + 5x_{3}^{2} - 6x_{1}x_{2} - 2x_{1}x_{3} + 2x_{2}x_{3}\); \\
\textbf{C3.} Matricası tómendegishe bolǵan sızıqlı túrlendiriwdiń menshikli mánisi hám menshikli vektorların tabıń: \(\begin{pmatrix} 0 & 2 & 1 \\  - 2 & 0 & 3 \\  - 1 & - 3 & 0 \end{pmatrix}\); \\

\end{tabular}
\vspace{1cm}


\begin{tabular}{m{17cm}}
\textbf{44-variant}
\newline

\textbf{T1.} Kompleks evklid keńislikleri.  (Kompleks vektorlı keńislik, Ermit kvadratlıq forma.) \\
\textbf{T2.} Túyinles túrlendiriw. ( Evklid keńisligindegi sızıqlı túrlendiriwler menen bisızıqlı formalar arasındaǵı baylanıs, Berilgen túrlendiriwge túyinles túrlendiriwler, Óz-ózine túyinles túrlendiriwler) \\
\textbf{A1.} \(\mathbb{R}^{2}\) keńislikte anıqlangan tómendegi sáwlelendiriw skalyar kóbeyme bolatuģının anıqlań: \((x,y) = x_{1}y_{1} - x_{2}y_{1} - x_{1}y_{2} + x_{2}y_{2}\) \\
\textbf{A2.} Tómendegi kvadratlıq forma oń anıqlangan bolatuģın\(\lambda\) nıń barlıq mánislerin tabıń: \(x_{1}^{2} + \lambda x_{2}^{2} + x_{3}^{2} - 4x_{1}x_{2} - 8x_{2}x_{3}\); \\
\textbf{A3.} Tómendegi sáykes túrlendiriwlerden qaysıları berilgen vektor keńislikte sızıqlı túrlendiriw boladı: sızıqlı keńislik,, bul jerde -fiksirlengen vektor; \\
\textbf{B1.} Ortogonallastırıw procesinen paydalanıp, berilgen vektorlar sistemasini ortogonallastirıń: \((1,1,0,0)\), (1, 0, 1, 1); \\
\textbf{B2.} Eger \(f\) sızıqlı funkciya\(e_{1},e_{2},e_{3}\) bazisde \(f(x) = 2x_{1} - 3x_{2} + x_{3}\) arqalı anıqlanǵan bolsa, onıń \(e_{1}^{'},e_{2}^{'},e_{3}^{'}\) bazisdegi kórinisin tabıń\(e_{1}^{'} = e_{1} + e_{2} - 2e_{3},\ e_{2}^{'} = e_{1} + e_{2} + 2e_{3},\ e_{3}^{'} = e_{2} + e_{3}\); \\
\textbf{B3.} Matricası tómendegishe bolǵan sızıqlı túrlendiriwdiń menshikli mánisi hám menshikli vektorların tabıń: \(\begin{pmatrix} 4 & - 5 & 2 \\ 0 & - 7 & 3 \\ 0 & 0 & 4 \end{pmatrix}\); \\
\textbf{C1.} Berilgen \(A\) bisızıqlı formanıń\(e_{1},e_{2},e_{3}\) bazisdegi matritsası hám\(e_{1}^{'},e_{2}^{'},e_{3}^{'}\) baziske ótiw formulaları berilgen bolsa, onda bul bisızıqli formanıń \(e_{1}^{'},e_{2}^{'},e_{3}^{'}\) bazisdegi matritsasini tabıń: \(\begin{pmatrix} 1 & 1 & 2 \\  - 1 & 2 & 1 \\  - 1 & 1 & - 1 \end{pmatrix},\begin{matrix}  & e_{1}^{'} = e_{1} + e_{2} - 2e_{3} \\  & e_{2}^{'} = e_{1} + e_{2} + 2e_{3} \\  & e_{3}^{'} = e_{2} + e_{3} \end{matrix}\) \\
\textbf{C2.} Tómendegi bisiziqli formalar ekvivalent emes ekenligin dálilleń:\(f_{1}(x,y) = 2x_{1}y_{2} - 3x_{1}y_{3} + x_{2}y_{3} - 2x_{2}y_{1} - x_{3}y_{2} - 3x_{3}y_{1}\),\(f_{2}(x,y) = x_{1}y_{2} - x_{2}y_{1} + 2x_{2}y_{2} + 3x_{1}y_{3} - 3x_{3}y_{1};\) \\
\textbf{C3.} Matricası tómendegishe bolǵan sızıqlı túrlendiriwdiń menshikli mánisi hám menshikli vektorların tabıń: \(\begin{pmatrix} 3 & - 1 & 0 & 0 \\ 1 & 1 & 0 & 0 \\ 3 & 0 & 5 & - 3 \\ 4 & - 1 & 3 & - 1 \end{pmatrix}\); \\

\end{tabular}
\vspace{1cm}


\begin{tabular}{m{17cm}}
\textbf{45-variant}
\newline

\textbf{T1.} Evklid keńisligi. (Skalyar kóbeyme, ortogonal vektorlar, ortonormal bazis.) \\
\textbf{T2.} Unitar túrlendiriwler. (Unitar sızıqlı túrlendiriwler túsinigi,  Unitar túrlendiriwge túyinles túrlendiriwlerdiń matricası,   Ortonormal baziste unitar túrlendiriwlerdiń matricası) \\
\textbf{A1.} \(\mathbb{R}^{2}\) keńislikte anıqlangan tómendegi sáwlelendiriw skalyar kóbeyme bolatuģının anıqlań: \((x,y) = x_{1}y_{1} - 2x_{2}y_{1} - 2x_{1}y_{2} + x_{2}y_{2}\) \\
\textbf{A2.} Tómendegi kvadratlıq forma oń anıqlangan bolatuģın\(\lambda\) nıń barlıq mánislerin tabıń: \(2x_{1}^{2} + 2x_{2}^{2} + x_{3}^{2} + 2\lambda x_{1}x_{2} + 6x_{1}x_{3} + 2x_{2}x_{3}\); \\
\textbf{A3.} Tómendegi sáwlelendiriw\(V = \mathbb{R}^{3}\) keńislikte sızıqlı túrlendiriw boladı: \(A\left( x_{1},x_{2},x_{3} \right) = \left( x_{1},x_{2} + 1,x_{3} + 2 \right)\); \\
\textbf{B1.} Ortogonallastırıw procesinen paydalanıp, berilgen vektorlar sistemasini ortogonallastirıń: \((2,0,1,1)\), ( \(1,2,0,1\) ), ( \(0,1, - 2,0\) ); \\
\textbf{B2.} Eger \(f\) sızıqlı funkciya\(e_{1},e_{2},e_{3}\) bazisde \(f(x) = 2x_{1} - 3x_{2} + x_{3}\) arqalı anıqlanǵan bolsa, onıń \(e_{1}^{'},e_{2}^{'},e_{3}^{'}\) bazisdegi kórinisin tabıń\(e_{1}^{'} = e_{1} + 3e_{2} - 2e_{3},\ e_{2}^{'} = 2e_{1} + e_{2} - e_{3},\ e_{3}^{'} = e_{1} + e_{2} - 3e_{3}\); \\
\textbf{B3.} Matricası tómendegishe bolǵan sızıqlı túrlendiriwdiń menshikli mánisi hám menshikli vektorların tabıń: \(\begin{pmatrix} 0 & 0 & 1 \\ 1 & 4 & 0 \\  - 2 & 0 & 2 \end{pmatrix}\); \\
\textbf{C1.} Berilgen \(A\) bisızıqlı formanıń\(e_{1},e_{2},e_{3}\) bazisdegi matritsası hám\(e_{1}^{'},e_{2}^{'},e_{3}^{'}\) baziske ótiw formulaları berilgen bolsa, onda bul bisızıqli formanıń\(e_{1}^{'},e_{2}^{'},e_{3}^{'}\) bazisdegi matritsasini tabıń: \(\ \) \(\begin{pmatrix} 0 & 2 & 1 \\  - 2 & 2 & 0 \\  - 1 & 0 & 3 \end{pmatrix}\), \(e_{1}^{'} = e_{1} + 2e_{2} - e_{3}\), \(e_{2}^{'} = e_{2} - e_{3}\), \(e_{3}^{'} = - e_{1} + e_{2} - 3e_{3}\) \\
\textbf{C2.} Bazi bir ortonormal bazisde berilgen kvadratlıq formani kanonik kóriniske keltiriwshi ortonormal bazisin tabıń: \(x_{1}^{2} + x_{2}^{2} + x_{3}^{2} + 4x_{1}x_{2} + 4x_{1}x_{3} + 4x_{2}x_{3}\); \\
\textbf{C3.} Ortogonallastırıw procesinen paydalanıp, berilgen vektorlar sistemasini ortogonallastirıń: \((2,1,3, - 1)\), ( \(7,4,3, - 3\) ), ( \(1,1, - 6,0\) ), (5, 7, 7, 8); \\

\end{tabular}
\vspace{1cm}


\begin{tabular}{m{17cm}}
\textbf{46-variant}
\newline

\textbf{T1.} İnertsiya nızamı. (invariantlar,  eki kvadratılıq forma arasindaǵı baylanıs ) \\
\textbf{T2.} Sızıqlı túrlendiriwler matritsasınıń Jordan normal kórinisi. (Jordan kletkasınıń xarakteristikalıq matricası, Jordan matricasınıń uqsaslıǵı haqqında teorema,  Matricalardı jordan normal kórinisine keltiriw) \\
\textbf{A1.} \(\mathbb{R}^{2}\) keńislikte anıqlangan tómendegi sáwlelendiriw skalyar kóbeyme bolatuģının anıqlań: \((x,y) = x_{1}y_{1} - x_{2}y_{2}\) \\
\textbf{A2.} Tómendegi kvadratlıq forma oń anıqlangan bolatuģın\(\lambda\) nıń barlıq mánislerin tabıń: \(2x_{1}^{2} + x_{2}^{2} + 3x_{3}^{2} + 2\lambda x_{1}x_{2} + 2x_{1}x_{3}\); \\
\textbf{A3.} Matricası tómendegishe bolgan sızıqlı túrlendiriwdin menshikli mánisi hám menshikli vektorların tabıń \(\begin{pmatrix} 3 & 4 \\ 5 & 2 \end{pmatrix}\); \\
\textbf{B1.} Ortogonallastırıw procesinen paydalanıp, berilgen vektorlar sistemasini ortogonallastirıń: \((1,2,2, - 1)\), ( \(1,1, - 5,3\) ), (3, 2, 8, -7); \\
\textbf{B2.} Tómendegi kvadratlıq formanıń kanonikalıq kórinisin hám bul túrge keltiriwshi menshikli mánislerin tabıń: \(x_{1}^{2} - 2x_{2}^{2} - 2x_{3}^{2} - 4x_{1}x_{2} - 4x_{1}x_{3} + 8x_{2}x_{3}\); \\
\textbf{B3.} Matricası tómendegishe bolǵan sızıqlı túrlendiriwdiń menshikli mánisi hám menshikli vektorların tabıń: \(\begin{pmatrix} 7 & 0 & 0 \\ 10 & - 19 & 0 \\ 12 & - 24 & 13 \end{pmatrix}\); \\
\textbf{C1.} Jordan normal formasını tabıń \(\begin{pmatrix} 4 & 1 & - 4 \\ 1 & 4 & 0 \\  - 4 & 0 & 4 \end{pmatrix}\); \\
\textbf{C2.} Tómendegi kvadratlıq forma oń anıqlangan bolatuģın\(\lambda\) parametrnıń barlıq mánislerin tabıń: \(\lambda x_{1}\overline{x_{1}} - ix_{1}\overline{x_{2}} + ix_{2}\overline{x_{1}} + 3x_{2}\overline{x_{2}}\); \\
\textbf{C3.} Matricası tómendegishe bolǵan sızıqlı túrlendiriwdiń menshikli mánisi hám menshikli vektorların tabıń: \(\begin{pmatrix} 1 & 0 & 0 & 0 \\ 0 & 0 & 0 & 0 \\ 1 & 0 & 0 & 0 \\ 0 & 0 & 0 & 1 \end{pmatrix}\); \\

\end{tabular}
\vspace{1cm}


\begin{tabular}{m{17cm}}
\textbf{47-variant}
\newline

\textbf{T1.} Sızıqlı keńislikler.   (Vektor,  sızıqlı baylanıs, bazis, ólshem, )  \\
\textbf{T2.} Sızıqlı túrlendiriwler.  (Sızıqlı túrlendiriw túsinigi, Sızıqlı túrlendiriwler ústinde ámeller, Sızıqlı túrlendiriwlerdiń obrazı hám yadrosı.) \\
\textbf{A1.} Tómendegi kvadratlıq formanıń rangni anıqlań: \(x_{1}x_{2} + x_{2}x_{3} + x_{3}x_{4} + x_{1}x_{4}\); \\
\textbf{A2.} Tómendegi kvadratlıq forma oń anıqlangan bolatuģın\(\lambda\) nıń barlıq mánislerin tabıń: \(x_{1}^{2} + x_{2}^{2} + 5x_{3}^{2} + 2\lambda x_{1}x_{2} - 2x_{1}x_{3} + 4x_{2}x_{3}\); \\
\textbf{A3.} Tómendegi sáwlelendiriw \(V\) vektor keńislikte sızıqlı túrlendiriw boladı: \(V\) sızıqlı keńislik,\(Ax = a\), bul jerde \(a\)-fiksirlengen vektor; \\
\textbf{B1.} \(\mathbb{R}^{3}\) keńislikte \((x,y) = x_{1}y_{1} + 3x_{2}y_{2} + 2x_{3}y_{3}\) berilgen skalyar kóbeyme ushın \(a = (1, - 3,2)\) va \(b = (2,1, - 1)\) \(b = (0,1)\) vektorlar arasındaǵı múyeshti tabıń . \\
\textbf{B2.} Tómendegi kvadratlıq formanıń kanonikalıq kórinisin hám bul túrge keltiriwshi menshikli mánislerin tabıń: \(2x_{1}^{2} + 3x_{2}^{2} + 4x_{3}^{2} - 2x_{1}x_{2} + 4x_{1}x_{3} - 3x_{2}x_{3}\); \\
\textbf{B3.} Tómendegi vektorlar sisteması óz ara ortogonallıqqa tekseriń hám olardı ortogonallıq baziske shekem toltırın: \((1,i, - i),\ \ ( - 2 - i,1 + i,2 - i)\); \\
\textbf{C1.} Jordan normal formasını tabıń \(\begin{pmatrix} 1 & - 2 & 1 \\  - 2 & 1 & 4 \\  - 1 & 4 & 1 \end{pmatrix}\). \\
\textbf{C2.} Tómendegi kvadratlıq forma oń anıqlangan bolatuģın\(\lambda\) parametrnıń barlıq mánislerin tabıń: \(x_{1}\overline{x_{1}} + ix_{1}\overline{x_{2}} - ix_{2}\overline{x_{1}} + \lambda x_{2}\overline{x_{2}}\); \\
\textbf{C3.} Matricası tómendegishe bolǵan sızıqlı túrlendiriwdiń menshikli mánisi hám menshikli vektorların tabıń: \(\begin{pmatrix} 5 & 6 & - 3 \\  - 1 & 0 & 1 \\ 1 & 2 & - 1 \end{pmatrix}\); \\

\end{tabular}
\vspace{1cm}


\begin{tabular}{m{17cm}}
\textbf{48-variant}
\newline

\textbf{T1.} Kvadratlıq forma. (Lagran usulı, Yakobi usulı kvadratlıq formanı keltiriw.) \\
\textbf{T2.} Óz-ara orın almasıwshı túrlendiriwler. (Óz-ara orın almasıwshı túrlendiriwler,  Ortogonal bazis haqqında teorema,  Normal túrlendiriwlerdiń kanonikalıq kórinisi) \\
\textbf{A1.} Tómendegi kvadratlıq formanıń rangni anıqlań: \(x_{1}^{2} - 2x_{2}^{2} - 2x_{3}^{2} - 4x_{1}x_{2} - 4x_{1}x_{3} + 8x_{2}x_{3}\); \\
\textbf{A2.} Tómendegi vektorlar sisteması óz ara ortogonallıqqa tekseriń hám olardı ortogonallıq baziske shekem toltırın: \((2,1,2),\ (1,2, - 2)\); \\
\textbf{A3.} Matricası tómendegishe: \\
\textbf{B1.} Ortogonallastırıw procesinen paydalanıp, berilgen vektorlar sistemasini ortogonallastirıń: \((1,1, - 1)\), (1, 1,1 ), \((3,2, - 1)\); \\
\textbf{B2.} Tómendegi kvadratlıq formanıń kanonikalıq kórinisin hám bul túrge keltiriwshi menshikli mánislerin tabıń: \(7x_{1}^{2} + 5x_{2}^{2} + 3x_{3}^{2} - 8x_{1}x_{2} + 8x_{2}x_{3}\); \\
\textbf{B3.} Tómendegi vektorlar sisteması óz ara ortogonallıqqa tekseriń hám olardı ortogonallıq baziske shekem toltırın. \((i,i,1, - 1),\ \ (1, - 1 + i,0,1),\ \ \); \\
\textbf{C1.} Jordan normal formasını tabıń \(\begin{pmatrix} 2 & - 1 & 2 \\ 0 & - 3 & 0 \\ 0 & 0 & 1 \end{pmatrix}\); \\
\textbf{C2.} Tómendegi kvadratlıq formalarga sáykes keliwshi ermit bisızıqlı formalardı tabiń: \(ix_{1}\overline{x_{2}} - ix_{2}\overline{x_{1}} + (3 - 2i)x_{1}\overline{x_{3}} + (3 + 2i)x_{3}\overline{x_{1}} + 2x_{2}\overline{x_{3}} + 2x_{3}\overline{x_{2}}\). \\
\textbf{C3.} Matricası tómendegishe bolǵan sızıqlı túrlendiriwdiń menshikli mánisi hám menshikli vektorların tabıń: \(\begin{pmatrix} 1 & 1 & 1 & 1 \\ 1 & 1 & - 1 & - 1 \\ 1 & - 1 & 1 & - 1 \\ 1 & - 1 & - 1 & 1 \end{pmatrix}\). \\

\end{tabular}
\vspace{1cm}


\begin{tabular}{m{17cm}}
\textbf{49-variant}
\newline

\textbf{T1.} Sızıqlı, bisızıqlı hám kvadratlıq formalar. (Bisızıqlı forma,  simmetriyalı bisızıqlı formalar)  \\
\textbf{T2.} Unitar túrlendiriwler. (Unitar sızıqlı túrlendiriwler túsinigi,  Unitar túrlendiriwge túyinles túrlendiriwlerdiń matricası,   Ortonormal baziste unitar túrlendiriwlerdiń matricası) \\
\textbf{A1.} \(\mathbb{R}^{2}\) keńislikte anıqlangan tómendegi sáwlelendiriw skalyar kóbeyme bolatuģının anıqlań: \((x,y) = x_{1}y_{1} + 2x_{2}y_{2}\) \\
\textbf{A2.} Tómendegi vektorlar sisteması óz ara ortogonallıqqa tekseriń hám olardı ortogonallıq baziske shekem toltırın: \((2,1,2),\ (1,2, - 2)\); \\
\textbf{A3.} Tómendegi sáwlelendiriw\(V = \mathbb{R}^{3}\) keńislikte sızıqlı túrlendiriw boladı: \(A\left( x_{1},x_{2},x_{3} \right) = \left( x_{1} + 3x_{3},x_{2}^{3},x_{1} + x_{3} \right)\). \\
\textbf{B1.} Ortogonallastırıw procesinen paydalanıp, berilgen vektorlar sistemasini ortogonallastirıń: \((1,2,1,3)\), (4, 1, 1, 1), (3, 1, 1, 0); \\
\textbf{B2.} Tómendegi kvadratlıq formanı kanonikalıq kóriniske keltiriń: \(2x_{1}^{2} + 18x_{2}^{2} + 8x_{3}^{2} - 12x_{1}x_{2} + 8x_{1}x_{3} - 27x_{2}x_{3}\); \\
\textbf{B3.} Tómendegi vektorlar sisteması óz ara ortogonallıqqa tekseriń hám olardı ortogonallıq baziske shekem toltırın: \((0,i,1,1),\ \ (1,2,1 + i, - 1 + i)\). \\
\textbf{C1.} Berilgen \(A\) bisızıqlı formanıń\(e_{1},e_{2},e_{3}\) bazisdegi matritsası hám\(e_{1}^{'},e_{2}^{'},e_{3}^{'}\) baziske ótiw formulaları berilgen bolsa, onda bul bisızıqli formanıń \(e_{1}^{'},e_{2}^{'},e_{3}^{'}\) bazisdegi matritsasini tabıń: \(\begin{pmatrix} 2 & 2 & 3 \\  - 4 & 3 & 1 \\ 3 & 1 & 2 \end{pmatrix},\ \begin{matrix}  & e_{1}^{'} = e_{1} + 3e_{2} - 2e_{3} \\  & e_{2}^{'} = 2e_{1} + e_{2} - e_{3} \\  & e_{3}^{'} = e_{1} + e_{2} - 3e_{3} \end{matrix}\) \\
\textbf{C2.} Tómendegi bisiziqli formalar ekvivalent emes ekenligin dálilleń:\(f_{1}(x,y) = x_{1}y_{1} + 2x_{1}y_{2} + 2x_{2}y_{1} + 5x_{2}y_{2} + 6x_{2}y_{3} + 8x_{3}y_{2} + 10x_{3}y_{3}\), \(f_{2}(x,y) = 2x_{1}y_{1} - x_{1}y_{3} + x_{2}y_{2} - x_{3}y_{1} + 5x_{3}y_{3}\). \\
\textbf{C3.} Matricası tómendegishe bolǵan sızıqlı túrlendiriwdiń menshikli mánisi hám menshikli vektorların tabıń: \(\begin{pmatrix} 1 & - 3 & 4 \\ 4 & - 7 & 8 \\ 6 & - 7 & 7 \end{pmatrix}\); \\

\end{tabular}
\vspace{1cm}


\begin{tabular}{m{17cm}}
\textbf{50-variant}
\newline

\textbf{T1.} İnertsiya nızamı. (invariantlar,  eki kvadratılıq forma arasindaǵı baylanıs ) \\
\textbf{T2.} Sızıqlı túrlendiriwler matritsasınıń Jordan normal kórinisi. (Jordan kletkasınıń xarakteristikalıq matricası, Jordan matricasınıń uqsaslıǵı haqqında teorema,  Matricalardı jordan normal kórinisine keltiriw) \\
\textbf{A1.} \(\mathbb{R}^{2}\) keńislikte anıqlangan tómendegi sáwlelendiriw skalyar kóbeyme bolatuģının anıqlań: \((x,y) = x_{1}y_{1} + 2x_{2}y_{1} + 2x_{1}y_{2} + 7x_{2}y_{2}\) \\
\textbf{A2.} Tómendegi kvadratlıq forma oń anıqlangan bolatuģın\(\lambda\) nıń barlıq mánislerin tabıń: \(x_{1}^{2} + \lambda x_{2}^{2} + x_{3}^{2} - 4x_{1}x_{2} - 8x_{2}x_{3}\); \\
\textbf{A3.} Tómendegi sáwlelendiriw\(V = \mathbb{R}^{3}\) keńislikte sızıqlı túrlendiriw boladı: \(A\left( x_{1},x_{2},x_{3} \right) = \left( x_{2} + x_{3},2x_{1} + x_{3},3x_{1} - x_{2} + x_{3} \right)\); \\
\textbf{B1.} Ortogonallastırıw procesinen paydalanıp, berilgen vektorlar sistemasini ortogonallastirıń: \((1,0,0)\), (0, 1, -1), (1, 1, 1); \\
\textbf{B2.} Tómendegi kvadratlıq formanı kanonikalıq kóriniske keltiriń: \(x_{1}x_{2} + x_{1}x_{3} + x_{1}x_{4} + x_{2}x_{3} + x_{2}x_{4} + x_{3}x_{4}\); \\
\textbf{B3.} Tómendegi vektorlar sisteması óz ara ortogonallıqqa tekseriń hám olardı ortogonallıq baziske shekem toltırın: \((0,1,i),\ \ (1 + i,i,1)\); \\
\textbf{C1.} Berilgen \(A\) bisızıqlı formanıń\(e_{1},e_{2},e_{3}\) bazisdegi matritsası hám\(e_{1}^{'},e_{2}^{'},e_{3}^{'}\) baziske ótiw formulaları berilgen bolsa, onda bul bisızıqli formanıń\(e_{1}^{'},e_{2}^{'},e_{3}^{'}\) bazisdegi matritsasini tabıń: \(\begin{pmatrix} 1 & 2 & 3 \\ 4 & 5 & 6 \\ 7 & 8 & 9 \end{pmatrix}\), \(e_{1}^{'} = e_{1} - e_{2}\), \(e_{2}^{'} = e_{1} + e_{3}\), \(e_{3}^{'} = e_{1} + e_{2} + e_{3}\) \\
\textbf{C2.} Tómendegi kvadratlıq forma oń anıqlangan bolatuģın\(\lambda\) parametrnıń barlıq mánislerin tabıń: \(x_{1}\overline{x_{1}} + ix_{1}\overline{x_{2}} - ix_{2}\overline{x_{1}} + \lambda x_{2}\overline{x_{2}}\); \\
\textbf{C3.} Matricası tómendegishe bolǵan sızıqlı túrlendiriwdiń menshikli mánisi hám menshikli vektorların tabıń: \(\begin{pmatrix} 2 & - 1 & 2 \\ 5 & - 3 & 3 \\  - 1 & 0 & - 2 \end{pmatrix}\); \\

\end{tabular}
\vspace{1cm}


\begin{tabular}{m{17cm}}
\textbf{51-variant}
\newline

\textbf{T1.} Evklid keńisligi. (Skalyar kóbeyme, ortogonal vektorlar, ortonormal bazis.) \\
\textbf{T2.} Keri túrlendiriwler. ( Keri túrlendiriw túsinigi,   Keri túrlendiriwdiń sızıqlılıǵı) \\
\textbf{A1.} Tómendegi kvadratlıq formanıń rangni anıqlań: \(x_{1}x_{2} + x_{2}x_{3} + x_{3}x_{4} + x_{1}x_{4}\); \\
\textbf{A2.} Tómendegi kvadratlıq forma oń anıqlangan bolatuģın\(\lambda\) nıń barlıq mánislerin tabıń: \(2x_{1}^{2} + 2x_{2}^{2} + x_{3}^{2} + 2\lambda x_{1}x_{2} + 6x_{1}x_{3} + 2x_{2}x_{3}\); \\
\textbf{A3.} Tómendegi sáwlelendiriwlerden qaysıları keńislikte sızıqlı túrlendiriw boladı:; \\
\textbf{B1.} Ortogonallastırıw procesinen paydalanıp, berilgen vektorlar sistemasini ortogonallastirıń: \((1,1, - 1)\), (1, 1,1 ), \((3,2, - 1)\); \\
\textbf{B2.} Tómendegi kvadratlıq formanıń kanonikalıq kórinisin hám bul túrge keltiriwshi menshikli mánislerin tabıń: \(x_{1}x_{2} + x_{1}x_{3} + x_{2}x_{3}\); \\
\textbf{B3.} Tómendegi vektorlar sisteması óz ara ortogonallıqqa tekseriń hám olardı ortogonallıq baziske shekem toltırın: \((0,1,i),\ \ (1 + i,i,1)\); \\
\textbf{C1.} Berilgen \(A\) bisızıqlı formanıń\(e_{1},e_{2},e_{3}\) bazisdegi matritsası hám\(e_{1}^{'},e_{2}^{'},e_{3}^{'}\) baziske ótiw formulaları berilgen bolsa, onda bul bisızıqli formanıń \(e_{1}^{'},e_{2}^{'},e_{3}^{'}\) bazisdegi matritsasini tabıń: \(\begin{pmatrix} 2 & 2 & 3 \\  - 4 & 3 & 1 \\ 3 & 1 & 2 \end{pmatrix},\ \begin{matrix}  & e_{1}^{'} = e_{1} + 3e_{2} - 2e_{3} \\  & e_{2}^{'} = 2e_{1} + e_{2} - e_{3} \\  & e_{3}^{'} = e_{1} + e_{2} - 3e_{3} \end{matrix}\) \\
\textbf{C2.} Bazi bir ortonormal bazisde berilgen kvadratlıq formani kanonik kóriniske keltiriwshi ortonormal bazisin tabıń: \(11x_{1}^{2} + 5x_{2}^{2} + 2x_{3}^{2} + 16x_{1}x_{2} + 4x_{1}x_{3} - 20x_{2}x_{3}\); \\
\textbf{C3.} Matricası tómendegishe bolǵan sızıqlı túrlendiriwdiń menshikli mánisi hám menshikli vektorların tabıń: \(\begin{pmatrix} 5 & 6 & - 3 \\  - 1 & 0 & 1 \\ 1 & 2 & - 1 \end{pmatrix}\); \\

\end{tabular}
\vspace{1cm}


\begin{tabular}{m{17cm}}
\textbf{52-variant}
\newline

\textbf{T1.} Sızıqlı, bisızıqlı hám kvadratlıq formalar. (Bisızıqlı forma,  simmetriyalı bisızıqlı formalar)  \\
\textbf{T2.} Sızıqlı túrlendiriwler.  (Sızıqlı túrlendiriw túsinigi, Sızıqlı túrlendiriwler ústinde ámeller, Sızıqlı túrlendiriwlerdiń obrazı hám yadrosı.) \\
\textbf{A1.} \(\mathbb{R}^{2}\) keńislikte anıqlangan tómendegi sáwlelendiriw skalyar kóbeyme bolatuģının anıqlań: \((x,y) = x_{1}y_{1} - x_{2}y_{2}\) \\
\textbf{A2.} Tómendegi kvadratlıq forma oń anıqlangan bolatuģın\(\lambda\) nıń barlıq mánislerin tabıń: \(5x_{1}^{2} + x_{2}^{2} + \lambda x_{3}^{2} + 4x_{1}x_{2} - 2x_{1}x_{3} - 2x_{2}x_{3}\); \\
\textbf{A3.} Tómendegi sáwlelendiriw\(V = \mathbb{R}^{3}\) keńislikte sızıqlı túrlendiriw boladı: \(A\left( x_{1},x_{2},x_{3} \right) = \left( x_{1},x_{2},x_{1} + x_{2} + x_{3} \right)\); \\
\textbf{B1.} Ortogonallastırıw procesinen paydalanıp, berilgen vektorlar sistemasini ortogonallastirıń: \((1,1, - 1,0)\), \((2,0, - 1,0)\), \((1, - 1,1, - 1)\), (2, \(0,1,1\) ). \\
\textbf{B2.} Eger \(f\) sızıqlı funkciya\(e_{1},e_{2},e_{3}\) bazisde \(f(x) = 2x_{1} - 3x_{2} + x_{3}\) arqalı anıqlanǵan bolsa, onıń \(e_{1}^{'},e_{2}^{'},e_{3}^{'}\) bazisdegi kórinisin tabıń\(e_{1}^{'} = e_{1} - e_{2},\ e_{2}^{'} = e_{1} + e_{3},\ \ e_{3}^{'} = e_{1} + e_{2} + e_{3}\); \\
\textbf{B3.} Matricası tómendegishe bolǵan sızıqlı túrlendiriwdiń menshikli mánisi hám menshikli vektorların tabıń: \(\begin{pmatrix} 0 & 0 & 1 \\ 1 & 4 & 0 \\  - 2 & 0 & 2 \end{pmatrix}\); \\
\textbf{C1.} Jordan normal formasını tabıń \(\begin{pmatrix} 2 & - 1 & 2 \\ 0 & - 3 & 0 \\ 0 & 0 & 1 \end{pmatrix}\); \\
\textbf{C2.} Bazi bir ortonormal bazisde berilgen kvadratlıq formani kanonik kóriniske keltiriwshi ortonormal bazisin tabıń: \(x_{1}^{2} - 5x_{2}^{2} + x_{3}^{2} + 4x_{1}x_{2} + 2x_{1}x_{3} + 4x_{2}x_{3}\); \\
\textbf{C3.} Matricası tómendegishe bolǵan sızıqlı túrlendiriwdiń menshikli mánisi hám menshikli vektorların tabıń: \(\begin{pmatrix} 1 & 0 & 0 & 0 \\ 0 & 0 & 0 & 0 \\ 1 & 0 & 0 & 0 \\ 0 & 0 & 0 & 1 \end{pmatrix}\); \\

\end{tabular}
\vspace{1cm}


\begin{tabular}{m{17cm}}
\textbf{53-variant}
\newline

\textbf{T1.} Sızıqlı keńislikler.   (Vektor,  sızıqlı baylanıs, bazis, ólshem, )  \\
\textbf{T2.} Túyinles túrlendiriw. ( Evklid keńisligindegi sızıqlı túrlendiriwler menen bisızıqlı formalar arasındaǵı baylanıs, Berilgen túrlendiriwge túyinles túrlendiriwler, Óz-ózine túyinles túrlendiriwler) \\
\textbf{A1.} \(\mathbb{R}^{2}\) keńislikte anıqlangan tómendegi sáwlelendiriw skalyar kóbeyme bolatuģının anıqlań: \((x,y) = x_{1}y_{1} + x_{2}y_{1} + 3x_{1}y_{2} + 2x_{2}y_{2}\) \\
\textbf{A2.} Tómendegi vektorlar sisteması óz ara ortogonallıqqa tekseriń hám olardı ortogonallıq baziske shekem toltırın: \((1,1,1,2)\), \((1,2,3, - 3)\). \\
\textbf{A3.} Tómendegi sáwlelendiriw\(V = \mathbb{R}^{3}\) keńislikte sızıqlı túrlendiriw boladı: \(A\left( x_{1},x_{2},x_{3} \right) = \left( x_{1} + 2,x_{2} + 5,x_{3} \right)\); \\
\textbf{B1.} Ortogonallastırıw procesinen paydalanıp, berilgen vektorlar sistemasini ortogonallastirıń: \((1,2,2, - 1)\), ( \(1,1, - 5,3\) ), (3, 2, 8, -7); \\
\textbf{B2.} Tómendegi kvadratlıq formanıń kanonikalıq kórinisin hám bul túrge keltiriwshi menshikli mánislerin tabıń: \(2x_{1}^{2} + 3x_{2}^{2} + 4x_{3}^{2} - 2x_{1}x_{2} + 4x_{1}x_{3} - 3x_{2}x_{3}\); \\
\textbf{B3.} Tómendegi vektorlar sisteması óz ara ortogonallıqqa tekseriń hám olardı ortogonallıq baziske shekem toltırın: \((1,2,0, - 1),(3, - 1,1,1),( - 1,2,2,3)\); \\
\textbf{C1.} Jordan normal formasını tabıń \(\begin{pmatrix} 1 & - 2 & 1 \\  - 2 & 1 & 4 \\  - 1 & 4 & 1 \end{pmatrix}\). \\
\textbf{C2.} Tómendegi kvadratlıq formalarga sáykes keliwshi ermit bisızıqlı formalardı tabiń: \(x_{1}\overline{x_{1}} + (2 + i)x_{1}\overline{x_{2}} + (2 - i)x_{2}\overline{x_{1}} + ix_{1}\overline{x_{3}} - ix_{3}\overline{x_{1}} - x_{3}\overline{x_{3}}\); \\
\textbf{C3.} Matricası tómendegishe bolǵan sızıqlı túrlendiriwdiń menshikli mánisi hám menshikli vektorların tabıń: \(\begin{pmatrix} 1 & - 3 & 4 \\ 4 & - 7 & 8 \\ 6 & - 7 & 7 \end{pmatrix}\); \\

\end{tabular}
\vspace{1cm}


\begin{tabular}{m{17cm}}
\textbf{54-variant}
\newline

\textbf{T1.} Ortogonal  tolıqtırıwshı. (Ortogonal tolıqtırıwshı,  ortogonal proekciya) \\
\textbf{T2.} Óz-ara orın almasıwshı túrlendiriwler. (Óz-ara orın almasıwshı túrlendiriwler,  Ortogonal bazis haqqında teorema,  Normal túrlendiriwlerdiń kanonikalıq kórinisi) \\
\textbf{A1.} Tómendegi kvadratlıq formanıń rangni anıqlań: \(x_{1}x_{2} + x_{1}x_{3} + x_{2}x_{3}\); \\
\textbf{A2.} Tómendegi funkciyalı haqiqiy sanlar maydanı ústinde anıqlangan \(V\) keńislikte sızıqlı funkciya boladı: \(V = M_{n}\left( \mathbb{R} \right),\ \ f(A) = \det(A)\); \\
\textbf{A3.} Tómendegi sáwlelendiriwlerden qaysıları keńislikte sızıqlı túrlendiriw boladı:; \\
\textbf{B1.} Ortogonallastırıw procesinen paydalanıp, berilgen vektorlar sistemasini ortogonallastirıń: \((1,2,1,3)\), (4, 1, 1, 1), (3, 1, 1, 0); \\
\textbf{B2.} Tómendegi kvadratlıq formanıń kanonikalıq kórinisin hám bul túrge keltiriwshi menshikli mánislerin tabıń: \(7x_{1}^{2} + 5x_{2}^{2} + 3x_{3}^{2} - 8x_{1}x_{2} + 8x_{2}x_{3}\); \\
\textbf{B3.} Tómendegi vektorlar sisteması óz ara ortogonallıqqa tekseriń hám olardı ortogonallıq baziske shekem toltırın: \((1,1,1,1),(1,1, - 1, - 1),(1, - 1,1, - 1)\); \\
\textbf{C1.} Jordan normal formasını tabıń \(\begin{pmatrix} 7 & 0 & 0 \\ 10 & - 19 & 0 \\ 12 & - 24 & 13 \end{pmatrix}\); \\
\textbf{C2.} Tómendegi kvadratlıq forma oń anıqlangan bolatuģın\(\lambda\) parametrnıń barlıq mánislerin tabıń: \(\lambda x_{1}\overline{x_{1}} - ix_{1}\overline{x_{2}} + ix_{2}\overline{x_{1}} + 3x_{2}\overline{x_{2}}\); \\
\textbf{C3.} Matricası tómendegishe bolǵan sızıqlı túrlendiriwdiń menshikli mánisi hám menshikli vektorların tabıń: \(\begin{pmatrix} 1 & 1 & 1 & 1 \\ 1 & 1 & - 1 & - 1 \\ 1 & - 1 & 1 & - 1 \\ 1 & - 1 & - 1 & 1 \end{pmatrix}\). \\

\end{tabular}
\vspace{1cm}


\begin{tabular}{m{17cm}}
\textbf{55-variant}
\newline

\textbf{T1.} Kvadratlıq forma. (Lagran usulı, Yakobi usulı kvadratlıq formanı keltiriw.) \\
\textbf{T2.} Sızıqlı túrlendiriwler matritsasınıń Jordan normal kórinisi. (Jordan kletkasınıń xarakteristikalıq matricası, Jordan matricasınıń uqsaslıǵı haqqında teorema,  Matricalardı jordan normal kórinisine keltiriw) \\
\textbf{A1.} Tómendegi kvadratlıq formanıń rangni anıqlań: \(x_{1}^{2} - 2x_{2}^{2} - 2x_{3}^{2} - 4x_{1}x_{2} - 4x_{1}x_{3} + 8x_{2}x_{3}\); \\
\textbf{A2.} Tómendegi kvadratlıq forma oń anıqlangan bolatuģın\(\lambda\) nıń barlıq mánislerin tabıń: \(x_{1}^{2} + x_{2}^{2} + 5x_{3}^{2} + 2\lambda x_{1}x_{2} - 2x_{1}x_{3} + 4x_{2}x_{3}\); \\
\textbf{A3.} Tómendegi sáwlelendiriwlerden qaysıları sáykes túrde berilgen vektor keńislikte sızıqlı túrlendiriw boladı: sızıqlı keńislik, bul jerde -fiksirlengen san; \\
\textbf{B1.} Ortogonallastırıw procesinen paydalanıp, berilgen vektorlar sistemasini ortogonallastirıń: \((1,1, - 1, - 2)\), \((5,8, - 2, - 3)\), (3, 9, 3, 8); \\
\textbf{B2.} Eger \(f\) sızıqlı funkciya\(e_{1},e_{2},e_{3}\) bazisde \(f(x) = 2x_{1} - 3x_{2} + x_{3}\) arqalı anıqlanǵan bolsa, onıń \(e_{1}^{'},e_{2}^{'},e_{3}^{'}\) bazisdegi kórinisin tabıń\(e_{1}^{'} = e_{1} + e_{2} - 2e_{3},\ e_{2}^{'} = e_{1} + e_{2} + 2e_{3},\ e_{3}^{'} = e_{2} + e_{3}\); \\
\textbf{B3.} Matricası tómendegishe bolǵan sızıqlı túrlendiriwdiń menshikli mánisi hám menshikli vektorların tabıń: \(\begin{pmatrix} 7 & 0 & 0 \\ 10 & - 19 & 0 \\ 12 & - 24 & 13 \end{pmatrix}\); \\
\textbf{C1.} Jordan normal formasını tabıń \(\begin{pmatrix} 0 & 0 & 1 \\ 1 & 4 & 0 \\  - 2 & 0 & 2 \end{pmatrix}\); \\
\textbf{C2.} Bazi bir ortonormal bazisde berilgen kvadratlıq formani kanonik kóriniske keltiriwshi ortonormal bazisin tabıń: \(x_{1}^{2} + x_{2}^{2} + 5x_{3}^{2} - 6x_{1}x_{2} - 2x_{1}x_{3} + 2x_{2}x_{3}\); \\
\textbf{C3.} Ortogonallastırıw procesinen paydalanıp, berilgen vektorlar sistemasini ortogonallastirıń: \((1,1, - 1,0)\), \((2,0, - 1,0)\), \((1, - 1,1, - 1)\), (2, \(0,1,1\) ). \\

\end{tabular}
\vspace{1cm}


\begin{tabular}{m{17cm}}
\textbf{56-variant}
\newline

\textbf{T1.} Kompleks evklid keńislikleri.  (Kompleks vektorlı keńislik, Ermit kvadratlıq forma.) \\
\textbf{T2.} Sızıqlı túrlendiriwler.  (Sızıqlı túrlendiriw túsinigi, Sızıqlı túrlendiriwler ústinde ámeller, Sızıqlı túrlendiriwlerdiń obrazı hám yadrosı.) \\
\textbf{A1.} Tómendegi kvadratlıq formanıń rangni anıqlań: \(x_{1}^{2} + x_{2}^{2} + x_{3}^{2} + x_{4}^{2} + 2x_{1}x_{2} - 2x_{1}x_{4} - 2x_{2}x_{3} + 2x_{3}x_{4}\). \\
\textbf{A2.} Tómendegi vektorlar sisteması óz ara ortogonallıqqa tekseriń hám olardı ortogonallıq baziske shekem toltırın: \((1,2, - 1),(3, - 1,1)\); \\
\textbf{A3.} Matricası tómendegishe bolgan sızıqlı túrlendiriwdin menshikli mánisi hám menshikli vektorların tabıń: \(\begin{pmatrix} 2 & 1 \\ 1 & 2 \end{pmatrix}\); \\
\textbf{B1.} \(\mathbb{R}^{3}\) keńislikte \((x,y) = x_{1}y_{1} + 3x_{2}y_{2} + 2x_{3}y_{3}\) berilgen skalyar kóbeyme ushın \(a = (1, - 3,2)\) va \(b = (2,1, - 1)\) \(b = (0,1)\) vektorlar arasındaǵı múyeshti tabıń . \\
\textbf{B2.} Tómendegi kvadratlıq formanıń kanonikalıq kórinisin hám bul túrge keltiriwshi menshikli mánislerin tabıń: \(x_{1}^{2} - 2x_{2}^{2} - 2x_{3}^{2} - 4x_{1}x_{2} - 4x_{1}x_{3} + 8x_{2}x_{3}\); \\
\textbf{B3.} Matricası tómendegishe bolǵan sızıqlı túrlendiriwdiń menshikli mánisi hám menshikli vektorların tabıń: \(\begin{pmatrix} 4 & - 5 & 2 \\ 0 & - 7 & 3 \\ 0 & 0 & 4 \end{pmatrix}\); \\
\textbf{C1.} Berilgen \(A\) bisızıqlı formanıń\(e_{1},e_{2},e_{3}\) bazisdegi matritsası hám\(e_{1}^{'},e_{2}^{'},e_{3}^{'}\) baziske ótiw formulaları berilgen bolsa, onda bul bisızıqli formanıń \(e_{1}^{'},e_{2}^{'},e_{3}^{'}\) bazisdegi matritsasini tabıń: \(\begin{pmatrix} 1 & 1 & 2 \\  - 1 & 2 & 1 \\  - 1 & 1 & - 1 \end{pmatrix},\begin{matrix}  & e_{1}^{'} = e_{1} + e_{2} - 2e_{3} \\  & e_{2}^{'} = e_{1} + e_{2} + 2e_{3} \\  & e_{3}^{'} = e_{2} + e_{3} \end{matrix}\) \\
\textbf{C2.} Tómendegi kvadratlıq formalarga sáykes keliwshi ermit bisızıqlı formalardı tabiń: \(ix_{1}\overline{x_{2}} - ix_{2}\overline{x_{1}} + (3 - 2i)x_{1}\overline{x_{3}} + (3 + 2i)x_{3}\overline{x_{1}} + 2x_{2}\overline{x_{3}} + 2x_{3}\overline{x_{2}}\). \\
\textbf{C3.} Matricası tómendegishe bolǵan sızıqlı túrlendiriwdiń menshikli mánisi hám menshikli vektorların tabıń: \(\begin{pmatrix} 3 & - 1 & 0 & 0 \\ 1 & 1 & 0 & 0 \\ 3 & 0 & 5 & - 3 \\ 4 & - 1 & 3 & - 1 \end{pmatrix}\); \\

\end{tabular}
\vspace{1cm}


\begin{tabular}{m{17cm}}
\textbf{57-variant}
\newline

\textbf{T1.} Kompleks evklid keńislikleri.  (Kompleks vektorlı keńislik, Ermit kvadratlıq forma.) \\
\textbf{T2.} Túyinles túrlendiriw. ( Evklid keńisligindegi sızıqlı túrlendiriwler menen bisızıqlı formalar arasındaǵı baylanıs, Berilgen túrlendiriwge túyinles túrlendiriwler, Óz-ózine túyinles túrlendiriwler) \\
\textbf{A1.} \(\mathbb{R}^{2}\) keńislikte anıqlangan tómendegi sáwlelendiriw skalyar kóbeyme bolatuģının anıqlań: \((x,y) = x_{1}y_{1} - x_{2}y_{1} - x_{1}y_{2} + x_{2}y_{2}\) \\
\textbf{A2.} Tómendegi kvadratlıq forma oń anıqlangan bolatuģın\(\lambda\) nıń barlıq mánislerin tabıń: \(2x_{1}^{2} + x_{2}^{2} + 3x_{3}^{2} + 2\lambda x_{1}x_{2} + 2x_{1}x_{3}\); \\
\textbf{A3.} Tómendegi sáwlelendiriw\(V = \mathbb{R}^{3}\) keńislikte sızıqlı túrlendiriw boladı: \(A\left( x_{1},x_{2},x_{3} \right) = \left( 2x_{1} + x_{2},x_{1} + x_{3},x_{3}^{2} \right)\); \\
\textbf{B1.} Ortogonallastırıw procesinen paydalanıp, berilgen vektorlar sistemasini ortogonallastirıń: \((1,1,0,0)\), (1, 0, 1, 1); \\
\textbf{B2.} Tómendegi kvadratlıq formanıń kanonikalıq kórinisin hám bul túrge keltiriwshi menshikli mánislerin tabıń: \(3x_{1}^{2} - 2x_{2}^{2} + 2x_{3}^{2} + 4x_{1}x_{2} - 3x_{1}x_{3} - x_{2}x_{3}\); \\
\textbf{B3.} Tómendegi vektorlar sisteması óz ara ortogonallıqqa tekseriń hám olardı ortogonallıq baziske shekem toltırın. \((i,i,1, - 1),\ \ (1, - 1 + i,0,1),\ \ \); \\
\textbf{C1.} Jordan narmal formasini tabıń\(\begin{pmatrix} 4 & - 5 & 2 \\ 0 & - 7 & 3 \\ 0 & 0 & 4 \end{pmatrix}\); \\
\textbf{C2.} Tómendegi kvadratlıq formalarga sáykes keliwshi ermit bisızıqlı formalardı tabiń:\((5 - i)x_{1}\overline{x_{2}} + (5 + i)\overline{x_{1}}x_{2} + x_{2}\overline{x_{2}}\); \\
\textbf{C3.} Ortogonallastırıw procesinen paydalanıp, berilgen vektorlar sistemasini ortogonallastirıń: \((2,1,3, - 1)\), ( \(7,4,3, - 3\) ), ( \(1,1, - 6,0\) ), (5, 7, 7, 8); \\

\end{tabular}
\vspace{1cm}


\begin{tabular}{m{17cm}}
\textbf{58-variant}
\newline

\textbf{T1.} Ortogonal  tolıqtırıwshı. (Ortogonal tolıqtırıwshı,  ortogonal proekciya) \\
\textbf{T2.} Keri túrlendiriwler. ( Keri túrlendiriw túsinigi,   Keri túrlendiriwdiń sızıqlılıǵı) \\
\textbf{A1.} \(\mathbb{R}^{2}\) keńislikte anıqlangan tómendegi sáwlelendiriw skalyar kóbeyme bolatuģının anıqlań: \((x,y) = x_{1}y_{1} - 2x_{2}y_{1} - 2x_{1}y_{2} + x_{2}y_{2}\) \\
\textbf{A2.} Tómendegi kvadratlıq forma oń anıqlangan bolatuģın\(\lambda\) nıń barlıq mánislerin tabıń: \(x_{1}^{2} + 4x_{2}^{2} + x_{3}^{2} + 2\lambda x_{1}x_{2} + 10x_{1}x_{3} + 6x_{2}x_{3}\); \\
\textbf{A3.} Tómendegi sáwlelendiriwlerden qaysıları keńislikte sızıqlı túrlendiriw boladı: 1. \\
\textbf{B1.} \(\mathbb{R}^{2}\) keńislikte \((x,y) = x_{1}y_{1} + 2x_{2}y_{1} + 2x_{1}y_{2} + 5x_{2}y_{2}\) berilgen skalyar kóbeyme ushın \(a = (1,0)\) hám \(b = (0,1)\) vektorlar arasındaǵı múyeshti tabıń \\
\textbf{B2.} Tómendegi kvadratlıq formanı kanonikalıq kóriniske keltiriń: \(12x_{1}^{2} + 3x_{2}^{2} + 12x_{3}^{2} - 12x_{1}x_{2} + 24x_{1}x_{3} - 8x_{2}x_{3}\); \\
\textbf{B3.} Tómendegi vektorlar sisteması óz ara ortogonallıqqa tekseriń hám olardı ortogonallıq baziske shekem toltırın: \((0,i,1,1),\ \ (1,2,1 + i, - 1 + i)\). \\
\textbf{C1.} Jordan normal formasını tabıń \(\begin{pmatrix} 4 & 1 & - 4 \\ 1 & 4 & 0 \\  - 4 & 0 & 4 \end{pmatrix}\); \\
\textbf{C2.} Tómendegi bisiziqli formalar ekvivalent emes ekenligin dálilleń:\(f_{1}(x,y) = 2x_{1}y_{2} - 3x_{1}y_{3} + x_{2}y_{3} - 2x_{2}y_{1} - x_{3}y_{2} - 3x_{3}y_{1}\),\(f_{2}(x,y) = x_{1}y_{2} - x_{2}y_{1} + 2x_{2}y_{2} + 3x_{1}y_{3} - 3x_{3}y_{1};\) \\
\textbf{C3.} Matricası tómendegishe bolǵan sızıqlı túrlendiriwdiń menshikli mánisi hám menshikli vektorların tabıń: \(\begin{pmatrix} 2 & - 1 & 2 \\ 5 & - 3 & 3 \\  - 1 & 0 & - 2 \end{pmatrix}\); \\

\end{tabular}
\vspace{1cm}


\begin{tabular}{m{17cm}}
\textbf{59-variant}
\newline

\textbf{T1.} Sızıqlı, bisızıqlı hám kvadratlıq formalar. (Bisızıqlı forma,  simmetriyalı bisızıqlı formalar)  \\
\textbf{T2.} Óz-ara orın almasıwshı túrlendiriwler. (Óz-ara orın almasıwshı túrlendiriwler,  Ortogonal bazis haqqında teorema,  Normal túrlendiriwlerdiń kanonikalıq kórinisi) \\
\textbf{A1.} Tómendegi kvadratlıq formanıń rangni anıqlań: \(3x_{1}^{2} - 2x_{2}^{2} + 2x_{3}^{2} + 4x_{1}x_{2} - 3x_{1}x_{3} - x_{2}x_{3}\); \\
\textbf{A2.} Tómendegi vektorlar sisteması óz ara ortogonallıqqa tekseriń hám olardı ortogonallıq baziske shekem toltırın: \((1, - 2,2, - 3),(2, - 3,2,4)\); \\
\textbf{A3.} Tómendegi sáwlelendiriwlerden qaysıları keńislikte sızıqlı túrlendiriw boladı:. \\
\textbf{B1.} Ortogonallastırıw procesinen paydalanıp, berilgen vektorlar sistemasini ortogonallastirıń: \((2,0,1,1)\), ( \(1,2,0,1\) ), ( \(0,1, - 2,0\) ); \\
\textbf{B2.} Eger \(f\) sızıqlı funkciya\(e_{1},e_{2},e_{3}\) bazisde \(f(x) = 2x_{1} - 3x_{2} + x_{3}\) arqalı anıqlanǵan bolsa, onıń \(e_{1}^{'},e_{2}^{'},e_{3}^{'}\) bazisdegi kórinisin tabıń\(e_{1}^{'} = e_{1} + 3e_{2} - 2e_{3},\ e_{2}^{'} = 2e_{1} + e_{2} - e_{3},\ e_{3}^{'} = e_{1} + e_{2} - 3e_{3}\); \\
\textbf{B3.} Matricası tómendegishe bolǵan sızıqlı túrlendiriwdiń menshikli mánisi hám menshikli vektorların tabıń: \(\begin{pmatrix} 2 & - 1 & 2 \\ 0 & - 3 & 0 \\ 0 & 0 & 1 \end{pmatrix}\); \\
\textbf{C1.} Berilgen \(A\) bisızıqlı formanıń\(e_{1},e_{2},e_{3}\) bazisdegi matritsası hám\(e_{1}^{'},e_{2}^{'},e_{3}^{'}\) baziske ótiw formulaları berilgen bolsa, onda bul bisızıqli formanıń\(e_{1}^{'},e_{2}^{'},e_{3}^{'}\) bazisdegi matritsasini tabıń: \(\ \) \(\begin{pmatrix} 0 & 2 & 1 \\  - 2 & 2 & 0 \\  - 1 & 0 & 3 \end{pmatrix}\), \(e_{1}^{'} = e_{1} + 2e_{2} - e_{3}\), \(e_{2}^{'} = e_{2} - e_{3}\), \(e_{3}^{'} = - e_{1} + e_{2} - 3e_{3}\) \\
\textbf{C2.} Tómendegi kvadratlıq formalarga sáykes keliwshi ermit bisızıqlı formalardı tabiń. \(x_{1}\overline{x_{1}} - ix_{1}\overline{x_{2}} - ix_{2}\overline{x_{1}} + 2x_{2}\overline{x_{2}}\); \\
\textbf{C3.} Matricası tómendegishe bolǵan sızıqlı túrlendiriwdiń menshikli mánisi hám menshikli vektorların tabıń: \(\begin{pmatrix} 1 & 0 & 0 & 0 \\ 0 & 0 & 0 & 0 \\ 0 & 0 & 0 & 0 \\ 1 & 0 & 0 & 0 \end{pmatrix}\); \\

\end{tabular}
\vspace{1cm}


\begin{tabular}{m{17cm}}
\textbf{60-variant}
\newline

\textbf{T1.} İnertsiya nızamı. (invariantlar,  eki kvadratılıq forma arasindaǵı baylanıs ) \\
\textbf{T2.} Unitar túrlendiriwler. (Unitar sızıqlı túrlendiriwler túsinigi,  Unitar túrlendiriwge túyinles túrlendiriwlerdiń matricası,   Ortonormal baziste unitar túrlendiriwlerdiń matricası) \\
\textbf{A1.} Tómendegi kvadratlıq formanıń rangni anıqlań: \(2x_{1}^{2} + 3x_{2}^{2} + 4x_{3}^{2} - 2x_{1}x_{2} + 4x_{1}x_{3} - 3x_{2}x_{3}\) \\
\textbf{A2.} Tómendegi funkciyalı haqiqiy sanlar maydanı ústinde anıqlangan \(V\) keńislikte sızıqlı funkciya boladı: \(V = \mathbb{R}^{3},\ \ f(x) = |x|\); \\
\textbf{A3.} Tómendegi sáwlelendiriwmos ravishda Berilgen \(V\) vektor keńislikte sızıqlı túrlendiriw boladı: \(V\) sızıqlı keńislik,\(Ax = x + a\), bul jerde \(a\)-fiksirlengen vektor; \\
\textbf{B1.} Ortogonallastırıw procesinen paydalanıp, berilgen vektorlar sistemasini ortogonallastirıń: \((1,0,0)\), (0, 1, -1), (1, 1, 1); \\
\textbf{B2.} Eger \(f\) sızıqlı funkciya\(e_{1},e_{2},e_{3}\) bazisde \(f(x) = 2x_{1} - 3x_{2} + x_{3}\) arqalı anıqlanǵan bolsa, onıń \(e_{1}^{'},e_{2}^{'},e_{3}^{'}\) bazisdegi kórinisin tabıń\(e_{1}^{'} = 4e_{1} - e_{2} - 3e_{3},\ e_{2}^{'} = 2e_{1} + e_{2},\ e_{3}^{'} = 3e_{1} + 2e_{2}\). \\
\textbf{B3.} Tómendegi vektorlar sisteması óz ara ortogonallıqqa tekseriń hám olardı ortogonallıq baziske shekem toltırın: \((1,i, - i),\ \ ( - 2 - i,1 + i,2 - i)\); \\
\textbf{C1.} Berilgen \(A\) bisızıqlı formanıń\(e_{1},e_{2},e_{3}\) bazisdegi matritsası hám\(e_{1}^{'},e_{2}^{'},e_{3}^{'}\) baziske ótiw formulaları berilgen bolsa, onda bul bisızıqli formanıń\(e_{1}^{'},e_{2}^{'},e_{3}^{'}\) bazisdegi matritsasini tabıń: \(\begin{pmatrix} 1 & 2 & 3 \\ 4 & 5 & 6 \\ 7 & 8 & 9 \end{pmatrix}\), \(e_{1}^{'} = e_{1} - e_{2}\), \(e_{2}^{'} = e_{1} + e_{3}\), \(e_{3}^{'} = e_{1} + e_{2} + e_{3}\) \\
\textbf{C2.} Bazi bir ortonormal bazisde berilgen kvadratlıq formani kanonik kóriniske keltiriwshi ortonormal bazisin tabıń: \(x_{1}^{2} + x_{2}^{2} + x_{3}^{2} + 4x_{1}x_{2} + 4x_{1}x_{3} + 4x_{2}x_{3}\); \\
\textbf{C3.} Matricası tómendegishe bolǵan sızıqlı túrlendiriwdiń menshikli mánisi hám menshikli vektorların tabıń: \(\begin{pmatrix} 0 & 2 & 1 \\  - 2 & 0 & 3 \\  - 1 & - 3 & 0 \end{pmatrix}\); \\

\end{tabular}
\vspace{1cm}


\begin{tabular}{m{17cm}}
\textbf{61-variant}
\newline

\textbf{T1.} Sızıqlı keńislikler.   (Vektor,  sızıqlı baylanıs, bazis, ólshem, )  \\
\textbf{T2.} Unitar túrlendiriwler. (Unitar sızıqlı túrlendiriwler túsinigi,  Unitar túrlendiriwge túyinles túrlendiriwlerdiń matricası,   Ortonormal baziste unitar túrlendiriwlerdiń matricası) \\
\textbf{A1.} \(\mathbb{R}^{2}\) keńislikte anıqlangan tómendegi sáwlelendiriw skalyar kóbeyme bolatuģının anıqlań: \((x,y) = x_{1}y_{1} - x_{2}y_{1} - x_{1}y_{2} + x_{2}y_{2}\) \\
\textbf{A2.} Tómendegi funkciyalı haqiqiy sanlar maydanı ústinde anıqlangan \(V\) keńislikte sızıqlı funkciya boladı: \(V = \mathbb{R}^{3},\ \ f(x) = |x|\); \\
\textbf{A3.} Tómendegi sáwlelendiriwlerden qaysıları keńislikte sızıqlı túrlendiriw boladı:; \\
\textbf{B1.} \(\mathbb{R}^{3}\) keńislikte \((x,y) = x_{1}y_{1} + 3x_{2}y_{2} + 2x_{3}y_{3}\) berilgen skalyar kóbeyme ushın \(a = (1, - 3,2)\) va \(b = (2,1, - 1)\) \(b = (0,1)\) vektorlar arasındaǵı múyeshti tabıń . \\
\textbf{B2.} Tómendegi kvadratlıq formanıń kanonikalıq kórinisin hám bul túrge keltiriwshi menshikli mánislerin tabıń: \(5x_{1}^{2} + 6x_{2}^{2} + 4x_{3}^{2} - 4x_{1}x_{2} - 4x_{1}x_{3}\); \\
\textbf{B3.} Matricası tómendegishe bolǵan sızıqlı túrlendiriwdiń menshikli mánisi hám menshikli vektorların tabıń: \(\begin{pmatrix} 7 & 0 & 0 \\ 10 & - 19 & 0 \\ 12 & - 24 & 13 \end{pmatrix}\); \\
\textbf{C1.} Jordan normal formasını tabıń \(\begin{pmatrix} 0 & 0 & 1 \\ 1 & 4 & 0 \\  - 2 & 0 & 2 \end{pmatrix}\); \\
\textbf{C2.} Tómendegi bisiziqli formalar ekvivalent emes ekenligin dálilleń:\(f_{1}(x,y) = 2x_{1}y_{2} - 3x_{1}y_{3} + x_{2}y_{3} - 2x_{2}y_{1} - x_{3}y_{2} - 3x_{3}y_{1}\),\(f_{2}(x,y) = x_{1}y_{2} - x_{2}y_{1} + 2x_{2}y_{2} + 3x_{1}y_{3} - 3x_{3}y_{1};\) \\
\textbf{C3.} Matricası tómendegishe bolǵan sızıqlı túrlendiriwdiń menshikli mánisi hám menshikli vektorların tabıń: \(\begin{pmatrix} 1 & 1 & 1 & 1 \\ 1 & 1 & - 1 & - 1 \\ 1 & - 1 & 1 & - 1 \\ 1 & - 1 & - 1 & 1 \end{pmatrix}\). \\

\end{tabular}
\vspace{1cm}


\begin{tabular}{m{17cm}}
\textbf{62-variant}
\newline

\textbf{T1.} Kvadratlıq forma. (Lagran usulı, Yakobi usulı kvadratlıq formanı keltiriw.) \\
\textbf{T2.} Sızıqlı túrlendiriwler matritsasınıń Jordan normal kórinisi. (Jordan kletkasınıń xarakteristikalıq matricası, Jordan matricasınıń uqsaslıǵı haqqında teorema,  Matricalardı jordan normal kórinisine keltiriw) \\
\textbf{A1.} \(\mathbb{R}^{2}\) keńislikte anıqlangan tómendegi sáwlelendiriw skalyar kóbeyme bolatuģının anıqlań: \((x,y) = x_{1}y_{1} - x_{2}y_{2}\) \\
\textbf{A2.} Tómendegi vektorlar sisteması óz ara ortogonallıqqa tekseriń hám olardı ortogonallıq baziske shekem toltırın: \((1, - 2,2, - 3),(2, - 3,2,4)\); \\
\textbf{A3.} Tómendegi ańlatpalardan qaysıları sáykes túrde berilgen vektor keńislikte sızıqlı túrlendiriw boladı: sızıqlı keńislik,, bul jerde -fiksirlengen vektor; \\
\textbf{B1.} \(\mathbb{R}^{2}\) keńislikte \((x,y) = x_{1}y_{1} + 2x_{2}y_{1} + 2x_{1}y_{2} + 5x_{2}y_{2}\) berilgen skalyar kóbeyme ushın \(a = (1,0)\) hám \(b = (0,1)\) vektorlar arasındaǵı múyeshti tabıń \\
\textbf{B2.} Eger \(f\) sızıqlı funkciya\(e_{1},e_{2},e_{3}\) bazisde \(f(x) = 2x_{1} - 3x_{2} + x_{3}\) arqalı anıqlanǵan bolsa, onıń \(e_{1}^{'},e_{2}^{'},e_{3}^{'}\) bazisdegi kórinisin tabıń\(e_{1}^{'} = e_{1} - e_{2},\ e_{2}^{'} = e_{1} + e_{3},\ \ e_{3}^{'} = e_{1} + e_{2} + e_{3}\); \\
\textbf{B3.} Tómendegi vektorlar sisteması óz ara ortogonallıqqa tekseriń hám olardı ortogonallıq baziske shekem toltırın: \((0,i,1,1),\ \ (1,2,1 + i, - 1 + i)\). \\
\textbf{C1.} Jordan normal formasını tabıń \(\begin{pmatrix} 1 & - 2 & 1 \\  - 2 & 1 & 4 \\  - 1 & 4 & 1 \end{pmatrix}\). \\
\textbf{C2.} Bazi bir ortonormal bazisde berilgen kvadratlıq formani kanonik kóriniske keltiriwshi ortonormal bazisin tabıń: \(x_{1}^{2} + x_{2}^{2} + 5x_{3}^{2} - 6x_{1}x_{2} - 2x_{1}x_{3} + 2x_{2}x_{3}\); \\
\textbf{C3.} Matricası tómendegishe bolǵan sızıqlı túrlendiriwdiń menshikli mánisi hám menshikli vektorların tabıń: \(\begin{pmatrix} 1 & 0 & 0 & 0 \\ 0 & 0 & 0 & 0 \\ 0 & 0 & 0 & 0 \\ 1 & 0 & 0 & 0 \end{pmatrix}\); \\

\end{tabular}
\vspace{1cm}


\begin{tabular}{m{17cm}}
\textbf{63-variant}
\newline

\textbf{T1.} Evklid keńisligi. (Skalyar kóbeyme, ortogonal vektorlar, ortonormal bazis.) \\
\textbf{T2.} Óz-ara orın almasıwshı túrlendiriwler. (Óz-ara orın almasıwshı túrlendiriwler,  Ortogonal bazis haqqında teorema,  Normal túrlendiriwlerdiń kanonikalıq kórinisi) \\
\textbf{A1.} \(\mathbb{R}^{2}\) keńislikte anıqlangan tómendegi sáwlelendiriw skalyar kóbeyme bolatuģının anıqlań: \((x,y) = x_{1}y_{1} + 2x_{2}y_{1} + 2x_{1}y_{2} + 7x_{2}y_{2}\) \\
\textbf{A2.} Tómendegi kvadratlıq forma oń anıqlangan bolatuģın\(\lambda\) nıń barlıq mánislerin tabıń: \(2x_{1}^{2} + 2x_{2}^{2} + x_{3}^{2} + 2\lambda x_{1}x_{2} + 6x_{1}x_{3} + 2x_{2}x_{3}\); \\
\textbf{A3.} Tómendegi sáwlelendiriwmos ravishda Berilgen \(V\) vektor keńislikte sızıqlı túrlendiriw boladı: \(V\) sızıqlı keńislik,\(Ax = \alpha x\) bul jerde \(\alpha\)-fiksirlangan son; \\
\textbf{B1.} Ortogonallastırıw procesinen paydalanıp, berilgen vektorlar sistemasini ortogonallastirıń: \((1,2,2, - 1)\), ( \(1,1, - 5,3\) ), (3, 2, 8, -7); \\
\textbf{B2.} Tómendegi kvadratlıq formanıń kanonikalıq kórinisin hám bul túrge keltiriwshi menshikli mánislerin tabıń: \(x_{1}x_{2} + x_{1}x_{3} + x_{2}x_{3}\); \\
\textbf{B3.} Matricası tómendegishe bolǵan sızıqlı túrlendiriwdiń menshikli mánisi hám menshikli vektorların tabıń: \(\begin{pmatrix} 0 & 0 & 1 \\ 1 & 4 & 0 \\  - 2 & 0 & 2 \end{pmatrix}\); \\
\textbf{C1.} Berilgen \(A\) bisızıqlı formanıń\(e_{1},e_{2},e_{3}\) bazisdegi matritsası hám\(e_{1}^{'},e_{2}^{'},e_{3}^{'}\) baziske ótiw formulaları berilgen bolsa, onda bul bisızıqli formanıń \(e_{1}^{'},e_{2}^{'},e_{3}^{'}\) bazisdegi matritsasini tabıń: \(\begin{pmatrix} 1 & 1 & 2 \\  - 1 & 2 & 1 \\  - 1 & 1 & - 1 \end{pmatrix},\begin{matrix}  & e_{1}^{'} = e_{1} + e_{2} - 2e_{3} \\  & e_{2}^{'} = e_{1} + e_{2} + 2e_{3} \\  & e_{3}^{'} = e_{2} + e_{3} \end{matrix}\) \\
\textbf{C2.} Tómendegi bisiziqli formalar ekvivalent emes ekenligin dálilleń:\(f_{1}(x,y) = x_{1}y_{1} + 2x_{1}y_{2} + 2x_{2}y_{1} + 5x_{2}y_{2} + 6x_{2}y_{3} + 8x_{3}y_{2} + 10x_{3}y_{3}\), \(f_{2}(x,y) = 2x_{1}y_{1} - x_{1}y_{3} + x_{2}y_{2} - x_{3}y_{1} + 5x_{3}y_{3}\). \\
\textbf{C3.} Matricası tómendegishe bolǵan sızıqlı túrlendiriwdiń menshikli mánisi hám menshikli vektorların tabıń: \(\begin{pmatrix} 5 & 6 & - 3 \\  - 1 & 0 & 1 \\ 1 & 2 & - 1 \end{pmatrix}\); \\

\end{tabular}
\vspace{1cm}


\begin{tabular}{m{17cm}}
\textbf{64-variant}
\newline

\textbf{T1.} Evklid keńisligi. (Skalyar kóbeyme, ortogonal vektorlar, ortonormal bazis.) \\
\textbf{T2.} Keri túrlendiriwler. ( Keri túrlendiriw túsinigi,   Keri túrlendiriwdiń sızıqlılıǵı) \\
\textbf{A1.} Tómendegi kvadratlıq formanıń rangni anıqlań: \(x_{1}^{2} - 2x_{2}^{2} - 2x_{3}^{2} - 4x_{1}x_{2} - 4x_{1}x_{3} + 8x_{2}x_{3}\); \\
\textbf{A2.} Tómendegi funkciyalı haqiqiy sanlar maydanı ústinde anıqlangan \(V\) keńislikte sızıqlı funkciya boladı: \(V = M_{n}\left( \mathbb{R} \right),\ \ f(A) = \det(A)\); \\
\textbf{A3.} Matricası tómendegishe bolgan sızıqlı túrlendiriwdin menshikli mánisi hám menshikli vektorların tabıń: \(\begin{pmatrix} 2 & 1 \\ 1 & 2 \end{pmatrix}\); \\
\textbf{B1.} Ortogonallastırıw procesinen paydalanıp, berilgen vektorlar sistemasini ortogonallastirıń: \((2,0,1,1)\), ( \(1,2,0,1\) ), ( \(0,1, - 2,0\) ); \\
\textbf{B2.} Tómendegi kvadratlıq formanı kanonikalıq kóriniske keltiriń: \(2x_{1}^{2} + 18x_{2}^{2} + 8x_{3}^{2} - 12x_{1}x_{2} + 8x_{1}x_{3} - 27x_{2}x_{3}\); \\
\textbf{B3.} Tómendegi vektorlar sisteması óz ara ortogonallıqqa tekseriń hám olardı ortogonallıq baziske shekem toltırın: \((1,1,1,1),(1,1, - 1, - 1),(1, - 1,1, - 1)\); \\
\textbf{C1.} Jordan normal formasını tabıń \(\begin{pmatrix} 7 & 0 & 0 \\ 10 & - 19 & 0 \\ 12 & - 24 & 13 \end{pmatrix}\); \\
\textbf{C2.} Tómendegi kvadratlıq forma oń anıqlangan bolatuģın\(\lambda\) parametrnıń barlıq mánislerin tabıń: \(x_{1}\overline{x_{1}} + ix_{1}\overline{x_{2}} - ix_{2}\overline{x_{1}} + \lambda x_{2}\overline{x_{2}}\); \\
\textbf{C3.} Ortogonallastırıw procesinen paydalanıp, berilgen vektorlar sistemasini ortogonallastirıń: \((1,1, - 1,0)\), \((2,0, - 1,0)\), \((1, - 1,1, - 1)\), (2, \(0,1,1\) ). \\

\end{tabular}
\vspace{1cm}


\begin{tabular}{m{17cm}}
\textbf{65-variant}
\newline

\textbf{T1.} Sızıqlı, bisızıqlı hám kvadratlıq formalar. (Bisızıqlı forma,  simmetriyalı bisızıqlı formalar)  \\
\textbf{T2.} Sızıqlı túrlendiriwler.  (Sızıqlı túrlendiriw túsinigi, Sızıqlı túrlendiriwler ústinde ámeller, Sızıqlı túrlendiriwlerdiń obrazı hám yadrosı.) \\
\textbf{A1.} Tómendegi kvadratlıq formanıń rangni anıqlań: \(3x_{1}^{2} - 2x_{2}^{2} + 2x_{3}^{2} + 4x_{1}x_{2} - 3x_{1}x_{3} - x_{2}x_{3}\); \\
\textbf{A2.} Tómendegi kvadratlıq forma oń anıqlangan bolatuģın\(\lambda\) nıń barlıq mánislerin tabıń: \(2x_{1}^{2} + x_{2}^{2} + 3x_{3}^{2} + 2\lambda x_{1}x_{2} + 2x_{1}x_{3}\); \\
\textbf{A3.} Tómendegi sáwlelendiriwlerden qaysıları keńislikte sızıqlı túrlendiriw boladı:; \\
\textbf{B1.} Ortogonallastırıw procesinen paydalanıp, berilgen vektorlar sistemasini ortogonallastirıń: \((1,1, - 1, - 2)\), \((5,8, - 2, - 3)\), (3, 9, 3, 8); \\
\textbf{B2.} Tómendegi kvadratlıq formanı kanonikalıq kóriniske keltiriń: \(x_{1}x_{2} + x_{1}x_{3} + x_{1}x_{4} + x_{2}x_{3} + x_{2}x_{4} + x_{3}x_{4}\); \\
\textbf{B3.} Tómendegi vektorlar sisteması óz ara ortogonallıqqa tekseriń hám olardı ortogonallıq baziske shekem toltırın: \((0,1,i),\ \ (1 + i,i,1)\); \\
\textbf{C1.} Berilgen \(A\) bisızıqlı formanıń\(e_{1},e_{2},e_{3}\) bazisdegi matritsası hám\(e_{1}^{'},e_{2}^{'},e_{3}^{'}\) baziske ótiw formulaları berilgen bolsa, onda bul bisızıqli formanıń\(e_{1}^{'},e_{2}^{'},e_{3}^{'}\) bazisdegi matritsasini tabıń: \(\begin{pmatrix} 1 & 2 & 3 \\ 4 & 5 & 6 \\ 7 & 8 & 9 \end{pmatrix}\), \(e_{1}^{'} = e_{1} - e_{2}\), \(e_{2}^{'} = e_{1} + e_{3}\), \(e_{3}^{'} = e_{1} + e_{2} + e_{3}\) \\
\textbf{C2.} Tómendegi kvadratlıq forma oń anıqlangan bolatuģın\(\lambda\) parametrnıń barlıq mánislerin tabıń: \(\lambda x_{1}\overline{x_{1}} - ix_{1}\overline{x_{2}} + ix_{2}\overline{x_{1}} + 3x_{2}\overline{x_{2}}\); \\
\textbf{C3.} Matricası tómendegishe bolǵan sızıqlı túrlendiriwdiń menshikli mánisi hám menshikli vektorların tabıń: \(\begin{pmatrix} 1 & - 3 & 4 \\ 4 & - 7 & 8 \\ 6 & - 7 & 7 \end{pmatrix}\); \\

\end{tabular}
\vspace{1cm}


\begin{tabular}{m{17cm}}
\textbf{66-variant}
\newline

\textbf{T1.} Sızıqlı keńislikler.   (Vektor,  sızıqlı baylanıs, bazis, ólshem, )  \\
\textbf{T2.} Túyinles túrlendiriw. ( Evklid keńisligindegi sızıqlı túrlendiriwler menen bisızıqlı formalar arasındaǵı baylanıs, Berilgen túrlendiriwge túyinles túrlendiriwler, Óz-ózine túyinles túrlendiriwler) \\
\textbf{A1.} \(\mathbb{R}^{2}\) keńislikte anıqlangan tómendegi sáwlelendiriw skalyar kóbeyme bolatuģının anıqlań: \((x,y) = x_{1}y_{1} + x_{2}y_{1} + 3x_{1}y_{2} + 2x_{2}y_{2}\) \\
\textbf{A2.} Tómendegi kvadratlıq forma oń anıqlangan bolatuģın\(\lambda\) nıń barlıq mánislerin tabıń: \(5x_{1}^{2} + x_{2}^{2} + \lambda x_{3}^{2} + 4x_{1}x_{2} - 2x_{1}x_{3} - 2x_{2}x_{3}\); \\
\textbf{A3.} Tómendegi sáwlelendiriwmos ravishda Berilgen \(V\) vektor keńislikte sızıqlı túrlendiriw boladı: \(V\) sızıqlı keńislik,\(Ax = x + a\), bul jerde \(a\)-fiksirlengen vektor; \\
\textbf{B1.} Ortogonallastırıw procesinen paydalanıp, berilgen vektorlar sistemasini ortogonallastirıń: \((1,1, - 1,0)\), \((2,0, - 1,0)\), \((1, - 1,1, - 1)\), (2, \(0,1,1\) ). \\
\textbf{B2.} Tómendegi kvadratlıq formanıń kanonikalıq kórinisin hám bul túrge keltiriwshi menshikli mánislerin tabıń: \(x_{1}^{2} - 2x_{2}^{2} - 2x_{3}^{2} - 4x_{1}x_{2} - 4x_{1}x_{3} + 8x_{2}x_{3}\); \\
\textbf{B3.} Tómendegi vektorlar sisteması óz ara ortogonallıqqa tekseriń hám olardı ortogonallıq baziske shekem toltırın: \((1,i, - i),\ \ ( - 2 - i,1 + i,2 - i)\); \\
\textbf{C1.} Jordan normal formasını tabıń \(\begin{pmatrix} 4 & 1 & - 4 \\ 1 & 4 & 0 \\  - 4 & 0 & 4 \end{pmatrix}\); \\
\textbf{C2.} Bazi bir ortonormal bazisde berilgen kvadratlıq formani kanonik kóriniske keltiriwshi ortonormal bazisin tabıń: \(x_{1}^{2} - 5x_{2}^{2} + x_{3}^{2} + 4x_{1}x_{2} + 2x_{1}x_{3} + 4x_{2}x_{3}\); \\
\textbf{C3.} Matricası tómendegishe bolǵan sızıqlı túrlendiriwdiń menshikli mánisi hám menshikli vektorların tabıń: \(\begin{pmatrix} 3 & - 1 & 0 & 0 \\ 1 & 1 & 0 & 0 \\ 3 & 0 & 5 & - 3 \\ 4 & - 1 & 3 & - 1 \end{pmatrix}\); \\

\end{tabular}
\vspace{1cm}


\begin{tabular}{m{17cm}}
\textbf{67-variant}
\newline

\textbf{T1.} Kvadratlıq forma. (Lagran usulı, Yakobi usulı kvadratlıq formanı keltiriw.) \\
\textbf{T2.} Keri túrlendiriwler. ( Keri túrlendiriw túsinigi,   Keri túrlendiriwdiń sızıqlılıǵı) \\
\textbf{A1.} \(\mathbb{R}^{2}\) keńislikte anıqlangan tómendegi sáwlelendiriw skalyar kóbeyme bolatuģının anıqlań: \((x,y) = x_{1}y_{1} + 2x_{2}y_{2}\) \\
\textbf{A2.} Tómendegi kvadratlıq forma oń anıqlangan bolatuģın\(\lambda\) nıń barlıq mánislerin tabıń: \(x_{1}^{2} + x_{2}^{2} + 5x_{3}^{2} + 2\lambda x_{1}x_{2} - 2x_{1}x_{3} + 4x_{2}x_{3}\); \\
\textbf{A3.} Tómendegi sáwlelendiriwlerden qaysıları keńislikte sızıqlı túrlendiriw boladı:. \\
\textbf{B1.} Ortogonallastırıw procesinen paydalanıp, berilgen vektorlar sistemasini ortogonallastirıń: \((1,0,0)\), (0, 1, -1), (1, 1, 1); \\
\textbf{B2.} Tómendegi kvadratlıq formanıń kanonikalıq kórinisin hám bul túrge keltiriwshi menshikli mánislerin tabıń: \(7x_{1}^{2} + 5x_{2}^{2} + 3x_{3}^{2} - 8x_{1}x_{2} + 8x_{2}x_{3}\); \\
\textbf{B3.} Tómendegi vektorlar sisteması óz ara ortogonallıqqa tekseriń hám olardı ortogonallıq baziske shekem toltırın: \((1,2,0, - 1),(3, - 1,1,1),( - 1,2,2,3)\); \\
\textbf{C1.} Jordan narmal formasini tabıń\(\begin{pmatrix} 4 & - 5 & 2 \\ 0 & - 7 & 3 \\ 0 & 0 & 4 \end{pmatrix}\); \\
\textbf{C2.} Tómendegi kvadratlıq formalarga sáykes keliwshi ermit bisızıqlı formalardı tabiń: \(ix_{1}\overline{x_{2}} - ix_{2}\overline{x_{1}} + (3 - 2i)x_{1}\overline{x_{3}} + (3 + 2i)x_{3}\overline{x_{1}} + 2x_{2}\overline{x_{3}} + 2x_{3}\overline{x_{2}}\). \\
\textbf{C3.} Matricası tómendegishe bolǵan sızıqlı túrlendiriwdiń menshikli mánisi hám menshikli vektorların tabıń: \(\begin{pmatrix} 2 & - 1 & 2 \\ 5 & - 3 & 3 \\  - 1 & 0 & - 2 \end{pmatrix}\); \\

\end{tabular}
\vspace{1cm}


\begin{tabular}{m{17cm}}
\textbf{68-variant}
\newline

\textbf{T1.} İnertsiya nızamı. (invariantlar,  eki kvadratılıq forma arasindaǵı baylanıs ) \\
\textbf{T2.} Sızıqlı túrlendiriwler.  (Sızıqlı túrlendiriw túsinigi, Sızıqlı túrlendiriwler ústinde ámeller, Sızıqlı túrlendiriwlerdiń obrazı hám yadrosı.) \\
\textbf{A1.} Tómendegi kvadratlıq formanıń rangni anıqlań: \(x_{1}x_{2} + x_{1}x_{3} + x_{2}x_{3}\); \\
\textbf{A2.} Tómendegi kvadratlıq forma oń anıqlangan bolatuģın\(\lambda\) nıń barlıq mánislerin tabıń: \(x_{1}^{2} + 4x_{2}^{2} + x_{3}^{2} + 2\lambda x_{1}x_{2} + 10x_{1}x_{3} + 6x_{2}x_{3}\); \\
\textbf{A3.} Tómendegi sáwlelendiriw\(V = \mathbb{R}^{3}\) keńislikte sızıqlı túrlendiriw boladı: \(A\left( x_{1},x_{2},x_{3} \right) = \left( x_{2} + x_{3},2x_{1} + x_{3},3x_{1} - x_{2} + x_{3} \right)\); \\
\textbf{B1.} Ortogonallastırıw procesinen paydalanıp, berilgen vektorlar sistemasini ortogonallastirıń: \((1,2,1,3)\), (4, 1, 1, 1), (3, 1, 1, 0); \\
\textbf{B2.} Tómendegi kvadratlıq formanı kanonikalıq kóriniske keltiriń: \(x_{1}x_{2} + x_{1}x_{3} + x_{1}x_{4} + x_{2}x_{3} + x_{2}x_{4} + x_{3}x_{4}\); \\
\textbf{B3.} Matricası tómendegishe bolǵan sızıqlı túrlendiriwdiń menshikli mánisi hám menshikli vektorların tabıń: \(\begin{pmatrix} 2 & - 1 & 2 \\ 0 & - 3 & 0 \\ 0 & 0 & 1 \end{pmatrix}\); \\
\textbf{C1.} Berilgen \(A\) bisızıqlı formanıń\(e_{1},e_{2},e_{3}\) bazisdegi matritsası hám\(e_{1}^{'},e_{2}^{'},e_{3}^{'}\) baziske ótiw formulaları berilgen bolsa, onda bul bisızıqli formanıń\(e_{1}^{'},e_{2}^{'},e_{3}^{'}\) bazisdegi matritsasini tabıń: \(\ \) \(\begin{pmatrix} 0 & 2 & 1 \\  - 2 & 2 & 0 \\  - 1 & 0 & 3 \end{pmatrix}\), \(e_{1}^{'} = e_{1} + 2e_{2} - e_{3}\), \(e_{2}^{'} = e_{2} - e_{3}\), \(e_{3}^{'} = - e_{1} + e_{2} - 3e_{3}\) \\
\textbf{C2.} Tómendegi kvadratlıq formalarga sáykes keliwshi ermit bisızıqlı formalardı tabiń:\((5 - i)x_{1}\overline{x_{2}} + (5 + i)\overline{x_{1}}x_{2} + x_{2}\overline{x_{2}}\); \\
\textbf{C3.} Ortogonallastırıw procesinen paydalanıp, berilgen vektorlar sistemasini ortogonallastirıń: \((2,1,3, - 1)\), ( \(7,4,3, - 3\) ), ( \(1,1, - 6,0\) ), (5, 7, 7, 8); \\

\end{tabular}
\vspace{1cm}


\begin{tabular}{m{17cm}}
\textbf{69-variant}
\newline

\textbf{T1.} Kompleks evklid keńislikleri.  (Kompleks vektorlı keńislik, Ermit kvadratlıq forma.) \\
\textbf{T2.} Túyinles túrlendiriw. ( Evklid keńisligindegi sızıqlı túrlendiriwler menen bisızıqlı formalar arasındaǵı baylanıs, Berilgen túrlendiriwge túyinles túrlendiriwler, Óz-ózine túyinles túrlendiriwler) \\
\textbf{A1.} Tómendegi kvadratlıq formanıń rangni anıqlań: \(x_{1}^{2} + x_{2}^{2} + x_{3}^{2} + x_{4}^{2} + 2x_{1}x_{2} - 2x_{1}x_{4} - 2x_{2}x_{3} + 2x_{3}x_{4}\). \\
\textbf{A2.} Tómendegi kvadratlıq forma oń anıqlangan bolatuģın\(\lambda\) nıń barlıq mánislerin tabıń: \(x_{1}^{2} + \lambda x_{2}^{2} + x_{3}^{2} - 4x_{1}x_{2} - 8x_{2}x_{3}\); \\
\textbf{A3.} Tómendegi sáwlelendiriw\(V = \mathbb{R}^{3}\) keńislikte sızıqlı túrlendiriw boladı: \(A\left( x_{1},x_{2},x_{3} \right) = \left( x_{1},x_{2},x_{1} + x_{2} + x_{3} \right)\); \\
\textbf{B1.} Ortogonallastırıw procesinen paydalanıp, berilgen vektorlar sistemasini ortogonallastirıń: \((1,1,0,0)\), (1, 0, 1, 1); \\
\textbf{B2.} Tómendegi kvadratlıq formanıń kanonikalıq kórinisin hám bul túrge keltiriwshi menshikli mánislerin tabıń: \(5x_{1}^{2} + 6x_{2}^{2} + 4x_{3}^{2} - 4x_{1}x_{2} - 4x_{1}x_{3}\); \\
\textbf{B3.} Tómendegi vektorlar sisteması óz ara ortogonallıqqa tekseriń hám olardı ortogonallıq baziske shekem toltırın. \((i,i,1, - 1),\ \ (1, - 1 + i,0,1),\ \ \); \\
\textbf{C1.} Berilgen \(A\) bisızıqlı formanıń\(e_{1},e_{2},e_{3}\) bazisdegi matritsası hám\(e_{1}^{'},e_{2}^{'},e_{3}^{'}\) baziske ótiw formulaları berilgen bolsa, onda bul bisızıqli formanıń \(e_{1}^{'},e_{2}^{'},e_{3}^{'}\) bazisdegi matritsasini tabıń: \(\begin{pmatrix} 2 & 2 & 3 \\  - 4 & 3 & 1 \\ 3 & 1 & 2 \end{pmatrix},\ \begin{matrix}  & e_{1}^{'} = e_{1} + 3e_{2} - 2e_{3} \\  & e_{2}^{'} = 2e_{1} + e_{2} - e_{3} \\  & e_{3}^{'} = e_{1} + e_{2} - 3e_{3} \end{matrix}\) \\
\textbf{C2.} Tómendegi kvadratlıq formalarga sáykes keliwshi ermit bisızıqlı formalardı tabiń. \(x_{1}\overline{x_{1}} - ix_{1}\overline{x_{2}} - ix_{2}\overline{x_{1}} + 2x_{2}\overline{x_{2}}\); \\
\textbf{C3.} Matricası tómendegishe bolǵan sızıqlı túrlendiriwdiń menshikli mánisi hám menshikli vektorların tabıń: \(\begin{pmatrix} 1 & 0 & 0 & 0 \\ 0 & 0 & 0 & 0 \\ 1 & 0 & 0 & 0 \\ 0 & 0 & 0 & 1 \end{pmatrix}\); \\

\end{tabular}
\vspace{1cm}


\begin{tabular}{m{17cm}}
\textbf{70-variant}
\newline

\textbf{T1.} Ortogonal  tolıqtırıwshı. (Ortogonal tolıqtırıwshı,  ortogonal proekciya) \\
\textbf{T2.} Sızıqlı túrlendiriwler matritsasınıń Jordan normal kórinisi. (Jordan kletkasınıń xarakteristikalıq matricası, Jordan matricasınıń uqsaslıǵı haqqında teorema,  Matricalardı jordan normal kórinisine keltiriw) \\
\textbf{A1.} Tómendegi kvadratlıq formanıń rangni anıqlań: \(x_{1}x_{2} + x_{2}x_{3} + x_{3}x_{4} + x_{1}x_{4}\); \\
\textbf{A2.} Tómendegi vektorlar sisteması óz ara ortogonallıqqa tekseriń hám olardı ortogonallıq baziske shekem toltırın: \((1,2, - 1),(3, - 1,1)\); \\
\textbf{A3.} Matricası tómendegishe: \\
\textbf{B1.} Ortogonallastırıw procesinen paydalanıp, berilgen vektorlar sistemasini ortogonallastirıń: \((1,1, - 1)\), (1, 1,1 ), \((3,2, - 1)\); \\
\textbf{B2.} Tómendegi kvadratlıq formanıń kanonikalıq kórinisin hám bul túrge keltiriwshi menshikli mánislerin tabıń: \(2x_{1}^{2} + 3x_{2}^{2} + 4x_{3}^{2} - 2x_{1}x_{2} + 4x_{1}x_{3} - 3x_{2}x_{3}\); \\
\textbf{B3.} Matricası tómendegishe bolǵan sızıqlı túrlendiriwdiń menshikli mánisi hám menshikli vektorların tabıń: \(\begin{pmatrix} 4 & - 5 & 2 \\ 0 & - 7 & 3 \\ 0 & 0 & 4 \end{pmatrix}\); \\
\textbf{C1.} Jordan normal formasını tabıń \(\begin{pmatrix} 2 & - 1 & 2 \\ 0 & - 3 & 0 \\ 0 & 0 & 1 \end{pmatrix}\); \\
\textbf{C2.} Tómendegi kvadratlıq formalarga sáykes keliwshi ermit bisızıqlı formalardı tabiń: \(x_{1}\overline{x_{1}} + (2 + i)x_{1}\overline{x_{2}} + (2 - i)x_{2}\overline{x_{1}} + ix_{1}\overline{x_{3}} - ix_{3}\overline{x_{1}} - x_{3}\overline{x_{3}}\); \\
\textbf{C3.} Matricası tómendegishe bolǵan sızıqlı túrlendiriwdiń menshikli mánisi hám menshikli vektorların tabıń: \(\begin{pmatrix} 0 & 2 & 1 \\  - 2 & 0 & 3 \\  - 1 & - 3 & 0 \end{pmatrix}\); \\

\end{tabular}
\vspace{1cm}


\begin{tabular}{m{17cm}}
\textbf{71-variant}
\newline

\textbf{T1.} İnertsiya nızamı. (invariantlar,  eki kvadratılıq forma arasindaǵı baylanıs ) \\
\textbf{T2.} Unitar túrlendiriwler. (Unitar sızıqlı túrlendiriwler túsinigi,  Unitar túrlendiriwge túyinles túrlendiriwlerdiń matricası,   Ortonormal baziste unitar túrlendiriwlerdiń matricası) \\
\textbf{A1.} Tómendegi kvadratlıq formanıń rangni anıqlań: \(2x_{1}^{2} + 3x_{2}^{2} + 4x_{3}^{2} - 2x_{1}x_{2} + 4x_{1}x_{3} - 3x_{2}x_{3}\) \\
\textbf{A2.} Tómendegi vektorlar sisteması óz ara ortogonallıqqa tekseriń hám olardı ortogonallıq baziske shekem toltırın: \((2,1,2),\ (1,2, - 2)\); \\
\textbf{A3.} Tómendegi sáwlelendiriw\(V = \mathbb{R}^{3}\) keńislikte sızıqlı túrlendiriw boladı: \(A\left( x_{1},x_{2},x_{3} \right) = \left( 2x_{1} + x_{2},x_{1} + x_{3},x_{3}^{2} \right)\); \\
\textbf{B1.} \(\mathbb{R}^{2}\) keńislikte \((x,y) = x_{1}y_{1} + 2x_{2}y_{1} + 2x_{1}y_{2} + 5x_{2}y_{2}\) berilgen skalyar kóbeyme ushın \(a = (1,0)\) hám \(b = (0,1)\) vektorlar arasındaǵı múyeshti tabıń \\
\textbf{B2.} Eger \(f\) sızıqlı funkciya\(e_{1},e_{2},e_{3}\) bazisde \(f(x) = 2x_{1} - 3x_{2} + x_{3}\) arqalı anıqlanǵan bolsa, onıń \(e_{1}^{'},e_{2}^{'},e_{3}^{'}\) bazisdegi kórinisin tabıń\(e_{1}^{'} = e_{1} + 3e_{2} - 2e_{3},\ e_{2}^{'} = 2e_{1} + e_{2} - e_{3},\ e_{3}^{'} = e_{1} + e_{2} - 3e_{3}\); \\
\textbf{B3.} Matricası tómendegishe bolǵan sızıqlı túrlendiriwdiń menshikli mánisi hám menshikli vektorların tabıń: \(\begin{pmatrix} 2 & - 1 & 2 \\ 0 & - 3 & 0 \\ 0 & 0 & 1 \end{pmatrix}\); \\
\textbf{C1.} Berilgen \(A\) bisızıqlı formanıń\(e_{1},e_{2},e_{3}\) bazisdegi matritsası hám\(e_{1}^{'},e_{2}^{'},e_{3}^{'}\) baziske ótiw formulaları berilgen bolsa, onda bul bisızıqli formanıń \(e_{1}^{'},e_{2}^{'},e_{3}^{'}\) bazisdegi matritsasini tabıń: \(\begin{pmatrix} 1 & 1 & 2 \\  - 1 & 2 & 1 \\  - 1 & 1 & - 1 \end{pmatrix},\begin{matrix}  & e_{1}^{'} = e_{1} + e_{2} - 2e_{3} \\  & e_{2}^{'} = e_{1} + e_{2} + 2e_{3} \\  & e_{3}^{'} = e_{2} + e_{3} \end{matrix}\) \\
\textbf{C2.} Bazi bir ortonormal bazisde berilgen kvadratlıq formani kanonik kóriniske keltiriwshi ortonormal bazisin tabıń: \(x_{1}^{2} + x_{2}^{2} + x_{3}^{2} + 4x_{1}x_{2} + 4x_{1}x_{3} + 4x_{2}x_{3}\); \\
\textbf{C3.} Matricası tómendegishe bolǵan sızıqlı túrlendiriwdiń menshikli mánisi hám menshikli vektorların tabıń: \(\begin{pmatrix} 1 & 0 & 0 & 0 \\ 0 & 0 & 0 & 0 \\ 0 & 0 & 0 & 0 \\ 1 & 0 & 0 & 0 \end{pmatrix}\); \\

\end{tabular}
\vspace{1cm}


\begin{tabular}{m{17cm}}
\textbf{72-variant}
\newline

\textbf{T1.} Kompleks evklid keńislikleri.  (Kompleks vektorlı keńislik, Ermit kvadratlıq forma.) \\
\textbf{T2.} Óz-ara orın almasıwshı túrlendiriwler. (Óz-ara orın almasıwshı túrlendiriwler,  Ortogonal bazis haqqında teorema,  Normal túrlendiriwlerdiń kanonikalıq kórinisi) \\
\textbf{A1.} \(\mathbb{R}^{2}\) keńislikte anıqlangan tómendegi sáwlelendiriw skalyar kóbeyme bolatuģının anıqlań: \((x,y) = x_{1}y_{1} - 2x_{2}y_{1} - 2x_{1}y_{2} + x_{2}y_{2}\) \\
\textbf{A2.} Tómendegi vektorlar sisteması óz ara ortogonallıqqa tekseriń hám olardı ortogonallıq baziske shekem toltırın: \((1,1,1,2)\), \((1,2,3, - 3)\). \\
\textbf{A3.} Tómendegi sáwlelendiriw\(V = \mathbb{R}^{3}\) keńislikte sızıqlı túrlendiriw boladı: \(A\left( x_{1},x_{2},x_{3} \right) = \left( x_{1},x_{2} + 1,x_{3} + 2 \right)\); \\
\textbf{B1.} Ortogonallastırıw procesinen paydalanıp, berilgen vektorlar sistemasini ortogonallastirıń: \((1,1, - 1)\), (1, 1,1 ), \((3,2, - 1)\); \\
\textbf{B2.} Tómendegi kvadratlıq formanı kanonikalıq kóriniske keltiriń: \(12x_{1}^{2} + 3x_{2}^{2} + 12x_{3}^{2} - 12x_{1}x_{2} + 24x_{1}x_{3} - 8x_{2}x_{3}\); \\
\textbf{B3.} Tómendegi vektorlar sisteması óz ara ortogonallıqqa tekseriń hám olardı ortogonallıq baziske shekem toltırın: \((1,i, - i),\ \ ( - 2 - i,1 + i,2 - i)\); \\
\textbf{C1.} Berilgen \(A\) bisızıqlı formanıń\(e_{1},e_{2},e_{3}\) bazisdegi matritsası hám\(e_{1}^{'},e_{2}^{'},e_{3}^{'}\) baziske ótiw formulaları berilgen bolsa, onda bul bisızıqli formanıń\(e_{1}^{'},e_{2}^{'},e_{3}^{'}\) bazisdegi matritsasini tabıń: \(\begin{pmatrix} 1 & 2 & 3 \\ 4 & 5 & 6 \\ 7 & 8 & 9 \end{pmatrix}\), \(e_{1}^{'} = e_{1} - e_{2}\), \(e_{2}^{'} = e_{1} + e_{3}\), \(e_{3}^{'} = e_{1} + e_{2} + e_{3}\) \\
\textbf{C2.} Bazi bir ortonormal bazisde berilgen kvadratlıq formani kanonik kóriniske keltiriwshi ortonormal bazisin tabıń: \(11x_{1}^{2} + 5x_{2}^{2} + 2x_{3}^{2} + 16x_{1}x_{2} + 4x_{1}x_{3} - 20x_{2}x_{3}\); \\
\textbf{C3.} Matricası tómendegishe bolǵan sızıqlı túrlendiriwdiń menshikli mánisi hám menshikli vektorların tabıń: \(\begin{pmatrix} 2 & - 1 & 2 \\ 5 & - 3 & 3 \\  - 1 & 0 & - 2 \end{pmatrix}\); \\

\end{tabular}
\vspace{1cm}


\begin{tabular}{m{17cm}}
\textbf{73-variant}
\newline

\textbf{T1.} Sızıqlı, bisızıqlı hám kvadratlıq formalar. (Bisızıqlı forma,  simmetriyalı bisızıqlı formalar)  \\
\textbf{T2.} Óz-ara orın almasıwshı túrlendiriwler. (Óz-ara orın almasıwshı túrlendiriwler,  Ortogonal bazis haqqında teorema,  Normal túrlendiriwlerdiń kanonikalıq kórinisi) \\
\textbf{A1.} \(\mathbb{R}^{2}\) keńislikte anıqlangan tómendegi sáwlelendiriw skalyar kóbeyme bolatuģının anıqlań: \((x,y) = x_{1}y_{1} - 2x_{2}y_{1} - 2x_{1}y_{2} + x_{2}y_{2}\) \\
\textbf{A2.} Tómendegi vektorlar sisteması óz ara ortogonallıqqa tekseriń hám olardı ortogonallıq baziske shekem toltırın: \((1, - 2,2, - 3),(2, - 3,2,4)\); \\
\textbf{A3.} Tómendegi sáwlelendiriwlerden qaysıları keńislikte sızıqlı túrlendiriw boladı:; \\
\textbf{B1.} Ortogonallastırıw procesinen paydalanıp, berilgen vektorlar sistemasini ortogonallastirıń: \((1,2,2, - 1)\), ( \(1,1, - 5,3\) ), (3, 2, 8, -7); \\
\textbf{B2.} Eger \(f\) sızıqlı funkciya\(e_{1},e_{2},e_{3}\) bazisde \(f(x) = 2x_{1} - 3x_{2} + x_{3}\) arqalı anıqlanǵan bolsa, onıń \(e_{1}^{'},e_{2}^{'},e_{3}^{'}\) bazisdegi kórinisin tabıń\(e_{1}^{'} = 4e_{1} - e_{2} - 3e_{3},\ e_{2}^{'} = 2e_{1} + e_{2},\ e_{3}^{'} = 3e_{1} + 2e_{2}\). \\
\textbf{B3.} Matricası tómendegishe bolǵan sızıqlı túrlendiriwdiń menshikli mánisi hám menshikli vektorların tabıń: \(\begin{pmatrix} 0 & 0 & 1 \\ 1 & 4 & 0 \\  - 2 & 0 & 2 \end{pmatrix}\); \\
\textbf{C1.} Jordan normal formasını tabıń \(\begin{pmatrix} 1 & - 2 & 1 \\  - 2 & 1 & 4 \\  - 1 & 4 & 1 \end{pmatrix}\). \\
\textbf{C2.} Tómendegi kvadratlıq formalarga sáykes keliwshi ermit bisızıqlı formalardı tabiń: \(ix_{1}\overline{x_{2}} - ix_{2}\overline{x_{1}} + (3 - 2i)x_{1}\overline{x_{3}} + (3 + 2i)x_{3}\overline{x_{1}} + 2x_{2}\overline{x_{3}} + 2x_{3}\overline{x_{2}}\). \\
\textbf{C3.} Ortogonallastırıw procesinen paydalanıp, berilgen vektorlar sistemasini ortogonallastirıń: \((2,1,3, - 1)\), ( \(7,4,3, - 3\) ), ( \(1,1, - 6,0\) ), (5, 7, 7, 8); \\

\end{tabular}
\vspace{1cm}


\begin{tabular}{m{17cm}}
\textbf{74-variant}
\newline

\textbf{T1.} Evklid keńisligi. (Skalyar kóbeyme, ortogonal vektorlar, ortonormal bazis.) \\
\textbf{T2.} Sızıqlı túrlendiriwler matritsasınıń Jordan normal kórinisi. (Jordan kletkasınıń xarakteristikalıq matricası, Jordan matricasınıń uqsaslıǵı haqqında teorema,  Matricalardı jordan normal kórinisine keltiriw) \\
\textbf{A1.} Tómendegi kvadratlıq formanıń rangni anıqlań: \(x_{1}x_{2} + x_{1}x_{3} + x_{2}x_{3}\); \\
\textbf{A2.} Tómendegi kvadratlıq forma oń anıqlangan bolatuģın\(\lambda\) nıń barlıq mánislerin tabıń: \(x_{1}^{2} + x_{2}^{2} + 5x_{3}^{2} + 2\lambda x_{1}x_{2} - 2x_{1}x_{3} + 4x_{2}x_{3}\); \\
\textbf{A3.} Tómendegi sáwlelendiriw\(V = \mathbb{R}^{3}\) keńislikte sızıqlı túrlendiriw boladı: \(A\left( x_{1},x_{2},x_{3} \right) = \left( x_{1} + 2,x_{2} + 5,x_{3} \right)\); \\
\textbf{B1.} Ortogonallastırıw procesinen paydalanıp, berilgen vektorlar sistemasini ortogonallastirıń: \((1,0,0)\), (0, 1, -1), (1, 1, 1); \\
\textbf{B2.} Tómendegi kvadratlıq formanı kanonikalıq kóriniske keltiriń: \(2x_{1}^{2} + 18x_{2}^{2} + 8x_{3}^{2} - 12x_{1}x_{2} + 8x_{1}x_{3} - 27x_{2}x_{3}\); \\
\textbf{B3.} Matricası tómendegishe bolǵan sızıqlı túrlendiriwdiń menshikli mánisi hám menshikli vektorların tabıń: \(\begin{pmatrix} 4 & - 5 & 2 \\ 0 & - 7 & 3 \\ 0 & 0 & 4 \end{pmatrix}\); \\
\textbf{C1.} Berilgen \(A\) bisızıqlı formanıń\(e_{1},e_{2},e_{3}\) bazisdegi matritsası hám\(e_{1}^{'},e_{2}^{'},e_{3}^{'}\) baziske ótiw formulaları berilgen bolsa, onda bul bisızıqli formanıń \(e_{1}^{'},e_{2}^{'},e_{3}^{'}\) bazisdegi matritsasini tabıń: \(\begin{pmatrix} 2 & 2 & 3 \\  - 4 & 3 & 1 \\ 3 & 1 & 2 \end{pmatrix},\ \begin{matrix}  & e_{1}^{'} = e_{1} + 3e_{2} - 2e_{3} \\  & e_{2}^{'} = 2e_{1} + e_{2} - e_{3} \\  & e_{3}^{'} = e_{1} + e_{2} - 3e_{3} \end{matrix}\) \\
\textbf{C2.} Bazi bir ortonormal bazisde berilgen kvadratlıq formani kanonik kóriniske keltiriwshi ortonormal bazisin tabıń: \(x_{1}^{2} + x_{2}^{2} + 5x_{3}^{2} - 6x_{1}x_{2} - 2x_{1}x_{3} + 2x_{2}x_{3}\); \\
\textbf{C3.} Ortogonallastırıw procesinen paydalanıp, berilgen vektorlar sistemasini ortogonallastirıń: \((1,1, - 1,0)\), \((2,0, - 1,0)\), \((1, - 1,1, - 1)\), (2, \(0,1,1\) ). \\

\end{tabular}
\vspace{1cm}


\begin{tabular}{m{17cm}}
\textbf{75-variant}
\newline

\textbf{T1.} Ortogonal  tolıqtırıwshı. (Ortogonal tolıqtırıwshı,  ortogonal proekciya) \\
\textbf{T2.} Túyinles túrlendiriw. ( Evklid keńisligindegi sızıqlı túrlendiriwler menen bisızıqlı formalar arasındaǵı baylanıs, Berilgen túrlendiriwge túyinles túrlendiriwler, Óz-ózine túyinles túrlendiriwler) \\
\textbf{A1.} \(\mathbb{R}^{2}\) keńislikte anıqlangan tómendegi sáwlelendiriw skalyar kóbeyme bolatuģının anıqlań: \((x,y) = x_{1}y_{1} - x_{2}y_{1} - x_{1}y_{2} + x_{2}y_{2}\) \\
\textbf{A2.} Tómendegi funkciyalı haqiqiy sanlar maydanı ústinde anıqlangan \(V\) keńislikte sızıqlı funkciya boladı: \(V = \mathbb{R}^{3},\ \ f(x) = |x|\); \\
\textbf{A3.} Tómendegi sáwlelendiriwmos ravishda Berilgen \(V\) vektor keńislikte sızıqlı túrlendiriw boladı: \(V\) sızıqlı keńislik,\(Ax = \alpha x\) bul jerde \(\alpha\)-fiksirlangan son; \\
\textbf{B1.} Ortogonallastırıw procesinen paydalanıp, berilgen vektorlar sistemasini ortogonallastirıń: \((1,1, - 1, - 2)\), \((5,8, - 2, - 3)\), (3, 9, 3, 8); \\
\textbf{B2.} Tómendegi kvadratlıq formanıń kanonikalıq kórinisin hám bul túrge keltiriwshi menshikli mánislerin tabıń: \(3x_{1}^{2} - 2x_{2}^{2} + 2x_{3}^{2} + 4x_{1}x_{2} - 3x_{1}x_{3} - x_{2}x_{3}\); \\
\textbf{B3.} Tómendegi vektorlar sisteması óz ara ortogonallıqqa tekseriń hám olardı ortogonallıq baziske shekem toltırın: \((1,2,0, - 1),(3, - 1,1,1),( - 1,2,2,3)\); \\
\textbf{C1.} Jordan normal formasını tabıń \(\begin{pmatrix} 2 & - 1 & 2 \\ 0 & - 3 & 0 \\ 0 & 0 & 1 \end{pmatrix}\); \\
\textbf{C2.} Tómendegi kvadratlıq forma oń anıqlangan bolatuģın\(\lambda\) parametrnıń barlıq mánislerin tabıń: \(\lambda x_{1}\overline{x_{1}} - ix_{1}\overline{x_{2}} + ix_{2}\overline{x_{1}} + 3x_{2}\overline{x_{2}}\); \\
\textbf{C3.} Matricası tómendegishe bolǵan sızıqlı túrlendiriwdiń menshikli mánisi hám menshikli vektorların tabıń: \(\begin{pmatrix} 1 & - 3 & 4 \\ 4 & - 7 & 8 \\ 6 & - 7 & 7 \end{pmatrix}\); \\

\end{tabular}
\vspace{1cm}


\begin{tabular}{m{17cm}}
\textbf{76-variant}
\newline

\textbf{T1.} Kvadratlıq forma. (Lagran usulı, Yakobi usulı kvadratlıq formanı keltiriw.) \\
\textbf{T2.} Sızıqlı túrlendiriwler.  (Sızıqlı túrlendiriw túsinigi, Sızıqlı túrlendiriwler ústinde ámeller, Sızıqlı túrlendiriwlerdiń obrazı hám yadrosı.) \\
\textbf{A1.} \(\mathbb{R}^{2}\) keńislikte anıqlangan tómendegi sáwlelendiriw skalyar kóbeyme bolatuģının anıqlań: \((x,y) = x_{1}y_{1} + 2x_{2}y_{1} + 2x_{1}y_{2} + 7x_{2}y_{2}\) \\
\textbf{A2.} Tómendegi kvadratlıq forma oń anıqlangan bolatuģın\(\lambda\) nıń barlıq mánislerin tabıń: \(2x_{1}^{2} + x_{2}^{2} + 3x_{3}^{2} + 2\lambda x_{1}x_{2} + 2x_{1}x_{3}\); \\
\textbf{A3.} Tómendegi sáwlelendiriwlerden qaysıları sáykes túrde berilgen vektor keńislikte sızıqlı túrlendiriw boladı: sızıqlı keńislik, bul jerde -fiksirlengen san; \\
\textbf{B1.} Ortogonallastırıw procesinen paydalanıp, berilgen vektorlar sistemasini ortogonallastirıń: \((1,1, - 1,0)\), \((2,0, - 1,0)\), \((1, - 1,1, - 1)\), (2, \(0,1,1\) ). \\
\textbf{B2.} Eger \(f\) sızıqlı funkciya\(e_{1},e_{2},e_{3}\) bazisde \(f(x) = 2x_{1} - 3x_{2} + x_{3}\) arqalı anıqlanǵan bolsa, onıń \(e_{1}^{'},e_{2}^{'},e_{3}^{'}\) bazisdegi kórinisin tabıń\(e_{1}^{'} = e_{1} - e_{2},\ e_{2}^{'} = e_{1} + e_{3},\ \ e_{3}^{'} = e_{1} + e_{2} + e_{3}\); \\
\textbf{B3.} Tómendegi vektorlar sisteması óz ara ortogonallıqqa tekseriń hám olardı ortogonallıq baziske shekem toltırın: \((0,1,i),\ \ (1 + i,i,1)\); \\
\textbf{C1.} Berilgen \(A\) bisızıqlı formanıń\(e_{1},e_{2},e_{3}\) bazisdegi matritsası hám\(e_{1}^{'},e_{2}^{'},e_{3}^{'}\) baziske ótiw formulaları berilgen bolsa, onda bul bisızıqli formanıń\(e_{1}^{'},e_{2}^{'},e_{3}^{'}\) bazisdegi matritsasini tabıń: \(\ \) \(\begin{pmatrix} 0 & 2 & 1 \\  - 2 & 2 & 0 \\  - 1 & 0 & 3 \end{pmatrix}\), \(e_{1}^{'} = e_{1} + 2e_{2} - e_{3}\), \(e_{2}^{'} = e_{2} - e_{3}\), \(e_{3}^{'} = - e_{1} + e_{2} - 3e_{3}\) \\
\textbf{C2.} Tómendegi kvadratlıq formalarga sáykes keliwshi ermit bisızıqlı formalardı tabiń: \(x_{1}\overline{x_{1}} + (2 + i)x_{1}\overline{x_{2}} + (2 - i)x_{2}\overline{x_{1}} + ix_{1}\overline{x_{3}} - ix_{3}\overline{x_{1}} - x_{3}\overline{x_{3}}\); \\
\textbf{C3.} Matricası tómendegishe bolǵan sızıqlı túrlendiriwdiń menshikli mánisi hám menshikli vektorların tabıń: \(\begin{pmatrix} 1 & 0 & 0 & 0 \\ 0 & 0 & 0 & 0 \\ 1 & 0 & 0 & 0 \\ 0 & 0 & 0 & 1 \end{pmatrix}\); \\

\end{tabular}
\vspace{1cm}


\begin{tabular}{m{17cm}}
\textbf{77-variant}
\newline

\textbf{T1.} Sızıqlı keńislikler.   (Vektor,  sızıqlı baylanıs, bazis, ólshem, )  \\
\textbf{T2.} Keri túrlendiriwler. ( Keri túrlendiriw túsinigi,   Keri túrlendiriwdiń sızıqlılıǵı) \\
\textbf{A1.} \(\mathbb{R}^{2}\) keńislikte anıqlangan tómendegi sáwlelendiriw skalyar kóbeyme bolatuģının anıqlań: \((x,y) = x_{1}y_{1} + 2x_{2}y_{2}\) \\
\textbf{A2.} Tómendegi vektorlar sisteması óz ara ortogonallıqqa tekseriń hám olardı ortogonallıq baziske shekem toltırın: \((1,1,1,2)\), \((1,2,3, - 3)\). \\
\textbf{A3.} Tómendegi sáwlelendiriwlerden qaysıları keńislikte sızıqlı túrlendiriw boladı:; \\
\textbf{B1.} \(\mathbb{R}^{3}\) keńislikte \((x,y) = x_{1}y_{1} + 3x_{2}y_{2} + 2x_{3}y_{3}\) berilgen skalyar kóbeyme ushın \(a = (1, - 3,2)\) va \(b = (2,1, - 1)\) \(b = (0,1)\) vektorlar arasındaǵı múyeshti tabıń . \\
\textbf{B2.} Eger \(f\) sızıqlı funkciya\(e_{1},e_{2},e_{3}\) bazisde \(f(x) = 2x_{1} - 3x_{2} + x_{3}\) arqalı anıqlanǵan bolsa, onıń \(e_{1}^{'},e_{2}^{'},e_{3}^{'}\) bazisdegi kórinisin tabıń\(e_{1}^{'} = e_{1} + e_{2} - 2e_{3},\ e_{2}^{'} = e_{1} + e_{2} + 2e_{3},\ e_{3}^{'} = e_{2} + e_{3}\); \\
\textbf{B3.} Matricası tómendegishe bolǵan sızıqlı túrlendiriwdiń menshikli mánisi hám menshikli vektorların tabıń: \(\begin{pmatrix} 7 & 0 & 0 \\ 10 & - 19 & 0 \\ 12 & - 24 & 13 \end{pmatrix}\); \\
\textbf{C1.} Jordan normal formasını tabıń \(\begin{pmatrix} 4 & 1 & - 4 \\ 1 & 4 & 0 \\  - 4 & 0 & 4 \end{pmatrix}\); \\
\textbf{C2.} Tómendegi kvadratlıq formalarga sáykes keliwshi ermit bisızıqlı formalardı tabiń. \(x_{1}\overline{x_{1}} - ix_{1}\overline{x_{2}} - ix_{2}\overline{x_{1}} + 2x_{2}\overline{x_{2}}\); \\
\textbf{C3.} Matricası tómendegishe bolǵan sızıqlı túrlendiriwdiń menshikli mánisi hám menshikli vektorların tabıń: \(\begin{pmatrix} 1 & 1 & 1 & 1 \\ 1 & 1 & - 1 & - 1 \\ 1 & - 1 & 1 & - 1 \\ 1 & - 1 & - 1 & 1 \end{pmatrix}\). \\

\end{tabular}
\vspace{1cm}


\begin{tabular}{m{17cm}}
\textbf{78-variant}
\newline

\textbf{T1.} Sızıqlı, bisızıqlı hám kvadratlıq formalar. (Bisızıqlı forma,  simmetriyalı bisızıqlı formalar)  \\
\textbf{T2.} Unitar túrlendiriwler. (Unitar sızıqlı túrlendiriwler túsinigi,  Unitar túrlendiriwge túyinles túrlendiriwlerdiń matricası,   Ortonormal baziste unitar túrlendiriwlerdiń matricası) \\
\textbf{A1.} \(\mathbb{R}^{2}\) keńislikte anıqlangan tómendegi sáwlelendiriw skalyar kóbeyme bolatuģının anıqlań: \((x,y) = x_{1}y_{1} - x_{2}y_{2}\) \\
\textbf{A2.} Tómendegi kvadratlıq forma oń anıqlangan bolatuģın\(\lambda\) nıń barlıq mánislerin tabıń: \(2x_{1}^{2} + 2x_{2}^{2} + x_{3}^{2} + 2\lambda x_{1}x_{2} + 6x_{1}x_{3} + 2x_{2}x_{3}\); \\
\textbf{A3.} Tómendegi sáwlelendiriwlerden qaysıları keńislikte sızıqlı túrlendiriw boladı: 1. \\
\textbf{B1.} Ortogonallastırıw procesinen paydalanıp, berilgen vektorlar sistemasini ortogonallastirıń: \((1,1,0,0)\), (1, 0, 1, 1); \\
\textbf{B2.} Tómendegi kvadratlıq formanıń kanonikalıq kórinisin hám bul túrge keltiriwshi menshikli mánislerin tabıń: \(x_{1}x_{2} + x_{1}x_{3} + x_{2}x_{3}\); \\
\textbf{B3.} Tómendegi vektorlar sisteması óz ara ortogonallıqqa tekseriń hám olardı ortogonallıq baziske shekem toltırın: \((0,i,1,1),\ \ (1,2,1 + i, - 1 + i)\). \\
\textbf{C1.} Jordan normal formasını tabıń \(\begin{pmatrix} 0 & 0 & 1 \\ 1 & 4 & 0 \\  - 2 & 0 & 2 \end{pmatrix}\); \\
\textbf{C2.} Tómendegi kvadratlıq forma oń anıqlangan bolatuģın\(\lambda\) parametrnıń barlıq mánislerin tabıń: \(x_{1}\overline{x_{1}} + ix_{1}\overline{x_{2}} - ix_{2}\overline{x_{1}} + \lambda x_{2}\overline{x_{2}}\); \\
\textbf{C3.} Matricası tómendegishe bolǵan sızıqlı túrlendiriwdiń menshikli mánisi hám menshikli vektorların tabıń: \(\begin{pmatrix} 5 & 6 & - 3 \\  - 1 & 0 & 1 \\ 1 & 2 & - 1 \end{pmatrix}\); \\

\end{tabular}
\vspace{1cm}


\begin{tabular}{m{17cm}}
\textbf{79-variant}
\newline

\textbf{T1.} Evklid keńisligi. (Skalyar kóbeyme, ortogonal vektorlar, ortonormal bazis.) \\
\textbf{T2.} Keri túrlendiriwler. ( Keri túrlendiriw túsinigi,   Keri túrlendiriwdiń sızıqlılıǵı) \\
\textbf{A1.} Tómendegi kvadratlıq formanıń rangni anıqlań: \(x_{1}^{2} + x_{2}^{2} + x_{3}^{2} + x_{4}^{2} + 2x_{1}x_{2} - 2x_{1}x_{4} - 2x_{2}x_{3} + 2x_{3}x_{4}\). \\
\textbf{A2.} Tómendegi kvadratlıq forma oń anıqlangan bolatuģın\(\lambda\) nıń barlıq mánislerin tabıń: \(5x_{1}^{2} + x_{2}^{2} + \lambda x_{3}^{2} + 4x_{1}x_{2} - 2x_{1}x_{3} - 2x_{2}x_{3}\); \\
\textbf{A3.} Matricası tómendegishe bolgan sızıqlı túrlendiriwdin menshikli mánisi hám menshikli vektorların tabıń \(\begin{pmatrix} 3 & 4 \\ 5 & 2 \end{pmatrix}\); \\
\textbf{B1.} Ortogonallastırıw procesinen paydalanıp, berilgen vektorlar sistemasini ortogonallastirıń: \((1,2,1,3)\), (4, 1, 1, 1), (3, 1, 1, 0); \\
\textbf{B2.} Tómendegi kvadratlıq formanıń kanonikalıq kórinisin hám bul túrge keltiriwshi menshikli mánislerin tabıń: \(7x_{1}^{2} + 5x_{2}^{2} + 3x_{3}^{2} - 8x_{1}x_{2} + 8x_{2}x_{3}\); \\
\textbf{B3.} Tómendegi vektorlar sisteması óz ara ortogonallıqqa tekseriń hám olardı ortogonallıq baziske shekem toltırın. \((i,i,1, - 1),\ \ (1, - 1 + i,0,1),\ \ \); \\
\textbf{C1.} Jordan normal formasını tabıń \(\begin{pmatrix} 7 & 0 & 0 \\ 10 & - 19 & 0 \\ 12 & - 24 & 13 \end{pmatrix}\); \\
\textbf{C2.} Tómendegi bisiziqli formalar ekvivalent emes ekenligin dálilleń:\(f_{1}(x,y) = 2x_{1}y_{2} - 3x_{1}y_{3} + x_{2}y_{3} - 2x_{2}y_{1} - x_{3}y_{2} - 3x_{3}y_{1}\),\(f_{2}(x,y) = x_{1}y_{2} - x_{2}y_{1} + 2x_{2}y_{2} + 3x_{1}y_{3} - 3x_{3}y_{1};\) \\
\textbf{C3.} Matricası tómendegishe bolǵan sızıqlı túrlendiriwdiń menshikli mánisi hám menshikli vektorların tabıń: \(\begin{pmatrix} 3 & - 1 & 0 & 0 \\ 1 & 1 & 0 & 0 \\ 3 & 0 & 5 & - 3 \\ 4 & - 1 & 3 & - 1 \end{pmatrix}\); \\

\end{tabular}
\vspace{1cm}


\begin{tabular}{m{17cm}}
\textbf{80-variant}
\newline

\textbf{T1.} Kompleks evklid keńislikleri.  (Kompleks vektorlı keńislik, Ermit kvadratlıq forma.) \\
\textbf{T2.} Túyinles túrlendiriw. ( Evklid keńisligindegi sızıqlı túrlendiriwler menen bisızıqlı formalar arasındaǵı baylanıs, Berilgen túrlendiriwge túyinles túrlendiriwler, Óz-ózine túyinles túrlendiriwler) \\
\textbf{A1.} Tómendegi kvadratlıq formanıń rangni anıqlań: \(x_{1}x_{2} + x_{2}x_{3} + x_{3}x_{4} + x_{1}x_{4}\); \\
\textbf{A2.} Tómendegi vektorlar sisteması óz ara ortogonallıqqa tekseriń hám olardı ortogonallıq baziske shekem toltırın: \((2,1,2),\ (1,2, - 2)\); \\
\textbf{A3.} Tómendegi sáwlelendiriw\(V = \mathbb{R}^{3}\) keńislikte sızıqlı túrlendiriw boladı: \(A\left( x_{1},x_{2},x_{3} \right) = \left( x_{1} + 3x_{3},x_{2}^{3},x_{1} + x_{3} \right)\). \\
\textbf{B1.} Ortogonallastırıw procesinen paydalanıp, berilgen vektorlar sistemasini ortogonallastirıń: \((2,0,1,1)\), ( \(1,2,0,1\) ), ( \(0,1, - 2,0\) ); \\
\textbf{B2.} Eger \(f\) sızıqlı funkciya\(e_{1},e_{2},e_{3}\) bazisde \(f(x) = 2x_{1} - 3x_{2} + x_{3}\) arqalı anıqlanǵan bolsa, onıń \(e_{1}^{'},e_{2}^{'},e_{3}^{'}\) bazisdegi kórinisin tabıń\(e_{1}^{'} = e_{1} + 3e_{2} - 2e_{3},\ e_{2}^{'} = 2e_{1} + e_{2} - e_{3},\ e_{3}^{'} = e_{1} + e_{2} - 3e_{3}\); \\
\textbf{B3.} Tómendegi vektorlar sisteması óz ara ortogonallıqqa tekseriń hám olardı ortogonallıq baziske shekem toltırın: \((1,1,1,1),(1,1, - 1, - 1),(1, - 1,1, - 1)\); \\
\textbf{C1.} Jordan narmal formasini tabıń\(\begin{pmatrix} 4 & - 5 & 2 \\ 0 & - 7 & 3 \\ 0 & 0 & 4 \end{pmatrix}\); \\
\textbf{C2.} Tómendegi bisiziqli formalar ekvivalent emes ekenligin dálilleń:\(f_{1}(x,y) = x_{1}y_{1} + 2x_{1}y_{2} + 2x_{2}y_{1} + 5x_{2}y_{2} + 6x_{2}y_{3} + 8x_{3}y_{2} + 10x_{3}y_{3}\), \(f_{2}(x,y) = 2x_{1}y_{1} - x_{1}y_{3} + x_{2}y_{2} - x_{3}y_{1} + 5x_{3}y_{3}\). \\
\textbf{C3.} Matricası tómendegishe bolǵan sızıqlı túrlendiriwdiń menshikli mánisi hám menshikli vektorların tabıń: \(\begin{pmatrix} 0 & 2 & 1 \\  - 2 & 0 & 3 \\  - 1 & - 3 & 0 \end{pmatrix}\); \\

\end{tabular}
\vspace{1cm}


\begin{tabular}{m{17cm}}
\textbf{81-variant}
\newline

\textbf{T1.} Sızıqlı keńislikler.   (Vektor,  sızıqlı baylanıs, bazis, ólshem, )  \\
\textbf{T2.} Sızıqlı túrlendiriwler matritsasınıń Jordan normal kórinisi. (Jordan kletkasınıń xarakteristikalıq matricası, Jordan matricasınıń uqsaslıǵı haqqında teorema,  Matricalardı jordan normal kórinisine keltiriw) \\
\textbf{A1.} Tómendegi kvadratlıq formanıń rangni anıqlań: \(2x_{1}^{2} + 3x_{2}^{2} + 4x_{3}^{2} - 2x_{1}x_{2} + 4x_{1}x_{3} - 3x_{2}x_{3}\) \\
\textbf{A2.} Tómendegi funkciyalı haqiqiy sanlar maydanı ústinde anıqlangan \(V\) keńislikte sızıqlı funkciya boladı: \(V = M_{n}\left( \mathbb{R} \right),\ \ f(A) = \det(A)\); \\
\textbf{A3.} Tómendegi ańlatpalardan qaysıları sáykes túrde berilgen vektor keńislikte sızıqlı túrlendiriw boladı: sızıqlı keńislik,, bul jerde -fiksirlengen vektor; \\
\textbf{B1.} Ortogonallastırıw procesinen paydalanıp, berilgen vektorlar sistemasini ortogonallastirıń: \((1,2,2, - 1)\), ( \(1,1, - 5,3\) ), (3, 2, 8, -7); \\
\textbf{B2.} Eger \(f\) sızıqlı funkciya\(e_{1},e_{2},e_{3}\) bazisde \(f(x) = 2x_{1} - 3x_{2} + x_{3}\) arqalı anıqlanǵan bolsa, onıń \(e_{1}^{'},e_{2}^{'},e_{3}^{'}\) bazisdegi kórinisin tabıń\(e_{1}^{'} = e_{1} - e_{2},\ e_{2}^{'} = e_{1} + e_{3},\ \ e_{3}^{'} = e_{1} + e_{2} + e_{3}\); \\
\textbf{B3.} Matricası tómendegishe bolǵan sızıqlı túrlendiriwdiń menshikli mánisi hám menshikli vektorların tabıń: \(\begin{pmatrix} 0 & 0 & 1 \\ 1 & 4 & 0 \\  - 2 & 0 & 2 \end{pmatrix}\); \\
\textbf{C1.} Jordan normal formasını tabıń \(\begin{pmatrix} 0 & 0 & 1 \\ 1 & 4 & 0 \\  - 2 & 0 & 2 \end{pmatrix}\); \\
\textbf{C2.} Bazi bir ortonormal bazisde berilgen kvadratlıq formani kanonik kóriniske keltiriwshi ortonormal bazisin tabıń: \(x_{1}^{2} - 5x_{2}^{2} + x_{3}^{2} + 4x_{1}x_{2} + 2x_{1}x_{3} + 4x_{2}x_{3}\); \\
\textbf{C3.} Ortogonallastırıw procesinen paydalanıp, berilgen vektorlar sistemasini ortogonallastirıń: \((1,1, - 1,0)\), \((2,0, - 1,0)\), \((1, - 1,1, - 1)\), (2, \(0,1,1\) ). \\

\end{tabular}
\vspace{1cm}


\begin{tabular}{m{17cm}}
\textbf{82-variant}
\newline

\textbf{T1.} İnertsiya nızamı. (invariantlar,  eki kvadratılıq forma arasindaǵı baylanıs ) \\
\textbf{T2.} Sızıqlı túrlendiriwler.  (Sızıqlı túrlendiriw túsinigi, Sızıqlı túrlendiriwler ústinde ámeller, Sızıqlı túrlendiriwlerdiń obrazı hám yadrosı.) \\
\textbf{A1.} \(\mathbb{R}^{2}\) keńislikte anıqlangan tómendegi sáwlelendiriw skalyar kóbeyme bolatuģının anıqlań: \((x,y) = x_{1}y_{1} + x_{2}y_{1} + 3x_{1}y_{2} + 2x_{2}y_{2}\) \\
\textbf{A2.} Tómendegi vektorlar sisteması óz ara ortogonallıqqa tekseriń hám olardı ortogonallıq baziske shekem toltırın: \((1,2, - 1),(3, - 1,1)\); \\
\textbf{A3.} Tómendegi sáwlelendiriwlerden qaysıları keńislikte sızıqlı túrlendiriw boladı:; \\
\textbf{B1.} Ortogonallastırıw procesinen paydalanıp, berilgen vektorlar sistemasini ortogonallastirıń: \((1,0,0)\), (0, 1, -1), (1, 1, 1); \\
\textbf{B2.} Tómendegi kvadratlıq formanıń kanonikalıq kórinisin hám bul túrge keltiriwshi menshikli mánislerin tabıń: \(x_{1}^{2} - 2x_{2}^{2} - 2x_{3}^{2} - 4x_{1}x_{2} - 4x_{1}x_{3} + 8x_{2}x_{3}\); \\
\textbf{B3.} Matricası tómendegishe bolǵan sızıqlı túrlendiriwdiń menshikli mánisi hám menshikli vektorların tabıń: \(\begin{pmatrix} 4 & - 5 & 2 \\ 0 & - 7 & 3 \\ 0 & 0 & 4 \end{pmatrix}\); \\
\textbf{C1.} Jordan narmal formasini tabıń\(\begin{pmatrix} 4 & - 5 & 2 \\ 0 & - 7 & 3 \\ 0 & 0 & 4 \end{pmatrix}\); \\
\textbf{C2.} Tómendegi kvadratlıq formalarga sáykes keliwshi ermit bisızıqlı formalardı tabiń:\((5 - i)x_{1}\overline{x_{2}} + (5 + i)\overline{x_{1}}x_{2} + x_{2}\overline{x_{2}}\); \\
\textbf{C3.} Matricası tómendegishe bolǵan sızıqlı túrlendiriwdiń menshikli mánisi hám menshikli vektorların tabıń: \(\begin{pmatrix} 3 & - 1 & 0 & 0 \\ 1 & 1 & 0 & 0 \\ 3 & 0 & 5 & - 3 \\ 4 & - 1 & 3 & - 1 \end{pmatrix}\); \\

\end{tabular}
\vspace{1cm}


\begin{tabular}{m{17cm}}
\textbf{83-variant}
\newline

\textbf{T1.} Ortogonal  tolıqtırıwshı. (Ortogonal tolıqtırıwshı,  ortogonal proekciya) \\
\textbf{T2.} Unitar túrlendiriwler. (Unitar sızıqlı túrlendiriwler túsinigi,  Unitar túrlendiriwge túyinles túrlendiriwlerdiń matricası,   Ortonormal baziste unitar túrlendiriwlerdiń matricası) \\
\textbf{A1.} Tómendegi kvadratlıq formanıń rangni anıqlań: \(3x_{1}^{2} - 2x_{2}^{2} + 2x_{3}^{2} + 4x_{1}x_{2} - 3x_{1}x_{3} - x_{2}x_{3}\); \\
\textbf{A2.} Tómendegi kvadratlıq forma oń anıqlangan bolatuģın\(\lambda\) nıń barlıq mánislerin tabıń: \(x_{1}^{2} + 4x_{2}^{2} + x_{3}^{2} + 2\lambda x_{1}x_{2} + 10x_{1}x_{3} + 6x_{2}x_{3}\); \\
\textbf{A3.} Tómendegi sáwlelendiriw \(V\) vektor keńislikte sızıqlı túrlendiriw boladı: \(V\) sızıqlı keńislik,\(Ax = a\), bul jerde \(a\)-fiksirlengen vektor; \\
\textbf{B1.} Ortogonallastırıw procesinen paydalanıp, berilgen vektorlar sistemasini ortogonallastirıń: \((1,1, - 1,0)\), \((2,0, - 1,0)\), \((1, - 1,1, - 1)\), (2, \(0,1,1\) ). \\
\textbf{B2.} Eger \(f\) sızıqlı funkciya\(e_{1},e_{2},e_{3}\) bazisde \(f(x) = 2x_{1} - 3x_{2} + x_{3}\) arqalı anıqlanǵan bolsa, onıń \(e_{1}^{'},e_{2}^{'},e_{3}^{'}\) bazisdegi kórinisin tabıń\(e_{1}^{'} = e_{1} + e_{2} - 2e_{3},\ e_{2}^{'} = e_{1} + e_{2} + 2e_{3},\ e_{3}^{'} = e_{2} + e_{3}\); \\
\textbf{B3.} Tómendegi vektorlar sisteması óz ara ortogonallıqqa tekseriń hám olardı ortogonallıq baziske shekem toltırın: \((0,i,1,1),\ \ (1,2,1 + i, - 1 + i)\). \\
\textbf{C1.} Berilgen \(A\) bisızıqlı formanıń\(e_{1},e_{2},e_{3}\) bazisdegi matritsası hám\(e_{1}^{'},e_{2}^{'},e_{3}^{'}\) baziske ótiw formulaları berilgen bolsa, onda bul bisızıqli formanıń \(e_{1}^{'},e_{2}^{'},e_{3}^{'}\) bazisdegi matritsasini tabıń: \(\begin{pmatrix} 2 & 2 & 3 \\  - 4 & 3 & 1 \\ 3 & 1 & 2 \end{pmatrix},\ \begin{matrix}  & e_{1}^{'} = e_{1} + 3e_{2} - 2e_{3} \\  & e_{2}^{'} = 2e_{1} + e_{2} - e_{3} \\  & e_{3}^{'} = e_{1} + e_{2} - 3e_{3} \end{matrix}\) \\
\textbf{C2.} Bazi bir ortonormal bazisde berilgen kvadratlıq formani kanonik kóriniske keltiriwshi ortonormal bazisin tabıń: \(x_{1}^{2} + x_{2}^{2} + x_{3}^{2} + 4x_{1}x_{2} + 4x_{1}x_{3} + 4x_{2}x_{3}\); \\
\textbf{C3.} Matricası tómendegishe bolǵan sızıqlı túrlendiriwdiń menshikli mánisi hám menshikli vektorların tabıń: \(\begin{pmatrix} 2 & - 1 & 2 \\ 5 & - 3 & 3 \\  - 1 & 0 & - 2 \end{pmatrix}\); \\

\end{tabular}
\vspace{1cm}


\begin{tabular}{m{17cm}}
\textbf{84-variant}
\newline

\textbf{T1.} Kvadratlıq forma. (Lagran usulı, Yakobi usulı kvadratlıq formanı keltiriw.) \\
\textbf{T2.} Óz-ara orın almasıwshı túrlendiriwler. (Óz-ara orın almasıwshı túrlendiriwler,  Ortogonal bazis haqqında teorema,  Normal túrlendiriwlerdiń kanonikalıq kórinisi) \\
\textbf{A1.} Tómendegi kvadratlıq formanıń rangni anıqlań: \(x_{1}^{2} - 2x_{2}^{2} - 2x_{3}^{2} - 4x_{1}x_{2} - 4x_{1}x_{3} + 8x_{2}x_{3}\); \\
\textbf{A2.} Tómendegi kvadratlıq forma oń anıqlangan bolatuģın\(\lambda\) nıń barlıq mánislerin tabıń: \(x_{1}^{2} + \lambda x_{2}^{2} + x_{3}^{2} - 4x_{1}x_{2} - 8x_{2}x_{3}\); \\
\textbf{A3.} Tómendegi sáykes túrlendiriwlerden qaysıları berilgen vektor keńislikte sızıqlı túrlendiriw boladı: sızıqlı keńislik,, bul jerde -fiksirlengen vektor; \\
\textbf{B1.} Ortogonallastırıw procesinen paydalanıp, berilgen vektorlar sistemasini ortogonallastirıń: \((1,1, - 1, - 2)\), \((5,8, - 2, - 3)\), (3, 9, 3, 8); \\
\textbf{B2.} Tómendegi kvadratlıq formanıń kanonikalıq kórinisin hám bul túrge keltiriwshi menshikli mánislerin tabıń: \(3x_{1}^{2} - 2x_{2}^{2} + 2x_{3}^{2} + 4x_{1}x_{2} - 3x_{1}x_{3} - x_{2}x_{3}\); \\
\textbf{B3.} Tómendegi vektorlar sisteması óz ara ortogonallıqqa tekseriń hám olardı ortogonallıq baziske shekem toltırın. \((i,i,1, - 1),\ \ (1, - 1 + i,0,1),\ \ \); \\
\textbf{C1.} Berilgen \(A\) bisızıqlı formanıń\(e_{1},e_{2},e_{3}\) bazisdegi matritsası hám\(e_{1}^{'},e_{2}^{'},e_{3}^{'}\) baziske ótiw formulaları berilgen bolsa, onda bul bisızıqli formanıń \(e_{1}^{'},e_{2}^{'},e_{3}^{'}\) bazisdegi matritsasini tabıń: \(\begin{pmatrix} 1 & 1 & 2 \\  - 1 & 2 & 1 \\  - 1 & 1 & - 1 \end{pmatrix},\begin{matrix}  & e_{1}^{'} = e_{1} + e_{2} - 2e_{3} \\  & e_{2}^{'} = e_{1} + e_{2} + 2e_{3} \\  & e_{3}^{'} = e_{2} + e_{3} \end{matrix}\) \\
\textbf{C2.} Bazi bir ortonormal bazisde berilgen kvadratlıq formani kanonik kóriniske keltiriwshi ortonormal bazisin tabıń: \(11x_{1}^{2} + 5x_{2}^{2} + 2x_{3}^{2} + 16x_{1}x_{2} + 4x_{1}x_{3} - 20x_{2}x_{3}\); \\
\textbf{C3.} Matricası tómendegishe bolǵan sızıqlı túrlendiriwdiń menshikli mánisi hám menshikli vektorların tabıń: \(\begin{pmatrix} 0 & 2 & 1 \\  - 2 & 0 & 3 \\  - 1 & - 3 & 0 \end{pmatrix}\); \\

\end{tabular}
\vspace{1cm}


\begin{tabular}{m{17cm}}
\textbf{85-variant}
\newline

\textbf{T1.} İnertsiya nızamı. (invariantlar,  eki kvadratılıq forma arasindaǵı baylanıs ) \\
\textbf{T2.} Túyinles túrlendiriw. ( Evklid keńisligindegi sızıqlı túrlendiriwler menen bisızıqlı formalar arasındaǵı baylanıs, Berilgen túrlendiriwge túyinles túrlendiriwler, Óz-ózine túyinles túrlendiriwler) \\
\textbf{A1.} \(\mathbb{R}^{2}\) keńislikte anıqlangan tómendegi sáwlelendiriw skalyar kóbeyme bolatuģının anıqlań: \((x,y) = x_{1}y_{1} + x_{2}y_{1} + 3x_{1}y_{2} + 2x_{2}y_{2}\) \\
\textbf{A2.} Tómendegi vektorlar sisteması óz ara ortogonallıqqa tekseriń hám olardı ortogonallıq baziske shekem toltırın: \((1,2, - 1),(3, - 1,1)\); \\
\textbf{A3.} Tómendegi sáwlelendiriw\(V = \mathbb{R}^{3}\) keńislikte sızıqlı túrlendiriw boladı: \(A\left( x_{1},x_{2},x_{3} \right) = \left( x_{1},x_{2} + 1,x_{3} + 2 \right)\); \\
\textbf{B1.} Ortogonallastırıw procesinen paydalanıp, berilgen vektorlar sistemasini ortogonallastirıń: \((2,0,1,1)\), ( \(1,2,0,1\) ), ( \(0,1, - 2,0\) ); \\
\textbf{B2.} Tómendegi kvadratlıq formanı kanonikalıq kóriniske keltiriń: \(12x_{1}^{2} + 3x_{2}^{2} + 12x_{3}^{2} - 12x_{1}x_{2} + 24x_{1}x_{3} - 8x_{2}x_{3}\); \\
\textbf{B3.} Matricası tómendegishe bolǵan sızıqlı túrlendiriwdiń menshikli mánisi hám menshikli vektorların tabıń: \(\begin{pmatrix} 2 & - 1 & 2 \\ 0 & - 3 & 0 \\ 0 & 0 & 1 \end{pmatrix}\); \\
\textbf{C1.} Jordan normal formasını tabıń \(\begin{pmatrix} 4 & 1 & - 4 \\ 1 & 4 & 0 \\  - 4 & 0 & 4 \end{pmatrix}\); \\
\textbf{C2.} Tómendegi kvadratlıq forma oń anıqlangan bolatuģın\(\lambda\) parametrnıń barlıq mánislerin tabıń: \(x_{1}\overline{x_{1}} + ix_{1}\overline{x_{2}} - ix_{2}\overline{x_{1}} + \lambda x_{2}\overline{x_{2}}\); \\
\textbf{C3.} Matricası tómendegishe bolǵan sızıqlı túrlendiriwdiń menshikli mánisi hám menshikli vektorların tabıń: \(\begin{pmatrix} 1 & - 3 & 4 \\ 4 & - 7 & 8 \\ 6 & - 7 & 7 \end{pmatrix}\); \\

\end{tabular}
\vspace{1cm}


\begin{tabular}{m{17cm}}
\textbf{86-variant}
\newline

\textbf{T1.} Kvadratlıq forma. (Lagran usulı, Yakobi usulı kvadratlıq formanı keltiriw.) \\
\textbf{T2.} Keri túrlendiriwler. ( Keri túrlendiriw túsinigi,   Keri túrlendiriwdiń sızıqlılıǵı) \\
\textbf{A1.} \(\mathbb{R}^{2}\) keńislikte anıqlangan tómendegi sáwlelendiriw skalyar kóbeyme bolatuģının anıqlań: \((x,y) = x_{1}y_{1} - 2x_{2}y_{1} - 2x_{1}y_{2} + x_{2}y_{2}\) \\
\textbf{A2.} Tómendegi kvadratlıq forma oń anıqlangan bolatuģın\(\lambda\) nıń barlıq mánislerin tabıń: \(2x_{1}^{2} + x_{2}^{2} + 3x_{3}^{2} + 2\lambda x_{1}x_{2} + 2x_{1}x_{3}\); \\
\textbf{A3.} Tómendegi sáwlelendiriw\(V = \mathbb{R}^{3}\) keńislikte sızıqlı túrlendiriw boladı: \(A\left( x_{1},x_{2},x_{3} \right) = \left( x_{1} + 3x_{3},x_{2}^{3},x_{1} + x_{3} \right)\). \\
\textbf{B1.} Ortogonallastırıw procesinen paydalanıp, berilgen vektorlar sistemasini ortogonallastirıń: \((1,1, - 1)\), (1, 1,1 ), \((3,2, - 1)\); \\
\textbf{B2.} Eger \(f\) sızıqlı funkciya\(e_{1},e_{2},e_{3}\) bazisde \(f(x) = 2x_{1} - 3x_{2} + x_{3}\) arqalı anıqlanǵan bolsa, onıń \(e_{1}^{'},e_{2}^{'},e_{3}^{'}\) bazisdegi kórinisin tabıń\(e_{1}^{'} = 4e_{1} - e_{2} - 3e_{3},\ e_{2}^{'} = 2e_{1} + e_{2},\ e_{3}^{'} = 3e_{1} + 2e_{2}\). \\
\textbf{B3.} Tómendegi vektorlar sisteması óz ara ortogonallıqqa tekseriń hám olardı ortogonallıq baziske shekem toltırın: \((1,1,1,1),(1,1, - 1, - 1),(1, - 1,1, - 1)\); \\
\textbf{C1.} Berilgen \(A\) bisızıqlı formanıń\(e_{1},e_{2},e_{3}\) bazisdegi matritsası hám\(e_{1}^{'},e_{2}^{'},e_{3}^{'}\) baziske ótiw formulaları berilgen bolsa, onda bul bisızıqli formanıń\(e_{1}^{'},e_{2}^{'},e_{3}^{'}\) bazisdegi matritsasini tabıń: \(\ \) \(\begin{pmatrix} 0 & 2 & 1 \\  - 2 & 2 & 0 \\  - 1 & 0 & 3 \end{pmatrix}\), \(e_{1}^{'} = e_{1} + 2e_{2} - e_{3}\), \(e_{2}^{'} = e_{2} - e_{3}\), \(e_{3}^{'} = - e_{1} + e_{2} - 3e_{3}\) \\
\textbf{C2.} Tómendegi bisiziqli formalar ekvivalent emes ekenligin dálilleń:\(f_{1}(x,y) = 2x_{1}y_{2} - 3x_{1}y_{3} + x_{2}y_{3} - 2x_{2}y_{1} - x_{3}y_{2} - 3x_{3}y_{1}\),\(f_{2}(x,y) = x_{1}y_{2} - x_{2}y_{1} + 2x_{2}y_{2} + 3x_{1}y_{3} - 3x_{3}y_{1};\) \\
\textbf{C3.} Matricası tómendegishe bolǵan sızıqlı túrlendiriwdiń menshikli mánisi hám menshikli vektorların tabıń: \(\begin{pmatrix} 1 & 0 & 0 & 0 \\ 0 & 0 & 0 & 0 \\ 0 & 0 & 0 & 0 \\ 1 & 0 & 0 & 0 \end{pmatrix}\); \\

\end{tabular}
\vspace{1cm}


\begin{tabular}{m{17cm}}
\textbf{87-variant}
\newline

\textbf{T1.} Kompleks evklid keńislikleri.  (Kompleks vektorlı keńislik, Ermit kvadratlıq forma.) \\
\textbf{T2.} Óz-ara orın almasıwshı túrlendiriwler. (Óz-ara orın almasıwshı túrlendiriwler,  Ortogonal bazis haqqında teorema,  Normal túrlendiriwlerdiń kanonikalıq kórinisi) \\
\textbf{A1.} \(\mathbb{R}^{2}\) keńislikte anıqlangan tómendegi sáwlelendiriw skalyar kóbeyme bolatuģının anıqlań: \((x,y) = x_{1}y_{1} - x_{2}y_{2}\) \\
\textbf{A2.} Tómendegi funkciyalı haqiqiy sanlar maydanı ústinde anıqlangan \(V\) keńislikte sızıqlı funkciya boladı: \(V = \mathbb{R}^{3},\ \ f(x) = |x|\); \\
\textbf{A3.} Tómendegi sáwlelendiriw \(V\) vektor keńislikte sızıqlı túrlendiriw boladı: \(V\) sızıqlı keńislik,\(Ax = a\), bul jerde \(a\)-fiksirlengen vektor; \\
\textbf{B1.} Ortogonallastırıw procesinen paydalanıp, berilgen vektorlar sistemasini ortogonallastirıń: \((1,2,1,3)\), (4, 1, 1, 1), (3, 1, 1, 0); \\
\textbf{B2.} Tómendegi kvadratlıq formanıń kanonikalıq kórinisin hám bul túrge keltiriwshi menshikli mánislerin tabıń: \(2x_{1}^{2} + 3x_{2}^{2} + 4x_{3}^{2} - 2x_{1}x_{2} + 4x_{1}x_{3} - 3x_{2}x_{3}\); \\
\textbf{B3.} Tómendegi vektorlar sisteması óz ara ortogonallıqqa tekseriń hám olardı ortogonallıq baziske shekem toltırın: \((0,1,i),\ \ (1 + i,i,1)\); \\
\textbf{C1.} Jordan normal formasını tabıń \(\begin{pmatrix} 2 & - 1 & 2 \\ 0 & - 3 & 0 \\ 0 & 0 & 1 \end{pmatrix}\); \\
\textbf{C2.} Bazi bir ortonormal bazisde berilgen kvadratlıq formani kanonik kóriniske keltiriwshi ortonormal bazisin tabıń: \(x_{1}^{2} - 5x_{2}^{2} + x_{3}^{2} + 4x_{1}x_{2} + 2x_{1}x_{3} + 4x_{2}x_{3}\); \\
\textbf{C3.} Matricası tómendegishe bolǵan sızıqlı túrlendiriwdiń menshikli mánisi hám menshikli vektorların tabıń: \(\begin{pmatrix} 5 & 6 & - 3 \\  - 1 & 0 & 1 \\ 1 & 2 & - 1 \end{pmatrix}\); \\

\end{tabular}
\vspace{1cm}


\begin{tabular}{m{17cm}}
\textbf{88-variant}
\newline

\textbf{T1.} Evklid keńisligi. (Skalyar kóbeyme, ortogonal vektorlar, ortonormal bazis.) \\
\textbf{T2.} Sızıqlı túrlendiriwler matritsasınıń Jordan normal kórinisi. (Jordan kletkasınıń xarakteristikalıq matricası, Jordan matricasınıń uqsaslıǵı haqqında teorema,  Matricalardı jordan normal kórinisine keltiriw) \\
\textbf{A1.} \(\mathbb{R}^{2}\) keńislikte anıqlangan tómendegi sáwlelendiriw skalyar kóbeyme bolatuģının anıqlań: \((x,y) = x_{1}y_{1} - x_{2}y_{1} - x_{1}y_{2} + x_{2}y_{2}\) \\
\textbf{A2.} Tómendegi vektorlar sisteması óz ara ortogonallıqqa tekseriń hám olardı ortogonallıq baziske shekem toltırın: \((1,1,1,2)\), \((1,2,3, - 3)\). \\
\textbf{A3.} Tómendegi sáwlelendiriwlerden qaysıları keńislikte sızıqlı túrlendiriw boladı:; \\
\textbf{B1.} \(\mathbb{R}^{3}\) keńislikte \((x,y) = x_{1}y_{1} + 3x_{2}y_{2} + 2x_{3}y_{3}\) berilgen skalyar kóbeyme ushın \(a = (1, - 3,2)\) va \(b = (2,1, - 1)\) \(b = (0,1)\) vektorlar arasındaǵı múyeshti tabıń . \\
\textbf{B2.} Tómendegi kvadratlıq formanıń kanonikalıq kórinisin hám bul túrge keltiriwshi menshikli mánislerin tabıń: \(5x_{1}^{2} + 6x_{2}^{2} + 4x_{3}^{2} - 4x_{1}x_{2} - 4x_{1}x_{3}\); \\
\textbf{B3.} Tómendegi vektorlar sisteması óz ara ortogonallıqqa tekseriń hám olardı ortogonallıq baziske shekem toltırın: \((1,2,0, - 1),(3, - 1,1,1),( - 1,2,2,3)\); \\
\textbf{C1.} Jordan normal formasını tabıń \(\begin{pmatrix} 1 & - 2 & 1 \\  - 2 & 1 & 4 \\  - 1 & 4 & 1 \end{pmatrix}\). \\
\textbf{C2.} Tómendegi kvadratlıq formalarga sáykes keliwshi ermit bisızıqlı formalardı tabiń. \(x_{1}\overline{x_{1}} - ix_{1}\overline{x_{2}} - ix_{2}\overline{x_{1}} + 2x_{2}\overline{x_{2}}\); \\
\textbf{C3.} Matricası tómendegishe bolǵan sızıqlı túrlendiriwdiń menshikli mánisi hám menshikli vektorların tabıń: \(\begin{pmatrix} 1 & 1 & 1 & 1 \\ 1 & 1 & - 1 & - 1 \\ 1 & - 1 & 1 & - 1 \\ 1 & - 1 & - 1 & 1 \end{pmatrix}\). \\

\end{tabular}
\vspace{1cm}


\begin{tabular}{m{17cm}}
\textbf{89-variant}
\newline

\textbf{T1.} Ortogonal  tolıqtırıwshı. (Ortogonal tolıqtırıwshı,  ortogonal proekciya) \\
\textbf{T2.} Sızıqlı túrlendiriwler.  (Sızıqlı túrlendiriw túsinigi, Sızıqlı túrlendiriwler ústinde ámeller, Sızıqlı túrlendiriwlerdiń obrazı hám yadrosı.) \\
\textbf{A1.} Tómendegi kvadratlıq formanıń rangni anıqlań: \(2x_{1}^{2} + 3x_{2}^{2} + 4x_{3}^{2} - 2x_{1}x_{2} + 4x_{1}x_{3} - 3x_{2}x_{3}\) \\
\textbf{A2.} Tómendegi kvadratlıq forma oń anıqlangan bolatuģın\(\lambda\) nıń barlıq mánislerin tabıń: \(5x_{1}^{2} + x_{2}^{2} + \lambda x_{3}^{2} + 4x_{1}x_{2} - 2x_{1}x_{3} - 2x_{2}x_{3}\); \\
\textbf{A3.} Tómendegi sáwlelendiriwlerden qaysıları sáykes túrde berilgen vektor keńislikte sızıqlı túrlendiriw boladı: sızıqlı keńislik, bul jerde -fiksirlengen san; \\
\textbf{B1.} Ortogonallastırıw procesinen paydalanıp, berilgen vektorlar sistemasini ortogonallastirıń: \((1,1,0,0)\), (1, 0, 1, 1); \\
\textbf{B2.} Tómendegi kvadratlıq formanı kanonikalıq kóriniske keltiriń: \(x_{1}x_{2} + x_{1}x_{3} + x_{1}x_{4} + x_{2}x_{3} + x_{2}x_{4} + x_{3}x_{4}\); \\
\textbf{B3.} Tómendegi vektorlar sisteması óz ara ortogonallıqqa tekseriń hám olardı ortogonallıq baziske shekem toltırın: \((1,i, - i),\ \ ( - 2 - i,1 + i,2 - i)\); \\
\textbf{C1.} Jordan normal formasını tabıń \(\begin{pmatrix} 7 & 0 & 0 \\ 10 & - 19 & 0 \\ 12 & - 24 & 13 \end{pmatrix}\); \\
\textbf{C2.} Tómendegi kvadratlıq formalarga sáykes keliwshi ermit bisızıqlı formalardı tabiń: \(ix_{1}\overline{x_{2}} - ix_{2}\overline{x_{1}} + (3 - 2i)x_{1}\overline{x_{3}} + (3 + 2i)x_{3}\overline{x_{1}} + 2x_{2}\overline{x_{3}} + 2x_{3}\overline{x_{2}}\). \\
\textbf{C3.} Matricası tómendegishe bolǵan sızıqlı túrlendiriwdiń menshikli mánisi hám menshikli vektorların tabıń: \(\begin{pmatrix} 1 & 0 & 0 & 0 \\ 0 & 0 & 0 & 0 \\ 1 & 0 & 0 & 0 \\ 0 & 0 & 0 & 1 \end{pmatrix}\); \\

\end{tabular}
\vspace{1cm}


\begin{tabular}{m{17cm}}
\textbf{90-variant}
\newline

\textbf{T1.} Sızıqlı keńislikler.   (Vektor,  sızıqlı baylanıs, bazis, ólshem, )  \\
\textbf{T2.} Unitar túrlendiriwler. (Unitar sızıqlı túrlendiriwler túsinigi,  Unitar túrlendiriwge túyinles túrlendiriwlerdiń matricası,   Ortonormal baziste unitar túrlendiriwlerdiń matricası) \\
\textbf{A1.} Tómendegi kvadratlıq formanıń rangni anıqlań: \(x_{1}^{2} - 2x_{2}^{2} - 2x_{3}^{2} - 4x_{1}x_{2} - 4x_{1}x_{3} + 8x_{2}x_{3}\); \\
\textbf{A2.} Tómendegi funkciyalı haqiqiy sanlar maydanı ústinde anıqlangan \(V\) keńislikte sızıqlı funkciya boladı: \(V = M_{n}\left( \mathbb{R} \right),\ \ f(A) = \det(A)\); \\
\textbf{A3.} Tómendegi sáwlelendiriwmos ravishda Berilgen \(V\) vektor keńislikte sızıqlı túrlendiriw boladı: \(V\) sızıqlı keńislik,\(Ax = x + a\), bul jerde \(a\)-fiksirlengen vektor; \\
\textbf{B1.} \(\mathbb{R}^{2}\) keńislikte \((x,y) = x_{1}y_{1} + 2x_{2}y_{1} + 2x_{1}y_{2} + 5x_{2}y_{2}\) berilgen skalyar kóbeyme ushın \(a = (1,0)\) hám \(b = (0,1)\) vektorlar arasındaǵı múyeshti tabıń \\
\textbf{B2.} Tómendegi kvadratlıq formanı kanonikalıq kóriniske keltiriń: \(2x_{1}^{2} + 18x_{2}^{2} + 8x_{3}^{2} - 12x_{1}x_{2} + 8x_{1}x_{3} - 27x_{2}x_{3}\); \\
\textbf{B3.} Matricası tómendegishe bolǵan sızıqlı túrlendiriwdiń menshikli mánisi hám menshikli vektorların tabıń: \(\begin{pmatrix} 7 & 0 & 0 \\ 10 & - 19 & 0 \\ 12 & - 24 & 13 \end{pmatrix}\); \\
\textbf{C1.} Berilgen \(A\) bisızıqlı formanıń\(e_{1},e_{2},e_{3}\) bazisdegi matritsası hám\(e_{1}^{'},e_{2}^{'},e_{3}^{'}\) baziske ótiw formulaları berilgen bolsa, onda bul bisızıqli formanıń\(e_{1}^{'},e_{2}^{'},e_{3}^{'}\) bazisdegi matritsasini tabıń: \(\begin{pmatrix} 1 & 2 & 3 \\ 4 & 5 & 6 \\ 7 & 8 & 9 \end{pmatrix}\), \(e_{1}^{'} = e_{1} - e_{2}\), \(e_{2}^{'} = e_{1} + e_{3}\), \(e_{3}^{'} = e_{1} + e_{2} + e_{3}\) \\
\textbf{C2.} Bazi bir ortonormal bazisde berilgen kvadratlıq formani kanonik kóriniske keltiriwshi ortonormal bazisin tabıń: \(x_{1}^{2} + x_{2}^{2} + x_{3}^{2} + 4x_{1}x_{2} + 4x_{1}x_{3} + 4x_{2}x_{3}\); \\
\textbf{C3.} Ortogonallastırıw procesinen paydalanıp, berilgen vektorlar sistemasini ortogonallastirıń: \((2,1,3, - 1)\), ( \(7,4,3, - 3\) ), ( \(1,1, - 6,0\) ), (5, 7, 7, 8); \\

\end{tabular}
\vspace{1cm}


\begin{tabular}{m{17cm}}
\textbf{91-variant}
\newline

\textbf{T1.} Sızıqlı, bisızıqlı hám kvadratlıq formalar. (Bisızıqlı forma,  simmetriyalı bisızıqlı formalar)  \\
\textbf{T2.} Óz-ara orın almasıwshı túrlendiriwler. (Óz-ara orın almasıwshı túrlendiriwler,  Ortogonal bazis haqqında teorema,  Normal túrlendiriwlerdiń kanonikalıq kórinisi) \\
\textbf{A1.} Tómendegi kvadratlıq formanıń rangni anıqlań: \(x_{1}^{2} + x_{2}^{2} + x_{3}^{2} + x_{4}^{2} + 2x_{1}x_{2} - 2x_{1}x_{4} - 2x_{2}x_{3} + 2x_{3}x_{4}\). \\
\textbf{A2.} Tómendegi kvadratlıq forma oń anıqlangan bolatuģın\(\lambda\) nıń barlıq mánislerin tabıń: \(x_{1}^{2} + \lambda x_{2}^{2} + x_{3}^{2} - 4x_{1}x_{2} - 8x_{2}x_{3}\); \\
\textbf{A3.} Tómendegi sáwlelendiriw\(V = \mathbb{R}^{3}\) keńislikte sızıqlı túrlendiriw boladı: \(A\left( x_{1},x_{2},x_{3} \right) = \left( x_{1},x_{2},x_{1} + x_{2} + x_{3} \right)\); \\
\textbf{B1.} Ortogonallastırıw procesinen paydalanıp, berilgen vektorlar sistemasini ortogonallastirıń: \((2,0,1,1)\), ( \(1,2,0,1\) ), ( \(0,1, - 2,0\) ); \\
\textbf{B2.} Tómendegi kvadratlıq formanıń kanonikalıq kórinisin hám bul túrge keltiriwshi menshikli mánislerin tabıń: \(x_{1}x_{2} + x_{1}x_{3} + x_{2}x_{3}\); \\
\textbf{B3.} Tómendegi vektorlar sisteması óz ara ortogonallıqqa tekseriń hám olardı ortogonallıq baziske shekem toltırın: \((1,2,0, - 1),(3, - 1,1,1),( - 1,2,2,3)\); \\
\textbf{C1.} Jordan normal formasını tabıń \(\begin{pmatrix} 0 & 0 & 1 \\ 1 & 4 & 0 \\  - 2 & 0 & 2 \end{pmatrix}\); \\
\textbf{C2.} Bazi bir ortonormal bazisde berilgen kvadratlıq formani kanonik kóriniske keltiriwshi ortonormal bazisin tabıń: \(11x_{1}^{2} + 5x_{2}^{2} + 2x_{3}^{2} + 16x_{1}x_{2} + 4x_{1}x_{3} - 20x_{2}x_{3}\); \\
\textbf{C3.} Ortogonallastırıw procesinen paydalanıp, berilgen vektorlar sistemasini ortogonallastirıń: \((1,1, - 1,0)\), \((2,0, - 1,0)\), \((1, - 1,1, - 1)\), (2, \(0,1,1\) ). \\

\end{tabular}
\vspace{1cm}


\begin{tabular}{m{17cm}}
\textbf{92-variant}
\newline

\textbf{T1.} Sızıqlı, bisızıqlı hám kvadratlıq formalar. (Bisızıqlı forma,  simmetriyalı bisızıqlı formalar)  \\
\textbf{T2.} Unitar túrlendiriwler. (Unitar sızıqlı túrlendiriwler túsinigi,  Unitar túrlendiriwge túyinles túrlendiriwlerdiń matricası,   Ortonormal baziste unitar túrlendiriwlerdiń matricası) \\
\textbf{A1.} Tómendegi kvadratlıq formanıń rangni anıqlań: \(x_{1}x_{2} + x_{2}x_{3} + x_{3}x_{4} + x_{1}x_{4}\); \\
\textbf{A2.} Tómendegi vektorlar sisteması óz ara ortogonallıqqa tekseriń hám olardı ortogonallıq baziske shekem toltırın: \((1, - 2,2, - 3),(2, - 3,2,4)\); \\
\textbf{A3.} Tómendegi sáwlelendiriw\(V = \mathbb{R}^{3}\) keńislikte sızıqlı túrlendiriw boladı: \(A\left( x_{1},x_{2},x_{3} \right) = \left( x_{2} + x_{3},2x_{1} + x_{3},3x_{1} - x_{2} + x_{3} \right)\); \\
\textbf{B1.} Ortogonallastırıw procesinen paydalanıp, berilgen vektorlar sistemasini ortogonallastirıń: \((1,1, - 1, - 2)\), \((5,8, - 2, - 3)\), (3, 9, 3, 8); \\
\textbf{B2.} Tómendegi kvadratlıq formanıń kanonikalıq kórinisin hám bul túrge keltiriwshi menshikli mánislerin tabıń: \(3x_{1}^{2} - 2x_{2}^{2} + 2x_{3}^{2} + 4x_{1}x_{2} - 3x_{1}x_{3} - x_{2}x_{3}\); \\
\textbf{B3.} Matricası tómendegishe bolǵan sızıqlı túrlendiriwdiń menshikli mánisi hám menshikli vektorların tabıń: \(\begin{pmatrix} 0 & 0 & 1 \\ 1 & 4 & 0 \\  - 2 & 0 & 2 \end{pmatrix}\); \\
\textbf{C1.} Berilgen \(A\) bisızıqlı formanıń\(e_{1},e_{2},e_{3}\) bazisdegi matritsası hám\(e_{1}^{'},e_{2}^{'},e_{3}^{'}\) baziske ótiw formulaları berilgen bolsa, onda bul bisızıqli formanıń\(e_{1}^{'},e_{2}^{'},e_{3}^{'}\) bazisdegi matritsasini tabıń: \(\begin{pmatrix} 1 & 2 & 3 \\ 4 & 5 & 6 \\ 7 & 8 & 9 \end{pmatrix}\), \(e_{1}^{'} = e_{1} - e_{2}\), \(e_{2}^{'} = e_{1} + e_{3}\), \(e_{3}^{'} = e_{1} + e_{2} + e_{3}\) \\
\textbf{C2.} Tómendegi bisiziqli formalar ekvivalent emes ekenligin dálilleń:\(f_{1}(x,y) = x_{1}y_{1} + 2x_{1}y_{2} + 2x_{2}y_{1} + 5x_{2}y_{2} + 6x_{2}y_{3} + 8x_{3}y_{2} + 10x_{3}y_{3}\), \(f_{2}(x,y) = 2x_{1}y_{1} - x_{1}y_{3} + x_{2}y_{2} - x_{3}y_{1} + 5x_{3}y_{3}\). \\
\textbf{C3.} Matricası tómendegishe bolǵan sızıqlı túrlendiriwdiń menshikli mánisi hám menshikli vektorların tabıń: \(\begin{pmatrix} 3 & - 1 & 0 & 0 \\ 1 & 1 & 0 & 0 \\ 3 & 0 & 5 & - 3 \\ 4 & - 1 & 3 & - 1 \end{pmatrix}\); \\

\end{tabular}
\vspace{1cm}


\begin{tabular}{m{17cm}}
\textbf{93-variant}
\newline

\textbf{T1.} Ortogonal  tolıqtırıwshı. (Ortogonal tolıqtırıwshı,  ortogonal proekciya) \\
\textbf{T2.} Keri túrlendiriwler. ( Keri túrlendiriw túsinigi,   Keri túrlendiriwdiń sızıqlılıǵı) \\
\textbf{A1.} \(\mathbb{R}^{2}\) keńislikte anıqlangan tómendegi sáwlelendiriw skalyar kóbeyme bolatuģının anıqlań: \((x,y) = x_{1}y_{1} + 2x_{2}y_{2}\) \\
\textbf{A2.} Tómendegi kvadratlıq forma oń anıqlangan bolatuģın\(\lambda\) nıń barlıq mánislerin tabıń: \(x_{1}^{2} + x_{2}^{2} + 5x_{3}^{2} + 2\lambda x_{1}x_{2} - 2x_{1}x_{3} + 4x_{2}x_{3}\); \\
\textbf{A3.} Tómendegi ańlatpalardan qaysıları sáykes túrde berilgen vektor keńislikte sızıqlı túrlendiriw boladı: sızıqlı keńislik,, bul jerde -fiksirlengen vektor; \\
\textbf{B1.} \(\mathbb{R}^{2}\) keńislikte \((x,y) = x_{1}y_{1} + 2x_{2}y_{1} + 2x_{1}y_{2} + 5x_{2}y_{2}\) berilgen skalyar kóbeyme ushın \(a = (1,0)\) hám \(b = (0,1)\) vektorlar arasındaǵı múyeshti tabıń \\
\textbf{B2.} Eger \(f\) sızıqlı funkciya\(e_{1},e_{2},e_{3}\) bazisde \(f(x) = 2x_{1} - 3x_{2} + x_{3}\) arqalı anıqlanǵan bolsa, onıń \(e_{1}^{'},e_{2}^{'},e_{3}^{'}\) bazisdegi kórinisin tabıń\(e_{1}^{'} = e_{1} - e_{2},\ e_{2}^{'} = e_{1} + e_{3},\ \ e_{3}^{'} = e_{1} + e_{2} + e_{3}\); \\
\textbf{B3.} Tómendegi vektorlar sisteması óz ara ortogonallıqqa tekseriń hám olardı ortogonallıq baziske shekem toltırın: \((0,1,i),\ \ (1 + i,i,1)\); \\
\textbf{C1.} Jordan normal formasını tabıń \(\begin{pmatrix} 7 & 0 & 0 \\ 10 & - 19 & 0 \\ 12 & - 24 & 13 \end{pmatrix}\); \\
\textbf{C2.} Tómendegi kvadratlıq formalarga sáykes keliwshi ermit bisızıqlı formalardı tabiń: \(x_{1}\overline{x_{1}} + (2 + i)x_{1}\overline{x_{2}} + (2 - i)x_{2}\overline{x_{1}} + ix_{1}\overline{x_{3}} - ix_{3}\overline{x_{1}} - x_{3}\overline{x_{3}}\); \\
\textbf{C3.} Matricası tómendegishe bolǵan sızıqlı túrlendiriwdiń menshikli mánisi hám menshikli vektorların tabıń: \(\begin{pmatrix} 0 & 2 & 1 \\  - 2 & 0 & 3 \\  - 1 & - 3 & 0 \end{pmatrix}\); \\

\end{tabular}
\vspace{1cm}


\begin{tabular}{m{17cm}}
\textbf{94-variant}
\newline

\textbf{T1.} Sızıqlı keńislikler.   (Vektor,  sızıqlı baylanıs, bazis, ólshem, )  \\
\textbf{T2.} Sızıqlı túrlendiriwler matritsasınıń Jordan normal kórinisi. (Jordan kletkasınıń xarakteristikalıq matricası, Jordan matricasınıń uqsaslıǵı haqqında teorema,  Matricalardı jordan normal kórinisine keltiriw) \\
\textbf{A1.} Tómendegi kvadratlıq formanıń rangni anıqlań: \(x_{1}x_{2} + x_{1}x_{3} + x_{2}x_{3}\); \\
\textbf{A2.} Tómendegi vektorlar sisteması óz ara ortogonallıqqa tekseriń hám olardı ortogonallıq baziske shekem toltırın: \((2,1,2),\ (1,2, - 2)\); \\
\textbf{A3.} Tómendegi sáwlelendiriwlerden qaysıları keńislikte sızıqlı túrlendiriw boladı:; \\
\textbf{B1.} \(\mathbb{R}^{3}\) keńislikte \((x,y) = x_{1}y_{1} + 3x_{2}y_{2} + 2x_{3}y_{3}\) berilgen skalyar kóbeyme ushın \(a = (1, - 3,2)\) va \(b = (2,1, - 1)\) \(b = (0,1)\) vektorlar arasındaǵı múyeshti tabıń . \\
\textbf{B2.} Tómendegi kvadratlıq formanıń kanonikalıq kórinisin hám bul túrge keltiriwshi menshikli mánislerin tabıń: \(7x_{1}^{2} + 5x_{2}^{2} + 3x_{3}^{2} - 8x_{1}x_{2} + 8x_{2}x_{3}\); \\
\textbf{B3.} Tómendegi vektorlar sisteması óz ara ortogonallıqqa tekseriń hám olardı ortogonallıq baziske shekem toltırın: \((1,i, - i),\ \ ( - 2 - i,1 + i,2 - i)\); \\
\textbf{C1.} Jordan narmal formasini tabıń\(\begin{pmatrix} 4 & - 5 & 2 \\ 0 & - 7 & 3 \\ 0 & 0 & 4 \end{pmatrix}\); \\
\textbf{C2.} Bazi bir ortonormal bazisde berilgen kvadratlıq formani kanonik kóriniske keltiriwshi ortonormal bazisin tabıń: \(x_{1}^{2} + x_{2}^{2} + 5x_{3}^{2} - 6x_{1}x_{2} - 2x_{1}x_{3} + 2x_{2}x_{3}\); \\
\textbf{C3.} Matricası tómendegishe bolǵan sızıqlı túrlendiriwdiń menshikli mánisi hám menshikli vektorların tabıń: \(\begin{pmatrix} 1 & 0 & 0 & 0 \\ 0 & 0 & 0 & 0 \\ 1 & 0 & 0 & 0 \\ 0 & 0 & 0 & 1 \end{pmatrix}\); \\

\end{tabular}
\vspace{1cm}


\begin{tabular}{m{17cm}}
\textbf{95-variant}
\newline

\textbf{T1.} Kompleks evklid keńislikleri.  (Kompleks vektorlı keńislik, Ermit kvadratlıq forma.) \\
\textbf{T2.} Túyinles túrlendiriw. ( Evklid keńisligindegi sızıqlı túrlendiriwler menen bisızıqlı formalar arasındaǵı baylanıs, Berilgen túrlendiriwge túyinles túrlendiriwler, Óz-ózine túyinles túrlendiriwler) \\
\textbf{A1.} Tómendegi kvadratlıq formanıń rangni anıqlań: \(3x_{1}^{2} - 2x_{2}^{2} + 2x_{3}^{2} + 4x_{1}x_{2} - 3x_{1}x_{3} - x_{2}x_{3}\); \\
\textbf{A2.} Tómendegi kvadratlıq forma oń anıqlangan bolatuģın\(\lambda\) nıń barlıq mánislerin tabıń: \(2x_{1}^{2} + 2x_{2}^{2} + x_{3}^{2} + 2\lambda x_{1}x_{2} + 6x_{1}x_{3} + 2x_{2}x_{3}\); \\
\textbf{A3.} Matricası tómendegishe bolgan sızıqlı túrlendiriwdin menshikli mánisi hám menshikli vektorların tabıń: \(\begin{pmatrix} 2 & 1 \\ 1 & 2 \end{pmatrix}\); \\
\textbf{B1.} Ortogonallastırıw procesinen paydalanıp, berilgen vektorlar sistemasini ortogonallastirıń: \((1,2,1,3)\), (4, 1, 1, 1), (3, 1, 1, 0); \\
\textbf{B2.} Tómendegi kvadratlıq formanı kanonikalıq kóriniske keltiriń: \(12x_{1}^{2} + 3x_{2}^{2} + 12x_{3}^{2} - 12x_{1}x_{2} + 24x_{1}x_{3} - 8x_{2}x_{3}\); \\
\textbf{B3.} Matricası tómendegishe bolǵan sızıqlı túrlendiriwdiń menshikli mánisi hám menshikli vektorların tabıń: \(\begin{pmatrix} 2 & - 1 & 2 \\ 0 & - 3 & 0 \\ 0 & 0 & 1 \end{pmatrix}\); \\
\textbf{C1.} Jordan normal formasını tabıń \(\begin{pmatrix} 1 & - 2 & 1 \\  - 2 & 1 & 4 \\  - 1 & 4 & 1 \end{pmatrix}\). \\
\textbf{C2.} Tómendegi kvadratlıq forma oń anıqlangan bolatuģın\(\lambda\) parametrnıń barlıq mánislerin tabıń: \(\lambda x_{1}\overline{x_{1}} - ix_{1}\overline{x_{2}} + ix_{2}\overline{x_{1}} + 3x_{2}\overline{x_{2}}\); \\
\textbf{C3.} Matricası tómendegishe bolǵan sızıqlı túrlendiriwdiń menshikli mánisi hám menshikli vektorların tabıń: \(\begin{pmatrix} 1 & 0 & 0 & 0 \\ 0 & 0 & 0 & 0 \\ 0 & 0 & 0 & 0 \\ 1 & 0 & 0 & 0 \end{pmatrix}\); \\

\end{tabular}
\vspace{1cm}


\begin{tabular}{m{17cm}}
\textbf{96-variant}
\newline

\textbf{T1.} Evklid keńisligi. (Skalyar kóbeyme, ortogonal vektorlar, ortonormal bazis.) \\
\textbf{T2.} Sızıqlı túrlendiriwler.  (Sızıqlı túrlendiriw túsinigi, Sızıqlı túrlendiriwler ústinde ámeller, Sızıqlı túrlendiriwlerdiń obrazı hám yadrosı.) \\
\textbf{A1.} \(\mathbb{R}^{2}\) keńislikte anıqlangan tómendegi sáwlelendiriw skalyar kóbeyme bolatuģının anıqlań: \((x,y) = x_{1}y_{1} + 2x_{2}y_{1} + 2x_{1}y_{2} + 7x_{2}y_{2}\) \\
\textbf{A2.} Tómendegi kvadratlıq forma oń anıqlangan bolatuģın\(\lambda\) nıń barlıq mánislerin tabıń: \(x_{1}^{2} + 4x_{2}^{2} + x_{3}^{2} + 2\lambda x_{1}x_{2} + 10x_{1}x_{3} + 6x_{2}x_{3}\); \\
\textbf{A3.} Matricası tómendegishe: \\
\textbf{B1.} Ortogonallastırıw procesinen paydalanıp, berilgen vektorlar sistemasini ortogonallastirıń: \((1,1,0,0)\), (1, 0, 1, 1); \\
\textbf{B2.} Tómendegi kvadratlıq formanıń kanonikalıq kórinisin hám bul túrge keltiriwshi menshikli mánislerin tabıń: \(x_{1}^{2} - 2x_{2}^{2} - 2x_{3}^{2} - 4x_{1}x_{2} - 4x_{1}x_{3} + 8x_{2}x_{3}\); \\
\textbf{B3.} Tómendegi vektorlar sisteması óz ara ortogonallıqqa tekseriń hám olardı ortogonallıq baziske shekem toltırın. \((i,i,1, - 1),\ \ (1, - 1 + i,0,1),\ \ \); \\
\textbf{C1.} Jordan normal formasını tabıń \(\begin{pmatrix} 2 & - 1 & 2 \\ 0 & - 3 & 0 \\ 0 & 0 & 1 \end{pmatrix}\); \\
\textbf{C2.} Tómendegi kvadratlıq formalarga sáykes keliwshi ermit bisızıqlı formalardı tabiń:\((5 - i)x_{1}\overline{x_{2}} + (5 + i)\overline{x_{1}}x_{2} + x_{2}\overline{x_{2}}\); \\
\textbf{C3.} Matricası tómendegishe bolǵan sızıqlı túrlendiriwdiń menshikli mánisi hám menshikli vektorların tabıń: \(\begin{pmatrix} 1 & 1 & 1 & 1 \\ 1 & 1 & - 1 & - 1 \\ 1 & - 1 & 1 & - 1 \\ 1 & - 1 & - 1 & 1 \end{pmatrix}\). \\

\end{tabular}
\vspace{1cm}


\begin{tabular}{m{17cm}}
\textbf{97-variant}
\newline

\textbf{T1.} Kvadratlıq forma. (Lagran usulı, Yakobi usulı kvadratlıq formanı keltiriw.) \\
\textbf{T2.} Keri túrlendiriwler. ( Keri túrlendiriw túsinigi,   Keri túrlendiriwdiń sızıqlılıǵı) \\
\textbf{A1.} \(\mathbb{R}^{2}\) keńislikte anıqlangan tómendegi sáwlelendiriw skalyar kóbeyme bolatuģının anıqlań: \((x,y) = x_{1}y_{1} - 2x_{2}y_{1} - 2x_{1}y_{2} + x_{2}y_{2}\) \\
\textbf{A2.} Tómendegi kvadratlıq forma oń anıqlangan bolatuģın\(\lambda\) nıń barlıq mánislerin tabıń: \(2x_{1}^{2} + x_{2}^{2} + 3x_{3}^{2} + 2\lambda x_{1}x_{2} + 2x_{1}x_{3}\); \\
\textbf{A3.} Tómendegi sáwlelendiriwmos ravishda Berilgen \(V\) vektor keńislikte sızıqlı túrlendiriw boladı: \(V\) sızıqlı keńislik,\(Ax = \alpha x\) bul jerde \(\alpha\)-fiksirlangan son; \\
\textbf{B1.} Ortogonallastırıw procesinen paydalanıp, berilgen vektorlar sistemasini ortogonallastirıń: \((1,2,2, - 1)\), ( \(1,1, - 5,3\) ), (3, 2, 8, -7); \\
\textbf{B2.} Tómendegi kvadratlıq formanıń kanonikalıq kórinisin hám bul túrge keltiriwshi menshikli mánislerin tabıń: \(2x_{1}^{2} + 3x_{2}^{2} + 4x_{3}^{2} - 2x_{1}x_{2} + 4x_{1}x_{3} - 3x_{2}x_{3}\); \\
\textbf{B3.} Matricası tómendegishe bolǵan sızıqlı túrlendiriwdiń menshikli mánisi hám menshikli vektorların tabıń: \(\begin{pmatrix} 7 & 0 & 0 \\ 10 & - 19 & 0 \\ 12 & - 24 & 13 \end{pmatrix}\); \\
\textbf{C1.} Berilgen \(A\) bisızıqlı formanıń\(e_{1},e_{2},e_{3}\) bazisdegi matritsası hám\(e_{1}^{'},e_{2}^{'},e_{3}^{'}\) baziske ótiw formulaları berilgen bolsa, onda bul bisızıqli formanıń \(e_{1}^{'},e_{2}^{'},e_{3}^{'}\) bazisdegi matritsasini tabıń: \(\begin{pmatrix} 1 & 1 & 2 \\  - 1 & 2 & 1 \\  - 1 & 1 & - 1 \end{pmatrix},\begin{matrix}  & e_{1}^{'} = e_{1} + e_{2} - 2e_{3} \\  & e_{2}^{'} = e_{1} + e_{2} + 2e_{3} \\  & e_{3}^{'} = e_{2} + e_{3} \end{matrix}\) \\
\textbf{C2.} Tómendegi bisiziqli formalar ekvivalent emes ekenligin dálilleń:\(f_{1}(x,y) = 2x_{1}y_{2} - 3x_{1}y_{3} + x_{2}y_{3} - 2x_{2}y_{1} - x_{3}y_{2} - 3x_{3}y_{1}\),\(f_{2}(x,y) = x_{1}y_{2} - x_{2}y_{1} + 2x_{2}y_{2} + 3x_{1}y_{3} - 3x_{3}y_{1};\) \\
\textbf{C3.} Matricası tómendegishe bolǵan sızıqlı túrlendiriwdiń menshikli mánisi hám menshikli vektorların tabıń: \(\begin{pmatrix} 1 & - 3 & 4 \\ 4 & - 7 & 8 \\ 6 & - 7 & 7 \end{pmatrix}\); \\

\end{tabular}
\vspace{1cm}


\begin{tabular}{m{17cm}}
\textbf{98-variant}
\newline

\textbf{T1.} İnertsiya nızamı. (invariantlar,  eki kvadratılıq forma arasindaǵı baylanıs ) \\
\textbf{T2.} Unitar túrlendiriwler. (Unitar sızıqlı túrlendiriwler túsinigi,  Unitar túrlendiriwge túyinles túrlendiriwlerdiń matricası,   Ortonormal baziste unitar túrlendiriwlerdiń matricası) \\
\textbf{A1.} \(\mathbb{R}^{2}\) keńislikte anıqlangan tómendegi sáwlelendiriw skalyar kóbeyme bolatuģının anıqlań: \((x,y) = x_{1}y_{1} - x_{2}y_{1} - x_{1}y_{2} + x_{2}y_{2}\) \\
\textbf{A2.} Tómendegi vektorlar sisteması óz ara ortogonallıqqa tekseriń hám olardı ortogonallıq baziske shekem toltırın: \((1, - 2,2, - 3),(2, - 3,2,4)\); \\
\textbf{A3.} Tómendegi sáwlelendiriwlerden qaysıları keńislikte sızıqlı túrlendiriw boladı: 1. \\
\textbf{B1.} Ortogonallastırıw procesinen paydalanıp, berilgen vektorlar sistemasini ortogonallastirıń: \((1,1, - 1,0)\), \((2,0, - 1,0)\), \((1, - 1,1, - 1)\), (2, \(0,1,1\) ). \\
\textbf{B2.} Tómendegi kvadratlıq formanıń kanonikalıq kórinisin hám bul túrge keltiriwshi menshikli mánislerin tabıń: \(x_{1}x_{2} + x_{1}x_{3} + x_{2}x_{3}\); \\
\textbf{B3.} Matricası tómendegishe bolǵan sızıqlı túrlendiriwdiń menshikli mánisi hám menshikli vektorların tabıń: \(\begin{pmatrix} 4 & - 5 & 2 \\ 0 & - 7 & 3 \\ 0 & 0 & 4 \end{pmatrix}\); \\
\textbf{C1.} Berilgen \(A\) bisızıqlı formanıń\(e_{1},e_{2},e_{3}\) bazisdegi matritsası hám\(e_{1}^{'},e_{2}^{'},e_{3}^{'}\) baziske ótiw formulaları berilgen bolsa, onda bul bisızıqli formanıń \(e_{1}^{'},e_{2}^{'},e_{3}^{'}\) bazisdegi matritsasini tabıń: \(\begin{pmatrix} 2 & 2 & 3 \\  - 4 & 3 & 1 \\ 3 & 1 & 2 \end{pmatrix},\ \begin{matrix}  & e_{1}^{'} = e_{1} + 3e_{2} - 2e_{3} \\  & e_{2}^{'} = 2e_{1} + e_{2} - e_{3} \\  & e_{3}^{'} = e_{1} + e_{2} - 3e_{3} \end{matrix}\) \\
\textbf{C2.} Tómendegi kvadratlıq formalarga sáykes keliwshi ermit bisızıqlı formalardı tabiń: \(ix_{1}\overline{x_{2}} - ix_{2}\overline{x_{1}} + (3 - 2i)x_{1}\overline{x_{3}} + (3 + 2i)x_{3}\overline{x_{1}} + 2x_{2}\overline{x_{3}} + 2x_{3}\overline{x_{2}}\). \\
\textbf{C3.} Ortogonallastırıw procesinen paydalanıp, berilgen vektorlar sistemasini ortogonallastirıń: \((2,1,3, - 1)\), ( \(7,4,3, - 3\) ), ( \(1,1, - 6,0\) ), (5, 7, 7, 8); \\

\end{tabular}
\vspace{1cm}


\begin{tabular}{m{17cm}}
\textbf{99-variant}
\newline

\textbf{T1.} İnertsiya nızamı. (invariantlar,  eki kvadratılıq forma arasindaǵı baylanıs ) \\
\textbf{T2.} Óz-ara orın almasıwshı túrlendiriwler. (Óz-ara orın almasıwshı túrlendiriwler,  Ortogonal bazis haqqında teorema,  Normal túrlendiriwlerdiń kanonikalıq kórinisi) \\
\textbf{A1.} \(\mathbb{R}^{2}\) keńislikte anıqlangan tómendegi sáwlelendiriw skalyar kóbeyme bolatuģının anıqlań: \((x,y) = x_{1}y_{1} + 2x_{2}y_{2}\) \\
\textbf{A2.} Tómendegi kvadratlıq forma oń anıqlangan bolatuģın\(\lambda\) nıń barlıq mánislerin tabıń: \(x_{1}^{2} + \lambda x_{2}^{2} + x_{3}^{2} - 4x_{1}x_{2} - 8x_{2}x_{3}\); \\
\textbf{A3.} Tómendegi sáwlelendiriwlerden qaysıları keńislikte sızıqlı túrlendiriw boladı:; \\
\textbf{B1.} Ortogonallastırıw procesinen paydalanıp, berilgen vektorlar sistemasini ortogonallastirıń: \((1,1, - 1)\), (1, 1,1 ), \((3,2, - 1)\); \\
\textbf{B2.} Tómendegi kvadratlıq formanı kanonikalıq kóriniske keltiriń: \(x_{1}x_{2} + x_{1}x_{3} + x_{1}x_{4} + x_{2}x_{3} + x_{2}x_{4} + x_{3}x_{4}\); \\
\textbf{B3.} Tómendegi vektorlar sisteması óz ara ortogonallıqqa tekseriń hám olardı ortogonallıq baziske shekem toltırın: \((0,i,1,1),\ \ (1,2,1 + i, - 1 + i)\). \\
\textbf{C1.} Berilgen \(A\) bisızıqlı formanıń\(e_{1},e_{2},e_{3}\) bazisdegi matritsası hám\(e_{1}^{'},e_{2}^{'},e_{3}^{'}\) baziske ótiw formulaları berilgen bolsa, onda bul bisızıqli formanıń\(e_{1}^{'},e_{2}^{'},e_{3}^{'}\) bazisdegi matritsasini tabıń: \(\ \) \(\begin{pmatrix} 0 & 2 & 1 \\  - 2 & 2 & 0 \\  - 1 & 0 & 3 \end{pmatrix}\), \(e_{1}^{'} = e_{1} + 2e_{2} - e_{3}\), \(e_{2}^{'} = e_{2} - e_{3}\), \(e_{3}^{'} = - e_{1} + e_{2} - 3e_{3}\) \\
\textbf{C2.} Tómendegi kvadratlıq forma oń anıqlangan bolatuģın\(\lambda\) parametrnıń barlıq mánislerin tabıń: \(\lambda x_{1}\overline{x_{1}} - ix_{1}\overline{x_{2}} + ix_{2}\overline{x_{1}} + 3x_{2}\overline{x_{2}}\); \\
\textbf{C3.} Matricası tómendegishe bolǵan sızıqlı túrlendiriwdiń menshikli mánisi hám menshikli vektorların tabıń: \(\begin{pmatrix} 2 & - 1 & 2 \\ 5 & - 3 & 3 \\  - 1 & 0 & - 2 \end{pmatrix}\); \\

\end{tabular}
\vspace{1cm}


\begin{tabular}{m{17cm}}
\textbf{100-variant}
\newline

\textbf{T1.} Sızıqlı, bisızıqlı hám kvadratlıq formalar. (Bisızıqlı forma,  simmetriyalı bisızıqlı formalar)  \\
\textbf{T2.} Túyinles túrlendiriw. ( Evklid keńisligindegi sızıqlı túrlendiriwler menen bisızıqlı formalar arasındaǵı baylanıs, Berilgen túrlendiriwge túyinles túrlendiriwler, Óz-ózine túyinles túrlendiriwler) \\
\textbf{A1.} Tómendegi kvadratlıq formanıń rangni anıqlań: \(3x_{1}^{2} - 2x_{2}^{2} + 2x_{3}^{2} + 4x_{1}x_{2} - 3x_{1}x_{3} - x_{2}x_{3}\); \\
\textbf{A2.} Tómendegi kvadratlıq forma oń anıqlangan bolatuģın\(\lambda\) nıń barlıq mánislerin tabıń: \(5x_{1}^{2} + x_{2}^{2} + \lambda x_{3}^{2} + 4x_{1}x_{2} - 2x_{1}x_{3} - 2x_{2}x_{3}\); \\
\textbf{A3.} Tómendegi sáwlelendiriwlerden qaysıları keńislikte sızıqlı túrlendiriw boladı:. \\
\textbf{B1.} Ortogonallastırıw procesinen paydalanıp, berilgen vektorlar sistemasini ortogonallastirıń: \((1,0,0)\), (0, 1, -1), (1, 1, 1); \\
\textbf{B2.} Eger \(f\) sızıqlı funkciya\(e_{1},e_{2},e_{3}\) bazisde \(f(x) = 2x_{1} - 3x_{2} + x_{3}\) arqalı anıqlanǵan bolsa, onıń \(e_{1}^{'},e_{2}^{'},e_{3}^{'}\) bazisdegi kórinisin tabıń\(e_{1}^{'} = e_{1} + e_{2} - 2e_{3},\ e_{2}^{'} = e_{1} + e_{2} + 2e_{3},\ e_{3}^{'} = e_{2} + e_{3}\); \\
\textbf{B3.} Tómendegi vektorlar sisteması óz ara ortogonallıqqa tekseriń hám olardı ortogonallıq baziske shekem toltırın: \((1,1,1,1),(1,1, - 1, - 1),(1, - 1,1, - 1)\); \\
\textbf{C1.} Jordan normal formasını tabıń \(\begin{pmatrix} 4 & 1 & - 4 \\ 1 & 4 & 0 \\  - 4 & 0 & 4 \end{pmatrix}\); \\
\textbf{C2.} Bazi bir ortonormal bazisde berilgen kvadratlıq formani kanonik kóriniske keltiriwshi ortonormal bazisin tabıń: \(11x_{1}^{2} + 5x_{2}^{2} + 2x_{3}^{2} + 16x_{1}x_{2} + 4x_{1}x_{3} - 20x_{2}x_{3}\); \\
\textbf{C3.} Matricası tómendegishe bolǵan sızıqlı túrlendiriwdiń menshikli mánisi hám menshikli vektorların tabıń: \(\begin{pmatrix} 5 & 6 & - 3 \\  - 1 & 0 & 1 \\ 1 & 2 & - 1 \end{pmatrix}\); \\

\end{tabular}
\vspace{1cm}



\end{document}
