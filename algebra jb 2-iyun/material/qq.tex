Sızıqlı keńislikler.   (Vektor,  sızıqlı baylanıs, bazis, ólshem, ) 
Evklid keńisligi. (Skalyar kóbeyme, ortogonal vektorlar, ortonormal bazis.)
Ortogonal  tolıqtırıwshı. (Ortogonal tolıqtırıwshı,  ortogonal proekciya)
Sızıqlı, bisızıqlı hám kvadratlıq formalar. (Bisızıqlı forma,  simmetriyalı bisızıqlı formalar) 
Kvadratlıq forma. (Lagran usulı, Yakobi usulı kvadratlıq formanı keltiriw.)
İnertsiya nızamı. (invariantlar,  eki kvadratılıq forma arasindaǵı baylanıs )
Kompleks evklid keńislikleri.  (Kompleks vektorlı keńislik, Ermit kvadratlıq forma.)
++++
Sızıqlı túrlendiriwler.  (Sızıqlı túrlendiriw túsinigi, Sızıqlı túrlendiriwler ústinde ámeller, Sızıqlı túrlendiriwlerdiń obrazı hám yadrosı.)
Keri túrlendiriwler. ( Keri túrlendiriw túsinigi,   Keri túrlendiriwdiń sızıqlılıǵı)
Túyinles túrlendiriw. ( Evklid keńisligindegi sızıqlı túrlendiriwler menen bisızıqlı formalar arasındaǵı baylanıs, Berilgen túrlendiriwge túyinles túrlendiriwler, Óz-ózine túyinles túrlendiriwler)
Unitar túrlendiriwler. (Unitar sızıqlı túrlendiriwler túsinigi,  Unitar túrlendiriwge túyinles túrlendiriwlerdiń matricası,   Ortonormal baziste unitar túrlendiriwlerdiń matricası)
Óz-ara orın almasıwshı túrlendiriwler. (Óz-ara orın almasıwshı túrlendiriwler,  Ortogonal bazis haqqında teorema,  Normal túrlendiriwlerdiń kanonikalıq kórinisi)
Sızıqlı túrlendiriwler matritsasınıń Jordan normal kórinisi. (Jordan kletkasınıń xarakteristikalıq matricası, Jordan matricasınıń uqsaslıǵı haqqında teorema,  Matricalardı jordan normal kórinisine keltiriw)
++++
Tómendegi kvadratlıq formanıń rangni anıqlań: \(x_{1}x_{2} + x_{1}x_{3} + x_{2}x_{3}\);
Tómendegi kvadratlıq formanıń rangni anıqlań: \(2x_{1}^{2} + 3x_{2}^{2} + 4x_{3}^{2} - 2x_{1}x_{2} + 4x_{1}x_{3} - 3x_{2}x_{3}\)
Tómendegi kvadratlıq formanıń rangni anıqlań: \(3x_{1}^{2} - 2x_{2}^{2} + 2x_{3}^{2} + 4x_{1}x_{2} - 3x_{1}x_{3} - x_{2}x_{3}\);
Tómendegi kvadratlıq formanıń rangni anıqlań: \(x_{1}^{2} - 2x_{2}^{2} - 2x_{3}^{2} - 4x_{1}x_{2} - 4x_{1}x_{3} + 8x_{2}x_{3}\);
Tómendegi kvadratlıq formanıń rangni anıqlań: \(x_{1}x_{2} + x_{2}x_{3} + x_{3}x_{4} + x_{1}x_{4}\);
Tómendegi kvadratlıq formanıń rangni anıqlań: \(x_{1}^{2} + x_{2}^{2} + x_{3}^{2} + x_{4}^{2} + 2x_{1}x_{2} - 2x_{1}x_{4} - 2x_{2}x_{3} + 2x_{3}x_{4}\).
\(\mathbb{R}^{2}\) keńislikte anıqlangan tómendegi sáwlelendiriw skalyar kóbeyme bolatuģının anıqlań: \((x,y) = x_{1}y_{1} + 2x_{2}y_{2}\)
\(\mathbb{R}^{2}\) keńislikte anıqlangan tómendegi sáwlelendiriw skalyar kóbeyme bolatuģının anıqlań: \((x,y) = x_{1}y_{1} - x_{2}y_{2}\)
\(\mathbb{R}^{2}\) keńislikte anıqlangan tómendegi sáwlelendiriw skalyar kóbeyme bolatuģının anıqlań: \((x,y) = x_{1}y_{1} + x_{2}y_{1} + 3x_{1}y_{2} + 2x_{2}y_{2}\)
\(\mathbb{R}^{2}\) keńislikte anıqlangan tómendegi sáwlelendiriw skalyar kóbeyme bolatuģının anıqlań: \((x,y) = x_{1}y_{1} - x_{2}y_{1} - x_{1}y_{2} + x_{2}y_{2}\)
\(\mathbb{R}^{2}\) keńislikte anıqlangan tómendegi sáwlelendiriw skalyar kóbeyme bolatuģının anıqlań: \((x,y) = x_{1}y_{1} + 2x_{2}y_{1} + 2x_{1}y_{2} + 7x_{2}y_{2}\)
\(\mathbb{R}^{2}\) keńislikte anıqlangan tómendegi sáwlelendiriw skalyar kóbeyme bolatuģının anıqlań: \((x,y) = x_{1}y_{1} - 2x_{2}y_{1} - 2x_{1}y_{2} + x_{2}y_{2}\)
++++
Tómendegi vektorlar sisteması óz ara ortogonallıqqa tekseriń hám olardı ortogonallıq baziske shekem toltırın: \((1,2, - 1),(3, - 1,1)\);
Tómendegi vektorlar sisteması óz ara ortogonallıqqa tekseriń hám olardı ortogonallıq baziske shekem toltırın: \((2,1,2),\ (1,2, - 2)\);
Tómendegi vektorlar sisteması óz ara ortogonallıqqa tekseriń hám olardı ortogonallıq baziske shekem toltırın: \((1, - 2,2, - 3),(2, - 3,2,4)\);
Tómendegi vektorlar sisteması óz ara ortogonallıqqa tekseriń hám olardı ortogonallıq baziske shekem toltırın: \((1,1,1,2)\), \((1,2,3, - 3)\).
Tómendegi funkciyalı haqiqiy sanlar maydanı ústinde anıqlangan \(V\) keńislikte sızıqlı funkciya boladı: \(V = \mathbb{R}^{3},\ \ f(x) = |x|\);
Tómendegi funkciyalı haqiqiy sanlar maydanı ústinde anıqlangan \(V\) keńislikte sızıqlı funkciya boladı: \(V = M_{n}\left( \mathbb{R} \right),\ \ f(A) = \det(A)\);
Tómendegi kvadratlıq forma oń anıqlangan bolatuģın\(\lambda\) nıń barlıq mánislerin tabıń: \(5x_{1}^{2} + x_{2}^{2} + \lambda x_{3}^{2} + 4x_{1}x_{2} - 2x_{1}x_{3} - 2x_{2}x_{3}\);
Tómendegi kvadratlıq forma oń anıqlangan bolatuģın\(\lambda\) nıń barlıq mánislerin tabıń: \(2x_{1}^{2} + x_{2}^{2} + 3x_{3}^{2} + 2\lambda x_{1}x_{2} + 2x_{1}x_{3}\);
Tómendegi kvadratlıq forma oń anıqlangan bolatuģın\(\lambda\) nıń barlıq mánislerin tabıń: \(x_{1}^{2} + x_{2}^{2} + 5x_{3}^{2} + 2\lambda x_{1}x_{2} - 2x_{1}x_{3} + 4x_{2}x_{3}\);
Tómendegi kvadratlıq forma oń anıqlangan bolatuģın\(\lambda\) nıń barlıq mánislerin tabıń: \(x_{1}^{2} + 4x_{2}^{2} + x_{3}^{2} + 2\lambda x_{1}x_{2} + 10x_{1}x_{3} + 6x_{2}x_{3}\);
Tómendegi kvadratlıq forma oń anıqlangan bolatuģın\(\lambda\) nıń barlıq mánislerin tabıń: \(2x_{1}^{2} + 2x_{2}^{2} + x_{3}^{2} + 2\lambda x_{1}x_{2} + 6x_{1}x_{3} + 2x_{2}x_{3}\);
Tómendegi kvadratlıq forma oń anıqlangan bolatuģın\(\lambda\) nıń barlıq mánislerin tabıń: \(x_{1}^{2} + \lambda x_{2}^{2} + x_{3}^{2} - 4x_{1}x_{2} - 8x_{2}x_{3}\);
++++
Tómendegi sáwlelendiriw\(V = \mathbb{R}^{3}\) keńislikte sızıqlı túrlendiriw boladı: \(A\left( x_{1},x_{2},x_{3} \right) = \left( x_{1} + 2,x_{2} + 5,x_{3} \right)\);
Tómendegi sáwlelendiriw\(V = \mathbb{R}^{3}\) keńislikte sızıqlı túrlendiriw boladı: \(A\left( x_{1},x_{2},x_{3} \right) = \left( x_{1},x_{2} + 1,x_{3} + 2 \right)\);
Tómendegi sáwlelendiriw\(V = \mathbb{R}^{3}\) keńislikte sızıqlı túrlendiriw boladı: \(A\left( x_{1},x_{2},x_{3} \right) = \left( x_{1},x_{2},x_{1} + x_{2} + x_{3} \right)\);
Tómendegi sáwlelendiriw\(V = \mathbb{R}^{3}\) keńislikte sızıqlı túrlendiriw boladı: \(A\left( x_{1},x_{2},x_{3} \right) = \left( x_{2} + x_{3},2x_{1} + x_{3},3x_{1} - x_{2} + x_{3} \right)\);
Tómendegi sáwlelendiriw\(V = \mathbb{R}^{3}\) keńislikte sızıqlı túrlendiriw boladı: \(A\left( x_{1},x_{2},x_{3} \right) = \left( 2x_{1} + x_{2},x_{1} + x_{3},x_{3}^{2} \right)\);
Tómendegi sáwlelendiriw\(V = \mathbb{R}^{3}\) keńislikte sızıqlı túrlendiriw boladı: \(A\left( x_{1},x_{2},x_{3} \right) = \left( x_{1} + 3x_{3},x_{2}^{3},x_{1} + x_{3} \right)\).
Tómendegi sáwlelendiriw \(V\) vektor keńislikte sızıqlı túrlendiriw boladı: \(V\) sızıqlı keńislik,\(Ax = a\), bul jerde \(a\)-fiksirlengen vektor;
Tómendegi sáwlelendiriwmos ravishda Berilgen \(V\) vektor keńislikte sızıqlı túrlendiriw boladı: \(V\) sızıqlı keńislik,\(Ax = x + a\), bul jerde \(a\)-fiksirlengen vektor;
Tómendegi sáwlelendiriwmos ravishda Berilgen \(V\) vektor keńislikte sızıqlı túrlendiriw boladı: \(V\) sızıqlı keńislik,\(Ax = \alpha x\) bul jerde \(\alpha\)-fiksirlangan son;
Matricası tómendegishe bolgan sızıqlı túrlendiriwdin menshikli mánisi hám menshikli vektorların tabıń: \(\begin{pmatrix} 2 & 1 \\ 1 & 2 \end{pmatrix}\);
Matricası tómendegishe bolgan sızıqlı túrlendiriwdin menshikli mánisi hám menshikli vektorların tabıń \(\begin{pmatrix} 3 & 4 \\ 5 & 2 \end{pmatrix}\);
Tómendegi sáwlelendiriwlerden qaysıları keńislikte sızıqlı túrlendiriw boladı: 1.
Tómendegi sáwlelendiriwlerden qaysıları keńislikte sızıqlı túrlendiriw boladı:;
Tómendegi sáwlelendiriwlerden qaysıları keńislikte sızıqlı túrlendiriw boladı:;
Tómendegi sáwlelendiriwlerden qaysıları keńislikte sızıqlı túrlendiriw boladı:;
Tómendegi sáwlelendiriwlerden qaysıları keńislikte sızıqlı túrlendiriw boladı:;
Tómendegi sáwlelendiriwlerden qaysıları keńislikte sızıqlı túrlendiriw boladı:.
Tómendegi ańlatpalardan qaysıları sáykes túrde berilgen vektor keńislikte sızıqlı túrlendiriw boladı: sızıqlı keńislik,, bul jerde -fiksirlengen vektor;
Tómendegi sáykes túrlendiriwlerden qaysıları berilgen vektor keńislikte sızıqlı túrlendiriw boladı: sızıqlı keńislik,, bul jerde -fiksirlengen vektor;
Tómendegi sáwlelendiriwlerden qaysıları sáykes túrde berilgen vektor keńislikte sızıqlı túrlendiriw boladı: sızıqlı keńislik, bul jerde -fiksirlengen san;
Matricası tómendegishe:
++++
\(\mathbb{R}^{2}\) keńislikte \((x,y) = x_{1}y_{1} + 2x_{2}y_{1} + 2x_{1}y_{2} + 5x_{2}y_{2}\) berilgen skalyar kóbeyme ushın \(a = (1,0)\) hám \(b = (0,1)\) vektorlar arasındaǵı múyeshti tabıń
\(\mathbb{R}^{3}\) keńislikte \((x,y) = x_{1}y_{1} + 3x_{2}y_{2} + 2x_{3}y_{3}\) berilgen skalyar kóbeyme ushın \(a = (1, - 3,2)\) va \(b = (2,1, - 1)\) \(b = (0,1)\) vektorlar arasındaǵı múyeshti tabıń .
Ortogonallastırıw procesinen paydalanıp, berilgen vektorlar sistemasini ortogonallastirıń: \((1,1,0,0)\), (1, 0, 1, 1);
Ortogonallastırıw procesinen paydalanıp, berilgen vektorlar sistemasini ortogonallastirıń: \((1,0,0)\), (0, 1, -1), (1, 1, 1);
Ortogonallastırıw procesinen paydalanıp, berilgen vektorlar sistemasini ortogonallastirıń: \((1,1, - 1)\), (1, 1,1 ), \((3,2, - 1)\);
Ortogonallastırıw procesinen paydalanıp, berilgen vektorlar sistemasini ortogonallastirıń: \((2,0,1,1)\), ( \(1,2,0,1\) ), ( \(0,1, - 2,0\) );
Ortogonallastırıw procesinen paydalanıp, berilgen vektorlar sistemasini ortogonallastirıń: \((1,2,1,3)\), (4, 1, 1, 1), (3, 1, 1, 0);
Ortogonallastırıw procesinen paydalanıp, berilgen vektorlar sistemasini ortogonallastirıń: \((1,2,2, - 1)\), ( \(1,1, - 5,3\) ), (3, 2, 8, -7);
Ortogonallastırıw procesinen paydalanıp, berilgen vektorlar sistemasini ortogonallastirıń: \((1,1, - 1, - 2)\), \((5,8, - 2, - 3)\), (3, 9, 3, 8);
Ortogonallastırıw procesinen paydalanıp, berilgen vektorlar sistemasini ortogonallastirıń: \((1,1, - 1,0)\), \((2,0, - 1,0)\), \((1, - 1,1, - 1)\), (2, \(0,1,1\) ).
++++
Eger \(f\) sızıqlı funkciya\(e_{1},e_{2},e_{3}\) bazisde \(f(x) = 2x_{1} - 3x_{2} + x_{3}\) arqalı anıqlanǵan bolsa, onıń \(e_{1}^{'},e_{2}^{'},e_{3}^{'}\) bazisdegi kórinisin tabıń\(e_{1}^{'} = e_{1} - e_{2},\ e_{2}^{'} = e_{1} + e_{3},\ \ e_{3}^{'} = e_{1} + e_{2} + e_{3}\);
Eger \(f\) sızıqlı funkciya\(e_{1},e_{2},e_{3}\) bazisde \(f(x) = 2x_{1} - 3x_{2} + x_{3}\) arqalı anıqlanǵan bolsa, onıń \(e_{1}^{'},e_{2}^{'},e_{3}^{'}\) bazisdegi kórinisin tabıń\(e_{1}^{'} = e_{1} + 3e_{2} - 2e_{3},\ e_{2}^{'} = 2e_{1} + e_{2} - e_{3},\ e_{3}^{'} = e_{1} + e_{2} - 3e_{3}\);
Eger \(f\) sızıqlı funkciya\(e_{1},e_{2},e_{3}\) bazisde \(f(x) = 2x_{1} - 3x_{2} + x_{3}\) arqalı anıqlanǵan bolsa, onıń \(e_{1}^{'},e_{2}^{'},e_{3}^{'}\) bazisdegi kórinisin tabıń\(e_{1}^{'} = e_{1} + e_{2} - 2e_{3},\ e_{2}^{'} = e_{1} + e_{2} + 2e_{3},\ e_{3}^{'} = e_{2} + e_{3}\);
Eger \(f\) sızıqlı funkciya\(e_{1},e_{2},e_{3}\) bazisde \(f(x) = 2x_{1} - 3x_{2} + x_{3}\) arqalı anıqlanǵan bolsa, onıń \(e_{1}^{'},e_{2}^{'},e_{3}^{'}\) bazisdegi kórinisin tabıń\(e_{1}^{'} = 4e_{1} - e_{2} - 3e_{3},\ e_{2}^{'} = 2e_{1} + e_{2},\ e_{3}^{'} = 3e_{1} + 2e_{2}\).
Tómendegi kvadratlıq formanıń kanonikalıq kórinisin hám bul túrge keltiriwshi menshikli mánislerin tabıń: \(x_{1}x_{2} + x_{1}x_{3} + x_{2}x_{3}\);
Tómendegi kvadratlıq formanıń kanonikalıq kórinisin hám bul túrge keltiriwshi menshikli mánislerin tabıń: \(2x_{1}^{2} + 3x_{2}^{2} + 4x_{3}^{2} - 2x_{1}x_{2} + 4x_{1}x_{3} - 3x_{2}x_{3}\);
Tómendegi kvadratlıq formanıń kanonikalıq kórinisin hám bul túrge keltiriwshi menshikli mánislerin tabıń: \(3x_{1}^{2} - 2x_{2}^{2} + 2x_{3}^{2} + 4x_{1}x_{2} - 3x_{1}x_{3} - x_{2}x_{3}\);
Tómendegi kvadratlıq formanıń kanonikalıq kórinisin hám bul túrge keltiriwshi menshikli mánislerin tabıń: \(x_{1}^{2} - 2x_{2}^{2} - 2x_{3}^{2} - 4x_{1}x_{2} - 4x_{1}x_{3} + 8x_{2}x_{3}\);
Tómendegi kvadratlıq formanıń kanonikalıq kórinisin hám bul túrge keltiriwshi menshikli mánislerin tabıń: \(5x_{1}^{2} + 6x_{2}^{2} + 4x_{3}^{2} - 4x_{1}x_{2} - 4x_{1}x_{3}\);
Tómendegi kvadratlıq formanıń kanonikalıq kórinisin hám bul túrge keltiriwshi menshikli mánislerin tabıń: \(7x_{1}^{2} + 5x_{2}^{2} + 3x_{3}^{2} - 8x_{1}x_{2} + 8x_{2}x_{3}\);
Tómendegi kvadratlıq formanı kanonikalıq kóriniske keltiriń: \(2x_{1}^{2} + 18x_{2}^{2} + 8x_{3}^{2} - 12x_{1}x_{2} + 8x_{1}x_{3} - 27x_{2}x_{3}\);
Tómendegi kvadratlıq formanı kanonikalıq kóriniske keltiriń: \(12x_{1}^{2} + 3x_{2}^{2} + 12x_{3}^{2} - 12x_{1}x_{2} + 24x_{1}x_{3} - 8x_{2}x_{3}\);
Tómendegi kvadratlıq formanı kanonikalıq kóriniske keltiriń: \(x_{1}x_{2} + x_{1}x_{3} + x_{1}x_{4} + x_{2}x_{3} + x_{2}x_{4} + x_{3}x_{4}\);
++++
Matricası tómendegishe bolǵan sızıqlı túrlendiriwdiń menshikli mánisi hám menshikli vektorların tabıń: \(\begin{pmatrix} 2 & - 1 & 2 \\ 0 & - 3 & 0 \\ 0 & 0 & 1 \end{pmatrix}\);
Matricası tómendegishe bolǵan sızıqlı túrlendiriwdiń menshikli mánisi hám menshikli vektorların tabıń: \(\begin{pmatrix} 0 & 0 & 1 \\ 1 & 4 & 0 \\  - 2 & 0 & 2 \end{pmatrix}\);
Matricası tómendegishe bolǵan sızıqlı túrlendiriwdiń menshikli mánisi hám menshikli vektorların tabıń: \(\begin{pmatrix} 4 & - 5 & 2 \\ 0 & - 7 & 3 \\ 0 & 0 & 4 \end{pmatrix}\);
Matricası tómendegishe bolǵan sızıqlı túrlendiriwdiń menshikli mánisi hám menshikli vektorların tabıń: \(\begin{pmatrix} 7 & 0 & 0 \\ 10 & - 19 & 0 \\ 12 & - 24 & 13 \end{pmatrix}\);
Tómendegi vektorlar sisteması óz ara ortogonallıqqa tekseriń hám olardı ortogonallıq baziske shekem toltırın: \((1,1,1,1),(1,1, - 1, - 1),(1, - 1,1, - 1)\);
Tómendegi vektorlar sisteması óz ara ortogonallıqqa tekseriń hám olardı ortogonallıq baziske shekem toltırın: \((1,2,0, - 1),(3, - 1,1,1),( - 1,2,2,3)\);
Tómendegi vektorlar sisteması óz ara ortogonallıqqa tekseriń hám olardı ortogonallıq baziske shekem toltırın: \((0,1,i),\ \ (1 + i,i,1)\);
Tómendegi vektorlar sisteması óz ara ortogonallıqqa tekseriń hám olardı ortogonallıq baziske shekem toltırın: \((1,i, - i),\ \ ( - 2 - i,1 + i,2 - i)\);
Tómendegi vektorlar sisteması óz ara ortogonallıqqa tekseriń hám olardı ortogonallıq baziske shekem toltırın. \((i,i,1, - 1),\ \ (1, - 1 + i,0,1),\ \ \);
Tómendegi vektorlar sisteması óz ara ortogonallıqqa tekseriń hám olardı ortogonallıq baziske shekem toltırın: \((0,i,1,1),\ \ (1,2,1 + i, - 1 + i)\).
++++
Berilgen \(A\) bisızıqlı formanıń\(e_{1},e_{2},e_{3}\) bazisdegi matritsası hám\(e_{1}^{'},e_{2}^{'},e_{3}^{'}\) baziske ótiw formulaları berilgen bolsa, onda bul bisızıqli formanıń\(e_{1}^{'},e_{2}^{'},e_{3}^{'}\) bazisdegi matritsasini tabıń: \(\begin{pmatrix} 1 & 2 & 3 \\ 4 & 5 & 6 \\ 7 & 8 & 9 \end{pmatrix}\), \(e_{1}^{'} = e_{1} - e_{2}\), \(e_{2}^{'} = e_{1} + e_{3}\), \(e_{3}^{'} = e_{1} + e_{2} + e_{3}\)
Berilgen \(A\) bisızıqlı formanıń\(e_{1},e_{2},e_{3}\) bazisdegi matritsası hám\(e_{1}^{'},e_{2}^{'},e_{3}^{'}\) baziske ótiw formulaları berilgen bolsa, onda bul bisızıqli formanıń\(e_{1}^{'},e_{2}^{'},e_{3}^{'}\) bazisdegi matritsasini tabıń: \(\ \) \(\begin{pmatrix} 0 & 2 & 1 \\  - 2 & 2 & 0 \\  - 1 & 0 & 3 \end{pmatrix}\), \(e_{1}^{'} = e_{1} + 2e_{2} - e_{3}\), \(e_{2}^{'} = e_{2} - e_{3}\), \(e_{3}^{'} = - e_{1} + e_{2} - 3e_{3}\)
Berilgen \(A\) bisızıqlı formanıń\(e_{1},e_{2},e_{3}\) bazisdegi matritsası hám\(e_{1}^{'},e_{2}^{'},e_{3}^{'}\) baziske ótiw formulaları berilgen bolsa, onda bul bisızıqli formanıń \(e_{1}^{'},e_{2}^{'},e_{3}^{'}\) bazisdegi matritsasini tabıń: \(\begin{pmatrix} 1 & 1 & 2 \\  - 1 & 2 & 1 \\  - 1 & 1 & - 1 \end{pmatrix},\begin{matrix}  & e_{1}^{'} = e_{1} + e_{2} - 2e_{3} \\  & e_{2}^{'} = e_{1} + e_{2} + 2e_{3} \\  & e_{3}^{'} = e_{2} + e_{3} \end{matrix}\)
Berilgen \(A\) bisızıqlı formanıń\(e_{1},e_{2},e_{3}\) bazisdegi matritsası hám\(e_{1}^{'},e_{2}^{'},e_{3}^{'}\) baziske ótiw formulaları berilgen bolsa, onda bul bisızıqli formanıń \(e_{1}^{'},e_{2}^{'},e_{3}^{'}\) bazisdegi matritsasini tabıń: \(\begin{pmatrix} 2 & 2 & 3 \\  - 4 & 3 & 1 \\ 3 & 1 & 2 \end{pmatrix},\ \begin{matrix}  & e_{1}^{'} = e_{1} + 3e_{2} - 2e_{3} \\  & e_{2}^{'} = 2e_{1} + e_{2} - e_{3} \\  & e_{3}^{'} = e_{1} + e_{2} - 3e_{3} \end{matrix}\)
Jordan normal formasını tabıń \(\begin{pmatrix} 2 & - 1 & 2 \\ 0 & - 3 & 0 \\ 0 & 0 & 1 \end{pmatrix}\);
Jordan normal formasını tabıń \(\begin{pmatrix} 0 & 0 & 1 \\ 1 & 4 & 0 \\  - 2 & 0 & 2 \end{pmatrix}\);
Jordan narmal formasini tabıń\(\begin{pmatrix} 4 & - 5 & 2 \\ 0 & - 7 & 3 \\ 0 & 0 & 4 \end{pmatrix}\);
Jordan normal formasını tabıń \(\begin{pmatrix} 7 & 0 & 0 \\ 10 & - 19 & 0 \\ 12 & - 24 & 13 \end{pmatrix}\);
Jordan normal formasını tabıń \(\begin{pmatrix} 4 & 1 & - 4 \\ 1 & 4 & 0 \\  - 4 & 0 & 4 \end{pmatrix}\);
Jordan normal formasını tabıń \(\begin{pmatrix} 1 & - 2 & 1 \\  - 2 & 1 & 4 \\  - 1 & 4 & 1 \end{pmatrix}\).
++++
Tómendegi bisiziqli formalar ekvivalent emes ekenligin dálilleń:\(f_{1}(x,y) = 2x_{1}y_{2} - 3x_{1}y_{3} + x_{2}y_{3} - 2x_{2}y_{1} - x_{3}y_{2} - 3x_{3}y_{1}\),\(f_{2}(x,y) = x_{1}y_{2} - x_{2}y_{1} + 2x_{2}y_{2} + 3x_{1}y_{3} - 3x_{3}y_{1};\)
Tómendegi bisiziqli formalar ekvivalent emes ekenligin dálilleń:\(f_{1}(x,y) = x_{1}y_{1} + 2x_{1}y_{2} + 2x_{2}y_{1} + 5x_{2}y_{2} + 6x_{2}y_{3} + 8x_{3}y_{2} + 10x_{3}y_{3}\), \(f_{2}(x,y) = 2x_{1}y_{1} - x_{1}y_{3} + x_{2}y_{2} - x_{3}y_{1} + 5x_{3}y_{3}\).
Tómendegi kvadratlıq formalarga sáykes keliwshi ermit bisızıqlı formalardı tabiń. \(x_{1}\overline{x_{1}} - ix_{1}\overline{x_{2}} - ix_{2}\overline{x_{1}} + 2x_{2}\overline{x_{2}}\);
Tómendegi kvadratlıq formalarga sáykes keliwshi ermit bisızıqlı formalardı tabiń:\((5 - i)x_{1}\overline{x_{2}} + (5 + i)\overline{x_{1}}x_{2} + x_{2}\overline{x_{2}}\);
Tómendegi kvadratlıq formalarga sáykes keliwshi ermit bisızıqlı formalardı tabiń: \(x_{1}\overline{x_{1}} + (2 + i)x_{1}\overline{x_{2}} + (2 - i)x_{2}\overline{x_{1}} + ix_{1}\overline{x_{3}} - ix_{3}\overline{x_{1}} - x_{3}\overline{x_{3}}\);
Tómendegi kvadratlıq formalarga sáykes keliwshi ermit bisızıqlı formalardı tabiń: \(ix_{1}\overline{x_{2}} - ix_{2}\overline{x_{1}} + (3 - 2i)x_{1}\overline{x_{3}} + (3 + 2i)x_{3}\overline{x_{1}} + 2x_{2}\overline{x_{3}} + 2x_{3}\overline{x_{2}}\).
Tómendegi kvadratlıq forma oń anıqlangan bolatuģın\(\lambda\) parametrnıń barlıq mánislerin tabıń: \(x_{1}\overline{x_{1}} + ix_{1}\overline{x_{2}} - ix_{2}\overline{x_{1}} + \lambda x_{2}\overline{x_{2}}\);
Tómendegi kvadratlıq forma oń anıqlangan bolatuģın\(\lambda\) parametrnıń barlıq mánislerin tabıń: \(\lambda x_{1}\overline{x_{1}} - ix_{1}\overline{x_{2}} + ix_{2}\overline{x_{1}} + 3x_{2}\overline{x_{2}}\);
Bazi bir ortonormal bazisde berilgen kvadratlıq formani kanonik kóriniske keltiriwshi ortonormal bazisin tabıń: \(x_{1}^{2} - 5x_{2}^{2} + x_{3}^{2} + 4x_{1}x_{2} + 2x_{1}x_{3} + 4x_{2}x_{3}\);
Bazi bir ortonormal bazisde berilgen kvadratlıq formani kanonik kóriniske keltiriwshi ortonormal bazisin tabıń: \(11x_{1}^{2} + 5x_{2}^{2} + 2x_{3}^{2} + 16x_{1}x_{2} + 4x_{1}x_{3} - 20x_{2}x_{3}\);
Bazi bir ortonormal bazisde berilgen kvadratlıq formani kanonik kóriniske keltiriwshi ortonormal bazisin tabıń: \(x_{1}^{2} + x_{2}^{2} + 5x_{3}^{2} - 6x_{1}x_{2} - 2x_{1}x_{3} + 2x_{2}x_{3}\);
Bazi bir ortonormal bazisde berilgen kvadratlıq formani kanonik kóriniske keltiriwshi ortonormal bazisin tabıń: \(x_{1}^{2} + x_{2}^{2} + x_{3}^{2} + 4x_{1}x_{2} + 4x_{1}x_{3} + 4x_{2}x_{3}\);
++++
Ortogonallastırıw procesinen paydalanıp, berilgen vektorlar sistemasini ortogonallastirıń: \((2,1,3, - 1)\), ( \(7,4,3, - 3\) ), ( \(1,1, - 6,0\) ), (5, 7, 7, 8);
Ortogonallastırıw procesinen paydalanıp, berilgen vektorlar sistemasini ortogonallastirıń: \((1,1, - 1,0)\), \((2,0, - 1,0)\), \((1, - 1,1, - 1)\), (2, \(0,1,1\) ).
Matricası tómendegishe bolǵan sızıqlı túrlendiriwdiń menshikli mánisi hám menshikli vektorların tabıń: \(\begin{pmatrix} 1 & - 3 & 4 \\ 4 & - 7 & 8 \\ 6 & - 7 & 7 \end{pmatrix}\);
Matricası tómendegishe bolǵan sızıqlı túrlendiriwdiń menshikli mánisi hám menshikli vektorların tabıń: \(\begin{pmatrix} 5 & 6 & - 3 \\  - 1 & 0 & 1 \\ 1 & 2 & - 1 \end{pmatrix}\);
Matricası tómendegishe bolǵan sızıqlı túrlendiriwdiń menshikli mánisi hám menshikli vektorların tabıń: \(\begin{pmatrix} 2 & - 1 & 2 \\ 5 & - 3 & 3 \\  - 1 & 0 & - 2 \end{pmatrix}\);
Matricası tómendegishe bolǵan sızıqlı túrlendiriwdiń menshikli mánisi hám menshikli vektorların tabıń: \(\begin{pmatrix} 0 & 2 & 1 \\  - 2 & 0 & 3 \\  - 1 & - 3 & 0 \end{pmatrix}\);
Matricası tómendegishe bolǵan sızıqlı túrlendiriwdiń menshikli mánisi hám menshikli vektorların tabıń: \(\begin{pmatrix} 1 & 0 & 0 & 0 \\ 0 & 0 & 0 & 0 \\ 0 & 0 & 0 & 0 \\ 1 & 0 & 0 & 0 \end{pmatrix}\);
Matricası tómendegishe bolǵan sızıqlı túrlendiriwdiń menshikli mánisi hám menshikli vektorların tabıń: \(\begin{pmatrix} 1 & 0 & 0 & 0 \\ 0 & 0 & 0 & 0 \\ 1 & 0 & 0 & 0 \\ 0 & 0 & 0 & 1 \end{pmatrix}\);
Matricası tómendegishe bolǵan sızıqlı túrlendiriwdiń menshikli mánisi hám menshikli vektorların tabıń: \(\begin{pmatrix} 3 & - 1 & 0 & 0 \\ 1 & 1 & 0 & 0 \\ 3 & 0 & 5 & - 3 \\ 4 & - 1 & 3 & - 1 \end{pmatrix}\);
Matricası tómendegishe bolǵan sızıqlı túrlendiriwdiń menshikli mánisi hám menshikli vektorların tabıń: \(\begin{pmatrix} 1 & 1 & 1 & 1 \\ 1 & 1 & - 1 & - 1 \\ 1 & - 1 & 1 & - 1 \\ 1 & - 1 & - 1 & 1 \end{pmatrix}\).