\documentclass{article}
\usepackage[fontsize=11pt]{fontsize}
\usepackage[utf8]{inputenc}
\usepackage[T2A]{fontenc}
% \usepackage{unicode-math}

\usepackage{array}
\usepackage[a4paper,
left=7mm,
right=5mm,
top=7mm,]{geometry}
\usepackage{amsmath}
% \usepackage{amssymbol}
\usepackage{amsfonts}
\usepackage{setspace}



\renewcommand{\baselinestretch}{1} 

\everymath{\displaystyle}
\everydisplay{\displaystyle}
% \linespread{1.25}

\DeclareMathOperator{\sign}{sign}


\begin{document}

\pagenumbering{gobble}


\begin{tabular}{m{17cm}}
\textbf{1-вариант}
\newline

T1. 
Понятие о векторе. Линейные операции над векторами.
 \\
T2. 
Взаимное расположение прямой на плоскости.
 \\
A1. 
Установить, какие из следующих пар прямых перпендикулярны: 1) \(3x - y + 5 = 0,x + 3y - 1 = 0\); 2) \(3x - 4y + 1 = 0,\ \ \ \ 4x - 3y + 7 = 0\); 3) \(6x - 15y + 7 = 0,\ \ \ \ 10x + 4y - 3 = 0\); 4) \(9x - 12y + 5 = 0,\ \ \ \ 8x + 6y - 13 = 0\); 5) \(7x - 2y + 1 = 0,4x + 6y + 17 = 0\); Решить задачу, не вычисляя угловых коэффициентов данных прямых.
 \\
A2. 
Даны вершины треугольника \(A(3;2; - 3)\), \(B(5;1; - 1)\) и \(C(1; - 2;1)\). Определить его внешний угол при вершине \(A\).
 \\
A3. 
Установить, какие из следующих пар уравнений определяют параллельные плоскости; 1) \(2x - 3y + 5z - 7 = 0,\ \ \ \ 2x - 3y + 5z + 3 = 0\); 2) \(4x + 2y - 4z + 5 = 0,\ \ \ \ 2x + y + 2z - 1 = 0\); 3) \(\ \ \ \ x - 3z + 2 = 0,\ \ \ \ 2x - 6z - 7 = 0\).
 \\
B1. 
Найти проекцию точки \(P( - 8;12)\) на прямую, проходящую через точки \(A(2; - 3)\) и \(B( - 5;1)\).
 \\
B2. 
Векторы \(\overrightarrow{a},\ \overrightarrow{b},\ \overrightarrow{c}\) образующие правую тройку, взаимно перпендикулярны. Зная, что \(|\overrightarrow{a}| = 4,\ \ |\overrightarrow{b}| = 2\), \(|\overrightarrow{c}| = 3\), вычислить \(\left( \left\lbrack \overrightarrow{a},\overrightarrow{b} \right\rbrack,\overrightarrow{c} \right)\).
 \\
B3. 
Составить параметрические уравнения следующих прямых: 1) \(2x + 3y - z - 4 = 0,3x - 5y + 2z + 1 = 0\); 2) \(x + 2y - z - 6 = 0,2x - y + z + 1 = 0\).
 \\
C1. 
Составить уравнения сторон треугольника, знал одну его вершину \(B(2; - 1)\), а также уравнения высоты \(3x - 4y + 27 = 0\) и биссектрисы \(x + 2y - 5 = 0\), проведенных из различных вершин.
 \\
C2. 
Доказать, что \(\lbrack\overrightarrow{a},\overrightarrow{b}\rbrack^{2} <  {\overrightarrow{a}}^{2}{\overrightarrow{b}}^{2}\); в каком случае здесь будет знак равенства?
 \\
C3. 
Составить уравнение плоскости, проходящей через прямую \(5x - y - 2z - 3 = 0,\ \ \ \ 3x - 2y - 5z + 2 = 0\) перпендикулярно плоскости \(x + 19y - 7z - 11 \doteq 0\).
 \\

\end{tabular}
\vspace{1cm}


\begin{tabular}{m{17cm}}
\textbf{2-вариант}
\newline

T1. 
Координаты вектора.
 \\
T2. 
Уравнения плоскости. Взаимное расположение плоскости.
 \\
A1. 
Вычислить величину отклонения \(\delta\) и расстояние \(d\) от точки до прямой в каждом из следующих случаев: 1) \(A(2; - 1),4x + 3y + 10 = 0;\) \(2)B(0; - 3),5x - 12y - 23 = 0;\); \(3)P( - 2;3),\ \ 3x - 4y - 2 = 0\); 4) \(Q(1; - 2),x - 2y - 5 = 0\).
 \\
A2. 
Даны вершины четырехугольника \(A(1; - 2;2)\), \(B(1;4;0),C( - 4;1;1)\) и \(D( - 5; - 5;3)\). Доказать, что его диагонали \(AC\) и \(BD\) взаимно перпендикулярны.
 \\
A3. 
Точка \(P(2; - 1; - 1)\) служит основанием перпендикуляра, опущенного из начала координат на плоскость. Составить уравнение этой плоскости.
 \\
B1. 
В каждом из следующих случаев составить уравнение прямой, параллельной двум данным прямым и проходящей посередине между ними: 1)  \(3x - 2y - 1 = 0,3x - 2y - 13 = 0\); 2)  \(5x + y + 3 = 0,5x + y - 17 = 0\); 3)  \(2x + 3y - 6 = 0,\ \ \ \ 4x + 6y + 17 = 0\).
 \\
B2. 
Доказать, что точки \(A(1;2; - 1),B(0;1;5)\), \(C( - 1;2;1),D(2;1;3)\) лежат в одной плоскости.
 \\
B3. Составить уравнение плоскости, проходящей через точку \(M_{0}(3;4; - 5)\) параллельно векторам \({\overrightarrow{a}}_{1} = \{ 3;1; - 1\}\) и \({\overrightarrow{a}}_{2} = \{ 1; - 2;1\}\).
 \\
C1. 
Составить уравнения сторон треугольника, зная одну из его вершин \(A(4; - 1)\) и уравнения двух биссектрис \(x - 1 = 0\) и \(x - y - 1 = 0\).
 \\
C2. 
Доказать тождество \(\lbrack\overrightarrow{a},\overrightarrow{b}\rbrack^{2} + (\overrightarrow{a},\overrightarrow{b})^{2} = {\overrightarrow{a}}^{2}{\overrightarrow{b}}^{2}\).
 \\
C3. 
Даны вершины треугольника \(A(3; - 1; - 3)\), \(B(1;2; - 7)\) и \(C( - 5;14; - 3)\). Составить канонические уравнения биссектрисы его внутреннего угла при вершине \(C\).
 \\

\end{tabular}
\vspace{1cm}


\begin{tabular}{m{17cm}}
\textbf{3-вариант}
\newline

T1. Предмет и методы аналитической геометрии.
 \\
T2. 
Расстояние от точки до прямой. Уравнение пучка прямых.
 \\
A1. 
Даны вершины \(A(2; - 1;4),B(3;2; - 6),C( - 5\); 0 ; 2) треугольника. Вычислить длину его медианы, проведенной из вершины \(A\). \\
A2. 
Векторы \(\overrightarrow{a}\) и \(\overrightarrow{b}\) образуют угол \(\varphi = \pi/6\). Зная, что \(|\overrightarrow{a}| = 6,|\overrightarrow{b}| = 5\), вычислить \(\left| \left\lbrack \overrightarrow{a},\overrightarrow{b} \right\rbrack \right|\)
 \\
A3. 
Составить параметрические уравнения прямой, проходящей через точку \(M_{1}(1; - 1; - 3)\) параллельно: 1) вектору \(\overrightarrow{a} = \{ 2; - 3;4\}\); 2) прямой \(\frac{x - 1}{2} = \frac{y + 2}{4} = \frac{z - 1}{0}\); 3) прямой \(x = 3t - 1,y = - 2t + 3,z = 5t + 2\).
 \\
B1. 
Отклонения точки \(M\) от прямых \(5x - 12y - 13 = 0\) и \(3x - 4y - 19 = 0\) равны соответственно - 3 и -5. Определить координаты точки \(M\).
 \\
B2. 
Даны векторы \(\overrightarrow{a} = \{ 3; - 1; - 2\}\) и \(\overrightarrow{b} = \{ 1;2; - 1\}\), Найти координаты векторных произведений: 1) \(\left\lbrack \overrightarrow{a},\overrightarrow{b} \right\rbrack\); 2)\(\left\lbrack 2\overrightarrow{a} + \overrightarrow{b},\overrightarrow{b} \right\rbrack\); 3) \(\left\lbrack 2\overrightarrow{a} - \overrightarrow{b},2\overrightarrow{a} + \overrightarrow{b} \right\rbrack\).
 \\
B3. 
Составить уравнение плоскости, которая проходит через прямую пересечения плоскостей \(3x - y + 2z + 9 = 0\), \(x + z - 3 = 0\): 1) и через точку \(M_{1}(4; - 2; - 3)\); 2)параллельно оси \(Ox\); 3) параллельно оси \(Oy\); 4) параллельно оси \(Oz\).
 \\
C1. 
Даны вершины треугольника \(A(1; - 1),B( - 2;1)\) и \(C(3;5)\). Составить уравнение перпендикуляра, опущенного из вершины \(A\) на медиану, проведенную из вершины \(B\).
 \\
C2. 
Доказать, что необходимым и достаточным условием компланарности векторов \(\overrightarrow{a},\ \overrightarrow{b},\ \overrightarrow{c}\) является зависимость \(\alpha\overrightarrow{a} + \beta\overrightarrow{b} + \gamma\overrightarrow{c} = 0\), где по крайней мере одно из чисел \(\alpha,\beta,\gamma\) не равно нулю. \\
C3. 
Составить уравнение плоскости, проходящей через прямую пересечения плоскостей \(3x - 2y + z - 3 = 0,x - 2z = 0\) перпендикулярно плоскости \(x - 2y + z + 5 = 0\).
 \\

\end{tabular}
\vspace{1cm}


\begin{tabular}{m{17cm}}
\textbf{4-вариант}
\newline

T1. 
Семейство линейно зависимых и линейно независимых векторов.
 \\
T2. Уравнении прямой на плоскости.
 \\
A1. 
Определить угол \(\varphi\) между двумя прямыми: 1) \(5x - y + 7 = 0,\ \ \ \ 3x + 2y = 0\); 2) \(3x - 2y + 7 = 0,2x + 3y - 3 = 0\); 3) \(x - 2y - 4 = 0,\ \ \ \ 2x - 4y + 3 = 0\);
 \\
A2. 
Вычислить косинус угла, образованного векторами \(\overrightarrow{a} = \{ 2; - 4;4\}\) и \(\overrightarrow{b} = \{ - 3;2;6\}\).
 \\
A3. 
При каком значении \(m\) прямая \(\frac{x + 1}{3} = \frac{y - 2}{m} = \frac{z + 3}{- 2}\) параллельна плоскости \(x - 3y + 6z + 7 = 0\) ?
 \\
B1. 
Площадь треугольника \(S = 3\), две его вершины суть точки \(A(3;1)\) и \(B(1; - 3)\), а третья вершина \(C\) лежит на оси \(Oy\). Определить координаты вершины \(C\).
 \\
B2. 
На плоскости даны три вектора \(\overrightarrow{a} = \{ 3; - 2\}\), \(\overrightarrow{b} = \{ - 2;1\}\) и \(\overrightarrow{c} = \{ 7; - 4\}\). Определить разложение каждого из этих трех векторов, принимая в качестве базиса два других.
 \\
B3. 
Даны прямые \(\frac{x + 2}{2} = \frac{y}{- 3} = \frac{z - 1}{4},\ \ \ \ \frac{x - 3}{l} = \frac{y - 1}{4} = \frac{z - 7}{2}\) при каком значении \(l\) они пересекаются?
 \\
C1. 
Даши вершины треугольника \(A(3; - 5),B( - 3;3)\) и \(C( - 1; - 2)\). Определить длину биссектрисы его внутреннего угла при вершине \(A\). (С помощью деление отрезка в данном отношение)
 \\
C2. 
Доказать, что вектор \(\overrightarrow{p} = \overrightarrow{b}(\overrightarrow{a},\overrightarrow{c}) - \overrightarrow{c}(\overrightarrow{a},\overrightarrow{b})\) перпендикулярен к вектору \(\overrightarrow{a}\).
 \\
C3. Составить уравнение плоскости, проходящей через прямую пересечения плоскостей \(2x - y + 3z - 5 = 0,x + 2y - z + 2 = 0\) параллельно вектору \(\overrightarrow{l} = \{ 2; - 1; - 2\}\).
 \\

\end{tabular}
\vspace{1cm}


\begin{tabular}{m{17cm}}
\textbf{5-вариант}
\newline

T1. 
Выражение скалярного, векторного и смешанного произведения векторов в координатах.
 \\
T2. 
Уравнения прямой в пространстве. Взаимное расположение прямых.
 \\
A1. 
Дано уравнение пучка прямых \(\alpha(3x + 2y - 9) +\) \(+ \beta(2x + 5y + 5) = 0\). Найти, при каком значении \(C\) прямая \(4x - 3y + C = 0\) будет принадлежать этому пучку.
 \\
A2. 
Даны: \(|\overrightarrow{a}| = 3,|\overrightarrow{b}| = 26\) и \(|\lbrack\overrightarrow{a},\overrightarrow{b}\rbrack| = 72\). Вычислить \(\left( \overrightarrow{a},\overrightarrow{b} \right)\).
 \\
A3. 
Найти точку пересечения прямой и плоскости: 1) \(\frac{x - 1}{1} = \frac{y + 1}{- 2} = \frac{z}{6},\ \ \ \ 2x + 3y + z - 1 = 0\); 2) \(\frac{x + 3}{3} = \frac{y - 2}{- 1} = \frac{z + 1}{- 5},\ \ \ \ x - 2y + z - 15 = 0\); 3) \(\frac{x + 2}{- 2} = \frac{y - 1}{3} = \frac{z - 3}{2},\ \ \ \ x + 2y - 2z + 6 = 0\).
 \\
B1. 
Площадь треугольника \(S = 3\), две его вершины суть точки \(A(3;1)\) и \(B(1; - 3)\), центр массе этого треугольника лежит на оси \(Ox\). Определить координаты третьей вершины \(C\).
 \\
B2. 
Векторы \(\overrightarrow{a}\) и \(\overrightarrow{b}\) взаимно перпендикулярны. Зная, что \(|\overrightarrow{a}| = 3,|\overrightarrow{b}| = 4\), вычислить \(1)|\lbrack\overrightarrow{a} + \overrightarrow{b},\overrightarrow{a} - \overrightarrow{b}\rbrack|;\ \ 2)|\lbrack 3\overrightarrow{a} - \overrightarrow{b},\overrightarrow{a} - 2\overrightarrow{b}\rbrack|\).
 \\
B3. 
Составить канонические уравнения следующих прямых: 1) \(x - 2y + 3z - 4 = 0,3x + 2y - 5z - 4 = 0\); 2) \(5x + y + z = 0,2x + 3y - 2z + 5 = 0\); 3) \(x - 2y + 3z + 1 = 0,2x + y - 4z - 8 = 0\).
 \\
C1. 
Даны вершины треугольника \(A(2; - 2),B(3; - 5)\) и \(C(5;7)\). Составить уравнение перпендикуляра, опущенного из вершины \(C\) на биссектрису внутреннего угла при вершине \(A\).
 \\
C2. 
Доказать, что вектор \(\overrightarrow{p} = \overrightarrow{b}(\overrightarrow{a},\overrightarrow{c}) - \overrightarrow{c}(\overrightarrow{a},\overrightarrow{b})\) перпендикулярен к вектору \(\overrightarrow{a}\).
 \\
C3. 
На плоскости \(Oxz\) найти такую точку \(P\), разность расстояний которой до точек \(M_{1}(3;2; - 5)\) и \(M_{2}(8; - 4\); -13) была бы наибольшей.
 \\

\end{tabular}
\vspace{1cm}


\begin{tabular}{m{17cm}}
\textbf{6-вариант}
\newline

T1. 
Векторное произведение и смешанное произведение векторов.
 \\
T2. 
Взаимное расположение прямой и плоскости в пространстве.
 \\
A1. 
Даны последовательные вершин однородной четырехугольной пластинки \(A(2;1),B(5;3),C( - 1;7)\) и \(D( - 7;5)\). Определить координат ее центра масс.
 \\
A2. 
Вычислив внутренние углы треугольника с вершинами \(A(1;2;1),B(3; - 1;7),C(7;4; - 2)\), убедиться, что этот треугольник равнобедренный.
 \\
A3. 
Составить канонические уравнения прямой, проходящей через данные точки: 1) \((1; - 2;1),(3;1; - 1)\); 2) \((3; - 1;0),(1;0, - 3);\) \(3)(0; - 2;3),(3; - 2;1)\); 4) \((1;2; - 4),( - 1;2; - 4)\).
 \\
B1. 
Дано уравнение пучка прямых \(\alpha(2x + y + 1) + \beta(x - 3y - 10) = 0\) . Найти прямые этого пучка, отсекающие на координатных осях отрезки равной длины (считая от начала координат).
 \\
B2. 
Векторы \(\overrightarrow{a}\) и \(\overrightarrow{b}\) взаимно перпендикулярны; вектор \(\overrightarrow{c}\) образует с ними углы, равные \(\pi/3\); зная, что \(|\overrightarrow{a}| = 3\), \(|\overrightarrow{b}| = 5,\ \ \ \ |\overrightarrow{c}| = 8\), вычислить: 1) \(\left( 3\overrightarrow{a} - 2\overrightarrow{b},\overrightarrow{b} + 3\overrightarrow{c} \right)\); 2) \((\overrightarrow{a} + \overrightarrow{b} + \overrightarrow{c})^{2};\) \(3)(\overrightarrow{a} + 2\overrightarrow{b} - 3\overrightarrow{c})^{2}\).
 \\
B3. 
Вычислить площадь треугольника, который отсекает плоскость \(5x - 6y + 3z + 120 = 0\) от координатного угла \(Oxy\).
 \\
C1. 
Составить уравнения сторон треугольника \(ABC\), зная одну его вершину \(A(2; - 1)\), а также уравнения высоты \(7x - 10y + 1 = 0\) и 6иссектрисы \(3x - 2y + 5 = 0\), проведенных из одной вершины. Решить задачу, не вычисляя координат вершин \(B\) и \(C\).
 \\
C2. 
Какому условия должны удовлетворять векторы \(\overrightarrow{a}\) и \(\overrightarrow{b}\), чтобы вектор \(\overrightarrow{a} + \overrightarrow{b}\) был перпендикулярен к вектору \(\overrightarrow{a} - \overrightarrow{b}\).
 \\
C3. 
Составить уравнение плоскости, проходящей через прямую пересечения плоскостей \(5x - 2y - z - 3 = 0,x + 3y - 2z + 5 = 0\) параллельно вектору \(\overrightarrow{l} = \{ 7;9;17\}\).
 \\

\end{tabular}
\vspace{1cm}


\begin{tabular}{m{17cm}}
\textbf{7-вариант}
\newline

T1. 
Скалярное произведение векторов.
 \\
T2. 
Расстояние от точки до плоскости, от точки до прямой в пространстве и между двумя скрещивающими прямыми. \\
A1. 
Центр масс однородного стержня находится в точке \(M(1;4)\), один из его концов в точке \(P( - 2;2)\). Определить координаты точки \(Q\) - другого конца этого стержня.
 \\
A2. 
Даны: \(|\overrightarrow{a}| = 10,|\overrightarrow{b}| = 2\) и \(\left( \overrightarrow{a},\overrightarrow{b} \right) = 12\). Вычислить \(\left| \left\lbrack \overrightarrow{a},\overrightarrow{b} \right\rbrack \right|\).
 \\
A3. 
Определить, при каком значении \(l\) следующие парь уравнений будут определять перпендикулярные плоскости: 1) \(3x - 5y + lz - 3 = 0,x + 3y + 2z + 5 = 0\); 2) \(5x + y - 3z - 3 = 0,2x + ly - 3z + 1 = 0\); 3) \(\ \ \ \ 7x - 2y - z = 0,\ \ \ \ lx + y - 3z - 1 = 0\).
 \\
B1. 
Площадь треугольника \(S = 4\), две его вершины суть точки \(A(2;1)\) и \(\dot{B}(3; - 2)\), а третья вершина \(C\) лежит на оси \(Ox\). Определить координаты вершины \(C\).
 \\
B2. 
Найти вектор \(\overrightarrow{x}\), коллинеарный вектору \(\overrightarrow{a} = \{ 2;1; - 1\}\) и удовлетворяющий условию \(\left( \overrightarrow{x},\overrightarrow{a} \right) = 3\).
 \\
B3. 
Составить уравнение плоскости, проходящей через точки \(M_{1}(2; - 1;3)\) и \(M_{2}(3;1;2)\) параллельно вектору \(\overrightarrow{a} = \{ 3; - 1;4\}\).
 \\
C1. 
В треугольнике \(ABC\) даны уравнения высоты \(AN\): \(x + 5y - 3 = 0\), высоты \(BN:x + y - 1 = 0\) и стороны \(AB:x + 3y - 1 = 0\) . Не определяя координат вершин и точки пересечения высот треугольника, составить уравнение двух других сторон н третьей высоты.
 \\
C2. 
Даны единичнье векторы \(\overrightarrow{a},\ \overrightarrow{b}\) и \(\overrightarrow{c}\), удовлетворяющие условию \(\overrightarrow{a} + \overrightarrow{b} + \overrightarrow{c} = 0\). Вычислить \(\left( \overrightarrow{a},\overrightarrow{b} \right) + \left( \overrightarrow{b},\overrightarrow{c} \right) + \left( \overrightarrow{c},\overrightarrow{a} \right)\).
 \\
C3. 
Даны вершины треугольника \(A(1; - 2; - 4)\), \(B(3;1; - 3)\) и \(C(5;1; - 7)\). Составить параметрические уравнения его высоты, опущенной из вершины \(B\) на противоположную сторону.
 \\

\end{tabular}
\vspace{1cm}


\begin{tabular}{m{17cm}}
\textbf{8-вариант}
\newline

T1. 
Преобразование декартовой системы координат на плоскости и пространстве. \\
T2. 
Расстояние от точки до плоскости, от точки до прямой в пространстве и между двумя скрещивающими прямыми. \\
A1. 
Доказать, что треугольник с вершинами \(A(3; - 1;2)\), \(B(0; - 4;2)\) и \(C( - 3;2;1)\) равнобедренный.
 \\
A2. 
Даны вершины треугольника \(A( - 1; - 2;4)\), \(B( - 4; - 2;0)\) и \(C(3; - 2;1)\). Определить его внутренний угол при вершине \(B\).
 \\
A3. 
Доказать, что прямая \(x = 3t - 2,y = - 4t + 1\), \(z = 4t - 5\) параллельна плоскости \(4x - 3y - 6z - 5 = 0\).
 \\
B1. 
Даны вершины треугольника \(M_{1}(2;1),M_{2}( - 1; - 1)\) и \(M_{3}(3;2)\). Составить уравнения его высот.
 \\
B2. 
Векторы \(\overrightarrow{a}\) и \(\overrightarrow{b}\) взаимно перпендикулярны; вектор \(\overrightarrow{c}\) образует с ними углы, равные \(\pi/3\); зная, что \(|\overrightarrow{a}| = 3\), \(|\overrightarrow{b}| = 5,\ \ \ \ |\overrightarrow{c}| = 8\), вычислить: 1) \(\left( 3\overrightarrow{a} - 2\overrightarrow{b},\overrightarrow{b} + 3\overrightarrow{c} \right)\); 2) \((\overrightarrow{a} + \overrightarrow{b} + \overrightarrow{c})^{2};\) \(3)(\overrightarrow{a} + 2\overrightarrow{b} - 3\overrightarrow{c})^{2}\).
 \\
B3. 
Составить уравнение плоскости, проходящей через точки \(M_{1}(3; - 1;2),M_{2}(4; - 1; - 1)\) и \(M_{3}(2;0;2)\).
 \\
C1. 
Доказать, что прямая \(2x + y + 3 = 0\) пересекает отрезок, ограниченный точками \(A( - 5;1)\) и \(B(3;7)\).
 \\
C2. 
Какому условию должны удовлетворять векторы \(\overrightarrow{a},\overrightarrow{b}\), чтобы векторы \(\overrightarrow{a} + \overrightarrow{b}\) и \(\overrightarrow{a} - \overrightarrow{b}\) были коллинеарны?
 \\
C3. 
Даны вершины треугольника \(A(2; - 1; - 3)\), \(B(5;2; - 7)\) и \(C( - 7;11;6)\). Составить канонические уравнения биссектрисы его внешнего угла при вершине \(A\).
 \\

\end{tabular}
\vspace{1cm}


\begin{tabular}{m{17cm}}
\textbf{9-вариант}
\newline

T1. Предмет и методы аналитической геометрии.
 \\
T2. 
Уравнения плоскости. Взаимное расположение плоскости.
 \\
A1. 
Даны вершины \(M_{1}(3;2; - 5),M_{2}(1; - 4;3)\) и \(M_{3}( - 3\); \(0;1)\) треугольника. Найти середины его сторон.
 \\
A2. 
Определить, при каком значении \(\alpha\) векторы \(\overrightarrow{a} = \alpha\overrightarrow{i} - 3\overrightarrow{j} + 2\overrightarrow{k}\) и \(\overrightarrow{b} = \overrightarrow{i} + 2\overrightarrow{j} - \alpha\overrightarrow{k}\) взаимно перпендикулярны.
 \\
A3. 
В каждом из следующих случаев вычислить расстояние между параллельными плоскостями: 1) \(x - 2y - 2z - 12 = 0,x - 2y - 2z - 6 = 0\); 2) \(2x - 3y + 6z - 14 = 0,4x - 6y + 12z + 21 = 0\); 3) \(2x - y + 2z + 9 = 0,4x - 2y + 4z - 21 = 0\); 4) \(16x + 12y - 15z + 50 = 0,\ \ \ \ 16x + 12y - 15z + 25 = 0\);
 \\
B1. 
Вершины треугольника суть точки \(A(3;6),B( - 1\); 3) и \(C(2; - 1)\). Вычислить длину его высоты, проведенной из вершины C. (С помощью площадью треугольника)
 \\
B2. 
Даны точки \(A(2; - 1;2),B(1;2; - 1)\) и \(C(3;2;1)\). Найти координаты векторных пронзведений: 1) \(\lbrack\overline{AB},\overline{BC}\rbrack\); 2) \(\lbrack\overline{BC} - 2\overline{CA},\overline{CB}\rbrack\).
 \\
B3. 
Вычислить объем пирамиды, ограниченной плоскостью \(2x - 3y + 6z - 12 = 0\) и координатными плоскостями.
 \\
C1. 
На оси ординат найти такую точку \(P\), чтобы разность расстояний ее до точек \(M( - 3;2)\) и \(N(2;5)\) была наибольшей.
 \\
C2. 
Векторы \(\overrightarrow{a},\ \overrightarrow{b},\ \overrightarrow{c}\) и \(\overrightarrow{d}\) связаны соотношениями \(\lbrack\overrightarrow{a},\overrightarrow{b}\rbrack = \lbrack\overrightarrow{c},\overrightarrow{d}\rbrack,\ \ \lbrack\overrightarrow{a},\overrightarrow{c}\rbrack = \lbrack\overrightarrow{b},\overrightarrow{d}\rbrack\). Доказать коллинеарность векторов \(\overrightarrow{a} - \overrightarrow{d}\) и \(\overrightarrow{b} - \overrightarrow{c}\).
 \\
C3. 
Составить уравнения прямой, которая проходит через точку \(M_{1}( - 1;2; - 3)\) перпендикулярно к вектору \(\overrightarrow{a} = \{ 6; - 2; - 3\}\) и пересекает прямую \(\frac{x - 1}{3} = \frac{y + 1}{2} = \frac{z - 3}{- 5}\).
 \\

\end{tabular}
\vspace{1cm}


\begin{tabular}{m{17cm}}
\textbf{10-вариант}
\newline

T1. 
Преобразование декартовой системы координат на плоскости и пространстве. \\
T2. 
Взаимное расположение прямой и плоскости в пространстве.
 \\
A1. 
Дано уравнение пучка прямых \(\alpha(3x + y - 1) + \beta(2x - y - 9) = 0\) . Доказать, что прямая \(x + 3y + 13 = 0\) при надлежит этому пучку.
 \\
A2. 
На плоскости даны два вектора \(\overrightarrow{p} = \{ 2; - 3\}\), \(\overrightarrow{q} = \{ 1;2\}\). Найти разложение вектора \(\overrightarrow{a} = \{ 9;4\}\) по базису \(\overrightarrow{p},\ \overrightarrow{q}\).
 \\
A3. Составить уравнение плоскости, которая проходит через точку \(M_{1}(2;1; - 1)\) и имеет нормальный вектор \(\overrightarrow{n} = \{ 1; - 2;3\}\).
 \\
B1. Даны три вершины параллелограмма \(A(3; - 5)\), \(B(5; - 3)\) и \(C( - 1;3)\). Определить четвертую вершину \(D\), противоположную \(B\).
 \\
B2. 
Дано, что \(|\overrightarrow{a}| = 3,|\overrightarrow{b}| = 5\). Определить, при каком значении \(\alpha\) векторы \(\overrightarrow{a} + \alpha\overrightarrow{b},\overrightarrow{a} - \alpha\overrightarrow{b}\) будут взаимно перпендикулярны.
 \\
B3. 
В пучке плоскостей \(2x - 3y + z - 3 + \lambda(x + 3y + 2z + 1) = 0\) найти плоскость, которая: 1) проходит через точку \(M_{1}(1; - 2;3)\); 2) параллельна оси \(Ox\); 3) параллельна оси \(Oy\); 4) параллельна оси \(Oz\).
 \\
C1. 
Дано уравнение пучка прямых \(\alpha(3x - 4y - 3) +\) \(+ \beta(2x + 3y - 1) = 0\). Написать уравнение прямой этого пучка, проходящей через центр масс однородной треугольной пластинки, вершины которой суть точки \(A( - 1;2),B(4; - 4)\) и \(C(6; - 1)\).
 \\
C2. 
Доказать тождество \(\lbrack\overrightarrow{a},\overrightarrow{b}\rbrack^{2} + (\overrightarrow{a},\overrightarrow{b})^{2} = {\overrightarrow{a}}^{2}{\overrightarrow{b}}^{2}\).
 \\
C3. 
Даны вершины треугольника \(A(3;6; - 7),B( - 5\); \(2;3)\) и \(C(4; - 7; - 2)\). Составить параметрические уравнения его медианы, проведенной из вершины \(C\).
 \\

\end{tabular}
\vspace{1cm}


\begin{tabular}{m{17cm}}
\textbf{11-вариант}
\newline

T1. 
Скалярное произведение векторов.
 \\
T2. 
Уравнения прямой в пространстве. Взаимное расположение прямых.
 \\
A1. 
Точки \(M(2; - 1),N( - 1;4)\) и \(P( - 2;2)\) являются серединами сторон треугольника. Определить его вершины.
 \\
A2. Может ли вектор составлять с координатными осями следующие углы: 1) \(\alpha = 45^{{^\circ}},\beta = 60^{{^\circ}},\gamma = 120^{{^\circ}}\); 2) \(\alpha = 45^{{^\circ}},\ \ \ \ \beta = 135^{{^\circ}},\ \ \ \ \gamma = 60^{{^\circ}}\); 3) \(\alpha = 90^{{^\circ}},\ \ \ \ \beta = 150^{{^\circ}}\), \(\gamma = 60^{{^\circ}}?\)
 \\
A3. 
Вычислить величину отклонения \(\delta\) и расстояние \(d\) от точки до плоскости в каждом из следующих случаев: 1) \(M_{1}( - 2; - 4;3)\), \(2x - y + 2z + 3 = 0;\) 2) \(M_{2}(2; - 1; - 1),16x - 12y + 15z - 4 = 0\); 3) \(M_{3}(1;2; - 3),\ \ \ \ 5x - 3y + z + 4 = 0\);
 \\
B1. 
Определить, при каких значениях \(m\) и \(n\) две прямые \(mx + 8y + n = 0,\ \ \ \ 2x + my - 1 = 0\) 1) параллельны; 2) совпадают; 3) перпендикулярны.
 \\
B2. 
Вектор \(\overrightarrow{x}\), коллинеарный вектору \(\overrightarrow{a} = \{ 6; - 8; - 7,5\}\), образует острый угол с осью Oz. Зная, что \(|\overrightarrow{x}| = 50\), найти его координаты.
 \\
B3. 
Доказать, что прямая \(5x - 3y + 2z - 5 = 0\), \(2x - y - z - 1 = 0\) лежит в плоскости \(4x - 3y + 7z - 7 = 0\).
 \\
C1. 
Определить координаты точки \(O^{'}\) - нового начала координат, если точка \(A(3; - 4)\) лежит на новой оси абсцисс, а точка \(B(2;3)\) лежит на новой оси ординат, причем оси старой и новой систем координат имеют соответственно одинаковые направления.
 \\
C2. 
Даны векторы \(\overrightarrow{AB} = \overrightarrow{b}\) и \(\overrightarrow{AC} = \overrightarrow{c}\), совпадающие со сторонами треугольника \(ABC\). Найти разложение вектора, приложенного к вершине \(B\) этого треугольника и совпадающего с его высотой \(BD\) по базису \(\overrightarrow{b},\ \overrightarrow{c}\).
 \\
C3. 
Найти проекцию точки \(C(3; - 4; - 2)\) на плоскость, проходящую через параллельные прямые \(\frac{x - 5}{13} = \frac{y - 6}{1} = \frac{z + 3}{- 4},\ \ \ \ \frac{x - 2}{13} = \frac{y - 3}{1} = \frac{z + 3}{- 4}\).
 \\

\end{tabular}
\vspace{1cm}


\begin{tabular}{m{17cm}}
\textbf{12-вариант}
\newline

T1. 
Понятие о векторе. Линейные операции над векторами.
 \\
T2. 
Взаимное расположение прямой на плоскости.
 \\
A1. 
Определить площадь параллелограмма, три вершины которого суть точки \(A( - 2;3),B(4; - 5)\) и \(C( - 3;1)\). (С помощью площадью треугольника)
 \\
A2. 
Даны векторы \(\overrightarrow{a} = \{ 1; - 1;3\},\ \ \ \overrightarrow{b} = \{ - 2;2;1\}\), \(\overrightarrow{c} = \{ 3; - 2;5\}\). Вычислить \((\lbrack\overrightarrow{a},\overrightarrow{b}\rbrack,\overrightarrow{c})\).
 \\
A3. 
Привести каждое из следующих уравнений плоскостей к нормальному виду: 1) \(2x - 2y + z - 18 = 0\); 2) \(\frac{3}{7}x - \frac{6}{7}y + \frac{2}{7}z + 3 = 0\); 3) \(4x - 6y - 12z - 11 = 0\); 4) \(- 4x - 4y + 2z + 1 = 0\); 5) \(5y - 12z + 26 = 0\);
 \\
B1. 
Составить уравнение прямой, которая проходит через точку пересечения прямых \(2x + y - 2 = 0,x - 5y - 23 = 0\) и делит пополам отрезок, ограниченный точками \(M_{1}(5; - 6)\) и \(M_{2}( - 1; - 4)\). Решить задачу, не вычисляя координат точки пересечения данных прямых.
 \\
B2. 
Даны векторы \(\overrightarrow{a},\ \overrightarrow{b}\) и \(\overrightarrow{c}\) удовлетворяющие условию \(\overrightarrow{a} + \overrightarrow{b} + \overrightarrow{c} = 0\). Зная, что \(|\overrightarrow{a}| = 3,\ \ \ \ |\overrightarrow{b}| = 1\) и \(|\overrightarrow{c}| = 4\), вычислить \(\left( \overrightarrow{a},\overrightarrow{b} \right) + \left( \overrightarrow{b},\overrightarrow{c} \right) + \left( \overrightarrow{c},\overrightarrow{a} \right)\).
 \\
B3. 
Составить уравнение плоскости, проходящей через точки \(M_{1}(3; - 1;2),M_{2}(4; - 1; - 1)\) и \(M_{3}(2;0;2)\).
 \\
C1. 
Стороны треугольника лежат на прямых \(x + 5y - 7 = 0,\) \(3x - 2y - 4 = 0,\) \(7x + y + 19 = 0\). Вычислить его площадь \(S\).
 \\
C2. Даны единичнье векторы \(\overrightarrow{a},\ \overrightarrow{b}\) и \(\overrightarrow{c}\), удовлетворяющие условию \(\overrightarrow{a} + \overrightarrow{b} + \overrightarrow{c} = 0\). Вычислить \(\left( \overrightarrow{a},\overrightarrow{b} \right) + \left( \overrightarrow{b},\overrightarrow{c} \right) + \left( \overrightarrow{c},\overrightarrow{a} \right)\).
 \\
C3. 
Найти точку \(Q\), симметричную точке \(P(4;1;6)\) относительно прямой \(x - y - 4z + 12 = 0,2x + y - 2z + 3 = 0\).
 \\

\end{tabular}
\vspace{1cm}


\begin{tabular}{m{17cm}}
\textbf{13-вариант}
\newline

T1. 
Выражение скалярного, векторного и смешанного произведения векторов в координатах.
 \\
T2. Уравнении прямой на плоскости.
 \\
A1. 
Определить, при каких значениях \(m\) и \(n\) прямая \((m + 2n - 3)x + (2m - n + 1)y + 6m + 9 = 0\) параллельна оси абсцисс и отсекает на оси ординат отрезок, равный -3 (считая от начала координат). Написать уравнение этой прямой.
 \\
A2. 
Установить, компланарны ли векторы \(\overrightarrow{a},\overrightarrow{b},\overrightarrow{c}\), если 1)\(a = \{ 2;3; - 1\},\ \ \ \ b = \{ 1; - 1;3\},\ \ \ \ c = \{ 1;9; - 11\}\); 2)\(a = \{ 3; - 2;1\},\ \ \ \ b = \{ 2;1;2\},\ \ \ \ c = \{ 3; - 1; - 2\}\); 3)\(a = \{ 2; - 1;2\},\ \ \ \ b = \{ 1;2; - 3\},\ \ \ \ c = \{ 3; - 4;7\}\). \\
A3. 
Доказать перпендикулярность прямых: 1) \(\frac{x}{1} = \frac{y - 1}{- 2} = \frac{z}{3}\) и \(3x + y - 5z + 1 = 0,2x + 3y - 8z + 3 = 0\); 2) \(x = 2t + 1,y = 3t - 2,z = - 6t + 1\) и \(2x + y - 4z + 2 = 0\), \(4x - y - 5z + 4 = 0\); 3) \(x + y - 3z - 1 = 0,\ \ \ \ 2x - y - 9z - 2 = 0\ \ \ \ \) и \(2x + y +\) \(+ 2z + 5 = 0,2x - 2y - z + 2 = 0\)
 \\
B1. 
Точка \(A(2; - 5)\) является вершиной квадрата, одна из сторон которого лежит на прямой \(x - 2y - 7 = 0\). Вычислить площадь этого квадрата.
 \\
B2. 
Даны векторы \(\overrightarrow{a} = \{ 3; - 1; - 2\}\) и \(\overrightarrow{b} = \{ 1;2; - 1\}\), Найти координаты векторных произведений: 1) \(\left\lbrack \overrightarrow{a},\overrightarrow{b} \right\rbrack\); 2)\(\left\lbrack 2\overrightarrow{a} + \overrightarrow{b},\overrightarrow{b} \right\rbrack\); 3) \(\left\lbrack 2\overrightarrow{a} - \overrightarrow{b},2\overrightarrow{a} + \overrightarrow{b} \right\rbrack\).
 \\
B3. 
Доказать, что прямая \(5x - 3y + 2z - 5 = 0\), \(2x - y - z - 1 = 0\) лежит в плоскости \(4x - 3y + 7z - 7 = 0\).
 \\
C1. 
Точка \(A( - 4;5)\) является вершиной квадрата, диагональ которого лежит на прямой \(7x - y + 8 = 0\). Составить уравнения сторон и второй диагонали этого квадрата.
 \\
C2. 
Даны векторы \(\overrightarrow{AB} = \overrightarrow{b}\) и \(\overrightarrow{AC} = \overrightarrow{c}\), совпадающие со сторонами треугольника \(ABC\). Найти разложение вектора, приложенного к вершине \(B\) этого треугольника и совпадающего с его высотой \(BD\) по базису \(\overrightarrow{b},\ \overrightarrow{c}\).
 \\
C3. 
Составить уравнение плоскости, проходящей через прямую \(5x - y - 2z - 3 = 0,\ \ \ \ 3x - 2y - 5z + 2 = 0\) перпендикулярно плоскости \(x + 19y - 7z - 11 \doteq 0\).
 \\

\end{tabular}
\vspace{1cm}


\begin{tabular}{m{17cm}}
\textbf{14-вариант}
\newline

T1. 
Векторное произведение и смешанное произведение векторов.
 \\
T2. 
Расстояние от точки до прямой. Уравнение пучка прямых.
 \\
A1. 
Даны вершины треугольника \(A(1; - 3),B(3; - 5)\) и \(C( - 5;7)\). Определить середины его сторон.
 \\
A2. 
Определить, при каком значении \(\alpha\) векторы \(\overrightarrow{a} = \alpha\overrightarrow{i} - 3\overrightarrow{j} + 2\overrightarrow{k}\) и \(\overrightarrow{b} = \overrightarrow{i} + 2\overrightarrow{j} - \alpha\overrightarrow{k}\) взаимно перпендикулярны.
 \\
A3. 
Вычислить расстояние \(d\) от точки \(P(2;3; - 1)\) до следующих прямых: 1) \(\frac{x - 5}{3} = \frac{y}{2} = \frac{z + 25}{- 2}\); 2) \(x = t + 1,y = t + 2,z = 4t + 13\).
 \\
B1. 
Отрезок, ограниченный точками \(A(1; - 3)\) и \(B(4;3)\), разделен на три равные части. Определить координаты точек деления.
 \\
B2. 
Вычислить объем тетраэдра, вершины которого находятся в точках \(A(2; - 1;1),B(5;5;4),C(3;2; - 1)\) и \(D(4;1;3)\). \\
B3. Составить уравнение плоскости, проходящей через точку \(M_{0}(3;4; - 5)\) параллельно векторам \({\overrightarrow{a}}_{1} = \{ 3;1; - 1\}\) и \({\overrightarrow{a}}_{2} = \{ 1; - 2;1\}\).
 \\
C1. 
Последовательные вершины четырехугольника суть точки \(A( - 3;5),B( - 1; - 4),C(7; - 1)\) и \(D(2;9)\). Установить, является ли этот четырехугольник выпуклым.
 \\
C2. 
Доказать, что необходимым и достаточным условием компланарности векторов \(\overrightarrow{a},\ \overrightarrow{b},\ \overrightarrow{c}\) является зависимость \(\alpha\overrightarrow{a} + \beta\overrightarrow{b} + \gamma\overrightarrow{c} = 0\), где по крайней мере одно из чисел \(\alpha,\beta,\gamma\) не равно нулю.
 \\
C3. 
Составить уравнения прямой, которая проходит через точку \(M_{1}( - 1;2; - 3)\) перпендикулярно к вектору \(\overrightarrow{a} = \{ 6; - 2; - 3\}\) и пересекает прямую \(\frac{x - 1}{3} = \frac{y + 1}{2} = \frac{z - 3}{- 5}\).
 \\

\end{tabular}
\vspace{1cm}


\begin{tabular}{m{17cm}}
\textbf{15-вариант}
\newline

T1. 
Координаты вектора.
 \\
T2. 
Взаимное расположение прямой на плоскости.
 \\
A1. 
Даны точки \(A(1; - 2; - 3),B(2; - 3;0),C(3;1\); \(- 9),D( - 1;1; - 12)\). Вычислить расстояние между: 1) \(A\) и \(C\); 2) \(B\) и \(D\); 3\()C\) и \(D\).
 \\
A2. 
Вычислить косинус угла, образованного векторами \(\overrightarrow{a} = \{ 2; - 4;4\}\) и \(\overrightarrow{b} = \{ - 3;2;6\}\).
 \\
A3. 
Составить канонические уравнения прямой, проходящей через точку \(M_{1}(2;0; - 3)\) параллельно: 1) вектору \(\overrightarrow{a} = \{ 2; - 3;5\}\); 2) прямой \(\frac{x - 1}{5} = \frac{y + 2}{2} = \frac{z + 1}{- 1}\); 3) оси \(Ox\); 4) оси \(Oy\); 5) оси \(Oz\).
 \\
B1. 
Найти точку \(Q\), симметричную точке \(P( - 5;13)\) относительно прямой \(2x - 3y - 3 = 0.\)
 \\
B2. 
Вектор \(\overrightarrow{c}\) перпендикулярен к векторам \(\overrightarrow{a}\) и \(\overrightarrow{b}\), угол между \(\overrightarrow{a}\) и \(\overrightarrow{b}\) равен \(30^{{^\circ}}\). Зная, что \(|\overrightarrow{a}| = 6,|\overrightarrow{b}| = 3\), \(|\overrightarrow{c}| = 3\), вычислить \(\left( \left\lbrack \overrightarrow{a},\overrightarrow{b} \right\rbrack,\overrightarrow{c} \right)\).
 \\
B3. 
Вычислить объем пирамиды, ограниченной плоскостью \(2x - 3y + 6z - 12 = 0\) и координатными плоскостями.
 \\
C1. 
Даны уравнения двух сторон прямоугольника \(x - 2y = 0,x - 2y + 15 = 0\) и уравнение одной из его диагоналей \(7x + y - 15 = 0\). Найти вершины прямоугольника.
 \\
C2. 
Доказать тождество \((\lbrack\overrightarrow{a},\overrightarrow{b}\rbrack,\overrightarrow{c} + \lambda\overrightarrow{a} + \mu\overrightarrow{b}) = (\lbrack\overrightarrow{a},\overrightarrow{b}\rbrack,\overrightarrow{c})\), где \(\lambda\) и \(\mu\)-какие угодно числа.
 \\
C3. 
Даны вершины треугольника \(A(3;6; - 7),B( - 5\); \(2;3)\) и \(C(4; - 7; - 2)\). Составить параметрические уравнения его медианы, проведенной из вершины \(C\).
 \\

\end{tabular}
\vspace{1cm}


\begin{tabular}{m{17cm}}
\textbf{16-вариант}
\newline

T1. 
Семейство линейно зависимых и линейно независимых векторов.
 \\
T2. 
Уравнения прямой в пространстве. Взаимное расположение прямых.
 \\
A1. 
Вычислить угловой коэффициент \(k\) прямой, проходящей через две данные точки: а) \(M_{1}(2; - 5),M_{2}(3;2)\); б) \(P( - 3;1),Q(7;8)\); в) \(A(5; - 3),B( - 1;6)\).
 \\
A2. 
На плоскости даны два вектора \(\overrightarrow{p} = \{ 2; - 3\}\), \(\overrightarrow{q} = \{ 1;2\}\). Найти разложение вектора \(\overrightarrow{a} = \{ 9;4\}\) по базису \(\overrightarrow{p},\ \overrightarrow{q}\).
 \\
A3. 
Две грани куба лежат на плоскостях \(2x - 2y + z - 1 = 0,\) \(2x - 2y + z + 5 = 0\). Вычислить объем этого куба.
 \\
B1. 
Найти уравнение прямой, принадлежащей пучку прямых \(\ \ \ \ \alpha(x + 2y - 5) + \beta(3x - 2y + 1) = 0\) и 1) проходящей через точку \(A(3; - 1)\); 2) проходящей через начало координат; 3) параллельной оси \(Ox\); 4) параллельной оси \(Oy\); 5) параллельной прямой \(4x + 3y + 5 = 0\); 6) перпендикулярной к прямой \(2x + 3y + 7 = 0\).
 \\
B2. 
Векторы \(\overrightarrow{a}\) и \(\overrightarrow{b}\) взаимно перпендикулярны. Зная, что \(|\overrightarrow{a}| = 3,|\overrightarrow{b}| = 4\), вычислить \(1)|\lbrack\overrightarrow{a} + \overrightarrow{b},\overrightarrow{a} - \overrightarrow{b}\rbrack|;\ \ 2)|\lbrack 3\overrightarrow{a} - \overrightarrow{b},\overrightarrow{a} - 2\overrightarrow{b}\rbrack|\).
 \\
B3. 
Составить уравнение плоскости, которая проходит через прямую пересечения плоскостей \(3x - y + 2z + 9 = 0\), \(x + z - 3 = 0\): 1) и через точку \(M_{1}(4; - 2; - 3)\); 2)параллельно оси \(Ox\); 3) параллельно оси \(Oy\); 4) параллельно оси \(Oz\).
 \\
C1. 
Составить уравнение биссектрисы угла между прямыми \(x + 2y - 11 = 0\) и \(3x - 6y - 5 = 0\), в котором лежит точка \(M(1; - 3)\).
 \\
C2. 
Доказать, что \(|(\lbrack\overrightarrow{a},b\rbrack,c)| <  |\overrightarrow{a}||\overrightarrow{b}||\overrightarrow{c}|;\) в каком случае здесь может иметь место знак равенства?
 \\
C3. 
Составить уравнение плоскости, проходящей через прямую пересечения плоскостей \(5x - 2y - z - 3 = 0,x + 3y - 2z + 5 = 0\) параллельно вектору \(\overrightarrow{l} = \{ 7;9;17\}\).
 \\

\end{tabular}
\vspace{1cm}


\begin{tabular}{m{17cm}}
\textbf{17-вариант}
\newline

T1. 
Векторное произведение и смешанное произведение векторов.
 \\
T2. 
Расстояние от точки до плоскости, от точки до прямой в пространстве и между двумя скрещивающими прямыми. \\
A1. 
Составить уравнение геометрического места точек, равноудаленных от двух параллельных прямых: 1) \(3x - y + 7 = 0,\ \ \ \ 3x - y - 3 = 0\); 2) \(x - 2y + 3 = 0,\ \ \ \ x - 2y + 7 = 0\); 3) \(5x - 2y - 6 = 0,\ \ \ \ 10x - 4y + 3 = 0\).
 \\
A2. 
Установить, компланарны ли векторы \(\overrightarrow{a},\overrightarrow{b},\overrightarrow{c}\), если 1)\(a = \{ 2;3; - 1\},\ \ \ \ b = \{ 1; - 1;3\},\ \ \ \ c = \{ 1;9; - 11\}\); 2)\(a = \{ 3; - 2;1\},\ \ \ \ b = \{ 2;1;2\},\ \ \ \ c = \{ 3; - 1; - 2\}\); 3)\(a = \{ 2; - 1;2\},\ \ \ \ b = \{ 1;2; - 3\},\ \ \ \ c = \{ 3; - 4;7\}\). \\
A3. 
Составить уравнение плоскости, которая проходит: 1) через точки \(M_{1}(7;2; - 3)\) и \(M_{2}(5;6; - 4)\) параллельно оси \(Ox\); 2) через точки \(P_{1}(2; - 1;1)\) и \(P_{2}(3;1;2)\) параллельно оси \(Oy\); 3) через точки \(Q_{1}(3; - 2;5)\) и \(Q_{2}(2;3;1)\) параллельно оси \(Oz\).
 \\
B1. 
Даны две точки \(P(2;3)\) и \(Q( - 1;0)\). Составить уравнение прямой, проходящей через точку \(Q\) перпендикулярно к отрезку \(\overline{PQ}\).
 \\
B2. 
Векторы \(\overrightarrow{a}\) и \(\overrightarrow{b}\) образуют угол \(\varphi = 2\pi/3\); зная, что \(|\overrightarrow{a}| = 3,|\overrightarrow{b}| = 4\), вычислить: 1) \(\left( \overrightarrow{a},\overrightarrow{b} \right)\); 2) \({\overrightarrow{a}}^{2}\); 3) \({\overrightarrow{b}}^{2}\); 4) \((\overrightarrow{a} + \overrightarrow{b})^{2}\); 5) \(\left( 3\overrightarrow{a} - 2\overrightarrow{b},\overrightarrow{a} + 2\overrightarrow{b} \right);\ 6)(\overrightarrow{a} - \overrightarrow{b})^{2};7)(3\overrightarrow{a} + 2\overrightarrow{b})^{2}\).
 \\
B3. 
В пучке плоскостей \(2x - 3y + z - 3 + \lambda(x + 3y + 2z + 1) = 0\) найти плоскость, которая: 1) проходит через точку \(M_{1}(1; - 2;3)\); 2) параллельна оси \(Ox\); 3) параллельна оси \(Oy\); 4) параллельна оси \(Oz\).
 \\
C1. 
Составить уравнение биссектрисы угла между прямыми \(2x - 3y - 5 = 0,6x - 4y + 7 = 0\), смежного с углом, содержащим точку \(C(2; - 1)\).
 \\
C2. 
Какому условия должны удовлетворять векторы \(\overrightarrow{a}\) и \(\overrightarrow{b}\), чтобы вектор \(\overrightarrow{a} + \overrightarrow{b}\) был перпендикулярен к вектору \(\overrightarrow{a} - \overrightarrow{b}\).
 \\
C3. 
Даны вершины треугольника \(A(1; - 2; - 4)\), \(B(3;1; - 3)\) и \(C(5;1; - 7)\). Составить параметрические уравнения его высоты, опущенной из вершины \(B\) на противоположную сторону.
 \\

\end{tabular}
\vspace{1cm}


\begin{tabular}{m{17cm}}
\textbf{18-вариант}
\newline

T1. 
Преобразование декартовой системы координат на плоскости и пространстве. \\
T2. Уравнении прямой на плоскости.
 \\
A1. 
Даны точки \(A(3; - 1)\) и \(B(2;1)\). Определить: координаты точки \(M\), симметричной точке \(A\) относительно точки \(B\); координаты точки \(N\), симметричной точке \(B\) относительно точки \(A\).
 \\
A2. 
Даны: \(|\overrightarrow{a}| = 3,|\overrightarrow{b}| = 26\) и \(|\lbrack\overrightarrow{a},\overrightarrow{b}\rbrack| = 72\). Вычислить \(\left( \overrightarrow{a},\overrightarrow{b} \right)\).
 \\
A3. 
Найти точку пересечения прямой и плоскости: 1) \(\frac{x - 1}{1} = \frac{y + 1}{- 2} = \frac{z}{6},\ \ \ \ 2x + 3y + z - 1 = 0\); 2) \(\frac{x + 3}{3} = \frac{y - 2}{- 1} = \frac{z + 1}{- 5},\ \ \ \ x - 2y + z - 15 = 0\); 3) \(\frac{x + 2}{- 2} = \frac{y - 1}{3} = \frac{z - 3}{2},\ \ \ \ x + 2y - 2z + 6 = 0\).
 \\
B1. 
Определить точки пересечения прямой \(2x - 3y - 12 = 0\) с координатными осями и построить эту прямую на чертеже.
 \\
B2. 
Даны векторы \(\overrightarrow{a},\ \overrightarrow{b}\) и \(\overrightarrow{c}\) удовлетворяющие условию \(\overrightarrow{a} + \overrightarrow{b} + \overrightarrow{c} = 0\). Зная, что \(|\overrightarrow{a}| = 3,\ \ \ \ |\overrightarrow{b}| = 1\) и \(|\overrightarrow{c}| = 4\), вычислить \(\left( \overrightarrow{a},\overrightarrow{b} \right) + \left( \overrightarrow{b},\overrightarrow{c} \right) + \left( \overrightarrow{c},\overrightarrow{a} \right)\).
 \\
B3. 
Составить уравнение плоскости, проходящей через точки \(M_{1}(2; - 1;3)\) и \(M_{2}(3;1;2)\) параллельно вектору \(\overrightarrow{a} = \{ 3; - 1;4\}\).
 \\
C1. 
На оси абсцисс найти такую точку \(P\), чтобы сумма ее расстояний до точек \(M(1;2)\) и \(\dot{N}(3;4)\) была наименьшей.
 \\
C2. 
Доказать, что \(|(\lbrack\overrightarrow{a},b\rbrack,c)| <  |\overrightarrow{a}||\overrightarrow{b}||\overrightarrow{c}|;\) в каком случае здесь может иметь место знак равенства?
 \\
C3. Составить уравнение плоскости, проходящей через прямую пересечения плоскостей \(2x - y + 3z - 5 = 0,x + 2y - z + 2 = 0\) параллельно вектору \(\overrightarrow{l} = \{ 2; - 1; - 2\}\).
 \\

\end{tabular}
\vspace{1cm}


\begin{tabular}{m{17cm}}
\textbf{19-вариант}
\newline

T1. 
Скалярное произведение векторов.
 \\
T2. 
Взаимное расположение прямой и плоскости в пространстве.
 \\
A1. 
Дана прямая \(2x + 3y + 4 = 0\). Составить уравнение прямой, проходящей через точку \(M_{0}(2;1)\): 1) параллельно данной прямой; 2) перпендикулярно к данной прямой.
 \\
A2. 
Даны векторы \(\overrightarrow{a} = \{ 1; - 1;3\},\ \ \ \overrightarrow{b} = \{ - 2;2;1\}\), \(\overrightarrow{c} = \{ 3; - 2;5\}\). Вычислить \((\lbrack\overrightarrow{a},\overrightarrow{b}\rbrack,\overrightarrow{c})\).
 \\
A3. 
Точка \(P(2; - 1; - 1)\) служит основанием перпендикуляра, опущенного из начала координат на плоскость. Составить уравнение этой плоскости.
 \\
B1. 
Стороны \(AB\), \(BC\) и \(AC\) треугольника \(ABC\) даны соответственно уравнениями \(4x + 3y - 5 = 0\), \(x - 3y + 10 = 0\), \(x - 2 = 0\). Определить координаты его вершин.
 \\
B2. 
Векторы \(\overrightarrow{a}\) и \(\overrightarrow{b}\) образуют угол \(\varphi = 2\pi/3\); зная, что \(|\overrightarrow{a}| = 3,|\overrightarrow{b}| = 4\), вычислить: 1) \(\left( \overrightarrow{a},\overrightarrow{b} \right)\); 2) \({\overrightarrow{a}}^{2}\); 3) \({\overrightarrow{b}}^{2}\); 4) \((\overrightarrow{a} + \overrightarrow{b})^{2}\); 5) \(\left( 3\overrightarrow{a} - 2\overrightarrow{b},\overrightarrow{a} + 2\overrightarrow{b} \right);\ 6)(\overrightarrow{a} - \overrightarrow{b})^{2};7)(3\overrightarrow{a} + 2\overrightarrow{b})^{2}\).
 \\
B3. 
Вычислить площадь треугольника, который отсекает плоскость \(5x - 6y + 3z + 120 = 0\) от координатного угла \(Oxy\).
 \\
C1. 
Даны две вершины \(A(3; - 1)\) и \(B(5;7)\) треугольника \(ABC\) и точка \(N(4; - 1)\) пересечения его высот. Составить уравнения сторон этого треугольника.
 \\
C2. 
Доказать, что векторы \(\overrightarrow{a},\ \overrightarrow{b},\ \overrightarrow{c}\) удовлетворяющие условию \(\lbrack\overrightarrow{a},\overrightarrow{b}\rbrack + \lbrack\overrightarrow{b},\overrightarrow{c}\rbrack + \lbrack\overrightarrow{c},\overrightarrow{a}\rbrack = 0\), компланарны.
 \\
C3. 
Даны вершины треугольника \(A(2; - 1; - 3)\), \(B(5;2; - 7)\) и \(C( - 7;11;6)\). Составить канонические уравнения биссектрисы его внешнего угла при вершине \(A\).
 \\

\end{tabular}
\vspace{1cm}


\begin{tabular}{m{17cm}}
\textbf{20-вариант}
\newline

T1. 
Выражение скалярного, векторного и смешанного произведения векторов в координатах.
 \\
T2. 
Расстояние от точки до прямой. Уравнение пучка прямых.
 \\
A1. 
Установить, лежат ли точка \(M(1; - 3)\) и начало координат по одну или по разные стороны каждой из следующих прямых: 1) \(2x - y + 5 = 0\); 2) \(x - 3y - 5 = 0\); 3) \(3x +\) \(+ 2y - 1 = 0\); 4) \(x - 3y + 2 = 0\); 5) \(10x + 24y + 15 = 0\).
 \\
A2. 
Даны: \(|\overrightarrow{a}| = 10,|\overrightarrow{b}| = 2\) и \(\left( \overrightarrow{a},\overrightarrow{b} \right) = 12\). Вычислить \(\left| \left\lbrack \overrightarrow{a},\overrightarrow{b} \right\rbrack \right|\).
 \\
A3. 
Привести каждое из следующих уравнений плоскостей к нормальному виду: 1) \(2x - 2y + z - 18 = 0\); 2) \(\frac{3}{7}x - \frac{6}{7}y + \frac{2}{7}z + 3 = 0\); 3) \(4x - 6y - 12z - 11 = 0\); 4) \(- 4x - 4y + 2z + 1 = 0\); 5) \(5y - 12z + 26 = 0\);
 \\
B1. 
Даны вершины треугольника \(A(1; - 1; - 3)\), \(B(2;1; - 2)\) и \(C( - 5;2; - 6)\). Вычислить длину биссектрисы его внутреннего угла при вершине \(A\). \\
B2. 
Вектор \(\overrightarrow{c}\) перпендикулярен к векторам \(\overrightarrow{a}\) и \(\overrightarrow{b}\), угол между \(\overrightarrow{a}\) и \(\overrightarrow{b}\) равен \(30^{{^\circ}}\). Зная, что \(|\overrightarrow{a}| = 6,|\overrightarrow{b}| = 3\), \(|\overrightarrow{c}| = 3\), вычислить \(\left( \left\lbrack \overrightarrow{a},\overrightarrow{b} \right\rbrack,\overrightarrow{c} \right)\).
 \\
B3. 
Даны прямые \(\frac{x + 2}{2} = \frac{y}{- 3} = \frac{z - 1}{4},\ \ \ \ \frac{x - 3}{l} = \frac{y - 1}{4} = \frac{z - 7}{2}\) при каком значении \(l\) они пересекаются?
 \\
C1. 
На прямой \(3x - y - 1 = 0\) найти такую точку \(P\), разность расстояний которой до точек \(A(4;1)\) и \(B(0;4)\) была бы наибольшей.
 \\
C2. 
Доказать, что векторы \(\overrightarrow{a},\ \overrightarrow{b},\ \overrightarrow{c}\) удовлетворяющие условию \(\lbrack\overrightarrow{a},\overrightarrow{b}\rbrack + \lbrack\overrightarrow{b},\overrightarrow{c}\rbrack + \lbrack\overrightarrow{c},\overrightarrow{a}\rbrack = 0\), компланарны.
 \\
C3. 
Даны вершины треугольника \(A(3; - 1; - 3)\), \(B(1;2; - 7)\) и \(C( - 5;14; - 3)\). Составить канонические уравнения биссектрисы его внутреннего угла при вершине \(C\).
 \\

\end{tabular}
\vspace{1cm}


\begin{tabular}{m{17cm}}
\textbf{21-вариант}
\newline

T1. 
Координаты вектора.
 \\
T2. 
Уравнения плоскости. Взаимное расположение плоскости.
 \\
A1. Даны концы \(A(3; - 5)\) и \(B( - 1;1)\) однородного стержня. Определить координаты его центра масс.
 \\
A2. 
Вычислив внутренние углы треугольника с вершинами \(A(1;2;1),B(3; - 1;7),C(7;4; - 2)\), убедиться, что этот треугольник равнобедренный.
 \\
A3. 
Доказать, что прямая \(x = 3t - 2,y = - 4t + 1\), \(z = 4t - 5\) параллельна плоскости \(4x - 3y - 6z - 5 = 0\).
 \\
B1. 
Составить уравнение прямой, проходящей через точку \(P( - 2;3)\) на одинаковых расстояниях от точек \(A(5\); \(- 1)\) и \(B(3;7)\).
 \\
B2. 
Дано, что \(|\overrightarrow{a}| = 3,|\overrightarrow{b}| = 5\). Определить, при каком значении \(\alpha\) векторы \(\overrightarrow{a} + \alpha\overrightarrow{b},\overrightarrow{a} - \alpha\overrightarrow{b}\) будут взаимно перпендикулярны.
 \\
B3. 
Составить канонические уравнения следующих прямых: 1) \(x - 2y + 3z - 4 = 0,3x + 2y - 5z - 4 = 0\); 2) \(5x + y + z = 0,2x + 3y - 2z + 5 = 0\); 3) \(x - 2y + 3z + 1 = 0,2x + y - 4z - 8 = 0\).
 \\
C1. 
Составить уравнения сторон треугольника \(ABC\), если даны одна из его вершин \(A(1;3)\) и уравнения двух медиан \(x - 2y + 1 = 0\) и \(y - 1 = 0\).
 \\
C2. 
Какому условию должны удовлетворять векторы \(\overrightarrow{a},\overrightarrow{b}\), чтобы векторы \(\overrightarrow{a} + \overrightarrow{b}\) и \(\overrightarrow{a} - \overrightarrow{b}\) были коллинеарны?
 \\
C3. 
Составить уравнение плоскости, проходящей через прямую пересечения плоскостей \(3x - 2y + z - 3 = 0,x - 2z = 0\) перпендикулярно плоскости \(x - 2y + z + 5 = 0\).
 \\

\end{tabular}
\vspace{1cm}


\begin{tabular}{m{17cm}}
\textbf{22-вариант}
\newline

T1. Предмет и методы аналитической геометрии.
 \\
T2. 
Уравнения плоскости. Взаимное расположение плоскости.
 \\
A1. 
Вычислить площадь треугольника, вершинами которого являются точки: 1) \(A(2; - 3),B(3;2)\) и \(C( - 2;5)\); 2) \(M_{1}( - 3;2),M_{2}(5; - 2)\) и \(\left. \ M_{3}(1;3);3 \right)M(3; - 4),N( - 2;3)\) н \(P(4;5)\).
 \\
A2. Может ли вектор составлять с координатными осями следующие углы: 1) \(\alpha = 45^{{^\circ}},\beta = 60^{{^\circ}},\gamma = 120^{{^\circ}}\); 2) \(\alpha = 45^{{^\circ}},\ \ \ \ \beta = 135^{{^\circ}},\ \ \ \ \gamma = 60^{{^\circ}}\); 3) \(\alpha = 90^{{^\circ}},\ \ \ \ \beta = 150^{{^\circ}}\), \(\gamma = 60^{{^\circ}}?\)
 \\
A3. 
Составить уравнение плоскости, которая проходит: 1) через точки \(M_{1}(7;2; - 3)\) и \(M_{2}(5;6; - 4)\) параллельно оси \(Ox\); 2) через точки \(P_{1}(2; - 1;1)\) и \(P_{2}(3;1;2)\) параллельно оси \(Oy\); 3) через точки \(Q_{1}(3; - 2;5)\) и \(Q_{2}(2;3;1)\) параллельно оси \(Oz\).
 \\
B1. 
Даны три вершины \(A(2;3),B(4; - 1)\) и \(C(0;5)\) параллелограмма \emph{ABCD}. Найти его четвертую вершину \(D\).
 \\
B2. 
Векторы \(\overrightarrow{a}\) и \(\overrightarrow{b}\) образуют угол \(\varphi = 2\pi/3\). Зная, что \(|\overrightarrow{a}| = 1,|\overrightarrow{b}| = 2\), вычислить: \(1)\left. \ \lbrack\overrightarrow{a},\overrightarrow{b}\rbrack^{2};\ \ \ 2 \right)\lbrack 2\overrightarrow{a} + \overrightarrow{b},\overrightarrow{a} + 2\overrightarrow{b}\rbrack^{2};\ \ \ \ \) 3) \(\lbrack\overrightarrow{a} + 3\overrightarrow{b},3\overrightarrow{a} - \overrightarrow{b}\rbrack^{2}\).
 \\
B3. 
Составить параметрические уравнения следующих прямых: 1) \(2x + 3y - z - 4 = 0,3x - 5y + 2z + 1 = 0\); 2) \(x + 2y - z - 6 = 0,2x - y + z + 1 = 0\).
 \\
C1. 
Площадь треугольника \(S = 8\), две его вершины суть точки \(A(1; - 2)\) и \(B(2;3)\), а третья вершина \(C\) лежит на прямой \(2x + y - 2 = 0\). Определить координаты вершины \(C\).
 \\
C2. 
Доказать, что \(\lbrack\overrightarrow{a},\overrightarrow{b}\rbrack^{2} <  {\overrightarrow{a}}^{2}{\overrightarrow{b}}^{2}\); в каком случае здесь будет знак равенства?
 \\
C3. 
Найти точку \(Q\), симметричную точке \(P(4;1;6)\) относительно прямой \(x - y - 4z + 12 = 0,2x + y - 2z + 3 = 0\).
 \\

\end{tabular}
\vspace{1cm}


\begin{tabular}{m{17cm}}
\textbf{23-вариант}
\newline

T1. 
Семейство линейно зависимых и линейно независимых векторов.
 \\
T2. 
Взаимное расположение прямой на плоскости.
 \\
A1. 
Привести общее уравнение прямой к нормальному виду в каждом из следующих случаев: 1) \(4x - 3y - 10 = 0\); 2) \(\frac{4}{5}x - \frac{3}{5}y + 10 = 0\); 3) \(12x - 5y + 13 = 0\); 4) \(x + 2 = 0\); 5) \(2x - y - \sqrt{5} = 0\).
 \\
A2. 
Векторы \(\overrightarrow{a}\) и \(\overrightarrow{b}\) образуют угол \(\varphi = \pi/6\). Зная, что \(|\overrightarrow{a}| = 6,|\overrightarrow{b}| = 5\), вычислить \(\left| \left\lbrack \overrightarrow{a},\overrightarrow{b} \right\rbrack \right|\)
 \\
A3. 
Составить канонические уравнения прямой, проходящей через точку \(M_{1}(2;0; - 3)\) параллельно: 1) вектору \(\overrightarrow{a} = \{ 2; - 3;5\}\); 2) прямой \(\frac{x - 1}{5} = \frac{y + 2}{2} = \frac{z + 1}{- 1}\); 3) оси \(Ox\); 4) оси \(Oy\); 5) оси \(Oz\).
 \\
B1. 
Даны уравнения двух сторон прямоугольника \(3x - 2y - 5 = 0,2x + 3y + 7 = 0\) и одна из его вершин \(A( - 2;1)\). Вычислить площадь этого прямоугольника.
 \\
B2. На плоскости даны три вектора \(\overrightarrow{a} = \{ 3; - 2\}\), \(\overrightarrow{b} = \{ - 2;1\}\) и \(\overrightarrow{c} = \{ 7; - 4\}\). Определить разложение каждого из этих трех векторов, принимая в качестве базиса два других.
 \\
B3. 
Вычислить площадь треугольника, который отсекает плоскость \(5x - 6y + 3z + 120 = 0\) от координатного угла \(Oxy\).
 \\
C1. 
Написать формулы преобразования координат, если точка \(M_{1}(2; - 3)\) лежит на новой оси абсцисс, а точка \(M_{2}(1; - 7)\) лежит на новой оси ординат, причем оси старой и новой систем координат имеют соответственно одинаковые направления.
 \\
C2. 
Доказать, что вектор \(\overrightarrow{p} = \overrightarrow{b} - \frac{\overrightarrow{a}(\overrightarrow{a},\overrightarrow{b})}{{\overrightarrow{a}}^{2}}\) перпендикулярен к вектору \(\overrightarrow{a}\).
 \\
C3. 
На плоскости \(Oxz\) найти такую точку \(P\), разность расстояний которой до точек \(M_{1}(3;2; - 5)\) и \(M_{2}(8; - 4\); -13) была бы наибольшей.
 \\

\end{tabular}
\vspace{1cm}


\begin{tabular}{m{17cm}}
\textbf{24-вариант}
\newline

T1. 
Понятие о векторе. Линейные операции над векторами.
 \\
T2. 
Расстояние от точки до прямой. Уравнение пучка прямых.
 \\
A1. 
Определить, при каком значении \(a\) прямая \((a + 2)x + \left( a^{2} - 9 \right)y + 3a^{2} - 8a + 5 = 0\) 1) параллельна оси абсцисс; 2) параллельна оси ординат; 3) проходит через начало координат. В каждом случае написать уравнение прямой.
 \\
A2. 
Даны вершины треугольника \(A( - 1; - 2;4)\), \(B( - 4; - 2;0)\) и \(C(3; - 2;1)\). Определить его внутренний угол при вершине \(B\).
 \\
A3. 
В каждом из следующих случаев вычислить расстояние между параллельными плоскостями: 1) \(x - 2y - 2z - 12 = 0,x - 2y - 2z - 6 = 0\); 2) \(2x - 3y + 6z - 14 = 0,4x - 6y + 12z + 21 = 0\); 3) \(2x - y + 2z + 9 = 0,4x - 2y + 4z - 21 = 0\); 4) \(16x + 12y - 15z + 50 = 0,\ \ \ \ 16x + 12y - 15z + 25 = 0\);
 \\
B1. 
Определить, при каком значении \(m\) две прямые \(\begin{matrix}
mx + (2m + 3)y + m + 6 = 0 \\
(2m + 1)x + (m - 1)y + m - 2 = 0
\end{matrix}\)пересекаются в точке, лежащей на оси ординат.
 \\
B2. 
Векторы \(\overrightarrow{a},\ \overrightarrow{b},\ \overrightarrow{c}\) образующие правую тройку, взаимно перпендикулярны. Зная, что \(|\overrightarrow{a}| = 4,\ \ |\overrightarrow{b}| = 2\), \(|\overrightarrow{c}| = 3\), вычислить \(\left( \left\lbrack \overrightarrow{a},\overrightarrow{b} \right\rbrack,\overrightarrow{c} \right)\).
 \\
B3. 
Составить уравнение плоскости, проходящей через точки \(M_{1}(2; - 1;3)\) и \(M_{2}(3;1;2)\) параллельно вектору \(\overrightarrow{a} = \{ 3; - 1;4\}\).
 \\
C1. 
Стороны треугольника даны уравнениями \(4x - y - 7 = 0,x + 3y - 31 = 0,x + 5y - 7 = 0\). Определить точку пересечения его высот.
 \\
C2. 
Векторы \(\overrightarrow{a},\ \overrightarrow{b},\ \overrightarrow{c}\) и \(\overrightarrow{d}\) связаны соотношениями \(\lbrack\overrightarrow{a},\overrightarrow{b}\rbrack = \lbrack\overrightarrow{c},\overrightarrow{d}\rbrack,\ \ \lbrack\overrightarrow{a},\overrightarrow{c}\rbrack = \lbrack\overrightarrow{b},\overrightarrow{d}\rbrack\). Доказать коллинеарность векторов \(\overrightarrow{a} - \overrightarrow{d}\) и \(\overrightarrow{b} - \overrightarrow{c}\).
 \\
C3. 
Найти проекцию точки \(C(3; - 4; - 2)\) на плоскость, проходящую через параллельные прямые \(\frac{x - 5}{13} = \frac{y - 6}{1} = \frac{z + 3}{- 4},\ \ \ \ \frac{x - 2}{13} = \frac{y - 3}{1} = \frac{z + 3}{- 4}\).
 \\

\end{tabular}
\vspace{1cm}


\begin{tabular}{m{17cm}}
\textbf{25-вариант}
\newline

T1. 
Скалярное произведение векторов.
 \\
T2. 
Взаимное расположение прямой и плоскости в пространстве.
 \\
A1. 
Даны последовательные вершины \(A(2;3),B(0;6)\), \(C( - 1;5),D(0;1)\) и \(E(1;1)\) однородной пятиугольной пластинки. Определить координаты ее центра масс.
 \\
A2. 
Даны вершины треугольника \(A(3;2; - 3)\), \(B(5;1; - 1)\) и \(C(1; - 2;1)\). Определить его внешний угол при вершине \(A\).
 \\
A3. 
Вычислить величину отклонения \(\delta\) и расстояние \(d\) от точки до плоскости в каждом из следующих случаев: 1) \(M_{1}( - 2; - 4;3)\), \(2x - y + 2z + 3 = 0;\) 2) \(M_{2}(2; - 1; - 1),16x - 12y + 15z - 4 = 0\); 3) \(M_{3}(1;2; - 3),\ \ \ \ 5x - 3y + z + 4 = 0\);
 \\
B1. 
Даны вершины треугольника \(A(1;2; - 1),B(2\); \(- 1;3)\) и \(C( - 4;7;5)\). Вычислить длину биссектрисы его внутреннего угла при вершине \(B\).
 \\
B2. 
Вычислить объем тетраэдра, вершины которого находятся в точках \(A(2; - 1;1),B(5;5;4),C(3;2; - 1)\) и \(D(4;1;3)\).
 \\
B3. 
Составить канонические уравнения следующих прямых: 1) \(x - 2y + 3z - 4 = 0,3x + 2y - 5z - 4 = 0\); 2) \(5x + y + z = 0,2x + 3y - 2z + 5 = 0\); 3) \(x - 2y + 3z + 1 = 0,2x + y - 4z - 8 = 0\).
 \\
C1. 
Даны уравнения двух сторон прямоугольника \(2x - 3y + 5 = 0,3x + 2y - 7 = 0\) и одна из его вершин \(A(2; - 3)\). Составить уравнения двух других сторон этого прямоугольника.
 \\
C2. 
Доказать тождество \((\lbrack\overrightarrow{a} + \overrightarrow{b},\overrightarrow{b} + \overrightarrow{c}\rbrack,\overrightarrow{c} + \overrightarrow{a}) = 2(\lbrack\overrightarrow{a},\overrightarrow{b}\rbrack,\overrightarrow{c})\).
 \\
C3. 
Составить уравнение плоскости, проходящей через прямую пересечения плоскостей \(3x - 2y + z - 3 = 0,x - 2z = 0\) перпендикулярно плоскости \(x - 2y + z + 5 = 0\).
 \\

\end{tabular}
\vspace{1cm}


\begin{tabular}{m{17cm}}
\textbf{26-вариант}
\newline

T1. 
Векторное произведение и смешанное произведение векторов.
 \\
T2. 
Уравнения прямой в пространстве. Взаимное расположение прямых.
 \\
A1. 
Определить угол \(\varphi\), образованный двумя прямыми: 1) \(3x - y + 5 = 0,2x + y - 7 = 0\); 2) \(x\sqrt{2} - y\sqrt{3} - 5 = 0\), \((3 + \sqrt{2})x + (\sqrt{6} - \sqrt{3})y + 7 = 0\); \(3)\) \(x\sqrt{3} + y\sqrt{2} - 2 = 0,x\sqrt{6} - 3y + 3 = 0\). Решить задачу, не вычисляя угловых коэффициентов данных прямых.
 \\
A2. 
Даны вершины четырехугольника \(A(1; - 2;2)\), \(B(1;4;0),C( - 4;1;1)\) и \(D( - 5; - 5;3)\). Доказать, что его диагонали \(AC\) и \(BD\) взаимно перпендикулярны.
 \\
A3. 
При каком значении \(m\) прямая \(\frac{x + 1}{3} = \frac{y - 2}{m} = \frac{z + 3}{- 2}\) параллельна плоскости \(x - 3y + 6z + 7 = 0\) ?
 \\
B1. 
Определить, лежат ли точки \(M(2;3)\) и \(N(5; - 1)\) в одном, в смежных или вертикальных углах, образованных при пересечении двух прямых: 1) \(x - 3y - 5 = 0,2x + 9y - 2 = 0\); 2) \(2x + 7y - 5 = 0,x + 3y + 7 = 0\); 3) \(12x + y - 1 = 0,\ \ \ \ 13x + 2y - 5 = 0\).
 \\
B2. 
Векторы \(\overrightarrow{a}\) и \(\overrightarrow{b}\) образуют угол \(\varphi = 2\pi/3\). Зная, что \(|\overrightarrow{a}| = 1,|\overrightarrow{b}| = 2\), вычислить: \(1)\left. \ \lbrack\overrightarrow{a},\overrightarrow{b}\rbrack^{2};\ \ \ 2 \right)\lbrack 2\overrightarrow{a} + \overrightarrow{b},\overrightarrow{a} + 2\overrightarrow{b}\rbrack^{2};\ \ \ \ \) 3) \(\lbrack\overrightarrow{a} + 3\overrightarrow{b},3\overrightarrow{a} - \overrightarrow{b}\rbrack^{2}\).
 \\
B3. Составить уравнение плоскости, проходящей через точку \(M_{0}(3;4; - 5)\) параллельно векторам \({\overrightarrow{a}}_{1} = \{ 3;1; - 1\}\) и \({\overrightarrow{a}}_{2} = \{ 1; - 2;1\}\).
 \\
C1. 
Составить уравнения сторон и медиан треугольника с вершинами \(A(3;2),B(5; - 2),C(1;0)\).
 \\
C2. 
Доказать тождество \((\lbrack\overrightarrow{a},\overrightarrow{b}\rbrack,\overrightarrow{c} + \lambda\overrightarrow{a} + \mu\overrightarrow{b}) = (\lbrack\overrightarrow{a},\overrightarrow{b}\rbrack,\overrightarrow{c})\), где \(\lambda\) и \(\mu\)-какие угодно числа.
 \\
C3. 
Даны вершины треугольника \(A(1; - 2; - 4)\), \(B(3;1; - 3)\) и \(C(5;1; - 7)\). Составить параметрические уравнения его высоты, опущенной из вершины \(B\) на противоположную сторону.
 \\

\end{tabular}
\vspace{1cm}


\begin{tabular}{m{17cm}}
\textbf{27-вариант}
\newline

T1. 
Выражение скалярного, векторного и смешанного произведения векторов в координатах.
 \\
T2. Уравнении прямой на плоскости.
 \\
A1. 
Точки \(P_{1},P_{2},P_{3},P_{3}\) и \(P_{5}\) расположены на прямой \(3x - 2y - 6 = 0\); их абсциссы соответственно равны числам 4 , \(0,2, - 2\) и -6 . Определить ординаты этих точек.
 \\
A2. 
Даны векторы \(\overrightarrow{a} = \{ 1; - 1;3\},\ \ \ \overrightarrow{b} = \{ - 2;2;1\}\), \(\overrightarrow{c} = \{ 3; - 2;5\}\). Вычислить \((\lbrack\overrightarrow{a},\overrightarrow{b}\rbrack,\overrightarrow{c})\).
 \\
A3. 
Составить канонические уравнения прямой, проходящей через данные точки: 1) \((1; - 2;1),(3;1; - 1)\); 2) \((3; - 1;0),(1;0, - 3);\) \(3)(0; - 2;3),(3; - 2;1)\); 4) \((1;2; - 4),( - 1;2; - 4)\).
 \\
B1. 
Центр пучка прямых \(\alpha(2x - 3y + 20) + \beta(3x + 5y - 27) = 0\) является вершиной квадрата, диагональ которого лежит на прямой \(x + 7y - 16 = 0\). Составить уравнения сторон и второй диагонали этого квадрата.
 \\
B2. 
Векторы \(a\) и \(b\) образует угол \(\varphi = \pi/6\); зная, что \(|\mathbf{a}| = \sqrt{3},|\mathbf{b}| = 1\), вычислить угол \(\alpha\) между векторами \(p = a + b\ \ \ \ \) и \(\ \ \ \ q = a - b\).
 \\
B3. 
Составить уравнение плоскости, проходящей через точки \(M_{1}(3; - 1;2),M_{2}(4; - 1; - 1)\) и \(M_{3}(2;0;2)\).
 \\
C1. Даны вершины треугольника \(A(1;4),B(3; - 9)\) и \(C( - 5;2)\). Определить длину его медианы, проведенной из вершины \(B\). (С помощью деление отрезка в данном отношение)
 \\
C2. 
Доказать, что вектор \(\overrightarrow{p} = \overrightarrow{b} - \frac{\overrightarrow{a}(\overrightarrow{a},\overrightarrow{b})}{{\overrightarrow{a}}^{2}}\) перпендикулярен к вектору \(\overrightarrow{a}\).
 \\
C3. 
Даны вершины треугольника \(A(3; - 1; - 3)\), \(B(1;2; - 7)\) и \(C( - 5;14; - 3)\). Составить канонические уравнения биссектрисы его внутреннего угла при вершине \(C\).
 \\

\end{tabular}
\vspace{1cm}


\begin{tabular}{m{17cm}}
\textbf{28-вариант}
\newline

T1. 
Преобразование декартовой системы координат на плоскости и пространстве. \\
T2. 
Расстояние от точки до плоскости, от точки до прямой в пространстве и между двумя скрещивающими прямыми. \\
A1. 
Определить, при каких значениях \(a\) и \(b\) две прямые \(ax - 2y - 1 = 0,\ \ \ \ 6x - 4y - b = 0\) 1) имеют одну общую точку; 2) параллельны; 3) совпадают.
 \\
A2. 
Определить, при каком значении \(\alpha\) векторы \(\overrightarrow{a} = \alpha\overrightarrow{i} - 3\overrightarrow{j} + 2\overrightarrow{k}\) и \(\overrightarrow{b} = \overrightarrow{i} + 2\overrightarrow{j} - \alpha\overrightarrow{k}\) взаимно перпендикулярны.
 \\
A3. 
Две грани куба лежат на плоскостях \(2x - 2y + z - 1 = 0,\) \(2x - 2y + z + 5 = 0\). Вычислить объем этого куба.
 \\
B1. 
Даны уравнения двух сторон параллелограмма \(8x + 3y + 1 = 0,2x + y - 1 = 0\) и уравнение одной из его диагоналей \(3x + 2y + 3 = 0\). Определить координаты вершин этого параллелограмма.
 \\
B2. 
Доказать, что точки \(A(1;2; - 1),B(0;1;5)\), \(C( - 1;2;1),D(2;1;3)\) лежат в одной плоскости.
 \\
B3. 
Составить параметрические уравнения следующих прямых: 1) \(2x + 3y - z - 4 = 0,3x - 5y + 2z + 1 = 0\); 2) \(x + 2y - z - 6 = 0,2x - y + z + 1 = 0\).
 \\
C1. 
Даны вершины треугольника \(A(2; - 5),B(1; - 2)\) и \(C(4;7)\). Найти точку пересечения биссектрисы его внутреннего угла при вершине \(B\) со стороной \emph{AC}. (С помощью деление отрезка в данном отношение)
 \\
C2. 
Доказать тождество \((\lbrack\overrightarrow{a} + \overrightarrow{b},\overrightarrow{b} + \overrightarrow{c}\rbrack,\overrightarrow{c} + \overrightarrow{a}) = 2(\lbrack\overrightarrow{a},\overrightarrow{b}\rbrack,\overrightarrow{c})\).
 \\
C3. 
Составить уравнение плоскости, проходящей через прямую пересечения плоскостей \(5x - 2y - z - 3 = 0,x + 3y - 2z + 5 = 0\) параллельно вектору \(\overrightarrow{l} = \{ 7;9;17\}\).
 \\

\end{tabular}
\vspace{1cm}


\begin{tabular}{m{17cm}}
\textbf{29-вариант}
\newline

T1. 
Понятие о векторе. Линейные операции над векторами.
 \\
T2. 
Взаимное расположение прямой на плоскости.
 \\
A1. 
Определить, какие из точек \(M_{1}(3;1),\) \(M_{2}(2;3)\), \(M_{3}(6;3),\) \(M_{4}( - 3; - 3),\) \(M_{5}(3; - 1),\) \(M_{6}( - 2;1)\) лежат на прямой \(2x - 3y - 3 = 0\) и какие не лежат на ней.
 \\
A2. 
Векторы \(\overrightarrow{a}\) и \(\overrightarrow{b}\) образуют угол \(\varphi = \pi/6\). Зная, что \(|\overrightarrow{a}| = 6,|\overrightarrow{b}| = 5\), вычислить \(\left| \left\lbrack \overrightarrow{a},\overrightarrow{b} \right\rbrack \right|\)
 \\
A3. Составить уравнение плоскости, которая проходит через точку \(M_{1}(2;1; - 1)\) и имеет нормальный вектор \(\overrightarrow{n} = \{ 1; - 2;3\}\).
 \\
B1. 
Составить уравнение прямой, проходящей через точку пересечения прямых \(2x + 7y - 8 = 0,3x + 2y + 5 = 0\) под углом \(45^{{^\circ}}\) к прямой \(2x + 3y - 7 = 0\). Решить задачу, не вычисляя координат точки пересечения данных прямых.
 \\
B2. 
Вектор \(\overrightarrow{x}\), коллинеарный вектору \(\overrightarrow{a} = \{ 6; - 8; - 7,5\}\), образует острый угол с осью Oz. Зная, что \(|\overrightarrow{x}| = 50\), найти его координаты.
 \\
B3. 
Доказать, что прямая \(5x - 3y + 2z - 5 = 0\), \(2x - y - z - 1 = 0\) лежит в плоскости \(4x - 3y + 7z - 7 = 0\).
 \\
C1. 
В треугольнике \(ABC\) даны: уравнение стороны \(AB:5x - 3y + 2 = 0\), уравнения высот \(AM:4x - 3y + 1 = 0\) и \(BN:7x + 2y - 22 = 0\). Составить уравнения двух других сторон и третьей высоты этого треугольника.
 \\
C2. 
Векторы \(\overrightarrow{a},\ \overrightarrow{b}\) и \(\overrightarrow{c}\) удовлетворяют условию \(\overrightarrow{a} + \overrightarrow{b} + \overrightarrow{c} = 0\). Доказать, что \(\lbrack\overrightarrow{a},\overrightarrow{b}\rbrack = \lbrack\overrightarrow{b},\overrightarrow{c}\rbrack = \lbrack\overrightarrow{c},\overrightarrow{a}\rbrack\)
 \\
C3. 
Составить уравнение плоскости, проходящей через прямую \(5x - y - 2z - 3 = 0,\ \ \ \ 3x - 2y - 5z + 2 = 0\) перпендикулярно плоскости \(x + 19y - 7z - 11 \doteq 0\).
 \\

\end{tabular}
\vspace{1cm}


\begin{tabular}{m{17cm}}
\textbf{30-вариант}
\newline

T1. 
Семейство линейно зависимых и линейно независимых векторов.
 \\
T2. 
Расстояние от точки до плоскости, от точки до прямой в пространстве и между двумя скрещивающими прямыми. \\
A1. 
Дана прямая \(5x + 3y - 3 = 0\). Определить угловой коэффициент \(k\) прямой: параллельной данной прямой; перпендикулярно к данной прямой.
 \\
A2. 
Даны: \(|\overrightarrow{a}| = 10,|\overrightarrow{b}| = 2\) и \(\left( \overrightarrow{a},\overrightarrow{b} \right) = 12\). Вычислить \(\left| \left\lbrack \overrightarrow{a},\overrightarrow{b} \right\rbrack \right|\).
 \\
A3. 
Доказать перпендикулярность прямых: 1) \(\frac{x}{1} = \frac{y - 1}{- 2} = \frac{z}{3}\) и \(3x + y - 5z + 1 = 0,2x + 3y - 8z + 3 = 0\); 2) \(x = 2t + 1,y = 3t - 2,z = - 6t + 1\) и \(2x + y - 4z + 2 = 0\), \(4x - y - 5z + 4 = 0\); 3) \(x + y - 3z - 1 = 0,\ \ \ \ 2x - y - 9z - 2 = 0\ \ \ \ \) и \(2x + y +\) \(+ 2z + 5 = 0,2x - 2y - z + 2 = 0\)
 \\
B1. 
Даны две вершины \(A(2; - 3; - 5),B( - 1;3;2)\) параллелограмма \(ABCD\) и точка пересечения его диагоналей \(E(4; - 1;7)\). Определить две другие вершины этого параллелограмма.
 \\
B2. 
Даны точки \(A(2; - 1;2),B(1;2; - 1)\) и \(C(3;2;1)\). Найти координаты векторных пронзведений: 1) \(\lbrack\overline{AB},\overline{BC}\rbrack\); 2) \(\lbrack\overline{BC} - 2\overline{CA},\overline{CB}\rbrack\).
 \\
B3. 
В пучке плоскостей \(2x - 3y + z - 3 + \lambda(x + 3y + 2z + 1) = 0\) найти плоскость, которая: 1) проходит через точку \(M_{1}(1; - 2;3)\); 2) параллельна оси \(Ox\); 3) параллельна оси \(Oy\); 4) параллельна оси \(Oz\).
 \\
C1. 
Составить уравнения сторон треугольника, зная одну его вершину \(B(2;6)\), а также уравнения высоты \(x - 7y + 15 = 0\) и биссектрисы \(7x + y + 5 = 0\), проведенных из одной вершины.
 \\
C2. 
Векторы \(\overrightarrow{a},\ \overrightarrow{b}\) и \(\overrightarrow{c}\) удовлетворяют условию \(\overrightarrow{a} + \overrightarrow{b} + \overrightarrow{c} = 0\). Доказать, что \(\lbrack\overrightarrow{a},\overrightarrow{b}\rbrack = \lbrack\overrightarrow{b},\overrightarrow{c}\rbrack = \lbrack\overrightarrow{c},\overrightarrow{a}\rbrack\)
 \\
C3. 
Составить уравнения прямой, которая проходит через точку \(M_{1}( - 1;2; - 3)\) перпендикулярно к вектору \(\overrightarrow{a} = \{ 6; - 2; - 3\}\) и пересекает прямую \(\frac{x - 1}{3} = \frac{y + 1}{2} = \frac{z - 3}{- 5}\).
 \\

\end{tabular}
\vspace{1cm}


\begin{tabular}{m{17cm}}
\textbf{31-вариант}
\newline

T1. Предмет и методы аналитической геометрии.
 \\
T2. 
Уравнения прямой в пространстве. Взаимное расположение прямых.
 \\
A1. 
Даны точки \(A(1; - 2; - 3),B(2; - 3;0),C(3;1\); \(- 9),D( - 1;1; - 12)\). Вычислить расстояние между: 1) \(A\) и \(C\); 2) \(B\) и \(D\); 3\()C\) и \(D\).
 \\
A2. 
Даны вершины четырехугольника \(A(1; - 2;2)\), \(B(1;4;0),C( - 4;1;1)\) и \(D( - 5; - 5;3)\). Доказать, что его диагонали \(AC\) и \(BD\) взаимно перпендикулярны.
 \\
A3. 
Определить, при каком значении \(l\) следующие парь уравнений будут определять перпендикулярные плоскости: 1) \(3x - 5y + lz - 3 = 0,x + 3y + 2z + 5 = 0\); 2) \(5x + y - 3z - 3 = 0,2x + ly - 3z + 1 = 0\); 3) \(\ \ \ \ 7x - 2y - z = 0,\ \ \ \ lx + y - 3z - 1 = 0\).
 \\
B1. 
Даны две смежные вершины параллелограмма \(A( - 3;5),B(1;7)\) и точка пересечения его диагоналей \(M(1;1)\). Определить две другие вершины.
 \\
B2. 
Найти вектор \(\overrightarrow{x}\), коллинеарный вектору \(\overrightarrow{a} = \{ 2;1; - 1\}\) и удовлетворяющий условию \(\left( \overrightarrow{x},\overrightarrow{a} \right) = 3\).
 \\
B3. 
Составить уравнение плоскости, которая проходит через прямую пересечения плоскостей \(3x - y + 2z + 9 = 0\), \(x + z - 3 = 0\): 1) и через точку \(M_{1}(4; - 2; - 3)\); 2)параллельно оси \(Ox\); 3) параллельно оси \(Oy\); 4) параллельно оси \(Oz\).
 \\
C1. 
Даны точки \(M_{1}(9; - 3)\) и \(M_{2}( - 6;5)\). Начало координат перенесено в точку \(M_{1}\), а координатные оси повернуты так, что положительное направление новой оси абсцисс совпадает с направлением отрезка \(\overrightarrow{M_{1}M_{2}}\). Вывести формулы преобразования координат.
 \\
C2. 
Векторы \(\overrightarrow{a},\ \overrightarrow{b},\ \overrightarrow{c}\) и \(\overrightarrow{d}\) связаны соотношениями \(\lbrack\overrightarrow{a},\overrightarrow{b}\rbrack = \lbrack\overrightarrow{c},\overrightarrow{d}\rbrack,\ \ \lbrack\overrightarrow{a},\overrightarrow{c}\rbrack = \lbrack\overrightarrow{b},\overrightarrow{d}\rbrack\). Доказать коллинеарность векторов \(\overrightarrow{a} - \overrightarrow{d}\) и \(\overrightarrow{b} - \overrightarrow{c}\).
 \\
C3. 
Найти точку \(Q\), симметричную точке \(P(4;1;6)\) относительно прямой \(x - y - 4z + 12 = 0,2x + y - 2z + 3 = 0\).
 \\

\end{tabular}
\vspace{1cm}


\begin{tabular}{m{17cm}}
\textbf{32-вариант}
\newline

T1. 
Координаты вектора.
 \\
T2. 
Уравнения плоскости. Взаимное расположение плоскости.
 \\
A1. 
Даны вершины треугольника \(A(1; - 3),B(3; - 5)\) и \(C( - 5;7)\). Определить середины его сторон.
 \\
A2. Может ли вектор составлять с координатными осями следующие углы: 1) \(\alpha = 45^{{^\circ}},\beta = 60^{{^\circ}},\gamma = 120^{{^\circ}}\); 2) \(\alpha = 45^{{^\circ}},\ \ \ \ \beta = 135^{{^\circ}},\ \ \ \ \gamma = 60^{{^\circ}}\); 3) \(\alpha = 90^{{^\circ}},\ \ \ \ \beta = 150^{{^\circ}}\), \(\gamma = 60^{{^\circ}}?\)
 \\
A3. 
Установить, какие из следующих пар уравнений определяют параллельные плоскости; 1) \(2x - 3y + 5z - 7 = 0,\ \ \ \ 2x - 3y + 5z + 3 = 0\); 2) \(4x + 2y - 4z + 5 = 0,\ \ \ \ 2x + y + 2z - 1 = 0\); 3) \(\ \ \ \ x - 3z + 2 = 0,\ \ \ \ 2x - 6z - 7 = 0\).
 \\
B1. 
Найти проекцию точки \(P( - 6;4)\) на прямую \(4x - 5y + 3 = 0\).
 \\
B2. 
Векторы \(a\) и \(b\) образует угол \(\varphi = \pi/6\); зная, что \(|\mathbf{a}| = \sqrt{3},|\mathbf{b}| = 1\), вычислить угол \(\alpha\) между векторами \(p = a + b\ \ \ \ \) и \(\ \ \ \ q = a - b\).
 \\
B3. 
Даны прямые \(\frac{x + 2}{2} = \frac{y}{- 3} = \frac{z - 1}{4},\ \ \ \ \frac{x - 3}{l} = \frac{y - 1}{4} = \frac{z - 7}{2}\) при каком значении \(l\) они пересекаются?
 \\
C1. 
Даны уравнения двух сторон прямоугольника \(5x + 2y - 7 = 0,\ \ \ \ 5x + 2y - 36 = 0\) и уравнение его диагонали \(3x + 7y - 10 = 0\). Составить уравнения остальных сторон и второй диагонали этого прямоугольника.
 \\
C2. 
Доказать тождество \(\lbrack\overrightarrow{a},\overrightarrow{b}\rbrack^{2} + (\overrightarrow{a},\overrightarrow{b})^{2} = {\overrightarrow{a}}^{2}{\overrightarrow{b}}^{2}\).
 \\
C3. 
Даны вершины треугольника \(A(3;6; - 7),B( - 5\); \(2;3)\) и \(C(4; - 7; - 2)\). Составить параметрические уравнения его медианы, проведенной из вершины \(C\).
 \\

\end{tabular}
\vspace{1cm}


\begin{tabular}{m{17cm}}
\textbf{33-вариант}
\newline

T1. Предмет и методы аналитической геометрии.
 \\
T2. Уравнении прямой на плоскости.
 \\
A1. 
Точки \(P_{1},P_{2},P_{3},P_{3}\) и \(P_{5}\) расположены на прямой \(3x - 2y - 6 = 0\); их абсциссы соответственно равны числам 4 , \(0,2, - 2\) и -6 . Определить ординаты этих точек.
 \\
A2. 
Даны вершины треугольника \(A(3;2; - 3)\), \(B(5;1; - 1)\) и \(C(1; - 2;1)\). Определить его внешний угол при вершине \(A\).
 \\
A3. 
Вычислить расстояние \(d\) от точки \(P(2;3; - 1)\) до следующих прямых: 1) \(\frac{x - 5}{3} = \frac{y}{2} = \frac{z + 25}{- 2}\); 2) \(x = t + 1,y = t + 2,z = 4t + 13\).
 \\
B1. 
Определить, лежит ли точка \(M( - 3;2)\) внутри или вне треугольника, стороны которого даны уравнениями \(x + y - 4 = 0,\ \ \ \ 3x - 7y + 8 = 0,\ \ \ \ 4x - y - 31 = 0\).
 \\
B2. 
На плоскости даны три вектора \(\overrightarrow{a} = \{ 3; - 2\}\), \(\overrightarrow{b} = \{ - 2;1\}\) и \(\overrightarrow{c} = \{ 7; - 4\}\). Определить разложение каждого из этих трех векторов, принимая в качестве базиса два других.
 \\
B3. 
Вычислить объем пирамиды, ограниченной плоскостью \(2x - 3y + 6z - 12 = 0\) и координатными плоскостями.
 \\
C1. 
Даны вершины треугольника \(A(1; - 2),B(5;4)\) и \(C( - 2;0)\). Составить уравнения биссектрис его внутреннего и внешнего углов при вершине \(A\).
 \\
C2. 
Доказать, что \(|(\lbrack\overrightarrow{a},b\rbrack,c)| <  |\overrightarrow{a}||\overrightarrow{b}||\overrightarrow{c}|;\) в каком случае здесь может иметь место знак равенства?
 \\
C3. 
Даны вершины треугольника \(A(2; - 1; - 3)\), \(B(5;2; - 7)\) и \(C( - 7;11;6)\). Составить канонические уравнения биссектрисы его внешнего угла при вершине \(A\).
 \\

\end{tabular}
\vspace{1cm}


\begin{tabular}{m{17cm}}
\textbf{34-вариант}
\newline

T1. 
Координаты вектора.
 \\
T2. 
Расстояние от точки до прямой. Уравнение пучка прямых.
 \\
A1. 
Вычислить площадь треугольника, вершинами которого являются точки: 1) \(A(2; - 3),B(3;2)\) и \(C( - 2;5)\); 2) \(M_{1}( - 3;2),M_{2}(5; - 2)\) и \(\left. \ M_{3}(1;3);3 \right)M(3; - 4),N( - 2;3)\) н \(P(4;5)\).
 \\
A2. 
Вычислив внутренние углы треугольника с вершинами \(A(1;2;1),B(3; - 1;7),C(7;4; - 2)\), убедиться, что этот треугольник равнобедренный.
 \\
A3. 
Составить параметрические уравнения прямой, проходящей через точку \(M_{1}(1; - 1; - 3)\) параллельно: 1) вектору \(\overrightarrow{a} = \{ 2; - 3;4\}\); 2) прямой \(\frac{x - 1}{2} = \frac{y + 2}{4} = \frac{z - 1}{0}\); 3) прямой \(x = 3t - 1,y = - 2t + 3,z = 5t + 2\).
 \\
B1. 
Даны середины сторон треугольника \(M_{1}(2;1)\), \(M_{2}(5;3)\) и \(M_{3}(3; - 4)\). Составить уравнение его сторон.
 \\
B2. 
Векторы \(\overrightarrow{a}\) и \(\overrightarrow{b}\) взаимно перпендикулярны. Зная, что \(|\overrightarrow{a}| = 3,|\overrightarrow{b}| = 4\), вычислить \(1)|\lbrack\overrightarrow{a} + \overrightarrow{b},\overrightarrow{a} - \overrightarrow{b}\rbrack|;\ \ 2)|\lbrack 3\overrightarrow{a} - \overrightarrow{b},\overrightarrow{a} - 2\overrightarrow{b}\rbrack|\).
 \\
B3. 
Составить уравнение плоскости, проходящей через точки \(M_{1}(3; - 1;2),M_{2}(4; - 1; - 1)\) и \(M_{3}(2;0;2)\).
 \\
C1. 
Три вершины параллелограмма суть точки \(A(3;7)\), \(B(2; - 3)\) и \(C( - 1;4)\), Вычислить длину его высоты, опущенной из вершины \(B\) на сторону \emph{AC}.
 \\
C2. 
Векторы \(\overrightarrow{a},\ \overrightarrow{b}\) и \(\overrightarrow{c}\) удовлетворяют условию \(\overrightarrow{a} + \overrightarrow{b} + \overrightarrow{c} = 0\). Доказать, что \(\lbrack\overrightarrow{a},\overrightarrow{b}\rbrack = \lbrack\overrightarrow{b},\overrightarrow{c}\rbrack = \lbrack\overrightarrow{c},\overrightarrow{a}\rbrack\)
 \\
C3. 
На плоскости \(Oxz\) найти такую точку \(P\), разность расстояний которой до точек \(M_{1}(3;2; - 5)\) и \(M_{2}(8; - 4\); -13) была бы наибольшей.
 \\

\end{tabular}
\vspace{1cm}


\begin{tabular}{m{17cm}}
\textbf{35-вариант}
\newline

T1. 
Выражение скалярного, векторного и смешанного произведения векторов в координатах.
 \\
T2. 
Взаимное расположение прямой и плоскости в пространстве.
 \\
A1. 
Дано уравнение пучка прямых \(\alpha(3x + y - 1) + \beta(2x - y - 9) = 0\) . Доказать, что прямая \(x + 3y + 13 = 0\) при надлежит этому пучку.
 \\
A2. 
На плоскости даны два вектора \(\overrightarrow{p} = \{ 2; - 3\}\), \(\overrightarrow{q} = \{ 1;2\}\). Найти разложение вектора \(\overrightarrow{a} = \{ 9;4\}\) по базису \(\overrightarrow{p},\ \overrightarrow{q}\).
 \\
A3. 
Составить параметрические уравнения прямой, проходящей через точку \(M_{1}(1; - 1; - 3)\) параллельно: 1) вектору \(\overrightarrow{a} = \{ 2; - 3;4\}\); 2) прямой \(\frac{x - 1}{2} = \frac{y + 2}{4} = \frac{z - 1}{0}\); 3) прямой \(x = 3t - 1,y = - 2t + 3,z = 5t + 2\).
 \\
B1. 
Определить, лежит ли начало координат внутри или вне треугольника, стороны которого даны уравнениями \(7x - 5y - 11 = 0,\ \ \ \ 8x + 3y + 31 = 0,\ \ \ \ x + 8y - 19 = 0\).
 \\
B2. 
Доказать, что точки \(A(1;2; - 1),B(0;1;5)\), \(C( - 1;2;1),D(2;1;3)\) лежат в одной плоскости.
 \\
B3. 
Вычислить площадь треугольника, который отсекает плоскость \(5x - 6y + 3z + 120 = 0\) от координатного угла \(Oxy\).
 \\
C1. 
Даны две вершины треугольника \(M_{1}( - 10;2)\) и \(M_{2}(6;4)\); его высоты пересекаются в точке \(N(5;2)\). Определить координаты третьей вершины \(M_{3}\).
 \\
C2. 
Доказать тождество \((\lbrack\overrightarrow{a},\overrightarrow{b}\rbrack,\overrightarrow{c} + \lambda\overrightarrow{a} + \mu\overrightarrow{b}) = (\lbrack\overrightarrow{a},\overrightarrow{b}\rbrack,\overrightarrow{c})\), где \(\lambda\) и \(\mu\)-какие угодно числа.
 \\
C3. 
Найти проекцию точки \(C(3; - 4; - 2)\) на плоскость, проходящую через параллельные прямые \(\frac{x - 5}{13} = \frac{y - 6}{1} = \frac{z + 3}{- 4},\ \ \ \ \frac{x - 2}{13} = \frac{y - 3}{1} = \frac{z + 3}{- 4}\).
 \\

\end{tabular}
\vspace{1cm}


\begin{tabular}{m{17cm}}
\textbf{36-вариант}
\newline

T1. 
Векторное произведение и смешанное произведение векторов.
 \\
T2. 
Расстояние от точки до прямой. Уравнение пучка прямых.
 \\
A1. Даны концы \(A(3; - 5)\) и \(B( - 1;1)\) однородного стержня. Определить координаты его центра масс.
 \\
A2. 
Даны вершины треугольника \(A( - 1; - 2;4)\), \(B( - 4; - 2;0)\) и \(C(3; - 2;1)\). Определить его внутренний угол при вершине \(B\).
 \\
A3. 
В каждом из следующих случаев вычислить расстояние между параллельными плоскостями: 1) \(x - 2y - 2z - 12 = 0,x - 2y - 2z - 6 = 0\); 2) \(2x - 3y + 6z - 14 = 0,4x - 6y + 12z + 21 = 0\); 3) \(2x - y + 2z + 9 = 0,4x - 2y + 4z - 21 = 0\); 4) \(16x + 12y - 15z + 50 = 0,\ \ \ \ 16x + 12y - 15z + 25 = 0\);
 \\
B1. 
Определить, при каком значении \(m\) две прямые \((m - 1)x + my - 5 = 0,\ \ \ \ mx + (2m - 1)y + 7 = 0\) пересекаются в точке, лежащей на ось абсцисс.
 \\
B2. 
Даны векторы \(\overrightarrow{a},\ \overrightarrow{b}\) и \(\overrightarrow{c}\) удовлетворяющие условию \(\overrightarrow{a} + \overrightarrow{b} + \overrightarrow{c} = 0\). Зная, что \(|\overrightarrow{a}| = 3,\ \ \ \ |\overrightarrow{b}| = 1\) и \(|\overrightarrow{c}| = 4\), вычислить \(\left( \overrightarrow{a},\overrightarrow{b} \right) + \left( \overrightarrow{b},\overrightarrow{c} \right) + \left( \overrightarrow{c},\overrightarrow{a} \right)\).
 \\
B3. 
Составить параметрические уравнения следующих прямых: 1) \(2x + 3y - z - 4 = 0,3x - 5y + 2z + 1 = 0\); 2) \(x + 2y - z - 6 = 0,2x - y + z + 1 = 0\).
 \\
C1. 
Составить уравнения сторон треугольника, если даны одна из его вершин \(B( - 4; - 5)\) и уравнения двух высот \(5x + 3y - 4 = 0\) и \(3x + 8y + 13 = 0\).
 \\
C2. 
Доказать, что вектор \(\overrightarrow{p} = \overrightarrow{b} - \frac{\overrightarrow{a}(\overrightarrow{a},\overrightarrow{b})}{{\overrightarrow{a}}^{2}}\) перпендикулярен к вектору \(\overrightarrow{a}\).
 \\
C3. Составить уравнение плоскости, проходящей через прямую пересечения плоскостей \(2x - y + 3z - 5 = 0,x + 2y - z + 2 = 0\) параллельно вектору \(\overrightarrow{l} = \{ 2; - 1; - 2\}\).
 \\

\end{tabular}
\vspace{1cm}


\begin{tabular}{m{17cm}}
\textbf{37-вариант}
\newline

T1. 
Преобразование декартовой системы координат на плоскости и пространстве. \\
T2. 
Взаимное расположение прямой и плоскости в пространстве.
 \\
A1. 
Определить, какие из точек \(M_{1}(3;1),\) \(M_{2}(2;3)\), \(M_{3}(6;3),\) \(M_{4}( - 3; - 3),\) \(M_{5}(3; - 1),\) \(M_{6}( - 2;1)\) лежат на прямой \(2x - 3y - 3 = 0\) и какие не лежат на ней.
 \\
A2. 
Установить, компланарны ли векторы \(\overrightarrow{a},\overrightarrow{b},\overrightarrow{c}\), если 1)\(a = \{ 2;3; - 1\},\ \ \ \ b = \{ 1; - 1;3\},\ \ \ \ c = \{ 1;9; - 11\}\); 2)\(a = \{ 3; - 2;1\},\ \ \ \ b = \{ 2;1;2\},\ \ \ \ c = \{ 3; - 1; - 2\}\); 3)\(a = \{ 2; - 1;2\},\ \ \ \ b = \{ 1;2; - 3\},\ \ \ \ c = \{ 3; - 4;7\}\). \\
A3. 
Составить уравнение плоскости, которая проходит: 1) через точки \(M_{1}(7;2; - 3)\) и \(M_{2}(5;6; - 4)\) параллельно оси \(Ox\); 2) через точки \(P_{1}(2; - 1;1)\) и \(P_{2}(3;1;2)\) параллельно оси \(Oy\); 3) через точки \(Q_{1}(3; - 2;5)\) и \(Q_{2}(2;3;1)\) параллельно оси \(Oz\).
 \\
B1. 
Найти проекцию точки \(P( - 6;4)\) на прямую \(4x - 5y + 3 = 0\).
 \\
B2. 
Вектор \(\overrightarrow{c}\) перпендикулярен к векторам \(\overrightarrow{a}\) и \(\overrightarrow{b}\), угол между \(\overrightarrow{a}\) и \(\overrightarrow{b}\) равен \(30^{{^\circ}}\). Зная, что \(|\overrightarrow{a}| = 6,|\overrightarrow{b}| = 3\), \(|\overrightarrow{c}| = 3\), вычислить \(\left( \left\lbrack \overrightarrow{a},\overrightarrow{b} \right\rbrack,\overrightarrow{c} \right)\).
 \\
B3. 
Составить уравнение плоскости, которая проходит через прямую пересечения плоскостей \(3x - y + 2z + 9 = 0\), \(x + z - 3 = 0\): 1) и через точку \(M_{1}(4; - 2; - 3)\); 2)параллельно оси \(Ox\); 3) параллельно оси \(Oy\); 4) параллельно оси \(Oz\).
 \\
C1. 
На прямой \(2x - y - 5 = 0\) найти такую точку \(P\), сумма расстояний которой до точек \(A( - 7;1),B( - 5;5)\) была бы наименьшей.
 \\
C2. 
Даны векторы \(\overrightarrow{AB} = \overrightarrow{b}\) и \(\overrightarrow{AC} = \overrightarrow{c}\), совпадающие со сторонами треугольника \(ABC\). Найти разложение вектора, приложенного к вершине \(B\) этого треугольника и совпадающего с его высотой \(BD\) по базису \(\overrightarrow{b},\ \overrightarrow{c}\).
 \\
C3. 
Составить уравнение плоскости, проходящей через прямую пересечения плоскостей \(3x - 2y + z - 3 = 0,x - 2z = 0\) перпендикулярно плоскости \(x - 2y + z + 5 = 0\).
 \\

\end{tabular}
\vspace{1cm}


\begin{tabular}{m{17cm}}
\textbf{38-вариант}
\newline

T1. 
Семейство линейно зависимых и линейно независимых векторов.
 \\
T2. 
Уравнения прямой в пространстве. Взаимное расположение прямых.
 \\
A1. 
Определить площадь параллелограмма, три вершины которого суть точки \(A( - 2;3),B(4; - 5)\) и \(C( - 3;1)\). (С помощью площадью треугольника)
 \\
A2. 
Вычислить косинус угла, образованного векторами \(\overrightarrow{a} = \{ 2; - 4;4\}\) и \(\overrightarrow{b} = \{ - 3;2;6\}\).
 \\
A3. 
Найти точку пересечения прямой и плоскости: 1) \(\frac{x - 1}{1} = \frac{y + 1}{- 2} = \frac{z}{6},\ \ \ \ 2x + 3y + z - 1 = 0\); 2) \(\frac{x + 3}{3} = \frac{y - 2}{- 1} = \frac{z + 1}{- 5},\ \ \ \ x - 2y + z - 15 = 0\); 3) \(\frac{x + 2}{- 2} = \frac{y - 1}{3} = \frac{z - 3}{2},\ \ \ \ x + 2y - 2z + 6 = 0\).
 \\
B1. 
Определить, при каком значении \(m\) две прямые \((m - 1)x + my - 5 = 0,\ \ \ \ mx + (2m - 1)y + 7 = 0\) пересекаются в точке, лежащей на ось абсцисс.
 \\
B2. 
Даны точки \(A(2; - 1;2),B(1;2; - 1)\) и \(C(3;2;1)\). Найти координаты векторных пронзведений: 1) \(\lbrack\overline{AB},\overline{BC}\rbrack\); 2) \(\lbrack\overline{BC} - 2\overline{CA},\overline{CB}\rbrack\).
 \\
B3. 
Вычислить объем пирамиды, ограниченной плоскостью \(2x - 3y + 6z - 12 = 0\) и координатными плоскостями.
 \\
C1. 
Даны две противоположные вершины квадрата \(A( - 1;3)\) и \(C(6;2)\). Составить уравнения его сторон.
 \\
C2. 
Доказать тождество \((\lbrack\overrightarrow{a},\overrightarrow{b}\rbrack,\overrightarrow{c} + \lambda\overrightarrow{a} + \mu\overrightarrow{b}) = (\lbrack\overrightarrow{a},\overrightarrow{b}\rbrack,\overrightarrow{c})\), где \(\lambda\) и \(\mu\)-какие угодно числа.
 \\
C3. 
Даны вершины треугольника \(A(3; - 1; - 3)\), \(B(1;2; - 7)\) и \(C( - 5;14; - 3)\). Составить канонические уравнения биссектрисы его внутреннего угла при вершине \(C\).
 \\

\end{tabular}
\vspace{1cm}


\begin{tabular}{m{17cm}}
\textbf{39-вариант}
\newline

T1. 
Понятие о векторе. Линейные операции над векторами.
 \\
T2. 
Взаимное расположение прямой на плоскости.
 \\
A1. 
Привести общее уравнение прямой к нормальному виду в каждом из следующих случаев: 1) \(4x - 3y - 10 = 0\); 2) \(\frac{4}{5}x - \frac{3}{5}y + 10 = 0\); 3) \(12x - 5y + 13 = 0\); 4) \(x + 2 = 0\); 5) \(2x - y - \sqrt{5} = 0\).
 \\
A2. 
Даны: \(|\overrightarrow{a}| = 3,|\overrightarrow{b}| = 26\) и \(|\lbrack\overrightarrow{a},\overrightarrow{b}\rbrack| = 72\). Вычислить \(\left( \overrightarrow{a},\overrightarrow{b} \right)\).
 \\
A3. 
Определить, при каком значении \(l\) следующие парь уравнений будут определять перпендикулярные плоскости: 1) \(3x - 5y + lz - 3 = 0,x + 3y + 2z + 5 = 0\); 2) \(5x + y - 3z - 3 = 0,2x + ly - 3z + 1 = 0\); 3) \(\ \ \ \ 7x - 2y - z = 0,\ \ \ \ lx + y - 3z - 1 = 0\).
 \\
B1. 
Даны уравнения двух сторон параллелограмма \(8x + 3y + 1 = 0,2x + y - 1 = 0\) и уравнение одной из его диагоналей \(3x + 2y + 3 = 0\). Определить координаты вершин этого параллелограмма.
 \\
B2. 
Вычислить объем тетраэдра, вершины которого находятся в точках \(A(2; - 1;1),B(5;5;4),C(3;2; - 1)\) и \(D(4;1;3)\).
 \\
B3. Составить уравнение плоскости, проходящей через точку \(M_{0}(3;4; - 5)\) параллельно векторам \({\overrightarrow{a}}_{1} = \{ 3;1; - 1\}\) и \({\overrightarrow{a}}_{2} = \{ 1; - 2;1\}\).
 \\
C1. 
Точка \(A( - 4;5)\) является вершиной квадрата, диагональ которого лежит на прямой \(7x - y + 8 = 0\). Составить уравнения сторон и второй диагонали этого квадрата.
 \\
C2. 
Векторы \(\overrightarrow{a},\ \overrightarrow{b},\ \overrightarrow{c}\) и \(\overrightarrow{d}\) связаны соотношениями \(\lbrack\overrightarrow{a},\overrightarrow{b}\rbrack = \lbrack\overrightarrow{c},\overrightarrow{d}\rbrack,\ \ \lbrack\overrightarrow{a},\overrightarrow{c}\rbrack = \lbrack\overrightarrow{b},\overrightarrow{d}\rbrack\). Доказать коллинеарность векторов \(\overrightarrow{a} - \overrightarrow{d}\) и \(\overrightarrow{b} - \overrightarrow{c}\).
 \\
C3. 
Составить уравнение плоскости, проходящей через прямую \(5x - y - 2z - 3 = 0,\ \ \ \ 3x - 2y - 5z + 2 = 0\) перпендикулярно плоскости \(x + 19y - 7z - 11 \doteq 0\).
 \\

\end{tabular}
\vspace{1cm}


\begin{tabular}{m{17cm}}
\textbf{40-вариант}
\newline

T1. 
Скалярное произведение векторов.
 \\
T2. 
Уравнения плоскости. Взаимное расположение плоскости.
 \\
A1. 
Доказать, что треугольник с вершинами \(A(3; - 1;2)\), \(B(0; - 4;2)\) и \(C( - 3;2;1)\) равнобедренный.
 \\
A2. 
Вычислить косинус угла, образованного векторами \(\overrightarrow{a} = \{ 2; - 4;4\}\) и \(\overrightarrow{b} = \{ - 3;2;6\}\).
 \\
A3. Составить уравнение плоскости, которая проходит через точку \(M_{1}(2;1; - 1)\) и имеет нормальный вектор \(\overrightarrow{n} = \{ 1; - 2;3\}\).
 \\
B1. 
Точка \(A(2; - 5)\) является вершиной квадрата, одна из сторон которого лежит на прямой \(x - 2y - 7 = 0\). Вычислить площадь этого квадрата.
 \\
B2. 
Даны векторы \(\overrightarrow{a},\ \overrightarrow{b}\) и \(\overrightarrow{c}\) удовлетворяющие условию \(\overrightarrow{a} + \overrightarrow{b} + \overrightarrow{c} = 0\). Зная, что \(|\overrightarrow{a}| = 3,\ \ \ \ |\overrightarrow{b}| = 1\) и \(|\overrightarrow{c}| = 4\), вычислить \(\left( \overrightarrow{a},\overrightarrow{b} \right) + \left( \overrightarrow{b},\overrightarrow{c} \right) + \left( \overrightarrow{c},\overrightarrow{a} \right)\).
 \\
B3. 
Составить канонические уравнения следующих прямых: 1) \(x - 2y + 3z - 4 = 0,3x + 2y - 5z - 4 = 0\); 2) \(5x + y + z = 0,2x + 3y - 2z + 5 = 0\); 3) \(x - 2y + 3z + 1 = 0,2x + y - 4z - 8 = 0\).
 \\
C1. 
Последовательные вершины четырехугольника суть точки \(A( - 3;5),B( - 1; - 4),C(7; - 1)\) и \(D(2;9)\). Установить, является ли этот четырехугольник выпуклым.
 \\
C2. 
Доказать тождество \((\lbrack\overrightarrow{a} + \overrightarrow{b},\overrightarrow{b} + \overrightarrow{c}\rbrack,\overrightarrow{c} + \overrightarrow{a}) = 2(\lbrack\overrightarrow{a},\overrightarrow{b}\rbrack,\overrightarrow{c})\).
 \\
C3. Составить уравнение плоскости, проходящей через прямую пересечения плоскостей \(2x - y + 3z - 5 = 0,x + 2y - z + 2 = 0\) параллельно вектору \(\overrightarrow{l} = \{ 2; - 1; - 2\}\).
 \\

\end{tabular}
\vspace{1cm}


\begin{tabular}{m{17cm}}
\textbf{41-вариант}
\newline

T1. 
Выражение скалярного, векторного и смешанного произведения векторов в координатах.
 \\
T2. 
Расстояние от точки до плоскости, от точки до прямой в пространстве и между двумя скрещивающими прямыми. \\
A1. 
Определить угол \(\varphi\) между двумя прямыми: 1) \(5x - y + 7 = 0,\ \ \ \ 3x + 2y = 0\); 2) \(3x - 2y + 7 = 0,2x + 3y - 3 = 0\); 3) \(x - 2y - 4 = 0,\ \ \ \ 2x - 4y + 3 = 0\);
 \\
A2. 
Установить, компланарны ли векторы \(\overrightarrow{a},\overrightarrow{b},\overrightarrow{c}\), если 1)\(a = \{ 2;3; - 1\},\ \ \ \ b = \{ 1; - 1;3\},\ \ \ \ c = \{ 1;9; - 11\}\); 2)\(a = \{ 3; - 2;1\},\ \ \ \ b = \{ 2;1;2\},\ \ \ \ c = \{ 3; - 1; - 2\}\); 3)\(a = \{ 2; - 1;2\},\ \ \ \ b = \{ 1;2; - 3\},\ \ \ \ c = \{ 3; - 4;7\}\). \\
A3. 
Две грани куба лежат на плоскостях \(2x - 2y + z - 1 = 0,\) \(2x - 2y + z + 5 = 0\). Вычислить объем этого куба.
 \\
B1. 
Составить уравнение прямой, которая проходит через точку пересечения прямых \(2x + y - 2 = 0,x - 5y - 23 = 0\) и делит пополам отрезок, ограниченный точками \(M_{1}(5; - 6)\) и \(M_{2}( - 1; - 4)\). Решить задачу, не вычисляя координат точки пересечения данных прямых.
 \\
B2. 
Векторы \(\overrightarrow{a}\) и \(\overrightarrow{b}\) взаимно перпендикулярны; вектор \(\overrightarrow{c}\) образует с ними углы, равные \(\pi/3\); зная, что \(|\overrightarrow{a}| = 3\), \(|\overrightarrow{b}| = 5,\ \ \ \ |\overrightarrow{c}| = 8\), вычислить: 1) \(\left( 3\overrightarrow{a} - 2\overrightarrow{b},\overrightarrow{b} + 3\overrightarrow{c} \right)\); 2) \((\overrightarrow{a} + \overrightarrow{b} + \overrightarrow{c})^{2};\) \(3)(\overrightarrow{a} + 2\overrightarrow{b} - 3\overrightarrow{c})^{2}\).
 \\
B3. 
В пучке плоскостей \(2x - 3y + z - 3 + \lambda(x + 3y + 2z + 1) = 0\) найти плоскость, которая: 1) проходит через точку \(M_{1}(1; - 2;3)\); 2) параллельна оси \(Ox\); 3) параллельна оси \(Oy\); 4) параллельна оси \(Oz\).
 \\
C1. 
Даны вершины треугольника \(A(2; - 2),B(3; - 5)\) и \(C(5;7)\). Составить уравнение перпендикуляра, опущенного из вершины \(C\) на биссектрису внутреннего угла при вершине \(A\).
 \\
C2. Даны единичнье векторы \(\overrightarrow{a},\ \overrightarrow{b}\) и \(\overrightarrow{c}\), удовлетворяющие условию \(\overrightarrow{a} + \overrightarrow{b} + \overrightarrow{c} = 0\). Вычислить \(\left( \overrightarrow{a},\overrightarrow{b} \right) + \left( \overrightarrow{b},\overrightarrow{c} \right) + \left( \overrightarrow{c},\overrightarrow{a} \right)\).
 \\
C3. 
Составить уравнение плоскости, проходящей через прямую пересечения плоскостей \(5x - 2y - z - 3 = 0,x + 3y - 2z + 5 = 0\) параллельно вектору \(\overrightarrow{l} = \{ 7;9;17\}\).
 \\

\end{tabular}
\vspace{1cm}


\begin{tabular}{m{17cm}}
\textbf{42-вариант}
\newline

T1. 
Координаты вектора.
 \\
T2. Уравнении прямой на плоскости.
 \\
A1. 
Дано уравнение пучка прямых \(\alpha(3x + 2y - 9) +\) \(+ \beta(2x + 5y + 5) = 0\). Найти, при каком значении \(C\) прямая \(4x - 3y + C = 0\) будет принадлежать этому пучку.
 \\
A2. 
Даны вершины треугольника \(A(3;2; - 3)\), \(B(5;1; - 1)\) и \(C(1; - 2;1)\). Определить его внешний угол при вершине \(A\).
 \\
A3. 
Составить канонические уравнения прямой, проходящей через данные точки: 1) \((1; - 2;1),(3;1; - 1)\); 2) \((3; - 1;0),(1;0, - 3);\) \(3)(0; - 2;3),(3; - 2;1)\); 4) \((1;2; - 4),( - 1;2; - 4)\).
 \\
B1. 
Отрезок, ограниченный точками \(A(1; - 3)\) и \(B(4;3)\), разделен на три равные части. Определить координаты точек деления.
 \\
B2. 
Векторы \(\overrightarrow{a}\) и \(\overrightarrow{b}\) образуют угол \(\varphi = 2\pi/3\). Зная, что \(|\overrightarrow{a}| = 1,|\overrightarrow{b}| = 2\), вычислить: \(1)\left. \ \lbrack\overrightarrow{a},\overrightarrow{b}\rbrack^{2};\ \ \ 2 \right)\lbrack 2\overrightarrow{a} + \overrightarrow{b},\overrightarrow{a} + 2\overrightarrow{b}\rbrack^{2};\ \ \ \ \) 3) \(\lbrack\overrightarrow{a} + 3\overrightarrow{b},3\overrightarrow{a} - \overrightarrow{b}\rbrack^{2}\).
 \\
B3. 
Даны прямые \(\frac{x + 2}{2} = \frac{y}{- 3} = \frac{z - 1}{4},\ \ \ \ \frac{x - 3}{l} = \frac{y - 1}{4} = \frac{z - 7}{2}\) при каком значении \(l\) они пересекаются?
 \\
C1. 
Стороны треугольника лежат на прямых \(x + 5y - 7 = 0,\) \(3x - 2y - 4 = 0,\) \(7x + y + 19 = 0\). Вычислить его площадь \(S\).
 \\
C2. 
Доказать, что необходимым и достаточным условием компланарности векторов \(\overrightarrow{a},\ \overrightarrow{b},\ \overrightarrow{c}\) является зависимость \(\alpha\overrightarrow{a} + \beta\overrightarrow{b} + \gamma\overrightarrow{c} = 0\), где по крайней мере одно из чисел \(\alpha,\beta,\gamma\) не равно нулю.
 \\
C3. 
Найти точку \(Q\), симметричную точке \(P(4;1;6)\) относительно прямой \(x - y - 4z + 12 = 0,2x + y - 2z + 3 = 0\).
 \\

\end{tabular}
\vspace{1cm}


\begin{tabular}{m{17cm}}
\textbf{43-вариант}
\newline

T1. 
Понятие о векторе. Линейные операции над векторами.
 \\
T2. 
Взаимное расположение прямой и плоскости в пространстве.
 \\
A1. 
Даны последовательные вершины \(A(2;3),B(0;6)\), \(C( - 1;5),D(0;1)\) и \(E(1;1)\) однородной пятиугольной пластинки. Определить координаты ее центра масс.
 \\
A2. 
Векторы \(\overrightarrow{a}\) и \(\overrightarrow{b}\) образуют угол \(\varphi = \pi/6\). Зная, что \(|\overrightarrow{a}| = 6,|\overrightarrow{b}| = 5\), вычислить \(\left| \left\lbrack \overrightarrow{a},\overrightarrow{b} \right\rbrack \right|\)
 \\
A3. 
Вычислить расстояние \(d\) от точки \(P(2;3; - 1)\) до следующих прямых: 1) \(\frac{x - 5}{3} = \frac{y}{2} = \frac{z + 25}{- 2}\); 2) \(x = t + 1,y = t + 2,z = 4t + 13\).
 \\
B1. 
Найти уравнение прямой, принадлежащей пучку прямых \(\ \ \ \ \alpha(x + 2y - 5) + \beta(3x - 2y + 1) = 0\) и 1) проходящей через точку \(A(3; - 1)\); 2) проходящей через начало координат; 3) параллельной оси \(Ox\); 4) параллельной оси \(Oy\); 5) параллельной прямой \(4x + 3y + 5 = 0\); 6) перпендикулярной к прямой \(2x + 3y + 7 = 0\).
 \\
B2. На плоскости даны три вектора \(\overrightarrow{a} = \{ 3; - 2\}\), \(\overrightarrow{b} = \{ - 2;1\}\) и \(\overrightarrow{c} = \{ 7; - 4\}\). Определить разложение каждого из этих трех векторов, принимая в качестве базиса два других.
 \\
B3. 
Доказать, что прямая \(5x - 3y + 2z - 5 = 0\), \(2x - y - z - 1 = 0\) лежит в плоскости \(4x - 3y + 7z - 7 = 0\).
 \\
C1. 
Дано уравнение пучка прямых \(\alpha(3x - 4y - 3) +\) \(+ \beta(2x + 3y - 1) = 0\). Написать уравнение прямой этого пучка, проходящей через центр масс однородной треугольной пластинки, вершины которой суть точки \(A( - 1;2),B(4; - 4)\) и \(C(6; - 1)\).
 \\
C2. 
Доказать, что векторы \(\overrightarrow{a},\ \overrightarrow{b},\ \overrightarrow{c}\) удовлетворяющие условию \(\lbrack\overrightarrow{a},\overrightarrow{b}\rbrack + \lbrack\overrightarrow{b},\overrightarrow{c}\rbrack + \lbrack\overrightarrow{c},\overrightarrow{a}\rbrack = 0\), компланарны.
 \\
C3. 
Даны вершины треугольника \(A(1; - 2; - 4)\), \(B(3;1; - 3)\) и \(C(5;1; - 7)\). Составить параметрические уравнения его высоты, опущенной из вершины \(B\) на противоположную сторону.
 \\

\end{tabular}
\vspace{1cm}


\begin{tabular}{m{17cm}}
\textbf{44-вариант}
\newline

T1. 
Векторное произведение и смешанное произведение векторов.
 \\
T2. Уравнении прямой на плоскости.
 \\
A1. 
Составить уравнение геометрического места точек, равноудаленных от двух параллельных прямых: 1) \(3x - y + 7 = 0,\ \ \ \ 3x - y - 3 = 0\); 2) \(x - 2y + 3 = 0,\ \ \ \ x - 2y + 7 = 0\); 3) \(5x - 2y - 6 = 0,\ \ \ \ 10x - 4y + 3 = 0\).
 \\
A2. 
Даны: \(|\overrightarrow{a}| = 3,|\overrightarrow{b}| = 26\) и \(|\lbrack\overrightarrow{a},\overrightarrow{b}\rbrack| = 72\). Вычислить \(\left( \overrightarrow{a},\overrightarrow{b} \right)\).
 \\
A3. 
Доказать, что прямая \(x = 3t - 2,y = - 4t + 1\), \(z = 4t - 5\) параллельна плоскости \(4x - 3y - 6z - 5 = 0\).
 \\
B1. 
Определить, лежат ли точки \(M(2;3)\) и \(N(5; - 1)\) в одном, в смежных или вертикальных углах, образованных при пересечении двух прямых: 1) \(x - 3y - 5 = 0,2x + 9y - 2 = 0\); 2) \(2x + 7y - 5 = 0,x + 3y + 7 = 0\); 3) \(12x + y - 1 = 0,\ \ \ \ 13x + 2y - 5 = 0\).
 \\
B2. 
Доказать, что точки \(A(1;2; - 1),B(0;1;5)\), \(C( - 1;2;1),D(2;1;3)\) лежат в одной плоскости.
 \\
B3. 
Составить уравнение плоскости, проходящей через точки \(M_{1}(2; - 1;3)\) и \(M_{2}(3;1;2)\) параллельно вектору \(\overrightarrow{a} = \{ 3; - 1;4\}\).
 \\
C1. 
Определить координаты точки \(O^{'}\) - нового начала координат, если точка \(A(3; - 4)\) лежит на новой оси абсцисс, а точка \(B(2;3)\) лежит на новой оси ординат, причем оси старой и новой систем координат имеют соответственно одинаковые направления.
 \\
C2. 
Даны векторы \(\overrightarrow{AB} = \overrightarrow{b}\) и \(\overrightarrow{AC} = \overrightarrow{c}\), совпадающие со сторонами треугольника \(ABC\). Найти разложение вектора, приложенного к вершине \(B\) этого треугольника и совпадающего с его высотой \(BD\) по базису \(\overrightarrow{b},\ \overrightarrow{c}\).
 \\
C3. 
Составить уравнения прямой, которая проходит через точку \(M_{1}( - 1;2; - 3)\) перпендикулярно к вектору \(\overrightarrow{a} = \{ 6; - 2; - 3\}\) и пересекает прямую \(\frac{x - 1}{3} = \frac{y + 1}{2} = \frac{z - 3}{- 5}\).
 \\

\end{tabular}
\vspace{1cm}


\begin{tabular}{m{17cm}}
\textbf{45-вариант}
\newline

T1. 
Преобразование декартовой системы координат на плоскости и пространстве. \\
T2. 
Расстояние от точки до плоскости, от точки до прямой в пространстве и между двумя скрещивающими прямыми. \\
A1. 
Даны последовательные вершин однородной четырехугольной пластинки \(A(2;1),B(5;3),C( - 1;7)\) и \(D( - 7;5)\). Определить координат ее центра масс.
 \\
A2. 
Вычислив внутренние углы треугольника с вершинами \(A(1;2;1),B(3; - 1;7),C(7;4; - 2)\), убедиться, что этот треугольник равнобедренный.
 \\
A3. 
Привести каждое из следующих уравнений плоскостей к нормальному виду: 1) \(2x - 2y + z - 18 = 0\); 2) \(\frac{3}{7}x - \frac{6}{7}y + \frac{2}{7}z + 3 = 0\); 3) \(4x - 6y - 12z - 11 = 0\); 4) \(- 4x - 4y + 2z + 1 = 0\); 5) \(5y - 12z + 26 = 0\);
 \\
B1. 
Стороны \(AB\), \(BC\) и \(AC\) треугольника \(ABC\) даны соответственно уравнениями \(4x + 3y - 5 = 0\), \(x - 3y + 10 = 0\), \(x - 2 = 0\). Определить координаты его вершин.
 \\
B2. 
Векторы \(\overrightarrow{a},\ \overrightarrow{b},\ \overrightarrow{c}\) образующие правую тройку, взаимно перпендикулярны. Зная, что \(|\overrightarrow{a}| = 4,\ \ |\overrightarrow{b}| = 2\), \(|\overrightarrow{c}| = 3\), вычислить \(\left( \left\lbrack \overrightarrow{a},\overrightarrow{b} \right\rbrack,\overrightarrow{c} \right)\).
 \\
B3. 
Вычислить объем пирамиды, ограниченной плоскостью \(2x - 3y + 6z - 12 = 0\) и координатными плоскостями.
 \\
C1. 
На оси абсцисс найти такую точку \(P\), чтобы сумма ее расстояний до точек \(M(1;2)\) и \(\dot{N}(3;4)\) была наименьшей.
 \\
C2. 
Доказать, что векторы \(\overrightarrow{a},\ \overrightarrow{b},\ \overrightarrow{c}\) удовлетворяющие условию \(\lbrack\overrightarrow{a},\overrightarrow{b}\rbrack + \lbrack\overrightarrow{b},\overrightarrow{c}\rbrack + \lbrack\overrightarrow{c},\overrightarrow{a}\rbrack = 0\), компланарны.
 \\
C3. 
На плоскости \(Oxz\) найти такую точку \(P\), разность расстояний которой до точек \(M_{1}(3;2; - 5)\) и \(M_{2}(8; - 4\); -13) была бы наибольшей.
 \\

\end{tabular}
\vspace{1cm}


\begin{tabular}{m{17cm}}
\textbf{46-вариант}
\newline

T1. 
Семейство линейно зависимых и линейно независимых векторов.
 \\
T2. 
Взаимное расположение прямой на плоскости.
 \\
A1. 
Точки \(M(2; - 1),N( - 1;4)\) и \(P( - 2;2)\) являются серединами сторон треугольника. Определить его вершины.
 \\
A2. 
Даны вершины треугольника \(A( - 1; - 2;4)\), \(B( - 4; - 2;0)\) и \(C(3; - 2;1)\). Определить его внутренний угол при вершине \(B\).
 \\
A3. 
При каком значении \(m\) прямая \(\frac{x + 1}{3} = \frac{y - 2}{m} = \frac{z + 3}{- 2}\) параллельна плоскости \(x - 3y + 6z + 7 = 0\) ?
 \\
B1. 
Определить, лежит ли точка \(M( - 3;2)\) внутри или вне треугольника, стороны которого даны уравнениями \(x + y - 4 = 0,\ \ \ \ 3x - 7y + 8 = 0,\ \ \ \ 4x - y - 31 = 0\).
 \\
B2. 
Векторы \(\overrightarrow{a}\) и \(\overrightarrow{b}\) взаимно перпендикулярны. Зная, что \(|\overrightarrow{a}| = 3,|\overrightarrow{b}| = 4\), вычислить \(1)|\lbrack\overrightarrow{a} + \overrightarrow{b},\overrightarrow{a} - \overrightarrow{b}\rbrack|;\ \ 2)|\lbrack 3\overrightarrow{a} - \overrightarrow{b},\overrightarrow{a} - 2\overrightarrow{b}\rbrack|\).
 \\
B3. 
Составить параметрические уравнения следующих прямых: 1) \(2x + 3y - z - 4 = 0,3x - 5y + 2z + 1 = 0\); 2) \(x + 2y - z - 6 = 0,2x - y + z + 1 = 0\).
 \\
C1. 
На прямой \(2x - y - 5 = 0\) найти такую точку \(P\), сумма расстояний которой до точек \(A( - 7;1),B( - 5;5)\) была бы наименьшей.
 \\
C2. 
Доказать тождество \(\lbrack\overrightarrow{a},\overrightarrow{b}\rbrack^{2} + (\overrightarrow{a},\overrightarrow{b})^{2} = {\overrightarrow{a}}^{2}{\overrightarrow{b}}^{2}\).
 \\
C3. 
Даны вершины треугольника \(A(2; - 1; - 3)\), \(B(5;2; - 7)\) и \(C( - 7;11;6)\). Составить канонические уравнения биссектрисы его внешнего угла при вершине \(A\).
 \\

\end{tabular}
\vspace{1cm}


\begin{tabular}{m{17cm}}
\textbf{47-вариант}
\newline

T1. 
Скалярное произведение векторов.
 \\
T2. 
Расстояние от точки до прямой. Уравнение пучка прямых.
 \\
A1. 
Установить, какие из следующих пар прямых перпендикулярны: 1) \(3x - y + 5 = 0,x + 3y - 1 = 0\); 2) \(3x - 4y + 1 = 0,\ \ \ \ 4x - 3y + 7 = 0\); 3) \(6x - 15y + 7 = 0,\ \ \ \ 10x + 4y - 3 = 0\); 4) \(9x - 12y + 5 = 0,\ \ \ \ 8x + 6y - 13 = 0\); 5) \(7x - 2y + 1 = 0,4x + 6y + 17 = 0\); Решить задачу, не вычисляя угловых коэффициентов данных прямых.
 \\
A2. Может ли вектор составлять с координатными осями следующие углы: 1) \(\alpha = 45^{{^\circ}},\beta = 60^{{^\circ}},\gamma = 120^{{^\circ}}\); 2) \(\alpha = 45^{{^\circ}},\ \ \ \ \beta = 135^{{^\circ}},\ \ \ \ \gamma = 60^{{^\circ}}\); 3) \(\alpha = 90^{{^\circ}},\ \ \ \ \beta = 150^{{^\circ}}\), \(\gamma = 60^{{^\circ}}?\)
 \\
A3. 
Составить канонические уравнения прямой, проходящей через точку \(M_{1}(2;0; - 3)\) параллельно: 1) вектору \(\overrightarrow{a} = \{ 2; - 3;5\}\); 2) прямой \(\frac{x - 1}{5} = \frac{y + 2}{2} = \frac{z + 1}{- 1}\); 3) оси \(Ox\); 4) оси \(Oy\); 5) оси \(Oz\).
 \\
B1. 
Найти проекцию точки \(P( - 8;12)\) на прямую, проходящую через точки \(A(2; - 3)\) и \(B( - 5;1)\).
 \\
B2. 
Дано, что \(|\overrightarrow{a}| = 3,|\overrightarrow{b}| = 5\). Определить, при каком значении \(\alpha\) векторы \(\overrightarrow{a} + \alpha\overrightarrow{b},\overrightarrow{a} - \alpha\overrightarrow{b}\) будут взаимно перпендикулярны.
 \\
B3. 
В пучке плоскостей \(2x - 3y + z - 3 + \lambda(x + 3y + 2z + 1) = 0\) найти плоскость, которая: 1) проходит через точку \(M_{1}(1; - 2;3)\); 2) параллельна оси \(Ox\); 3) параллельна оси \(Oy\); 4) параллельна оси \(Oz\).
 \\
C1. 
Составить уравнения сторон треугольника, зная одну из его вершин \(A(4; - 1)\) и уравнения двух биссектрис \(x - 1 = 0\) и \(x - y - 1 = 0\).
 \\
C2. 
Какому условию должны удовлетворять векторы \(\overrightarrow{a},\overrightarrow{b}\), чтобы векторы \(\overrightarrow{a} + \overrightarrow{b}\) и \(\overrightarrow{a} - \overrightarrow{b}\) были коллинеарны?
 \\
C3. 
Найти проекцию точки \(C(3; - 4; - 2)\) на плоскость, проходящую через параллельные прямые \(\frac{x - 5}{13} = \frac{y - 6}{1} = \frac{z + 3}{- 4},\ \ \ \ \frac{x - 2}{13} = \frac{y - 3}{1} = \frac{z + 3}{- 4}\).
 \\

\end{tabular}
\vspace{1cm}


\begin{tabular}{m{17cm}}
\textbf{48-вариант}
\newline

T1. Предмет и методы аналитической геометрии.
 \\
T2. 
Уравнения прямой в пространстве. Взаимное расположение прямых.
 \\
A1. 
Вычислить величину отклонения \(\delta\) и расстояние \(d\) от точки до прямой в каждом из следующих случаев: 1) \(A(2; - 1),4x + 3y + 10 = 0;\) \(2)B(0; - 3),5x - 12y - 23 = 0;\); \(3)P( - 2;3),\ \ 3x - 4y - 2 = 0\); 4) \(Q(1; - 2),x - 2y - 5 = 0\).
 \\
A2. 
На плоскости даны два вектора \(\overrightarrow{p} = \{ 2; - 3\}\), \(\overrightarrow{q} = \{ 1;2\}\). Найти разложение вектора \(\overrightarrow{a} = \{ 9;4\}\) по базису \(\overrightarrow{p},\ \overrightarrow{q}\).
 \\
A3. 
Вычислить величину отклонения \(\delta\) и расстояние \(d\) от точки до плоскости в каждом из следующих случаев: 1) \(M_{1}( - 2; - 4;3)\), \(2x - y + 2z + 3 = 0;\) 2) \(M_{2}(2; - 1; - 1),16x - 12y + 15z - 4 = 0\); 3) \(M_{3}(1;2; - 3),\ \ \ \ 5x - 3y + z + 4 = 0\);
 \\
B1. 
Даны две смежные вершины параллелограмма \(A( - 3;5),B(1;7)\) и точка пересечения его диагоналей \(M(1;1)\). Определить две другие вершины.
 \\
B2. 
Дано, что \(|\overrightarrow{a}| = 3,|\overrightarrow{b}| = 5\). Определить, при каком значении \(\alpha\) векторы \(\overrightarrow{a} + \alpha\overrightarrow{b},\overrightarrow{a} - \alpha\overrightarrow{b}\) будут взаимно перпендикулярны.
 \\
B3. 
Составить уравнение плоскости, которая проходит через прямую пересечения плоскостей \(3x - y + 2z + 9 = 0\), \(x + z - 3 = 0\): 1) и через точку \(M_{1}(4; - 2; - 3)\); 2)параллельно оси \(Ox\); 3) параллельно оси \(Oy\); 4) параллельно оси \(Oz\).
 \\
C1. 
В треугольнике \(ABC\) даны: уравнение стороны \(AB:5x - 3y + 2 = 0\), уравнения высот \(AM:4x - 3y + 1 = 0\) и \(BN:7x + 2y - 22 = 0\). Составить уравнения двух других сторон и третьей высоты этого треугольника.
 \\
C2. 
Доказать, что необходимым и достаточным условием компланарности векторов \(\overrightarrow{a},\ \overrightarrow{b},\ \overrightarrow{c}\) является зависимость \(\alpha\overrightarrow{a} + \beta\overrightarrow{b} + \gamma\overrightarrow{c} = 0\), где по крайней мере одно из чисел \(\alpha,\beta,\gamma\) не равно нулю. \\
C3. 
Даны вершины треугольника \(A(3;6; - 7),B( - 5\); \(2;3)\) и \(C(4; - 7; - 2)\). Составить параметрические уравнения его медианы, проведенной из вершины \(C\).
 \\

\end{tabular}
\vspace{1cm}


\begin{tabular}{m{17cm}}
\textbf{49-вариант}
\newline

T1. 
Векторное произведение и смешанное произведение векторов.
 \\
T2. 
Уравнения плоскости. Взаимное расположение плоскости.
 \\
A1. 
Даны вершины \(M_{1}(3;2; - 5),M_{2}(1; - 4;3)\) и \(M_{3}( - 3\); \(0;1)\) треугольника. Найти середины его сторон.
 \\
A2. 
Даны вершины четырехугольника \(A(1; - 2;2)\), \(B(1;4;0),C( - 4;1;1)\) и \(D( - 5; - 5;3)\). Доказать, что его диагонали \(AC\) и \(BD\) взаимно перпендикулярны.
 \\
A3. 
Точка \(P(2; - 1; - 1)\) служит основанием перпендикуляра, опущенного из начала координат на плоскость. Составить уравнение этой плоскости.
 \\
B1. 
Определить, при каком значении \(m\) две прямые \(\begin{matrix}
mx + (2m + 3)y + m + 6 = 0 \\
(2m + 1)x + (m - 1)y + m - 2 = 0
\end{matrix}\)пересекаются в точке, лежащей на оси ординат.
 \\
B2. 
Даны точки \(A(2; - 1;2),B(1;2; - 1)\) и \(C(3;2;1)\). Найти координаты векторных пронзведений: 1) \(\lbrack\overline{AB},\overline{BC}\rbrack\); 2) \(\lbrack\overline{BC} - 2\overline{CA},\overline{CB}\rbrack\).
 \\
B3. 
Доказать, что прямая \(5x - 3y + 2z - 5 = 0\), \(2x - y - z - 1 = 0\) лежит в плоскости \(4x - 3y + 7z - 7 = 0\).
 \\
C1. 
Составить уравнения сторон треугольника \(ABC\), зная одну его вершину \(A(2; - 1)\), а также уравнения высоты \(7x - 10y + 1 = 0\) и 6иссектрисы \(3x - 2y + 5 = 0\), проведенных из одной вершины. Решить задачу, не вычисляя координат вершин \(B\) и \(C\).
 \\
C2. 
Доказать, что вектор \(\overrightarrow{p} = \overrightarrow{b} - \frac{\overrightarrow{a}(\overrightarrow{a},\overrightarrow{b})}{{\overrightarrow{a}}^{2}}\) перпендикулярен к вектору \(\overrightarrow{a}\).
 \\
C3. 
Составить уравнения прямой, которая проходит через точку \(M_{1}( - 1;2; - 3)\) перпендикулярно к вектору \(\overrightarrow{a} = \{ 6; - 2; - 3\}\) и пересекает прямую \(\frac{x - 1}{3} = \frac{y + 1}{2} = \frac{z - 3}{- 5}\).
 \\

\end{tabular}
\vspace{1cm}


\begin{tabular}{m{17cm}}
\textbf{50-вариант}
\newline

T1. 
Координаты вектора.
 \\
T2. 
Уравнения прямой в пространстве. Взаимное расположение прямых.
 \\
A1. 
Даны точки \(A(3; - 1)\) и \(B(2;1)\). Определить: координаты точки \(M\), симметричной точке \(A\) относительно точки \(B\); координаты точки \(N\), симметричной точке \(B\) относительно точки \(A\).
 \\
A2. 
Даны векторы \(\overrightarrow{a} = \{ 1; - 1;3\},\ \ \ \overrightarrow{b} = \{ - 2;2;1\}\), \(\overrightarrow{c} = \{ 3; - 2;5\}\). Вычислить \((\lbrack\overrightarrow{a},\overrightarrow{b}\rbrack,\overrightarrow{c})\).
 \\
A3. 
Установить, какие из следующих пар уравнений определяют параллельные плоскости; 1) \(2x - 3y + 5z - 7 = 0,\ \ \ \ 2x - 3y + 5z + 3 = 0\); 2) \(4x + 2y - 4z + 5 = 0,\ \ \ \ 2x + y + 2z - 1 = 0\); 3) \(\ \ \ \ x - 3z + 2 = 0,\ \ \ \ 2x - 6z - 7 = 0\).
 \\
B1. 
Определить, при каких значениях \(m\) и \(n\) две прямые \(mx + 8y + n = 0,\ \ \ \ 2x + my - 1 = 0\) 1) параллельны; 2) совпадают; 3) перпендикулярны.
 \\
B2. 
Векторы \(\overrightarrow{a}\) и \(\overrightarrow{b}\) образуют угол \(\varphi = 2\pi/3\). Зная, что \(|\overrightarrow{a}| = 1,|\overrightarrow{b}| = 2\), вычислить: \(1)\left. \ \lbrack\overrightarrow{a},\overrightarrow{b}\rbrack^{2};\ \ \ 2 \right)\lbrack 2\overrightarrow{a} + \overrightarrow{b},\overrightarrow{a} + 2\overrightarrow{b}\rbrack^{2};\ \ \ \ \) 3) \(\lbrack\overrightarrow{a} + 3\overrightarrow{b},3\overrightarrow{a} - \overrightarrow{b}\rbrack^{2}\).
 \\
B3. 
Составить канонические уравнения следующих прямых: 1) \(x - 2y + 3z - 4 = 0,3x + 2y - 5z - 4 = 0\); 2) \(5x + y + z = 0,2x + 3y - 2z + 5 = 0\); 3) \(x - 2y + 3z + 1 = 0,2x + y - 4z - 8 = 0\).
 \\
C1. 
Даны вершины треугольника \(A(1; - 1),B( - 2;1)\) и \(C(3;5)\). Составить уравнение перпендикуляра, опущенного из вершины \(A\) на медиану, проведенную из вершины \(B\).
 \\
C2. 
Доказать тождество \((\lbrack\overrightarrow{a} + \overrightarrow{b},\overrightarrow{b} + \overrightarrow{c}\rbrack,\overrightarrow{c} + \overrightarrow{a}) = 2(\lbrack\overrightarrow{a},\overrightarrow{b}\rbrack,\overrightarrow{c})\).
 \\
C3. 
Даны вершины треугольника \(A(3;6; - 7),B( - 5\); \(2;3)\) и \(C(4; - 7; - 2)\). Составить параметрические уравнения его медианы, проведенной из вершины \(C\).
 \\

\end{tabular}
\vspace{1cm}



\end{document}
