Составить уравнение плоскости, проходящей через прямую пересечения плоскостей \(2x - y + 3z - 5 = 0,x + 2y - z + 2 = 0\) параллельно вектору \(\overrightarrow{l} = \{ 2; - 1; - 2\}\).
====
Составить уравнение плоскости, проходящей через прямую пересечения плоскостей \(5x - 2y - z - 3 = 0,x + 3y - 2z + 5 = 0\) параллельно вектору \(\overrightarrow{l} = \{ 7;9;17\}\).
====
Составить уравнение плоскости, проходящей через прямую пересечения плоскостей \(3x - 2y + z - 3 = 0,x - 2z = 0\) перпендикулярно плоскости \(x - 2y + z + 5 = 0\).
====
Составить уравнение плоскости, проходящей через прямую \(5x - y - 2z - 3 = 0,\ \ \ \ 3x - 2y - 5z + 2 = 0\) перпендикулярно плоскости \(x + 19y - 7z - 11 \doteq 0\).
====
Даны вершины треугольника \(A(3;6; - 7),B( - 5\); \(2;3)\) и \(C(4; - 7; - 2)\). Составить параметрические уравнения его медианы, проведенной из вершины \(C\).
====
Даны вершины треугольника \(A(3; - 1; - 3)\), \(B(1;2; - 7)\) и \(C( - 5;14; - 3)\). Составить канонические уравнения биссектрисы его внутреннего угла при вершине \(C\).
====
Даны вершины треугольника \(A(2; - 1; - 3)\), \(B(5;2; - 7)\) и \(C( - 7;11;6)\). Составить канонические уравнения биссектрисы его внешнего угла при вершине \(A\).
====
Даны вершины треугольника \(A(1; - 2; - 4)\), \(B(3;1; - 3)\) и \(C(5;1; - 7)\). Составить параметрические уравнения его высоты, опущенной из вершины \(B\) на противоположную сторону.
====
Составить уравнения прямой, которая проходит через точку \(M_{1}( - 1;2; - 3)\) перпендикулярно к вектору \(\overrightarrow{a} = \{ 6; - 2; - 3\}\) и пересекает прямую \(\frac{x - 1}{3} = \frac{y + 1}{2} = \frac{z - 3}{- 5}\).
====
Найти точку \(Q\), симметричную точке \(P(4;1;6)\) относительно прямой \(x - y - 4z + 12 = 0,2x + y - 2z + 3 = 0\).
====
На плоскости \(Oxz\) найти такую точку \(P\), разность расстояний которой до точек \(M_{1}(3;2; - 5)\) и \(M_{2}(8; - 4\); -13) была бы наибольшей.
====
Найти проекцию точки \(C(3; - 4; - 2)\) на плоскость, проходящую через параллельные прямые \(\frac{x - 5}{13} = \frac{y - 6}{1} = \frac{z + 3}{- 4},\ \ \ \ \frac{x - 2}{13} = \frac{y - 3}{1} = \frac{z + 3}{- 4}\).
