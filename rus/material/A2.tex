Может ли вектор составлять с координатными осями следующие углы: 1) \(\alpha = 45^{{^\circ}},\beta = 60^{{^\circ}},\gamma = 120^{{^\circ}}\); 2) \(\alpha = 45^{{^\circ}},\ \ \ \ \beta = 135^{{^\circ}},\ \ \ \ \gamma = 60^{{^\circ}}\); 3) \(\alpha = 90^{{^\circ}},\ \ \ \ \beta = 150^{{^\circ}}\), \(\gamma = 60^{{^\circ}}?\)
====
На плоскости даны два вектора \(\overrightarrow{p} = \{ 2; - 3\}\), \(\overrightarrow{q} = \{ 1;2\}\). Найти разложение вектора \(\overrightarrow{a} = \{ 9;4\}\) по базису \(\overrightarrow{p},\ \overrightarrow{q}\).
====
Даны вершины четырехугольника \(A(1; - 2;2)\), \(B(1;4;0),C( - 4;1;1)\) и \(D( - 5; - 5;3)\). Доказать, что его диагонали \(AC\) и \(BD\) взаимно перпендикулярны.
====
Определить, при каком значении \(\alpha\) векторы \(\overrightarrow{a} = \alpha\overrightarrow{i} - 3\overrightarrow{j} + 2\overrightarrow{k}\) и \(\overrightarrow{b} = \overrightarrow{i} + 2\overrightarrow{j} - \alpha\overrightarrow{k}\) взаимно перпендикулярны.
====
Вычислить косинус угла, образованного векторами \(\overrightarrow{a} = \{ 2; - 4;4\}\) и \(\overrightarrow{b} = \{ - 3;2;6\}\).
====
Даны вершины треугольника \(A( - 1; - 2;4)\), \(B( - 4; - 2;0)\) и \(C(3; - 2;1)\). Определить его внутренний угол при вершине \(B\).
====
Даны вершины треугольника \(A(3;2; - 3)\), \(B(5;1; - 1)\) и \(C(1; - 2;1)\). Определить его внешний угол при вершине \(A\).
====
Вычислив внутренние углы треугольника с вершинами \(A(1;2;1),B(3; - 1;7),C(7;4; - 2)\), убедиться, что этот треугольник равнобедренный.
====
Векторы \(\overrightarrow{a}\) и \(\overrightarrow{b}\) образуют угол \(\varphi = \pi/6\). Зная, что \(|\overrightarrow{a}| = 6,|\overrightarrow{b}| = 5\), вычислить \(\left| \left\lbrack \overrightarrow{a},\overrightarrow{b} \right\rbrack \right|\)
====
Даны: \(|\overrightarrow{a}| = 10,|\overrightarrow{b}| = 2\) и \(\left( \overrightarrow{a},\overrightarrow{b} \right) = 12\). Вычислить \(\left| \left\lbrack \overrightarrow{a},\overrightarrow{b} \right\rbrack \right|\).
====
Даны: \(|\overrightarrow{a}| = 3,|\overrightarrow{b}| = 26\) и \(|\lbrack\overrightarrow{a},\overrightarrow{b}\rbrack| = 72\). Вычислить \(\left( \overrightarrow{a},\overrightarrow{b} \right)\).
====
Даны векторы \(\overrightarrow{a} = \{ 1; - 1;3\},\ \ \ \overrightarrow{b} = \{ - 2;2;1\}\), \(\overrightarrow{c} = \{ 3; - 2;5\}\). Вычислить \((\lbrack\overrightarrow{a},\overrightarrow{b}\rbrack,\overrightarrow{c})\).
====
Установить, компланарны ли векторы \(\overrightarrow{a},\overrightarrow{b},\overrightarrow{c}\), если 1)\(a = \{ 2;3; - 1\},\ \ \ \ b = \{ 1; - 1;3\},\ \ \ \ c = \{ 1;9; - 11\}\); 2)\(a = \{ 3; - 2;1\},\ \ \ \ b = \{ 2;1;2\},\ \ \ \ c = \{ 3; - 1; - 2\}\); 3)\(a = \{ 2; - 1;2\},\ \ \ \ b = \{ 1;2; - 3\},\ \ \ \ c = \{ 3; - 4;7\}\).