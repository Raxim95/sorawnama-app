Даны три вершины параллелограмма \(A(3; - 5)\), \(B(5; - 3)\) и \(C( - 1;3)\). Определить четвертую вершину \(D\), противоположную \(B\).
====
Даны две смежные вершины параллелограмма \(A( - 3;5),B(1;7)\) и точка пересечения его диагоналей \(M(1;1)\). Определить две другие вершины.
====
Даны три вершины \(A(2;3),B(4; - 1)\) и \(C(0;5)\) параллелограмма \emph{ABCD}. Найти его четвертую вершину \(D\).
====
Отрезок, ограниченный точками \(A(1; - 3)\) и \(B(4;3)\), разделен на три равные части. Определить координаты точек деления.
====
Вершины треугольника суть точки \(A(3;6),B( - 1\); 3) и \(C(2; - 1)\). Вычислить длину его высоты, проведенной из вершины C. (С помощью площадью треугольника)
====
Площадь треугольника \(S = 3\), две его вершины суть точки \(A(3;1)\) и \(B(1; - 3)\), а третья вершина \(C\) лежит на оси \(Oy\). Определить координаты вершины \(C\).
====
Площадь треугольника \(S = 4\), две его вершины суть точки \(A(2;1)\) и \(\dot{B}(3; - 2)\), а третья вершина \(C\) лежит на оси \(Ox\). Определить координаты вершины \(C\).
====
Площадь треугольника \(S = 3\), две его вершины суть точки \(A(3;1)\) и \(B(1; - 3)\), центр массе этого треугольника лежит на оси \(Ox\). Определить координаты третьей вершины \(C\).
====
Определить точки пересечения прямой \(2x - 3y - 12 = 0\) с координатными осями и построить эту прямую на чертеже.
====
Стороны \(AB\), \(BC\) и \(AC\) треугольника \(ABC\) даны соответственно уравнениями \(4x + 3y - 5 = 0\), \(x - 3y + 10 = 0\), \(x - 2 = 0\). Определить координаты его вершин.
====
Даны уравнения двух сторон параллелограмма \(8x + 3y + 1 = 0,2x + y - 1 = 0\) и уравнение одной из его диагоналей \(3x + 2y + 3 = 0\). Определить координаты вершин этого параллелограмма.
====
Найти проекцию точки \(P( - 6;4)\) на прямую \(4x - 5y + 3 = 0\).
====
Найти точку \(Q\), симметричную точке \(P( - 5;13)\) относительно прямой \(2x - 3y - 3 = 0.\)
====
В каждом из следующих случаев составить уравнение прямой, параллельной двум данным прямым и проходящей посередине между ними: 1)  \(3x - 2y - 1 = 0,3x - 2y - 13 = 0\); 2)  \(5x + y + 3 = 0,5x + y - 17 = 0\); 3)  \(2x + 3y - 6 = 0,\ \ \ \ 4x + 6y + 17 = 0\).
====
Даны середины сторон треугольника \(M_{1}(2;1)\), \(M_{2}(5;3)\) и \(M_{3}(3; - 4)\). Составить уравнение его сторон.
====
Даны две точки \(P(2;3)\) и \(Q( - 1;0)\). Составить уравнение прямой, проходящей через точку \(Q\) перпендикулярно к отрезку \(\overline{PQ}\).
====
Даны вершины треугольника \(M_{1}(2;1),M_{2}( - 1; - 1)\) и \(M_{3}(3;2)\). Составить уравнения его высот.
====
Найти проекцию точки \(P( - 8;12)\) на прямую, проходящую через точки \(A(2; - 3)\) и \(B( - 5;1)\).
====
Определить, при каких значениях \(m\) и \(n\) две прямые \(mx + 8y + n = 0,\ \ \ \ 2x + my - 1 = 0\) 1) параллельны; 2) совпадают; 3) перпендикулярны.
====
Определить, при каком значении \(m\) две прямые \((m - 1)x + my - 5 = 0,\ \ \ \ mx + (2m - 1)y + 7 = 0\) пересекаются в точке, лежащей на ось абсцисс.
====
Определить, при каком значении \(m\) две прямые \(\begin{matrix}
mx + (2m + 3)y + m + 6 = 0 \\
(2m + 1)x + (m - 1)y + m - 2 = 0
\end{matrix}\)пересекаются в точке, лежащей на оси ординат.
====
Точка \(A(2; - 5)\) является вершиной квадрата, одна из сторон которого лежит на прямой \(x - 2y - 7 = 0\). Вычислить площадь этого квадрата.
====
Даны уравнения двух сторон прямоугольника \(3x - 2y - 5 = 0,2x + 3y + 7 = 0\) и одна из его вершин \(A( - 2;1)\). Вычислить площадь этого прямоугольника.
====
Отклонения точки \(M\) от прямых \(5x - 12y - 13 = 0\) и \(3x - 4y - 19 = 0\) равны соответственно - 3 и -5. Определить координаты точки \(M\).
====
Составить уравнение прямой, проходящей через точку \(P( - 2;3)\) на одинаковых расстояниях от точек \(A(5\); \(- 1)\) и \(B(3;7)\).
====
Определить, лежат ли точки \(M(2;3)\) и \(N(5; - 1)\) в одном, в смежных или вертикальных углах, образованных при пересечении двух прямых: 1) \(x - 3y - 5 = 0,2x + 9y - 2 = 0\); 2) \(2x + 7y - 5 = 0,x + 3y + 7 = 0\); 3) \(12x + y - 1 = 0,\ \ \ \ 13x + 2y - 5 = 0\).
====
Определить, лежит ли начало координат внутри или вне треугольника, стороны которого даны уравнениями \(7x - 5y - 11 = 0,\ \ \ \ 8x + 3y + 31 = 0,\ \ \ \ x + 8y - 19 = 0\).
====
Определить, лежит ли точка \(M( - 3;2)\) внутри или вне треугольника, стороны которого даны уравнениями \(x + y - 4 = 0,\ \ \ \ 3x - 7y + 8 = 0,\ \ \ \ 4x - y - 31 = 0\).
====
Найти уравнение прямой, принадлежащей пучку прямых \(\ \ \ \ \alpha(x + 2y - 5) + \beta(3x - 2y + 1) = 0\) и 1) проходящей через точку \(A(3; - 1)\); 2) проходящей через начало координат; 3) параллельной оси \(Ox\); 4) параллельной оси \(Oy\); 5) параллельной прямой \(4x + 3y + 5 = 0\); 6) перпендикулярной к прямой \(2x + 3y + 7 = 0\).
====
Составить уравнение прямой, которая проходит через точку пересечения прямых \(2x + y - 2 = 0,x - 5y - 23 = 0\) и делит пополам отрезок, ограниченный точками \(M_{1}(5; - 6)\) и \(M_{2}( - 1; - 4)\). Решить задачу, не вычисляя координат точки пересечения данных прямых.
====
Составить уравнение прямой, проходящей через точку пересечения прямых \(2x + 7y - 8 = 0,3x + 2y + 5 = 0\) под углом \(45^{{^\circ}}\) к прямой \(2x + 3y - 7 = 0\). Решить задачу, не вычисляя координат точки пересечения данных прямых.
====
Центр пучка прямых \(\alpha(2x - 3y + 20) + \beta(3x + 5y - 27) = 0\) является вершиной квадрата, диагональ которого лежит на прямой \(x + 7y - 16 = 0\). Составить уравнения сторон и второй диагонали этого квадрата.
====
Дано уравнение пучка прямых \(\alpha(2x + y + 1) + \beta(x - 3y - 10) = 0\) . Найти прямые этого пучка, отсекающие на координатных осях отрезки равной длины (считая от начала координат).
====
Даны две вершины \(A(2; - 3; - 5),B( - 1;3;2)\) параллелограмма \(ABCD\) и точка пересечения его диагоналей \(E(4; - 1;7)\). Определить две другие вершины этого параллелограмма.
====
Даны вершины треугольника \(A(1;2; - 1),B(2\); \(- 1;3)\) и \(C( - 4;7;5)\). Вычислить длину биссектрисы его внутреннего угла при вершине \(B\).
====
Даны вершины треугольника \(A(1; - 1; - 3)\), \(B(2;1; - 2)\) и \(C( - 5;2; - 6)\). Вычислить длину биссектрисы его внутреннего угла при вершине \(A\).