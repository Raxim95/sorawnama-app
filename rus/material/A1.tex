Даны концы \(A(3; - 5)\) и \(B( - 1;1)\) однородного стержня. Определить координаты его центра масс.
====
Центр масс однородного стержня находится в точке \(M(1;4)\), один из его концов в точке \(P( - 2;2)\). Определить координаты точки \(Q\) - другого конца этого стержня.
====
Даны вершины треугольника \(A(1; - 3),B(3; - 5)\) и \(C( - 5;7)\). Определить середины его сторон.
====
Даны точки \(A(3; - 1)\) и \(B(2;1)\). Определить: координаты точки \(M\), симметричной точке \(A\) относительно точки \(B\); координаты точки \(N\), симметричной точке \(B\) относительно точки \(A\).
====
Точки \(M(2; - 1),N( - 1;4)\) и \(P( - 2;2)\) являются серединами сторон треугольника. Определить его вершины.
====
Вычислить площадь треугольника, вершинами которого являются точки: 1) \(A(2; - 3),B(3;2)\) и \(C( - 2;5)\); 2) \(M_{1}( - 3;2),M_{2}(5; - 2)\) и \(\left. \ M_{3}(1;3);3 \right)M(3; - 4),N( - 2;3)\) н \(P(4;5)\).
====
Определить площадь параллелограмма, три вершины которого суть точки \(A( - 2;3),B(4; - 5)\) и \(C( - 3;1)\). (С помощью площадью треугольника)
====
Даны последовательные вершин однородной четырехугольной пластинки \(A(2;1),B(5;3),C( - 1;7)\) и \(D( - 7;5)\). Определить координат ее центра масс.
====
Даны последовательные вершины \(A(2;3),B(0;6)\), \(C( - 1;5),D(0;1)\) и \(E(1;1)\) однородной пятиугольной пластинки. Определить координаты ее центра масс.
====
Определить, какие из точек \(M_{1}(3;1),\) \(M_{2}(2;3)\), \(M_{3}(6;3),\) \(M_{4}( - 3; - 3),\) \(M_{5}(3; - 1),\) \(M_{6}( - 2;1)\) лежат на прямой \(2x - 3y - 3 = 0\) и какие не лежат на ней.
====
Точки \(P_{1},P_{2},P_{3},P_{3}\) и \(P_{5}\) расположены на прямой \(3x - 2y - 6 = 0\); их абсциссы соответственно равны числам 4 , \(0,2, - 2\) и -6 . Определить ординаты этих точек.
====
Дана прямая \(5x + 3y - 3 = 0\). Определить угловой коэффициент \(k\) прямой: параллельной данной прямой; перпендикулярно к данной прямой.
====
Дана прямая \(2x + 3y + 4 = 0\). Составить уравнение прямой, проходящей через точку \(M_{0}(2;1)\): 1) параллельно данной прямой; 2) перпендикулярно к данной прямой.
====
Вычислить угловой коэффициент \(k\) прямой, проходящей через две данные точки: а) \(M_{1}(2; - 5),M_{2}(3;2)\); б) \(P( - 3;1),Q(7;8)\); в) \(A(5; - 3),B( - 1;6)\).
====
Определить угол \(\varphi\) между двумя прямыми: 1) \(5x - y + 7 = 0,\ \ \ \ 3x + 2y = 0\); 2) \(3x - 2y + 7 = 0,2x + 3y - 3 = 0\); 3) \(x - 2y - 4 = 0,\ \ \ \ 2x - 4y + 3 = 0\);
====
Установить, какие из следующих пар прямых перпендикулярны: 1) \(3x - y + 5 = 0,x + 3y - 1 = 0\); 2) \(3x - 4y + 1 = 0,\ \ \ \ 4x - 3y + 7 = 0\); 3) \(6x - 15y + 7 = 0,\ \ \ \ 10x + 4y - 3 = 0\); 4) \(9x - 12y + 5 = 0,\ \ \ \ 8x + 6y - 13 = 0\); 5) \(7x - 2y + 1 = 0,4x + 6y + 17 = 0\); Решить задачу, не вычисляя угловых коэффициентов данных прямых.
====
Определить угол \(\varphi\), образованный двумя прямыми: 1) \(3x - y + 5 = 0,2x + y - 7 = 0\); 2) \(x\sqrt{2} - y\sqrt{3} - 5 = 0\), \((3 + \sqrt{2})x + (\sqrt{6} - \sqrt{3})y + 7 = 0\); \(3)\) \(x\sqrt{3} + y\sqrt{2} - 2 = 0,x\sqrt{6} - 3y + 3 = 0\). Решить задачу, не вычисляя угловых коэффициентов данных прямых.
====
Определить, при каком значении \(a\) прямая \((a + 2)x + \left( a^{2} - 9 \right)y + 3a^{2} - 8a + 5 = 0\) 1) параллельна оси абсцисс; 2) параллельна оси ординат; 3) проходит через начало координат. В каждом случае написать уравнение прямой.
====
Определить, при каких значениях \(m\) и \(n\) прямая \((m + 2n - 3)x + (2m - n + 1)y + 6m + 9 = 0\) параллельна оси абсцисс и отсекает на оси ординат отрезок, равный -3 (считая от начала координат). Написать уравнение этой прямой.
====
Определить, при каких значениях \(a\) и \(b\) две прямые \(ax - 2y - 1 = 0,\ \ \ \ 6x - 4y - b = 0\) 1) имеют одну общую точку; 2) параллельны; 3) совпадают.
====
Привести общее уравнение прямой к нормальному виду в каждом из следующих случаев: 1) \(4x - 3y - 10 = 0\); 2) \(\frac{4}{5}x - \frac{3}{5}y + 10 = 0\); 3) \(12x - 5y + 13 = 0\); 4) \(x + 2 = 0\); 5) \(2x - y - \sqrt{5} = 0\).
====
Вычислить величину отклонения \(\delta\) и расстояние \(d\) от точки до прямой в каждом из следующих случаев: 1) \(A(2; - 1),4x + 3y + 10 = 0;\) \(2)B(0; - 3),5x - 12y - 23 = 0;\); \(3)P( - 2;3),\ \ 3x - 4y - 2 = 0\); 4) \(Q(1; - 2),x - 2y - 5 = 0\).
====
Установить, лежат ли точка \(M(1; - 3)\) и начало координат по одну или по разные стороны каждой из следующих прямых: 1) \(2x - y + 5 = 0\); 2) \(x - 3y - 5 = 0\); 3) \(3x +\) \(+ 2y - 1 = 0\); 4) \(x - 3y + 2 = 0\); 5) \(10x + 24y + 15 = 0\).
====
Составить уравнение геометрического места точек, равноудаленных от двух параллельных прямых: 1) \(3x - y + 7 = 0,\ \ \ \ 3x - y - 3 = 0\); 2) \(x - 2y + 3 = 0,\ \ \ \ x - 2y + 7 = 0\); 3) \(5x - 2y - 6 = 0,\ \ \ \ 10x - 4y + 3 = 0\).
====
Дано уравнение пучка прямых \(\alpha(3x + y - 1) + \beta(2x - y - 9) = 0\) . Доказать, что прямая \(x + 3y + 13 = 0\) при надлежит этому пучку.
====
Дано уравнение пучка прямых \(\alpha(3x + 2y - 9) +\) \(+ \beta(2x + 5y + 5) = 0\). Найти, при каком значении \(C\) прямая \(4x - 3y + C = 0\) будет принадлежать этому пучку.
====
Даны точки \(A(1; - 2; - 3),B(2; - 3;0),C(3;1\); \(- 9),D( - 1;1; - 12)\). Вычислить расстояние между: 1) \(A\) и \(C\); 2) \(B\) и \(D\); 3\()C\) и \(D\).
====
Доказать, что треугольник с вершинами \(A(3; - 1;2)\), \(B(0; - 4;2)\) и \(C( - 3;2;1)\) равнобедренный.
====
Даны вершины \(M_{1}(3;2; - 5),M_{2}(1; - 4;3)\) и \(M_{3}( - 3\); \(0;1)\) треугольника. Найти середины его сторон.
====
Даны вершины \(A(2; - 1;4),B(3;2; - 6),C( - 5\); 0 ; 2) треугольника. Вычислить длину его медианы, проведенной из вершины \(A\).