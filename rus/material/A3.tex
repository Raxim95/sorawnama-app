Составить уравнение плоскости, которая проходит через точку \(M_{1}(2;1; - 1)\) и имеет нормальный вектор \(\overrightarrow{n} = \{ 1; - 2;3\}\).
====
Точка \(P(2; - 1; - 1)\) служит основанием перпендикуляра, опущенного из начала координат на плоскость. Составить уравнение этой плоскости.
====
Установить, какие из следующих пар уравнений определяют параллельные плоскости; 1) \(2x - 3y + 5z - 7 = 0,\ \ \ \ 2x - 3y + 5z + 3 = 0\); 2) \(4x + 2y - 4z + 5 = 0,\ \ \ \ 2x + y + 2z - 1 = 0\); 3) \(\ \ \ \ x - 3z + 2 = 0,\ \ \ \ 2x - 6z - 7 = 0\).
====
Определить, при каком значении \(l\) следующие парь уравнений будут определять перпендикулярные плоскости: 1) \(3x - 5y + lz - 3 = 0,x + 3y + 2z + 5 = 0\); 2) \(5x + y - 3z - 3 = 0,2x + ly - 3z + 1 = 0\); 3) \(\ \ \ \ 7x - 2y - z = 0,\ \ \ \ lx + y - 3z - 1 = 0\).
====
Составить уравнение плоскости, которая проходит: 1) через точки \(M_{1}(7;2; - 3)\) и \(M_{2}(5;6; - 4)\) параллельно оси \(Ox\); 2) через точки \(P_{1}(2; - 1;1)\) и \(P_{2}(3;1;2)\) параллельно оси \(Oy\); 3) через точки \(Q_{1}(3; - 2;5)\) и \(Q_{2}(2;3;1)\) параллельно оси \(Oz\).
====
Привести каждое из следующих уравнений плоскостей к нормальному виду: 1) \(2x - 2y + z - 18 = 0\); 2) \(\frac{3}{7}x - \frac{6}{7}y + \frac{2}{7}z + 3 = 0\); 3) \(4x - 6y - 12z - 11 = 0\); 4) \(- 4x - 4y + 2z + 1 = 0\); 5) \(5y - 12z + 26 = 0\);
====
Вычислить величину отклонения \(\delta\) и расстояние \(d\) от точки до плоскости в каждом из следующих случаев: 1) \(M_{1}( - 2; - 4;3)\), \(2x - y + 2z + 3 = 0;\) 2) \(M_{2}(2; - 1; - 1),16x - 12y + 15z - 4 = 0\); 3) \(M_{3}(1;2; - 3),\ \ \ \ 5x - 3y + z + 4 = 0\);
====
Составить канонические уравнения прямой, проходящей через точку \(M_{1}(2;0; - 3)\) параллельно: 1) вектору \(\overrightarrow{a} = \{ 2; - 3;5\}\); 2) прямой \(\frac{x - 1}{5} = \frac{y + 2}{2} = \frac{z + 1}{- 1}\); 3) оси \(Ox\); 4) оси \(Oy\); 5) оси \(Oz\).
====
Составить канонические уравнения прямой, проходящей через данные точки: 1) \((1; - 2;1),(3;1; - 1)\); 2) \((3; - 1;0),(1;0, - 3);\) \(3)(0; - 2;3),(3; - 2;1)\); 4) \((1;2; - 4),( - 1;2; - 4)\).
====
Составить параметрические уравнения прямой, проходящей через точку \(M_{1}(1; - 1; - 3)\) параллельно: 1) вектору \(\overrightarrow{a} = \{ 2; - 3;4\}\); 2) прямой \(\frac{x - 1}{2} = \frac{y + 2}{4} = \frac{z - 1}{0}\); 3) прямой \(x = 3t - 1,y = - 2t + 3,z = 5t + 2\).
====
Доказать перпендикулярность прямых: 1) \(\frac{x}{1} = \frac{y - 1}{- 2} = \frac{z}{3}\) и \(3x + y - 5z + 1 = 0,2x + 3y - 8z + 3 = 0\); 2) \(x = 2t + 1,y = 3t - 2,z = - 6t + 1\) и \(2x + y - 4z + 2 = 0\), \(4x - y - 5z + 4 = 0\); 3) \(x + y - 3z - 1 = 0,\ \ \ \ 2x - y - 9z - 2 = 0\ \ \ \ \) и \(2x + y +\) \(+ 2z + 5 = 0,2x - 2y - z + 2 = 0\)
====
Доказать, что прямая \(x = 3t - 2,y = - 4t + 1\), \(z = 4t - 5\) параллельна плоскости \(4x - 3y - 6z - 5 = 0\).
====
Найти точку пересечения прямой и плоскости: 1) \(\frac{x - 1}{1} = \frac{y + 1}{- 2} = \frac{z}{6},\ \ \ \ 2x + 3y + z - 1 = 0\); 2) \(\frac{x + 3}{3} = \frac{y - 2}{- 1} = \frac{z + 1}{- 5},\ \ \ \ x - 2y + z - 15 = 0\); 3) \(\frac{x + 2}{- 2} = \frac{y - 1}{3} = \frac{z - 3}{2},\ \ \ \ x + 2y - 2z + 6 = 0\).
====
При каком значении \(m\) прямая \(\frac{x + 1}{3} = \frac{y - 2}{m} = \frac{z + 3}{- 2}\) параллельна плоскости \(x - 3y + 6z + 7 = 0\) ?
====
Вычислить расстояние \(d\) от точки \(P(2;3; - 1)\) до следующих прямых: 1) \(\frac{x - 5}{3} = \frac{y}{2} = \frac{z + 25}{- 2}\); 2) \(x = t + 1,y = t + 2,z = 4t + 13\).
====
В каждом из следующих случаев вычислить расстояние между параллельными плоскостями: 1) \(x - 2y - 2z - 12 = 0,x - 2y - 2z - 6 = 0\); 2) \(2x - 3y + 6z - 14 = 0,4x - 6y + 12z + 21 = 0\); 3) \(2x - y + 2z + 9 = 0,4x - 2y + 4z - 21 = 0\); 4) \(16x + 12y - 15z + 50 = 0,\ \ \ \ 16x + 12y - 15z + 25 = 0\);
====
Две грани куба лежат на плоскостях \(2x - 2y + z - 1 = 0,\) \(2x - 2y + z + 5 = 0\). Вычислить объем этого куба.
