Даны единичнье векторы \(\overrightarrow{a},\ \overrightarrow{b}\) и \(\overrightarrow{c}\), удовлетворяющие условию \(\overrightarrow{a} + \overrightarrow{b} + \overrightarrow{c} = 0\). Вычислить \(\left( \overrightarrow{a},\overrightarrow{b} \right) + \left( \overrightarrow{b},\overrightarrow{c} \right) + \left( \overrightarrow{c},\overrightarrow{a} \right)\).
====
Какому условия должны удовлетворять векторы \(\overrightarrow{a}\) и \(\overrightarrow{b}\), чтобы вектор \(\overrightarrow{a} + \overrightarrow{b}\) был перпендикулярен к вектору \(\overrightarrow{a} - \overrightarrow{b}\).
====
Доказать, что вектор \(\overrightarrow{p} = \overrightarrow{b}(\overrightarrow{a},\overrightarrow{c}) - \overrightarrow{c}(\overrightarrow{a},\overrightarrow{b})\) перпендикулярен к вектору \(\overrightarrow{a}\).
====
Доказать, что вектор \(\overrightarrow{p} = \overrightarrow{b} - \frac{\overrightarrow{a}(\overrightarrow{a},\overrightarrow{b})}{{\overrightarrow{a}}^{2}}\) перпендикулярен к вектору \(\overrightarrow{a}\).
====
Даны векторы \(\overrightarrow{AB} = \overrightarrow{b}\) и \(\overrightarrow{AC} = \overrightarrow{c}\), совпадающие со сторонами треугольника \(ABC\). Найти разложение вектора, приложенного к вершине \(B\) этого треугольника и совпадающего с его высотой \(BD\) по базису \(\overrightarrow{b},\ \overrightarrow{c}\).
====
Какому условию должны удовлетворять векторы \(\overrightarrow{a},\overrightarrow{b}\), чтобы векторы \(\overrightarrow{a} + \overrightarrow{b}\) и \(\overrightarrow{a} - \overrightarrow{b}\) были коллинеарны?
====
Доказать тождество \(\lbrack\overrightarrow{a},\overrightarrow{b}\rbrack^{2} + (\overrightarrow{a},\overrightarrow{b})^{2} = {\overrightarrow{a}}^{2}{\overrightarrow{b}}^{2}\).
====
Доказать, что \(\lbrack\overrightarrow{a},\overrightarrow{b}\rbrack^{2} <  {\overrightarrow{a}}^{2}{\overrightarrow{b}}^{2}\); в каком случае здесь будет знак равенства?
====
Векторы \(\overrightarrow{a},\ \overrightarrow{b}\) и \(\overrightarrow{c}\) удовлетворяют условию \(\overrightarrow{a} + \overrightarrow{b} + \overrightarrow{c} = 0\). Доказать, что \(\lbrack\overrightarrow{a},\overrightarrow{b}\rbrack = \lbrack\overrightarrow{b},\overrightarrow{c}\rbrack = \lbrack\overrightarrow{c},\overrightarrow{a}\rbrack\)
====
Векторы \(\overrightarrow{a},\ \overrightarrow{b},\ \overrightarrow{c}\) и \(\overrightarrow{d}\) связаны соотношениями \(\lbrack\overrightarrow{a},\overrightarrow{b}\rbrack = \lbrack\overrightarrow{c},\overrightarrow{d}\rbrack,\ \ \lbrack\overrightarrow{a},\overrightarrow{c}\rbrack = \lbrack\overrightarrow{b},\overrightarrow{d}\rbrack\). Доказать коллинеарность векторов \(\overrightarrow{a} - \overrightarrow{d}\) и \(\overrightarrow{b} - \overrightarrow{c}\).
====
Доказать, что \(|(\lbrack\overrightarrow{a},b\rbrack,c)| <  |\overrightarrow{a}||\overrightarrow{b}||\overrightarrow{c}|;\) в каком случае здесь может иметь место знак равенства?
====
Доказать тождество \((\lbrack\overrightarrow{a} + \overrightarrow{b},\overrightarrow{b} + \overrightarrow{c}\rbrack,\overrightarrow{c} + \overrightarrow{a}) = 2(\lbrack\overrightarrow{a},\overrightarrow{b}\rbrack,\overrightarrow{c})\).
====
Доказать тождество \((\lbrack\overrightarrow{a},\overrightarrow{b}\rbrack,\overrightarrow{c} + \lambda\overrightarrow{a} + \mu\overrightarrow{b}) = (\lbrack\overrightarrow{a},\overrightarrow{b}\rbrack,\overrightarrow{c})\), где \(\lambda\) и \(\mu\)-какие угодно числа.
====
Доказать, что векторы \(\overrightarrow{a},\ \overrightarrow{b},\ \overrightarrow{c}\) удовлетворяющие условию \(\lbrack\overrightarrow{a},\overrightarrow{b}\rbrack + \lbrack\overrightarrow{b},\overrightarrow{c}\rbrack + \lbrack\overrightarrow{c},\overrightarrow{a}\rbrack = 0\), компланарны.
====
Доказать, что необходимым и достаточным условием компланарности векторов \(\overrightarrow{a},\ \overrightarrow{b},\ \overrightarrow{c}\) является зависимость \(\alpha\overrightarrow{a} + \beta\overrightarrow{b} + \gamma\overrightarrow{c} = 0\), где по крайней мере одно из чисел \(\alpha,\beta,\gamma\) не равно нулю.
====
Даны единичнье векторы \(\overrightarrow{a},\ \overrightarrow{b}\) и \(\overrightarrow{c}\), удовлетворяющие условию \(\overrightarrow{a} + \overrightarrow{b} + \overrightarrow{c} = 0\). Вычислить \(\left( \overrightarrow{a},\overrightarrow{b} \right) + \left( \overrightarrow{b},\overrightarrow{c} \right) + \left( \overrightarrow{c},\overrightarrow{a} \right)\).
====
Какому условия должны удовлетворять векторы \(\overrightarrow{a}\) и \(\overrightarrow{b}\), чтобы вектор \(\overrightarrow{a} + \overrightarrow{b}\) был перпендикулярен к вектору \(\overrightarrow{a} - \overrightarrow{b}\).
====
Доказать, что вектор \(\overrightarrow{p} = \overrightarrow{b}(\overrightarrow{a},\overrightarrow{c}) - \overrightarrow{c}(\overrightarrow{a},\overrightarrow{b})\) перпендикулярен к вектору \(\overrightarrow{a}\).
====
Доказать, что вектор \(\overrightarrow{p} = \overrightarrow{b} - \frac{\overrightarrow{a}(\overrightarrow{a},\overrightarrow{b})}{{\overrightarrow{a}}^{2}}\) перпендикулярен к вектору \(\overrightarrow{a}\).
====
Даны векторы \(\overrightarrow{AB} = \overrightarrow{b}\) и \(\overrightarrow{AC} = \overrightarrow{c}\), совпадающие со сторонами треугольника \(ABC\). Найти разложение вектора, приложенного к вершине \(B\) этого треугольника и совпадающего с его высотой \(BD\) по базису \(\overrightarrow{b},\ \overrightarrow{c}\).
====
Какому условию должны удовлетворять векторы \(\overrightarrow{a},\overrightarrow{b}\), чтобы векторы \(\overrightarrow{a} + \overrightarrow{b}\) и \(\overrightarrow{a} - \overrightarrow{b}\) были коллинеарны?
====
Доказать тождество \(\lbrack\overrightarrow{a},\overrightarrow{b}\rbrack^{2} + (\overrightarrow{a},\overrightarrow{b})^{2} = {\overrightarrow{a}}^{2}{\overrightarrow{b}}^{2}\).
====
Доказать, что \(\lbrack\overrightarrow{a},\overrightarrow{b}\rbrack^{2} <  {\overrightarrow{a}}^{2}{\overrightarrow{b}}^{2}\); в каком случае здесь будет знак равенства?
====
Векторы \(\overrightarrow{a},\ \overrightarrow{b}\) и \(\overrightarrow{c}\) удовлетворяют условию \(\overrightarrow{a} + \overrightarrow{b} + \overrightarrow{c} = 0\). Доказать, что \(\lbrack\overrightarrow{a},\overrightarrow{b}\rbrack = \lbrack\overrightarrow{b},\overrightarrow{c}\rbrack = \lbrack\overrightarrow{c},\overrightarrow{a}\rbrack\)
====
Векторы \(\overrightarrow{a},\ \overrightarrow{b},\ \overrightarrow{c}\) и \(\overrightarrow{d}\) связаны соотношениями \(\lbrack\overrightarrow{a},\overrightarrow{b}\rbrack = \lbrack\overrightarrow{c},\overrightarrow{d}\rbrack,\ \ \lbrack\overrightarrow{a},\overrightarrow{c}\rbrack = \lbrack\overrightarrow{b},\overrightarrow{d}\rbrack\). Доказать коллинеарность векторов \(\overrightarrow{a} - \overrightarrow{d}\) и \(\overrightarrow{b} - \overrightarrow{c}\).
====
Доказать, что \(|(\lbrack\overrightarrow{a},b\rbrack,c)| <  |\overrightarrow{a}||\overrightarrow{b}||\overrightarrow{c}|;\) в каком случае здесь может иметь место знак равенства?
====
Доказать тождество \((\lbrack\overrightarrow{a} + \overrightarrow{b},\overrightarrow{b} + \overrightarrow{c}\rbrack,\overrightarrow{c} + \overrightarrow{a}) = 2(\lbrack\overrightarrow{a},\overrightarrow{b}\rbrack,\overrightarrow{c})\).
====
Доказать тождество \((\lbrack\overrightarrow{a},\overrightarrow{b}\rbrack,\overrightarrow{c} + \lambda\overrightarrow{a} + \mu\overrightarrow{b}) = (\lbrack\overrightarrow{a},\overrightarrow{b}\rbrack,\overrightarrow{c})\), где \(\lambda\) и \(\mu\)-какие угодно числа.
====
Доказать, что векторы \(\overrightarrow{a},\ \overrightarrow{b},\ \overrightarrow{c}\) удовлетворяющие условию \(\lbrack\overrightarrow{a},\overrightarrow{b}\rbrack + \lbrack\overrightarrow{b},\overrightarrow{c}\rbrack + \lbrack\overrightarrow{c},\overrightarrow{a}\rbrack = 0\), компланарны.
====
Доказать, что необходимым и достаточным условием компланарности векторов \(\overrightarrow{a},\ \overrightarrow{b},\ \overrightarrow{c}\) является зависимость \(\alpha\overrightarrow{a} + \beta\overrightarrow{b} + \gamma\overrightarrow{c} = 0\), где по крайней мере одно из чисел \(\alpha,\beta,\gamma\) не равно нулю.