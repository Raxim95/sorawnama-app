На плоскости даны три вектора \(\overrightarrow{a} = \{ 3; - 2\}\), \(\overrightarrow{b} = \{ - 2;1\}\) и \(\overrightarrow{c} = \{ 7; - 4\}\). Определить разложение каждого из этих трех векторов, принимая в качестве базиса два других.
====
Векторы \(\overrightarrow{a}\) и \(\overrightarrow{b}\) образуют угол \(\varphi = 2\pi/3\); зная, что \(|\overrightarrow{a}| = 3,|\overrightarrow{b}| = 4\), вычислить: 1) \(\left( \overrightarrow{a},\overrightarrow{b} \right)\); 2) \({\overrightarrow{a}}^{2}\); 3) \({\overrightarrow{b}}^{2}\); 4) \((\overrightarrow{a} + \overrightarrow{b})^{2}\); 5) \(\left( 3\overrightarrow{a} - 2\overrightarrow{b},\overrightarrow{a} + 2\overrightarrow{b} \right);\ 6)(\overrightarrow{a} - \overrightarrow{b})^{2};7)(3\overrightarrow{a} + 2\overrightarrow{b})^{2}\).
====
Векторы \(\overrightarrow{a}\) и \(\overrightarrow{b}\) взаимно перпендикулярны; вектор \(\overrightarrow{c}\) образует с ними углы, равные \(\pi/3\); зная, что \(|\overrightarrow{a}| = 3\), \(|\overrightarrow{b}| = 5,\ \ \ \ |\overrightarrow{c}| = 8\), вычислить: 1) \(\left( 3\overrightarrow{a} - 2\overrightarrow{b},\overrightarrow{b} + 3\overrightarrow{c} \right)\); 2) \((\overrightarrow{a} + \overrightarrow{b} + \overrightarrow{c})^{2};\) \(3)(\overrightarrow{a} + 2\overrightarrow{b} - 3\overrightarrow{c})^{2}\).
====
Даны векторы \(\overrightarrow{a},\ \overrightarrow{b}\) и \(\overrightarrow{c}\) удовлетворяющие условию \(\overrightarrow{a} + \overrightarrow{b} + \overrightarrow{c} = 0\). Зная, что \(|\overrightarrow{a}| = 3,\ \ \ \ |\overrightarrow{b}| = 1\) и \(|\overrightarrow{c}| = 4\), вычислить \(\left( \overrightarrow{a},\overrightarrow{b} \right) + \left( \overrightarrow{b},\overrightarrow{c} \right) + \left( \overrightarrow{c},\overrightarrow{a} \right)\).
====
Дано, что \(|\overrightarrow{a}| = 3,|\overrightarrow{b}| = 5\). Определить, при каком значении \(\alpha\) векторы \(\overrightarrow{a} + \alpha\overrightarrow{b},\overrightarrow{a} - \alpha\overrightarrow{b}\) будут взаимно перпендикулярны.
====
Векторы \(a\) и \(b\) образует угол \(\varphi = \pi/6\); зная, что \(|\mathbf{a}| = \sqrt{3},|\mathbf{b}| = 1\), вычислить угол \(\alpha\) между векторами \(p = a + b\ \ \ \ \) и \(\ \ \ \ q = a - b\).
====
Вектор \(\overrightarrow{x}\), коллинеарный вектору \(\overrightarrow{a} = \{ 6; - 8; - 7,5\}\), образует острый угол с осью Oz. Зная, что \(|\overrightarrow{x}| = 50\), найти его координаты.
====
Найти вектор \(\overrightarrow{x}\), коллинеарный вектору \(\overrightarrow{a} = \{ 2;1; - 1\}\) и удовлетворяющий условию \(\left( \overrightarrow{x},\overrightarrow{a} \right) = 3\).
====
Векторы \(\overrightarrow{a}\) и \(\overrightarrow{b}\) взаимно перпендикулярны. Зная, что \(|\overrightarrow{a}| = 3,|\overrightarrow{b}| = 4\), вычислить \(1)|\lbrack\overrightarrow{a} + \overrightarrow{b},\overrightarrow{a} - \overrightarrow{b}\rbrack|;\ \ 2)|\lbrack 3\overrightarrow{a} - \overrightarrow{b},\overrightarrow{a} - 2\overrightarrow{b}\rbrack|\).
====
Векторы \(\overrightarrow{a}\) и \(\overrightarrow{b}\) образуют угол \(\varphi = 2\pi/3\). Зная, что \(|\overrightarrow{a}| = 1,|\overrightarrow{b}| = 2\), вычислить: \(1)\left. \ \lbrack\overrightarrow{a},\overrightarrow{b}\rbrack^{2};\ \ \ 2 \right)\lbrack 2\overrightarrow{a} + \overrightarrow{b},\overrightarrow{a} + 2\overrightarrow{b}\rbrack^{2};\ \ \ \ \) 3) \(\lbrack\overrightarrow{a} + 3\overrightarrow{b},3\overrightarrow{a} - \overrightarrow{b}\rbrack^{2}\).
====
Даны векторы \(\overrightarrow{a} = \{ 3; - 1; - 2\}\) и \(\overrightarrow{b} = \{ 1;2; - 1\}\), Найти координаты векторных произведений: 1) \(\left\lbrack \overrightarrow{a},\overrightarrow{b} \right\rbrack\); 2)\(\left\lbrack 2\overrightarrow{a} + \overrightarrow{b},\overrightarrow{b} \right\rbrack\); 3) \(\left\lbrack 2\overrightarrow{a} - \overrightarrow{b},2\overrightarrow{a} + \overrightarrow{b} \right\rbrack\).
====
Даны точки \(A(2; - 1;2),B(1;2; - 1)\) и \(C(3;2;1)\). Найти координаты векторных пронзведений: 1) \(\lbrack\overline{AB},\overline{BC}\rbrack\); 2) \(\lbrack\overline{BC} - 2\overline{CA},\overline{CB}\rbrack\).
====
Векторы \(\overrightarrow{a},\ \overrightarrow{b},\ \overrightarrow{c}\) образующие правую тройку, взаимно перпендикулярны. Зная, что \(|\overrightarrow{a}| = 4,\ \ |\overrightarrow{b}| = 2\), \(|\overrightarrow{c}| = 3\), вычислить \(\left( \left\lbrack \overrightarrow{a},\overrightarrow{b} \right\rbrack,\overrightarrow{c} \right)\).
====
Вектор \(\overrightarrow{c}\) перпендикулярен к векторам \(\overrightarrow{a}\) и \(\overrightarrow{b}\), угол между \(\overrightarrow{a}\) и \(\overrightarrow{b}\) равен \(30^{{^\circ}}\). Зная, что \(|\overrightarrow{a}| = 6,|\overrightarrow{b}| = 3\), \(|\overrightarrow{c}| = 3\), вычислить \(\left( \left\lbrack \overrightarrow{a},\overrightarrow{b} \right\rbrack,\overrightarrow{c} \right)\).
====
Доказать, что точки \(A(1;2; - 1),B(0;1;5)\), \(C( - 1;2;1),D(2;1;3)\) лежат в одной плоскости.
====
Вычислить объем тетраэдра, вершины которого находятся в точках \(A(2; - 1;1),B(5;5;4),C(3;2; - 1)\) и \(D(4;1;3)\).
====
На плоскости даны три вектора \(\overrightarrow{a} = \{ 3; - 2\}\), \(\overrightarrow{b} = \{ - 2;1\}\) и \(\overrightarrow{c} = \{ 7; - 4\}\). Определить разложение каждого из этих трех векторов, принимая в качестве базиса два других.
====
Векторы \(\overrightarrow{a}\) и \(\overrightarrow{b}\) образуют угол \(\varphi = 2\pi/3\); зная, что \(|\overrightarrow{a}| = 3,|\overrightarrow{b}| = 4\), вычислить: 1) \(\left( \overrightarrow{a},\overrightarrow{b} \right)\); 2) \({\overrightarrow{a}}^{2}\); 3) \({\overrightarrow{b}}^{2}\); 4) \((\overrightarrow{a} + \overrightarrow{b})^{2}\); 5) \(\left( 3\overrightarrow{a} - 2\overrightarrow{b},\overrightarrow{a} + 2\overrightarrow{b} \right);\ 6)(\overrightarrow{a} - \overrightarrow{b})^{2};7)(3\overrightarrow{a} + 2\overrightarrow{b})^{2}\).
====
Векторы \(\overrightarrow{a}\) и \(\overrightarrow{b}\) взаимно перпендикулярны; вектор \(\overrightarrow{c}\) образует с ними углы, равные \(\pi/3\); зная, что \(|\overrightarrow{a}| = 3\), \(|\overrightarrow{b}| = 5,\ \ \ \ |\overrightarrow{c}| = 8\), вычислить: 1) \(\left( 3\overrightarrow{a} - 2\overrightarrow{b},\overrightarrow{b} + 3\overrightarrow{c} \right)\); 2) \((\overrightarrow{a} + \overrightarrow{b} + \overrightarrow{c})^{2};\) \(3)(\overrightarrow{a} + 2\overrightarrow{b} - 3\overrightarrow{c})^{2}\).
====
Даны векторы \(\overrightarrow{a},\ \overrightarrow{b}\) и \(\overrightarrow{c}\) удовлетворяющие условию \(\overrightarrow{a} + \overrightarrow{b} + \overrightarrow{c} = 0\). Зная, что \(|\overrightarrow{a}| = 3,\ \ \ \ |\overrightarrow{b}| = 1\) и \(|\overrightarrow{c}| = 4\), вычислить \(\left( \overrightarrow{a},\overrightarrow{b} \right) + \left( \overrightarrow{b},\overrightarrow{c} \right) + \left( \overrightarrow{c},\overrightarrow{a} \right)\).
====
Дано, что \(|\overrightarrow{a}| = 3,|\overrightarrow{b}| = 5\). Определить, при каком значении \(\alpha\) векторы \(\overrightarrow{a} + \alpha\overrightarrow{b},\overrightarrow{a} - \alpha\overrightarrow{b}\) будут взаимно перпендикулярны.
====
Векторы \(a\) и \(b\) образует угол \(\varphi = \pi/6\); зная, что \(|\mathbf{a}| = \sqrt{3},|\mathbf{b}| = 1\), вычислить угол \(\alpha\) между векторами \(p = a + b\ \ \ \ \) и \(\ \ \ \ q = a - b\).
====
Вектор \(\overrightarrow{x}\), коллинеарный вектору \(\overrightarrow{a} = \{ 6; - 8; - 7,5\}\), образует острый угол с осью Oz. Зная, что \(|\overrightarrow{x}| = 50\), найти его координаты.
====
Найти вектор \(\overrightarrow{x}\), коллинеарный вектору \(\overrightarrow{a} = \{ 2;1; - 1\}\) и удовлетворяющий условию \(\left( \overrightarrow{x},\overrightarrow{a} \right) = 3\).
====
Векторы \(\overrightarrow{a}\) и \(\overrightarrow{b}\) взаимно перпендикулярны. Зная, что \(|\overrightarrow{a}| = 3,|\overrightarrow{b}| = 4\), вычислить \(1)|\lbrack\overrightarrow{a} + \overrightarrow{b},\overrightarrow{a} - \overrightarrow{b}\rbrack|;\ \ 2)|\lbrack 3\overrightarrow{a} - \overrightarrow{b},\overrightarrow{a} - 2\overrightarrow{b}\rbrack|\).
====
Векторы \(\overrightarrow{a}\) и \(\overrightarrow{b}\) образуют угол \(\varphi = 2\pi/3\). Зная, что \(|\overrightarrow{a}| = 1,|\overrightarrow{b}| = 2\), вычислить: \(1)\left. \ \lbrack\overrightarrow{a},\overrightarrow{b}\rbrack^{2};\ \ \ 2 \right)\lbrack 2\overrightarrow{a} + \overrightarrow{b},\overrightarrow{a} + 2\overrightarrow{b}\rbrack^{2};\ \ \ \ \) 3) \(\lbrack\overrightarrow{a} + 3\overrightarrow{b},3\overrightarrow{a} - \overrightarrow{b}\rbrack^{2}\).
====
Даны векторы \(\overrightarrow{a} = \{ 3; - 1; - 2\}\) и \(\overrightarrow{b} = \{ 1;2; - 1\}\), Найти координаты векторных произведений: 1) \(\left\lbrack \overrightarrow{a},\overrightarrow{b} \right\rbrack\); 2)\(\left\lbrack 2\overrightarrow{a} + \overrightarrow{b},\overrightarrow{b} \right\rbrack\); 3) \(\left\lbrack 2\overrightarrow{a} - \overrightarrow{b},2\overrightarrow{a} + \overrightarrow{b} \right\rbrack\).
====
Даны точки \(A(2; - 1;2),B(1;2; - 1)\) и \(C(3;2;1)\). Найти координаты векторных пронзведений: 1) \(\lbrack\overline{AB},\overline{BC}\rbrack\); 2) \(\lbrack\overline{BC} - 2\overline{CA},\overline{CB}\rbrack\).
====
Векторы \(\overrightarrow{a},\ \overrightarrow{b},\ \overrightarrow{c}\) образующие правую тройку, взаимно перпендикулярны. Зная, что \(|\overrightarrow{a}| = 4,\ \ |\overrightarrow{b}| = 2\), \(|\overrightarrow{c}| = 3\), вычислить \(\left( \left\lbrack \overrightarrow{a},\overrightarrow{b} \right\rbrack,\overrightarrow{c} \right)\).
====
Вектор \(\overrightarrow{c}\) перпендикулярен к векторам \(\overrightarrow{a}\) и \(\overrightarrow{b}\), угол между \(\overrightarrow{a}\) и \(\overrightarrow{b}\) равен \(30^{{^\circ}}\). Зная, что \(|\overrightarrow{a}| = 6,|\overrightarrow{b}| = 3\), \(|\overrightarrow{c}| = 3\), вычислить \(\left( \left\lbrack \overrightarrow{a},\overrightarrow{b} \right\rbrack,\overrightarrow{c} \right)\).
====
Доказать, что точки \(A(1;2; - 1),B(0;1;5)\), \(C( - 1;2;1),D(2;1;3)\) лежат в одной плоскости.
====
Вычислить объем тетраэдра, вершины которого находятся в точках \(A(2; - 1;1),B(5;5;4),C(3;2; - 1)\) и \(D(4;1;3)\).