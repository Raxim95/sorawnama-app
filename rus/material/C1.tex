Даны вершины треугольника \(A(1;4),B(3; - 9)\) и \(C( - 5;2)\). Определить длину его медианы, проведенной из вершины \(B\). (С помощью деление отрезка в данном отношение)
====
Даны вершины треугольника \(A(2; - 5),B(1; - 2)\) и \(C(4;7)\). Найти точку пересечения биссектрисы его внутреннего угла при вершине \(B\) со стороной \emph{AC}. (С помощью деление отрезка в данном отношение)
====
Даши вершины треугольника \(A(3; - 5),B( - 3;3)\) и \(C( - 1; - 2)\). Определить длину биссектрисы его внутреннего угла при вершине \(A\). (С помощью деление отрезка в данном отношение)
====
Три вершины параллелограмма суть точки \(A(3;7)\), \(B(2; - 3)\) и \(C( - 1;4)\), Вычислить длину его высоты, опущенной из вершины \(B\) на сторону \emph{AC}.
====
Определить координаты точки \(O^{'}\) - нового начала координат, если точка \(A(3; - 4)\) лежит на новой оси абсцисс, а точка \(B(2;3)\) лежит на новой оси ординат, причем оси старой и новой систем координат имеют соответственно одинаковые направления.
====
Написать формулы преобразования координат, если точка \(M_{1}(2; - 3)\) лежит на новой оси абсцисс, а точка \(M_{2}(1; - 7)\) лежит на новой оси ординат, причем оси старой и новой систем координат имеют соответственно одинаковые направления.
====
Даны точки \(M_{1}(9; - 3)\) и \(M_{2}( - 6;5)\). Начало координат перенесено в точку \(M_{1}\), а координатные оси повернуты так, что положительное направление новой оси абсцисс совпадает с направлением отрезка \(\overrightarrow{M_{1}M_{2}}\). Вывести формулы преобразования координат.
====
Стороны треугольника лежат на прямых \(x + 5y - 7 = 0,\) \(3x - 2y - 4 = 0,\) \(7x + y + 19 = 0\). Вычислить его площадь \(S\).
====
Площадь треугольника \(S = 8\), две его вершины суть точки \(A(1; - 2)\) и \(B(2;3)\), а третья вершина \(C\) лежит на прямой \(2x + y - 2 = 0\). Определить координаты вершины \(C\).
====
Даны уравнения двух сторон прямоугольника \(2x - 3y + 5 = 0,3x + 2y - 7 = 0\) и одна из его вершин \(A(2; - 3)\). Составить уравнения двух других сторон этого прямоугольника.
====
Даны уравнения двух сторон прямоугольника \(x - 2y = 0,x - 2y + 15 = 0\) и уравнение одной из его диагоналей \(7x + y - 15 = 0\). Найти вершины прямоугольника.
====
Стороны треугольника даны уравнениями \(4x - y - 7 = 0,x + 3y - 31 = 0,x + 5y - 7 = 0\). Определить точку пересечения его высот.
====
Даны вершины треугольника \(A(1; - 1),B( - 2;1)\) и \(C(3;5)\). Составить уравнение перпендикуляра, опущенного из вершины \(A\) на медиану, проведенную из вершины \(B\).
====
Даны вершины треугольника \(A(2; - 2),B(3; - 5)\) и \(C(5;7)\). Составить уравнение перпендикуляра, опущенного из вершины \(C\) на биссектрису внутреннего угла при вершине \(A\).
====
Составить уравнения сторон и медиан треугольника с вершинами \(A(3;2),B(5; - 2),C(1;0)\).
====
Даны уравнения двух сторон прямоугольника \(5x + 2y - 7 = 0,\ \ \ \ 5x + 2y - 36 = 0\) и уравнение его диагонали \(3x + 7y - 10 = 0\). Составить уравнения остальных сторон и второй диагонали этого прямоугольника.
====
Даны вершины треугольника \(A(1; - 2),B(5;4)\) и \(C( - 2;0)\). Составить уравнения биссектрис его внутреннего и внешнего углов при вершине \(A\).
====
На оси абсцисс найти такую точку \(P\), чтобы сумма ее расстояний до точек \(M(1;2)\) и \(\dot{N}(3;4)\) была наименьшей.
====
На оси ординат найти такую точку \(P\), чтобы разность расстояний ее до точек \(M( - 3;2)\) и \(N(2;5)\) была наибольшей.
====
На прямой \(2x - y - 5 = 0\) найти такую точку \(P\), сумма расстояний которой до точек \(A( - 7;1),B( - 5;5)\) была бы наименьшей.
====
На прямой \(3x - y - 1 = 0\) найти такую точку \(P\), разность расстояний которой до точек \(A(4;1)\) и \(B(0;4)\) была бы наибольшей.
====
Точка \(A( - 4;5)\) является вершиной квадрата, диагональ которого лежит на прямой \(7x - y + 8 = 0\). Составить уравнения сторон и второй диагонали этого квадрата.
====
Даны две противоположные вершины квадрата \(A( - 1;3)\) и \(C(6;2)\). Составить уравнения его сторон.
====
Даны две вершины треугольника \(M_{1}( - 10;2)\) и \(M_{2}(6;4)\); его высоты пересекаются в точке \(N(5;2)\). Определить координаты третьей вершины \(M_{3}\).
====
Даны две вершины \(A(3; - 1)\) и \(B(5;7)\) треугольника \(ABC\) и точка \(N(4; - 1)\) пересечения его высот. Составить уравнения сторон этого треугольника.
====
В треугольнике \(ABC\) даны: уравнение стороны \(AB:5x - 3y + 2 = 0\), уравнения высот \(AM:4x - 3y + 1 = 0\) и \(BN:7x + 2y - 22 = 0\). Составить уравнения двух других сторон и третьей высоты этого треугольника.
====
Составить уравнения сторон треугольника \(ABC\), если даны одна из его вершин \(A(1;3)\) и уравнения двух медиан \(x - 2y + 1 = 0\) и \(y - 1 = 0\).
====
Составить уравнения сторон треугольника, если даны одна из его вершин \(B( - 4; - 5)\) и уравнения двух высот \(5x + 3y - 4 = 0\) и \(3x + 8y + 13 = 0\).
====
Составить уравнения сторон треугольника, зная одну из его вершин \(A(4; - 1)\) и уравнения двух биссектрис \(x - 1 = 0\) и \(x - y - 1 = 0\).
====
Составить уравнения сторон треугольника, зная одну его вершину \(B(2;6)\), а также уравнения высоты \(x - 7y + 15 = 0\) и биссектрисы \(7x + y + 5 = 0\), проведенных из одной вершины.
====
Составить уравнения сторон треугольника, знал одну его вершину \(B(2; - 1)\), а также уравнения высоты \(3x - 4y + 27 = 0\) и биссектрисы \(x + 2y - 5 = 0\), проведенных из различных вершин.
====
Доказать, что прямая \(2x + y + 3 = 0\) пересекает отрезок, ограниченный точками \(A( - 5;1)\) и \(B(3;7)\).
====
Последовательные вершины четырехугольника суть точки \(A( - 3;5),B( - 1; - 4),C(7; - 1)\) и \(D(2;9)\). Установить, является ли этот четырехугольник выпуклым.
====
Составить уравнение биссектрисы угла между прямыми \(x + 2y - 11 = 0\) и \(3x - 6y - 5 = 0\), в котором лежит точка \(M(1; - 3)\).
====
Составить уравнение биссектрисы угла между прямыми \(2x - 3y - 5 = 0,6x - 4y + 7 = 0\), смежного с углом, содержащим точку \(C(2; - 1)\).
====
Дано уравнение пучка прямых \(\alpha(3x - 4y - 3) +\) \(+ \beta(2x + 3y - 1) = 0\). Написать уравнение прямой этого пучка, проходящей через центр масс однородной треугольной пластинки, вершины которой суть точки \(A( - 1;2),B(4; - 4)\) и \(C(6; - 1)\).
====
В треугольнике \(ABC\) даны уравнения высоты \(AN\): \(x + 5y - 3 = 0\), высоты \(BN:x + y - 1 = 0\) и стороны \(AB:x + 3y - 1 = 0\) . Не определяя координат вершин и точки пересечения высот треугольника, составить уравнение двух других сторон н третьей высоты.
====
Составить уравнения сторон треугольника \(ABC\), зная одну его вершину \(A(2; - 1)\), а также уравнения высоты \(7x - 10y + 1 = 0\) и 6иссектрисы \(3x - 2y + 5 = 0\), проведенных из одной вершины. Решить задачу, не вычисляя координат вершин \(B\) и \(C\).
