Составить уравнение плоскости, проходящей через точку \(M_{0}(3;4; - 5)\) параллельно векторам \({\overrightarrow{a}}_{1} = \{ 3;1; - 1\}\) и \({\overrightarrow{a}}_{2} = \{ 1; - 2;1\}\).
====
Составить уравнение плоскости, проходящей через точки \(M_{1}(2; - 1;3)\) и \(M_{2}(3;1;2)\) параллельно вектору \(\overrightarrow{a} = \{ 3; - 1;4\}\).
====
Составить уравнение плоскости, проходящей через точки \(M_{1}(3; - 1;2),M_{2}(4; - 1; - 1)\) и \(M_{3}(2;0;2)\).
====
Вычислить площадь треугольника, который отсекает плоскость \(5x - 6y + 3z + 120 = 0\) от координатного угла \(Oxy\).
====
Вычислить объем пирамиды, ограниченной плоскостью \(2x - 3y + 6z - 12 = 0\) и координатными плоскостями.
====
В пучке плоскостей \(2x - 3y + z - 3 + \lambda(x + 3y + 2z + 1) = 0\) найти плоскость, которая: 1) проходит через точку \(M_{1}(1; - 2;3)\); 2) параллельна оси \(Ox\); 3) параллельна оси \(Oy\); 4) параллельна оси \(Oz\).
====
Составить уравнение плоскости, которая проходит через прямую пересечения плоскостей \(3x - y + 2z + 9 = 0\), \(x + z - 3 = 0\): 1) и через точку \(M_{1}(4; - 2; - 3)\); 2)параллельно оси \(Ox\); 3) параллельно оси \(Oy\); 4) параллельно оси \(Oz\).
====
Составить канонические уравнения следующих прямых: 1) \(x - 2y + 3z - 4 = 0,3x + 2y - 5z - 4 = 0\); 2) \(5x + y + z = 0,2x + 3y - 2z + 5 = 0\); 3) \(x - 2y + 3z + 1 = 0,2x + y - 4z - 8 = 0\).
====
Составить параметрические уравнения следующих прямых: 1) \(2x + 3y - z - 4 = 0,3x - 5y + 2z + 1 = 0\); 2) \(x + 2y - z - 6 = 0,2x - y + z + 1 = 0\).
====
Даны прямые \(\frac{x + 2}{2} = \frac{y}{- 3} = \frac{z - 1}{4},\ \ \ \ \frac{x - 3}{l} = \frac{y - 1}{4} = \frac{z - 7}{2}\) при каком значении \(l\) они пересекаются?
====
Доказать, что прямая \(5x - 3y + 2z - 5 = 0\), \(2x - y - z - 1 = 0\) лежит в плоскости \(4x - 3y + 7z - 7 = 0\).
