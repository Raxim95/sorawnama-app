Analitik geometriya fanining predmeti va metodlari.
====
Vektor tushunchasi. Vektorlar ustida chiziqli amallar.
====
Chiziqli bog‘liq va chiziqli bog‘lanmagan vektorlar.
====
Vektorning koordinatalari.
====
Vektorlarning skalyar ko‘paytmasi.
====
Vektorlarning vektor ko‘paytmasi va aralash ko‘paytmasi.
====
Koordinatalari bilan berilgan vektorlarning skalyar, vektor va aralash ko‘paytmalari.
++++
Tekislikda va fazoda dekart koordinatalar sistemasini almashtirish.
====
Tekislikda to‘g‘ri chiziqning tenglamalari.
====
Tekislikdagi to‘g‘ri chiziqlarning o‘zaro joylashishi.
====
Nuqtadan to‘g‘ri chiziqqacha bo‘lgan masofa. To‘g‘rilar dastasi.
====
Tekislikning tenglamalari. Tekisliklarning o‘zaro joylashishi.
====
Fazoviy to‘g‘ri chiziqning tenglamalari. To‘g‘ri chiziqlarning o‘zaro joylashishi.
====
Tekislik va to‘g‘ri chiziqlarning o‘zaro joylashishi.
====
Nuqtadan tekislikkacha, fazoda nuqtadan to‘g‘ri chiziqqacha va ayqash to‘g‘ri chiziqlar orasidagi masofa.
++++
$M_1 (1; -2) $, $M_2 (2; 1) $ nuqtalar berilgan.
Quyidagi kesmalarning koordinata o‘qlariga proyeksiyalarini toping: $\overline{M_1M_2}$ \\
====
Kvadratning ikkita qo‘shni uchlari $A (3; -7)$ va
$B (-1;4) $ berilgan. Uning yuzini hisoblang.
====
Kvadratning ikkita qarama-qarshi uchlari $P (3; 5) $ va
$Q (1; -3) $ berilgan. Uning yuzini hisoblang.
====
Ikkita uchi $A (-3; 2) $ va $B (1; 6) $ nuqtalarda
joylashgan muntazam uchburchakning yuzini hisoblang.
====
$ABCD$ parallelogrammning uchta uchi $A (3; -7) $,
$B (5; -7) $, $C (-2; 5) $ berilgan, to‘rtinchi uchi $D$,
$B$ uchiga qarama-qarshi. Shu parallelogrammning diagonallari
uzunliklarini aniqlang.
====
Berilgan $A (3; -5) $, $B (-2; -7)$ va
$C (18; 1) $ nuqtalar bir to‘g‘ri chiziqda yotishini isbotlang.
====
$A (2;2) $, $B (-1;6) $, $C (-5;3) $ va $D (-2;-1) $
nuqtalari kvadrat uchlari ekanini isbotlang.
====
Bir jinsli elementdan yasalgan qatorning uchlari
$A (3;-5) $ va $B (-1;1) $ nuqtalarda joylashgan. Uning og‘irligi
markazi koordinatasini aniqlang.
====
Bir jinsli elementdan yasalgan qatorning og‘irlik markazi
$M (1;4) $ nuqtada, bir uchi $P (-2;2) $ nuqtada joylashgan. Shu
qatorning ikkinchi uchi $Q$ ning koordinatalarini aniqlang.
====
Uchburchak uchlarining koordinatalari berilgan
$A (1;-3) $, $B (3;-5) $ va $C (-5;7) $. Tomonlarining o‘rtalarini
aniqlang.
====
$M (2;-1) $, $N (-1;4) $ va $P (-2;2) $ nuqtalar
uchburchak tomonlarining o‘rtalari. Uchlarining koordinatalarini
aniqlang.
====
Parallelogrammning uchlari
$A (3;-5) $, $B (5;-3) $, $C (-1;3) $ berilgan. $B$ tepasiga
qarama-qarshi joylashgan $D$ uchini aniqlang.
====
Parallelogrammning ikkita qo‘shni uchlari $A (-3;5) $, $B (1;7) $
va dioganallarining kesishish nuqtasi $M (1;1)$ berilgan. Qolgan ikki
cho‘qqisini aniqlang.
====
$ABCD$-parallelogrammning uchta uchi
$A (2;3) $, $B (4;-1) $ va $C (0;5) $ berilgan. To‘rtinchi $D$
cho‘qqisini toping.
====
Uchburchakning uchlari $A (1;4) $, $B (3;-9) $, $C (-5;2) $
berilgan. $B$ uchidan o‘tkazilgan mediana uzunligini aniqlang.
====
$A (1;-3) $ va $B (4;3) $ nuqtalarni tutashtiruvchi
kesma teng uch bo‘lakka bo‘lindi. Bo‘luvchi nuqtalarning koordinatalarini
aniqlang.
====
$A (4;2) $, $B (7;-2) $ va $C (1;6) $ nuqtalar bir jinsli
simdan yasalgan uchburchak uchlari. Shu uchburchakning og‘irligi
====
Uchlari $A (2;-3) $, $B (3;2) $ va $C (-2;5) $
nuqtalarida joylashgan uchburchaklarning yuzini hisoblang.
====
Uchlari $M_1 (-3;2) $, $M_2 (5;-2) $ va $M_3 (1;3) $
nuqtalarida joylashgan uchburchaklarning yuzini hisoblang.
====
Uchlari $M (3;-4) $, $N (-2;3) $ va $P (4;5) $
nuqtalarida joylashgan uchburchaklarning yuzini hisoblang.
====
Uch uchi $A (-2;3), \ B (4;-5) $ va
$C (-3;1)$ nuqtalarda joylashgan parallelogrammning yuzini aniqlang.
====
Bir jinsli to‘rtburchakli plastinkaning uchlari berilgan:
$A (2;1), \ B (5;3), \ C (-1;7) $ va $D (-7;5) $. Uning og‘irlik markazi
koordinatalarini aniqlang.
====
Bir jinsli beshburchakli plastinkaning uchlari berilgan:
$A (2;3), \ B (0;6), \ C (-1;5), \ D (0;1) $ va $E (1;1) $. Uning og‘irligi
markazi koordinatalarini aniqlang.
====
Ikkala uchi $A (3;1) $ va $B (1;-3) $ nuqtalarda, a
uchinchi $C$ uchi $Oy$ o‘qiga tegishli uchburchakning
yuzi $S=3$ ga teng. $C$ uchining koordinatalarini aniqlang.
====
Ikkala uchi $A (2;1) $ va $B (3;-2) $ nuqtalarda, va
uchinchi $C$ uchi $Ox$ o‘qiga tegishli bo‘lgan uchburchakning
yuzi $S=4$ ga teng. $C$ uchining koordinatalarini aniqlang.
++++
Berilgan $M_1 (3; 1) $, $M_2 (2; 3) $, $M_3 (6; 3) $,
$M_4 (-3;-3) $. $M_5 (3;-1) $, $M_6 (-2; 1) $ nuqtalarning qaysilari
$2x-3y-3 = 0$ to‘g‘ri chiziqqa tegishli va qaysilari tegishli
emas.
====
$P1$, $P2$, $P3$, $P4$, $P5$ nuqtalar
3x-2y-6=0 to‘g‘ri chiziqqa tegishli va abssissalari mos ravishda
4, 0, 2, -2, -6 ga teng. Ularning ordinatalarini toping.
====
$Q_1$, $Q_2$, $Q_3$, $Q_4$, $Q_5$ nuqtalar
$x-3y+2=0$ to‘g‘ri chiziqqa tegishli va ordinatalari mos ravishda
1, 0, 2, -1, 3 ga teng. Ularning abssissalarini toping.
====
$5x-y+3=0$ to‘g‘ri chiziqning $k$ burchagi
koeffitsiyentini va $Oy$ o‘qidan kesib olgan kesmaning algebraik
qiymati $b$ ni aniqlang.
====
$2x+3y-6=0$ to‘g‘ri chiziqning $k$ burchagi
koeffitsiyentini va $Oy$ o‘qidan kesib olgan kesmaning algebraik
qiymati $b$ ni aniqlang.
====
$5x+3y+2=0$ to‘g‘ri chiziqning $k$ burchagi
koeffitsiyentini va $Oy$ o‘qidan kesib olgan kesmaning algebraik
qiymati $b$ ni aniqlang.
====
$3x+2y=0$ to‘g‘ri chiziqning $k$ burchagi
koeffitsiyentini va $Oy$ o‘qidan kesib olgan kesmaning algebraik
qiymati $b$ ni aniqlang.
====
$y-3=0$ to‘g‘ri chiziqning $k$ burchagi
koeffitsiyentini va $Oy$ o‘qidan kesib olgan kesmaning algebraik
qiymati $b$ ni aniqlang.
====
Umumiy tenglama bilan berilgan to‘g‘ri chiziqlarning
o‘zaro joylashuvini aniqlang, agar kesishadigan bo‘lsa kesishish nuqtasini
toping: $12x+15y-39=0, 16x-9y-23=0$.
====
Umumiy tenglama bilan berilgan to‘g‘ri chiziqlarning
o‘zaro joylashuvini aniqlang, agar kesishadigan bo‘lsa kesishish nuqtasini
toping: $3x+2y-27=0, x+5y-35=0$.
====
Umumiy tenglama bilan berilgan to‘g‘ri chiziqlarning
o‘zaro joylashuvini aniqlang, agar kesishadigan bo‘lsa kesishish nuqtasini
toping: $12x+59y-19=0, 8x+33y-19=0$.
====
Umumiy tenglama bilan berilgan to‘g‘ri chiziqlarning
o‘zaro joylashuvini aniqlang, agar kesishadigan bo‘lsa kesishish nuqtasini
toping: $6x+10y+9=0, 3x+5y-6=0$.
====
Umumiy tenglama bilan berilgan to‘g‘ri chiziqlarning
o‘zaro joylashuvini aniqlang, agar kesishadigan bo‘lsa kesishish nuqtasini
toping: $14x-9y-24=0, 7x-2y-17=0$.
====
Umumiy tenglama bilan berilgan to‘g‘ri chiziqlarning
o‘zaro joylashuvini aniqlang, agar kesishadigan bo‘lsa kesishish nuqtasini
toping: $2x-3y+12=0, 4x-6y-21=0$.
====
Umumiy tenglama bilan berilgan to‘g‘ri chiziqlarning
o‘zaro joylashuvini aniqlang, agar kesishadigan bo‘lsa kesishish nuqtasini
toping: $2y+9=0, y-5=0$.
====
Umumiy tenglama bilan berilgan to‘g‘ri chiziqlarning
o‘zaro joylashuvini aniqlang, agar kesishadigan bo‘lsa kesishish nuqtasini
toping: $4x-7=0, 3x+8=0$.
====
Umumiy tenglama bilan berilgan to‘g‘ri chiziqlarning
o‘zaro joylashuvini aniqlang, agar kesishadigan bo‘lsa kesishish nuqtasini
toping: $2x-5y+1=0, 6x-15y+3=0$.
====
Umumiy tenglama bilan berilgan to‘g‘ri chiziqlarning
o‘zaro joylashuvini aniqlang, agar kesishadigan bo‘lsa kesishish nuqtasini
toping: $x-5=0, y+12=0$.
====
Umumiy tenglama bilan berilgan to‘g‘ri chiziqlarning
o‘zaro joylashuvini aniqlang, agar kesishadigan bo‘lsa kesishish nuqtasini
toping: $x\sqrt{2}+12=0, 4x+24\sqrt{2}=0$.
====
Umumiy tenglama bilan berilgan to‘g‘ri chiziqlarning
o‘zaro joylashuvini aniqlang, agar kesishadigan bo‘lsa kesishish nuqtasini
toping: $3x+y\sqrt{3}=0, x\sqrt{3}+3y-6=0$.
====
$a$ va $b$ parametrlarining qanday qiymatlarida
$ax-2y-1=0$, $6x-4y-b=0$ to‘g‘ri chiziqlar umumiy nuqtaga ega bo‘ladi?
====
$a$ va $b$ parametrlarining qanday qiymatlarida
$ax-2y-1=0$, $6x-4y-b=0$ to‘g‘ri chiziqlar parallel bo‘ladi?
====
$a$ va $b$ parametrlarining qanday qiymatlarida
$ax-2y-1=0$, $6x-4y-b=0$ to‘g‘ri chiziqlar kesishadimi?
====
$m$ va $n$ parametrlarining qanday qiymatlarida
$mx+8y+n=0$, $2x+my-1=0$ to‘g‘ri chiziqlar parallel bo‘ladi?
====
$m$ parametrining qanday qiymatlarida
$ (m-1) x+my-5=0$, $mx+ (2m-1) y+7=0$ to‘g‘ri chiziqlar abssissa
o‘qida yotuvchi nuqtada kesishadi.
====
$m$ parametrining qanday qiymatlarida
$mx+ (2m+3) y+m+6=0$, $ (2m+1) x+ (m-1) y+m-2=0$ to‘g‘ri chiziqlar ordinata
o‘qida yotuvchi nuqtada kesishadi.
====
$3x-y+2=0$, $4x-5y+5=0$, $2x+3y-1=0$
to‘g‘ri chiziqlar bir nuqtada kesishishadimi?
====
$5x+3y-7=0$, $x-2y-4=0$, $3x-y+3=0$
to‘g‘ri chiziqlar bir nuqtada kesishishadimi?
====
$x+2y-17=0$, $2x-y+1=0$, $x+2y-3=0$
to‘g‘ri chiziqlar bir nuqtada kesishishadimi?
====
$2x-y+2=0$, $4x-2y+4=0$, $6x-3y+6=0$
to‘g‘ri chiziqlar bir nuqtada kesishishadimi?
====
5x-3y+15=0 to‘g‘ri chiziqning koordinata burchagidan
kesib olgan uchburchakning yuzini hisoblang.
====
$M (-3;8) $ nuqtadan o‘tib, koordinata o‘qlaridan
teng kesmalarni kesib oladigan to‘g‘ri chiziqlarning tenglamasini tuzing.
====
$M (3;3)$ nuqtadan o‘tib, koordinata o‘qlaridan teng
kesmalarni kesib oladigan to‘g‘ri chiziqlarning tenglamasini tuzing.
====
$P (2;2)$ nuqtadan o‘tib, koordinata burchagidan
yuzi 1 ga teng uchburchak kesib oladigan to‘g‘ri chiziqlarning
tenglamasini tuzing.
====
$B (-5;5)$ nuqtadan o‘tib, koordinata burchagidan
yuzi 50 ga teng uchburchak kesib oladigan to‘g‘ri chiziqlarning tenglamasini
tuzing.
====
$P (8;6) $ nuqtadan o‘tib, koordinata burchagidan
yuzi 12 ga teng uchburchak kesib oladigan to‘g‘ri chiziqlarning tenglamasini
tuzing.
====
$P (12;6)$ nuqtadan o‘tib, koordinata burchagidan
yuzi 150 ga teng uchburchak kesib oladigan to‘g‘ri chiziqlarning
tenglamasini tuzing.
====
$M (4;3) $ nuqtadan, koordinata burchagidan
yuzi 3 ga teng uchburchak kesib oladigan to‘g‘ri chiziq o‘tkazildi.
Shu to‘g‘ri chiziqning koordinata o‘qlari bilan kesishish nuqtalari
koordinatalarini aniqlang.
====
$A (3;-2) $ nuqtadan $3x+4y-15=0$ to‘g‘ri chiziqqa
gacha siljishni va masofani hisoblang.
++++
Vektor koordinata o‘qlari bilan quyidagi burchaklarni hosil qila oladimi:
$\alpha = 45^{{^\circ}},\beta = 60^{{^\circ}},\gamma = 120^{{^\circ}}$.
====
Vektor koordinata o‘qlari bilan quyidagi burchaklarni hosil qila oladimi:
$\alpha = 45^{{^\circ}},\ \ \ \ \beta = 135^{{^\circ}},\ \gamma = 60^{{^\circ}}$.
====
Vektor koordinata o‘qlari bilan quyidagi burchaklarni hosil qilishi
mumkinmi: $\alpha = 90^{{^\circ}},\ \beta = 150^{{^\circ}}$,
$\gamma = 60^{{^\circ}}?$
====
Tekislikda ikkita vektor
$\overrightarrow{p} = \{ 2; - 3\}$, $\overrightarrow{q} = \{ 1;2\}$.
$\overrightarrow{a} = \{9;4\}$ vektorning
$\overrightarrow{p},\ \overrightarrow{q}$ bazis bo‘yicha yoyilmasi topilsin.
====
To‘rtburchakning uchlari berilgan:
$A (1; - 2;2) $, $B (1;4;0),C (- 4;1;1) $ va $D (- 5; -5;3) $. Uning diagonallari $AC$ va $BD$ o‘zaro
perpendikulyarligini isbotlang.
====
$\alpha$
qanday qiymatlarida 
$\overrightarrow{a} = \alpha\overrightarrow{i} - 3\overrightarrow{j} + 2\overrightarrow{k}$
va
$\overrightarrow{b} = \overrightarrow{i} + 2\overrightarrow{j} - \alpha\overrightarrow{k}$
vektorlar o‘zaro perpendikulyar bo‘lishini aniqlang.
====
$\overrightarrow{a} = \{ 2; - 4;4\}$ va $\overrightarrow{b} = \{ - 3;2;6\}$
vektorlar hosil qilgan burchak kosinusini hisoblang.
====
Uchburchakning uchlari
$A (- 1; - 2;4) $, $B (- 4; - 2;0) $ va $C (3; -2;1) $. Uning $B$ uchidagi
ichki burchakni aniqlang.
====
Uchburchakning uchlari
$A (3;2; 3) $, $B (5;1; - 1) $ va $C (1; -2;1) $. Uning $A$ uchidagi tashqi burchagi aniqlansin.
====
Uchlari $A (1;2;1), B (3;-1;7) $ va $C (7;4;-2) $ bo‘lgan uchburchakning
ichki burchaklarini hisoblab toping. Bu uchburchakning teng yonli ekanligini isbotlang.
====
$\overrightarrow{a}$ va $\overrightarrow{b}$ vektorlar
$\varphi = \pi/6$ burchak hosil qiladi.
$|\overrightarrow{a}| = 6,|\overrightarrow{b}| = 5$ ekanini bilib,
$\left| \left\lbrack \overrightarrow{a},\overrightarrow{b} \right\rbrack \right|$ kattalikni hisoblang.
====
Berilgan: $\overrightarrow{a}| = 10,|\overrightarrow{b}| = 2$ va
$\left(\overrightarrow{a},\overrightarrow{b} \right) = 12$. Hisoblang
$\left| \left\lbrack \overrightarrow{a},\overrightarrow{b} \right\rbrack \right|$.
====
Berilgan: $\overrightarrow{a}| = 3,|\overrightarrow{b}| = 26$ va
$\lbrack\overrightarrow{a},\overrightarrow{b}\rbrack| = 72$. Hisoblang
$\left(\overrightarrow{a},\overrightarrow{b} \right) $.
====
$\overrightarrow{a}
= \{ 1; - 1;3\}, \ \ \ \ \ \overrightarrow{b} = \{ - 2;1\}$, $\overrightarrow{c} = \{3; -2;5\}$ vektorlar berilgan. Hisoblang:
$ (\lbrack\overrightarrow{a},\overrightarrow{b}\rbrack,\overrightarrow{c}) $.
====
Agar \(a = \{ 2;3; - 1\}, \ \ \ \ b = \{ 1; - 1;3\}, \ \ \ \ c = \{ 1;9; - 11\}\) bo‘lsa, $\overrightarrow{a}, \overrightarrow{b}, \overrightarrow{c}$ vektorlar komplanar bo‘lishini tekshiring.
====
Agar \(a = \{ 3; - 2;1\},\ \ \ \ \ b = \{ 2;1;2\},\ \ \ \ c = \{ 3; - 1; - 2\}\) bo‘lsa, $\overrightarrow{a}, \overrightarrow{b}, \overrightarrow{c}$ vektorlar komplanar bo‘lishini tekshiring.
====
Agar \(a = \{ 2; - 1;2\}, \ \ \ \ b = \{ 1;2; - 3\}, \ \ \ \ c = \{ 3; - 4;7\}\) bo‘lsa, $\overrightarrow{a}, \overrightarrow{b}, \overrightarrow{c}$ vektorlar komplanar bo‘lishini tekshiring.
++++
Ikkita qarama-qarshi uchlari \(P (4;9) \) va \(Q (-2; 1) \) nuqtalarida joylashgan romning tomon uzunligi \(5\sqrt{10}\). Shu
romba yuzini hisoblang.
====
Ikkita qarama-qarshi uchlari $P (3; -4) $ va $Q (l;2) $ nuqtalarda joylashgan rombaning tomon uzunligi \(5\sqrt{2}\). Shu romb balandligining uzunligini hisoblang.
====
Uchlari $A_1 (1; 1), A_2 (2; 3) $ va $A (5;-1) $
nuqtalarida joylashgan uchburchakning to‘g‘ri burchakli ekanini isbotlang.
====
Uchlari \(M_{1} (1;1), M_{2} (0,2) \) va
\(M_{3} (2;-1) \) nuqtalarda joylashgan uchburchakning ichki 
burchaklari orasida o‘tmas burchak bor yoki yo‘qligini aniqlang.
====
Uchlari \(M (-1;3),\ N (1,2) \ \) va \(P (0;4) \)
nuqtalarida joylashgan uchburchakning ichki burchaklari o‘tkir burchak
ekanligini isbotlang.
====
Uchburchakning uchlari \(A (5;0),\ B (0;1) \) va \(C (3;3) \)
nuqtalarida. Uning ichki burchaklarini toping.
====
Uchburchakning uchlari
\(A\left(-\sqrt{3};1 \right),\ B (0;2) \) va
\(C\left(-2\sqrt{3};2 \right) \) nuqtalarda. Uning $A$
uchidagi tashqi burchakni toping.
====
Abssissa o‘qida shunday $M$ nuqtani topingki,
\(N (2;-3) \) nuqtadan uzoqligi 5 ga teng bo‘lgan.
====
Ordinata o‘qida shunday $M$ nuqtani toping.
\(N (-8;13) \) nuqtadan uzoqligi 17 ga teng bo‘lgan.
====
Ikkita nuqta berilgan \(M (2;2) \) va \(N (5;-2) \); abssissa o‘qida shunday $P$ nuqtani topingki, $MPN$ burchak to‘g‘ri burchak bo‘lsin.
====
\(M_{1} (1;2) \) nuqtaga, \(A (1;0) \) va \(B (-1;-2) \)
nuqtalaridan o‘tuvchi to‘g‘ri chiziqqa nisbatan simmetrik bo‘lgan \(M_{2}\) nuqtaning koordinatalarini toping.
====
Uchburchakning uchlari \(A (2;-5),\ B (1;-2),\ C (4;7) \)
berilgan. $AC$ tomoni bilan $B$ uchining ichki burchagi
bissektrisasining kesishish nuqtasini toping.
====
Uchburchakning uchlari
\(A (3;-5),\ B (-3;3),\ C (-1;-2) \) berilgan. $A$ uchining ichki qismi
burchakli bessektrisaning uzunligini aniqlang.
++++
Bir to‘g‘ri chiziqqa tegishli \(A (1;-1),\ B (3;3) \) va
\(C (4;5) \) nuqtalar berilgan. Har bir nuqtaning, qolgan ikki nuqta orqali aniqlanuvchi kesmani bo‘lish nisbati $\lambda$ ni aniqlang.
====
\(P (2;2) \) va \(Q (1;5) \) nuqtalar bilan teng uchta
bo‘lingan kesmaning uchlari $A$ va $B$ nuqtalarning
koordinatalarini aniqlang.
====
To‘g‘ri \(M_{1} (-12;-13) \) va \(M_{2} (-2;-5) \)
nuqtalaridan o‘tadi. Shu to‘g‘ri chiziqda abssissasi 3 ga teng nuqtani toping.
====
To‘g‘ri chiziq \(M (2;-3) \) va \(N (-6;5) \) nuqtalardan o‘tadi.
Shu to‘g‘ri chiziqda ordinatasi $-5$ ga teng nuqtani toping.
====
To‘g‘ri chiziq \(A (7;-3) \) va \(B (23;-6) \) nuqtalardan o‘tadi.
Shu to‘g‘ri chiziqning abssissa o‘qi bilan kesishish nuqtasini toping.
====
To‘g‘ri \(A (5;2) \) va \(B (-4; -7) \) nuqtalaridan o‘tadi.
Shu to‘g‘ri chiziqning ordinata o‘qi bilan kesishish nuqtasini toping.
====
To‘rtburchakning uchlari
\(A (-3;12),\ B (3;-4),\ C (5;-4) \) va \(D (5;8) \) berilgan. Shu
to‘rtburchakning $AC$ diagonali $BD$ diagonali qanday
nisbatda bo'lishini aniqlang.
====
To‘rtburchakning uchlari
\(A (-2;14),\ B (4;-2),\ C (6;-2) \) va \(D (6;10) \) berilgan. Shu
to‘rtburchakning $AC$ va $BD$ diagonallarining kesishishi
nuqtani toping.
====
Uchburchakning uchlari \(A (3;6),\ B (-1;3) \) va
\(C (2:-1) \) nuqtalarda joylashgan. $C$ uchidan tushirilgan balandlik uzunligini hisoblang.
====
Parallelogrammning uchta uchi \(A (3;7),\ B (2;-3) \) va
\(C (-1;4) \) nuqtalarda joylashgan. $B$ uchidan $AC$
tomonidan tushirilgan balandlik uzunligini hisoblang.
====
Ikkala uchi \(A (3;1) \) va \(B (1;-3) \) nuqtalarda, va
og‘irlik markazi $Ox$ o‘qiga tegishli uchburchakning yuzi
\(S=3\) ga teng. Uchinchi $C$ uchining koordinatalarini aniqlang.
++++
Berilgan to‘g‘ri chiziqlarning kesishish nuqtasini toping:
$(3x-4y-29=0, 2x+5y+19=0)$.
====
$ABC$ uchburchakning tomonlari:
\(AB:4x+3y-5=0,\ BC:x-3y+10=0,\ AC:x-2=0\) 
tenglamalari bilan berilgan. Uchlarining koordinatalarini aniqlang.
====
Parallelogrammning ikki tomoni tenglamalari
\(8x+3y+1=0,\ 2x+y-1=0\) va bir diagonali tenglamasi
\(3x+2y+3=0\) berilgan. Parallelogramm uchlari koordinatalarini
aniqlang
====
Uchburchakning tomonlari \(x+5y-7=0\),
\(3x-2y-4=0\), \(7x+y+19=0\) to‘g‘ri chiziqlarda yotadi. Uning
yuzini hisoblang.
====
Umumiy tenglamasi \(2x-5y+4=0\) bo‘lgan to‘g‘ri
berilgan. \(M (-3,5) \) nuqtadan o‘tib, berilgan to‘g‘ri chiziqqa: a) parallel;
b) perpendikular bo‘lgan to‘g‘ri chiziqlar tenglamasini tuzing.
====
To‘g‘ri to‘rtburchakning bir uchi \(A (2;-3) \), va ikkita tarafining
ning tenglamalari \(2x+3y+9=0,\ 3x-2y-7=0\)
berilgan. Qolgan ikki tomonning tenglamalarini tuzing.
====
\(N (5;8) \) nuqtaning, \(5x-11y-43=0\) to‘g‘ri chizig‘idagi
proyeksiyasini toping.
====
Quyidagi har bir to‘g‘ri chiziqlar jufti uchun, ularga parallel
bo‘lib, aynan o‘rtasidan o‘tuvchi to‘g‘ri tenglamani tuzing: $3x-2y-3=0$, $3x-2y-17=0$.
====
Berilgan ikki nuqtadan o‘tuvchi to‘g‘ri chiziqning burchagi
koeffitsiyenti $k$ ni hisoblang: $A (-4;3) $, $B (1;8) $.
====
Uchburchak uchlari \(A (1;0),\ B (5;-2),\ C (3;2) \)
koordinatalari bilan berilgan. Uchburchaklar tomonlarining va
medianalarining tenglamalarini tuzing.
====
\(P (3;8) \) va \(Q (-1;-6) \) nuqtalardan o‘tgan
to‘g‘ri chiziqning koordinata o‘qlari bilan kesishish nuqtalarini toping.
====
Doiraviy to‘rtburchakning uchlari
\(A (-2;-6),\ B (7;6),\ C (3;9) \) va \(D (-3;1) \) nuqtalarda
joylashgan. Diagonallarining kesishish nuqtasi topilsin.
====
$ABCD$ parallelogrammning ikkita qo‘shni uchlari
\(A (3,3),\ B (-1;7) \) va diagonallarining kesishish nuqtasi
\(E (2;-4) \) berilgan. Shu parallelogramm tomonlarining tenglamalarini
tuzing.
++++
To‘g‘ri to‘rtburchakning ikki tomoni
\(5x+2y-7=0,\ 5x+2y-36=0\) va diagonali
\(3x+7y-10=0\) tenglamalar bilan berilgan. Qolgan ikki tomoni
tenglamalarni tuzing.
====
Berilgan to‘g‘ri chiziqlar orasidagi burchakni aniqlang: $3x+2y+4=0, 5x-y+1=0$.
====
Qirralari
\(7x+y+31=0,\ 3x+4y-1=0,\ x-7y-17=0\) tenglamalar
bilan berilgan uchburchakning teng yonli ekanini isbotlang.
Masalani uchburchakning
burchaklarini topish orqali yeching.
====
\(N (4;-5) \) nuqtadan o‘tib, $2x+5y-7=0$
to‘g‘ri chiziqlariga parallel to‘g‘ri chiziqlarning tenglamasini tuzing. Masalani burchaklik
koeffitsiyentni hisoblamasdan yeching.
====
Quyida berilgan to‘g‘ri chiziqlar juftlarining qaysilari
perpendikular ekanini aniqlang: $4x+y+6=0, 2x-8y-13=0$.
====
Ikki to‘g‘ri chiziqning chetidagi burchakni toping: $2x+y-9=0, 3x-y+11=0$.
====
Parallel to‘g‘ri chiziqlar orasidagi masofani hisoblang: $5x-12y+13=0, 5x-12y-26=0$.
====
Kvadratning ikki tomoni
\(5x-12y+65=0,\ 5x-12y-26=0\) to‘g‘ri chiziqlarda
yotishini bilgan holda, yuzini hisoblang.
====
\(P (2;7) \) nuqtadan o‘tib, \(Q (1;2) \) nuqtagacha
masofasi 5 ga teng bo‘lgan to‘g‘ri chiziqlarning tenglamasini tuzing.
====
\(M (7;-2) \) nuqtadan o‘tib, \(N (4;-6) \) nuqtaga
gacha bo‘lgan masofasi 5 ga teng bo‘lgan to‘g‘ri chiziqlarning tenglamasini tuzing.
====
\(A (4;-5) \) nuqtadan o‘tib, \(B (-2;3) \) nuqtaga
gacha masofasi 12 ga teng bo‘lgan to‘g‘ri chiziqlarning tenglamasini tuzing.
====
Berilgan \(8x-15y-25=0\) to‘g‘ri chiziqdan og‘ishi -2 ga teng
teng bo‘lgan nuqtalarning geometrik o‘rni tenglamasini tuzing.
====
Berilgan \(3x-4y-10=0\) to‘g‘ri chiziqqa parallel va undan
$d=3$ masofada yotuvchi to‘g‘ri chiziqlarning tenglamasini tuzing.
====
Berilgan parallel to‘g‘ri chiziqlardan teng masofada yotuvchi
nuqtalarning geometrik o‘rni tenglamasini tuzing: $2x+y+7=0, 2x+y-3=0$.
====
\(P (1;-2) \) nuqta va koordinatalar boshi, berilgan ikkita
to‘g‘ri yozing: $12x-5y-7=0, 3x+4y-8=0$.
kesishishidan hosil bo‘lgan bir xil burchakdami, qo‘shni burchakdami yoki vertikal
burchaklarda yotadimi?
====
\(P (2;3) \) va \(Q (5;-1) \) nuqtalar, berilgan ikkita
to‘g‘ri: $12x-y-7=0,\ 13x+4y-5=0$.
kesishishidan hosil bo‘lgan bir xil burchakdami, qo‘shni burchakdami yoki vertikal
burchaklarda yotadimi?
====
Koordinata boshi, tomonlarining tenglamalari
\(8x+3y+31=0,\ x+8y-19=0,\ 7x-5y-11=0\) bilan
berilgan uchburchakning tashqarisida yoki ichida yotishini aniqlang.
====
\(P (-3;2) \) nuqta, tomonlarining tenglamalari
\(x+y-4=0,\ 3x-7y+8=0,\ 4x-y-31=0\) bilan
berilgan uchburchakning tashqarisida yoki ichida yotishini aniqlang.
====
Koordinata boshi, berilgan to‘g‘ri chiziqlarning:
\(3x+y-4=0\) va \(3x-2y+6=0\) kesishmasida hosil bo‘ladi
bo‘lgan o‘tkir yoki o‘tmas burchakka tegishli bo‘lishini aniqlang.
====
\(M (2;-5) \) nuqta, berilgan to‘g‘ri chiziqlarning:
\(3x+5y-4=0\) va \(x-2y+3=0\) kesishmasida hosil bo‘ladi
bo‘lgan o‘tkir yoki o‘tmas burchakka tegishli bo‘lishini aniqlang.
====
\(4x+3y-1=0\) va \(3x-2y+5=0\)
to‘g‘ri chiziqlarning kesishish nuqtasidan o‘tib (bu nuqtani aniqlamay), ordinata
o‘qidan \(b=4\) kesmani kesib oladigan to‘g‘ri chiziq tenglamasini tuzing.
====
\(2x+y-2=0\) va \(x-5y-3=0\)
to‘g‘ri chiziqlarning kesishish nuqtasidan o‘tib (bu nuqtani aniqlamay), uchlari
\(A (-1;-4) \) va \(B (5;-6) \) nuqtalarda joylashgan kesmaning
to‘g‘ri o‘rtasidan o‘tuvchi to‘g‘ri chiziqning tenglamasini tuzing.
====
Uchlari \(A (4;-4),\ B (6;-1) \) va \(C (-1;2) \)
nuqtalarida joylashgan bir jinsli plastinkadan yasalgan uchburchakning
og‘irlik markazidan o‘tib, quyida berilgan
\(\alpha (2x+3y-1) +\beta (3x-4y-3) =0\) to‘g‘ri chiziqlar dasturiga
tegishli to‘g‘ri chiziqning tenglamasini tuzing.
++++
Tekislikda uchta vektor $\vec{a} = \{ 3; - 2\}$, $\vec{b} = \{ - 2;1\}$ va $\vec{c} = \{ 7; - 4\}$ berilgan. Bu uchta vektorning har birining qolgan ikkitasini bazis sifatida qabul qilib yoyilmasini toping.
====
$\vec{a}$ va $\vec{b}$ vektorlar $\varphi = 2\pi/3$ burchak hosil qiladi. $|\vec{a}| = 3,|\vec{b}| = 4$ ekani ma’lum. Hisoblang:
$\left(\vec{a},\vec{b} \right) $.
====
$\vec{a}$ va $\vec{b}$ vektorlar $\varphi = 2\pi/3$ burchak hosil qiladi. $|\vec{a}| = 3,|\vec{b}| = 4$ ekani ma’lum. Hisoblang:
${\vec{a}}^{2}$.
====
$\vec{a}$ va $\vec{b}$ vektorlar $\varphi = 2\pi/3$ burchak hosil qiladi. $|\vec{a}| = 3,|\vec{b}| = 4$ ekani ma’lum. Hisoblang:
${\vec{b}}^{2}$.
====
$\vec{a}$ va $\vec{b}$ vektorlar $\varphi = 2\pi/3$ burchak hosil qiladi. $|\vec{a}| = 3,|\vec{b}| = 4$ ekani ma’lum. Hisoblang:
$ (\vec{a} + \vec{b}) ^{2}$.
====
$\vec{a}$ va $\vec{b}$ vektorlar $\varphi = 2\pi/3$ burchak hosil qiladi. $|\vec{a}| = 3,|\vec{b}| = 4$ ekani ma’lum. Hisoblang:
$\left(3\vec{a} - 2\vec{b},\vec{a} + 2\vec{b} \right) $.
====
$\vec{a}$ va $\vec{b}$ vektorlar $\varphi = 2\pi/3$ burchak hosil qiladi. $|\vec{a}| = 3,|\vec{b}| = 4$ ekani ma’lum. Hisoblang:
$ (\vec{a} - \vec{b}) ^{2};$ 7) $ (3\vec{a} + 2\vec{b}) ^{2}$.
====
$\vec{a}$ va $\vec{b}$ vektorlar o‘zaro perpendikulyar; $\vec{c}$ vektor ular bilan $\pi/3$ ga teng bo‘lgan burchaklar hosil qiladi; $|\vec{a}| = 3$, $|\vec{b}| = 5,\ |\vec{c}| = 8$ ekani ma’lum, quyidagilarni hisoblang:
$\left(3\vec{a} - 2\vec{b},\vec{b} + 3\vec{c} \right) $.
====
$\vec{a}$ va $\vec{b}$ vektorlar o‘zaro perpendikulyar; $\vec{c}$ vektor ular bilan $\pi/3$ ga teng bo‘lgan burchaklar hosil qiladi; $|\vec{a}| = 3$, $|\vec{b}| = 5,\ |\vec{c}| = 8$ ekani ma’lum, quyidagilarni hisoblang:
$ (\vec{a} + \vec{b} + \vec{c}) ^{2}$.
====
$\vec{a}$ va $\vec{b}$ vektorlar o‘zaro perpendikulyar; $\vec{c}$ vektor ular bilan $\pi/3$ ga teng bo‘lgan burchaklar hosil qiladi; $|\vec{a}| = 3$, $|\vec{b}| = 5,\ |\vec{c}| = 8$ ekani ma’lum, quyidagilarni hisoblang:
$ (\vec{a} + 2\vec{b} - 3\vec{c}) ^{2}$.
====
$\vec{a} + \vec{b} + \vec{c} = 0$ shartni qanoatlantiruvchi $\vec{a},\ \vec{b}$ va $\vec{c}$ vektorlar berilgan. $|\vec{a}| = 3,\ |\vec{b}| = 1$ va $|\vec{c}| = 4$ ekani ma’lum, $\left(\vec{a},\vec{b} \right) + \left(\vec{b},\vec{c} \right) + (\vec{c}) $ ifodani hisoblang.
====
$|\vec{a}| = 3,|\vec{b}| = 5$ berilgan. $\alpha$ ning qanday qiymatida $\vec{a} + \alpha\vec{b}$, $\vec{a} - \alpha\vec{b}$ vektorlar o‘zaro perpendikulyar bo‘lishini aniqlang.
====
$a$ va $b$ vektorlar $\varphi = \pi/6$ burchak hosil qiladi; $|a| = \sqrt{3},|b| = 1$ ekani ma’lum. $p = a + b$ va $q = a - b$ vektorlar orasidagi $\alpha$ burchakni hisoblang.
====
$\vec{a} = \{ 6; - 8; - 7,5\}$ vektorga kollinear bo‘lgan $\vec{x}$ vektor $Oz$ o‘qi bilan o‘tkir burchak hosil qiladi. $|\vec{x}| = 50$ ekanini bilgan holda uning koordinatalarini toping.
++++
$\vec{a} = \{ 2;1; - 1\}$ vektorga kollinear bo‘lgan va $\left(\vec{x},\vec{a} \right) = 3$ shartni qanoatlantiruvchi $\vec{x}$ vektorni toping.
====
$\vec{a}$ va $\vec{b}$ vektorlar o‘zaro perpendikulyar. $|\vec{a}| = 3,|\vec{b}| = 4$ ekani ma’lum, quyidagilarni hisoblang:
$|\lbrack\vec{a} + \vec{b},\vec{a} - \vec{b}\rbrack|$.
====
$\vec{a}$ va $\vec{b}$ vektorlar o‘zaro perpendikulyar. $|\vec{a}| = 3,|\vec{b}| = 4$ ekani ma’lum, quyidagilarni hisoblang:
$|\lbrack 3\vec{a} - \vec{b},\vec{a}-2\vec{b}\rbrack|$.
====
$\vec{a}$ va $\vec{b}$ vektorlar $\varphi = 2\pi/3$ burchak hosil qiladi. $|\vec{a}| = 1,|\vec{b}| = 2$ ekanini bilib, quyidagilarni hisoblang:
$\lbrack\vec{a},\vec{b}\rbrack^{2}$.
====
$\vec{a}$ va $\vec{b}$ vektorlar $\varphi = 2\pi/3$ burchak hosil qiladi. $|\vec{a}| = 1,|\vec{b}| = 2$ ekanini bilib, quyidagilarni hisoblang:
$\lbrack 2\overrightarrow{a} + \overrightarrow{b},\overrightarrow{a} + 2\overrightarrow{b}\rbrack^{2}$.
====
$\vec{a}$ va $\vec{b}$ vektorlar $\varphi = 2\pi/3$ burchak hosil qiladi. $|\vec{a}| = 1,|\vec{b}| = 2$ ekanini bilib, quyidagilarni hisoblang:
$\lbrack\overrightarrow{a} + 3\overrightarrow{b},3\overrightarrow{a} - \overrightarrow{b}\rbrack^{2}$
====
$\vec{a} = \{ 3; - 1; - 2\}$ va $\vec{b} = \{ 1;2; - 1\}$ vektorlar berilgan. Quyidagi vektor ko‘paytmalarning koordinatalarini toping:
$\left\lbrack \vec{a},\vec{b} \right\rbrack$.
====
$\vec{a} = \{ 3; - 1; - 2\}$ va $\vec{b} = \{ 1;2; - 1\}$ vektorlar berilgan. Quyidagi vektor ko‘paytmalarning koordinatalarini toping:
$\left\lbrack 2\vec{a} + \vec{b},\vec{b} \right\rbrack$.
====
$\vec{a} = \{ 3; - 1; - 2\}$ va $\vec{b} = \{ 1;2; - 1\}$ vektorlar berilgan. Quyidagi vektor ko‘paytmalarning koordinatalarini toping:
$\left\lbrack 2\vec{a} - \vec{b},2\vec{a} + \vec{b} \right\rbrack$.
====
$A (2; -1;2),B (1;2; 1) $ va $C (3;2;1)$ nuqtalar berilgan. Quyidagi vektor ko‘paytmalarning koordinatalarini toping:
$\lbrack\overline{AB},\overline{BC}\rbrack$.
====
$A (2; -1;2),B (1;2; 1) $ va $C (3;2;1) $ nuqtalar berilgan. Quyidagi vektor ko‘paytmalarning koordinatalarini toping:
$\lbrack\overline{BC} - 2\overline{CA},\overline{CB}\rbrack$.