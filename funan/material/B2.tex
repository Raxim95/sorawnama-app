\(\mathbb{R}^{2}\) kóplikte \(x = \left( x_{1},x_{2} \right)\) hám \(y = \left( y_{1},y_{2} \right)\) elementler ushın keltirilgen \(\rho(x,y) = \left| x_{1} - y_{1} \right| + \left| x_{2} - y_{2} \right|\) sáwlelendiriw metrika bolıwın kórsetiń.
Natural sanlar kópliginde \(\rho(n,m) = \left\{ \begin{matrix} 1 + \frac{1}{n + m},\ \ \ \text{eger}\ n \neq m \\ 0,\ \ \ \ \ \ \ \ \ \ \ \ \ \ \ \ \text{eger}\ n = m \end{matrix} \right.\) sáwlelendiriw metrika bolıwın kórsetiń.
\(\rho(x,y) = \sqrt{\sum_{i = 1}^{n}\left| x_{i} - y_{i} \right|^{2}}\), \(x,y \in \mathbb{R}^{n}\) sáwlelendiriwdiń metrika shártlerin qanaatlandırıwın tekseriń.
\(\rho(x,y) = (x - y)^{2}\), \(x,y\mathbb{\in R}\) sáwlelendiriw metrikanıń qaysı shártin qanaatlandırmawın anıqlań.
\(X = AC\lbrack 0,\pi\rbrack\), \(x(t) = \sin t\), \(y(t) = 0\) metrikalıq keńislikte \(x \in X\) hám \(y \in X\) elementler arasındaǵı aralıqtı tabıń.
Eger haqıyqıy sanlar arasındaǵı aralıq \(\rho(x,y) = |x - y|\) kórinisinde anıqlansa, onda bul aralıq metrika bolıwın kórsetiń.
Eger haqıyqıy sanlar arasındaǵı aralıq \(\rho(x,y) = \sqrt{|x - y|}\) kórinisinde anıqlansa, onda bul aralıq metrika bolıwın kórsetiń.
Úlken radiuslı shar ózinen kishirek bolǵan shardıń úlesi bolıwı múmkinbe? Mısal keltiriń.
\(\mathbb{R}^{3}\) kóplikte \(\rho(x,y) = \sum_{i = 1}^{3}{sgn\left| x_{i} - y_{i} \right|}\) metrika kiritilgen. Orayı \((0,1,2)\)noqatta bolǵan, radiusı 1 ge teń bolǵan sferanı sızıń.