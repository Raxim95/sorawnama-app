Berilgen \(f:\mathbb{R}^{2}\mathbb{\rightarrow R}\) funkciyanıń ólshewli ekenligin dálilleń: \(f(x,y) = \sign\left( \cos\pi\left( x^{2} + y^{2} \right) \right)\).
Berilgen \(f:\mathbb{R}^{2}\mathbb{\rightarrow R}\) funkciyanıń ólshewli ekenligin dálilleń: \(f(x,y) = \left( |x| + |y| \right)e^{\lbrack y\rbrack}\).
Berilgen \(f:\mathbb{R}^{2}\mathbb{\rightarrow R}\) funkciyanıń ólshewli ekenligin dálilleń: \(f(x,y) = \lbrack x\rbrack^{2} + \lbrack y\rbrack^{3}\).
Berilgen \(f:\mathbb{R}^{2}\mathbb{\rightarrow R}\) funkciyanıń ólshewli ekenligin dálilleń: \(f(x,y) = \ln\left( 1 + \left\lbrack x^{2} + y^{2} \right\rbrack \right)\).
Berilgen \(f:\mathbb{R \rightarrow R}\) funkciya ushın sonday \(g:\mathbb{R \rightarrow R}\) funkciyanı tabıń, nátiyjede derlik barlıq \(x\mathbb{\in R}\) noqatlar ushın \(f(x) = g(x)\) bolsın: \(f(x) = \left\{ \begin{matrix} \sin x,\ \ \ \ x\mathbb{\in Q} \\ 0,\ \ \ \ x\mathbb{\in R}\backslash\mathbb{Q} \end{matrix} \right.\ \).
Berilgen \(f:\mathbb{R \rightarrow R}\) funkciya ushın sonday \(g:\mathbb{R \rightarrow R}\) funkciyanı tabıń, nátiyjede derlik barlıq \(x\mathbb{\in R}\) noqatlar ushın \(f(x) = g(x)\) bolsın: \(f(x) = \left\{ \begin{matrix} \ln\left( 1 + |x| \right),\ \ \ \ e^{x}\mathbb{\in R}\backslash\mathbb{Q} \\ \sin x^{2},\ \ \ \ e^{x}\mathbb{\in Q} \end{matrix} \right.\ \).
Berilgen \(f:\mathbb{R \rightarrow R}\) funkciya ushın sonday \(g:\mathbb{R \rightarrow R}\) funkciyanı tabıń, nátiyjede derlik barlıq \(x\mathbb{\in R}\) noqatlar ushın \(f(x) = g(x)\) bolsın: \(f(x) = \left\{ \begin{matrix} \arctan x,\ \ \ \ x\mathbb{\in Z} \\ \pi,\ \ \ \ x\mathbb{\in R}\backslash\mathbb{Z} \end{matrix} \right.\ \).
Berilgen \(f:\mathbb{R \rightarrow R}\) funkciya ushın sonday \(g:\mathbb{R \rightarrow R}\) funkciyanı tabıń, nátiyjede derlik barlıq \(x\mathbb{\in R}\) noqatlar ushın \(f(x) = g(x)\) bolsın: \(f(x) = \left\{ \begin{matrix} x^{2},\ \ \ \ x\mathbb{\in Q} \\ 0,\ \ \ \ x\mathbb{\in R}\backslash\mathbb{Q} \end{matrix} \right.\ \).