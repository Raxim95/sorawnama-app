\(f(x) = \lbrack 2x\rbrack\), \(A = \lbrack 0,\ 2)\) berilgen \(f:A\rightarrow\mathbb{R}\) funkciyanıń ápiwayi ekenligin kórsetip, onıń integralın esaplań.
\(f(x) = \sign \ x\), \(A = \lbrack - 1,\ 3\rbrack\) berilgen \(f:A\rightarrow\mathbb{R}\) funkciyanıń ápiwayi ekenligin kórsetip, onıń integralın esaplań.
\(f(x) = \chi_{\lbrack 0,\ 1\rbrack\backslash\mathbb{Q}}(x)\), \(A = \lbrack - 1,\ 3\rbrack\) berilgen \(f:A\rightarrow\mathbb{R}\) funkciyanıń ápiwayi ekenligin kórsetip, onıń integralın esaplań.
\(f(x) = \lbrack x\rbrack + \sign \ x\), \(A = \lbrack - 1,\ 2\rbrack\) berilgen \(f:A\rightarrow\mathbb{R}\) funkciyanıń ápiwayi ekenligin kórsetip, onıń integralın esaplań.
\(f(x) = \sign \ x + \chi_{\lbrack 1,\ 2\rbrack}(x)\), \(A = \lbrack - 1,\ 4\rbrack\) berilgen \(f:A\rightarrow\mathbb{R}\) funkciyanıń ápiwayi ekenligin kórsetip, onıń integralın esaplań.