\documentclass{article}
\usepackage[fontsize=12pt]{fontsize}
\usepackage[utf8]{inputenc}
\usepackage[T2A]{fontenc}
% \usepackage{unicode-math}

\usepackage{array}
\usepackage[a4paper,
left=7mm,
right=5mm,
top=7mm,]{geometry}
\usepackage{amsmath}
% \usepackage{amssymbol}
\usepackage{amsfonts}
\usepackage{setspace}
\onehalfspacing



\renewcommand{\baselinestretch}{1} 

\everymath{\displaystyle}
\everydisplay{\displaystyle}
% \linespread{1.25}

\DeclareMathOperator{\sign}{sign}


\begin{document}

\pagenumbering{gobble}


\begin{tabular}{m{17cm}}
\textbf{1-variant}
\newline

\textbf{T1.} Ekinshi tártipli sızıq orayı (Oraylıq sızıqlar (ellips, giperbola), Oray koordinataları: simmetriya orayı) \\
\textbf{T2.} Tegislikte ekinshi tártipli sızıqlar (Ekinshi tártipli teńleme, Kvadrat kórinisindegi teńleme, Konik sızıqlar (konuslar kesimi)) \\
\textbf{A1.} Fokusları abscissa kósherinde jatqan hám koordinatalar basına salıstırģanda simmetriyalı bolǵan ellipstiń teńlemesin dúziń, bunda: kishi kósheri 6, direktrisaları arasındaǵı aralıq 13; \\
\textbf{A2.} Koordinatalar sistemasın túrlendirmesten, tómendegi teńlemelerdiń hár biri parabolanı anıqlawın kórsetiń hám parametrin tabıń: $9 x^2+24 x y+16 y^2-120 x+90 y=0$; \\
\textbf{A3.} Tómendegi sızıqlardan qaysı biri oraylıq (yaǵnıy birden-bir orayǵa iye), qaysı biri orayǵa iye emes, qaysı biri sheksiz kóp orayǵa iye ekenligin anıqlań: $25 x^2-10 x y+y^2+40 x-8 y+7=0$. \\
\textbf{B1.} Teń tárepli giperbolanıń ekssentrisiteti anıqlansın. \\
\textbf{B2.} Berilgen teńleme parabolik ekenligin kórsetiń; ápiwayı túrge keltiriń; qanday geometriyalıq obrazdı anlatıwın anıqlań, eski hám de jańa koordinata kósherlerine salıstırģanda sızılmada súwretleń:$9 x^2+12 x y+4 y^2-24 x-16 y+3=0$; \\
\textbf{B3.} ITECH túri, ólshemleri hám jaylasıwın anıqlań: $x^2+2 x y+y^2-8 x+4=0$; \\
\textbf{C1.} Giperbolanıń bir diametr tóbelerinen ótkerilgen urınbalar parallel bolıwın dálilleń. \\
\textbf{C2.} Tómendegi betliklerdiń kanonikalıq teńlemesi hám jaylasıwın anıqlań.: $2 x^2+10 y^2-2 z^2+12 x y+8 y z+12 x+4 y+8 z-1=0$. \\
\textbf{C3.} $m$ nıń qanday mánislerinde $y=-x+m$ sızıq: 1) $\frac{x^2}{20}+\frac{y^2}{5}=1$ ellipsti kesip ótedi; 2) ellipske urınadı; 3) ellipsti kesip ótpeydi. \\

\end{tabular}
\vspace{1cm}


\begin{tabular}{m{17cm}}
\textbf{2-variant}
\newline

\textbf{T1.} Bir gewekli giperboloid hám giperbolik paraboloidtıń tuwrı sızıqlı jasawshıları (Giperboloid, Giperbolik paraboloid, Sızıqlı jasawshılar) \\
\textbf{T2.} Parabola hám onıń kanonikalıq teńlemeleri (Fokus (bagdarlawshı noqat), Direktrisa (bagdarlawshı sızıq), Kósher (simmetriya kósheri)) \\
\textbf{A1.} $\frac{x^2}{100}+\frac{y^2}{225}=1$ ellips hám $y^2=24 x$ parabolanıń kesilisiw noqatların anıqlań. \\
\textbf{A2.} $y+6=0$ tegislik $\frac{x^2}{5}-\frac{y^2}{4}=6 z$ giperbolik paraboloidti parabola boyınsha kesip ótetuģının kórsetiń; parametrin hám tóbesin tabıń. \\
\textbf{A3.} Fokusları abscissa kósherinde, koordinatalar basına qarata simmetriyalı jaylasqan giperbolanıń teńlemesi dúzilsin, bunda: asimptotanıń teńlemeleri $y= \pm \frac{3}{4} x$ hám direktrisalarınıń teńlemeleri $x= \pm \frac{16}{5}$. \\
\textbf{B1.} Lagranj usılınan paydalanıp, teńlemelerdi kvadratlar qosındısı túrine keltirip, tómendegi betlerdiń kórinisin anıqlań: $4 x y+2 x+4 y-6 z-3=0$; \\
\textbf{B2.} Parabola tóbesiniń koordinataların, parametrin hám kósheriniń baǵıtın anıqlań: $x^2-6 x-4 y+29=0$, \\
\textbf{B3.} Berilgen teńlemeler oraylıq iymek sızıqlar ekenligin kórsetiń hám hárbir teńlemeni koordinata basın orayģa kóshiriń: $4 x^2+6 x y+y^2-10 x-10=0$; \\
\textbf{C1.} Giperbolanıń asimptotalarınan direktrisaları ajıratqan kesindiler (giperbolanıń orayınan esaplaganda) giperbolanıń haqıyqıy yarım kósherine teń ekenligin dálilleń. Bul qásiyetten paydalanıp, giperbolanıń direktrisaların jasań. \\
\textbf{C2.} $\frac{x^2}{a^2}+\frac{y^2}{b^2}=1$ ellipske ishley sızılgan kvadrat tárepiniń uzınlıǵın esaplań. \\
\textbf{C3.} $\frac{x^2}{9}+\frac{z^2}{4}=2 y$ elliptik paraboloid $2 x-2 y-z-10=0$ tegislik penen bir ulıwma noqatqa iye ekenligin dálilleń hám onıń koordinataların tabıń. \\

\end{tabular}
\vspace{1cm}


\begin{tabular}{m{17cm}}
\textbf{3-variant}
\newline

\textbf{T1.} Ekinshi tártipli betlik orayı, urınba tegisligi hám diametral tegisligi (Oray, Urınba tegislik, Diametral tegislik.) \\
\textbf{T2.} Ekinshi tártipli sızıqlardıń ulıwma teńlemesin invariantlar járdeminde kanonikalıq túrge keltiriw \\
\textbf{A1.} Koordinatalar sistemasın túrlendirmesten tómendegi teńlemelerdiń hár biri birden-bir noqattı anıqlawın kórsetiń hám onıń koordinataların tabıń: $5 x^2+4 x y+y^2-6 x-2 y+2=0$; \\
\textbf{A2.} Berilgen teńleme qanday iymek sızıq ekenligin tabıń: $y=-3 \sqrt{x^2+1}$; \\
\textbf{A3.} Tómendegi sızıqlardan qaysı biri oraylıq (yaǵnıy birden-bir orayǵa iye), qaysı biri orayǵa iye emes, qaysı biri sheksiz kóp orayǵa iye ekenligin anıqlań: $x^2-6 x y+9 y^2-12 x+36 y+20=0$; \\
\textbf{B1.} Ellips fokuslarınıń birinen úlken kósheri tóbelerine shekemgi aralıqlar sáykes túrde 7 hám 1 ge teń. Bul ellipstiń tenlemesin dúziń. \\
\textbf{B2.} Berilgen sızıqlar oraylıq ekenligin kórsetiń hám hárbir iymek sızıq ushın orayınıń koordinataların tabıń: $2 x^2-6 x y+5 y^2+22 x-36 y+11=0$. \\
\textbf{B3.} ITECH túri, ólshemleri hám jaylasıwın anıqlań: $7 x^2+16 x y-23 y^2-14 x-16 y-218=0$; \\
\textbf{C1.} Berilgen $y=k x+b$ tuwri sızıqqa parallel hám $y^2=2 p x$ parabolaga urinatuģın tuwri sızıqtıń teńlemesin jazıń. \\
\textbf{C2.} Giperbolanıń asimptotaların tabıń: $10 x^2+21 x y+9 y^2-41 x-39 y+4=0$. \\
\textbf{C3.} Elliptik túrdegi ($\delta>0$) teńleme $a_{11}$ hám $\Delta$ lardıń hár qıylı belgige iye sanlar bolǵanda ǵana ellipsti anıqlawın dálilleń. \\

\end{tabular}
\vspace{1cm}


\begin{tabular}{m{17cm}}
\textbf{4-variant}
\newline

\textbf{T1.} Parabola hám onıń kanonikalıq teńlemeleri (Fokus (bagdarlawshı noqat), Direktrisa (bagdarlawshı sızıq), Kósher (simmetriya kósheri)) \\
\textbf{T2.} Ekinshi tártipli betliklerdiń kanonikalıq teńlemeleri (Ellipsoid, Giperboloid (1 gewekli), Giperboloid (2 gewekli)) \\
\textbf{A1.} Diskriminantın esaplaw arqalı tómendegi teńlemelerdiń hár biriniń tipin anıqlań: $5 x^2+14 x y+11 y^2+12 x-7 y+19=0$; \\
\textbf{A2.} Koordinatalar sistemasın túrlendirmesten, tómendegi teńlemelerdiń hár biri parabolanı anıqlawın kórsetiń hám parametrin tabıń: $x^2-2 x y+y^2+6 x-14 y+29=0$; \\
\textbf{A3.} $y^2=6 x$ parabola direktrisası teńlemesin dúziń. \\
\textbf{B1.} $\frac{x^2}{16}-\frac{y^2}{9}=1$ giperbolada fokal radiusları óz ara perpendikulyar bolgan noqat tabılsın. \\
\textbf{B2.} Berilgen teńleme parabolik ekenligin kórsetiń; ápiwayı túrge keltiriń; qanday geometriyalıq obrazdı anlatıwın anıqlań, eski hám de jańa koordinata kósherlerine salıstırģanda sızılmada súwretleń: $x^2-2 x y+y^2-12 x+12 y-14=0$ \\
\textbf{B3.} Ellipstiń ekssentrisiteti $\varepsilon=\frac{1}{3}$, onıń orayı koordinatalar bası menen ústpe-úst túsedi, fokuslarınan biri $ (-2; 0) $. Abcissası 2 ge teń bolǵan ellipstiń $M_1$ noqatınan berilgen fokusqa sáykes direktrisaģa shekem bolǵan aralıqtı tabıń. \\
\textbf{C1.} $4 x^2-4 x y+y^2+6 x+1=0$ ETIS teńlemesi berilgen. Múyesh koefficienti $k$ tiń qanday mánislerinde $y=kx$ tuwrı sızıq: 1) bul iymek sızıqtı bir noqatta kesip ótiwi; 2) urınadı; 3) eki noqatta kesip ótiwin; 4) bul tuwrı menen ulıwma noqatqa iye bolmaytuģının anıqlań. \\
\textbf{C2.} $y^2=2 p x$ parabolaǵa onıń $M_1\left(x_1; y_1\right) $ noqatındaǵı urınbasınıń teńlemesin dúziń. \\
\textbf{C3.} Ekinshi dárejeli teńleme tek hám tek $\Delta=0$ bolǵanda ǵana aynıǵan iymek sızıq teńlemesi bolatuģının dálilleń. \\

\end{tabular}
\vspace{1cm}


\begin{tabular}{m{17cm}}
\textbf{5-variant}
\newline

\textbf{T1.} Ekinshi tártipli sızıqqa urınba, túyinles diametri teńlemesi (Urınba teńlemesi, Túyinles diametr: oraydan ótiwshi simmetriya kósherleri) \\
\textbf{T2.} Ekinshi tártipli sızıq hám tuwrı sızıqtıń óz ara jaylasıwı (Kesilisiw noqatları, Urınba (urınıw) jaģdayı) \\
\textbf{A1.} $\frac{x^2}{100}+\frac{y^2}{36}=1$ ellipsinde jaylasqan hám oń fokusına shekemgi aralıǵı 14 ke teń noqattı tabıń. \\
\textbf{A2.} Berilgen teńleme menen qanday iymek sızıq anıqlanıwın tabıń: $\left\{\begin{array}{l}\frac{x^2}{.4}+\frac{y^2}{9}-\frac{z^2}{36}=1, \\ 9 x-6 y+2 z-28=0,\end{array}\right.$ \\
\textbf{A3.} Fokusları abscissa kósherinde, koordinatalar basına salıstırģanda simmetriyalı jaylasqan giperbolanıń teńlemesin dúziń, bunda: $M_1 (6;-1) $ hám $M_2 (-8; 2 \sqrt{2}) noqatlar $ giperbolaga tiyisli; \\
\textbf{B1.} Lagranj usılınan paydalanıp, teńlemelerdi kvadratlar qosındısı túrine keltirip, tómendegi betlerdiń kórinisin anıqlań: $2 x^2+y^2+2 z^2-2 x y-2 y z+x-4 y-3 z+2=0$; \\
\textbf{B2.} Parabola tóbesiniń koordinataların, parametrin hám kósheriniń baǵıtın anıqlań: $y^2-10 x-2 y-19=0$; \\
\textbf{B3.} Giperbolanıń asimptotaları arasındaǵı múyeshin tabıń, eger: ekssentrisiteti $e=2$; \\
\textbf{C1.} Ulıwma fokusqa hám ústpe-úst túsken, biraq qarama-qarsı baǵıtlangan kósherlerge iye bolǵan parabolalardıń tuwrı múyesh astında kesilisiwin dálilleń. \\
\textbf{C2.} Giperbolanıń asimptotaların tabıń: $3 x^2+2 x y-y^2+8 x+10 y-14=0$; \\
\textbf{C3.} Hár qanday parabolik teńleme $ (\alpha x+\beta y) ^2+2a_{13}x+2a_{23}y+a_{33}=0$ kórinisinde jazılıwı múmkinligin dálilleń. Sonday-aq, elliptikalıq hám giperbolikalıq teńlemelerdi bunday kóriniste jazıp bolmaytuģının dálilleń. \\

\end{tabular}
\vspace{1cm}


\begin{tabular}{m{17cm}}
\textbf{6-variant}
\newline

\textbf{T1.} Ekinshi tártipli betliklerdiń ulıwma teńlemeleri (Ulıwma teńleme) \\
\textbf{T2.} Tegislikte ekinshi tártipli sızıqlar (Ekinshi tártipli teńleme, Kvadrat kórinisindegi teńleme, Konik sızıqlar (konuslar kesimi)) \\
\textbf{A1.} $y^2=-8 x$ parabola fokusınıń koordinataların anıqlań. \\
\textbf{A2.} $x-2=0$ tegislik $\frac{x^2}{16}+\frac{y^2}{12}+\frac{z^2}{4}=1$ ellipsoidti ellips boyınsha kesip ótetuģının kórsetiń; onıń yarım kósherleri hám tóbelerin tabıń. \\
\textbf{A3.} Tómendegi sızıqlardan qaysı biri oraylıq (yaǵnıy birden-bir orayǵa iye), qaysı biri orayǵa iye emes, qaysı biri sheksiz kóp orayǵa iye ekenligin anıqlań:  $x^2-2 x y+y^2-6 x+6 y-3=0$; \\
\textbf{B1.} Berilgen teńlemeni ápiwayı túrge keltiriń; tipin anıqlań; qanday geometriyalıq obrazdı ańlatıwın anıqlań, eski hám de jańa koordinata kósherlerine qarata sızılmada súwretleń: $5 x^2-6 x y+5 y^2+8=0$. \\
\textbf{B2.} Berilgen teńleme parabolik ekenligin kórsetiń; ápiwayı túrge keltiriń; qanday geometriyalıq obrazdı anlatıwın anıqlań, eski hám de jańa koordinata kósherlerine salıstırģanda sızılmada súwretleń: $9 x^2+24 x y+16 y^2-18 x+226 y+209=0$; \\
\textbf{B3.} $A (5;9) $ noqattan $y^2=5x$ parabolaǵa júrgizilgen urınbalardıń urınıw noqatların tutastırıwshı xordanıń teńlemesin dúziń. \\
\textbf{C1.} $\frac{x^2}{a^2}-\frac{y^2}{b^2}=1$ giperbolanıń fokuslarınan urınbasına shekemgi aralıqlardıń kóbeymesin tabıń. \\
\textbf{C2.} $A x+B y+C=0$ tuwrı sızıqtıń $\frac{x^2}{a^2}+\frac{y^2}{b^2}=1$, ellipske urınba bolıwı ushın zárúrli hám jeterli shárti tabılsın. \\
\textbf{C3.} $m$ hám $n$ tiń qanday mánislerinde $m x^2+12 x y+9 y^2+4 x+n y-13=0$ teńleme: 1) oraylıq sızıqtı; 2) orayga iye bolmaǵan sızıq; 3) sheksiz kóp orayǵa iye bolǵan sızıqtı ańlatadı. \\

\end{tabular}
\vspace{1cm}


\begin{tabular}{m{17cm}}
\textbf{7-variant}
\newline

\textbf{T1.} Parabola hám onıń kanonikalıq teńlemeleri (Fokus (bagdarlawshı noqat), Direktrisa (bagdarlawshı sızıq), Kósher (simmetriya kósheri)) \\
\textbf{T2.} Ekinshi tártipli sızıqlardıń ulıwma teńlemeleri (Ulıwma teńleme) \\
\textbf{A1.} Koordinatalar sistemasın túrlendirmesten, tómendegi teńlemelerdiń hár biri parabolanı anıqlawın kórsetiń hám parametrin tabıń: $9 x^2-24 x y+16 y^2-54 x-178 y+181=0$; \\
\textbf{A2.} Koordinatalar sistemasın túrlendirmesten tómendegi teńlemelerdiń hár biri kesilisiwshi eki tuwrını anıqlawın kórsetiń hám onıń koordinataların tabıń: $x^2+4 x y+3 y^2-6 x-12 y+9=0$. \\
\textbf{A3.} Fokusları abscissa kósherinde jatqan hám koordinatalar basına qarata simmetriyalı bolǵan ellipstiń teńlemesin dúziń, bunda: $M_1 (-\sqrt{5}; 2)$ noqatı ellipske tiyisli hám onıń direktrisaları arasındaǵı aralıq 10 ģa teń. \\
\textbf{B1.} Lagranj usılınan paydalanıp, teńlemelerdi kvadratlar qosındısı túrine keltirip, tómendegi betlerdiń kórinisin anıqlań: $2 x^2+y^2+2 z^2-2 x y-2 y z+4 x-2 y=0$; \\
\textbf{B2.} Berilgen sızıqlar oraylıq ekenligin kórsetiń hám hárbir iymek sızıq ushın orayınıń koordinataların tabıń: $3x^2+5xy+y^2-8x-11y-7=0$. \\
\textbf{B3.} Úlken kósheri 2 birlikke teń, fokusları $F_1 (0,1), F_2 (1,0) $ noqatlarda bolgan ellipstiń teńlemesin dúziń. \\
\textbf{C1.} Tómendegi betliklerdiń kanonikalıq teńlemesi hám jaylasıwın anıqlań.: $5 x^2-y^2+z^2+4 x y+6 x z+2 x+4 y+6 z-8=0$. \\
\textbf{C2.} $m$ niń qanday mánislerinde $x+mz-1=0$ tegislik tómendegi $x^2+y^2z^2=1$ eki gewekli giperboloidti a) ellips boyınsha, b) giperbola boyınsha kesedi? \\
\textbf{C3.} Eger giperbolanıń yarım kósherleri $a$ hám $b$, orayı $C\left(x_0; y_0\right) $ hám fokuslar tómendegi tuwrı sızıqta jaylasqan: 1) $O x$ kósherine parallel; 2) $O y$ kósherine parallel bolsa, onıń teńlemesin dúziń. \\

\end{tabular}
\vspace{1cm}


\begin{tabular}{m{17cm}}
\textbf{8-variant}
\newline

\textbf{T1.} Ekinshi tártipli betliklerdiń ulıwmalıq teńlemesin kanonikalıq túrge keltiriw (invariantlar járdeminde) \\
\textbf{T2.} Tegislikte ekinshi tártipli sızıqlar (Ekinshi tártipli teńleme, Kvadrat kórinisindegi teńleme, Konik sızıqlar (konuslar kesimi)) \\
\textbf{A1.} Koordinatalar sistemasın túrlendirmesten, tómendegi teńlemeler menen qanday geometriyalıq obrazdı anıqlanıwın tabıń: $17 x^2-18 x y-7 y^2+34 x-18 y+7=0$; \\
\textbf{A2.} $y^2+z^2=x$ elliptik paraboloidtıń $x+2 y-z=0$ tegislik penen kesilisiwiniń koordinata tegisliklerindegi proekciyalarınıń teńlemelerin tabıń. \\
\textbf{A3.} Fokusları ordinata kósherinde jatqan hám koordinatalar basına qarata simmetriyalı bolǵan ellipstiń teńlemesin dúziń, bunda: kishi kósheri 16, a ekssentrisiteti $\varepsilon=\frac{3}{5}$; \\
\textbf{B1.} Lagranj usılınan paydalanıp, teńlemelerdi kvadratlar qosındısı túrine keltirip, tómendegi betlerdiń kórinisin anıqlań: $x^2+y^2-3 z^2-2 x y-6 x z-6 y z+2 x+2 y+4 z=0$; \\
\textbf{B2.} $5 x^2-3 x y+y^2-3 x+2 y-5=0$ sızıqtıń $x-2 y-1=0$ tuwri sızıq penen kesilisiwinen payda bolgan xordanıń ortasınan ótetuģin diametr teńlemesi jazılsın. \\
\textbf{B3.} Berilgen sızıqlar oraylıq ekenligin kórsetiń hám hárbir iymek sızıq ushın orayınıń koordinataların tabıń:$9 x^2-4 x y-7 y^2-12=0$; \\
\textbf{C1.} $\frac{x^2}{30}+\frac{y^2}{24}=1$ ellipske $4x-2y+23=0$ parallel bolǵan urınbalardı júrgiziń hám olar arasındaģı aralıqtı esaplań. \\
\textbf{C2.} Kósherleri óz ara perpendikulyar bolǵan eki parabola tórt noqatta kesilisse, bul noqatlar bir sheńberde jatıwın dálilleń. \\
\textbf{C3.} Berilgen teńleme kanonikalıq kóriniske keltirilsin; tipi anıqlansın; qanday geometriyalıq obrazdı anlatıwı anıqlansın; eski hám jana koordinatalar sistemasında geometriyalıq obrazı súwretlensin: $7 x^2+60 x y+32 y^2-14 x-60 y+7=0$; \\

\end{tabular}
\vspace{1cm}


\begin{tabular}{m{17cm}}
\textbf{9-variant}
\newline

\textbf{T1.} Ekinshi tártipli betliklerdiń kanonikalıq teńlemeleri (Paraboloid (ellipstik), Paraboloid (giperbolik), Konus, Cilindr) \\
\textbf{T2.} Ekinshi tártipli sızıqqa urınba, túyinles diametri teńlemesi (Urınba teńlemesi, Túyinles diametr: oraydan ótiwshi simmetriya kósherleri) \\
\textbf{A1.} Koordinatalar sistemasın túrlendirmesten, tómendegi teńlemelerdiń hár biri parabolanı anıqlawın kórsetiń hám parametrin tabıń: $9 x^2-6 x y+y^2-50 x+50 y-275=0$. \\
\textbf{A2.} $\frac{x^2}{25}-\frac{y^2}{144}=1$ giperbolanıń fokusların tabıń. \\
\textbf{A3.} Parabolanıń teńlemesin dúziń, eger: parabola $O y$ kósherine qarata simmetriyalı bolıp, $M (6,-2) $ noqatınan hám koordinatalar basınan ótedi. \\
\textbf{B1.} Berilgen teńleme parabolik ekenligin kórsetiń; ápiwayı túrge keltiriń; qanday geometriyalıq obrazdı anlatıwın anıqlań, eski hám de jańa koordinata kósherlerine salıstırģanda sızılmada súwretleń:$9 x^2-24 x y+16 y^2-20 x+110 y-50=0$; \\
\textbf{B2.} Parabola tóbesiniń koordinataların, parametrin hám kósheriniń baǵıtın anıqlań: $y^2+8 x-16=0$, \\
\textbf{B3.} Giperbolanıń yarım kósherlerin tabıń, eger: asimptotaları $y= \pm 2 x$ tenlemeleri menen berilgen hám fokusları oraydan 5 birlik aralıqta; \\
\textbf{C1.} $\frac{x^2}{a^2}+\frac{y^2}{b^2}=1$ ellipstiń $M_1 (x_1; y_1) $ noqatındaǵı urınbasınıń teńlemesin dúziń. \\
\textbf{C2.} Eki gewekli $\frac{x^2}{3}+\frac{y^2}{4}-\frac{z^2}{25}=-1$ giperboloid $5 x+2 z+5=0$ tegislik penen bir ulıwma noqatqa iye ekenligin dálilleń hám onıń koordinataların tabıń. \\
\textbf{C3.} Giperbolanıń asimptotaların tabıń: $3 x^2+7 x y+4 y^2+5 x+2 y-6=0$; \\

\end{tabular}
\vspace{1cm}


\begin{tabular}{m{17cm}}
\textbf{10-variant}
\newline

\textbf{T1.} Tegislikte ekinshi tártipli sızıqlar (Ekinshi tártipli teńleme, Kvadrat kórinisindegi teńleme, Konik sızıqlar (konuslar kesimi)) \\
\textbf{T2.} Ekinshi tártipli sızıq orayı (Oraylıq sızıqlar (ellips, giperbola), Oray koordinataları: simmetriya orayı) \\
\textbf{A1.} Tómendegi sızıqlardan qaysı biri oraylıq (yaǵnıy birden-bir orayǵa iye), qaysı biri orayǵa iye emes, qaysı biri sheksiz kóp orayǵa iye ekenligin anıqlań:  $4 x^2+4 x y+y^2-8 x-4 y-21=0$; \\
\textbf{A2.} Koordinatalar sistemasın túrlendirmesten, tómendegi teńlemelerdiń hár biri parabolanı anıqlawın kórsetiń hám parametrin tabıń: $9 x^2+24 x y+16 y^2-120 x+90 y=0$; \\
\textbf{A3.} $z+1=0$ tegislik bir qabatlı $\frac{x^2}{32}-\frac{y^2}{18}+\frac{z^2}{2}=1$ giperboloidti giperbola boyınsha kesip ótetuģının kórsetiń; onıń yarım kósherleri hám tóbelerin tabıń. \\
\textbf{B1.} Ellipstegi ekssentrisitetti anıqlań, eger: direktrisalar arasındaǵı aralıq fokuslar arasındaǵı aralıqtan úsh ese úlken bolsa; \\
\textbf{B2.} Parabola tóbesiniń koordinataların, parametrin hám kósheriniń baǵıtın anıqlań: $y=x^2+6 x$. \\
\textbf{B3.} $\frac{x^2}{16}-\frac{y^2}{64}=1$ giperbolaǵa $10 x-3 y+9=0$ tuwrısına parallel bolǵan urınbalardıń teńlemelerin dúziń. \\
\textbf{C1.} $m$ hám $n$ tiń qanday mánislerinde $m x^2+12 x y+9 y^2+4 x+n y-13=0$ teńleme: 1) oraylıq sızıqtı; 2) orayga iye bolmaǵan sızıq; 3) sheksiz kóp orayǵa iye bolǵan sızıqtı ańlatadı. \\
\textbf{C2.} $y^2=2 p x$ parabolaǵa $y=k x+b$ tuwrı sızıq urınıw shártin keltirip shigarıń. \\
\textbf{C3.} $\frac{x^2}{a^2}-\frac{y^2}{b^2}=1$ giperbola hám onıń qanday da bir urınbası berilgen: $P$-urınbasınıń $O x$ kósheri menen kesilisiw noqatı, $Q$ - urınba noqatınıń sol kósherdegi proekciyası. $O P \cdot O Q=a^2$ ekenligin dálilleń. \\

\end{tabular}
\vspace{1cm}


\begin{tabular}{m{17cm}}
\textbf{11-variant}
\newline

\textbf{T1.} Ekinshi tártipli betliklerdiń ulıwmalıq teńlemesin kanonikalıq túrge keltiriw (invariantlar járdeminde) \\
\textbf{T2.} Bir gewekli giperboloid hám giperbolik paraboloidtıń tuwrı sızıqlı jasawshıları (Giperboloid, Giperbolik paraboloid, Sızıqlı jasawshılar) \\
\textbf{A1.} $\frac{x^2}{32}+\frac{y^2}{18}=1$ ellipsiniń $M (4,3) $ noqatında júrgizilgen urinbasınıń teńlemesin dúziń. \\
\textbf{A2.} Berilgen teńleme qanday iymek sızıq ekenligin tabıń: $x=-\frac{4}{3} \sqrt{y^2+9} ;$ \\
\textbf{A3.} Diskriminantın esaplaw arqalı tómendegi teńlemelerdiń hár biriniń tipin anıqlań: $2 x^2+10 x y+12 y^2-7 x+18 y-15=0$; \\
\textbf{B1.} Berilgen teńlemeniń tipin anıqlań, koordinata kósherlerin parallel kóshiriw arqalı ápiwayı túrge keltiriń; qanday geometriyalıq obrazdı ańlatıwın anıqlań, eski hám jańa koordinata kósherlerine salıstırģanda sızılmada súwretleń: $9 x^2-16 y^2-54 x-64 y-127=0$; \\
\textbf{B2.} $\varepsilon=\frac{2}{3}$ ellipsiniń ekssentrisiteti, $M$ ellips noqatınıń fokal radiusı 10 ģa teń. $M$ noqattan usı fokusqa sáykes direktrisaǵa shekem bolǵan aralıqtı esaplań. \\
\textbf{B3.} Berilgen teńlemelerdiń parabolik ekenligin kórsetiń hám olardıń hár birin $(\alpha x+\beta y)^2+2 a_{13} x+2 a_{23} y+a_{33}=0$ kórinisinde jazıń:  $16 x^2+16 x y+4 y^2-5 x+7 y=0$; \\
\textbf{C1.} Tómendegi betliklerdiń kanonikalıq teńlemesi hám jaylasıwın anıqlań.: $5 x^2+2 y^2+5 z^2-4 x y-2 x y-4 y z+10 x-4 y-2 z+4=0$; \\
\textbf{C2.} $\frac{x^2}{a^2}-\frac{y^2}{b^2}=1$ giperbolanıń fokusınan asimptotasına shekemgi aralıq $b$ qa teń ekenligin dálilleń. \\
\textbf{C3.} $y^2=4 x$ parabola menen $\frac{x^2}{8}+\frac{y^2}{2}=1$ ellipstiń uliwma urinbaların anıqlań. \\

\end{tabular}
\vspace{1cm}


\begin{tabular}{m{17cm}}
\textbf{12-variant}
\newline

\textbf{T1.} Ekinshi tártipli sızıqlardıń ulıwma teńlemeleri (Ulıwma teńleme) \\
\textbf{T2.} Parabola hám onıń kanonikalıq teńlemeleri (Fokus (bagdarlawshı noqat), Direktrisa (bagdarlawshı sızıq), Kósher (simmetriya kósheri)) \\
\textbf{A1.} Parabolanıń teńlemesin dúziń, eger: parabolanıń fokusi $ (0,2) $ noqatında hám tóbesi koordinatalar basında jatadı; \\
\textbf{A2.} Tómendegi sızıqlardan qaysı biri oraylıq (yaǵnıy birden-bir orayǵa iye), qaysı biri orayǵa iye emes, qaysı biri sheksiz kóp orayǵa iye ekenligin anıqlań:  $4 x^2-6 x y-9 y^2+3 x-7 y+12=0$. \\
\textbf{A3.} Tóbesi koordinatalar basında bolǵan parabolanıń teńlemesin dúziń, bunda: parabola $Ox$ kósherine simmetriyalı jaylasqan hám $A (9; 6) $ noqatınan ótedi; \\
\textbf{B1.} Parallel kóshiriw hám burıw túrlendiriwleri yamasa aǵzalardı gruppalaw járdeminde tómendegi betliklerdiń kórinisi hám jaylasıwı anıqlansın: $2 x y+z^2-2 z+1=0$; \\
\textbf{B2.} Berilgen teńlemeler oraylıq iymek sızıqlar ekenligin kórsetiń hám hárbir teńlemeni koordinata basın orayģa kóshiriń: $3x^2-6xy+2y^2-4x+2y+1=0$. \\
\textbf{B3.} Berilgen teńlemeni ápiwayı túrge keltiriń; tipin anıqlań; qanday geometriyalıq obrazdı ańlatıwın anıqlań, eski hám de jańa koordinata kósherlerine qarata sızılmada súwretleń: $32 x^2+52 x y-7 y^2+180=0$; \\
\textbf{C1.} Berilgen teńleme kanonikalıq kóriniske keltirilsin; tipi anıqlansın; qanday geometriyalıq obrazdı anlatıwı anıqlansın; eski hám jana koordinatalar sistemasında geometriyalıq obrazı súwretlensin: $41 x^2+24 x y+34 y^2+34 x-112 y+129=0$; \\
\textbf{C2.} Hár qanday parabolik teńleme ushın $a_{11}$ hám $a_{22}$ koefficientler hár qıylı belgige iye sanlar bola almaytuģının hám olar bir waqıtta nolge aylana almaytuģının dálilleń. \\
\textbf{C3.} $\frac{x^2}{a^2}+\frac{y^2}{b^2}=1$ ellipstiń $F(c, 0)$ fokusı arqalı úlken kósherine perpendikulyar bolǵan xorda ótkerilgen. Bul xordıń uzınlıǵın tabıń. \\

\end{tabular}
\vspace{1cm}


\begin{tabular}{m{17cm}}
\textbf{13-variant}
\newline

\textbf{T1.} Tegislikte ekinshi tártipli sızıqlar (Ekinshi tártipli teńleme, Kvadrat kórinisindegi teńleme, Konik sızıqlar (konuslar kesimi)) \\
\textbf{T2.} Ekinshi tártipli betliklerdiń kanonikalıq teńlemeleri (Paraboloid (ellipstik), Paraboloid (giperbolik), Konus, Cilindr) \\
\textbf{A1.} Koordinatalar sistemasın túrlendirmesten tómendegi teńlemelerdiń hár biri giperbolanı anıqlawın kórsetiń hám onıń koordinataların tabıń: $3 x^2+4 x y-12 x+16=0$; \\
\textbf{A2.} Fokusları ordinata kósherinde jatqan hám koordinatalar basına salıstırģanda simmetriyalı bolǵan ellipstiń teńlemesin dúziń, bunda: direktrisaları arasındaģı qashıqlıq $frac{2}{3}$ hám ekssentrisiteti $frac{3}{4}$. \\
\textbf{A3.} Koordinatalar sistemasın túrlendirmesten, tómendegi teńlemelerdiń hár biri parabolanı anıqlawın kórsetiń hám parametrin tabıń: $9 x^2-6 x y+y^2-50 x+50 y-275=0$. \\
\textbf{B1.} Eger giperbolanıń ekssentrisiteti $\varepsilon=\sqrt{5}$, fokusı $F (2;-3) $ oǵan sáykes direktrisasınıń teńlemesi $3 x-y+3=0$ belgili bolsa, onıń teńlemesin dúziń \\
\textbf{B2.} Eger parabolanıń fokusı $F (7; 2) $ hám direktrisa $x-5=0$ teńlemesi berilgen bolsa, onıń teńlemesin dúziń. \\
\textbf{B3.} Berilgen teńlemeler oraylıq iymek sızıqlar ekenligin kórsetiń hám hárbir teńlemeni koordinata basın orayģa kóshiriń: $6 x^2+4 x y+y^2+4 x-2 y+2=0$; \\
\textbf{C1.} Berilgen teńleme kanonikalıq kóriniske keltirilsin; tipi anıqlansın; qanday geometriyalıq obrazdı anlatıwı anıqlansın; eski hám jana koordinatalar sistemasında geometriyalıq obrazı súwretlensin: $19 x^2+6 x y+11 y^2+38 x+6 y+29=0$; \\
\textbf{C2.} $A x+B y+C=0$ tuwrı sızıq qanday zárúrli hám jeterli shárt orınlanǵanda $\frac{x^2}{a^2}+\frac{y^2}{b^2}=1$ ellips penen 1) kesilisedi; 2) kesilispeydi. \\
\textbf{C3.} Tómendegi betliklerdiń kanonikalıq teńlemesi hám jaylasıwın anıqlań.: $2 x^2+2 y^2+3 z^2+4 x y+2 x z+2 y z-4 x+6 y-2 z+3=0$. \\

\end{tabular}
\vspace{1cm}


\begin{tabular}{m{17cm}}
\textbf{14-variant}
\newline

\textbf{T1.} Ekinshi tártipli sızıqlardıń ulıwma teńlemesin invariantlar járdeminde kanonikalıq túrge keltiriw \\
\textbf{T2.} Ekinshi tártipli sızıq hám tuwrı sızıqtıń óz ara jaylasıwı (Kesilisiw noqatları, Urınba (urınıw) jaģdayı) \\
\textbf{A1.} Berilgen teńleme menen qanday iymek sızıq anıqlanıwın tabıń: $\left\{\begin{array}{l}\frac{x^2}{4}-\frac{y^2}{3}=2 z \\ x-2 y+2=0 ;\end{array}\right.$ \\
\textbf{A2.} Tómendegi sızıqlardan qaysı biri oraylıq (yaǵnıy birden-bir orayǵa iye), qaysı biri orayǵa iye emes, qaysı biri sheksiz kóp orayǵa iye ekenligin anıqlań: $4 x^2-4 x y+y^2-6 x+8 y+13=0$; \\
\textbf{A3.} $A (-3;-5) $ noqatı fokusı $F (-1;-4) $ bolǵan ellipste jatadı hám oǵan sáykes direktrisa $x-2=0$ teńlemesi menen berilgen. Usi ellipstiń teńlemesin dúziń. \\
\textbf{B1.} Berilgen teńlemelerdiń parabolik ekenligin kórsetiń hám olardıń hár birin $(\alpha x+\beta y)^2+2 a_{13} x+2 a_{23} y+a_{33}=0$ kórinisinde jazıń: $9 x^2-6 x y+y^2-x+2 y-14=0$; \\
\textbf{B2.} Parallel kóshiriw hám burıw túrlendiriwleri yamasa aǵzalardı gruppalaw járdeminde tómendegi betliklerdiń kórinisi hám jaylasıwı anıqlansın: $x^2+2 y^2-3 z^2+2 x+4 y-6 z=0$; \\
\textbf{B3.} Tómendegilerdi bilgen halda ellips teńlemesin dúziń: onıń fokusları $F_1 (1; 3), F_2 (3; 1) $ hám direktrisalar arasındaģı aralıq $12 \sqrt{2}$ qa teń. \\
\textbf{C1.} Giperbolanıń asimptotaların tabıń: $x^2-3 x y-10 y^2+6 x-8 y=0$; \\
\textbf{C2.} $\frac{x^2}{a^2}-\frac{y^2}{b^2}=1$ giperbolaģa onıń $M_1\left(x_1; y_1\right) $ noqatındaǵı urınbasınıń teńlemesin dúziń. \\
\textbf{C3.} Giperbola asimptotalarınıń tenlemeleri $y= \pm \frac{1}{2} x$ hám urinbalardan biriniń teńlemesi. $5 x-6 y-8=0$ belgili bolsa, giperbola teńlemesin dúziń. \\

\end{tabular}
\vspace{1cm}


\begin{tabular}{m{17cm}}
\textbf{15-variant}
\newline

\textbf{T1.} Parabola hám onıń kanonikalıq teńlemeleri (Fokus (bagdarlawshı noqat), Direktrisa (bagdarlawshı sızıq), Kósher (simmetriya kósheri)) \\
\textbf{T2.} Ekinshi tártipli betliklerdiń ulıwma teńlemeleri (Ulıwma teńleme) \\
\textbf{A1.} Koordinatalar sistemasın túrlendirmesten, tómendegi teńlemelerdiń hár biri parabolanı anıqlawın kórsetiń hám parametrin tabıń: $x^2-2 x y+y^2+6 x-14 y+29=0$; \\
\textbf{A2.} Eki tóbesi $9 x^2+5 y^2=1$ ellipstiń fokuslarında, qalǵan eki tóbesi onıń kishi kósheriniń tóbelerinde jaylasqan tórtmúyeshliktiń maydanın esaplań. \\
\textbf{A3.} Koordinatalar sistemasın túrlendirmesten, tómendegi teńlemeler menen qanday geometriyalıq obrazdı anıqlanıwın tabıń: $2 x^2+3 x y-2 y^2+5 x+10 y=0$; \\
\textbf{B1.} Berilgen sızıqlar oraylıq ekenligin kórsetiń hám hárbir iymek sızıq ushın orayınıń koordinataların tabıń:$5 x^2+4 x y+2 y^2+20 x+20 y-18=0$; \\
\textbf{B2.} Berilgen teńlemeni ápiwayı túrge keltiriń; tipin anıqlań; qanday geometriyalıq obrazdı ańlatıwın anıqlań, eski hám de jańa koordinata kósherlerine qarata sızılmada súwretleń: $17 x^2-12 x y+8 y^2=0$; \\
\textbf{B3.} Berilgen teńlemelerdiń parabolik ekenligin kórsetiń hám olardıń hár birin $(\alpha x+\beta y)^2+2 a_{13} x+2 a_{23} y+a_{33}=0$ kórinisinde jazıń:  $9 x^2-42 x y+49 y^2+3 x-2 y-24=0$. \\
\textbf{C1.} $A\left(\frac{10}{3}; \frac{5}{3}\right)$ noqatınan $\frac{x2}{20}+\frac{y2}{5}=1$ ellipske urınbalar ótkerilgen. Olardıń teńlemelerin dúziń. \\
\textbf{C2.} $A x+B y+C=0$ tuwri sızıq $y^2=2 p x$ parabolaga urinıwı ushin zárúrli hám jeterli shártti tabıń. \\
\textbf{C3.} Múyesh koefficienti $k$ tiń qanday mánislerinde $y=kx+2$ tuwrısı: 1) $y^2=4x$ parabolanı kesip ótedi; 2) oǵan urınadı; 3) bul parabola sırtınan ótedi. \\

\end{tabular}
\vspace{1cm}


\begin{tabular}{m{17cm}}
\textbf{16-variant}
\newline

\textbf{T1.} Ekinshi tártipli sızıqlardıń ulıwma teńlemesin invariantlar járdeminde kanonikalıq túrge keltiriw \\
\textbf{T2.} Ekinshi tártipli betliklerdiń kanonikalıq teńlemeleri (Ellipsoid, Giperboloid (1 gewekli), Giperboloid (2 gewekli)) \\
\textbf{A1.} Tómendegi maglıwmatlar boyınsha giperbolanıń kanonikalıq teńlemesin dúziń: úlken kósheri $a=5$, kishi kósheri $b=3$; \\
\textbf{A2.} Berilgen teńleme menen qanday iymek sızıq anıqlanıwın tabıń: $\left\{\begin{array}{l}\frac{x^2}{3}+\frac{y^2}{6}=2 z, \\ 3 x-y+6 z-14=0\end{array}\right.$ \\
\textbf{A3.} Tóbesi koordinatalar basında bolǵan parabolanıń teńlemesin dúziń, bunda: parabola $Ox$ kósherine simmetriyalı jaylasqan hám $B (-1; 2) $ noqatınan ótedi; \\
\textbf{B1.} Ellipstiń ekssentrisiteti $\varepsilon=\frac{1}{2}$, onıń orayı koordinatalar bası menen ústpe-úst túsedi, direktrisalardan biri $x=16$ teńleme menen berilgen. Abcissası $-4$ ke teń bolǵan ellipstiń $M_1$ noqatınan berilgen direktrisa menen bir tárepleme fokusqa shekem bolǵan aralıqtı esaplań. \\
\textbf{B2.} Parabola tóbesi $A (-2;-1) $ hám onıń direktrisasınıń teńlemesi $x+2y-1=0$ berilgen. Bul parabolanıń teńlemesin dúziń. \\
\textbf{B3.} Tómendegilerdi bilgen halda giperbolanıń teńlemesin dúziń: onıń tóbeleri arasındaǵı aralıq 24 ke teń hám fokusları $F_1 (-10; 2), F_2 (16; 2) $; \\
\textbf{C1.} Parabolik teńleme $\Delta \neq 0$ bolǵanda hám tek sonda ǵana parabolanı anıqlaytuģının dálilleń. Bul jaǵdayda parabolanıń parametri $p=\sqrt{\frac{-\Delta}{ (a_{11}+a_{33}) ^3}}$ formula menen anıqlanıwın dálilleń. \\
\textbf{C2.} $4 x^2-4 x y+y^2+6 x+1=0$ ETIS teńlemesi berilgen. Múyesh koefficienti $k$ tiń qanday mánislerinde $y=kx$ tuwrı sızıq: 1) bul iymek sızıqtı bir noqatta kesip ótiwi; 2) urınadı; 3) eki noqatta kesip ótiwin; 4) bul tuwrı menen ulıwma noqatqa iye bolmaytuģının anıqlań. \\
\textbf{C3.} $m$ nıń qanday mánislerinde $x-2 y-2 z+m=0$ tegislik $\frac{x^2}{144}+\frac{y^2}{36}+\frac{z^2}{9}=1$ ellipsoidqa urınıwın anıqlań. \\

\end{tabular}
\vspace{1cm}


\begin{tabular}{m{17cm}}
\textbf{17-variant}
\newline

\textbf{T1.} Parabola hám onıń kanonikalıq teńlemeleri (Fokus (bagdarlawshı noqat), Direktrisa (bagdarlawshı sızıq), Kósher (simmetriya kósheri)) \\
\textbf{T2.} Ekinshi tártipli betlik orayı, urınba tegisligi hám diametral tegisligi (Oray, Urınba tegislik, Diametral tegislik.) \\
\textbf{A1.} Tómendegi sızıqlardan qaysı biri oraylıq (yaǵnıy birden-bir orayǵa iye), qaysı biri orayǵa iye emes, qaysı biri sheksiz kóp orayǵa iye ekenligin anıqlań: $4 x^2+5 x y+3 y^2-x+9 y-12=0$; \\
\textbf{A2.} $y+6=0$ tegislik $\frac{x^2}{5}-\frac{y^2}{4}=6 z$ giperbolik paraboloidti parabola boyınsha kesip ótetuģının kórsetiń; parametrin hám tóbesin tabıń. \\
\textbf{A3.} $C (-3; 2)$ eki koordinata kósherine urınıwshı ellipstiń orayı. Bul ellipstiń simmetriya kósherleri koordinata kósherlerine parallel ekenligin bilgen halda onıń teńlemesin dúziń. \\
\textbf{B1.} Parallel kóshiriw hám burıw túrlendiriwleri yamasa aǵzalardı gruppalaw járdeminde tómendegi betliklerdiń kórinisi hám jaylasıwı anıqlansın: $z=x^2+2 x y+y^2+1$; \\
\textbf{B2.} Berilgen teńleme parabolik ekenligin kórsetiń; ápiwayı túrge keltiriń; qanday geometriyalıq obrazdı anlatıwın anıqlań, eski hám de jańa koordinata kósherlerine salıstırģanda sızılmada súwretleń:$16 x^2-24 x y+9 y^2-160 x+120 y+425=0$. \\
\textbf{B3.} Giperbolanıń haqıyqıy kósherine perpendikulyar bolǵan hám giperbola fokusınan ótken xorda uzınlıǵın tabıń. \\
\textbf{C1.} Ellipstiń yarım kósherleri $a$, $b$ hám orayı $C\left(x_0; y_0\right) $ noqatında bolıp, simmetriya kósherleri koordinata kósherlerine parallel ekenligi belgili bolsa, onıń teńlemesin dúziń. \\
\textbf{C2.} Eger ekinshi dárejeli teńleme parabolik bolıp, $ (\alpha x+\beta y) ^2+2a_{13}x+2a_{23}y+a_{33}=0$ kórinisinde jazılsa, onıń shep tárepindegi diskriminant $\Delta=- (a_{13} \beta-a_{23} \alpha) ^2$ formula menen anıqlanıwın dálilleń. \\
\textbf{C3.} $\frac{x^2}{a^2}-\frac{y^2}{b^2}=1$ giperbolanıń fokuslarınan urınbasına shekemgi aralıqlardıń kóbeymesin tabıń. \\

\end{tabular}
\vspace{1cm}


\begin{tabular}{m{17cm}}
\textbf{18-variant}
\newline

\textbf{T1.} Ekinshi tártipli sızıqqa urınba, túyinles diametri teńlemesi (Urınba teńlemesi, Túyinles diametr: oraydan ótiwshi simmetriya kósherleri) \\
\textbf{T2.} Tegislikte ekinshi tártipli sızıqlar (Ekinshi tártipli teńleme, Kvadrat kórinisindegi teńleme, Konik sızıqlar (konuslar kesimi)) \\
\textbf{A1.} Tóbesi koordinatalar basında bolǵan parabolanıń teńlemesin dúziń, bunda: parabola oń yarım tegislikte hám $Ox$ kósherine simmetriyalı jaylasqan, hám parametri $p=3$; \\
\textbf{A2.} Tómendegi sızıqlardan qaysı biri oraylıq (yaǵnıy birden-bir orayǵa iye), qaysı biri orayǵa iye emes, qaysı biri sheksiz kóp orayǵa iye ekenligin anıqlań: $4 x^2-4 x y+y^2-12 x+6 y-11=0$; \\
\textbf{A3.} Koordinatalar sistemasın túrlendirmesten tómendegi teńlemelerdiń hár biri ellipsti anıqlawın kórsetiń hám onıń yarım kósherlerin tabıń: $41 x^2+24 x y+9 y^2+24 x+18 y-36=0$; \\
\textbf{B1.} Berilgen teńlemeniń tipin anıqlań, koordinata kósherlerin parallel kóshiriw arqalı ápiwayı túrge keltiriń; qanday geometriyalıq obrazdı ańlatıwın anıqlań, eski hám jańa koordinata kósherlerine salıstırģanda sızılmada súwretleń: $4 x^2+9 y^2-40 x+36 y+100=0$; \\
\textbf{B2.} Berilgen teńlemeler oraylıq iymek sızıqlar ekenligin kórsetiń hám hárbir teńlemeni koordinata basın orayģa kóshiriń:  $4 x^2+2 x y+6 y^2+6 x-10 y+9=0$. \\
\textbf{B3.} Parallel kóshiriw hám burıw túrlendiriwleri yamasa aǵzalardı gruppalaw járdeminde tómendegi betliklerdiń kórinisi hám jaylasıwı anıqlansın: $2 x y+2 x+2 y+2 z-1=0$; \\
\textbf{C1.} $4 x^2-4 x y+y^2+6 x+1=0$ ETIS teńlemesi berilgen. Múyesh koefficienti $k$ tiń qanday mánislerinde $y=kx$ tuwrı sızıq: 1) bul iymek sızıqtı bir noqatta kesip ótiwi; 2) urınadı; 3) eki noqatta kesip ótiwin; 4) bul tuwrı menen ulıwma noqatqa iye bolmaytuģının anıqlań. \\
\textbf{C2.} Tómendegi betliklerdiń kanonikalıq teńlemesi hám jaylasıwın anıqlań.: $x^2-2 y^2+z^2+4 x y-10 x z+4 y z+2 x+4 y-10 z-1=0$. \\
\textbf{C3.} Parabolanıń qálegen urinbasınıń direktrisası hám kósherge perpendikulyar bolǵan fokal xordanı fokustan teńdey uzaqlıqtaģı noqatlarda kesetuģının dálilleń. \\

\end{tabular}
\vspace{1cm}


\begin{tabular}{m{17cm}}
\textbf{19-variant}
\newline

\textbf{T1.} Ekinshi tártipli sızıq hám tuwrı sızıqtıń óz ara jaylasıwı (Kesilisiw noqatları, Urınba (urınıw) jaģdayı) \\
\textbf{T2.} Ekinshi tártipli betliklerdiń kanonikalıq teńlemeleri (Ellipsoid, Giperboloid (1 gewekli), Giperboloid (2 gewekli)) \\
\textbf{A1.} Fokusları ordinata kósherinde jaylasqan, koordinatalar basına salıstırģanda simmetriyalı bolǵan giperbolanıń teńlemesin dúziń, bunda: asimtotalarınıń teńlemesi $7 \frac{1}{7}$ hám ekssentrisiteti $\varepsilon=\frac{7}{5}$; \\
\textbf{A2.} Koordinatalar sistemasın túrlendirmesten, tómendegi teńlemelerdiń hár biri parabolanı anıqlawın kórsetiń hám parametrin tabıń: $9 x^2-24 x y+16 y^2-54 x-178 y+181=0$; \\
\textbf{A3.} Fokusları abscissa kósherinde jaylasqan, koordinatalar basına qarata simmetriyalı bolǵan giperbolanıń teńlemesin dúziń, bunda: fokusları arasındaǵı aralıq $2 c=6$ hám ekscentrisiteti $\varepsilon=\frac{3}{2}$; \\
\textbf{B1.} Berilgen parabola tóbesi $A (6;-3) $ hám onıń direktrisasınıń teńlemesi $3x-5y+1=0$ berilgen. Bul parabolanıń $F$ fokusın tabıń. \\
\textbf{B2.} $\frac{x^2}{25}+\frac{y^2}{15}=1$ ellipstiń fokusı arqalı onıń úlken kósherine perpendikulyar ótkerilgen. Bul perpendikulyardıń ellips penen kesilisken noqatlarınan fokuslarga shekem bolǵan aralıqlardı anıqlań. \\
\textbf{B3.} Eger ellipstiń ekssentrisiteti $\varepsilon=\frac{1}{2}$ hám fokusı $F (3; 0) $ hám oǵan sáykes direktrisa teńlemesi $x+y-1=0$ belgili bolsa, onıń teńlemesin dúziń. \\
\textbf{C1.} $\frac{x^2}{81}+\frac{y^2}{36}+\frac{z^2}{9}=1$ ellipsoid $4 x-3 y+12 z-54=0$ tegislik penen bir ulıwma noqatqa iye ekenligin dálilleń hám onıń koordinataların tabıń. \\
\textbf{C2.} $\frac{x^2}{100}+\frac{y^2}{64}=1$ ellipstiń $2 x-y+7=0,2 x-y-1=0$ xordalarınıń ortaları arqalı ótetuģın tuwrı sızıqtıń teńlemesin dúziń. \\
\textbf{C3.} Giperbolanıń asimptotaların tabıń: $10 x y-2 y^2+6 x+4 y+21=0$ \\

\end{tabular}
\vspace{1cm}


\begin{tabular}{m{17cm}}
\textbf{20-variant}
\newline

\textbf{T1.} Tegislikte ekinshi tártipli sızıqlar (Ekinshi tártipli teńleme, Kvadrat kórinisindegi teńleme, Konik sızıqlar (konuslar kesimi)) \\
\textbf{T2.} Ekinshi tártipli betlik orayı, urınba tegisligi hám diametral tegisligi (Oray, Urınba tegislik, Diametral tegislik.) \\
\textbf{A1.} Tómendegi sızıqlardan qaysı biri oraylıq (yaǵnıy birden-bir orayǵa iye), qaysı biri orayǵa iye emes, qaysı biri sheksiz kóp orayǵa iye ekenligin anıqlań: $3 x^2-4 x y-2 y^2+3 x-12 y-7=0$; \\
\textbf{A2.} $M$ noqatı $y^2=20 x$ parabolaga tiyisli, eger onıń abscissası 7 ge teń bolsa fokal radiusların tabıń. \\
\textbf{A3.} Koordinatalar sistemasın túrlendirmesten, tómendegi teńlemelerdiń hár biri parabolanı anıqlawın kórsetiń hám parametrin tabıń: $9 x^2-6 x y+y^2-50 x+50 y-275=0$. \\
\textbf{B1.} Berilgen sızıqlar oraylıq ekenligin kórsetiń hám hárbir iymek sızıq ushın orayınıń koordinataların tabıń:$9 x^2-4 x y-7 y^2-12=0$; \\
\textbf{B2.} Parallel kóshiriw hám burıw túrlendiriwleri yamasa aǵzalardı gruppalaw járdeminde tómendegi betliklerdiń kórinisi hám jaylasıwı anıqlansın: $z=2 x^2-4 y^2-6 x+8 y+1$; \\
\textbf{B3.} Parabola tóbesiniń koordinataların, parametrin hám kósheriniń baǵıtın anıqlań: $y=A x^2+B x+C$, \\
\textbf{C1.} $\frac{x^2}{a^2}-\frac{y^2}{b^2}=1$ giperbolanıń asimptotaları hám onıń qálegen noqatınan asimptotalarga parallel etip ótkerilgen tuwrı sızıqlar menen shegaralanǵan parallelogrammnıń maydanı turaqlı san bolıp, $\frac{a b}{2}$ ga teń bolatuģının dálilleń. \\
\textbf{C2.} Ulıwma kósherge hám tóbeleri arasında jaylasqan ulıwma fokusqa iye bolǵan eki parabola tuwrı múyesh astında kesilisetuģının dálilleń. \\
\textbf{C3.} Berilgen teńleme kanonikalıq kóriniske keltirilsin; tipi anıqlansın; qanday geometriyalıq obrazdı anlatıwı anıqlansın; eski hám jana koordinatalar sistemasında geometriyalıq obrazı súwretlensin: $50 x^2-8 x y+35 y^2+100 x-8 y+67=0$; \\

\end{tabular}
\vspace{1cm}


\begin{tabular}{m{17cm}}
\textbf{21-variant}
\newline

\textbf{T1.} Ekinshi tártipli sızıqlardıń ulıwma teńlemeleri (Ulıwma teńleme) \\
\textbf{T2.} Parabola hám onıń kanonikalıq teńlemeleri (Fokus (bagdarlawshı noqat), Direktrisa (bagdarlawshı sızıq), Kósher (simmetriya kósheri)) \\
\textbf{A1.} $z+1=0$ tegislik bir qabatlı $\frac{x^2}{32}-\frac{y^2}{18}+\frac{z^2}{2}=1$ giperboloidti giperbola boyınsha kesip ótetuģının kórsetiń; onıń yarım kósherleri hám tóbelerin tabıń. \\
\textbf{A2.} Diskriminantın esaplaw arqalı tómendegi teńlemelerdiń hár biriniń tipin anıqlań: $x^2-4 x y+4 y^2+7 x-12=0$; \\
\textbf{A3.} $\frac{x^2}{16}+\frac{y^2}{7}=1$, ellipsinde jaylasqan hám shep fokusına shekemgi aralıǵı 2,5 ge teń noqattı tabıń. \\
\textbf{B1.} ITECH túri, ólshemleri hám jaylasıwın anıqlań: $9 x^2+24 x y+16 y^2-40 x-30 y=0$; \\
\textbf{B2.} Berilgen teńlemelerdiń parabolik ekenligin kórsetiń hám olardıń hár birin $(\alpha x+\beta y)^2+2 a_{13} x+2 a_{23} y+a_{33}=0$ kórinisinde jazıń: $x^2+4 x y+4 y^2+4 x+y-15=0 ;$ \\
\textbf{B3.} Giperbolanıń yarım kósherlerin tabıń, eger: asimptotaları $y= \pm \frac{5}{3} x$ tenlemeleri menen berilgen hám giperbola $N (6,9) $ noqatınan ótedi. \\
\textbf{C1.} $m$ nıń qanday mánislerinde $x+m y-2=0$ tegislik $\frac{x^2}{2}+\frac{z^2}{3}=y$ elliptik paraboloidti a) ellips boyınsha, b) parabola boyınsha kesip ótetuǵınlıǵın anıqlań. \\
\textbf{C2.} $y=k x+m$ tuwrı sızıqtıń $\frac{x^2}{a^2}-\frac{y^2}{b^2}=1$ giperbolaģa urınıw shártin keltirip shıģarıń. \\
\textbf{C3.} $m$ hám $n$ tiń qanday mánislerinde $m x^2+12 x y+9 y^2+4 x+n y-13=0$ teńleme: 1) oraylıq sızıqtı; 2) orayga iye bolmaǵan sızıq; 3) sheksiz kóp orayǵa iye bolǵan sızıqtı ańlatadı. \\

\end{tabular}
\vspace{1cm}


\begin{tabular}{m{17cm}}
\textbf{22-variant}
\newline

\textbf{T1.} Ekinshi tártipli sızıq orayı (Oraylıq sızıqlar (ellips, giperbola), Oray koordinataları: simmetriya orayı) \\
\textbf{T2.} Tegislikte ekinshi tártipli sızıqlar (Ekinshi tártipli teńleme, Kvadrat kórinisindegi teńleme, Konik sızıqlar (konuslar kesimi)) \\
\textbf{A1.} Tómendegi sızıqlardan qaysı biri oraylıq (yaǵnıy birden-bir orayǵa iye), qaysı biri orayǵa iye emes, qaysı biri sheksiz kóp orayǵa iye ekenligin anıqlań: $4 x^2-20 x y+25 y^2-14 x+2 y-15=0$; \\
\textbf{A2.} Fokusları ordinata kósherinde jatqan hám koordinatalar basına qarata simmetriyalı bolǵan ellipstiń teńlemesin dúziń, bunda: fokusları arasındaǵı aralıq $2 c=6$, direktrisaları arasındaǵı aralıq $16 \frac{2}{3}$; \\
\textbf{A3.} Koordinatalar sistemasın túrlendirmesten, tómendegi teńlemelerdiń hár biri parabolanı anıqlawın kórsetiń hám parametrin tabıń: $9 x^2-24 x y+16 y^2-54 x-178 y+181=0$; \\
\textbf{B1.} Eger parabolanıń fokusı $F(2;-1) $ hám direktrisa $x-y-1=0$ teńlemesi berilgen bolsa, onıń teńlemesin dúziń. \\
\textbf{B2.} $\frac{x^2}{2}-\frac{z^2}{3}=y$ giperbolik paraboloidi hám $3x-3y+4z+2=0$ tegisliginiń kesilisiw sızıǵı qanday iymek sızıq ekenligin anıqlań hám onıń orayın tabıń. \\
\textbf{B3.} Ellipstegi ekssentrisitetti anıqlań, eger: onıń kishi kósheri fokuslardan $60^{\circ}$ múyesh astında kórinse; \\
\textbf{C1.} $y^2=4 x$ parabola menen $\frac{x^2}{8}+\frac{y^2}{2}=1$ ellipstiń uliwma urinbaların anıqlań. \\
\textbf{C2.} Hár qanday parabolik teńleme ushın $a_{11}$ hám $a_{22}$ koefficientler hár qıylı belgige iye sanlar bola almaytuģının hám olar bir waqıtta nolge aylana almaytuģının dálilleń. \\
\textbf{C3.} Berilgen teńleme kanonikalıq kóriniske keltirilsin; tipi anıqlansın; qanday geometriyalıq obrazdı anlatıwı anıqlansın; eski hám jana koordinatalar sistemasında geometriyalıq obrazı súwretlensin: $11 x^2-20 x y-4 y^2-20 x-8 y+1=0$; \\

\end{tabular}
\vspace{1cm}


\begin{tabular}{m{17cm}}
\textbf{23-variant}
\newline

\textbf{T1.} Ekinshi tártipli betliklerdiń ulıwma teńlemeleri (Ulıwma teńleme) \\
\textbf{T2.} Ekinshi tártipli betliklerdiń ulıwmalıq teńlemesin kanonikalıq túrge keltiriw (invariantlar járdeminde) \\
\textbf{A1.} $x+y-3=0$ tuwrı sızıģi hám $x^2=4 y$ parabolasınıń kesilisiw noqatın tabıń. \\
\textbf{A2.} Koordinatalar sistemasın túrlendirmesten tómendegi teńlemelerdiń hár biri ellipsti anıqlawın kórsetiń hám onıń yarım kósherlerin tabıń: $13 x^2+18 x y+37 y^2-26 x-18 y+3=0$; \\
\textbf{A3.} $x-2=0$ tegislik $\frac{x^2}{16}+\frac{y^2}{12}+\frac{z^2}{4}=1$ ellipsoidti ellips boyınsha kesip ótetuģının kórsetiń; onıń yarım kósherleri hám tóbelerin tabıń. \\
\textbf{B1.} $x^2-y^2=16$ giperbolaǵa $A (-1;-7)$ noqattan ótkerilgen urınbalar teńlemesin dúziń. \\
\textbf{B2.} Berilgen teńlemeler oraylıq iymek sızıqlar ekenligin kórsetiń hám hárbir teńlemeni koordinata basın orayģa kóshiriń: $6 x^2+4 x y+y^2+4 x-2 y+2=0$; \\
\textbf{B3.} ITECH túri, ólshemleri hám jaylasıwın anıqlań: $x^2-2 x y+y^2-10 x-6 y+25=0$. \\
\textbf{C1.} Tómendegi betliklerdiń kanonikalıq teńlemesi hám jaylasıwın anıqlań.: $x^2+y^2+4 z^2+2 x y+4 x z+4 y z-6 z+1=0$. \\
\textbf{C2.} Giperbolanıń asimptotaların tabıń: $10 x y-2 y^2+6 x+4 y+21=0$ \\
\textbf{C3.} Múyesh koefficienti $k$ tiń qanday mánislerinde $y=kx+2$ tuwrısı: 1) $y^2=4x$ parabolanı kesip ótedi; 2) oǵan urınadı; 3) bul parabola sırtınan ótedi. \\

\end{tabular}
\vspace{1cm}


\begin{tabular}{m{17cm}}
\textbf{24-variant}
\newline

\textbf{T1.} Ekinshi tártipli sızıqlardıń ulıwma teńlemeleri (Ulıwma teńleme) \\
\textbf{T2.} Parabola hám onıń kanonikalıq teńlemeleri (Fokus (bagdarlawshı noqat), Direktrisa (bagdarlawshı sızıq), Kósher (simmetriya kósheri)) \\
\textbf{A1.} $\frac{x^2}{4}-\frac{y^2}{9}=1$ giperbolanıń asimptotalarınan hám $9 x+2 y-24=0$ tuwrı sızıqtan payda bolǵan úshmúyeshlik maydanın esaplań. \\
\textbf{A2.} Fokusları ordinata kósherinde jatqan hám koordinatalar basına salıstırģanda simmetriyalı bolǵan ellipstiń teńlemesin dúziń, bunda: fokusları arasındaǵı aralıq $2 c=24$, ekssentrisiteti $\varepsilon=\frac{12}{13}$; \\
\textbf{A3.} Tómendegi sızıqlardan qaysı biri oraylıq (yaǵnıy birden-bir orayǵa iye), qaysı biri orayǵa iye emes, qaysı biri sheksiz kóp orayǵa iye ekenligin anıqlań:: $x^2-2 x y+4 y^2+5 x-7 y+12=0$; \\
\textbf{B1.} Berilgen teńleme parabolik ekenligin kórsetiń; ápiwayı túrge keltiriń; qanday geometriyalıq obrazdı anlatıwın anıqlań, eski hám de jańa koordinata kósherlerine salıstırģanda sızılmada súwretleń:$4 x^2+12 x y+9 y^2-4 x-6 y+1=0$. \\
\textbf{B2.} Berilgen teńlemelerdiń parabolik ekenligin kórsetiń hám olardıń hár birin $(\alpha x+\beta y)^2+2 a_{13} x+2 a_{23} y+a_{33}=0$ kórinisinde jazıń: $25 x^2-20 x y+4 y^2+3 x-y+11=0$; \\
\textbf{B3.} Berilgen sızıqlar oraylıq ekenligin kórsetiń hám hárbir iymek sızıq ushın orayınıń koordinataların tabıń: $2 x^2-6 x y+5 y^2+22 x-36 y+11=0$. \\
\textbf{C1.} Giperbolanıń asimptotalarınan direktrisaları ajıratqan kesindiler (giperbolanıń orayınan esaplaganda) giperbolanıń haqıyqıy yarım kósherine teń ekenligin dálilleń. Bul qásiyetten paydalanıp, giperbolanıń direktrisaların jasań. \\
\textbf{C2.} Fokuslardan ellipstiń qálegen urınbasına shekem bolǵan aralıqlar kóbeymesi kishi yarım kósherdiń kvadratına teń ekenligin dálilleń. \\
\textbf{C3.} $\frac{x^2}{100}+\frac{y^2}{64}=1$ ellipstiń $2 x-y+7=0,2 x-y-1=0$ xordalarınıń ortaları arqalı ótetuģın tuwrı sızıqtıń teńlemesin dúziń. \\

\end{tabular}
\vspace{1cm}


\begin{tabular}{m{17cm}}
\textbf{25-variant}
\newline

\textbf{T1.} Tegislikte ekinshi tártipli sızıqlar (Ekinshi tártipli teńleme, Kvadrat kórinisindegi teńleme, Konik sızıqlar (konuslar kesimi)) \\
\textbf{T2.} Ekinshi tártipli sızıq orayı (Oraylıq sızıqlar (ellips, giperbola), Oray koordinataları: simmetriya orayı) \\
\textbf{A1.} Fokusları abscissa kósherinde jaylasqan, koordinatalar basına salıstırģanda simmetriyalı bolǵan giperbolanıń teńlemesin dúziń, bunda: asimtotalarınıń teńlemesi $y= \pm \frac{3}{4} x$ hám direktrisalarınıń arasındaģı aralıq $12 \frac{4}{5}$. \\
\textbf{A2.} Koordinatalar sistemasın túrlendirmesten, tómendegi teńlemelerdiń hár biri parabolanı anıqlawın kórsetiń hám parametrin tabıń: $9 x^2+24 x y+16 y^2-120 x+90 y=0$; \\
\textbf{A3.} Parabolanıń tóbesi ($\alpha;\beta$) noqat penen ústpe-úst túsetuģının bilgen halda onıń teńlemesin dúziń. Parametri $p$ ǵa teń. Onıń kósheri $O y$ kósherine parallel bolıp, $O y$ kósheriniń oń baǵıtında sheksizlikke sozilgan; \\
\textbf{B1.} Lagranj usılınan paydalanıp, teńlemelerdi kvadratlar qosındısı túrine keltirip, tómendegi betlerdiń kórinisin anıqlań: $x^2-2 y^2+z^2+4 x y-8 x z-4 y z-14 x-4 y+14 z+16=0$; \\
\textbf{B2.} ITECH túri, ólshemleri hám jaylasıwın anıqlań: $6 x y-8 y^2+12 x-26 y-11=0$; \\
\textbf{B3.} Giperbolanıń yarım kósherlerin tabıń, eger: fokusları arasındaǵı aralıq 8 ge hám direktrisaları arasındaǵı aralıq 6 ģa teń. \\
\textbf{C1.} Giperbolanıń asimptotaların tabıń: $x^2-3 x y-10 y^2+6 x-8 y=0$; \\
\textbf{C2.} $\frac{x^2}{a^2}-\frac{y^2}{b^2}=1$ giperbolanıń fokusınan asimptotasına shekemgi aralıq $b$ qa teń ekenligin dálilleń. \\
\textbf{C3.} Hár qanday parabolik teńleme $ (\alpha x+\beta y) ^2+2a_{13}x+2a_{23}y+a_{33}=0$ kórinisinde jazılıwı múmkinligin dálilleń. Sonday-aq, elliptikalıq hám giperbolikalıq teńlemelerdi bunday kóriniste jazıp bolmaytuģının dálilleń. \\

\end{tabular}
\vspace{1cm}


\begin{tabular}{m{17cm}}
\textbf{26-variant}
\newline

\textbf{T1.} Bir gewekli giperboloid hám giperbolik paraboloidtıń tuwrı sızıqlı jasawshıları (Giperboloid, Giperbolik paraboloid, Sızıqlı jasawshılar) \\
\textbf{T2.} Parabola hám onıń kanonikalıq teńlemeleri (Fokus (bagdarlawshı noqat), Direktrisa (bagdarlawshı sızıq), Kósher (simmetriya kósheri)) \\
\textbf{A1.} Berilgen teńleme menen qanday iymek sızıq anıqlanıwın tabıń: $\left\{\begin{array}{l}\frac{x^2}{4}-\frac{y^2}{3}=2 z \\ x-2 y+2=0 ;\end{array}\right.$ \\
\textbf{A2.} Diskriminantın esaplaw arqalı tómendegi teńlemelerdiń hár biriniń tipin anıqlań: $25 x^2-20 x y+4 y^2-12 x+20 y-17=0$; \\
\textbf{A3.} Koordinatalar sistemasın túrlendirmesten tómendegi teńlemelerdiń hár biri ellipsti anıqlawın kórsetiń hám onıń yarım kósherlerin tabıń: $13 x^2+10 x y+13 y^2+46 x+62 y+13=0$. \\
\textbf{B1.} Ellipstegi ekssentrisitetti anıqlań, eger: fokusları arasındaǵı kesindiniń ózi kishi kósherdiń tóbesinen tuwrı múyesh astında kórinse; \\
\textbf{B2.} Parabola tóbesiniń koordinataların, parametrin hám kósheriniń baǵıtın anıqlań: $y=x^2-8 x+15$, \\
\textbf{B3.} Parallel kóshiriw hám burıw túrlendiriwleri yamasa aǵzalardı gruppalaw járdeminde tómendegi betliklerdiń kórinisi hám jaylasıwı anıqlansın: $z^2=x^2+2 x y+y^2+1$; \\
\textbf{C1.} $4 x^2-4 x y+y^2+6 x+1=0$ ETIS teńlemesi berilgen. Múyesh koefficienti $k$ tiń qanday mánislerinde $y=kx$ tuwrı sızıq: 1) bul iymek sızıqtı bir noqatta kesip ótiwi; 2) urınadı; 3) eki noqatta kesip ótiwin; 4) bul tuwrı menen ulıwma noqatqa iye bolmaytuģının anıqlań. \\
\textbf{C2.} Elliptik túrdegi ($\delta>0$) teńleme $\Delta=0$ bolǵanda ǵana eki bir-birin kesip ótiwshi jormal tuwrı sızıq bolatuģının dálilleń. \\
\textbf{C3.} $m$ nıń qanday mánislerinde $y=\frac{5}{2} x+m$ tuwrı sızıq $\frac{x^2}{9}-\frac{y^2}{36}=1$ giperbolanı 1) kesip ótiwin; 2) oǵan urınıwın; 3) sırtınan ótiwin anıqlań. \\

\end{tabular}
\vspace{1cm}


\begin{tabular}{m{17cm}}
\textbf{27-variant}
\newline

\textbf{T1.} Ekinshi tártipli sızıqlardıń ulıwma teńlemesin invariantlar járdeminde kanonikalıq túrge keltiriw \\
\textbf{T2.} Ekinshi tártipli betliklerdiń kanonikalıq teńlemeleri (Paraboloid (ellipstik), Paraboloid (giperbolik), Konus, Cilindr) \\
\textbf{A1.} Fokusları abscissa kósherinde jatqan hám koordinatalar basına salıstırǵanda simmetriyalı, yarım kósherleri 5 hám 2 bolǵan ellipstiń teńlemesin dúziń; \\
\textbf{A2.} Fokusları abscissa kósherinde, koordinatalar basına qarata simmetriyalı jaylasqan giperbolanıń teńlemesin dúziń, bunda: $M_1\left(-3; \frac{5}{2}\right)$ noqatı giperbolaģa tiyisli hám direktrisalarınıń teńlemesi $x= \pm \frac{4}{3}$; \\
\textbf{A3.} $y^2=36 x$ parabolanıń $A (2; 9) $ noqatındaǵı urınbasınıń teńlemesin dúziń. \\
\textbf{B1.} $\frac{x^2}{64}-\frac{y^2}{36}=1$ giperbolanıń oń fokusına shekemgi aralıǵı 4,5 ke teń bolǵan noqatların anıqlań. \\
\textbf{B2.} Berilgen teńlemeni ápiwayı túrge keltiriń; tipin anıqlań; qanday geometriyalıq obrazdı ańlatıwın anıqlań, eski hám de jańa koordinata kósherlerine qarata sızılmada súwretleń: $5 x^2-6 x y+5 y^2-32=0$; \\
\textbf{B3.} Berilgen teńlemeler oraylıq iymek sızıqlar ekenligin kórsetiń hám hárbir teńlemeni koordinata basın orayģa kóshiriń: $4 x^2+6 x y+y^2-10 x-10=0$; \\
\textbf{C1.} $\frac{x^2}{a^2}+\frac{y^2}{b^2}=1$ ellipstiń $F(c, 0)$ fokusı arqalı úlken kósherine perpendikulyar bolǵan xorda ótkerilgen. Bul xordıń uzınlıǵın tabıń. \\
\textbf{C2.} Tómendegi betliklerdiń kanonikalıq teńlemesi hám jaylasıwın anıqlań.: $7 x^2+6 y^2+5 z^2-4 x y-4 y z-6 x-24 y+18 z+30=0$. \\
\textbf{C3.} $A x+B y+C=0$ tuwri sızıq $y^2=2 p x$ parabolaga urinıwı ushin zárúrli hám jeterli shártti tabıń. \\

\end{tabular}
\vspace{1cm}


\begin{tabular}{m{17cm}}
\textbf{28-variant}
\newline

\textbf{T1.} Ekinshi tártipli betliklerdiń ulıwmalıq teńlemesin kanonikalıq túrge keltiriw (invariantlar járdeminde) \\
\textbf{T2.} Ekinshi tártipli sızıqqa urınba, túyinles diametri teńlemesi (Urınba teńlemesi, Túyinles diametr: oraydan ótiwshi simmetriya kósherleri) \\
\textbf{A1.} Tómendegi sızıqlardan qaysı biri oraylıq (yaǵnıy birden-bir orayǵa iye), qaysı biri orayǵa iye emes, qaysı biri sheksiz kóp orayǵa iye ekenligin anıqlań:: $x^2-2 x y+4 y^2+5 x-7 y+12=0$; \\
\textbf{A2.} Koordinatalar sistemasın túrlendirmesten, tómendegi teńlemelerdiń hár biri parabolanı anıqlawın kórsetiń hám parametrin tabıń: $x^2-2 x y+y^2+6 x-14 y+29=0$; \\
\textbf{A3.} $y^2+z^2=x$ elliptik paraboloidtıń $x+2 y-z=0$ tegislik penen kesilisiwiniń koordinata tegisliklerindegi proekciyalarınıń teńlemelerin tabıń. \\
\textbf{B1.} Eger parabolanıń fokusı $F (4;3) $ hám direktrisa $x-1=0$ teńlemesi berilgen bolsa, onıń teńlemesin dúziń. \\
\textbf{B2.} Tómendegilerdi bilgen halda ellips teńlemesin dúziń: onıń fokusları $F_1\left(-2; \frac{3}{2}\right), F_2\left(2;-\frac{3}{2}\right) $ hám ekssentrisitet $\varepsilon=\frac{\sqrt{2}}{2}$; \\
\textbf{B3.} Berilgen teńleme parabolik ekenligin kórsetiń; ápiwayı túrge keltiriń; qanday geometriyalıq obrazdı anlatıwın anıqlań, eski hám de jańa koordinata kósherlerine salıstırģanda sızılmada súwretleń:$16 x^2-24 x y+9 y^2-160 x+120 y+425=0$. \\
\textbf{C1.} $\frac{x^2}{81}+\frac{y^2}{36}+\frac{z^2}{9}=1$ ellipsoid $4 x-3 y+12 z-54=0$ tegislik penen bir ulıwma noqatqa iye ekenligin dálilleń hám onıń koordinataların tabıń. \\
\textbf{C2.} $\frac{x^2}{a^2}+\frac{y^2}{b^2}=1$ ellipstiń bir diametriniń tóbelerine júrgizilgen urınbalar parallel bolıwın dálilleń (ellipstiń diametri dep onıń orayınan ótiwshi xordaǵa aytıladı). \\
\textbf{C3.} $y^2=2 p x$ parabolaǵa $y=k x+b$ tuwrı sızıq urınıw shártin keltirip shigarıń. \\

\end{tabular}
\vspace{1cm}


\begin{tabular}{m{17cm}}
\textbf{29-variant}
\newline

\textbf{T1.} Tegislikte ekinshi tártipli sızıqlar (Ekinshi tártipli teńleme, Kvadrat kórinisindegi teńleme, Konik sızıqlar (konuslar kesimi)) \\
\textbf{T2.} Ekinshi tártipli sızıq hám tuwrı sızıqtıń óz ara jaylasıwı (Kesilisiw noqatları, Urınba (urınıw) jaģdayı) \\
\textbf{A1.} Koordinatalar sistemasın túrlendirmesten tómendegi teńlemelerdiń hár biri giperbolanı anıqlawın kórsetiń hám onıń koordinataların tabıń: $12 x^2+26 x y+12 y^2-52 x-48 y+73=0$ \\
\textbf{A2.} Fokusları abscissa kósherinde jatqan hám koordinatalar basına qarata simmetriyalı bolǵan ellipstiń teńlemesi dúzilsin, bunda: $M_1 (2;-2) $ noqatı ellipske tiyisli hám úlken yarım kósheri $a=4$; \\
\textbf{A3.} $y^2=24 x$ parabolanıń $F$ fokusın hám direktrisasınıń teńlemesin tabıń. \\
\textbf{B1.} Parabola tóbesiniń koordinataların, parametrin hám kósheriniń baǵıtın anıqlań: $y^2-6 x+14 y+49=0$, \\
\textbf{B2.} Berilgen teńlemeler oraylıq iymek sızıqlar ekenligin kórsetiń hám hárbir teńlemeni koordinata basın orayģa kóshiriń:  $4 x^2+2 x y+6 y^2+6 x-10 y+9=0$. \\
\textbf{B3.} $\frac{x^2}{80}-\frac{y^2}{20}=1$ giperbolada $M_1 (10;-\sqrt{5}) $ noqat berilgen. $M_1$ noqatınıń fokal radiusları jatqan tuwrı sızıqlardıń teńlemelerin dúziń. \\
\textbf{C1.} Berilgen teńleme kanonikalıq kóriniske keltirilsin; tipi anıqlansın; qanday geometriyalıq obrazdı anlatıwı anıqlansın; eski hám jana koordinatalar sistemasında geometriyalıq obrazı súwretlensin: $7 x^2+6 x y-y^2+28 x+12 y+28=0$; \\
\textbf{C2.} $\frac{x^2}{a^2}+\frac{y^2}{b^2}=1$ ellipske ishley sızılgan kvadrat tárepiniń uzınlıǵın esaplań. \\
\textbf{C3.} Giperbola asimptotalarınıń tenlemeleri $y= \pm \frac{1}{2} x$ hám urinbalardan biriniń teńlemesi. $5 x-6 y-8=0$ belgili bolsa, giperbola teńlemesin dúziń. \\

\end{tabular}
\vspace{1cm}


\begin{tabular}{m{17cm}}
\textbf{30-variant}
\newline

\textbf{T1.} Parabola hám onıń kanonikalıq teńlemeleri (Fokus (bagdarlawshı noqat), Direktrisa (bagdarlawshı sızıq), Kósher (simmetriya kósheri)) \\
\textbf{T2.} Ekinshi tártipli betliklerdiń kanonikalıq teńlemeleri (Paraboloid (ellipstik), Paraboloid (giperbolik), Konus, Cilindr) \\
\textbf{A1.} Tómendegi sızıqlardan qaysı biri oraylıq (yaǵnıy birden-bir orayǵa iye), qaysı biri orayǵa iye emes, qaysı biri sheksiz kóp orayǵa iye ekenligin anıqlań:  $4 x^2+4 x y+y^2-8 x-4 y-21=0$; \\
\textbf{A2.} Fokusları abscissa kósherinde jaylasqan, koordinatalar basına salıstırģanda simmetriyalı bolǵan giperbolanıń teńlemesin dúziń, bunda: onıń kósherleri $2 a=10$ hám $2 b=8$; \\
\textbf{A3.} Koordinatalar sistemasın túrlendirmesten, tómendegi teńlemelerdiń hár biri parabolanı anıqlawın kórsetiń hám parametrin tabıń: $x^2-2 x y+y^2+6 x-14 y+29=0$; \\
\textbf{B1.} Berilgen teńlemeniń tipin anıqlań, koordinata kósherlerin parallel kóshiriw arqalı ápiwayı túrge keltiriń; qanday geometriyalıq obrazdı ańlatıwın anıqlań, eski hám jańa koordinata kósherlerine salıstırģanda sızılmada súwretleń:  $9 x^2+4 y^2+18 x-8 y+49=0$; \\
\textbf{B2.} Berilgen teńleme parabolik ekenligin kórsetiń; ápiwayı túrge keltiriń; qanday geometriyalıq obrazdı anlatıwın anıqlań, eski hám de jańa koordinata kósherlerine salıstırģanda sızılmada súwretleń: $x^2-2 x y+y^2-12 x+12 y-14=0$ \\
\textbf{B3.} Tómendegilerdi bilgen halda ellips teńlemesin dúziń: onıń úlken kósheri 26 ģa teń hám fokusları $F_1 (-10; 0), F_2 (14; 0) $; \\
\textbf{C1.} Tómendegi eki tuwrı sızıqqa urınatuģın giperbolanıń teńlemesin dúziń: $5x-6y-16=0$, $13x-10y-48=0$, bunda onıń kósherleri koordinata kósherleri menen ústpe-úst túsedi. \\
\textbf{C2.} Tómendegi betliklerdiń kanonikalıq teńlemesi hám jaylasıwın anıqlań.: $x^2-2 y^2+z^2+4 x y-8 x z-4 y z-14 x-4 y+14 z+16=0$. \\
\textbf{C3.} Giperbolanıń asimptotaların tabıń: $3 x^2+2 x y-y^2+8 x+10 y-14=0$; \\

\end{tabular}
\vspace{1cm}


\begin{tabular}{m{17cm}}
\textbf{31-variant}
\newline

\textbf{T1.} Ekinshi tártipli sızıq hám tuwrı sızıqtıń óz ara jaylasıwı (Kesilisiw noqatları, Urınba (urınıw) jaģdayı) \\
\textbf{T2.} Ekinshi tártipli betliklerdiń kanonikalıq teńlemeleri (Ellipsoid, Giperboloid (1 gewekli), Giperboloid (2 gewekli)) \\
\textbf{A1.} Berilgen teńleme menen qanday iymek sızıq anıqlanıwın tabıń: $\left\{\begin{array}{l}\frac{x^2}{3}+\frac{y^2}{6}=2 z, \\ 3 x-y+6 z-14=0\end{array}\right.$ \\
\textbf{A2.} Koordinatalar sistemasın túrlendirmesten tómendegi teńlemelerdiń hár biri giperbolanı anıqlawın kórsetiń hám onıń koordinataların tabıń: $x^2-6 x y-7 y^2+10 x-30 y+23=0$. \\
\textbf{A3.} Koordinatalar sistemasın túrlendirmesten, tómendegi teńlemelerdiń hár biri parabolanı anıqlawın kórsetiń hám parametrin tabıń: $9 x^2+24 x y+16 y^2-120 x+90 y=0$; \\
\textbf{B1.} $\frac{x^2}{12}+\frac{y^2}{4}+\frac{z^2}{3}=1$ ellipsoidı hám $2x-3y+4z-11=0$ tegisliginiń kesilisiw sızıǵı qanday iymek sızıq yekenligin anıqlań hám onıń orayın tabıń. \\
\textbf{B2.} $\frac{x^2}{9}-\frac{y^2}{4}=1$ giperbolanıń $M (5,1) $ noqatında teń ekige bólinetuģın xordasınıń teńlemesi dúzilsin. \\
\textbf{B3.} Berilgen teńlemelerdiń parabolik ekenligin kórsetiń hám olardıń hár birin $(\alpha x+\beta y)^2+2 a_{13} x+2 a_{23} y+a_{33}=0$ kórinisinde jazıń: $x^2+4 x y+4 y^2+4 x+y-15=0 ;$ \\
\textbf{C1.} Parabolanıń qálegen urinbasınıń direktrisası hám kósherge perpendikulyar bolǵan fokal xordanı fokustan teńdey uzaqlıqtaģı noqatlarda kesetuģının dálilleń. \\
\textbf{C2.} $m$ niń qanday mánislerinde $x+mz-1=0$ tegislik tómendegi $x^2+y^2z^2=1$ eki gewekli giperboloidti a) ellips boyınsha, b) giperbola boyınsha kesedi? \\
\textbf{C3.} Parabolik teńleme $\Delta \neq 0$ bolǵanda hám tek sonda ǵana parabolanı anıqlaytuģının dálilleń. Bul jaǵdayda parabolanıń parametri $p=\sqrt{\frac{-\Delta}{ (a_{11}+a_{33}) ^3}}$ formula menen anıqlanıwın dálilleń. \\

\end{tabular}
\vspace{1cm}


\begin{tabular}{m{17cm}}
\textbf{32-variant}
\newline

\textbf{T1.} Tegislikte ekinshi tártipli sızıqlar (Ekinshi tártipli teńleme, Kvadrat kórinisindegi teńleme, Konik sızıqlar (konuslar kesimi)) \\
\textbf{T2.} Ekinshi tártipli betliklerdiń ulıwma teńlemeleri (Ulıwma teńleme) \\
\textbf{A1.} Fokusları abscissa kósherinde jatqan hám koordinatalar basına salıstırģanda simmetriyalı bolǵan ellipstiń teńlemesin dúziń, bunda: direktrisaları arasındaǵı aralıq 32 hám $\varepsilon=\frac{1}{2}$. \\
\textbf{A2.} Fokusları ordinata kósherinde jaylasqan, koordinatalar basına salıstırģanda simmetriyalı bolǵan giperbolanıń teńlemesin dúziń, bunda: asimtotalarınıń teńlemesi $y= \pm \frac{4}{3} x$ ham direktrisalarınıń arasındaģı aralıq $6 \frac{2}{5}$. \\
\textbf{A3.} Tóbesi koordinatalar basında bolǵan parabolanıń teńlemesin dúziń, bunda: parabola oń yarım tegislikte hám $Ox$ kósherine simmetriyalı jaylasqan, hám parametri $p=0,5$; \\
\textbf{B1.} Parallel kóshiriw hám burıw túrlendiriwleri yamasa aǵzalardı gruppalaw járdeminde tómendegi betliklerdiń kórinisi hám jaylasıwı anıqlansın: $4 x^2-y^2-4 x+4 y-3=0$; \\
\textbf{B2.} Tómendegilerdi bilgen halda ellips teńlemesin dúziń: onıń kishi kósheri 2 ge teń hám fokusları $F_1 (-1;-1) $, $F_2 (1; 1) $; \\
\textbf{B3.} Berilgen teńlemeniń tipin anıqlań, koordinata kósherlerin parallel kóshiriw arqalı ápiwayı túrge keltiriń; qanday geometriyalıq obrazdı ańlatıwın anıqlań, eski hám jańa koordinata kósherlerine salıstırģanda sızılmada súwretleń: $4 x^2-y^2+8 x-2 y+3=0$; \\
\textbf{C1.} Ellips orayınan onıń qálegen urınbasınıń fokal kósher menen kesilisiw noqatına shekemgi hám urınıw noqatınan fokal kósherge túsirilgen perpendikulyar ultanına shekemgi aralıqlar kóbeymesi turaqlı shama bolıp, ellips úlken yarım kósheriniń kvadratına teń ekenligin dálilleń. \\
\textbf{C2.} $m$ hám $n$ tiń qanday mánislerinde $m x^2+12 x y+9 y^2+4 x+n y-13=0$ teńleme: 1) oraylıq sızıqtı; 2) orayga iye bolmaǵan sızıq; 3) sheksiz kóp orayǵa iye bolǵan sızıqtı ańlatadı. \\
\textbf{C3.} Kósherleri óz ara perpendikulyar bolǵan eki parabola tórt noqatta kesilisse, bul noqatlar bir sheńberde jatıwın dálilleń. \\

\end{tabular}
\vspace{1cm}


\begin{tabular}{m{17cm}}
\textbf{33-variant}
\newline

\textbf{T1.} Ekinshi tártipli sızıqlardıń ulıwma teńlemeleri (Ulıwma teńleme) \\
\textbf{T2.} Parabola hám onıń kanonikalıq teńlemeleri (Fokus (bagdarlawshı noqat), Direktrisa (bagdarlawshı sızıq), Kósher (simmetriya kósheri)) \\
\textbf{A1.} Berilgen teńleme menen qanday iymek sızıq anıqlanıwın tabıń: $\left\{\begin{array}{l}\frac{x^2}{.4}+\frac{y^2}{9}-\frac{z^2}{36}=1, \\ 9 x-6 y+2 z-28=0,\end{array}\right.$ \\
\textbf{A2.} Tómendegi sızıqlardan qaysı biri oraylıq (yaǵnıy birden-bir orayǵa iye), qaysı biri orayǵa iye emes, qaysı biri sheksiz kóp orayǵa iye ekenligin anıqlań: $4 x^2-4 x y+y^2-12 x+6 y-11=0$; \\
\textbf{A3.} Koordinatalar sistemasın túrlendirmesten tómendegi teńlemelerdiń hár biri birden-bir noqattı anıqlawın kórsetiń hám onıń koordinataların tabıń: $3 x^2+4 x y+y^2-2 x-1=0$; \\
\textbf{B1.} Berilgen sızıqlar oraylıq ekenligin kórsetiń hám hárbir iymek sızıq ushın orayınıń koordinataların tabıń: $3x^2+5xy+y^2-8x-11y-7=0$. \\
\textbf{B2.} $x^2=16y$ parabolanıń $2x+4y+7=0$ tuwrısına perpendikulyar bolǵan urınbasınıń teńlemesin dúziń. \\
\textbf{B3.} $y^2=8x$ parabolanıń $2x+2y-3=0$ tuwrısına parallel urınbasınıń teńlemesin dúziń. \\
\textbf{C1.} Ulıwma kósherge hám tóbeleri arasında jaylasqan ulıwma fokusqa iye bolǵan eki parabola tuwrı múyesh astında kesilisetuģının dálilleń. \\
\textbf{C2.} Giperbolanıń asimptotaların tabıń: $3 x^2+7 x y+4 y^2+5 x+2 y-6=0$; \\
\textbf{C3.} Ekinshi dárejeli teńleme tek hám tek $\Delta=0$ bolǵanda ǵana aynıǵan iymek sızıq teńlemesi bolatuģının dálilleń. \\

\end{tabular}
\vspace{1cm}


\begin{tabular}{m{17cm}}
\textbf{34-variant}
\newline

\textbf{T1.} Tegislikte ekinshi tártipli sızıqlar (Ekinshi tártipli teńleme, Kvadrat kórinisindegi teńleme, Konik sızıqlar (konuslar kesimi)) \\
\textbf{T2.} Ekinshi tártipli betlik orayı, urınba tegisligi hám diametral tegisligi (Oray, Urınba tegislik, Diametral tegislik.) \\
\textbf{A1.} Koordinatalar sistemasın túrlendirmesten, tómendegi teńlemelerdiń hár biri parabolanı anıqlawın kórsetiń hám parametrin tabıń: $9 x^2-24 x y+16 y^2-54 x-178 y+181=0$; \\
\textbf{A2.} Fokusları abscissa kósherinde jatqan hám koordinatalar basına qarata simmetriyalı bolǵan ellipstiń teńlemesin dúziń, bunda: $M_1 (4;-\sqrt{3}) $ hám $M_2 (2 \sqrt{2}; 3)$ noqatları ellipske tiyisli; \\
\textbf{A3.} Tómendegi sızıqlardan qaysı biri oraylıq (yaǵnıy birden-bir orayǵa iye), qaysı biri orayǵa iye emes, qaysı biri sheksiz kóp orayǵa iye ekenligin anıqlań: $3 x^2-4 x y-2 y^2+3 x-12 y-7=0$; \\
\textbf{B1.} $M_1 (2;-1) $ noqatı fokusı $F (1;0) $ bolǵan ellipste jatadı. Bul fokusqa sáykes direktrisa bolsa $2x-y-10=0$ teńleme menen berilgen. Usı ellipstiń teńlemesin dúziń. \\
\textbf{B2.} Berilgen teńlemeler oraylıq iymek sızıqlar ekenligin kórsetiń hám hárbir teńlemeni koordinata basın orayģa kóshiriń: $3x^2-6xy+2y^2-4x+2y+1=0$. \\
\textbf{B3.} $M_1 (1;-2) $ noqat fokusı $F (-2; 2) $, oǵan sáykes direktrisa bolsa $2x-y-1=0$ teńleme menen berilgen giperbolaǵa tiyisli. Bul giperbolanıń teńlemesin dúziń. \\
\textbf{C1.} $m$ nıń qanday mánislerinde $y=-x+m$ sızıq: 1) $\frac{x^2}{20}+\frac{y^2}{5}=1$ ellipsti kesip ótedi; 2) ellipske urınadı; 3) ellipsti kesip ótpeydi. \\
\textbf{C2.} $A x+B y+C=0$ tuwrı sızıqtıń $\frac{x^2}{a^2}+\frac{y^2}{b^2}=1$, ellipske urınba bolıwı ushın zárúrli hám jeterli shárti tabılsın. \\
\textbf{C3.} Berilgen $y=k x+b$ tuwri sızıqqa parallel hám $y^2=2 p x$ parabolaga urinatuģın tuwri sızıqtıń teńlemesin jazıń. \\

\end{tabular}
\vspace{1cm}


\begin{tabular}{m{17cm}}
\textbf{35-variant}
\newline

\textbf{T1.} Ekinshi tártipli sızıqlardıń ulıwma teńlemesin invariantlar járdeminde kanonikalıq túrge keltiriw \\
\textbf{T2.} Ekinshi tártipli sızıqqa urınba, túyinles diametri teńlemesi (Urınba teńlemesi, Túyinles diametr: oraydan ótiwshi simmetriya kósherleri) \\
\textbf{A1.} $\frac{x^2}{20}-\frac{y^2}{5}=-1$ giperbola hám $y^2=3 x$ parabolanıń kesilisiw noqatların anıqlań. \\
\textbf{A2.} $z+1=0$ tegislik bir qabatlı $\frac{x^2}{32}-\frac{y^2}{18}+\frac{z^2}{2}=1$ giperboloidti giperbola boyınsha kesip ótetuģının kórsetiń; onıń yarım kósherleri hám tóbelerin tabıń. \\
\textbf{A3.} Fokusları abscissa kósherinde jaylasqan, koordinatalar basına salıstırģanda simmetriyalı bolǵan giperbolanıń teńlemesin dúziń, bunda: direktrisalarınıń arasındaģı aralıq $\frac{32}{5}$ ham kishi kósheri $2 b=6$; \\
\textbf{B1.} Berilgen teńlemelerdiń parabolik ekenligin kórsetiń hám olardıń hár birin $(\alpha x+\beta y)^2+2 a_{13} x+2 a_{23} y+a_{33}=0$ kórinisinde jazıń:  $9 x^2-42 x y+49 y^2+3 x-2 y-24=0$. \\
\textbf{B2.} ITECH túri, ólshemleri hám jaylasıwın anıqlań: $4 x^2-4 x y+y^2-2 x-14 y+7=0$. \\
\textbf{B3.} Parallel kóshiriw hám burıw túrlendiriwleri yamasa aǵzalardı gruppalaw járdeminde tómendegi betliklerdiń kórinisi hám jaylasıwı anıqlansın: $3 x^2+6 x-8 y+6 z-7=0$; \\
\textbf{C1.} $\frac{x^2}{a^2}-\frac{y^2}{b^2}=1$ giperbolanıń qálegen noqatınan onıń eki asimptotasına shekemgi aralıqlar kóbeymesi $\frac{a^2 b^2}{a^2+b^2}$ ǵa teń turaqlı shama ekenligin dálilleń. \\
\textbf{C2.} Giperbolanıń asimptotalarınan direktrisaları ajıratqan kesindiler (giperbolanıń orayınan esaplaganda) giperbolanıń haqıyqıy yarım kósherine teń ekenligin dálilleń. Bul qásiyetten paydalanıp, giperbolanıń direktrisaların jasań. \\
\textbf{C3.} Berilgen teńleme kanonikalıq kóriniske keltirilsin; tipi anıqlansın; qanday geometriyalıq obrazdı anlatıwı anıqlansın; eski hám jana koordinatalar sistemasında geometriyalıq obrazı súwretlensin: $14 x^2+24 x y+21 y^2-4 x+18 y-139=0$; \\

\end{tabular}
\vspace{1cm}


\begin{tabular}{m{17cm}}
\textbf{36-variant}
\newline

\textbf{T1.} Parabola hám onıń kanonikalıq teńlemeleri (Fokus (bagdarlawshı noqat), Direktrisa (bagdarlawshı sızıq), Kósher (simmetriya kósheri)) \\
\textbf{T2.} Bir gewekli giperboloid hám giperbolik paraboloidtıń tuwrı sızıqlı jasawshıları (Giperboloid, Giperbolik paraboloid, Sızıqlı jasawshılar) \\
\textbf{A1.} Teń tárepli giperbolanıń ekssentrisiteti esaplansın. \\
\textbf{A2.} Tómendegi sızıqlardan qaysı biri oraylıq (yaǵnıy birden-bir orayǵa iye), qaysı biri orayǵa iye emes, qaysı biri sheksiz kóp orayǵa iye ekenligin anıqlań:  $4 x^2-6 x y-9 y^2+3 x-7 y+12=0$. \\
\textbf{A3.} Parabolanıń tóbesi ($\alpha;\beta$) noqat penen ústpe-úst túsetuģının bilgen halda onıń teńlemesin dúziń. Parametri $p$ ǵa teń. Onıń kósheri $O x$ kósherine parallel bolıp, $O x$ kósheriniń teris baǵıtında sheksizlikke sozilgan;. \\
\textbf{B1.} Berilgen sızıqlar oraylıq ekenligin kórsetiń hám hárbir iymek sızıq ushın orayınıń koordinataların tabıń:$5 x^2+4 x y+2 y^2+20 x+20 y-18=0$; \\
\textbf{B2.} ITECH túri, ólshemleri hám jaylasıwın anıqlań: $4 x^2-12 x y+9 y^2-2 x+3 y-2=0$. \\
\textbf{B3.} $\frac{x^2}{49}+\frac{y^2}{24}=1$ ellips penen fokuslas hám ekssentrisiteti $e=\frac{5}{4}$ bolgan giperbolanıń teńlemesi jazılsın. \\
\textbf{C1.} $m$ nıń qanday mánislerinde $x+m y-2=0$ tegislik $\frac{x^2}{2}+\frac{z^2}{3}=y$ elliptik paraboloidti a) ellips boyınsha, b) parabola boyınsha kesip ótetuǵınlıǵın anıqlań. \\
\textbf{C2.} Tómendegi betliklerdiń kanonikalıq teńlemesi hám jaylasıwın anıqlań.: $x^2+5 y^2+z^2+2 x y+6 x z+2 y z-2 x+6 y+2 z=0$; \\
\textbf{C3.} $4 x^2-4 x y+y^2+6 x+1=0$ ETIS teńlemesi berilgen. Múyesh koefficienti $k$ tiń qanday mánislerinde $y=kx$ tuwrı sızıq: 1) bul iymek sızıqtı bir noqatta kesip ótiwi; 2) urınadı; 3) eki noqatta kesip ótiwin; 4) bul tuwrı menen ulıwma noqatqa iye bolmaytuģının anıqlań. \\

\end{tabular}
\vspace{1cm}


\begin{tabular}{m{17cm}}
\textbf{37-variant}
\newline

\textbf{T1.} Ekinshi tártipli betliklerdiń ulıwmalıq teńlemesin kanonikalıq túrge keltiriw (invariantlar járdeminde) \\
\textbf{T2.} Parabola hám onıń kanonikalıq teńlemeleri (Fokus (bagdarlawshı noqat), Direktrisa (bagdarlawshı sızıq), Kósher (simmetriya kósheri)) \\
\textbf{A1.} Fokusları abscissa kósherinde jatqan hám koordinatalar basına salıstırģanda simmetriyalı bolǵan ellipstiń teńlemesin dúziń, bunda onıń kishi kósheri 24 ke, fokusları arasındaǵı aralıq bolsa $c = 10$ ga teń; \\
\textbf{A2.} Koordinatalar sistemasın túrlendirmesten, tómendegi teńlemelerdiń hár biri parabolanı anıqlawın kórsetiń hám parametrin tabıń: $9 x^2-6 x y+y^2-50 x+50 y-275=0$. \\
\textbf{A3.} Berilgen teńleme menen qanday iymek sızıq anıqlanıwın tabıń: $\left\{\begin{array}{l}\frac{x^2}{4}-\frac{y^2}{3}=2 z \\ x-2 y+2=0 ;\end{array}\right.$ \\
\textbf{B1.} Ellipstegi ekssentrisitetti anıqlań, eger: ellips orayınan onıń direktrisasına túsirilgen perpendikulyar kesindisi ellipstiń tóbesi menen teń ekige bólinedi. \\
\textbf{B2.} Eger parabolanıń fokusı $F(2;-1) $ hám direktrisa $x-y-1=0$ teńlemesi berilgen bolsa, onıń teńlemesin dúziń. \\
\textbf{B3.} Parallel kóshiriw hám burıw túrlendiriwleri yamasa aǵzalardı gruppalaw járdeminde tómendegi betliklerdiń kórinisi hám jaylasıwı anıqlansın: $x^2+2 x y+y^2-z^2=0$; \\
\textbf{C1.} Ulıwma fokusqa hám ústpe-úst túsken, biraq qarama-qarsı baǵıtlangan kósherlerge iye bolǵan parabolalardıń tuwrı múyesh astında kesilisiwin dálilleń. \\
\textbf{C2.} $A x+B y+C=0$ tuwrı sızıq qanday zárúrli hám jeterli shárt orınlanǵanda $\frac{x^2}{a^2}+\frac{y^2}{b^2}=1$ ellips penen 1) kesilisedi; 2) kesilispeydi. \\
\textbf{C3.} Tómendegi betliklerdiń kanonikalıq teńlemesi hám jaylasıwın anıqlań.: $2 x^2+5 y^2+2 z^2-2 x y+2 y z-4 x z+2 x-10 y-2 z-1=0$. \\

\end{tabular}
\vspace{1cm}


\begin{tabular}{m{17cm}}
\textbf{38-variant}
\newline

\textbf{T1.} Ekinshi tártipli sızıq orayı (Oraylıq sızıqlar (ellips, giperbola), Oray koordinataları: simmetriya orayı) \\
\textbf{T2.} Ekinshi tártipli sızıq orayı (Oraylıq sızıqlar (ellips, giperbola), Oray koordinataları: simmetriya orayı) \\
\textbf{A1.} Koordinatalar sistemasın túrlendirmesten tómendegi teńlemelerdiń hár biri ellipsti anıqlawın kórsetiń hám onıń yarım kósherlerin tabıń: $8 x^2+4 x y+5 y^2+16 x+4 y-28=0$; \\
\textbf{A2.} Tómendegi sızıqlardan qaysı biri oraylıq (yaǵnıy birden-bir orayǵa iye), qaysı biri orayǵa iye emes, qaysı biri sheksiz kóp orayǵa iye ekenligin anıqlań: $4 x^2+5 x y+3 y^2-x+9 y-12=0$; \\
\textbf{A3.} Tóbesi koordinatalar basında bolǵan parabolanıń teńlemesin dúziń, bunda: parabola oń yarım tegislikte hám $Oy$ kósherine simmetriyalı jaylasqan, hám parametri $p=3$. \\
\textbf{B1.} Berilgen teńlemelerdiń parabolik ekenligin kórsetiń hám olardıń hár birin $(\alpha x+\beta y)^2+2 a_{13} x+2 a_{23} y+a_{33}=0$ kórinisinde jazıń:  $16 x^2+16 x y+4 y^2-5 x+7 y=0$; \\
\textbf{B2.} Lagranj usılınan paydalanıp, teńlemelerdi kvadratlar qosındısı túrine keltirip, tómendegi betlerdiń kórinisin anıqlań: $x^2+5 y^2+z^2+2 x y+6 x z+2 y z-2 x+6 y-10 z=0$; \\
\textbf{B3.} Berilgen sızıqlar oraylıq ekenligin kórsetiń hám hárbir iymek sızıq ushın orayınıń koordinataların tabıń: $2 x^2-6 x y+5 y^2+22 x-36 y+11=0$. \\
\textbf{C1.} Eger ekinshi dárejeli teńleme parabolik bolıp, $ (\alpha x+\beta y) ^2+2a_{13}x+2a_{23}y+a_{33}=0$ kórinisinde jazılsa, onıń shep tárepindegi diskriminant $\Delta=- (a_{13} \beta-a_{23} \alpha) ^2$ formula menen anıqlanıwın dálilleń. \\
\textbf{C2.} $y^2=2 p x$ parabolaǵa onıń $M_1\left(x_1; y_1\right) $ noqatındaǵı urınbasınıń teńlemesin dúziń. \\
\textbf{C3.} $m$ hám $n$ tiń qanday mánislerinde $m x^2+12 x y+9 y^2+4 x+n y-13=0$ teńleme: 1) oraylıq sızıqtı; 2) orayga iye bolmaǵan sızıq; 3) sheksiz kóp orayǵa iye bolǵan sızıqtı ańlatadı. \\

\end{tabular}
\vspace{1cm}


\begin{tabular}{m{17cm}}
\textbf{39-variant}
\newline

\textbf{T1.} Tegislikte ekinshi tártipli sızıqlar (Ekinshi tártipli teńleme, Kvadrat kórinisindegi teńleme, Konik sızıqlar (konuslar kesimi)) \\
\textbf{T2.} Ekinshi tártipli betliklerdiń ulıwma teńlemeleri (Ulıwma teńleme) \\
\textbf{A1.} Fokusları abscissa kósherinde jatqan hám koordinatalar basına salıstırģanda simmetriyalı bolǵan ellipstiń teńlemesin dúziń, bunda: úlken kósheri 8, direktrisaları arasındaǵı aralıq 16; \\
\textbf{A2.} $x-2=0$ tegislik $\frac{x^2}{16}+\frac{y^2}{12}+\frac{z^2}{4}=1$ ellipsoidti ellips boyınsha kesip ótetuģının kórsetiń; onıń yarım kósherleri hám tóbelerin tabıń. \\
\textbf{A3.} Berilgen teńleme qanday iymek sızıq ekenligin tabıń: $y=+\frac{2}{5} \sqrt{x^2+25}$ \\
\textbf{B1.} $\frac{x^2}{16}-\frac{y^2}{8}=-1$ giperbolaǵa $2 x+4 y-5=0$ tuwrısına parallel urınbalar ótkiziń hám olar arasındaǵı $d$ aralıqtı esaplań. \\
\textbf{B2.} Kósherleri koordinata kósherleri menen ústpe-úst túsetuģin hám $P (2,2); Q (3,1) $ noqatlar arqalı ótiwshi ellips teńlemesin dúziń. \\
\textbf{B3.} Eger parabolanıń fokusı $F (7; 2) $ hám direktrisa $x-5=0$ teńlemesi berilgen bolsa, onıń teńlemesin dúziń. \\
\textbf{C1.} $\frac{x^2}{9}+\frac{z^2}{4}=2 y$ elliptik paraboloid $2 x-2 y-z-10=0$ tegislik penen bir ulıwma noqatqa iye ekenligin dálilleń hám onıń koordinataların tabıń. \\
\textbf{C2.} $\frac{x^2}{a^2}-\frac{y^2}{b^2}=1$ giperbolanıń asimptotaları hám onıń qálegen noqatınan asimptotalarga parallel etip ótkerilgen tuwrı sızıqlar menen shegaralanǵan parallelogrammnıń maydanı turaqlı san bolıp, $\frac{a b}{2}$ ga teń bolatuģının dálilleń. \\
\textbf{C3.} $A\left(\frac{10}{3}; \frac{5}{3}\right)$ noqatınan $\frac{x2}{20}+\frac{y2}{5}=1$ ellipske urınbalar ótkerilgen. Olardıń teńlemelerin dúziń. \\

\end{tabular}
\vspace{1cm}


\begin{tabular}{m{17cm}}
\textbf{40-variant}
\newline

\textbf{T1.} Ekinshi tártipli sızıqlardıń ulıwma teńlemeleri (Ulıwma teńleme) \\
\textbf{T2.} Tegislikte ekinshi tártipli sızıqlar (Ekinshi tártipli teńleme, Kvadrat kórinisindegi teńleme, Konik sızıqlar (konuslar kesimi)) \\
\textbf{A1.} Koordinatalar sistemasın túrlendirmesten, tómendegi teńlemelerdiń hár biri parabolanı anıqlawın kórsetiń hám parametrin tabıń: $x^2-2 x y+y^2+6 x-14 y+29=0$; \\
\textbf{A2.} Koordinatalar sistemasın túrlendirmesten tómendegi teńlemelerdiń hár biri birden-bir noqattı anıqlawın kórsetiń hám onıń koordinataların tabıń: $5 x^2-6 x y+2 y^2-2 x+2=0$; \\
\textbf{A3.} Fokusları abscissa kósherinde jatqan hám koordinatalar basına qarata simmetriyalı bolǵan ellipstiń teńlemesin dúziń, bunda: $M_1 (\sqrt{15};-1) $ noqatı ellipske tiyisli hám fokusları arasındaǵı aralıq $2 c=8$; \\
\textbf{B1.} Berilgen teńleme parabolik ekenligin kórsetiń; ápiwayı túrge keltiriń; qanday geometriyalıq obrazdı anlatıwın anıqlań, eski hám de jańa koordinata kósherlerine salıstırģanda sızılmada súwretleń: $9 x^2+24 x y+16 y^2-18 x+226 y+209=0$; \\
\textbf{B2.} ITECH túri, ólshemleri hám jaylasıwın anıqlań: $5 x^2+6 x y+5 y^2-16 x-16 y-16=0$; \\
\textbf{B3.} Berilgen teńleme parabolik ekenligin kórsetiń; ápiwayı túrge keltiriń; qanday geometriyalıq obrazdı anlatıwın anıqlań, eski hám de jańa koordinata kósherlerine salıstırģanda sızılmada súwretleń:$4 x^2+12 x y+9 y^2-4 x-6 y+1=0$. \\
\textbf{C1.} $\frac{x^2}{a^2}-\frac{y^2}{b^2}=1$ giperbolanıń fokusınan asimptotasına shekemgi aralıq $b$ qa teń ekenligin dálilleń. \\
\textbf{C2.} Berilgen teńleme kanonikalıq kóriniske keltirilsin; tipi anıqlansın; qanday geometriyalıq obrazdı anlatıwı anıqlansın; eski hám jana koordinatalar sistemasında geometriyalıq obrazı súwretlensin: $25 x^2-14 x y+25 y^2+64 x-64 y-224=0$; \\
\textbf{C3.} Giperbolanıń asimptotaların tabıń: $10 x^2+21 x y+9 y^2-41 x-39 y+4=0$. \\

\end{tabular}
\vspace{1cm}


\begin{tabular}{m{17cm}}
\textbf{41-variant}
\newline

\textbf{T1.} Ekinshi tártipli betliklerdiń kanonikalıq teńlemeleri (Ellipsoid, Giperboloid (1 gewekli), Giperboloid (2 gewekli)) \\
\textbf{T2.} Parabola hám onıń kanonikalıq teńlemeleri (Fokus (bagdarlawshı noqat), Direktrisa (bagdarlawshı sızıq), Kósher (simmetriya kósheri)) \\
\textbf{A1.} Diskriminantın esaplaw arqalı tómendegi teńlemelerdiń hár biriniń tipin anıqlań: $3 x^2-2 x y-3 y^2+12 y-15=0$. \\
\textbf{A2.} Koordinatalar sistemasın túrlendirmesten, tómendegi teńlemelerdiń hár biri parabolanı anıqlawın kórsetiń hám parametrin tabıń: $9 x^2-6 x y+y^2-50 x+50 y-275=0$. \\
\textbf{A3.} Berilgen teńleme menen qanday iymek sızıq anıqlanıwın tabıń: $\left\{\begin{array}{l}\frac{x^2}{3}+\frac{y^2}{6}=2 z, \\ 3 x-y+6 z-14=0\end{array}\right.$ \\
\textbf{B1.} Parallel kóshiriw hám burıw túrlendiriwleri yamasa aǵzalardı gruppalaw járdeminde tómendegi betliklerdiń kórinisi hám jaylasıwı anıqlansın: $3 x^2+3 y^2-3 z^2-6 x+4 y+4 z+3=0$; \\
\textbf{B2.} $\frac{x^2}{16}+\frac{y^2}{9}=1$ ellipsning $x+y-1=0$ to'g'ri chiziqqa parallel bo'lgan urinmalarini aniqlang. \\
\textbf{B3.} Berilgen sızıqlar oraylıq ekenligin kórsetiń hám hárbir iymek sızıq ushın orayınıń koordinataların tabıń: $3x^2+5xy+y^2-8x-11y-7=0$. \\
\textbf{C1.} $m$ nıń qanday mánislerinde $y=\frac{5}{2} x+m$ tuwrı sızıq $\frac{x^2}{9}-\frac{y^2}{36}=1$ giperbolanı 1) kesip ótiwin; 2) oǵan urınıwın; 3) sırtınan ótiwin anıqlań. \\
\textbf{C2.} Berilgen $y=k x+b$ tuwri sızıqqa parallel hám $y^2=2 p x$ parabolaga urinatuģın tuwri sızıqtıń teńlemesin jazıń. \\
\textbf{C3.} $\frac{x^2}{a^2}+\frac{y^2}{b^2}=1$ ellipstiń $M_1 (x_1; y_1) $ noqatındaǵı urınbasınıń teńlemesin dúziń. \\

\end{tabular}
\vspace{1cm}


\begin{tabular}{m{17cm}}
\textbf{42-variant}
\newline

\textbf{T1.} Ekinshi tártipli betlik orayı, urınba tegisligi hám diametral tegisligi (Oray, Urınba tegislik, Diametral tegislik.) \\
\textbf{T2.} Ekinshi tártipli sızıqqa urınba, túyinles diametri teńlemesi (Urınba teńlemesi, Túyinles diametr: oraydan ótiwshi simmetriya kósherleri) \\
\textbf{A1.} Tómendegi sızıqlardan qaysı biri oraylıq (yaǵnıy birden-bir orayǵa iye), qaysı biri orayǵa iye emes, qaysı biri sheksiz kóp orayǵa iye ekenligin anıqlań: $4 x^2-20 x y+25 y^2-14 x+2 y-15=0$; \\
\textbf{A2.} Tómendegi maglıwmatlar boyınsha giperbolanıń kanonikalıq teńlemesin dúziń: asimptotaları arasındaǵı múyesh $60^{\circ}$ qa teń hám $c=2 \sqrt{3}$. \\
\textbf{A3.} $x^2=4 y$ parabola fokusınıń koordinataların anıqlań. \\
\textbf{B1.} Berilgen teńlemeni ápiwayı túrge keltiriń; tipin anıqlań; qanday geometriyalıq obrazdı ańlatıwın anıqlań, eski hám de jańa koordinata kósherlerine qarata sızılmada súwretleń:  $5 x^2+24 x y-5 y^2=0$; \\
\textbf{B2.} $\frac{x^2}{5}-\frac{y^2}{4}=1$ giperbolaģa $ (5,-4) $ noqatta urınatuģın tuwrı sızıq teńlemesi jazılsın. \\
\textbf{B3.} Eger parabolanıń fokusı $F (4;3) $ hám direktrisa $x-1=0$ teńlemesi berilgen bolsa, onıń teńlemesin dúziń. \\
\textbf{C1.} $A x+B y+C=0$ tuwrı sızıq qanday zárúrli hám jeterli shárt orınlanǵanda $\frac{x^2}{a^2}+\frac{y^2}{b^2}=1$ ellips penen 1) kesilisedi; 2) kesilispeydi. \\
\textbf{C2.} Giperbola asimptotalarınıń tenlemeleri $y= \pm \frac{1}{2} x$ hám urinbalardan biriniń teńlemesi. $5 x-6 y-8=0$ belgili bolsa, giperbola teńlemesin dúziń. \\
\textbf{C3.} $m$ hám $n$ tiń qanday mánislerinde $m x^2+12 x y+9 y^2+4 x+n y-13=0$ teńleme: 1) oraylıq sızıqtı; 2) orayga iye bolmaǵan sızıq; 3) sheksiz kóp orayǵa iye bolǵan sızıqtı ańlatadı. \\

\end{tabular}
\vspace{1cm}


\begin{tabular}{m{17cm}}
\textbf{43-variant}
\newline

\textbf{T1.} Ekinshi tártipli betliklerdiń kanonikalıq teńlemeleri (Paraboloid (ellipstik), Paraboloid (giperbolik), Konus, Cilindr) \\
\textbf{T2.} Parabola hám onıń kanonikalıq teńlemeleri (Fokus (bagdarlawshı noqat), Direktrisa (bagdarlawshı sızıq), Kósher (simmetriya kósheri)) \\
\textbf{A1.} Tóbesi koordinatalar basında bolǵan parabolanıń teńlemesin dúziń, bunda: parabola $Oy$ kósherine simmetriyalı jaylasqan hám $C (1; 1) $ noqatınan ótedi; \\
\textbf{A2.} Fokusları ordinata kósherinde, koordinatalar basına salıstırǵanda simmetriyalı jaylasqan giperbolanıń teńlemesi dúzilsin, bunda: onıń yarım kósherleri. $a=6, b=18$; \\
\textbf{A3.} Berilgen teńleme menen qanday iymek sızıq anıqlanıwın tabıń: $\left\{\begin{array}{l}\frac{x^2}{.4}+\frac{y^2}{9}-\frac{z^2}{36}=1, \\ 9 x-6 y+2 z-28=0,\end{array}\right.$ \\
\textbf{B1.} Tómendegilerdi bilgen halda giperbolanıń teńlemesin dúziń: Asimptotaları arasındaǵı múyesh $90^{\circ}$ qa teń hám fokuslar $F_1 (4;-4), F_2 (-2; 2) $. \\
\textbf{B2.} Berilgen sızıqlar oraylıq ekenligin kórsetiń hám hárbir iymek sızıq ushın orayınıń koordinataların tabıń:$5 x^2+4 x y+2 y^2+20 x+20 y-18=0$; \\
\textbf{B3.} Berilgen teńlemelerdiń parabolik ekenligin kórsetiń hám olardıń hár birin $(\alpha x+\beta y)^2+2 a_{13} x+2 a_{23} y+a_{33}=0$ kórinisinde jazıń: $9 x^2-6 x y+y^2-x+2 y-14=0$; \\
\textbf{C1.} Tómendegi betliklerdiń kanonikalıq teńlemesi hám jaylasıwın anıqlań.: $2 x^2+y^2+2 z^2-2 x y+2 y z+4 x-2 y=0$. \\
\textbf{C2.} Eki gewekli $\frac{x^2}{3}+\frac{y^2}{4}-\frac{z^2}{25}=-1$ giperboloid $5 x+2 z+5=0$ tegislik penen bir ulıwma noqatqa iye ekenligin dálilleń hám onıń koordinataların tabıń. \\
\textbf{C3.} Giperbolanıń asimptotaların tabıń: $3 x^2+7 x y+4 y^2+5 x+2 y-6=0$; \\

\end{tabular}
\vspace{1cm}


\begin{tabular}{m{17cm}}
\textbf{44-variant}
\newline

\textbf{T1.} Ekinshi tártipli sızıqlardıń ulıwma teńlemesin invariantlar járdeminde kanonikalıq túrge keltiriw \\
\textbf{T2.} Tegislikte ekinshi tártipli sızıqlar (Ekinshi tártipli teńleme, Kvadrat kórinisindegi teńleme, Konik sızıqlar (konuslar kesimi)) \\
\textbf{A1.} Tómendegi sızıqlardan qaysı biri oraylıq (yaǵnıy birden-bir orayǵa iye), qaysı biri orayǵa iye emes, qaysı biri sheksiz kóp orayǵa iye ekenligin anıqlań: $x^2-6 x y+9 y^2-12 x+36 y+20=0$; \\
\textbf{A2.} Fokusları abscissa kósherinde jatqan hám koordinatalar basına qarata simmetriyalı bolǵan ellipstiń teńlemesin dúziń, bunda: $M_1 (-2 \sqrt{5}; 2) $ noqatı ellipske tiyisli hám kishi yarım kósheri $b=3$; \\
\textbf{A3.} Koordinatalar sistemasın túrlendirmesten tómendegi teńlemelerdiń hár biri kesilisiwshi eki tuwrını anıqlawın kórsetiń hám onıń koordinataların tabıń: $x^2-6 x y+8 y^2-4 y-4=0$; \\
\textbf{B1.} Parabola tóbesi $A (-2;-1) $ hám onıń direktrisasınıń teńlemesi $x+2y-1=0$ berilgen. Bul parabolanıń teńlemesin dúziń. \\
\textbf{B2.} ITECH túri, ólshemleri hám jaylasıwın anıqlań: $9 x^2+24 x y+16 y^2-230 x+110 y-475=0$. \\
\textbf{B3.} Parallel kóshiriw hám burıw túrlendiriwleri yamasa aǵzalardı gruppalaw járdeminde tómendegi betliklerdiń kórinisi hám jaylasıwı anıqlansın: $z^2=3 x+4 y+5$; \\
\textbf{C1.} Berilgen teńleme kanonikalıq kóriniske keltirilsin; tipi anıqlansın; qanday geometriyalıq obrazdı anlatıwı anıqlansın; eski hám jana koordinatalar sistemasında geometriyalıq obrazı súwretlensin: $3 x^2+10 x y+3 y^2-2 x-14 y-13=0$; \\
\textbf{C2.} $y^2=2 p x$ parabolaǵa $y=k x+b$ tuwrı sızıq urınıw shártin keltirip shigarıń. \\
\textbf{C3.} Ekinshi dárejeli teńleme tek hám tek $\Delta=0$ bolǵanda ǵana aynıǵan iymek sızıq teńlemesi bolatuģının dálilleń. \\

\end{tabular}
\vspace{1cm}


\begin{tabular}{m{17cm}}
\textbf{45-variant}
\newline

\textbf{T1.} Ekinshi tártipli sızıq hám tuwrı sızıqtıń óz ara jaylasıwı (Kesilisiw noqatları, Urınba (urınıw) jaģdayı) \\
\textbf{T2.} Bir gewekli giperboloid hám giperbolik paraboloidtıń tuwrı sızıqlı jasawshıları (Giperboloid, Giperbolik paraboloid, Sızıqlı jasawshılar) \\
\textbf{A1.} Koordinatalar sistemasın túrlendirmesten, tómendegi teńlemelerdiń hár biri parabolanı anıqlawın kórsetiń hám parametrin tabıń: $9 x^2-24 x y+16 y^2-54 x-178 y+181=0$; \\
\textbf{A2.} $y^2+z^2=x$ elliptik paraboloidtıń $x+2 y-z=0$ tegislik penen kesilisiwiniń koordinata tegisliklerindegi proekciyalarınıń teńlemelerin tabıń. \\
\textbf{A3.} Koordinatalar sistemasın túrlendirmesten, tómendegi teńlemelerdiń hár biri parabolanı anıqlawın kórsetiń hám parametrin tabıń: $9 x^2+24 x y+16 y^2-120 x+90 y=0$; \\
\textbf{B1.} Eki tóbesi $x^2+5 y^2=20$ ellipstiń fokuslarında jatıwshı, qalǵan ekewi bolsa onıń kishi kósheri tóbeleri menen ústpe-úst túsiwshi tórtmúyeshliktiń maydanın esaplań. \\
\textbf{B2.} $\varepsilon=\frac{2}{5}$ ellipstiń ekssentrisiteti, ellipstiń $M$ noqatınan direktrisaģa shekemgi aralıq 20 ģa teń. $M$ noqattan usı direktrisa menen bir tárepleme fokusqa shekem bolǵan aralıqtı esaplań. \\
\textbf{B3.} Parallel kóshiriw hám burıw túrlendiriwleri yamasa aǵzalardı gruppalaw járdeminde tómendegi betliklerdiń kórinisi hám jaylasıwı anıqlansın: $x^2+y^2+2 z^2+2 x y+4 z=0$; \\
\textbf{C1.} $4 x^2-4 x y+y^2+6 x+1=0$ ETIS teńlemesi berilgen. Múyesh koefficienti $k$ tiń qanday mánislerinde $y=kx$ tuwrı sızıq: 1) bul iymek sızıqtı bir noqatta kesip ótiwi; 2) urınadı; 3) eki noqatta kesip ótiwin; 4) bul tuwrı menen ulıwma noqatqa iye bolmaytuģının anıqlań. \\
\textbf{C2.} $A x+B y+C=0$ tuwri sızıq $y^2=2 p x$ parabolaga urinıwı ushin zárúrli hám jeterli shártti tabıń. \\
\textbf{C3.} Giperbolanıń asimptotaların tabıń: $10 x^2+21 x y+9 y^2-41 x-39 y+4=0$. \\

\end{tabular}
\vspace{1cm}


\begin{tabular}{m{17cm}}
\textbf{46-variant}
\newline

\textbf{T1.} Tegislikte ekinshi tártipli sızıqlar (Ekinshi tártipli teńleme, Kvadrat kórinisindegi teńleme, Konik sızıqlar (konuslar kesimi)) \\
\textbf{T2.} Ekinshi tártipli betliklerdiń kanonikalıq teńlemeleri (Paraboloid (ellipstik), Paraboloid (giperbolik), Konus, Cilindr) \\
\textbf{A1.} Parabolanıń tóbesi ($\alpha;\beta$) noqat penen ústpe-úst túsetuģının bilgen halda onıń teńlemesin dúziń. Parametri $p$ ǵa teń. Onıń kósheri $O x$ kósherine parallel bolıp, $O x$ kósheriniń oń baǵıtında sheksizlikke sozilgan; \\
\textbf{A2.} Koordinatalar sistemasın túrlendirmesten, tómendegi teńlemeler menen qanday geometriyalıq obrazdı anıqlanıwın tabıń: $8 x^2-12 x y+17 y^2+16 x-12 y+3=0$; \\
\textbf{A3.} Fokusları abscissa kósherinde jaylasqan, koordinatalar basına salıstırģanda simmetriyalı bolǵan giperbolanıń teńlemesin dúziń, bunda: direktrisalarınıń arasındaģı aralıq $\frac{8}{3}$ hám ekssentrisiteti $\varepsilon=\frac{3}{2}$; \\
\textbf{B1.} Fokusları $\frac{x^2}{100}+\frac{y^2}{64}=1$ ellipstiń tóbelerinde jatıwshı, direktrisaları bolsa usı ellipstiń fokuslarınan ótiwshi giperbolanıń teńlemesin dúziń. \\
\textbf{B2.} Berilgen teńlemeler oraylıq iymek sızıqlar ekenligin kórsetiń hám hárbir teńlemeni koordinata basın orayģa kóshiriń: $6 x^2+4 x y+y^2+4 x-2 y+2=0$; \\
\textbf{B3.} Berilgen parabola tóbesi $A (6;-3) $ hám onıń direktrisasınıń teńlemesi $3x-5y+1=0$ berilgen. Bul parabolanıń $F$ fokusın tabıń. \\
\textbf{C1.} Kósherleri óz ara perpendikulyar bolǵan eki parabola tórt noqatta kesilisse, bul noqatlar bir sheńberde jatıwın dálilleń. \\
\textbf{C2.} Tómendegi betliklerdiń kanonikalıq teńlemesi hám jaylasıwın anıqlań.: $2 x^2+2 y^2-5 z^2+2 x y-2 x-4 y-4 z+2=0$. \\
\textbf{C3.} $\frac{x^2}{a^2}-\frac{y^2}{b^2}=1$ giperbolanıń fokuslarınan urınbasına shekemgi aralıqlardıń kóbeymesin tabıń. \\

\end{tabular}
\vspace{1cm}


\begin{tabular}{m{17cm}}
\textbf{47-variant}
\newline

\textbf{T1.} Ekinshi tártipli sızıqlardıń ulıwma teńlemesin invariantlar járdeminde kanonikalıq túrge keltiriw \\
\textbf{T2.} Ekinshi tártipli betliklerdiń ulıwmalıq teńlemesin kanonikalıq túrge keltiriw (invariantlar járdeminde) \\
\textbf{A1.} Tómendegi sızıqlardan qaysı biri oraylıq (yaǵnıy birden-bir orayǵa iye), qaysı biri orayǵa iye emes, qaysı biri sheksiz kóp orayǵa iye ekenligin anıqlań:  $x^2-2 x y+y^2-6 x+6 y-3=0$; \\
\textbf{A2.} Fokusları abscissa kósherinde jatqan hám koordinatalar basına salıstırģanda simmetriyalı bolǵan ellipstiń teńlemesin dúziń, bunda: direktrisaları arasındaǵı aralıq 32 hám $\varepsilon=\frac{1}{2}$. \\
\textbf{A3.} Tómendegi sızıqlardan qaysı biri oraylıq (yaǵnıy birden-bir orayǵa iye), qaysı biri orayǵa iye emes, qaysı biri sheksiz kóp orayǵa iye ekenligin anıqlań: $4 x^2-4 x y+y^2-6 x+8 y+13=0$; \\
\textbf{B1.} Berilgen teńleme parabolik ekenligin kórsetiń; ápiwayı túrge keltiriń; qanday geometriyalıq obrazdı anlatıwın anıqlań, eski hám de jańa koordinata kósherlerine salıstırģanda sızılmada súwretleń:$9 x^2+12 x y+4 y^2-24 x-16 y+3=0$; \\
\textbf{B2.} Berilgen teńlemeniń tipin anıqlań, koordinata kósherlerin parallel kóshiriw arqalı ápiwayı túrge keltiriń; qanday geometriyalıq obrazdı ańlatıwın anıqlań, eski hám jańa koordinata kósherlerine salıstırģanda sızılmada súwretleń: $2 x^2+3 y^2+8 x-6 y+11=0$. \\
\textbf{B3.} Berilgen teńleme parabolik ekenligin kórsetiń; ápiwayı túrge keltiriń; qanday geometriyalıq obrazdı anlatıwın anıqlań, eski hám de jańa koordinata kósherlerine salıstırģanda sızılmada súwretleń:$9 x^2-24 x y+16 y^2-20 x+110 y-50=0$; \\
\textbf{C1.} $m$ nıń qanday mánislerinde $x-2 y-2 z+m=0$ tegislik $\frac{x^2}{144}+\frac{y^2}{36}+\frac{z^2}{9}=1$ ellipsoidqa urınıwın anıqlań. \\
\textbf{C2.} Hár qanday parabolik teńleme ushın $a_{11}$ hám $a_{22}$ koefficientler hár qıylı belgige iye sanlar bola almaytuģının hám olar bir waqıtta nolge aylana almaytuģının dálilleń. \\
\textbf{C3.} $A x+B y+C=0$ tuwrı sızıqtıń $\frac{x^2}{a^2}+\frac{y^2}{b^2}=1$, ellipske urınba bolıwı ushın zárúrli hám jeterli shárti tabılsın. \\

\end{tabular}
\vspace{1cm}


\begin{tabular}{m{17cm}}
\textbf{48-variant}
\newline

\textbf{T1.} Parabola hám onıń kanonikalıq teńlemeleri (Fokus (bagdarlawshı noqat), Direktrisa (bagdarlawshı sızıq), Kósher (simmetriya kósheri)) \\
\textbf{T2.} Ekinshi tártipli sızıqlardıń ulıwma teńlemeleri (Ulıwma teńleme) \\
\textbf{A1.} Diskriminantın esaplaw arqalı tómendegi teńlemelerdiń hár biriniń tipin anıqlań: $3 x^2-8 x y+7 y^2+8 x-15 y+20=0$; \\
\textbf{A2.} $y+6=0$ tegislik $\frac{x^2}{5}-\frac{y^2}{4}=6 z$ giperbolik paraboloidti parabola boyınsha kesip ótetuģının kórsetiń; parametrin hám tóbesin tabıń. \\
\textbf{A3.} Giperbolanıń ekssentrisiteti $\varepsilon=2$ ǵa teń, $M$ noqatınıń bazı bir fokal radiusı 16 ǵa teń. $M$ noqattan sáykes direktrisaģa shekem bolǵan aralıqtı tabıń. \\
\textbf{B1.} Parabola tóbesiniń koordinataların, parametrin hám kósheriniń baǵıtın anıqlań: $y=x^2+6 x$. \\
\textbf{B2.} Berilgen teńlemeler oraylıq iymek sızıqlar ekenligin kórsetiń hám hárbir teńlemeni koordinata basın orayģa kóshiriń: $4 x^2+6 x y+y^2-10 x-10=0$; \\
\textbf{B3.} Giperbolanıń asimptotaları arasındaǵı múyeshin tabıń, eger: fokusları arasındaǵı qashıqlıq direktrisaları arasındaǵı qashıqlıqtan eki ese úlken bolsa. \\
\textbf{C1.} Qálegen elliptik teńleme ushın $a_{11}$ hám $a_{22}$ koefficientleriniń hesh biri nolge aylana almaytuģınlıǵın hám olar birdey belgige iye sanlar ekenligin dálilleń. \\
\textbf{C2.} Eger giperbolanıń yarım kósherleri $a$ hám $b$, orayı $C\left(x_0; y_0\right) $ hám fokuslar tómendegi tuwrı sızıqta jaylasqan: 1) $O x$ kósherine parallel; 2) $O y$ kósherine parallel bolsa, onıń teńlemesin dúziń. \\
\textbf{C3.} $\frac{x^2}{a^2}+\frac{y^2}{b^2}=1$ ellipstiń bir diametriniń tóbelerine júrgizilgen urınbalar parallel bolıwın dálilleń (ellipstiń diametri dep onıń orayınan ótiwshi xordaǵa aytıladı). \\

\end{tabular}
\vspace{1cm}


\begin{tabular}{m{17cm}}
\textbf{49-variant}
\newline

\textbf{T1.} Ekinshi tártipli sızıqqa urınba, túyinles diametri teńlemesi (Urınba teńlemesi, Túyinles diametr: oraydan ótiwshi simmetriya kósherleri) \\
\textbf{T2.} Bir gewekli giperboloid hám giperbolik paraboloidtıń tuwrı sızıqlı jasawshıları (Giperboloid, Giperbolik paraboloid, Sızıqlı jasawshılar) \\
\textbf{A1.} Parabolanıń teńlemesin dúziń, eger: parabolanıń tóbesinen fokusına shekemgi aralıq 3 ke teń hám parabola $O x$ kósherine qarata simmetriyalı bolıp, $O y$ kósherine urınsa; \\
\textbf{A2.} Koordinatalar sistemasın túrlendirmesten, tómendegi teńlemelerdiń hár biri parabolanı anıqlawın kórsetiń hám parametrin tabıń: $9 x^2-24 x y+16 y^2-54 x-178 y+181=0$; \\
\textbf{A3.} Ekscentrisiteti $\varepsilon=\frac{1}{2}$, fokusı $F (-4; 1) $ hám usı fokus tárepindegi direktrisası $y+3=0$ bolǵan ellipstiń teńlemesin dúziń. \\
\textbf{B1.} ITECH túri, ólshemleri hám jaylasıwın anıqlań: $5 x^2+8 x y+5 y^2-18 x-18 y+9=0$; \\
\textbf{B2.} Úlken kósheri 2 birlikke teń, fokusları $F_1 (0,1), F_2 (1,0) $ noqatlarda bolgan ellipstiń teńlemesin dúziń. \\
\textbf{B3.} Parallel kóshiriw hám burıw túrlendiriwleri yamasa aǵzalardı gruppalaw járdeminde tómendegi betliklerdiń kórinisi hám jaylasıwı anıqlansın: $x^2+y^2-z^2-2 x y+2 z-1=0$; \\
\textbf{C1.} Parabolanıń qálegen urinbasınıń direktrisası hám kósherge perpendikulyar bolǵan fokal xordanı fokustan teńdey uzaqlıqtaģı noqatlarda kesetuģının dálilleń. \\
\textbf{C2.} Fokuslardan ellipstiń qálegen urınbasına shekem bolǵan aralıqlar kóbeymesi kishi yarım kósherdiń kvadratına teń ekenligin dálilleń. \\
\textbf{C3.} Berilgen teńleme kanonikalıq kóriniske keltirilsin; tipi anıqlansın; qanday geometriyalıq obrazdı anlatıwı anıqlansın; eski hám jana koordinatalar sistemasında geometriyalıq obrazı súwretlensin: $29 x^2-24 x y+36 y^2+82 x-96 y-91=0$; \\

\end{tabular}
\vspace{1cm}


\begin{tabular}{m{17cm}}
\textbf{50-variant}
\newline

\textbf{T1.} Tegislikte ekinshi tártipli sızıqlar (Ekinshi tártipli teńleme, Kvadrat kórinisindegi teńleme, Konik sızıqlar (konuslar kesimi)) \\
\textbf{T2.} Ekinshi tártipli sızıq orayı (Oraylıq sızıqlar (ellips, giperbola), Oray koordinataları: simmetriya orayı) \\
\textbf{A1.} Fokusları ordinata kósherinde jatqan hám koordinatalar basına salıstırģanda simmetriyalı bolǵan ellipstiń teńlemesin dúziń, bunda: yarım kósherleri 7 hám 2; \\
\textbf{A2.} Koordinatalar sistemasın túrlendirmesten tómendegi teńlemelerdiń hár biri giperbolanı anıqlawın kórsetiń hám onıń koordinataların tabıń: $4 x^2+24 x y+11 y^2+64 x+42 y+51=0$; \\
\textbf{A3.} $y^2+z^2=x$ elliptik paraboloidtıń $x+2 y-z=0$ tegislik penen kesilisiwiniń koordinata tegisliklerindegi proekciyalarınıń teńlemelerin tabıń. \\
\textbf{B1.} Lagranj usılınan paydalanıp, teńlemelerdi kvadratlar qosındısı túrine keltirip, tómendegi betlerdiń kórinisin anıqlań: $x^2-2 y^2+z^2+4 x y-10 x z+4 y z+2 x+4 y-10 z-1=0$; \\
\textbf{B2.} ITECH túri, ólshemleri hám jaylasıwın anıqlań: $5 x^2+12 x y-12 x-22 y-19=0$. \\
\textbf{B3.} Eki tóbesi $x^2+5 y^2=20$ ellipstiń fokuslarında jatıwshı, qalǵan ekewi bolsa onıń kishi kósheri tóbeleri menen ústpe-úst túsiwshi tórtmúyeshliktiń maydanın esaplań. \\
\textbf{C1.} Eki gewekli $\frac{x^2}{3}+\frac{y^2}{4}-\frac{z^2}{25}=-1$ giperboloid $5 x+2 z+5=0$ tegislik penen bir ulıwma noqatqa iye ekenligin dálilleń hám onıń koordinataların tabıń. \\
\textbf{C2.} $m$ hám $n$ tiń qanday mánislerinde $m x^2+12 x y+9 y^2+4 x+n y-13=0$ teńleme: 1) oraylıq sızıqtı; 2) orayga iye bolmaǵan sızıq; 3) sheksiz kóp orayǵa iye bolǵan sızıqtı ańlatadı. \\
\textbf{C3.} $\frac{x^2}{100}+\frac{y^2}{64}=1$ ellipstiń $2 x-y+7=0,2 x-y-1=0$ xordalarınıń ortaları arqalı ótetuģın tuwrı sızıqtıń teńlemesin dúziń. \\

\end{tabular}
\vspace{1cm}


\begin{tabular}{m{17cm}}
\textbf{51-variant}
\newline

\textbf{T1.} Parabola hám onıń kanonikalıq teńlemeleri (Fokus (bagdarlawshı noqat), Direktrisa (bagdarlawshı sızıq), Kósher (simmetriya kósheri)) \\
\textbf{T2.} Ekinshi tártipli betliklerdiń ulıwma teńlemeleri (Ulıwma teńleme) \\
\textbf{A1.} $y^2=4 x$ parabola fokusınıń koordinataların anıqlań. \\
\textbf{A2.} Koordinatalar sistemasın túrlendirmesten, tómendegi teńlemelerdiń hár biri parabolanı anıqlawın kórsetiń hám parametrin tabıń: $9 x^2-6 x y+y^2-50 x+50 y-275=0$. \\
\textbf{A3.} Ekscentrisiteti $\varepsilon=\frac{5}{4}$, bir fokusı $F (5; 0) $ hám oǵan sáykes direktrisasınıń teńlemesi $5x-16=0$ bolǵan giperbolanıń teńlemesin dúziń. \\
\textbf{B1.} Berilgen teńlemelerdiń parabolik ekenligin kórsetiń hám olardıń hár birin $(\alpha x+\beta y)^2+2 a_{13} x+2 a_{23} y+a_{33}=0$ kórinisinde jazıń: $25 x^2-20 x y+4 y^2+3 x-y+11=0$; \\
\textbf{B2.} Parabola tóbesiniń koordinataların, parametrin hám kósheriniń baǵıtın anıqlań: $y^2-10 x-2 y-19=0$; \\
\textbf{B3.} $x^2-y^2=8$ giperbolaga $M(3,-1)$ noqatında urınatuģın tuwrı sızıqtıń teńlemesin jazıń. \\
\textbf{C1.} Giperbolanıń asimptotaların tabıń: $x^2-3 x y-10 y^2+6 x-8 y=0$; \\
\textbf{C2.} Parabolik teńleme $\Delta \neq 0$ bolǵanda hám tek sonda ǵana parabolanı anıqlaytuģının dálilleń. Bul jaǵdayda parabolanıń parametri $p=\sqrt{\frac{-\Delta}{ (a_{11}+a_{33}) ^3}}$ formula menen anıqlanıwın dálilleń. \\
\textbf{C3.} Tómendegi betliklerdiń kanonikalıq teńlemesi hám jaylasıwın anıqlań.: $4 x^2+9 y^2+z^2-12 x y-6 y z+4 z x+4 x-6 y+2 z-5=0$. \\

\end{tabular}
\vspace{1cm}


\begin{tabular}{m{17cm}}
\textbf{52-variant}
\newline

\textbf{T1.} Ekinshi tártipli betlik orayı, urınba tegisligi hám diametral tegisligi (Oray, Urınba tegislik, Diametral tegislik.) \\
\textbf{T2.} Parabola hám onıń kanonikalıq teńlemeleri (Fokus (bagdarlawshı noqat), Direktrisa (bagdarlawshı sızıq), Kósher (simmetriya kósheri)) \\
\textbf{A1.} Tómendegi sızıqlardan qaysı biri oraylıq (yaǵnıy birden-bir orayǵa iye), qaysı biri orayǵa iye emes, qaysı biri sheksiz kóp orayǵa iye ekenligin anıqlań: $25 x^2-10 x y+y^2+40 x-8 y+7=0$. \\
\textbf{A2.} Koordinatalar sistemasın túrlendirmesten, tómendegi teńlemeler menen qanday geometriyalıq obrazdı anıqlanıwın tabıń: $6 x^2-6 x y+9 y^2-4 x+18 y+14=0$; \\
\textbf{A3.} Parabolanıń teńlemesin dúziń, eger: parabola $O x$ kósherine qarata simmetriyalı bolıp, $M (1;-4) $ noqatınan hám koordinatalar basınan ótedi; \\
\textbf{B1.} Berilgen teńlemeler oraylıq iymek sızıqlar ekenligin kórsetiń hám hárbir teńlemeni koordinata basın orayģa kóshiriń: $3x^2-6xy+2y^2-4x+2y+1=0$. \\
\textbf{B2.} Bes noqattan ótiwshi ekinshi tártipli sızıqtıń teńlemesin dúziń: $(0,0),(0,1),(1,0),(2,-5),(-5,2)$. \\
\textbf{B3.} $A (5;9) $ noqattan $y^2=5x$ parabolaǵa júrgizilgen urınbalardıń urınıw noqatların tutastırıwshı xordanıń teńlemesin dúziń. \\
\textbf{C1.} $\frac{x^2}{a^2}-\frac{y^2}{b^2}=1$ giperbolanıń qálegen noqatınan onıń eki asimptotasına shekemgi aralıqlar kóbeymesi $\frac{a^2 b^2}{a^2+b^2}$ ǵa teń turaqlı shama ekenligin dálilleń. \\
\textbf{C2.} Ulıwma kósherge hám tóbeleri arasında jaylasqan ulıwma fokusqa iye bolǵan eki parabola tuwrı múyesh astında kesilisetuģının dálilleń. \\
\textbf{C3.} $\frac{x^2}{a^2}-\frac{y^2}{b^2}=1$ giperbolanıń fokusınan asimptotasına shekemgi aralıq $b$ qa teń ekenligin dálilleń. \\

\end{tabular}
\vspace{1cm}


\begin{tabular}{m{17cm}}
\textbf{53-variant}
\newline

\textbf{T1.} Ekinshi tártipli sızıq hám tuwrı sızıqtıń óz ara jaylasıwı (Kesilisiw noqatları, Urınba (urınıw) jaģdayı) \\
\textbf{T2.} Ekinshi tártipli sızıq hám tuwrı sızıqtıń óz ara jaylasıwı (Kesilisiw noqatları, Urınba (urınıw) jaģdayı) \\
\textbf{A1.} Tómendegi sızıqlardan qaysı biri oraylıq (yaǵnıy birden-bir orayǵa iye), qaysı biri orayǵa iye emes, qaysı biri sheksiz kóp orayǵa iye ekenligin anıqlań:: $x^2-2 x y+4 y^2+5 x-7 y+12=0$; \\
\textbf{A2.} $y+6=0$ tegislik $\frac{x^2}{5}-\frac{y^2}{4}=6 z$ giperbolik paraboloidti parabola boyınsha kesip ótetuģının kórsetiń; parametrin hám tóbesin tabıń. \\
\textbf{A3.} Koordinatalar sistemasın túrlendirmesten, tómendegi teńlemelerdiń hár biri parabolanı anıqlawın kórsetiń hám parametrin tabıń: $x^2-2 x y+y^2+6 x-14 y+29=0$; \\
\textbf{B1.} Berilgen teńleme parabolik ekenligin kórsetiń; ápiwayı túrge keltiriń; qanday geometriyalıq obrazdı anlatıwın anıqlań, eski hám de jańa koordinata kósherlerine salıstırģanda sızılmada súwretleń: $9 x^2+24 x y+16 y^2-18 x+226 y+209=0$; \\
\textbf{B2.} Lagranj usılınan paydalanıp, teńlemelerdi kvadratlar qosındısı túrine keltirip, tómendegi betlerdiń kórinisin anıqlań: $x y+x z+y z+2 x+2 y-2 z=0$. \\
\textbf{B3.} Giperbolanın yarım kósherlerin tabiń, eger: direktrisaları $x= \pm 3 \sqrt{2}$ tenlemeler menen berilgen hám asimptotaları arasındaģi múyesh tuwri múyesh; \\
\textbf{C1.} Hár qanday parabolik teńleme $ (\alpha x+\beta y) ^2+2a_{13}x+2a_{23}y+a_{33}=0$ kórinisinde jazılıwı múmkinligin dálilleń. Sonday-aq, elliptikalıq hám giperbolikalıq teńlemelerdi bunday kóriniste jazıp bolmaytuģının dálilleń. \\
\textbf{C2.} Tómendegi eki tuwrı sızıqqa urınatuģın giperbolanıń teńlemesin dúziń: $5x-6y-16=0$, $13x-10y-48=0$, bunda onıń kósherleri koordinata kósherleri menen ústpe-úst túsedi. \\
\textbf{C3.} Múyesh koefficienti $k$ tiń qanday mánislerinde $y=kx+2$ tuwrısı: 1) $y^2=4x$ parabolanı kesip ótedi; 2) oǵan urınadı; 3) bul parabola sırtınan ótedi. \\

\end{tabular}
\vspace{1cm}


\begin{tabular}{m{17cm}}
\textbf{54-variant}
\newline

\textbf{T1.} Tegislikte ekinshi tártipli sızıqlar (Ekinshi tártipli teńleme, Kvadrat kórinisindegi teńleme, Konik sızıqlar (konuslar kesimi)) \\
\textbf{T2.} Ekinshi tártipli betliklerdiń kanonikalıq teńlemeleri (Ellipsoid, Giperboloid (1 gewekli), Giperboloid (2 gewekli)) \\
\textbf{A1.} Fokusları abscissa kósherinde, koordinatalar basına qarata simmetriyalı jaylasqan giperbolanıń teńlemesi dúzilsin, bunda: $M_1\left(\frac{9}{2};-1\right) $ noqatı giperbolaga tiyisli hám asimtota teńlemeleri $y= \pm \frac{2}{3} x$; \\
\textbf{A2.} Fokusları abscissa kósherinde jatqan hám koordinatalar basına salıstırģanda simmetriyalı bolǵan ellipstiń teńlemesin dúziń, bunda: fokusları arasındaǵı aralıq $2 c=6$ hám ekssentrisiteti $\varepsilon=\frac{3}{5}$; \\
\textbf{A3.} Koordinatalar sistemasın túrlendirmesten, tómendegi teńlemelerdiń hár biri parabolanı anıqlawın kórsetiń hám parametrin tabıń: $9 x^2+24 x y+16 y^2-120 x+90 y=0$; \\
\textbf{B1.} Ellips fokuslarınıń birinen úlken kósheri tóbelerine shekemgi aralıqlar sáykes túrde 7 hám 1 ge teń. Bul ellipstiń tenlemesin dúziń. \\
\textbf{B2.} Berilgen teńlemeler oraylıq iymek sızıqlar ekenligin kórsetiń hám hárbir teńlemeni koordinata basın orayģa kóshiriń:  $4 x^2+2 x y+6 y^2+6 x-10 y+9=0$. \\
\textbf{B3.} Tómendegilerdi bilgen halda giperbolanıń teńlemesin dúziń: fokuslar $F_1 (3; 4), F_2 (-3;-4) $ hám direktrisalar arasındaǵı aralıq 3,6; \\
\textbf{C1.} $m$ nıń qanday mánislerinde $x+m y-2=0$ tegislik $\frac{x^2}{2}+\frac{z^2}{3}=y$ elliptik paraboloidti a) ellips boyınsha, b) parabola boyınsha kesip ótetuǵınlıǵın anıqlań. \\
\textbf{C2.} Ellipstiń yarım kósherleri $a$, $b$ hám orayı $C\left(x_0; y_0\right) $ noqatında bolıp, simmetriya kósherleri koordinata kósherlerine parallel ekenligi belgili bolsa, onıń teńlemesin dúziń. \\
\textbf{C3.} Ulıwma fokusqa hám ústpe-úst túsken, biraq qarama-qarsı baǵıtlangan kósherlerge iye bolǵan parabolalardıń tuwrı múyesh astında kesilisiwin dálilleń. \\

\end{tabular}
\vspace{1cm}


\begin{tabular}{m{17cm}}
\textbf{55-variant}
\newline

\textbf{T1.} Bir gewekli giperboloid hám giperbolik paraboloidtıń tuwrı sızıqlı jasawshıları (Giperboloid, Giperbolik paraboloid, Sızıqlı jasawshılar) \\
\textbf{T2.} Parabola hám onıń kanonikalıq teńlemeleri (Fokus (bagdarlawshı noqat), Direktrisa (bagdarlawshı sızıq), Kósher (simmetriya kósheri)) \\
\textbf{A1.} Polat tros eki ushınan ildirilgen; bekkemlew noqatları birdey biyiklikte jaylasqan; olar arasındaǵı aralıq 20 m ge teń. Onıń bekkemleniw noqatınan 2 m aralıqtaǵı iyiliw shaması, gorizontal boyınsha esaplaǵanda, 14,4 sm ge teń. Trosttı shama menen parabola doǵası formasında dep esaplap, bekkemlew noqatları arasındaǵı bul trostıń iyiliw shamasın anıqlańız. \\
\textbf{A2.} Koordinatalar sistemasın túrlendirmesten tómendegi teńlemelerdiń hár biri kesilisiwshi eki tuwrını anıqlawın kórsetiń hám onıń koordinataların tabıń: $x^2-4 x y+3 y^2=0$; \\
\textbf{A3.} Fokusları ordinata kósherinde jaylasqan, koordinatalar basına salıstırģanda simmetriyalı bolǵan giperbolanıń teńlemesin dúziń, bunda: asimtotalarınıń teńlemesi $y= \pm \frac{12}{5} x$ hám ushlarınıń arasındaģı aralıq 48; \\
\textbf{B1.} Parabola tóbesiniń koordinataların, parametrin hám kósheriniń baǵıtın anıqlań: $y^2-6 x+14 y+49=0$, \\
\textbf{B2.} Berilgen sızıqlar oraylıq ekenligin kórsetiń hám hárbir iymek sızıq ushın orayınıń koordinataların tabıń:$9 x^2-4 x y-7 y^2-12=0$; \\
\textbf{B3.} Lagranj usılınan paydalanıp, teńlemelerdi kvadratlar qosındısı túrine keltirip, tómendegi betlerdiń kórinisin anıqlań: $4 x^2+6 y^2+4 z^2+4 x z-8 y-4 z+3=0$; \\
\textbf{C1.} Berilgen teńleme kanonikalıq kóriniske keltirilsin; tipi anıqlansın; qanday geometriyalıq obrazdı anlatıwı anıqlansın; eski hám jana koordinatalar sistemasında geometriyalıq obrazı súwretlensin: $4 x^2+24 x y+11 y^2+64 x+42 y+51=0$; \\
\textbf{C2.} Giperbola asimptotalarınıń tenlemeleri $y= \pm \frac{1}{2} x$ hám urinbalardan biriniń teńlemesi. $5 x-6 y-8=0$ belgili bolsa, giperbola teńlemesin dúziń. \\
\textbf{C3.} $\frac{x^2}{a^2}+\frac{y^2}{b^2}=1$ ellipske ishley sızılgan kvadrat tárepiniń uzınlıǵın esaplań. \\

\end{tabular}
\vspace{1cm}


\begin{tabular}{m{17cm}}
\textbf{56-variant}
\newline

\textbf{T1.} Ekinshi tártipli sızıq orayı (Oraylıq sızıqlar (ellips, giperbola), Oray koordinataları: simmetriya orayı) \\
\textbf{T2.} Ekinshi tártipli betliklerdiń kanonikalıq teńlemeleri (Paraboloid (ellipstik), Paraboloid (giperbolik), Konus, Cilindr) \\
\textbf{A1.} Fokusları abscissa kósherinde jatqan hám koordinatalar basına salıstırģanda simmetriyalı bolǵan ellipstiń teńlemesin dúziń, bunda: direktrisaları arasındaǵı aralıq 5 hám fokusları arasındaǵı aralıq $2c=4$; \\
\textbf{A2.} Tómendegi sızıqlardan qaysı biri oraylıq (yaǵnıy birden-bir orayǵa iye), qaysı biri orayǵa iye emes, qaysı biri sheksiz kóp orayǵa iye ekenligin anıqlań:  $x^2-2 x y+y^2-6 x+6 y-3=0$; \\
\textbf{A3.} Berilgen teńleme menen qanday iymek sızıq anıqlanıwın tabıń: $\left\{\begin{array}{l}\frac{x^2}{3}+\frac{y^2}{6}=2 z, \\ 3 x-y+6 z-14=0\end{array}\right.$ \\
\textbf{B1.} Berilgen teńlemelerdiń parabolik ekenligin kórsetiń hám olardıń hár birin $(\alpha x+\beta y)^2+2 a_{13} x+2 a_{23} y+a_{33}=0$ kórinisinde jazıń:  $9 x^2-42 x y+49 y^2+3 x-2 y-24=0$. \\
\textbf{B2.} Eger ellipstiń ekssentrisiteti $\varepsilon=\frac{1}{2}$ hám fokusı $F (3; 0) $ hám oǵan sáykes direktrisa teńlemesi $x+y-1=0$ belgili bolsa, onıń teńlemesin dúziń. \\
\textbf{B3.} ITECH túri, ólshemleri hám jaylasıwın anıqlań: $7 x^2-24 x y-38 x+24 y+175=0$; \\
\textbf{C1.} Tómendegi betliklerdiń kanonikalıq teńlemesi hám jaylasıwın anıqlań.: $x^2+5 y^2+z^2+2 x y+6 x z+2 y z-2 x+6 y+2 z=0$. \\
\textbf{C2.} Giperbolanıń asimptotaların tabıń: $3 x^2+2 x y-y^2+8 x+10 y-14=0$; \\
\textbf{C3.} $4 x^2-4 x y+y^2+6 x+1=0$ ETIS teńlemesi berilgen. Múyesh koefficienti $k$ tiń qanday mánislerinde $y=kx$ tuwrı sızıq: 1) bul iymek sızıqtı bir noqatta kesip ótiwi; 2) urınadı; 3) eki noqatta kesip ótiwin; 4) bul tuwrı menen ulıwma noqatqa iye bolmaytuģının anıqlań. \\

\end{tabular}
\vspace{1cm}


\begin{tabular}{m{17cm}}
\textbf{57-variant}
\newline

\textbf{T1.} Tegislikte ekinshi tártipli sızıqlar (Ekinshi tártipli teńleme, Kvadrat kórinisindegi teńleme, Konik sızıqlar (konuslar kesimi)) \\
\textbf{T2.} Ekinshi tártipli sızıqqa urınba, túyinles diametri teńlemesi (Urınba teńlemesi, Túyinles diametr: oraydan ótiwshi simmetriya kósherleri) \\
\textbf{A1.} Tómendegi sızıqlardan qaysı biri oraylıq (yaǵnıy birden-bir orayǵa iye), qaysı biri orayǵa iye emes, qaysı biri sheksiz kóp orayǵa iye ekenligin anıqlań:  $4 x^2-6 x y-9 y^2+3 x-7 y+12=0$. \\
\textbf{A2.} Fokusları abscissa kósherinde jatqan hám koordinatalar basına salıstırganda simmetriyalı bolǵan ellipstiń teńlemesin dúziń, bunda: kishi kósheri 10, ekscentrisiteti $\varepsilon=\frac{12}{13}$; \\
\textbf{A3.} Koordinatalar sistemasın túrlendirmesten tómendegi teńlemelerdiń hár biri birden-bir noqattı anıqlawın kórsetiń hám onıń koordinataların tabıń: $x^2+2 x y+2 y^2+6 y+9=0$; \\
\textbf{B1.} Berilgen teńlemelerdiń parabolik ekenligin kórsetiń hám olardıń hár birin $(\alpha x+\beta y)^2+2 a_{13} x+2 a_{23} y+a_{33}=0$ kórinisinde jazıń: $25 x^2-20 x y+4 y^2+3 x-y+11=0$; \\
\textbf{B2.} Parallel kóshiriw hám burıw túrlendiriwleri yamasa aǵzalardı gruppalaw járdeminde tómendegi betliklerdiń kórinisi hám jaylasıwı anıqlansın: $3 x^2+3 y^2-6 x+4 y-1=0$; \\
\textbf{B3.} ITECH túri, ólshemleri hám jaylasıwın anıqlań: $2 x^2+4 x y+5 y^2-6 x-8 y-1=0$; \\
\textbf{C1.} $y^2=4 x$ parabola menen $\frac{x^2}{8}+\frac{y^2}{2}=1$ ellipstiń uliwma urinbaların anıqlań. \\
\textbf{C2.} $\frac{x^2}{a^2}-\frac{y^2}{b^2}=1$ giperbola hám onıń qanday da bir urınbası berilgen: $P$-urınbasınıń $O x$ kósheri menen kesilisiw noqatı, $Q$ - urınba noqatınıń sol kósherdegi proekciyası. $O P \cdot O Q=a^2$ ekenligin dálilleń. \\
\textbf{C3.} Elliptik túrdegi ($\delta>0$) teńleme $a_{11}$ hám $\Delta$ birdey belgige iye san bolǵanda ǵana jormal ellips teńlemesi bolatuģının dálilleń. \\

\end{tabular}
\vspace{1cm}


\begin{tabular}{m{17cm}}
\textbf{58-variant}
\newline

\textbf{T1.} Ekinshi tártipli sızıqlardıń ulıwma teńlemesin invariantlar járdeminde kanonikalıq túrge keltiriw \\
\textbf{T2.} Ekinshi tártipli betliklerdiń kanonikalıq teńlemeleri (Ellipsoid, Giperboloid (1 gewekli), Giperboloid (2 gewekli)) \\
\textbf{A1.} Ekscentrisiteti $\varepsilon=\frac{13}{12}$, bir fokusi $F (0; 13) $ hám ogan sáykes direktrisasınıń teńlemesi $13 y-144=0$ bolǵ an giperbolanıń teńlemesin dúziń. \\
\textbf{A2.} Parabolanıń teńlemesin dúziń, eger: fokusı $ (5,0) $ noqatta bolıp, ordinatalar kósheri direktrisa bolsa; \\
\textbf{A3.} Koordinatalar sistemasın túrlendirmesten, tómendegi teńlemelerdiń hár biri parabolanı anıqlawın kórsetiń hám parametrin tabıń: $x^2-2 x y+y^2+6 x-14 y+29=0$; \\
\textbf{B1.} Berilgen sızıqlar oraylıq ekenligin kórsetiń hám hárbir iymek sızıq ushın orayınıń koordinataların tabıń:$9 x^2-4 x y-7 y^2-12=0$; \\
\textbf{B2.} $x^2-y^2=16$ giperbolaǵa $A (-1;-7)$ noqattan ótkerilgen urınbalar teńlemesin dúziń. \\
\textbf{B3.} Tómendegilerdi bilgen halda ellips teńlemesin dúziń: onıń fokusları $F_1 (1; 3), F_2 (3; 1) $ hám direktrisalar arasındaģı aralıq $12 \sqrt{2}$ qa teń. \\
\textbf{C1.} $\frac{x^2}{a^2}-\frac{y^2}{b^2}=1$ giperbolanıń fokuslarınan urınbasına shekemgi aralıqlardıń kóbeymesin tabıń. \\
\textbf{C2.} $m$ nıń qanday mánislerinde $x-2 y-2 z+m=0$ tegislik $\frac{x^2}{144}+\frac{y^2}{36}+\frac{z^2}{9}=1$ ellipsoidqa urınıwın anıqlań. \\
\textbf{C3.} $y^2=2 p x$ parabolaǵa onıń $M_1\left(x_1; y_1\right) $ noqatındaǵı urınbasınıń teńlemesin dúziń. \\

\end{tabular}
\vspace{1cm}


\begin{tabular}{m{17cm}}
\textbf{59-variant}
\newline

\textbf{T1.} Parabola hám onıń kanonikalıq teńlemeleri (Fokus (bagdarlawshı noqat), Direktrisa (bagdarlawshı sızıq), Kósher (simmetriya kósheri)) \\
\textbf{T2.} Tegislikte ekinshi tártipli sızıqlar (Ekinshi tártipli teńleme, Kvadrat kórinisindegi teńleme, Konik sızıqlar (konuslar kesimi)) \\
\textbf{A1.} $z+1=0$ tegislik bir qabatlı $\frac{x^2}{32}-\frac{y^2}{18}+\frac{z^2}{2}=1$ giperboloidti giperbola boyınsha kesip ótetuģının kórsetiń; onıń yarım kósherleri hám tóbelerin tabıń. \\
\textbf{A2.} Koordinatalar sistemasın túrlendirmesten, tómendegi teńlemelerdiń hár biri parabolanı anıqlawın kórsetiń hám parametrin tabıń: $9 x^2-6 x y+y^2-50 x+50 y-275=0$. \\
\textbf{A3.} Koordinatalar sistemasın túrlendirmesten tómendegi teńlemelerdiń hár biri birden-bir noqattı anıqlawın kórsetiń hám onıń koordinataların tabıń: $x^2-6 x y+10 y^2+10 x-32 y+26=0$. \\
\textbf{B1.} $y^2=8x$ parabolanıń $2x+2y-3=0$ tuwrısına parallel urınbasınıń teńlemesin dúziń. \\
\textbf{B2.} Ellipstiń ekssentrisiteti $\varepsilon=\frac{1}{3}$, onıń orayı koordinatalar bası menen ústpe-úst túsedi, fokuslarınan biri $ (-2; 0) $. Abcissası 2 ge teń bolǵan ellipstiń $M_1$ noqatınan berilgen fokusqa sáykes direktrisaģa shekem bolǵan aralıqtı tabıń. \\
\textbf{B3.} ITECH túri, ólshemleri hám jaylasıwın anıqlań: $5 x^2+4 x y+8 y^2-32 x-56 y+80=0$. \\
\textbf{C1.} Eger ekinshi dárejeli teńleme parabolik bolıp, $ (\alpha x+\beta y) ^2+2a_{13}x+2a_{23}y+a_{33}=0$ kórinisinde jazılsa, onıń shep tárepindegi diskriminant $\Delta=- (a_{13} \beta-a_{23} \alpha) ^2$ formula menen anıqlanıwın dálilleń. \\
\textbf{C2.} Giperbolanıń asimptotaların tabıń: $10 x y-2 y^2+6 x+4 y+21=0$ \\
\textbf{C3.} $\frac{x^2}{a^2}+\frac{y^2}{b^2}=1$ ellipstiń $F(c, 0)$ fokusı arqalı úlken kósherine perpendikulyar bolǵan xorda ótkerilgen. Bul xordıń uzınlıǵın tabıń. \\

\end{tabular}
\vspace{1cm}


\begin{tabular}{m{17cm}}
\textbf{60-variant}
\newline

\textbf{T1.} Ekinshi tártipli sızıqlardıń ulıwma teńlemeleri (Ulıwma teńleme) \\
\textbf{T2.} Ekinshi tártipli betlik orayı, urınba tegisligi hám diametral tegisligi (Oray, Urınba tegislik, Diametral tegislik.) \\
\textbf{A1.} Tómendegi sızıqlardan qaysı biri oraylıq (yaǵnıy birden-bir orayǵa iye), qaysı biri orayǵa iye emes, qaysı biri sheksiz kóp orayǵa iye ekenligin anıqlań: $25 x^2-10 x y+y^2+40 x-8 y+7=0$. \\
\textbf{A2.} Berilgen teńleme menen qanday iymek sızıq anıqlanıwın tabıń: $\left\{\begin{array}{l}\frac{x^2}{.4}+\frac{y^2}{9}-\frac{z^2}{36}=1, \\ 9 x-6 y+2 z-28=0,\end{array}\right.$ \\
\textbf{A3.} Tóbesi koordinatalar basında bolǵan parabolanıń teńlemesin dúziń, bunda: parabola oń yarım tegislikte hám $Oy$ kósherine simmetriyalı jaylasqan, hám parametri $p=\frac{1}{4}$; \\
\textbf{B1.} Giperbolanıń haqıyqıy kósherine perpendikulyar bolǵan hám giperbola fokusınan ótken xorda uzınlıǵın tabıń. \\
\textbf{B2.} Berilgen sızıqlar oraylıq ekenligin kórsetiń hám hárbir iymek sızıq ushın orayınıń koordinataların tabıń: $3x^2+5xy+y^2-8x-11y-7=0$. \\
\textbf{B3.} Berilgen teńleme parabolik ekenligin kórsetiń; ápiwayı túrge keltiriń; qanday geometriyalıq obrazdı anlatıwın anıqlań, eski hám de jańa koordinata kósherlerine salıstırģanda sızılmada súwretleń:$9 x^2+12 x y+4 y^2-24 x-16 y+3=0$; \\
\textbf{C1.} Tómendegi betliklerdiń kanonikalıq teńlemesi hám jaylasıwın anıqlań.: $2 x^2+2 y^2-5 z^2+2 x y-2 x-4 y-4 z+2=0$. \\
\textbf{C2.} Ellips orayınan onıń qálegen urınbasınıń fokal kósher menen kesilisiw noqatına shekemgi hám urınıw noqatınan fokal kósherge túsirilgen perpendikulyar ultanına shekemgi aralıqlar kóbeymesi turaqlı shama bolıp, ellips úlken yarım kósheriniń kvadratına teń ekenligin dálilleń. \\
\textbf{C3.} $m$ hám $n$ tiń qanday mánislerinde $m x^2+12 x y+9 y^2+4 x+n y-13=0$ teńleme: 1) oraylıq sızıqtı; 2) orayga iye bolmaǵan sızıq; 3) sheksiz kóp orayǵa iye bolǵan sızıqtı ańlatadı. \\

\end{tabular}
\vspace{1cm}


\begin{tabular}{m{17cm}}
\textbf{61-variant}
\newline

\textbf{T1.} Ekinshi tártipli betliklerdiń ulıwmalıq teńlemesin kanonikalıq túrge keltiriw (invariantlar járdeminde) \\
\textbf{T2.} Ekinshi tártipli sızıqlardıń ulıwma teńlemeleri (Ulıwma teńleme) \\
\textbf{A1.} Fokusları abscissa kósherinde jatqan hám koordinatalar basına salıstırģanda simmetriyalı bolǵan ellipstiń teńlemesin dúziń, bunda onıń úlken kósheri 10 ģa, fokusları arasındaǵı aralıq bolsa $2c = 8$ ge teń; \\
\textbf{A2.} Fokusları abscissa kósherinde jaylasqan, koordinatalar basına qarata simmetriyalı bolǵan giperbolanıń teńlemesin dúziń, bunda: $2 a=16$ hám ekssentrisiteti $\varepsilon=\frac{5}{4}$; \\
\textbf{A3.} Koordinatalar sistemasın túrlendirmesten, tómendegi teńlemelerdiń hár biri parabolanı anıqlawın kórsetiń hám parametrin tabıń: $9 x^2+24 x y+16 y^2-120 x+90 y=0$; \\
\textbf{B1.} $x^2=16y$ parabolanıń $2x+4y+7=0$ tuwrısına perpendikulyar bolǵan urınbasınıń teńlemesin dúziń. \\
\textbf{B2.} Parallel kóshiriw hám burıw túrlendiriwleri yamasa aǵzalardı gruppalaw járdeminde tómendegi betliklerdiń kórinisi hám jaylasıwı anıqlansın: $x^2+4 y^2+9 z^2-6 x+8 y-18 z-14=0$; \\
\textbf{B3.} Berilgen teńlemelerdiń parabolik ekenligin kórsetiń hám olardıń hár birin $(\alpha x+\beta y)^2+2 a_{13} x+2 a_{23} y+a_{33}=0$ kórinisinde jazıń: $x^2+4 x y+4 y^2+4 x+y-15=0 ;$ \\
\textbf{C1.} $A x+B y+C=0$ tuwrı sızıqtıń $\frac{x^2}{a^2}+\frac{y^2}{b^2}=1$, ellipske urınba bolıwı ushın zárúrli hám jeterli shárti tabılsın. \\
\textbf{C2.} $\frac{x^2}{81}+\frac{y^2}{36}+\frac{z^2}{9}=1$ ellipsoid $4 x-3 y+12 z-54=0$ tegislik penen bir ulıwma noqatqa iye ekenligin dálilleń hám onıń koordinataların tabıń. \\
\textbf{C3.} Berilgen teńleme kanonikalıq kóriniske keltirilsin; tipi anıqlansın; qanday geometriyalıq obrazdı anlatıwı anıqlansın; eski hám jana koordinatalar sistemasında geometriyalıq obrazı súwretlensin: $4 x y+3 y^2+16 x+12 y-36=0$; \\

\end{tabular}
\vspace{1cm}


\begin{tabular}{m{17cm}}
\textbf{62-variant}
\newline

\textbf{T1.} Parabola hám onıń kanonikalıq teńlemeleri (Fokus (bagdarlawshı noqat), Direktrisa (bagdarlawshı sızıq), Kósher (simmetriya kósheri)) \\
\textbf{T2.} Ekinshi tártipli betliklerdiń ulıwma teńlemeleri (Ulıwma teńleme) \\
\textbf{A1.} Fokusları abscissa kósherinde, koordinatalar basına qarata simmetriyalı jaylasqan giperbolanıń teńlemesin dúziń, bunda: $M_1 (-5; 3)$ noqat giperbolaģa tiyisli hám ekssentrisiteti $\varepsilon=\sqrt{2}$; \\
\textbf{A2.} Tómendegi sızıqlardan qaysı biri oraylıq (yaǵnıy birden-bir orayǵa iye), qaysı biri orayǵa iye emes, qaysı biri sheksiz kóp orayǵa iye ekenligin anıqlań: $4 x^2-4 x y+y^2-12 x+6 y-11=0$; \\
\textbf{A3.} Fokusları abscissa kósherinde jatqan hám koordinatalar basına qarata simmetriyalı bolǵan ellipstiń teńlemesi dúzilsin, bunda: $M_1 \left(2;-\frac{5}{3}\right) $ noqatı ellipske tiyisli hám ekscentrisiteti $\varepsilon=\frac{2}{3}$; \\
\textbf{B1.} Tómendegilerdi bilgen halda giperbolanıń teńlemesin dúziń: fokuslar $F_1 (3; 4), F_2 (-3;-4) $ hám direktrisalar arasındaǵı aralıq 3,6; \\
\textbf{B2.} Berilgen sızıqlar oraylıq ekenligin kórsetiń hám hárbir iymek sızıq ushın orayınıń koordinataların tabıń:$5 x^2+4 x y+2 y^2+20 x+20 y-18=0$; \\
\textbf{B3.} ITECH túri, ólshemleri hám jaylasıwın anıqlań: $x^2-5 x y+4 y^2+x+2 y-2=0$. \\
\textbf{C1.} $y^2=2 p x$ parabolaǵa $y=k x+b$ tuwrı sızıq urınıw shártin keltirip shigarıń. \\
\textbf{C2.} Tómendegi betliklerdiń kanonikalıq teńlemesi hám jaylasıwın anıqlań.: $x^2+y^2+4 z^2+2 x y+4 x z+4 y z-6 z+1=0$. \\
\textbf{C3.} $y=k x+m$ tuwrı sızıqtıń $\frac{x^2}{a^2}-\frac{y^2}{b^2}=1$ giperbolaģa urınıw shártin keltirip shıģarıń. \\

\end{tabular}
\vspace{1cm}


\begin{tabular}{m{17cm}}
\textbf{63-variant}
\newline

\textbf{T1.} Tegislikte ekinshi tártipli sızıqlar (Ekinshi tártipli teńleme, Kvadrat kórinisindegi teńleme, Konik sızıqlar (konuslar kesimi)) \\
\textbf{T2.} Ekinshi tártipli sızıq orayı (Oraylıq sızıqlar (ellips, giperbola), Oray koordinataları: simmetriya orayı) \\
\textbf{A1.} Diskriminantın esaplaw arqalı tómendegi teńlemelerdiń hár biriniń tipin anıqlań: $3 x^2-2 x y-3 y^2+12 y-15=0$. \\
\textbf{A2.} $x-2=0$ tegislik $\frac{x^2}{16}+\frac{y^2}{12}+\frac{z^2}{4}=1$ ellipsoidti ellips boyınsha kesip ótetuģının kórsetiń; onıń yarım kósherleri hám tóbelerin tabıń. \\
\textbf{A3.} $y^2=8 x$ paraboladaǵı fokal radius vektorı 20 ga teń bolgan noqat tabılsın. \\
\textbf{B1.} Parallel kóshiriw hám burıw túrlendiriwleri yamasa aǵzalardı gruppalaw járdeminde tómendegi betliklerdiń kórinisi hám jaylasıwı anıqlansın: $z=x^2+3 y^2-6 y+1$; \\
\textbf{B2.} Parabola tóbesiniń koordinataların, parametrin hám kósheriniń baǵıtın anıqlań: $x^2-6 x-4 y+29=0$, \\
\textbf{B3.} Ellipstegi ekssentrisitetti anıqlań, eger: direktrisalar arasındaǵı aralıq fokuslar arasındaǵı aralıqtan úsh ese úlken bolsa; \\
\textbf{C1.} $4 x^2-4 x y+y^2+6 x+1=0$ ETIS teńlemesi berilgen. Múyesh koefficienti $k$ tiń qanday mánislerinde $y=kx$ tuwrı sızıq: 1) bul iymek sızıqtı bir noqatta kesip ótiwi; 2) urınadı; 3) eki noqatta kesip ótiwin; 4) bul tuwrı menen ulıwma noqatqa iye bolmaytuģının anıqlań. \\
\textbf{C2.} $\frac{x^2}{30}+\frac{y^2}{24}=1$ ellipske $4x-2y+23=0$ parallel bolǵan urınbalardı júrgiziń hám olar arasındaģı aralıqtı esaplań. \\
\textbf{C3.} Giperbolanıń asimptotaların tabıń: $10 x y-2 y^2+6 x+4 y+21=0$ \\

\end{tabular}
\vspace{1cm}


\begin{tabular}{m{17cm}}
\textbf{64-variant}
\newline

\textbf{T1.} Ekinshi tártipli betliklerdiń ulıwmalıq teńlemesin kanonikalıq túrge keltiriw (invariantlar járdeminde) \\
\textbf{T2.} Parabola hám onıń kanonikalıq teńlemeleri (Fokus (bagdarlawshı noqat), Direktrisa (bagdarlawshı sızıq), Kósher (simmetriya kósheri)) \\
\textbf{A1.} Ekscentrisiteti $\varepsilon=\frac{2}{3}$, fokusı $F (2; 1) $ hám usı fokus tárepindegi direktrisası $x-5=0$ bolǵan ellipstiń teńlemesin dúziń. \\
\textbf{A2.} Koordinatalar sistemasın túrlendirmesten, tómendegi teńlemeler menen qanday geometriyalıq obrazdı anıqlanıwın tabıń: $6 x^2-6 x y+9 y^2-4 x+18 y+14=0$; \\
\textbf{A3.} Berilgen teńleme menen qanday iymek sızıq anıqlanıwın tabıń: $\left\{\begin{array}{l}\frac{x^2}{4}-\frac{y^2}{3}=2 z \\ x-2 y+2=0 ;\end{array}\right.$ \\
\textbf{B1.} Giperbolanıń asimptotaları arasındaǵı múyeshin tabıń, eger: ekssentrisiteti $e=2$; \\
\textbf{B2.} Berilgen teńlemeler oraylıq iymek sızıqlar ekenligin kórsetiń hám hárbir teńlemeni koordinata basın orayģa kóshiriń:  $4 x^2+2 x y+6 y^2+6 x-10 y+9=0$. \\
\textbf{B3.} Ellipstiń ekssentrisiteti $\varepsilon=\frac{1}{2}$, onıń orayı koordinatalar bası menen ústpe-úst túsedi, direktrisalardan biri $x=16$ teńleme menen berilgen. Abcissası $-4$ ke teń bolǵan ellipstiń $M_1$ noqatınan berilgen direktrisa menen bir tárepleme fokusqa shekem bolǵan aralıqtı esaplań. \\
\textbf{C1.} Hár qanday parabolik teńleme $ (\alpha x+\beta y) ^2+2a_{13}x+2a_{23}y+a_{33}=0$ kórinisinde jazılıwı múmkinligin dálilleń. Sonday-aq, elliptikalıq hám giperbolikalıq teńlemelerdi bunday kóriniste jazıp bolmaytuģının dálilleń. \\
\textbf{C2.} Parabolanıń qálegen urinbasınıń direktrisası hám kósherge perpendikulyar bolǵan fokal xordanı fokustan teńdey uzaqlıqtaģı noqatlarda kesetuģının dálilleń. \\
\textbf{C3.} Giperbolanıń asimptotalarınan direktrisaları ajıratqan kesindiler (giperbolanıń orayınan esaplaganda) giperbolanıń haqıyqıy yarım kósherine teń ekenligin dálilleń. Bul qásiyetten paydalanıp, giperbolanıń direktrisaların jasań. \\

\end{tabular}
\vspace{1cm}


\begin{tabular}{m{17cm}}
\textbf{65-variant}
\newline

\textbf{T1.} Ekinshi tártipli sızıqlardıń ulıwma teńlemesin invariantlar járdeminde kanonikalıq túrge keltiriw \\
\textbf{T2.} Ekinshi tártipli sızıq hám tuwrı sızıqtıń óz ara jaylasıwı (Kesilisiw noqatları, Urınba (urınıw) jaģdayı) \\
\textbf{A1.} Giperbola asimptotalarınıń teńlemeleri $y= \pm \frac{5}{12} x$ hám giperbolada jatıwshı $M (24,5) $ noqatı berilgen. Giperbola teńlemesin dúziń. \\
\textbf{A2.} Koordinatalar sistemasın túrlendirmesten, tómendegi teńlemelerdiń hár biri parabolanı anıqlawın kórsetiń hám parametrin tabıń: $9 x^2-24 x y+16 y^2-54 x-178 y+181=0$; \\
\textbf{A3.} Parabolanıń tóbesi ($\alpha;\beta$) noqat penen ústpe-úst túsetuģının bilgen halda onıń teńlemesin dúziń. Parametri $p$ ǵa teń. Onıń kósheri $O y$ kósherine parallel bolıp, $O y$ kósheriniń teris baǵıtında sheksizlikke sozilgan; \\
\textbf{B1.} $5 x^2-3 x y+y^2-3 x+2 y-5=0$ sızıqtıń $x-2 y-1=0$ tuwri sızıq penen kesilisiwinen payda bolgan xordanıń ortasınan ótetuģin diametr teńlemesi jazılsın. \\
\textbf{B2.} Parabola tóbesiniń koordinataların, parametrin hám kósheriniń baǵıtın anıqlań: $y^2+8 x-16=0$, \\
\textbf{B3.} Parallel kóshiriw hám burıw túrlendiriwleri yamasa aǵzalardı gruppalaw járdeminde tómendegi betliklerdiń kórinisi hám jaylasıwı anıqlansın: $3 x^2+3 y^2+3 z^2-6 x+4 y-1=0$; \\
\textbf{C1.} Tómendegi betliklerdiń kanonikalıq teńlemesi hám jaylasıwın anıqlań.: $2 x^2+10 y^2-2 z^2+12 x y+8 y z+12 x+4 y+8 z-1=0$. \\
\textbf{C2.} $\frac{x^2}{100}+\frac{y^2}{64}=1$ ellipstiń $2 x-y+7=0,2 x-y-1=0$ xordalarınıń ortaları arqalı ótetuģın tuwrı sızıqtıń teńlemesin dúziń. \\
\textbf{C3.} $m$ niń qanday mánislerinde $x+mz-1=0$ tegislik tómendegi $x^2+y^2z^2=1$ eki gewekli giperboloidti a) ellips boyınsha, b) giperbola boyınsha kesedi? \\

\end{tabular}
\vspace{1cm}


\begin{tabular}{m{17cm}}
\textbf{66-variant}
\newline

\textbf{T1.} Ekinshi tártipli betliklerdiń kanonikalıq teńlemeleri (Paraboloid (ellipstik), Paraboloid (giperbolik), Konus, Cilindr) \\
\textbf{T2.} Tegislikte ekinshi tártipli sızıqlar (Ekinshi tártipli teńleme, Kvadrat kórinisindegi teńleme, Konik sızıqlar (konuslar kesimi)) \\
\textbf{A1.} Tómendegi sızıqlardan qaysı biri oraylıq (yaǵnıy birden-bir orayǵa iye), qaysı biri orayǵa iye emes, qaysı biri sheksiz kóp orayǵa iye ekenligin anıqlań: $3 x^2-4 x y-2 y^2+3 x-12 y-7=0$; \\
\textbf{A2.} $y+6=0$ tegislik $\frac{x^2}{5}-\frac{y^2}{4}=6 z$ giperbolik paraboloidti parabola boyınsha kesip ótetuģının kórsetiń; parametrin hám tóbesin tabıń. \\
\textbf{A3.} Koordinatalar sistemasın túrlendirmesten tómendegi teńlemelerdiń hár biri ellipsti anıqlawın kórsetiń hám onıń yarım kósherlerin tabıń: $13 x^2+10 x y+13 y^2+46 x+62 y+13=0$. \\
\textbf{B1.} Berilgen teńleme parabolik ekenligin kórsetiń; ápiwayı túrge keltiriń; qanday geometriyalıq obrazdı anlatıwın anıqlań, eski hám de jańa koordinata kósherlerine salıstırģanda sızılmada súwretleń: $x^2-2 x y+y^2-12 x+12 y-14=0$ \\
\textbf{B2.} Bes noqattan ótiwshi ekinshi tártipli sızıqtıń teńlemesin dúziń: $(0,0),(0,1),(1,0),(2,-5),(-5,2)$. \\
\textbf{B3.} $\frac{x^2}{25}+\frac{y^2}{15}=1$ ellipstiń fokusı arqalı onıń úlken kósherine perpendikulyar ótkerilgen. Bul perpendikulyardıń ellips penen kesilisken noqatlarınan fokuslarga shekem bolǵan aralıqlardı anıqlań. \\
\textbf{C1.} $4 x^2-4 x y+y^2+6 x+1=0$ ETIS teńlemesi berilgen. Múyesh koefficienti $k$ tiń qanday mánislerinde $y=kx$ tuwrı sızıq: 1) bul iymek sızıqtı bir noqatta kesip ótiwi; 2) urınadı; 3) eki noqatta kesip ótiwin; 4) bul tuwrı menen ulıwma noqatqa iye bolmaytuģının anıqlań. \\
\textbf{C2.} Ulıwma fokusqa hám ústpe-úst túsken, biraq qarama-qarsı baǵıtlangan kósherlerge iye bolǵan parabolalardıń tuwrı múyesh astında kesilisiwin dálilleń. \\
\textbf{C3.} Berilgen teńleme kanonikalıq kóriniske keltirilsin; tipi anıqlansın; qanday geometriyalıq obrazdı anlatıwı anıqlansın; eski hám jana koordinatalar sistemasında geometriyalıq obrazı súwretlensin: $5 x^2-2 x y+5 y^2-4 x+20 y+20=0$. \\

\end{tabular}
\vspace{1cm}


\begin{tabular}{m{17cm}}
\textbf{67-variant}
\newline

\textbf{T1.} Ekinshi tártipli betliklerdiń ulıwma teńlemeleri (Ulıwma teńleme) \\
\textbf{T2.} Parabola hám onıń kanonikalıq teńlemeleri (Fokus (bagdarlawshı noqat), Direktrisa (bagdarlawshı sızıq), Kósher (simmetriya kósheri)) \\
\textbf{A1.} Tóbesi koordinatalar basında bolǵan parabolanıń teńlemesin dúziń, bunda: parabola $Oy$ kósherine simmetriyalı jaylasqan hám $D (4; -8) $ noqatınan ótedi; \\
\textbf{A2.} $\frac{x^2}{225}-\frac{y^2}{64}=-1$ giperbolanıń fokusların tabıń. \\
\textbf{A3.} Tómendegi sızıqlardan qaysı biri oraylıq (yaǵnıy birden-bir orayǵa iye), qaysı biri orayǵa iye emes, qaysı biri sheksiz kóp orayǵa iye ekenligin anıqlań: $4 x^2+5 x y+3 y^2-x+9 y-12=0$; \\
\textbf{B1.} Parabola tóbesiniń koordinataların, parametrin hám kósheriniń baǵıtın anıqlań: $y=x^2-8 x+15$, \\
\textbf{B2.} $x^2-y^2=8$ giperbolaga $M(3,-1)$ noqatında urınatuģın tuwrı sızıqtıń teńlemesin jazıń. \\
\textbf{B3.} Berilgen teńleme parabolik ekenligin kórsetiń; ápiwayı túrge keltiriń; qanday geometriyalıq obrazdı anlatıwın anıqlań, eski hám de jańa koordinata kósherlerine salıstırģanda sızılmada súwretleń:$9 x^2-24 x y+16 y^2-20 x+110 y-50=0$; \\
\textbf{C1.} Ekinshi dárejeli teńleme tek hám tek $\Delta=0$ bolǵanda ǵana aynıǵan iymek sızıq teńlemesi bolatuģının dálilleń. \\
\textbf{C2.} $m$ nıń qanday mánislerinde $y=-x+m$ sızıq: 1) $\frac{x^2}{20}+\frac{y^2}{5}=1$ ellipsti kesip ótedi; 2) ellipske urınadı; 3) ellipsti kesip ótpeydi. \\
\textbf{C3.} $\frac{x^2}{a^2}-\frac{y^2}{b^2}=1$ giperbolaģa onıń $M_1\left(x_1; y_1\right) $ noqatındaǵı urınbasınıń teńlemesin dúziń. \\

\end{tabular}
\vspace{1cm}


\begin{tabular}{m{17cm}}
\textbf{68-variant}
\newline

\textbf{T1.} Ekinshi tártipli sızıqqa urınba, túyinles diametri teńlemesi (Urınba teńlemesi, Túyinles diametr: oraydan ótiwshi simmetriya kósherleri) \\
\textbf{T2.} Bir gewekli giperboloid hám giperbolik paraboloidtıń tuwrı sızıqlı jasawshıları (Giperboloid, Giperbolik paraboloid, Sızıqlı jasawshılar) \\
\textbf{A1.} Fokusları abscissa kósherinde jatqan hám koordinatalar basına qarata simmetriyalı bolǵan ellipstiń teńlemesin dúziń, bunda: $M_1 (8; 12) $ ellipske tiyisli hám bul noqattan shep fokusına shekemgi aralıq $r_1=20$ qa teń; \\
\textbf{A2.} Koordinatalar sistemasın túrlendirmesten, tómendegi teńlemelerdiń hár biri parabolanı anıqlawın kórsetiń hám parametrin tabıń: $9 x^2+24 x y+16 y^2-120 x+90 y=0$; \\
\textbf{A3.} Tómendegi sızıqlardan qaysı biri oraylıq (yaǵnıy birden-bir orayǵa iye), qaysı biri orayǵa iye emes, qaysı biri sheksiz kóp orayǵa iye ekenligin anıqlań:  $4 x^2+4 x y+y^2-8 x-4 y-21=0$; \\
\textbf{B1.} Lagranj usılınan paydalanıp, teńlemelerdi kvadratlar qosındısı túrine keltirip, tómendegi betlerdiń kórinisin anıqlań: $x^2+y^2+4 z^2+2 x y+4 x z+4 y z-6 z+1=0$; \\
\textbf{B2.} Berilgen sızıqlar oraylıq ekenligin kórsetiń hám hárbir iymek sızıq ushın orayınıń koordinataların tabıń: $2 x^2-6 x y+5 y^2+22 x-36 y+11=0$. \\
\textbf{B3.} Kósherleri koordinata kósherleri menen ústpe-úst túsetuģin hám $P (2,2); Q (3,1) $ noqatlar arqalı ótiwshi ellips teńlemesin dúziń. \\
\textbf{C1.} $y^2=2 p x$ parabolaǵa onıń $M_1\left(x_1; y_1\right) $ noqatındaǵı urınbasınıń teńlemesin dúziń. \\
\textbf{C2.} $\frac{x^2}{a^2}-\frac{y^2}{b^2}=1$ giperbolanıń fokuslarınan urınbasına shekemgi aralıqlardıń kóbeymesin tabıń. \\
\textbf{C3.} Giperbolanıń asimptotaların tabıń: $10 x^2+21 x y+9 y^2-41 x-39 y+4=0$. \\

\end{tabular}
\vspace{1cm}


\begin{tabular}{m{17cm}}
\textbf{69-variant}
\newline

\textbf{T1.} Ekinshi tártipli sızıqlardıń ulıwma teńlemesin invariantlar járdeminde kanonikalıq túrge keltiriw \\
\textbf{T2.} Tegislikte ekinshi tártipli sızıqlar (Ekinshi tártipli teńleme, Kvadrat kórinisindegi teńleme, Konik sızıqlar (konuslar kesimi)) \\
\textbf{A1.} Fokusları abscissa kósherinde jaylasqan, koordinatalar basına qarata simmetriyalı bolǵan giperbolanıń teńlemesin dúziń, bunda: fokusları arasındaǵı aralıq $2 c=10$ hám kishi kósheri $2 b=8$; \\
\textbf{A2.} Koordinatalar sistemasın túrlendirmesten, tómendegi teńlemelerdiń hár biri parabolanı anıqlawın kórsetiń hám parametrin tabıń: $9 x^2-24 x y+16 y^2-54 x-178 y+181=0$; \\
\textbf{A3.} $x-2=0$ tegislik $\frac{x^2}{16}+\frac{y^2}{12}+\frac{z^2}{4}=1$ ellipsoidti ellips boyınsha kesip ótetuģının kórsetiń; onıń yarım kósherleri hám tóbelerin tabıń. \\
\textbf{B1.} Lagranj usılınan paydalanıp, teńlemelerdi kvadratlar qosındısı túrine keltirip, tómendegi betlerdiń kórinisin anıqlań: $x^2-2 y^2+z^2+4 x y-10 x z+4 y z+x+y-z=0$; \\
\textbf{B2.} ITECH túri, ólshemleri hám jaylasıwın anıqlań: $7 x^2+16 x y-23 y^2-14 x-16 y-218=0$; \\
\textbf{B3.} Parabola tóbesiniń koordinataların, parametrin hám kósheriniń baǵıtın anıqlań: $y=A x^2+B x+C$, \\
\textbf{C1.} $y^2=4 x$ parabola menen $\frac{x^2}{8}+\frac{y^2}{2}=1$ ellipstiń uliwma urinbaların anıqlań. \\
\textbf{C2.} Berilgen teńleme kanonikalıq kóriniske keltirilsin; tipi anıqlansın; qanday geometriyalıq obrazdı anlatıwı anıqlansın; eski hám jana koordinatalar sistemasında geometriyalıq obrazı súwretlensin: $41 x^2+24 x y+9 y^2+24 x+18 y-36=0$. \\
\textbf{C3.} Ulıwma kósherge hám tóbeleri arasında jaylasqan ulıwma fokusqa iye bolǵan eki parabola tuwrı múyesh astında kesilisetuģının dálilleń. \\

\end{tabular}
\vspace{1cm}


\begin{tabular}{m{17cm}}
\textbf{70-variant}
\newline

\textbf{T1.} Ekinshi tártipli betlik orayı, urınba tegisligi hám diametral tegisligi (Oray, Urınba tegislik, Diametral tegislik.) \\
\textbf{T2.} Tegislikte ekinshi tártipli sızıqlar (Ekinshi tártipli teńleme, Kvadrat kórinisindegi teńleme, Konik sızıqlar (konuslar kesimi)) \\
\textbf{A1.} Diskriminantın esaplaw arqalı tómendegi teńlemelerdiń hár biriniń tipin anıqlań: $x^2-4 x y+4 y^2+7 x-12=0$; \\
\textbf{A2.} $\frac{x^2}{36}+\frac{y^2}{20}=1$ ellips direktrisalarınıń teńlemelerin jazıń. \\
\textbf{A3.} Tóbesi koordinatalar basında bolǵan parabolanıń teńlemesin dúziń, bunda: parabola oń yarım tegislikte hám $Oy$ kósherine simmetriyalı jaylasqan, hám parametri $p=\frac{1}{4}$; \\
\textbf{B1.} Berilgen teńlemelerdiń parabolik ekenligin kórsetiń hám olardıń hár birin $(\alpha x+\beta y)^2+2 a_{13} x+2 a_{23} y+a_{33}=0$ kórinisinde jazıń: $9 x^2-6 x y+y^2-x+2 y-14=0$; \\
\textbf{B2.} Tómendegilerdi bilgen halda giperbolanıń teńlemesin dúziń: onıń tóbeleri arasındaǵı aralıq 24 ke teń hám fokusları $F_1 (-10; 2), F_2 (16; 2) $; \\
\textbf{B3.} Berilgen teńlemeler oraylıq iymek sızıqlar ekenligin kórsetiń hám hárbir teńlemeni koordinata basın orayģa kóshiriń: $4 x^2+6 x y+y^2-10 x-10=0$; \\
\textbf{C1.} $\frac{x^2}{9}+\frac{z^2}{4}=2 y$ elliptik paraboloid $2 x-2 y-z-10=0$ tegislik penen bir ulıwma noqatqa iye ekenligin dálilleń hám onıń koordinataların tabıń. \\
\textbf{C2.} $\frac{x^2}{a^2}+\frac{y^2}{b^2}=1$ ellipstiń $M_1 (x_1; y_1) $ noqatındaǵı urınbasınıń teńlemesin dúziń. \\
\textbf{C3.} Giperbolanıń bir diametr tóbelerinen ótkerilgen urınbalar parallel bolıwın dálilleń. \\

\end{tabular}
\vspace{1cm}


\begin{tabular}{m{17cm}}
\textbf{71-variant}
\newline

\textbf{T1.} Ekinshi tártipli sızıqlardıń ulıwma teńlemeleri (Ulıwma teńleme) \\
\textbf{T2.} Parabola hám onıń kanonikalıq teńlemeleri (Fokus (bagdarlawshı noqat), Direktrisa (bagdarlawshı sızıq), Kósher (simmetriya kósheri)) \\
\textbf{A1.} Fokusları ordinata kósherinde jatqan hám koordinatalar basına salıstırģanda simmetriyalı bolǵan ellipstiń teńlemesin dúziń, bunda: úlken yarım kósheri 10, fokusları arasındaǵı aralıq $2 c=8$; \\
\textbf{A2.} Berilgen teńleme menen qanday iymek sızıq anıqlanıwın tabıń: $\left\{\begin{array}{l}\frac{x^2}{3}+\frac{y^2}{6}=2 z, \\ 3 x-y+6 z-14=0\end{array}\right.$ \\
\textbf{A3.} Tómendegi maglıwmatlar boyınsha giperbolanıń kanonikalıq teńlemesin dúziń: fokusları arasındaǵı aralıq 10 ģa, úlken kósheri bolsa 8 ge teń. \\
\textbf{B1.} Berilgen teńlemeler oraylıq iymek sızıqlar ekenligin kórsetiń hám hárbir teńlemeni koordinata basın orayģa kóshiriń: $6 x^2+4 x y+y^2+4 x-2 y+2=0$; \\
\textbf{B2.} Parallel kóshiriw hám burıw túrlendiriwleri yamasa aǵzalardı gruppalaw járdeminde tómendegi betliklerdiń kórinisi hám jaylasıwı anıqlansın: $x^2+4 y^2-z^2-10 x-16 y+6 z+16=0$; \\
\textbf{B3.} ITECH túri, ólshemleri hám jaylasıwın anıqlań: $x^2-5 x y+4 y^2+x+2 y-2=0$. \\
\textbf{C1.} Hár qanday parabolik teńleme ushın $a_{11}$ hám $a_{22}$ koefficientler hár qıylı belgige iye sanlar bola almaytuģının hám olar bir waqıtta nolge aylana almaytuģının dálilleń. \\
\textbf{C2.} Tómendegi betliklerdiń kanonikalıq teńlemesi hám jaylasıwın anıqlań.: $x^2+5 y^2+z^2+2 x y+6 x z+2 y z-2 x+6 y+2 z=0$; \\
\textbf{C3.} $m$ hám $n$ tiń qanday mánislerinde $m x^2+12 x y+9 y^2+4 x+n y-13=0$ teńleme: 1) oraylıq sızıqtı; 2) orayga iye bolmaǵan sızıq; 3) sheksiz kóp orayǵa iye bolǵan sızıqtı ańlatadı. \\

\end{tabular}
\vspace{1cm}


\begin{tabular}{m{17cm}}
\textbf{72-variant}
\newline

\textbf{T1.} Ekinshi tártipli sızıq hám tuwrı sızıqtıń óz ara jaylasıwı (Kesilisiw noqatları, Urınba (urınıw) jaģdayı) \\
\textbf{T2.} Ekinshi tártipli betliklerdiń kanonikalıq teńlemeleri (Ellipsoid, Giperboloid (1 gewekli), Giperboloid (2 gewekli)) \\
\textbf{A1.} $y^2=24 x$ parabolanıń $F$ fokusın hám direktrisasınıń teńlemesin tabıń. \\
\textbf{A2.} Tómendegi sızıqlardan qaysı biri oraylıq (yaǵnıy birden-bir orayǵa iye), qaysı biri orayǵa iye emes, qaysı biri sheksiz kóp orayǵa iye ekenligin anıqlań: $4 x^2-4 x y+y^2-6 x+8 y+13=0$; \\
\textbf{A3.} Koordinatalar sistemasın túrlendirmesten tómendegi teńlemelerdiń hár biri birden-bir noqattı anıqlawın kórsetiń hám onıń koordinataların tabıń: $3 x^2+4 x y+y^2-2 x-1=0$; \\
\textbf{B1.} Ellipstegi ekssentrisitetti anıqlań, eger: ellips orayınan onıń direktrisasına túsirilgen perpendikulyar kesindisi ellipstiń tóbesi menen teń ekige bólinedi. \\
\textbf{B2.} $x^2=16y$ parabolanıń $2x+4y+7=0$ tuwrısına perpendikulyar bolǵan urınbasınıń teńlemesin dúziń. \\
\textbf{B3.} Berilgen teńleme parabolik ekenligin kórsetiń; ápiwayı túrge keltiriń; qanday geometriyalıq obrazdı anlatıwın anıqlań, eski hám de jańa koordinata kósherlerine salıstırģanda sızılmada súwretleń:$16 x^2-24 x y+9 y^2-160 x+120 y+425=0$. \\
\textbf{C1.} Giperbolanıń asimptotaların tabıń: $3 x^2+2 x y-y^2+8 x+10 y-14=0$; \\
\textbf{C2.} $\frac{x^2}{a^2}+\frac{y^2}{b^2}=1$ ellipske ishley sızılgan kvadrat tárepiniń uzınlıǵın esaplań. \\
\textbf{C3.} $\frac{x^2}{a^2}-\frac{y^2}{b^2}=1$ giperbolanıń fokusınan asimptotasına shekemgi aralıq $b$ qa teń ekenligin dálilleń. \\

\end{tabular}
\vspace{1cm}


\begin{tabular}{m{17cm}}
\textbf{73-variant}
\newline

\textbf{T1.} Ekinshi tártipli sızıqqa urınba, túyinles diametri teńlemesi (Urınba teńlemesi, Túyinles diametr: oraydan ótiwshi simmetriya kósherleri) \\
\textbf{T2.} Ekinshi tártipli betliklerdiń ulıwmalıq teńlemesin kanonikalıq túrge keltiriw (invariantlar járdeminde) \\
\textbf{A1.} Koordinatalar sistemasın túrlendirmesten, tómendegi teńlemelerdiń hár biri parabolanı anıqlawın kórsetiń hám parametrin tabıń: $9 x^2-6 x y+y^2-50 x+50 y-275=0$. \\
\textbf{A2.} Fokusları abscissa kósherinde jaylasqan, koordinatalar basına salıstırģanda simmetriyalı bolǵan giperbolanıń teńlemesin dúziń, bunda: asimtotalarınıń teńlemesi $y= \pm \frac{4}{3} x$ hám fokusları arasındaǵı aralıq $2 c=20$; \\
\textbf{A3.} Tómendegi sızıqlardan qaysı biri oraylıq (yaǵnıy birden-bir orayǵa iye), qaysı biri orayǵa iye emes, qaysı biri sheksiz kóp orayǵa iye ekenligin anıqlań: $4 x^2-20 x y+25 y^2-14 x+2 y-15=0$; \\
\textbf{B1.} Giperbolanıń yarım kósherlerin tabıń, eger: fokusları arasındaǵı aralıq 8 ge hám direktrisaları arasındaǵı aralıq 6 ģa teń. \\
\textbf{B2.} Berilgen teńlemeler oraylıq iymek sızıqlar ekenligin kórsetiń hám hárbir teńlemeni koordinata basın orayģa kóshiriń: $3x^2-6xy+2y^2-4x+2y+1=0$. \\
\textbf{B3.} ITECH túri, ólshemleri hám jaylasıwın anıqlań: $9 x^2+24 x y+16 y^2-230 x+110 y-475=0$. \\
\textbf{C1.} $A x+B y+C=0$ tuwrı sızıq qanday zárúrli hám jeterli shárt orınlanǵanda $\frac{x^2}{a^2}+\frac{y^2}{b^2}=1$ ellips penen 1) kesilisedi; 2) kesilispeydi. \\
\textbf{C2.} Eger ekinshi dárejeli teńleme parabolik bolıp, $ (\alpha x+\beta y) ^2+2a_{13}x+2a_{23}y+a_{33}=0$ kórinisinde jazılsa, onıń shep tárepindegi diskriminant $\Delta=- (a_{13} \beta-a_{23} \alpha) ^2$ formula menen anıqlanıwın dálilleń. \\
\textbf{C3.} Giperbolanıń asimptotalarınan direktrisaları ajıratqan kesindiler (giperbolanıń orayınan esaplaganda) giperbolanıń haqıyqıy yarım kósherine teń ekenligin dálilleń. Bul qásiyetten paydalanıp, giperbolanıń direktrisaların jasań. \\

\end{tabular}
\vspace{1cm}


\begin{tabular}{m{17cm}}
\textbf{74-variant}
\newline

\textbf{T1.} Parabola hám onıń kanonikalıq teńlemeleri (Fokus (bagdarlawshı noqat), Direktrisa (bagdarlawshı sızıq), Kósher (simmetriya kósheri)) \\
\textbf{T2.} Ekinshi tártipli sızıq orayı (Oraylıq sızıqlar (ellips, giperbola), Oray koordinataları: simmetriya orayı) \\
\textbf{A1.} Tóbesi koordinatalar basında bolǵan parabolanıń teńlemesin dúziń, bunda: parabola $Ox$ kósherine simmetriyalı jaylasqan hám $B (-1; 2) $ noqatınan ótedi; \\
\textbf{A2.} Fokusları abscissa kósherinde jatqan hám koordinatalar basına salıstırģanda simmetriyalı bolǵan ellipstiń teńlemesin dúziń, bunda: direktrisaları arasındaǵı aralıq 32 hám $\varepsilon=\frac{1}{2}$. \\
\textbf{A3.} Koordinatalar sistemasın túrlendirmesten, tómendegi teńlemelerdiń hár biri parabolanı anıqlawın kórsetiń hám parametrin tabıń: $x^2-2 x y+y^2+6 x-14 y+29=0$; \\
\textbf{B1.} Eger parabolanıń fokusı $F (7; 2) $ hám direktrisa $x-5=0$ teńlemesi berilgen bolsa, onıń teńlemesin dúziń. \\
\textbf{B2.} Ellipstegi ekssentrisitetti anıqlań, eger: fokusları arasındaǵı kesindiniń ózi kishi kósherdiń tóbesinen tuwrı múyesh astında kórinse; \\
\textbf{B3.} $\frac{x^2}{16}-\frac{y^2}{9}=1$ giperbolada fokal radiusları óz ara perpendikulyar bolgan noqat tabılsın. \\
\textbf{C1.} Giperbolanıń asimptotaların tabıń: $x^2-3 x y-10 y^2+6 x-8 y=0$; \\
\textbf{C2.} Múyesh koefficienti $k$ tiń qanday mánislerinde $y=kx+2$ tuwrısı: 1) $y^2=4x$ parabolanı kesip ótedi; 2) oǵan urınadı; 3) bul parabola sırtınan ótedi. \\
\textbf{C3.} $y=k x+m$ tuwrı sızıqtıń $\frac{x^2}{a^2}-\frac{y^2}{b^2}=1$ giperbolaģa urınıw shártin keltirip shıģarıń. \\

\end{tabular}
\vspace{1cm}


\begin{tabular}{m{17cm}}
\textbf{75-variant}
\newline

\textbf{T1.} Ekinshi tártipli betlik orayı, urınba tegisligi hám diametral tegisligi (Oray, Urınba tegislik, Diametral tegislik.) \\
\textbf{T2.} Tegislikte ekinshi tártipli sızıqlar (Ekinshi tártipli teńleme, Kvadrat kórinisindegi teńleme, Konik sızıqlar (konuslar kesimi)) \\
\textbf{A1.} Koordinatalar sistemasın túrlendirmesten tómendegi teńlemelerdiń hár biri giperbolanı anıqlawın kórsetiń hám onıń koordinataların tabıń: $3 x^2+4 x y-12 x+16=0$; \\
\textbf{A2.} Berilgen teńleme menen qanday iymek sızıq anıqlanıwın tabıń: $\left\{\begin{array}{l}\frac{x^2}{.4}+\frac{y^2}{9}-\frac{z^2}{36}=1, \\ 9 x-6 y+2 z-28=0,\end{array}\right.$ \\
\textbf{A3.} Tómendegi sızıqlardan qaysı biri oraylıq (yaǵnıy birden-bir orayǵa iye), qaysı biri orayǵa iye emes, qaysı biri sheksiz kóp orayǵa iye ekenligin anıqlań: $x^2-6 x y+9 y^2-12 x+36 y+20=0$; \\
\textbf{B1.} Lagranj usılınan paydalanıp, teńlemelerdi kvadratlar qosındısı túrine keltirip, tómendegi betlerdiń kórinisin anıqlań: $x^2-2 y^2+z^2+4 x y-8 x z-4 y z-14 x-4 y+14 z+16=0$; \\
\textbf{B2.} Berilgen teńlemelerdiń parabolik ekenligin kórsetiń hám olardıń hár birin $(\alpha x+\beta y)^2+2 a_{13} x+2 a_{23} y+a_{33}=0$ kórinisinde jazıń:  $16 x^2+16 x y+4 y^2-5 x+7 y=0$; \\
\textbf{B3.} $\varepsilon=\frac{2}{3}$ ellipsiniń ekssentrisiteti, $M$ ellips noqatınıń fokal radiusı 10 ģa teń. $M$ noqattan usı fokusqa sáykes direktrisaǵa shekem bolǵan aralıqtı esaplań. \\
\textbf{C1.} Berilgen teńleme kanonikalıq kóriniske keltirilsin; tipi anıqlansın; qanday geometriyalıq obrazdı anlatıwı anıqlansın; eski hám jana koordinatalar sistemasında geometriyalıq obrazı súwretlensin: $25 x^2-14 x y+25 y^2+64 x-64 y-224=0$; \\
\textbf{C2.} Tómendegi betliklerdiń kanonikalıq teńlemesi hám jaylasıwın anıqlań.: $4 x^2+9 y^2+z^2-12 x y-6 y z+4 z x+4 x-6 y+2 z-5=0$. \\
\textbf{C3.} Berilgen $y=k x+b$ tuwri sızıqqa parallel hám $y^2=2 p x$ parabolaga urinatuģın tuwri sızıqtıń teńlemesin jazıń. \\

\end{tabular}
\vspace{1cm}


\begin{tabular}{m{17cm}}
\textbf{76-variant}
\newline

\textbf{T1.} Bir gewekli giperboloid hám giperbolik paraboloidtıń tuwrı sızıqlı jasawshıları (Giperboloid, Giperbolik paraboloid, Sızıqlı jasawshılar) \\
\textbf{T2.} Tegislikte ekinshi tártipli sızıqlar (Ekinshi tártipli teńleme, Kvadrat kórinisindegi teńleme, Konik sızıqlar (konuslar kesimi)) \\
\textbf{A1.} Fokusları abscissa kósherinde jatqan hám koordinatalar basına salıstırģanda simmetriyalı bolǵan ellipstiń teńlemesin dúziń, bunda onıń kishi kósheri 24 ke, fokusları arasındaǵı aralıq bolsa $c = 10$ ga teń; \\
\textbf{A2.} Koordinatalar sistemasın túrlendirmesten, tómendegi teńlemelerdiń hár biri parabolanı anıqlawın kórsetiń hám parametrin tabıń: $9 x^2+24 x y+16 y^2-120 x+90 y=0$; \\
\textbf{A3.} Parabolanıń teńlemesin dúziń, eger: parabolanıń tóbesinen fokusına shekemgi aralıq 3 ke teń hám parabola $O x$ kósherine qarata simmetriyalı bolıp, $O y$ kósherine urınsa; \\
\textbf{B1.} $\frac{x^2}{64}-\frac{y^2}{36}=1$ giperbolanıń oń fokusına shekemgi aralıǵı 4,5 ke teń bolǵan noqatların anıqlań. \\
\textbf{B2.} Berilgen teńlemeler oraylıq iymek sızıqlar ekenligin kórsetiń hám hárbir teńlemeni koordinata basın orayģa kóshiriń: $6 x^2+4 x y+y^2+4 x-2 y+2=0$; \\
\textbf{B3.} ITECH túri, ólshemleri hám jaylasıwın anıqlań: $2 x^2+4 x y+5 y^2-6 x-8 y-1=0$; \\
\textbf{C1.} $m$ nıń qanday mánislerinde $x+m y-2=0$ tegislik $\frac{x^2}{2}+\frac{z^2}{3}=y$ elliptik paraboloidti a) ellips boyınsha, b) parabola boyınsha kesip ótetuǵınlıǵın anıqlań. \\
\textbf{C2.} $4 x^2-4 x y+y^2+6 x+1=0$ ETIS teńlemesi berilgen. Múyesh koefficienti $k$ tiń qanday mánislerinde $y=kx$ tuwrı sızıq: 1) bul iymek sızıqtı bir noqatta kesip ótiwi; 2) urınadı; 3) eki noqatta kesip ótiwin; 4) bul tuwrı menen ulıwma noqatqa iye bolmaytuģının anıqlań. \\
\textbf{C3.} $\frac{x^2}{30}+\frac{y^2}{24}=1$ ellipske $4x-2y+23=0$ parallel bolǵan urınbalardı júrgiziń hám olar arasındaģı aralıqtı esaplań. \\

\end{tabular}
\vspace{1cm}


\begin{tabular}{m{17cm}}
\textbf{77-variant}
\newline

\textbf{T1.} Ekinshi tártipli sızıq orayı (Oraylıq sızıqlar (ellips, giperbola), Oray koordinataları: simmetriya orayı) \\
\textbf{T2.} Ekinshi tártipli sızıq hám tuwrı sızıqtıń óz ara jaylasıwı (Kesilisiw noqatları, Urınba (urınıw) jaģdayı) \\
\textbf{A1.} $y^2+z^2=x$ elliptik paraboloidtıń $x+2 y-z=0$ tegislik penen kesilisiwiniń koordinata tegisliklerindegi proekciyalarınıń teńlemelerin tabıń. \\
\textbf{A2.} Tómendegi maglıwmatlar boyınsha giperbolanıń kanonikalıq teńlemesin dúziń: haqıyqıy kósheri 16 ǵa, asimptotası menen abscissa kósheri arasındaǵı múyesh $\varphi$. \\
\textbf{A3.} Koordinatalar sistemasın túrlendirmesten tómendegi teńlemelerdiń hár biri kesilisiwshi eki tuwrını anıqlawın kórsetiń hám onıń koordinataların tabıń: $x^2+4 x y+3 y^2-6 x-12 y+9=0$. \\
\textbf{B1.} $A (5;9) $ noqattan $y^2=5x$ parabolaǵa júrgizilgen urınbalardıń urınıw noqatların tutastırıwshı xordanıń teńlemesin dúziń. \\
\textbf{B2.} Lagranj usılınan paydalanıp, teńlemelerdi kvadratlar qosındısı túrine keltirip, tómendegi betlerdiń kórinisin anıqlań: $x^2+y^2+4 z^2+2 x y+4 x z+4 y z-6 z+1=0$; \\
\textbf{B3.} Berilgen teńleme parabolik ekenligin kórsetiń; ápiwayı túrge keltiriń; qanday geometriyalıq obrazdı anlatıwın anıqlań, eski hám de jańa koordinata kósherlerine salıstırģanda sızılmada súwretleń:$4 x^2+12 x y+9 y^2-4 x-6 y+1=0$. \\
\textbf{C1.} $m$ hám $n$ tiń qanday mánislerinde $m x^2+12 x y+9 y^2+4 x+n y-13=0$ teńleme: 1) oraylıq sızıqtı; 2) orayga iye bolmaǵan sızıq; 3) sheksiz kóp orayǵa iye bolǵan sızıqtı ańlatadı. \\
\textbf{C2.} $A x+B y+C=0$ tuwri sızıq $y^2=2 p x$ parabolaga urinıwı ushin zárúrli hám jeterli shártti tabıń. \\
\textbf{C3.} Berilgen teńleme kanonikalıq kóriniske keltirilsin; tipi anıqlansın; qanday geometriyalıq obrazdı anlatıwı anıqlansın; eski hám jana koordinatalar sistemasında geometriyalıq obrazı súwretlensin: $50 x^2-8 x y+35 y^2+100 x-8 y+67=0$; \\

\end{tabular}
\vspace{1cm}


\begin{tabular}{m{17cm}}
\textbf{78-variant}
\newline

\textbf{T1.} Ekinshi tártipli betliklerdiń kanonikalıq teńlemeleri (Paraboloid (ellipstik), Paraboloid (giperbolik), Konus, Cilindr) \\
\textbf{T2.} Parabola hám onıń kanonikalıq teńlemeleri (Fokus (bagdarlawshı noqat), Direktrisa (bagdarlawshı sızıq), Kósher (simmetriya kósheri)) \\
\textbf{A1.} Koordinatalar sistemasın túrlendirmesten, tómendegi teńlemelerdiń hár biri parabolanı anıqlawın kórsetiń hám parametrin tabıń: $9 x^2-24 x y+16 y^2-54 x-178 y+181=0$; \\
\textbf{A2.} Tómendegi sızıqlardan qaysı biri oraylıq (yaǵnıy birden-bir orayǵa iye), qaysı biri orayǵa iye emes, qaysı biri sheksiz kóp orayǵa iye ekenligin anıqlań:  $4 x^2+4 x y+y^2-8 x-4 y-21=0$; \\
\textbf{A3.} Berilgen teńleme menen qanday iymek sızıq anıqlanıwın tabıń: $\left\{\begin{array}{l}\frac{x^2}{4}-\frac{y^2}{3}=2 z \\ x-2 y+2=0 ;\end{array}\right.$ \\
\textbf{B1.} Eger parabolanıń fokusı $F(2;-1) $ hám direktrisa $x-y-1=0$ teńlemesi berilgen bolsa, onıń teńlemesin dúziń. \\
\textbf{B2.} Tómendegilerdi bilgen halda ellips teńlemesin dúziń: onıń kishi kósheri 2 ge teń hám fokusları $F_1 (-1;-1) $, $F_2 (1; 1) $; \\
\textbf{B3.} Berilgen teńlemeni ápiwayı túrge keltiriń; tipin anıqlań; qanday geometriyalıq obrazdı ańlatıwın anıqlań, eski hám de jańa koordinata kósherlerine qarata sızılmada súwretleń: $32 x^2+52 x y-7 y^2+180=0$; \\
\textbf{C1.} Giperbola asimptotalarınıń tenlemeleri $y= \pm \frac{1}{2} x$ hám urinbalardan biriniń teńlemesi. $5 x-6 y-8=0$ belgili bolsa, giperbola teńlemesin dúziń. \\
\textbf{C2.} Giperbolanıń asimptotaların tabıń: $3 x^2+7 x y+4 y^2+5 x+2 y-6=0$; \\
\textbf{C3.} $\frac{x^2}{a^2}+\frac{y^2}{b^2}=1$ ellipstiń bir diametriniń tóbelerine júrgizilgen urınbalar parallel bolıwın dálilleń (ellipstiń diametri dep onıń orayınan ótiwshi xordaǵa aytıladı). \\

\end{tabular}
\vspace{1cm}


\begin{tabular}{m{17cm}}
\textbf{79-variant}
\newline

\textbf{T1.} Tegislikte ekinshi tártipli sızıqlar (Ekinshi tártipli teńleme, Kvadrat kórinisindegi teńleme, Konik sızıqlar (konuslar kesimi)) \\
\textbf{T2.} Ekinshi tártipli sızıqlardıń ulıwma teńlemeleri (Ulıwma teńleme) \\
\textbf{A1.} Parabolanıń teńlemesin dúziń, eger: parabola $O x$ kósherine qarata simmetriyalı bolıp, $M (1;-4) $ noqatınan hám koordinatalar basınan ótedi; \\
\textbf{A2.} Fokusları abscissa kósherinde jatqan hám koordinatalar basına salıstırģanda simmetriyalı bolǵan ellipstiń teńlemesin dúziń, bunda: direktrisaları arasındaǵı aralıq 5 hám fokusları arasındaǵı aralıq $2c=4$; \\
\textbf{A3.} $16 x^2-9 y^2=144$ giperbola berilgen. Tabıń: 1) yarım kósherlerin; 2) fokusların; 3) ekssentrisitetin; 4) asimtotalarınıń teńlemesi; 5) direktrisaları tenlemelerin. \\
\textbf{B1.} Lagranj usılınan paydalanıp, teńlemelerdi kvadratlar qosındısı túrine keltirip, tómendegi betlerdiń kórinisin anıqlań: $x^2-2 y^2+z^2+4 x y-10 x z+4 y z+2 x+4 y-10 z-1=0$; \\
\textbf{B2.} Giperbolanıń yarım kósherlerin tabıń, eger: asimptotaları $y= \pm \frac{5}{3} x$ tenlemeleri menen berilgen hám giperbola $N (6,9) $ noqatınan ótedi. \\
\textbf{B3.} Berilgen teńlemeler oraylıq iymek sızıqlar ekenligin kórsetiń hám hárbir teńlemeni koordinata basın orayģa kóshiriń:  $4 x^2+2 x y+6 y^2+6 x-10 y+9=0$. \\
\textbf{C1.} Tómendegi betliklerdiń kanonikalıq teńlemesi hám jaylasıwın anıqlań.: $2 x^2+5 y^2+2 z^2-2 x y+2 y z-4 x z+2 x-10 y-2 z-1=0$. \\
\textbf{C2.} Tómendegi eki tuwrı sızıqqa urınatuģın giperbolanıń teńlemesin dúziń: $5x-6y-16=0$, $13x-10y-48=0$, bunda onıń kósherleri koordinata kósherleri menen ústpe-úst túsedi. \\
\textbf{C3.} $\frac{x^2}{a^2}+\frac{y^2}{b^2}=1$ ellipstiń $F(c, 0)$ fokusı arqalı úlken kósherine perpendikulyar bolǵan xorda ótkerilgen. Bul xordıń uzınlıǵın tabıń. \\

\end{tabular}
\vspace{1cm}


\begin{tabular}{m{17cm}}
\textbf{80-variant}
\newline

\textbf{T1.} Ekinshi tártipli betliklerdiń kanonikalıq teńlemeleri (Ellipsoid, Giperboloid (1 gewekli), Giperboloid (2 gewekli)) \\
\textbf{T2.} Parabola hám onıń kanonikalıq teńlemeleri (Fokus (bagdarlawshı noqat), Direktrisa (bagdarlawshı sızıq), Kósher (simmetriya kósheri)) \\
\textbf{A1.} Diskriminantın esaplaw arqalı tómendegi teńlemelerdiń hár biriniń tipin anıqlań: $3 x^2-8 x y+7 y^2+8 x-15 y+20=0$; \\
\textbf{A2.} Fokusları abscissa kósherinde jaylasqan, koordinatalar basına salıstırģanda simmetriyalı bolǵan giperbolanıń teńlemesin dúziń, bunda: direktrisalarınıń arasındaģı aralıq $22 \frac{2}{13}$ hám fokusları arasındaǵı aralıq $2 c=26$; \\
\textbf{A3.} Koordinatalar sistemasın túrlendirmesten, tómendegi teńlemelerdiń hár biri parabolanı anıqlawın kórsetiń hám parametrin tabıń: $x^2-2 x y+y^2+6 x-14 y+29=0$; \\
\textbf{B1.} Berilgen teńlemelerdiń parabolik ekenligin kórsetiń hám olardıń hár birin $(\alpha x+\beta y)^2+2 a_{13} x+2 a_{23} y+a_{33}=0$ kórinisinde jazıń: $25 x^2-20 x y+4 y^2+3 x-y+11=0$; \\
\textbf{B2.} Parabola tóbesiniń koordinataların, parametrin hám kósheriniń baǵıtın anıqlań: $y=x^2-8 x+15$, \\
\textbf{B3.} Berilgen teńlemelerdiń parabolik ekenligin kórsetiń hám olardıń hár birin $(\alpha x+\beta y)^2+2 a_{13} x+2 a_{23} y+a_{33}=0$ kórinisinde jazıń:  $16 x^2+16 x y+4 y^2-5 x+7 y=0$; \\
\textbf{C1.} $\frac{x^2}{9}+\frac{z^2}{4}=2 y$ elliptik paraboloid $2 x-2 y-z-10=0$ tegislik penen bir ulıwma noqatqa iye ekenligin dálilleń hám onıń koordinataların tabıń. \\
\textbf{C2.} Kósherleri óz ara perpendikulyar bolǵan eki parabola tórt noqatta kesilisse, bul noqatlar bir sheńberde jatıwın dálilleń. \\
\textbf{C3.} Parabolik teńleme $\Delta \neq 0$ bolǵanda hám tek sonda ǵana parabolanı anıqlaytuģının dálilleń. Bul jaǵdayda parabolanıń parametri $p=\sqrt{\frac{-\Delta}{ (a_{11}+a_{33}) ^3}}$ formula menen anıqlanıwın dálilleń. \\

\end{tabular}
\vspace{1cm}


\begin{tabular}{m{17cm}}
\textbf{81-variant}
\newline

\textbf{T1.} Ekinshi tártipli betliklerdiń ulıwma teńlemeleri (Ulıwma teńleme) \\
\textbf{T2.} Ekinshi tártipli sızıqlardıń ulıwma teńlemesin invariantlar járdeminde kanonikalıq túrge keltiriw \\
\textbf{A1.} Diskriminantın esaplaw arqalı tómendegi teńlemelerdiń hár biriniń tipin anıqlań: $5 x^2+14 x y+11 y^2+12 x-7 y+19=0$; \\
\textbf{A2.} Tómendegi sızıqlardan qaysı biri oraylıq (yaǵnıy birden-bir orayǵa iye), qaysı biri orayǵa iye emes, qaysı biri sheksiz kóp orayǵa iye ekenligin anıqlań: $4 x^2-4 x y+y^2-12 x+6 y-11=0$; \\
\textbf{A3.} Fokusları ordinata kósherinde jatqan hám koordinatalar basına qarata simmetriyalı bolǵan ellipstiń teńlemesin dúziń, bunda: kishi kósheri 16, a ekssentrisiteti $\varepsilon=\frac{3}{5}$; \\
\textbf{B1.} Berilgen sızıqlar oraylıq ekenligin kórsetiń hám hárbir iymek sızıq ushın orayınıń koordinataların tabıń: $2 x^2-6 x y+5 y^2+22 x-36 y+11=0$. \\
\textbf{B2.} ITECH túri, ólshemleri hám jaylasıwın anıqlań: $5 x^2+8 x y+5 y^2-18 x-18 y+9=0$; \\
\textbf{B3.} Fokusları $\frac{x^2}{100}+\frac{y^2}{64}=1$ ellipstiń tóbelerinde jatıwshı, direktrisaları bolsa usı ellipstiń fokuslarınan ótiwshi giperbolanıń teńlemesin dúziń. \\
\textbf{C1.} $m$ nıń qanday mánislerinde $x-2 y-2 z+m=0$ tegislik $\frac{x^2}{144}+\frac{y^2}{36}+\frac{z^2}{9}=1$ ellipsoidqa urınıwın anıqlań. \\
\textbf{C2.} Tómendegi betliklerdiń kanonikalıq teńlemesi hám jaylasıwın anıqlań.: $x^2-2 y^2+z^2+4 x y-8 x z-4 y z-14 x-4 y+14 z+16=0$. \\
\textbf{C3.} $A x+B y+C=0$ tuwrı sızıqtıń $\frac{x^2}{a^2}+\frac{y^2}{b^2}=1$, ellipske urınba bolıwı ushın zárúrli hám jeterli shárti tabılsın. \\

\end{tabular}
\vspace{1cm}


\begin{tabular}{m{17cm}}
\textbf{82-variant}
\newline

\textbf{T1.} Tegislikte ekinshi tártipli sızıqlar (Ekinshi tártipli teńleme, Kvadrat kórinisindegi teńleme, Konik sızıqlar (konuslar kesimi)) \\
\textbf{T2.} Ekinshi tártipli betliklerdiń kanonikalıq teńlemeleri (Ellipsoid, Giperboloid (1 gewekli), Giperboloid (2 gewekli)) \\
\textbf{A1.} $z+1=0$ tegislik bir qabatlı $\frac{x^2}{32}-\frac{y^2}{18}+\frac{z^2}{2}=1$ giperboloidti giperbola boyınsha kesip ótetuģının kórsetiń; onıń yarım kósherleri hám tóbelerin tabıń. \\
\textbf{A2.} Parabolanıń tóbesi ($\alpha;\beta$) noqat penen ústpe-úst túsetuģının bilgen halda onıń teńlemesin dúziń. Parametri $p$ ǵa teń. Onıń kósheri $O y$ kósherine parallel bolıp, $O y$ kósheriniń oń baǵıtında sheksizlikke sozilgan; \\
\textbf{A3.} Tómendegi sızıqlardan qaysı biri oraylıq (yaǵnıy birden-bir orayǵa iye), qaysı biri orayǵa iye emes, qaysı biri sheksiz kóp orayǵa iye ekenligin anıqlań: $x^2-6 x y+9 y^2-12 x+36 y+20=0$; \\
\textbf{B1.} Tómendegilerdi bilgen halda ellips teńlemesin dúziń: onıń úlken kósheri 26 ģa teń hám fokusları $F_1 (-10; 0), F_2 (14; 0) $; \\
\textbf{B2.} Lagranj usılınan paydalanıp, teńlemelerdi kvadratlar qosındısı túrine keltirip, tómendegi betlerdiń kórinisin anıqlań: $x y+x z+y z+2 x+2 y-2 z=0$. \\
\textbf{B3.} Berilgen teńlemeler oraylıq iymek sızıqlar ekenligin kórsetiń hám hárbir teńlemeni koordinata basın orayģa kóshiriń: $4 x^2+6 x y+y^2-10 x-10=0$; \\
\textbf{C1.} Giperbolanıń asimptotaların tabıń: $3 x^2+7 x y+4 y^2+5 x+2 y-6=0$; \\
\textbf{C2.} Elliptik túrdegi ($\delta>0$) teńleme $a_{11}$ hám $\Delta$ lardıń hár qıylı belgige iye sanlar bolǵanda ǵana ellipsti anıqlawın dálilleń. \\
\textbf{C3.} $A\left(\frac{10}{3}; \frac{5}{3}\right)$ noqatınan $\frac{x2}{20}+\frac{y2}{5}=1$ ellipske urınbalar ótkerilgen. Olardıń teńlemelerin dúziń. \\

\end{tabular}
\vspace{1cm}


\begin{tabular}{m{17cm}}
\textbf{83-variant}
\newline

\textbf{T1.} Ekinshi tártipli sızıqqa urınba, túyinles diametri teńlemesi (Urınba teńlemesi, Túyinles diametr: oraydan ótiwshi simmetriya kósherleri) \\
\textbf{T2.} Ekinshi tártipli sızıq hám tuwrı sızıqtıń óz ara jaylasıwı (Kesilisiw noqatları, Urınba (urınıw) jaģdayı) \\
\textbf{A1.} Koordinatalar sistemasın túrlendirmesten, tómendegi teńlemeler menen qanday geometriyalıq obrazdı anıqlanıwın tabıń: $2 x^2+3 x y-2 y^2+5 x+10 y=0$; \\
\textbf{A2.} Giperbolanıń ekssentrisiteti $varepsilon=3$, $M$ noqatınıń bazı bir fokal radiusı 4 ke teń. $M$ noqattan sáykes direktrisaģa shekem bolǵan aralıqtı tabıń. \\
\textbf{A3.} Koordinatalar sistemasın túrlendirmesten, tómendegi teńlemelerdiń hár biri parabolanı anıqlawın kórsetiń hám parametrin tabıń: $9 x^2-6 x y+y^2-50 x+50 y-275=0$. \\
\textbf{B1.} Giperbolanıń yarım kósherlerin tabıń, eger: asimptotaları $y= \pm 2 x$ tenlemeleri menen berilgen hám fokusları oraydan 5 birlik aralıqta; \\
\textbf{B2.} ITECH túri, ólshemleri hám jaylasıwın anıqlań: $5 x^2+12 x y-12 x-22 y-19=0$. \\
\textbf{B3.} Berilgen teńleme parabolik ekenligin kórsetiń; ápiwayı túrge keltiriń; qanday geometriyalıq obrazdı anlatıwın anıqlań, eski hám de jańa koordinata kósherlerine salıstırģanda sızılmada súwretleń: $x^2-2 x y+y^2-12 x+12 y-14=0$ \\
\textbf{C1.} $\frac{x^2}{a^2}-\frac{y^2}{b^2}=1$ giperbolanıń fokusınan asimptotasına shekemgi aralıq $b$ qa teń ekenligin dálilleń. \\
\textbf{C2.} $m$ hám $n$ tiń qanday mánislerinde $m x^2+12 x y+9 y^2+4 x+n y-13=0$ teńleme: 1) oraylıq sızıqtı; 2) orayga iye bolmaǵan sızıq; 3) sheksiz kóp orayǵa iye bolǵan sızıqtı ańlatadı. \\
\textbf{C3.} $\frac{x^2}{a^2}-\frac{y^2}{b^2}=1$ giperbola hám onıń qanday da bir urınbası berilgen: $P$-urınbasınıń $O x$ kósheri menen kesilisiw noqatı, $Q$ - urınba noqatınıń sol kósherdegi proekciyası. $O P \cdot O Q=a^2$ ekenligin dálilleń. \\

\end{tabular}
\vspace{1cm}


\begin{tabular}{m{17cm}}
\textbf{84-variant}
\newline

\textbf{T1.} Parabola hám onıń kanonikalıq teńlemeleri (Fokus (bagdarlawshı noqat), Direktrisa (bagdarlawshı sızıq), Kósher (simmetriya kósheri)) \\
\textbf{T2.} Ekinshi tártipli betliklerdiń kanonikalıq teńlemeleri (Paraboloid (ellipstik), Paraboloid (giperbolik), Konus, Cilindr) \\
\textbf{A1.} Fokusları abscissa kósherinde jatqan hám koordinatalar basına qarata simmetriyalı bolǵan ellipstiń teńlemesi dúzilsin, bunda: $M_1 (2;-2) $ noqatı ellipske tiyisli hám úlken yarım kósheri $a=4$; \\
\textbf{A2.} $\frac{x^2}{20}-\frac{y^2}{5}=-1$ giperbola hám $y^2=3 x$ parabolanıń kesilisiw noqatların anıqlań. \\
\textbf{A3.} Berilgen teńleme menen qanday iymek sızıq anıqlanıwın tabıń: $\left\{\begin{array}{l}\frac{x^2}{3}+\frac{y^2}{6}=2 z, \\ 3 x-y+6 z-14=0\end{array}\right.$ \\
\textbf{B1.} Parabola tóbesiniń koordinataların, parametrin hám kósheriniń baǵıtın anıqlań: $y=x^2+6 x$. \\
\textbf{B2.} $\frac{x^2}{16}+\frac{y^2}{9}=1$ ellipsning $x+y-1=0$ to'g'ri chiziqqa parallel bo'lgan urinmalarini aniqlang. \\
\textbf{B3.} Parallel kóshiriw hám burıw túrlendiriwleri yamasa aǵzalardı gruppalaw járdeminde tómendegi betliklerdiń kórinisi hám jaylasıwı anıqlansın: $3 x^2+3 y^2-3 z^2-6 x+4 y+4 z+3=0$; \\
\textbf{C1.} Kósherleri óz ara perpendikulyar bolǵan eki parabola tórt noqatta kesilisse, bul noqatlar bir sheńberde jatıwın dálilleń. \\
\textbf{C2.} Hár qanday parabolik teńleme $ (\alpha x+\beta y) ^2+2a_{13}x+2a_{23}y+a_{33}=0$ kórinisinde jazılıwı múmkinligin dálilleń. Sonday-aq, elliptikalıq hám giperbolikalıq teńlemelerdi bunday kóriniste jazıp bolmaytuģının dálilleń. \\
\textbf{C3.} Berilgen $y=k x+b$ tuwri sızıqqa parallel hám $y^2=2 p x$ parabolaga urinatuģın tuwri sızıqtıń teńlemesin jazıń. \\

\end{tabular}
\vspace{1cm}


\begin{tabular}{m{17cm}}
\textbf{85-variant}
\newline

\textbf{T1.} Ekinshi tártipli sızıqlardıń ulıwma teńlemesin invariantlar járdeminde kanonikalıq túrge keltiriw \\
\textbf{T2.} Parabola hám onıń kanonikalıq teńlemeleri (Fokus (bagdarlawshı noqat), Direktrisa (bagdarlawshı sızıq), Kósher (simmetriya kósheri)) \\
\textbf{A1.} Tómendegi maglıwmatlar boyınsha giperbolanıń kanonikalıq teńlemesin dúziń: ekssentrisiteti $e=\frac{12}{13}$ úlken kósheri 48 ge teń. \\
\textbf{A2.} Koordinatalar sistemasın túrlendirmesten, tómendegi teńlemelerdiń hár biri parabolanı anıqlawın kórsetiń hám parametrin tabıń: $x^2-2 x y+y^2+6 x-14 y+29=0$; \\
\textbf{A3.} Koordinatalar sistemasın túrlendirmesten tómendegi teńlemelerdiń hár biri ellipsti anıqlawın kórsetiń hám onıń yarım kósherlerin tabıń: $13 x^2+18 x y+37 y^2-26 x-18 y+3=0$; \\
\textbf{B1.} Berilgen teńleme parabolik ekenligin kórsetiń; ápiwayı túrge keltiriń; qanday geometriyalıq obrazdı anlatıwın anıqlań, eski hám de jańa koordinata kósherlerine salıstırģanda sızılmada súwretleń:$9 x^2+12 x y+4 y^2-24 x-16 y+3=0$; \\
\textbf{B2.} Berilgen sızıqlar oraylıq ekenligin kórsetiń hám hárbir iymek sızıq ushın orayınıń koordinataların tabıń:$9 x^2-4 x y-7 y^2-12=0$; \\
\textbf{B3.} $\frac{x^2}{80}-\frac{y^2}{20}=1$ giperbolada $M_1 (10;-\sqrt{5}) $ noqat berilgen. $M_1$ noqatınıń fokal radiusları jatqan tuwrı sızıqlardıń teńlemelerin dúziń. \\
\textbf{C1.} Múyesh koefficienti $k$ tiń qanday mánislerinde $y=kx+2$ tuwrısı: 1) $y^2=4x$ parabolanı kesip ótedi; 2) oǵan urınadı; 3) bul parabola sırtınan ótedi. \\
\textbf{C2.} Fokuslardan ellipstiń qálegen urınbasına shekem bolǵan aralıqlar kóbeymesi kishi yarım kósherdiń kvadratına teń ekenligin dálilleń. \\
\textbf{C3.} $y^2=4 x$ parabola menen $\frac{x^2}{8}+\frac{y^2}{2}=1$ ellipstiń uliwma urinbaların anıqlań. \\

\end{tabular}
\vspace{1cm}


\begin{tabular}{m{17cm}}
\textbf{86-variant}
\newline

\textbf{T1.} Bir gewekli giperboloid hám giperbolik paraboloidtıń tuwrı sızıqlı jasawshıları (Giperboloid, Giperbolik paraboloid, Sızıqlı jasawshılar) \\
\textbf{T2.} Ekinshi tártipli betlik orayı, urınba tegisligi hám diametral tegisligi (Oray, Urınba tegislik, Diametral tegislik.) \\
\textbf{A1.} Tóbesi koordinatalar basında bolǵan parabolanıń teńlemesin dúziń, bunda: parabola $Ox$ kósherine simmetriyalı jaylasqan hám $A (9; 6) $ noqatınan ótedi; \\
\textbf{A2.} Fokusları ordinata kósherinde jatqan hám koordinatalar basına qarata simmetriyalı bolǵan ellipstiń teńlemesin dúziń, bunda: fokusları arasındaǵı aralıq $2 c=6$, direktrisaları arasındaǵı aralıq $16 \frac{2}{3}$; \\
\textbf{A3.} Tómendegi sızıqlardan qaysı biri oraylıq (yaǵnıy birden-bir orayǵa iye), qaysı biri orayǵa iye emes, qaysı biri sheksiz kóp orayǵa iye ekenligin anıqlań: $4 x^2+5 x y+3 y^2-x+9 y-12=0$; \\
\textbf{B1.} Parabola tóbesi $A (-2;-1) $ hám onıń direktrisasınıń teńlemesi $x+2y-1=0$ berilgen. Bul parabolanıń teńlemesin dúziń. \\
\textbf{B2.} Parallel kóshiriw hám burıw túrlendiriwleri yamasa aǵzalardı gruppalaw járdeminde tómendegi betliklerdiń kórinisi hám jaylasıwı anıqlansın: $z=x^2+2 x y+y^2+1$; \\
\textbf{B3.} $\varepsilon=\frac{2}{5}$ ellipstiń ekssentrisiteti, ellipstiń $M$ noqatınan direktrisaģa shekemgi aralıq 20 ģa teń. $M$ noqattan usı direktrisa menen bir tárepleme fokusqa shekem bolǵan aralıqtı esaplań. \\
\textbf{C1.} $\frac{x^2}{a^2}-\frac{y^2}{b^2}=1$ giperbolanıń asimptotaları hám onıń qálegen noqatınan asimptotalarga parallel etip ótkerilgen tuwrı sızıqlar menen shegaralanǵan parallelogrammnıń maydanı turaqlı san bolıp, $\frac{a b}{2}$ ga teń bolatuģının dálilleń. \\
\textbf{C2.} Berilgen teńleme kanonikalıq kóriniske keltirilsin; tipi anıqlansın; qanday geometriyalıq obrazdı anlatıwı anıqlansın; eski hám jana koordinatalar sistemasında geometriyalıq obrazı súwretlensin: $5 x^2-2 x y+5 y^2-4 x+20 y+20=0$. \\
\textbf{C3.} Eki gewekli $\frac{x^2}{3}+\frac{y^2}{4}-\frac{z^2}{25}=-1$ giperboloid $5 x+2 z+5=0$ tegislik penen bir ulıwma noqatqa iye ekenligin dálilleń hám onıń koordinataların tabıń. \\

\end{tabular}
\vspace{1cm}


\begin{tabular}{m{17cm}}
\textbf{87-variant}
\newline

\textbf{T1.} Tegislikte ekinshi tártipli sızıqlar (Ekinshi tártipli teńleme, Kvadrat kórinisindegi teńleme, Konik sızıqlar (konuslar kesimi)) \\
\textbf{T2.} Ekinshi tártipli sızıqqa urınba, túyinles diametri teńlemesi (Urınba teńlemesi, Túyinles diametr: oraydan ótiwshi simmetriya kósherleri) \\
\textbf{A1.} $x-2=0$ tegislik $\frac{x^2}{16}+\frac{y^2}{12}+\frac{z^2}{4}=1$ ellipsoidti ellips boyınsha kesip ótetuģının kórsetiń; onıń yarım kósherleri hám tóbelerin tabıń. \\
\textbf{A2.} Berilgen teńleme menen qanday iymek sızıq anıqlanıwın tabıń: $\left\{\begin{array}{l}\frac{x^2}{.4}+\frac{y^2}{9}-\frac{z^2}{36}=1, \\ 9 x-6 y+2 z-28=0,\end{array}\right.$ \\
\textbf{A3.} Koordinatalar sistemasın túrlendirmesten, tómendegi teńlemelerdiń hár biri parabolanı anıqlawın kórsetiń hám parametrin tabıń: $9 x^2-24 x y+16 y^2-54 x-178 y+181=0$; \\
\textbf{B1.} Berilgen teńlemeniń tipin anıqlań, koordinata kósherlerin parallel kóshiriw arqalı ápiwayı túrge keltiriń; qanday geometriyalıq obrazdı ańlatıwın anıqlań, eski hám jańa koordinata kósherlerine salıstırģanda sızılmada súwretleń: $9 x^2-16 y^2-54 x-64 y-127=0$; \\
\textbf{B2.} ITECH túri, ólshemleri hám jaylasıwın anıqlań: $5 x^2+4 x y+8 y^2-32 x-56 y+80=0$. \\
\textbf{B3.} Berilgen sızıqlar oraylıq ekenligin kórsetiń hám hárbir iymek sızıq ushın orayınıń koordinataların tabıń:$5 x^2+4 x y+2 y^2+20 x+20 y-18=0$; \\
\textbf{C1.} Giperbolanıń asimptotaların tabıń: $10 x^2+21 x y+9 y^2-41 x-39 y+4=0$. \\
\textbf{C2.} Tómendegi betliklerdiń kanonikalıq teńlemesi hám jaylasıwın anıqlań.: $5 x^2-y^2+z^2+4 x y+6 x z+2 x+4 y+6 z-8=0$. \\
\textbf{C3.} Parabolik teńleme $\Delta \neq 0$ bolǵanda hám tek sonda ǵana parabolanı anıqlaytuģının dálilleń. Bul jaǵdayda parabolanıń parametri $p=\sqrt{\frac{-\Delta}{ (a_{11}+a_{33}) ^3}}$ formula menen anıqlanıwın dálilleń. \\

\end{tabular}
\vspace{1cm}


\begin{tabular}{m{17cm}}
\textbf{88-variant}
\newline

\textbf{T1.} Ekinshi tártipli betliklerdiń ulıwma teńlemeleri (Ulıwma teńleme) \\
\textbf{T2.} Ekinshi tártipli sızıqlardıń ulıwma teńlemeleri (Ulıwma teńleme) \\
\textbf{A1.} Tóbesi koordinatalar basında bolǵan parabolanıń teńlemesin dúziń, bunda: parabola oń yarım tegislikte hám $Ox$ kósherine simmetriyalı jaylasqan, hám parametri $p=3$; \\
\textbf{A2.} Tómendegi sızıqlardan qaysı biri oraylıq (yaǵnıy birden-bir orayǵa iye), qaysı biri orayǵa iye emes, qaysı biri sheksiz kóp orayǵa iye ekenligin anıqlań: $4 x^2-20 x y+25 y^2-14 x+2 y-15=0$; \\
\textbf{A3.} Fokusları ordinata kósherinde jaylasqan, koordinatalar basına salıstırģanda simmetriyalı bolǵan giperbolanıń teńlemesin dúziń, bunda: fokusları arasındaģı aralıq $2 c=10$ hám ekssentrisiteti $\varepsilon=\frac{5}{3}$; \\
\textbf{B1.} Parabola tóbesiniń koordinataların, parametrin hám kósheriniń baǵıtın anıqlań: $y^2-10 x-2 y-19=0$; \\
\textbf{B2.} Berilgen teńleme parabolik ekenligin kórsetiń; ápiwayı túrge keltiriń; qanday geometriyalıq obrazdı anlatıwın anıqlań, eski hám de jańa koordinata kósherlerine salıstırģanda sızılmada súwretleń:$4 x^2+12 x y+9 y^2-4 x-6 y+1=0$. \\
\textbf{B3.} Parallel kóshiriw hám burıw túrlendiriwleri yamasa aǵzalardı gruppalaw járdeminde tómendegi betliklerdiń kórinisi hám jaylasıwı anıqlansın: $3 x^2+6 x-8 y+6 z-7=0$; \\
\textbf{C1.} $\frac{x^2}{100}+\frac{y^2}{64}=1$ ellipstiń $2 x-y+7=0,2 x-y-1=0$ xordalarınıń ortaları arqalı ótetuģın tuwrı sızıqtıń teńlemesin dúziń. \\
\textbf{C2.} $4 x^2-4 x y+y^2+6 x+1=0$ ETIS teńlemesi berilgen. Múyesh koefficienti $k$ tiń qanday mánislerinde $y=kx$ tuwrı sızıq: 1) bul iymek sızıqtı bir noqatta kesip ótiwi; 2) urınadı; 3) eki noqatta kesip ótiwin; 4) bul tuwrı menen ulıwma noqatqa iye bolmaytuģının anıqlań. \\
\textbf{C3.} Giperbolanıń asimptotalarınan direktrisaları ajıratqan kesindiler (giperbolanıń orayınan esaplaganda) giperbolanıń haqıyqıy yarım kósherine teń ekenligin dálilleń. Bul qásiyetten paydalanıp, giperbolanıń direktrisaların jasań. \\

\end{tabular}
\vspace{1cm}


\begin{tabular}{m{17cm}}
\textbf{89-variant}
\newline

\textbf{T1.} Parabola hám onıń kanonikalıq teńlemeleri (Fokus (bagdarlawshı noqat), Direktrisa (bagdarlawshı sızıq), Kósher (simmetriya kósheri)) \\
\textbf{T2.} Tegislikte ekinshi tártipli sızıqlar (Ekinshi tártipli teńleme, Kvadrat kórinisindegi teńleme, Konik sızıqlar (konuslar kesimi)) \\
\textbf{A1.} Ekscentrisiteti $\varepsilon=\frac{2}{3}$, fokusı $F (2; 1) $ hám usı fokus tárepindegi direktrisası $x-5=0$ bolǵan ellipstiń teńlemesin dúziń. \\
\textbf{A2.} Diskriminantın esaplaw arqalı tómendegi teńlemelerdiń hár biriniń tipin anıqlań: $2 x^2+10 x y+12 y^2-7 x+18 y-15=0$; \\
\textbf{A3.} Tómendegi sızıqlardan qaysı biri oraylıq (yaǵnıy birden-bir orayǵa iye), qaysı biri orayǵa iye emes, qaysı biri sheksiz kóp orayǵa iye ekenligin anıqlań: $3 x^2-4 x y-2 y^2+3 x-12 y-7=0$; \\
\textbf{B1.} Ellipstegi ekssentrisitetti anıqlań, eger: onıń kishi kósheri fokuslardan $60^{\circ}$ múyesh astında kórinse; \\
\textbf{B2.} Giperbolanıń asimptotaları arasındaǵı múyeshin tabıń, eger: fokusları arasındaǵı qashıqlıq direktrisaları arasındaǵı qashıqlıqtan eki ese úlken bolsa. \\
\textbf{B3.} Eger parabolanıń fokusı $F (4;3) $ hám direktrisa $x-1=0$ teńlemesi berilgen bolsa, onıń teńlemesin dúziń. \\
\textbf{C1.} Qálegen elliptik teńleme ushın $a_{11}$ hám $a_{22}$ koefficientleriniń hesh biri nolge aylana almaytuģınlıǵın hám olar birdey belgige iye sanlar ekenligin dálilleń. \\
\textbf{C2.} $4 x^2-4 x y+y^2+6 x+1=0$ ETIS teńlemesi berilgen. Múyesh koefficienti $k$ tiń qanday mánislerinde $y=kx$ tuwrı sızıq: 1) bul iymek sızıqtı bir noqatta kesip ótiwi; 2) urınadı; 3) eki noqatta kesip ótiwin; 4) bul tuwrı menen ulıwma noqatqa iye bolmaytuģının anıqlań. \\
\textbf{C3.} Ekinshi dárejeli teńleme tek hám tek $\Delta=0$ bolǵanda ǵana aynıǵan iymek sızıq teńlemesi bolatuģının dálilleń. \\

\end{tabular}
\vspace{1cm}


\begin{tabular}{m{17cm}}
\textbf{90-variant}
\newline

\textbf{T1.} Ekinshi tártipli sızıq orayı (Oraylıq sızıqlar (ellips, giperbola), Oray koordinataları: simmetriya orayı) \\
\textbf{T2.} Ekinshi tártipli betliklerdiń ulıwmalıq teńlemesin kanonikalıq túrge keltiriw (invariantlar járdeminde) \\
\textbf{A1.} Berilgen teńleme qanday iymek sızıq ekenligin tabıń: $y=+\frac{2}{3} \sqrt{x^2-9}$ \\
\textbf{A2.} Berilgen teńleme menen qanday iymek sızıq anıqlanıwın tabıń: $\left\{\begin{array}{l}\frac{x^2}{4}-\frac{y^2}{3}=2 z \\ x-2 y+2=0 ;\end{array}\right.$ \\
\textbf{A3.} Tóbesi koordinatalar basında bolǵan parabolanıń teńlemesin dúziń, bunda: parabola $Oy$ kósherine simmetriyalı jaylasqan hám $C (1; 1) $ noqatınan ótedi; \\
\textbf{B1.} Tómendegilerdi bilgen halda giperbolanıń teńlemesin dúziń: Asimptotaları arasındaǵı múyesh $90^{\circ}$ qa teń hám fokuslar $F_1 (4;-4), F_2 (-2; 2) $. \\
\textbf{B2.} Berilgen teńlemelerdiń parabolik ekenligin kórsetiń hám olardıń hár birin $(\alpha x+\beta y)^2+2 a_{13} x+2 a_{23} y+a_{33}=0$ kórinisinde jazıń: $9 x^2-6 x y+y^2-x+2 y-14=0$; \\
\textbf{B3.} Berilgen sızıqlar oraylıq ekenligin kórsetiń hám hárbir iymek sızıq ushın orayınıń koordinataların tabıń: $3x^2+5xy+y^2-8x-11y-7=0$. \\
\textbf{C1.} $\frac{x^2}{81}+\frac{y^2}{36}+\frac{z^2}{9}=1$ ellipsoid $4 x-3 y+12 z-54=0$ tegislik penen bir ulıwma noqatqa iye ekenligin dálilleń hám onıń koordinataların tabıń. \\
\textbf{C2.} $\frac{x^2}{a^2}-\frac{y^2}{b^2}=1$ giperbolanıń fokuslarınan urınbasına shekemgi aralıqlardıń kóbeymesin tabıń. \\
\textbf{C3.} Giperbolanıń asimptotaların tabıń: $10 x y-2 y^2+6 x+4 y+21=0$ \\

\end{tabular}
\vspace{1cm}


\begin{tabular}{m{17cm}}
\textbf{91-variant}
\newline

\textbf{T1.} Ekinshi tártipli sızıq orayı (Oraylıq sızıqlar (ellips, giperbola), Oray koordinataları: simmetriya orayı) \\
\textbf{T2.} Ekinshi tártipli betlik orayı, urınba tegisligi hám diametral tegisligi (Oray, Urınba tegislik, Diametral tegislik.) \\
\textbf{A1.} Koordinatalar sistemasın túrlendirmesten tómendegi teńlemelerdiń hár biri ellipsti anıqlawın kórsetiń hám onıń yarım kósherlerin tabıń: $8 x^2+4 x y+5 y^2+16 x+4 y-28=0$; \\
\textbf{A2.} Koordinatalar sistemasın túrlendirmesten, tómendegi teńlemelerdiń hár biri parabolanı anıqlawın kórsetiń hám parametrin tabıń: $9 x^2+24 x y+16 y^2-120 x+90 y=0$; \\
\textbf{A3.} $\frac{x^2}{100}+\frac{y^2}{36}=1$ ellipsinde jaylasqan hám oń fokusına shekemgi aralıǵı 14 ke teń noqattı tabıń. \\
\textbf{B1.} Parallel kóshiriw hám burıw túrlendiriwleri yamasa aǵzalardı gruppalaw járdeminde tómendegi betliklerdiń kórinisi hám jaylasıwı anıqlansın: $z^2=3 x+4 y+5$; \\
\textbf{B2.} $M_1 (2;-1) $ noqatı fokusı $F (1;0) $ bolǵan ellipste jatadı. Bul fokusqa sáykes direktrisa bolsa $2x-y-10=0$ teńleme menen berilgen. Usı ellipstiń teńlemesin dúziń. \\
\textbf{B3.} ITECH túri, ólshemleri hám jaylasıwın anıqlań: $4 x^2-4 x y+y^2-2 x-14 y+7=0$. \\
\textbf{C1.} Ellipstiń yarım kósherleri $a$, $b$ hám orayı $C\left(x_0; y_0\right) $ noqatında bolıp, simmetriya kósherleri koordinata kósherlerine parallel ekenligi belgili bolsa, onıń teńlemesin dúziń. \\
\textbf{C2.} $y^2=2 p x$ parabolaǵa onıń $M_1\left(x_1; y_1\right) $ noqatındaǵı urınbasınıń teńlemesin dúziń. \\
\textbf{C3.} Tómendegi betliklerdiń kanonikalıq teńlemesi hám jaylasıwın anıqlań.: $5 x^2+2 y^2+5 z^2-4 x y-2 x y-4 y z+10 x-4 y-2 z+4=0$; \\

\end{tabular}
\vspace{1cm}


\begin{tabular}{m{17cm}}
\textbf{92-variant}
\newline

\textbf{T1.} Parabola hám onıń kanonikalıq teńlemeleri (Fokus (bagdarlawshı noqat), Direktrisa (bagdarlawshı sızıq), Kósher (simmetriya kósheri)) \\
\textbf{T2.} Ekinshi tártipli sızıqlardıń ulıwma teńlemeleri (Ulıwma teńleme) \\
\textbf{A1.} Tóbesi koordinatalar basında bolǵan parabolanıń teńlemesin dúziń, bunda: parabola $Oy$ kósherine simmetriyalı jaylasqan hám $D (4; -8) $ noqatınan ótedi; \\
\textbf{A2.} Koordinatalar sistemasın túrlendirmesten tómendegi teńlemelerdiń hár biri ellipsti anıqlawın kórsetiń hám onıń yarım kósherlerin tabıń: $41 x^2+24 x y+9 y^2+24 x+18 y-36=0$; \\
\textbf{A3.} $y^2+z^2=x$ elliptik paraboloidtıń $x+2 y-z=0$ tegislik penen kesilisiwiniń koordinata tegisliklerindegi proekciyalarınıń teńlemelerin tabıń. \\
\textbf{B1.} Berilgen parabola tóbesi $A (6;-3) $ hám onıń direktrisasınıń teńlemesi $3x-5y+1=0$ berilgen. Bul parabolanıń $F$ fokusın tabıń. \\
\textbf{B2.} Berilgen teńlemeler oraylıq iymek sızıqlar ekenligin kórsetiń hám hárbir teńlemeni koordinata basın orayģa kóshiriń: $3x^2-6xy+2y^2-4x+2y+1=0$. \\
\textbf{B3.} Tómendegilerdi bilgen halda ellips teńlemesin dúziń: onıń fokusları $F_1\left(-2; \frac{3}{2}\right), F_2\left(2;-\frac{3}{2}\right) $ hám ekssentrisitet $\varepsilon=\frac{\sqrt{2}}{2}$; \\
\textbf{C1.} Parabolanıń qálegen urinbasınıń direktrisası hám kósherge perpendikulyar bolǵan fokal xordanı fokustan teńdey uzaqlıqtaģı noqatlarda kesetuģının dálilleń. \\
\textbf{C2.} $\frac{x^2}{a^2}+\frac{y^2}{b^2}=1$ ellipske ishley sızılgan kvadrat tárepiniń uzınlıǵın esaplań. \\
\textbf{C3.} $\frac{x^2}{a^2}-\frac{y^2}{b^2}=1$ giperbolanıń qálegen noqatınan onıń eki asimptotasına shekemgi aralıqlar kóbeymesi $\frac{a^2 b^2}{a^2+b^2}$ ǵa teń turaqlı shama ekenligin dálilleń. \\

\end{tabular}
\vspace{1cm}


\begin{tabular}{m{17cm}}
\textbf{93-variant}
\newline

\textbf{T1.} Ekinshi tártipli betliklerdiń ulıwma teńlemeleri (Ulıwma teńleme) \\
\textbf{T2.} Tegislikte ekinshi tártipli sızıqlar (Ekinshi tártipli teńleme, Kvadrat kórinisindegi teńleme, Konik sızıqlar (konuslar kesimi)) \\
\textbf{A1.} Fokusları ordinata kósherinde jatqan hám koordinatalar basına salıstırģanda simmetriyalı bolǵan ellipstiń teńlemesin dúziń, bunda: direktrisaları arasındaģı qashıqlıq $frac{2}{3}$ hám ekssentrisiteti $frac{3}{4}$. \\
\textbf{A2.} Koordinatalar sistemasın túrlendirmesten, tómendegi teńlemelerdiń hár biri parabolanı anıqlawın kórsetiń hám parametrin tabıń: $9 x^2-6 x y+y^2-50 x+50 y-275=0$. \\
\textbf{A3.} Tómendegi sızıqlardan qaysı biri oraylıq (yaǵnıy birden-bir orayǵa iye), qaysı biri orayǵa iye emes, qaysı biri sheksiz kóp orayǵa iye ekenligin anıqlań:: $x^2-2 x y+4 y^2+5 x-7 y+12=0$; \\
\textbf{B1.} Berilgen teńleme parabolik ekenligin kórsetiń; ápiwayı túrge keltiriń; qanday geometriyalıq obrazdı anlatıwın anıqlań, eski hám de jańa koordinata kósherlerine salıstırģanda sızılmada súwretleń: $9 x^2+24 x y+16 y^2-18 x+226 y+209=0$; \\
\textbf{B2.} $\frac{x^2}{2}-\frac{z^2}{3}=y$ giperbolik paraboloidi hám $3x-3y+4z+2=0$ tegisliginiń kesilisiw sızıǵı qanday iymek sızıq ekenligin anıqlań hám onıń orayın tabıń. \\
\textbf{B3.} Giperbolanın yarım kósherlerin tabiń, eger: direktrisaları $x= \pm 3 \sqrt{2}$ tenlemeler menen berilgen hám asimptotaları arasındaģi múyesh tuwri múyesh; \\
\textbf{C1.} Giperbolanıń asimptotaların tabıń: $3 x^2+2 x y-y^2+8 x+10 y-14=0$; \\
\textbf{C2.} Elliptik túrdegi ($\delta>0$) teńleme $a_{11}$ hám $\Delta$ birdey belgige iye san bolǵanda ǵana jormal ellips teńlemesi bolatuģının dálilleń. \\
\textbf{C3.} Ellips orayınan onıń qálegen urınbasınıń fokal kósher menen kesilisiw noqatına shekemgi hám urınıw noqatınan fokal kósherge túsirilgen perpendikulyar ultanına shekemgi aralıqlar kóbeymesi turaqlı shama bolıp, ellips úlken yarım kósheriniń kvadratına teń ekenligin dálilleń. \\

\end{tabular}
\vspace{1cm}


\begin{tabular}{m{17cm}}
\textbf{94-variant}
\newline

\textbf{T1.} Ekinshi tártipli sızıq hám tuwrı sızıqtıń óz ara jaylasıwı (Kesilisiw noqatları, Urınba (urınıw) jaģdayı) \\
\textbf{T2.} Ekinshi tártipli betliklerdiń kanonikalıq teńlemeleri (Ellipsoid, Giperboloid (1 gewekli), Giperboloid (2 gewekli)) \\
\textbf{A1.} Tómendegi maglıwmatlar boyınsha giperbolanıń kanonikalıq teńlemesin dúziń: direktrisaları arasındaǵı aralıq $\frac{32}{5}$ ģa teń hám ekssentrisiteti $e=\frac{5}{4}$; \\
\textbf{A2.} $y+6=0$ tegislik $\frac{x^2}{5}-\frac{y^2}{4}=6 z$ giperbolik paraboloidti parabola boyınsha kesip ótetuģının kórsetiń; parametrin hám tóbesin tabıń. \\
\textbf{A3.} Koordinatalar sistemasın túrlendirmesten, tómendegi teńlemelerdiń hár biri parabolanı anıqlawın kórsetiń hám parametrin tabıń: $x^2-2 x y+y^2+6 x-14 y+29=0$; \\
\textbf{B1.} ITECH túri, ólshemleri hám jaylasıwın anıqlań: $x^2+2 x y+y^2-8 x+4=0$; \\
\textbf{B2.} Parallel kóshiriw hám burıw túrlendiriwleri yamasa aǵzalardı gruppalaw járdeminde tómendegi betliklerdiń kórinisi hám jaylasıwı anıqlansın: $2 x y+2 x+2 y+2 z-1=0$; \\
\textbf{B3.} $M_1 (2;-1) $ noqatı fokusı $F (1;0) $ bolǵan ellipste jatadı. Bul fokusqa sáykes direktrisa bolsa $2x-y-10=0$ teńleme menen berilgen. Usı ellipstiń teńlemesin dúziń. \\
\textbf{C1.} $y^2=2 p x$ parabolaǵa $y=k x+b$ tuwrı sızıq urınıw shártin keltirip shigarıń. \\
\textbf{C2.} $A x+B y+C=0$ tuwri sızıq $y^2=2 p x$ parabolaga urinıwı ushin zárúrli hám jeterli shártti tabıń. \\
\textbf{C3.} Eger giperbolanıń yarım kósherleri $a$ hám $b$, orayı $C\left(x_0; y_0\right) $ hám fokuslar tómendegi tuwrı sızıqta jaylasqan: 1) $O x$ kósherine parallel; 2) $O y$ kósherine parallel bolsa, onıń teńlemesin dúziń. \\

\end{tabular}
\vspace{1cm}


\begin{tabular}{m{17cm}}
\textbf{95-variant}
\newline

\textbf{T1.} Tegislikte ekinshi tártipli sızıqlar (Ekinshi tártipli teńleme, Kvadrat kórinisindegi teńleme, Konik sızıqlar (konuslar kesimi)) \\
\textbf{T2.} Parabola hám onıń kanonikalıq teńlemeleri (Fokus (bagdarlawshı noqat), Direktrisa (bagdarlawshı sızıq), Kósher (simmetriya kósheri)) \\
\textbf{A1.} Tómendegi sızıqlardan qaysı biri oraylıq (yaǵnıy birden-bir orayǵa iye), qaysı biri orayǵa iye emes, qaysı biri sheksiz kóp orayǵa iye ekenligin anıqlań:  $x^2-2 x y+y^2-6 x+6 y-3=0$; \\
\textbf{A2.} Parabolanıń teńlemesin dúziń, eger: parabolanıń fokusi $ (0,2) $ noqatında hám tóbesi koordinatalar basında jatadı; \\
\textbf{A3.} Fokusları abscissa kósherinde jatqan hám koordinatalar basına salıstırģanda simmetriyalı bolǵan ellipstiń teńlemesin dúziń, bunda: direktrisaları arasındaǵı aralıq 32 hám $\varepsilon=\frac{1}{2}$. \\
\textbf{B1.} Berilgen teńlemeniń tipin anıqlań, koordinata kósherlerin parallel kóshiriw arqalı ápiwayı túrge keltiriń; qanday geometriyalıq obrazdı ańlatıwın anıqlań, eski hám jańa koordinata kósherlerine salıstırģanda sızılmada súwretleń: $4 x^2-y^2+8 x-2 y+3=0$; \\
\textbf{B2.} $\frac{x^2}{16}-\frac{y^2}{64}=1$ giperbolaǵa $10 x-3 y+9=0$ tuwrısına parallel bolǵan urınbalardıń teńlemelerin dúziń. \\
\textbf{B3.} Berilgen teńlemeler oraylıq iymek sızıqlar ekenligin kórsetiń hám hárbir teńlemeni koordinata basın orayģa kóshiriń: $3x^2-6xy+2y^2-4x+2y+1=0$. \\
\textbf{C1.} $\frac{x^2}{a^2}+\frac{y^2}{b^2}=1$ ellipstiń $F(c, 0)$ fokusı arqalı úlken kósherine perpendikulyar bolǵan xorda ótkerilgen. Bul xordıń uzınlıǵın tabıń. \\
\textbf{C2.} Tómendegi betliklerdiń kanonikalıq teńlemesi hám jaylasıwın anıqlań.: $x^2+5 y^2+z^2+2 x y+6 x z+2 y z-2 x+6 y+2 z=0$. \\
\textbf{C3.} Giperbola asimptotalarınıń tenlemeleri $y= \pm \frac{1}{2} x$ hám urinbalardan biriniń teńlemesi. $5 x-6 y-8=0$ belgili bolsa, giperbola teńlemesin dúziń. \\

\end{tabular}
\vspace{1cm}


\begin{tabular}{m{17cm}}
\textbf{96-variant}
\newline

\textbf{T1.} Ekinshi tártipli betliklerdiń ulıwmalıq teńlemesin kanonikalıq túrge keltiriw (invariantlar járdeminde) \\
\textbf{T2.} Ekinshi tártipli sızıqqa urınba, túyinles diametri teńlemesi (Urınba teńlemesi, Túyinles diametr: oraydan ótiwshi simmetriya kósherleri) \\
\textbf{A1.} $\frac{x^2}{25}-\frac{y^2}{144}=1$ giperbolanıń fokusların tabıń. \\
\textbf{A2.} Diskriminantın esaplaw arqalı tómendegi teńlemelerdiń hár biriniń tipin anıqlań: $25 x^2-20 x y+4 y^2-12 x+20 y-17=0$; \\
\textbf{A3.} Fokusları abscissa kósherinde jatqan hám koordinatalar basına qarata simmetriyalı bolǵan ellipstiń teńlemesin dúziń, bunda: $M_1 (4;-\sqrt{3}) $ hám $M_2 (2 \sqrt{2}; 3)$ noqatları ellipske tiyisli; \\
\textbf{B1.} Berilgen teńlemelerdiń parabolik ekenligin kórsetiń hám olardıń hár birin $(\alpha x+\beta y)^2+2 a_{13} x+2 a_{23} y+a_{33}=0$ kórinisinde jazıń:  $9 x^2-42 x y+49 y^2+3 x-2 y-24=0$. \\
\textbf{B2.} Parabola tóbesiniń koordinataların, parametrin hám kósheriniń baǵıtın anıqlań: $y=A x^2+B x+C$, \\
\textbf{B3.} Berilgen sızıqlar oraylıq ekenligin kórsetiń hám hárbir iymek sızıq ushın orayınıń koordinataların tabıń: $2 x^2-6 x y+5 y^2+22 x-36 y+11=0$. \\
\textbf{C1.} $m$ niń qanday mánislerinde $x+mz-1=0$ tegislik tómendegi $x^2+y^2z^2=1$ eki gewekli giperboloidti a) ellips boyınsha, b) giperbola boyınsha kesedi? \\
\textbf{C2.} $m$ hám $n$ tiń qanday mánislerinde $m x^2+12 x y+9 y^2+4 x+n y-13=0$ teńleme: 1) oraylıq sızıqtı; 2) orayga iye bolmaǵan sızıq; 3) sheksiz kóp orayǵa iye bolǵan sızıqtı ańlatadı. \\
\textbf{C3.} Eger ekinshi dárejeli teńleme parabolik bolıp, $ (\alpha x+\beta y) ^2+2a_{13}x+2a_{23}y+a_{33}=0$ kórinisinde jazılsa, onıń shep tárepindegi diskriminant $\Delta=- (a_{13} \beta-a_{23} \alpha) ^2$ formula menen anıqlanıwın dálilleń. \\

\end{tabular}
\vspace{1cm}


\begin{tabular}{m{17cm}}
\textbf{97-variant}
\newline

\textbf{T1.} Bir gewekli giperboloid hám giperbolik paraboloidtıń tuwrı sızıqlı jasawshıları (Giperboloid, Giperbolik paraboloid, Sızıqlı jasawshılar) \\
\textbf{T2.} Tegislikte ekinshi tártipli sızıqlar (Ekinshi tártipli teńleme, Kvadrat kórinisindegi teńleme, Konik sızıqlar (konuslar kesimi)) \\
\textbf{A1.} Tómendegi sızıqlardan qaysı biri oraylıq (yaǵnıy birden-bir orayǵa iye), qaysı biri orayǵa iye emes, qaysı biri sheksiz kóp orayǵa iye ekenligin anıqlań: $4 x^2-4 x y+y^2-6 x+8 y+13=0$; \\
\textbf{A2.} Tómendegi maglıwmatlar boyınsha giperbolanıń kanonikalıq teńlemesin dúziń: fokusları arasındaǵı aralıq 10 ģa, úlken kósheri bolsa 8 ge teń. \\
\textbf{A3.} Koordinatalar sistemasın túrlendirmesten tómendegi teńlemelerdiń hár biri kesilisiwshi eki tuwrını anıqlawın kórsetiń hám onıń koordinataların tabıń: $x^2-6 x y+8 y^2-4 y-4=0$; \\
\textbf{B1.} Teń tárepli giperbolanıń ekssentrisiteti anıqlansın. \\
\textbf{B2.} Berilgen teńleme parabolik ekenligin kórsetiń; ápiwayı túrge keltiriń; qanday geometriyalıq obrazdı anlatıwın anıqlań, eski hám de jańa koordinata kósherlerine salıstırģanda sızılmada súwretleń:$9 x^2-24 x y+16 y^2-20 x+110 y-50=0$; \\
\textbf{B3.} Berilgen teńlemeniń tipin anıqlań, koordinata kósherlerin parallel kóshiriw arqalı ápiwayı túrge keltiriń; qanday geometriyalıq obrazdı ańlatıwın anıqlań, eski hám jańa koordinata kósherlerine salıstırģanda sızılmada súwretleń: $2 x^2+3 y^2+8 x-6 y+11=0$. \\
\textbf{C1.} Ulıwma kósherge hám tóbeleri arasında jaylasqan ulıwma fokusqa iye bolǵan eki parabola tuwrı múyesh astında kesilisetuģının dálilleń. \\
\textbf{C2.} Hár qanday parabolik teńleme ushın $a_{11}$ hám $a_{22}$ koefficientler hár qıylı belgige iye sanlar bola almaytuģının hám olar bir waqıtta nolge aylana almaytuģının dálilleń. \\
\textbf{C3.} $\frac{x^2}{9}+\frac{z^2}{4}=2 y$ elliptik paraboloid $2 x-2 y-z-10=0$ tegislik penen bir ulıwma noqatqa iye ekenligin dálilleń hám onıń koordinataların tabıń. \\

\end{tabular}
\vspace{1cm}


\begin{tabular}{m{17cm}}
\textbf{98-variant}
\newline

\textbf{T1.} Ekinshi tártipli sızıqlardıń ulıwma teńlemesin invariantlar járdeminde kanonikalıq túrge keltiriw \\
\textbf{T2.} Parabola hám onıń kanonikalıq teńlemeleri (Fokus (bagdarlawshı noqat), Direktrisa (bagdarlawshı sızıq), Kósher (simmetriya kósheri)) \\
\textbf{A1.} Koordinatalar sistemasın túrlendirmesten, tómendegi teńlemelerdiń hár biri parabolanı anıqlawın kórsetiń hám parametrin tabıń: $9 x^2-6 x y+y^2-50 x+50 y-275=0$. \\
\textbf{A2.} Parabolanıń teńlemesin dúziń, eger: fokusı $ (5,0) $ noqatta bolıp, ordinatalar kósheri direktrisa bolsa; \\
\textbf{A3.} $z+1=0$ tegislik bir qabatlı $\frac{x^2}{32}-\frac{y^2}{18}+\frac{z^2}{2}=1$ giperboloidti giperbola boyınsha kesip ótetuģının kórsetiń; onıń yarım kósherleri hám tóbelerin tabıń. \\
\textbf{B1.} Parabola tóbesiniń koordinataların, parametrin hám kósheriniń baǵıtın anıqlań: $y^2-6 x+14 y+49=0$, \\
\textbf{B2.} Ellipstegi ekssentrisitetti anıqlań, eger: fokusları arasındaǵı kesindiniń ózi kishi kósherdiń tóbesinen tuwrı múyesh astında kórinse; \\
\textbf{B3.} Lagranj usılınan paydalanıp, teńlemelerdi kvadratlar qosındısı túrine keltirip, tómendegi betlerdiń kórinisin anıqlań: $4 x y+2 x+4 y-6 z-3=0$; \\
\textbf{C1.} Berilgen teńleme kanonikalıq kóriniske keltirilsin; tipi anıqlansın; qanday geometriyalıq obrazdı anlatıwı anıqlansın; eski hám jana koordinatalar sistemasında geometriyalıq obrazı súwretlensin: $11 x^2-20 x y-4 y^2-20 x-8 y+1=0$; \\
\textbf{C2.} $\frac{x^2}{30}+\frac{y^2}{24}=1$ ellipske $4x-2y+23=0$ parallel bolǵan urınbalardı júrgiziń hám olar arasındaģı aralıqtı esaplań. \\
\textbf{C3.} $A x+B y+C=0$ tuwrı sızıq qanday zárúrli hám jeterli shárt orınlanǵanda $\frac{x^2}{a^2}+\frac{y^2}{b^2}=1$ ellips penen 1) kesilisedi; 2) kesilispeydi. \\

\end{tabular}
\vspace{1cm}


\begin{tabular}{m{17cm}}
\textbf{99-variant}
\newline

\textbf{T1.} Ekinshi tártipli sızıqlardıń ulıwma teńlemeleri (Ulıwma teńleme) \\
\textbf{T2.} Ekinshi tártipli betliklerdiń kanonikalıq teńlemeleri (Paraboloid (ellipstik), Paraboloid (giperbolik), Konus, Cilindr) \\
\textbf{A1.} $y^2+z^2=x$ elliptik paraboloidtıń $x+2 y-z=0$ tegislik penen kesilisiwiniń koordinata tegisliklerindegi proekciyalarınıń teńlemelerin tabıń. \\
\textbf{A2.} $\frac{x^2}{100}+\frac{y^2}{225}=1$ ellips hám $y^2=24 x$ parabolanıń kesilisiw noqatların anıqlań. \\
\textbf{A3.} Koordinatalar sistemasın túrlendirmesten, tómendegi teńlemelerdiń hár biri parabolanı anıqlawın kórsetiń hám parametrin tabıń: $9 x^2+24 x y+16 y^2-120 x+90 y=0$; \\
\textbf{B1.} Berilgen teńlemelerdiń parabolik ekenligin kórsetiń hám olardıń hár birin $(\alpha x+\beta y)^2+2 a_{13} x+2 a_{23} y+a_{33}=0$ kórinisinde jazıń: $x^2+4 x y+4 y^2+4 x+y-15=0 ;$ \\
\textbf{B2.} Parallel kóshiriw hám burıw túrlendiriwleri yamasa aǵzalardı gruppalaw járdeminde tómendegi betliklerdiń kórinisi hám jaylasıwı anıqlansın: $2 x y+z^2-2 z+1=0$; \\
\textbf{B3.} Berilgen teńlemeler oraylıq iymek sızıqlar ekenligin kórsetiń hám hárbir teńlemeni koordinata basın orayģa kóshiriń: $6 x^2+4 x y+y^2+4 x-2 y+2=0$; \\
\textbf{C1.} Tómendegi betliklerdiń kanonikalıq teńlemesi hám jaylasıwın anıqlań.: $x^2-2 y^2+z^2+4 x y-10 x z+4 y z+2 x+4 y-10 z-1=0$. \\
\textbf{C2.} Giperbolanıń asimptotaların tabıń: $x^2-3 x y-10 y^2+6 x-8 y=0$; \\
\textbf{C3.} Giperbolanıń bir diametr tóbelerinen ótkerilgen urınbalar parallel bolıwın dálilleń. \\

\end{tabular}
\vspace{1cm}


\begin{tabular}{m{17cm}}
\textbf{100-variant}
\newline

\textbf{T1.} Ekinshi tártipli sızıqlardıń ulıwma teńlemesin invariantlar járdeminde kanonikalıq túrge keltiriw \\
\textbf{T2.} Tegislikte ekinshi tártipli sızıqlar (Ekinshi tártipli teńleme, Kvadrat kórinisindegi teńleme, Konik sızıqlar (konuslar kesimi)) \\
\textbf{A1.} Fokusları abscissa kósherinde jatqan hám koordinatalar basına qarata simmetriyalı bolǵan ellipstiń teńlemesin dúziń, bunda: $M_1 (\sqrt{15};-1) $ noqatı ellipske tiyisli hám fokusları arasındaǵı aralıq $2 c=8$; \\
\textbf{A2.} Koordinatalar sistemasın túrlendirmesten tómendegi teńlemelerdiń hár biri giperbolanı anıqlawın kórsetiń hám onıń koordinataların tabıń: $x^2-6 x y-7 y^2+10 x-30 y+23=0$. \\
\textbf{A3.} Fokusları abscissa kósherinde, koordinatalar basına salıstırģanda simmetriyalı jaylasqan giperbolanıń teńlemesin dúziń, bunda: $M_1 (6;-1) $ hám $M_2 (-8; 2 \sqrt{2}) noqatlar $ giperbolaga tiyisli; \\
\textbf{B1.} Tómendegilerdi bilgen halda ellips teńlemesin dúziń: onıń úlken kósheri 26 ģa teń hám fokusları $F_1 (-10; 0), F_2 (14; 0) $; \\
\textbf{B2.} $M_1 (1;-2) $ noqat fokusı $F (-2; 2) $, oǵan sáykes direktrisa bolsa $2x-y-1=0$ teńleme menen berilgen giperbolaǵa tiyisli. Bul giperbolanıń teńlemesin dúziń. \\
\textbf{B3.} Parabola tóbesiniń koordinataların, parametrin hám kósheriniń baǵıtın anıqlań: $x^2-6 x-4 y+29=0$, \\
\textbf{C1.} Ulıwma fokusqa hám ústpe-úst túsken, biraq qarama-qarsı baǵıtlangan kósherlerge iye bolǵan parabolalardıń tuwrı múyesh astında kesilisiwin dálilleń. \\
\textbf{C2.} $m$ hám $n$ tiń qanday mánislerinde $m x^2+12 x y+9 y^2+4 x+n y-13=0$ teńleme: 1) oraylıq sızıqtı; 2) orayga iye bolmaǵan sızıq; 3) sheksiz kóp orayǵa iye bolǵan sızıqtı ańlatadı. \\
\textbf{C3.} $\frac{x^2}{a^2}-\frac{y^2}{b^2}=1$ giperbolanıń fokuslarınan urınbasına shekemgi aralıqlardıń kóbeymesin tabıń. \\

\end{tabular}
\vspace{1cm}



\end{document}
