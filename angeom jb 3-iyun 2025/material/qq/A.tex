Fokusları abscissa kósherinde jatqan hám koordinatalar basına salıstırǵanda simmetriyalı, yarım kósherleri 5 hám 2 bolǵan ellipstiń teńlemesin dúziń;
Fokusları abscissa kósherinde jatqan hám koordinatalar basına salıstırģanda simmetriyalı bolǵan ellipstiń teńlemesin dúziń, bunda onıń úlken kósheri 10 ģa, fokusları arasındaǵı aralıq bolsa $2c = 8$ ge teń;
Fokusları abscissa kósherinde jatqan hám koordinatalar basına salıstırģanda simmetriyalı bolǵan ellipstiń teńlemesin dúziń, bunda onıń kishi kósheri 24 ke, fokusları arasındaǵı aralıq bolsa $c = 10$ ga teń;
Fokusları abscissa kósherinde jatqan hám koordinatalar basına salıstırģanda simmetriyalı bolǵan ellipstiń teńlemesin dúziń, bunda: fokusları arasındaǵı aralıq $2 c=6$ hám ekssentrisiteti $\varepsilon=\frac{3}{5}$;
Fokusları abscissa kósherinde jatqan hám koordinatalar basına salıstırganda simmetriyalı bolǵan ellipstiń teńlemesin dúziń, bunda: kishi kósheri 10, ekscentrisiteti $\varepsilon=\frac{12}{13}$;
Fokusları abscissa kósherinde jatqan hám koordinatalar basına salıstırģanda simmetriyalı bolǵan ellipstiń teńlemesin dúziń, bunda: direktrisaları arasındaǵı aralıq 5 hám fokusları arasındaǵı aralıq $2c=4$;
Fokusları abscissa kósherinde jatqan hám koordinatalar basına salıstırģanda simmetriyalı bolǵan ellipstiń teńlemesin dúziń, bunda: úlken kósheri 8, direktrisaları arasındaǵı aralıq 16;
Fokusları abscissa kósherinde jatqan hám koordinatalar basına salıstırģanda simmetriyalı bolǵan ellipstiń teńlemesin dúziń, bunda: kishi kósheri 6, direktrisaları arasındaǵı aralıq 13;
Fokusları abscissa kósherinde jatqan hám koordinatalar basına salıstırģanda simmetriyalı bolǵan ellipstiń teńlemesin dúziń, bunda: direktrisaları arasındaǵı aralıq 32 hám $\varepsilon=\frac{1}{2}$.
Fokusları ordinata kósherinde jatqan hám koordinatalar basına salıstırģanda simmetriyalı bolǵan ellipstiń teńlemesin dúziń, bunda: yarım kósherleri 7 hám 2;
Fokusları ordinata kósherinde jatqan hám koordinatalar basına salıstırģanda simmetriyalı bolǵan ellipstiń teńlemesin dúziń, bunda: úlken yarım kósheri 10, fokusları arasındaǵı aralıq $2 c=8$;
Fokusları ordinata kósherinde jatqan hám koordinatalar basına salıstırģanda simmetriyalı bolǵan ellipstiń teńlemesin dúziń, bunda: fokusları arasındaǵı aralıq $2 c=24$, ekssentrisiteti $\varepsilon=\frac{12}{13}$;
Fokusları abscissa kósherinde jatqan hám koordinatalar basına salıstırģanda simmetriyalı bolǵan ellipstiń teńlemesin dúziń, bunda: direktrisaları arasındaǵı aralıq 32 hám $\varepsilon=\frac{1}{2}$.
Fokusları ordinata kósherinde jatqan hám koordinatalar basına qarata simmetriyalı bolǵan ellipstiń teńlemesin dúziń, bunda: kishi kósheri 16, a ekssentrisiteti $\varepsilon=\frac{3}{5}$;
Fokusları ordinata kósherinde jatqan hám koordinatalar basına qarata simmetriyalı bolǵan ellipstiń teńlemesin dúziń, bunda: fokusları arasındaǵı aralıq $2 c=6$, direktrisaları arasındaǵı aralıq $16 \frac{2}{3}$;
Fokusları ordinata kósherinde jatqan hám koordinatalar basına salıstırģanda simmetriyalı bolǵan ellipstiń teńlemesin dúziń, bunda: direktrisaları arasındaģı qashıqlıq $frac{2}{3}$ hám ekssentrisiteti $frac{3}{4}$.
Eki tóbesi $9 x^2+5 y^2=1$ ellipstiń fokuslarında, qalǵan eki tóbesi onıń kishi kósheriniń tóbelerinde jaylasqan tórtmúyeshliktiń maydanın esaplań.
$\frac{x^2}{100}+\frac{y^2}{36}=1$ ellipsinde jaylasqan hám oń fokusına shekemgi aralıǵı 14 ke teń noqattı tabıń.
$\frac{x^2}{16}+\frac{y^2}{7}=1$, ellipsinde jaylasqan hám shep fokusına shekemgi aralıǵı 2,5 ge teń noqattı tabıń.
Fokusları abscissa kósherinde jatqan hám koordinatalar basına qarata simmetriyalı bolǵan ellipstiń teńlemesin dúziń, bunda: $M_1 (-2 \sqrt{5}; 2) $ noqatı ellipske tiyisli hám kishi yarım kósheri $b=3$;
Fokusları abscissa kósherinde jatqan hám koordinatalar basına qarata simmetriyalı bolǵan ellipstiń teńlemesi dúzilsin, bunda: $M_1 (2;-2) $ noqatı ellipske tiyisli hám úlken yarım kósheri $a=4$;
Fokusları abscissa kósherinde jatqan hám koordinatalar basına qarata simmetriyalı bolǵan ellipstiń teńlemesin dúziń, bunda: $M_1 (4;-\sqrt{3}) $ hám $M_2 (2 \sqrt{2}; 3)$ noqatları ellipske tiyisli;
Fokusları abscissa kósherinde jatqan hám koordinatalar basına qarata simmetriyalı bolǵan ellipstiń teńlemesin dúziń, bunda: $M_1 (\sqrt{15};-1) $ noqatı ellipske tiyisli hám fokusları arasındaǵı aralıq $2 c=8$;
Fokusları abscissa kósherinde jatqan hám koordinatalar basına qarata simmetriyalı bolǵan ellipstiń teńlemesi dúzilsin, bunda: $M_1 \left(2;-\frac{5}{3}\right) $ noqatı ellipske tiyisli hám ekscentrisiteti $\varepsilon=\frac{2}{3}$;
Fokusları abscissa kósherinde jatqan hám koordinatalar basına qarata simmetriyalı bolǵan ellipstiń teńlemesin dúziń, bunda: $M_1 (8; 12) $ ellipske tiyisli hám bul noqattan shep fokusına shekemgi aralıq $r_1=20$ qa teń;
Fokusları abscissa kósherinde jatqan hám koordinatalar basına qarata simmetriyalı bolǵan ellipstiń teńlemesin dúziń, bunda: $M_1 (-\sqrt{5}; 2)$ noqatı ellipske tiyisli hám onıń direktrisaları arasındaǵı aralıq 10 ģa teń.
$C (-3; 2)$ eki koordinata kósherine urınıwshı ellipstiń orayı. Bul ellipstiń simmetriya kósherleri koordinata kósherlerine parallel ekenligin bilgen halda onıń teńlemesin dúziń.
Ekscentrisiteti $\varepsilon=\frac{2}{3}$, fokusı $F (2; 1) $ hám usı fokus tárepindegi direktrisası $x-5=0$ bolǵan ellipstiń teńlemesin dúziń.
Ekscentrisiteti $\varepsilon=\frac{1}{2}$, fokusı $F (-4; 1) $ hám usı fokus tárepindegi direktrisası $y+3=0$ bolǵan ellipstiń teńlemesin dúziń.
$\frac{x^2}{36}+\frac{y^2}{20}=1$ ellips direktrisalarınıń teńlemelerin jazıń.
$\frac{x^2}{32}+\frac{y^2}{18}=1$ ellipsiniń $M (4,3) $ noqatında júrgizilgen urinbasınıń teńlemesin dúziń.
++++
$A (-3;-5) $ noqatı fokusı $F (-1;-4) $ bolǵan ellipste jatadı hám oǵan sáykes direktrisa $x-2=0$ teńlemesi menen berilgen. Usi ellipstiń teńlemesin dúziń.
Fokusları abscissa kósherinde jaylasqan, koordinatalar basına salıstırģanda simmetriyalı bolǵan giperbolanıń teńlemesin dúziń, bunda: onıń kósherleri $2 a=10$ hám $2 b=8$;
Fokusları abscissa kósherinde jaylasqan, koordinatalar basına qarata simmetriyalı bolǵan giperbolanıń teńlemesin dúziń, bunda: fokusları arasındaǵı aralıq $2 c=10$ hám kishi kósheri $2 b=8$;
Fokusları abscissa kósherinde jaylasqan, koordinatalar basına qarata simmetriyalı bolǵan giperbolanıń teńlemesin dúziń, bunda: fokusları arasındaǵı aralıq $2 c=6$ hám ekscentrisiteti $\varepsilon=\frac{3}{2}$;
Fokusları abscissa kósherinde jaylasqan, koordinatalar basına qarata simmetriyalı bolǵan giperbolanıń teńlemesin dúziń, bunda: $2 a=16$ hám ekssentrisiteti $\varepsilon=\frac{5}{4}$;
Fokusları abscissa kósherinde jaylasqan, koordinatalar basına salıstırģanda simmetriyalı bolǵan giperbolanıń teńlemesin dúziń, bunda: asimtotalarınıń teńlemesi $y= \pm \frac{4}{3} x$ hám fokusları arasındaǵı aralıq $2 c=20$;
Fokusları abscissa kósherinde jaylasqan, koordinatalar basına salıstırģanda simmetriyalı bolǵan giperbolanıń teńlemesin dúziń, bunda: direktrisalarınıń arasındaģı aralıq $22 \frac{2}{13}$ hám fokusları arasındaǵı aralıq $2 c=26$;
Fokusları abscissa kósherinde jaylasqan, koordinatalar basına salıstırģanda simmetriyalı bolǵan giperbolanıń teńlemesin dúziń, bunda: direktrisalarınıń arasındaģı aralıq $\frac{32}{5}$ ham kishi kósheri $2 b=6$;
Fokusları abscissa kósherinde jaylasqan, koordinatalar basına salıstırģanda simmetriyalı bolǵan giperbolanıń teńlemesin dúziń, bunda: direktrisalarınıń arasındaģı aralıq $\frac{8}{3}$ hám ekssentrisiteti $\varepsilon=\frac{3}{2}$;
Fokusları abscissa kósherinde jaylasqan, koordinatalar basına salıstırģanda simmetriyalı bolǵan giperbolanıń teńlemesin dúziń, bunda: asimtotalarınıń teńlemesi $y= \pm \frac{3}{4} x$ hám direktrisalarınıń arasındaģı aralıq $12 \frac{4}{5}$.
Fokusları ordinata kósherinde, koordinatalar basına salıstırǵanda simmetriyalı jaylasqan giperbolanıń teńlemesi dúzilsin, bunda: onıń yarım kósherleri. $a=6, b=18$;
Fokusları ordinata kósherinde jaylasqan, koordinatalar basına salıstırģanda simmetriyalı bolǵan giperbolanıń teńlemesin dúziń, bunda: fokusları arasındaģı aralıq $2 c=10$ hám ekssentrisiteti $\varepsilon=\frac{5}{3}$;
Fokusları ordinata kósherinde jaylasqan, koordinatalar basına salıstırģanda simmetriyalı bolǵan giperbolanıń teńlemesin dúziń, bunda: asimtotalarınıń teńlemesi $y= \pm \frac{12}{5} x$ hám ushlarınıń arasındaģı aralıq 48;
Fokusları ordinata kósherinde jaylasqan, koordinatalar basına salıstırģanda simmetriyalı bolǵan giperbolanıń teńlemesin dúziń, bunda: asimtotalarınıń teńlemesi $7 \frac{1}{7}$ hám ekssentrisiteti $\varepsilon=\frac{7}{5}$;
Fokusları ordinata kósherinde jaylasqan, koordinatalar basına salıstırģanda simmetriyalı bolǵan giperbolanıń teńlemesin dúziń, bunda: asimtotalarınıń teńlemesi $y= \pm \frac{4}{3} x$ ham direktrisalarınıń arasındaģı aralıq $6 \frac{2}{5}$.
$16 x^2-9 y^2=144$ giperbola berilgen. Tabıń: 1) yarım kósherlerin; 2) fokusların; 3) ekssentrisitetin; 4) asimtotalarınıń teńlemesi; 5) direktrisaları tenlemelerin.
$\frac{x^2}{4}-\frac{y^2}{9}=1$ giperbolanıń asimptotalarınan hám $9 x+2 y-24=0$ tuwrı sızıqtan payda bolǵan úshmúyeshlik maydanın esaplań.
Berilgen teńleme qanday iymek sızıq ekenligin tabıń: $y=+\frac{2}{3} \sqrt{x^2-9}$
Berilgen teńleme qanday iymek sızıq ekenligin tabıń: $y=-3 \sqrt{x^2+1}$;
Berilgen teńleme qanday iymek sızıq ekenligin tabıń: $x=-\frac{4}{3} \sqrt{y^2+9} ;$
Berilgen teńleme qanday iymek sızıq ekenligin tabıń: $y=+\frac{2}{5} \sqrt{x^2+25}$
Giperbolanıń ekssentrisiteti $\varepsilon=2$ ǵa teń, $M$ noqatınıń bazı bir fokal radiusı 16 ǵa teń. $M$ noqattan sáykes direktrisaģa shekem bolǵan aralıqtı tabıń.
Giperbolanıń ekssentrisiteti $varepsilon=3$, $M$ noqatınıń bazı bir fokal radiusı 4 ke teń. $M$ noqattan sáykes direktrisaģa shekem bolǵan aralıqtı tabıń.
Fokusları abscissa kósherinde, koordinatalar basına salıstırģanda simmetriyalı jaylasqan giperbolanıń teńlemesin dúziń, bunda: $M_1 (6;-1) $ hám $M_2 (-8; 2 \sqrt{2}) noqatlar $ giperbolaga tiyisli;
Fokusları abscissa kósherinde, koordinatalar basına qarata simmetriyalı jaylasqan giperbolanıń teńlemesin dúziń, bunda: $M_1 (-5; 3)$ noqat giperbolaģa tiyisli hám ekssentrisiteti $\varepsilon=\sqrt{2}$;
Fokusları abscissa kósherinde, koordinatalar basına qarata simmetriyalı jaylasqan giperbolanıń teńlemesi dúzilsin, bunda: $M_1\left(\frac{9}{2};-1\right) $ noqatı giperbolaga tiyisli hám asimtota teńlemeleri $y= \pm \frac{2}{3} x$;
Fokusları abscissa kósherinde, koordinatalar basına qarata simmetriyalı jaylasqan giperbolanıń teńlemesin dúziń, bunda: $M_1\left(-3; \frac{5}{2}\right)$ noqatı giperbolaģa tiyisli hám direktrisalarınıń teńlemesi $x= \pm \frac{4}{3}$;
Fokusları abscissa kósherinde, koordinatalar basına qarata simmetriyalı jaylasqan giperbolanıń teńlemesi dúzilsin, bunda: asimptotanıń teńlemeleri $y= \pm \frac{3}{4} x$ hám direktrisalarınıń teńlemeleri $x= \pm \frac{16}{5}$.
Ekscentrisiteti $\varepsilon=\frac{5}{4}$, bir fokusı $F (5; 0) $ hám oǵan sáykes direktrisasınıń teńlemesi $5x-16=0$ bolǵan giperbolanıń teńlemesin dúziń.
Ekscentrisiteti $\varepsilon=\frac{13}{12}$, bir fokusi $F (0; 13) $ hám ogan sáykes direktrisasınıń teńlemesi $13 y-144=0$ bolǵ an giperbolanıń teńlemesin dúziń.
Tómendegi maglıwmatlar boyınsha giperbolanıń kanonikalıq teńlemesin dúziń: úlken kósheri $a=5$, kishi kósheri $b=3$;
Tómendegi maglıwmatlar boyınsha giperbolanıń kanonikalıq teńlemesin dúziń: fokusları arasındaǵı aralıq 10 ģa, úlken kósheri bolsa 8 ge teń.
Tómendegi maglıwmatlar boyınsha giperbolanıń kanonikalıq teńlemesin dúziń: ekssentrisiteti $e=\frac{12}{13}$ úlken kósheri 48 ge teń.
Tómendegi maglıwmatlar boyınsha giperbolanıń kanonikalıq teńlemesin dúziń: haqıyqıy kósheri 16 ǵa, asimptotası menen abscissa kósheri arasındaǵı múyesh $\varphi$.
Teń tárepli giperbolanıń ekssentrisiteti esaplansın.
Giperbola asimptotalarınıń teńlemeleri $y= \pm \frac{5}{12} x$ hám giperbolada jatıwshı $M (24,5) $ noqatı berilgen. Giperbola teńlemesin dúziń.
$\frac{x^2}{25}-\frac{y^2}{144}=1$ giperbolanıń fokusların tabıń.
$\frac{x^2}{225}-\frac{y^2}{64}=-1$ giperbolanıń fokusların tabıń.
Tómendegi maglıwmatlar boyınsha giperbolanıń kanonikalıq teńlemesin dúziń: direktrisaları arasındaǵı aralıq $\frac{32}{5}$ ģa teń hám ekssentrisiteti $e=\frac{5}{4}$;
Tómendegi maglıwmatlar boyınsha giperbolanıń kanonikalıq teńlemesin dúziń: asimptotaları arasındaǵı múyesh $60^{\circ}$ qa teń hám $c=2 \sqrt{3}$.
++++
Tóbesi koordinatalar basında bolǵan parabolanıń teńlemesin dúziń, bunda: parabola oń yarım tegislikte hám $Ox$ kósherine simmetriyalı jaylasqan, hám parametri $p=3$;
Tóbesi koordinatalar basında bolǵan parabolanıń teńlemesin dúziń, bunda: parabola oń yarım tegislikte hám $Ox$ kósherine simmetriyalı jaylasqan, hám parametri $p=0,5$;
Tóbesi koordinatalar basında bolǵan parabolanıń teńlemesin dúziń, bunda: parabola oń yarım tegislikte hám $Oy$ kósherine simmetriyalı jaylasqan, hám parametri $p=\frac{1}{4}$;
Tóbesi koordinatalar basında bolǵan parabolanıń teńlemesin dúziń, bunda: parabola oń yarım tegislikte hám $Oy$ kósherine simmetriyalı jaylasqan, hám parametri $p=3$.
Tóbesi koordinatalar basında bolǵan parabolanıń teńlemesin dúziń, bunda: parabola $Ox$ kósherine simmetriyalı jaylasqan hám $A (9; 6) $ noqatınan ótedi;
Tóbesi koordinatalar basında bolǵan parabolanıń teńlemesin dúziń, bunda: parabola $Ox$ kósherine simmetriyalı jaylasqan hám $B (-1; 2) $ noqatınan ótedi;
Tóbesi koordinatalar basında bolǵan parabolanıń teńlemesin dúziń, bunda: parabola $Oy$ kósherine simmetriyalı jaylasqan hám $C (1; 1) $ noqatınan ótedi;
Tóbesi koordinatalar basında bolǵan parabolanıń teńlemesin dúziń, bunda: parabola $Oy$ kósherine simmetriyalı jaylasqan hám $D (4; -8) $ noqatınan ótedi;
Polat tros eki ushınan ildirilgen; bekkemlew noqatları birdey biyiklikte jaylasqan; olar arasındaǵı aralıq 20 m ge teń. Onıń bekkemleniw noqatınan 2 m aralıqtaǵı iyiliw shaması, gorizontal boyınsha esaplaǵanda, 14,4 sm ge teń. Trosttı shama menen parabola doǵası formasında dep esaplap, bekkemlew noqatları arasındaǵı bul trostıń iyiliw shamasın anıqlańız.
$y^2=24 x$ parabolanıń $F$ fokusın hám direktrisasınıń teńlemesin tabıń.
$M$ noqatı $y^2=20 x$ parabolaga tiyisli, eger onıń abscissası 7 ge teń bolsa fokal radiusların tabıń.
Parabolanıń tóbesi ($\alpha;\beta$) noqat penen ústpe-úst túsetuģının bilgen halda onıń teńlemesin dúziń. Parametri $p$ ǵa teń. Onıń kósheri $O x$ kósherine parallel bolıp, $O x$ kósheriniń oń baǵıtında sheksizlikke sozilgan;
Parabolanıń tóbesi ($\alpha;\beta$) noqat penen ústpe-úst túsetuģının bilgen halda onıń teńlemesin dúziń. Parametri $p$ ǵa teń. Onıń kósheri $O x$ kósherine parallel bolıp, $O x$ kósheriniń teris baǵıtında sheksizlikke sozilgan;.
Parabolanıń tóbesi ($\alpha;\beta$) noqat penen ústpe-úst túsetuģının bilgen halda onıń teńlemesin dúziń. Parametri $p$ ǵa teń. Onıń kósheri $O y$ kósherine parallel bolıp, $O y$ kósheriniń oń baǵıtında sheksizlikke sozilgan;
Parabolanıń tóbesi ($\alpha;\beta$) noqat penen ústpe-úst túsetuģının bilgen halda onıń teńlemesin dúziń. Parametri $p$ ǵa teń. Onıń kósheri $O y$ kósherine parallel bolıp, $O y$ kósheriniń teris baǵıtında sheksizlikke sozilgan;
$x+y-3=0$ tuwrı sızıģi hám $x^2=4 y$ parabolasınıń kesilisiw noqatın tabıń.
$y^2=36 x$ parabolanıń $A (2; 9) $ noqatındaǵı urınbasınıń teńlemesin dúziń.
$\frac{x^2}{100}+\frac{y^2}{225}=1$ ellips hám $y^2=24 x$ parabolanıń kesilisiw noqatların anıqlań.
$\frac{x^2}{20}-\frac{y^2}{5}=-1$ giperbola hám $y^2=3 x$ parabolanıń kesilisiw noqatların anıqlań.
$y^2=4 x$ parabola fokusınıń koordinataların anıqlań.
$x^2=4 y$ parabola fokusınıń koordinataların anıqlań.
$y^2=-8 x$ parabola fokusınıń koordinataların anıqlań.
$y^2=6 x$ parabola direktrisası teńlemesin dúziń.
Parabolanıń teńlemesin dúziń, eger: parabolanıń tóbesinen fokusına shekemgi aralıq 3 ke teń hám parabola $O x$ kósherine qarata simmetriyalı bolıp, $O y$ kósherine urınsa;
Parabolanıń teńlemesin dúziń, eger: fokusı $ (5,0) $ noqatta bolıp, ordinatalar kósheri direktrisa bolsa;
Parabolanıń teńlemesin dúziń, eger: parabola $O x$ kósherine qarata simmetriyalı bolıp, $M (1;-4) $ noqatınan hám koordinatalar basınan ótedi;
Parabolanıń teńlemesin dúziń, eger: parabolanıń fokusi $ (0,2) $ noqatında hám tóbesi koordinatalar basında jatadı;
Parabolanıń teńlemesin dúziń, eger: parabola $O y$ kósherine qarata simmetriyalı bolıp, $M (6,-2) $ noqatınan hám koordinatalar basınan ótedi.
$y^2=8 x$ paraboladaǵı fokal radius vektorı 20 ga teń bolgan noqat tabılsın.
++++
Tómendegi sızıqlardan qaysı biri oraylıq (yaǵnıy birden-bir orayǵa iye), qaysı biri orayǵa iye emes, qaysı biri sheksiz kóp orayǵa iye ekenligin anıqlań: $3 x^2-4 x y-2 y^2+3 x-12 y-7=0$;
Tómendegi sızıqlardan qaysı biri oraylıq (yaǵnıy birden-bir orayǵa iye), qaysı biri orayǵa iye emes, qaysı biri sheksiz kóp orayǵa iye ekenligin anıqlań: $4 x^2+5 x y+3 y^2-x+9 y-12=0$;
Tómendegi sızıqlardan qaysı biri oraylıq (yaǵnıy birden-bir orayǵa iye), qaysı biri orayǵa iye emes, qaysı biri sheksiz kóp orayǵa iye ekenligin anıqlań: $4 x^2-4 x y+y^2-6 x+8 y+13=0$;
Tómendegi sızıqlardan qaysı biri oraylıq (yaǵnıy birden-bir orayǵa iye), qaysı biri orayǵa iye emes, qaysı biri sheksiz kóp orayǵa iye ekenligin anıqlań: $4 x^2-4 x y+y^2-12 x+6 y-11=0$;
Tómendegi sızıqlardan qaysı biri oraylıq (yaǵnıy birden-bir orayǵa iye), qaysı biri orayǵa iye emes, qaysı biri sheksiz kóp orayǵa iye ekenligin anıqlań:: $x^2-2 x y+4 y^2+5 x-7 y+12=0$;
Tómendegi sızıqlardan qaysı biri oraylıq (yaǵnıy birden-bir orayǵa iye), qaysı biri orayǵa iye emes, qaysı biri sheksiz kóp orayǵa iye ekenligin anıqlań:  $x^2-2 x y+y^2-6 x+6 y-3=0$;
Tómendegi sızıqlardan qaysı biri oraylıq (yaǵnıy birden-bir orayǵa iye), qaysı biri orayǵa iye emes, qaysı biri sheksiz kóp orayǵa iye ekenligin anıqlań: $4 x^2-20 x y+25 y^2-14 x+2 y-15=0$;
Tómendegi sızıqlardan qaysı biri oraylıq (yaǵnıy birden-bir orayǵa iye), qaysı biri orayǵa iye emes, qaysı biri sheksiz kóp orayǵa iye ekenligin anıqlań:  $4 x^2-6 x y-9 y^2+3 x-7 y+12=0$.
Tómendegi sızıqlardan qaysı biri oraylıq (yaǵnıy birden-bir orayǵa iye), qaysı biri orayǵa iye emes, qaysı biri sheksiz kóp orayǵa iye ekenligin anıqlań: $x^2-6 x y+9 y^2-12 x+36 y+20=0$;
Tómendegi sızıqlardan qaysı biri oraylıq (yaǵnıy birden-bir orayǵa iye), qaysı biri orayǵa iye emes, qaysı biri sheksiz kóp orayǵa iye ekenligin anıqlań:  $4 x^2+4 x y+y^2-8 x-4 y-21=0$;
Tómendegi sızıqlardan qaysı biri oraylıq (yaǵnıy birden-bir orayǵa iye), qaysı biri orayǵa iye emes, qaysı biri sheksiz kóp orayǵa iye ekenligin anıqlań: $25 x^2-10 x y+y^2+40 x-8 y+7=0$.
++++
Diskriminantın esaplaw arqalı tómendegi teńlemelerdiń hár biriniń tipin anıqlań: $2 x^2+10 x y+12 y^2-7 x+18 y-15=0$;
Diskriminantın esaplaw arqalı tómendegi teńlemelerdiń hár biriniń tipin anıqlań: $3 x^2-8 x y+7 y^2+8 x-15 y+20=0$;
Diskriminantın esaplaw arqalı tómendegi teńlemelerdiń hár biriniń tipin anıqlań: $25 x^2-20 x y+4 y^2-12 x+20 y-17=0$;
Diskriminantın esaplaw arqalı tómendegi teńlemelerdiń hár biriniń tipin anıqlań: $5 x^2+14 x y+11 y^2+12 x-7 y+19=0$;
Diskriminantın esaplaw arqalı tómendegi teńlemelerdiń hár biriniń tipin anıqlań: $x^2-4 x y+4 y^2+7 x-12=0$;
Diskriminantın esaplaw arqalı tómendegi teńlemelerdiń hár biriniń tipin anıqlań: $3 x^2-2 x y-3 y^2+12 y-15=0$.
Koordinatalar sistemasın túrlendirmesten tómendegi teńlemelerdiń hár biri ellipsti anıqlawın kórsetiń hám onıń yarım kósherlerin tabıń: $41 x^2+24 x y+9 y^2+24 x+18 y-36=0$;
Koordinatalar sistemasın túrlendirmesten tómendegi teńlemelerdiń hár biri ellipsti anıqlawın kórsetiń hám onıń yarım kósherlerin tabıń: $8 x^2+4 x y+5 y^2+16 x+4 y-28=0$;
Koordinatalar sistemasın túrlendirmesten tómendegi teńlemelerdiń hár biri ellipsti anıqlawın kórsetiń hám onıń yarım kósherlerin tabıń: $13 x^2+18 x y+37 y^2-26 x-18 y+3=0$;
Koordinatalar sistemasın túrlendirmesten tómendegi teńlemelerdiń hár biri ellipsti anıqlawın kórsetiń hám onıń yarım kósherlerin tabıń: $13 x^2+10 x y+13 y^2+46 x+62 y+13=0$.
Koordinatalar sistemasın túrlendirmesten tómendegi teńlemelerdiń hár biri birden-bir noqattı anıqlawın kórsetiń hám onıń koordinataların tabıń: $5 x^2-6 x y+2 y^2-2 x+2=0$;
Koordinatalar sistemasın túrlendirmesten tómendegi teńlemelerdiń hár biri birden-bir noqattı anıqlawın kórsetiń hám onıń koordinataların tabıń: $x^2+2 x y+2 y^2+6 y+9=0$;
Koordinatalar sistemasın túrlendirmesten tómendegi teńlemelerdiń hár biri birden-bir noqattı anıqlawın kórsetiń hám onıń koordinataların tabıń: $5 x^2+4 x y+y^2-6 x-2 y+2=0$;
Koordinatalar sistemasın túrlendirmesten tómendegi teńlemelerdiń hár biri birden-bir noqattı anıqlawın kórsetiń hám onıń koordinataların tabıń: $x^2-6 x y+10 y^2+10 x-32 y+26=0$.
Koordinatalar sistemasın túrlendirmesten tómendegi teńlemelerdiń hár biri giperbolanı anıqlawın kórsetiń hám onıń koordinataların tabıń: $4 x^2+24 x y+11 y^2+64 x+42 y+51=0$;
Koordinatalar sistemasın túrlendirmesten tómendegi teńlemelerdiń hár biri giperbolanı anıqlawın kórsetiń hám onıń koordinataların tabıń: $12 x^2+26 x y+12 y^2-52 x-48 y+73=0$
Koordinatalar sistemasın túrlendirmesten tómendegi teńlemelerdiń hár biri giperbolanı anıqlawın kórsetiń hám onıń koordinataların tabıń: $3 x^2+4 x y-12 x+16=0$;
Koordinatalar sistemasın túrlendirmesten tómendegi teńlemelerdiń hár biri giperbolanı anıqlawın kórsetiń hám onıń koordinataların tabıń: $x^2-6 x y-7 y^2+10 x-30 y+23=0$.
Koordinatalar sistemasın túrlendirmesten tómendegi teńlemelerdiń hár biri birden-bir noqattı anıqlawın kórsetiń hám onıń koordinataların tabıń: $3 x^2+4 x y+y^2-2 x-1=0$;
Koordinatalar sistemasın túrlendirmesten tómendegi teńlemelerdiń hár biri kesilisiwshi eki tuwrını anıqlawın kórsetiń hám onıń koordinataların tabıń: $x^2-6 x y+8 y^2-4 y-4=0$;
Koordinatalar sistemasın túrlendirmesten tómendegi teńlemelerdiń hár biri kesilisiwshi eki tuwrını anıqlawın kórsetiń hám onıń koordinataların tabıń: $x^2-4 x y+3 y^2=0$;
Koordinatalar sistemasın túrlendirmesten tómendegi teńlemelerdiń hár biri kesilisiwshi eki tuwrını anıqlawın kórsetiń hám onıń koordinataların tabıń: $x^2+4 x y+3 y^2-6 x-12 y+9=0$.
Koordinatalar sistemasın túrlendirmesten, tómendegi teńlemeler menen qanday geometriyalıq obrazdı anıqlanıwın tabıń: $8 x^2-12 x y+17 y^2+16 x-12 y+3=0$;
Koordinatalar sistemasın túrlendirmesten, tómendegi teńlemeler menen qanday geometriyalıq obrazdı anıqlanıwın tabıń: $17 x^2-18 x y-7 y^2+34 x-18 y+7=0$;
Koordinatalar sistemasın túrlendirmesten, tómendegi teńlemeler menen qanday geometriyalıq obrazdı anıqlanıwın tabıń: $2 x^2+3 x y-2 y^2+5 x+10 y=0$;
Koordinatalar sistemasın túrlendirmesten, tómendegi teńlemeler menen qanday geometriyalıq obrazdı anıqlanıwın tabıń: $6 x^2-6 x y+9 y^2-4 x+18 y+14=0$;
++++
Koordinatalar sistemasın túrlendirmesten, tómendegi teńlemelerdiń hár biri parabolanı anıqlawın kórsetiń hám parametrin tabıń: $9 x^2+24 x y+16 y^2-120 x+90 y=0$;
Koordinatalar sistemasın túrlendirmesten, tómendegi teńlemelerdiń hár biri parabolanı anıqlawın kórsetiń hám parametrin tabıń: $9 x^2-24 x y+16 y^2-54 x-178 y+181=0$;
Koordinatalar sistemasın túrlendirmesten, tómendegi teńlemelerdiń hár biri parabolanı anıqlawın kórsetiń hám parametrin tabıń: $x^2-2 x y+y^2+6 x-14 y+29=0$;
Koordinatalar sistemasın túrlendirmesten, tómendegi teńlemelerdiń hár biri parabolanı anıqlawın kórsetiń hám parametrin tabıń: $9 x^2-6 x y+y^2-50 x+50 y-275=0$.
++++
$x-2=0$ tegislik $\frac{x^2}{16}+\frac{y^2}{12}+\frac{z^2}{4}=1$ ellipsoidti ellips boyınsha kesip ótetuģının kórsetiń; onıń yarım kósherleri hám tóbelerin tabıń.
$z+1=0$ tegislik bir qabatlı $\frac{x^2}{32}-\frac{y^2}{18}+\frac{z^2}{2}=1$ giperboloidti giperbola boyınsha kesip ótetuģının kórsetiń; onıń yarım kósherleri hám tóbelerin tabıń.
$y+6=0$ tegislik $\frac{x^2}{5}-\frac{y^2}{4}=6 z$ giperbolik paraboloidti parabola boyınsha kesip ótetuģının kórsetiń; parametrin hám tóbesin tabıń.
$y^2+z^2=x$ elliptik paraboloidtıń $x+2 y-z=0$ tegislik penen kesilisiwiniń koordinata tegisliklerindegi proekciyalarınıń teńlemelerin tabıń.
Berilgen teńleme menen qanday iymek sızıq anıqlanıwın tabıń: $\left\{\begin{array}{l}\frac{x^2}{3}+\frac{y^2}{6}=2 z, \\ 3 x-y+6 z-14=0\end{array}\right.$
Berilgen teńleme menen qanday iymek sızıq anıqlanıwın tabıń: $\left\{\begin{array}{l}\frac{x^2}{4}-\frac{y^2}{3}=2 z \\ x-2 y+2=0 ;\end{array}\right.$
Berilgen teńleme menen qanday iymek sızıq anıqlanıwın tabıń: $\left\{\begin{array}{l}\frac{x^2}{.4}+\frac{y^2}{9}-\frac{z^2}{36}=1, \\ 9 x-6 y+2 z-28=0,\end{array}\right.$