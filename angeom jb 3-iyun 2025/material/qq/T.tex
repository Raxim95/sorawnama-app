Tegislikte ekinshi tártipli sızıqlar (Ekinshi tártipli teńleme, Kvadrat kórinisindegi teńleme, Konik sızıqlar (konuslar kesimi))
Parabola hám onıń kanonikalıq teńlemeleri (Fokus (bagdarlawshı noqat), Direktrisa (bagdarlawshı sızıq), Kósher (simmetriya kósheri))
++++
Ekinshi tártipli sızıqlardıń ulıwma teńlemeleri (Ulıwma teńleme)
Ekinshi tártipli sızıq hám tuwrı sızıqtıń óz ara jaylasıwı (Kesilisiw noqatları, Urınba (urınıw) jaģdayı)
Ekinshi tártipli sızıqqa urınba, túyinles diametri teńlemesi (Urınba teńlemesi, Túyinles diametr: oraydan ótiwshi simmetriya kósherleri)
Ekinshi tártipli sızıq orayı (Oraylıq sızıqlar (ellips, giperbola), Oray koordinataları: simmetriya orayı)
Ekinshi tártipli sızıqlardıń ulıwma teńlemesin invariantlar járdeminde kanonikalıq túrge keltiriw
++++
Ekinshi tártipli betliklerdiń kanonikalıq teńlemeleri (Ellipsoid, Giperboloid (1 gewekli), Giperboloid (2 gewekli))
Ekinshi tártipli betliklerdiń kanonikalıq teńlemeleri (Paraboloid (ellipstik), Paraboloid (giperbolik), Konus, Cilindr)
Bir gewekli giperboloid hám giperbolik paraboloidtıń tuwrı sızıqlı jasawshıları (Giperboloid, Giperbolik paraboloid, Sızıqlı jasawshılar)
Ekinshi tártipli betliklerdiń ulıwma teńlemeleri (Ulıwma teńleme)
Ekinshi tártipli betliklerdiń ulıwmalıq teńlemesin kanonikalıq túrge keltiriw (invariantlar járdeminde)
Ekinshi tártipli betlik orayı, urınba tegisligi hám diametral tegisligi (Oray, Urınba tegislik, Diametral tegislik.)