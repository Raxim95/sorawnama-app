Eki tóbesi $x^2+5 y^2=20$ ellipstiń fokuslarında jatıwshı, qalǵan ekewi bolsa onıń kishi kósheri tóbeleri menen ústpe-úst túsiwshi tórtmúyeshliktiń maydanın esaplań.
$\varepsilon=\frac{2}{3}$ ellipsiniń ekssentrisiteti, $M$ ellips noqatınıń fokal radiusı 10 ģa teń. $M$ noqattan usı fokusqa sáykes direktrisaǵa shekem bolǵan aralıqtı esaplań.
$\varepsilon=\frac{2}{5}$ ellipstiń ekssentrisiteti, ellipstiń $M$ noqatınan direktrisaģa shekemgi aralıq 20 ģa teń. $M$ noqattan usı direktrisa menen bir tárepleme fokusqa shekem bolǵan aralıqtı esaplań.
Ellipstiń ekssentrisiteti $\varepsilon=\frac{1}{3}$, onıń orayı koordinatalar bası menen ústpe-úst túsedi, fokuslarınan biri $ (-2; 0) $. Abcissası 2 ge teń bolǵan ellipstiń $M_1$ noqatınan berilgen fokusqa sáykes direktrisaģa shekem bolǵan aralıqtı tabıń.
Ellipstiń ekssentrisiteti $\varepsilon=\frac{1}{2}$, onıń orayı koordinatalar bası menen ústpe-úst túsedi, direktrisalardan biri $x=16$ teńleme menen berilgen. Abcissası $-4$ ke teń bolǵan ellipstiń $M_1$ noqatınan berilgen direktrisa menen bir tárepleme fokusqa shekem bolǵan aralıqtı esaplań.
$\frac{x^2}{25}+\frac{y^2}{15}=1$ ellipstiń fokusı arqalı onıń úlken kósherine perpendikulyar ótkerilgen. Bul perpendikulyardıń ellips penen kesilisken noqatlarınan fokuslarga shekem bolǵan aralıqlardı anıqlań.
Ellipstegi ekssentrisitetti anıqlań, eger: onıń kishi kósheri fokuslardan $60^{\circ}$ múyesh astında kórinse;
Ellipstegi ekssentrisitetti anıqlań, eger: fokusları arasındaǵı kesindiniń ózi kishi kósherdiń tóbesinen tuwrı múyesh astında kórinse;
Ellipstegi ekssentrisitetti anıqlań, eger: direktrisalar arasındaǵı aralıq fokuslar arasındaǵı aralıqtan úsh ese úlken bolsa;
Ellipstegi ekssentrisitetti anıqlań, eger: ellips orayınan onıń direktrisasına túsirilgen perpendikulyar kesindisi ellipstiń tóbesi menen teń ekige bólinedi.
Tómendegilerdi bilgen halda ellips teńlemesin dúziń: onıń úlken kósheri 26 ģa teń hám fokusları $F_1 (-10; 0), F_2 (14; 0) $;
Tómendegilerdi bilgen halda ellips teńlemesin dúziń: onıń kishi kósheri 2 ge teń hám fokusları $F_1 (-1;-1) $, $F_2 (1; 1) $;
Tómendegilerdi bilgen halda ellips teńlemesin dúziń: onıń fokusları $F_1\left(-2; \frac{3}{2}\right), F_2\left(2;-\frac{3}{2}\right) $ hám ekssentrisitet $\varepsilon=\frac{\sqrt{2}}{2}$;
Tómendegilerdi bilgen halda ellips teńlemesin dúziń: onıń fokusları $F_1 (1; 3), F_2 (3; 1) $ hám direktrisalar arasındaģı aralıq $12 \sqrt{2}$ qa teń.
Eger ellipstiń ekssentrisiteti $\varepsilon=\frac{1}{2}$ hám fokusı $F (3; 0) $ hám oǵan sáykes direktrisa teńlemesi $x+y-1=0$ belgili bolsa, onıń teńlemesin dúziń.
$M_1 (2;-1) $ noqatı fokusı $F (1;0) $ bolǵan ellipste jatadı. Bul fokusqa sáykes direktrisa bolsa $2x-y-10=0$ teńleme menen berilgen. Usı ellipstiń teńlemesin dúziń.
Kósherleri koordinata kósherleri menen ústpe-úst túsetuģin hám $P (2,2); Q (3,1) $ noqatlar arqalı ótiwshi ellips teńlemesin dúziń.
Úlken kósheri 2 birlikke teń, fokusları $F_1 (0,1), F_2 (1,0) $ noqatlarda bolgan ellipstiń teńlemesin dúziń.
Ellips fokuslarınıń birinen úlken kósheri tóbelerine shekemgi aralıqlar sáykes túrde 7 hám 1 ge teń. Bul ellipstiń tenlemesin dúziń.
$\frac{x^2}{16}+\frac{y^2}{9}=1$ ellipsning $x+y-1=0$ to'g'ri chiziqqa parallel bo'lgan urinmalarini aniqlang.
++++
Giperbolanıń haqıyqıy kósherine perpendikulyar bolǵan hám giperbola fokusınan ótken xorda uzınlıǵın tabıń.
$\frac{x^2}{49}+\frac{y^2}{24}=1$ ellips penen fokuslas hám ekssentrisiteti $e=\frac{5}{4}$ bolgan giperbolanıń teńlemesi jazılsın.
Giperbolanıń yarım kósherlerin tabıń, eger: fokusları arasındaǵı aralıq 8 ge hám direktrisaları arasındaǵı aralıq 6 ģa teń.
Giperbolanın yarım kósherlerin tabiń, eger: direktrisaları $x= \pm 3 \sqrt{2}$ tenlemeler menen berilgen hám asimptotaları arasındaģi múyesh tuwri múyesh;
Giperbolanıń yarım kósherlerin tabıń, eger: asimptotaları $y= \pm 2 x$ tenlemeleri menen berilgen hám fokusları oraydan 5 birlik aralıqta;
Giperbolanıń yarım kósherlerin tabıń, eger: asimptotaları $y= \pm \frac{5}{3} x$ tenlemeleri menen berilgen hám giperbola $N (6,9) $ noqatınan ótedi.
Giperbolanıń asimptotaları arasındaǵı múyeshin tabıń, eger: ekssentrisiteti $e=2$;
Giperbolanıń asimptotaları arasındaǵı múyeshin tabıń, eger: fokusları arasındaǵı qashıqlıq direktrisaları arasındaǵı qashıqlıqtan eki ese úlken bolsa.
$\frac{x^2}{16}-\frac{y^2}{9}=1$ giperbolada fokal radiusları óz ara perpendikulyar bolgan noqat tabılsın.
$\frac{x^2}{9}-\frac{y^2}{4}=1$ giperbolanıń $M (5,1) $ noqatında teń ekige bólinetuģın xordasınıń teńlemesi dúzilsin.
$\frac{x^2}{5}-\frac{y^2}{4}=1$ giperbolaģa $ (5,-4) $ noqatta urınatuģın tuwrı sızıq teńlemesi jazılsın.
$x^2-y^2=8$ giperbolaga $M(3,-1)$ noqatında urınatuģın tuwrı sızıqtıń teńlemesin jazıń.
$\frac{x^2}{80}-\frac{y^2}{20}=1$ giperbolada $M_1 (10;-\sqrt{5}) $ noqat berilgen. $M_1$ noqatınıń fokal radiusları jatqan tuwrı sızıqlardıń teńlemelerin dúziń.
$\frac{x^2}{64}-\frac{y^2}{36}=1$ giperbolanıń oń fokusına shekemgi aralıǵı 4,5 ke teń bolǵan noqatların anıqlań.
Teń tárepli giperbolanıń ekssentrisiteti anıqlansın.
Fokusları $\frac{x^2}{100}+\frac{y^2}{64}=1$ ellipstiń tóbelerinde jatıwshı, direktrisaları bolsa usı ellipstiń fokuslarınan ótiwshi giperbolanıń teńlemesin dúziń.
Tómendegilerdi bilgen halda giperbolanıń teńlemesin dúziń: onıń tóbeleri arasındaǵı aralıq 24 ke teń hám fokusları $F_1 (-10; 2), F_2 (16; 2) $;
Tómendegilerdi bilgen halda giperbolanıń teńlemesin dúziń: fokuslar $F_1 (3; 4), F_2 (-3;-4) $ hám direktrisalar arasındaǵı aralıq 3,6;
Tómendegilerdi bilgen halda giperbolanıń teńlemesin dúziń: Asimptotaları arasındaǵı múyesh $90^{\circ}$ qa teń hám fokuslar $F_1 (4;-4), F_2 (-2; 2) $.
Eger giperbolanıń ekssentrisiteti $\varepsilon=\sqrt{5}$, fokusı $F (2;-3) $ oǵan sáykes direktrisasınıń teńlemesi $3 x-y+3=0$ belgili bolsa, onıń teńlemesin dúziń
$M_1 (1;-2) $ noqat fokusı $F (-2; 2) $, oǵan sáykes direktrisa bolsa $2x-y-1=0$ teńleme menen berilgen giperbolaǵa tiyisli. Bul giperbolanıń teńlemesin dúziń.
$\frac{x^2}{16}-\frac{y^2}{64}=1$ giperbolaǵa $10 x-3 y+9=0$ tuwrısına parallel bolǵan urınbalardıń teńlemelerin dúziń.
$\frac{x^2}{16}-\frac{y^2}{8}=-1$ giperbolaǵa $2 x+4 y-5=0$ tuwrısına parallel urınbalar ótkiziń hám olar arasındaǵı $d$ aralıqtı esaplań.
$x^2-y^2=16$ giperbolaǵa $A (-1;-7)$ noqattan ótkerilgen urınbalar teńlemesin dúziń.
++++
Parabola tóbesiniń koordinataların, parametrin hám kósheriniń baǵıtın anıqlań: $y^2-10 x-2 y-19=0$;
Parabola tóbesiniń koordinataların, parametrin hám kósheriniń baǵıtın anıqlań: $y^2-6 x+14 y+49=0$,
Parabola tóbesiniń koordinataların, parametrin hám kósheriniń baǵıtın anıqlań: $y^2+8 x-16=0$,
Parabola tóbesiniń koordinataların, parametrin hám kósheriniń baǵıtın anıqlań: $x^2-6 x-4 y+29=0$,
Parabola tóbesiniń koordinataların, parametrin hám kósheriniń baǵıtın anıqlań: $y=A x^2+B x+C$,
Parabola tóbesiniń koordinataların, parametrin hám kósheriniń baǵıtın anıqlań: $y=x^2-8 x+15$,
Parabola tóbesiniń koordinataların, parametrin hám kósheriniń baǵıtın anıqlań: $y=x^2+6 x$.
Eger parabolanıń fokusı $F (7; 2) $ hám direktrisa $x-5=0$ teńlemesi berilgen bolsa, onıń teńlemesin dúziń.
Eger parabolanıń fokusı $F (4;3) $ hám direktrisa $x-1=0$ teńlemesi berilgen bolsa, onıń teńlemesin dúziń.
Eger parabolanıń fokusı $F(2;-1) $ hám direktrisa $x-y-1=0$ teńlemesi berilgen bolsa, onıń teńlemesin dúziń.
Berilgen parabola tóbesi $A (6;-3) $ hám onıń direktrisasınıń teńlemesi $3x-5y+1=0$ berilgen. Bul parabolanıń $F$ fokusın tabıń.
Parabola tóbesi $A (-2;-1) $ hám onıń direktrisasınıń teńlemesi $x+2y-1=0$ berilgen. Bul parabolanıń teńlemesin dúziń.
$y^2=8x$ parabolanıń $2x+2y-3=0$ tuwrısına parallel urınbasınıń teńlemesin dúziń.
$x^2=16y$ parabolanıń $2x+4y+7=0$ tuwrısına perpendikulyar bolǵan urınbasınıń teńlemesin dúziń.
$A (5;9) $ noqattan $y^2=5x$ parabolaǵa júrgizilgen urınbalardıń urınıw noqatların tutastırıwshı xordanıń teńlemesin dúziń.
++++
Berilgen sızıqlar oraylıq ekenligin kórsetiń hám hárbir iymek sızıq ushın orayınıń koordinataların tabıń: $3x^2+5xy+y^2-8x-11y-7=0$.
Berilgen sızıqlar oraylıq ekenligin kórsetiń hám hárbir iymek sızıq ushın orayınıń koordinataların tabıń:$5 x^2+4 x y+2 y^2+20 x+20 y-18=0$;
Berilgen sızıqlar oraylıq ekenligin kórsetiń hám hárbir iymek sızıq ushın orayınıń koordinataların tabıń:$9 x^2-4 x y-7 y^2-12=0$;
Berilgen sızıqlar oraylıq ekenligin kórsetiń hám hárbir iymek sızıq ushın orayınıń koordinataların tabıń: $2 x^2-6 x y+5 y^2+22 x-36 y+11=0$.
Berilgen teńlemeler oraylıq iymek sızıqlar ekenligin kórsetiń hám hárbir teńlemeni koordinata basın orayģa kóshiriń: $3x^2-6xy+2y^2-4x+2y+1=0$.
Berilgen teńlemeler oraylıq iymek sızıqlar ekenligin kórsetiń hám hárbir teńlemeni koordinata basın orayģa kóshiriń: $6 x^2+4 x y+y^2+4 x-2 y+2=0$;
Berilgen teńlemeler oraylıq iymek sızıqlar ekenligin kórsetiń hám hárbir teńlemeni koordinata basın orayģa kóshiriń: $4 x^2+6 x y+y^2-10 x-10=0$;
Berilgen teńlemeler oraylıq iymek sızıqlar ekenligin kórsetiń hám hárbir teńlemeni koordinata basın orayģa kóshiriń:  $4 x^2+2 x y+6 y^2+6 x-10 y+9=0$.
++++
Bes noqattan ótiwshi ekinshi tártipli sızıqtıń teńlemesin dúziń: $(0,0),(0,1),(1,0),(2,-5),(-5,2)$.
$5 x^2-3 x y+y^2-3 x+2 y-5=0$ sızıqtıń $x-2 y-1=0$ tuwri sızıq penen kesilisiwinen payda bolgan xordanıń ortasınan ótetuģin diametr teńlemesi jazılsın.
ITECH túri, ólshemleri hám jaylasıwın anıqlań: $5 x^2+4 x y+8 y^2-32 x-56 y+80=0$.
ITECH túri, ólshemleri hám jaylasıwın anıqlań: $9 x^2+24 x y+16 y^2-230 x+110 y-475=0$.
ITECH túri, ólshemleri hám jaylasıwın anıqlań: $5 x^2+12 x y-12 x-22 y-19=0$.
ITECH túri, ólshemleri hám jaylasıwın anıqlań: $x^2-2 x y+y^2-10 x-6 y+25=0$.
ITECH túri, ólshemleri hám jaylasıwın anıqlań: $x^2-5 x y+4 y^2+x+2 y-2=0$.
ITECH túri, ólshemleri hám jaylasıwın anıqlań: $4 x^2-12 x y+9 y^2-2 x+3 y-2=0$.
ITECH túri, ólshemleri hám jaylasıwın anıqlań: $2 x^2+4 x y+5 y^2-6 x-8 y-1=0$;
ITECH túri, ólshemleri hám jaylasıwın anıqlań: $5 x^2+8 x y+5 y^2-18 x-18 y+9=0$;
ITECH túri, ólshemleri hám jaylasıwın anıqlań: $5 x^2+6 x y+5 y^2-16 x-16 y-16=0$;
ITECH túri, ólshemleri hám jaylasıwın anıqlań: $6 x y-8 y^2+12 x-26 y-11=0$;
ITECH túri, ólshemleri hám jaylasıwın anıqlań: $7 x^2+16 x y-23 y^2-14 x-16 y-218=0$;
ITECH túri, ólshemleri hám jaylasıwın anıqlań: $7 x^2-24 x y-38 x+24 y+175=0$;
ITECH túri, ólshemleri hám jaylasıwın anıqlań: $9 x^2+24 x y+16 y^2-40 x-30 y=0$;
ITECH túri, ólshemleri hám jaylasıwın anıqlań: $x^2+2 x y+y^2-8 x+4=0$;
ITECH túri, ólshemleri hám jaylasıwın anıqlań: $4 x^2-4 x y+y^2-2 x-14 y+7=0$.
Berilgen teńlemeniń tipin anıqlań, koordinata kósherlerin parallel kóshiriw arqalı ápiwayı túrge keltiriń; qanday geometriyalıq obrazdı ańlatıwın anıqlań, eski hám jańa koordinata kósherlerine salıstırģanda sızılmada súwretleń: $4 x^2+9 y^2-40 x+36 y+100=0$;
Berilgen teńlemeniń tipin anıqlań, koordinata kósherlerin parallel kóshiriw arqalı ápiwayı túrge keltiriń; qanday geometriyalıq obrazdı ańlatıwın anıqlań, eski hám jańa koordinata kósherlerine salıstırģanda sızılmada súwretleń: $9 x^2-16 y^2-54 x-64 y-127=0$;
Berilgen teńlemeniń tipin anıqlań, koordinata kósherlerin parallel kóshiriw arqalı ápiwayı túrge keltiriń; qanday geometriyalıq obrazdı ańlatıwın anıqlań, eski hám jańa koordinata kósherlerine salıstırģanda sızılmada súwretleń:  $9 x^2+4 y^2+18 x-8 y+49=0$;
Berilgen teńlemeniń tipin anıqlań, koordinata kósherlerin parallel kóshiriw arqalı ápiwayı túrge keltiriń; qanday geometriyalıq obrazdı ańlatıwın anıqlań, eski hám jańa koordinata kósherlerine salıstırģanda sızılmada súwretleń: $4 x^2-y^2+8 x-2 y+3=0$;
Berilgen teńlemeniń tipin anıqlań, koordinata kósherlerin parallel kóshiriw arqalı ápiwayı túrge keltiriń; qanday geometriyalıq obrazdı ańlatıwın anıqlań, eski hám jańa koordinata kósherlerine salıstırģanda sızılmada súwretleń: $2 x^2+3 y^2+8 x-6 y+11=0$.
Berilgen teńlemeni ápiwayı túrge keltiriń; tipin anıqlań; qanday geometriyalıq obrazdı ańlatıwın anıqlań, eski hám de jańa koordinata kósherlerine qarata sızılmada súwretleń: $32 x^2+52 x y-7 y^2+180=0$;
Berilgen teńlemeni ápiwayı túrge keltiriń; tipin anıqlań; qanday geometriyalıq obrazdı ańlatıwın anıqlań, eski hám de jańa koordinata kósherlerine qarata sızılmada súwretleń: $5 x^2-6 x y+5 y^2-32=0$;
Berilgen teńlemeni ápiwayı túrge keltiriń; tipin anıqlań; qanday geometriyalıq obrazdı ańlatıwın anıqlań, eski hám de jańa koordinata kósherlerine qarata sızılmada súwretleń: $17 x^2-12 x y+8 y^2=0$;
Berilgen teńlemeni ápiwayı túrge keltiriń; tipin anıqlań; qanday geometriyalıq obrazdı ańlatıwın anıqlań, eski hám de jańa koordinata kósherlerine qarata sızılmada súwretleń:  $5 x^2+24 x y-5 y^2=0$;
Berilgen teńlemeni ápiwayı túrge keltiriń; tipin anıqlań; qanday geometriyalıq obrazdı ańlatıwın anıqlań, eski hám de jańa koordinata kósherlerine qarata sızılmada súwretleń: $5 x^2-6 x y+5 y^2+8=0$.
++++
Berilgen teńleme parabolik ekenligin kórsetiń; ápiwayı túrge keltiriń; qanday geometriyalıq obrazdı anlatıwın anıqlań, eski hám de jańa koordinata kósherlerine salıstırģanda sızılmada súwretleń:$9 x^2-24 x y+16 y^2-20 x+110 y-50=0$;
Berilgen teńleme parabolik ekenligin kórsetiń; ápiwayı túrge keltiriń; qanday geometriyalıq obrazdı anlatıwın anıqlań, eski hám de jańa koordinata kósherlerine salıstırģanda sızılmada súwretleń:$9 x^2+12 x y+4 y^2-24 x-16 y+3=0$;
Berilgen teńleme parabolik ekenligin kórsetiń; ápiwayı túrge keltiriń; qanday geometriyalıq obrazdı anlatıwın anıqlań, eski hám de jańa koordinata kósherlerine salıstırģanda sızılmada súwretleń:$16 x^2-24 x y+9 y^2-160 x+120 y+425=0$.
Berilgen teńleme parabolik ekenligin kórsetiń; ápiwayı túrge keltiriń; qanday geometriyalıq obrazdı anlatıwın anıqlań, eski hám de jańa koordinata kósherlerine salıstırģanda sızılmada súwretleń: $9 x^2+24 x y+16 y^2-18 x+226 y+209=0$;
Berilgen teńleme parabolik ekenligin kórsetiń; ápiwayı túrge keltiriń; qanday geometriyalıq obrazdı anlatıwın anıqlań, eski hám de jańa koordinata kósherlerine salıstırģanda sızılmada súwretleń: $x^2-2 x y+y^2-12 x+12 y-14=0$
Berilgen teńleme parabolik ekenligin kórsetiń; ápiwayı túrge keltiriń; qanday geometriyalıq obrazdı anlatıwın anıqlań, eski hám de jańa koordinata kósherlerine salıstırģanda sızılmada súwretleń:$4 x^2+12 x y+9 y^2-4 x-6 y+1=0$.
Berilgen teńlemelerdiń parabolik ekenligin kórsetiń hám olardıń hár birin $(\alpha x+\beta y)^2+2 a_{13} x+2 a_{23} y+a_{33}=0$ kórinisinde jazıń: $x^2+4 x y+4 y^2+4 x+y-15=0 ;$
Berilgen teńlemelerdiń parabolik ekenligin kórsetiń hám olardıń hár birin $(\alpha x+\beta y)^2+2 a_{13} x+2 a_{23} y+a_{33}=0$ kórinisinde jazıń: $9 x^2-6 x y+y^2-x+2 y-14=0$;
Berilgen teńlemelerdiń parabolik ekenligin kórsetiń hám olardıń hár birin $(\alpha x+\beta y)^2+2 a_{13} x+2 a_{23} y+a_{33}=0$ kórinisinde jazıń: $25 x^2-20 x y+4 y^2+3 x-y+11=0$;
Berilgen teńlemelerdiń parabolik ekenligin kórsetiń hám olardıń hár birin $(\alpha x+\beta y)^2+2 a_{13} x+2 a_{23} y+a_{33}=0$ kórinisinde jazıń:  $16 x^2+16 x y+4 y^2-5 x+7 y=0$;
Berilgen teńlemelerdiń parabolik ekenligin kórsetiń hám olardıń hár birin $(\alpha x+\beta y)^2+2 a_{13} x+2 a_{23} y+a_{33}=0$ kórinisinde jazıń:  $9 x^2-42 x y+49 y^2+3 x-2 y-24=0$.
++++
$\frac{x^2}{12}+\frac{y^2}{4}+\frac{z^2}{3}=1$ ellipsoidı hám $2x-3y+4z-11=0$ tegisliginiń kesilisiw sızıǵı qanday iymek sızıq yekenligin anıqlań hám onıń orayın tabıń.
$\frac{x^2}{2}-\frac{z^2}{3}=y$ giperbolik paraboloidi hám $3x-3y+4z+2=0$ tegisliginiń kesilisiw sızıǵı qanday iymek sızıq ekenligin anıqlań hám onıń orayın tabıń.
Lagranj usılınan paydalanıp, teńlemelerdi kvadratlar qosındısı túrine keltirip, tómendegi betlerdiń kórinisin anıqlań: $4 x^2+6 y^2+4 z^2+4 x z-8 y-4 z+3=0$;
Lagranj usılınan paydalanıp, teńlemelerdi kvadratlar qosındısı túrine keltirip, tómendegi betlerdiń kórinisin anıqlań: $x^2+5 y^2+z^2+2 x y+6 x z+2 y z-2 x+6 y-10 z=0$;
Lagranj usılınan paydalanıp, teńlemelerdi kvadratlar qosındısı túrine keltirip, tómendegi betlerdiń kórinisin anıqlań: $x^2+y^2-3 z^2-2 x y-6 x z-6 y z+2 x+2 y+4 z=0$;
Lagranj usılınan paydalanıp, teńlemelerdi kvadratlar qosındısı túrine keltirip, tómendegi betlerdiń kórinisin anıqlań: $x^2-2 y^2+z^2+4 x y-8 x z-4 y z-14 x-4 y+14 z+16=0$;
Lagranj usılınan paydalanıp, teńlemelerdi kvadratlar qosındısı túrine keltirip, tómendegi betlerdiń kórinisin anıqlań: $2 x^2+y^2+2 z^2-2 x y-2 y z+x-4 y-3 z+2=0$;
Lagranj usılınan paydalanıp, teńlemelerdi kvadratlar qosındısı túrine keltirip, tómendegi betlerdiń kórinisin anıqlań: $x^2-2 y^2+z^2+4 x y-10 x z+4 y z+x+y-z=0$;
Lagranj usılınan paydalanıp, teńlemelerdi kvadratlar qosındısı túrine keltirip, tómendegi betlerdiń kórinisin anıqlań: $2 x^2+y^2+2 z^2-2 x y-2 y z+4 x-2 y=0$;
Lagranj usılınan paydalanıp, teńlemelerdi kvadratlar qosındısı túrine keltirip, tómendegi betlerdiń kórinisin anıqlań: $x^2-2 y^2+z^2+4 x y-10 x z+4 y z+2 x+4 y-10 z-1=0$;
Lagranj usılınan paydalanıp, teńlemelerdi kvadratlar qosındısı túrine keltirip, tómendegi betlerdiń kórinisin anıqlań: $x^2+y^2+4 z^2+2 x y+4 x z+4 y z-6 z+1=0$;
Lagranj usılınan paydalanıp, teńlemelerdi kvadratlar qosındısı túrine keltirip, tómendegi betlerdiń kórinisin anıqlań: $4 x y+2 x+4 y-6 z-3=0$;
Lagranj usılınan paydalanıp, teńlemelerdi kvadratlar qosındısı túrine keltirip, tómendegi betlerdiń kórinisin anıqlań: $x y+x z+y z+2 x+2 y-2 z=0$.
Parallel kóshiriw hám burıw túrlendiriwleri yamasa aǵzalardı gruppalaw járdeminde tómendegi betliklerdiń kórinisi hám jaylasıwı anıqlansın: $z=2 x^2-4 y^2-6 x+8 y+1$;
Parallel kóshiriw hám burıw túrlendiriwleri yamasa aǵzalardı gruppalaw járdeminde tómendegi betliklerdiń kórinisi hám jaylasıwı anıqlansın: $z=x^2+3 y^2-6 y+1$;
Parallel kóshiriw hám burıw túrlendiriwleri yamasa aǵzalardı gruppalaw járdeminde tómendegi betliklerdiń kórinisi hám jaylasıwı anıqlansın: $x^2+2 y^2-3 z^2+2 x+4 y-6 z=0$;
Parallel kóshiriw hám burıw túrlendiriwleri yamasa aǵzalardı gruppalaw járdeminde tómendegi betliklerdiń kórinisi hám jaylasıwı anıqlansın: $x^2+2 x y+y^2-z^2=0$;
Parallel kóshiriw hám burıw túrlendiriwleri yamasa aǵzalardı gruppalaw járdeminde tómendegi betliklerdiń kórinisi hám jaylasıwı anıqlansın: $z^2=3 x+4 y+5$;
Parallel kóshiriw hám burıw túrlendiriwleri yamasa aǵzalardı gruppalaw járdeminde tómendegi betliklerdiń kórinisi hám jaylasıwı anıqlansın: $z=x^2+2 x y+y^2+1$;
Parallel kóshiriw hám burıw túrlendiriwleri yamasa aǵzalardı gruppalaw járdeminde tómendegi betliklerdiń kórinisi hám jaylasıwı anıqlansın: $z^2=x^2+2 x y+y^2+1$;
Parallel kóshiriw hám burıw túrlendiriwleri yamasa aǵzalardı gruppalaw járdeminde tómendegi betliklerdiń kórinisi hám jaylasıwı anıqlansın: $x^2+4 y^2+9 z^2-6 x+8 y-18 z-14=0$;
Parallel kóshiriw hám burıw túrlendiriwleri yamasa aǵzalardı gruppalaw járdeminde tómendegi betliklerdiń kórinisi hám jaylasıwı anıqlansın: $2 x y+z^2-2 z+1=0$;
Parallel kóshiriw hám burıw túrlendiriwleri yamasa aǵzalardı gruppalaw járdeminde tómendegi betliklerdiń kórinisi hám jaylasıwı anıqlansın: $x^2+y^2-z^2-2 x y+2 z-1=0$;
Parallel kóshiriw hám burıw túrlendiriwleri yamasa aǵzalardı gruppalaw járdeminde tómendegi betliklerdiń kórinisi hám jaylasıwı anıqlansın: $x^2+4 y^2-z^2-10 x-16 y+6 z+16=0$;
Parallel kóshiriw hám burıw túrlendiriwleri yamasa aǵzalardı gruppalaw járdeminde tómendegi betliklerdiń kórinisi hám jaylasıwı anıqlansın: $2 x y+2 x+2 y+2 z-1=0$;
Parallel kóshiriw hám burıw túrlendiriwleri yamasa aǵzalardı gruppalaw járdeminde tómendegi betliklerdiń kórinisi hám jaylasıwı anıqlansın: $3 x^2+6 x-8 y+6 z-7=0$;
Parallel kóshiriw hám burıw túrlendiriwleri yamasa aǵzalardı gruppalaw járdeminde tómendegi betliklerdiń kórinisi hám jaylasıwı anıqlansın: $x^2+y^2+2 z^2+2 x y+4 z=0$;
Parallel kóshiriw hám burıw túrlendiriwleri yamasa aǵzalardı gruppalaw járdeminde tómendegi betliklerdiń kórinisi hám jaylasıwı anıqlansın: $3 x^2+3 y^2+3 z^2-6 x+4 y-1=0$;
Parallel kóshiriw hám burıw túrlendiriwleri yamasa aǵzalardı gruppalaw járdeminde tómendegi betliklerdiń kórinisi hám jaylasıwı anıqlansın: $3 x^2+3 y^2-6 x+4 y-1=0$;
Parallel kóshiriw hám burıw túrlendiriwleri yamasa aǵzalardı gruppalaw járdeminde tómendegi betliklerdiń kórinisi hám jaylasıwı anıqlansın: $3 x^2+3 y^2-3 z^2-6 x+4 y+4 z+3=0$;
Parallel kóshiriw hám burıw túrlendiriwleri yamasa aǵzalardı gruppalaw járdeminde tómendegi betliklerdiń kórinisi hám jaylasıwı anıqlansın: $4 x^2-y^2-4 x+4 y-3=0$;