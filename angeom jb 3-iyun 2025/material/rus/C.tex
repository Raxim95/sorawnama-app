Составить уравнение эллипса с полуосями $a, b$ и центром $C\left(x_0 ; y_0\right)$, если известно, что оси симметрии эллипса параллельны осям координат.
Определить, при каких значениях $m$ прямая $y=-x+m$ 1) пересекает эллипс $\frac{x^2}{20}+\frac{y^2}{5}=1$; 2) касается его; 3) проходит вне этого эллипса.
Составить уравнение касательной к эллипсу $\frac{x^2}{a^2}+\frac{y^2}{b^2}=1$ в его точке $M_1\left(x_1 ; y_1\right)$.
Доказать, что касательные к эллипсу $\frac{x^2}{a^2}+\frac{y^2}{b^2}=1$, проведенные в концах одного и того же диаметра, параллельны. (Диаметром эллипса называется его хорда, проходящая через центр.)
Провести касательные к эллиису $\frac{x^2}{30}+\frac{y^2}{24}=1$ параллельно прямой $4 x-2 y+23=0$ и вычислить расстояние $d$ между ними.
Из точки $A\left(\frac{10}{3} ; \frac{5}{3}\right)$ проведены касательные к эллипсу $\frac{x^2}{20}+\frac{y^2}{5}=1$. Составить их уравнения.
Доказать, что произведение расстояний от центра эллипса до точки пересечения любой его касательной с фокальной осью и до основания перпендикуляря, опущенного из точки касания на фокальную ось, есть величина постоянная, равная квадрату большой полуоси эллипса.
Доказать, что произведение расстояний от фокусов до любой касательной к эллипсу равно квадрату малой полуоси.
++++
Доказать, что расстояние от фокуса гиперболы $\frac{x^2}{a^2}-\frac{y^2}{b^2}=1$ до ее асимптоты равно $b$.
Доказать что произведение расстояний от любой точки гиперболы $\frac{x^2}{a^2}-\frac{y^2}{b^2}=1$ до двух ее асимптот есть величина постоянная, равная $\frac{a^2 b^2}{a^2+b^2}$.
Доказать, что площадь параллелограмма, ограниченного асимптотами гиперболы $\frac{x^2}{a^2}-\frac{y^2}{b^2}=1$ и прямыми, проведенными через любую ее точку параллельно асимптотам, есть величина постоянная, равная $\frac{a b}{2}$.
Составить уравнение гиперболы, если известны ее полуоси $a$ и $b$, центр $C\left(x_0 ; y_0\right)$ и фокусы расположены на прямой: 1) параллельной оси $O x$; 2) параллельной оси $O y$.
Определить, при каких значениях $m$ прямая $y=\frac{5}{2} x+m$ пересекает гиперболу $\frac{x^2}{9}-\frac{y^2}{36}=1$; 2) касается ее; 3) проходит вне этой гиперболы
Вывести условие, при котором прямая $y=k x+m$ касается гиперболы $\frac{x^2}{a^2}-\frac{y^2}{b^2}=1$.
Составить уравнение касательной к гиперболе $\frac{x^2}{a^2}-\frac{y^2}{b^2}=1$ в ее точке $M_1\left(x_1 ; y_1\right)$.
Доказать, что касательные к гиперболе, проведенные в концах одного и того же диаметра, параллельны.
Составить уравнение гиперболы, касающейся двух прямых: $\quad 5 x-6 y-16=0, \quad 13 x-10 y-48=0$, при условии, что ее оси совпадают с осями координат.
Даны гиперболы $\frac{x^2}{a^2}-\frac{y^2}{b^2}=1$ и какая-нибудь ее касательная: $P$-точка пересечения касательной с осью $O x, Q$ - проекция точки касания на ту же ось. Доказать, что $O P \cdot O Q=a^2$.
++++
Определить, при каких значениях углового коэффициента $k$ прямая $y=k x+2$ 1) пересекает параболу $y^2=4 x$; 2) касается ее; 3) проходит вне этсй параболы.
Вывести условие, при котором прямая $y=k x+b$ касается параболы $y^2=2 p x$.
Составить уравнение касагельной к параболе $y^2=2 p x$ в ее точке $M_1\left(x_1 ; y_1\right)$.
Доказать, что две параболы, имеющие общую ось и общий фокус, расположенный между их вершинами, пересекаются под прямым углом.
Доказать, что если две параболы со взаимно перпендикулярными осями пересекаются в четырех точках, то эти точки лежат на одной окружности.
++++
При каких значениях $m$ и $n$ уравнение $m x^2+12 x y+9 y^2+4 x+n y-13=0$ определяет: 1) центральную линию; 2) линию без центра; 3) линию, имеющую бесконечно много центров.
Дано уравнение линии $4 x^2-4 x y+y^2+6 x+1=0$. Определить, при каких значениях углового коэффициента $k$ прямая $y=k x:$ 1) пересекает эту линию в одной точке; 2) касается этой линии; 3) пересекает эту линию в двух точках; 4) не имеет общих точек с этой линией.
++++
Каждое из следующих уравнений привести к каноническому виду; определить тип каждого из них; установить, какие геометрические образы они определяют; для каждого случая изобразить на чертеже оси первоначальной координатной системы, оси других координатных систем, которые вводятся по ходу решения, и геометрический образ, определяемый данным уравнением: $3 x^2+10 x y+3 y^2-2 x-14 y-13=0$;
Каждое из следующих уравнений привести к каноническому виду; определить тип каждого из них; установить, какие геометрические образы они определяют; для каждого случая изобразить на чертеже оси первоначальной координатной системы, оси других координатных систем, которые вводятся по ходу решения, и геометрический образ, определяемый данным уравнением: $25 x^2-14 x y+25 y^2+64 x-64 y-224=0$;
Каждое из следующих уравнений привести к каноническому виду; определить тип каждого из них; установить, какие геометрические образы они определяют; для каждого случая изобразить на чертеже оси первоначальной координатной системы, оси других координатных систем, которые вводятся по ходу решения, и геометрический образ, определяемый данным уравнением: $4 x y+3 y^2+16 x+12 y-36=0$;
Каждое из следующих уравнений привести к каноническому виду; определить тип каждого из них; установить, какие геометрические образы они определяют; для каждого случая изобразить на чертеже оси первоначальной координатной системы, оси других координатных систем, которые вводятся по ходу решения, и геометрический образ, определяемый данным уравнением: $7 x^2+6 x y-y^2+28 x+12 y+28=0$;
Каждое из следующих уравнений привести к каноническому виду; определить тип каждого из них; установить, какие геометрические образы они определяют; для каждого случая изобразить на чертеже оси первоначальной координатной системы, оси других координатных систем, которые вводятся по ходу решения, и геометрический образ, определяемый данным уравнением: $19 x^2+6 x y+11 y^2+38 x+6 y+29=0$;
Каждое из следующих уравнений привести к каноническому виду; определить тип каждого из них; установить, какие геометрические образы они определяют; для каждого случая изобразить на чертеже оси первоначальной координатной системы, оси других координатных систем, которые вводятся по ходу решения, и геометрический образ, определяемый данным уравнением: $5 x^2-2 x y+5 y^2-4 x+20 y+20=0$.
Каждое из следующих уравнений привести к каноническому виду; определить тип каждого из них; установить, какие геометрические образы они определяют; для каждого случая изобразить на чертеже оси первоначальной координатной системы, оси других координатных систем, которые вводятся по ходу решения, и геометрический образ, определяемый данным уравнением: $14 x^2+24 x y+21 y^2-4 x+18 y-139=0$;
Каждое из следующих уравнений привести к каноническому виду; определить тип каждого из них; установить, какие геометрические образы они определяют; для каждого случая изобразить на чертеже оси первоначальной координатной системы, оси других координатных систем, которые вводятся по ходу решения, и геометрический образ, определяемый данным уравнением: $11 x^2-20 x y-4 y^2-20 x-8 y+1=0$;
Каждое из следующих уравнений привести к каноническому виду; определить тип каждого из них; установить, какие геометрические образы они определяют; для каждого случая изобразить на чертеже оси первоначальной координатной системы, оси других координатных систем, которые вводятся по ходу решения, и геометрический образ, определяемый данным уравнением: $7 x^2+60 x y+32 y^2-14 x-60 y+7=0$;
Каждое из следующих уравнений привести к каноническому виду; определить тип каждого из них; установить, какие геометрические образы они определяют; для каждого случая изобразить на чертеже оси первоначальной координатной системы, оси других координатных систем, которые вводятся по ходу решения, и геометрический образ, определяемый данным уравнением: $50 x^2-8 x y+35 y^2+100 x-8 y+67=0$;
Каждое из следующих уравнений привести к каноническому виду; определить тип каждого из них; установить, какие геометрические образы они определяют; для каждого случая изобразить на чертеже оси первоначальной координатной системы, оси других координатных систем, которые вводятся по ходу решения, и геометрический образ, определяемый данным уравнением: $41 x^2+24 x y+34 y^2+34 x-112 y+129=0$;
Каждое из следующих уравнений привести к каноническому виду; определить тип каждого из них; установить, какие геометрические образы они определяют; для каждого случая изобразить на чертеже оси первоначальной координатной системы, оси других координатных систем, которые вводятся по ходу решения, и геометрический образ, определяемый данным уравнением: $29 x^2-24 x y+36 y^2+82 x-96 y-91=0$;
Каждое из следующих уравнений привести к каноническому виду; определить тип каждого из них; установить, какие геометрические образы они определяют; для каждого случая изобразить на чертеже оси первоначальной координатной системы, оси других координатных систем, которые вводятся по ходу решения, и геометрический образ, определяемый данным уравнением: $4 x^2+24 x y+11 y^2+64 x+42 y+51=0$;
Каждое из следующих уравнений привести к каноническому виду; определить тип каждого из них; установить, какие геометрические образы они определяют; для каждого случая изобразить на чертеже оси первоначальной координатной системы, оси других координатных систем, которые вводятся по ходу решения, и геометрический образ, определяемый данным уравнением: $41 x^2+24 x y+9 y^2+24 x+18 y-36=0$.
Для любого эллиптического уравнения доказать, что ни один из коэффициентов $a_{11}$ и $a_{22}$ не может обрашаться в нуль и что они суть числа одного знака.
Доказать, что эллиптическое уравнение второй степени ( $\delta>0$ ) определяет эллипс в том и только в том случае, когда $a_{11}$ и $\Delta$ суть числа разных знаков.
Доказать, что эллиптичсское уравнение второй степсни ( $\delta>0$ ) является уравпением мнимого эллипса в том и только в том случае, когда $a_{11}$ и $\Delta$ суть числа одинаковых знаков.
Доказать, что эллиптическое уравнение второй степени ( $\delta>0$ ) определяет вырожденный эллипс (точку) в том и только в том случае, когда $\Delta=0$.
++++
Для любого параболического уравнения доказать, что коэффициенты $a_{11}$ и $a_{22}$ не могут быть числами разных знаков и что они одновременно не могут обрашаться в нуль.
Доказать, что любое параболическое уравнение может быть написано в виде: $ (\alpha x+\beta y) ^2+2a_{13}x+2a_{23}y+a_{33}=0$. Доказать также, что эллиптические и гиперболические уравнения в таком виде не могут быть написаны.
Доказать, что если уравнение второй степени является параболическим и написано в виде $ (\alpha x+\beta y) ^2+2a_{13}x+2a_{23}y+a_{33}=0$ то дискриминант его левой части определяется формулой $\Delta=- (a_{13} \beta-a_{23} \alpha) ^2$.
Доказать, что параболическое уравнение определяет параболу в том и только в том случае, когда $\Delta \neq 0$. Доказать, что в этом случае параметр параболы определяется формулой $p=\sqrt{\frac{-\Delta}{ (a_{11}+a_{33}) ^3}}$.
Доказать, что уравнение второй степени является уравнением вырожденной линии в том и только в том случае, когда $\Delta=0$.
++++
Установить, при каких значениях $m$ плоскость $x+m z-1=0$ пересекает двухполостный гиперболоид $x^2+y^2-z^2=-1$ а) по эллипсу, б) по гиперболе.
Установить, при каких значениях $m$ плоскость $x+m y-2=0$ пересекает эллиптический параболоид $\frac{x^2}{2}+\frac{z^2}{3}=y$ а) по эллипсу, б) по параболе.
Доказать, что эллиптический параболоид $\frac{x^2}{9}+\frac{z^2}{4}=2 y$ имеет одну общую точку с плоскостью $2 x-2 y-z-10=0$, и найти ее координаты.
Доказать, что двухполостный гиперболоид $\frac{x^2}{3}+\frac{y^2}{4}-\frac{z^2}{25}=-1$ имеет одну общую точку с плоскостью $5 x+2 z+5=0$, и найти ее координаты.
Доказать, что эллипсоид $\frac{x^2}{81}+\frac{y^2}{36}+\frac{z^2}{9}=1$ имеет одну общую точку с плоскостью $4 x-3 y+12 z-54=0$, и найти ее координаты.
Определить, при каком значении $m$ плоскость $x-2 y-2 z+m=0$ касается эллипсоида $\frac{x^2}{144}+\frac{y^2}{36}+\frac{z^2}{9}=1$.