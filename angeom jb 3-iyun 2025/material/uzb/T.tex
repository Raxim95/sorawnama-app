Tekislikda ikkinchi tartibli chiziqlar (Ikkinchi tartibli tenglama, Kvadrat shakldagi tenglama, Konik chiziqlar (konuslar kesimi))
Parabola va uning kanonik tenglamalari (Fokus (yo’naluvchi nuqta), Direktrisa (yo’naltiruvchi chiziq), O’q (simmetriya o’qi))
++++
Ikkinchi tartibli chiziqlarning umumiy tenglamalari (Umumiy tenglama)
Ikkinchi tartibli chiziq va to‘g‘ri chiziqning o‘zaro vaziyati (Kesishish nuqtalari, Urinma (tegish) holat)
Ikkinchi tartibli chiziq urinmasi, qo‘shma diametri tenglamasi (Urinma tenglama, Qo‘shma diametr: markazdan o‘tuvchi simmetriya o‘qlari)
Ikkinchi tartibli chiziq markazi (Markazli chiziqlar (ellips, giperbola), Markaz koordinatalari: simmetriya markazi)
Ikkinchi tartibli chiziqlarning umumiy tenglamasini invariantlar yordamida kanonik ko‘rinishga keltirish
++++
Ikkinchi tartibli sirtlarning kanonik tenglamalari (Ellipsoid, Giperboloid (1 pallali), Giperboloid (2 pallali))
Ikkinchi tartibli sirtlarning kanonik tenglamalari (Paraboloid (elliptik), Paraboloid (giperbolik), Konus, Silindr)
Bir pallali giperboloid va giperbolik paraboloidning to‘g‘ri chiziqli yasovchilari (Giperboloid, Giperbolik paraboloid, Chiziqli yasovchilar)
Ikkinchi tartibli sirtlarning umumiy tenglamalari (Umumiy tenglama)
Ikkinchi tartibli sirtlarning umumiy tenglamasini kanonik ko‘rinishga invariantlar yordamida keltirish
Ikkinchi tartibli sirt markazi, urinma tekisligi va diametral tekisligi (Markaz, Urinma tekislik, Diametral tekislik)