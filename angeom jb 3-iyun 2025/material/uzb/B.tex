Ikkita uchi $x^2+5 y^2=20$ ellipsning fokuslarida yotuvchi, qolgan ikkitasi esa uning kichik o‘qi uchlari bilan ustma-ust tushuvchi to‘rtburchakning yuzi hisoblansin.
$\varepsilon=\frac{2}{3}$ ellipsning ekssentrisiteti, $M$ ellips nuqtasining fokal radiusi 10 ga teng. $M$ nuqtadan shu fokusga mos direktrisagacha bo‘lgan masofani hisoblang.
$\varepsilon=\frac{2}{5}$ ellipsning ekssentrisiteti, ellipsning $M$ nuqtasidan direktrisagacha bo‘lgan masofa 20 ga teng. $M$ nuqtadan shu direktrisa bilan bir tomonlama fokusgacha bo‘lgan masofani hisoblang.
Ellipsning ekssentrisiteti $\varepsilon=\frac{1}{3}$, uning markazi koordinatalar boshi bilan ustma-ust tushadi, fokuslaridan biri $ (-2; 0) $. Abssissasi 2 ga teng bo‘lgan ellipsning $M_1$ nuqtasidan berilgan fokusga mos direktrisagacha bo‘lgan masofani ayiring.
Ellipsning ekssentrisiteti $\varepsilon=\frac{1}{2}$, uning markazi koordinatalar boshi bilan ustma-ust tushadi, direktrisalardan biri $x=16$ tenglama bilan berilgan. Abssissasi -4 ga teng bo‘lgan ellipsning $M_1$ nuqtasidan berilgan direktrisa bilan bir tomonlama fokusgacha bo‘lgan masofani hisoblang.
$\frac{x^2}{25}+\frac{y^2}{15}=1$ ellipsning fokusi orqali uning katta o‘qiga perpendikulyar o‘tkazilgan. Bu perpendikulyarning ellips bilan kesishgan nuqtalaridan fokuslargacha bo‘lgan masofalar aniqlansin.
Ellipsdagi ekssentrisitetni aniqlang, agar: uning kichik o‘qi fokuslardan $60^{\circ}$ burchak ostida ko‘rinadi;
Ellipsdagi ekssentrisitetni aniqlang, agar: fokuslari orasidagi kesmaning o‘zi kichik o‘qning uchidan to‘g‘ri burchak ostida ko‘rinadi.;
Ellipsdagi ekssentrisitetni aniqlang, agar: direktrisalar orasidagi masofa fokuslar orasidagi masofadan uch marta katta.;
Ellipsdagi ekssentrisitetni aniqlang, agar: ellips markazidan uning direktrisasiga tushirilgan perpendikulyar kesmasi ellipsning uchi bilan teng ikkiga bo‘linadi.
Quyidagilarni bilgan holda ellips tenglamasini tuzing: uning katta o‘qi 26 ga teng va fokuslari $F_1 (-10; 0), F2 (14; 0) $;
Quyidagilarni bilgan holda ellips tenglamasini tuzing: uning kichik o‘qi 2 ga teng va fokuslari $F_1 (-1;-1) $, $F_2 (1; 1) $;
Quyidagilarni bilgan holda ellips tenglamasini tuzing: uning fokuslari $F_1\left(-2; \frac{3}{2}\right), F_2\left(2;-\frac{3}{2}\right) $ va ekssentrisitet $\varepsilon=\frac{\sqrt{2}}{2}$;
Quyidagilarni bilgan holda ellips tenglamasini tuzing: uning fokuslari $F_1 (1; 3), F_2 (3; 1) $ va direktrisalar orasidagi masofa $12 \sqrt{2}$ ga teng.
Agar ellipsning ekssentrisiteti $\varepsilon=\frac{1}{2}$ va fokusi $F(3 ; 0)$ va unga mos direktrisa tenglamasi $x+y-1=0$ ma’lum bo‘lsa, uning tenglamasi tuzilsin.
$M_1 (2;-1)$ nuqta fokusi $F (1;0)$ bo‘lgan ellipsda yotadi. Bu fokusga mos direktrisa esa $2x-y-10=0$ tenglama bilan berilgan. Shu ellipsning tenglamasi tuzilsin.
O'qlari koordinata o'qlari bilan ustma - ust tushuvchi va $P(2,2) ; Q(3,1)$ nuqtalar orqali o'tuvchi ellips tenglamasi tuzilsin.
Katta o'qi 2 birlikka teng, fokuslari $F_1(0,1), F_2(1,0)$ nuqtalarda bo'lgan ellipsning tenglamasi tuzilsin.
Ellips fokuslarining biridan katta o'qi uchlarigacha masofalar mos ravishda 7 va 1 ga teng. Bu ellips ning tenglamasini tuzing.
$\frac{x^2}{16}+\frac{y^2}{9}=1$ ellipsning $x+y-1=0$ to'g'ri chiziqqa parallel bo'lgan urinmalarini aniqlang.
++++
$\frac{x^2}{80}-\frac{y^2}{20}=1$ giperbolada $M_1 (10;-\sqrt{5}) $ nuqta berilgan. $M_1$ nuqtaning fokal radiuslari yotgan to‘g‘ri chiziqlarning tenglamalari tuzilsin.
$\frac{x^2}{64}-\frac{y^2}{36}=1$ giperbolaning o‘ng fokusigacha bo‘lgan masofasi 4,5 ga teng bo‘lgan nuqtalari aniqlansin.
Teng tomonli giperbolaning ekssentrisiteti aniqlansin.
Fokuslari $\frac{x^2}{100}+\frac{y^2}{64}=1$ ellipsning uchlarida yotuvchi, direktrisalari esa shu ellipsning fokuslaridan o‘tuvchi giperbolaning tenglamasi tuzilsin.
Quyidagilarni bilgan holda giperbola tenglamasini tuzing: uning uchlari orasidagi masofa 24 ga teng va fokuslari $F_1 (-10; 2), F_2 (16; 2) $;
Quyidagilarni bilgan holda giperbola tenglamasini tuzing: fokuslar $F_1 (3; 4), F_2 (-3;-4)$ va direktrisalar orasidagi masofa 3,6;
Quyidagilarni bilgan holda giperbola tenglamasini tuzing: Asimptotalar orasidagi burchak $90^{\circ}$ ga teng va fokuslar $F_1 (4;-4), F_2 (-2; 2) $.
Agar giperbolaning ekssentrisiteti $\varepsilon=\sqrt{5}$, fokusi $F (2;-3) $ va unga mos direktrisasining tenglamasi $3 x-y+3=0$ ma’lum bo‘lsa, uning tenglamasini tuzing.
$M_1 (1;-2) $ nuqta fokusi $F (-2; 2) $, unga mos direktrisa esa $2x-y-1=0$ tenglama bilan berilgan giperbolaga tegishli. Bu giperbolaning tenglamasi tuzilsin.
$\frac{x^2}{16}-\frac{y^2}{64}=1$ giperbolaga $10 x-3 y+9=0$ to‘g‘ri chiziqqa parallel bo‘lgan urinmalarning tenglamalarini tuzing.
$\frac{x^2}{16}-\frac{y^2}{8}=-1$ giperbolaga $2 x+4 y-5=0$ to‘g‘ri chiziqqa parallel urinmalar o‘tkazing va ular orasidagi $d$ masofani hisoblang.
Ushbu $x^2-y^2=16$ giperbolaga $A (-1;-7)$ nuqtadan o‘tkazilgan urinmalar tenglamasi tuzilsin.
Giperbolaning haqiqiy o'qiga perpendikular bo'lgan va giperbola fokusidan o'tgan vatar uzunligi topilsin.
$\frac{x^2}{49}+\frac{y^2}{24}=1$ ellips bilan fokusdosh va ekssentrisiteti $e=\frac{5}{4}$ bo'lgan giperbolaning tenglamasi yozilsin.
Giperbolaning yarim o'qlarini toping, agar: fokuslari orasidagi masofa 8 ga va direktrisalari orasidagi masofa 6 ga teng;
Giperbolaning yarim o'qlarini toping, agar: direktrisalari $x= \pm 3 \sqrt{2}$ tenglamalar bilan berilgan va asimptotalari orasidagi burchak - to'g'ri burchak;
Giperbolaning yarim o'qlarini toping, agar: asimptotalari $y= \pm 2 x$ tenglamalar bilan berilgan va fokuslari markazdan 5 birlik masofada;
Giperbolaning yarim o'qlarini toping, agar: asimptotalari $y= \pm \frac{5}{3} x$ tenglamalar bilan berilgan va giperbola $N(6,9)$ nuqtadan o'tadi.
Giperbolaning asimptotalari orasidagi burchagi topilsin, agar: ekssentrisiteti $e=2$;
Giperbolaning asimptotalari orasidagi burchagi topilsin, agar: fokuslari orasidagi masofa direktrisalari orasidagi masofadan ikki marta katta.
$\frac{x^2}{16}-\frac{y^2}{9}=1$ giperbolada fokal radiuslari o'zaro perpendikular bo'lgan nuqta topilsin.
$\frac{x^2}{16}-\frac{y^2}{9}=1$ giperbolada chap fokusgacha bo'lgan masofasi o'ng fokusgacha bo'lgan nuqta topilsin.
$\frac{x^2}{9}-\frac{y^2}{4}=1$ giperbolaning $M(5,1)$ nuqtada teng ikkiga bo'linadigan vatarining tenglamasi tuzilsin.
$\frac{x^2}{5}-\frac{y^2}{4}=1$ giperbolaga $(5,-4)$ nuqtada urinadigan to'g'ri chiziq tenglamasi yozilsin.
$x^2-y^2=8$ giperbolaga $M(3,-1)$ nuqtada urinadigan to'g'ri chiziq tenglamasi yozilsin.
++++
Agar parabolaning fokusi $F (7; 2) $ va direktrisa $x-5=0$ tenglamasi berilgan bo'lsa uning tenglamasini tuzing.
Agar parabolaning fokusi $F (4;3) $ va direktrisa $y+1=0$ tenglamasi berilgan bo'lsa uning tenglamasini tuzing.
Agar parabolaning fokusi $F(2;-1) $ va direktrisa $x-y-1=0$ tenglamasi berilgan bo'lsa uning tenglamasini tuzing.
Berilgan parabola uchi $A(6;-3)$ va uning direktrisasining tenglamasi $3x-5y+1=0$ berilgan. Ushbu parabolaning $F$ fokusini toping.
Parabola uchi $A(-2;-1)$ va uning direktrisasining tenglamasi $x+2y-1=0$ berilgan. Ushbu parabolaning tenglamasini tuzing.
$y^2=8x$ parabolaning $2x+2y-3=0$ to'g'ri chizig'iga parallel urinmasining tenglamasini tuzing.
$x^2=16y$ parabolaning $2x+4y+7=0$ to'g'ri chizig'iga perpendikulyar bo'lgan urinmasining tenglamasini tuzing.
$A(5;9)$ nuqtadan $y^2=5x$ parabolaga o'tkazilgan urinmalarning urinish nuqtalarini tutashtiruvchi xordaning tenglamasini tuzing.
Parabola uchining koordinatalari, parametri va o'qining yo'nalishi aniqlansin: $y^2-10 x-2 y-19=0$;
Parabola uchining koordinatalari, parametri va o'qining yo'nalishi aniqlansin: $y^2-6 x+14 y+49=0$,
Parabola uchining koordinatalari, parametri va o'qining yo'nalishi aniqlansin: $y^2+8 x-16=0$,
Parabola uchining koordinatalari, parametri va o'qining yo'nalishi aniqlansin: $x^2-6 x-4 y+29=0$,
Parabola uchining koordinatalari, parametri va o'qining yo'nalishi aniqlansin: $y=A x^2+B x+C$,
Parabola uchining koordinatalari, parametri va o'qining yo'nalishi aniqlansin: $y=x^2-8 x+15$,
Parabola uchining koordinatalari, parametri va o'qining yo'nalishi aniqlansin: $y=x^2+6 x$.
Beshta nuqtadan o'tuvchi ikkinchi tartibli chiziqning tenglamasi tuzilsin: $(0,0),(0,1),(1,0),(2,-5),(-5,2)$.
$5 x^2-3 x y+y^2-3 x+2 y-5=0$ chiziqning $x-2 y-1=0$ to'g'ri chiziq bilan kesishishidan hosil qilingan vatarning o'rtasidan o'tadigan diametr tenglamasi yozilsin.
++++
Ushbu chiziqlar markaziy ekanligini ko'rsating va har bir chiziq uchun markaz koordinatalarini toping: $3x^2+5xy+y^2-8x-11y-7=0$.
Ushbu chiziqlar markaziy ekanligini ko'rsating va har bir chiziq uchun markaz koordinatalarini toping: $5 x^2+4 x y+2 y^2+20 x+20 y-18=0$;
Ushbu chiziqlar markaziy ekanligini ko'rsating va har bir chiziq uchun markaz koordinatalarini toping: $9 x^2-4 x y-7 y^2-12=0$;
Ushbu chiziqlar markaziy ekanligini ko'rsating va har bir chiziq uchun markaz koordinatalarini toping: $2 x^2-6 x y+5 y^2+22 x-36 y+11=0$.
Ushbu tenglamalar markaziy chiziqlarni ifodalashini ko‘rsating va har bir tenglamani koordinatalar boshini markazga ko‘chirgan holda o‘zgartiring: $3x^2-6xy+2y^2-4x+2y+1=0$.
Ushbu tenglamalar markaziy chiziqlarni ifodalashini ko‘rsating va har bir tenglamani koordinatalar boshini markazga ko‘chirgan holda o‘zgartiring: $6 x^2+4 x y+y^2+4 x-2 y+2=0$;
Ushbu tenglamalar markaziy chiziqlarni ifodalashini ko‘rsating va har bir tenglamani koordinatalar boshini markazga ko‘chirgan holda o‘zgartiring: $4 x^2+6 x y+y^2-10 x-10=0$;
Ushbu tenglamalar markaziy chiziqlarni ifodalashini ko‘rsating va har bir tenglamani koordinatalar boshini markazga ko‘chirgan holda o‘zgartiring: $4 x^2+2 x y+6 y^2+6 x-10 y+9=0$.
++++
Quyidagi tenglamaning tipini aniqlang, koordinata o‘qlarini parallel ko‘chirish orqali sodda shaklga keltiring; qanday geometrik obrazni ifodalashini aniqlang va eski hamda yangi koordinata o‘qlariga nisbatan chizmada tasvirlang: $4 x^2+9 y^2-40 x+36 y+100=0$;
Quyidagi tenglamaning tipini aniqlang, koordinata o‘qlarini parallel ko‘chirish orqali sodda shaklga keltiring; qanday geometrik obrazni ifodalashini aniqlang va eski hamda yangi koordinata o‘qlariga nisbatan chizmada tasvirlang: $9 x^2-16 y^2-54 x-64 y-127=0$;
Quyidagi tenglamaning tipini aniqlang, koordinata o‘qlarini parallel ko‘chirish orqali sodda shaklga keltiring; qanday geometrik obrazni ifodalashini aniqlang va eski hamda yangi koordinata o‘qlariga nisbatan chizmada tasvirlang: $9 x^2+4 y^2+18 x-8 y+49=0$;
Quyidagi tenglamaning tipini aniqlang, koordinata o‘qlarini parallel ko‘chirish orqali sodda shaklga keltiring; qanday geometrik obrazni ifodalashini aniqlang va eski hamda yangi koordinata o‘qlariga nisbatan chizmada tasvirlang: $4 x^2-y^2+8 x-2 y+3=0$;
Quyidagi tenglamaning tipini aniqlang, koordinata o‘qlarini parallel ko‘chirish orqali sodda shaklga keltiring; qanday geometrik obrazni ifodalashini aniqlang va eski hamda yangi koordinata o‘qlariga nisbatan chizmada tasvirlang: $2 x^2+3 y^2+8 x-6 y+11=0$.
Berilgan tenglamani sodda shaklga keltiring; tipini aniqlang; qanday geometrik obrazni ifodalashini aniqlang, eski hamda yangi koordinata o‘qlariga nisbatan chizmada tasvirlang: $32x^2+52xy-7y^2+180=0$.: $32 x^2+52 x y-7 y^2+180=0$;
Berilgan tenglamani sodda shaklga keltiring; tipini aniqlang; qanday geometrik obrazni ifodalashini aniqlang, eski hamda yangi koordinata o‘qlariga nisbatan chizmada tasvirlang: $32x^2+52xy-7y^2+180=0$.: $5 x^2-6 x y+5 y^2-32=0$;
Berilgan tenglamani sodda shaklga keltiring; tipini aniqlang; qanday geometrik obrazni ifodalashini aniqlang, eski hamda yangi koordinata o‘qlariga nisbatan chizmada tasvirlang: $32x^2+52xy-7y^2+180=0$.: $17 x^2-12 x y+8 y^2=0$;
Berilgan tenglamani sodda shaklga keltiring; tipini aniqlang; qanday geometrik obrazni ifodalashini aniqlang, eski hamda yangi koordinata o‘qlariga nisbatan chizmada tasvirlang: $32x^2+52xy-7y^2+180=0$.: $5 x^2+24 x y-5 y^2=0$;
Berilgan tenglamani sodda shaklga keltiring; tipini aniqlang; qanday geometrik obrazni ifodalashini aniqlang, eski hamda yangi koordinata o‘qlariga nisbatan chizmada tasvirlang: $32x^2+52xy-7y^2+180=0$.: $5 x^2-6 x y+5 y^2+8=0$.
ITECH turi, o'lchovlari va joylashishi aniqlansin: $5 x^2+4 x y+8 y^2-32 x-56 y+80=0$.
ITECH turi, o'lchovlari va joylashishi aniqlansin: $9 x^2+24 x y+16 y^2-230 x+110 y-475=0$.
ITECH turi, o'lchovlari va joylashishi aniqlansin: $5 x^2+12 x y-12 x-22 y-19=0$.
ITECH turi, o'lchovlari va joylashishi aniqlansin: $x^2-2 x y+y^2-10 x-6 y+25=0$.
ITECH turi, o'lchovlari va joylashishi aniqlansin: $x^2-5 x y+4 y^2+x+2 y-2=0$.
ITECH turi, o'lchovlari va joylashishi aniqlansin: $4 x^2-12 x y+9 y^2-2 x+3 y-2=0$.
ITECH turi, o'lchovlari va joylashishi aniqlansin: $2 x^2+4 x y+5 y^2-6 x-8 y-1=0$;
ITECH turi, o'lchovlari va joylashishi aniqlansin: $5 x^2+8 x y+5 y^2-18 x-18 y+9=0$;
ITECH turi, o'lchovlari va joylashishi aniqlansin: $5 x^2+6 x y+5 y^2-16 x-16 y-16=0$;
ITECH turi, o'lchovlari va joylashishi aniqlansin: $6 x y-8 y^2+12 x-26 y-11=0$;
ITECH turi, o'lchovlari va joylashishi aniqlansin: $7 x^2+16 x y-23 y^2-14 x-16 y-218=0$;
ITECH turi, o'lchovlari va joylashishi aniqlansin: $7 x^2-24 x y-38 x+24 y+175=0$;
ITECH turi, o'lchovlari va joylashishi aniqlansin: $9 x^2+24 x y+16 y^2-40 x-30 y=0$;
ITECH turi, o'lchovlari va joylashishi aniqlansin: $x^2+2 x y+y^2-8 x+4=0$;
ITECH turi, o'lchovlari va joylashishi aniqlansin: $4 x^2-4 x y+y^2-2 x-14 y+7=0$.
++++
Berilgan tenglama parabolik ekanligini ko'rsating; sodda shaklga keltiring; qanday geometrik obrazni ifodalashini aniqlang, eski hamda yangi koordinata o‘qlariga nisbatan chizmada tasvirlang: $9 x^2-24 x y+16 y^2-20 x+110 y-50=0$;
Berilgan tenglama parabolik ekanligini ko'rsating; sodda shaklga keltiring; qanday geometrik obrazni ifodalashini aniqlang, eski hamda yangi koordinata o‘qlariga nisbatan chizmada tasvirlang: $9 x^2+12 x y+4 y^2-24 x-16 y+3=0$;
Berilgan tenglama parabolik ekanligini ko'rsating; sodda shaklga keltiring; qanday geometrik obrazni ifodalashini aniqlang, eski hamda yangi koordinata o‘qlariga nisbatan chizmada tasvirlang: $16 x^2-24 x y+9 y^2-160 x+120 y+425=0$.
Berilgan tenglama parabolik ekanligini ko'rsating; sodda shaklga keltiring; qanday geometrik obrazni ifodalashini aniqlang, eski hamda yangi koordinata o‘qlariga nisbatan chizmada tasvirlang: $9 x^2+24 x y+16 y^2-18 x+226 y+209=0$;
Berilgan tenglama parabolik ekanligini ko'rsating; sodda shaklga keltiring; qanday geometrik obrazni ifodalashini aniqlang, eski hamda yangi koordinata o‘qlariga nisbatan chizmada tasvirlang: $x^2-2 x y+y^2-12 x+12 y-14=0$
Berilgan tenglama parabolik ekanligini ko'rsating; sodda shaklga keltiring; qanday geometrik obrazni ifodalashini aniqlang, eski hamda yangi koordinata o‘qlariga nisbatan chizmada tasvirlang: $4 x^2+12 x y+9 y^2-4 x-6 y+1=0$.
Berilgan tenglamalarning parabolik ekanligini ko‘rsating va ularning har birini $(\alpha x+\beta y)^2+2 a_{13} x+2 a_{23} y+a_{33}=0$ ko‘rinishda yozing: $x^2+4 x y+4 y^2+4 x+y-15=0 ;$
Berilgan tenglamalarning parabolik ekanligini ko‘rsating va ularning har birini $(\alpha x+\beta y)^2+2 a_{13} x+2 a_{23} y+a_{33}=0$ ko‘rinishda yozing: $9 x^2-6 x y+y^2-x+2 y-14=0$;
Berilgan tenglamalarning parabolik ekanligini ko‘rsating va ularning har birini $(\alpha x+\beta y)^2+2 a_{13} x+2 a_{23} y+a_{33}=0$ ko‘rinishda yozing: $25 x^2-20 x y+4 y^2+3 x-y+11=0$;
Berilgan tenglamalarning parabolik ekanligini ko‘rsating va ularning har birini $(\alpha x+\beta y)^2+2 a_{13} x+2 a_{23} y+a_{33}=0$ ko‘rinishda yozing: $16 x^2+16 x y+4 y^2-5 x+7 y=0$;
Berilgan tenglamalarning parabolik ekanligini ko‘rsating va ularning har birini $(\alpha x+\beta y)^2+2 a_{13} x+2 a_{23} y+a_{33}=0$ ko‘rinishda yozing: $9 x^2-42 x y+49 y^2+3 x-2 y-24=0$.
++++
$\frac{x^2}{12}+\frac{y^2}{4}+\frac{z^2}{3}=1$ ellipsoidi va $2x-3y+4z-11=0$ tekisligining kesishish chizig‘i qanday chiziq ekanligini aniqlang va uning markazini toping.
$\frac{x^2}{2}-\frac{z^2}{3}=y$ giperbolik paraboloidi va $3x-3y+4z+2=0$ tekisligining kesish chizig‘i qanday chiziq ekanligini aniqlang va uning markazini toping.
Lagranj usulidan foydalanib, tenglamalarni kvadratlar yig'indisi shakliga keltirib, quyidagi sirtlarning ko'rinishi aniqlansin: $4 x^2+6 y^2+4 z^2+4 x z-8 y-4 z+3=0$;
Lagranj usulidan foydalanib, tenglamalarni kvadratlar yig'indisi shakliga keltirib, quyidagi sirtlarning ko'rinishi aniqlansin: $x^2+5 y^2+z^2+2 x y+6 x z+2 y z-2 x+6 y-10 z=0$;
Lagranj usulidan foydalanib, tenglamalarni kvadratlar yig'indisi shakliga keltirib, quyidagi sirtlarning ko'rinishi aniqlansin: $x^2+y^2-3 z^2-2 x y-6 x z-6 y z+2 x+2 y+4 z=0$;
Lagranj usulidan foydalanib, tenglamalarni kvadratlar yig'indisi shakliga keltirib, quyidagi sirtlarning ko'rinishi aniqlansin: $x^2-2 y^2+z^2+4 x y-8 x z-4 y z-14 x-4 y+14 z+16=0$;
Lagranj usulidan foydalanib, tenglamalarni kvadratlar yig'indisi shakliga keltirib, quyidagi sirtlarning ko'rinishi aniqlansin: $2 x^2+y^2+2 z^2-2 x y-2 y z+x-4 y-3 z+2=0$;
Lagranj usulidan foydalanib, tenglamalarni kvadratlar yig'indisi shakliga keltirib, quyidagi sirtlarning ko'rinishi aniqlansin: $x^2-2 y^2+z^2+4 x y-10 x z+4 y z+x+y-z=0$;
Lagranj usulidan foydalanib, tenglamalarni kvadratlar yig'indisi shakliga keltirib, quyidagi sirtlarning ko'rinishi aniqlansin: $2 x^2+y^2+2 z^2-2 x y-2 y z+4 x-2 y=0$;
Lagranj usulidan foydalanib, tenglamalarni kvadratlar yig'indisi shakliga keltirib, quyidagi sirtlarning ko'rinishi aniqlansin: $x^2-2 y^2+z^2+4 x y-10 x z+4 y z+2 x+4 y-10 z-1=0$;
Lagranj usulidan foydalanib, tenglamalarni kvadratlar yig'indisi shakliga keltirib, quyidagi sirtlarning ko'rinishi aniqlansin: $x^2+y^2+4 z^2+2 x y+4 x z+4 y z-6 z+1=0$;
Lagranj usulidan foydalanib, tenglamalarni kvadratlar yig'indisi shakliga keltirib, quyidagi sirtlarning ko'rinishi aniqlansin: $4 x y+2 x+4 y-6 z-3=0$;
Lagranj usulidan foydalanib, tenglamalarni kvadratlar yig'indisi shakliga keltirib, quyidagi sirtlarning ko'rinishi aniqlansin: $x y+x z+y z+2 x+2 y-2 z=0$.
Parallel ko'chirish va burish almashtirishlari yoki hadlarni gruppalash yordamida quyidagi sirtlarning ko'rinishi va joylashishi aniqlansin: $z=2 x^2-4 y^2-6 x+8 y+1$;
Parallel ko'chirish va burish almashtirishlari yoki hadlarni gruppalash yordamida quyidagi sirtlarning ko'rinishi va joylashishi aniqlansin: $z=x^2+3 y^2-6 y+1$;
Parallel ko'chirish va burish almashtirishlari yoki hadlarni gruppalash yordamida quyidagi sirtlarning ko'rinishi va joylashishi aniqlansin: $x^2+2 y^2-3 z^2+2 x+4 y-6 z=0$;
Parallel ko'chirish va burish almashtirishlari yoki hadlarni gruppalash yordamida quyidagi sirtlarning ko'rinishi va joylashishi aniqlansin: $x^2+2 x y+y^2-z^2=0$;
Parallel ko'chirish va burish almashtirishlari yoki hadlarni gruppalash yordamida quyidagi sirtlarning ko'rinishi va joylashishi aniqlansin: $z^2=3 x+4 y+5$;
Parallel ko'chirish va burish almashtirishlari yoki hadlarni gruppalash yordamida quyidagi sirtlarning ko'rinishi va joylashishi aniqlansin: $z=x^2+2 x y+y^2+1$;
Parallel ko'chirish va burish almashtirishlari yoki hadlarni gruppalash yordamida quyidagi sirtlarning ko'rinishi va joylashishi aniqlansin: $z^2=x^2+2 x y+y^2+1$;
Parallel ko'chirish va burish almashtirishlari yoki hadlarni gruppalash yordamida quyidagi sirtlarning ko'rinishi va joylashishi aniqlansin: $x^2+4 y^2+9 z^2-6 x+8 y-18 z-14=0$;
Parallel ko'chirish va burish almashtirishlari yoki hadlarni gruppalash yordamida quyidagi sirtlarning ko'rinishi va joylashishi aniqlansin: $2 x y+z^2-2 z+1=0$;
Parallel ko'chirish va burish almashtirishlari yoki hadlarni gruppalash yordamida quyidagi sirtlarning ko'rinishi va joylashishi aniqlansin: $x^2+y^2-z^2-2 x y+2 z-1=0$;
Parallel ko'chirish va burish almashtirishlari yoki hadlarni gruppalash yordamida quyidagi sirtlarning ko'rinishi va joylashishi aniqlansin: $x^2+4 y^2-z^2-10 x-16 y+6 z+16=0$;
Parallel ko'chirish va burish almashtirishlari yoki hadlarni gruppalash yordamida quyidagi sirtlarning ko'rinishi va joylashishi aniqlansin: $2 x y+2 x+2 y+2 z-1=0$;
Parallel ko'chirish va burish almashtirishlari yoki hadlarni gruppalash yordamida quyidagi sirtlarning ko'rinishi va joylashishi aniqlansin: $3 x^2+6 x-8 y+6 z-7=0$;
Parallel ko'chirish va burish almashtirishlari yoki hadlarni gruppalash yordamida quyidagi sirtlarning ko'rinishi va joylashishi aniqlansin: $x^2+y^2+2 z^2+2 x y+4 z=0$;
Parallel ko'chirish va burish almashtirishlari yoki hadlarni gruppalash yordamida quyidagi sirtlarning ko'rinishi va joylashishi aniqlansin: $3 x^2+3 y^2+3 z^2-6 x+4 y-1=0$;
Parallel ko'chirish va burish almashtirishlari yoki hadlarni gruppalash yordamida quyidagi sirtlarning ko'rinishi va joylashishi aniqlansin: $3 x^2+3 y^2-6 x+4 y-1=0$;
Parallel ko'chirish va burish almashtirishlari yoki hadlarni gruppalash yordamida quyidagi sirtlarning ko'rinishi va joylashishi aniqlansin: $3 x^2+3 y^2-3 z^2-6 x+4 y+4 z+3=0$;
Parallel ko'chirish va burish almashtirishlari yoki hadlarni gruppalash yordamida quyidagi sirtlarning ko'rinishi va joylashishi aniqlansin: $4 x^2-y^2-4 x+4 y-3=0$;