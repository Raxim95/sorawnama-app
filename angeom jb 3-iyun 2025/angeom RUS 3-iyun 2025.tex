\documentclass{article}
\usepackage[fontsize=12pt]{fontsize}
\usepackage[utf8]{inputenc}
\usepackage[T2A]{fontenc}
% \usepackage{unicode-math}

\usepackage{array}
\usepackage[a4paper,
left=7mm,
right=5mm,
top=7mm,]{geometry}
\usepackage{amsmath}
% \usepackage{amssymbol}
\usepackage{amsfonts}
\usepackage{setspace}
\onehalfspacing



\renewcommand{\baselinestretch}{1} 

\everymath{\displaystyle}
\everydisplay{\displaystyle}
% \linespread{1.25}

\DeclareMathOperator{\sign}{sign}


\begin{document}

\pagenumbering{gobble}


\begin{tabular}{m{17cm}}
\textbf{1-variant}
\newline

\textbf{T1.} Канонические уравнения поверхностей второго порядка (Параболоид (эллиптический), Параболоид (гиперболический), Конус, Цилиндр) \\
\textbf{T2.} Парабола и её канонические уравнения (Фокус (направляющая точка), Директриса (направляющая линия), Ось (ось симметрии)) \\
\textbf{A1.} Стальной трос подвешен за два конца; точки креп.тения расположены на одинаковой высоте; расстояние между ними равно 20 м. Величина его прогиба на расстолиии 2 m от точки крепления, считая по горизонтали, равна 14,4 см. Определить величину прогиба этого троса в ссредине между точками крепления, приближенно считая, что трос имеет форму дуги параболы. \\
\textbf{A2.} Не проводя преобразования координат, установить, что каждое из следующих уравнений определяет единственную точку (вырожденный эллипс), и найти ее координаты: $x^2+2 x y+2 y^2+6 y+9=0$; \\
\textbf{A3.} Установить, что плоскость $y+6=0$ пересекает гиперболический параболоид $\frac{x^2}{5}-\frac{y^2}{4}=6 z$ по параболе; найти ее параметр и вершину. \\
\textbf{B1.} Установить, какая линия является сечением гиперболического параболоида $\frac{x^2}{2}-\frac{z^2}{3}=y$ плоскостью $3 x-3 y+4 z+2=0$, и найти ее центр. \\
\textbf{B2.} Составить уравнение параболы, если даны ее фокус $F(4 ; 3)$ и директриса $y+1=0$. \\
\textbf{B3.} Определить тип каждого из следующих уравнений каждое из них путем параллельного переноса осей координат привести к простейшему виду; установить, какие геометрические образы они определяют, и изобразить на чертеже расположение этих образов относительно старых и новых осей координат: $9 x^2-16 y^2-54 x-64 y-127=0$; \\
\textbf{C1.} Определить, при каких значениях $m$ прямая $y=-x+m$ 1) пересекает эллипс $\frac{x^2}{20}+\frac{y^2}{5}=1$; 2) касается его; 3) проходит вне этого эллипса. \\
\textbf{C2.} Для любого параболического уравнения доказать, что коэффициенты $a_{11}$ и $a_{22}$ не могут быть числами разных знаков и что они одновременно не могут обрашаться в нуль. \\
\textbf{C3.} Установить, при каких значениях $m$ плоскость $x+m z-1=0$ пересекает двухполостный гиперболоид $x^2+y^2-z^2=-1$ а) по эллипсу, б) по гиперболе. \\

\end{tabular}
\vspace{1cm}


\begin{tabular}{m{17cm}}
\textbf{2-variant}
\newline

\textbf{T1.} Приведение общего уравнения кривой второго порядка к каноническому виду с помощью инвариантов \\
\textbf{T2.} Взаимное расположение линии второго порядка и прямой (Точки пересечения, касательное положение) \\
\textbf{A1.} Установить, какие из следующих линий являются центральными (т.е. имеют единственный центр), какие имеют центра, какие имеют бесконечно много центров: $4 x^2-6 x y-9 y^2+3 x-7 y+12=0$. \\
\textbf{A2.} Составить уравнение гиперболы, фокусы которой лежат на оси абсцисс симметрично относительно начала координат, если даны: точка $M_1(-5 ; 3)$ гиперболы и эксцентриситет $\varepsilon=\sqrt{2}$; \\
\textbf{A3.} Не проводя преобразования координат, установить, что каждое из следующих уравнений определяет параболу, и найти параметр этой параболы: $x^2-2 x y+y^2+6 x-14 y+29=0$; \\
\textbf{B1.} Установить, что следующие уравнения являются параболическими, и записать каждое из них в виде $(\alpha x+\beta y)^2+2 a_{13} x+2 a_{23} y+a_{33}=0$: $9 x^2-42 x y+49 y^2+3 x-2 y-24=0$. \\
\textbf{B2.} Составить уравнение гиперболы, если известны ее эксцентриситет $\varepsilon=\sqrt{5}$, фокус $F(2 ;-3)$ и уравнение соответствующей директрисы $3 x-y+3=0$. \\
\textbf{B3.} Составить уравнение эллипса, зная, что: его малая ось равна 2 и фокусы суть $F_1(-1 ;-1)$, $F_2(1 ; 1)$; \\
\textbf{C1.} Доказать, что расстояние от фокуса гиперболы $\frac{x^2}{a^2}-\frac{y^2}{b^2}=1$ до ее асимптоты равно $b$. \\
\textbf{C2.} Доказать, что если две параболы со взаимно перпендикулярными осями пересекаются в четырех точках, то эти точки лежат на одной окружности. \\
\textbf{C3.} Каждое из следующих уравнений привести к каноническому виду; определить тип каждого из них; установить, какие геометрические образы они определяют; для каждого случая изобразить на чертеже оси первоначальной координатной системы, оси других координатных систем, которые вводятся по ходу решения, и геометрический образ, определяемый данным уравнением: $41 x^2+24 x y+9 y^2+24 x+18 y-36=0$. \\

\end{tabular}
\vspace{1cm}


\begin{tabular}{m{17cm}}
\textbf{3-variant}
\newline

\textbf{T1.} Линии второго порядка на плоскости (Уравнение второго порядка, Уравнение квадратной формы, Конические линии (сечение конусов)) \\
\textbf{T2.} Приведение общего уравнения поверхности второго порядка к каноническому виду с помощью инвариантов \\
\textbf{A1.} Составить уравнение эллипса, фокусы которого расположены на оси абсцисс, симметрично относительно начала координат, если даны: точка $M_1(\sqrt{15} ;-1)$ эллипса и расстояние между его фокусами $2 c=8$; \\
\textbf{A2.} Установить, какие линии определяются следующими уравнениями: $\left\{\begin{array}{l}\frac{x^2}{3}+\frac{y^2}{6}=2 z, \\ 3 x-y+6 z-14=0\end{array}\right.$ \\
\textbf{A3.} Установить, какие из следующих линий являются центральными (т.е. имеют единственный центр), какие имеют центра, какие имеют бесконечно много центров: $4 x^2+5 x y+3 y^2-x+9 y-12=0$; \\
\textbf{B1.} Установить, что следующие линии являются центральными, и для каждой из них найти координаты центра: $2 x^2-6 x y+5 y^2+22 x-36 y+11=0$. \\
\textbf{B2.} Установить, что следующие уравнения определяют центральные линии; преобразовать каждое из них путем переноса начала координат в центр: $3 x^2-6 x y+2 y^2-4 x+2 y+1=0$; \\
\textbf{B3.} Установить, что каждое из следующих уравнений является параболическим; каждое из них привести к простейшему виду; установить, какие геометрические образы они определяют; для каждого случая изобразить на чертеже оси первоначальной координатной системы, оси других координатных систем, которые вводятся по ходу решения, и геометрический образ, определяемый данным уравнением: $16 x^2-24 x y+9 y^2-160 x+120 y+425=0$. \\
\textbf{C1.} Дано уравнение линии $4 x^2-4 x y+y^2+6 x+1=0$. Определить, при каких значениях углового коэффициента $k$ прямая $y=k x:$ 1) пересекает эту линию в одной точке; 2) касается этой линии; 3) пересекает эту линию в двух точках; 4) не имеет общих точек с этой линией. \\
\textbf{C2.} Доказать, что любое параболическое уравнение может быть написано в виде: $ (\alpha x+\beta y) ^2+2a_{13}x+2a_{23}y+a_{33}=0$. Доказать также, что эллиптические и гиперболические уравнения в таком виде не могут быть написаны. \\
\textbf{C3.} Каждое из следующих уравнений привести к каноническому виду; определить тип каждого из них; установить, какие геометрические образы они определяют; для каждого случая изобразить на чертеже оси первоначальной координатной системы, оси других координатных систем, которые вводятся по ходу решения, и геометрический образ, определяемый данным уравнением: $25 x^2-14 x y+25 y^2+64 x-64 y-224=0$; \\

\end{tabular}
\vspace{1cm}


\begin{tabular}{m{17cm}}
\textbf{4-variant}
\newline

\textbf{T1.} Парабола и её канонические уравнения (Фокус (направляющая точка), Директриса (направляющая линия), Ось (ось симметрии)) \\
\textbf{T2.} Общие уравнения линий второго порядка (Общее уравнение) \\
\textbf{A1.} Составить уравнение гиперболы, фокусы которой расположены на оси ординат симметрично относительно начала координат, зная, кроме того, что: расстояние между фокусами $2 c=10$ и эксцентриситет $\varepsilon=\frac{5}{3}$; \\
\textbf{A2.} Точка $C(-3 ; 2)$ является центром эллипса, касающегося обеих координатных осей. Составить уравнение этого эллипса, зная, что его оси симметрии параллельны координатным осям. \\
\textbf{A3.} Не проводя преобразования координат, установить, какие геометрические образы определяются следующими уравнениями: $2 x^2+3 x y-2 y^2+5 x+10 y=0$; \\
\textbf{B1.} Каждое из следующих уравнений привести к простейшему виду; определить тип каждого из них; установить, какие геометрические образы они определяют, и изобразить на чертеже расположение этих образов относительно старых и новых осей координат: $17 x^2-12 x y+8 y^2=0$; \\
\textbf{B2.} Даны вершина параболы $A(-2 ;-1)$ и урав нение ее директрисы $x+2 y-1=0$. Составить уравнение этой параболы. \\
\textbf{B3.} Установить, какая линия является сечением эллипсоида $\frac{x^2}{12}+\frac{y^2}{4}+\frac{z^2}{3}=1$ плоскостью $2 x-3 y+4 z-11=0$, и найти ее центр. \\
\textbf{C1.} Определить, при каком значении $m$ плоскость $x-2 y-2 z+m=0$ касается эллипсоида $\frac{x^2}{144}+\frac{y^2}{36}+\frac{z^2}{9}=1$. \\
\textbf{C2.} Доказать что произведение расстояний от любой точки гиперболы $\frac{x^2}{a^2}-\frac{y^2}{b^2}=1$ до двух ее асимптот есть величина постоянная, равная $\frac{a^2 b^2}{a^2+b^2}$. \\
\textbf{C3.} При каких значениях $m$ и $n$ уравнение $m x^2+12 x y+9 y^2+4 x+n y-13=0$ определяет: 1) центральную линию; 2) линию без центра; 3) линию, имеющую бесконечно много центров. \\

\end{tabular}
\vspace{1cm}


\begin{tabular}{m{17cm}}
\textbf{5-variant}
\newline

\textbf{T1.} Канонические уравнения поверхностей второго порядка (эллипсоид, гиперболоид (1-полостный), гиперболоид (2-полостный)) \\
\textbf{T2.} Центр линии второго порядка (Центровые линии (эллипс, гипербола), Координаты центра: центр симметрии) \\
\textbf{A1.} Составить уравнение параболы, вершина которой находится в начале координат, зная, что: парабола расположена в нижней полуплоскости симметрично относительно оси $O y$, и ее параметр $p=3$. \\
\textbf{A2.} Не проводя преобразования координат, установить, что каждое из следующих уравнений определяет параболу, и найти параметр этой параболы: $9 x^2-6 x y+y^2-50 x+50 y-275=0$. \\
\textbf{A3.} Составить уравнение эллипса, фокусы которого расположены на оси абсцисс, симметрично относительно начала координат, если даны: точка $M_1(8 ; 12)$ эллипса и расстояние $r_1=20$ от нее до левого фокуса; \\
\textbf{B1.} Определить эксцентриситет в эллипса, если: отрезок перпендикуляра, опущенного из центра эллипса на его директрису, делится вершиной эллипса пополам. \\
\textbf{B2.} Дана точка $M_1(10 ;-\sqrt{5})$ на гиперболе $\frac{x^2}{80}-\frac{y^2}{20}=1$. Составить уравнения прямых, на которых лежат фокальные радиусы точки $M_1$. \\
\textbf{B3.} Каждое из следующих уравнений привести к простейшему виду; определить тип каждого из них; установить, какие геометрические образы они определяют, и изобразить на чертеже расположение этих образов относительно старых и новых осей координат: $5 x^2+24 x y-5 y^2=0$; \\
\textbf{C1.} Составить уравнение касательной к эллипсу $\frac{x^2}{a^2}+\frac{y^2}{b^2}=1$ в его точке $M_1\left(x_1 ; y_1\right)$. \\
\textbf{C2.} Вывести условие, при котором прямая $y=k x+b$ касается параболы $y^2=2 p x$. \\
\textbf{C3.} Доказать, что касательные к гиперболе, проведенные в концах одного и того же диаметра, параллельны. \\

\end{tabular}
\vspace{1cm}


\begin{tabular}{m{17cm}}
\textbf{6-variant}
\newline

\textbf{T1.} Центр, касательная плоскость и диаметральная плоскость поверхности второго порядка (Центр, касательная плоскость, диаметральная плоскость) \\
\textbf{T2.} Линии второго порядка на плоскости (Уравнение второго порядка, Уравнение квадратной формы, Конические линии (сечение конусов)) \\
\textbf{A1.} Установить, что каждая из следующих линий имєет бесконечно много цєнтров; для каждой их них составить уравнение геометрического места центров: $x^2-6 x y+9 y^2-12 x+36 y+20=0$; \\
\textbf{A2.} Установить, что плоскость $z+1=0$ пересекает однополостный гиперболоид $\frac{x^2}{32}-\frac{y^2}{18}+\frac{z^2}{2}=1$ по гиперболе; найти ее полуоси и вершины. \\
\textbf{A3.} Не проводя преобразования координат, установить, какие геометрические образы определяются следующими уравнениями: $8 x^2-12 x y+17 y^2+16 x-12 y+3=0$; \\
\textbf{B1.} Точка $M_1(2 ;-1)$ лежит на эллипсе, фокус которого $F(1 ; 0)$, а соответствующая директриса дана уравнением $2 x-y-10=0$. Составить уравнение этого эллипса. \\
\textbf{B2.} Установить, какая линия является сечением гиперболического параболоида $\frac{x^2}{2}-\frac{z^2}{3}=y$ плоскостью $3 x-3 y+4 z+2=0$, и найти ее центр. \\
\textbf{B3.} Составить уравнение параболы, если даны ее фокус $F(2 ;-1)$ и директриса $x-y-1=0$. \\
\textbf{C1.} Доказать, что касательные к эллипсу $\frac{x^2}{a^2}+\frac{y^2}{b^2}=1$, проведенные в концах одного и того же диаметра, параллельны. (Диаметром эллипса называется его хорда, проходящая через центр.) \\
\textbf{C2.} Установить, при каких значениях $m$ плоскость $x+m y-2=0$ пересекает эллиптический параболоид $\frac{x^2}{2}+\frac{z^2}{3}=y$ а) по эллипсу, б) по параболе. \\
\textbf{C3.} Доказать, что если уравнение второй степени является параболическим и написано в виде $ (\alpha x+\beta y) ^2+2a_{13}x+2a_{23}y+a_{33}=0$ то дискриминант его левой части определяется формулой $\Delta=- (a_{13} \beta-a_{23} \alpha) ^2$. \\

\end{tabular}
\vspace{1cm}


\begin{tabular}{m{17cm}}
\textbf{7-variant}
\newline

\textbf{T1.} Уравнение касательной линии второго порядка, сопряжённого диаметра (Уравнение касательной, сопряжённый диаметр: оси симметрии, проходящие через центр) \\
\textbf{T2.} Парабола и её канонические уравнения (Фокус (направляющая точка), Директриса (направляющая линия), Ось (ось симметрии)) \\
\textbf{A1.} Составить уравнение гиперболы, фокусы когорой расположены на оси абсцисс симметрично относительно начала координат, зная, кроме того, что: уравнения асимптот $y= \pm \frac{3}{4} x$ и расстояние между директрисами равно $12 \frac{4}{5}$. \\
\textbf{A2.} Составить уравнение параболы, зная, что .ее вершина совпадает с точкой ( $\alpha ; \beta$ ), параметр равен $p$, ось параллельна оси $O x$ и парабола простирается в бесконечность: в положительном направлении оси $O y$; \\
\textbf{A3.} Не проводя преобразования координат, установить, что каждое из следующих уравнений определяет параболу, и найти параметр этой параболы: $9 x^2-24 x y+16 y^2-54 x-178 y+181=0$; \\
\textbf{B1.} Установить, что следующие уравнения являются параболическими, и записать каждое из них в виде $(\alpha x+\beta y)^2+2 a_{13} x+2 a_{23} y+a_{33}=0$: $16 x^2+16 x y+4 y^2-5 x+7 y=0$; \\
\textbf{B2.} Установить, что следующие уравнения определяют центральные линии; преобразовать каждое из них путем переноса начала координат в центр: $6 x^2+4 x y+y^2+4 x-2 y+2=0$; \\
\textbf{B3.} Точка $M_1(1 ;-2)$ лежит на гиперболе, фокус которой $F(-2 ; 2)$, а соответствующая директриса дана уравнением $2 x-y-1=0$. Составить уравнение этой гиперболы. \\
\textbf{C1.} Для любого эллиптического уравнения доказать, что ни один из коэффициентов $a_{11}$ и $a_{22}$ не может обрашаться в нуль и что они суть числа одного знака. \\
\textbf{C2.} Составить уравнение касагельной к параболе $y^2=2 p x$ в ее точке $M_1\left(x_1 ; y_1\right)$. \\
\textbf{C3.} Дано уравнение линии $4 x^2-4 x y+y^2+6 x+1=0$. Определить, при каких значениях углового коэффициента $k$ прямая $y=k x:$ 1) пересекает эту линию в одной точке; 2) касается этой линии; 3) пересекает эту линию в двух точках; 4) не имеет общих точек с этой линией. \\

\end{tabular}
\vspace{1cm}


\begin{tabular}{m{17cm}}
\textbf{8-variant}
\newline

\textbf{T1.} Общие уравнения поверхностей второго порядка (Общее уравнение) \\
\textbf{T2.} Прямые образующие однополостного гиперболоида и гиперболического параболоида (Гиперболоид, Гиперболический параболоид, Линейные образующие) \\
\textbf{A1.} Составить уравнение гиперболы, если известны ее эксцентриситет $\varepsilon=\frac{5}{4}$, фокус $F(5 ; 0)$ и уравнение состветствующей директрисы $5 x-16=0$. \\
\textbf{A2.} Не проводя преобразования координат, установить, что каждое из следующих уравнений определяет параболу, и найти параметр этой параболы: $9 x^2+24 x y+16 y^2-120 x+90 y=0$; \\
\textbf{A3.} Вычислить фокальный радиус точки $M$ параболы $y^2=20 x$, если абсцисса точки $M$ равна 7 . \\
\textbf{B1.} Установить, какая линия является сечением эллипсоида $\frac{x^2}{12}+\frac{y^2}{4}+\frac{z^2}{3}=1$ плоскостью $2 x-3 y+4 z-11=0$, и найти ее центр. \\
\textbf{B2.} Установить, что следующие уравнения являются параболическими, и записать каждое из них в виде $(\alpha x+\beta y)^2+2 a_{13} x+2 a_{23} y+a_{33}=0$: $x^2+4 x y+4 y^2+4 x+y-15=0 ;$ \\
\textbf{B3.} Эксцентриситет эллипса $\varepsilon=\frac{1}{2}$, центр его совпадает с началом координат, одна из директрис дана уравнением $x=16$. Вычислить расстояние от точки $M_1$ эллипса с абсциссой, равной -4, до фокуса, одностороннего с данной директрисой. \\
\textbf{C1.} Определить, при каких значениях углового коэффициента $k$ прямая $y=k x+2$ 1) пересекает параболу $y^2=4 x$; 2) касается ее; 3) проходит вне этсй параболы. \\
\textbf{C2.} При каких значениях $m$ и $n$ уравнение $m x^2+12 x y+9 y^2+4 x+n y-13=0$ определяет: 1) центральную линию; 2) линию без центра; 3) линию, имеющую бесконечно много центров. \\
\textbf{C3.} Доказать, что уравнение второй степени является уравнением вырожденной линии в том и только в том случае, когда $\Delta=0$. \\

\end{tabular}
\vspace{1cm}


\begin{tabular}{m{17cm}}
\textbf{9-variant}
\newline

\textbf{T1.} Взаимное расположение линии второго порядка и прямой (Точки пересечения, касательное положение) \\
\textbf{T2.} Линии второго порядка на плоскости (Уравнение второго порядка, Уравнение квадратной формы, Конические линии (сечение конусов)) \\
\textbf{A1.} Установить, какие из следующих линий являются центральными (т.е. имеют единственный центр), какие имеют центра, какие имеют бесконечно много центров: $4 x^2-20 x y+25 y^2-14 x+2 y-15=0$; \\
\textbf{A2.} Не проводя преобразования координат, установить, что каждое из следующих уравнений определяет пару пересекающихся прямых (вырожденную гиперболу), и найти их уравнения: $3 x^2+4 x y+y^2-2 x-1=0$; \\
\textbf{A3.} Составить уравнение эллипса, фокусы которого расположены на оси абсцисс, симметрично относительно начала координат, если даны: точка $M_1(-2 \sqrt{5} ; 2)$ эллипса и его малая полуось $b=3$; \\
\textbf{B1.} Установить, что следующие уравнения определяют центральные линии; преобразовать каждое из них путем переноса начала координат в центр: $4 x^2+2 x y+6 y^2+6 x-10 y+9=0$. \\
\textbf{B2.} Составить уравнение параболы, если даны ее фокус $F(7 ; 2)$ и директриса $x-5=0$ \\
\textbf{B3.} Определить тип каждого из следующих уравнений каждое из них путем параллельного переноса осей координат привести к простейшему виду; установить, какие геометрические образы они определяют, и изобразить на чертеже расположение этих образов относительно старых и новых осей координат: $9 x^2+4 y^2+18 x-8 y+49=0$; \\
\textbf{C1.} Составить уравнение эллипса с полуосями $a, b$ и центром $C\left(x_0 ; y_0\right)$, если известно, что оси симметрии эллипса параллельны осям координат. \\
\textbf{C2.} Составить уравнение касательной к гиперболе $\frac{x^2}{a^2}-\frac{y^2}{b^2}=1$ в ее точке $M_1\left(x_1 ; y_1\right)$. \\
\textbf{C3.} Каждое из следующих уравнений привести к каноническому виду; определить тип каждого из них; установить, какие геометрические образы они определяют; для каждого случая изобразить на чертеже оси первоначальной координатной системы, оси других координатных систем, которые вводятся по ходу решения, и геометрический образ, определяемый данным уравнением: $29 x^2-24 x y+36 y^2+82 x-96 y-91=0$; \\

\end{tabular}
\vspace{1cm}


\begin{tabular}{m{17cm}}
\textbf{10-variant}
\newline

\textbf{T1.} Парабола и её канонические уравнения (Фокус (направляющая точка), Директриса (направляющая линия), Ось (ось симметрии)) \\
\textbf{T2.} Центр линии второго порядка (Центровые линии (эллипс, гипербола), Координаты центра: центр симметрии) \\
\textbf{A1.} Установить, какие линии определяются следующими уравнениями: $\left\{\begin{array}{l}\frac{x^2}{4}-\frac{y^2}{3}=2 z \\ x-2 y+2=0 ;\end{array}\right.$ \\
\textbf{A2.} Составить уравнение гиперболы, фокусы когорой расположены на оси абсцисс симметрично относительно начала координат, зная, кроме того, что: ее оси $2 a=10$ и $2 b=8$; \\
\textbf{A3.} Составить уравнение параболы, вершина которой находится в начале координат, зная, что: парабола расположена в правой полуплоскости симметрично относительно оси $O x$, и ее параметр $p=3$; \\
\textbf{B1.} Составить уравнение гиперболы, фокусы которой лежат в вершинах эллинса $\frac{x^2}{100}+\frac{y^2}{64}=1$, а директрисы проходят через фокусы этого эллипса. \\
\textbf{B2.} Установить, какая линия является сечением гиперболического параболоида $\frac{x^2}{2}-\frac{z^2}{3}=y$ плоскостью $3 x-3 y+4 z+2=0$, и найти ее центр. \\
\textbf{B3.} Составить уравнение прямой, которая касается параболы $y^2=8 x$ и параллельна прямой $2 x+2 y-3=0$. \\
\textbf{C1.} Доказать, что эллипсоид $\frac{x^2}{81}+\frac{y^2}{36}+\frac{z^2}{9}=1$ имеет одну общую точку с плоскостью $4 x-3 y+12 z-54=0$, и найти ее координаты. \\
\textbf{C2.} Провести касательные к эллиису $\frac{x^2}{30}+\frac{y^2}{24}=1$ параллельно прямой $4 x-2 y+23=0$ и вычислить расстояние $d$ между ними. \\
\textbf{C3.} Доказать, что параболическое уравнение определяет параболу в том и только в том случае, когда $\Delta \neq 0$. Доказать, что в этом случае параметр параболы определяется формулой $p=\sqrt{\frac{-\Delta}{ (a_{11}+a_{33}) ^3}}$. \\

\end{tabular}
\vspace{1cm}


\begin{tabular}{m{17cm}}
\textbf{11-variant}
\newline

\textbf{T1.} Общие уравнения поверхностей второго порядка (Общее уравнение) \\
\textbf{T2.} Приведение общего уравнения поверхности второго порядка к каноническому виду с помощью инвариантов \\
\textbf{A1.} Составить уравнение эллипса, фокусы которого лежат на оси абсцисс, симметрично относительно начала координат, зная, кроме того, что: его малая ось равна 6, а расстояние между дирек трисами равно 13 ; \\
\textbf{A2.} Не проводя преобразования координат, установить, что каждое из следующих уравнений определяет пару пересекающихся прямых (вырожденную гиперболу), и найти их уравнения: $x^2+4 x y+3 y^2-6 x-12 y+9=0$. \\
\textbf{A3.} Не проводя преобразования координат, установить, что каждое из следующих уравнений определяет параболу, и найти параметр этой параболы: $x^2-2 x y+y^2+6 x-14 y+29=0$; \\
\textbf{B1.} Определить эксцентриситет равносторонней гиперболы. \\
\textbf{B2.} Составить уравнение эллипса, зная, что: его большая ось равна 26 и фокусы суть $F_1(-10 ; 0), F_2(14 ; 0)$; \\
\textbf{B3.} Установить, что следующие линии являются центральными, и для каждой из них найти координаты центра: $5 x^2+4 x y+2 y^2+20 x+20 y-18=0$; \\
\textbf{C1.} Каждое из следующих уравнений привести к каноническому виду; определить тип каждого из них; установить, какие геометрические образы они определяют; для каждого случая изобразить на чертеже оси первоначальной координатной системы, оси других координатных систем, которые вводятся по ходу решения, и геометрический образ, определяемый данным уравнением: $7 x^2+60 x y+32 y^2-14 x-60 y+7=0$; \\
\textbf{C2.} Доказать, что эллиптический параболоид $\frac{x^2}{9}+\frac{z^2}{4}=2 y$ имеет одну общую точку с плоскостью $2 x-2 y-z-10=0$, и найти ее координаты. \\
\textbf{C3.} Доказать, что две параболы, имеющие общую ось и общий фокус, расположенный между их вершинами, пересекаются под прямым углом. \\

\end{tabular}
\vspace{1cm}


\begin{tabular}{m{17cm}}
\textbf{12-variant}
\newline

\textbf{T1.} Линии второго порядка на плоскости (Уравнение второго порядка, Уравнение квадратной формы, Конические линии (сечение конусов)) \\
\textbf{T2.} Общие уравнения линий второго порядка (Общее уравнение) \\
\textbf{A1.} Найти уравнения проекций на координатные плоскости сечения эллиптического параболоида $y^2+z^2=x$ плоскостью $x+2 y-z=0$ \\
\textbf{A2.} Установить, какие из следующих линий являются центральными (т.е. имеют единственный центр), какие имеют центра, какие имеют бесконечно много центров: $x^2-2 x y+4 y^2+5 x-7 y+12=0$; \\
\textbf{A3.} Установить, какие из следующих линий являются центральными (т.е. имеют единственный центр), какие имеют центра, какие имеют бесконечно много центров: $3 x^2-4 x y-2 y^2+3 x-12 y-7=0$; \\
\textbf{B1.} Установить, что каждое из следующих уравнений является параболическим; каждое из них привести к простейшему виду; установить, какие геометрические образы они определяют; для каждого случая изобразить на чертеже оси первоначальной координатной системы, оси других координатных систем, которые вводятся по ходу решения, и геометрический образ, определяемый данным уравнением: $9 x^2+12 x y+4 y^2-24 x-16 y+3=0$; \\
\textbf{B2.} Каждое из следующих уравнений привести к простейшему виду; определить тип каждого из них; установить, какие геометрические образы они определяют, и изобразить на чертеже расположение этих образов относительно старых и новых осей координат: $5 x^2-6 x y+5 y^2-32=0$; \\
\textbf{B3.} Установить, какая линия является сечением эллипсоида $\frac{x^2}{12}+\frac{y^2}{4}+\frac{z^2}{3}=1$ плоскостью $2 x-3 y+4 z-11=0$, и найти ее центр. \\
\textbf{C1.} Составить уравнение гиперболы, касающейся двух прямых: $\quad 5 x-6 y-16=0, \quad 13 x-10 y-48=0$, при условии, что ее оси совпадают с осями координат. \\
\textbf{C2.} При каких значениях $m$ и $n$ уравнение $m x^2+12 x y+9 y^2+4 x+n y-13=0$ определяет: 1) центральную линию; 2) линию без центра; 3) линию, имеющую бесконечно много центров. \\
\textbf{C3.} Каждое из следующих уравнений привести к каноническому виду; определить тип каждого из них; установить, какие геометрические образы они определяют; для каждого случая изобразить на чертеже оси первоначальной координатной системы, оси других координатных систем, которые вводятся по ходу решения, и геометрический образ, определяемый данным уравнением: $50 x^2-8 x y+35 y^2+100 x-8 y+67=0$; \\

\end{tabular}
\vspace{1cm}


\begin{tabular}{m{17cm}}
\textbf{13-variant}
\newline

\textbf{T1.} Приведение общего уравнения кривой второго порядка к каноническому виду с помощью инвариантов \\
\textbf{T2.} Центр, касательная плоскость и диаметральная плоскость поверхности второго порядка (Центр, касательная плоскость, диаметральная плоскость) \\
\textbf{A1.} Не проводя преобразования координат, установить, что каждое из следующих уравнений определяет параболу, и найти параметр этой параболы: $9 x^2+24 x y+16 y^2-120 x+90 y=0$; \\
\textbf{A2.} Определить тип каждого из следующих уравнений при помощи вычисления дискриминанта старших членов: $3 x^2-8 x y+7 y^2+8 x-15 y+20=0$; \\
\textbf{A3.} Составить уравнение эллипса, єсли известны его эксцентриситет $\varepsilon=\frac{2}{3}$, фокус $F(2 ; 1)$ и уравнение соответствующей директрисы $x-5=0$. \\
\textbf{B1.} Из точки $A(5 ; 9)$ проведены касательные к параболе $y^2=5 x$. Составить уравнение хорды, соединяющей точки касания. \\
\textbf{B2.} Установить, что следующие уравнения определяют центральные линии; преобразовать каждое из них путем переноса начала координат в центр: $4 x^2+6 x y+y^2-10 x-10=0$; \\
\textbf{B3.} Установить, что каждое из следующих уравнений является параболическим; каждое из них привести к простейшему виду; установить, какие геометрические образы они определяют; для каждого случая изобразить на чертеже оси первоначальной координатной системы, оси других координатных систем, которые вводятся по ходу решения, и геометрический образ, определяемый данным уравнением: $9 x^2-24 x y+16 y^2-20 x+110 y-50=0$; \\
\textbf{C1.} Доказать, что параболическое уравнение определяет параболу в том и только в том случае, когда $\Delta \neq 0$. Доказать, что в этом случае параметр параболы определяется формулой $p=\sqrt{\frac{-\Delta}{ (a_{11}+a_{33}) ^3}}$. \\
\textbf{C2.} Доказать, что двухполостный гиперболоид $\frac{x^2}{3}+\frac{y^2}{4}-\frac{z^2}{25}=-1$ имеет одну общую точку с плоскостью $5 x+2 z+5=0$, и найти ее координаты. \\
\textbf{C3.} Дано уравнение линии $4 x^2-4 x y+y^2+6 x+1=0$. Определить, при каких значениях углового коэффициента $k$ прямая $y=k x:$ 1) пересекает эту линию в одной точке; 2) касается этой линии; 3) пересекает эту линию в двух точках; 4) не имеет общих точек с этой линией. \\

\end{tabular}
\vspace{1cm}


\begin{tabular}{m{17cm}}
\textbf{14-variant}
\newline

\textbf{T1.} Линии второго порядка на плоскости (Уравнение второго порядка, Уравнение квадратной формы, Конические линии (сечение конусов)) \\
\textbf{T2.} Уравнение касательной линии второго порядка, сопряжённого диаметра (Уравнение касательной, сопряжённый диаметр: оси симметрии, проходящие через центр) \\
\textbf{A1.} Установить, какие линии определяются следующими уравнениями: $\left\{\begin{array}{l}\frac{x^2}{.4}+\frac{y^2}{9}-\frac{z^2}{36}=1, \\ 9 x-6 y+2 z-28=0,\end{array}\right.$ \\
\textbf{A2.} Составить уравнение параболы, вершина которой находится в начале координат, зная, что: парабола расположена симметрично относительно оси $O y$ и проходит через точку $C(1 ; 1)$. \\
\textbf{A3.} Составить уравнение гиперболы, фокусы когорой расположены на оси абсцисс симметрично относительно начала координат, зная, кроме того, что: расстояние между директрисами равно $\frac{32}{5}$ и ось $2 b=6$; \\
\textbf{B1.} Составить уравнение гиперболы, зная, что: расстояние между ее вершинами равно 24 и фокусы суть $F_1(-10 ; 2), F_2(16 ; 2)$; \\
\textbf{B2.} Определить тип каждого из следующих уравнений каждое из них путем параллельного переноса осей координат привести к простейшему виду; установить, какие геометрические образы они определяют, и изобразить на чертеже расположение этих образов относительно старых и новых осей координат: $2 x^2+3 y^2+8 x-6 y+11=0$. \\
\textbf{B3.} Эксцентриситет эллипса $\varepsilon=\frac{2}{3}$, фокальный радиус точки $M$ эллипса равен 10 . Вычислить расстояние от точки $M$ до односторонней с этим фокусом директрисы. \\
\textbf{C1.} Из точки $A\left(\frac{10}{3} ; \frac{5}{3}\right)$ проведены касательные к эллипсу $\frac{x^2}{20}+\frac{y^2}{5}=1$. Составить их уравнения. \\
\textbf{C2.} Вывести условие, при котором прямая $y=k x+m$ касается гиперболы $\frac{x^2}{a^2}-\frac{y^2}{b^2}=1$. \\
\textbf{C3.} Доказать, что две параболы, имеющие общую ось и общий фокус, расположенный между их вершинами, пересекаются под прямым углом. \\

\end{tabular}
\vspace{1cm}


\begin{tabular}{m{17cm}}
\textbf{15-variant}
\newline

\textbf{T1.} Канонические уравнения поверхностей второго порядка (Параболоид (эллиптический), Параболоид (гиперболический), Конус, Цилиндр) \\
\textbf{T2.} Парабола и её канонические уравнения (Фокус (направляющая точка), Директриса (направляющая линия), Ось (ось симметрии)) \\
\textbf{A1.} Составить уравнение гиперболы, фокусы когорой расположены на оси абсцисс симметрично относительно начала координат, зная, кроме того, что: расстояние между фокусами $2 c=6$ и эксцентриситет $\varepsilon=\frac{3}{2}$; \\
\textbf{A2.} Составить уравнение эллипса, фокусы которого лежат на оси абсцисс, симметрично относительно начала координат, зная, кроме того, что: его малая ось равна 10 , а эксцентриситет $\varepsilon=\frac{12}{13}$; \\
\textbf{A3.} Установить, что каждая из следующих линий имєет бесконечно много цєнтров; для каждой их них составить уравнение геометрического места центров: $25 x^2-10 x y+y^2+40 x-8 y+7=0$. \\
\textbf{B1.} Определить тип каждого из следующих уравнений каждое из них путем параллельного переноса осей координат привести к простейшему виду; установить, какие геометрические образы они определяют, и изобразить на чертеже расположение этих образов относительно старых и новых осей координат: $4 x^2+9 y^2-40 x+36 y+100=0$; \\
\textbf{B2.} Провести касательные к гиперболе $\frac{x^2}{16}-\frac{y^2}{8}=-1$ параллельно прямой $2 x+4 y-5=0$ и вычис лить расстояние $d$ между ними. \\
\textbf{B3.} То же задание, что и в предыдушей задаче, выполнить для уравнений: $4 x^2+12 x y+9 y^2-4 x-6 y+1=0$. \\
\textbf{C1.} Доказать, что эллиптичсское уравнение второй степсни ( $\delta>0$ ) является уравпением мнимого эллипса в том и только в том случае, когда $a_{11}$ и $\Delta$ суть числа одинаковых знаков. \\
\textbf{C2.} Доказать, что эллипсоид $\frac{x^2}{81}+\frac{y^2}{36}+\frac{z^2}{9}=1$ имеет одну общую точку с плоскостью $4 x-3 y+12 z-54=0$, и найти ее координаты. \\
\textbf{C3.} Доказать, что произведение расстояний от фокусов до любой касательной к эллипсу равно квадрату малой полуоси. \\

\end{tabular}
\vspace{1cm}


\begin{tabular}{m{17cm}}
\textbf{16-variant}
\newline

\textbf{T1.} Парабола и её канонические уравнения (Фокус (направляющая точка), Директриса (направляющая линия), Ось (ось симметрии)) \\
\textbf{T2.} Центр линии второго порядка (Центровые линии (эллипс, гипербола), Координаты центра: центр симметрии) \\
\textbf{A1.} Установить, что плоскость $x-2=0$ пересекает эллипсоид $\frac{x^2}{16}+\frac{y^2}{12}+\frac{z^2}{4}=1$ по эллипсу; найти его полуоси и вершины. \\
\textbf{A2.} Не проводя преобразования координат, установить, что каждое из следующих уравнений определяет параболу, и найти параметр этой параболы: $9 x^2-6 x y+y^2-50 x+50 y-275=0$. \\
\textbf{A3.} Составить уравнение параболы, вершина которой находится в начале координат, зная, что: парабола расположена симметрично относительно оси $O x$ и проходит через точку $B(-1 ; 3)$; \\
\textbf{B1.} Вычислить площадь четырехугольника, две вершины которого лежат в фокусах эллипса $x^2+5 y^2=20$, а две другие совпадают с концами его малой оси. \\
\textbf{B2.} Установить, что следующие линии являются центральными, и для каждой из них найти координаты центра: $9 x^2-4 x y-7 y^2-12=0$; \\
\textbf{B3.} Установить, какая линия является сечением эллипсоида $\frac{x^2}{12}+\frac{y^2}{4}+\frac{z^2}{3}=1$ плоскостью $2 x-3 y+4 z-11=0$, и найти ее центр. \\
\textbf{C1.} Определить, при каких значениях углового коэффициента $k$ прямая $y=k x+2$ 1) пересекает параболу $y^2=4 x$; 2) касается ее; 3) проходит вне этсй параболы. \\
\textbf{C2.} Определить, при каких значениях $m$ прямая $y=\frac{5}{2} x+m$ пересекает гиперболу $\frac{x^2}{9}-\frac{y^2}{36}=1$; 2) касается ее; 3) проходит вне этой гиперболы \\
\textbf{C3.} Доказать, что если уравнение второй степени является параболическим и написано в виде $ (\alpha x+\beta y) ^2+2a_{13}x+2a_{23}y+a_{33}=0$ то дискриминант его левой части определяется формулой $\Delta=- (a_{13} \beta-a_{23} \alpha) ^2$. \\

\end{tabular}
\vspace{1cm}


\begin{tabular}{m{17cm}}
\textbf{17-variant}
\newline

\textbf{T1.} Прямые образующие однополостного гиперболоида и гиперболического параболоида (Гиперболоид, Гиперболический параболоид, Линейные образующие) \\
\textbf{T2.} Взаимное расположение линии второго порядка и прямой (Точки пересечения, касательное положение) \\
\textbf{A1.} Определить тип каждого из следующих уравнений при помощи вычисления дискриминанта старших членов: $x^2-4 x y+4 y^2+7 x-12=0$; \\
\textbf{A2.} Эксцентриситет гиперболы $\varepsilon=3$, расстояние от точки. $M$ гиперболы до директрисы равно 4 . Вычислить расстояние от точки $M$ до фокуса, одностороннего с этой директрисой. \\
\textbf{A3.} Не проводя преобразования координат, установить, что каждое из следующих уравнений определяет параболу, и найти параметр этой параболы: $9 x^2-24 x y+16 y^2-54 x-178 y+181=0$; \\
\textbf{B1.} Составить уравнение прямой, которая касается параболы $x^2=16 y$ и перпендикулярна к прямой $2 x+4 y+7=0$. \\
\textbf{B2.} Определить эксцентриситет в эллипса, если: расстояние между директрисами в три раза больше расстояния между фокусами; \\
\textbf{B3.} Установить, какая линия является сечением гиперболического параболоида $\frac{x^2}{2}-\frac{z^2}{3}=y$ плоскостью $3 x-3 y+4 z+2=0$, и найти ее центр. \\
\textbf{C1.} При каких значениях $m$ и $n$ уравнение $m x^2+12 x y+9 y^2+4 x+n y-13=0$ определяет: 1) центральную линию; 2) линию без центра; 3) линию, имеющую бесконечно много центров. \\
\textbf{C2.} Составить уравнение касагельной к параболе $y^2=2 p x$ в ее точке $M_1\left(x_1 ; y_1\right)$. \\
\textbf{C3.} Составить уравнение гиперболы, если известны ее полуоси $a$ и $b$, центр $C\left(x_0 ; y_0\right)$ и фокусы расположены на прямой: 1) параллельной оси $O x$; 2) параллельной оси $O y$. \\

\end{tabular}
\vspace{1cm}


\begin{tabular}{m{17cm}}
\textbf{18-variant}
\newline

\textbf{T1.} Канонические уравнения поверхностей второго порядка (эллипсоид, гиперболоид (1-полостный), гиперболоид (2-полостный)) \\
\textbf{T2.} Линии второго порядка на плоскости (Уравнение второго порядка, Уравнение квадратной формы, Конические линии (сечение конусов)) \\
\textbf{A1.} Составить уравнение эллипса, фокусы которого лежат на оси абсцисс, симметрично относительно начала координат, зная, кроме того, что: расстояние между его директрисами равно 5 и расстояние между фокусами $2 c=4$; \\
\textbf{A2.} Найти уравнения проекций на координатные плоскости сечения эллиптического параболоида $y^2+z^2=x$ плоскостью $x+2 y-z=0$ \\
\textbf{A3.} Установить, какие из следующих линий являются центральными (т.е. имеют единственный центр), какие имеют центра, какие имеют бесконечно много центров: $x^2-2 x y+y^2-6 x+6 y-3=0$; \\
\textbf{B1.} То же задание, что и в предыдушей задаче, выполнить для уравнений: $x^2-2 x y+y^2-12 x+12 y-14=0$ \\
\textbf{B2.} Даны вершина параболы $A(6 ;-3)$ и уравнение ее директрисы $3 x-5 y+1=0$. Найти фокус $F$ этой параболы. \\
\textbf{B3.} Установить, что следующие линии являются центральными, и для каждой из них найти координаты центра: $3 x^2+5 x y+y^2-8 x-11 y-7=0$; \\
\textbf{C1.} Доказать, что уравнение второй степени является уравнением вырожденной линии в том и только в том случае, когда $\Delta=0$. \\
\textbf{C2.} Доказать, что эллиптический параболоид $\frac{x^2}{9}+\frac{z^2}{4}=2 y$ имеет одну общую точку с плоскостью $2 x-2 y-z-10=0$, и найти ее координаты. \\
\textbf{C3.} Каждое из следующих уравнений привести к каноническому виду; определить тип каждого из них; установить, какие геометрические образы они определяют; для каждого случая изобразить на чертеже оси первоначальной координатной системы, оси других координатных систем, которые вводятся по ходу решения, и геометрический образ, определяемый данным уравнением: $14 x^2+24 x y+21 y^2-4 x+18 y-139=0$; \\

\end{tabular}
\vspace{1cm}


\begin{tabular}{m{17cm}}
\textbf{19-variant}
\newline

\textbf{T1.} Прямые образующие однополостного гиперболоида и гиперболического параболоида (Гиперболоид, Гиперболический параболоид, Линейные образующие) \\
\textbf{T2.} Уравнение касательной линии второго порядка, сопряжённого диаметра (Уравнение касательной, сопряжённый диаметр: оси симметрии, проходящие через центр) \\
\textbf{A1.} Не проводя преобразования координат, установить, что каждое из следующих уравнений определяет единственную точку (вырожденный эллипс), и найти ее координаты: $x^2-6 x y+10 y^2+10 x-32 y+26=0$. \\
\textbf{A2.} Определить точки пересечения прямой $x+y$ -$-3=0$ и параболы $x^2=4 y$. \\
\textbf{A3.} Составить уравнение гиперболы, фокусы которой лежат на оси абсцисс симметрично относительно начала координат, если даны: точка $M_1\left(\frac{9}{2} ;-1\right)$ гиперболы и уравнения асимптот $y= \pm \frac{2}{3} x$; \\
\textbf{B1.} Составить уравнение касательных к гиперболе $x^2-y^2=16$, проведенных из точки $A(-1 ;-7)$. \\
\textbf{B2.} Определить тип каждого из следующих уравнений каждое из них путем параллельного переноса осей координат привести к простейшему виду; установить, какие геометрические образы они определяют, и изобразить на чертеже расположение этих образов относительно старых и новых осей координат: $4 x^2-y^2+8 x-2 y+3=0$; \\
\textbf{B3.} Составить уравнение прямой, которая касается параболы $y^2=8 x$ и параллельна прямой $2 x+2 y-3=0$. \\
\textbf{C1.} Дано уравнение линии $4 x^2-4 x y+y^2+6 x+1=0$. Определить, при каких значениях углового коэффициента $k$ прямая $y=k x:$ 1) пересекает эту линию в одной точке; 2) касается этой линии; 3) пересекает эту линию в двух точках; 4) не имеет общих точек с этой линией. \\
\textbf{C2.} Доказать, что произведение расстояний от центра эллипса до точки пересечения любой его касательной с фокальной осью и до основания перпендикуляря, опущенного из точки касания на фокальную ось, есть величина постоянная, равная квадрату большой полуоси эллипса. \\
\textbf{C3.} Дано уравнение линии $4 x^2-4 x y+y^2+6 x+1=0$. Определить, при каких значениях углового коэффициента $k$ прямая $y=k x:$ 1) пересекает эту линию в одной точке; 2) касается этой линии; 3) пересекает эту линию в двух точках; 4) не имеет общих точек с этой линией. \\

\end{tabular}
\vspace{1cm}


\begin{tabular}{m{17cm}}
\textbf{20-variant}
\newline

\textbf{T1.} Парабола и её канонические уравнения (Фокус (направляющая точка), Директриса (направляющая линия), Ось (ось симметрии)) \\
\textbf{T2.} Линии второго порядка на плоскости (Уравнение второго порядка, Уравнение квадратной формы, Конические линии (сечение конусов)) \\
\textbf{A1.} Не проводя преобразования координат, установить, что каждое из следующих уравнений определяет гиперболу, и найти величины ее полуосей: $x^2-6 x y-7 y^2+10 x-30 y+23=0$. \\
\textbf{A2.} Установить, что каждая из следующих линий имєет бесконечно много цєнтров; для каждой их них составить уравнение геометрического места центров: $4 x^2+4 x y+y^2-8 x-4 y-21=0$; \\
\textbf{A3.} Составить уравнение эллипса, фокусы которого лежат на оси ординат, симметрично относительно начала координат, зная, кроме того, что: расстояние между его фокусами $2 c=24$ и эксцентриситет $\varepsilon=\frac{12}{13}$; \\
\textbf{B1.} Каждое из следующих уравнений привести к простейшему виду; определить тип каждого из них; установить, какие геометрические образы они определяют, и изобразить на чертеже расположение этих образов относительно старых и новых осей координат: $5 x^2-6 x y+5 y^2+8=0$. \\
\textbf{B2.} Установить, какая линия является сечением гиперболического параболоида $\frac{x^2}{2}-\frac{z^2}{3}=y$ плоскостью $3 x-3 y+4 z+2=0$, и найти ее центр. \\
\textbf{B3.} То же задание, что и в предыдушей задаче, выполнить для уравнений: $9 x^2+24 x y+16 y^2-18 x+226 y+209=0$; \\
\textbf{C1.} Установить, при каких значениях $m$ плоскость $x+m y-2=0$ пересекает эллиптический параболоид $\frac{x^2}{2}+\frac{z^2}{3}=y$ а) по эллипсу, б) по параболе. \\
\textbf{C2.} Каждое из следующих уравнений привести к каноническому виду; определить тип каждого из них; установить, какие геометрические образы они определяют; для каждого случая изобразить на чертеже оси первоначальной координатной системы, оси других координатных систем, которые вводятся по ходу решения, и геометрический образ, определяемый данным уравнением: $4 x y+3 y^2+16 x+12 y-36=0$; \\
\textbf{C3.} Доказать, что любое параболическое уравнение может быть написано в виде: $ (\alpha x+\beta y) ^2+2a_{13}x+2a_{23}y+a_{33}=0$. Доказать также, что эллиптические и гиперболические уравнения в таком виде не могут быть написаны. \\

\end{tabular}
\vspace{1cm}


\begin{tabular}{m{17cm}}
\textbf{21-variant}
\newline

\textbf{T1.} Общие уравнения линий второго порядка (Общее уравнение) \\
\textbf{T2.} Центр, касательная плоскость и диаметральная плоскость поверхности второго порядка (Центр, касательная плоскость, диаметральная плоскость) \\
\textbf{A1.} Составить уравнение параболы, зная, что .ее вершина совпадает с точкой ( $\alpha ; \beta$ ), параметр равен $p$, ось параллельна оси $O x$ и парабола простирается в бесконечность: в положительном направлении оси $O x$; \\
\textbf{A2.} Установить, что плоскость $y+6=0$ пересекает гиперболический параболоид $\frac{x^2}{5}-\frac{y^2}{4}=6 z$ по параболе; найти ее параметр и вершину. \\
\textbf{A3.} Не проводя преобразования координат, установить, что каждое из следующих уравнений определяет параболу, и найти параметр этой параболы: $9 x^2-6 x y+y^2-50 x+50 y-275=0$. \\
\textbf{B1.} Составить уравнение эллипса, если известны его эксцентриситет $\varepsilon=\frac{1}{2}$, фокус $F(3 ; 0)$ и уравнение соответствующей директрисы $x+y-1=0$. \\
\textbf{B2.} Определить точки гиперболы $\frac{x^2}{64}-\frac{y^2}{36}=1$, расстояние которых до правого фокуса равно 4,5 . \\
\textbf{B3.} Установить, что следующие уравнения определяют центральные линии; преобразовать каждое из них путем переноса начала координат в центр: $6 x^2+4 x y+y^2+4 x-2 y+2=0$; \\
\textbf{C1.} Доказать, что если две параболы со взаимно перпендикулярными осями пересекаются в четырех точках, то эти точки лежат на одной окружности. \\
\textbf{C2.} Доказать, что произведение расстояний от фокусов до любой касательной к эллипсу равно квадрату малой полуоси. \\
\textbf{C3.} Доказать, что площадь параллелограмма, ограниченного асимптотами гиперболы $\frac{x^2}{a^2}-\frac{y^2}{b^2}=1$ и прямыми, проведенными через любую ее точку параллельно асимптотам, есть величина постоянная, равная $\frac{a b}{2}$. \\

\end{tabular}
\vspace{1cm}


\begin{tabular}{m{17cm}}
\textbf{22-variant}
\newline

\textbf{T1.} Приведение общего уравнения кривой второго порядка к каноническому виду с помощью инвариантов \\
\textbf{T2.} Канонические уравнения поверхностей второго порядка (эллипсоид, гиперболоид (1-полостный), гиперболоид (2-полостный)) \\
\textbf{A1.} Составить уравнения касательных к параболе $y^2=36 x$, проведенных из точки $A(2 ; 9)$. \\
\textbf{A2.} Не проводя преобразования координат, установить, что каждое из следующих уравнений определяет параболу, и найти параметр этой параболы: $9 x^2-24 x y+16 y^2-54 x-178 y+181=0$; \\
\textbf{A3.} Составить уравнение гиперболы, фокусы которой расположены на оси ординат симметрично относительно начала координат, зная, кроме того, что: уравнения асимптот $y= \pm \frac{4}{3} x$ и расстояние между директрисами равно $6 \frac{2}{5}$. \\
\textbf{B1.} Установить, что следующие линии являются центральными, и для каждой из них найти координаты центра: $3 x^2+5 x y+y^2-8 x-11 y-7=0$; \\
\textbf{B2.} Каждое из следующих уравнений привести к простейшему виду; определить тип каждого из них; установить, какие геометрические образы они определяют, и изобразить на чертеже расположение этих образов относительно старых и новых осей координат: $32 x^2+52 x y-7 y^2+180=0$; \\
\textbf{B3.} Установить, какая линия является сечением эллипсоида $\frac{x^2}{12}+\frac{y^2}{4}+\frac{z^2}{3}=1$ плоскостью $2 x-3 y+4 z-11=0$, и найти ее центр. \\
\textbf{C1.} Каждое из следующих уравнений привести к каноническому виду; определить тип каждого из них; установить, какие геометрические образы они определяют; для каждого случая изобразить на чертеже оси первоначальной координатной системы, оси других координатных систем, которые вводятся по ходу решения, и геометрический образ, определяемый данным уравнением: $41 x^2+24 x y+34 y^2+34 x-112 y+129=0$; \\
\textbf{C2.} Для любого параболического уравнения доказать, что коэффициенты $a_{11}$ и $a_{22}$ не могут быть числами разных знаков и что они одновременно не могут обрашаться в нуль. \\
\textbf{C3.} Провести касательные к эллиису $\frac{x^2}{30}+\frac{y^2}{24}=1$ параллельно прямой $4 x-2 y+23=0$ и вычислить расстояние $d$ между ними. \\

\end{tabular}
\vspace{1cm}


\begin{tabular}{m{17cm}}
\textbf{23-variant}
\newline

\textbf{T1.} Парабола и её канонические уравнения (Фокус (направляющая точка), Директриса (направляющая линия), Ось (ось симметрии)) \\
\textbf{T2.} Линии второго порядка на плоскости (Уравнение второго порядка, Уравнение квадратной формы, Конические линии (сечение конусов)) \\
\textbf{A1.} He проводя преобразования координат, установить, что каждое из следующих уравнений определяет эллипс, и найти величины его полуосей: $8 x^2+4 x y+5 y^2+16 x+4 y-28=0$; \\
\textbf{A2.} Составить уравнение эллипса, фокусы которого расположены на оси абсцисс, симметрично относительно начала координат, если даны: точки $M_1(4 ;-\sqrt{3})$ и $M_2(2 \sqrt{2} ; 3)$ эллипса; \\
\textbf{A3.} Установить, что плоскость $x-2=0$ пересекает эллипсоид $\frac{x^2}{16}+\frac{y^2}{12}+\frac{z^2}{4}=1$ по эллипсу; найти его полуоси и вершины. \\
\textbf{B1.} Даны вершина параболы $A(6 ;-3)$ и уравнение ее директрисы $3 x-5 y+1=0$. Найти фокус $F$ этой параболы. \\
\textbf{B2.} Определить эксцентриситет в эллипса, если: отрезок между фоку сами виден из вєршин малой оси под прямым углом; \\
\textbf{B3.} Установить, что следующие уравнения являются параболическими, и записать каждое из них в виде $(\alpha x+\beta y)^2+2 a_{13} x+2 a_{23} y+a_{33}=0$: $25 x^2-20 x y+4 y^2+3 x-y+11=0$; \\
\textbf{C1.} Даны гиперболы $\frac{x^2}{a^2}-\frac{y^2}{b^2}=1$ и какая-нибудь ее касательная: $P$-точка пересечения касательной с осью $O x, Q$ - проекция точки касания на ту же ось. Доказать, что $O P \cdot O Q=a^2$. \\
\textbf{C2.} При каких значениях $m$ и $n$ уравнение $m x^2+12 x y+9 y^2+4 x+n y-13=0$ определяет: 1) центральную линию; 2) линию без центра; 3) линию, имеющую бесконечно много центров. \\
\textbf{C3.} Доказать, что двухполостный гиперболоид $\frac{x^2}{3}+\frac{y^2}{4}-\frac{z^2}{25}=-1$ имеет одну общую точку с плоскостью $5 x+2 z+5=0$, и найти ее координаты. \\

\end{tabular}
\vspace{1cm}


\begin{tabular}{m{17cm}}
\textbf{24-variant}
\newline

\textbf{T1.} Взаимное расположение линии второго порядка и прямой (Точки пересечения, касательное положение) \\
\textbf{T2.} Канонические уравнения поверхностей второго порядка (Параболоид (эллиптический), Параболоид (гиперболический), Конус, Цилиндр) \\
\textbf{A1.} Установить, какие из следующих линий являются центральными (т.е. имеют единственный центр), какие имеют центра, какие имеют бесконечно много центров: $4 x^2-4 x y+y^2-12 x+6 y-11=0$; \\
\textbf{A2.} Составить уравнение гиперболы, фокусы когорой расположены на оси абсцисс симметрично относительно начала координат, зная, кроме того, что: расстояние между директрисами равно $22 \frac{2}{13}$ и расстояние между фокусами $2 c=26$; \\
\textbf{A3.} Составить уравнение эллипса, фокусы которого лежат на оси абсцисс, симметрично относительно начала координат, зная, кроме того, что: его большая ось равна 20, а эксцентриситет $\varepsilon=\frac{3}{5}$; \\
\textbf{B1.} Составить уравнения касательных к гиперболе $\frac{x^2}{16}-\frac{y^2}{64}=1$, параллельных прямой $10 x-3 y+9=0$. \\
\textbf{B2.} Установить, что следующие линии являются центральными, и для каждой из них найти координаты центра: $9 x^2-4 x y-7 y^2-12=0$; \\
\textbf{B3.} Составить уравнение гиперболы, зная, что: угол между асимптотами равен $90^{\circ}$ и фокусы суть $F_1(4 ;-4), F_2(-2 ; 2)$. \\
\textbf{C1.} Вывести условие, при котором прямая $y=k x+b$ касается параболы $y^2=2 p x$. \\
\textbf{C2.} Каждое из следующих уравнений привести к каноническому виду; определить тип каждого из них; установить, какие геометрические образы они определяют; для каждого случая изобразить на чертеже оси первоначальной координатной системы, оси других координатных систем, которые вводятся по ходу решения, и геометрический образ, определяемый данным уравнением: $3 x^2+10 x y+3 y^2-2 x-14 y-13=0$; \\
\textbf{C3.} Установить, при каких значениях $m$ плоскость $x+m z-1=0$ пересекает двухполостный гиперболоид $x^2+y^2-z^2=-1$ а) по эллипсу, б) по гиперболе. \\

\end{tabular}
\vspace{1cm}


\begin{tabular}{m{17cm}}
\textbf{25-variant}
\newline

\textbf{T1.} Центр линии второго порядка (Центровые линии (эллипс, гипербола), Координаты центра: центр симметрии) \\
\textbf{T2.} Общие уравнения поверхностей второго порядка (Общее уравнение) \\
\textbf{A1.} Не проводя преобразования координат, установить, что каждое из следующих уравнений определяет пару пересекающихся прямых (вырожденную гиперболу), и найти их уравнения: $x^2-4 x y+3 y^2=0$; \\
\textbf{A2.} Установить, какие из следующих линий являются центральными (т.е. имеют единственный центр), какие имеют центра, какие имеют бесконечно много центров: $4 x^2-4 x y+y^2-6 x+8 y+13=0$; \\
\textbf{A3.} Не проводя преобразования координат, установить, что каждое из следующих уравнений определяет параболу, и найти параметр этой параболы: $9 x^2+24 x y+16 y^2-120 x+90 y=0$; \\
\textbf{B1.} Установить, что следующие уравнения являются параболическими, и записать каждое из них в виде $(\alpha x+\beta y)^2+2 a_{13} x+2 a_{23} y+a_{33}=0$: $9 x^2-6 x y+y^2-x+2 y-14=0$; \\
\textbf{B2.} Установить, какая линия является сечением гиперболического параболоида $\frac{x^2}{2}-\frac{z^2}{3}=y$ плоскостью $3 x-3 y+4 z+2=0$, и найти ее центр. \\
\textbf{B3.} Определить эксцентриситет в эллипса, если: его малая ось видна из фокусов под углом в $60^{\circ}$; \\
\textbf{C1.} Доказать, что если две параболы со взаимно перпендикулярными осями пересекаются в четырех точках, то эти точки лежат на одной окружности. \\
\textbf{C2.} Доказать, что площадь параллелограмма, ограниченного асимптотами гиперболы $\frac{x^2}{a^2}-\frac{y^2}{b^2}=1$ и прямыми, проведенными через любую ее точку параллельно асимптотам, есть величина постоянная, равная $\frac{a b}{2}$. \\
\textbf{C3.} Доказать, что если уравнение второй степени является параболическим и написано в виде $ (\alpha x+\beta y) ^2+2a_{13}x+2a_{23}y+a_{33}=0$ то дискриминант его левой части определяется формулой $\Delta=- (a_{13} \beta-a_{23} \alpha) ^2$. \\

\end{tabular}
\vspace{1cm}


\begin{tabular}{m{17cm}}
\textbf{26-variant}
\newline

\textbf{T1.} Линии второго порядка на плоскости (Уравнение второго порядка, Уравнение квадратной формы, Конические линии (сечение конусов)) \\
\textbf{T2.} Приведение общего уравнения поверхности второго порядка к каноническому виду с помощью инвариантов \\
\textbf{A1.} Составить уравнение параболы, зная, что .ее вершина совпадает с точкой ( $\alpha ; \beta$ ), параметр равен $p$, ось параллельна оси $O x$ и парабола простирается в бесконечность: в отрицательном направлении оси $O y$. \\
\textbf{A2.} Установить, какие линии определяются следующими уравнениями: $\left\{\begin{array}{l}\frac{x^2}{4}-\frac{y^2}{3}=2 z \\ x-2 y+2=0 ;\end{array}\right.$ \\
\textbf{A3.} Составить уравнение эллипса, фокусы которого лежат на оси абсцисс, симметрично относительно начала координат, зная, кроме того, что: расстояние между его директрисами равно 32 и $\varepsilon=\frac{1}{2}$. \\
\textbf{B1.} Даны вершина параболы $A(-2 ;-1)$ и урав нение ее директрисы $x+2 y-1=0$. Составить уравнение этой параболы. \\
\textbf{B2.} Каждое из следующих уравнений привести к простейшему виду; определить тип каждого из них; установить, какие геометрические образы они определяют, и изобразить на чертеже расположение этих образов относительно старых и новых осей координат: $5 x^2+24 x y-5 y^2=0$; \\
\textbf{B3.} Установить, что следующие уравнения являются параболическими, и записать каждое из них в виде $(\alpha x+\beta y)^2+2 a_{13} x+2 a_{23} y+a_{33}=0$: $9 x^2-6 x y+y^2-x+2 y-14=0$; \\
\textbf{C1.} Дано уравнение линии $4 x^2-4 x y+y^2+6 x+1=0$. Определить, при каких значениях углового коэффициента $k$ прямая $y=k x:$ 1) пересекает эту линию в одной точке; 2) касается этой линии; 3) пересекает эту линию в двух точках; 4) не имеет общих точек с этой линией. \\
\textbf{C2.} Из точки $A\left(\frac{10}{3} ; \frac{5}{3}\right)$ проведены касательные к эллипсу $\frac{x^2}{20}+\frac{y^2}{5}=1$. Составить их уравнения. \\
\textbf{C3.} Определить, при каких значениях $m$ прямая $y=-x+m$ 1) пересекает эллипс $\frac{x^2}{20}+\frac{y^2}{5}=1$; 2) касается его; 3) проходит вне этого эллипса. \\

\end{tabular}
\vspace{1cm}


\begin{tabular}{m{17cm}}
\textbf{27-variant}
\newline

\textbf{T1.} Парабола и её канонические уравнения (Фокус (направляющая точка), Директриса (направляющая линия), Ось (ось симметрии)) \\
\textbf{T2.} Общие уравнения линий второго порядка (Общее уравнение) \\
\textbf{A1.} Не проводя преобразования координат, установить, какие геометрические образы определяются следующими уравнениями: $6 x^2-6 x y+9 y^2-4 x+18 y+14=0$; \\
\textbf{A2.} Составить уравнение гиперболы, фокусы которой лежат на оси абсцисс симметрично относительно начала координат, если даны: точки $M_1(6 ;-1)$ и $M_2(-8 ; 2 \sqrt{2})$ гиперболы; \\
\textbf{A3.} Не проводя преобразования координат, установить, что каждое из следующих уравнений определяет параболу, и найти параметр этой параболы: $x^2-2 x y+y^2+6 x-14 y+29=0$; \\
\textbf{B1.} Из точки $A(5 ; 9)$ проведены касательные к параболе $y^2=5 x$. Составить уравнение хорды, соединяющей точки касания. \\
\textbf{B2.} Установить, что следующие линии являются центральными, и для каждой из них найти координаты центра: $5 x^2+4 x y+2 y^2+20 x+20 y-18=0$; \\
\textbf{B3.} Каждое из следующих уравнений привести к простейшему виду; определить тип каждого из них; установить, какие геометрические образы они определяют, и изобразить на чертеже расположение этих образов относительно старых и новых осей координат: $17 x^2-12 x y+8 y^2=0$; \\
\textbf{C1.} Даны гиперболы $\frac{x^2}{a^2}-\frac{y^2}{b^2}=1$ и какая-нибудь ее касательная: $P$-точка пересечения касательной с осью $O x, Q$ - проекция точки касания на ту же ось. Доказать, что $O P \cdot O Q=a^2$. \\
\textbf{C2.} Доказать, что любое параболическое уравнение может быть написано в виде: $ (\alpha x+\beta y) ^2+2a_{13}x+2a_{23}y+a_{33}=0$. Доказать также, что эллиптические и гиперболические уравнения в таком виде не могут быть написаны. \\
\textbf{C3.} Доказать, что эллиптическое уравнение второй степени ( $\delta>0$ ) определяет эллипс в том и только в том случае, когда $a_{11}$ и $\Delta$ суть числа разных знаков. \\

\end{tabular}
\vspace{1cm}


\begin{tabular}{m{17cm}}
\textbf{28-variant}
\newline

\textbf{T1.} Линии второго порядка на плоскости (Уравнение второго порядка, Уравнение квадратной формы, Конические линии (сечение конусов)) \\
\textbf{T2.} Уравнение касательной линии второго порядка, сопряжённого диаметра (Уравнение касательной, сопряжённый диаметр: оси симметрии, проходящие через центр) \\
\textbf{A1.} Установить, что плоскость $z+1=0$ пересекает однополостный гиперболоид $\frac{x^2}{32}-\frac{y^2}{18}+\frac{z^2}{2}=1$ по гиперболе; найти ее полуоси и вершины. \\
\textbf{A2.} Установить, какие из следующих линий являются центральными (т.е. имеют единственный центр), какие имеют центра, какие имеют бесконечно много центров: $4 x^2-4 x y+y^2-12 x+6 y-11=0$; \\
\textbf{A3.} Составить уравнение параболы, зная, что .ее вершина совпадает с точкой ( $\alpha ; \beta$ ), параметр равен $p$, ось параллельна оси $O x$ и парабола простирается в бесконечность: в отрицательном направлении оси $O x$. \\
\textbf{B1.} Составить уравнение гиперболы, зная, что: фокусы суть $F_1(3 ; 4), F_2(-3 ;-4)$ и расстояние между директрисами равно 3,6 ; \\
\textbf{B2.} Составить уравнение эллипса, зная, что: его фокусы суть $F_1(1 ; 3), F_2(3 ; 1)$ и расстояние между директрисами равно $12 \sqrt{2}$. \\
\textbf{B3.} Установить, какая линия является сечением эллипсоида $\frac{x^2}{12}+\frac{y^2}{4}+\frac{z^2}{3}=1$ плоскостью $2 x-3 y+4 z-11=0$, и найти ее центр. \\
\textbf{C1.} При каких значениях $m$ и $n$ уравнение $m x^2+12 x y+9 y^2+4 x+n y-13=0$ определяет: 1) центральную линию; 2) линию без центра; 3) линию, имеющую бесконечно много центров. \\
\textbf{C2.} Определить, при каком значении $m$ плоскость $x-2 y-2 z+m=0$ касается эллипсоида $\frac{x^2}{144}+\frac{y^2}{36}+\frac{z^2}{9}=1$. \\
\textbf{C3.} Составить уравнение касагельной к параболе $y^2=2 p x$ в ее точке $M_1\left(x_1 ; y_1\right)$. \\

\end{tabular}
\vspace{1cm}


\begin{tabular}{m{17cm}}
\textbf{29-variant}
\newline

\textbf{T1.} Приведение общего уравнения поверхности второго порядка к каноническому виду с помощью инвариантов \\
\textbf{T2.} Парабола и её канонические уравнения (Фокус (направляющая точка), Директриса (направляющая линия), Ось (ось симметрии)) \\
\textbf{A1.} Определить точки пересечения гиперболы $\frac{x^2}{20}$ -$-\frac{y^2}{5}=-1$ и параболы $y^2=3 x$ \\
\textbf{A2.} Установить, какие линии определяются следующими уравнениями: $y=+\frac{2}{3} \sqrt{x^2-9}$ \\
\textbf{A3.} Не проводя преобразования координат, установить, что каждое из следующих уравнений определяет параболу, и найти параметр этой параболы: $9 x^2-6 x y+y^2-50 x+50 y-275=0$. \\
\textbf{B1.} Эксцентриситет эллипса $\varepsilon=\frac{2}{5}$, расстояние от точки $M$ эллипса до директрисы равно 20. Вычислить расстояние от точки $M$ до фокуса, одностороннего с этой директрисой. \\
\textbf{B2.} Установить, что следующие уравнения определяют центральные линии; преобразовать каждое из них путем переноса начала координат в центр: $3 x^2-6 x y+2 y^2-4 x+2 y+1=0$; \\
\textbf{B3.} Составить уравнение прямой, которая касается параболы $x^2=16 y$ и перпендикулярна к прямой $2 x+4 y+7=0$. \\
\textbf{C1.} Вывести условие, при котором прямая $y=k x+b$ касается параболы $y^2=2 p x$. \\
\textbf{C2.} Установить, при каких значениях $m$ плоскость $x+m y-2=0$ пересекает эллиптический параболоид $\frac{x^2}{2}+\frac{z^2}{3}=y$ а) по эллипсу, б) по параболе. \\
\textbf{C3.} Доказать, что уравнение второй степени является уравнением вырожденной линии в том и только в том случае, когда $\Delta=0$. \\

\end{tabular}
\vspace{1cm}


\begin{tabular}{m{17cm}}
\textbf{30-variant}
\newline

\textbf{T1.} Приведение общего уравнения кривой второго порядка к каноническому виду с помощью инвариантов \\
\textbf{T2.} Центр, касательная плоскость и диаметральная плоскость поверхности второго порядка (Центр, касательная плоскость, диаметральная плоскость) \\
\textbf{A1.} Составить уравнение эллипса, фокусы которого лежат на оси ординат, симметрично относительно начала координат, зная, кроме того, что: его малая ось равна 16 , а эксцентриситет $\varepsilon=\frac{3}{5}$; \\
\textbf{A2.} Установить, какие линии определяются следующими уравнениями: $\left\{\begin{array}{l}\frac{x^2}{.4}+\frac{y^2}{9}-\frac{z^2}{36}=1, \\ 9 x-6 y+2 z-28=0,\end{array}\right.$ \\
\textbf{A3.} Не проводя преобразования координат, установить, что каждое из следующих уравнений определяет пару пересекающихся прямых (вырожденную гиперболу), и найти их уравнения: $x^2-6 x y+8 y^2-4 y-4=0$; \\
\textbf{B1.} Установить, какая линия является сечением эллипсоида $\frac{x^2}{12}+\frac{y^2}{4}+\frac{z^2}{3}=1$ плоскостью $2 x-3 y+4 z-11=0$, и найти ее центр. \\
\textbf{B2.} Определить тип каждого из следующих уравнений каждое из них путем параллельного переноса осей координат привести к простейшему виду; установить, какие геометрические образы они определяют, и изобразить на чертеже расположение этих образов относительно старых и новых осей координат: $4 x^2+9 y^2-40 x+36 y+100=0$; \\
\textbf{B3.} То же задание, что и в предыдушей задаче, выполнить для уравнений: $x^2-2 x y+y^2-12 x+12 y-14=0$ \\
\textbf{C1.} Доказать, что произведение расстояний от центра эллипса до точки пересечения любой его касательной с фокальной осью и до основания перпендикуляря, опущенного из точки касания на фокальную ось, есть величина постоянная, равная квадрату большой полуоси эллипса. \\
\textbf{C2.} Доказать, что касательные к гиперболе, проведенные в концах одного и того же диаметра, параллельны. \\
\textbf{C3.} Дано уравнение линии $4 x^2-4 x y+y^2+6 x+1=0$. Определить, при каких значениях углового коэффициента $k$ прямая $y=k x:$ 1) пересекает эту линию в одной точке; 2) касается этой линии; 3) пересекает эту линию в двух точках; 4) не имеет общих точек с этой линией. \\

\end{tabular}
\vspace{1cm}


\begin{tabular}{m{17cm}}
\textbf{31-variant}
\newline

\textbf{T1.} Центр линии второго порядка (Центровые линии (эллипс, гипербола), Координаты центра: центр симметрии) \\
\textbf{T2.} Канонические уравнения поверхностей второго порядка (Параболоид (эллиптический), Параболоид (гиперболический), Конус, Цилиндр) \\
\textbf{A1.} Установить, что каждая из следующих линий имєет бесконечно много цєнтров; для каждой их них составить уравнение геометрического места центров: $4 x^2+4 x y+y^2-8 x-4 y-21=0$; \\
\textbf{A2.} Составить уравнение эллипса, фокусы которого расположены на оси абсцисс, симметрично относительно начала координат, если даны: точка $M_1(-\sqrt{5} ; 2)$ эллипса и расстояние между его директрисами равно 10. \\
\textbf{A3.} Установить, какие линии определяются следующими уравнениями: $\left\{\begin{array}{l}\frac{x^2}{3}+\frac{y^2}{6}=2 z, \\ 3 x-y+6 z-14=0\end{array}\right.$ \\
\textbf{B1.} Провести касательные к гиперболе $\frac{x^2}{16}-\frac{y^2}{8}=-1$ параллельно прямой $2 x+4 y-5=0$ и вычис лить расстояние $d$ между ними. \\
\textbf{B2.} Установить, что следующие уравнения определяют центральные линии; преобразовать каждое из них путем переноса начала координат в центр: $4 x^2+6 x y+y^2-10 x-10=0$; \\
\textbf{B3.} Установить, какая линия является сечением гиперболического параболоида $\frac{x^2}{2}-\frac{z^2}{3}=y$ плоскостью $3 x-3 y+4 z+2=0$, и найти ее центр. \\
\textbf{C1.} Доказать, что эллиптическое уравнение второй степени ( $\delta>0$ ) определяет вырожденный эллипс (точку) в том и только в том случае, когда $\Delta=0$. \\
\textbf{C2.} Доказать, что касательные к эллипсу $\frac{x^2}{a^2}+\frac{y^2}{b^2}=1$, проведенные в концах одного и того же диаметра, параллельны. (Диаметром эллипса называется его хорда, проходящая через центр.) \\
\textbf{C3.} Установить, при каких значениях $m$ плоскость $x+m z-1=0$ пересекает двухполостный гиперболоид $x^2+y^2-z^2=-1$ а) по эллипсу, б) по гиперболе. \\

\end{tabular}
\vspace{1cm}


\begin{tabular}{m{17cm}}
\textbf{32-variant}
\newline

\textbf{T1.} Парабола и её канонические уравнения (Фокус (направляющая точка), Директриса (направляющая линия), Ось (ось симметрии)) \\
\textbf{T2.} Общие уравнения поверхностей второго порядка (Общее уравнение) \\
\textbf{A1.} Составить уравнение гиперболы, фокусы когорой расположены на оси абсцисс симметрично относительно начала координат, зная, кроме того, что: уравнения асимптот $y= \pm \frac{4}{3} x$ и расстояние между фокусами $2 c=20$; \\
\textbf{A2.} Составить уравнение параболы, вершина которой находится в начале координат, зная, что: парабола расположена симметрично относительно оси $O x$ и проходит через точку $A(9 ; 6)$; \\
\textbf{A3.} He проводя преобразования координат, установить, что каждое из следующих уравнений определяет эллипс, и найти величины его полуосей: $13 x^2+10 x y+13 y^2+46 x+62 y+13=0$. \\
\textbf{B1.} Составить уравнение параболы, если даны ее фокус $F(2 ;-1)$ и директриса $x-y-1=0$. \\
\textbf{B2.} Через фокус эллипса $\frac{x^2}{25}+\frac{y^2}{15}=1$ проведен перпендикуляр к его большой оси. Определить расстояния от точек пересечения этого перпендикуляра с эллипсом до фокусов. \\
\textbf{B3.} Определить тип каждого из следующих уравнений каждое из них путем параллельного переноса осей координат привести к простейшему виду; установить, какие геометрические образы они определяют, и изобразить на чертеже расположение этих образов относительно старых и новых осей координат: $4 x^2-y^2+8 x-2 y+3=0$; \\
\textbf{C1.} При каких значениях $m$ и $n$ уравнение $m x^2+12 x y+9 y^2+4 x+n y-13=0$ определяет: 1) центральную линию; 2) линию без центра; 3) линию, имеющую бесконечно много центров. \\
\textbf{C2.} Составить уравнение гиперболы, касающейся двух прямых: $\quad 5 x-6 y-16=0, \quad 13 x-10 y-48=0$, при условии, что ее оси совпадают с осями координат. \\
\textbf{C3.} Доказать, что две параболы, имеющие общую ось и общий фокус, расположенный между их вершинами, пересекаются под прямым углом. \\

\end{tabular}
\vspace{1cm}


\begin{tabular}{m{17cm}}
\textbf{33-variant}
\newline

\textbf{T1.} Линии второго порядка на плоскости (Уравнение второго порядка, Уравнение квадратной формы, Конические линии (сечение конусов)) \\
\textbf{T2.} Общие уравнения линий второго порядка (Общее уравнение) \\
\textbf{A1.} Не проводя преобразования координат, установить, что каждое из следующих уравнений определяет параболу, и найти параметр этой параболы: $9 x^2-24 x y+16 y^2-54 x-178 y+181=0$; \\
\textbf{A2.} Установить, какие из следующих линий являются центральными (т.е. имеют единственный центр), какие имеют центра, какие имеют бесконечно много центров: $4 x^2-20 x y+25 y^2-14 x+2 y-15=0$; \\
\textbf{A3.} Установить, какие из следующих линий являются центральными (т.е. имеют единственный центр), какие имеют центра, какие имеют бесконечно много центров: $x^2-2 x y+4 y^2+5 x-7 y+12=0$; \\
\textbf{B1.} Определить эксцентриситет равносторонней гиперболы. \\
\textbf{B2.} Установить, что следующие уравнения являются параболическими, и записать каждое из них в виде $(\alpha x+\beta y)^2+2 a_{13} x+2 a_{23} y+a_{33}=0$: $25 x^2-20 x y+4 y^2+3 x-y+11=0$; \\
\textbf{B3.} Каждое из следующих уравнений привести к простейшему виду; определить тип каждого из них; установить, какие геометрические образы они определяют, и изобразить на чертеже расположение этих образов относительно старых и новых осей координат: $5 x^2-6 x y+5 y^2-32=0$; \\
\textbf{C1.} Доказать, что параболическое уравнение определяет параболу в том и только в том случае, когда $\Delta \neq 0$. Доказать, что в этом случае параметр параболы определяется формулой $p=\sqrt{\frac{-\Delta}{ (a_{11}+a_{33}) ^3}}$. \\
\textbf{C2.} Каждое из следующих уравнений привести к каноническому виду; определить тип каждого из них; установить, какие геометрические образы они определяют; для каждого случая изобразить на чертеже оси первоначальной координатной системы, оси других координатных систем, которые вводятся по ходу решения, и геометрический образ, определяемый данным уравнением: $7 x^2+6 x y-y^2+28 x+12 y+28=0$; \\
\textbf{C3.} Составить уравнение гиперболы, если известны ее полуоси $a$ и $b$, центр $C\left(x_0 ; y_0\right)$ и фокусы расположены на прямой: 1) параллельной оси $O x$; 2) параллельной оси $O y$. \\

\end{tabular}
\vspace{1cm}


\begin{tabular}{m{17cm}}
\textbf{34-variant}
\newline

\textbf{T1.} Канонические уравнения поверхностей второго порядка (эллипсоид, гиперболоид (1-полостный), гиперболоид (2-полостный)) \\
\textbf{T2.} Парабола и её канонические уравнения (Фокус (направляющая точка), Директриса (направляющая линия), Ось (ось симметрии)) \\
\textbf{A1.} Установить, что плоскость $x-2=0$ пересекает эллипсоид $\frac{x^2}{16}+\frac{y^2}{12}+\frac{z^2}{4}=1$ по эллипсу; найти его полуоси и вершины. \\
\textbf{A2.} Найти фокус $F$ и уравнение директрисы параболы $y^2=24 x$. \\
\textbf{A3.} Точка $A(-3 ;-5)$ лежит на эллипсе, фокус которого $F(-1 ;-4)$, а соответствующая директриса дана уравнением $x-2=0$. Составить уравнение этого эллипса. \\
\textbf{B1.} Установить, что следующие линии являются центральными, и для каждой из них найти координаты центра: $2 x^2-6 x y+5 y^2+22 x-36 y+11=0$. \\
\textbf{B2.} Составить уравнение гиперболы, если известны ее эксцентриситет $\varepsilon=\sqrt{5}$, фокус $F(2 ;-3)$ и уравнение соответствующей директрисы $3 x-y+3=0$. \\
\textbf{B3.} Составить уравнение параболы, если даны ее фокус $F(7 ; 2)$ и директриса $x-5=0$ \\
\textbf{C1.} Составить уравнение касательной к эллипсу $\frac{x^2}{a^2}+\frac{y^2}{b^2}=1$ в его точке $M_1\left(x_1 ; y_1\right)$. \\
\textbf{C2.} Для любого параболического уравнения доказать, что коэффициенты $a_{11}$ и $a_{22}$ не могут быть числами разных знаков и что они одновременно не могут обрашаться в нуль. \\
\textbf{C3.} Доказать, что двухполостный гиперболоид $\frac{x^2}{3}+\frac{y^2}{4}-\frac{z^2}{25}=-1$ имеет одну общую точку с плоскостью $5 x+2 z+5=0$, и найти ее координаты. \\

\end{tabular}
\vspace{1cm}


\begin{tabular}{m{17cm}}
\textbf{35-variant}
\newline

\textbf{T1.} Уравнение касательной линии второго порядка, сопряжённого диаметра (Уравнение касательной, сопряжённый диаметр: оси симметрии, проходящие через центр) \\
\textbf{T2.} Линии второго порядка на плоскости (Уравнение второго порядка, Уравнение квадратной формы, Конические линии (сечение конусов)) \\
\textbf{A1.} Определить тип каждого из следующих уравнений при помощи вычисления дискриминанта старших членов: $3 x^2-2 x y-3 y^2+12 y-15=0$. \\
\textbf{A2.} Составить уравнение гиперболы, фокусы которой расположены на оси ординат симметрично относительно начала координат, зная, кроме того, что: расстояние между директрисами равно $7 \frac{1}{7}$ и эксцентриситет $\varepsilon=\frac{7}{5}$; \\
\textbf{A3.} Не проводя преобразования координат, установить, что каждое из следующих уравнений определяет параболу, и найти параметр этой параболы: $x^2-2 x y+y^2+6 x-14 y+29=0$; \\
\textbf{B1.} Установить, что каждое из следующих уравнений является параболическим; каждое из них привести к простейшему виду; установить, какие геометрические образы они определяют; для каждого случая изобразить на чертеже оси первоначальной координатной системы, оси других координатных систем, которые вводятся по ходу решения, и геометрический образ, определяемый данным уравнением: $16 x^2-24 x y+9 y^2-160 x+120 y+425=0$. \\
\textbf{B2.} Установить, какая линия является сечением эллипсоида $\frac{x^2}{12}+\frac{y^2}{4}+\frac{z^2}{3}=1$ плоскостью $2 x-3 y+4 z-11=0$, и найти ее центр. \\
\textbf{B3.} Эксцентриситет эллипса $\varepsilon=\frac{1}{3}$, центр его совпадает с началом координат, один из фокусов $(-2 ; 0)$. Вычиелить расстояние от точки $M_1$ эллипса с абсциссой, равной 2, до директрисы, односторонней с данным фокусом. \\
\textbf{C1.} Каждое из следующих уравнений привести к каноническому виду; определить тип каждого из них; установить, какие геометрические образы они определяют; для каждого случая изобразить на чертеже оси первоначальной координатной системы, оси других координатных систем, которые вводятся по ходу решения, и геометрический образ, определяемый данным уравнением: $19 x^2+6 x y+11 y^2+38 x+6 y+29=0$; \\
\textbf{C2.} Определить, при каких значениях углового коэффициента $k$ прямая $y=k x+2$ 1) пересекает параболу $y^2=4 x$; 2) касается ее; 3) проходит вне этсй параболы. \\
\textbf{C3.} Дано уравнение линии $4 x^2-4 x y+y^2+6 x+1=0$. Определить, при каких значениях углового коэффициента $k$ прямая $y=k x:$ 1) пересекает эту линию в одной точке; 2) касается этой линии; 3) пересекает эту линию в двух точках; 4) не имеет общих точек с этой линией. \\

\end{tabular}
\vspace{1cm}


\begin{tabular}{m{17cm}}
\textbf{36-variant}
\newline

\textbf{T1.} Прямые образующие однополостного гиперболоида и гиперболического параболоида (Гиперболоид, Гиперболический параболоид, Линейные образующие) \\
\textbf{T2.} Приведение общего уравнения кривой второго порядка к каноническому виду с помощью инвариантов \\
\textbf{A1.} Установить, какие из следующих линий являются центральными (т.е. имеют единственный центр), какие имеют центра, какие имеют бесконечно много центров: $4 x^2-6 x y-9 y^2+3 x-7 y+12=0$. \\
\textbf{A2.} Составить уравнение эллипса, фокусы которого расположены на оси абсцисс, симметрично относительно начала координат, если даны: точка $M_1(2 ;-2)$ эллипса и его большая полуось $a=4$; \\
\textbf{A3.} Не проводя преобразования координат, установить, что каждое из следующих уравнений определяет параболу, и найти параметр этой параболы: $9 x^2+24 x y+16 y^2-120 x+90 y=0$; \\
\textbf{B1.} Определить тип каждого из следующих уравнений каждое из них путем параллельного переноса осей координат привести к простейшему виду; установить, какие геометрические образы они определяют, и изобразить на чертеже расположение этих образов относительно старых и новых осей координат: $9 x^2-16 y^2-54 x-64 y-127=0$; \\
\textbf{B2.} Установить, что следующие уравнения определяют центральные линии; преобразовать каждое из них путем переноса начала координат в центр: $4 x^2+2 x y+6 y^2+6 x-10 y+9=0$. \\
\textbf{B3.} Составить уравнение параболы, если даны ее фокус $F(4 ; 3)$ и директриса $y+1=0$. \\
\textbf{C1.} Каждое из следующих уравнений привести к каноническому виду; определить тип каждого из них; установить, какие геометрические образы они определяют; для каждого случая изобразить на чертеже оси первоначальной координатной системы, оси других координатных систем, которые вводятся по ходу решения, и геометрический образ, определяемый данным уравнением: $4 x^2+24 x y+11 y^2+64 x+42 y+51=0$; \\
\textbf{C2.} При каких значениях $m$ и $n$ уравнение $m x^2+12 x y+9 y^2+4 x+n y-13=0$ определяет: 1) центральную линию; 2) линию без центра; 3) линию, имеющую бесконечно много центров. \\
\textbf{C3.} Доказать что произведение расстояний от любой точки гиперболы $\frac{x^2}{a^2}-\frac{y^2}{b^2}=1$ до двух ее асимптот есть величина постоянная, равная $\frac{a^2 b^2}{a^2+b^2}$. \\

\end{tabular}
\vspace{1cm}


\begin{tabular}{m{17cm}}
\textbf{37-variant}
\newline

\textbf{T1.} Центр, касательная плоскость и диаметральная плоскость поверхности второго порядка (Центр, касательная плоскость, диаметральная плоскость) \\
\textbf{T2.} Линии второго порядка на плоскости (Уравнение второго порядка, Уравнение квадратной формы, Конические линии (сечение конусов)) \\
\textbf{A1.} Определить тип каждого из следующих уравнений при помощи вычисления дискриминанта старших членов: $5 x^2+14 x y+11 y^2+12 x-7 y+19=0$; \\
\textbf{A2.} Составить уравнение параболы, вершина которой находится в начале координат, зная, что: парабола расположена в левой полуплоскости симметрично относительно оси $O x$, и ее параметр $p=0,5$; \\
\textbf{A3.} Установить, что плоскость $z+1=0$ пересекает однополостный гиперболоид $\frac{x^2}{32}-\frac{y^2}{18}+\frac{z^2}{2}=1$ по гиперболе; найти ее полуоси и вершины. \\
\textbf{B1.} Установить, какая линия является сечением гиперболического параболоида $\frac{x^2}{2}-\frac{z^2}{3}=y$ плоскостью $3 x-3 y+4 z+2=0$, и найти ее центр. \\
\textbf{B2.} То же задание, что и в предыдушей задаче, выполнить для уравнений: $4 x^2+12 x y+9 y^2-4 x-6 y+1=0$. \\
\textbf{B3.} Составить уравнение эллипса, зная, что: его фокусы суть $F_1\left(-2 ; \frac{3}{2}\right), F_2\left(2 ;-\frac{3}{2}\right)$ и эксцентриситет $\varepsilon=\frac{\sqrt{2}}{2}$; \\
\textbf{C1.} Составить уравнение эллипса с полуосями $a, b$ и центром $C\left(x_0 ; y_0\right)$, если известно, что оси симметрии эллипса параллельны осям координат. \\
\textbf{C2.} Определить, при каком значении $m$ плоскость $x-2 y-2 z+m=0$ касается эллипсоида $\frac{x^2}{144}+\frac{y^2}{36}+\frac{z^2}{9}=1$. \\
\textbf{C3.} Составить уравнение касагельной к параболе $y^2=2 p x$ в ее точке $M_1\left(x_1 ; y_1\right)$. \\

\end{tabular}
\vspace{1cm}


\begin{tabular}{m{17cm}}
\textbf{38-variant}
\newline

\textbf{T1.} Взаимное расположение линии второго порядка и прямой (Точки пересечения, касательное положение) \\
\textbf{T2.} Взаимное расположение линии второго порядка и прямой (Точки пересечения, касательное положение) \\
\textbf{A1.} Эксцентриситет гиперболы $\varepsilon=2$, фокальный радиус ее точки $M$, проведенный из некоторого фокуса, равен 16. Вычислить расстояние от точки $M$ до односто* ронней с этим фокусом директрисы. \\
\textbf{A2.} Установить, что плоскость $y+6=0$ пересекает гиперболический параболоид $\frac{x^2}{5}-\frac{y^2}{4}=6 z$ по параболе; найти ее параметр и вершину. \\
\textbf{A3.} Составить уравнение эллипса, если известны его өксцентриситет $\varepsilon=\frac{1}{2}$, фокус $F(-4 ; 1)$ и уравнение соответствующей директрисы $y+3=0$. \\
\textbf{B1.} Точка $M_1(1 ;-2)$ лежит на гиперболе, фокус которой $F(-2 ; 2)$, а соответствующая директриса дана уравнением $2 x-y-1=0$. Составить уравнение этой гиперболы. \\
\textbf{B2.} Установить, какая линия является сечением эллипсоида $\frac{x^2}{12}+\frac{y^2}{4}+\frac{z^2}{3}=1$ плоскостью $2 x-3 y+4 z-11=0$, и найти ее центр. \\
\textbf{B3.} Составить уравнение гиперболы, зная, что: фокусы суть $F_1(3 ; 4), F_2(-3 ;-4)$ и расстояние между директрисами равно 3,6 ; \\
\textbf{C1.} Доказать, что любое параболическое уравнение может быть написано в виде: $ (\alpha x+\beta y) ^2+2a_{13}x+2a_{23}y+a_{33}=0$. Доказать также, что эллиптические и гиперболические уравнения в таком виде не могут быть написаны. \\
\textbf{C2.} Дано уравнение линии $4 x^2-4 x y+y^2+6 x+1=0$. Определить, при каких значениях углового коэффициента $k$ прямая $y=k x:$ 1) пересекает эту линию в одной точке; 2) касается этой линии; 3) пересекает эту линию в двух точках; 4) не имеет общих точек с этой линией. \\
\textbf{C3.} Вывести условие, при котором прямая $y=k x+m$ касается гиперболы $\frac{x^2}{a^2}-\frac{y^2}{b^2}=1$. \\

\end{tabular}
\vspace{1cm}


\begin{tabular}{m{17cm}}
\textbf{39-variant}
\newline

\textbf{T1.} Парабола и её канонические уравнения (Фокус (направляющая точка), Директриса (направляющая линия), Ось (ось симметрии)) \\
\textbf{T2.} Канонические уравнения поверхностей второго порядка (Параболоид (эллиптический), Параболоид (гиперболический), Конус, Цилиндр) \\
\textbf{A1.} Не проводя преобразования координат, установить, что каждое из следующих уравнений определяет гиперболу, и найти величины ее полуосей: $4 x^2+24 x y+11 y^2+64 x+42 y+51=0$; \\
\textbf{A2.} Составить уравнение параболы, вершина которой находится в начале координат, зная, что: парабола расположена в верхней полуплоскости симметррично относительно оси $O y$, и ее параметр $p=\frac{1}{4}$; \\
\textbf{A3.} Установить, какие из следующих линий являются центральными (т.е. имеют единственный центр), какие имеют центра, какие имеют бесконечно много центров: $4 x^2-4 x y+y^2-6 x+8 y+13=0$; \\
\textbf{B1.} Установить, что каждое из следующих уравнений является параболическим; каждое из них привести к простейшему виду; установить, какие геометрические образы они определяют; для каждого случая изобразить на чертеже оси первоначальной координатной системы, оси других координатных систем, которые вводятся по ходу решения, и геометрический образ, определяемый данным уравнением: $9 x^2+12 x y+4 y^2-24 x-16 y+3=0$; \\
\textbf{B2.} Определить эксцентриситет в эллипса, если: отрезок перпендикуляра, опущенного из центра эллипса на его директрису, делится вершиной эллипса пополам. \\
\textbf{B3.} Определить тип каждого из следующих уравнений каждое из них путем параллельного переноса осей координат привести к простейшему виду; установить, какие геометрические образы они определяют, и изобразить на чертеже расположение этих образов относительно старых и новых осей координат: $9 x^2+4 y^2+18 x-8 y+49=0$; \\
\textbf{C1.} Доказать, что уравнение второй степени является уравнением вырожденной линии в том и только в том случае, когда $\Delta=0$. \\
\textbf{C2.} Вывести условие, при котором прямая $y=k x+b$ касается параболы $y^2=2 p x$. \\
\textbf{C3.} Каждое из следующих уравнений привести к каноническому виду; определить тип каждого из них; установить, какие геометрические образы они определяют; для каждого случая изобразить на чертеже оси первоначальной координатной системы, оси других координатных систем, которые вводятся по ходу решения, и геометрический образ, определяемый данным уравнением: $5 x^2-2 x y+5 y^2-4 x+20 y+20=0$. \\

\end{tabular}
\vspace{1cm}


\begin{tabular}{m{17cm}}
\textbf{40-variant}
\newline

\textbf{T1.} Уравнение касательной линии второго порядка, сопряжённого диаметра (Уравнение касательной, сопряжённый диаметр: оси симметрии, проходящие через центр) \\
\textbf{T2.} Линии второго порядка на плоскости (Уравнение второго порядка, Уравнение квадратной формы, Конические линии (сечение конусов)) \\
\textbf{A1.} Составить уравнение гиперболы, фокусы которой лежат на оси абсцисс симметрично относительно начала координат, если даны: точка $M_1\left(-3 ; \frac{5}{2}\right)$ гиперболы и уравнения директрис $x= \pm \frac{4}{3}$; \\
\textbf{A2.} Не проводя преобразования координат, установить, что каждое из следующих уравнений определяет параболу, и найти параметр этой параболы: $9 x^2-24 x y+16 y^2-54 x-178 y+181=0$; \\
\textbf{A3.} Не проводя преобразования координат, установить, что каждое из следующих уравнений определяет параболу, и найти параметр этой параболы: $x^2-2 x y+y^2+6 x-14 y+29=0$; \\
\textbf{B1.} Установить, что следующие линии являются центральными, и для каждой из них найти координаты центра: $9 x^2-4 x y-7 y^2-12=0$; \\
\textbf{B2.} Составить уравнение прямой, которая касается параболы $x^2=16 y$ и перпендикулярна к прямой $2 x+4 y+7=0$. \\
\textbf{B3.} Определить эксцентриситет в эллипса, если: его малая ось видна из фокусов под углом в $60^{\circ}$; \\
\textbf{C1.} Доказать, что касательные к эллипсу $\frac{x^2}{a^2}+\frac{y^2}{b^2}=1$, проведенные в концах одного и того же диаметра, параллельны. (Диаметром эллипса называется его хорда, проходящая через центр.) \\
\textbf{C2.} Доказать, что эллипсоид $\frac{x^2}{81}+\frac{y^2}{36}+\frac{z^2}{9}=1$ имеет одну общую точку с плоскостью $4 x-3 y+12 z-54=0$, и найти ее координаты. \\
\textbf{C3.} Доказать, что если две параболы со взаимно перпендикулярными осями пересекаются в четырех точках, то эти точки лежат на одной окружности. \\

\end{tabular}
\vspace{1cm}


\begin{tabular}{m{17cm}}
\textbf{41-variant}
\newline

\textbf{T1.} Общие уравнения поверхностей второго порядка (Общее уравнение) \\
\textbf{T2.} Приведение общего уравнения кривой второго порядка к каноническому виду с помощью инвариантов \\
\textbf{A1.} Установить, что каждая из следующих линий имєет бесконечно много цєнтров; для каждой их них составить уравнение геометрического места центров: $25 x^2-10 x y+y^2+40 x-8 y+7=0$. \\
\textbf{A2.} Составить уравнение параболы, вершина которой находится в начале координат, зная, что: парабола расположена симметрично отнлсительно оси $O y$ и проходит через точку $D(4 ;-8)$. \\
\textbf{A3.} Составить уравнение эллипса, фокусы которого лежат на оси ординат, симметрично относительно начала координат, зная, кроме того, что: расстояние между его директрисами равно $10 \frac{2}{3}$ и эксцентриситет $\varepsilon=\frac{3}{4}$. \\
\textbf{B1.} Установить, что следующие линии являются центральными, и для каждой из них найти координаты центра: $5 x^2+4 x y+2 y^2+20 x+20 y-18=0$; \\
\textbf{B2.} Составить уравнение параболы, если даны ее фокус $F(4 ; 3)$ и директриса $y+1=0$. \\
\textbf{B3.} Составить уравнения касательных к гиперболе $\frac{x^2}{16}-\frac{y^2}{64}=1$, параллельных прямой $10 x-3 y+9=0$. \\
\textbf{C1.} Из точки $A\left(\frac{10}{3} ; \frac{5}{3}\right)$ проведены касательные к эллипсу $\frac{x^2}{20}+\frac{y^2}{5}=1$. Составить их уравнения. \\
\textbf{C2.} Каждое из следующих уравнений привести к каноническому виду; определить тип каждого из них; установить, какие геометрические образы они определяют; для каждого случая изобразить на чертеже оси первоначальной координатной системы, оси других координатных систем, которые вводятся по ходу решения, и геометрический образ, определяемый данным уравнением: $11 x^2-20 x y-4 y^2-20 x-8 y+1=0$; \\
\textbf{C3.} Доказать, что параболическое уравнение определяет параболу в том и только в том случае, когда $\Delta \neq 0$. Доказать, что в этом случае параметр параболы определяется формулой $p=\sqrt{\frac{-\Delta}{ (a_{11}+a_{33}) ^3}}$. \\

\end{tabular}
\vspace{1cm}


\begin{tabular}{m{17cm}}
\textbf{42-variant}
\newline

\textbf{T1.} Приведение общего уравнения поверхности второго порядка к каноническому виду с помощью инвариантов \\
\textbf{T2.} Парабола и её канонические уравнения (Фокус (направляющая точка), Директриса (направляющая линия), Ось (ось симметрии)) \\
\textbf{A1.} Найти уравнения проекций на координатные плоскости сечения эллиптического параболоида $y^2+z^2=x$ плоскостью $x+2 y-z=0$ \\
\textbf{A2.} Установить, какие линии определяются следующими уравнениями: $y=-3 \sqrt{x^2+1}$; \\
\textbf{A3.} Определить тип каждого из следующих уравнений при помощи вычисления дискриминанта старших членов: $2 x^2+10 x y+12 y^2-7 x+18 y-15=0$; \\
\textbf{B1.} Установить, что каждое из следующих уравнений является параболическим; каждое из них привести к простейшему виду; установить, какие геометрические образы они определяют; для каждого случая изобразить на чертеже оси первоначальной координатной системы, оси других координатных систем, которые вводятся по ходу решения, и геометрический образ, определяемый данным уравнением: $9 x^2-24 x y+16 y^2-20 x+110 y-50=0$; \\
\textbf{B2.} Каждое из следующих уравнений привести к простейшему виду; определить тип каждого из них; установить, какие геометрические образы они определяют, и изобразить на чертеже расположение этих образов относительно старых и новых осей координат: $5 x^2-6 x y+5 y^2+8=0$. \\
\textbf{B3.} Установить, какая линия является сечением гиперболического параболоида $\frac{x^2}{2}-\frac{z^2}{3}=y$ плоскостью $3 x-3 y+4 z+2=0$, и найти ее центр. \\
\textbf{C1.} Определить, при каких значениях $m$ прямая $y=\frac{5}{2} x+m$ пересекает гиперболу $\frac{x^2}{9}-\frac{y^2}{36}=1$; 2) касается ее; 3) проходит вне этой гиперболы \\
\textbf{C2.} При каких значениях $m$ и $n$ уравнение $m x^2+12 x y+9 y^2+4 x+n y-13=0$ определяет: 1) центральную линию; 2) линию без центра; 3) линию, имеющую бесконечно много центров. \\
\textbf{C3.} Доказать, что эллиптический параболоид $\frac{x^2}{9}+\frac{z^2}{4}=2 y$ имеет одну общую точку с плоскостью $2 x-2 y-z-10=0$, и найти ее координаты. \\

\end{tabular}
\vspace{1cm}


\begin{tabular}{m{17cm}}
\textbf{43-variant}
\newline

\textbf{T1.} Общие уравнения линий второго порядка (Общее уравнение) \\
\textbf{T2.} Линии второго порядка на плоскости (Уравнение второго порядка, Уравнение квадратной формы, Конические линии (сечение конусов)) \\
\textbf{A1.} Определить точки эллипса $\frac{x^2}{16}+\frac{y^2}{7}=1$, pacстояние которых до левого фокуса равно $2,5$. \\
\textbf{A2.} Установить, какие линии определяются следующими уравнениями: $\left\{\begin{array}{l}\frac{x^2}{3}+\frac{y^2}{6}=2 z, \\ 3 x-y+6 z-14=0\end{array}\right.$ \\
\textbf{A3.} Определить тип каждого из следующих уравнений при помощи вычисления дискриминанта старших членов: $25 x^2-20 x y+4 y^2-12 x+20 y-17=0$; \\
\textbf{B1.} Установить, что следующие уравнения определяют центральные линии; преобразовать каждое из них путем переноса начала координат в центр: $3 x^2-6 x y+2 y^2-4 x+2 y+1=0$; \\
\textbf{B2.} Установить, что следующие уравнения являются параболическими, и записать каждое из них в виде $(\alpha x+\beta y)^2+2 a_{13} x+2 a_{23} y+a_{33}=0$: $16 x^2+16 x y+4 y^2-5 x+7 y=0$; \\
\textbf{B3.} Эксцентриситет эллипса $\varepsilon=\frac{2}{5}$, расстояние от точки $M$ эллипса до директрисы равно 20. Вычислить расстояние от точки $M$ до фокуса, одностороннего с этой директрисой. \\
\textbf{C1.} Определить, при каких значениях $m$ прямая $y=-x+m$ 1) пересекает эллипс $\frac{x^2}{20}+\frac{y^2}{5}=1$; 2) касается его; 3) проходит вне этого эллипса. \\
\textbf{C2.} Доказать, что если уравнение второй степени является параболическим и написано в виде $ (\alpha x+\beta y) ^2+2a_{13}x+2a_{23}y+a_{33}=0$ то дискриминант его левой части определяется формулой $\Delta=- (a_{13} \beta-a_{23} \alpha) ^2$. \\
\textbf{C3.} Установить, при каких значениях $m$ плоскость $x+m y-2=0$ пересекает эллиптический параболоид $\frac{x^2}{2}+\frac{z^2}{3}=y$ а) по эллипсу, б) по параболе. \\

\end{tabular}
\vspace{1cm}


\begin{tabular}{m{17cm}}
\textbf{44-variant}
\newline

\textbf{T1.} Канонические уравнения поверхностей второго порядка (эллипсоид, гиперболоид (1-полостный), гиперболоид (2-полостный)) \\
\textbf{T2.} Прямые образующие однополостного гиперболоида и гиперболического параболоида (Гиперболоид, Гиперболический параболоид, Линейные образующие) \\
\textbf{A1.} Составить уравнение гиперболы, фокусы которой лежат на оси абсцисс симметрично относительно начала координат, если даны: уравнения асимптот $y= \pm \frac{3}{4} x$ и уравнения директрис $x= \pm \frac{16}{5}$. \\
\textbf{A2.} Не проводя преобразования координат, установить, что каждое из следующих уравнений определяет параболу, и найти параметр этой параболы: $9 x^2-6 x y+y^2-50 x+50 y-275=0$. \\
\textbf{A3.} Определить точки пересечения эллипса $\frac{x^2}{100}+\frac{y^2}{225}=1$ и параболы $y^2=24 x$ \\
\textbf{B1.} Определить тип каждого из следующих уравнений каждое из них путем параллельного переноса осей координат привести к простейшему виду; установить, какие геометрические образы они определяют, и изобразить на чертеже расположение этих образов относительно старых и новых осей координат: $2 x^2+3 y^2+8 x-6 y+11=0$. \\
\textbf{B2.} Установить, какая линия является сечением эллипсоида $\frac{x^2}{12}+\frac{y^2}{4}+\frac{z^2}{3}=1$ плоскостью $2 x-3 y+4 z-11=0$, и найти ее центр. \\
\textbf{B3.} Дана точка $M_1(10 ;-\sqrt{5})$ на гиперболе $\frac{x^2}{80}-\frac{y^2}{20}=1$. Составить уравнения прямых, на которых лежат фокальные радиусы точки $M_1$. \\
\textbf{C1.} При каких значениях $m$ и $n$ уравнение $m x^2+12 x y+9 y^2+4 x+n y-13=0$ определяет: 1) центральную линию; 2) линию без центра; 3) линию, имеющую бесконечно много центров. \\
\textbf{C2.} Составить уравнение касательной к гиперболе $\frac{x^2}{a^2}-\frac{y^2}{b^2}=1$ в ее точке $M_1\left(x_1 ; y_1\right)$. \\
\textbf{C3.} Определить, при каких значениях углового коэффициента $k$ прямая $y=k x+2$ 1) пересекает параболу $y^2=4 x$; 2) касается ее; 3) проходит вне этсй параболы. \\

\end{tabular}
\vspace{1cm}


\begin{tabular}{m{17cm}}
\textbf{45-variant}
\newline

\textbf{T1.} Центр линии второго порядка (Центровые линии (эллипс, гипербола), Координаты центра: центр симметрии) \\
\textbf{T2.} Парабола и её канонические уравнения (Фокус (направляющая точка), Директриса (направляющая линия), Ось (ось симметрии)) \\
\textbf{A1.} Установить, какие из следующих линий являются центральными (т.е. имеют единственный центр), какие имеют центра, какие имеют бесконечно много центров: $3 x^2-4 x y-2 y^2+3 x-12 y-7=0$; \\
\textbf{A2.} Составить уравнение гиперболы, фокусы которой расположены на оси ординат симметрично относительно начала координат, зная, кроме того, что: ее полуоси $a=6, b=18$ (буквой $a$ мы обозначаем полуось гинерболы, расположенную на оси абсцисс) ; \\
\textbf{A3.} Установить, какие из следующих линий являются центральными (т.е. имеют единственный центр), какие имеют центра, какие имеют бесконечно много центров: $x^2-2 x y+y^2-6 x+6 y-3=0$; \\
\textbf{B1.} Составить уравнение параболы, если даны ее фокус $F(2 ;-1)$ и директриса $x-y-1=0$. \\
\textbf{B2.} Составить уравнение гиперболы, фокусы которой лежат в вершинах эллинса $\frac{x^2}{100}+\frac{y^2}{64}=1$, а директрисы проходят через фокусы этого эллипса. \\
\textbf{B3.} Установить, что следующие уравнения определяют центральные линии; преобразовать каждое из них путем переноса начала координат в центр: $4 x^2+2 x y+6 y^2+6 x-10 y+9=0$. \\
\textbf{C1.} Доказать, что эллиптическое уравнение второй степени ( $\delta>0$ ) определяет вырожденный эллипс (точку) в том и только в том случае, когда $\Delta=0$. \\
\textbf{C2.} Дано уравнение линии $4 x^2-4 x y+y^2+6 x+1=0$. Определить, при каких значениях углового коэффициента $k$ прямая $y=k x:$ 1) пересекает эту линию в одной точке; 2) касается этой линии; 3) пересекает эту линию в двух точках; 4) не имеет общих точек с этой линией. \\
\textbf{C3.} Провести касательные к эллиису $\frac{x^2}{30}+\frac{y^2}{24}=1$ параллельно прямой $4 x-2 y+23=0$ и вычислить расстояние $d$ между ними. \\

\end{tabular}
\vspace{1cm}


\begin{tabular}{m{17cm}}
\textbf{46-variant}
\newline

\textbf{T1.} Парабола и её канонические уравнения (Фокус (направляющая точка), Директриса (направляющая линия), Ось (ось симметрии)) \\
\textbf{T2.} Приведение общего уравнения кривой второго порядка к каноническому виду с помощью инвариантов \\
\textbf{A1.} Установить, какие линии определяются следующими уравнениями: $\left\{\begin{array}{l}\frac{x^2}{.4}+\frac{y^2}{9}-\frac{z^2}{36}=1, \\ 9 x-6 y+2 z-28=0,\end{array}\right.$ \\
\textbf{A2.} Не проводя преобразования координат, установить, что каждое из следующих уравнений определяет параболу, и найти параметр этой параболы: $9 x^2+24 x y+16 y^2-120 x+90 y=0$; \\
\textbf{A3.} Составить уравнение эллипса, фокусы которого лежат на оси ординат, симметрично относительно начала координат, зная, кроме того, что: его большая ось равна 10 , а расстояние между фокусами $2 c=8$; \\
\textbf{B1.} Каждое из следующих уравнений привести к простейшему виду; определить тип каждого из них; установить, какие геометрические образы они определяют, и изобразить на чертеже расположение этих образов относительно старых и новых осей координат: $32 x^2+52 x y-7 y^2+180=0$; \\
\textbf{B2.} Установить, какая линия является сечением гиперболического параболоида $\frac{x^2}{2}-\frac{z^2}{3}=y$ плоскостью $3 x-3 y+4 z+2=0$, и найти ее центр. \\
\textbf{B3.} То же задание, что и в предыдушей задаче, выполнить для уравнений: $9 x^2+24 x y+16 y^2-18 x+226 y+209=0$; \\
\textbf{C1.} Каждое из следующих уравнений привести к каноническому виду; определить тип каждого из них; установить, какие геометрические образы они определяют; для каждого случая изобразить на чертеже оси первоначальной координатной системы, оси других координатных систем, которые вводятся по ходу решения, и геометрический образ, определяемый данным уравнением: $29 x^2-24 x y+36 y^2+82 x-96 y-91=0$; \\
\textbf{C2.} Доказать, что двухполостный гиперболоид $\frac{x^2}{3}+\frac{y^2}{4}-\frac{z^2}{25}=-1$ имеет одну общую точку с плоскостью $5 x+2 z+5=0$, и найти ее координаты. \\
\textbf{C3.} Доказать, что расстояние от фокуса гиперболы $\frac{x^2}{a^2}-\frac{y^2}{b^2}=1$ до ее асимптоты равно $b$. \\

\end{tabular}
\vspace{1cm}


\begin{tabular}{m{17cm}}
\textbf{47-variant}
\newline

\textbf{T1.} Канонические уравнения поверхностей второго порядка (Параболоид (эллиптический), Параболоид (гиперболический), Конус, Цилиндр) \\
\textbf{T2.} Линии второго порядка на плоскости (Уравнение второго порядка, Уравнение квадратной формы, Конические линии (сечение конусов)) \\
\textbf{A1.} Составить уравнение параболы, вершина которой находится в начале координат, зная, что: парабола расположена симметрично относительно оси $O x$ и проходит через точку $A(9 ; 6)$; \\
\textbf{A2.} Не проводя преобразования координат, установить, что каждое из следующих уравнений определяет единственную точку (вырожденный эллипс), и найти ее координаты: $5 x^2+4 x y+y^2-6 x-2 y+2=0$; \\
\textbf{A3.} Не проводя преобразования координат, установить, какие геометрические образы определяются следующими уравнениями: $17 x^2-18 x y-7 y^2+34 x-18 y+7=0$; \\
\textbf{B1.} Эксцентриситет эллипса $\varepsilon=\frac{1}{3}$, центр его совпадает с началом координат, один из фокусов $(-2 ; 0)$. Вычиелить расстояние от точки $M_1$ эллипса с абсциссой, равной 2, до директрисы, односторонней с данным фокусом. \\
\textbf{B2.} Даны вершина параболы $A(-2 ;-1)$ и урав нение ее директрисы $x+2 y-1=0$. Составить уравнение этой параболы. \\
\textbf{B3.} Составить уравнение эллипса, зная, что: его фокусы суть $F_1(1 ; 3), F_2(3 ; 1)$ и расстояние между директрисами равно $12 \sqrt{2}$. \\
\textbf{C1.} Доказать, что две параболы, имеющие общую ось и общий фокус, расположенный между их вершинами, пересекаются под прямым углом. \\
\textbf{C2.} Для любого параболического уравнения доказать, что коэффициенты $a_{11}$ и $a_{22}$ не могут быть числами разных знаков и что они одновременно не могут обрашаться в нуль. \\
\textbf{C3.} Вывести условие, при котором прямая $y=k x+b$ касается параболы $y^2=2 p x$. \\

\end{tabular}
\vspace{1cm}


\begin{tabular}{m{17cm}}
\textbf{48-variant}
\newline

\textbf{T1.} Приведение общего уравнения поверхности второго порядка к каноническому виду с помощью инвариантов \\
\textbf{T2.} Общие уравнения линий второго порядка (Общее уравнение) \\
\textbf{A1.} Не проводя преобразования координат, установить, что каждое из следующих уравнений определяет параболу, и найти параметр этой параболы: $9 x^2+24 x y+16 y^2-120 x+90 y=0$; \\
\textbf{A2.} Установить, какие из следующих линий являются центральными (т.е. имеют единственный центр), какие имеют центра, какие имеют бесконечно много центров: $4 x^2+5 x y+3 y^2-x+9 y-12=0$; \\
\textbf{A3.} Установить, какие линии определяются следующими уравнениями: $\left\{\begin{array}{l}\frac{x^2}{4}-\frac{y^2}{3}=2 z \\ x-2 y+2=0 ;\end{array}\right.$ \\
\textbf{B1.} Составить уравнение касательных к гиперболе $x^2-y^2=16$, проведенных из точки $A(-1 ;-7)$. \\
\textbf{B2.} Каждое из следующих уравнений привести к простейшему виду; определить тип каждого из них; установить, какие геометрические образы они определяют, и изобразить на чертеже расположение этих образов относительно старых и новых осей координат: $32 x^2+52 x y-7 y^2+180=0$; \\
\textbf{B3.} Установить, что следующие уравнения определяют центральные линии; преобразовать каждое из них путем переноса начала координат в центр: $6 x^2+4 x y+y^2+4 x-2 y+2=0$; \\
\textbf{C1.} Каждое из следующих уравнений привести к каноническому виду; определить тип каждого из них; установить, какие геометрические образы они определяют; для каждого случая изобразить на чертеже оси первоначальной координатной системы, оси других координатных систем, которые вводятся по ходу решения, и геометрический образ, определяемый данным уравнением: $41 x^2+24 x y+9 y^2+24 x+18 y-36=0$. \\
\textbf{C2.} Доказать, что параболическое уравнение определяет параболу в том и только в том случае, когда $\Delta \neq 0$. Доказать, что в этом случае параметр параболы определяется формулой $p=\sqrt{\frac{-\Delta}{ (a_{11}+a_{33}) ^3}}$. \\
\textbf{C3.} Составить уравнение эллипса с полуосями $a, b$ и центром $C\left(x_0 ; y_0\right)$, если известно, что оси симметрии эллипса параллельны осям координат. \\

\end{tabular}
\vspace{1cm}


\begin{tabular}{m{17cm}}
\textbf{49-variant}
\newline

\textbf{T1.} Центр, касательная плоскость и диаметральная плоскость поверхности второго порядка (Центр, касательная плоскость, диаметральная плоскость) \\
\textbf{T2.} Линии второго порядка на плоскости (Уравнение второго порядка, Уравнение квадратной формы, Конические линии (сечение конусов)) \\
\textbf{A1.} Составить уравнение эллипса, фокусы которого лежат на оси абсцисс, симметрично относительно начала координат, зная, кроме того, что: его полуоси равны 5 и 2 ; \\
\textbf{A2.} Составить уравнение гиперболы, если известны ее эксцентриситет $\varepsilon=\frac{13}{12}$, фокус $F(0 ; 13)$ и уравнение соответствующей директрисы $13 y-144=0$. \\
\textbf{A3.} Составить уравнение параболы, зная, что .ее вершина совпадает с точкой ( $\alpha ; \beta$ ), параметр равен $p$, ось параллельна оси $O x$ и парабола простирается в бесконечность: в отрицательном направлении оси $O x$. \\
\textbf{B1.} Установить, что следующие уравнения являются параболическими, и записать каждое из них в виде $(\alpha x+\beta y)^2+2 a_{13} x+2 a_{23} y+a_{33}=0$: $9 x^2-42 x y+49 y^2+3 x-2 y-24=0$. \\
\textbf{B2.} Составить уравнение прямой, которая касается параболы $y^2=8 x$ и параллельна прямой $2 x+2 y-3=0$. \\
\textbf{B3.} Установить, какая линия является сечением эллипсоида $\frac{x^2}{12}+\frac{y^2}{4}+\frac{z^2}{3}=1$ плоскостью $2 x-3 y+4 z-11=0$, и найти ее центр. \\
\textbf{C1.} Дано уравнение линии $4 x^2-4 x y+y^2+6 x+1=0$. Определить, при каких значениях углового коэффициента $k$ прямая $y=k x:$ 1) пересекает эту линию в одной точке; 2) касается этой линии; 3) пересекает эту линию в двух точках; 4) не имеет общих точек с этой линией. \\
\textbf{C2.} Установить, при каких значениях $m$ плоскость $x+m z-1=0$ пересекает двухполостный гиперболоид $x^2+y^2-z^2=-1$ а) по эллипсу, б) по гиперболе. \\
\textbf{C3.} Составить уравнение гиперболы, если известны ее полуоси $a$ и $b$, центр $C\left(x_0 ; y_0\right)$ и фокусы расположены на прямой: 1) параллельной оси $O x$; 2) параллельной оси $O y$. \\

\end{tabular}
\vspace{1cm}


\begin{tabular}{m{17cm}}
\textbf{50-variant}
\newline

\textbf{T1.} Уравнение касательной линии второго порядка, сопряжённого диаметра (Уравнение касательной, сопряжённый диаметр: оси симметрии, проходящие через центр) \\
\textbf{T2.} Парабола и её канонические уравнения (Фокус (направляющая точка), Директриса (направляющая линия), Ось (ось симметрии)) \\
\textbf{A1.} Составить уравнение гиперболы, фокусы когорой расположены на оси абсцисс симметрично относительно начала координат, зная, кроме того, что: ось $2 a=16$ и эксцентриситет $\varepsilon=\frac{5}{4}$; \\
\textbf{A2.} Составить уравнения касательных к параболе $y^2=36 x$, проведенных из точки $A(2 ; 9)$. \\
\textbf{A3.} Не проводя преобразования координат, установить, что каждое из следующих уравнений определяет гиперболу, и найти величины ее полуосей: $12 x^2+26 x y+12 y^2-52 x-48 y+73=0$ \\
\textbf{B1.} Определить эксцентриситет в эллипса, если: расстояние между директрисами в три раза больше расстояния между фокусами; \\
\textbf{B2.} Установить, что следующие уравнения являются параболическими, и записать каждое из них в виде $(\alpha x+\beta y)^2+2 a_{13} x+2 a_{23} y+a_{33}=0$: $x^2+4 x y+4 y^2+4 x+y-15=0 ;$ \\
\textbf{B3.} Составить уравнение параболы, если даны ее фокус $F(7 ; 2)$ и директриса $x-5=0$ \\
\textbf{C1.} Каждое из следующих уравнений привести к каноническому виду; определить тип каждого из них; установить, какие геометрические образы они определяют; для каждого случая изобразить на чертеже оси первоначальной координатной системы, оси других координатных систем, которые вводятся по ходу решения, и геометрический образ, определяемый данным уравнением: $7 x^2+60 x y+32 y^2-14 x-60 y+7=0$; \\
\textbf{C2.} Вывести условие, при котором прямая $y=k x+m$ касается гиперболы $\frac{x^2}{a^2}-\frac{y^2}{b^2}=1$. \\
\textbf{C3.} Составить уравнение касательной к эллипсу $\frac{x^2}{a^2}+\frac{y^2}{b^2}=1$ в его точке $M_1\left(x_1 ; y_1\right)$. \\

\end{tabular}
\vspace{1cm}


\begin{tabular}{m{17cm}}
\textbf{51-variant}
\newline

\textbf{T1.} Канонические уравнения поверхностей второго порядка (эллипсоид, гиперболоид (1-полостный), гиперболоид (2-полостный)) \\
\textbf{T2.} Центр линии второго порядка (Центровые линии (эллипс, гипербола), Координаты центра: центр симметрии) \\
\textbf{A1.} Составить уравнение эллипса, фокусы которого расположены на оси абсцисс, симметрично относительно начала координат, если даны: точка $M_1\left(2 ;-\frac{5}{3}\right)$ эллипса и его эксцентриситет $\varepsilon=\frac{2}{3}$; \\
\textbf{A2.} Найти уравнения проекций на координатные плоскости сечения эллиптического параболоида $y^2+z^2=x$ плоскостью $x+2 y-z=0$ \\
\textbf{A3.} Не проводя преобразования координат, установить, что каждое из следующих уравнений определяет параболу, и найти параметр этой параболы: $x^2-2 x y+y^2+6 x-14 y+29=0$; \\
\textbf{B1.} Составить уравнение гиперболы, зная, что: расстояние между ее вершинами равно 24 и фокусы суть $F_1(-10 ; 2), F_2(16 ; 2)$; \\
\textbf{B2.} Установить, какая линия является сечением гиперболического параболоида $\frac{x^2}{2}-\frac{z^2}{3}=y$ плоскостью $3 x-3 y+4 z+2=0$, и найти ее центр. \\
\textbf{B3.} Установить, что следующие уравнения определяют центральные линии; преобразовать каждое из них путем переноса начала координат в центр: $4 x^2+6 x y+y^2-10 x-10=0$; \\
\textbf{C1.} Доказать, что эллиптический параболоид $\frac{x^2}{9}+\frac{z^2}{4}=2 y$ имеет одну общую точку с плоскостью $2 x-2 y-z-10=0$, и найти ее координаты. \\
\textbf{C2.} При каких значениях $m$ и $n$ уравнение $m x^2+12 x y+9 y^2+4 x+n y-13=0$ определяет: 1) центральную линию; 2) линию без центра; 3) линию, имеющую бесконечно много центров. \\
\textbf{C3.} Доказать, что уравнение второй степени является уравнением вырожденной линии в том и только в том случае, когда $\Delta=0$. \\

\end{tabular}
\vspace{1cm}


\begin{tabular}{m{17cm}}
\textbf{52-variant}
\newline

\textbf{T1.} Линии второго порядка на плоскости (Уравнение второго порядка, Уравнение квадратной формы, Конические линии (сечение конусов)) \\
\textbf{T2.} Взаимное расположение линии второго порядка и прямой (Точки пересечения, касательное положение) \\
\textbf{A1.} Установить, что каждая из следующих линий имєет бесконечно много цєнтров; для каждой их них составить уравнение геометрического места центров: $x^2-6 x y+9 y^2-12 x+36 y+20=0$; \\
\textbf{A2.} Установить, что плоскость $x-2=0$ пересекает эллипсоид $\frac{x^2}{16}+\frac{y^2}{12}+\frac{z^2}{4}=1$ по эллипсу; найти его полуоси и вершины. \\
\textbf{A3.} He проводя преобразования координат, установить, что каждое из следующих уравнений определяет эллипс, и найти величины его полуосей: $13 x^2+18 x y+37 y^2-26 x-18 y+3=0$; \\
\textbf{B1.} Определить тип каждого из следующих уравнений каждое из них путем параллельного переноса осей координат привести к простейшему виду; установить, какие геометрические образы они определяют, и изобразить на чертеже расположение этих образов относительно старых и новых осей координат: $9 x^2+4 y^2+18 x-8 y+49=0$; \\
\textbf{B2.} Установить, какая линия является сечением гиперболического параболоида $\frac{x^2}{2}-\frac{z^2}{3}=y$ плоскостью $3 x-3 y+4 z+2=0$, и найти ее центр. \\
\textbf{B3.} Каждое из следующих уравнений привести к простейшему виду; определить тип каждого из них; установить, какие геометрические образы они определяют, и изобразить на чертеже расположение этих образов относительно старых и новых осей координат: $17 x^2-12 x y+8 y^2=0$; \\
\textbf{C1.} Доказать, что если две параболы со взаимно перпендикулярными осями пересекаются в четырех точках, то эти точки лежат на одной окружности. \\
\textbf{C2.} Доказать, что две параболы, имеющие общую ось и общий фокус, расположенный между их вершинами, пересекаются под прямым углом. \\
\textbf{C3.} Дано уравнение линии $4 x^2-4 x y+y^2+6 x+1=0$. Определить, при каких значениях углового коэффициента $k$ прямая $y=k x:$ 1) пересекает эту линию в одной точке; 2) касается этой линии; 3) пересекает эту линию в двух точках; 4) не имеет общих точек с этой линией. \\

\end{tabular}
\vspace{1cm}


\begin{tabular}{m{17cm}}
\textbf{53-variant}
\newline

\textbf{T1.} Прямые образующие однополостного гиперболоида и гиперболического параболоида (Гиперболоид, Гиперболический параболоид, Линейные образующие) \\
\textbf{T2.} Общие уравнения поверхностей второго порядка (Общее уравнение) \\
\textbf{A1.} Установить, какие из следующих линий являются центральными (т.е. имеют единственный центр), какие имеют центра, какие имеют бесконечно много центров: $4 x^2-4 x y+y^2-12 x+6 y-11=0$; \\
\textbf{A2.} Составить уравнение гиперболы, фокусы когорой расположены на оси абсцисс симметрично относительно начала координат, зная, кроме того, что: расстояние между директрисами равно $\frac{8}{3}$ и эксцентриситет $\varepsilon=\frac{3}{2}$; \\
\textbf{A3.} Не проводя преобразования координат, установить, что каждое из следующих уравнений определяет параболу, и найти параметр этой параболы: $9 x^2-6 x y+y^2-50 x+50 y-275=0$. \\
\textbf{B1.} Определить эксцентриситет в эллипса, если: отрезок между фоку сами виден из вєршин малой оси под прямым углом; \\
\textbf{B2.} Определить точки гиперболы $\frac{x^2}{64}-\frac{y^2}{36}=1$, расстояние которых до правого фокуса равно 4,5 . \\
\textbf{B3.} Даны вершина параболы $A(6 ;-3)$ и уравнение ее директрисы $3 x-5 y+1=0$. Найти фокус $F$ этой параболы. \\
\textbf{C1.} Составить уравнение касательной к гиперболе $\frac{x^2}{a^2}-\frac{y^2}{b^2}=1$ в ее точке $M_1\left(x_1 ; y_1\right)$. \\
\textbf{C2.} Доказать, что если уравнение второй степени является параболическим и написано в виде $ (\alpha x+\beta y) ^2+2a_{13}x+2a_{23}y+a_{33}=0$ то дискриминант его левой части определяется формулой $\Delta=- (a_{13} \beta-a_{23} \alpha) ^2$. \\
\textbf{C3.} Каждое из следующих уравнений привести к каноническому виду; определить тип каждого из них; установить, какие геометрические образы они определяют; для каждого случая изобразить на чертеже оси первоначальной координатной системы, оси других координатных систем, которые вводятся по ходу решения, и геометрический образ, определяемый данным уравнением: $14 x^2+24 x y+21 y^2-4 x+18 y-139=0$; \\

\end{tabular}
\vspace{1cm}


\begin{tabular}{m{17cm}}
\textbf{54-variant}
\newline

\textbf{T1.} Приведение общего уравнения кривой второго порядка к каноническому виду с помощью инвариантов \\
\textbf{T2.} Парабола и её канонические уравнения (Фокус (направляющая точка), Директриса (направляющая линия), Ось (ось симметрии)) \\
\textbf{A1.} Составить уравнение параболы, вершина которой находится в начале координат, зная, что: парабола расположена в правой полуплоскости симметрично относительно оси $O x$, и ее параметр $p=3$; \\
\textbf{A2.} Составить уравнение эллипса, фокусы которого лежат на оси абсцисс, симметрично относительно начала координат, зная, кроме того, что: его большая ось равна 8, а расстояние между директрисами равно 16 ; \\
\textbf{A3.} Определить точки пересечения прямой $x+y$ -$-3=0$ и параболы $x^2=4 y$. \\
\textbf{B1.} Установить, что следующие линии являются центральными, и для каждой из них найти координаты центра: $2 x^2-6 x y+5 y^2+22 x-36 y+11=0$. \\
\textbf{B2.} То же задание, что и в предыдушей задаче, выполнить для уравнений: $x^2-2 x y+y^2-12 x+12 y-14=0$ \\
\textbf{B3.} Составить уравнение эллипса, зная, что: его малая ось равна 2 и фокусы суть $F_1(-1 ;-1)$, $F_2(1 ; 1)$; \\
\textbf{C1.} Доказать, что эллипсоид $\frac{x^2}{81}+\frac{y^2}{36}+\frac{z^2}{9}=1$ имеет одну общую точку с плоскостью $4 x-3 y+12 z-54=0$, и найти ее координаты. \\
\textbf{C2.} Доказать, что произведение расстояний от центра эллипса до точки пересечения любой его касательной с фокальной осью и до основания перпендикуляря, опущенного из точки касания на фокальную ось, есть величина постоянная, равная квадрату большой полуоси эллипса. \\
\textbf{C3.} При каких значениях $m$ и $n$ уравнение $m x^2+12 x y+9 y^2+4 x+n y-13=0$ определяет: 1) центральную линию; 2) линию без центра; 3) линию, имеющую бесконечно много центров. \\

\end{tabular}
\vspace{1cm}


\begin{tabular}{m{17cm}}
\textbf{55-variant}
\newline

\textbf{T1.} Парабола и её канонические уравнения (Фокус (направляющая точка), Директриса (направляющая линия), Ось (ось симметрии)) \\
\textbf{T2.} Общие уравнения линий второго порядка (Общее уравнение) \\
\textbf{A1.} Не проводя преобразования координат, установить, что каждое из следующих уравнений определяет гиперболу, и найти величины ее полуосей: $3 x^2+4 x y-12 x+16=0$; \\
\textbf{A2.} Составить уравнение эллипса, фокусы которого лежат на оси ординат, симметрично относительно начала координат, зная, кроме того, что: его полуоси равны соответственно 7 и 2 ; \\
\textbf{A3.} Установить, что плоскость $y+6=0$ пересекает гиперболический параболоид $\frac{x^2}{5}-\frac{y^2}{4}=6 z$ по параболе; найти ее параметр и вершину. \\
\textbf{B1.} Составить уравнение гиперболы, зная, что: угол между асимптотами равен $90^{\circ}$ и фокусы суть $F_1(4 ;-4), F_2(-2 ; 2)$. \\
\textbf{B2.} Определить тип каждого из следующих уравнений каждое из них путем параллельного переноса осей координат привести к простейшему виду; установить, какие геометрические образы они определяют, и изобразить на чертеже расположение этих образов относительно старых и новых осей координат: $9 x^2-16 y^2-54 x-64 y-127=0$; \\
\textbf{B3.} Установить, какая линия является сечением эллипсоида $\frac{x^2}{12}+\frac{y^2}{4}+\frac{z^2}{3}=1$ плоскостью $2 x-3 y+4 z-11=0$, и найти ее центр. \\
\textbf{C1.} Определить, при каком значении $m$ плоскость $x-2 y-2 z+m=0$ касается эллипсоида $\frac{x^2}{144}+\frac{y^2}{36}+\frac{z^2}{9}=1$. \\
\textbf{C2.} Доказать, что любое параболическое уравнение может быть написано в виде: $ (\alpha x+\beta y) ^2+2a_{13}x+2a_{23}y+a_{33}=0$. Доказать также, что эллиптические и гиперболические уравнения в таком виде не могут быть написаны. \\
\textbf{C3.} Каждое из следующих уравнений привести к каноническому виду; определить тип каждого из них; установить, какие геометрические образы они определяют; для каждого случая изобразить на чертеже оси первоначальной координатной системы, оси других координатных систем, которые вводятся по ходу решения, и геометрический образ, определяемый данным уравнением: $7 x^2+6 x y-y^2+28 x+12 y+28=0$; \\

\end{tabular}
\vspace{1cm}


\begin{tabular}{m{17cm}}
\textbf{56-variant}
\newline

\textbf{T1.} Приведение общего уравнения поверхности второго порядка к каноническому виду с помощью инвариантов \\
\textbf{T2.} Линии второго порядка на плоскости (Уравнение второго порядка, Уравнение квадратной формы, Конические линии (сечение конусов)) \\
\textbf{A1.} Не проводя преобразования координат, установить, что каждое из следующих уравнений определяет параболу, и найти параметр этой параболы: $9 x^2-24 x y+16 y^2-54 x-178 y+181=0$; \\
\textbf{A2.} Составить уравнение гиперболы, фокусы которой расположены на оси ординат симметрично относительно начала координат, зная, кроме того, что: уравнения асимптот $y= \pm \frac{12}{5} x$ и расстояние между вершинами равно 48; \\
\textbf{A3.} Установить, какие из следующих линий являются центральными (т.е. имеют единственный центр), какие имеют центра, какие имеют бесконечно много центров: $x^2-2 x y+4 y^2+5 x-7 y+12=0$; \\
\textbf{B1.} Установить, что следующие линии являются центральными, и для каждой из них найти координаты центра: $3 x^2+5 x y+y^2-8 x-11 y-7=0$; \\
\textbf{B2.} Установить, что каждое из следующих уравнений является параболическим; каждое из них привести к простейшему виду; установить, какие геометрические образы они определяют; для каждого случая изобразить на чертеже оси первоначальной координатной системы, оси других координатных систем, которые вводятся по ходу решения, и геометрический образ, определяемый данным уравнением: $16 x^2-24 x y+9 y^2-160 x+120 y+425=0$. \\
\textbf{B3.} Из точки $A(5 ; 9)$ проведены касательные к параболе $y^2=5 x$. Составить уравнение хорды, соединяющей точки касания. \\
\textbf{C1.} Доказать, что касательные к гиперболе, проведенные в концах одного и того же диаметра, параллельны. \\
\textbf{C2.} Доказать, что произведение расстояний от фокусов до любой касательной к эллипсу равно квадрату малой полуоси. \\
\textbf{C3.} Составить уравнение касагельной к параболе $y^2=2 p x$ в ее точке $M_1\left(x_1 ; y_1\right)$. \\

\end{tabular}
\vspace{1cm}


\begin{tabular}{m{17cm}}
\textbf{57-variant}
\newline

\textbf{T1.} Центр линии второго порядка (Центровые линии (эллипс, гипербола), Координаты центра: центр симметрии) \\
\textbf{T2.} Центр, касательная плоскость и диаметральная плоскость поверхности второго порядка (Центр, касательная плоскость, диаметральная плоскость) \\
\textbf{A1.} Не проводя преобразования координат, установить, что каждое из следующих уравнений определяет параболу, и найти параметр этой параболы: $9 x^2+24 x y+16 y^2-120 x+90 y=0$; \\
\textbf{A2.} Установить, какие из следующих линий являются центральными (т.е. имеют единственный центр), какие имеют центра, какие имеют бесконечно много центров: $4 x^2+5 x y+3 y^2-x+9 y-12=0$; \\
\textbf{A3.} Не проводя преобразования координат, установить, что каждое из следующих уравнений определяет единственную точку (вырожденный эллипс), и найти ее координаты: $5 x^2-6 x y+2 y^2-2 x+2=0$; \\
\textbf{B1.} Составить уравнение эллипса, если известны его эксцентриситет $\varepsilon=\frac{1}{2}$, фокус $F(3 ; 0)$ и уравнение соответствующей директрисы $x+y-1=0$. \\
\textbf{B2.} Даны вершина параболы $A(6 ;-3)$ и уравнение ее директрисы $3 x-5 y+1=0$. Найти фокус $F$ этой параболы. \\
\textbf{B3.} Каждое из следующих уравнений привести к простейшему виду; определить тип каждого из них; установить, какие геометрические образы они определяют, и изобразить на чертеже расположение этих образов относительно старых и новых осей координат: $5 x^2+24 x y-5 y^2=0$; \\
\textbf{C1.} Для любого параболического уравнения доказать, что коэффициенты $a_{11}$ и $a_{22}$ не могут быть числами разных знаков и что они одновременно не могут обрашаться в нуль. \\
\textbf{C2.} Доказать, что эллипсоид $\frac{x^2}{81}+\frac{y^2}{36}+\frac{z^2}{9}=1$ имеет одну общую точку с плоскостью $4 x-3 y+12 z-54=0$, и найти ее координаты. \\
\textbf{C3.} Для любого эллиптического уравнения доказать, что ни один из коэффициентов $a_{11}$ и $a_{22}$ не может обрашаться в нуль и что они суть числа одного знака. \\

\end{tabular}
\vspace{1cm}


\begin{tabular}{m{17cm}}
\textbf{58-variant}
\newline

\textbf{T1.} Линии второго порядка на плоскости (Уравнение второго порядка, Уравнение квадратной формы, Конические линии (сечение конусов)) \\
\textbf{T2.} Общие уравнения поверхностей второго порядка (Общее уравнение) \\
\textbf{A1.} Определить точки пересечения гиперболы $\frac{x^2}{20}$ -$-\frac{y^2}{5}=-1$ и параболы $y^2=3 x$ \\
\textbf{A2.} Составить уравнение эллипса, фокусы которого лежат на оси абсцисс, симметрично относительно начала координат, зная, кроме того, что: расстояние между его фокусами $2 c=6$ и эксцентриситет $\varepsilon=\frac{3}{5}$; \\
\textbf{A3.} Составить уравнение гиперболы, фокусы когорой расположены на оси абсцисс симметрично относительно начала координат, зная, кроме того, что: расстсяние между фокусами $2 c=10$ и ось $2 b=8$; \\
\textbf{B1.} Установить, что следующие линии являются центральными, и для каждой из них найти координаты центра: $5 x^2+4 x y+2 y^2+20 x+20 y-18=0$; \\
\textbf{B2.} Составить уравнение гиперболы, зная, что: фокусы суть $F_1(3 ; 4), F_2(-3 ;-4)$ и расстояние между директрисами равно 3,6 ; \\
\textbf{B3.} Установить, что следующие уравнения являются параболическими, и записать каждое из них в виде $(\alpha x+\beta y)^2+2 a_{13} x+2 a_{23} y+a_{33}=0$: $16 x^2+16 x y+4 y^2-5 x+7 y=0$; \\
\textbf{C1.} Определить, при каких значениях углового коэффициента $k$ прямая $y=k x+2$ 1) пересекает параболу $y^2=4 x$; 2) касается ее; 3) проходит вне этсй параболы. \\
\textbf{C2.} Доказать, что касательные к эллипсу $\frac{x^2}{a^2}+\frac{y^2}{b^2}=1$, проведенные в концах одного и того же диаметра, параллельны. (Диаметром эллипса называется его хорда, проходящая через центр.) \\
\textbf{C3.} Дано уравнение линии $4 x^2-4 x y+y^2+6 x+1=0$. Определить, при каких значениях углового коэффициента $k$ прямая $y=k x:$ 1) пересекает эту линию в одной точке; 2) касается этой линии; 3) пересекает эту линию в двух точках; 4) не имеет общих точек с этой линией. \\

\end{tabular}
\vspace{1cm}


\begin{tabular}{m{17cm}}
\textbf{59-variant}
\newline

\textbf{T1.} Взаимное расположение линии второго порядка и прямой (Точки пересечения, касательное положение) \\
\textbf{T2.} Парабола и её канонические уравнения (Фокус (направляющая точка), Директриса (направляющая линия), Ось (ось симметрии)) \\
\textbf{A1.} Установить, какие линии определяются следующими уравнениями: $\left\{\begin{array}{l}\frac{x^2}{4}-\frac{y^2}{3}=2 z \\ x-2 y+2=0 ;\end{array}\right.$ \\
\textbf{A2.} Не проводя преобразования координат, установить, что каждое из следующих уравнений определяет параболу, и найти параметр этой параболы: $9 x^2-24 x y+16 y^2-54 x-178 y+181=0$; \\
\textbf{A3.} Вычислить площадь треугольника, образованного асимптотами гиперболы $\frac{x^2}{4}-\frac{y^2}{9}=1$ и прямой $9 x+2 y-24=0$ \\
\textbf{B1.} Установить, какая линия является сечением гиперболического параболоида $\frac{x^2}{2}-\frac{z^2}{3}=y$ плоскостью $3 x-3 y+4 z+2=0$, и найти ее центр. \\
\textbf{B2.} Каждое из следующих уравнений привести к простейшему виду; определить тип каждого из них; установить, какие геометрические образы они определяют, и изобразить на чертеже расположение этих образов относительно старых и новых осей координат: $5 x^2-6 x y+5 y^2-32=0$; \\
\textbf{B3.} Составить уравнение прямой, которая касается параболы $x^2=16 y$ и перпендикулярна к прямой $2 x+4 y+7=0$. \\
\textbf{C1.} Определить, при каких значениях $m$ прямая $y=\frac{5}{2} x+m$ пересекает гиперболу $\frac{x^2}{9}-\frac{y^2}{36}=1$; 2) касается ее; 3) проходит вне этой гиперболы \\
\textbf{C2.} Доказать, что площадь параллелограмма, ограниченного асимптотами гиперболы $\frac{x^2}{a^2}-\frac{y^2}{b^2}=1$ и прямыми, проведенными через любую ее точку параллельно асимптотам, есть величина постоянная, равная $\frac{a b}{2}$. \\
\textbf{C3.} Доказать, что двухполостный гиперболоид $\frac{x^2}{3}+\frac{y^2}{4}-\frac{z^2}{25}=-1$ имеет одну общую точку с плоскостью $5 x+2 z+5=0$, и найти ее координаты. \\

\end{tabular}
\vspace{1cm}


\begin{tabular}{m{17cm}}
\textbf{60-variant}
\newline

\textbf{T1.} Канонические уравнения поверхностей второго порядка (эллипсоид, гиперболоид (1-полостный), гиперболоид (2-полостный)) \\
\textbf{T2.} Уравнение касательной линии второго порядка, сопряжённого диаметра (Уравнение касательной, сопряжённый диаметр: оси симметрии, проходящие через центр) \\
\textbf{A1.} He проводя преобразования координат, установить, что каждое из следующих уравнений определяет эллипс, и найти величины его полуосей: $41 x^2+24 x y+9 y^2+24 x+18 y-36=0$; \\
\textbf{A2.} Вычислить площадь четырехугольника, две вершины которого лежат в фокусах эллипса $9 x^2+5 y^2=1$, две другие совпадают с концами его малой оси. \\
\textbf{A3.} Установить, что плоскость $z+1=0$ пересекает однополостный гиперболоид $\frac{x^2}{32}-\frac{y^2}{18}+\frac{z^2}{2}=1$ по гиперболе; найти ее полуоси и вершины. \\
\textbf{B1.} Установить, что каждое из следующих уравнений является параболическим; каждое из них привести к простейшему виду; установить, какие геометрические образы они определяют; для каждого случая изобразить на чертеже оси первоначальной координатной системы, оси других координатных систем, которые вводятся по ходу решения, и геометрический образ, определяемый данным уравнением: $9 x^2+12 x y+4 y^2-24 x-16 y+3=0$; \\
\textbf{B2.} Установить, какая линия является сечением эллипсоида $\frac{x^2}{12}+\frac{y^2}{4}+\frac{z^2}{3}=1$ плоскостью $2 x-3 y+4 z-11=0$, и найти ее центр. \\
\textbf{B3.} Провести касательные к гиперболе $\frac{x^2}{16}-\frac{y^2}{8}=-1$ параллельно прямой $2 x+4 y-5=0$ и вычис лить расстояние $d$ между ними. \\
\textbf{C1.} При каких значениях $m$ и $n$ уравнение $m x^2+12 x y+9 y^2+4 x+n y-13=0$ определяет: 1) центральную линию; 2) линию без центра; 3) линию, имеющую бесконечно много центров. \\
\textbf{C2.} Доказать, что если две параболы со взаимно перпендикулярными осями пересекаются в четырех точках, то эти точки лежат на одной окружности. \\
\textbf{C3.} Каждое из следующих уравнений привести к каноническому виду; определить тип каждого из них; установить, какие геометрические образы они определяют; для каждого случая изобразить на чертеже оси первоначальной координатной системы, оси других координатных систем, которые вводятся по ходу решения, и геометрический образ, определяемый данным уравнением: $25 x^2-14 x y+25 y^2+64 x-64 y-224=0$; \\

\end{tabular}
\vspace{1cm}


\begin{tabular}{m{17cm}}
\textbf{61-variant}
\newline

\textbf{T1.} Канонические уравнения поверхностей второго порядка (Параболоид (эллиптический), Параболоид (гиперболический), Конус, Цилиндр) \\
\textbf{T2.} Парабола и её канонические уравнения (Фокус (направляющая точка), Директриса (направляющая линия), Ось (ось симметрии)) \\
\textbf{A1.} Установить, что каждая из следующих линий имєет бесконечно много цєнтров; для каждой их них составить уравнение геометрического места центров: $4 x^2+4 x y+y^2-8 x-4 y-21=0$; \\
\textbf{A2.} Найти фокус $F$ и уравнение директрисы параболы $y^2=24 x$. \\
\textbf{A3.} Определить точки эллипса $\frac{x^2}{100}+\frac{y^2}{36}=1$, pacстояние которых до правого фокуса равно 14. \\
\textbf{B1.} Составить уравнение эллипса, зная, что: его фокусы суть $F_1\left(-2 ; \frac{3}{2}\right), F_2\left(2 ;-\frac{3}{2}\right)$ и эксцентриситет $\varepsilon=\frac{\sqrt{2}}{2}$; \\
\textbf{B2.} Установить, что следующие линии являются центральными, и для каждой из них найти координаты центра: $3 x^2+5 x y+y^2-8 x-11 y-7=0$; \\
\textbf{B3.} Через фокус эллипса $\frac{x^2}{25}+\frac{y^2}{15}=1$ проведен перпендикуляр к его большой оси. Определить расстояния от точек пересечения этого перпендикуляра с эллипсом до фокусов. \\
\textbf{C1.} Из точки $A\left(\frac{10}{3} ; \frac{5}{3}\right)$ проведены касательные к эллипсу $\frac{x^2}{20}+\frac{y^2}{5}=1$. Составить их уравнения. \\
\textbf{C2.} Доказать, что любое параболическое уравнение может быть написано в виде: $ (\alpha x+\beta y) ^2+2a_{13}x+2a_{23}y+a_{33}=0$. Доказать также, что эллиптические и гиперболические уравнения в таком виде не могут быть написаны. \\
\textbf{C3.} Каждое из следующих уравнений привести к каноническому виду; определить тип каждого из них; установить, какие геометрические образы они определяют; для каждого случая изобразить на чертеже оси первоначальной координатной системы, оси других координатных систем, которые вводятся по ходу решения, и геометрический образ, определяемый данным уравнением: $11 x^2-20 x y-4 y^2-20 x-8 y+1=0$; \\

\end{tabular}
\vspace{1cm}


\begin{tabular}{m{17cm}}
\textbf{62-variant}
\newline

\textbf{T1.} Уравнение касательной линии второго порядка, сопряжённого диаметра (Уравнение касательной, сопряжённый диаметр: оси симметрии, проходящие через центр) \\
\textbf{T2.} Общие уравнения линий второго порядка (Общее уравнение) \\
\textbf{A1.} Не проводя преобразования координат, установить, что каждое из следующих уравнений определяет параболу, и найти параметр этой параболы: $9 x^2-6 x y+y^2-50 x+50 y-275=0$. \\
\textbf{A2.} Не проводя преобразования координат, установить, что каждое из следующих уравнений определяет пару пересекающихся прямых (вырожденную гиперболу), и найти их уравнения: $x^2-4 x y+3 y^2=0$; \\
\textbf{A3.} Установить, какие из следующих линий являются центральными (т.е. имеют единственный центр), какие имеют центра, какие имеют бесконечно много центров: $3 x^2-4 x y-2 y^2+3 x-12 y-7=0$; \\
\textbf{B1.} Даны вершина параболы $A(-2 ;-1)$ и урав нение ее директрисы $x+2 y-1=0$. Составить уравнение этой параболы. \\
\textbf{B2.} То же задание, что и в предыдушей задаче, выполнить для уравнений: $9 x^2+24 x y+16 y^2-18 x+226 y+209=0$; \\
\textbf{B3.} Установить, какая линия является сечением эллипсоида $\frac{x^2}{12}+\frac{y^2}{4}+\frac{z^2}{3}=1$ плоскостью $2 x-3 y+4 z-11=0$, и найти ее центр. \\
\textbf{C1.} Определить, при каком значении $m$ плоскость $x-2 y-2 z+m=0$ касается эллипсоида $\frac{x^2}{144}+\frac{y^2}{36}+\frac{z^2}{9}=1$. \\
\textbf{C2.} Доказать, что произведение расстояний от центра эллипса до точки пересечения любой его касательной с фокальной осью и до основания перпендикуляря, опущенного из точки касания на фокальную ось, есть величина постоянная, равная квадрату большой полуоси эллипса. \\
\textbf{C3.} Доказать, что если уравнение второй степени является параболическим и написано в виде $ (\alpha x+\beta y) ^2+2a_{13}x+2a_{23}y+a_{33}=0$ то дискриминант его левой части определяется формулой $\Delta=- (a_{13} \beta-a_{23} \alpha) ^2$. \\

\end{tabular}
\vspace{1cm}


\begin{tabular}{m{17cm}}
\textbf{63-variant}
\newline

\textbf{T1.} Линии второго порядка на плоскости (Уравнение второго порядка, Уравнение квадратной формы, Конические линии (сечение конусов)) \\
\textbf{T2.} Прямые образующие однополостного гиперболоида и гиперболического параболоида (Гиперболоид, Гиперболический параболоид, Линейные образующие) \\
\textbf{A1.} Составить уравнение параболы, вершина которой находится в начале координат, зная, что: парабола расположена в нижней полуплоскости симметрично относительно оси $O y$, и ее параметр $p=3$. \\
\textbf{A2.} Установить, какие линии определяются следующими уравнениями: $y=+\frac{2}{5} \sqrt{x^2+25}$ \\
\textbf{A3.} Установить, какие линии определяются следующими уравнениями: $\left\{\begin{array}{l}\frac{x^2}{3}+\frac{y^2}{6}=2 z, \\ 3 x-y+6 z-14=0\end{array}\right.$ \\
\textbf{B1.} Каждое из следующих уравнений привести к простейшему виду; определить тип каждого из них; установить, какие геометрические образы они определяют, и изобразить на чертеже расположение этих образов относительно старых и новых осей координат: $5 x^2-6 x y+5 y^2+8=0$. \\
\textbf{B2.} Определить точки гиперболы $\frac{x^2}{64}-\frac{y^2}{36}=1$, расстояние которых до правого фокуса равно 4,5 . \\
\textbf{B3.} Установить, что следующие уравнения определяют центральные линии; преобразовать каждое из них путем переноса начала координат в центр: $6 x^2+4 x y+y^2+4 x-2 y+2=0$; \\
\textbf{C1.} При каких значениях $m$ и $n$ уравнение $m x^2+12 x y+9 y^2+4 x+n y-13=0$ определяет: 1) центральную линию; 2) линию без центра; 3) линию, имеющую бесконечно много центров. \\
\textbf{C2.} Даны гиперболы $\frac{x^2}{a^2}-\frac{y^2}{b^2}=1$ и какая-нибудь ее касательная: $P$-точка пересечения касательной с осью $O x, Q$ - проекция точки касания на ту же ось. Доказать, что $O P \cdot O Q=a^2$. \\
\textbf{C3.} Доказать, что две параболы, имеющие общую ось и общий фокус, расположенный между их вершинами, пересекаются под прямым углом. \\

\end{tabular}
\vspace{1cm}


\begin{tabular}{m{17cm}}
\textbf{64-variant}
\newline

\textbf{T1.} Взаимное расположение линии второго порядка и прямой (Точки пересечения, касательное положение) \\
\textbf{T2.} Линии второго порядка на плоскости (Уравнение второго порядка, Уравнение квадратной формы, Конические линии (сечение конусов)) \\
\textbf{A1.} Установить, какие линии определяются следующими уравнениями: $\left\{\begin{array}{l}\frac{x^2}{.4}+\frac{y^2}{9}-\frac{z^2}{36}=1, \\ 9 x-6 y+2 z-28=0,\end{array}\right.$ \\
\textbf{A2.} Установить, какие из следующих линий являются центральными (т.е. имеют единственный центр), какие имеют центра, какие имеют бесконечно много центров: $4 x^2-6 x y-9 y^2+3 x-7 y+12=0$. \\
\textbf{A3.} Не проводя преобразования координат, установить, что каждое из следующих уравнений определяет параболу, и найти параметр этой параболы: $x^2-2 x y+y^2+6 x-14 y+29=0$; \\
\textbf{B1.} Установить, что следующие уравнения определяют центральные линии; преобразовать каждое из них путем переноса начала координат в центр: $4 x^2+2 x y+6 y^2+6 x-10 y+9=0$. \\
\textbf{B2.} Вычислить площадь четырехугольника, две вершины которого лежат в фокусах эллипса $x^2+5 y^2=20$, а две другие совпадают с концами его малой оси. \\
\textbf{B3.} Установить, что следующие уравнения являются параболическими, и записать каждое из них в виде $(\alpha x+\beta y)^2+2 a_{13} x+2 a_{23} y+a_{33}=0$: $x^2+4 x y+4 y^2+4 x+y-15=0 ;$ \\
\textbf{C1.} Определить, при каких значениях углового коэффициента $k$ прямая $y=k x+2$ 1) пересекает параболу $y^2=4 x$; 2) касается ее; 3) проходит вне этсй параболы. \\
\textbf{C2.} Дано уравнение линии $4 x^2-4 x y+y^2+6 x+1=0$. Определить, при каких значениях углового коэффициента $k$ прямая $y=k x:$ 1) пересекает эту линию в одной точке; 2) касается этой линии; 3) пересекает эту линию в двух точках; 4) не имеет общих точек с этой линией. \\
\textbf{C3.} Каждое из следующих уравнений привести к каноническому виду; определить тип каждого из них; установить, какие геометрические образы они определяют; для каждого случая изобразить на чертеже оси первоначальной координатной системы, оси других координатных систем, которые вводятся по ходу решения, и геометрический образ, определяемый данным уравнением: $3 x^2+10 x y+3 y^2-2 x-14 y-13=0$; \\

\end{tabular}
\vspace{1cm}


\begin{tabular}{m{17cm}}
\textbf{65-variant}
\newline

\textbf{T1.} Канонические уравнения поверхностей второго порядка (эллипсоид, гиперболоид (1-полостный), гиперболоид (2-полостный)) \\
\textbf{T2.} Приведение общего уравнения кривой второго порядка к каноническому виду с помощью инвариантов \\
\textbf{A1.} Стальной трос подвешен за два конца; точки креп.тения расположены на одинаковой высоте; расстояние между ними равно 20 м. Величина его прогиба на расстолиии 2 m от точки крепления, считая по горизонтали, равна 14,4 см. Определить величину прогиба этого троса в ссредине между точками крепления, приближенно считая, что трос имеет форму дуги параболы. \\
\textbf{A2.} Установить, какие линии определяются следующими уравнениями: $x=-\frac{4}{3} \sqrt{y^2+9} ;$ \\
\textbf{A3.} Не проводя преобразования координат, установить, что каждое из следующих уравнений определяет пару пересекающихся прямых (вырожденную гиперболу), и найти их уравнения: $x^2-6 x y+8 y^2-4 y-4=0$; \\
\textbf{B1.} Установить, какая линия является сечением гиперболического параболоида $\frac{x^2}{2}-\frac{z^2}{3}=y$ плоскостью $3 x-3 y+4 z+2=0$, и найти ее центр. \\
\textbf{B2.} Составить уравнение гиперболы, зная, что: расстояние между ее вершинами равно 24 и фокусы суть $F_1(-10 ; 2), F_2(16 ; 2)$; \\
\textbf{B3.} Из точки $A(5 ; 9)$ проведены касательные к параболе $y^2=5 x$. Составить уравнение хорды, соединяющей точки касания. \\
\textbf{C1.} Доказать что произведение расстояний от любой точки гиперболы $\frac{x^2}{a^2}-\frac{y^2}{b^2}=1$ до двух ее асимптот есть величина постоянная, равная $\frac{a^2 b^2}{a^2+b^2}$. \\
\textbf{C2.} Установить, при каких значениях $m$ плоскость $x+m z-1=0$ пересекает двухполостный гиперболоид $x^2+y^2-z^2=-1$ а) по эллипсу, б) по гиперболе. \\
\textbf{C3.} Составить уравнение эллипса с полуосями $a, b$ и центром $C\left(x_0 ; y_0\right)$, если известно, что оси симметрии эллипса параллельны осям координат. \\

\end{tabular}
\vspace{1cm}


\begin{tabular}{m{17cm}}
\textbf{66-variant}
\newline

\textbf{T1.} Общие уравнения поверхностей второго порядка (Общее уравнение) \\
\textbf{T2.} Парабола и её канонические уравнения (Фокус (направляющая точка), Директриса (направляющая линия), Ось (ось симметрии)) \\
\textbf{A1.} Составить уравнение эллипса, фокусы которого лежат на оси абсцисс, симметрично относительно начала координат, зная, кроме того, что: его большая ось равна 10 , а расстояние между фокусами $2 c=8$; \\
\textbf{A2.} Дана гипербола $16 x^2-9 y^2=144$. Найти: 1) полуоси $a$ и $b ; 2$ ) фокусы; 3) эксцентриситет; 4) уравнения асимптот; 5) уравнения директрис. \\
\textbf{A3.} Составить уравнение параболы, зная, что .ее вершина совпадает с точкой ( $\alpha ; \beta$ ), параметр равен $p$, ось параллельна оси $O x$ и парабола простирается в бесконечность: в отрицательном направлении оси $O y$. \\
\textbf{B1.} Определить тип каждого из следующих уравнений каждое из них путем параллельного переноса осей координат привести к простейшему виду; установить, какие геометрические образы они определяют, и изобразить на чертеже расположение этих образов относительно старых и новых осей координат: $4 x^2+9 y^2-40 x+36 y+100=0$; \\
\textbf{B2.} Составить уравнение параболы, если даны ее фокус $F(2 ;-1)$ и директриса $x-y-1=0$. \\
\textbf{B3.} Установить, что следующие линии являются центральными, и для каждой из них найти координаты центра: $2 x^2-6 x y+5 y^2+22 x-36 y+11=0$. \\
\textbf{C1.} Для любого параболического уравнения доказать, что коэффициенты $a_{11}$ и $a_{22}$ не могут быть числами разных знаков и что они одновременно не могут обрашаться в нуль. \\
\textbf{C2.} Составить уравнение гиперболы, касающейся двух прямых: $\quad 5 x-6 y-16=0, \quad 13 x-10 y-48=0$, при условии, что ее оси совпадают с осями координат. \\
\textbf{C3.} Определить, при каких значениях $m$ прямая $y=-x+m$ 1) пересекает эллипс $\frac{x^2}{20}+\frac{y^2}{5}=1$; 2) касается его; 3) проходит вне этого эллипса. \\

\end{tabular}
\vspace{1cm}


\begin{tabular}{m{17cm}}
\textbf{67-variant}
\newline

\textbf{T1.} Линии второго порядка на плоскости (Уравнение второго порядка, Уравнение квадратной формы, Конические линии (сечение конусов)) \\
\textbf{T2.} Центр линии второго порядка (Центровые линии (эллипс, гипербола), Координаты центра: центр симметрии) \\
\textbf{A1.} Не проводя преобразования координат, установить, что каждое из следующих уравнений определяет параболу, и найти параметр этой параболы: $9 x^2-24 x y+16 y^2-54 x-178 y+181=0$; \\
\textbf{A2.} Не проводя преобразования координат, установить, какие геометрические образы определяются следующими уравнениями: $8 x^2-12 x y+17 y^2+16 x-12 y+3=0$; \\
\textbf{A3.} Установить, что плоскость $x-2=0$ пересекает эллипсоид $\frac{x^2}{16}+\frac{y^2}{12}+\frac{z^2}{4}=1$ по эллипсу; найти его полуоси и вершины. \\
\textbf{B1.} Эксцентриситет эллипса $\varepsilon=\frac{1}{2}$, центр его совпадает с началом координат, одна из директрис дана уравнением $x=16$. Вычислить расстояние от точки $M_1$ эллипса с абсциссой, равной -4, до фокуса, одностороннего с данной директрисой. \\
\textbf{B2.} Составить уравнение касательных к гиперболе $x^2-y^2=16$, проведенных из точки $A(-1 ;-7)$. \\
\textbf{B3.} Установить, какая линия является сечением гиперболического параболоида $\frac{x^2}{2}-\frac{z^2}{3}=y$ плоскостью $3 x-3 y+4 z+2=0$, и найти ее центр. \\
\textbf{C1.} Дано уравнение линии $4 x^2-4 x y+y^2+6 x+1=0$. Определить, при каких значениях углового коэффициента $k$ прямая $y=k x:$ 1) пересекает эту линию в одной точке; 2) касается этой линии; 3) пересекает эту линию в двух точках; 4) не имеет общих точек с этой линией. \\
\textbf{C2.} Вывести условие, при котором прямая $y=k x+b$ касается параболы $y^2=2 p x$. \\
\textbf{C3.} Доказать, что уравнение второй степени является уравнением вырожденной линии в том и только в том случае, когда $\Delta=0$. \\

\end{tabular}
\vspace{1cm}


\begin{tabular}{m{17cm}}
\textbf{68-variant}
\newline

\textbf{T1.} Канонические уравнения поверхностей второго порядка (Параболоид (эллиптический), Параболоид (гиперболический), Конус, Цилиндр) \\
\textbf{T2.} Приведение общего уравнения кривой второго порядка к каноническому виду с помощью инвариантов \\
\textbf{A1.} Установить, что каждая из следующих линий имєет бесконечно много цєнтров; для каждой их них составить уравнение геометрического места центров: $x^2-6 x y+9 y^2-12 x+36 y+20=0$; \\
\textbf{A2.} Составить уравнение эллипса, фокусы которого лежат на оси ординат, симметрично относительно начала координат, зная, кроме того, что: расстояние между его фокусами $2 c=6$ и расстояние между директрисами равно $16 \frac{2}{3}$; \\
\textbf{A3.} Установить, какие из следующих линий являются центральными (т.е. имеют единственный центр), какие имеют центра, какие имеют бесконечно много центров: $4 x^2-20 x y+25 y^2-14 x+2 y-15=0$; \\
\textbf{B1.} Определить тип каждого из следующих уравнений каждое из них путем параллельного переноса осей координат привести к простейшему виду; установить, какие геометрические образы они определяют, и изобразить на чертеже расположение этих образов относительно старых и новых осей координат: $4 x^2-y^2+8 x-2 y+3=0$; \\
\textbf{B2.} Установить, что каждое из следующих уравнений является параболическим; каждое из них привести к простейшему виду; установить, какие геометрические образы они определяют; для каждого случая изобразить на чертеже оси первоначальной координатной системы, оси других координатных систем, которые вводятся по ходу решения, и геометрический образ, определяемый данным уравнением: $9 x^2-24 x y+16 y^2-20 x+110 y-50=0$; \\
\textbf{B3.} Эксцентриситет эллипса $\varepsilon=\frac{2}{3}$, фокальный радиус точки $M$ эллипса равен 10 . Вычислить расстояние от точки $M$ до односторонней с этим фокусом директрисы. \\
\textbf{C1.} Установить, при каких значениях $m$ плоскость $x+m y-2=0$ пересекает эллиптический параболоид $\frac{x^2}{2}+\frac{z^2}{3}=y$ а) по эллипсу, б) по параболе. \\
\textbf{C2.} Доказать, что эллиптическое уравнение второй степени ( $\delta>0$ ) определяет эллипс в том и только в том случае, когда $a_{11}$ и $\Delta$ суть числа разных знаков. \\
\textbf{C3.} Каждое из следующих уравнений привести к каноническому виду; определить тип каждого из них; установить, какие геометрические образы они определяют; для каждого случая изобразить на чертеже оси первоначальной координатной системы, оси других координатных систем, которые вводятся по ходу решения, и геометрический образ, определяемый данным уравнением: $4 x y+3 y^2+16 x+12 y-36=0$; \\

\end{tabular}
\vspace{1cm}


\begin{tabular}{m{17cm}}
\textbf{69-variant}
\newline

\textbf{T1.} Парабола и её канонические уравнения (Фокус (направляющая точка), Директриса (направляющая линия), Ось (ось симметрии)) \\
\textbf{T2.} Прямые образующие однополостного гиперболоида и гиперболического параболоида (Гиперболоид, Гиперболический параболоид, Линейные образующие) \\
\textbf{A1.} Определить тип каждого из следующих уравнений при помощи вычисления дискриминанта старших членов: $25 x^2-20 x y+4 y^2-12 x+20 y-17=0$; \\
\textbf{A2.} Установить, что плоскость $z+1=0$ пересекает однополостный гиперболоид $\frac{x^2}{32}-\frac{y^2}{18}+\frac{z^2}{2}=1$ по гиперболе; найти ее полуоси и вершины. \\
\textbf{A3.} Не проводя преобразования координат, установить, что каждое из следующих уравнений определяет параболу, и найти параметр этой параболы: $9 x^2+24 x y+16 y^2-120 x+90 y=0$; \\
\textbf{B1.} Установить, что следующие уравнения определяют центральные линии; преобразовать каждое из них путем переноса начала координат в центр: $4 x^2+6 x y+y^2-10 x-10=0$; \\
\textbf{B2.} Установить, какая линия является сечением эллипсоида $\frac{x^2}{12}+\frac{y^2}{4}+\frac{z^2}{3}=1$ плоскостью $2 x-3 y+4 z-11=0$, и найти ее центр. \\
\textbf{B3.} Составить уравнения касательных к гиперболе $\frac{x^2}{16}-\frac{y^2}{64}=1$, параллельных прямой $10 x-3 y+9=0$. \\
\textbf{C1.} Доказать, что эллиптический параболоид $\frac{x^2}{9}+\frac{z^2}{4}=2 y$ имеет одну общую точку с плоскостью $2 x-2 y-z-10=0$, и найти ее координаты. \\
\textbf{C2.} Составить уравнение касательной к эллипсу $\frac{x^2}{a^2}+\frac{y^2}{b^2}=1$ в его точке $M_1\left(x_1 ; y_1\right)$. \\
\textbf{C3.} Составить уравнение касагельной к параболе $y^2=2 p x$ в ее точке $M_1\left(x_1 ; y_1\right)$. \\

\end{tabular}
\vspace{1cm}


\begin{tabular}{m{17cm}}
\textbf{70-variant}
\newline

\textbf{T1.} Взаимное расположение линии второго порядка и прямой (Точки пересечения, касательное положение) \\
\textbf{T2.} Приведение общего уравнения поверхности второго порядка к каноническому виду с помощью инвариантов \\
\textbf{A1.} Составить уравнение гиперболы, фокусы которой расположены на оси ординат симметрично относительно начала координат, зная, кроме того, что: ее полуоси $a=6, b=18$ (буквой $a$ мы обозначаем полуось гинерболы, расположенную на оси абсцисс) ; \\
\textbf{A2.} Составить уравнение параболы, зная, что .ее вершина совпадает с точкой ( $\alpha ; \beta$ ), параметр равен $p$, ось параллельна оси $O x$ и парабола простирается в бесконечность: в положительном направлении оси $O y$; \\
\textbf{A3.} Составить уравнение эллипса, фокусы которого лежат на оси абсцисс, симметрично относительно начала координат, зная, кроме того, что: его малая ось равна 24 , а расстояние между фокусами $2 c=10$; \\
\textbf{B1.} Составить уравнение прямой, которая касается параболы $y^2=8 x$ и параллельна прямой $2 x+2 y-3=0$. \\
\textbf{B2.} Определить тип каждого из следующих уравнений каждое из них путем параллельного переноса осей координат привести к простейшему виду; установить, какие геометрические образы они определяют, и изобразить на чертеже расположение этих образов относительно старых и новых осей координат: $2 x^2+3 y^2+8 x-6 y+11=0$. \\
\textbf{B3.} То же задание, что и в предыдушей задаче, выполнить для уравнений: $4 x^2+12 x y+9 y^2-4 x-6 y+1=0$. \\
\textbf{C1.} При каких значениях $m$ и $n$ уравнение $m x^2+12 x y+9 y^2+4 x+n y-13=0$ определяет: 1) центральную линию; 2) линию без центра; 3) линию, имеющую бесконечно много центров. \\
\textbf{C2.} Доказать, что расстояние от фокуса гиперболы $\frac{x^2}{a^2}-\frac{y^2}{b^2}=1$ до ее асимптоты равно $b$. \\
\textbf{C3.} Доказать, что параболическое уравнение определяет параболу в том и только в том случае, когда $\Delta \neq 0$. Доказать, что в этом случае параметр параболы определяется формулой $p=\sqrt{\frac{-\Delta}{ (a_{11}+a_{33}) ^3}}$. \\

\end{tabular}
\vspace{1cm}


\begin{tabular}{m{17cm}}
\textbf{71-variant}
\newline

\textbf{T1.} Линии второго порядка на плоскости (Уравнение второго порядка, Уравнение квадратной формы, Конические линии (сечение конусов)) \\
\textbf{T2.} Парабола и её канонические уравнения (Фокус (направляющая точка), Директриса (направляющая линия), Ось (ось симметрии)) \\
\textbf{A1.} He проводя преобразования координат, установить, что каждое из следующих уравнений определяет эллипс, и найти величины его полуосей: $13 x^2+18 x y+37 y^2-26 x-18 y+3=0$; \\
\textbf{A2.} Установить, какие из следующих линий являются центральными (т.е. имеют единственный центр), какие имеют центра, какие имеют бесконечно много центров: $4 x^2-4 x y+y^2-6 x+8 y+13=0$; \\
\textbf{A3.} Не проводя преобразования координат, установить, что каждое из следующих уравнений определяет параболу, и найти параметр этой параболы: $x^2-2 x y+y^2+6 x-14 y+29=0$; \\
\textbf{B1.} Составить уравнение параболы, если даны ее фокус $F(4 ; 3)$ и директриса $y+1=0$. \\
\textbf{B2.} Каждое из следующих уравнений привести к простейшему виду; определить тип каждого из них; установить, какие геометрические образы они определяют, и изобразить на чертеже расположение этих образов относительно старых и новых осей координат: $17 x^2-12 x y+8 y^2=0$; \\
\textbf{B3.} Установить, что следующие линии являются центральными, и для каждой из них найти координаты центра: $9 x^2-4 x y-7 y^2-12=0$; \\
\textbf{C1.} Доказать что произведение расстояний от любой точки гиперболы $\frac{x^2}{a^2}-\frac{y^2}{b^2}=1$ до двух ее асимптот есть величина постоянная, равная $\frac{a^2 b^2}{a^2+b^2}$. \\
\textbf{C2.} Доказать, что эллипсоид $\frac{x^2}{81}+\frac{y^2}{36}+\frac{z^2}{9}=1$ имеет одну общую точку с плоскостью $4 x-3 y+12 z-54=0$, и найти ее координаты. \\
\textbf{C3.} Доказать, что уравнение второй степени является уравнением вырожденной линии в том и только в том случае, когда $\Delta=0$. \\

\end{tabular}
\vspace{1cm}


\begin{tabular}{m{17cm}}
\textbf{72-variant}
\newline

\textbf{T1.} Центр, касательная плоскость и диаметральная плоскость поверхности второго порядка (Центр, касательная плоскость, диаметральная плоскость) \\
\textbf{T2.} Общие уравнения линий второго порядка (Общее уравнение) \\
\textbf{A1.} Эксцентриситет гиперболы $\varepsilon=3$, расстояние от точки. $M$ гиперболы до директрисы равно 4 . Вычислить расстояние от точки $M$ до фокуса, одностороннего с этой директрисой. \\
\textbf{A2.} Определить точки пересечения эллипса $\frac{x^2}{100}+\frac{y^2}{225}=1$ и параболы $y^2=24 x$ \\
\textbf{A3.} Составить уравнение эллипса, фокусы которого лежат на оси абсцисс, симметрично относительно начала координат, зная, кроме того, что: его большая ось равна 8, а расстояние между директрисами равно 16 ; \\
\textbf{B1.} Установить, что следующие уравнения являются параболическими, и записать каждое из них в виде $(\alpha x+\beta y)^2+2 a_{13} x+2 a_{23} y+a_{33}=0$: $9 x^2-42 x y+49 y^2+3 x-2 y-24=0$. \\
\textbf{B2.} Точка $M_1(2 ;-1)$ лежит на эллипсе, фокус которого $F(1 ; 0)$, а соответствующая директриса дана уравнением $2 x-y-10=0$. Составить уравнение этого эллипса. \\
\textbf{B3.} Установить, какая линия является сечением эллипсоида $\frac{x^2}{12}+\frac{y^2}{4}+\frac{z^2}{3}=1$ плоскостью $2 x-3 y+4 z-11=0$, и найти ее центр. \\
\textbf{C1.} Доказать, что произведение расстояний от фокусов до любой касательной к эллипсу равно квадрату малой полуоси. \\
\textbf{C2.} При каких значениях $m$ и $n$ уравнение $m x^2+12 x y+9 y^2+4 x+n y-13=0$ определяет: 1) центральную линию; 2) линию без центра; 3) линию, имеющую бесконечно много центров. \\
\textbf{C3.} Каждое из следующих уравнений привести к каноническому виду; определить тип каждого из них; установить, какие геометрические образы они определяют; для каждого случая изобразить на чертеже оси первоначальной координатной системы, оси других координатных систем, которые вводятся по ходу решения, и геометрический образ, определяемый данным уравнением: $5 x^2-2 x y+5 y^2-4 x+20 y+20=0$. \\

\end{tabular}
\vspace{1cm}


\begin{tabular}{m{17cm}}
\textbf{73-variant}
\newline

\textbf{T1.} Линии второго порядка на плоскости (Уравнение второго порядка, Уравнение квадратной формы, Конические линии (сечение конусов)) \\
\textbf{T2.} Общие уравнения поверхностей второго порядка (Общее уравнение) \\
\textbf{A1.} Установить, какие линии определяются следующими уравнениями: $\left\{\begin{array}{l}\frac{x^2}{.4}+\frac{y^2}{9}-\frac{z^2}{36}=1, \\ 9 x-6 y+2 z-28=0,\end{array}\right.$ \\
\textbf{A2.} Составить уравнение эллипса, фокусы которого лежат на оси абсцисс, симметрично относительно начала координат, зная, кроме того, что: его малая ось равна 6, а расстояние между дирек трисами равно 13 ; \\
\textbf{A3.} Не проводя преобразования координат, установить, что каждое из следующих уравнений определяет параболу, и найти параметр этой параболы: $9 x^2-6 x y+y^2-50 x+50 y-275=0$. \\
\textbf{B1.} Дана точка $M_1(10 ;-\sqrt{5})$ на гиперболе $\frac{x^2}{80}-\frac{y^2}{20}=1$. Составить уравнения прямых, на которых лежат фокальные радиусы точки $M_1$. \\
\textbf{B2.} Каждое из следующих уравнений привести к простейшему виду; определить тип каждого из них; установить, какие геометрические образы они определяют, и изобразить на чертеже расположение этих образов относительно старых и новых осей координат: $5 x^2-6 x y+5 y^2+8=0$. \\
\textbf{B3.} Составить уравнение эллипса, зная, что: его большая ось равна 26 и фокусы суть $F_1(-10 ; 0), F_2(14 ; 0)$; \\
\textbf{C1.} Определить, при каких значениях углового коэффициента $k$ прямая $y=k x+2$ 1) пересекает параболу $y^2=4 x$; 2) касается ее; 3) проходит вне этсй параболы. \\
\textbf{C2.} Доказать, что две параболы, имеющие общую ось и общий фокус, расположенный между их вершинами, пересекаются под прямым углом. \\
\textbf{C3.} Доказать, что площадь параллелограмма, ограниченного асимптотами гиперболы $\frac{x^2}{a^2}-\frac{y^2}{b^2}=1$ и прямыми, проведенными через любую ее точку параллельно асимптотам, есть величина постоянная, равная $\frac{a b}{2}$. \\

\end{tabular}
\vspace{1cm}


\begin{tabular}{m{17cm}}
\textbf{74-variant}
\newline

\textbf{T1.} Уравнение касательной линии второго порядка, сопряжённого диаметра (Уравнение касательной, сопряжённый диаметр: оси симметрии, проходящие через центр) \\
\textbf{T2.} Парабола и её канонические уравнения (Фокус (направляющая точка), Директриса (направляющая линия), Ось (ось симметрии)) \\
\textbf{A1.} Установить, какие линии определяются следующими уравнениями: $\left\{\begin{array}{l}\frac{x^2}{4}-\frac{y^2}{3}=2 z \\ x-2 y+2=0 ;\end{array}\right.$ \\
\textbf{A2.} Составить уравнение гиперболы, если известны ее эксцентриситет $\varepsilon=\frac{13}{12}$, фокус $F(0 ; 13)$ и уравнение соответствующей директрисы $13 y-144=0$. \\
\textbf{A3.} Установить, что каждая из следующих линий имєет бесконечно много цєнтров; для каждой их них составить уравнение геометрического места центров: $25 x^2-10 x y+y^2+40 x-8 y+7=0$. \\
\textbf{B1.} Установить, какая линия является сечением гиперболического параболоида $\frac{x^2}{2}-\frac{z^2}{3}=y$ плоскостью $3 x-3 y+4 z+2=0$, и найти ее центр. \\
\textbf{B2.} Составить уравнение гиперболы, если известны ее эксцентриситет $\varepsilon=\sqrt{5}$, фокус $F(2 ;-3)$ и уравнение соответствующей директрисы $3 x-y+3=0$. \\
\textbf{B3.} Составить уравнение параболы, если даны ее фокус $F(7 ; 2)$ и директриса $x-5=0$ \\
\textbf{C1.} Доказать, что если уравнение второй степени является параболическим и написано в виде $ (\alpha x+\beta y) ^2+2a_{13}x+2a_{23}y+a_{33}=0$ то дискриминант его левой части определяется формулой $\Delta=- (a_{13} \beta-a_{23} \alpha) ^2$. \\
\textbf{C2.} Доказать, что двухполостный гиперболоид $\frac{x^2}{3}+\frac{y^2}{4}-\frac{z^2}{25}=-1$ имеет одну общую точку с плоскостью $5 x+2 z+5=0$, и найти ее координаты. \\
\textbf{C3.} Провести касательные к эллиису $\frac{x^2}{30}+\frac{y^2}{24}=1$ параллельно прямой $4 x-2 y+23=0$ и вычислить расстояние $d$ между ними. \\

\end{tabular}
\vspace{1cm}


\begin{tabular}{m{17cm}}
\textbf{75-variant}
\newline

\textbf{T1.} Прямые образующие однополостного гиперболоида и гиперболического параболоида (Гиперболоид, Гиперболический параболоид, Линейные образующие) \\
\textbf{T2.} Центр линии второго порядка (Центровые линии (эллипс, гипербола), Координаты центра: центр симметрии) \\
\textbf{A1.} Не проводя преобразования координат, установить, что каждое из следующих уравнений определяет пару пересекающихся прямых (вырожденную гиперболу), и найти их уравнения: $3 x^2+4 x y+y^2-2 x-1=0$; \\
\textbf{A2.} Составить уравнение параболы, вершина которой находится в начале координат, зная, что: парабола расположена в верхней полуплоскости симметррично относительно оси $O y$, и ее параметр $p=\frac{1}{4}$; \\
\textbf{A3.} Вычислить площадь треугольника, образованного асимптотами гиперболы $\frac{x^2}{4}-\frac{y^2}{9}=1$ и прямой $9 x+2 y-24=0$ \\
\textbf{B1.} Установить, что следующие уравнения определяют центральные линии; преобразовать каждое из них путем переноса начала координат в центр: $3 x^2-6 x y+2 y^2-4 x+2 y+1=0$; \\
\textbf{B2.} Установить, что следующие уравнения являются параболическими, и записать каждое из них в виде $(\alpha x+\beta y)^2+2 a_{13} x+2 a_{23} y+a_{33}=0$: $9 x^2-6 x y+y^2-x+2 y-14=0$; \\
\textbf{B3.} Установить, какая линия является сечением эллипсоида $\frac{x^2}{12}+\frac{y^2}{4}+\frac{z^2}{3}=1$ плоскостью $2 x-3 y+4 z-11=0$, и найти ее центр. \\
\textbf{C1.} Дано уравнение линии $4 x^2-4 x y+y^2+6 x+1=0$. Определить, при каких значениях углового коэффициента $k$ прямая $y=k x:$ 1) пересекает эту линию в одной точке; 2) касается этой линии; 3) пересекает эту линию в двух точках; 4) не имеет общих точек с этой линией. \\
\textbf{C2.} Каждое из следующих уравнений привести к каноническому виду; определить тип каждого из них; установить, какие геометрические образы они определяют; для каждого случая изобразить на чертеже оси первоначальной координатной системы, оси других координатных систем, которые вводятся по ходу решения, и геометрический образ, определяемый данным уравнением: $41 x^2+24 x y+34 y^2+34 x-112 y+129=0$; \\
\textbf{C3.} Каждое из следующих уравнений привести к каноническому виду; определить тип каждого из них; установить, какие геометрические образы они определяют; для каждого случая изобразить на чертеже оси первоначальной координатной системы, оси других координатных систем, которые вводятся по ходу решения, и геометрический образ, определяемый данным уравнением: $4 x^2+24 x y+11 y^2+64 x+42 y+51=0$; \\

\end{tabular}
\vspace{1cm}


\begin{tabular}{m{17cm}}
\textbf{76-variant}
\newline

\textbf{T1.} Линии второго порядка на плоскости (Уравнение второго порядка, Уравнение квадратной формы, Конические линии (сечение конусов)) \\
\textbf{T2.} Центр, касательная плоскость и диаметральная плоскость поверхности второго порядка (Центр, касательная плоскость, диаметральная плоскость) \\
\textbf{A1.} Установить, какие из следующих линий являются центральными (т.е. имеют единственный центр), какие имеют центра, какие имеют бесконечно много центров: $x^2-2 x y+y^2-6 x+6 y-3=0$; \\
\textbf{A2.} Составить уравнение параболы, зная, что .ее вершина совпадает с точкой ( $\alpha ; \beta$ ), параметр равен $p$, ось параллельна оси $O x$ и парабола простирается в бесконечность: в положительном направлении оси $O x$; \\
\textbf{A3.} Найти уравнения проекций на координатные плоскости сечения эллиптического параболоида $y^2+z^2=x$ плоскостью $x+2 y-z=0$ \\
\textbf{B1.} Определить тип каждого из следующих уравнений каждое из них путем параллельного переноса осей координат привести к простейшему виду; установить, какие геометрические образы они определяют, и изобразить на чертеже расположение этих образов относительно старых и новых осей координат: $4 x^2-y^2+8 x-2 y+3=0$; \\
\textbf{B2.} Составить уравнение параболы, если даны ее фокус $F(2 ;-1)$ и директриса $x-y-1=0$. \\
\textbf{B3.} Составить уравнение эллипса, зная, что: его малая ось равна 2 и фокусы суть $F_1(-1 ;-1)$, $F_2(1 ; 1)$; \\
\textbf{C1.} Для любого параболического уравнения доказать, что коэффициенты $a_{11}$ и $a_{22}$ не могут быть числами разных знаков и что они одновременно не могут обрашаться в нуль. \\
\textbf{C2.} Определить, при каких значениях $m$ прямая $y=-x+m$ 1) пересекает эллипс $\frac{x^2}{20}+\frac{y^2}{5}=1$; 2) касается его; 3) проходит вне этого эллипса. \\
\textbf{C3.} Дано уравнение линии $4 x^2-4 x y+y^2+6 x+1=0$. Определить, при каких значениях углового коэффициента $k$ прямая $y=k x:$ 1) пересекает эту линию в одной точке; 2) касается этой линии; 3) пересекает эту линию в двух точках; 4) не имеет общих точек с этой линией. \\

\end{tabular}
\vspace{1cm}


\begin{tabular}{m{17cm}}
\textbf{77-variant}
\newline

\textbf{T1.} Взаимное расположение линии второго порядка и прямой (Точки пересечения, касательное положение) \\
\textbf{T2.} Уравнение касательной линии второго порядка, сопряжённого диаметра (Уравнение касательной, сопряжённый диаметр: оси симметрии, проходящие через центр) \\
\textbf{A1.} Не проводя преобразования координат, установить, что каждое из следующих уравнений определяет параболу, и найти параметр этой параболы: $x^2-2 x y+y^2+6 x-14 y+29=0$; \\
\textbf{A2.} Составить уравнение эллипса, фокусы которого лежат на оси абсцисс, симметрично относительно начала координат, зная, кроме того, что: расстояние между его директрисами равно 5 и расстояние между фокусами $2 c=4$; \\
\textbf{A3.} He проводя преобразования координат, установить, что каждое из следующих уравнений определяет эллипс, и найти величины его полуосей: $41 x^2+24 x y+9 y^2+24 x+18 y-36=0$; \\
\textbf{B1.} Определить эксцентриситет равносторонней гиперболы. \\
\textbf{B2.} Установить, что следующие линии являются центральными, и для каждой из них найти координаты центра: $2 x^2-6 x y+5 y^2+22 x-36 y+11=0$. \\
\textbf{B3.} Установить, что следующие уравнения являются параболическими, и записать каждое из них в виде $(\alpha x+\beta y)^2+2 a_{13} x+2 a_{23} y+a_{33}=0$: $25 x^2-20 x y+4 y^2+3 x-y+11=0$; \\
\textbf{C1.} Составить уравнение касагельной к параболе $y^2=2 p x$ в ее точке $M_1\left(x_1 ; y_1\right)$. \\
\textbf{C2.} Установить, при каких значениях $m$ плоскость $x+m y-2=0$ пересекает эллиптический параболоид $\frac{x^2}{2}+\frac{z^2}{3}=y$ а) по эллипсу, б) по параболе. \\
\textbf{C3.} Составить уравнение гиперболы, если известны ее полуоси $a$ и $b$, центр $C\left(x_0 ; y_0\right)$ и фокусы расположены на прямой: 1) параллельной оси $O x$; 2) параллельной оси $O y$. \\

\end{tabular}
\vspace{1cm}


\begin{tabular}{m{17cm}}
\textbf{78-variant}
\newline

\textbf{T1.} Парабола и её канонические уравнения (Фокус (направляющая точка), Директриса (направляющая линия), Ось (ось симметрии)) \\
\textbf{T2.} Канонические уравнения поверхностей второго порядка (Параболоид (эллиптический), Параболоид (гиперболический), Конус, Цилиндр) \\
\textbf{A1.} Установить, какие из следующих линий являются центральными (т.е. имеют единственный центр), какие имеют центра, какие имеют бесконечно много центров: $4 x^2+5 x y+3 y^2-x+9 y-12=0$; \\
\textbf{A2.} Установить, какие линии определяются следующими уравнениями: $y=+\frac{2}{3} \sqrt{x^2-9}$ \\
\textbf{A3.} Составить уравнение параболы, вершина которой находится в начале координат, зная, что: парабола расположена в левой полуплоскости симметрично относительно оси $O x$, и ее параметр $p=0,5$; \\
\textbf{B1.} Определить тип каждого из следующих уравнений каждое из них путем параллельного переноса осей координат привести к простейшему виду; установить, какие геометрические образы они определяют, и изобразить на чертеже расположение этих образов относительно старых и новых осей координат: $9 x^2-16 y^2-54 x-64 y-127=0$; \\
\textbf{B2.} Установить, что следующие линии являются центральными, и для каждой из них найти координаты центра: $3 x^2+5 x y+y^2-8 x-11 y-7=0$; \\
\textbf{B3.} Эксцентриситет эллипса $\varepsilon=\frac{1}{2}$, центр его совпадает с началом координат, одна из директрис дана уравнением $x=16$. Вычислить расстояние от точки $M_1$ эллипса с абсциссой, равной -4, до фокуса, одностороннего с данной директрисой. \\
\textbf{C1.} При каких значениях $m$ и $n$ уравнение $m x^2+12 x y+9 y^2+4 x+n y-13=0$ определяет: 1) центральную линию; 2) линию без центра; 3) линию, имеющую бесконечно много центров. \\
\textbf{C2.} Определить, при каких значениях $m$ прямая $y=\frac{5}{2} x+m$ пересекает гиперболу $\frac{x^2}{9}-\frac{y^2}{36}=1$; 2) касается ее; 3) проходит вне этой гиперболы \\
\textbf{C3.} Доказать, что произведение расстояний от центра эллипса до точки пересечения любой его касательной с фокальной осью и до основания перпендикуляря, опущенного из точки касания на фокальную ось, есть величина постоянная, равная квадрату большой полуоси эллипса. \\

\end{tabular}
\vspace{1cm}


\begin{tabular}{m{17cm}}
\textbf{79-variant}
\newline

\textbf{T1.} Приведение общего уравнения поверхности второго порядка к каноническому виду с помощью инвариантов \\
\textbf{T2.} Линии второго порядка на плоскости (Уравнение второго порядка, Уравнение квадратной формы, Конические линии (сечение конусов)) \\
\textbf{A1.} Установить, какие линии определяются следующими уравнениями: $\left\{\begin{array}{l}\frac{x^2}{3}+\frac{y^2}{6}=2 z, \\ 3 x-y+6 z-14=0\end{array}\right.$ \\
\textbf{A2.} Составить уравнение эллипса, фокусы которого расположены на оси абсцисс, симметрично относительно начала координат, если даны: точка $M_1(\sqrt{15} ;-1)$ эллипса и расстояние между его фокусами $2 c=8$; \\
\textbf{A3.} Не проводя преобразования координат, установить, что каждое из следующих уравнений определяет параболу, и найти параметр этой параболы: $9 x^2-24 x y+16 y^2-54 x-178 y+181=0$; \\
\textbf{B1.} Составить уравнение прямой, которая касается параболы $x^2=16 y$ и перпендикулярна к прямой $2 x+4 y+7=0$. \\
\textbf{B2.} Установить, что следующие уравнения являются параболическими, и записать каждое из них в виде $(\alpha x+\beta y)^2+2 a_{13} x+2 a_{23} y+a_{33}=0$: $25 x^2-20 x y+4 y^2+3 x-y+11=0$; \\
\textbf{B3.} Установить, какая линия является сечением гиперболического параболоида $\frac{x^2}{2}-\frac{z^2}{3}=y$ плоскостью $3 x-3 y+4 z+2=0$, и найти ее центр. \\
\textbf{C1.} Доказать, что любое параболическое уравнение может быть написано в виде: $ (\alpha x+\beta y) ^2+2a_{13}x+2a_{23}y+a_{33}=0$. Доказать также, что эллиптические и гиперболические уравнения в таком виде не могут быть написаны. \\
\textbf{C2.} Доказать, что эллиптический параболоид $\frac{x^2}{9}+\frac{z^2}{4}=2 y$ имеет одну общую точку с плоскостью $2 x-2 y-z-10=0$, и найти ее координаты. \\
\textbf{C3.} Каждое из следующих уравнений привести к каноническому виду; определить тип каждого из них; установить, какие геометрические образы они определяют; для каждого случая изобразить на чертеже оси первоначальной координатной системы, оси других координатных систем, которые вводятся по ходу решения, и геометрический образ, определяемый данным уравнением: $19 x^2+6 x y+11 y^2+38 x+6 y+29=0$; \\

\end{tabular}
\vspace{1cm}


\begin{tabular}{m{17cm}}
\textbf{80-variant}
\newline

\textbf{T1.} Общие уравнения линий второго порядка (Общее уравнение) \\
\textbf{T2.} Парабола и её канонические уравнения (Фокус (направляющая точка), Директриса (направляющая линия), Ось (ось симметрии)) \\
\textbf{A1.} Не проводя преобразования координат, установить, что каждое из следующих уравнений определяет гиперболу, и найти величины ее полуосей: $3 x^2+4 x y-12 x+16=0$; \\
\textbf{A2.} Установить, какие линии определяются следующими уравнениями: $y=-3 \sqrt{x^2+1}$; \\
\textbf{A3.} Не проводя преобразования координат, установить, что каждое из следующих уравнений определяет единственную точку (вырожденный эллипс), и найти ее координаты: $5 x^2-6 x y+2 y^2-2 x+2=0$; \\
\textbf{B1.} Составить уравнение гиперболы, зная, что: угол между асимптотами равен $90^{\circ}$ и фокусы суть $F_1(4 ;-4), F_2(-2 ; 2)$. \\
\textbf{B2.} То же задание, что и в предыдушей задаче, выполнить для уравнений: $x^2-2 x y+y^2-12 x+12 y-14=0$ \\
\textbf{B3.} Установить, какая линия является сечением гиперболического параболоида $\frac{x^2}{2}-\frac{z^2}{3}=y$ плоскостью $3 x-3 y+4 z+2=0$, и найти ее центр. \\
\textbf{C1.} Доказать, что если две параболы со взаимно перпендикулярными осями пересекаются в четырех точках, то эти точки лежат на одной окружности. \\
\textbf{C2.} Каждое из следующих уравнений привести к каноническому виду; определить тип каждого из них; установить, какие геометрические образы они определяют; для каждого случая изобразить на чертеже оси первоначальной координатной системы, оси других координатных систем, которые вводятся по ходу решения, и геометрический образ, определяемый данным уравнением: $50 x^2-8 x y+35 y^2+100 x-8 y+67=0$; \\
\textbf{C3.} Доказать, что параболическое уравнение определяет параболу в том и только в том случае, когда $\Delta \neq 0$. Доказать, что в этом случае параметр параболы определяется формулой $p=\sqrt{\frac{-\Delta}{ (a_{11}+a_{33}) ^3}}$. \\

\end{tabular}
\vspace{1cm}


\begin{tabular}{m{17cm}}
\textbf{81-variant}
\newline

\textbf{T1.} Канонические уравнения поверхностей второго порядка (эллипсоид, гиперболоид (1-полостный), гиперболоид (2-полостный)) \\
\textbf{T2.} Приведение общего уравнения кривой второго порядка к каноническому виду с помощью инвариантов \\
\textbf{A1.} Не проводя преобразования координат, установить, что каждое из следующих уравнений определяет параболу, и найти параметр этой параболы: $9 x^2-6 x y+y^2-50 x+50 y-275=0$. \\
\textbf{A2.} Определить точки эллипса $\frac{x^2}{100}+\frac{y^2}{36}=1$, pacстояние которых до правого фокуса равно 14. \\
\textbf{A3.} Установить, что плоскость $y+6=0$ пересекает гиперболический параболоид $\frac{x^2}{5}-\frac{y^2}{4}=6 z$ по параболе; найти ее параметр и вершину. \\
\textbf{B1.} Составить уравнение гиперболы, фокусы которой лежат в вершинах эллинса $\frac{x^2}{100}+\frac{y^2}{64}=1$, а директрисы проходят через фокусы этого эллипса. \\
\textbf{B2.} Даны вершина параболы $A(-2 ;-1)$ и урав нение ее директрисы $x+2 y-1=0$. Составить уравнение этой параболы. \\
\textbf{B3.} Каждое из следующих уравнений привести к простейшему виду; определить тип каждого из них; установить, какие геометрические образы они определяют, и изобразить на чертеже расположение этих образов относительно старых и новых осей координат: $5 x^2+24 x y-5 y^2=0$; \\
\textbf{C1.} Составить уравнение касательной к эллипсу $\frac{x^2}{a^2}+\frac{y^2}{b^2}=1$ в его точке $M_1\left(x_1 ; y_1\right)$. \\
\textbf{C2.} Дано уравнение линии $4 x^2-4 x y+y^2+6 x+1=0$. Определить, при каких значениях углового коэффициента $k$ прямая $y=k x:$ 1) пересекает эту линию в одной точке; 2) касается этой линии; 3) пересекает эту линию в двух точках; 4) не имеет общих точек с этой линией. \\
\textbf{C3.} Вывести условие, при котором прямая $y=k x+b$ касается параболы $y^2=2 p x$. \\

\end{tabular}
\vspace{1cm}


\begin{tabular}{m{17cm}}
\textbf{82-variant}
\newline

\textbf{T1.} Парабола и её канонические уравнения (Фокус (направляющая точка), Директриса (направляющая линия), Ось (ось симметрии)) \\
\textbf{T2.} Общие уравнения поверхностей второго порядка (Общее уравнение) \\
\textbf{A1.} Составить уравнение параболы, вершина которой находится в начале координат, зная, что: парабола расположена симметрично относительно оси $O x$ и проходит через точку $B(-1 ; 3)$; \\
\textbf{A2.} Установить, какие из следующих линий являются центральными (т.е. имеют единственный центр), какие имеют центра, какие имеют бесконечно много центров: $4 x^2-4 x y+y^2-6 x+8 y+13=0$; \\
\textbf{A3.} Установить, что каждая из следующих линий имєет бесконечно много цєнтров; для каждой их них составить уравнение геометрического места центров: $x^2-6 x y+9 y^2-12 x+36 y+20=0$; \\
\textbf{B1.} Эксцентриситет эллипса $\varepsilon=\frac{1}{3}$, центр его совпадает с началом координат, один из фокусов $(-2 ; 0)$. Вычиелить расстояние от точки $M_1$ эллипса с абсциссой, равной 2, до директрисы, односторонней с данным фокусом. \\
\textbf{B2.} Установить, что следующие уравнения определяют центральные линии; преобразовать каждое из них путем переноса начала координат в центр: $3 x^2-6 x y+2 y^2-4 x+2 y+1=0$; \\
\textbf{B3.} Каждое из следующих уравнений привести к простейшему виду; определить тип каждого из них; установить, какие геометрические образы они определяют, и изобразить на чертеже расположение этих образов относительно старых и новых осей координат: $5 x^2-6 x y+5 y^2-32=0$; \\
\textbf{C1.} Вывести условие, при котором прямая $y=k x+m$ касается гиперболы $\frac{x^2}{a^2}-\frac{y^2}{b^2}=1$. \\
\textbf{C2.} Установить, при каких значениях $m$ плоскость $x+m z-1=0$ пересекает двухполостный гиперболоид $x^2+y^2-z^2=-1$ а) по эллипсу, б) по гиперболе. \\
\textbf{C3.} Доказать, что эллиптичсское уравнение второй степсни ( $\delta>0$ ) является уравпением мнимого эллипса в том и только в том случае, когда $a_{11}$ и $\Delta$ суть числа одинаковых знаков. \\

\end{tabular}
\vspace{1cm}


\begin{tabular}{m{17cm}}
\textbf{83-variant}
\newline

\textbf{T1.} Центр линии второго порядка (Центровые линии (эллипс, гипербола), Координаты центра: центр симметрии) \\
\textbf{T2.} Линии второго порядка на плоскости (Уравнение второго порядка, Уравнение квадратной формы, Конические линии (сечение конусов)) \\
\textbf{A1.} Составить уравнение эллипса, фокусы которого лежат на оси абсцисс, симметрично относительно начала координат, зная, кроме того, что: его большая ось равна 10 , а расстояние между фокусами $2 c=8$; \\
\textbf{A2.} Составить уравнение гиперболы, фокусы которой расположены на оси ординат симметрично относительно начала координат, зная, кроме того, что: уравнения асимптот $y= \pm \frac{4}{3} x$ и расстояние между директрисами равно $6 \frac{2}{5}$. \\
\textbf{A3.} Не проводя преобразования координат, установить, что каждое из следующих уравнений определяет параболу, и найти параметр этой параболы: $9 x^2+24 x y+16 y^2-120 x+90 y=0$; \\
\textbf{B1.} Установить, что следующие уравнения определяют центральные линии; преобразовать каждое из них путем переноса начала координат в центр: $6 x^2+4 x y+y^2+4 x-2 y+2=0$; \\
\textbf{B2.} Установить, что следующие уравнения являются параболическими, и записать каждое из них в виде $(\alpha x+\beta y)^2+2 a_{13} x+2 a_{23} y+a_{33}=0$: $9 x^2-6 x y+y^2-x+2 y-14=0$; \\
\textbf{B3.} Составить уравнение эллипса, если известны его эксцентриситет $\varepsilon=\frac{1}{2}$, фокус $F(3 ; 0)$ и уравнение соответствующей директрисы $x+y-1=0$. \\
\textbf{C1.} Доказать, что параболическое уравнение определяет параболу в том и только в том случае, когда $\Delta \neq 0$. Доказать, что в этом случае параметр параболы определяется формулой $p=\sqrt{\frac{-\Delta}{ (a_{11}+a_{33}) ^3}}$. \\
\textbf{C2.} Составить уравнение касагельной к параболе $y^2=2 p x$ в ее точке $M_1\left(x_1 ; y_1\right)$. \\
\textbf{C3.} Составить уравнение касательной к гиперболе $\frac{x^2}{a^2}-\frac{y^2}{b^2}=1$ в ее точке $M_1\left(x_1 ; y_1\right)$. \\

\end{tabular}
\vspace{1cm}


\begin{tabular}{m{17cm}}
\textbf{84-variant}
\newline

\textbf{T1.} Приведение общего уравнения поверхности второго порядка к каноническому виду с помощью инвариантов \\
\textbf{T2.} Общие уравнения линий второго порядка (Общее уравнение) \\
\textbf{A1.} Не проводя преобразования координат, установить, что каждое из следующих уравнений определяет гиперболу, и найти величины ее полуосей: $x^2-6 x y-7 y^2+10 x-30 y+23=0$. \\
\textbf{A2.} Вычислить фокальный радиус точки $M$ параболы $y^2=20 x$, если абсцисса точки $M$ равна 7 . \\
\textbf{A3.} Установить, что плоскость $z+1=0$ пересекает однополостный гиперболоид $\frac{x^2}{32}-\frac{y^2}{18}+\frac{z^2}{2}=1$ по гиперболе; найти ее полуоси и вершины. \\
\textbf{B1.} Точка $M_1(1 ;-2)$ лежит на гиперболе, фокус которой $F(-2 ; 2)$, а соответствующая директриса дана уравнением $2 x-y-1=0$. Составить уравнение этой гиперболы. \\
\textbf{B2.} Установить, какая линия является сечением эллипсоида $\frac{x^2}{12}+\frac{y^2}{4}+\frac{z^2}{3}=1$ плоскостью $2 x-3 y+4 z-11=0$, и найти ее центр. \\
\textbf{B3.} Из точки $A(5 ; 9)$ проведены касательные к параболе $y^2=5 x$. Составить уравнение хорды, соединяющей точки касания. \\
\textbf{C1.} Доказать, что касательные к эллипсу $\frac{x^2}{a^2}+\frac{y^2}{b^2}=1$, проведенные в концах одного и того же диаметра, параллельны. (Диаметром эллипса называется его хорда, проходящая через центр.) \\
\textbf{C2.} При каких значениях $m$ и $n$ уравнение $m x^2+12 x y+9 y^2+4 x+n y-13=0$ определяет: 1) центральную линию; 2) линию без центра; 3) линию, имеющую бесконечно много центров. \\
\textbf{C3.} Определить, при каком значении $m$ плоскость $x-2 y-2 z+m=0$ касается эллипсоида $\frac{x^2}{144}+\frac{y^2}{36}+\frac{z^2}{9}=1$. \\

\end{tabular}
\vspace{1cm}


\begin{tabular}{m{17cm}}
\textbf{85-variant}
\newline

\textbf{T1.} Прямые образующие однополостного гиперболоида и гиперболического параболоида (Гиперболоид, Гиперболический параболоид, Линейные образующие) \\
\textbf{T2.} Линии второго порядка на плоскости (Уравнение второго порядка, Уравнение квадратной формы, Конические линии (сечение конусов)) \\
\textbf{A1.} Установить, что каждая из следующих линий имєет бесконечно много цєнтров; для каждой их них составить уравнение геометрического места центров: $4 x^2+4 x y+y^2-8 x-4 y-21=0$; \\
\textbf{A2.} Не проводя преобразования координат, установить, какие геометрические образы определяются следующими уравнениями: $17 x^2-18 x y-7 y^2+34 x-18 y+7=0$; \\
\textbf{A3.} Составить уравнение гиперболы, фокусы которой лежат на оси абсцисс симметрично относительно начала координат, если даны: точка $M_1\left(-3 ; \frac{5}{2}\right)$ гиперболы и уравнения директрис $x= \pm \frac{4}{3}$; \\
\textbf{B1.} Установить, какая линия является сечением эллипсоида $\frac{x^2}{12}+\frac{y^2}{4}+\frac{z^2}{3}=1$ плоскостью $2 x-3 y+4 z-11=0$, и найти ее центр. \\
\textbf{B2.} Эксцентриситет эллипса $\varepsilon=\frac{2}{5}$, расстояние от точки $M$ эллипса до директрисы равно 20. Вычислить расстояние от точки $M$ до фокуса, одностороннего с этой директрисой. \\
\textbf{B3.} Установить, что следующие уравнения являются параболическими, и записать каждое из них в виде $(\alpha x+\beta y)^2+2 a_{13} x+2 a_{23} y+a_{33}=0$: $16 x^2+16 x y+4 y^2-5 x+7 y=0$; \\
\textbf{C1.} Для любого параболического уравнения доказать, что коэффициенты $a_{11}$ и $a_{22}$ не могут быть числами разных знаков и что они одновременно не могут обрашаться в нуль. \\
\textbf{C2.} Установить, при каких значениях $m$ плоскость $x+m z-1=0$ пересекает двухполостный гиперболоид $x^2+y^2-z^2=-1$ а) по эллипсу, б) по гиперболе. \\
\textbf{C3.} Каждое из следующих уравнений привести к каноническому виду; определить тип каждого из них; установить, какие геометрические образы они определяют; для каждого случая изобразить на чертеже оси первоначальной координатной системы, оси других координатных систем, которые вводятся по ходу решения, и геометрический образ, определяемый данным уравнением: $4 x^2+24 x y+11 y^2+64 x+42 y+51=0$; \\

\end{tabular}
\vspace{1cm}


\begin{tabular}{m{17cm}}
\textbf{86-variant}
\newline

\textbf{T1.} Взаимное расположение линии второго порядка и прямой (Точки пересечения, касательное положение) \\
\textbf{T2.} Парабола и её канонические уравнения (Фокус (направляющая точка), Директриса (направляющая линия), Ось (ось симметрии)) \\
\textbf{A1.} Не проводя преобразования координат, установить, что каждое из следующих уравнений определяет параболу, и найти параметр этой параболы: $9 x^2+24 x y+16 y^2-120 x+90 y=0$; \\
\textbf{A2.} Установить, что плоскость $y+6=0$ пересекает гиперболический параболоид $\frac{x^2}{5}-\frac{y^2}{4}=6 z$ по параболе; найти ее параметр и вершину. \\
\textbf{A3.} Составить уравнение параболы, вершина которой находится в начале координат, зная, что: парабола расположена симметрично относительно оси $O y$ и проходит через точку $C(1 ; 1)$. \\
\textbf{B1.} Определить тип каждого из следующих уравнений каждое из них путем параллельного переноса осей координат привести к простейшему виду; установить, какие геометрические образы они определяют, и изобразить на чертеже расположение этих образов относительно старых и новых осей координат: $9 x^2+4 y^2+18 x-8 y+49=0$; \\
\textbf{B2.} Установить, что следующие уравнения определяют центральные линии; преобразовать каждое из них путем переноса начала координат в центр: $4 x^2+6 x y+y^2-10 x-10=0$; \\
\textbf{B3.} Даны вершина параболы $A(6 ;-3)$ и уравнение ее директрисы $3 x-5 y+1=0$. Найти фокус $F$ этой параболы. \\
\textbf{C1.} Доказать, что две параболы, имеющие общую ось и общий фокус, расположенный между их вершинами, пересекаются под прямым углом. \\
\textbf{C2.} Дано уравнение линии $4 x^2-4 x y+y^2+6 x+1=0$. Определить, при каких значениях углового коэффициента $k$ прямая $y=k x:$ 1) пересекает эту линию в одной точке; 2) касается этой линии; 3) пересекает эту линию в двух точках; 4) не имеет общих точек с этой линией. \\
\textbf{C3.} Составить уравнение гиперболы, касающейся двух прямых: $\quad 5 x-6 y-16=0, \quad 13 x-10 y-48=0$, при условии, что ее оси совпадают с осями координат. \\

\end{tabular}
\vspace{1cm}


\begin{tabular}{m{17cm}}
\textbf{87-variant}
\newline

\textbf{T1.} Приведение общего уравнения кривой второго порядка к каноническому виду с помощью инвариантов \\
\textbf{T2.} Канонические уравнения поверхностей второго порядка (Параболоид (эллиптический), Параболоид (гиперболический), Конус, Цилиндр) \\
\textbf{A1.} Составить уравнение эллипса, фокусы которого лежат на оси ординат, симметрично относительно начала координат, зная, кроме того, что: его полуоси равны соответственно 7 и 2 ; \\
\textbf{A2.} Не проводя преобразования координат, установить, что каждое из следующих уравнений определяет гиперболу, и найти величины ее полуосей: $12 x^2+26 x y+12 y^2-52 x-48 y+73=0$ \\
\textbf{A3.} Установить, какие из следующих линий являются центральными (т.е. имеют единственный центр), какие имеют центра, какие имеют бесконечно много центров: $x^2-2 x y+4 y^2+5 x-7 y+12=0$; \\
\textbf{B1.} Составить уравнение гиперболы, фокусы которой лежат в вершинах эллинса $\frac{x^2}{100}+\frac{y^2}{64}=1$, а директрисы проходят через фокусы этого эллипса. \\
\textbf{B2.} Составить уравнение гиперболы, если известны ее эксцентриситет $\varepsilon=\sqrt{5}$, фокус $F(2 ;-3)$ и уравнение соответствующей директрисы $3 x-y+3=0$. \\
\textbf{B3.} Определить тип каждого из следующих уравнений каждое из них путем параллельного переноса осей координат привести к простейшему виду; установить, какие геометрические образы они определяют, и изобразить на чертеже расположение этих образов относительно старых и новых осей координат: $2 x^2+3 y^2+8 x-6 y+11=0$. \\
\textbf{C1.} Доказать, что произведение расстояний от фокусов до любой касательной к эллипсу равно квадрату малой полуоси. \\
\textbf{C2.} Определить, при каком значении $m$ плоскость $x-2 y-2 z+m=0$ касается эллипсоида $\frac{x^2}{144}+\frac{y^2}{36}+\frac{z^2}{9}=1$. \\
\textbf{C3.} Доказать, что если уравнение второй степени является параболическим и написано в виде $ (\alpha x+\beta y) ^2+2a_{13}x+2a_{23}y+a_{33}=0$ то дискриминант его левой части определяется формулой $\Delta=- (a_{13} \beta-a_{23} \alpha) ^2$. \\

\end{tabular}
\vspace{1cm}


\begin{tabular}{m{17cm}}
\textbf{88-variant}
\newline

\textbf{T1.} Центр, касательная плоскость и диаметральная плоскость поверхности второго порядка (Центр, касательная плоскость, диаметральная плоскость) \\
\textbf{T2.} Центр линии второго порядка (Центровые линии (эллипс, гипербола), Координаты центра: центр симметрии) \\
\textbf{A1.} Составить уравнение гиперболы, фокусы которой лежат на оси абсцисс симметрично относительно начала координат, если даны: точка $M_1\left(\frac{9}{2} ;-1\right)$ гиперболы и уравнения асимптот $y= \pm \frac{2}{3} x$; \\
\textbf{A2.} Не проводя преобразования координат, установить, что каждое из следующих уравнений определяет параболу, и найти параметр этой параболы: $9 x^2-6 x y+y^2-50 x+50 y-275=0$. \\
\textbf{A3.} Найти уравнения проекций на координатные плоскости сечения эллиптического параболоида $y^2+z^2=x$ плоскостью $x+2 y-z=0$ \\
\textbf{B1.} Составить уравнение параболы, если даны ее фокус $F(7 ; 2)$ и директриса $x-5=0$ \\
\textbf{B2.} Установить, что следующие линии являются центральными, и для каждой из них найти координаты центра: $5 x^2+4 x y+2 y^2+20 x+20 y-18=0$; \\
\textbf{B3.} Установить, какая линия является сечением гиперболического параболоида $\frac{x^2}{2}-\frac{z^2}{3}=y$ плоскостью $3 x-3 y+4 z+2=0$, и найти ее центр. \\
\textbf{C1.} При каких значениях $m$ и $n$ уравнение $m x^2+12 x y+9 y^2+4 x+n y-13=0$ определяет: 1) центральную линию; 2) линию без центра; 3) линию, имеющую бесконечно много центров. \\
\textbf{C2.} Вывести условие, при котором прямая $y=k x+b$ касается параболы $y^2=2 p x$. \\
\textbf{C3.} Каждое из следующих уравнений привести к каноническому виду; определить тип каждого из них; установить, какие геометрические образы они определяют; для каждого случая изобразить на чертеже оси первоначальной координатной системы, оси других координатных систем, которые вводятся по ходу решения, и геометрический образ, определяемый данным уравнением: $4 x y+3 y^2+16 x+12 y-36=0$; \\

\end{tabular}
\vspace{1cm}


\begin{tabular}{m{17cm}}
\textbf{89-variant}
\newline

\textbf{T1.} Линии второго порядка на плоскости (Уравнение второго порядка, Уравнение квадратной формы, Конические линии (сечение конусов)) \\
\textbf{T2.} Уравнение касательной линии второго порядка, сопряжённого диаметра (Уравнение касательной, сопряжённый диаметр: оси симметрии, проходящие через центр) \\
\textbf{A1.} Составить уравнение параболы, вершина которой находится в начале координат, зная, что: парабола расположена симметрично отнлсительно оси $O y$ и проходит через точку $D(4 ;-8)$. \\
\textbf{A2.} Составить уравнение эллипса, фокусы которого лежат на оси ординат, симметрично относительно начала координат, зная, кроме того, что: расстояние между его директрисами равно $10 \frac{2}{3}$ и эксцентриситет $\varepsilon=\frac{3}{4}$. \\
\textbf{A3.} Установить, какие линии определяются следующими уравнениями: $\left\{\begin{array}{l}\frac{x^2}{.4}+\frac{y^2}{9}-\frac{z^2}{36}=1, \\ 9 x-6 y+2 z-28=0,\end{array}\right.$ \\
\textbf{B1.} Определить эксцентриситет в эллипса, если: отрезок перпендикуляра, опущенного из центра эллипса на его директрису, делится вершиной эллипса пополам. \\
\textbf{B2.} Установить, что каждое из следующих уравнений является параболическим; каждое из них привести к простейшему виду; установить, какие геометрические образы они определяют; для каждого случая изобразить на чертеже оси первоначальной координатной системы, оси других координатных систем, которые вводятся по ходу решения, и геометрический образ, определяемый данным уравнением: $9 x^2+12 x y+4 y^2-24 x-16 y+3=0$; \\
\textbf{B3.} Дана точка $M_1(10 ;-\sqrt{5})$ на гиперболе $\frac{x^2}{80}-\frac{y^2}{20}=1$. Составить уравнения прямых, на которых лежат фокальные радиусы точки $M_1$. \\
\textbf{C1.} Доказать, что касательные к гиперболе, проведенные в концах одного и того же диаметра, параллельны. \\
\textbf{C2.} Из точки $A\left(\frac{10}{3} ; \frac{5}{3}\right)$ проведены касательные к эллипсу $\frac{x^2}{20}+\frac{y^2}{5}=1$. Составить их уравнения. \\
\textbf{C3.} Доказать, что эллипсоид $\frac{x^2}{81}+\frac{y^2}{36}+\frac{z^2}{9}=1$ имеет одну общую точку с плоскостью $4 x-3 y+12 z-54=0$, и найти ее координаты. \\

\end{tabular}
\vspace{1cm}


\begin{tabular}{m{17cm}}
\textbf{90-variant}
\newline

\textbf{T1.} Канонические уравнения поверхностей второго порядка (эллипсоид, гиперболоид (1-полостный), гиперболоид (2-полостный)) \\
\textbf{T2.} Парабола и её канонические уравнения (Фокус (направляющая точка), Директриса (направляющая линия), Ось (ось симметрии)) \\
\textbf{A1.} Определить тип каждого из следующих уравнений при помощи вычисления дискриминанта старших членов: $5 x^2+14 x y+11 y^2+12 x-7 y+19=0$; \\
\textbf{A2.} Не проводя преобразования координат, установить, что каждое из следующих уравнений определяет параболу, и найти параметр этой параболы: $x^2-2 x y+y^2+6 x-14 y+29=0$; \\
\textbf{A3.} Составить уравнение гиперболы, фокусы когорой расположены на оси абсцисс симметрично относительно начала координат, зная, кроме того, что: ось $2 a=16$ и эксцентриситет $\varepsilon=\frac{5}{4}$; \\
\textbf{B1.} Эксцентриситет эллипса $\varepsilon=\frac{2}{3}$, фокальный радиус точки $M$ эллипса равен 10 . Вычислить расстояние от точки $M$ до односторонней с этим фокусом директрисы. \\
\textbf{B2.} Установить, что следующие линии являются центральными, и для каждой из них найти координаты центра: $9 x^2-4 x y-7 y^2-12=0$; \\
\textbf{B3.} Установить, что следующие уравнения являются параболическими, и записать каждое из них в виде $(\alpha x+\beta y)^2+2 a_{13} x+2 a_{23} y+a_{33}=0$: $x^2+4 x y+4 y^2+4 x+y-15=0 ;$ \\
\textbf{C1.} Каждое из следующих уравнений привести к каноническому виду; определить тип каждого из них; установить, какие геометрические образы они определяют; для каждого случая изобразить на чертеже оси первоначальной координатной системы, оси других координатных систем, которые вводятся по ходу решения, и геометрический образ, определяемый данным уравнением: $41 x^2+24 x y+34 y^2+34 x-112 y+129=0$; \\
\textbf{C2.} Определить, при каких значениях углового коэффициента $k$ прямая $y=k x+2$ 1) пересекает параболу $y^2=4 x$; 2) касается ее; 3) проходит вне этсй параболы. \\
\textbf{C3.} Даны гиперболы $\frac{x^2}{a^2}-\frac{y^2}{b^2}=1$ и какая-нибудь ее касательная: $P$-точка пересечения касательной с осью $O x, Q$ - проекция точки касания на ту же ось. Доказать, что $O P \cdot O Q=a^2$. \\

\end{tabular}
\vspace{1cm}


\begin{tabular}{m{17cm}}
\textbf{91-variant}
\newline

\textbf{T1.} Парабола и её канонические уравнения (Фокус (направляющая точка), Директриса (направляющая линия), Ось (ось симметрии)) \\
\textbf{T2.} Взаимное расположение линии второго порядка и прямой (Точки пересечения, касательное положение) \\
\textbf{A1.} Точка $C(-3 ; 2)$ является центром эллипса, касающегося обеих координатных осей. Составить уравнение этого эллипса, зная, что его оси симметрии параллельны координатным осям. \\
\textbf{A2.} Установить, что каждая из следующих линий имєет бесконечно много цєнтров; для каждой их них составить уравнение геометрического места центров: $25 x^2-10 x y+y^2+40 x-8 y+7=0$. \\
\textbf{A3.} Составить уравнение параболы, вершина которой находится в начале координат, зная, что: парабола расположена в верхней полуплоскости симметррично относительно оси $O y$, и ее параметр $p=\frac{1}{4}$; \\
\textbf{B1.} Определить тип каждого из следующих уравнений каждое из них путем параллельного переноса осей координат привести к простейшему виду; установить, какие геометрические образы они определяют, и изобразить на чертеже расположение этих образов относительно старых и новых осей координат: $4 x^2+9 y^2-40 x+36 y+100=0$; \\
\textbf{B2.} Составить уравнение прямой, которая касается параболы $y^2=8 x$ и параллельна прямой $2 x+2 y-3=0$. \\
\textbf{B3.} Установить, какая линия является сечением эллипсоида $\frac{x^2}{12}+\frac{y^2}{4}+\frac{z^2}{3}=1$ плоскостью $2 x-3 y+4 z-11=0$, и найти ее центр. \\
\textbf{C1.} Доказать, что уравнение второй степени является уравнением вырожденной линии в том и только в том случае, когда $\Delta=0$. \\
\textbf{C2.} Провести касательные к эллиису $\frac{x^2}{30}+\frac{y^2}{24}=1$ параллельно прямой $4 x-2 y+23=0$ и вычислить расстояние $d$ между ними. \\
\textbf{C3.} Дано уравнение линии $4 x^2-4 x y+y^2+6 x+1=0$. Определить, при каких значениях углового коэффициента $k$ прямая $y=k x:$ 1) пересекает эту линию в одной точке; 2) касается этой линии; 3) пересекает эту линию в двух точках; 4) не имеет общих точек с этой линией. \\

\end{tabular}
\vspace{1cm}


\begin{tabular}{m{17cm}}
\textbf{92-variant}
\newline

\textbf{T1.} Центр, касательная плоскость и диаметральная плоскость поверхности второго порядка (Центр, касательная плоскость, диаметральная плоскость) \\
\textbf{T2.} Общие уравнения поверхностей второго порядка (Общее уравнение) \\
\textbf{A1.} Составить уравнение эллипса, фокусы которого лежат на оси абсцисс, симметрично относительно начала координат, зная, кроме того, что: его малая ось равна 24 , а расстояние между фокусами $2 c=10$; \\
\textbf{A2.} Не проводя преобразования координат, установить, что каждое из следующих уравнений определяет единственную точку (вырожденный эллипс), и найти ее координаты: $5 x^2+4 x y+y^2-6 x-2 y+2=0$; \\
\textbf{A3.} Установить, какие из следующих линий являются центральными (т.е. имеют единственный центр), какие имеют центра, какие имеют бесконечно много центров: $4 x^2-4 x y+y^2-12 x+6 y-11=0$; \\
\textbf{B1.} Определить эксцентриситет в эллипса, если: отрезок между фоку сами виден из вєршин малой оси под прямым углом; \\
\textbf{B2.} Каждое из следующих уравнений привести к простейшему виду; определить тип каждого из них; установить, какие геометрические образы они определяют, и изобразить на чертеже расположение этих образов относительно старых и новых осей координат: $32 x^2+52 x y-7 y^2+180=0$; \\
\textbf{B3.} Установить, какая линия является сечением гиперболического параболоида $\frac{x^2}{2}-\frac{z^2}{3}=y$ плоскостью $3 x-3 y+4 z+2=0$, и найти ее центр. \\
\textbf{C1.} Доказать, что эллиптическое уравнение второй степени ( $\delta>0$ ) определяет эллипс в том и только в том случае, когда $a_{11}$ и $\Delta$ суть числа разных знаков. \\
\textbf{C2.} При каких значениях $m$ и $n$ уравнение $m x^2+12 x y+9 y^2+4 x+n y-13=0$ определяет: 1) центральную линию; 2) линию без центра; 3) линию, имеющую бесконечно много центров. \\
\textbf{C3.} Доказать, что любое параболическое уравнение может быть написано в виде: $ (\alpha x+\beta y) ^2+2a_{13}x+2a_{23}y+a_{33}=0$. Доказать также, что эллиптические и гиперболические уравнения в таком виде не могут быть написаны. \\

\end{tabular}
\vspace{1cm}


\begin{tabular}{m{17cm}}
\textbf{93-variant}
\newline

\textbf{T1.} Центр линии второго порядка (Центровые линии (эллипс, гипербола), Координаты центра: центр симметрии) \\
\textbf{T2.} Линии второго порядка на плоскости (Уравнение второго порядка, Уравнение квадратной формы, Конические линии (сечение конусов)) \\
\textbf{A1.} Составить уравнение гиперболы, фокусы которой лежат на оси абсцисс симметрично относительно начала координат, если даны: точки $M_1(6 ;-1)$ и $M_2(-8 ; 2 \sqrt{2})$ гиперболы; \\
\textbf{A2.} Не проводя преобразования координат, установить, что каждое из следующих уравнений определяет параболу, и найти параметр этой параболы: $9 x^2-24 x y+16 y^2-54 x-178 y+181=0$; \\
\textbf{A3.} Составить уравнение параболы, вершина которой находится в начале координат, зная, что: парабола расположена в левой полуплоскости симметрично относительно оси $O x$, и ее параметр $p=0,5$; \\
\textbf{B1.} Установить, что следующие уравнения определяют центральные линии; преобразовать каждое из них путем переноса начала координат в центр: $4 x^2+2 x y+6 y^2+6 x-10 y+9=0$. \\
\textbf{B2.} Составить уравнение гиперболы, зная, что: фокусы суть $F_1(3 ; 4), F_2(-3 ;-4)$ и расстояние между директрисами равно 3,6 ; \\
\textbf{B3.} Установить, что каждое из следующих уравнений является параболическим; каждое из них привести к простейшему виду; установить, какие геометрические образы они определяют; для каждого случая изобразить на чертеже оси первоначальной координатной системы, оси других координатных систем, которые вводятся по ходу решения, и геометрический образ, определяемый данным уравнением: $16 x^2-24 x y+9 y^2-160 x+120 y+425=0$. \\
\textbf{C1.} Составить уравнение эллипса с полуосями $a, b$ и центром $C\left(x_0 ; y_0\right)$, если известно, что оси симметрии эллипса параллельны осям координат. \\
\textbf{C2.} Доказать, что эллиптический параболоид $\frac{x^2}{9}+\frac{z^2}{4}=2 y$ имеет одну общую точку с плоскостью $2 x-2 y-z-10=0$, и найти ее координаты. \\
\textbf{C3.} Доказать, что расстояние от фокуса гиперболы $\frac{x^2}{a^2}-\frac{y^2}{b^2}=1$ до ее асимптоты равно $b$. \\

\end{tabular}
\vspace{1cm}


\begin{tabular}{m{17cm}}
\textbf{94-variant}
\newline

\textbf{T1.} Уравнение касательной линии второго порядка, сопряжённого диаметра (Уравнение касательной, сопряжённый диаметр: оси симметрии, проходящие через центр) \\
\textbf{T2.} Прямые образующие однополостного гиперболоида и гиперболического параболоида (Гиперболоид, Гиперболический параболоид, Линейные образующие) \\
\textbf{A1.} Установить, какие линии определяются следующими уравнениями: $\left\{\begin{array}{l}\frac{x^2}{3}+\frac{y^2}{6}=2 z, \\ 3 x-y+6 z-14=0\end{array}\right.$ \\
\textbf{A2.} Составить уравнение эллипса, фокусы которого лежат на оси абсцисс, симметрично относительно начала координат, зная, кроме того, что: расстояние между его директрисами равно 32 и $\varepsilon=\frac{1}{2}$. \\
\textbf{A3.} Не проводя преобразования координат, установить, что каждое из следующих уравнений определяет параболу, и найти параметр этой параболы: $x^2-2 x y+y^2+6 x-14 y+29=0$; \\
\textbf{B1.} Составить уравнение параболы, если даны ее фокус $F(4 ; 3)$ и директриса $y+1=0$. \\
\textbf{B2.} Составить уравнение эллипса, зная, что: его фокусы суть $F_1\left(-2 ; \frac{3}{2}\right), F_2\left(2 ;-\frac{3}{2}\right)$ и эксцентриситет $\varepsilon=\frac{\sqrt{2}}{2}$; \\
\textbf{B3.} Установить, какая линия является сечением эллипсоида $\frac{x^2}{12}+\frac{y^2}{4}+\frac{z^2}{3}=1$ плоскостью $2 x-3 y+4 z-11=0$, и найти ее центр. \\
\textbf{C1.} Доказать, что если две параболы со взаимно перпендикулярными осями пересекаются в четырех точках, то эти точки лежат на одной окружности. \\
\textbf{C2.} Доказать, что параболическое уравнение определяет параболу в том и только в том случае, когда $\Delta \neq 0$. Доказать, что в этом случае параметр параболы определяется формулой $p=\sqrt{\frac{-\Delta}{ (a_{11}+a_{33}) ^3}}$. \\
\textbf{C3.} При каких значениях $m$ и $n$ уравнение $m x^2+12 x y+9 y^2+4 x+n y-13=0$ определяет: 1) центральную линию; 2) линию без центра; 3) линию, имеющую бесконечно много центров. \\

\end{tabular}
\vspace{1cm}


\begin{tabular}{m{17cm}}
\textbf{95-variant}
\newline

\textbf{T1.} Линии второго порядка на плоскости (Уравнение второго порядка, Уравнение квадратной формы, Конические линии (сечение конусов)) \\
\textbf{T2.} Парабола и её канонические уравнения (Фокус (направляющая точка), Директриса (направляющая линия), Ось (ось симметрии)) \\
\textbf{A1.} Установить, какие из следующих линий являются центральными (т.е. имеют единственный центр), какие имеют центра, какие имеют бесконечно много центров: $4 x^2-6 x y-9 y^2+3 x-7 y+12=0$. \\
\textbf{A2.} Установить, какие линии определяются следующими уравнениями: $\left\{\begin{array}{l}\frac{x^2}{4}-\frac{y^2}{3}=2 z \\ x-2 y+2=0 ;\end{array}\right.$ \\
\textbf{A3.} Эксцентриситет гиперболы $\varepsilon=2$, фокальный радиус ее точки $M$, проведенный из некоторого фокуса, равен 16. Вычислить расстояние от точки $M$ до односто* ронней с этим фокусом директрисы. \\
\textbf{B1.} Определить тип каждого из следующих уравнений каждое из них путем параллельного переноса осей координат привести к простейшему виду; установить, какие геометрические образы они определяют, и изобразить на чертеже расположение этих образов относительно старых и новых осей координат: $2 x^2+3 y^2+8 x-6 y+11=0$. \\
\textbf{B2.} Составить уравнения касательных к гиперболе $\frac{x^2}{16}-\frac{y^2}{64}=1$, параллельных прямой $10 x-3 y+9=0$. \\
\textbf{B3.} То же задание, что и в предыдушей задаче, выполнить для уравнений: $4 x^2+12 x y+9 y^2-4 x-6 y+1=0$. \\
\textbf{C1.} Доказать, что произведение расстояний от центра эллипса до точки пересечения любой его касательной с фокальной осью и до основания перпендикуляря, опущенного из точки касания на фокальную ось, есть величина постоянная, равная квадрату большой полуоси эллипса. \\
\textbf{C2.} Вывести условие, при котором прямая $y=k x+m$ касается гиперболы $\frac{x^2}{a^2}-\frac{y^2}{b^2}=1$. \\
\textbf{C3.} Каждое из следующих уравнений привести к каноническому виду; определить тип каждого из них; установить, какие геометрические образы они определяют; для каждого случая изобразить на чертеже оси первоначальной координатной системы, оси других координатных систем, которые вводятся по ходу решения, и геометрический образ, определяемый данным уравнением: $5 x^2-2 x y+5 y^2-4 x+20 y+20=0$. \\

\end{tabular}
\vspace{1cm}


\begin{tabular}{m{17cm}}
\textbf{96-variant}
\newline

\textbf{T1.} Общие уравнения линий второго порядка (Общее уравнение) \\
\textbf{T2.} Канонические уравнения поверхностей второго порядка (Параболоид (эллиптический), Параболоид (гиперболический), Конус, Цилиндр) \\
\textbf{A1.} Не проводя преобразования координат, установить, что каждое из следующих уравнений определяет гиперболу, и найти величины ее полуосей: $4 x^2+24 x y+11 y^2+64 x+42 y+51=0$; \\
\textbf{A2.} Составить уравнение параболы, зная, что .ее вершина совпадает с точкой ( $\alpha ; \beta$ ), параметр равен $p$, ось параллельна оси $O x$ и парабола простирается в бесконечность: в положительном направлении оси $O x$; \\
\textbf{A3.} Составить уравнение гиперболы, фокусы когорой расположены на оси абсцисс симметрично относительно начала координат, зная, кроме того, что: расстояние между директрисами равно $\frac{8}{3}$ и эксцентриситет $\varepsilon=\frac{3}{2}$; \\
\textbf{B1.} Установить, что следующие уравнения определяют центральные линии; преобразовать каждое из них путем переноса начала координат в центр: $4 x^2+6 x y+y^2-10 x-10=0$; \\
\textbf{B2.} Составить уравнение прямой, которая касается параболы $x^2=16 y$ и перпендикулярна к прямой $2 x+4 y+7=0$. \\
\textbf{B3.} То же задание, что и в предыдушей задаче, выполнить для уравнений: $9 x^2+24 x y+16 y^2-18 x+226 y+209=0$; \\
\textbf{C1.} Доказать, что двухполостный гиперболоид $\frac{x^2}{3}+\frac{y^2}{4}-\frac{z^2}{25}=-1$ имеет одну общую точку с плоскостью $5 x+2 z+5=0$, и найти ее координаты. \\
\textbf{C2.} Определить, при каких значениях углового коэффициента $k$ прямая $y=k x+2$ 1) пересекает параболу $y^2=4 x$; 2) касается ее; 3) проходит вне этсй параболы. \\
\textbf{C3.} Доказать, что две параболы, имеющие общую ось и общий фокус, расположенный между их вершинами, пересекаются под прямым углом. \\

\end{tabular}
\vspace{1cm}


\begin{tabular}{m{17cm}}
\textbf{97-variant}
\newline

\textbf{T1.} Приведение общего уравнения поверхности второго порядка к каноническому виду с помощью инвариантов \\
\textbf{T2.} Приведение общего уравнения кривой второго порядка к каноническому виду с помощью инвариантов \\
\textbf{A1.} Составить уравнение эллипса, фокусы которого лежат на оси абсцисс, симметрично относительно начала координат, зная, кроме того, что: расстояние между его фокусами $2 c=6$ и эксцентриситет $\varepsilon=\frac{3}{5}$; \\
\textbf{A2.} Определить тип каждого из следующих уравнений при помощи вычисления дискриминанта старших членов: $x^2-4 x y+4 y^2+7 x-12=0$; \\
\textbf{A3.} Не проводя преобразования координат, установить, что каждое из следующих уравнений определяет параболу, и найти параметр этой параболы: $9 x^2+24 x y+16 y^2-120 x+90 y=0$; \\
\textbf{B1.} Вычислить площадь четырехугольника, две вершины которого лежат в фокусах эллипса $x^2+5 y^2=20$, а две другие совпадают с концами его малой оси. \\
\textbf{B2.} Определить тип каждого из следующих уравнений каждое из них путем параллельного переноса осей координат привести к простейшему виду; установить, какие геометрические образы они определяют, и изобразить на чертеже расположение этих образов относительно старых и новых осей координат: $4 x^2-y^2+8 x-2 y+3=0$; \\
\textbf{B3.} Составить уравнение касательных к гиперболе $x^2-y^2=16$, проведенных из точки $A(-1 ;-7)$. \\
\textbf{C1.} Дано уравнение линии $4 x^2-4 x y+y^2+6 x+1=0$. Определить, при каких значениях углового коэффициента $k$ прямая $y=k x:$ 1) пересекает эту линию в одной точке; 2) касается этой линии; 3) пересекает эту линию в двух точках; 4) не имеет общих точек с этой линией. \\
\textbf{C2.} Каждое из следующих уравнений привести к каноническому виду; определить тип каждого из них; установить, какие геометрические образы они определяют; для каждого случая изобразить на чертеже оси первоначальной координатной системы, оси других координатных систем, которые вводятся по ходу решения, и геометрический образ, определяемый данным уравнением: $7 x^2+6 x y-y^2+28 x+12 y+28=0$; \\
\textbf{C3.} Из точки $A\left(\frac{10}{3} ; \frac{5}{3}\right)$ проведены касательные к эллипсу $\frac{x^2}{20}+\frac{y^2}{5}=1$. Составить их уравнения. \\

\end{tabular}
\vspace{1cm}


\begin{tabular}{m{17cm}}
\textbf{98-variant}
\newline

\textbf{T1.} Парабола и её канонические уравнения (Фокус (направляющая точка), Директриса (направляющая линия), Ось (ось симметрии)) \\
\textbf{T2.} Линии второго порядка на плоскости (Уравнение второго порядка, Уравнение квадратной формы, Конические линии (сечение конусов)) \\
\textbf{A1.} Установить, что плоскость $x-2=0$ пересекает эллипсоид $\frac{x^2}{16}+\frac{y^2}{12}+\frac{z^2}{4}=1$ по эллипсу; найти его полуоси и вершины. \\
\textbf{A2.} Установить, какие из следующих линий являются центральными (т.е. имеют единственный центр), какие имеют центра, какие имеют бесконечно много центров: $3 x^2-4 x y-2 y^2+3 x-12 y-7=0$; \\
\textbf{A3.} Составить уравнение параболы, зная, что .ее вершина совпадает с точкой ( $\alpha ; \beta$ ), параметр равен $p$, ось параллельна оси $O x$ и парабола простирается в бесконечность: в отрицательном направлении оси $O y$. \\
\textbf{B1.} Установить, какая линия является сечением гиперболического параболоида $\frac{x^2}{2}-\frac{z^2}{3}=y$ плоскостью $3 x-3 y+4 z+2=0$, и найти ее центр. \\
\textbf{B2.} Установить, что следующие уравнения определяют центральные линии; преобразовать каждое из них путем переноса начала координат в центр: $4 x^2+2 x y+6 y^2+6 x-10 y+9=0$. \\
\textbf{B3.} Из точки $A(5 ; 9)$ проведены касательные к параболе $y^2=5 x$. Составить уравнение хорды, соединяющей точки касания. \\
\textbf{C1.} Доказать, что любое параболическое уравнение может быть написано в виде: $ (\alpha x+\beta y) ^2+2a_{13}x+2a_{23}y+a_{33}=0$. Доказать также, что эллиптические и гиперболические уравнения в таком виде не могут быть написаны. \\
\textbf{C2.} Установить, при каких значениях $m$ плоскость $x+m y-2=0$ пересекает эллиптический параболоид $\frac{x^2}{2}+\frac{z^2}{3}=y$ а) по эллипсу, б) по параболе. \\
\textbf{C3.} Составить уравнение гиперболы, если известны ее полуоси $a$ и $b$, центр $C\left(x_0 ; y_0\right)$ и фокусы расположены на прямой: 1) параллельной оси $O x$; 2) параллельной оси $O y$. \\

\end{tabular}
\vspace{1cm}


\begin{tabular}{m{17cm}}
\textbf{99-variant}
\newline

\textbf{T1.} Общие уравнения линий второго порядка (Общее уравнение) \\
\textbf{T2.} Канонические уравнения поверхностей второго порядка (эллипсоид, гиперболоид (1-полостный), гиперболоид (2-полостный)) \\
\textbf{A1.} Установить, какие линии определяются следующими уравнениями: $\left\{\begin{array}{l}\frac{x^2}{4}-\frac{y^2}{3}=2 z \\ x-2 y+2=0 ;\end{array}\right.$ \\
\textbf{A2.} Составить уравнение эллипса, фокусы которого лежат на оси ординат, симметрично относительно начала координат, зная, кроме того, что: его малая ось равна 16 , а эксцентриситет $\varepsilon=\frac{3}{5}$; \\
\textbf{A3.} Дана гипербола $16 x^2-9 y^2=144$. Найти: 1) полуоси $a$ и $b ; 2$ ) фокусы; 3) эксцентриситет; 4) уравнения асимптот; 5) уравнения директрис. \\
\textbf{B1.} Составить уравнение гиперболы, зная, что: расстояние между ее вершинами равно 24 и фокусы суть $F_1(-10 ; 2), F_2(16 ; 2)$; \\
\textbf{B2.} Установить, что каждое из следующих уравнений является параболическим; каждое из них привести к простейшему виду; установить, какие геометрические образы они определяют; для каждого случая изобразить на чертеже оси первоначальной координатной системы, оси других координатных систем, которые вводятся по ходу решения, и геометрический образ, определяемый данным уравнением: $9 x^2-24 x y+16 y^2-20 x+110 y-50=0$; \\
\textbf{B3.} Определить тип каждого из следующих уравнений каждое из них путем параллельного переноса осей координат привести к простейшему виду; установить, какие геометрические образы они определяют, и изобразить на чертеже расположение этих образов относительно старых и новых осей координат: $9 x^2-16 y^2-54 x-64 y-127=0$; \\
\textbf{C1.} Каждое из следующих уравнений привести к каноническому виду; определить тип каждого из них; установить, какие геометрические образы они определяют; для каждого случая изобразить на чертеже оси первоначальной координатной системы, оси других координатных систем, которые вводятся по ходу решения, и геометрический образ, определяемый данным уравнением: $14 x^2+24 x y+21 y^2-4 x+18 y-139=0$; \\
\textbf{C2.} Составить уравнение эллипса с полуосями $a, b$ и центром $C\left(x_0 ; y_0\right)$, если известно, что оси симметрии эллипса параллельны осям координат. \\
\textbf{C3.} Доказать, что уравнение второй степени является уравнением вырожденной линии в том и только в том случае, когда $\Delta=0$. \\

\end{tabular}
\vspace{1cm}


\begin{tabular}{m{17cm}}
\textbf{100-variant}
\newline

\textbf{T1.} Прямые образующие однополостного гиперболоида и гиперболического параболоида (Гиперболоид, Гиперболический параболоид, Линейные образующие) \\
\textbf{T2.} Центр линии второго порядка (Центровые линии (эллипс, гипербола), Координаты центра: центр симметрии) \\
\textbf{A1.} Составить уравнение параболы, вершина которой находится в начале координат, зная, что: парабола расположена симметрично относительно оси $O x$ и проходит через точку $A(9 ; 6)$; \\
\textbf{A2.} Установить, какие из следующих линий являются центральными (т.е. имеют единственный центр), какие имеют центра, какие имеют бесконечно много центров: $4 x^2-20 x y+25 y^2-14 x+2 y-15=0$; \\
\textbf{A3.} Не проводя преобразования координат, установить, что каждое из следующих уравнений определяет единственную точку (вырожденный эллипс), и найти ее координаты: $x^2+2 x y+2 y^2+6 y+9=0$; \\
\textbf{B1.} Точка $M_1(2 ;-1)$ лежит на эллипсе, фокус которого $F(1 ; 0)$, а соответствующая директриса дана уравнением $2 x-y-10=0$. Составить уравнение этого эллипса. \\
\textbf{B2.} Установить, что следующие линии являются центральными, и для каждой из них найти координаты центра: $3 x^2+5 x y+y^2-8 x-11 y-7=0$; \\
\textbf{B3.} Составить уравнение параболы, если даны ее фокус $F(4 ; 3)$ и директриса $y+1=0$. \\
\textbf{C1.} При каких значениях $m$ и $n$ уравнение $m x^2+12 x y+9 y^2+4 x+n y-13=0$ определяет: 1) центральную линию; 2) линию без центра; 3) линию, имеющую бесконечно много центров. \\
\textbf{C2.} Составить уравнение касагельной к параболе $y^2=2 p x$ в ее точке $M_1\left(x_1 ; y_1\right)$. \\
\textbf{C3.} Определить, при каком значении $m$ плоскость $x-2 y-2 z+m=0$ касается эллипсоида $\frac{x^2}{144}+\frac{y^2}{36}+\frac{z^2}{9}=1$. \\

\end{tabular}
\vspace{1cm}



\end{document}
