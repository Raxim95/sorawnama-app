\documentclass{article}
\usepackage[fontsize=14pt]{fontsize}
\usepackage[utf8]{inputenc}

\usepackage{array}
\usepackage[a4paper,
left=7mm,
right=5mm,
top=7mm,]{geometry}
\usepackage{amsmath}
\usepackage{amssymb}
\usepackage{amsfonts}
\usepackage{setspace}



\renewcommand{\baselinestretch}{1} 

\everymath{\displaystyle}
\everydisplay{\displaystyle}
% \linespread{1.25}

\DeclareMathOperator{\sign}{sign}


\begin{document}

\onehalfspacing
\pagenumbering{gobble}


\begin{tabular}{m{17cm}}
\textbf{1-variant}
\newline

\textbf{T1.} 
Poligon hám gistogramma(salıstirmalı jiyilik, intervallıq qatar, grafik)
 \\
\textbf{T2.} 
Statistikalıq baha qásiyetleri. (Jıljımaytuǵın, tiykarlı, effektiv)
 \\
\textbf{A1.} 
Kólemi \(n = 20\) ǵa teń bolǵan tańlanba berilgen: 4,3; 4,9; 13,4; 13,4; 6,5; 4,9; 4,9; 4,3; 5,1; 6,5; 6,5; 7,0; 4,3; 4,9; 6,5; 6,5; 5,1; 5,1; 4,9; 13,4. Bul tańlanbanıń statistikalıq bólistiriliwin tabıń.
 \\
\textbf{A2.} 
Kólemi \(n = 20\) ǵa teń bolǵan tańlanba berilgen: 4,2; 4,9; 13,8; 13,8; 6,6; 4,9; 4,9; 4,2; 5,3; 6,6; 6,6; 7,5; 4,2; 4,9; 6,6; 6,6; 5,3; 5,3; 4,9; 13,8. Bul tańlanbanıń empirikalıq bólistiriw funkciyasın tabıń.
 \\
\textbf{A3.} 
Joqarı matematika páninen 10 dana student test sınaqların tapsırǵan. Hárbir student 10 balǵa shekem toplawı múmkin. Eger test sınaqları nátiyjeleri boyınsha \{9, 10, 6, 7, 4, 8, 10, 7, 9, 10\} tańlanba alınǵan bolsa, onda tańlanba ortasha hám tańlanba dispersiyalardı tabıń.
 \\
\textbf{B1.} 
Eger ortasha kvadratlıq shetleniwi \(\sigma = 2\) bolǵan normal bólistirilgen bas toplamnan alınǵan kólemi \(n = 10\) ǵa teń tańlanba boyınsha \(\overline{x} = 5,4\) tańlanba ortasha mánisi tabılǵan bolsa, onda \(\gamma = 0,95\) isenimlilik penen belgisiz \(\theta\) matematikalıq kútiliwdi qaplaytuǵın isenimlilik intervalın dúziń.
 \\
\textbf{B2.} 
\(\lbrack 0,\theta\rbrack\) aralıqta teń ólshewli bólistirilgen \(\theta\) parametri ushın momentler usulı bahasın tabıń.
 \\
\textbf{B3.} 
Eger \(X^{(n)} = \left( X_{1},...,X_{n} \right)\) tańlanba \(\theta\) parametrli kórsetkishli bólistiriliwden alınǵan bolsa, onda belgisiz \(\theta\) parametrdiń shınlıqqa maksimal uqsaslıq usılı bahasın tabıń.
 \\
\textbf{C1.} 
Eger \(X^{(n)} = \left( X_{1},...,X_{n} \right)\) tańlanba \(\lbrack 0,\theta\rbrack\) aralıqta teń ólshemli bólistiriliwden alýnǵan bolsa, onda belgisiz \(\theta\) parametr ushın \((n + 1)X_{(1)}\) bahasın jıljımaǵanlıq hám tiykarlılıqqa tekseriń.
 \\
\textbf{C2.} 
Eger \(X^{(n)} = \left( X_{1},...,X_{n} \right)\) tańlanba tıǵızlıq funkciyası
$f(x,\theta) = \left\{ \begin{array}{r}
\theta_{1}^{- 1}e^{- \frac{x - \theta_{2}}{\theta_{1}}},\ \ \ x \geq \theta_{2}, \\
0,\ \ \ x < \theta_{2}
\end{array} \right.\ $
bolǵan bólistiriliwden alınǵan bolsa, onda belgisiz \(\left( \theta_{1},\theta_{2} \right)\) \(\theta_{1} > 0,\) \(\theta_{2} \in R\) vektor parametr ushın momentler usılı bahasın tabıń.
 \\
\textbf{C3.} 
Eger \(X^{(n)} = \left( X_{1},...,X_{n} \right)\) tańlanba tıǵızlıq funkciyası
$f(x;\theta) = \frac{\theta}{2}e^{- \theta|x|},\ x \in R$
bolǵan bólistiriliwden alınǵan bolsa, onda belgisiz \(\theta > 0\) parametrdiń shınlıqqa maksimal uqsaslıq bahasın tabıń.
 \\

\end{tabular}
\vspace{1cm}


\begin{tabular}{m{17cm}}
\textbf{2-variant}
\newline

\textbf{T1.} 
Tańlanba xarakteristikalar. (Variaciyalıq qatar, salıstırmalı jiyilik).
 \\
\textbf{T2.} 
Pirsonnıń xi-kvadrat kelisimlilik belgisi (Pirson teoreması).
 \\
\textbf{A1.} 
Kólemi \(n = 20\) ǵa teń bolǵan tańlanba berilgen: -2,1; 1,7; 3,3; 3,3; 11,7; 4,7; 1,7; 4,7; -2,1; 4,7; 4,7; 4,7; 8,0; -2,1; 1,7; 4,7; 8,0; 11,7; 1,7; 8,0. Bul tańlanbanıń statistikalıq bólistiriliwin tabıń.
 \\
\textbf{A2.} 
Kólemi \(n = 20\) ǵa teń bolǵan tańlanba berilgen: -2,2; 1,3; 3,8; 3,8; 11,5; 4,1; 1,3; 4,1; -2,2; 4,1; 4,1; 4,1; 8,4; -2,2; 1,3; 4,1; 8,4; 11,5; 1,3; 8,4. Bul tańlanbanıń empirikalıq bólistiriw funkciyasın tabıń.
 \\
\textbf{A3.} 
Joqarı matematika páninen 10 dana student test sınaqların tapsırǵan. Hárbir student 10 balǵa shekem toplawı múmkin. Eger test sınaqları nátiyjeleri boyınsha \{4, 1, 2, 4, 6, 4, 5, 3, 6, 5\} tańlanba alınǵan bolsa, onda tańlanba ortasha hám tańlanba dispersiyalardı tabıń.
 \\
\textbf{B1.} 
Eger normal bólistirilgen bas toplamnan alınǵan kólemi \(n = 16\) ǵa teń tańlanba boyınsha \(\overline{x} = 20,2\) tańlanba ortasha hám \({\overline{S}}^{2} = 0,64\) dúzetilgen tańlanba dispersiyalar tabılǵan bolsa, onda \(\gamma = 0,95\) isenimlilik penen belgisiz \(\theta\) matematikalıq kútiliw ushın isenimlilik interval dúziń.
 \\
\textbf{B2.} 
Eger \(X^{(n)} = \left( X_{1},...,X_{n} \right)\) tańlanba \(\theta\) parametrli kórsetkishli bólistiriliwden alınǵan bolsa, onda belgisiz \(\theta\) parametr ushın momentler usılı bahasın tabıń.
 \\
\textbf{B3.} 
Eger (4,8,5,3) tańlanba \(\left( a,\theta^{2} \right)\) parametrli normal bólistiriliwden alınǵan bolsa, onda belgisiz \(\theta^{2}\) parametrdiń shınlıqqa maksimal uqsaslıq bahasın tabıń.
 \\
\textbf{C1.} 
Eger \(X^{(n)} = \left( X_{1},...,X_{n} \right)\) tańlanba \(\lbrack 0,\theta\rbrack\) aralıqta teń ólshemli bólistiriliwden alınǵan bolsa, onda belgisiz \(\theta\) parametr ushın \(\frac{n + 1}{n}X_{(n)}\) bahasın jıljımaǵanlıq hám tiykarlılıqqa tekseriń.
 \\
\textbf{C2.} 
Eger \(X^{(n)} = \left( X_{1},...,X_{n} \right)\) tańlanba \(\frac{1}{\theta}\) parametrli kórsetkishli bólistiriliwden alınǵan bolsa, onda belgisiz \(\theta\) parametr ushın momentler usılı bahasın\({\ g(x) = x}^{k},\) \(k \in N\)funkciyası járdeminde tabıń.
 \\
\textbf{C3.} 
Eger \(X^{(n)} = \left( X_{1},...,X_{n} \right)\) tańlanba tıǵızlıq funkciyası
$f(x;\theta) = \frac{e^{x}}{\sqrt{2\pi}}\exp\left\{ - \frac{\left( e^{x} - \theta \right)^{2}}{2} \right\},\ x \in R$
bolǵan bólistiriliwden alınǵan bolsa, onda belgisiz \(\theta\) parametrdiń shınlıqqa maksimal uqsaslıq bahasın tabıń.
 \\

\end{tabular}
\vspace{1cm}


\begin{tabular}{m{17cm}}
\textbf{3-variant}
\newline

\textbf{T1.} 
Tańlanba xarakteristikaları.(tańlanba orta, tańlanba dispersiya)
 \\
\textbf{T2.} 
Statistikalıq gipotezalardı tekseriw (kritikalıq kóplik, 1 hám 2-túr qátelik)
 \\
\textbf{A1.} 
Kólemi \(n = 20\) ǵa teń bolǵan tańlanba berilgen: -11,0; -4,1; 0; 2,3; 1,2; 0; 1,2; 2,3; 2,3; 1,2; 2,3; -11,0; 3,4; 1,2; 3,4; 3,4; 0; 3,4; 2,3; 0. Bul tańlanbanıń statistikalıq bólistiriliwin tabıń.
 \\
\textbf{A2.} 
Kólemi \(n = 20\) ǵa teń bolǵan tańlanba berilgen: -11,2; -4,5; 0; 2,9; 1,7; 0; 1,7; 2,9; 2,9; 1,7; 2,9; -11,2; 3,1; 1,7; 3,1; 3,1; 0; 3,1; 2,9; 0. Bul tańlanbanıń empirikalıq bólistiriw funkciyasın tabıń.
 \\
\textbf{A3.} 
Joqarı matematika páninen 10 dana student test sınaqların tapsırǵan. Hárbir student 10 balǵa shekem toplawı múmkin. Eger test sınaqları nátiyjeleri boyınsha \{8, 9, 10, 4, 9, 7, 6, 7, 6, 4\} tańlanba alınǵan bolsa, onda tańlanba ortasha hám tańlanba dispersiyalardı tabıń.
 \\
\textbf{B1.} 
Eger normal bólistirilgen bas toplamnan alınǵan kólemi \(n = 11\) ǵa teń bolǵan tańlanba boyınsha \({\overline{S}}^{2} = 0,5\) dúzetilgen tańlanba dispersiya tabılǵan bolsa, onda \(\gamma = 0,90\) isenimlilik penen belgisiz \(\theta_{2}^{2}\) dispersiya ushın isenimlilik interval dúziń.
 \\
\textbf{B2.} 
Eger (3,0,-2,0,-2,3,-2,0,0,3,0,0,0,3,-2,0,0,-2,3,0) tańlanba tómende berilgen bólistiriliwden alınǵan bolsa, onda belgisiz \(\left( \theta_{1},\theta_{2} \right)\) vektor parametr ushın momentler usılı bahalasın tabıń.
\begin{tabular}{|c|c|c|c|}
  \hline
$\xi$ &
$- 2$ &
$0$ &
$3$\\
\hline
\(P_{\theta}\) & \({2\theta}_{1}\) & \(0,5 + \theta_{1} + \theta_{2}\) & \(\theta_{2}\) \\
\hline
\end{tabular}
 \\
\textbf{B3.} 
Eger \(X^{(n)} = \left( X_{1},...,X_{n} \right)\) tańlanba \(\lbrack - \theta,\theta\rbrack\) aralıqta teń ólshemli bólistiriliwden alınǵan bolsa, onda belgisiz \(\theta > 0\) parametrdiń shınlıqqa maksimal uqsaslıq usılı bahasın tabıń.
 \\
\textbf{C1.} 
Eger \(X^{(n)} = \left( X_{1},...,X_{n} \right)\) tańlanba \(M\xi = a\) belgili hám \(M\xi^{2}\) shekli bolǵan bólistiriliwden alınǵan bolsa, onda belgisiz \(D\xi\) dispersiya ushın \({\overline{S}}^{2}\) bahasın jıljımaǵanlıq hám tiykarlılıqqa tekseriń.
 \\
\textbf{C2.} 
Eger \(X^{(n)} = \left( X_{1},...,X_{n} \right)\) tańlanba \({\lbrack\theta}_{1},\theta_{1}{+ \theta}_{2}\rbrack\) aralıqta teń ólshemli bólistiriliwden alınǵan bolsa, onda belgisiz \(\left( \theta_{1},\theta_{2} \right)\) vektor parametr ushın momentler usılı bahasın tabıń.
 \\
\textbf{C3.} 
Eger \(X^{(n)} = \left( X_{1},...,X_{n} \right)\) tańlanba tıǵızlıq funkciyası
$f(x;\theta) = \frac{4x^{3}}{\sqrt{2\pi}\theta_{2}}\exp\left\{ - \frac{\left( x^{4} - \theta_{1} \right)^{2}}{2{\theta_{2}}^{2}} \right\},\ x \in R$
bolǵan bólistiriliwden alınǵan bolsa, onda belgisiz \(\left( \theta_{1},\theta_{2}^{2} \right)\) vektor parametrdiń shınlıqqa maksimal uqsaslıq usılı bahaların tabıń.
 \\

\end{tabular}
\vspace{1cm}


\begin{tabular}{m{17cm}}
\textbf{4-variant}
\newline

\textbf{T1.} 
Neyman-Pirson teoreması
 \\
\textbf{T2.} 
Normal nızamnıń dispersiyası ushın isenimlilik intervalın dúziw. (Isenimlilik itimallıǵı, interval)
 \\
\textbf{A1.} 
Kólemi \(n = 20\) ǵa teń bolǵan tańlanba berilgen: 2,5; 3,8; 4,3; 2,5; 3,8; 2,5; 3,1; 4,3; 4,3; 5,5; 6,2; 2,5; 3,1; 6,2; 5,5; 6,2; 3,1; 3,1; 6,2; 3,1. Bul tańlanbanıń statistikalıq bólistiriliwin tabıń.
 \\
\textbf{A2.} 
Kólemi \(n = 20\) ǵa teń bolǵan tańlanba berilgen: 2,7; 4,2; 4,8; 2,7; 4,2; 2,7; 3,9; 4,8; 4,8; 5,9; 6,5; 2,7; 3,9; 6,5; 5,9; 6,5; 3,9; 3,9; 6,5; 3,9. Bul tańlanbanıń empirikalıq bólistiriw funkciyasın tabıń.
 \\
\textbf{A3.} 
Joqarı matematika páninen 10 dana student test sınaqların tapsırǵan. Hárbir student 10 balǵa shekem toplawı múmkin. Eger test sınaqları nátiyjeleri boyınsha \{7, 8, 7, 6, 4, 8, 4, 7, 9, 10\} tańlanba alınǵan bolsa, onda tańlanba ortasha hám tańlanba dispersiyalardı tabıń.
 \\
\textbf{B1.} 
Eger ortasha kvadratlıq shetleniwi \(\sigma = 3\) bolǵan normal bólistirilgen bas toplamnan alınǵan kólemi \(n = 9\) ǵa teń tańlanba boyınsha \(\overline{x} = 4,5\) tańlanba ortasha mánisi tabılǵan bolsa, onda \(\gamma = 0,95\) isenimlilik penen belgisiz \(\theta\) matematikalıq kútiliwdi qaplaytuǵın isenimlilik intervalın dúziń.
 \\
\textbf{B2.} 
Eger (3,-2,-2,0,-2,-2,-2,0,-2,3,-2,0,3,0,3,-2,0,-2,3,-2,-2,-2,-2,3,3,3,-2,-2,3,3) tańlanba tómende berilgen bólistiriliwden alınǵan bolsa, onda belgisiz \(\theta\) parametr ushın momentler usılı bahasın \(g(x) = |x|\) funkciyası járdeminde tabıń.
\begin{tabular}{|c|c|c|c|}
  \hline
$\xi$ &
$- 2$ &
$0$ &
$3$ \\
\hline
\(P_{\theta}\) & \(3\theta\) & \(1 - 5\theta\) & \(2\theta\) \\
\hline
\end{tabular}
 \\
\textbf{B3.} 
Eger \(X^{(n)} = \left( X_{1},...,X_{n} \right)\) tańlanba \(\left\lbrack - \theta,\theta^{2} \right\rbrack\) aralıqta teń ólshemli bólistiriliwden alınǵan bolsa, onda belgisiz \(\theta > 0\) parametrdiń shınlıqqa maksimal uqsaslıq usılı bahasın tabıń.
 \\
\textbf{C1.} 
Eger \(X^{(n)} = \left( X_{1},...,X_{n} \right)\) tańlanba \(M\xi = a\) belgili hám \(M\xi^{2}\) shekli bolǵan bólistiriliwden alınǵan bolsa, onda belgisiz \(D\xi\) dispersiya ushın \(\frac{1}{n - 1}\sum_{i = 1}^{n}\left( X_{i} - a \right)^{2}\) bahasın jıljımaǵanlıq hám tiykarlılıqqa tekseriń.
 \\
\textbf{C2.} 
Eger \(X^{(n)} = \left( X_{1},...,X_{n} \right)\) tańlanba \(\left( \theta_{1},\theta_{2} \right)\) parametrli gamma bólistiriliwden alınǵan bolsa, onda belgisiz \(\left( \theta_{1},\theta_{2} \right)\) vektor parametr ushın momentler usılı bahasın tabıń.
 \\
\textbf{C3.} 
Eger \(X^{(n)} = \left( X_{1},...,X_{n} \right)\) tańlanba \(\left\lbrack \theta_{1},\theta_{2} \right\rbrack\) aralıqta teń ólshemli bólistiriliwden alınǵan bolsa, onda belgisiz \(\left( \theta_{1},\theta_{2} \right)\) vektor parametrdiń shınlıqqa maksimal uqsaslıq bahasın tabıń.
 \\

\end{tabular}
\vspace{1cm}


\begin{tabular}{m{17cm}}
\textbf{5-variant}
\newline

\textbf{T1.} 
Empirikalıq bólistiriw funkciyası. (Tańlanba, eksperiment)
 \\
\textbf{T2.} 
Sızıqlı korrelyaciya teńlemesi (anıqlaması, regressiya tuwrı sızıǵınıń tańlanba teńlemeleri)
 \\
\textbf{A1.} 
Kólemi \(n = 20\) ǵa teń bolǵan tańlanba berilgen: -4,3; 2,6; 0; -2,5; 2,6; 1,9; 2,2; 0; -4,3; -2,5; 1,9; -2,5; 1,9; 2,2; 2,6; 1,9; 2,6; 2,2; 2,2; 1,9. Bul tańlanbanıń statistikalıq bólistiriliwin tabıń.
 \\
\textbf{A2.} 
Kólemi \(n = 20\) ǵa teń bolǵan tańlanba berilgen: -4,9; 2,6; 0,5; -2,6; 2,6; 1,7; 2,3; 0,5; -4,9; -2,6; 1,7; -2,6; 1,7; 2,3; 2,6; 1,7; 2,6; 2,3; 2,3; 1,7. Bul tańlanbanıń empirikalıq bólistiriw funkciyasın tabıń.
 \\
\textbf{A3.} 
Joqarı matematika páninen 10 dana student test sınaqların tapsırǵan. Hárbir student 10 balǵa shekem toplawı múmkin. Eger test sınaqları nátiyjeleri boyınsha \{9, 5, 6, 8, 4, 7, 4, 6, 9, 7\} tańlanba alınǵan bolsa, onda tańlanba ortasha hám tańlanba dispersiyalardı tabıń.
 \\
\textbf{B1.} 
Eger normal bólistirilgen bas toplamnan alınǵan kólemi \(n = 25\) ǵa teń tańlanba boyınsha \(\overline{x} = 18,6\) tańlanba ortasha hám \({\overline{S}}^{2} = 0,49\) dúzetilgen tańlanba dispersiyalar tabılǵan bolsa, onda \(\gamma = 0,95\) isenimlilik penen belgisiz \(\theta\) matematikalıq kútiliw ushın isenimlilik interval dúziń.
 \\
\textbf{B2.} 
\(\lbrack\theta_{1},\theta_{2}\rbrack\) aralıqta teń ólshewli bólistiriw parametrleri ushın momentler usulı bahaların tabıń.
 \\
\textbf{B3.} 
\(\ f(x) = \frac{2x}{\theta}e^{- \frac{x^{2}}{\theta}},\ \ x \geq 0\) model ushın \(\theta\) parametri haqıyqatqa maksimal uqsaslıq usılı bahası tabılsın.
 \\
\textbf{C1.} 
Eger \(X^{(n)} = \left( X_{1},...,X_{n} \right)\) tańlanba \(M\xi = a\) belgili hám \(M\xi^{2}\) shekli bolǵan bólistiriliwden alınǵan bolsa, onda belgisiz \(D\xi\) dispersiya ushın \(\frac{1}{n}\sum_{i = 1}^{n}\left( X_{i} - a \right)^{2}\) bahasın jıljımaǵanlıq hám tiykarlılıqqa tekseriń.
 \\
\textbf{C2.} 
Eger \(X^{(n)} = \left( X_{1},...,X_{n} \right)\) tańlanba \({(\theta,\theta}^{2})\) parametrli normal bólistiriliwden \({\ g(x) = (x)}^{2}\ \)alınǵan bolsa, onda belgisiz \(\theta > 0\) parametr ushın momentler usılı bahasın funkciyası járdeminde tabıń.
 \\
\textbf{C3.} 
\(\ f(x,\theta) = \frac{4x^{3}}{\theta_{2}\sqrt{2\pi}}\exp\left\{ - \frac{\left( x^{4} - \theta_{1} \right)^{2}}{2{\theta_{2}}^{2}} \right\}\) model ushın \(\theta_{1}\) hám \({\theta_{2}}^{2}\) parametrler haqıyqatqa maksimal uqsaslıq usılı bahaları tabılsın.
 \\

\end{tabular}
\vspace{1cm}


\begin{tabular}{m{17cm}}
\textbf{6-variant}
\newline

\textbf{T1.} 
Glivenko-Kantelli teoreması. (empirikalıq bólistiriw funkciyası, 1itimallıq penen jaqınlasıw)
 \\
\textbf{T2.} 
Statistikalıq gipotezalardı tekseriw (kritikalıq kóplik, 1 hám 2-túr qátelik).
 \\
\textbf{A1.} 
Kólemi \(n = 20\) ǵa teń bolǵan tańlanba berilgen: -2,9; -3,8; 2,3; 1,8; 1,8; 0,7; -3,8; -1,5; 2,3; 0,7; -2,9; -1,5; 1,8; -2,9; -1,5; -3,8; 1,8; 1,8; -3,8; 1,8. Bul tańlanbanıń statistikalıq bólistiriliwin tabıń.
 \\
\textbf{A2.} 
Kólemi \(n = 20\) ǵa teń bolǵan tańlanba berilgen: -2,4; -3,5; 2,8; 1,4; 1,4; 0,1; -3,5; -1,9; 2,8; 0,1; -2,4; -1,9; 1,4; -2,4; -1,9; -3,5; 1,4; 1,4; -3,5; 1,4. Bul tańlanbanıń empirikalıq bólistiriw funkciyasın tabıń.
 \\
\textbf{A3.} 
Joqarı matematika páninen 10 dana student test sınaqların tapsırǵan. Hárbir student 10 balǵa shekem toplawı múmkin. Eger test sınaqları nátiyjeleri boyınsha \{8, 9, 7, 10, 6, 8, 10, 3, 10, 9\} tańlanba alınǵan bolsa, onda tańlanba ortasha hám tańlanba dispersiyalardı tabıń.
 \\
\textbf{B1.} 
Eger normal bólistirilgen bas toplamnan alınǵan kólemi \(n = 12\) ǵa teń bolǵan tańlanba boyınsha \({\overline{S}}^{2} = 0,4\) dúzetilgen tańlanba dispersiya tabılǵan bolsa, onda \(\gamma = 0,90\) isenimlilik penen belgisiz \(\theta_{2}^{2}\) dispersiya ushın isenimlilik interval dúziń.
 \\
\textbf{B2.} 
Eger \(X^{(n)} = \left( X_{1},...,X_{n} \right)\) tańlanba \(\theta\) parametrli Bernulli bólistiriliwinen alınǵan bolsa, onda belgisiz \(\theta\) parametr ushın momentler usılı bahasın tabıń.
 \\
\textbf{B3.} 
Eger (0,1,2,0) tańlanba tómende berilgen bólistiriliwden alınǵan bolsa, onda belgisiz \(\theta\) parametrdiń shınlıqqa maksimal uqsaslıq bahasın tabıń.
\begin{tabular}{|c|c|c|c|}
  \hline
$\xi$
&
$0$
&
$1$
&
$2$\\
\hline
\(P_{\theta}\) & \(\theta\) & \(2\theta\) & \(1 - 3\theta\) \\
\hline
\end{tabular}
 \\
\textbf{C1.} 
Eger \(X^{(n)} = \left( X_{1},...,X_{n} \right)\) tańlanba \(M\xi = a\) belgili hám \(M\xi^{2}\) shekli bolǵan bólistiriliwden alınǵan bolsa, onda belgisiz \(D\xi\) dispersiya ushın \(\overline{x^{2}} - a^{2}\) bahasın jıljımaǵanlıq hám tiykarlılıqqa tekseriń.
 \\
\textbf{C2.} 
Eger \(X^{(n)} = \left( X_{1},...,X_{n} \right)\) tańlanba {[}\(0,2\theta\rbrack\) aralıqta teń ólshemli bólistiriliwden alınǵan bolsa, onda belgisiz \(\theta > 0\) parametr ushın momentler usılı bahasın tabıń.
 \\
\textbf{C3.} 
Eger \(X^{(n)} = \left( X_{1},...,X_{n} \right)\) tańlanba tıǵızlıq funkciyası
$f(x;\theta) = \left\{ \begin{matrix}
e^{\theta - x},\ \ x \geq \theta, \\
\ \ 0,\ \ \ \ \ \ \ x < \theta
\end{matrix} \right.\ $
bolǵan bólistiriliwden alınǵan bolsa, onda belgisiz \(\theta\) parametrdiń shınlıqqa maksimal uqsaslıq bahasın tabıń.
 \\

\end{tabular}
\vspace{1cm}


\begin{tabular}{m{17cm}}
\textbf{7-variant}
\newline

\textbf{T1.} Matematikalıq statistikanıń tiykarǵı máseleleri. (Statistikalıq maǵlıwmatlar, gruppalaw)
 \\
\textbf{T2.} 
Kolmogorovtıń kelisimlilik belgisi (Kolmogorov teoreması)
 \\
\textbf{A1.} 
Kólemi \(n = 20\) ǵa teń bolǵan tańlanba berilgen: 3,6; 2,9; 3,6; 3,2; 1,1; 0,3; 1,1; 3,6; 1,7; 1,1; 0,3; 1,7; 1,1; 0,3; 2,9; 2,9; 2,9; 1,1; 2,9; 1,7. Bul tańlanbanıń statistikalıq bólistiriliwin tabıń.
 \\
\textbf{A2.} 
Kólemi \(n = 20\) ǵa teń bolǵan tańlanba berilgen: 4,6; 2,5; 4,6; 3,3; 1,8; 0,3; 1,8; 4,6; 2,1; 1,8; 0,3; 2,1; 1,8; 0,3; 2,5; 2,5; 2,5; 1,8; 2,5; 2,1. Bul tańlanbanıń empirikalıq bólistiriw funkciyasın tabıń.
 \\
\textbf{A3.} 
Joqarı matematika páninen 10 dana student test sınaqların tapsırǵan. Hárbir student 10 balǵa shekem toplawı múmkin. Eger test sınaqları nátiyjeleri boyınsha \{5, 7, 5, 9, 5, 8, 10, 6, 7, 8\} tańlanba alınǵan bolsa, onda tańlanba ortasha hám tańlanba dispersiyalardı tabıń.
 \\
\textbf{B1.} 
Eger ortasha kvadratlıq shetleniwi \(\sigma = 1\) bolǵan normal bólistirilgen bas toplamnan alınǵan kólemi \(n = 15\) ǵa teń tańlanba boyınsha \(\overline{x} = 5,8\) tańlanba ortasha mánisi tabılǵan bolsa, onda \(\gamma = 0,90\) isenimlilik penen belgisiz \(\theta\) matematikalıq kútiliwdi qaplaytuǵın isenimlilik intervalın dúziń.
 \\
\textbf{B2.} 
Kórsetkishli bólistiriw belgisiz \(\theta > 0\) parametri momentlar usulı bahasın tabıń.
 \\
\textbf{B3.} 
\(\ f(x) = \frac{\theta}{2}e^{- \theta|x|}\) model ushın \(\theta\) parametri haqıyqatqa maksimal uqsaslıq usılı bahası tabılsın.
 \\
\textbf{C1.} 
Eger \(X^{(n)} = \left( X_{1},...,X_{n} \right)\) tańlanba tıǵızlıq funkciyası: \(f(x,\theta) = \left\{ \begin{matrix}
e^{\theta - x},\ \ x \geq \theta, \\
\ \ 0,\ \ \ \ \ \ \ x < \theta
\end{matrix} \right.\ \)
bolǵan bólistiriliwden alınǵan bolsa, onda belgisiz \(\theta\) parametr ushın \(X_{(1)}\) bahasın jıljımaǵanlıq hám tiykarlılıqqa tekseriń.
 \\
\textbf{C2.} 
Eger \(X^{(n)} = \left( X_{1},...,X_{n} \right)\) tańlanba \({(\theta,\theta}^{2})\ \) parametrli normal bólistiriliwden alınǵan bolsa, onda belgisiz \(\theta > 0\) parametr ushın momentler usılı bahasın tabıń.
 \\
\textbf{C3.} 
Eger \(X^{(n)} = \left( X_{1},...,X_{n} \right)\) tańlanba tıǵızlıq funkciyası
$f(x;\theta) = \left\{ \begin{matrix}
3x^{2}\theta^{- 3}e^{- \ \left( \frac{x}{\theta} \right)^{3}},\ \ x \geq 0, \\
\ \ \ \ \ \ \ \ \ \ \ \ \ \ 0,\ \ \ \ \ \ \ \ \ x < 0
\end{matrix} \right.\ $
bolǵan bólistiriliwden alınǵan bolsa, onda belgisiz \(\theta > 0\) parametrdiń shınlıqqa maksimal uqsaslıq bahasın tabıń.
 \\

\end{tabular}
\vspace{1cm}


\begin{tabular}{m{17cm}}
\textbf{8-variant}
\newline

\textbf{T1.} 
Gruppalanǵan hám intervallıq variaciyalıq qatarlar.
 \\
\textbf{T2.} 
Momentler usulı. (tańlanba momentleri, belgisiz parametrlerdi bahalaw).
 \\
\textbf{A1.} 
Kólemi \(n = 20\) ǵa teń bolǵan tańlanba berilgen: -1,3; 0; 0,8; 2,3; 1,1; 0,8; 0,8; 2,3; 1,1; 0,8; -1,3; 1,8; 1,1; -1,3; 1,1; 1,8; 1,8; 1,1; 1,8; 1,8. Bul tańlanbanıń statistikalıq bólistiriliwin tabıń.
 \\
\textbf{A2.} 
Kólemi \(n = 20\) ǵa teń bolǵan tańlanba berilgen: -1,9; 0,7; 0,9; 2,8; 1,3; 0,9; 0,9; 2,8; 1,3; 0,9; -1,9; 1,6; 1,3; -1,9; 1,3; 1,6; 1,6; 1,3; 1,6; 1,6. Bul tańlanbanıń empirikalıq bólistiriw funkciyasın tabıń.
 \\
\textbf{A3.} 
Joqarı matematika páninen 10 dana student test sınaqların tapsırǵan. Hárbir student 10 balǵa shekem toplawı múmkin. Eger test sınaqları nátiyjeleri boyınsha \{8, 4, 3, 7, 3, 6, 5, 3, 5, 6\} tańlanba alınǵan bolsa, onda tańlanba ortasha hám tańlanba dispersiyalardı tabıń.
 \\
\textbf{B1.} 
Eger normal bólistirilgen bas toplamnan alınǵan kólemi \(n = 20\) ǵa teń tańlanba boyınsha \(\overline{x} = 16,6\) tańlanba ortasha hám \({\overline{S}}^{2} = 0,64\) dúzetilgen tańlanba dispersiyalar tabılǵan bolsa, onda \(\gamma = 0,95\) isenimlilik penen belgisiz \(\theta\) matematikalıq kútiliw ushın isenimlilik interval dúziń.
 \\
\textbf{B2.} 
Eger (-2,0,-2,0,-2,3,-2,0,0,3,0,0,0,3,-2,0,0,-2,3,0) tańlanba tómende berilgen bólistiriliwden alınǵan bolsa, onda belgisiz \(\left( \theta_{1},\theta_{2} \right)\) vektor parametr ushın momentler usılı bahalasın tabıń.
\begin{tabular}{|c|c|c|c|}
  \hline
$\xi$ &
$- 2$ &
$0$ &
$3$\\
\hline
\(P_{\theta}\) & \(\theta_{1}\) & \(1 - \theta_{1} - \theta_{2}\) & \(\theta_{2}\) \\
\hline
\end{tabular}
 \\
\textbf{B3.} 
Eger \(X^{(n)} = \left( X_{1},...,X_{n} \right)\) tańlanba \(\theta\) parametrli Bernulli bólistiriliwinen alınǵan bolsa, onda belgisiz \(\theta\) parametrdiń shınlıqqa maksimal uqsaslıq usılı bahasın tabıń.
 \\
\textbf{C1.} 
Eger \(X^{(n)} = \left( X_{1},...,X_{n} \right)\) tańlanba tıǵızlıq funkciyası: \(f(x,\theta) = \left\{ \begin{matrix}
e^{\theta - x},\ \ x \geq \theta, \\
\ \ 0,\ \ \ \ \ \ \ x < \theta
\end{matrix} \right.\ \)
bolǵan bólistiriliwden alınǵan bolsa, onda belgisiz \(\theta\) parametr ushın \(\overline{x} - 1\) bahasın jıljımaǵanlıq hám tiykarlılıqqa tekseriń.
 \\
\textbf{C2.} 
Eger \(X^{(n)} = \left( X_{1},...,X_{n} \right)\) tańlanba \({\lbrack\theta}_{1},\theta_{2}\rbrack\) aralıqta teń ólshemli bólistiriliwden alınǵan bolsa, onda belgisiz \(\left( \theta_{1},\theta_{2} \right)\) vektor parametr ushın momentler usılı bahasın tabıń.
 \\
\textbf{C3.} 
Eger \(X^{(n)} = \left( X_{1},...,X_{n} \right)\) tańlanba \(\lbrack\theta,\theta + 2\rbrack\) aralıqta teń ólshemli bólistiriliwden alınǵan bolsa, onda belgisiz \(\theta\) parametrdiń shınlıqqa maksimal uqsaslıq usılı bahasın tabıń.
 \\

\end{tabular}
\vspace{1cm}


\begin{tabular}{m{17cm}}
\textbf{9-variant}
\newline

\textbf{T1.} 
Tańlanba momentleri (\(k -\)tártipli baslanǵısh, baslanǵısh absolyut, oraylıq hám oraylıq absolyut momentler).
 \\
\textbf{T2.} 
Haqiyqatqa maksimal uqsaslıq usulı. (haqiyqatqa maksimal uqsaslıq funkciyası, belgisiz parametrlerdi bahalaw).
 \\
\textbf{A1.} 
Kólemi \(n = 20\) ǵa teń bolǵan tańlanba berilgen: -2,4; 5,6; 5,6; -5,2; -6,7; 5,1; -5,2; -2,4; 4,3; 5,1; -6,7; 4,3; -2,4; -6,7; 4,3; 5,1; 4,3; 5,6; -6,7; 5,6. Bul tańlanbanıń statistikalıq bólistiriliwin tabıń.
 \\
\textbf{A2.} 
Kólemi \(n = 20\) ǵa teń bolǵan tańlanba berilgen: -2,9; 7,6; 7,6; -5,7; -6,1; 5,5; -5,7; -2,9; 4,2; 5,5; -6,1; 4,2; -2,9; -6,1; 4,2; 5,5; 4,2; 7,6; -6,1; 7,6. Bul tańlanbanıń empirikalıq bólistiriw funkciyasın tabıń.
 \\
\textbf{A3.} 
Joqarı matematika páninen 10 dana student test sınaqların tapsırǵan. Hárbir student 10 balǵa shekem toplawı múmkin. Eger test sınaqları nátiyjeleri boyınsha \{9, 8, 6, 7, 5, 8, 5, 7, 4, 6\} tańlanba alınǵan bolsa, onda tańlanba ortasha hám tańlanba dispersiyalardı tabıń.
 \\
\textbf{B1.} 
Eger normal bólistirilgen bas toplamnan alınǵan kólemi \(n = 13\) ǵa teń bolǵan tańlanba boyınsha \({\overline{S}}^{2} = 1,2\) dúzetilgen tańlanba dispersiya tabılǵan bolsa, onda \(\gamma = 0,90\) isenimlilik penen belgisiz \(\theta_{2}^{2}\) dispersiya ushın isenimlilik interval dúziń.
 \\
\textbf{B2.} 
Eger (0,-2,0,-2,3,-2,0,0,3,0,0,0,3,-2,0,0,-2,3,0,3) tańlanba tómende berilgen bólistiriliwden alınǵan bolsa, onda belgisiz \(\theta\) parametr ushın momentler usılı bahasın tabıń.
\begin{tabular}{|c|c|c|c|}
  \hline
$\xi$ & $- 2$  & $0$  & $3$ \\
\hline
\(P_{\theta}\) & \(\theta\) & \(1 - 2\theta\) & \(\theta\) \\
\hline
\end{tabular}
 \\
\textbf{B3.} 
Eger \(X^{(n)} = \left( X_{1},...,X_{n} \right)\) tańlanba tıǵızlıq funkciyası \(f(x;\theta) = \frac{2x}{\theta}e^{- \frac{x^{2}}{\theta}},\ x \geq 0\). bolǵan bólistiriliwden alınǵan bolsa, onda belgisiz \(\theta > 0\) parametrdiń shınlıqqa maksimal uqsaslıq bahasın tabıń.
 \\
\textbf{C1.} 
Eger \(X^{(n)} = \left( X_{1},...,X_{n} \right)\) tańlanba \(\lbrack - 3\theta,\theta\rbrack\) aralıqta teń ólshemli bólistiriliwden alınǵan bolsa, onda belgisiz \(\theta\) parametr ushın \(4X_{(n)} + X_{(1)}\) bahasın jıljımaǵanlıq hám tiykarlılıqqa tekseriń.
 \\
\textbf{C2.} 
Eger \(X^{(n)} = \left( X_{1},...,X_{n} \right)\) tańlanba \(\theta\) parametrli geometriyalıq bólistiriliwden alınǵan bolsa, onda belgisiz \(\theta\) parametr ushın momentler usılı bahasın tabıń.
 \\
\textbf{C3.} 
Eger \(X^{(n)} = \left( X_{1},...,X_{n} \right)\) tańlanba tıǵızlıq funkciyası
$f(x;\theta) = \frac{3x^{2}}{\sqrt{2\pi}}\exp\left\{ - \frac{\left( x^{3} - \theta \right)^{2}}{2} \right\},\ x \in R$
bolǵan bólistiriliwden alınǵan bolsa, onda belgisiz \(\theta\) parametrdiń shınlıqqa maksimal uqsaslıq bahasın tabıń.
 \\

\end{tabular}
\vspace{1cm}


\begin{tabular}{m{17cm}}
\textbf{10-variant}
\newline

\textbf{T1.} 
Momentler usulı. (tańlanba momentleri, belgisiz parametrlerdi bahalaw).
 \\
\textbf{T2.} 
Isenimlilik intervalların qurıw. Anıq isenimli intervallar
 \\
\textbf{A1.} 
Kólemi \(n = 20\) ǵa teń bolǵan tańlanba berilgen:-3,3; 0; 4,4; 2,2; -2,7; 4,4; 2,2; 4,4;-3,3; 2,2; -2,7; 2,2; -3,3; -2,7; 2,2; 3,4; 4,4; 0; -3,3; 0. Bul tańlanbanıń statistikalıq bólistiriliwin tabıń.
 \\
\textbf{A2.} 
Kólemi \(n = 20\) ǵa teń bolǵan tańlanba berilgen:-3,3; 0; 4,9; 2,8; -2,6; 4,9; 2,8; 4,9;-3,3; 2,8; -2,6; 2,8; -3,3; -2,6; 2,8; 3,1; 4,9; 0; -3,3; 0. Bul tańlanbanıń empirikalıq bólistiriw funkciyasın tabıń.
 \\
\textbf{A3.} 
Joqarı matematika páninen 10 dana student test sınaqların tapsırǵan. Hárbir student 10 balǵa shekem toplawı múmkin. Eger test sınaqları nátiyjeleri boyınsha \{4, 7, 6, 9, 3, 8, 3, 7, 4, 9\} tańlanba alınǵan bolsa, onda tańlanba ortasha hám tańlanba dispersiyalardı tabıń.
 \\
\textbf{B1.} 
Eger ortasha kvadratlıq shetleniwi \(\sigma = 4\) bolǵan normal bólistirilgen bas toplamnan alınǵan kólemi \(n = 12\) ǵa teń tańlanba boyınsha \(\overline{x} = 3\) tańlanba ortasha mánisi tabılǵan bolsa, onda \(\gamma = 0,95\) isenimlilik penen belgisiz \(\theta\) matematikalıq kútiliwdi qaplaytuǵın isenimlilik intervalın dúziń.
 \\
\textbf{B2.} 
Puasson bólistiriliwi belgisiz \(\theta > 0\) parametri momentlar usuli bahasin tabıń.
 \\
\textbf{B3.} 
Eger (-1,-1,0,-1,0,-1,-1,5,-1,0,-1,0,5,-1,-1,-1,5,-1,-1,-1,5,0,-1,-1,5) tańlanba tómende berilgen bólistiriliwden alınǵan bolsa, onda belgisiz \(\theta\) parametrdiń shınlıqqa maksimal uqsaslıq usılı bahasın tabıń.
\begin{tabular}{|c|c|c|c|}
  \hline
$\xi$
&
$- 1$
&
$0$
&
$5$\\
\hline
\(P_{\theta}\) & \(1 - \theta\) & \(\theta/2\) & \(\theta/2\ \) \\
\hline
\end{tabular}
 \\
\textbf{C1.} 
Eger \(X^{(n)} = \left( X_{1},...,X_{n} \right)\) tańlanba bólistiriw funkciyası \(F(x)\) bolǵan bólistiriliwden alınǵan bolsa, onda belgisiz \(F(x)\) ushın \(F_{n}(x)\) empirikalıq bólistiriw funkciyasın jıljımaǵanlıq hám tiykarlılıqqa tekseriń.
 \\
\textbf{C2.} 
Eger \(X^{(n)} = \left( X_{1},...,X_{n} \right)\) tańlanba \(\theta\) parametrli Puasson bólistiriliwinen alınǵan bolsa, onda belgisiz \(\theta\) parametr ushın momentler usılı bahasın tabıń. Eger \(X^{(n)} = \left( X_{1},...,X_{n} \right)\) tańlanba \(\theta\) parametrli Puasson bólistiriliwinen alınǵan bolsa, onda belgisiz \(\theta\) parametr ushın momentler usılı bahasın\({\ g(x) = x}^{2}\) funkciyası járdeminde tabıń.
 \\
\textbf{C3.} 
Eger \(X^{(n)} = \left( X_{1},...,X_{n} \right)\) tańlanba \((\theta,2\theta)\) parametrli normal bólistiriliwden alınǵan bolsa, onda belgisiz \(\theta > 0\) parametrdiń shınlıqqa maksimal uqsaslıq bahasın tabıń.
 \\

\end{tabular}
\vspace{1cm}


\begin{tabular}{m{17cm}}
\textbf{11-variant}
\newline

\textbf{T1.} 
Poligon hám gistogramma(salıstirmalı jiyilik, intervallıq qatar, grafik)
 \\
\textbf{T2.} 
Momentler usulı. (tańlanba momentleri, belgisiz parametrlerdi bahalaw).
 \\
\textbf{A1.} 
Kólemi \(n = 20\) ǵa teń bolǵan tańlanba berilgen: 3,7; 3,1; 4,8; 2,8; 3,1; 4,3; 3,7; 4,3; 2,4; 3,1; 2,4; 4,3; 3,1; 3,7; 4,8; 2,8; 2,4; 2,8; 2,4; 3,1. Bul tańlanbanıń statistikalıq bólistiriliwin tabıń.
 \\
\textbf{A2.} 
Kólemi \(n = 20\) ǵa teń bolǵan tańlanba berilgen: 3,8; 3,4; 4,8; 2,9; 3,4; 4,6; 3,8; 4,6; 2,1; 3,4; 2,1; 4,6; 3,4; 3,8; 4,8; 2,9; 2,1; 2,9; 2,1; 3,4. Bul tańlanbanıń empirikalıq bólistiriw funkciyasın tabıń.
 \\
\textbf{A3.} 
Joqarı matematika páninen 10 dana student test sınaqların tapsırǵan. Hárbir student 10 balǵa shekem toplawı múmkin. Eger test sınaqları nátiyjeleri boyınsha \{6, 5, 6, 9, 5, 7, 10, 5, 9, 8\} tańlanba alınǵan bolsa, onda tańlanba ortasha hám tańlanba dispersiyalardı tabıń.
 \\
\textbf{B1.} 
Eger normal bólistirilgen bas toplamnan alınǵan kólemi \(n = 25\) ǵa teń tańlanba boyınsha \(\overline{x} = 9\) tańlanba ortasha hám \({\overline{S}}^{2} = 0,64\) dúzetilgen tańlanba dispersiyalar tabılǵan bolsa, onda \(\gamma = 0,95\) isenimlilik penen belgisiz \(\theta\) matematikalıq kútiliw ushın isenimlilik interval dúziń.
 \\
\textbf{B2.} 
Eger tıǵızlıq funkciyası \(f(x) = \frac{2x}{\theta}e^{- \frac{x^{2}}{\theta}},\ \ x \geq 0\) kóriniske iye bolsa, onda \(\theta\) parametr momentler usulı bahasın tabıń.
 \\
\textbf{B3.} 
Eger \(x_{1} = 1,1;\ x_{2} = 2,7;\ldots;x_{100} = 1,5\) tańlanba \(\theta\) parametrli kórsetkishli bólistiriliwden alınǵan bolıp, \(\sum_{k = 1}^{100}x_{k} = 200\) bolsa, onda belgisiz \(\theta\) parametrdiń shınlıqqa maksimal uqsaslıq bahasın tabıń.
 \\
\textbf{C1.} 
Eger \(X^{(n)} = \left( X_{1},...,X_{n} \right)\) tańlanba \(\left( a,\theta^{2} \right)\) parametrli normal bólistiriliwden alınǵan bolsa (\(a -\)belgili), onda belgisiz \(\theta\) parametr ushın \(\sqrt{\frac{\pi}{2}}\left| \overline{x - a} \right|\) bahasın jıljımaǵanlıq hám tiykarlılıqqa tekseriń.
 \\
\textbf{C2.} 
Eger \(X^{(n)} = \left( X_{1},...,X_{n} \right)\) tańlanba \(\frac{1}{\sqrt{\theta}}\) parametrli kórsetkishli bólistiriliwden alınǵan bolsa, onda belgisiz \(\theta\) parametr ushın momentler usılı bahasın tabıń.
 \\
\textbf{C3.} 
Eger \(X^{(n)} = \left( X_{1},...,X_{n} \right)\) tańlanba \(\theta\) parametrli geometriyalıq bólistiriliwden alınǵan bolsa, onda belgisiz \(\theta\) parametrdiń shınlıqqa maksimal uqsaslıq usılı bahasın tabıń.
 \\

\end{tabular}
\vspace{1cm}


\begin{tabular}{m{17cm}}
\textbf{12-variant}
\newline

\textbf{T1.} 
Glivenko-Kantelli teoreması. (empirikalıq bólistiriw funkciyası, 1itimallıq penen jaqınlasıw)
 \\
\textbf{T2.} 
Statistikalıq baha qásiyetleri. (Jıljımaytuǵın, tiykarlı, effektiv)
 \\
\textbf{A1.} 
Kólemi \(n = 20\) ǵa teń bolǵan tańlanba berilgen: 1,5; -0,9; -2,4; -0,9; 0,7; 1,5; -0,9; -0,2; -2,4; 0,7; -2,4; 0,7; -0,9; 1,5; -1,7; -0,9; -0,2; 0,7; -1,7; -0,9. Bul tańlanbanıń statistikalıq bólistiriliwin tabıń.
 \\
\textbf{A2.} 
Kólemi \(n = 20\) ǵa teń bolǵan tańlanba berilgen: 1,9; -0,3; -2,7; -0,3; 0,6; 1,9; -0,3; -0,1; -2,7; 0,6; -2,7; 0,6; -0,3; 1,9; -1,8; -0,3; -0,1; 0,6; -1,8; -0,3. Bul tańlanbanıń empirikalıq bólistiriw funkciyasın tabıń.
 \\
\textbf{A3.} 
Joqarı matematika páninen 10 dana student test sınaqların tapsırǵan. Hárbir student 10 balǵa shekem toplawı múmkin. Eger test sınaqları nátiyjeleri boyınsha \{4, 6, 6, 9, 5, 8, 4, 7, 5, 6\} tańlanba alınǵan bolsa, onda tańlanba ortasha hám tańlanba dispersiyalardı tabıń.
 \\
\textbf{B1.} 
Eger normal bólistirilgen bas toplamnan alınǵan kólemi \(n = 10\) ǵa teń bolǵan tańlanba boyınsha \({\overline{S}}^{2} = 0,6\) dúzetilgen tańlanba dispersiya tabılǵan bolsa, onda \(\gamma = 0,95\) isenimlilik penen belgisiz \(\theta_{2}^{2}\) dispersiya ushın isenimlilik interval dúziń.
 \\
\textbf{B2.} 
Eger \(X^{(n)} = \left( X_{1},...,X_{n} \right)\) tańlanba \(\theta\) parametrli Bernulli bólistiriliwinen alınǵan bolsa, onda belgisiz \(\theta\) parametr ushın momentler usılı bahasın tabıń.
 \\
\textbf{B3.} 
Eger \(X^{(n)} = \left( X_{1},...,X_{n} \right)\) tańlanba \(\left( a,\theta^{2} \right)\) parametrli normal bólistiriliwden alınǵan bolsa (\(\alpha -\)belgili), onda belgisiz \(\theta^{2}\) parametrdiń shınlıqqa maksimal uqsaslıq bahasın tabıń.
 \\
\textbf{C1.} 
Eger \(X^{(n)} = \left( X_{1},...,X_{n} \right)\) tańlanba \(\theta\) parametrli kórsetkishli bólistiriliwinen alınǵan bolsa, onda belgisiz \(\theta\) parametr ushın \(\frac{1}{\overline{x}}\) bahasın jıljımaǵanlıq hám tiykarlılıqqa tekseriń.
 \\
\textbf{C2.} 
Eger \(X^{(n)} = \left( X_{1},...,X_{n} \right)\) tańlanba \((\theta,2\theta)\) parametrli normal bólistiriliwden alınǵan bolsa, onda belgisiz \(\theta > 0\) parametr ushın momentler usılı bahasın \({\ g(x) = (x)}^{2}\) funkciyası járdeminde tabıń.
 \\
\textbf{C3.} 
Eger \(X^{(n)} = \left( X_{1},...,X_{n} \right)\) tańlanba tıǵızlıq funkciyası
$f(x;\theta) = \left\{ \begin{matrix}
\theta_{1}^{- 1}e^{\frac{x - \theta_{2}}{\theta_{1}}},\ \ x \geq \theta_{2}, \\
\ \ \ \ \ \ \ \ \ \ \ \ 0,\ \ \ \ \ \ \ x < \theta_{2}
\end{matrix} \right.\ $
bolǵan bólistiriliwden alınǵan bolsa, onda belgisiz \(\left( \theta_{1},\theta_{2} \right),\) \(\theta_{1} > 0,\) \(\theta_{2} \in R\) vektor parametrdiń shınlıqqa maksimal uqsaslıq bahasın tabıń.
 \\

\end{tabular}
\vspace{1cm}


\begin{tabular}{m{17cm}}
\textbf{13-variant}
\newline

\textbf{T1.} 
Momentler usulı. (tańlanba momentleri, belgisiz parametrlerdi bahalaw).
 \\
\textbf{T2.} 
Normal nızamnıń dispersiyası ushın isenimlilik intervalın dúziw. (Isenimlilik itimallıǵı, interval)
 \\
\textbf{A1.} 
Kólemi \(n = 20\) ǵa teń bolǵan tańlanba berilgen:9,4; 6,8; -8,5; 9,4; 2,9; 9,4; -8,5; -6,4; 6,8; -8,5; 9,4; -6,4; 6,8; 9,4; 2,9; 9,4; -3,6; -8,5; 2,9; -6,4. Bul tańlanbanıń statistikalıq bólistiriliwin tabıń.
 \\
\textbf{A2.} 
Kólemi \(n = 20\) ǵa teń bolǵan tańlanba berilgen:9,1; 6,4; -8,6; 9,1; 2,3; 9,1; -8,6; -6,2; 6,4; -8,6; 9,1; -6,2; 6,4; 9,1; 2,3; 9,1; -3,9; -8,6; 2,3; -6,2. Bul tańlanbanıń empirikalıq bólistiriw funkciyasın tabıń.
 \\
\textbf{A3.} 
Joqarı matematika páninen 10 dana student test sınaqların tapsırǵan. Hárbir student 10 balǵa shekem toplawı múmkin. Eger test sınaqları nátiyjeleri boyınsha \{3, 7, 6, 4, 5, 4, 3, 7, 8, 3\} tańlanba alınǵan bolsa, onda tańlanba ortasha hám tańlanba dispersiyalardı tabıń.
 \\
\textbf{B1.} 
Eger ortasha kvadratlıq shetleniwi \(\sigma = 5\) bolǵan normal bólistirilgen bas toplamnan alınǵan kólemi \(n = 16\) ǵa teń tańlanba boyınsha \(\overline{x} = 3,6\) tańlanba ortasha mánisi tabılǵan bolsa, onda \(\gamma = 0,90\) isenimlilik penen belgisiz \(\theta\) matematikalıq kútiliwdi qaplaytuǵın isenimlilik intervalın dúziń.
 \\
\textbf{B2.} 
Eger \(X^{(n)} = \left( X_{1},...,X_{n} \right)\) tańlanba \(\theta\) parametrli kórsetkishli bólistiriliwden alınǵan bolsa, onda belgisiz \(\theta\) parametr ushın momentler usılı bahasın tabıń.
 \\
\textbf{B3.} 
Eger \(x_{1} = 1,1;\ x_{2} = 2,7;\ldots;x_{100} = 1,5\) tańlanba \(\theta\) parametrli kórsetkishli bólistiriliwden alınǵan bolıp, \(\sum_{k = 1}^{100}x_{k} = 200\) bolsa, onda belgisiz \(\theta\) parametrdiń shınlıqqa maksimal uqsaslıq bahasın tabıń.
 \\
\textbf{C1.} 
Eger \(X^{(n)} = \left( X_{1},...,X_{n} \right)\) tańlanba \(\frac{1}{\sqrt{\theta}}\) parametrli kórsetkishli bólistiriliwinen alınǵan bolsa, onda belgisiz \(\theta\) parametr ushın \((\overline{x})^{2}\) bahasın jıljımaǵanlıq hám tiykarlılıqqa tekseriń.
 \\
\textbf{C2.} 
Eger \(X^{(n)} = \left( X_{1},...,X_{n} \right)\) tańlanba tıǵızlıq funkciyası
$f(x,\theta) = \frac{2x}{\theta^{2}},x \in \lbrack 0,\theta\rbrack$
bolǵan bólistiriliwden alınǵan bolsa, onda belgisiz \(\theta\) parametr ushın momentler usılı bahasın tabıń.
 \\
\textbf{C3.} 
Eger \(X^{(n)} = \left( X_{1},...,X_{n} \right)\) tańlanba tıǵızlıq funkciyası
$f(x;\theta) = \frac{\theta}{\sqrt{2\pi x^{3}}}e^{\frac{- \ \theta^{2}}{2x}},\ x \geq 0$
bolǵan bólistiriliwden alınǵan bolsa, onda belgisiz \(\theta > 0\) parametrdiń shınlıqqa maksimal uqsaslıq bahasın tabıń.
 \\

\end{tabular}
\vspace{1cm}


\begin{tabular}{m{17cm}}
\textbf{14-variant}
\newline

\textbf{T1.} 
Gruppalanǵan hám intervallıq variaciyalıq qatarlar.
 \\
\textbf{T2.} 
Statistikalıq gipotezalardı tekseriw (kritikalıq kóplik, 1 hám 2-túr qátelik)
 \\
\textbf{A1.} 
Kólemi \(n = 20\) ǵa teń bolǵan tańlanba berilgen: 6,2; -5,3; 7,2; 3,7; -2,2; 6,2; 3,7; -7,6; 3,7; 7,2; 6,2; -5,3; -7,6; -5,3; -7,6; 6,2; 7,2; -2,2; -7,6; 7,2. Bul tańlanbanıń statistikalıq bólistiriliwin tabıń.
 \\
\textbf{A2.} 
Kólemi \(n = 20\) ǵa teń bolǵan tańlanba berilgen: 6,1; -5,8; 7,9; 3,5; -2,5; 6,1; 3,5; -7,2; 3,5; 7,9; 6,1; -5,8; -7,2; -5,8; -7,2; 6,1; 7,9; -2,5; -7,2; 7,9. Bul tańlanbanıń empirikalıq bólistiriw funkciyasın tabıń.
 \\
\textbf{A3.} 
Joqarı matematika páninen 10 dana student test sınaqların tapsırǵan. Hárbir student 10 balǵa shekem toplawı múmkin. Eger test sınaqları nátiyjeleri boyınsha \{10, 8, 6, 5, 4, 8, 10, 7, 5, 7\} tańlanba alınǵan bolsa, onda tańlanba ortasha hám tańlanba dispersiyalardı tabıń.
 \\
\textbf{B1.} 
Eger normal bólistirilgen bas toplamnan alınǵan kólemi \(n = 16\) ǵa teń tańlanba boyınsha \(\overline{x} = 15,2\) tańlanba ortasha hám \({\overline{S}}^{2} = 0,81\) dúzetilgen tańlanba dispersiyalar tabılǵan bolsa, onda \(\gamma = 0,90\) isenimlilik penen belgisiz \(\theta\) matematikalıq kútiliw ushın isenimlilik interval dúziń.
 \\
\textbf{B2.} 
Eger (0,-2,0,-2,3,-2,0,0,3,0,0,0,3,-2,0,0,-2,3,0,3) tańlanba tómende berilgen bólistiriliwden alınǵan bolsa, onda belgisiz \(\theta\) parametr ushın momentler usılı bahasın tabıń.
\begin{tabular}{|c|c|c|c|}
  \hline
$\xi$ & $- 2$  & $0$  & $3$ \\
\hline
\(P_{\theta}\) & \(\theta\) & \(1 - 2\theta\) & \(\theta\) \\
\hline
\end{tabular}
 \\
\textbf{B3.} 
Eger \(X^{(n)} = \left( X_{1},...,X_{n} \right)\) tańlanba \(\left\lbrack - \theta,\theta^{2} \right\rbrack\) aralıqta teń ólshemli bólistiriliwden alınǵan bolsa, onda belgisiz \(\theta > 0\) parametrdiń shınlıqqa maksimal uqsaslıq usılı bahasın tabıń.
 \\
\textbf{C1.} 
Eger \(X^{(n)} = \left( X_{1},...,X_{n} \right)\) tańlanba \(\sqrt{\theta}\) parametrli Bernulli bólistiriliwinen alınǵan bolsa, onda belgisiz \(\theta\) parametr ushın \((\overline{x})^{2}\) bahasın jıljımaǵanlıq hám tiykarlılıqqa tekseriń.
 \\
\textbf{C2.} 
Eger \(X^{(n)} = \left( X_{1},...,X_{n} \right)\) tańlanba\({\ \ (a,\theta}^{2})\) parametrli normal bólistiriliwden alınǵan bolsa (\(\alpha -\)belgili), onda belgisiz\({\ \ \theta}^{2}\) parametr ushın momentler usılı bahasın tabıń.
 \\
\textbf{C3.} 
\(\ f(x;\theta) = \frac{7x^{6}}{\sqrt{2\pi}}exp\{ - \frac{(x^{7} - \theta)^{2}}{2}\}\) model ushın \(\theta\) parametri haqıyqatqa maksimal uqsaslıq usılı bahası tabılsın.
 \\

\end{tabular}
\vspace{1cm}


\begin{tabular}{m{17cm}}
\textbf{15-variant}
\newline

\textbf{T1.} 
Tańlanba momentleri (\(k -\)tártipli baslanǵısh, baslanǵısh absolyut, oraylıq hám oraylıq absolyut momentler).
 \\
\textbf{T2.} 
Haqiyqatqa maksimal uqsaslıq usulı. (haqiyqatqa maksimal uqsaslıq funkciyası, belgisiz parametrlerdi bahalaw).
 \\
\textbf{A1.} 
Kólemi \(n = 20\) ǵa teń bolǵan tańlanba berilgen: 9,6; 1,5; 7,4; 9,6; 2,8; 1,5; 6,3; 1,5; 9,6; 6,3; 2,8; 4,1; 6,3; 9,6; 1,5; 1,5; 6,3; 7,4; 4,1; 7,4. Bul tańlanbanıń statistikalıq bólistiriliwin tabıń.
 \\
\textbf{A2.} 
Kólemi \(n = 20\) ǵa teń bolǵan tańlanba berilgen: 9,8; 1,2; 7,1; 9,8; 2,9; 1,2; 6,7; 1,2; 9,8; 6,7; 2,9; 4,6; 6,7; 9,8; 1,2; 1,2; 6,7; 7,1; 4,6; 7,1. Bul tańlanbanıń empirikalıq bólistiriw funkciyasın tabıń.
 \\
\textbf{A3.} 
Joqarı matematika páninen 10 dana student test sınaqların tapsırǵan. Hárbir student 10 balǵa shekem toplawı múmkin. Eger test sınaqları nátiyjeleri boyınsha \{9, 10, 5, 6, 4, 8, 4, 6, 10, 8\} tańlanba alınǵan bolsa, onda tańlanba ortasha hám tańlanba dispersiyalardı tabıń.
 \\
\textbf{B1.} 
Eger normal bólistirilgen bas toplamnan alınǵan kólemi \(n = 10\) ǵa teń bolǵan tańlanba boyınsha \({\overline{S}}^{2} = 0,45\) dúzetilgen tańlanba dispersiya tabılǵan bolsa, onda \(\gamma = 0,95\) isenimlilik penen belgisiz \(\theta_{2}^{2}\) dispersiya ushın isenimlilik interval dúziń.
 \\
\textbf{B2.} 
Eger tıǵızlıq funkciyası \(f(x) = \frac{2x}{\theta}e^{- \frac{x^{2}}{\theta}},\ \ x \geq 0\) kóriniske iye bolsa, onda \(\theta\) parametr momentler usulı bahasın tabıń.
 \\
\textbf{B3.} 
Eger \(X^{(n)} = \left( X_{1},...,X_{n} \right)\) tańlanba tıǵızlıq funkciyası \(f(x;\theta) = \frac{2x}{\theta}e^{- \frac{x^{2}}{\theta}},\ x \geq 0\). bolǵan bólistiriliwden alınǵan bolsa, onda belgisiz \(\theta > 0\) parametrdiń shınlıqqa maksimal uqsaslıq bahasın tabıń.
 \\
\textbf{C1.} 
Eger \(X^{(n)} = \left( X_{1},...,X_{n} \right)\) tańlanba \(\theta\) parametrli Bernulli bólistiriliwinen alınǵan bolsa, onda belgisiz \(\theta\) parametr ushın \(X_{n}\) bahasın jıljımaǵanlıq hám tiykarlılıqqa tekseriń.
 \\
\textbf{C2.} 
Eger \(X^{(n)} = \left( X_{1},...,X_{n} \right)\) tańlanba \({\ \ (a,\theta}^{2})\ \)parametrli normal bólistiriliwden alınǵan bolsa (\(\alpha -\)belgili), onda belgisiz \({\ \theta}^{2}\) parametr ushın momentler usılı bahasın \({\ g(x) = (x - a)}^{2}\) funkciyası járdeminde tabıń.
 \\
\textbf{C3.} 
\(\ f(x,\theta) = \frac{e^{x}}{\sqrt{2\pi}}\exp\left\{ - \frac{\left( e^{x} - \theta \right)^{2}}{2} \right\}\) model ushın \(\theta\) parametri haqıyqatqa maksimal uqsaslıq usılı bahası tabılsın.
 \\

\end{tabular}
\vspace{1cm}


\begin{tabular}{m{17cm}}
\textbf{16-variant}
\newline

\textbf{T1.} 
Tańlanba xarakteristikaları.(tańlanba orta, tańlanba dispersiya)
 \\
\textbf{T2.} 
Kolmogorovtıń kelisimlilik belgisi (Kolmogorov teoreması)
 \\
\textbf{A1.} 
Kólemi \(n = 20\) ǵa teń bolǵan tańlanba berilgen:1,8; -8,4; 7,3; 4,7; -3,9; 1,8; 4,7; -10,4; -8,4; 7,3; -10,4; 4,7; -8,4; 1,8; 4,7; -10,4; 7,3; -3,9; 4,7; -8,4. Bul tańlanbanıń statistikalıq bólistiriliwin tabıń.
 \\
\textbf{A2.} 
Kólemi \(n = 20\) ǵa teń bolǵan tańlanba berilgen:1,6; -8,3; 7,6; 4,2; -3,1; 1,6; 4,2; -10,5; -8,3; 7,6; -10,5; 4,2; -8,3; 1,6; 4,2; -10,5; 7,6; -3,1; 4,2; -8,3. Bul tańlanbanıń empirikalıq bólistiriw funkciyasın tabıń.
 \\
\textbf{A3.} 
Joqarı matematika páninen 10 dana student test sınaqların tapsırǵan. Hárbir student 10 balǵa shekem toplawı múmkin. Eger test sınaqları nátiyjeleri boyınsha \{9, 3, 6, 3, 7, 6, 4, 6, 10, 6\} tańlanba alınǵan bolsa, onda tańlanba ortasha hám tańlanba dispersiyalardı tabıń.
 \\
\textbf{B1.} 
Eger ortasha kvadratlıq shetleniwi \(\sigma = 2\) bolǵan normal bólistirilgen bas toplamnan alınǵan kólemi \(n = 18\) ǵa teń tańlanba boyınsha \(\overline{x} = 5,2\) tańlanba ortasha mánisi tabılǵan bolsa, onda \(\gamma = 0,90\) isenimlilik penen belgisiz \(\theta\) matematikalıq kútiliwdi qaplaytuǵın isenimlilik intervalın dúziń.
 \\
\textbf{B2.} 
\(\lbrack\theta_{1},\theta_{2}\rbrack\) aralıqta teń ólshewli bólistiriw parametrleri ushın momentler usulı bahaların tabıń.
 \\
\textbf{B3.} 
Eger \(X^{(n)} = \left( X_{1},...,X_{n} \right)\) tańlanba \(\theta\) parametrli Bernulli bólistiriliwinen alınǵan bolsa, onda belgisiz \(\theta\) parametrdiń shınlıqqa maksimal uqsaslıq usılı bahasın tabıń.
 \\
\textbf{C1.} 
Eger \(X^{(n)} = \left( X_{1},...,X_{n} \right)\) tańlanba \(\theta\) parametrli Bernulli bólistiriliwinen alınǵan bolsa, onda belgisiz \(\theta(1 - \theta)\) parametr ushın \(X_{1}\left( 1 - X_{n} \right)\) bahasın jıljımaǵanlıq hám tiykarlılıqqa tekseriń.
 \\
\textbf{C2.} 
Eger \(X^{(n)} = \left( X_{1},...,X_{n} \right)\) tańlanba tıǵızlıq funkciyası
$
{f(x,\theta) = \left\{ \begin{array}{r}
e^{\theta - x},\ \ \ x \geq \theta, \\
0,\ \ \ x < \theta
\end{array} \right.\ }$
bolǵan bólistiriliwden alınǵan bolsa, onda belgisiz \(\theta\) parametr ushın momentler usılı bahasın tabıń.
 \\
\textbf{C3.} 
Eger \(X^{(n)} = \left( X_{1},...,X_{n} \right)\) tańlanba tıǵızlıq funkciyası
$f(x;\theta) = \frac{\theta ln^{\theta - 1}x}{x},\ x \in \lbrack 1,e\rbrack$
bolǵan bólistiriliwden alınǵan bolsa, onda belgisiz \(\theta > 0\) parametr ushın shınlıqqa maksimal uqsaslıq bahasın tabıń.
 \\

\end{tabular}
\vspace{1cm}


\begin{tabular}{m{17cm}}
\textbf{17-variant}
\newline

\textbf{T1.} 
Tańlanba xarakteristikalar. (Variaciyalıq qatar, salıstırmalı jiyilik).
 \\
\textbf{T2.} 
Statistikalıq gipotezalardı tekseriw (kritikalıq kóplik, 1 hám 2-túr qátelik).
 \\
\textbf{A1.} 
Kólemi \(n = 20\) ǵa teń bolǵan tańlanba berilgen: 2,7; -13,5; 1,2; 2,7; 1,2; 4,9; -9,5; 1,2; 2,7; 4,9; -9,5; 2,7; -3,5; 1,2; 2,7; 4,9; -3,5; 2,7; 4,9; 1,2;. Bul tańlanbanıń statistikalıq bólistiriliwin tabıń.
 \\
\textbf{A2.} 
Kólemi \(n = 20\) ǵa teń bolǵan tańlanba berilgen: 2,8; -13,9; 1,9; 2,8; 1,9; 4,3; -9,4; 1,9; 2,8; 4,3; -9,4; 2,8; -3,7; 1,9; 2,8; 4,3; -3,7; 2,8; 4,3; 1,9. Bul tańlanbanıń empirikalıq bólistiriw funkciyasın tabıń.
 \\
\textbf{A3.} 
Joqarı matematika páninen 10 dana student test sınaqların tapsırǵan. Hárbir student 10 balǵa shekem toplawı múmkin. Eger test sınaqları nátiyjeleri boyınsha \{10, 7, 5, 9, 3, 8, 10, 7, 8, 3\} tańlanba alınǵan bolsa, onda tańlanba ortasha hám tańlanba dispersiyalardı tabıń.
 \\
\textbf{B1.} 
Eger normal bólistirilgen bas toplamnan alınǵan kólemi \(n = 36\) ǵa teń tańlanba boyınsha \(\overline{x} = 20,2\) tańlanba ortasha hám \({\overline{S}}^{2} = 0,81\) dúzetilgen tańlanba dispersiyalar tabılǵan bolsa, onda \(\gamma = 0,95\) isenimlilik penen belgisiz \(\theta\) matematikalıq kútiliw ushın isenimlilik interval dúziń.
 \\
\textbf{B2.} 
\(\lbrack 0,\theta\rbrack\) aralıqta teń ólshewli bólistirilgen \(\theta\) parametri ushın momentler usulı bahasın tabıń.
 \\
\textbf{B3.} 
Eger (0,1,2,0) tańlanba tómende berilgen bólistiriliwden alınǵan bolsa, onda belgisiz \(\theta\) parametrdiń shınlıqqa maksimal uqsaslıq bahasın tabıń.
\begin{tabular}{|c|c|c|c|}
  \hline
$\xi$
&
$0$
&
$1$
&
$2$\\
\hline
\(P_{\theta}\) & \(\theta\) & \(2\theta\) & \(1 - 3\theta\) \\
\hline
\end{tabular}
 \\
\textbf{C1.} 
Eger \(X^{(n)} = \left( X_{1},...,X_{n} \right)\) tańlanba \(\theta\) parametrli Bernulli bólistiriliwinen alınǵan bolsa, onda belgisiz \(\theta^{2}\) parametr ushın \(X_{1}X_{n}\) bahasın jıljımaǵanlıq hám tiykarlılıqqa tekseriń.
 \\
\textbf{C2.} 
Eger \(X^{(n)} = \left( X_{1},...,X_{n} \right)\) tańlanba\(\ (\theta,2\theta)\ \) parametrli normal bólistiriliwden alınǵan bolsa, onda belgisiz \(\theta > 0\) parametr ushın momentler usılı bahasın tabıń.
 \\
\textbf{C3.} 
Eger \(X^{(n)} = \left( X_{1},...,X_{n} \right)\) tańlanba \(\left( \theta,\theta^{2} \right)\) parametrli normal bólistiriliwden alınǵan bolsa, onda belgisiz \(\theta > 0\) parametrdiń shınlıqqa maksimal uqsaslıq bahasın tabıń.
 \\

\end{tabular}
\vspace{1cm}


\begin{tabular}{m{17cm}}
\textbf{18-variant}
\newline

\textbf{T1.} 
Neyman-Pirson teoreması
 \\
\textbf{T2.} 
Pirsonnıń xi-kvadrat kelisimlilik belgisi (Pirson teoreması).
 \\
\textbf{A1.} 
Kólemi \(n = 20\) ǵa teń bolǵan tańlanba berilgen: 9,9; 5,7; 3,2; 2,8; 5,7; 9,9; 7,5; 3,7; 9,9; 3,2; 2,8; 3,7; 7,5; 5,7; 3,2; 2,8; 7,5; 3,2; 9,9; 7,5. Bul tańlanbanıń statistikalıq bólistiriliwin tabıń.
 \\
\textbf{A2.} 
Kólemi \(n = 20\) ǵa teń bolǵan tańlanba berilgen: 9,7; 5,2; 3,2; 2,4; 5,2; 9,7; 7,5; 3,7; 9,7; 3,2; 2,4; 3,7; 7,5; 5,2; 3,2; 2,4; 7,5; 3,2; 9,7; 7,5. Bul tańlanbanıń empirikalıq bólistiriw funkciyasın tabıń.
 \\
\textbf{A3.} 
Joqarı matematika páninen 10 dana student test sınaqların tapsırǵan. Hárbir student 10 balǵa shekem toplawı múmkin. Eger test sınaqları nátiyjeleri boyınsha \{1, 6, 2, 6, 3, 6, 4, 6, 10, 6\} tańlanba alınǵan bolsa, onda tańlanba ortasha hám tańlanba dispersiyalardı tabıń.
 \\
\textbf{B1.} 
Eger normal bólistirilgen bas toplamnan alınǵan kólemi \(n = 10\) ǵa teń bolǵan tańlanba boyınsha \({\overline{S}}^{2} = 0,7\) dúzetilgen tańlanba dispersiya tabılǵan bolsa, onda \(\gamma = 0,95\) isenimlilik penen belgisiz \(\theta_{2}^{2}\) dispersiya ushın isenimlilik interval dúziń.
 \\
\textbf{B2.} 
Kórsetkishli bólistiriw belgisiz \(\theta > 0\) parametri momentlar usulı bahasın tabıń.
 \\
\textbf{B3.} 
Eger \(X^{(n)} = \left( X_{1},...,X_{n} \right)\) tańlanba \(\left( a,\theta^{2} \right)\) parametrli normal bólistiriliwden alınǵan bolsa (\(\alpha -\)belgili), onda belgisiz \(\theta^{2}\) parametrdiń shınlıqqa maksimal uqsaslıq bahasın tabıń.
 \\
\textbf{C1.} 
Eger \(X^{(n)} = \left( X_{1},...,X_{n} \right)\) tańlanba \((\alpha,\theta)\) parametrli Veybull bólistiriliwinen alınǵan bolsa (\(\alpha -\)belgili), onda belgisiz \(\theta\) parametr ushın \(\frac{1}{\overline{x^{\alpha}}}\) bahasın jıljımaǵanlıq hám tiykarlılıqqa tekseriń.
 \\
\textbf{C2.} 
Eger \(X^{(n)} = \left( X_{1},...,X_{n} \right)\) tańlanba tıǵızlıq funkciyası
${f(x,\theta) = \theta x}^{\theta - 1},x \in \lbrack 0,1\rbrack$
bolǵan bólistiriliwden alınǵan bolsa, onda belgisiz \(\theta\) parametr ushın momentler usılı bahasın tabıń.
 \\
\textbf{C3.} 
Eger \(X^{(n)} = \left( X_{1},...,X_{n} \right)\) tańlanba tıǵızlıq funkciyası
$f(x;\theta) = \frac{1}{2}e^{- \ |x - \theta|},\ x \in R$
bolǵan Laplas bólistiriliwinen alınǵan bolsa, onda belgisiz \(\theta \in R\) parametrdiń shınlıqqa maksimal uqsaslıq bahasın tabıń.
 \\

\end{tabular}
\vspace{1cm}


\begin{tabular}{m{17cm}}
\textbf{19-variant}
\newline

\textbf{T1.} Matematikalıq statistikanıń tiykarǵı máseleleri. (Statistikalıq maǵlıwmatlar, gruppalaw)
 \\
\textbf{T2.} 
Isenimlilik intervalların qurıw. Anıq isenimli intervallar
 \\
\textbf{A1.} 
Kólemi \(n = 20\) ǵa teń bolǵan tańlanba berilgen: 3,6; 1,1; -1,8; 0,4; 3,6; 0; 5,3; 1,1; 0; -1,8; 3,6; 0,4; 1,1; 0; 0,4; 1,1; 3,6; -1,8; 3,6; 0. Bul tańlanbanıń statistikalıq bólistiriliwin tabıń.
 \\
\textbf{A2.} 
Kólemi \(n = 20\) ǵa teń bolǵan tańlanba berilgen: 3,2; 1,8; -1,1; 0,9; 3,2; 0; 5,6; 1,8; 0; -1,1; 3,2; 0,9; 1,8; 0; 0,9; 1,8; 3,2; -1,1; 3,2; 0. Bul tańlanbanıń empirikalıq bólistiriw funkciyasın tabıń.
 \\
\textbf{A3.} 
Joqarı matematika páninen 10 dana student test sınaqların tapsırǵan. Hárbir student 10 balǵa shekem toplawı múmkin. Eger test sınaqları nátiyjeleri boyınsha \{2, 7, 3, 7, 6, 7, 4, 7, 7, 10\} tańlanba alınǵan bolsa, onda tańlanba ortasha hám tańlanba dispersiyalardı tabıń.
 \\
\textbf{B1.} 
Eger ortasha kvadratlıq shetleniwi \(\sigma = 3\) bolǵan normal bólistirilgen bas toplamnan alınǵan kólemi \(n = 14\) ǵa teń tańlanba boyınsha \(\overline{x} = 5,5\) tańlanba ortasha mánisi tabılǵan bolsa, onda \(\gamma = 0,90\) isenimlilik penen belgisiz \(\theta\) matematikalıq kútiliwdi qaplaytuǵın isenimlilik intervalın dúziń.
 \\
\textbf{B2.} 
Eger (3,-2,-2,0,-2,-2,-2,0,-2,3,-2,0,3,0,3,-2,0,-2,3,-2,-2,-2,-2,3,3,3,-2,-2,3,3) tańlanba tómende berilgen bólistiriliwden alınǵan bolsa, onda belgisiz \(\theta\) parametr ushın momentler usılı bahasın \(g(x) = |x|\) funkciyası járdeminde tabıń.
\begin{tabular}{|c|c|c|c|}
  \hline
$\xi$ &
$- 2$ &
$0$ &
$3$ \\
\hline
\(P_{\theta}\) & \(3\theta\) & \(1 - 5\theta\) & \(2\theta\) \\
\hline
\end{tabular}
 \\
\textbf{B3.} 
\(\ f(x) = \frac{2x}{\theta}e^{- \frac{x^{2}}{\theta}},\ \ x \geq 0\) model ushın \(\theta\) parametri haqıyqatqa maksimal uqsaslıq usılı bahası tabılsın.
 \\
\textbf{C1.} 
Eger \(X^{(n)} = \left( X_{1},...,X_{n} \right)\) tańlanba \(\theta\) parametrli geometriyalıq bólistiriliwden alınǵan bolsa, onda belgisiz \(\theta\) parametr ushın \(\frac{1}{(1 + \overline{x})}\) bahasın jıljımaǵanlıq hám tiykarlılıqqa tekseriń.
 \\
\textbf{C2.} 
Eger \(X^{(n)} = \left( X_{1},...,X_{n} \right)\) tańlanba\(\ (\theta,2\theta)\ \) parametrli normal bólistiriliwden alınǵan bolsa, onda belgisiz \(\theta > 0\) parametr ushın momentler usılı bahasın tabıń.
 \\
\textbf{C3.} 
\(\ f(x;\theta) = \frac{7x^{6}}{\sqrt{2\pi}}exp\{ - \frac{(x^{7} - \theta)^{2}}{2}\}\) model ushın \(\theta\) parametri haqıyqatqa maksimal uqsaslıq usılı bahası tabılsın.
 \\

\end{tabular}
\vspace{1cm}


\begin{tabular}{m{17cm}}
\textbf{20-variant}
\newline

\textbf{T1.} 
Empirikalıq bólistiriw funkciyası. (Tańlanba, eksperiment)
 \\
\textbf{T2.} 
Sızıqlı korrelyaciya teńlemesi (anıqlaması, regressiya tuwrı sızıǵınıń tańlanba teńlemeleri)
 \\
\textbf{A1.} 
Kólemi \(n = 20\) ǵa teń bolǵan tańlanba berilgen: 7,1; 3,9; 6,3; 4,6; 7,1; 2,3; 6,3; 3,9; 4,6; 7,1; 2,3; 3,9; 7,6; 2,3; 4,6; 3,9; 2,3; 3,9; 7,6; 4,6. Bul tańlanbanıń statistikalıq bólistiriliwin tabıń.
 \\
\textbf{A2.} 
Kólemi \(n = 20\) ǵa teń bolǵan tańlanba berilgen: 7,9; 3,8; 6,1; 4,2; 7,9; 2,4; 6,1; 3,8; 4,2; 7,9; 2,4; 3,8; 10,2; 2,4; 4,2; 3,8; 2,4; 3,8; 10,2; 4,2. Bul tańlanbanıń empirikalıq bólistiriw funkciyasın tabıń.
 \\
\textbf{A3.} 
Joqarı matematika páninen 10 dana student test sınaqların tapsırǵan. Hárbir student 10 balǵa shekem toplawı múmkin. Eger test sınaqları nátiyjeleri boyınsha \{9, 8, 6, 8, 6, 4, 5, 4, 7, 4\} tańlanba alınǵan bolsa, onda tańlanba ortasha hám tańlanba dispersiyalardı tabıń.
 \\
\textbf{B1.} 
Eger normal bólistirilgen bas toplamnan alınǵan kólemi \(n = 49\) ǵa teń tańlanba boyınsha \(\overline{x} = 14,2\) tańlanba ortasha hám \({\overline{S}}^{2} = 0,64\) dúzetilgen tańlanba dispersiyalar tabılǵan bolsa, onda \(\gamma = 0,95\) isenimlilik penen belgisiz \(\theta\) matematikalıq kútiliw ushın isenimlilik interval dúziń.
 \\
\textbf{B2.} 
Puasson bólistiriliwi belgisiz \(\theta > 0\) parametri momentlar usuli bahasin tabıń.
 \\
\textbf{B3.} 
Eger (4,8,5,3) tańlanba \(\left( a,\theta^{2} \right)\) parametrli normal bólistiriliwden alınǵan bolsa, onda belgisiz \(\theta^{2}\) parametrdiń shınlıqqa maksimal uqsaslıq bahasın tabıń.
 \\
\textbf{C1.} 
Eger \(X^{(n)} = \left( X_{1},...,X_{n} \right)\) tańlanba \(\theta\) parametrli Puasson bólistiriliwinen alınǵan bolsa, onda belgisiz \(\theta\) parametr ushın \(\frac{n + 3}{n + 4}\overline{x}\) bahasın jıljımaǵanlıq hám tiykarlılıqqa tekseriń.
 \\
\textbf{C2.} 
Eger \(X^{(n)} = \left( X_{1},...,X_{n} \right)\) tańlanba \({\lbrack\theta}_{1},\theta_{1}{+ \theta}_{2}\rbrack\) aralıqta teń ólshemli bólistiriliwden alınǵan bolsa, onda belgisiz \(\left( \theta_{1},\theta_{2} \right)\) vektor parametr ushın momentler usılı bahasın tabıń.
 \\
\textbf{C3.} 
\(\ f(x,\theta) = \frac{e^{x}}{\sqrt{2\pi}}\exp\left\{ - \frac{\left( e^{x} - \theta \right)^{2}}{2} \right\}\) model ushın \(\theta\) parametri haqıyqatqa maksimal uqsaslıq usılı bahası tabılsın.
 \\

\end{tabular}
\vspace{1cm}


\begin{tabular}{m{17cm}}
\textbf{21-variant}
\newline

\textbf{T1.} 
Glivenko-Kantelli teoreması. (empirikalıq bólistiriw funkciyası, 1itimallıq penen jaqınlasıw)
 \\
\textbf{T2.} 
Pirsonnıń xi-kvadrat kelisimlilik belgisi (Pirson teoreması).
 \\
\textbf{A1.} 
Kólemi \(n = 20\) ǵa teń bolǵan tańlanba berilgen: 0,6; -3,8; -2,3; -4,3; 2,8; 4,7; -2,3; 0,6; -3,8; 2,8; -2,3; -4,3; 0,6; -2,3; 2,8; -3,8; -4,3; -2,3; 2,8; -3,8. Bul tańlanbanıń statistikalıq bólistiriliwin tabıń.
 \\
\textbf{A2.} 
Kólemi \(n = 20\) ǵa teń bolǵan tańlanba berilgen: 0,7; -3,1; -2,3; -4,8; 2,6; 4,9; -2,3; 0,7; -3,1; 2,6; -2,3; -4,8; 0,7; -2,3; 2,6; -3,1; -4,8; -2,3; 2,6; -3,1. Bul tańlanbanıń empirikalıq bólistiriw funkciyasın tabıń.
 \\
\textbf{A3.} 
Joqarı matematika páninen 10 dana student test sınaqların tapsırǵan. Hárbir student 10 balǵa shekem toplawı múmkin. Eger test sınaqları nátiyjeleri boyınsha \{10, 4, 6, 5, 5, 4, 10, 7, 9, 10\} tańlanba alınǵan bolsa, onda tańlanba ortasha hám tańlanba dispersiyalardı tabıń.
 \\
\textbf{B1.} 
Eger normal bólistirilgen bas toplamnan alınǵan kólemi \(n = 8\) ǵa teń bolǵan tańlanba boyınsha \({\overline{S}}^{2} = 0,35\) dúzetilgen tańlanba dispersiya tabılǵan bolsa, onda \(\gamma = 0,90\) isenimlilik penen belgisiz \(\theta_{2}^{2}\) dispersiya ushın isenimlilik interval dúziń.
 \\
\textbf{B2.} 
Eger (-2,0,-2,0,-2,3,-2,0,0,3,0,0,0,3,-2,0,0,-2,3,0) tańlanba tómende berilgen bólistiriliwden alınǵan bolsa, onda belgisiz \(\left( \theta_{1},\theta_{2} \right)\) vektor parametr ushın momentler usılı bahalasın tabıń.
\begin{tabular}{|c|c|c|c|}
  \hline
$\xi$ &
$- 2$ &
$0$ &
$3$\\
\hline
\(P_{\theta}\) & \(\theta_{1}\) & \(1 - \theta_{1} - \theta_{2}\) & \(\theta_{2}\) \\
\hline
\end{tabular}
 \\
\textbf{B3.} 
Eger (-1,-1,0,-1,0,-1,-1,5,-1,0,-1,0,5,-1,-1,-1,5,-1,-1,-1,5,0,-1,-1,5) tańlanba tómende berilgen bólistiriliwden alınǵan bolsa, onda belgisiz \(\theta\) parametrdiń shınlıqqa maksimal uqsaslıq usılı bahasın tabıń.
\begin{tabular}{|c|c|c|c|}
  \hline
$\xi$
&
$- 1$
&
$0$
&
$5$\\
\hline
\(P_{\theta}\) & \(1 - \theta\) & \(\theta/2\) & \(\theta/2\ \) \\
\hline
\end{tabular}
 \\
\textbf{C1.} 
Eger \(X^{(n)} = \left( X_{1},...,X_{n} \right)\) tańlanba \(\theta\) parametrli Puasson bólistiriliwinen alınǵan bolsa, onda belgisiz \(\theta\) parametr ushın \(\frac{X_{1} + X_{3}}{2}\) bahasın jıljımaǵanlıq hám tiykarlılıqqa tekseriń.
 \\
\textbf{C2.} 
Eger \(X^{(n)} = \left( X_{1},...,X_{n} \right)\) tańlanba \({\lbrack\theta}_{1},\theta_{2}\rbrack\) aralıqta teń ólshemli bólistiriliwden alınǵan bolsa, onda belgisiz \(\left( \theta_{1},\theta_{2} \right)\) vektor parametr ushın momentler usılı bahasın tabıń.
 \\
\textbf{C3.} 
Eger \(X^{(n)} = \left( X_{1},...,X_{n} \right)\) tańlanba \(\theta\) parametrli geometriyalıq bólistiriliwden alınǵan bolsa, onda belgisiz \(\theta\) parametrdiń shınlıqqa maksimal uqsaslıq usılı bahasın tabıń.
 \\

\end{tabular}
\vspace{1cm}


\begin{tabular}{m{17cm}}
\textbf{22-variant}
\newline

\textbf{T1.} 
Momentler usulı. (tańlanba momentleri, belgisiz parametrlerdi bahalaw).
 \\
\textbf{T2.} 
Haqiyqatqa maksimal uqsaslıq usulı. (haqiyqatqa maksimal uqsaslıq funkciyası, belgisiz parametrlerdi bahalaw).
 \\
\textbf{A1.} 
Kólemi \(n = 20\) ǵa teń bolǵan tańlanba berilgen: 8,9; 2,7; 1,7; 2,2; 5,6; 1,7; 5,6; 2,7; 1,7; 2,2; 5,6; 8,9; 1,7; 2,2; 1,7; 2,7; 1,7; 5,6; 6,1; 8,9. Bul tańlanbanıń statistikalıq bólistiriliwin tabıń.
 \\
\textbf{A2.} 
Kólemi \(n = 20\) ǵa teń bolǵan tańlanba berilgen: 8,7; 2,7; 1,5; 2,2; 5,7; 1,5; 5,7; 2,7; 1,5; 2,2; 5,7; 8,7; 1,5; 2,2; 1,5; 2,7; 1,5; 5,7; 6,3; 8,7. Bul tańlanbanıń empirikalıq bólistiriw funkciyasın tabıń.
 \\
\textbf{A3.} 
Joqarı matematika páninen 10 dana student test sınaqların tapsırǵan. Hárbir student 10 balǵa shekem toplawı múmkin. Eger test sınaqları nátiyjeleri boyınsha \{9, 8, 6, 9, 5, 4, 5, 7, 8, 9\} tańlanba alınǵan bolsa, onda tańlanba ortasha hám tańlanba dispersiyalardı tabıń.
 \\
\textbf{B1.} 
Eger ortasha kvadratlıq shetleniwi \(\sigma = 4\) bolǵan normal bólistirilgen bas toplamnan alınǵan kólemi \(n = 16\) ǵa teń tańlanba boyınsha \(\overline{x} = 5,8\) tańlanba ortasha mánisi tabılǵan bolsa, onda \(\gamma = 0,90\) isenimlilik penen belgisiz \(\theta\) matematikalıq kútiliwdi qaplaytuǵın isenimlilik intervalın dúziń.
 \\
\textbf{B2.} 
Eger (3,0,-2,0,-2,3,-2,0,0,3,0,0,0,3,-2,0,0,-2,3,0) tańlanba tómende berilgen bólistiriliwden alınǵan bolsa, onda belgisiz \(\left( \theta_{1},\theta_{2} \right)\) vektor parametr ushın momentler usılı bahalasın tabıń.
\begin{tabular}{|c|c|c|c|}
  \hline
$\xi$ &
$- 2$ &
$0$ &
$3$\\
\hline
\(P_{\theta}\) & \({2\theta}_{1}\) & \(0,5 + \theta_{1} + \theta_{2}\) & \(\theta_{2}\) \\
\hline
\end{tabular}
 \\
\textbf{B3.} 
\(\ f(x) = \frac{\theta}{2}e^{- \theta|x|}\) model ushın \(\theta\) parametri haqıyqatqa maksimal uqsaslıq usılı bahası tabılsın.
 \\
\textbf{C1.} 
Eger \(X^{(n)} = \left( X_{1},...,X_{n} \right)\) tańlanba \(\ln\theta\) parametrli Puasson bólistiriliwinen alınǵan bolsa, onda belgisiz \(\theta\) parametr ushın \(e^{\overline{x}}\) bahasın jıljımaǵanlıq hám tiykarlılıqqa tekseriń.
 \\
\textbf{C2.} 
Eger \(X^{(n)} = \left( X_{1},...,X_{n} \right)\) tańlanba \(\frac{1}{\theta}\) parametrli kórsetkishli bólistiriliwden alınǵan bolsa, onda belgisiz \(\theta\) parametr ushın momentler usılı bahasın\({\ g(x) = x}^{k},\) \(k \in N\)funkciyası járdeminde tabıń.
 \\
\textbf{C3.} 
Eger \(X^{(n)} = \left( X_{1},...,X_{n} \right)\) tańlanba tıǵızlıq funkciyası
$f(x;\theta) = \frac{e^{x}}{\sqrt{2\pi}}\exp\left\{ - \frac{\left( e^{x} - \theta \right)^{2}}{2} \right\},\ x \in R$
bolǵan bólistiriliwden alınǵan bolsa, onda belgisiz \(\theta\) parametrdiń shınlıqqa maksimal uqsaslıq bahasın tabıń.
 \\

\end{tabular}
\vspace{1cm}


\begin{tabular}{m{17cm}}
\textbf{23-variant}
\newline

\textbf{T1.} 
Gruppalanǵan hám intervallıq variaciyalıq qatarlar.
 \\
\textbf{T2.} 
Statistikalıq gipotezalardı tekseriw (kritikalıq kóplik, 1 hám 2-túr qátelik)
 \\
\textbf{A1.} 
Kólemi \(n = 20\) ǵa teń bolǵan tańlanba berilgen: 1,8; -1,9; 2,4; 1,8; 2,4; 1,8; 2,4; -0,6; -1,9; 1,8; -0,6; 2,4; -3,3; -1,9; 4,0; -3,3; -3,3; -1,9; -3,3; -1,9. Bul tańlanbanıń statistikalıq bólistiriliwin tabıń.
 \\
\textbf{A2.} 
Kólemi \(n = 20\) ǵa teń bolǵan tańlanba berilgen: 1,4; -1,9; 2,5; 1,4; 2,5; 1,4; 2,5; -0,4; -1,9; 1,4; -0,4; 2,5; -3,7; -1,9; 4,5; -3,7; -3,7; -1,9; -3,7; -1,9. Bul tańlanbanıń empirikalıq bólistiriw funkciyasın tabıń.
 \\
\textbf{A3.} 
Joqarı matematika páninen 10 dana student test sınaqların tapsırǵan. Hárbir student 10 balǵa shekem toplawı múmkin. Eger test sınaqları nátiyjeleri boyınsha \{4, 3, 8, 4, 8, 3, 9, 4, 7, 10\} tańlanba alınǵan bolsa, onda tańlanba ortasha hám tańlanba dispersiyalardı tabıń.
 \\
\textbf{B1.} 
Eger normal bólistirilgen bas toplamnan alınǵan kólemi \(n = 36\) ǵa teń tańlanba boyınsha \(\overline{x} = 20,2\) tańlanba ortasha hám \({\overline{S}}^{2} = 0,64\) dúzetilgen tańlanba dispersiyalar tabılǵan bolsa, onda \(\gamma = 0,90\) isenimlilik penen belgisiz \(\theta\) matematikalıq kútiliw ushın isenimlilik interval dúziń.
 \\
\textbf{B2.} 
Eger (0,-2,0,-2,3,-2,0,0,3,0,0,0,3,-2,0,0,-2,3,0,3) tańlanba tómende berilgen bólistiriliwden alınǵan bolsa, onda belgisiz \(\theta\) parametr ushın momentler usılı bahasın tabıń.
\begin{tabular}{|c|c|c|c|}
  \hline
$\xi$ & $- 2$  & $0$  & $3$ \\
\hline
\(P_{\theta}\) & \(\theta\) & \(1 - 2\theta\) & \(\theta\) \\
\hline
\end{tabular}
 \\
\textbf{B3.} 
Eger \(X^{(n)} = \left( X_{1},...,X_{n} \right)\) tańlanba \(\theta\) parametrli kórsetkishli bólistiriliwden alınǵan bolsa, onda belgisiz \(\theta\) parametrdiń shınlıqqa maksimal uqsaslıq usılı bahasın tabıń.
 \\
\textbf{C1.} 
Eger \(X^{(n)} = \left( X_{1},...,X_{n} \right)\) tańlanba \((\alpha,\theta)\) parametrli Pareto bólistiriliwinen alınǵan bolsa (\(\alpha -\)belgili), onda belgisiz \(\theta\) parametr ushın \(X_{(1)}\) bahasın jıljımaǵanlıq hám tiykarlılıqqa tekseriń.
 \\
\textbf{C2.} 
Eger \(X^{(n)} = \left( X_{1},...,X_{n} \right)\) tańlanba {[}\(0,2\theta\rbrack\) aralıqta teń ólshemli bólistiriliwden alınǵan bolsa, onda belgisiz \(\theta > 0\) parametr ushın momentler usılı bahasın tabıń.
 \\
\textbf{C3.} 
Eger \(X^{(n)} = \left( X_{1},...,X_{n} \right)\) tańlanba tıǵızlıq funkciyası
$f(x;\theta) = \frac{\theta}{2}e^{- \theta|x|},\ x \in R$
bolǵan bólistiriliwden alınǵan bolsa, onda belgisiz \(\theta > 0\) parametrdiń shınlıqqa maksimal uqsaslıq bahasın tabıń.
 \\

\end{tabular}
\vspace{1cm}


\begin{tabular}{m{17cm}}
\textbf{24-variant}
\newline

\textbf{T1.} 
Neyman-Pirson teoreması
 \\
\textbf{T2.} 
Kolmogorovtıń kelisimlilik belgisi (Kolmogorov teoreması)
 \\
\textbf{A1.} 
Kólemi \(n = 20\) ǵa teń bolǵan tańlanba berilgen: 2,9; -3,2; 5,3; -4,3; 4,1; 5,3; -1,2; 2,9; -3,2; 4,1; -4,3; 5,3; -3,2; 2,9; -4,3; 4,1; -1,2; 5,3; 2,9; -3,2. Bul tańlanbanıń statistikalıq bólistiriliwin tabıń.
 \\
\textbf{A2.} 
Kólemi \(n = 20\) ǵa teń bolǵan tańlanba berilgen: 2,7; -5,6; 5,2; -8,1; 4,8; 5,2; -1,6; 2,7; -5,6; 4,8; -8,1; 5,2; -5,6; 2,7; -8,1; 4,8; -1,6; 5,2; 2,7; -5,6. Bul tańlanbanıń empirikalıq bólistiriw funkciyasın tabıń.
 \\
\textbf{A3.} 
Joqarı matematika páninen 10 dana student test sınaqların tapsırǵan. Hárbir student 10 balǵa shekem toplawı múmkin. Eger test sınaqları nátiyjeleri boyınsha \{7, 9, 4, 9, 7, 5, 4, 7, 2, 6\} tańlanba alınǵan bolsa, onda tańlanba ortasha hám tańlanba dispersiyalardı tabıń.
 \\
\textbf{B1.} 
Eger normal bólistirilgen bas toplamnan alınǵan kólemi \(n = 11\) ǵa teń bolǵan tańlanba boyınsha \({\overline{S}}^{2} = 0,3\) dúzetilgen tańlanba dispersiya tabılǵan bolsa, onda \(\gamma = 0,95\) isenimlilik penen belgisiz \(\theta_{2}^{2}\) dispersiya ushın isenimlilik interval dúziń.
 \\
\textbf{B2.} 
Eger (-2,0,-2,0,-2,3,-2,0,0,3,0,0,0,3,-2,0,0,-2,3,0) tańlanba tómende berilgen bólistiriliwden alınǵan bolsa, onda belgisiz \(\left( \theta_{1},\theta_{2} \right)\) vektor parametr ushın momentler usılı bahalasın tabıń.
\begin{tabular}{|c|c|c|c|}
  \hline
$\xi$ &
$- 2$ &
$0$ &
$3$\\
\hline
\(P_{\theta}\) & \(\theta_{1}\) & \(1 - \theta_{1} - \theta_{2}\) & \(\theta_{2}\) \\
\hline
\end{tabular}
 \\
\textbf{B3.} 
Eger \(X^{(n)} = \left( X_{1},...,X_{n} \right)\) tańlanba \(\lbrack - \theta,\theta\rbrack\) aralıqta teń ólshemli bólistiriliwden alınǵan bolsa, onda belgisiz \(\theta > 0\) parametrdiń shınlıqqa maksimal uqsaslıq usılı bahasın tabıń.
 \\
\textbf{C1.} 
Eger \(X^{(n)} = \left( X_{1},...,X_{n} \right)\) tańlanba tıǵızlıq funkciyası: \(f(x;\theta) = e^{- x + \theta}\left( 1 + e^{- x + \theta} \right)^{2},\ x \in R\)
bolǵan bólistiriliwden alınǵan bolsa, onda belgisiz \(\theta\) parametr ushın \(\overline{x}\) bahasın jıljımaǵanlıq hám tiykarlılıqqa tekseriń.
 \\
\textbf{C2.} 
Eger \(X^{(n)} = \left( X_{1},...,X_{n} \right)\) tańlanba \({(\theta,\theta}^{2})\ \) parametrli normal bólistiriliwden alınǵan bolsa, onda belgisiz \(\theta > 0\) parametr ushın momentler usılı bahasın tabıń.
 \\
\textbf{C3.} 
Eger \(X^{(n)} = \left( X_{1},...,X_{n} \right)\) tańlanba tıǵızlıq funkciyası
$f(x;\theta) = \left\{ \begin{matrix}
\theta_{1}^{- 1}e^{\frac{x - \theta_{2}}{\theta_{1}}},\ \ x \geq \theta_{2}, \\
\ \ \ \ \ \ \ \ \ \ \ \ 0,\ \ \ \ \ \ \ x < \theta_{2}
\end{matrix} \right.\ $
bolǵan bólistiriliwden alınǵan bolsa, onda belgisiz \(\left( \theta_{1},\theta_{2} \right),\) \(\theta_{1} > 0,\) \(\theta_{2} \in R\) vektor parametrdiń shınlıqqa maksimal uqsaslıq bahasın tabıń.
 \\

\end{tabular}
\vspace{1cm}


\begin{tabular}{m{17cm}}
\textbf{25-variant}
\newline

\textbf{T1.} 
Tańlanba xarakteristikalar. (Variaciyalıq qatar, salıstırmalı jiyilik).
 \\
\textbf{T2.} 
Statistikalıq gipotezalardı tekseriw (kritikalıq kóplik, 1 hám 2-túr qátelik).
 \\
\textbf{A1.} 
Kólemi \(n = 20\) ǵa teń bolǵan tańlanba berilgen: 14,7; 7,3; 16,6; 9,8; 11,2; 16,6; 6,7; 7,3; 11,2; 14,7; 6,7; 16,6; 7,3; 11,2; 14,7; 16,6; 6,7; 7,3; 11,2; 16,6. Bul tańlanbanıń statistikalıq bólistiriliwin tabıń.
 \\
\textbf{A2.} 
Kólemi \(n = 20\) ǵa teń bolǵan tańlanba berilgen: 14,4; 7,6; 16,7; 9,1; 11,8; 16,7; 6,4; 7,6; 11,8; 14,4; 6,4; 16,7; 7,6; 11,8; 14,4; 16,7; 6,4; 7,6; 11,8; 16,7. Bul tańlanbanıń empirikalıq bólistiriw funkciyasın tabıń.
 \\
\textbf{A3.} 
Joqarı matematika páninen 10 dana student test sınaqların tapsırǵan. Hárbir student 10 balǵa shekem toplawı múmkin. Eger test sınaqları nátiyjeleri boyınsha \{10, 8, 4, 6, 2, 8, 5, 10, 2, 5\} tańlanba alınǵan bolsa, onda tańlanba ortasha hám tańlanba dispersiyalardı tabıń.
 \\
\textbf{B1.} 
Eger ortasha kvadratlıq shetleniwi \(\sigma = 4\) bolǵan normal bólistirilgen bas toplamnan alınǵan kólemi \(n = 49\) ǵa teń tańlanba boyınsha \(\overline{x} = 9,4\) tańlanba ortasha mánisi tabılǵan bolsa, onda \(\gamma = 0,90\) isenimlilik penen belgisiz \(\theta\) matematikalıq kútiliwdi qaplaytuǵın isenimlilik intervalın dúziń.
 \\
\textbf{B2.} 
Puasson bólistiriliwi belgisiz \(\theta > 0\) parametri momentlar usuli bahasin tabıń.
 \\
\textbf{B3.} 
\(\ f(x) = \frac{2x}{\theta}e^{- \frac{x^{2}}{\theta}},\ \ x \geq 0\) model ushın \(\theta\) parametri haqıyqatqa maksimal uqsaslıq usılı bahası tabılsın.
 \\
\textbf{C1.} 
Eger \(X^{(n)} = \left( X_{1},...,X_{n} \right)\) tańlanba tıǵızlıq funkciyası
$f(x;\theta) = \left\{ \begin{array}{r}
\alpha^{- 1}e^{- \ \frac{x - \theta}{\alpha}},\ \ x \geq \theta, \\
0,\ \ \ \ \ \ \ x < \theta
\end{array} \right.\ $
bolǵan bólistiriliwden alınǵan bolsa (\(\alpha -\)belgili), onda belgisiz \(\theta\) parametr ushın \(X_{(1)}\) bahasın jıljımaǵanlıq hám tiykarlılıqqa tekseriń.
 \\
\textbf{C2.} 
Eger \(X^{(n)} = \left( X_{1},...,X_{n} \right)\) tańlanba \(\theta\) parametrli geometriyalıq bólistiriliwden alınǵan bolsa, onda belgisiz \(\theta\) parametr ushın momentler usılı bahasın tabıń.
 \\
\textbf{C3.} 
\(\ f(x,\theta) = \frac{4x^{3}}{\theta_{2}\sqrt{2\pi}}\exp\left\{ - \frac{\left( x^{4} - \theta_{1} \right)^{2}}{2{\theta_{2}}^{2}} \right\}\) model ushın \(\theta_{1}\) hám \({\theta_{2}}^{2}\) parametrler haqıyqatqa maksimal uqsaslıq usılı bahaları tabılsın.
 \\

\end{tabular}
\vspace{1cm}


\begin{tabular}{m{17cm}}
\textbf{26-variant}
\newline

\textbf{T1.} 
Empirikalıq bólistiriw funkciyası. (Tańlanba, eksperiment)
 \\
\textbf{T2.} 
Momentler usulı. (tańlanba momentleri, belgisiz parametrlerdi bahalaw).
 \\
\textbf{A1.} 
Kólemi \(n = 20\) ǵa teń bolǵan tańlanba berilgen: 4,3; 4,9; 13,4; 13,4; 6,5; 4,9; 4,9; 4,3; 5,1; 6,5; 6,5; 7,0; 4,3; 4,9; 6,5; 6,5; 5,1; 5,1; 4,9; 13,4. Bul tańlanbanıń statistikalıq bólistiriliwin tabıń.
 \\
\textbf{A2.} 
Kólemi \(n = 20\) ǵa teń bolǵan tańlanba berilgen: 4,2; 4,9; 13,8; 13,8; 6,6; 4,9; 4,9; 4,2; 5,3; 6,6; 6,6; 7,5; 4,2; 4,9; 6,6; 6,6; 5,3; 5,3; 4,9; 13,8. Bul tańlanbanıń empirikalıq bólistiriw funkciyasın tabıń.
 \\
\textbf{A3.} 
Joqarı matematika páninen 10 dana student test sınaqların tapsırǵan. Hárbir student 10 balǵa shekem toplawı múmkin. Eger test sınaqları nátiyjeleri boyınsha \{9, 10, 6, 7, 4, 8, 10, 7, 9, 10\} tańlanba alınǵan bolsa, onda tańlanba ortasha hám tańlanba dispersiyalardı tabıń.
 \\
\textbf{B1.} 
Eger ortasha kvadratlıq shetleniwi \(\sigma = 2\) bolǵan normal bólistirilgen bas toplamnan alınǵan kólemi \(n = 10\) ǵa teń tańlanba boyınsha \(\overline{x} = 5,4\) tańlanba ortasha mánisi tabılǵan bolsa, onda \(\gamma = 0,95\) isenimlilik penen belgisiz \(\theta\) matematikalıq kútiliwdi qaplaytuǵın isenimlilik intervalın dúziń.
 \\
\textbf{B2.} 
Eger (3,0,-2,0,-2,3,-2,0,0,3,0,0,0,3,-2,0,0,-2,3,0) tańlanba tómende berilgen bólistiriliwden alınǵan bolsa, onda belgisiz \(\left( \theta_{1},\theta_{2} \right)\) vektor parametr ushın momentler usılı bahalasın tabıń.
\begin{tabular}{|c|c|c|c|}
  \hline
$\xi$ &
$- 2$ &
$0$ &
$3$\\
\hline
\(P_{\theta}\) & \({2\theta}_{1}\) & \(0,5 + \theta_{1} + \theta_{2}\) & \(\theta_{2}\) \\
\hline
\end{tabular}
 \\
\textbf{B3.} 
Eger \(x_{1} = 1,1;\ x_{2} = 2,7;\ldots;x_{100} = 1,5\) tańlanba \(\theta\) parametrli kórsetkishli bólistiriliwden alınǵan bolıp, \(\sum_{k = 1}^{100}x_{k} = 200\) bolsa, onda belgisiz \(\theta\) parametrdiń shınlıqqa maksimal uqsaslıq bahasın tabıń.
 \\
\textbf{C1.} 
Eger \(X^{(n)} = \left( X_{1},...,X_{n} \right)\) tańlanba \(\lbrack 0,\theta\rbrack\) aralıqta teń ólshemli bólistiriliwden alýnǵan bolsa, onda belgisiz \(\theta\) parametr ushın \((n + 1)X_{(1)}\) bahasın jıljımaǵanlıq hám tiykarlılıqqa tekseriń.
 \\
\textbf{C2.} 
Eger \(X^{(n)} = \left( X_{1},...,X_{n} \right)\) tańlanba \({\ \ (a,\theta}^{2})\ \)parametrli normal bólistiriliwden alınǵan bolsa (\(\alpha -\)belgili), onda belgisiz \({\ \theta}^{2}\) parametr ushın momentler usılı bahasın \({\ g(x) = (x - a)}^{2}\) funkciyası járdeminde tabıń.
 \\
\textbf{C3.} 
Eger \(X^{(n)} = \left( X_{1},...,X_{n} \right)\) tańlanba tıǵızlıq funkciyası
$f(x;\theta) = \frac{1}{2}e^{- \ |x - \theta|},\ x \in R$
bolǵan Laplas bólistiriliwinen alınǵan bolsa, onda belgisiz \(\theta \in R\) parametrdiń shınlıqqa maksimal uqsaslıq bahasın tabıń.
 \\

\end{tabular}
\vspace{1cm}


\begin{tabular}{m{17cm}}
\textbf{27-variant}
\newline

\textbf{T1.} 
Tańlanba momentleri (\(k -\)tártipli baslanǵısh, baslanǵısh absolyut, oraylıq hám oraylıq absolyut momentler).
 \\
\textbf{T2.} 
Statistikalıq baha qásiyetleri. (Jıljımaytuǵın, tiykarlı, effektiv)
 \\
\textbf{A1.} 
Kólemi \(n = 20\) ǵa teń bolǵan tańlanba berilgen: -2,1; 1,7; 3,3; 3,3; 11,7; 4,7; 1,7; 4,7; -2,1; 4,7; 4,7; 4,7; 8,0; -2,1; 1,7; 4,7; 8,0; 11,7; 1,7; 8,0. Bul tańlanbanıń statistikalıq bólistiriliwin tabıń.
 \\
\textbf{A2.} 
Kólemi \(n = 20\) ǵa teń bolǵan tańlanba berilgen: -2,2; 1,3; 3,8; 3,8; 11,5; 4,1; 1,3; 4,1; -2,2; 4,1; 4,1; 4,1; 8,4; -2,2; 1,3; 4,1; 8,4; 11,5; 1,3; 8,4. Bul tańlanbanıń empirikalıq bólistiriw funkciyasın tabıń.
 \\
\textbf{A3.} 
Joqarı matematika páninen 10 dana student test sınaqların tapsırǵan. Hárbir student 10 balǵa shekem toplawı múmkin. Eger test sınaqları nátiyjeleri boyınsha \{4, 1, 2, 4, 6, 4, 5, 3, 6, 5\} tańlanba alınǵan bolsa, onda tańlanba ortasha hám tańlanba dispersiyalardı tabıń.
 \\
\textbf{B1.} 
Eger normal bólistirilgen bas toplamnan alınǵan kólemi \(n = 16\) ǵa teń tańlanba boyınsha \(\overline{x} = 20,2\) tańlanba ortasha hám \({\overline{S}}^{2} = 0,64\) dúzetilgen tańlanba dispersiyalar tabılǵan bolsa, onda \(\gamma = 0,95\) isenimlilik penen belgisiz \(\theta\) matematikalıq kútiliw ushın isenimlilik interval dúziń.
 \\
\textbf{B2.} 
Eger (3,-2,-2,0,-2,-2,-2,0,-2,3,-2,0,3,0,3,-2,0,-2,3,-2,-2,-2,-2,3,3,3,-2,-2,3,3) tańlanba tómende berilgen bólistiriliwden alınǵan bolsa, onda belgisiz \(\theta\) parametr ushın momentler usılı bahasın \(g(x) = |x|\) funkciyası járdeminde tabıń.
\begin{tabular}{|c|c|c|c|}
  \hline
$\xi$ &
$- 2$ &
$0$ &
$3$ \\
\hline
\(P_{\theta}\) & \(3\theta\) & \(1 - 5\theta\) & \(2\theta\) \\
\hline
\end{tabular}
 \\
\textbf{B3.} 
Eger \(X^{(n)} = \left( X_{1},...,X_{n} \right)\) tańlanba \(\theta\) parametrli kórsetkishli bólistiriliwden alınǵan bolsa, onda belgisiz \(\theta\) parametrdiń shınlıqqa maksimal uqsaslıq usılı bahasın tabıń.
 \\
\textbf{C1.} 
Eger \(X^{(n)} = \left( X_{1},...,X_{n} \right)\) tańlanba \(\lbrack 0,\theta\rbrack\) aralıqta teń ólshemli bólistiriliwden alınǵan bolsa, onda belgisiz \(\theta\) parametr ushın \(\frac{n + 1}{n}X_{(n)}\) bahasın jıljımaǵanlıq hám tiykarlılıqqa tekseriń.
 \\
\textbf{C2.} 
Eger \(X^{(n)} = \left( X_{1},...,X_{n} \right)\) tańlanba tıǵızlıq funkciyası
$
{f(x,\theta) = \left\{ \begin{array}{r}
e^{\theta - x},\ \ \ x \geq \theta, \\
0,\ \ \ x < \theta
\end{array} \right.\ }$
bolǵan bólistiriliwden alınǵan bolsa, onda belgisiz \(\theta\) parametr ushın momentler usılı bahasın tabıń.
 \\
\textbf{C3.} 
Eger \(X^{(n)} = \left( X_{1},...,X_{n} \right)\) tańlanba \(\left( \theta,\theta^{2} \right)\) parametrli normal bólistiriliwden alınǵan bolsa, onda belgisiz \(\theta > 0\) parametrdiń shınlıqqa maksimal uqsaslıq bahasın tabıń.
 \\

\end{tabular}
\vspace{1cm}


\begin{tabular}{m{17cm}}
\textbf{28-variant}
\newline

\textbf{T1.} 
Poligon hám gistogramma(salıstirmalı jiyilik, intervallıq qatar, grafik)
 \\
\textbf{T2.} 
Isenimlilik intervalların qurıw. Anıq isenimli intervallar
 \\
\textbf{A1.} 
Kólemi \(n = 20\) ǵa teń bolǵan tańlanba berilgen: -11,0; -4,1; 0; 2,3; 1,2; 0; 1,2; 2,3; 2,3; 1,2; 2,3; -11,0; 3,4; 1,2; 3,4; 3,4; 0; 3,4; 2,3; 0. Bul tańlanbanıń statistikalıq bólistiriliwin tabıń.
 \\
\textbf{A2.} 
Kólemi \(n = 20\) ǵa teń bolǵan tańlanba berilgen: -11,2; -4,5; 0; 2,9; 1,7; 0; 1,7; 2,9; 2,9; 1,7; 2,9; -11,2; 3,1; 1,7; 3,1; 3,1; 0; 3,1; 2,9; 0. Bul tańlanbanıń empirikalıq bólistiriw funkciyasın tabıń.
 \\
\textbf{A3.} 
Joqarı matematika páninen 10 dana student test sınaqların tapsırǵan. Hárbir student 10 balǵa shekem toplawı múmkin. Eger test sınaqları nátiyjeleri boyınsha \{8, 9, 10, 4, 9, 7, 6, 7, 6, 4\} tańlanba alınǵan bolsa, onda tańlanba ortasha hám tańlanba dispersiyalardı tabıń.
 \\
\textbf{B1.} 
Eger normal bólistirilgen bas toplamnan alınǵan kólemi \(n = 11\) ǵa teń bolǵan tańlanba boyınsha \({\overline{S}}^{2} = 0,5\) dúzetilgen tańlanba dispersiya tabılǵan bolsa, onda \(\gamma = 0,90\) isenimlilik penen belgisiz \(\theta_{2}^{2}\) dispersiya ushın isenimlilik interval dúziń.
 \\
\textbf{B2.} 
Eger tıǵızlıq funkciyası \(f(x) = \frac{2x}{\theta}e^{- \frac{x^{2}}{\theta}},\ \ x \geq 0\) kóriniske iye bolsa, onda \(\theta\) parametr momentler usulı bahasın tabıń.
 \\
\textbf{B3.} 
\(\ f(x) = \frac{\theta}{2}e^{- \theta|x|}\) model ushın \(\theta\) parametri haqıyqatqa maksimal uqsaslıq usılı bahası tabılsın.
 \\
\textbf{C1.} 
Eger \(X^{(n)} = \left( X_{1},...,X_{n} \right)\) tańlanba \(M\xi = a\) belgili hám \(M\xi^{2}\) shekli bolǵan bólistiriliwden alınǵan bolsa, onda belgisiz \(D\xi\) dispersiya ushın \({\overline{S}}^{2}\) bahasın jıljımaǵanlıq hám tiykarlılıqqa tekseriń.
 \\
\textbf{C2.} 
Eger \(X^{(n)} = \left( X_{1},...,X_{n} \right)\) tańlanba \(\theta\) parametrli Puasson bólistiriliwinen alınǵan bolsa, onda belgisiz \(\theta\) parametr ushın momentler usılı bahasın tabıń. Eger \(X^{(n)} = \left( X_{1},...,X_{n} \right)\) tańlanba \(\theta\) parametrli Puasson bólistiriliwinen alınǵan bolsa, onda belgisiz \(\theta\) parametr ushın momentler usılı bahasın\({\ g(x) = x}^{2}\) funkciyası járdeminde tabıń.
 \\
\textbf{C3.} 
Eger \(X^{(n)} = \left( X_{1},...,X_{n} \right)\) tańlanba tıǵızlıq funkciyası
$f(x;\theta) = \left\{ \begin{matrix}
e^{\theta - x},\ \ x \geq \theta, \\
\ \ 0,\ \ \ \ \ \ \ x < \theta
\end{matrix} \right.\ $
bolǵan bólistiriliwden alınǵan bolsa, onda belgisiz \(\theta\) parametrdiń shınlıqqa maksimal uqsaslıq bahasın tabıń.
 \\

\end{tabular}
\vspace{1cm}


\begin{tabular}{m{17cm}}
\textbf{29-variant}
\newline

\textbf{T1.} 
Tańlanba xarakteristikaları.(tańlanba orta, tańlanba dispersiya)
 \\
\textbf{T2.} 
Sızıqlı korrelyaciya teńlemesi (anıqlaması, regressiya tuwrı sızıǵınıń tańlanba teńlemeleri)
 \\
\textbf{A1.} 
Kólemi \(n = 20\) ǵa teń bolǵan tańlanba berilgen: 2,5; 3,8; 4,3; 2,5; 3,8; 2,5; 3,1; 4,3; 4,3; 5,5; 6,2; 2,5; 3,1; 6,2; 5,5; 6,2; 3,1; 3,1; 6,2; 3,1. Bul tańlanbanıń statistikalıq bólistiriliwin tabıń.
 \\
\textbf{A2.} 
Kólemi \(n = 20\) ǵa teń bolǵan tańlanba berilgen: 2,7; 4,2; 4,8; 2,7; 4,2; 2,7; 3,9; 4,8; 4,8; 5,9; 6,5; 2,7; 3,9; 6,5; 5,9; 6,5; 3,9; 3,9; 6,5; 3,9. Bul tańlanbanıń empirikalıq bólistiriw funkciyasın tabıń.
 \\
\textbf{A3.} 
Joqarı matematika páninen 10 dana student test sınaqların tapsırǵan. Hárbir student 10 balǵa shekem toplawı múmkin. Eger test sınaqları nátiyjeleri boyınsha \{7, 8, 7, 6, 4, 8, 4, 7, 9, 10\} tańlanba alınǵan bolsa, onda tańlanba ortasha hám tańlanba dispersiyalardı tabıń.
 \\
\textbf{B1.} 
Eger ortasha kvadratlıq shetleniwi \(\sigma = 3\) bolǵan normal bólistirilgen bas toplamnan alınǵan kólemi \(n = 9\) ǵa teń tańlanba boyınsha \(\overline{x} = 4,5\) tańlanba ortasha mánisi tabılǵan bolsa, onda \(\gamma = 0,95\) isenimlilik penen belgisiz \(\theta\) matematikalıq kútiliwdi qaplaytuǵın isenimlilik intervalın dúziń.
 \\
\textbf{B2.} 
Eger \(X^{(n)} = \left( X_{1},...,X_{n} \right)\) tańlanba \(\theta\) parametrli kórsetkishli bólistiriliwden alınǵan bolsa, onda belgisiz \(\theta\) parametr ushın momentler usılı bahasın tabıń.
 \\
\textbf{B3.} 
Eger (4,8,5,3) tańlanba \(\left( a,\theta^{2} \right)\) parametrli normal bólistiriliwden alınǵan bolsa, onda belgisiz \(\theta^{2}\) parametrdiń shınlıqqa maksimal uqsaslıq bahasın tabıń.
 \\
\textbf{C1.} 
Eger \(X^{(n)} = \left( X_{1},...,X_{n} \right)\) tańlanba \(M\xi = a\) belgili hám \(M\xi^{2}\) shekli bolǵan bólistiriliwden alınǵan bolsa, onda belgisiz \(D\xi\) dispersiya ushın \(\frac{1}{n - 1}\sum_{i = 1}^{n}\left( X_{i} - a \right)^{2}\) bahasın jıljımaǵanlıq hám tiykarlılıqqa tekseriń.
 \\
\textbf{C2.} 
Eger \(X^{(n)} = \left( X_{1},...,X_{n} \right)\) tańlanba tıǵızlıq funkciyası
$f(x,\theta) = \left\{ \begin{array}{r}
\theta_{1}^{- 1}e^{- \frac{x - \theta_{2}}{\theta_{1}}},\ \ \ x \geq \theta_{2}, \\
0,\ \ \ x < \theta_{2}
\end{array} \right.\ $
bolǵan bólistiriliwden alınǵan bolsa, onda belgisiz \(\left( \theta_{1},\theta_{2} \right)\) \(\theta_{1} > 0,\) \(\theta_{2} \in R\) vektor parametr ushın momentler usılı bahasın tabıń.
 \\
\textbf{C3.} 
Eger \(X^{(n)} = \left( X_{1},...,X_{n} \right)\) tańlanba tıǵızlıq funkciyası
$f(x;\theta) = \left\{ \begin{matrix}
3x^{2}\theta^{- 3}e^{- \ \left( \frac{x}{\theta} \right)^{3}},\ \ x \geq 0, \\
\ \ \ \ \ \ \ \ \ \ \ \ \ \ 0,\ \ \ \ \ \ \ \ \ x < 0
\end{matrix} \right.\ $
bolǵan bólistiriliwden alınǵan bolsa, onda belgisiz \(\theta > 0\) parametrdiń shınlıqqa maksimal uqsaslıq bahasın tabıń.
 \\

\end{tabular}
\vspace{1cm}


\begin{tabular}{m{17cm}}
\textbf{30-variant}
\newline

\textbf{T1.} Matematikalıq statistikanıń tiykarǵı máseleleri. (Statistikalıq maǵlıwmatlar, gruppalaw)
 \\
\textbf{T2.} 
Normal nızamnıń dispersiyası ushın isenimlilik intervalın dúziw. (Isenimlilik itimallıǵı, interval)
 \\
\textbf{A1.} 
Kólemi \(n = 20\) ǵa teń bolǵan tańlanba berilgen: -4,3; 2,6; 0; -2,5; 2,6; 1,9; 2,2; 0; -4,3; -2,5; 1,9; -2,5; 1,9; 2,2; 2,6; 1,9; 2,6; 2,2; 2,2; 1,9. Bul tańlanbanıń statistikalıq bólistiriliwin tabıń.
 \\
\textbf{A2.} 
Kólemi \(n = 20\) ǵa teń bolǵan tańlanba berilgen: -4,9; 2,6; 0,5; -2,6; 2,6; 1,7; 2,3; 0,5; -4,9; -2,6; 1,7; -2,6; 1,7; 2,3; 2,6; 1,7; 2,6; 2,3; 2,3; 1,7. Bul tańlanbanıń empirikalıq bólistiriw funkciyasın tabıń.
 \\
\textbf{A3.} 
Joqarı matematika páninen 10 dana student test sınaqların tapsırǵan. Hárbir student 10 balǵa shekem toplawı múmkin. Eger test sınaqları nátiyjeleri boyınsha \{9, 5, 6, 8, 4, 7, 4, 6, 9, 7\} tańlanba alınǵan bolsa, onda tańlanba ortasha hám tańlanba dispersiyalardı tabıń.
 \\
\textbf{B1.} 
Eger normal bólistirilgen bas toplamnan alınǵan kólemi \(n = 25\) ǵa teń tańlanba boyınsha \(\overline{x} = 18,6\) tańlanba ortasha hám \({\overline{S}}^{2} = 0,49\) dúzetilgen tańlanba dispersiyalar tabılǵan bolsa, onda \(\gamma = 0,95\) isenimlilik penen belgisiz \(\theta\) matematikalıq kútiliw ushın isenimlilik interval dúziń.
 \\
\textbf{B2.} 
\(\lbrack\theta_{1},\theta_{2}\rbrack\) aralıqta teń ólshewli bólistiriw parametrleri ushın momentler usulı bahaların tabıń.
 \\
\textbf{B3.} 
Eger \(X^{(n)} = \left( X_{1},...,X_{n} \right)\) tańlanba tıǵızlıq funkciyası \(f(x;\theta) = \frac{2x}{\theta}e^{- \frac{x^{2}}{\theta}},\ x \geq 0\). bolǵan bólistiriliwden alınǵan bolsa, onda belgisiz \(\theta > 0\) parametrdiń shınlıqqa maksimal uqsaslıq bahasın tabıń.
 \\
\textbf{C1.} 
Eger \(X^{(n)} = \left( X_{1},...,X_{n} \right)\) tańlanba \(M\xi = a\) belgili hám \(M\xi^{2}\) shekli bolǵan bólistiriliwden alınǵan bolsa, onda belgisiz \(D\xi\) dispersiya ushın \(\frac{1}{n}\sum_{i = 1}^{n}\left( X_{i} - a \right)^{2}\) bahasın jıljımaǵanlıq hám tiykarlılıqqa tekseriń.
 \\
\textbf{C2.} 
Eger \(X^{(n)} = \left( X_{1},...,X_{n} \right)\) tańlanba tıǵızlıq funkciyası
$f(x,\theta) = \frac{2x}{\theta^{2}},x \in \lbrack 0,\theta\rbrack$
bolǵan bólistiriliwden alınǵan bolsa, onda belgisiz \(\theta\) parametr ushın momentler usılı bahasın tabıń.
 \\
\textbf{C3.} 
Eger \(X^{(n)} = \left( X_{1},...,X_{n} \right)\) tańlanba \((\theta,2\theta)\) parametrli normal bólistiriliwden alınǵan bolsa, onda belgisiz \(\theta > 0\) parametrdiń shınlıqqa maksimal uqsaslıq bahasın tabıń.
 \\

\end{tabular}
\vspace{1cm}


\begin{tabular}{m{17cm}}
\textbf{31-variant}
\newline

\textbf{T1.} 
Tańlanba momentleri (\(k -\)tártipli baslanǵısh, baslanǵısh absolyut, oraylıq hám oraylıq absolyut momentler).
 \\
\textbf{T2.} 
Haqiyqatqa maksimal uqsaslıq usulı. (haqiyqatqa maksimal uqsaslıq funkciyası, belgisiz parametrlerdi bahalaw).
 \\
\textbf{A1.} 
Kólemi \(n = 20\) ǵa teń bolǵan tańlanba berilgen: -2,9; -3,8; 2,3; 1,8; 1,8; 0,7; -3,8; -1,5; 2,3; 0,7; -2,9; -1,5; 1,8; -2,9; -1,5; -3,8; 1,8; 1,8; -3,8; 1,8. Bul tańlanbanıń statistikalıq bólistiriliwin tabıń.
 \\
\textbf{A2.} 
Kólemi \(n = 20\) ǵa teń bolǵan tańlanba berilgen: -2,4; -3,5; 2,8; 1,4; 1,4; 0,1; -3,5; -1,9; 2,8; 0,1; -2,4; -1,9; 1,4; -2,4; -1,9; -3,5; 1,4; 1,4; -3,5; 1,4. Bul tańlanbanıń empirikalıq bólistiriw funkciyasın tabıń.
 \\
\textbf{A3.} 
Joqarı matematika páninen 10 dana student test sınaqların tapsırǵan. Hárbir student 10 balǵa shekem toplawı múmkin. Eger test sınaqları nátiyjeleri boyınsha \{8, 9, 7, 10, 6, 8, 10, 3, 10, 9\} tańlanba alınǵan bolsa, onda tańlanba ortasha hám tańlanba dispersiyalardı tabıń.
 \\
\textbf{B1.} 
Eger normal bólistirilgen bas toplamnan alınǵan kólemi \(n = 12\) ǵa teń bolǵan tańlanba boyınsha \({\overline{S}}^{2} = 0,4\) dúzetilgen tańlanba dispersiya tabılǵan bolsa, onda \(\gamma = 0,90\) isenimlilik penen belgisiz \(\theta_{2}^{2}\) dispersiya ushın isenimlilik interval dúziń.
 \\
\textbf{B2.} 
\(\lbrack 0,\theta\rbrack\) aralıqta teń ólshewli bólistirilgen \(\theta\) parametri ushın momentler usulı bahasın tabıń.
 \\
\textbf{B3.} 
Eger \(X^{(n)} = \left( X_{1},...,X_{n} \right)\) tańlanba \(\lbrack - \theta,\theta\rbrack\) aralıqta teń ólshemli bólistiriliwden alınǵan bolsa, onda belgisiz \(\theta > 0\) parametrdiń shınlıqqa maksimal uqsaslıq usılı bahasın tabıń.
 \\
\textbf{C1.} 
Eger \(X^{(n)} = \left( X_{1},...,X_{n} \right)\) tańlanba \(M\xi = a\) belgili hám \(M\xi^{2}\) shekli bolǵan bólistiriliwden alınǵan bolsa, onda belgisiz \(D\xi\) dispersiya ushın \(\overline{x^{2}} - a^{2}\) bahasın jıljımaǵanlıq hám tiykarlılıqqa tekseriń.
 \\
\textbf{C2.} 
Eger \(X^{(n)} = \left( X_{1},...,X_{n} \right)\) tańlanba \(\frac{1}{\sqrt{\theta}}\) parametrli kórsetkishli bólistiriliwden alınǵan bolsa, onda belgisiz \(\theta\) parametr ushın momentler usılı bahasın tabıń.
 \\
\textbf{C3.} 
Eger \(X^{(n)} = \left( X_{1},...,X_{n} \right)\) tańlanba tıǵızlıq funkciyası
$f(x;\theta) = \frac{3x^{2}}{\sqrt{2\pi}}\exp\left\{ - \frac{\left( x^{3} - \theta \right)^{2}}{2} \right\},\ x \in R$
bolǵan bólistiriliwden alınǵan bolsa, onda belgisiz \(\theta\) parametrdiń shınlıqqa maksimal uqsaslıq bahasın tabıń.
 \\

\end{tabular}
\vspace{1cm}


\begin{tabular}{m{17cm}}
\textbf{32-variant}
\newline

\textbf{T1.} 
Tańlanba xarakteristikalar. (Variaciyalıq qatar, salıstırmalı jiyilik).
 \\
\textbf{T2.} 
Momentler usulı. (tańlanba momentleri, belgisiz parametrlerdi bahalaw).
 \\
\textbf{A1.} 
Kólemi \(n = 20\) ǵa teń bolǵan tańlanba berilgen: 3,6; 2,9; 3,6; 3,2; 1,1; 0,3; 1,1; 3,6; 1,7; 1,1; 0,3; 1,7; 1,1; 0,3; 2,9; 2,9; 2,9; 1,1; 2,9; 1,7. Bul tańlanbanıń statistikalıq bólistiriliwin tabıń.
 \\
\textbf{A2.} 
Kólemi \(n = 20\) ǵa teń bolǵan tańlanba berilgen: 4,6; 2,5; 4,6; 3,3; 1,8; 0,3; 1,8; 4,6; 2,1; 1,8; 0,3; 2,1; 1,8; 0,3; 2,5; 2,5; 2,5; 1,8; 2,5; 2,1. Bul tańlanbanıń empirikalıq bólistiriw funkciyasın tabıń.
 \\
\textbf{A3.} 
Joqarı matematika páninen 10 dana student test sınaqların tapsırǵan. Hárbir student 10 balǵa shekem toplawı múmkin. Eger test sınaqları nátiyjeleri boyınsha \{5, 7, 5, 9, 5, 8, 10, 6, 7, 8\} tańlanba alınǵan bolsa, onda tańlanba ortasha hám tańlanba dispersiyalardı tabıń.
 \\
\textbf{B1.} 
Eger ortasha kvadratlıq shetleniwi \(\sigma = 1\) bolǵan normal bólistirilgen bas toplamnan alınǵan kólemi \(n = 15\) ǵa teń tańlanba boyınsha \(\overline{x} = 5,8\) tańlanba ortasha mánisi tabılǵan bolsa, onda \(\gamma = 0,90\) isenimlilik penen belgisiz \(\theta\) matematikalıq kútiliwdi qaplaytuǵın isenimlilik intervalın dúziń.
 \\
\textbf{B2.} 
Eger \(X^{(n)} = \left( X_{1},...,X_{n} \right)\) tańlanba \(\theta\) parametrli Bernulli bólistiriliwinen alınǵan bolsa, onda belgisiz \(\theta\) parametr ushın momentler usılı bahasın tabıń.
 \\
\textbf{B3.} 
Eger (-1,-1,0,-1,0,-1,-1,5,-1,0,-1,0,5,-1,-1,-1,5,-1,-1,-1,5,0,-1,-1,5) tańlanba tómende berilgen bólistiriliwden alınǵan bolsa, onda belgisiz \(\theta\) parametrdiń shınlıqqa maksimal uqsaslıq usılı bahasın tabıń.
\begin{tabular}{|c|c|c|c|}
  \hline
$\xi$
&
$- 1$
&
$0$
&
$5$\\
\hline
\(P_{\theta}\) & \(1 - \theta\) & \(\theta/2\) & \(\theta/2\ \) \\
\hline
\end{tabular}
 \\
\textbf{C1.} 
Eger \(X^{(n)} = \left( X_{1},...,X_{n} \right)\) tańlanba tıǵızlıq funkciyası: \(f(x,\theta) = \left\{ \begin{matrix}
e^{\theta - x},\ \ x \geq \theta, \\
\ \ 0,\ \ \ \ \ \ \ x < \theta
\end{matrix} \right.\ \)
bolǵan bólistiriliwden alınǵan bolsa, onda belgisiz \(\theta\) parametr ushın \(X_{(1)}\) bahasın jıljımaǵanlıq hám tiykarlılıqqa tekseriń.
 \\
\textbf{C2.} 
Eger \(X^{(n)} = \left( X_{1},...,X_{n} \right)\) tańlanba \(\left( \theta_{1},\theta_{2} \right)\) parametrli gamma bólistiriliwden alınǵan bolsa, onda belgisiz \(\left( \theta_{1},\theta_{2} \right)\) vektor parametr ushın momentler usılı bahasın tabıń.
 \\
\textbf{C3.} 
Eger \(X^{(n)} = \left( X_{1},...,X_{n} \right)\) tańlanba tıǵızlıq funkciyası
$f(x;\theta) = \frac{\theta ln^{\theta - 1}x}{x},\ x \in \lbrack 1,e\rbrack$
bolǵan bólistiriliwden alınǵan bolsa, onda belgisiz \(\theta > 0\) parametr ushın shınlıqqa maksimal uqsaslıq bahasın tabıń.
 \\

\end{tabular}
\vspace{1cm}


\begin{tabular}{m{17cm}}
\textbf{33-variant}
\newline

\textbf{T1.} 
Empirikalıq bólistiriw funkciyası. (Tańlanba, eksperiment)
 \\
\textbf{T2.} 
Kolmogorovtıń kelisimlilik belgisi (Kolmogorov teoreması)
 \\
\textbf{A1.} 
Kólemi \(n = 20\) ǵa teń bolǵan tańlanba berilgen: -1,3; 0; 0,8; 2,3; 1,1; 0,8; 0,8; 2,3; 1,1; 0,8; -1,3; 1,8; 1,1; -1,3; 1,1; 1,8; 1,8; 1,1; 1,8; 1,8. Bul tańlanbanıń statistikalıq bólistiriliwin tabıń.
 \\
\textbf{A2.} 
Kólemi \(n = 20\) ǵa teń bolǵan tańlanba berilgen: -1,9; 0,7; 0,9; 2,8; 1,3; 0,9; 0,9; 2,8; 1,3; 0,9; -1,9; 1,6; 1,3; -1,9; 1,3; 1,6; 1,6; 1,3; 1,6; 1,6. Bul tańlanbanıń empirikalıq bólistiriw funkciyasın tabıń.
 \\
\textbf{A3.} 
Joqarı matematika páninen 10 dana student test sınaqların tapsırǵan. Hárbir student 10 balǵa shekem toplawı múmkin. Eger test sınaqları nátiyjeleri boyınsha \{8, 4, 3, 7, 3, 6, 5, 3, 5, 6\} tańlanba alınǵan bolsa, onda tańlanba ortasha hám tańlanba dispersiyalardı tabıń.
 \\
\textbf{B1.} 
Eger normal bólistirilgen bas toplamnan alınǵan kólemi \(n = 20\) ǵa teń tańlanba boyınsha \(\overline{x} = 16,6\) tańlanba ortasha hám \({\overline{S}}^{2} = 0,64\) dúzetilgen tańlanba dispersiyalar tabılǵan bolsa, onda \(\gamma = 0,95\) isenimlilik penen belgisiz \(\theta\) matematikalıq kútiliw ushın isenimlilik interval dúziń.
 \\
\textbf{B2.} 
Kórsetkishli bólistiriw belgisiz \(\theta > 0\) parametri momentlar usulı bahasın tabıń.
 \\
\textbf{B3.} 
Eger \(X^{(n)} = \left( X_{1},...,X_{n} \right)\) tańlanba \(\left( a,\theta^{2} \right)\) parametrli normal bólistiriliwden alınǵan bolsa (\(\alpha -\)belgili), onda belgisiz \(\theta^{2}\) parametrdiń shınlıqqa maksimal uqsaslıq bahasın tabıń.
 \\
\textbf{C1.} 
Eger \(X^{(n)} = \left( X_{1},...,X_{n} \right)\) tańlanba tıǵızlıq funkciyası: \(f(x,\theta) = \left\{ \begin{matrix}
e^{\theta - x},\ \ x \geq \theta, \\
\ \ 0,\ \ \ \ \ \ \ x < \theta
\end{matrix} \right.\ \)
bolǵan bólistiriliwden alınǵan bolsa, onda belgisiz \(\theta\) parametr ushın \(\overline{x} - 1\) bahasın jıljımaǵanlıq hám tiykarlılıqqa tekseriń.
 \\
\textbf{C2.} 
Eger \(X^{(n)} = \left( X_{1},...,X_{n} \right)\) tańlanba tıǵızlıq funkciyası
${f(x,\theta) = \theta x}^{\theta - 1},x \in \lbrack 0,1\rbrack$
bolǵan bólistiriliwden alınǵan bolsa, onda belgisiz \(\theta\) parametr ushın momentler usılı bahasın tabıń.
 \\
\textbf{C3.} 
Eger \(X^{(n)} = \left( X_{1},...,X_{n} \right)\) tańlanba \(\left\lbrack \theta_{1},\theta_{2} \right\rbrack\) aralıqta teń ólshemli bólistiriliwden alınǵan bolsa, onda belgisiz \(\left( \theta_{1},\theta_{2} \right)\) vektor parametrdiń shınlıqqa maksimal uqsaslıq bahasın tabıń.
 \\

\end{tabular}
\vspace{1cm}


\begin{tabular}{m{17cm}}
\textbf{34-variant}
\newline

\textbf{T1.} 
Gruppalanǵan hám intervallıq variaciyalıq qatarlar.
 \\
\textbf{T2.} 
Normal nızamnıń dispersiyası ushın isenimlilik intervalın dúziw. (Isenimlilik itimallıǵı, interval)
 \\
\textbf{A1.} 
Kólemi \(n = 20\) ǵa teń bolǵan tańlanba berilgen: -2,4; 5,6; 5,6; -5,2; -6,7; 5,1; -5,2; -2,4; 4,3; 5,1; -6,7; 4,3; -2,4; -6,7; 4,3; 5,1; 4,3; 5,6; -6,7; 5,6. Bul tańlanbanıń statistikalıq bólistiriliwin tabıń.
 \\
\textbf{A2.} 
Kólemi \(n = 20\) ǵa teń bolǵan tańlanba berilgen: -2,9; 7,6; 7,6; -5,7; -6,1; 5,5; -5,7; -2,9; 4,2; 5,5; -6,1; 4,2; -2,9; -6,1; 4,2; 5,5; 4,2; 7,6; -6,1; 7,6. Bul tańlanbanıń empirikalıq bólistiriw funkciyasın tabıń.
 \\
\textbf{A3.} 
Joqarı matematika páninen 10 dana student test sınaqların tapsırǵan. Hárbir student 10 balǵa shekem toplawı múmkin. Eger test sınaqları nátiyjeleri boyınsha \{9, 8, 6, 7, 5, 8, 5, 7, 4, 6\} tańlanba alınǵan bolsa, onda tańlanba ortasha hám tańlanba dispersiyalardı tabıń.
 \\
\textbf{B1.} 
Eger normal bólistirilgen bas toplamnan alınǵan kólemi \(n = 13\) ǵa teń bolǵan tańlanba boyınsha \({\overline{S}}^{2} = 1,2\) dúzetilgen tańlanba dispersiya tabılǵan bolsa, onda \(\gamma = 0,90\) isenimlilik penen belgisiz \(\theta_{2}^{2}\) dispersiya ushın isenimlilik interval dúziń.
 \\
\textbf{B2.} 
Eger \(X^{(n)} = \left( X_{1},...,X_{n} \right)\) tańlanba \(\theta\) parametrli kórsetkishli bólistiriliwden alınǵan bolsa, onda belgisiz \(\theta\) parametr ushın momentler usılı bahasın tabıń.
 \\
\textbf{B3.} 
Eger (0,1,2,0) tańlanba tómende berilgen bólistiriliwden alınǵan bolsa, onda belgisiz \(\theta\) parametrdiń shınlıqqa maksimal uqsaslıq bahasın tabıń.
\begin{tabular}{|c|c|c|c|}
  \hline
$\xi$
&
$0$
&
$1$
&
$2$\\
\hline
\(P_{\theta}\) & \(\theta\) & \(2\theta\) & \(1 - 3\theta\) \\
\hline
\end{tabular}
 \\
\textbf{C1.} 
Eger \(X^{(n)} = \left( X_{1},...,X_{n} \right)\) tańlanba \(\lbrack - 3\theta,\theta\rbrack\) aralıqta teń ólshemli bólistiriliwden alınǵan bolsa, onda belgisiz \(\theta\) parametr ushın \(4X_{(n)} + X_{(1)}\) bahasın jıljımaǵanlıq hám tiykarlılıqqa tekseriń.
 \\
\textbf{C2.} 
Eger \(X^{(n)} = \left( X_{1},...,X_{n} \right)\) tańlanba\({\ \ (a,\theta}^{2})\) parametrli normal bólistiriliwden alınǵan bolsa (\(\alpha -\)belgili), onda belgisiz\({\ \ \theta}^{2}\) parametr ushın momentler usılı bahasın tabıń.
 \\
\textbf{C3.} 
Eger \(X^{(n)} = \left( X_{1},...,X_{n} \right)\) tańlanba \(\lbrack\theta,\theta + 2\rbrack\) aralıqta teń ólshemli bólistiriliwden alınǵan bolsa, onda belgisiz \(\theta\) parametrdiń shınlıqqa maksimal uqsaslıq usılı bahasın tabıń.
 \\

\end{tabular}
\vspace{1cm}


\begin{tabular}{m{17cm}}
\textbf{35-variant}
\newline

\textbf{T1.} 
Neyman-Pirson teoreması
 \\
\textbf{T2.} 
Isenimlilik intervalların qurıw. Anıq isenimli intervallar
 \\
\textbf{A1.} 
Kólemi \(n = 20\) ǵa teń bolǵan tańlanba berilgen:-3,3; 0; 4,4; 2,2; -2,7; 4,4; 2,2; 4,4;-3,3; 2,2; -2,7; 2,2; -3,3; -2,7; 2,2; 3,4; 4,4; 0; -3,3; 0. Bul tańlanbanıń statistikalıq bólistiriliwin tabıń.
 \\
\textbf{A2.} 
Kólemi \(n = 20\) ǵa teń bolǵan tańlanba berilgen:-3,3; 0; 4,9; 2,8; -2,6; 4,9; 2,8; 4,9;-3,3; 2,8; -2,6; 2,8; -3,3; -2,6; 2,8; 3,1; 4,9; 0; -3,3; 0. Bul tańlanbanıń empirikalıq bólistiriw funkciyasın tabıń.
 \\
\textbf{A3.} 
Joqarı matematika páninen 10 dana student test sınaqların tapsırǵan. Hárbir student 10 balǵa shekem toplawı múmkin. Eger test sınaqları nátiyjeleri boyınsha \{4, 7, 6, 9, 3, 8, 3, 7, 4, 9\} tańlanba alınǵan bolsa, onda tańlanba ortasha hám tańlanba dispersiyalardı tabıń.
 \\
\textbf{B1.} 
Eger ortasha kvadratlıq shetleniwi \(\sigma = 4\) bolǵan normal bólistirilgen bas toplamnan alınǵan kólemi \(n = 12\) ǵa teń tańlanba boyınsha \(\overline{x} = 3\) tańlanba ortasha mánisi tabılǵan bolsa, onda \(\gamma = 0,95\) isenimlilik penen belgisiz \(\theta\) matematikalıq kútiliwdi qaplaytuǵın isenimlilik intervalın dúziń.
 \\
\textbf{B2.} 
Eger (-2,0,-2,0,-2,3,-2,0,0,3,0,0,0,3,-2,0,0,-2,3,0) tańlanba tómende berilgen bólistiriliwden alınǵan bolsa, onda belgisiz \(\left( \theta_{1},\theta_{2} \right)\) vektor parametr ushın momentler usılı bahalasın tabıń.
\begin{tabular}{|c|c|c|c|}
  \hline
$\xi$ &
$- 2$ &
$0$ &
$3$\\
\hline
\(P_{\theta}\) & \(\theta_{1}\) & \(1 - \theta_{1} - \theta_{2}\) & \(\theta_{2}\) \\
\hline
\end{tabular}
 \\
\textbf{B3.} 
Eger \(X^{(n)} = \left( X_{1},...,X_{n} \right)\) tańlanba \(\theta\) parametrli Bernulli bólistiriliwinen alınǵan bolsa, onda belgisiz \(\theta\) parametrdiń shınlıqqa maksimal uqsaslıq usılı bahasın tabıń.
 \\
\textbf{C1.} 
Eger \(X^{(n)} = \left( X_{1},...,X_{n} \right)\) tańlanba bólistiriw funkciyası \(F(x)\) bolǵan bólistiriliwden alınǵan bolsa, onda belgisiz \(F(x)\) ushın \(F_{n}(x)\) empirikalıq bólistiriw funkciyasın jıljımaǵanlıq hám tiykarlılıqqa tekseriń.
 \\
\textbf{C2.} 
Eger \(X^{(n)} = \left( X_{1},...,X_{n} \right)\) tańlanba \({(\theta,\theta}^{2})\) parametrli normal bólistiriliwden \({\ g(x) = (x)}^{2}\ \)alınǵan bolsa, onda belgisiz \(\theta > 0\) parametr ushın momentler usılı bahasın funkciyası járdeminde tabıń.
 \\
\textbf{C3.} 
Eger \(X^{(n)} = \left( X_{1},...,X_{n} \right)\) tańlanba tıǵızlıq funkciyası
$f(x;\theta) = \frac{4x^{3}}{\sqrt{2\pi}\theta_{2}}\exp\left\{ - \frac{\left( x^{4} - \theta_{1} \right)^{2}}{2{\theta_{2}}^{2}} \right\},\ x \in R$
bolǵan bólistiriliwden alınǵan bolsa, onda belgisiz \(\left( \theta_{1},\theta_{2}^{2} \right)\) vektor parametrdiń shınlıqqa maksimal uqsaslıq usılı bahaların tabıń.
 \\

\end{tabular}
\vspace{1cm}


\begin{tabular}{m{17cm}}
\textbf{36-variant}
\newline

\textbf{T1.} 
Glivenko-Kantelli teoreması. (empirikalıq bólistiriw funkciyası, 1itimallıq penen jaqınlasıw)
 \\
\textbf{T2.} 
Statistikalıq gipotezalardı tekseriw (kritikalıq kóplik, 1 hám 2-túr qátelik)
 \\
\textbf{A1.} 
Kólemi \(n = 20\) ǵa teń bolǵan tańlanba berilgen: 3,7; 3,1; 4,8; 2,8; 3,1; 4,3; 3,7; 4,3; 2,4; 3,1; 2,4; 4,3; 3,1; 3,7; 4,8; 2,8; 2,4; 2,8; 2,4; 3,1. Bul tańlanbanıń statistikalıq bólistiriliwin tabıń.
 \\
\textbf{A2.} 
Kólemi \(n = 20\) ǵa teń bolǵan tańlanba berilgen: 3,8; 3,4; 4,8; 2,9; 3,4; 4,6; 3,8; 4,6; 2,1; 3,4; 2,1; 4,6; 3,4; 3,8; 4,8; 2,9; 2,1; 2,9; 2,1; 3,4. Bul tańlanbanıń empirikalıq bólistiriw funkciyasın tabıń.
 \\
\textbf{A3.} 
Joqarı matematika páninen 10 dana student test sınaqların tapsırǵan. Hárbir student 10 balǵa shekem toplawı múmkin. Eger test sınaqları nátiyjeleri boyınsha \{6, 5, 6, 9, 5, 7, 10, 5, 9, 8\} tańlanba alınǵan bolsa, onda tańlanba ortasha hám tańlanba dispersiyalardı tabıń.
 \\
\textbf{B1.} 
Eger normal bólistirilgen bas toplamnan alınǵan kólemi \(n = 25\) ǵa teń tańlanba boyınsha \(\overline{x} = 9\) tańlanba ortasha hám \({\overline{S}}^{2} = 0,64\) dúzetilgen tańlanba dispersiyalar tabılǵan bolsa, onda \(\gamma = 0,95\) isenimlilik penen belgisiz \(\theta\) matematikalıq kútiliw ushın isenimlilik interval dúziń.
 \\
\textbf{B2.} 
Kórsetkishli bólistiriw belgisiz \(\theta > 0\) parametri momentlar usulı bahasın tabıń.
 \\
\textbf{B3.} 
Eger \(X^{(n)} = \left( X_{1},...,X_{n} \right)\) tańlanba \(\left\lbrack - \theta,\theta^{2} \right\rbrack\) aralıqta teń ólshemli bólistiriliwden alınǵan bolsa, onda belgisiz \(\theta > 0\) parametrdiń shınlıqqa maksimal uqsaslıq usılı bahasın tabıń.
 \\
\textbf{C1.} 
Eger \(X^{(n)} = \left( X_{1},...,X_{n} \right)\) tańlanba \(\left( a,\theta^{2} \right)\) parametrli normal bólistiriliwden alınǵan bolsa (\(a -\)belgili), onda belgisiz \(\theta\) parametr ushın \(\sqrt{\frac{\pi}{2}}\left| \overline{x - a} \right|\) bahasın jıljımaǵanlıq hám tiykarlılıqqa tekseriń.
 \\
\textbf{C2.} 
Eger \(X^{(n)} = \left( X_{1},...,X_{n} \right)\) tańlanba \((\theta,2\theta)\) parametrli normal bólistiriliwden alınǵan bolsa, onda belgisiz \(\theta > 0\) parametr ushın momentler usılı bahasın \({\ g(x) = (x)}^{2}\) funkciyası járdeminde tabıń.
 \\
\textbf{C3.} 
Eger \(X^{(n)} = \left( X_{1},...,X_{n} \right)\) tańlanba tıǵızlıq funkciyası
$f(x;\theta) = \frac{\theta}{\sqrt{2\pi x^{3}}}e^{\frac{- \ \theta^{2}}{2x}},\ x \geq 0$
bolǵan bólistiriliwden alınǵan bolsa, onda belgisiz \(\theta > 0\) parametrdiń shınlıqqa maksimal uqsaslıq bahasın tabıń.
 \\

\end{tabular}
\vspace{1cm}


\begin{tabular}{m{17cm}}
\textbf{37-variant}
\newline

\textbf{T1.} 
Momentler usulı. (tańlanba momentleri, belgisiz parametrlerdi bahalaw).
 \\
\textbf{T2.} 
Statistikalıq gipotezalardı tekseriw (kritikalıq kóplik, 1 hám 2-túr qátelik).
 \\
\textbf{A1.} 
Kólemi \(n = 20\) ǵa teń bolǵan tańlanba berilgen: 1,5; -0,9; -2,4; -0,9; 0,7; 1,5; -0,9; -0,2; -2,4; 0,7; -2,4; 0,7; -0,9; 1,5; -1,7; -0,9; -0,2; 0,7; -1,7; -0,9. Bul tańlanbanıń statistikalıq bólistiriliwin tabıń.
 \\
\textbf{A2.} 
Kólemi \(n = 20\) ǵa teń bolǵan tańlanba berilgen: 1,9; -0,3; -2,7; -0,3; 0,6; 1,9; -0,3; -0,1; -2,7; 0,6; -2,7; 0,6; -0,3; 1,9; -1,8; -0,3; -0,1; 0,6; -1,8; -0,3. Bul tańlanbanıń empirikalıq bólistiriw funkciyasın tabıń.
 \\
\textbf{A3.} 
Joqarı matematika páninen 10 dana student test sınaqların tapsırǵan. Hárbir student 10 balǵa shekem toplawı múmkin. Eger test sınaqları nátiyjeleri boyınsha \{4, 6, 6, 9, 5, 8, 4, 7, 5, 6\} tańlanba alınǵan bolsa, onda tańlanba ortasha hám tańlanba dispersiyalardı tabıń.
 \\
\textbf{B1.} 
Eger normal bólistirilgen bas toplamnan alınǵan kólemi \(n = 10\) ǵa teń bolǵan tańlanba boyınsha \({\overline{S}}^{2} = 0,6\) dúzetilgen tańlanba dispersiya tabılǵan bolsa, onda \(\gamma = 0,95\) isenimlilik penen belgisiz \(\theta_{2}^{2}\) dispersiya ushın isenimlilik interval dúziń.
 \\
\textbf{B2.} 
Eger (0,-2,0,-2,3,-2,0,0,3,0,0,0,3,-2,0,0,-2,3,0,3) tańlanba tómende berilgen bólistiriliwden alınǵan bolsa, onda belgisiz \(\theta\) parametr ushın momentler usılı bahasın tabıń.
\begin{tabular}{|c|c|c|c|}
  \hline
$\xi$ & $- 2$  & $0$  & $3$ \\
\hline
\(P_{\theta}\) & \(\theta\) & \(1 - 2\theta\) & \(\theta\) \\
\hline
\end{tabular}
 \\
\textbf{B3.} 
Eger (4,8,5,3) tańlanba \(\left( a,\theta^{2} \right)\) parametrli normal bólistiriliwden alınǵan bolsa, onda belgisiz \(\theta^{2}\) parametrdiń shınlıqqa maksimal uqsaslıq bahasın tabıń.
 \\
\textbf{C1.} 
Eger \(X^{(n)} = \left( X_{1},...,X_{n} \right)\) tańlanba \(\theta\) parametrli kórsetkishli bólistiriliwinen alınǵan bolsa, onda belgisiz \(\theta\) parametr ushın \(\frac{1}{\overline{x}}\) bahasın jıljımaǵanlıq hám tiykarlılıqqa tekseriń.
 \\
\textbf{C2.} 
Eger \(X^{(n)} = \left( X_{1},...,X_{n} \right)\) tańlanba \({\lbrack\theta}_{1},\theta_{1}{+ \theta}_{2}\rbrack\) aralıqta teń ólshemli bólistiriliwden alınǵan bolsa, onda belgisiz \(\left( \theta_{1},\theta_{2} \right)\) vektor parametr ushın momentler usılı bahasın tabıń.
 \\
\textbf{C3.} 
Eger \(X^{(n)} = \left( X_{1},...,X_{n} \right)\) tańlanba \(\left( \theta,\theta^{2} \right)\) parametrli normal bólistiriliwden alınǵan bolsa, onda belgisiz \(\theta > 0\) parametrdiń shınlıqqa maksimal uqsaslıq bahasın tabıń.
 \\

\end{tabular}
\vspace{1cm}


\begin{tabular}{m{17cm}}
\textbf{38-variant}
\newline

\textbf{T1.} 
Tańlanba xarakteristikaları.(tańlanba orta, tańlanba dispersiya)
 \\
\textbf{T2.} 
Pirsonnıń xi-kvadrat kelisimlilik belgisi (Pirson teoreması).
 \\
\textbf{A1.} 
Kólemi \(n = 20\) ǵa teń bolǵan tańlanba berilgen:9,4; 6,8; -8,5; 9,4; 2,9; 9,4; -8,5; -6,4; 6,8; -8,5; 9,4; -6,4; 6,8; 9,4; 2,9; 9,4; -3,6; -8,5; 2,9; -6,4. Bul tańlanbanıń statistikalıq bólistiriliwin tabıń.
 \\
\textbf{A2.} 
Kólemi \(n = 20\) ǵa teń bolǵan tańlanba berilgen:9,1; 6,4; -8,6; 9,1; 2,3; 9,1; -8,6; -6,2; 6,4; -8,6; 9,1; -6,2; 6,4; 9,1; 2,3; 9,1; -3,9; -8,6; 2,3; -6,2. Bul tańlanbanıń empirikalıq bólistiriw funkciyasın tabıń.
 \\
\textbf{A3.} 
Joqarı matematika páninen 10 dana student test sınaqların tapsırǵan. Hárbir student 10 balǵa shekem toplawı múmkin. Eger test sınaqları nátiyjeleri boyınsha \{3, 7, 6, 4, 5, 4, 3, 7, 8, 3\} tańlanba alınǵan bolsa, onda tańlanba ortasha hám tańlanba dispersiyalardı tabıń.
 \\
\textbf{B1.} 
Eger ortasha kvadratlıq shetleniwi \(\sigma = 5\) bolǵan normal bólistirilgen bas toplamnan alınǵan kólemi \(n = 16\) ǵa teń tańlanba boyınsha \(\overline{x} = 3,6\) tańlanba ortasha mánisi tabılǵan bolsa, onda \(\gamma = 0,90\) isenimlilik penen belgisiz \(\theta\) matematikalıq kútiliwdi qaplaytuǵın isenimlilik intervalın dúziń.
 \\
\textbf{B2.} 
Eger (3,0,-2,0,-2,3,-2,0,0,3,0,0,0,3,-2,0,0,-2,3,0) tańlanba tómende berilgen bólistiriliwden alınǵan bolsa, onda belgisiz \(\left( \theta_{1},\theta_{2} \right)\) vektor parametr ushın momentler usılı bahalasın tabıń.
\begin{tabular}{|c|c|c|c|}
  \hline
$\xi$ &
$- 2$ &
$0$ &
$3$\\
\hline
\(P_{\theta}\) & \({2\theta}_{1}\) & \(0,5 + \theta_{1} + \theta_{2}\) & \(\theta_{2}\) \\
\hline
\end{tabular}
 \\
\textbf{B3.} 
Eger \(x_{1} = 1,1;\ x_{2} = 2,7;\ldots;x_{100} = 1,5\) tańlanba \(\theta\) parametrli kórsetkishli bólistiriliwden alınǵan bolıp, \(\sum_{k = 1}^{100}x_{k} = 200\) bolsa, onda belgisiz \(\theta\) parametrdiń shınlıqqa maksimal uqsaslıq bahasın tabıń.
 \\
\textbf{C1.} 
Eger \(X^{(n)} = \left( X_{1},...,X_{n} \right)\) tańlanba \(\frac{1}{\sqrt{\theta}}\) parametrli kórsetkishli bólistiriliwinen alınǵan bolsa, onda belgisiz \(\theta\) parametr ushın \((\overline{x})^{2}\) bahasın jıljımaǵanlıq hám tiykarlılıqqa tekseriń.
 \\
\textbf{C2.} 
Eger \(X^{(n)} = \left( X_{1},...,X_{n} \right)\) tańlanba \({\ \ (a,\theta}^{2})\ \)parametrli normal bólistiriliwden alınǵan bolsa (\(\alpha -\)belgili), onda belgisiz \({\ \theta}^{2}\) parametr ushın momentler usılı bahasın \({\ g(x) = (x - a)}^{2}\) funkciyası járdeminde tabıń.
 \\
\textbf{C3.} 
Eger \(X^{(n)} = \left( X_{1},...,X_{n} \right)\) tańlanba tıǵızlıq funkciyası
$f(x;\theta) = \left\{ \begin{matrix}
3x^{2}\theta^{- 3}e^{- \ \left( \frac{x}{\theta} \right)^{3}},\ \ x \geq 0, \\
\ \ \ \ \ \ \ \ \ \ \ \ \ \ 0,\ \ \ \ \ \ \ \ \ x < 0
\end{matrix} \right.\ $
bolǵan bólistiriliwden alınǵan bolsa, onda belgisiz \(\theta > 0\) parametrdiń shınlıqqa maksimal uqsaslıq bahasın tabıń.
 \\

\end{tabular}
\vspace{1cm}


\begin{tabular}{m{17cm}}
\textbf{39-variant}
\newline

\textbf{T1.} 
Poligon hám gistogramma(salıstirmalı jiyilik, intervallıq qatar, grafik)
 \\
\textbf{T2.} 
Statistikalıq baha qásiyetleri. (Jıljımaytuǵın, tiykarlı, effektiv)
 \\
\textbf{A1.} 
Kólemi \(n = 20\) ǵa teń bolǵan tańlanba berilgen: 6,2; -5,3; 7,2; 3,7; -2,2; 6,2; 3,7; -7,6; 3,7; 7,2; 6,2; -5,3; -7,6; -5,3; -7,6; 6,2; 7,2; -2,2; -7,6; 7,2. Bul tańlanbanıń statistikalıq bólistiriliwin tabıń.
 \\
\textbf{A2.} 
Kólemi \(n = 20\) ǵa teń bolǵan tańlanba berilgen: 6,1; -5,8; 7,9; 3,5; -2,5; 6,1; 3,5; -7,2; 3,5; 7,9; 6,1; -5,8; -7,2; -5,8; -7,2; 6,1; 7,9; -2,5; -7,2; 7,9. Bul tańlanbanıń empirikalıq bólistiriw funkciyasın tabıń.
 \\
\textbf{A3.} 
Joqarı matematika páninen 10 dana student test sınaqların tapsırǵan. Hárbir student 10 balǵa shekem toplawı múmkin. Eger test sınaqları nátiyjeleri boyınsha \{10, 8, 6, 5, 4, 8, 10, 7, 5, 7\} tańlanba alınǵan bolsa, onda tańlanba ortasha hám tańlanba dispersiyalardı tabıń.
 \\
\textbf{B1.} 
Eger normal bólistirilgen bas toplamnan alınǵan kólemi \(n = 16\) ǵa teń tańlanba boyınsha \(\overline{x} = 15,2\) tańlanba ortasha hám \({\overline{S}}^{2} = 0,81\) dúzetilgen tańlanba dispersiyalar tabılǵan bolsa, onda \(\gamma = 0,90\) isenimlilik penen belgisiz \(\theta\) matematikalıq kútiliw ushın isenimlilik interval dúziń.
 \\
\textbf{B2.} 
Eger tıǵızlıq funkciyası \(f(x) = \frac{2x}{\theta}e^{- \frac{x^{2}}{\theta}},\ \ x \geq 0\) kóriniske iye bolsa, onda \(\theta\) parametr momentler usulı bahasın tabıń.
 \\
\textbf{B3.} 
Eger \(X^{(n)} = \left( X_{1},...,X_{n} \right)\) tańlanba tıǵızlıq funkciyası \(f(x;\theta) = \frac{2x}{\theta}e^{- \frac{x^{2}}{\theta}},\ x \geq 0\). bolǵan bólistiriliwden alınǵan bolsa, onda belgisiz \(\theta > 0\) parametrdiń shınlıqqa maksimal uqsaslıq bahasın tabıń.
 \\
\textbf{C1.} 
Eger \(X^{(n)} = \left( X_{1},...,X_{n} \right)\) tańlanba \(\sqrt{\theta}\) parametrli Bernulli bólistiriliwinen alınǵan bolsa, onda belgisiz \(\theta\) parametr ushın \((\overline{x})^{2}\) bahasın jıljımaǵanlıq hám tiykarlılıqqa tekseriń.
 \\
\textbf{C2.} 
Eger \(X^{(n)} = \left( X_{1},...,X_{n} \right)\) tańlanba tıǵızlıq funkciyası
$
{f(x,\theta) = \left\{ \begin{array}{r}
e^{\theta - x},\ \ \ x \geq \theta, \\
0,\ \ \ x < \theta
\end{array} \right.\ }$
bolǵan bólistiriliwden alınǵan bolsa, onda belgisiz \(\theta\) parametr ushın momentler usılı bahasın tabıń.
 \\
\textbf{C3.} 
Eger \(X^{(n)} = \left( X_{1},...,X_{n} \right)\) tańlanba tıǵızlıq funkciyası
$f(x;\theta) = \left\{ \begin{matrix}
e^{\theta - x},\ \ x \geq \theta, \\
\ \ 0,\ \ \ \ \ \ \ x < \theta
\end{matrix} \right.\ $
bolǵan bólistiriliwden alınǵan bolsa, onda belgisiz \(\theta\) parametrdiń shınlıqqa maksimal uqsaslıq bahasın tabıń.
 \\

\end{tabular}
\vspace{1cm}


\begin{tabular}{m{17cm}}
\textbf{40-variant}
\newline

\textbf{T1.} Matematikalıq statistikanıń tiykarǵı máseleleri. (Statistikalıq maǵlıwmatlar, gruppalaw)
 \\
\textbf{T2.} 
Sızıqlı korrelyaciya teńlemesi (anıqlaması, regressiya tuwrı sızıǵınıń tańlanba teńlemeleri)
 \\
\textbf{A1.} 
Kólemi \(n = 20\) ǵa teń bolǵan tańlanba berilgen: 9,6; 1,5; 7,4; 9,6; 2,8; 1,5; 6,3; 1,5; 9,6; 6,3; 2,8; 4,1; 6,3; 9,6; 1,5; 1,5; 6,3; 7,4; 4,1; 7,4. Bul tańlanbanıń statistikalıq bólistiriliwin tabıń.
 \\
\textbf{A2.} 
Kólemi \(n = 20\) ǵa teń bolǵan tańlanba berilgen: 9,8; 1,2; 7,1; 9,8; 2,9; 1,2; 6,7; 1,2; 9,8; 6,7; 2,9; 4,6; 6,7; 9,8; 1,2; 1,2; 6,7; 7,1; 4,6; 7,1. Bul tańlanbanıń empirikalıq bólistiriw funkciyasın tabıń.
 \\
\textbf{A3.} 
Joqarı matematika páninen 10 dana student test sınaqların tapsırǵan. Hárbir student 10 balǵa shekem toplawı múmkin. Eger test sınaqları nátiyjeleri boyınsha \{9, 10, 5, 6, 4, 8, 4, 6, 10, 8\} tańlanba alınǵan bolsa, onda tańlanba ortasha hám tańlanba dispersiyalardı tabıń.
 \\
\textbf{B1.} 
Eger normal bólistirilgen bas toplamnan alınǵan kólemi \(n = 10\) ǵa teń bolǵan tańlanba boyınsha \({\overline{S}}^{2} = 0,45\) dúzetilgen tańlanba dispersiya tabılǵan bolsa, onda \(\gamma = 0,95\) isenimlilik penen belgisiz \(\theta_{2}^{2}\) dispersiya ushın isenimlilik interval dúziń.
 \\
\textbf{B2.} 
\(\lbrack\theta_{1},\theta_{2}\rbrack\) aralıqta teń ólshewli bólistiriw parametrleri ushın momentler usulı bahaların tabıń.
 \\
\textbf{B3.} 
Eger \(X^{(n)} = \left( X_{1},...,X_{n} \right)\) tańlanba \(\left\lbrack - \theta,\theta^{2} \right\rbrack\) aralıqta teń ólshemli bólistiriliwden alınǵan bolsa, onda belgisiz \(\theta > 0\) parametrdiń shınlıqqa maksimal uqsaslıq usılı bahasın tabıń.
 \\
\textbf{C1.} 
Eger \(X^{(n)} = \left( X_{1},...,X_{n} \right)\) tańlanba \(\theta\) parametrli Bernulli bólistiriliwinen alınǵan bolsa, onda belgisiz \(\theta\) parametr ushın \(X_{n}\) bahasın jıljımaǵanlıq hám tiykarlılıqqa tekseriń.
 \\
\textbf{C2.} 
Eger \(X^{(n)} = \left( X_{1},...,X_{n} \right)\) tańlanba tıǵızlıq funkciyası
$f(x,\theta) = \frac{2x}{\theta^{2}},x \in \lbrack 0,\theta\rbrack$
bolǵan bólistiriliwden alınǵan bolsa, onda belgisiz \(\theta\) parametr ushın momentler usılı bahasın tabıń.
 \\
\textbf{C3.} 
\(\ f(x,\theta) = \frac{4x^{3}}{\theta_{2}\sqrt{2\pi}}\exp\left\{ - \frac{\left( x^{4} - \theta_{1} \right)^{2}}{2{\theta_{2}}^{2}} \right\}\) model ushın \(\theta_{1}\) hám \({\theta_{2}}^{2}\) parametrler haqıyqatqa maksimal uqsaslıq usılı bahaları tabılsın.
 \\

\end{tabular}
\vspace{1cm}


\begin{tabular}{m{17cm}}
\textbf{41-variant}
\newline

\textbf{T1.} 
Tańlanba xarakteristikaları.(tańlanba orta, tańlanba dispersiya)
 \\
\textbf{T2.} 
Momentler usulı. (tańlanba momentleri, belgisiz parametrlerdi bahalaw).
 \\
\textbf{A1.} 
Kólemi \(n = 20\) ǵa teń bolǵan tańlanba berilgen:1,8; -8,4; 7,3; 4,7; -3,9; 1,8; 4,7; -10,4; -8,4; 7,3; -10,4; 4,7; -8,4; 1,8; 4,7; -10,4; 7,3; -3,9; 4,7; -8,4. Bul tańlanbanıń statistikalıq bólistiriliwin tabıń.
 \\
\textbf{A2.} 
Kólemi \(n = 20\) ǵa teń bolǵan tańlanba berilgen:1,6; -8,3; 7,6; 4,2; -3,1; 1,6; 4,2; -10,5; -8,3; 7,6; -10,5; 4,2; -8,3; 1,6; 4,2; -10,5; 7,6; -3,1; 4,2; -8,3. Bul tańlanbanıń empirikalıq bólistiriw funkciyasın tabıń.
 \\
\textbf{A3.} 
Joqarı matematika páninen 10 dana student test sınaqların tapsırǵan. Hárbir student 10 balǵa shekem toplawı múmkin. Eger test sınaqları nátiyjeleri boyınsha \{9, 3, 6, 3, 7, 6, 4, 6, 10, 6\} tańlanba alınǵan bolsa, onda tańlanba ortasha hám tańlanba dispersiyalardı tabıń.
 \\
\textbf{B1.} 
Eger ortasha kvadratlıq shetleniwi \(\sigma = 2\) bolǵan normal bólistirilgen bas toplamnan alınǵan kólemi \(n = 18\) ǵa teń tańlanba boyınsha \(\overline{x} = 5,2\) tańlanba ortasha mánisi tabılǵan bolsa, onda \(\gamma = 0,90\) isenimlilik penen belgisiz \(\theta\) matematikalıq kútiliwdi qaplaytuǵın isenimlilik intervalın dúziń.
 \\
\textbf{B2.} 
Eger (3,-2,-2,0,-2,-2,-2,0,-2,3,-2,0,3,0,3,-2,0,-2,3,-2,-2,-2,-2,3,3,3,-2,-2,3,3) tańlanba tómende berilgen bólistiriliwden alınǵan bolsa, onda belgisiz \(\theta\) parametr ushın momentler usılı bahasın \(g(x) = |x|\) funkciyası járdeminde tabıń.
\begin{tabular}{|c|c|c|c|}
  \hline
$\xi$ &
$- 2$ &
$0$ &
$3$ \\
\hline
\(P_{\theta}\) & \(3\theta\) & \(1 - 5\theta\) & \(2\theta\) \\
\hline
\end{tabular}
 \\
\textbf{B3.} 
Eger \(X^{(n)} = \left( X_{1},...,X_{n} \right)\) tańlanba \(\theta\) parametrli Bernulli bólistiriliwinen alınǵan bolsa, onda belgisiz \(\theta\) parametrdiń shınlıqqa maksimal uqsaslıq usılı bahasın tabıń.
 \\
\textbf{C1.} 
Eger \(X^{(n)} = \left( X_{1},...,X_{n} \right)\) tańlanba \(\theta\) parametrli Bernulli bólistiriliwinen alınǵan bolsa, onda belgisiz \(\theta(1 - \theta)\) parametr ushın \(X_{1}\left( 1 - X_{n} \right)\) bahasın jıljımaǵanlıq hám tiykarlılıqqa tekseriń.
 \\
\textbf{C2.} 
Eger \(X^{(n)} = \left( X_{1},...,X_{n} \right)\) tańlanba \({(\theta,\theta}^{2})\) parametrli normal bólistiriliwden \({\ g(x) = (x)}^{2}\ \)alınǵan bolsa, onda belgisiz \(\theta > 0\) parametr ushın momentler usılı bahasın funkciyası járdeminde tabıń.
 \\
\textbf{C3.} 
\(\ f(x,\theta) = \frac{e^{x}}{\sqrt{2\pi}}\exp\left\{ - \frac{\left( e^{x} - \theta \right)^{2}}{2} \right\}\) model ushın \(\theta\) parametri haqıyqatqa maksimal uqsaslıq usılı bahası tabılsın.
 \\

\end{tabular}
\vspace{1cm}


\begin{tabular}{m{17cm}}
\textbf{42-variant}
\newline

\textbf{T1.} 
Tańlanba xarakteristikalar. (Variaciyalıq qatar, salıstırmalı jiyilik).
 \\
\textbf{T2.} 
Statistikalıq gipotezalardı tekseriw (kritikalıq kóplik, 1 hám 2-túr qátelik)
 \\
\textbf{A1.} 
Kólemi \(n = 20\) ǵa teń bolǵan tańlanba berilgen: 2,7; -13,5; 1,2; 2,7; 1,2; 4,9; -9,5; 1,2; 2,7; 4,9; -9,5; 2,7; -3,5; 1,2; 2,7; 4,9; -3,5; 2,7; 4,9; 1,2;. Bul tańlanbanıń statistikalıq bólistiriliwin tabıń.
 \\
\textbf{A2.} 
Kólemi \(n = 20\) ǵa teń bolǵan tańlanba berilgen: 2,8; -13,9; 1,9; 2,8; 1,9; 4,3; -9,4; 1,9; 2,8; 4,3; -9,4; 2,8; -3,7; 1,9; 2,8; 4,3; -3,7; 2,8; 4,3; 1,9. Bul tańlanbanıń empirikalıq bólistiriw funkciyasın tabıń.
 \\
\textbf{A3.} 
Joqarı matematika páninen 10 dana student test sınaqların tapsırǵan. Hárbir student 10 balǵa shekem toplawı múmkin. Eger test sınaqları nátiyjeleri boyınsha \{10, 7, 5, 9, 3, 8, 10, 7, 8, 3\} tańlanba alınǵan bolsa, onda tańlanba ortasha hám tańlanba dispersiyalardı tabıń.
 \\
\textbf{B1.} 
Eger normal bólistirilgen bas toplamnan alınǵan kólemi \(n = 36\) ǵa teń tańlanba boyınsha \(\overline{x} = 20,2\) tańlanba ortasha hám \({\overline{S}}^{2} = 0,81\) dúzetilgen tańlanba dispersiyalar tabılǵan bolsa, onda \(\gamma = 0,95\) isenimlilik penen belgisiz \(\theta\) matematikalıq kútiliw ushın isenimlilik interval dúziń.
 \\
\textbf{B2.} 
\(\lbrack 0,\theta\rbrack\) aralıqta teń ólshewli bólistirilgen \(\theta\) parametri ushın momentler usulı bahasın tabıń.
 \\
\textbf{B3.} 
Eger (0,1,2,0) tańlanba tómende berilgen bólistiriliwden alınǵan bolsa, onda belgisiz \(\theta\) parametrdiń shınlıqqa maksimal uqsaslıq bahasın tabıń.
\begin{tabular}{|c|c|c|c|}
  \hline
$\xi$
&
$0$
&
$1$
&
$2$\\
\hline
\(P_{\theta}\) & \(\theta\) & \(2\theta\) & \(1 - 3\theta\) \\
\hline
\end{tabular}
 \\
\textbf{C1.} 
Eger \(X^{(n)} = \left( X_{1},...,X_{n} \right)\) tańlanba \(\theta\) parametrli Bernulli bólistiriliwinen alınǵan bolsa, onda belgisiz \(\theta^{2}\) parametr ushın \(X_{1}X_{n}\) bahasın jıljımaǵanlıq hám tiykarlılıqqa tekseriń.
 \\
\textbf{C2.} 
Eger \(X^{(n)} = \left( X_{1},...,X_{n} \right)\) tańlanba {[}\(0,2\theta\rbrack\) aralıqta teń ólshemli bólistiriliwden alınǵan bolsa, onda belgisiz \(\theta > 0\) parametr ushın momentler usılı bahasın tabıń.
 \\
\textbf{C3.} 
Eger \(X^{(n)} = \left( X_{1},...,X_{n} \right)\) tańlanba tıǵızlıq funkciyası
$f(x;\theta) = \frac{1}{2}e^{- \ |x - \theta|},\ x \in R$
bolǵan Laplas bólistiriliwinen alınǵan bolsa, onda belgisiz \(\theta \in R\) parametrdiń shınlıqqa maksimal uqsaslıq bahasın tabıń.
 \\

\end{tabular}
\vspace{1cm}


\begin{tabular}{m{17cm}}
\textbf{43-variant}
\newline

\textbf{T1.} 
Empirikalıq bólistiriw funkciyası. (Tańlanba, eksperiment)
 \\
\textbf{T2.} 
Statistikalıq baha qásiyetleri. (Jıljımaytuǵın, tiykarlı, effektiv)
 \\
\textbf{A1.} 
Kólemi \(n = 20\) ǵa teń bolǵan tańlanba berilgen: 9,9; 5,7; 3,2; 2,8; 5,7; 9,9; 7,5; 3,7; 9,9; 3,2; 2,8; 3,7; 7,5; 5,7; 3,2; 2,8; 7,5; 3,2; 9,9; 7,5. Bul tańlanbanıń statistikalıq bólistiriliwin tabıń.
 \\
\textbf{A2.} 
Kólemi \(n = 20\) ǵa teń bolǵan tańlanba berilgen: 9,7; 5,2; 3,2; 2,4; 5,2; 9,7; 7,5; 3,7; 9,7; 3,2; 2,4; 3,7; 7,5; 5,2; 3,2; 2,4; 7,5; 3,2; 9,7; 7,5. Bul tańlanbanıń empirikalıq bólistiriw funkciyasın tabıń.
 \\
\textbf{A3.} 
Joqarı matematika páninen 10 dana student test sınaqların tapsırǵan. Hárbir student 10 balǵa shekem toplawı múmkin. Eger test sınaqları nátiyjeleri boyınsha \{1, 6, 2, 6, 3, 6, 4, 6, 10, 6\} tańlanba alınǵan bolsa, onda tańlanba ortasha hám tańlanba dispersiyalardı tabıń.
 \\
\textbf{B1.} 
Eger normal bólistirilgen bas toplamnan alınǵan kólemi \(n = 10\) ǵa teń bolǵan tańlanba boyınsha \({\overline{S}}^{2} = 0,7\) dúzetilgen tańlanba dispersiya tabılǵan bolsa, onda \(\gamma = 0,95\) isenimlilik penen belgisiz \(\theta_{2}^{2}\) dispersiya ushın isenimlilik interval dúziń.
 \\
\textbf{B2.} 
Puasson bólistiriliwi belgisiz \(\theta > 0\) parametri momentlar usuli bahasin tabıń.
 \\
\textbf{B3.} 
Eger \(X^{(n)} = \left( X_{1},...,X_{n} \right)\) tańlanba \(\lbrack - \theta,\theta\rbrack\) aralıqta teń ólshemli bólistiriliwden alınǵan bolsa, onda belgisiz \(\theta > 0\) parametrdiń shınlıqqa maksimal uqsaslıq usılı bahasın tabıń.
 \\
\textbf{C1.} 
Eger \(X^{(n)} = \left( X_{1},...,X_{n} \right)\) tańlanba \((\alpha,\theta)\) parametrli Veybull bólistiriliwinen alınǵan bolsa (\(\alpha -\)belgili), onda belgisiz \(\theta\) parametr ushın \(\frac{1}{\overline{x^{\alpha}}}\) bahasın jıljımaǵanlıq hám tiykarlılıqqa tekseriń.
 \\
\textbf{C2.} 
Eger \(X^{(n)} = \left( X_{1},...,X_{n} \right)\) tańlanba \((\theta,2\theta)\) parametrli normal bólistiriliwden alınǵan bolsa, onda belgisiz \(\theta > 0\) parametr ushın momentler usılı bahasın \({\ g(x) = (x)}^{2}\) funkciyası járdeminde tabıń.
 \\
\textbf{C3.} 
Eger \(X^{(n)} = \left( X_{1},...,X_{n} \right)\) tańlanba tıǵızlıq funkciyası
$f(x;\theta) = \frac{e^{x}}{\sqrt{2\pi}}\exp\left\{ - \frac{\left( e^{x} - \theta \right)^{2}}{2} \right\},\ x \in R$
bolǵan bólistiriliwden alınǵan bolsa, onda belgisiz \(\theta\) parametrdiń shınlıqqa maksimal uqsaslıq bahasın tabıń.
 \\

\end{tabular}
\vspace{1cm}


\begin{tabular}{m{17cm}}
\textbf{44-variant}
\newline

\textbf{T1.} Matematikalıq statistikanıń tiykarǵı máseleleri. (Statistikalıq maǵlıwmatlar, gruppalaw)
 \\
\textbf{T2.} 
Pirsonnıń xi-kvadrat kelisimlilik belgisi (Pirson teoreması).
 \\
\textbf{A1.} 
Kólemi \(n = 20\) ǵa teń bolǵan tańlanba berilgen: 3,6; 1,1; -1,8; 0,4; 3,6; 0; 5,3; 1,1; 0; -1,8; 3,6; 0,4; 1,1; 0; 0,4; 1,1; 3,6; -1,8; 3,6; 0. Bul tańlanbanıń statistikalıq bólistiriliwin tabıń.
 \\
\textbf{A2.} 
Kólemi \(n = 20\) ǵa teń bolǵan tańlanba berilgen: 3,2; 1,8; -1,1; 0,9; 3,2; 0; 5,6; 1,8; 0; -1,1; 3,2; 0,9; 1,8; 0; 0,9; 1,8; 3,2; -1,1; 3,2; 0. Bul tańlanbanıń empirikalıq bólistiriw funkciyasın tabıń.
 \\
\textbf{A3.} 
Joqarı matematika páninen 10 dana student test sınaqların tapsırǵan. Hárbir student 10 balǵa shekem toplawı múmkin. Eger test sınaqları nátiyjeleri boyınsha \{2, 7, 3, 7, 6, 7, 4, 7, 7, 10\} tańlanba alınǵan bolsa, onda tańlanba ortasha hám tańlanba dispersiyalardı tabıń.
 \\
\textbf{B1.} 
Eger ortasha kvadratlıq shetleniwi \(\sigma = 3\) bolǵan normal bólistirilgen bas toplamnan alınǵan kólemi \(n = 14\) ǵa teń tańlanba boyınsha \(\overline{x} = 5,5\) tańlanba ortasha mánisi tabılǵan bolsa, onda \(\gamma = 0,90\) isenimlilik penen belgisiz \(\theta\) matematikalıq kútiliwdi qaplaytuǵın isenimlilik intervalın dúziń.
 \\
\textbf{B2.} 
Eger \(X^{(n)} = \left( X_{1},...,X_{n} \right)\) tańlanba \(\theta\) parametrli Bernulli bólistiriliwinen alınǵan bolsa, onda belgisiz \(\theta\) parametr ushın momentler usılı bahasın tabıń.
 \\
\textbf{B3.} 
Eger (-1,-1,0,-1,0,-1,-1,5,-1,0,-1,0,5,-1,-1,-1,5,-1,-1,-1,5,0,-1,-1,5) tańlanba tómende berilgen bólistiriliwden alınǵan bolsa, onda belgisiz \(\theta\) parametrdiń shınlıqqa maksimal uqsaslıq usılı bahasın tabıń.
\begin{tabular}{|c|c|c|c|}
  \hline
$\xi$
&
$- 1$
&
$0$
&
$5$\\
\hline
\(P_{\theta}\) & \(1 - \theta\) & \(\theta/2\) & \(\theta/2\ \) \\
\hline
\end{tabular}
 \\
\textbf{C1.} 
Eger \(X^{(n)} = \left( X_{1},...,X_{n} \right)\) tańlanba \(\theta\) parametrli geometriyalıq bólistiriliwden alınǵan bolsa, onda belgisiz \(\theta\) parametr ushın \(\frac{1}{(1 + \overline{x})}\) bahasın jıljımaǵanlıq hám tiykarlılıqqa tekseriń.
 \\
\textbf{C2.} 
Eger \(X^{(n)} = \left( X_{1},...,X_{n} \right)\) tańlanba \({(\theta,\theta}^{2})\ \) parametrli normal bólistiriliwden alınǵan bolsa, onda belgisiz \(\theta > 0\) parametr ushın momentler usılı bahasın tabıń.
 \\
\textbf{C3.} 
Eger \(X^{(n)} = \left( X_{1},...,X_{n} \right)\) tańlanba tıǵızlıq funkciyası
$f(x;\theta) = \frac{\theta}{\sqrt{2\pi x^{3}}}e^{\frac{- \ \theta^{2}}{2x}},\ x \geq 0$
bolǵan bólistiriliwden alınǵan bolsa, onda belgisiz \(\theta > 0\) parametrdiń shınlıqqa maksimal uqsaslıq bahasın tabıń.
 \\

\end{tabular}
\vspace{1cm}


\begin{tabular}{m{17cm}}
\textbf{45-variant}
\newline

\textbf{T1.} 
Tańlanba momentleri (\(k -\)tártipli baslanǵısh, baslanǵısh absolyut, oraylıq hám oraylıq absolyut momentler).
 \\
\textbf{T2.} 
Haqiyqatqa maksimal uqsaslıq usulı. (haqiyqatqa maksimal uqsaslıq funkciyası, belgisiz parametrlerdi bahalaw).
 \\
\textbf{A1.} 
Kólemi \(n = 20\) ǵa teń bolǵan tańlanba berilgen: 7,1; 3,9; 6,3; 4,6; 7,1; 2,3; 6,3; 3,9; 4,6; 7,1; 2,3; 3,9; 7,6; 2,3; 4,6; 3,9; 2,3; 3,9; 7,6; 4,6. Bul tańlanbanıń statistikalıq bólistiriliwin tabıń.
 \\
\textbf{A2.} 
Kólemi \(n = 20\) ǵa teń bolǵan tańlanba berilgen: 7,9; 3,8; 6,1; 4,2; 7,9; 2,4; 6,1; 3,8; 4,2; 7,9; 2,4; 3,8; 10,2; 2,4; 4,2; 3,8; 2,4; 3,8; 10,2; 4,2. Bul tańlanbanıń empirikalıq bólistiriw funkciyasın tabıń.
 \\
\textbf{A3.} 
Joqarı matematika páninen 10 dana student test sınaqların tapsırǵan. Hárbir student 10 balǵa shekem toplawı múmkin. Eger test sınaqları nátiyjeleri boyınsha \{9, 8, 6, 8, 6, 4, 5, 4, 7, 4\} tańlanba alınǵan bolsa, onda tańlanba ortasha hám tańlanba dispersiyalardı tabıń.
 \\
\textbf{B1.} 
Eger normal bólistirilgen bas toplamnan alınǵan kólemi \(n = 49\) ǵa teń tańlanba boyınsha \(\overline{x} = 14,2\) tańlanba ortasha hám \({\overline{S}}^{2} = 0,64\) dúzetilgen tańlanba dispersiyalar tabılǵan bolsa, onda \(\gamma = 0,95\) isenimlilik penen belgisiz \(\theta\) matematikalıq kútiliw ushın isenimlilik interval dúziń.
 \\
\textbf{B2.} 
\(\lbrack\theta_{1},\theta_{2}\rbrack\) aralıqta teń ólshewli bólistiriw parametrleri ushın momentler usulı bahaların tabıń.
 \\
\textbf{B3.} 
\(\ f(x) = \frac{2x}{\theta}e^{- \frac{x^{2}}{\theta}},\ \ x \geq 0\) model ushın \(\theta\) parametri haqıyqatqa maksimal uqsaslıq usılı bahası tabılsın.
 \\
\textbf{C1.} 
Eger \(X^{(n)} = \left( X_{1},...,X_{n} \right)\) tańlanba \(\theta\) parametrli Puasson bólistiriliwinen alınǵan bolsa, onda belgisiz \(\theta\) parametr ushın \(\frac{n + 3}{n + 4}\overline{x}\) bahasın jıljımaǵanlıq hám tiykarlılıqqa tekseriń.
 \\
\textbf{C2.} 
Eger \(X^{(n)} = \left( X_{1},...,X_{n} \right)\) tańlanba \(\left( \theta_{1},\theta_{2} \right)\) parametrli gamma bólistiriliwden alınǵan bolsa, onda belgisiz \(\left( \theta_{1},\theta_{2} \right)\) vektor parametr ushın momentler usılı bahasın tabıń.
 \\
\textbf{C3.} 
Eger \(X^{(n)} = \left( X_{1},...,X_{n} \right)\) tańlanba \(\left\lbrack \theta_{1},\theta_{2} \right\rbrack\) aralıqta teń ólshemli bólistiriliwden alınǵan bolsa, onda belgisiz \(\left( \theta_{1},\theta_{2} \right)\) vektor parametrdiń shınlıqqa maksimal uqsaslıq bahasın tabıń.
 \\

\end{tabular}
\vspace{1cm}


\begin{tabular}{m{17cm}}
\textbf{46-variant}
\newline

\textbf{T1.} 
Neyman-Pirson teoreması
 \\
\textbf{T2.} 
Isenimlilik intervalların qurıw. Anıq isenimli intervallar
 \\
\textbf{A1.} 
Kólemi \(n = 20\) ǵa teń bolǵan tańlanba berilgen: 0,6; -3,8; -2,3; -4,3; 2,8; 4,7; -2,3; 0,6; -3,8; 2,8; -2,3; -4,3; 0,6; -2,3; 2,8; -3,8; -4,3; -2,3; 2,8; -3,8. Bul tańlanbanıń statistikalıq bólistiriliwin tabıń.
 \\
\textbf{A2.} 
Kólemi \(n = 20\) ǵa teń bolǵan tańlanba berilgen: 0,7; -3,1; -2,3; -4,8; 2,6; 4,9; -2,3; 0,7; -3,1; 2,6; -2,3; -4,8; 0,7; -2,3; 2,6; -3,1; -4,8; -2,3; 2,6; -3,1. Bul tańlanbanıń empirikalıq bólistiriw funkciyasın tabıń.
 \\
\textbf{A3.} 
Joqarı matematika páninen 10 dana student test sınaqların tapsırǵan. Hárbir student 10 balǵa shekem toplawı múmkin. Eger test sınaqları nátiyjeleri boyınsha \{10, 4, 6, 5, 5, 4, 10, 7, 9, 10\} tańlanba alınǵan bolsa, onda tańlanba ortasha hám tańlanba dispersiyalardı tabıń.
 \\
\textbf{B1.} 
Eger normal bólistirilgen bas toplamnan alınǵan kólemi \(n = 8\) ǵa teń bolǵan tańlanba boyınsha \({\overline{S}}^{2} = 0,35\) dúzetilgen tańlanba dispersiya tabılǵan bolsa, onda \(\gamma = 0,90\) isenimlilik penen belgisiz \(\theta_{2}^{2}\) dispersiya ushın isenimlilik interval dúziń.
 \\
\textbf{B2.} 
Eger \(X^{(n)} = \left( X_{1},...,X_{n} \right)\) tańlanba \(\theta\) parametrli kórsetkishli bólistiriliwden alınǵan bolsa, onda belgisiz \(\theta\) parametr ushın momentler usılı bahasın tabıń.
 \\
\textbf{B3.} 
Eger \(X^{(n)} = \left( X_{1},...,X_{n} \right)\) tańlanba \(\theta\) parametrli kórsetkishli bólistiriliwden alınǵan bolsa, onda belgisiz \(\theta\) parametrdiń shınlıqqa maksimal uqsaslıq usılı bahasın tabıń.
 \\
\textbf{C1.} 
Eger \(X^{(n)} = \left( X_{1},...,X_{n} \right)\) tańlanba \(\theta\) parametrli Puasson bólistiriliwinen alınǵan bolsa, onda belgisiz \(\theta\) parametr ushın \(\frac{X_{1} + X_{3}}{2}\) bahasın jıljımaǵanlıq hám tiykarlılıqqa tekseriń.
 \\
\textbf{C2.} 
Eger \(X^{(n)} = \left( X_{1},...,X_{n} \right)\) tańlanba \(\frac{1}{\theta}\) parametrli kórsetkishli bólistiriliwden alınǵan bolsa, onda belgisiz \(\theta\) parametr ushın momentler usılı bahasın\({\ g(x) = x}^{k},\) \(k \in N\)funkciyası járdeminde tabıń.
 \\
\textbf{C3.} 
Eger \(X^{(n)} = \left( X_{1},...,X_{n} \right)\) tańlanba tıǵızlıq funkciyası
$f(x;\theta) = \frac{4x^{3}}{\sqrt{2\pi}\theta_{2}}\exp\left\{ - \frac{\left( x^{4} - \theta_{1} \right)^{2}}{2{\theta_{2}}^{2}} \right\},\ x \in R$
bolǵan bólistiriliwden alınǵan bolsa, onda belgisiz \(\left( \theta_{1},\theta_{2}^{2} \right)\) vektor parametrdiń shınlıqqa maksimal uqsaslıq usılı bahaların tabıń.
 \\

\end{tabular}
\vspace{1cm}


\begin{tabular}{m{17cm}}
\textbf{47-variant}
\newline

\textbf{T1.} 
Momentler usulı. (tańlanba momentleri, belgisiz parametrlerdi bahalaw).
 \\
\textbf{T2.} 
Sızıqlı korrelyaciya teńlemesi (anıqlaması, regressiya tuwrı sızıǵınıń tańlanba teńlemeleri)
 \\
\textbf{A1.} 
Kólemi \(n = 20\) ǵa teń bolǵan tańlanba berilgen: 8,9; 2,7; 1,7; 2,2; 5,6; 1,7; 5,6; 2,7; 1,7; 2,2; 5,6; 8,9; 1,7; 2,2; 1,7; 2,7; 1,7; 5,6; 6,1; 8,9. Bul tańlanbanıń statistikalıq bólistiriliwin tabıń.
 \\
\textbf{A2.} 
Kólemi \(n = 20\) ǵa teń bolǵan tańlanba berilgen: 8,7; 2,7; 1,5; 2,2; 5,7; 1,5; 5,7; 2,7; 1,5; 2,2; 5,7; 8,7; 1,5; 2,2; 1,5; 2,7; 1,5; 5,7; 6,3; 8,7. Bul tańlanbanıń empirikalıq bólistiriw funkciyasın tabıń.
 \\
\textbf{A3.} 
Joqarı matematika páninen 10 dana student test sınaqların tapsırǵan. Hárbir student 10 balǵa shekem toplawı múmkin. Eger test sınaqları nátiyjeleri boyınsha \{9, 8, 6, 9, 5, 4, 5, 7, 8, 9\} tańlanba alınǵan bolsa, onda tańlanba ortasha hám tańlanba dispersiyalardı tabıń.
 \\
\textbf{B1.} 
Eger ortasha kvadratlıq shetleniwi \(\sigma = 4\) bolǵan normal bólistirilgen bas toplamnan alınǵan kólemi \(n = 16\) ǵa teń tańlanba boyınsha \(\overline{x} = 5,8\) tańlanba ortasha mánisi tabılǵan bolsa, onda \(\gamma = 0,90\) isenimlilik penen belgisiz \(\theta\) matematikalıq kútiliwdi qaplaytuǵın isenimlilik intervalın dúziń.
 \\
\textbf{B2.} 
Eger \(X^{(n)} = \left( X_{1},...,X_{n} \right)\) tańlanba \(\theta\) parametrli Bernulli bólistiriliwinen alınǵan bolsa, onda belgisiz \(\theta\) parametr ushın momentler usılı bahasın tabıń.
 \\
\textbf{B3.} 
Eger \(X^{(n)} = \left( X_{1},...,X_{n} \right)\) tańlanba \(\left( a,\theta^{2} \right)\) parametrli normal bólistiriliwden alınǵan bolsa (\(\alpha -\)belgili), onda belgisiz \(\theta^{2}\) parametrdiń shınlıqqa maksimal uqsaslıq bahasın tabıń.
 \\
\textbf{C1.} 
Eger \(X^{(n)} = \left( X_{1},...,X_{n} \right)\) tańlanba \(\ln\theta\) parametrli Puasson bólistiriliwinen alınǵan bolsa, onda belgisiz \(\theta\) parametr ushın \(e^{\overline{x}}\) bahasın jıljımaǵanlıq hám tiykarlılıqqa tekseriń.
 \\
\textbf{C2.} 
Eger \(X^{(n)} = \left( X_{1},...,X_{n} \right)\) tańlanba \(\theta\) parametrli Puasson bólistiriliwinen alınǵan bolsa, onda belgisiz \(\theta\) parametr ushın momentler usılı bahasın tabıń. Eger \(X^{(n)} = \left( X_{1},...,X_{n} \right)\) tańlanba \(\theta\) parametrli Puasson bólistiriliwinen alınǵan bolsa, onda belgisiz \(\theta\) parametr ushın momentler usılı bahasın\({\ g(x) = x}^{2}\) funkciyası járdeminde tabıń.
 \\
\textbf{C3.} 
\(\ f(x;\theta) = \frac{7x^{6}}{\sqrt{2\pi}}exp\{ - \frac{(x^{7} - \theta)^{2}}{2}\}\) model ushın \(\theta\) parametri haqıyqatqa maksimal uqsaslıq usılı bahası tabılsın.
 \\

\end{tabular}
\vspace{1cm}


\begin{tabular}{m{17cm}}
\textbf{48-variant}
\newline

\textbf{T1.} 
Gruppalanǵan hám intervallıq variaciyalıq qatarlar.
 \\
\textbf{T2.} 
Normal nızamnıń dispersiyası ushın isenimlilik intervalın dúziw. (Isenimlilik itimallıǵı, interval)
 \\
\textbf{A1.} 
Kólemi \(n = 20\) ǵa teń bolǵan tańlanba berilgen: 1,8; -1,9; 2,4; 1,8; 2,4; 1,8; 2,4; -0,6; -1,9; 1,8; -0,6; 2,4; -3,3; -1,9; 4,0; -3,3; -3,3; -1,9; -3,3; -1,9. Bul tańlanbanıń statistikalıq bólistiriliwin tabıń.
 \\
\textbf{A2.} 
Kólemi \(n = 20\) ǵa teń bolǵan tańlanba berilgen: 1,4; -1,9; 2,5; 1,4; 2,5; 1,4; 2,5; -0,4; -1,9; 1,4; -0,4; 2,5; -3,7; -1,9; 4,5; -3,7; -3,7; -1,9; -3,7; -1,9. Bul tańlanbanıń empirikalıq bólistiriw funkciyasın tabıń.
 \\
\textbf{A3.} 
Joqarı matematika páninen 10 dana student test sınaqların tapsırǵan. Hárbir student 10 balǵa shekem toplawı múmkin. Eger test sınaqları nátiyjeleri boyınsha \{4, 3, 8, 4, 8, 3, 9, 4, 7, 10\} tańlanba alınǵan bolsa, onda tańlanba ortasha hám tańlanba dispersiyalardı tabıń.
 \\
\textbf{B1.} 
Eger normal bólistirilgen bas toplamnan alınǵan kólemi \(n = 36\) ǵa teń tańlanba boyınsha \(\overline{x} = 20,2\) tańlanba ortasha hám \({\overline{S}}^{2} = 0,64\) dúzetilgen tańlanba dispersiyalar tabılǵan bolsa, onda \(\gamma = 0,90\) isenimlilik penen belgisiz \(\theta\) matematikalıq kútiliw ushın isenimlilik interval dúziń.
 \\
\textbf{B2.} 
Eger (3,-2,-2,0,-2,-2,-2,0,-2,3,-2,0,3,0,3,-2,0,-2,3,-2,-2,-2,-2,3,3,3,-2,-2,3,3) tańlanba tómende berilgen bólistiriliwden alınǵan bolsa, onda belgisiz \(\theta\) parametr ushın momentler usılı bahasın \(g(x) = |x|\) funkciyası járdeminde tabıń.
\begin{tabular}{|c|c|c|c|}
  \hline
$\xi$ &
$- 2$ &
$0$ &
$3$ \\
\hline
\(P_{\theta}\) & \(3\theta\) & \(1 - 5\theta\) & \(2\theta\) \\
\hline
\end{tabular}
 \\
\textbf{B3.} 
\(\ f(x) = \frac{\theta}{2}e^{- \theta|x|}\) model ushın \(\theta\) parametri haqıyqatqa maksimal uqsaslıq usılı bahası tabılsın.
 \\
\textbf{C1.} 
Eger \(X^{(n)} = \left( X_{1},...,X_{n} \right)\) tańlanba \((\alpha,\theta)\) parametrli Pareto bólistiriliwinen alınǵan bolsa (\(\alpha -\)belgili), onda belgisiz \(\theta\) parametr ushın \(X_{(1)}\) bahasın jıljımaǵanlıq hám tiykarlılıqqa tekseriń.
 \\
\textbf{C2.} 
Eger \(X^{(n)} = \left( X_{1},...,X_{n} \right)\) tańlanba \({\lbrack\theta}_{1},\theta_{2}\rbrack\) aralıqta teń ólshemli bólistiriliwden alınǵan bolsa, onda belgisiz \(\left( \theta_{1},\theta_{2} \right)\) vektor parametr ushın momentler usılı bahasın tabıń.
 \\
\textbf{C3.} 
Eger \(X^{(n)} = \left( X_{1},...,X_{n} \right)\) tańlanba \(\lbrack\theta,\theta + 2\rbrack\) aralıqta teń ólshemli bólistiriliwden alınǵan bolsa, onda belgisiz \(\theta\) parametrdiń shınlıqqa maksimal uqsaslıq usılı bahasın tabıń.
 \\

\end{tabular}
\vspace{1cm}


\begin{tabular}{m{17cm}}
\textbf{49-variant}
\newline

\textbf{T1.} 
Glivenko-Kantelli teoreması. (empirikalıq bólistiriw funkciyası, 1itimallıq penen jaqınlasıw)
 \\
\textbf{T2.} 
Statistikalıq gipotezalardı tekseriw (kritikalıq kóplik, 1 hám 2-túr qátelik).
 \\
\textbf{A1.} 
Kólemi \(n = 20\) ǵa teń bolǵan tańlanba berilgen: 2,9; -3,2; 5,3; -4,3; 4,1; 5,3; -1,2; 2,9; -3,2; 4,1; -4,3; 5,3; -3,2; 2,9; -4,3; 4,1; -1,2; 5,3; 2,9; -3,2. Bul tańlanbanıń statistikalıq bólistiriliwin tabıń.
 \\
\textbf{A2.} 
Kólemi \(n = 20\) ǵa teń bolǵan tańlanba berilgen: 2,7; -5,6; 5,2; -8,1; 4,8; 5,2; -1,6; 2,7; -5,6; 4,8; -8,1; 5,2; -5,6; 2,7; -8,1; 4,8; -1,6; 5,2; 2,7; -5,6. Bul tańlanbanıń empirikalıq bólistiriw funkciyasın tabıń.
 \\
\textbf{A3.} 
Joqarı matematika páninen 10 dana student test sınaqların tapsırǵan. Hárbir student 10 balǵa shekem toplawı múmkin. Eger test sınaqları nátiyjeleri boyınsha \{7, 9, 4, 9, 7, 5, 4, 7, 2, 6\} tańlanba alınǵan bolsa, onda tańlanba ortasha hám tańlanba dispersiyalardı tabıń.
 \\
\textbf{B1.} 
Eger normal bólistirilgen bas toplamnan alınǵan kólemi \(n = 11\) ǵa teń bolǵan tańlanba boyınsha \({\overline{S}}^{2} = 0,3\) dúzetilgen tańlanba dispersiya tabılǵan bolsa, onda \(\gamma = 0,95\) isenimlilik penen belgisiz \(\theta_{2}^{2}\) dispersiya ushın isenimlilik interval dúziń.
 \\
\textbf{B2.} 
Eger (3,0,-2,0,-2,3,-2,0,0,3,0,0,0,3,-2,0,0,-2,3,0) tańlanba tómende berilgen bólistiriliwden alınǵan bolsa, onda belgisiz \(\left( \theta_{1},\theta_{2} \right)\) vektor parametr ushın momentler usılı bahalasın tabıń.
\begin{tabular}{|c|c|c|c|}
  \hline
$\xi$ &
$- 2$ &
$0$ &
$3$\\
\hline
\(P_{\theta}\) & \({2\theta}_{1}\) & \(0,5 + \theta_{1} + \theta_{2}\) & \(\theta_{2}\) \\
\hline
\end{tabular}
 \\
\textbf{B3.} 
Eger (-1,-1,0,-1,0,-1,-1,5,-1,0,-1,0,5,-1,-1,-1,5,-1,-1,-1,5,0,-1,-1,5) tańlanba tómende berilgen bólistiriliwden alınǵan bolsa, onda belgisiz \(\theta\) parametrdiń shınlıqqa maksimal uqsaslıq usılı bahasın tabıń.
\begin{tabular}{|c|c|c|c|}
  \hline
$\xi$
&
$- 1$
&
$0$
&
$5$\\
\hline
\(P_{\theta}\) & \(1 - \theta\) & \(\theta/2\) & \(\theta/2\ \) \\
\hline
\end{tabular}
 \\
\textbf{C1.} 
Eger \(X^{(n)} = \left( X_{1},...,X_{n} \right)\) tańlanba tıǵızlıq funkciyası: \(f(x;\theta) = e^{- x + \theta}\left( 1 + e^{- x + \theta} \right)^{2},\ x \in R\)
bolǵan bólistiriliwden alınǵan bolsa, onda belgisiz \(\theta\) parametr ushın \(\overline{x}\) bahasın jıljımaǵanlıq hám tiykarlılıqqa tekseriń.
 \\
\textbf{C2.} 
Eger \(X^{(n)} = \left( X_{1},...,X_{n} \right)\) tańlanba\(\ (\theta,2\theta)\ \) parametrli normal bólistiriliwden alınǵan bolsa, onda belgisiz \(\theta > 0\) parametr ushın momentler usılı bahasın tabıń.
 \\
\textbf{C3.} 
Eger \(X^{(n)} = \left( X_{1},...,X_{n} \right)\) tańlanba tıǵızlıq funkciyası
$f(x;\theta) = \frac{3x^{2}}{\sqrt{2\pi}}\exp\left\{ - \frac{\left( x^{3} - \theta \right)^{2}}{2} \right\},\ x \in R$
bolǵan bólistiriliwden alınǵan bolsa, onda belgisiz \(\theta\) parametrdiń shınlıqqa maksimal uqsaslıq bahasın tabıń.
 \\

\end{tabular}
\vspace{1cm}


\begin{tabular}{m{17cm}}
\textbf{50-variant}
\newline

\textbf{T1.} 
Poligon hám gistogramma(salıstirmalı jiyilik, intervallıq qatar, grafik)
 \\
\textbf{T2.} 
Kolmogorovtıń kelisimlilik belgisi (Kolmogorov teoreması)
 \\
\textbf{A1.} 
Kólemi \(n = 20\) ǵa teń bolǵan tańlanba berilgen: 14,7; 7,3; 16,6; 9,8; 11,2; 16,6; 6,7; 7,3; 11,2; 14,7; 6,7; 16,6; 7,3; 11,2; 14,7; 16,6; 6,7; 7,3; 11,2; 16,6. Bul tańlanbanıń statistikalıq bólistiriliwin tabıń.
 \\
\textbf{A2.} 
Kólemi \(n = 20\) ǵa teń bolǵan tańlanba berilgen: 14,4; 7,6; 16,7; 9,1; 11,8; 16,7; 6,4; 7,6; 11,8; 14,4; 6,4; 16,7; 7,6; 11,8; 14,4; 16,7; 6,4; 7,6; 11,8; 16,7. Bul tańlanbanıń empirikalıq bólistiriw funkciyasın tabıń.
 \\
\textbf{A3.} 
Joqarı matematika páninen 10 dana student test sınaqların tapsırǵan. Hárbir student 10 balǵa shekem toplawı múmkin. Eger test sınaqları nátiyjeleri boyınsha \{10, 8, 4, 6, 2, 8, 5, 10, 2, 5\} tańlanba alınǵan bolsa, onda tańlanba ortasha hám tańlanba dispersiyalardı tabıń.
 \\
\textbf{B1.} 
Eger ortasha kvadratlıq shetleniwi \(\sigma = 4\) bolǵan normal bólistirilgen bas toplamnan alınǵan kólemi \(n = 49\) ǵa teń tańlanba boyınsha \(\overline{x} = 9,4\) tańlanba ortasha mánisi tabılǵan bolsa, onda \(\gamma = 0,90\) isenimlilik penen belgisiz \(\theta\) matematikalıq kútiliwdi qaplaytuǵın isenimlilik intervalın dúziń.
 \\
\textbf{B2.} 
Eger tıǵızlıq funkciyası \(f(x) = \frac{2x}{\theta}e^{- \frac{x^{2}}{\theta}},\ \ x \geq 0\) kóriniske iye bolsa, onda \(\theta\) parametr momentler usulı bahasın tabıń.
 \\
\textbf{B3.} 
Eger \(X^{(n)} = \left( X_{1},...,X_{n} \right)\) tańlanba \(\left( a,\theta^{2} \right)\) parametrli normal bólistiriliwden alınǵan bolsa (\(\alpha -\)belgili), onda belgisiz \(\theta^{2}\) parametrdiń shınlıqqa maksimal uqsaslıq bahasın tabıń.
 \\
\textbf{C1.} 
Eger \(X^{(n)} = \left( X_{1},...,X_{n} \right)\) tańlanba tıǵızlıq funkciyası
$f(x;\theta) = \left\{ \begin{array}{r}
\alpha^{- 1}e^{- \ \frac{x - \theta}{\alpha}},\ \ x \geq \theta, \\
0,\ \ \ \ \ \ \ x < \theta
\end{array} \right.\ $
bolǵan bólistiriliwden alınǵan bolsa (\(\alpha -\)belgili), onda belgisiz \(\theta\) parametr ushın \(X_{(1)}\) bahasın jıljımaǵanlıq hám tiykarlılıqqa tekseriń.
 \\
\textbf{C2.} 
Eger \(X^{(n)} = \left( X_{1},...,X_{n} \right)\) tańlanba \(\frac{1}{\sqrt{\theta}}\) parametrli kórsetkishli bólistiriliwden alınǵan bolsa, onda belgisiz \(\theta\) parametr ushın momentler usılı bahasın tabıń.
 \\
\textbf{C3.} 
Eger \(X^{(n)} = \left( X_{1},...,X_{n} \right)\) tańlanba tıǵızlıq funkciyası
$f(x;\theta) = \frac{\theta}{2}e^{- \theta|x|},\ x \in R$
bolǵan bólistiriliwden alınǵan bolsa, onda belgisiz \(\theta > 0\) parametrdiń shınlıqqa maksimal uqsaslıq bahasın tabıń.
 \\

\end{tabular}
\vspace{1cm}


\begin{tabular}{m{17cm}}
\textbf{51-variant}
\newline

\textbf{T1.} 
Empirikalıq bólistiriw funkciyası. (Tańlanba, eksperiment)
 \\
\textbf{T2.} 
Statistikalıq gipotezalardı tekseriw (kritikalıq kóplik, 1 hám 2-túr qátelik).
 \\
\textbf{A1.} 
Kólemi \(n = 20\) ǵa teń bolǵan tańlanba berilgen: 4,3; 4,9; 13,4; 13,4; 6,5; 4,9; 4,9; 4,3; 5,1; 6,5; 6,5; 7,0; 4,3; 4,9; 6,5; 6,5; 5,1; 5,1; 4,9; 13,4. Bul tańlanbanıń statistikalıq bólistiriliwin tabıń.
 \\
\textbf{A2.} 
Kólemi \(n = 20\) ǵa teń bolǵan tańlanba berilgen: 4,2; 4,9; 13,8; 13,8; 6,6; 4,9; 4,9; 4,2; 5,3; 6,6; 6,6; 7,5; 4,2; 4,9; 6,6; 6,6; 5,3; 5,3; 4,9; 13,8. Bul tańlanbanıń empirikalıq bólistiriw funkciyasın tabıń.
 \\
\textbf{A3.} 
Joqarı matematika páninen 10 dana student test sınaqların tapsırǵan. Hárbir student 10 balǵa shekem toplawı múmkin. Eger test sınaqları nátiyjeleri boyınsha \{9, 10, 6, 7, 4, 8, 10, 7, 9, 10\} tańlanba alınǵan bolsa, onda tańlanba ortasha hám tańlanba dispersiyalardı tabıń.
 \\
\textbf{B1.} 
Eger ortasha kvadratlıq shetleniwi \(\sigma = 2\) bolǵan normal bólistirilgen bas toplamnan alınǵan kólemi \(n = 10\) ǵa teń tańlanba boyınsha \(\overline{x} = 5,4\) tańlanba ortasha mánisi tabılǵan bolsa, onda \(\gamma = 0,95\) isenimlilik penen belgisiz \(\theta\) matematikalıq kútiliwdi qaplaytuǵın isenimlilik intervalın dúziń.
 \\
\textbf{B2.} 
Eger (0,-2,0,-2,3,-2,0,0,3,0,0,0,3,-2,0,0,-2,3,0,3) tańlanba tómende berilgen bólistiriliwden alınǵan bolsa, onda belgisiz \(\theta\) parametr ushın momentler usılı bahasın tabıń.
\begin{tabular}{|c|c|c|c|}
  \hline
$\xi$ & $- 2$  & $0$  & $3$ \\
\hline
\(P_{\theta}\) & \(\theta\) & \(1 - 2\theta\) & \(\theta\) \\
\hline
\end{tabular}
 \\
\textbf{B3.} 
Eger \(X^{(n)} = \left( X_{1},...,X_{n} \right)\) tańlanba \(\left\lbrack - \theta,\theta^{2} \right\rbrack\) aralıqta teń ólshemli bólistiriliwden alınǵan bolsa, onda belgisiz \(\theta > 0\) parametrdiń shınlıqqa maksimal uqsaslıq usılı bahasın tabıń.
 \\
\textbf{C1.} 
Eger \(X^{(n)} = \left( X_{1},...,X_{n} \right)\) tańlanba \(\lbrack 0,\theta\rbrack\) aralıqta teń ólshemli bólistiriliwden alýnǵan bolsa, onda belgisiz \(\theta\) parametr ushın \((n + 1)X_{(1)}\) bahasın jıljımaǵanlıq hám tiykarlılıqqa tekseriń.
 \\
\textbf{C2.} 
Eger \(X^{(n)} = \left( X_{1},...,X_{n} \right)\) tańlanba\({\ \ (a,\theta}^{2})\) parametrli normal bólistiriliwden alınǵan bolsa (\(\alpha -\)belgili), onda belgisiz\({\ \ \theta}^{2}\) parametr ushın momentler usılı bahasın tabıń.
 \\
\textbf{C3.} 
Eger \(X^{(n)} = \left( X_{1},...,X_{n} \right)\) tańlanba \(\theta\) parametrli geometriyalıq bólistiriliwden alınǵan bolsa, onda belgisiz \(\theta\) parametrdiń shınlıqqa maksimal uqsaslıq usılı bahasın tabıń.
 \\

\end{tabular}
\vspace{1cm}


\begin{tabular}{m{17cm}}
\textbf{52-variant}
\newline

\textbf{T1.} 
Tańlanba xarakteristikalar. (Variaciyalıq qatar, salıstırmalı jiyilik).
 \\
\textbf{T2.} 
Statistikalıq gipotezalardı tekseriw (kritikalıq kóplik, 1 hám 2-túr qátelik)
 \\
\textbf{A1.} 
Kólemi \(n = 20\) ǵa teń bolǵan tańlanba berilgen: -2,1; 1,7; 3,3; 3,3; 11,7; 4,7; 1,7; 4,7; -2,1; 4,7; 4,7; 4,7; 8,0; -2,1; 1,7; 4,7; 8,0; 11,7; 1,7; 8,0. Bul tańlanbanıń statistikalıq bólistiriliwin tabıń.
 \\
\textbf{A2.} 
Kólemi \(n = 20\) ǵa teń bolǵan tańlanba berilgen: -2,2; 1,3; 3,8; 3,8; 11,5; 4,1; 1,3; 4,1; -2,2; 4,1; 4,1; 4,1; 8,4; -2,2; 1,3; 4,1; 8,4; 11,5; 1,3; 8,4. Bul tańlanbanıń empirikalıq bólistiriw funkciyasın tabıń.
 \\
\textbf{A3.} 
Joqarı matematika páninen 10 dana student test sınaqların tapsırǵan. Hárbir student 10 balǵa shekem toplawı múmkin. Eger test sınaqları nátiyjeleri boyınsha \{4, 1, 2, 4, 6, 4, 5, 3, 6, 5\} tańlanba alınǵan bolsa, onda tańlanba ortasha hám tańlanba dispersiyalardı tabıń.
 \\
\textbf{B1.} 
Eger normal bólistirilgen bas toplamnan alınǵan kólemi \(n = 16\) ǵa teń tańlanba boyınsha \(\overline{x} = 20,2\) tańlanba ortasha hám \({\overline{S}}^{2} = 0,64\) dúzetilgen tańlanba dispersiyalar tabılǵan bolsa, onda \(\gamma = 0,95\) isenimlilik penen belgisiz \(\theta\) matematikalıq kútiliw ushın isenimlilik interval dúziń.
 \\
\textbf{B2.} 
Eger (-2,0,-2,0,-2,3,-2,0,0,3,0,0,0,3,-2,0,0,-2,3,0) tańlanba tómende berilgen bólistiriliwden alınǵan bolsa, onda belgisiz \(\left( \theta_{1},\theta_{2} \right)\) vektor parametr ushın momentler usılı bahalasın tabıń.
\begin{tabular}{|c|c|c|c|}
  \hline
$\xi$ &
$- 2$ &
$0$ &
$3$\\
\hline
\(P_{\theta}\) & \(\theta_{1}\) & \(1 - \theta_{1} - \theta_{2}\) & \(\theta_{2}\) \\
\hline
\end{tabular}
 \\
\textbf{B3.} 
Eger \(X^{(n)} = \left( X_{1},...,X_{n} \right)\) tańlanba \(\theta\) parametrli Bernulli bólistiriliwinen alınǵan bolsa, onda belgisiz \(\theta\) parametrdiń shınlıqqa maksimal uqsaslıq usılı bahasın tabıń.
 \\
\textbf{C1.} 
Eger \(X^{(n)} = \left( X_{1},...,X_{n} \right)\) tańlanba \(\lbrack 0,\theta\rbrack\) aralıqta teń ólshemli bólistiriliwden alınǵan bolsa, onda belgisiz \(\theta\) parametr ushın \(\frac{n + 1}{n}X_{(n)}\) bahasın jıljımaǵanlıq hám tiykarlılıqqa tekseriń.
 \\
\textbf{C2.} 
Eger \(X^{(n)} = \left( X_{1},...,X_{n} \right)\) tańlanba \(\theta\) parametrli geometriyalıq bólistiriliwden alınǵan bolsa, onda belgisiz \(\theta\) parametr ushın momentler usılı bahasın tabıń.
 \\
\textbf{C3.} 
Eger \(X^{(n)} = \left( X_{1},...,X_{n} \right)\) tańlanba tıǵızlıq funkciyası
$f(x;\theta) = \left\{ \begin{matrix}
\theta_{1}^{- 1}e^{\frac{x - \theta_{2}}{\theta_{1}}},\ \ x \geq \theta_{2}, \\
\ \ \ \ \ \ \ \ \ \ \ \ 0,\ \ \ \ \ \ \ x < \theta_{2}
\end{matrix} \right.\ $
bolǵan bólistiriliwden alınǵan bolsa, onda belgisiz \(\left( \theta_{1},\theta_{2} \right),\) \(\theta_{1} > 0,\) \(\theta_{2} \in R\) vektor parametrdiń shınlıqqa maksimal uqsaslıq bahasın tabıń.
 \\

\end{tabular}
\vspace{1cm}


\begin{tabular}{m{17cm}}
\textbf{53-variant}
\newline

\textbf{T1.} 
Poligon hám gistogramma(salıstirmalı jiyilik, intervallıq qatar, grafik)
 \\
\textbf{T2.} 
Haqiyqatqa maksimal uqsaslıq usulı. (haqiyqatqa maksimal uqsaslıq funkciyası, belgisiz parametrlerdi bahalaw).
 \\
\textbf{A1.} 
Kólemi \(n = 20\) ǵa teń bolǵan tańlanba berilgen: -11,0; -4,1; 0; 2,3; 1,2; 0; 1,2; 2,3; 2,3; 1,2; 2,3; -11,0; 3,4; 1,2; 3,4; 3,4; 0; 3,4; 2,3; 0. Bul tańlanbanıń statistikalıq bólistiriliwin tabıń.
 \\
\textbf{A2.} 
Kólemi \(n = 20\) ǵa teń bolǵan tańlanba berilgen: -11,2; -4,5; 0; 2,9; 1,7; 0; 1,7; 2,9; 2,9; 1,7; 2,9; -11,2; 3,1; 1,7; 3,1; 3,1; 0; 3,1; 2,9; 0. Bul tańlanbanıń empirikalıq bólistiriw funkciyasın tabıń.
 \\
\textbf{A3.} 
Joqarı matematika páninen 10 dana student test sınaqların tapsırǵan. Hárbir student 10 balǵa shekem toplawı múmkin. Eger test sınaqları nátiyjeleri boyınsha \{8, 9, 10, 4, 9, 7, 6, 7, 6, 4\} tańlanba alınǵan bolsa, onda tańlanba ortasha hám tańlanba dispersiyalardı tabıń.
 \\
\textbf{B1.} 
Eger normal bólistirilgen bas toplamnan alınǵan kólemi \(n = 11\) ǵa teń bolǵan tańlanba boyınsha \({\overline{S}}^{2} = 0,5\) dúzetilgen tańlanba dispersiya tabılǵan bolsa, onda \(\gamma = 0,90\) isenimlilik penen belgisiz \(\theta_{2}^{2}\) dispersiya ushın isenimlilik interval dúziń.
 \\
\textbf{B2.} 
Puasson bólistiriliwi belgisiz \(\theta > 0\) parametri momentlar usuli bahasin tabıń.
 \\
\textbf{B3.} 
Eger \(X^{(n)} = \left( X_{1},...,X_{n} \right)\) tańlanba \(\lbrack - \theta,\theta\rbrack\) aralıqta teń ólshemli bólistiriliwden alınǵan bolsa, onda belgisiz \(\theta > 0\) parametrdiń shınlıqqa maksimal uqsaslıq usılı bahasın tabıń.
 \\
\textbf{C1.} 
Eger \(X^{(n)} = \left( X_{1},...,X_{n} \right)\) tańlanba \(M\xi = a\) belgili hám \(M\xi^{2}\) shekli bolǵan bólistiriliwden alınǵan bolsa, onda belgisiz \(D\xi\) dispersiya ushın \({\overline{S}}^{2}\) bahasın jıljımaǵanlıq hám tiykarlılıqqa tekseriń.
 \\
\textbf{C2.} 
Eger \(X^{(n)} = \left( X_{1},...,X_{n} \right)\) tańlanba tıǵızlıq funkciyası
$f(x,\theta) = \left\{ \begin{array}{r}
\theta_{1}^{- 1}e^{- \frac{x - \theta_{2}}{\theta_{1}}},\ \ \ x \geq \theta_{2}, \\
0,\ \ \ x < \theta_{2}
\end{array} \right.\ $
bolǵan bólistiriliwden alınǵan bolsa, onda belgisiz \(\left( \theta_{1},\theta_{2} \right)\) \(\theta_{1} > 0,\) \(\theta_{2} \in R\) vektor parametr ushın momentler usılı bahasın tabıń.
 \\
\textbf{C3.} 
Eger \(X^{(n)} = \left( X_{1},...,X_{n} \right)\) tańlanba tıǵızlıq funkciyası
$f(x;\theta) = \frac{\theta ln^{\theta - 1}x}{x},\ x \in \lbrack 1,e\rbrack$
bolǵan bólistiriliwden alınǵan bolsa, onda belgisiz \(\theta > 0\) parametr ushın shınlıqqa maksimal uqsaslıq bahasın tabıń.
 \\

\end{tabular}
\vspace{1cm}


\begin{tabular}{m{17cm}}
\textbf{54-variant}
\newline

\textbf{T1.} 
Neyman-Pirson teoreması
 \\
\textbf{T2.} 
Isenimlilik intervalların qurıw. Anıq isenimli intervallar
 \\
\textbf{A1.} 
Kólemi \(n = 20\) ǵa teń bolǵan tańlanba berilgen: 2,5; 3,8; 4,3; 2,5; 3,8; 2,5; 3,1; 4,3; 4,3; 5,5; 6,2; 2,5; 3,1; 6,2; 5,5; 6,2; 3,1; 3,1; 6,2; 3,1. Bul tańlanbanıń statistikalıq bólistiriliwin tabıń.
 \\
\textbf{A2.} 
Kólemi \(n = 20\) ǵa teń bolǵan tańlanba berilgen: 2,7; 4,2; 4,8; 2,7; 4,2; 2,7; 3,9; 4,8; 4,8; 5,9; 6,5; 2,7; 3,9; 6,5; 5,9; 6,5; 3,9; 3,9; 6,5; 3,9. Bul tańlanbanıń empirikalıq bólistiriw funkciyasın tabıń.
 \\
\textbf{A3.} 
Joqarı matematika páninen 10 dana student test sınaqların tapsırǵan. Hárbir student 10 balǵa shekem toplawı múmkin. Eger test sınaqları nátiyjeleri boyınsha \{7, 8, 7, 6, 4, 8, 4, 7, 9, 10\} tańlanba alınǵan bolsa, onda tańlanba ortasha hám tańlanba dispersiyalardı tabıń.
 \\
\textbf{B1.} 
Eger ortasha kvadratlıq shetleniwi \(\sigma = 3\) bolǵan normal bólistirilgen bas toplamnan alınǵan kólemi \(n = 9\) ǵa teń tańlanba boyınsha \(\overline{x} = 4,5\) tańlanba ortasha mánisi tabılǵan bolsa, onda \(\gamma = 0,95\) isenimlilik penen belgisiz \(\theta\) matematikalıq kútiliwdi qaplaytuǵın isenimlilik intervalın dúziń.
 \\
\textbf{B2.} 
\(\lbrack 0,\theta\rbrack\) aralıqta teń ólshewli bólistirilgen \(\theta\) parametri ushın momentler usulı bahasın tabıń.
 \\
\textbf{B3.} 
\(\ f(x) = \frac{2x}{\theta}e^{- \frac{x^{2}}{\theta}},\ \ x \geq 0\) model ushın \(\theta\) parametri haqıyqatqa maksimal uqsaslıq usılı bahası tabılsın.
 \\
\textbf{C1.} 
Eger \(X^{(n)} = \left( X_{1},...,X_{n} \right)\) tańlanba \(M\xi = a\) belgili hám \(M\xi^{2}\) shekli bolǵan bólistiriliwden alınǵan bolsa, onda belgisiz \(D\xi\) dispersiya ushın \(\frac{1}{n - 1}\sum_{i = 1}^{n}\left( X_{i} - a \right)^{2}\) bahasın jıljımaǵanlıq hám tiykarlılıqqa tekseriń.
 \\
\textbf{C2.} 
Eger \(X^{(n)} = \left( X_{1},...,X_{n} \right)\) tańlanba tıǵızlıq funkciyası
${f(x,\theta) = \theta x}^{\theta - 1},x \in \lbrack 0,1\rbrack$
bolǵan bólistiriliwden alınǵan bolsa, onda belgisiz \(\theta\) parametr ushın momentler usılı bahasın tabıń.
 \\
\textbf{C3.} 
Eger \(X^{(n)} = \left( X_{1},...,X_{n} \right)\) tańlanba \((\theta,2\theta)\) parametrli normal bólistiriliwden alınǵan bolsa, onda belgisiz \(\theta > 0\) parametrdiń shınlıqqa maksimal uqsaslıq bahasın tabıń.
 \\

\end{tabular}
\vspace{1cm}


\begin{tabular}{m{17cm}}
\textbf{55-variant}
\newline

\textbf{T1.} Matematikalıq statistikanıń tiykarǵı máseleleri. (Statistikalıq maǵlıwmatlar, gruppalaw)
 \\
\textbf{T2.} 
Kolmogorovtıń kelisimlilik belgisi (Kolmogorov teoreması)
 \\
\textbf{A1.} 
Kólemi \(n = 20\) ǵa teń bolǵan tańlanba berilgen: -4,3; 2,6; 0; -2,5; 2,6; 1,9; 2,2; 0; -4,3; -2,5; 1,9; -2,5; 1,9; 2,2; 2,6; 1,9; 2,6; 2,2; 2,2; 1,9. Bul tańlanbanıń statistikalıq bólistiriliwin tabıń.
 \\
\textbf{A2.} 
Kólemi \(n = 20\) ǵa teń bolǵan tańlanba berilgen: -4,9; 2,6; 0,5; -2,6; 2,6; 1,7; 2,3; 0,5; -4,9; -2,6; 1,7; -2,6; 1,7; 2,3; 2,6; 1,7; 2,6; 2,3; 2,3; 1,7. Bul tańlanbanıń empirikalıq bólistiriw funkciyasın tabıń.
 \\
\textbf{A3.} 
Joqarı matematika páninen 10 dana student test sınaqların tapsırǵan. Hárbir student 10 balǵa shekem toplawı múmkin. Eger test sınaqları nátiyjeleri boyınsha \{9, 5, 6, 8, 4, 7, 4, 6, 9, 7\} tańlanba alınǵan bolsa, onda tańlanba ortasha hám tańlanba dispersiyalardı tabıń.
 \\
\textbf{B1.} 
Eger normal bólistirilgen bas toplamnan alınǵan kólemi \(n = 25\) ǵa teń tańlanba boyınsha \(\overline{x} = 18,6\) tańlanba ortasha hám \({\overline{S}}^{2} = 0,49\) dúzetilgen tańlanba dispersiyalar tabılǵan bolsa, onda \(\gamma = 0,95\) isenimlilik penen belgisiz \(\theta\) matematikalıq kútiliw ushın isenimlilik interval dúziń.
 \\
\textbf{B2.} 
Kórsetkishli bólistiriw belgisiz \(\theta > 0\) parametri momentlar usulı bahasın tabıń.
 \\
\textbf{B3.} 
Eger (4,8,5,3) tańlanba \(\left( a,\theta^{2} \right)\) parametrli normal bólistiriliwden alınǵan bolsa, onda belgisiz \(\theta^{2}\) parametrdiń shınlıqqa maksimal uqsaslıq bahasın tabıń.
 \\
\textbf{C1.} 
Eger \(X^{(n)} = \left( X_{1},...,X_{n} \right)\) tańlanba \(M\xi = a\) belgili hám \(M\xi^{2}\) shekli bolǵan bólistiriliwden alınǵan bolsa, onda belgisiz \(D\xi\) dispersiya ushın \(\frac{1}{n}\sum_{i = 1}^{n}\left( X_{i} - a \right)^{2}\) bahasın jıljımaǵanlıq hám tiykarlılıqqa tekseriń.
 \\
\textbf{C2.} 
Eger \(X^{(n)} = \left( X_{1},...,X_{n} \right)\) tańlanba tıǵızlıq funkciyası
$
{f(x,\theta) = \left\{ \begin{array}{r}
e^{\theta - x},\ \ \ x \geq \theta, \\
0,\ \ \ x < \theta
\end{array} \right.\ }$
bolǵan bólistiriliwden alınǵan bolsa, onda belgisiz \(\theta\) parametr ushın momentler usılı bahasın tabıń.
 \\
\textbf{C3.} 
\(\ f(x,\theta) = \frac{4x^{3}}{\theta_{2}\sqrt{2\pi}}\exp\left\{ - \frac{\left( x^{4} - \theta_{1} \right)^{2}}{2{\theta_{2}}^{2}} \right\}\) model ushın \(\theta_{1}\) hám \({\theta_{2}}^{2}\) parametrler haqıyqatqa maksimal uqsaslıq usılı bahaları tabılsın.
 \\

\end{tabular}
\vspace{1cm}


\begin{tabular}{m{17cm}}
\textbf{56-variant}
\newline

\textbf{T1.} 
Gruppalanǵan hám intervallıq variaciyalıq qatarlar.
 \\
\textbf{T2.} 
Statistikalıq baha qásiyetleri. (Jıljımaytuǵın, tiykarlı, effektiv)
 \\
\textbf{A1.} 
Kólemi \(n = 20\) ǵa teń bolǵan tańlanba berilgen: -2,9; -3,8; 2,3; 1,8; 1,8; 0,7; -3,8; -1,5; 2,3; 0,7; -2,9; -1,5; 1,8; -2,9; -1,5; -3,8; 1,8; 1,8; -3,8; 1,8. Bul tańlanbanıń statistikalıq bólistiriliwin tabıń.
 \\
\textbf{A2.} 
Kólemi \(n = 20\) ǵa teń bolǵan tańlanba berilgen: -2,4; -3,5; 2,8; 1,4; 1,4; 0,1; -3,5; -1,9; 2,8; 0,1; -2,4; -1,9; 1,4; -2,4; -1,9; -3,5; 1,4; 1,4; -3,5; 1,4. Bul tańlanbanıń empirikalıq bólistiriw funkciyasın tabıń.
 \\
\textbf{A3.} 
Joqarı matematika páninen 10 dana student test sınaqların tapsırǵan. Hárbir student 10 balǵa shekem toplawı múmkin. Eger test sınaqları nátiyjeleri boyınsha \{8, 9, 7, 10, 6, 8, 10, 3, 10, 9\} tańlanba alınǵan bolsa, onda tańlanba ortasha hám tańlanba dispersiyalardı tabıń.
 \\
\textbf{B1.} 
Eger normal bólistirilgen bas toplamnan alınǵan kólemi \(n = 12\) ǵa teń bolǵan tańlanba boyınsha \({\overline{S}}^{2} = 0,4\) dúzetilgen tańlanba dispersiya tabılǵan bolsa, onda \(\gamma = 0,90\) isenimlilik penen belgisiz \(\theta_{2}^{2}\) dispersiya ushın isenimlilik interval dúziń.
 \\
\textbf{B2.} 
Kórsetkishli bólistiriw belgisiz \(\theta > 0\) parametri momentlar usulı bahasın tabıń.
 \\
\textbf{B3.} 
Eger \(x_{1} = 1,1;\ x_{2} = 2,7;\ldots;x_{100} = 1,5\) tańlanba \(\theta\) parametrli kórsetkishli bólistiriliwden alınǵan bolıp, \(\sum_{k = 1}^{100}x_{k} = 200\) bolsa, onda belgisiz \(\theta\) parametrdiń shınlıqqa maksimal uqsaslıq bahasın tabıń.
 \\
\textbf{C1.} 
Eger \(X^{(n)} = \left( X_{1},...,X_{n} \right)\) tańlanba \(M\xi = a\) belgili hám \(M\xi^{2}\) shekli bolǵan bólistiriliwden alınǵan bolsa, onda belgisiz \(D\xi\) dispersiya ushın \(\overline{x^{2}} - a^{2}\) bahasın jıljımaǵanlıq hám tiykarlılıqqa tekseriń.
 \\
\textbf{C2.} 
Eger \(X^{(n)} = \left( X_{1},...,X_{n} \right)\) tańlanba\(\ (\theta,2\theta)\ \) parametrli normal bólistiriliwden alınǵan bolsa, onda belgisiz \(\theta > 0\) parametr ushın momentler usılı bahasın tabıń.
 \\
\textbf{C3.} 
Eger \(X^{(n)} = \left( X_{1},...,X_{n} \right)\) tańlanba tıǵızlıq funkciyası
$f(x;\theta) = \frac{4x^{3}}{\sqrt{2\pi}\theta_{2}}\exp\left\{ - \frac{\left( x^{4} - \theta_{1} \right)^{2}}{2{\theta_{2}}^{2}} \right\},\ x \in R$
bolǵan bólistiriliwden alınǵan bolsa, onda belgisiz \(\left( \theta_{1},\theta_{2}^{2} \right)\) vektor parametrdiń shınlıqqa maksimal uqsaslıq usılı bahaların tabıń.
 \\

\end{tabular}
\vspace{1cm}


\begin{tabular}{m{17cm}}
\textbf{57-variant}
\newline

\textbf{T1.} 
Tańlanba momentleri (\(k -\)tártipli baslanǵısh, baslanǵısh absolyut, oraylıq hám oraylıq absolyut momentler).
 \\
\textbf{T2.} 
Sızıqlı korrelyaciya teńlemesi (anıqlaması, regressiya tuwrı sızıǵınıń tańlanba teńlemeleri)
 \\
\textbf{A1.} 
Kólemi \(n = 20\) ǵa teń bolǵan tańlanba berilgen: 3,6; 2,9; 3,6; 3,2; 1,1; 0,3; 1,1; 3,6; 1,7; 1,1; 0,3; 1,7; 1,1; 0,3; 2,9; 2,9; 2,9; 1,1; 2,9; 1,7. Bul tańlanbanıń statistikalıq bólistiriliwin tabıń.
 \\
\textbf{A2.} 
Kólemi \(n = 20\) ǵa teń bolǵan tańlanba berilgen: 4,6; 2,5; 4,6; 3,3; 1,8; 0,3; 1,8; 4,6; 2,1; 1,8; 0,3; 2,1; 1,8; 0,3; 2,5; 2,5; 2,5; 1,8; 2,5; 2,1. Bul tańlanbanıń empirikalıq bólistiriw funkciyasın tabıń.
 \\
\textbf{A3.} 
Joqarı matematika páninen 10 dana student test sınaqların tapsırǵan. Hárbir student 10 balǵa shekem toplawı múmkin. Eger test sınaqları nátiyjeleri boyınsha \{5, 7, 5, 9, 5, 8, 10, 6, 7, 8\} tańlanba alınǵan bolsa, onda tańlanba ortasha hám tańlanba dispersiyalardı tabıń.
 \\
\textbf{B1.} 
Eger ortasha kvadratlıq shetleniwi \(\sigma = 1\) bolǵan normal bólistirilgen bas toplamnan alınǵan kólemi \(n = 15\) ǵa teń tańlanba boyınsha \(\overline{x} = 5,8\) tańlanba ortasha mánisi tabılǵan bolsa, onda \(\gamma = 0,90\) isenimlilik penen belgisiz \(\theta\) matematikalıq kútiliwdi qaplaytuǵın isenimlilik intervalın dúziń.
 \\
\textbf{B2.} 
\(\lbrack\theta_{1},\theta_{2}\rbrack\) aralıqta teń ólshewli bólistiriw parametrleri ushın momentler usulı bahaların tabıń.
 \\
\textbf{B3.} 
Eger (0,1,2,0) tańlanba tómende berilgen bólistiriliwden alınǵan bolsa, onda belgisiz \(\theta\) parametrdiń shınlıqqa maksimal uqsaslıq bahasın tabıń.
\begin{tabular}{|c|c|c|c|}
  \hline
$\xi$
&
$0$
&
$1$
&
$2$\\
\hline
\(P_{\theta}\) & \(\theta\) & \(2\theta\) & \(1 - 3\theta\) \\
\hline
\end{tabular}
 \\
\textbf{C1.} 
Eger \(X^{(n)} = \left( X_{1},...,X_{n} \right)\) tańlanba tıǵızlıq funkciyası: \(f(x,\theta) = \left\{ \begin{matrix}
e^{\theta - x},\ \ x \geq \theta, \\
\ \ 0,\ \ \ \ \ \ \ x < \theta
\end{matrix} \right.\ \)
bolǵan bólistiriliwden alınǵan bolsa, onda belgisiz \(\theta\) parametr ushın \(X_{(1)}\) bahasın jıljımaǵanlıq hám tiykarlılıqqa tekseriń.
 \\
\textbf{C2.} 
Eger \(X^{(n)} = \left( X_{1},...,X_{n} \right)\) tańlanba tıǵızlıq funkciyası
$f(x,\theta) = \left\{ \begin{array}{r}
\theta_{1}^{- 1}e^{- \frac{x - \theta_{2}}{\theta_{1}}},\ \ \ x \geq \theta_{2}, \\
0,\ \ \ x < \theta_{2}
\end{array} \right.\ $
bolǵan bólistiriliwden alınǵan bolsa, onda belgisiz \(\left( \theta_{1},\theta_{2} \right)\) \(\theta_{1} > 0,\) \(\theta_{2} \in R\) vektor parametr ushın momentler usılı bahasın tabıń.
 \\
\textbf{C3.} 
Eger \(X^{(n)} = \left( X_{1},...,X_{n} \right)\) tańlanba tıǵızlıq funkciyası
$f(x;\theta) = \frac{e^{x}}{\sqrt{2\pi}}\exp\left\{ - \frac{\left( e^{x} - \theta \right)^{2}}{2} \right\},\ x \in R$
bolǵan bólistiriliwden alınǵan bolsa, onda belgisiz \(\theta\) parametrdiń shınlıqqa maksimal uqsaslıq bahasın tabıń.
 \\

\end{tabular}
\vspace{1cm}


\begin{tabular}{m{17cm}}
\textbf{58-variant}
\newline

\textbf{T1.} 
Tańlanba xarakteristikaları.(tańlanba orta, tańlanba dispersiya)
 \\
\textbf{T2.} 
Momentler usulı. (tańlanba momentleri, belgisiz parametrlerdi bahalaw).
 \\
\textbf{A1.} 
Kólemi \(n = 20\) ǵa teń bolǵan tańlanba berilgen: -1,3; 0; 0,8; 2,3; 1,1; 0,8; 0,8; 2,3; 1,1; 0,8; -1,3; 1,8; 1,1; -1,3; 1,1; 1,8; 1,8; 1,1; 1,8; 1,8. Bul tańlanbanıń statistikalıq bólistiriliwin tabıń.
 \\
\textbf{A2.} 
Kólemi \(n = 20\) ǵa teń bolǵan tańlanba berilgen: -1,9; 0,7; 0,9; 2,8; 1,3; 0,9; 0,9; 2,8; 1,3; 0,9; -1,9; 1,6; 1,3; -1,9; 1,3; 1,6; 1,6; 1,3; 1,6; 1,6. Bul tańlanbanıń empirikalıq bólistiriw funkciyasın tabıń.
 \\
\textbf{A3.} 
Joqarı matematika páninen 10 dana student test sınaqların tapsırǵan. Hárbir student 10 balǵa shekem toplawı múmkin. Eger test sınaqları nátiyjeleri boyınsha \{8, 4, 3, 7, 3, 6, 5, 3, 5, 6\} tańlanba alınǵan bolsa, onda tańlanba ortasha hám tańlanba dispersiyalardı tabıń.
 \\
\textbf{B1.} 
Eger normal bólistirilgen bas toplamnan alınǵan kólemi \(n = 20\) ǵa teń tańlanba boyınsha \(\overline{x} = 16,6\) tańlanba ortasha hám \({\overline{S}}^{2} = 0,64\) dúzetilgen tańlanba dispersiyalar tabılǵan bolsa, onda \(\gamma = 0,95\) isenimlilik penen belgisiz \(\theta\) matematikalıq kútiliw ushın isenimlilik interval dúziń.
 \\
\textbf{B2.} 
\(\lbrack 0,\theta\rbrack\) aralıqta teń ólshewli bólistirilgen \(\theta\) parametri ushın momentler usulı bahasın tabıń.
 \\
\textbf{B3.} 
Eger \(X^{(n)} = \left( X_{1},...,X_{n} \right)\) tańlanba \(\theta\) parametrli kórsetkishli bólistiriliwden alınǵan bolsa, onda belgisiz \(\theta\) parametrdiń shınlıqqa maksimal uqsaslıq usılı bahasın tabıń.
 \\
\textbf{C1.} 
Eger \(X^{(n)} = \left( X_{1},...,X_{n} \right)\) tańlanba tıǵızlıq funkciyası: \(f(x,\theta) = \left\{ \begin{matrix}
e^{\theta - x},\ \ x \geq \theta, \\
\ \ 0,\ \ \ \ \ \ \ x < \theta
\end{matrix} \right.\ \)
bolǵan bólistiriliwden alınǵan bolsa, onda belgisiz \(\theta\) parametr ushın \(\overline{x} - 1\) bahasın jıljımaǵanlıq hám tiykarlılıqqa tekseriń.
 \\
\textbf{C2.} 
Eger \(X^{(n)} = \left( X_{1},...,X_{n} \right)\) tańlanba \({\ \ (a,\theta}^{2})\ \)parametrli normal bólistiriliwden alınǵan bolsa (\(\alpha -\)belgili), onda belgisiz \({\ \theta}^{2}\) parametr ushın momentler usılı bahasın \({\ g(x) = (x - a)}^{2}\) funkciyası járdeminde tabıń.
 \\
\textbf{C3.} 
Eger \(X^{(n)} = \left( X_{1},...,X_{n} \right)\) tańlanba \(\left( \theta,\theta^{2} \right)\) parametrli normal bólistiriliwden alınǵan bolsa, onda belgisiz \(\theta > 0\) parametrdiń shınlıqqa maksimal uqsaslıq bahasın tabıń.
 \\

\end{tabular}
\vspace{1cm}


\begin{tabular}{m{17cm}}
\textbf{59-variant}
\newline

\textbf{T1.} 
Momentler usulı. (tańlanba momentleri, belgisiz parametrlerdi bahalaw).
 \\
\textbf{T2.} 
Pirsonnıń xi-kvadrat kelisimlilik belgisi (Pirson teoreması).
 \\
\textbf{A1.} 
Kólemi \(n = 20\) ǵa teń bolǵan tańlanba berilgen: -2,4; 5,6; 5,6; -5,2; -6,7; 5,1; -5,2; -2,4; 4,3; 5,1; -6,7; 4,3; -2,4; -6,7; 4,3; 5,1; 4,3; 5,6; -6,7; 5,6. Bul tańlanbanıń statistikalıq bólistiriliwin tabıń.
 \\
\textbf{A2.} 
Kólemi \(n = 20\) ǵa teń bolǵan tańlanba berilgen: -2,9; 7,6; 7,6; -5,7; -6,1; 5,5; -5,7; -2,9; 4,2; 5,5; -6,1; 4,2; -2,9; -6,1; 4,2; 5,5; 4,2; 7,6; -6,1; 7,6. Bul tańlanbanıń empirikalıq bólistiriw funkciyasın tabıń.
 \\
\textbf{A3.} 
Joqarı matematika páninen 10 dana student test sınaqların tapsırǵan. Hárbir student 10 balǵa shekem toplawı múmkin. Eger test sınaqları nátiyjeleri boyınsha \{9, 8, 6, 7, 5, 8, 5, 7, 4, 6\} tańlanba alınǵan bolsa, onda tańlanba ortasha hám tańlanba dispersiyalardı tabıń.
 \\
\textbf{B1.} 
Eger normal bólistirilgen bas toplamnan alınǵan kólemi \(n = 13\) ǵa teń bolǵan tańlanba boyınsha \({\overline{S}}^{2} = 1,2\) dúzetilgen tańlanba dispersiya tabılǵan bolsa, onda \(\gamma = 0,90\) isenimlilik penen belgisiz \(\theta_{2}^{2}\) dispersiya ushın isenimlilik interval dúziń.
 \\
\textbf{B2.} 
Eger (3,0,-2,0,-2,3,-2,0,0,3,0,0,0,3,-2,0,0,-2,3,0) tańlanba tómende berilgen bólistiriliwden alınǵan bolsa, onda belgisiz \(\left( \theta_{1},\theta_{2} \right)\) vektor parametr ushın momentler usılı bahalasın tabıń.
\begin{tabular}{|c|c|c|c|}
  \hline
$\xi$ &
$- 2$ &
$0$ &
$3$\\
\hline
\(P_{\theta}\) & \({2\theta}_{1}\) & \(0,5 + \theta_{1} + \theta_{2}\) & \(\theta_{2}\) \\
\hline
\end{tabular}
 \\
\textbf{B3.} 
Eger \(X^{(n)} = \left( X_{1},...,X_{n} \right)\) tańlanba tıǵızlıq funkciyası \(f(x;\theta) = \frac{2x}{\theta}e^{- \frac{x^{2}}{\theta}},\ x \geq 0\). bolǵan bólistiriliwden alınǵan bolsa, onda belgisiz \(\theta > 0\) parametrdiń shınlıqqa maksimal uqsaslıq bahasın tabıń.
 \\
\textbf{C1.} 
Eger \(X^{(n)} = \left( X_{1},...,X_{n} \right)\) tańlanba \(\lbrack - 3\theta,\theta\rbrack\) aralıqta teń ólshemli bólistiriliwden alınǵan bolsa, onda belgisiz \(\theta\) parametr ushın \(4X_{(n)} + X_{(1)}\) bahasın jıljımaǵanlıq hám tiykarlılıqqa tekseriń.
 \\
\textbf{C2.} 
Eger \(X^{(n)} = \left( X_{1},...,X_{n} \right)\) tańlanba \({(\theta,\theta}^{2})\ \) parametrli normal bólistiriliwden alınǵan bolsa, onda belgisiz \(\theta > 0\) parametr ushın momentler usılı bahasın tabıń.
 \\
\textbf{C3.} 
Eger \(X^{(n)} = \left( X_{1},...,X_{n} \right)\) tańlanba \(\left\lbrack \theta_{1},\theta_{2} \right\rbrack\) aralıqta teń ólshemli bólistiriliwden alınǵan bolsa, onda belgisiz \(\left( \theta_{1},\theta_{2} \right)\) vektor parametrdiń shınlıqqa maksimal uqsaslıq bahasın tabıń.
 \\

\end{tabular}
\vspace{1cm}


\begin{tabular}{m{17cm}}
\textbf{60-variant}
\newline

\textbf{T1.} 
Glivenko-Kantelli teoreması. (empirikalıq bólistiriw funkciyası, 1itimallıq penen jaqınlasıw)
 \\
\textbf{T2.} 
Normal nızamnıń dispersiyası ushın isenimlilik intervalın dúziw. (Isenimlilik itimallıǵı, interval)
 \\
\textbf{A1.} 
Kólemi \(n = 20\) ǵa teń bolǵan tańlanba berilgen:-3,3; 0; 4,4; 2,2; -2,7; 4,4; 2,2; 4,4;-3,3; 2,2; -2,7; 2,2; -3,3; -2,7; 2,2; 3,4; 4,4; 0; -3,3; 0. Bul tańlanbanıń statistikalıq bólistiriliwin tabıń.
 \\
\textbf{A2.} 
Kólemi \(n = 20\) ǵa teń bolǵan tańlanba berilgen:-3,3; 0; 4,9; 2,8; -2,6; 4,9; 2,8; 4,9;-3,3; 2,8; -2,6; 2,8; -3,3; -2,6; 2,8; 3,1; 4,9; 0; -3,3; 0. Bul tańlanbanıń empirikalıq bólistiriw funkciyasın tabıń.
 \\
\textbf{A3.} 
Joqarı matematika páninen 10 dana student test sınaqların tapsırǵan. Hárbir student 10 balǵa shekem toplawı múmkin. Eger test sınaqları nátiyjeleri boyınsha \{4, 7, 6, 9, 3, 8, 3, 7, 4, 9\} tańlanba alınǵan bolsa, onda tańlanba ortasha hám tańlanba dispersiyalardı tabıń.
 \\
\textbf{B1.} 
Eger ortasha kvadratlıq shetleniwi \(\sigma = 4\) bolǵan normal bólistirilgen bas toplamnan alınǵan kólemi \(n = 12\) ǵa teń tańlanba boyınsha \(\overline{x} = 3\) tańlanba ortasha mánisi tabılǵan bolsa, onda \(\gamma = 0,95\) isenimlilik penen belgisiz \(\theta\) matematikalıq kútiliwdi qaplaytuǵın isenimlilik intervalın dúziń.
 \\
\textbf{B2.} 
Eger (-2,0,-2,0,-2,3,-2,0,0,3,0,0,0,3,-2,0,0,-2,3,0) tańlanba tómende berilgen bólistiriliwden alınǵan bolsa, onda belgisiz \(\left( \theta_{1},\theta_{2} \right)\) vektor parametr ushın momentler usılı bahalasın tabıń.
\begin{tabular}{|c|c|c|c|}
  \hline
$\xi$ &
$- 2$ &
$0$ &
$3$\\
\hline
\(P_{\theta}\) & \(\theta_{1}\) & \(1 - \theta_{1} - \theta_{2}\) & \(\theta_{2}\) \\
\hline
\end{tabular}
 \\
\textbf{B3.} 
\(\ f(x) = \frac{\theta}{2}e^{- \theta|x|}\) model ushın \(\theta\) parametri haqıyqatqa maksimal uqsaslıq usılı bahası tabılsın.
 \\
\textbf{C1.} 
Eger \(X^{(n)} = \left( X_{1},...,X_{n} \right)\) tańlanba bólistiriw funkciyası \(F(x)\) bolǵan bólistiriliwden alınǵan bolsa, onda belgisiz \(F(x)\) ushın \(F_{n}(x)\) empirikalıq bólistiriw funkciyasın jıljımaǵanlıq hám tiykarlılıqqa tekseriń.
 \\
\textbf{C2.} 
Eger \(X^{(n)} = \left( X_{1},...,X_{n} \right)\) tańlanba \({(\theta,\theta}^{2})\) parametrli normal bólistiriliwden \({\ g(x) = (x)}^{2}\ \)alınǵan bolsa, onda belgisiz \(\theta > 0\) parametr ushın momentler usılı bahasın funkciyası járdeminde tabıń.
 \\
\textbf{C3.} 
Eger \(X^{(n)} = \left( X_{1},...,X_{n} \right)\) tańlanba tıǵızlıq funkciyası
$f(x;\theta) = \left\{ \begin{matrix}
3x^{2}\theta^{- 3}e^{- \ \left( \frac{x}{\theta} \right)^{3}},\ \ x \geq 0, \\
\ \ \ \ \ \ \ \ \ \ \ \ \ \ 0,\ \ \ \ \ \ \ \ \ x < 0
\end{matrix} \right.\ $
bolǵan bólistiriliwden alınǵan bolsa, onda belgisiz \(\theta > 0\) parametrdiń shınlıqqa maksimal uqsaslıq bahasın tabıń.
 \\

\end{tabular}
\vspace{1cm}


\begin{tabular}{m{17cm}}
\textbf{61-variant}
\newline

\textbf{T1.} 
Glivenko-Kantelli teoreması. (empirikalıq bólistiriw funkciyası, 1itimallıq penen jaqınlasıw)
 \\
\textbf{T2.} 
Kolmogorovtıń kelisimlilik belgisi (Kolmogorov teoreması)
 \\
\textbf{A1.} 
Kólemi \(n = 20\) ǵa teń bolǵan tańlanba berilgen: 3,7; 3,1; 4,8; 2,8; 3,1; 4,3; 3,7; 4,3; 2,4; 3,1; 2,4; 4,3; 3,1; 3,7; 4,8; 2,8; 2,4; 2,8; 2,4; 3,1. Bul tańlanbanıń statistikalıq bólistiriliwin tabıń.
 \\
\textbf{A2.} 
Kólemi \(n = 20\) ǵa teń bolǵan tańlanba berilgen: 3,8; 3,4; 4,8; 2,9; 3,4; 4,6; 3,8; 4,6; 2,1; 3,4; 2,1; 4,6; 3,4; 3,8; 4,8; 2,9; 2,1; 2,9; 2,1; 3,4. Bul tańlanbanıń empirikalıq bólistiriw funkciyasın tabıń.
 \\
\textbf{A3.} 
Joqarı matematika páninen 10 dana student test sınaqların tapsırǵan. Hárbir student 10 balǵa shekem toplawı múmkin. Eger test sınaqları nátiyjeleri boyınsha \{6, 5, 6, 9, 5, 7, 10, 5, 9, 8\} tańlanba alınǵan bolsa, onda tańlanba ortasha hám tańlanba dispersiyalardı tabıń.
 \\
\textbf{B1.} 
Eger normal bólistirilgen bas toplamnan alınǵan kólemi \(n = 25\) ǵa teń tańlanba boyınsha \(\overline{x} = 9\) tańlanba ortasha hám \({\overline{S}}^{2} = 0,64\) dúzetilgen tańlanba dispersiyalar tabılǵan bolsa, onda \(\gamma = 0,95\) isenimlilik penen belgisiz \(\theta\) matematikalıq kútiliw ushın isenimlilik interval dúziń.
 \\
\textbf{B2.} 
Eger (3,-2,-2,0,-2,-2,-2,0,-2,3,-2,0,3,0,3,-2,0,-2,3,-2,-2,-2,-2,3,3,3,-2,-2,3,3) tańlanba tómende berilgen bólistiriliwden alınǵan bolsa, onda belgisiz \(\theta\) parametr ushın momentler usılı bahasın \(g(x) = |x|\) funkciyası járdeminde tabıń.
\begin{tabular}{|c|c|c|c|}
  \hline
$\xi$ &
$- 2$ &
$0$ &
$3$ \\
\hline
\(P_{\theta}\) & \(3\theta\) & \(1 - 5\theta\) & \(2\theta\) \\
\hline
\end{tabular}
 \\
\textbf{B3.} 
Eger \(x_{1} = 1,1;\ x_{2} = 2,7;\ldots;x_{100} = 1,5\) tańlanba \(\theta\) parametrli kórsetkishli bólistiriliwden alınǵan bolıp, \(\sum_{k = 1}^{100}x_{k} = 200\) bolsa, onda belgisiz \(\theta\) parametrdiń shınlıqqa maksimal uqsaslıq bahasın tabıń.
 \\
\textbf{C1.} 
Eger \(X^{(n)} = \left( X_{1},...,X_{n} \right)\) tańlanba \(\left( a,\theta^{2} \right)\) parametrli normal bólistiriliwden alınǵan bolsa (\(a -\)belgili), onda belgisiz \(\theta\) parametr ushın \(\sqrt{\frac{\pi}{2}}\left| \overline{x - a} \right|\) bahasın jıljımaǵanlıq hám tiykarlılıqqa tekseriń.
 \\
\textbf{C2.} 
Eger \(X^{(n)} = \left( X_{1},...,X_{n} \right)\) tańlanba \({\lbrack\theta}_{1},\theta_{1}{+ \theta}_{2}\rbrack\) aralıqta teń ólshemli bólistiriliwden alınǵan bolsa, onda belgisiz \(\left( \theta_{1},\theta_{2} \right)\) vektor parametr ushın momentler usılı bahasın tabıń.
 \\
\textbf{C3.} 
Eger \(X^{(n)} = \left( X_{1},...,X_{n} \right)\) tańlanba tıǵızlıq funkciyası
$f(x;\theta) = \left\{ \begin{matrix}
\theta_{1}^{- 1}e^{\frac{x - \theta_{2}}{\theta_{1}}},\ \ x \geq \theta_{2}, \\
\ \ \ \ \ \ \ \ \ \ \ \ 0,\ \ \ \ \ \ \ x < \theta_{2}
\end{matrix} \right.\ $
bolǵan bólistiriliwden alınǵan bolsa, onda belgisiz \(\left( \theta_{1},\theta_{2} \right),\) \(\theta_{1} > 0,\) \(\theta_{2} \in R\) vektor parametrdiń shınlıqqa maksimal uqsaslıq bahasın tabıń.
 \\

\end{tabular}
\vspace{1cm}


\begin{tabular}{m{17cm}}
\textbf{62-variant}
\newline

\textbf{T1.} 
Tańlanba xarakteristikalar. (Variaciyalıq qatar, salıstırmalı jiyilik).
 \\
\textbf{T2.} 
Isenimlilik intervalların qurıw. Anıq isenimli intervallar
 \\
\textbf{A1.} 
Kólemi \(n = 20\) ǵa teń bolǵan tańlanba berilgen: 1,5; -0,9; -2,4; -0,9; 0,7; 1,5; -0,9; -0,2; -2,4; 0,7; -2,4; 0,7; -0,9; 1,5; -1,7; -0,9; -0,2; 0,7; -1,7; -0,9. Bul tańlanbanıń statistikalıq bólistiriliwin tabıń.
 \\
\textbf{A2.} 
Kólemi \(n = 20\) ǵa teń bolǵan tańlanba berilgen: 1,9; -0,3; -2,7; -0,3; 0,6; 1,9; -0,3; -0,1; -2,7; 0,6; -2,7; 0,6; -0,3; 1,9; -1,8; -0,3; -0,1; 0,6; -1,8; -0,3. Bul tańlanbanıń empirikalıq bólistiriw funkciyasın tabıń.
 \\
\textbf{A3.} 
Joqarı matematika páninen 10 dana student test sınaqların tapsırǵan. Hárbir student 10 balǵa shekem toplawı múmkin. Eger test sınaqları nátiyjeleri boyınsha \{4, 6, 6, 9, 5, 8, 4, 7, 5, 6\} tańlanba alınǵan bolsa, onda tańlanba ortasha hám tańlanba dispersiyalardı tabıń.
 \\
\textbf{B1.} 
Eger normal bólistirilgen bas toplamnan alınǵan kólemi \(n = 10\) ǵa teń bolǵan tańlanba boyınsha \({\overline{S}}^{2} = 0,6\) dúzetilgen tańlanba dispersiya tabılǵan bolsa, onda \(\gamma = 0,95\) isenimlilik penen belgisiz \(\theta_{2}^{2}\) dispersiya ushın isenimlilik interval dúziń.
 \\
\textbf{B2.} 
Eger (0,-2,0,-2,3,-2,0,0,3,0,0,0,3,-2,0,0,-2,3,0,3) tańlanba tómende berilgen bólistiriliwden alınǵan bolsa, onda belgisiz \(\theta\) parametr ushın momentler usılı bahasın tabıń.
\begin{tabular}{|c|c|c|c|}
  \hline
$\xi$ & $- 2$  & $0$  & $3$ \\
\hline
\(P_{\theta}\) & \(\theta\) & \(1 - 2\theta\) & \(\theta\) \\
\hline
\end{tabular}
 \\
\textbf{B3.} 
Eger \(X^{(n)} = \left( X_{1},...,X_{n} \right)\) tańlanba tıǵızlıq funkciyası \(f(x;\theta) = \frac{2x}{\theta}e^{- \frac{x^{2}}{\theta}},\ x \geq 0\). bolǵan bólistiriliwden alınǵan bolsa, onda belgisiz \(\theta > 0\) parametrdiń shınlıqqa maksimal uqsaslıq bahasın tabıń.
 \\
\textbf{C1.} 
Eger \(X^{(n)} = \left( X_{1},...,X_{n} \right)\) tańlanba \(\theta\) parametrli kórsetkishli bólistiriliwinen alınǵan bolsa, onda belgisiz \(\theta\) parametr ushın \(\frac{1}{\overline{x}}\) bahasın jıljımaǵanlıq hám tiykarlılıqqa tekseriń.
 \\
\textbf{C2.} 
Eger \(X^{(n)} = \left( X_{1},...,X_{n} \right)\) tańlanba\({\ \ (a,\theta}^{2})\) parametrli normal bólistiriliwden alınǵan bolsa (\(\alpha -\)belgili), onda belgisiz\({\ \ \theta}^{2}\) parametr ushın momentler usılı bahasın tabıń.
 \\
\textbf{C3.} 
Eger \(X^{(n)} = \left( X_{1},...,X_{n} \right)\) tańlanba tıǵızlıq funkciyası
$f(x;\theta) = \frac{3x^{2}}{\sqrt{2\pi}}\exp\left\{ - \frac{\left( x^{3} - \theta \right)^{2}}{2} \right\},\ x \in R$
bolǵan bólistiriliwden alınǵan bolsa, onda belgisiz \(\theta\) parametrdiń shınlıqqa maksimal uqsaslıq bahasın tabıń.
 \\

\end{tabular}
\vspace{1cm}


\begin{tabular}{m{17cm}}
\textbf{63-variant}
\newline

\textbf{T1.} 
Gruppalanǵan hám intervallıq variaciyalıq qatarlar.
 \\
\textbf{T2.} 
Statistikalıq baha qásiyetleri. (Jıljımaytuǵın, tiykarlı, effektiv)
 \\
\textbf{A1.} 
Kólemi \(n = 20\) ǵa teń bolǵan tańlanba berilgen:9,4; 6,8; -8,5; 9,4; 2,9; 9,4; -8,5; -6,4; 6,8; -8,5; 9,4; -6,4; 6,8; 9,4; 2,9; 9,4; -3,6; -8,5; 2,9; -6,4. Bul tańlanbanıń statistikalıq bólistiriliwin tabıń.
 \\
\textbf{A2.} 
Kólemi \(n = 20\) ǵa teń bolǵan tańlanba berilgen:9,1; 6,4; -8,6; 9,1; 2,3; 9,1; -8,6; -6,2; 6,4; -8,6; 9,1; -6,2; 6,4; 9,1; 2,3; 9,1; -3,9; -8,6; 2,3; -6,2. Bul tańlanbanıń empirikalıq bólistiriw funkciyasın tabıń.
 \\
\textbf{A3.} 
Joqarı matematika páninen 10 dana student test sınaqların tapsırǵan. Hárbir student 10 balǵa shekem toplawı múmkin. Eger test sınaqları nátiyjeleri boyınsha \{3, 7, 6, 4, 5, 4, 3, 7, 8, 3\} tańlanba alınǵan bolsa, onda tańlanba ortasha hám tańlanba dispersiyalardı tabıń.
 \\
\textbf{B1.} 
Eger ortasha kvadratlıq shetleniwi \(\sigma = 5\) bolǵan normal bólistirilgen bas toplamnan alınǵan kólemi \(n = 16\) ǵa teń tańlanba boyınsha \(\overline{x} = 3,6\) tańlanba ortasha mánisi tabılǵan bolsa, onda \(\gamma = 0,90\) isenimlilik penen belgisiz \(\theta\) matematikalıq kútiliwdi qaplaytuǵın isenimlilik intervalın dúziń.
 \\
\textbf{B2.} 
Eger \(X^{(n)} = \left( X_{1},...,X_{n} \right)\) tańlanba \(\theta\) parametrli kórsetkishli bólistiriliwden alınǵan bolsa, onda belgisiz \(\theta\) parametr ushın momentler usılı bahasın tabıń.
 \\
\textbf{B3.} 
Eger \(X^{(n)} = \left( X_{1},...,X_{n} \right)\) tańlanba \(\left( a,\theta^{2} \right)\) parametrli normal bólistiriliwden alınǵan bolsa (\(\alpha -\)belgili), onda belgisiz \(\theta^{2}\) parametrdiń shınlıqqa maksimal uqsaslıq bahasın tabıń.
 \\
\textbf{C1.} 
Eger \(X^{(n)} = \left( X_{1},...,X_{n} \right)\) tańlanba \(\frac{1}{\sqrt{\theta}}\) parametrli kórsetkishli bólistiriliwinen alınǵan bolsa, onda belgisiz \(\theta\) parametr ushın \((\overline{x})^{2}\) bahasın jıljımaǵanlıq hám tiykarlılıqqa tekseriń.
 \\
\textbf{C2.} 
Eger \(X^{(n)} = \left( X_{1},...,X_{n} \right)\) tańlanba \((\theta,2\theta)\) parametrli normal bólistiriliwden alınǵan bolsa, onda belgisiz \(\theta > 0\) parametr ushın momentler usılı bahasın \({\ g(x) = (x)}^{2}\) funkciyası járdeminde tabıń.
 \\
\textbf{C3.} 
Eger \(X^{(n)} = \left( X_{1},...,X_{n} \right)\) tańlanba tıǵızlıq funkciyası
$f(x;\theta) = \frac{\theta ln^{\theta - 1}x}{x},\ x \in \lbrack 1,e\rbrack$
bolǵan bólistiriliwden alınǵan bolsa, onda belgisiz \(\theta > 0\) parametr ushın shınlıqqa maksimal uqsaslıq bahasın tabıń.
 \\

\end{tabular}
\vspace{1cm}


\begin{tabular}{m{17cm}}
\textbf{64-variant}
\newline

\textbf{T1.} Matematikalıq statistikanıń tiykarǵı máseleleri. (Statistikalıq maǵlıwmatlar, gruppalaw)
 \\
\textbf{T2.} 
Statistikalıq gipotezalardı tekseriw (kritikalıq kóplik, 1 hám 2-túr qátelik)
 \\
\textbf{A1.} 
Kólemi \(n = 20\) ǵa teń bolǵan tańlanba berilgen: 6,2; -5,3; 7,2; 3,7; -2,2; 6,2; 3,7; -7,6; 3,7; 7,2; 6,2; -5,3; -7,6; -5,3; -7,6; 6,2; 7,2; -2,2; -7,6; 7,2. Bul tańlanbanıń statistikalıq bólistiriliwin tabıń.
 \\
\textbf{A2.} 
Kólemi \(n = 20\) ǵa teń bolǵan tańlanba berilgen: 6,1; -5,8; 7,9; 3,5; -2,5; 6,1; 3,5; -7,2; 3,5; 7,9; 6,1; -5,8; -7,2; -5,8; -7,2; 6,1; 7,9; -2,5; -7,2; 7,9. Bul tańlanbanıń empirikalıq bólistiriw funkciyasın tabıń.
 \\
\textbf{A3.} 
Joqarı matematika páninen 10 dana student test sınaqların tapsırǵan. Hárbir student 10 balǵa shekem toplawı múmkin. Eger test sınaqları nátiyjeleri boyınsha \{10, 8, 6, 5, 4, 8, 10, 7, 5, 7\} tańlanba alınǵan bolsa, onda tańlanba ortasha hám tańlanba dispersiyalardı tabıń.
 \\
\textbf{B1.} 
Eger normal bólistirilgen bas toplamnan alınǵan kólemi \(n = 16\) ǵa teń tańlanba boyınsha \(\overline{x} = 15,2\) tańlanba ortasha hám \({\overline{S}}^{2} = 0,81\) dúzetilgen tańlanba dispersiyalar tabılǵan bolsa, onda \(\gamma = 0,90\) isenimlilik penen belgisiz \(\theta\) matematikalıq kútiliw ushın isenimlilik interval dúziń.
 \\
\textbf{B2.} 
Puasson bólistiriliwi belgisiz \(\theta > 0\) parametri momentlar usuli bahasin tabıń.
 \\
\textbf{B3.} 
Eger (4,8,5,3) tańlanba \(\left( a,\theta^{2} \right)\) parametrli normal bólistiriliwden alınǵan bolsa, onda belgisiz \(\theta^{2}\) parametrdiń shınlıqqa maksimal uqsaslıq bahasın tabıń.
 \\
\textbf{C1.} 
Eger \(X^{(n)} = \left( X_{1},...,X_{n} \right)\) tańlanba \(\sqrt{\theta}\) parametrli Bernulli bólistiriliwinen alınǵan bolsa, onda belgisiz \(\theta\) parametr ushın \((\overline{x})^{2}\) bahasın jıljımaǵanlıq hám tiykarlılıqqa tekseriń.
 \\
\textbf{C2.} 
Eger \(X^{(n)} = \left( X_{1},...,X_{n} \right)\) tańlanba \(\theta\) parametrli geometriyalıq bólistiriliwden alınǵan bolsa, onda belgisiz \(\theta\) parametr ushın momentler usılı bahasın tabıń.
 \\
\textbf{C3.} 
Eger \(X^{(n)} = \left( X_{1},...,X_{n} \right)\) tańlanba \(\theta\) parametrli geometriyalıq bólistiriliwden alınǵan bolsa, onda belgisiz \(\theta\) parametrdiń shınlıqqa maksimal uqsaslıq usılı bahasın tabıń.
 \\

\end{tabular}
\vspace{1cm}


\begin{tabular}{m{17cm}}
\textbf{65-variant}
\newline

\textbf{T1.} 
Tańlanba momentleri (\(k -\)tártipli baslanǵısh, baslanǵısh absolyut, oraylıq hám oraylıq absolyut momentler).
 \\
\textbf{T2.} 
Normal nızamnıń dispersiyası ushın isenimlilik intervalın dúziw. (Isenimlilik itimallıǵı, interval)
 \\
\textbf{A1.} 
Kólemi \(n = 20\) ǵa teń bolǵan tańlanba berilgen: 9,6; 1,5; 7,4; 9,6; 2,8; 1,5; 6,3; 1,5; 9,6; 6,3; 2,8; 4,1; 6,3; 9,6; 1,5; 1,5; 6,3; 7,4; 4,1; 7,4. Bul tańlanbanıń statistikalıq bólistiriliwin tabıń.
 \\
\textbf{A2.} 
Kólemi \(n = 20\) ǵa teń bolǵan tańlanba berilgen: 9,8; 1,2; 7,1; 9,8; 2,9; 1,2; 6,7; 1,2; 9,8; 6,7; 2,9; 4,6; 6,7; 9,8; 1,2; 1,2; 6,7; 7,1; 4,6; 7,1. Bul tańlanbanıń empirikalıq bólistiriw funkciyasın tabıń.
 \\
\textbf{A3.} 
Joqarı matematika páninen 10 dana student test sınaqların tapsırǵan. Hárbir student 10 balǵa shekem toplawı múmkin. Eger test sınaqları nátiyjeleri boyınsha \{9, 10, 5, 6, 4, 8, 4, 6, 10, 8\} tańlanba alınǵan bolsa, onda tańlanba ortasha hám tańlanba dispersiyalardı tabıń.
 \\
\textbf{B1.} 
Eger normal bólistirilgen bas toplamnan alınǵan kólemi \(n = 10\) ǵa teń bolǵan tańlanba boyınsha \({\overline{S}}^{2} = 0,45\) dúzetilgen tańlanba dispersiya tabılǵan bolsa, onda \(\gamma = 0,95\) isenimlilik penen belgisiz \(\theta_{2}^{2}\) dispersiya ushın isenimlilik interval dúziń.
 \\
\textbf{B2.} 
Eger tıǵızlıq funkciyası \(f(x) = \frac{2x}{\theta}e^{- \frac{x^{2}}{\theta}},\ \ x \geq 0\) kóriniske iye bolsa, onda \(\theta\) parametr momentler usulı bahasın tabıń.
 \\
\textbf{B3.} 
\(\ f(x) = \frac{\theta}{2}e^{- \theta|x|}\) model ushın \(\theta\) parametri haqıyqatqa maksimal uqsaslıq usılı bahası tabılsın.
 \\
\textbf{C1.} 
Eger \(X^{(n)} = \left( X_{1},...,X_{n} \right)\) tańlanba \(\theta\) parametrli Bernulli bólistiriliwinen alınǵan bolsa, onda belgisiz \(\theta\) parametr ushın \(X_{n}\) bahasın jıljımaǵanlıq hám tiykarlılıqqa tekseriń.
 \\
\textbf{C2.} 
Eger \(X^{(n)} = \left( X_{1},...,X_{n} \right)\) tańlanba \({\lbrack\theta}_{1},\theta_{2}\rbrack\) aralıqta teń ólshemli bólistiriliwden alınǵan bolsa, onda belgisiz \(\left( \theta_{1},\theta_{2} \right)\) vektor parametr ushın momentler usılı bahasın tabıń.
 \\
\textbf{C3.} 
Eger \(X^{(n)} = \left( X_{1},...,X_{n} \right)\) tańlanba tıǵızlıq funkciyası
$f(x;\theta) = \left\{ \begin{matrix}
e^{\theta - x},\ \ x \geq \theta, \\
\ \ 0,\ \ \ \ \ \ \ x < \theta
\end{matrix} \right.\ $
bolǵan bólistiriliwden alınǵan bolsa, onda belgisiz \(\theta\) parametrdiń shınlıqqa maksimal uqsaslıq bahasın tabıń.
 \\

\end{tabular}
\vspace{1cm}


\begin{tabular}{m{17cm}}
\textbf{66-variant}
\newline

\textbf{T1.} 
Neyman-Pirson teoreması
 \\
\textbf{T2.} 
Pirsonnıń xi-kvadrat kelisimlilik belgisi (Pirson teoreması).
 \\
\textbf{A1.} 
Kólemi \(n = 20\) ǵa teń bolǵan tańlanba berilgen:1,8; -8,4; 7,3; 4,7; -3,9; 1,8; 4,7; -10,4; -8,4; 7,3; -10,4; 4,7; -8,4; 1,8; 4,7; -10,4; 7,3; -3,9; 4,7; -8,4. Bul tańlanbanıń statistikalıq bólistiriliwin tabıń.
 \\
\textbf{A2.} 
Kólemi \(n = 20\) ǵa teń bolǵan tańlanba berilgen:1,6; -8,3; 7,6; 4,2; -3,1; 1,6; 4,2; -10,5; -8,3; 7,6; -10,5; 4,2; -8,3; 1,6; 4,2; -10,5; 7,6; -3,1; 4,2; -8,3. Bul tańlanbanıń empirikalıq bólistiriw funkciyasın tabıń.
 \\
\textbf{A3.} 
Joqarı matematika páninen 10 dana student test sınaqların tapsırǵan. Hárbir student 10 balǵa shekem toplawı múmkin. Eger test sınaqları nátiyjeleri boyınsha \{9, 3, 6, 3, 7, 6, 4, 6, 10, 6\} tańlanba alınǵan bolsa, onda tańlanba ortasha hám tańlanba dispersiyalardı tabıń.
 \\
\textbf{B1.} 
Eger ortasha kvadratlıq shetleniwi \(\sigma = 2\) bolǵan normal bólistirilgen bas toplamnan alınǵan kólemi \(n = 18\) ǵa teń tańlanba boyınsha \(\overline{x} = 5,2\) tańlanba ortasha mánisi tabılǵan bolsa, onda \(\gamma = 0,90\) isenimlilik penen belgisiz \(\theta\) matematikalıq kútiliwdi qaplaytuǵın isenimlilik intervalın dúziń.
 \\
\textbf{B2.} 
Eger \(X^{(n)} = \left( X_{1},...,X_{n} \right)\) tańlanba \(\theta\) parametrli Bernulli bólistiriliwinen alınǵan bolsa, onda belgisiz \(\theta\) parametr ushın momentler usılı bahasın tabıń.
 \\
\textbf{B3.} 
Eger \(X^{(n)} = \left( X_{1},...,X_{n} \right)\) tańlanba \(\left\lbrack - \theta,\theta^{2} \right\rbrack\) aralıqta teń ólshemli bólistiriliwden alınǵan bolsa, onda belgisiz \(\theta > 0\) parametrdiń shınlıqqa maksimal uqsaslıq usılı bahasın tabıń.
 \\
\textbf{C1.} 
Eger \(X^{(n)} = \left( X_{1},...,X_{n} \right)\) tańlanba \(\theta\) parametrli Bernulli bólistiriliwinen alınǵan bolsa, onda belgisiz \(\theta(1 - \theta)\) parametr ushın \(X_{1}\left( 1 - X_{n} \right)\) bahasın jıljımaǵanlıq hám tiykarlılıqqa tekseriń.
 \\
\textbf{C2.} 
Eger \(X^{(n)} = \left( X_{1},...,X_{n} \right)\) tańlanba \(\frac{1}{\theta}\) parametrli kórsetkishli bólistiriliwden alınǵan bolsa, onda belgisiz \(\theta\) parametr ushın momentler usılı bahasın\({\ g(x) = x}^{k},\) \(k \in N\)funkciyası járdeminde tabıń.
 \\
\textbf{C3.} 
\(\ f(x;\theta) = \frac{7x^{6}}{\sqrt{2\pi}}exp\{ - \frac{(x^{7} - \theta)^{2}}{2}\}\) model ushın \(\theta\) parametri haqıyqatqa maksimal uqsaslıq usılı bahası tabılsın.
 \\

\end{tabular}
\vspace{1cm}


\begin{tabular}{m{17cm}}
\textbf{67-variant}
\newline

\textbf{T1.} 
Tańlanba xarakteristikaları.(tańlanba orta, tańlanba dispersiya)
 \\
\textbf{T2.} 
Haqiyqatqa maksimal uqsaslıq usulı. (haqiyqatqa maksimal uqsaslıq funkciyası, belgisiz parametrlerdi bahalaw).
 \\
\textbf{A1.} 
Kólemi \(n = 20\) ǵa teń bolǵan tańlanba berilgen: 2,7; -13,5; 1,2; 2,7; 1,2; 4,9; -9,5; 1,2; 2,7; 4,9; -9,5; 2,7; -3,5; 1,2; 2,7; 4,9; -3,5; 2,7; 4,9; 1,2;. Bul tańlanbanıń statistikalıq bólistiriliwin tabıń.
 \\
\textbf{A2.} 
Kólemi \(n = 20\) ǵa teń bolǵan tańlanba berilgen: 2,8; -13,9; 1,9; 2,8; 1,9; 4,3; -9,4; 1,9; 2,8; 4,3; -9,4; 2,8; -3,7; 1,9; 2,8; 4,3; -3,7; 2,8; 4,3; 1,9. Bul tańlanbanıń empirikalıq bólistiriw funkciyasın tabıń.
 \\
\textbf{A3.} 
Joqarı matematika páninen 10 dana student test sınaqların tapsırǵan. Hárbir student 10 balǵa shekem toplawı múmkin. Eger test sınaqları nátiyjeleri boyınsha \{10, 7, 5, 9, 3, 8, 10, 7, 8, 3\} tańlanba alınǵan bolsa, onda tańlanba ortasha hám tańlanba dispersiyalardı tabıń.
 \\
\textbf{B1.} 
Eger normal bólistirilgen bas toplamnan alınǵan kólemi \(n = 36\) ǵa teń tańlanba boyınsha \(\overline{x} = 20,2\) tańlanba ortasha hám \({\overline{S}}^{2} = 0,81\) dúzetilgen tańlanba dispersiyalar tabılǵan bolsa, onda \(\gamma = 0,95\) isenimlilik penen belgisiz \(\theta\) matematikalıq kútiliw ushın isenimlilik interval dúziń.
 \\
\textbf{B2.} 
Eger (0,-2,0,-2,3,-2,0,0,3,0,0,0,3,-2,0,0,-2,3,0,3) tańlanba tómende berilgen bólistiriliwden alınǵan bolsa, onda belgisiz \(\theta\) parametr ushın momentler usılı bahasın tabıń.
\begin{tabular}{|c|c|c|c|}
  \hline
$\xi$ & $- 2$  & $0$  & $3$ \\
\hline
\(P_{\theta}\) & \(\theta\) & \(1 - 2\theta\) & \(\theta\) \\
\hline
\end{tabular}
 \\
\textbf{B3.} 
\(\ f(x) = \frac{2x}{\theta}e^{- \frac{x^{2}}{\theta}},\ \ x \geq 0\) model ushın \(\theta\) parametri haqıyqatqa maksimal uqsaslıq usılı bahası tabılsın.
 \\
\textbf{C1.} 
Eger \(X^{(n)} = \left( X_{1},...,X_{n} \right)\) tańlanba \(\theta\) parametrli Bernulli bólistiriliwinen alınǵan bolsa, onda belgisiz \(\theta^{2}\) parametr ushın \(X_{1}X_{n}\) bahasın jıljımaǵanlıq hám tiykarlılıqqa tekseriń.
 \\
\textbf{C2.} 
Eger \(X^{(n)} = \left( X_{1},...,X_{n} \right)\) tańlanba {[}\(0,2\theta\rbrack\) aralıqta teń ólshemli bólistiriliwden alınǵan bolsa, onda belgisiz \(\theta > 0\) parametr ushın momentler usılı bahasın tabıń.
 \\
\textbf{C3.} 
Eger \(X^{(n)} = \left( X_{1},...,X_{n} \right)\) tańlanba tıǵızlıq funkciyası
$f(x;\theta) = \frac{1}{2}e^{- \ |x - \theta|},\ x \in R$
bolǵan Laplas bólistiriliwinen alınǵan bolsa, onda belgisiz \(\theta \in R\) parametrdiń shınlıqqa maksimal uqsaslıq bahasın tabıń.
 \\

\end{tabular}
\vspace{1cm}


\begin{tabular}{m{17cm}}
\textbf{68-variant}
\newline

\textbf{T1.} 
Poligon hám gistogramma(salıstirmalı jiyilik, intervallıq qatar, grafik)
 \\
\textbf{T2.} 
Statistikalıq gipotezalardı tekseriw (kritikalıq kóplik, 1 hám 2-túr qátelik).
 \\
\textbf{A1.} 
Kólemi \(n = 20\) ǵa teń bolǵan tańlanba berilgen: 9,9; 5,7; 3,2; 2,8; 5,7; 9,9; 7,5; 3,7; 9,9; 3,2; 2,8; 3,7; 7,5; 5,7; 3,2; 2,8; 7,5; 3,2; 9,9; 7,5. Bul tańlanbanıń statistikalıq bólistiriliwin tabıń.
 \\
\textbf{A2.} 
Kólemi \(n = 20\) ǵa teń bolǵan tańlanba berilgen: 9,7; 5,2; 3,2; 2,4; 5,2; 9,7; 7,5; 3,7; 9,7; 3,2; 2,4; 3,7; 7,5; 5,2; 3,2; 2,4; 7,5; 3,2; 9,7; 7,5. Bul tańlanbanıń empirikalıq bólistiriw funkciyasın tabıń.
 \\
\textbf{A3.} 
Joqarı matematika páninen 10 dana student test sınaqların tapsırǵan. Hárbir student 10 balǵa shekem toplawı múmkin. Eger test sınaqları nátiyjeleri boyınsha \{1, 6, 2, 6, 3, 6, 4, 6, 10, 6\} tańlanba alınǵan bolsa, onda tańlanba ortasha hám tańlanba dispersiyalardı tabıń.
 \\
\textbf{B1.} 
Eger normal bólistirilgen bas toplamnan alınǵan kólemi \(n = 10\) ǵa teń bolǵan tańlanba boyınsha \({\overline{S}}^{2} = 0,7\) dúzetilgen tańlanba dispersiya tabılǵan bolsa, onda \(\gamma = 0,95\) isenimlilik penen belgisiz \(\theta_{2}^{2}\) dispersiya ushın isenimlilik interval dúziń.
 \\
\textbf{B2.} 
\(\lbrack 0,\theta\rbrack\) aralıqta teń ólshewli bólistirilgen \(\theta\) parametri ushın momentler usulı bahasın tabıń.
 \\
\textbf{B3.} 
Eger \(X^{(n)} = \left( X_{1},...,X_{n} \right)\) tańlanba \(\theta\) parametrli kórsetkishli bólistiriliwden alınǵan bolsa, onda belgisiz \(\theta\) parametrdiń shınlıqqa maksimal uqsaslıq usılı bahasın tabıń.
 \\
\textbf{C1.} 
Eger \(X^{(n)} = \left( X_{1},...,X_{n} \right)\) tańlanba \((\alpha,\theta)\) parametrli Veybull bólistiriliwinen alınǵan bolsa (\(\alpha -\)belgili), onda belgisiz \(\theta\) parametr ushın \(\frac{1}{\overline{x^{\alpha}}}\) bahasın jıljımaǵanlıq hám tiykarlılıqqa tekseriń.
 \\
\textbf{C2.} 
Eger \(X^{(n)} = \left( X_{1},...,X_{n} \right)\) tańlanba \(\frac{1}{\sqrt{\theta}}\) parametrli kórsetkishli bólistiriliwden alınǵan bolsa, onda belgisiz \(\theta\) parametr ushın momentler usılı bahasın tabıń.
 \\
\textbf{C3.} 
Eger \(X^{(n)} = \left( X_{1},...,X_{n} \right)\) tańlanba tıǵızlıq funkciyası
$f(x;\theta) = \frac{\theta}{2}e^{- \theta|x|},\ x \in R$
bolǵan bólistiriliwden alınǵan bolsa, onda belgisiz \(\theta > 0\) parametrdiń shınlıqqa maksimal uqsaslıq bahasın tabıń.
 \\

\end{tabular}
\vspace{1cm}


\begin{tabular}{m{17cm}}
\textbf{69-variant}
\newline

\textbf{T1.} 
Empirikalıq bólistiriw funkciyası. (Tańlanba, eksperiment)
 \\
\textbf{T2.} 
Sızıqlı korrelyaciya teńlemesi (anıqlaması, regressiya tuwrı sızıǵınıń tańlanba teńlemeleri)
 \\
\textbf{A1.} 
Kólemi \(n = 20\) ǵa teń bolǵan tańlanba berilgen: 3,6; 1,1; -1,8; 0,4; 3,6; 0; 5,3; 1,1; 0; -1,8; 3,6; 0,4; 1,1; 0; 0,4; 1,1; 3,6; -1,8; 3,6; 0. Bul tańlanbanıń statistikalıq bólistiriliwin tabıń.
 \\
\textbf{A2.} 
Kólemi \(n = 20\) ǵa teń bolǵan tańlanba berilgen: 3,2; 1,8; -1,1; 0,9; 3,2; 0; 5,6; 1,8; 0; -1,1; 3,2; 0,9; 1,8; 0; 0,9; 1,8; 3,2; -1,1; 3,2; 0. Bul tańlanbanıń empirikalıq bólistiriw funkciyasın tabıń.
 \\
\textbf{A3.} 
Joqarı matematika páninen 10 dana student test sınaqların tapsırǵan. Hárbir student 10 balǵa shekem toplawı múmkin. Eger test sınaqları nátiyjeleri boyınsha \{2, 7, 3, 7, 6, 7, 4, 7, 7, 10\} tańlanba alınǵan bolsa, onda tańlanba ortasha hám tańlanba dispersiyalardı tabıń.
 \\
\textbf{B1.} 
Eger ortasha kvadratlıq shetleniwi \(\sigma = 3\) bolǵan normal bólistirilgen bas toplamnan alınǵan kólemi \(n = 14\) ǵa teń tańlanba boyınsha \(\overline{x} = 5,5\) tańlanba ortasha mánisi tabılǵan bolsa, onda \(\gamma = 0,90\) isenimlilik penen belgisiz \(\theta\) matematikalıq kútiliwdi qaplaytuǵın isenimlilik intervalın dúziń.
 \\
\textbf{B2.} 
Puasson bólistiriliwi belgisiz \(\theta > 0\) parametri momentlar usuli bahasin tabıń.
 \\
\textbf{B3.} 
Eger \(X^{(n)} = \left( X_{1},...,X_{n} \right)\) tańlanba \(\lbrack - \theta,\theta\rbrack\) aralıqta teń ólshemli bólistiriliwden alınǵan bolsa, onda belgisiz \(\theta > 0\) parametrdiń shınlıqqa maksimal uqsaslıq usılı bahasın tabıń.
 \\
\textbf{C1.} 
Eger \(X^{(n)} = \left( X_{1},...,X_{n} \right)\) tańlanba \(\theta\) parametrli geometriyalıq bólistiriliwden alınǵan bolsa, onda belgisiz \(\theta\) parametr ushın \(\frac{1}{(1 + \overline{x})}\) bahasın jıljımaǵanlıq hám tiykarlılıqqa tekseriń.
 \\
\textbf{C2.} 
Eger \(X^{(n)} = \left( X_{1},...,X_{n} \right)\) tańlanba tıǵızlıq funkciyası
${f(x,\theta) = \theta x}^{\theta - 1},x \in \lbrack 0,1\rbrack$
bolǵan bólistiriliwden alınǵan bolsa, onda belgisiz \(\theta\) parametr ushın momentler usılı bahasın tabıń.
 \\
\textbf{C3.} 
Eger \(X^{(n)} = \left( X_{1},...,X_{n} \right)\) tańlanba \((\theta,2\theta)\) parametrli normal bólistiriliwden alınǵan bolsa, onda belgisiz \(\theta > 0\) parametrdiń shınlıqqa maksimal uqsaslıq bahasın tabıń.
 \\

\end{tabular}
\vspace{1cm}


\begin{tabular}{m{17cm}}
\textbf{70-variant}
\newline

\textbf{T1.} 
Momentler usulı. (tańlanba momentleri, belgisiz parametrlerdi bahalaw).
 \\
\textbf{T2.} 
Momentler usulı. (tańlanba momentleri, belgisiz parametrlerdi bahalaw).
 \\
\textbf{A1.} 
Kólemi \(n = 20\) ǵa teń bolǵan tańlanba berilgen: 7,1; 3,9; 6,3; 4,6; 7,1; 2,3; 6,3; 3,9; 4,6; 7,1; 2,3; 3,9; 7,6; 2,3; 4,6; 3,9; 2,3; 3,9; 7,6; 4,6. Bul tańlanbanıń statistikalıq bólistiriliwin tabıń.
 \\
\textbf{A2.} 
Kólemi \(n = 20\) ǵa teń bolǵan tańlanba berilgen: 7,9; 3,8; 6,1; 4,2; 7,9; 2,4; 6,1; 3,8; 4,2; 7,9; 2,4; 3,8; 10,2; 2,4; 4,2; 3,8; 2,4; 3,8; 10,2; 4,2. Bul tańlanbanıń empirikalıq bólistiriw funkciyasın tabıń.
 \\
\textbf{A3.} 
Joqarı matematika páninen 10 dana student test sınaqların tapsırǵan. Hárbir student 10 balǵa shekem toplawı múmkin. Eger test sınaqları nátiyjeleri boyınsha \{9, 8, 6, 8, 6, 4, 5, 4, 7, 4\} tańlanba alınǵan bolsa, onda tańlanba ortasha hám tańlanba dispersiyalardı tabıń.
 \\
\textbf{B1.} 
Eger normal bólistirilgen bas toplamnan alınǵan kólemi \(n = 49\) ǵa teń tańlanba boyınsha \(\overline{x} = 14,2\) tańlanba ortasha hám \({\overline{S}}^{2} = 0,64\) dúzetilgen tańlanba dispersiyalar tabılǵan bolsa, onda \(\gamma = 0,95\) isenimlilik penen belgisiz \(\theta\) matematikalıq kútiliw ushın isenimlilik interval dúziń.
 \\
\textbf{B2.} 
Eger (-2,0,-2,0,-2,3,-2,0,0,3,0,0,0,3,-2,0,0,-2,3,0) tańlanba tómende berilgen bólistiriliwden alınǵan bolsa, onda belgisiz \(\left( \theta_{1},\theta_{2} \right)\) vektor parametr ushın momentler usılı bahalasın tabıń.
\begin{tabular}{|c|c|c|c|}
  \hline
$\xi$ &
$- 2$ &
$0$ &
$3$\\
\hline
\(P_{\theta}\) & \(\theta_{1}\) & \(1 - \theta_{1} - \theta_{2}\) & \(\theta_{2}\) \\
\hline
\end{tabular}
 \\
\textbf{B3.} 
Eger \(X^{(n)} = \left( X_{1},...,X_{n} \right)\) tańlanba \(\theta\) parametrli Bernulli bólistiriliwinen alınǵan bolsa, onda belgisiz \(\theta\) parametrdiń shınlıqqa maksimal uqsaslıq usılı bahasın tabıń.
 \\
\textbf{C1.} 
Eger \(X^{(n)} = \left( X_{1},...,X_{n} \right)\) tańlanba \(\theta\) parametrli Puasson bólistiriliwinen alınǵan bolsa, onda belgisiz \(\theta\) parametr ushın \(\frac{n + 3}{n + 4}\overline{x}\) bahasın jıljımaǵanlıq hám tiykarlılıqqa tekseriń.
 \\
\textbf{C2.} 
Eger \(X^{(n)} = \left( X_{1},...,X_{n} \right)\) tańlanba \(\left( \theta_{1},\theta_{2} \right)\) parametrli gamma bólistiriliwden alınǵan bolsa, onda belgisiz \(\left( \theta_{1},\theta_{2} \right)\) vektor parametr ushın momentler usılı bahasın tabıń.
 \\
\textbf{C3.} 
Eger \(X^{(n)} = \left( X_{1},...,X_{n} \right)\) tańlanba \(\lbrack\theta,\theta + 2\rbrack\) aralıqta teń ólshemli bólistiriliwden alınǵan bolsa, onda belgisiz \(\theta\) parametrdiń shınlıqqa maksimal uqsaslıq usılı bahasın tabıń.
 \\

\end{tabular}
\vspace{1cm}


\begin{tabular}{m{17cm}}
\textbf{71-variant}
\newline

\textbf{T1.} 
Glivenko-Kantelli teoreması. (empirikalıq bólistiriw funkciyası, 1itimallıq penen jaqınlasıw)
 \\
\textbf{T2.} 
Normal nızamnıń dispersiyası ushın isenimlilik intervalın dúziw. (Isenimlilik itimallıǵı, interval)
 \\
\textbf{A1.} 
Kólemi \(n = 20\) ǵa teń bolǵan tańlanba berilgen: 0,6; -3,8; -2,3; -4,3; 2,8; 4,7; -2,3; 0,6; -3,8; 2,8; -2,3; -4,3; 0,6; -2,3; 2,8; -3,8; -4,3; -2,3; 2,8; -3,8. Bul tańlanbanıń statistikalıq bólistiriliwin tabıń.
 \\
\textbf{A2.} 
Kólemi \(n = 20\) ǵa teń bolǵan tańlanba berilgen: 0,7; -3,1; -2,3; -4,8; 2,6; 4,9; -2,3; 0,7; -3,1; 2,6; -2,3; -4,8; 0,7; -2,3; 2,6; -3,1; -4,8; -2,3; 2,6; -3,1. Bul tańlanbanıń empirikalıq bólistiriw funkciyasın tabıń.
 \\
\textbf{A3.} 
Joqarı matematika páninen 10 dana student test sınaqların tapsırǵan. Hárbir student 10 balǵa shekem toplawı múmkin. Eger test sınaqları nátiyjeleri boyınsha \{10, 4, 6, 5, 5, 4, 10, 7, 9, 10\} tańlanba alınǵan bolsa, onda tańlanba ortasha hám tańlanba dispersiyalardı tabıń.
 \\
\textbf{B1.} 
Eger normal bólistirilgen bas toplamnan alınǵan kólemi \(n = 8\) ǵa teń bolǵan tańlanba boyınsha \({\overline{S}}^{2} = 0,35\) dúzetilgen tańlanba dispersiya tabılǵan bolsa, onda \(\gamma = 0,90\) isenimlilik penen belgisiz \(\theta_{2}^{2}\) dispersiya ushın isenimlilik interval dúziń.
 \\
\textbf{B2.} 
Eger \(X^{(n)} = \left( X_{1},...,X_{n} \right)\) tańlanba \(\theta\) parametrli kórsetkishli bólistiriliwden alınǵan bolsa, onda belgisiz \(\theta\) parametr ushın momentler usılı bahasın tabıń.
 \\
\textbf{B3.} 
Eger (-1,-1,0,-1,0,-1,-1,5,-1,0,-1,0,5,-1,-1,-1,5,-1,-1,-1,5,0,-1,-1,5) tańlanba tómende berilgen bólistiriliwden alınǵan bolsa, onda belgisiz \(\theta\) parametrdiń shınlıqqa maksimal uqsaslıq usılı bahasın tabıń.
\begin{tabular}{|c|c|c|c|}
  \hline
$\xi$
&
$- 1$
&
$0$
&
$5$\\
\hline
\(P_{\theta}\) & \(1 - \theta\) & \(\theta/2\) & \(\theta/2\ \) \\
\hline
\end{tabular}
 \\
\textbf{C1.} 
Eger \(X^{(n)} = \left( X_{1},...,X_{n} \right)\) tańlanba \(\theta\) parametrli Puasson bólistiriliwinen alınǵan bolsa, onda belgisiz \(\theta\) parametr ushın \(\frac{X_{1} + X_{3}}{2}\) bahasın jıljımaǵanlıq hám tiykarlılıqqa tekseriń.
 \\
\textbf{C2.} 
Eger \(X^{(n)} = \left( X_{1},...,X_{n} \right)\) tańlanba \(\theta\) parametrli Puasson bólistiriliwinen alınǵan bolsa, onda belgisiz \(\theta\) parametr ushın momentler usılı bahasın tabıń. Eger \(X^{(n)} = \left( X_{1},...,X_{n} \right)\) tańlanba \(\theta\) parametrli Puasson bólistiriliwinen alınǵan bolsa, onda belgisiz \(\theta\) parametr ushın momentler usılı bahasın\({\ g(x) = x}^{2}\) funkciyası járdeminde tabıń.
 \\
\textbf{C3.} 
Eger \(X^{(n)} = \left( X_{1},...,X_{n} \right)\) tańlanba tıǵızlıq funkciyası
$f(x;\theta) = \frac{\theta}{\sqrt{2\pi x^{3}}}e^{\frac{- \ \theta^{2}}{2x}},\ x \geq 0$
bolǵan bólistiriliwden alınǵan bolsa, onda belgisiz \(\theta > 0\) parametrdiń shınlıqqa maksimal uqsaslıq bahasın tabıń.
 \\

\end{tabular}
\vspace{1cm}


\begin{tabular}{m{17cm}}
\textbf{72-variant}
\newline

\textbf{T1.} 
Tańlanba momentleri (\(k -\)tártipli baslanǵısh, baslanǵısh absolyut, oraylıq hám oraylıq absolyut momentler).
 \\
\textbf{T2.} 
Pirsonnıń xi-kvadrat kelisimlilik belgisi (Pirson teoreması).
 \\
\textbf{A1.} 
Kólemi \(n = 20\) ǵa teń bolǵan tańlanba berilgen: 8,9; 2,7; 1,7; 2,2; 5,6; 1,7; 5,6; 2,7; 1,7; 2,2; 5,6; 8,9; 1,7; 2,2; 1,7; 2,7; 1,7; 5,6; 6,1; 8,9. Bul tańlanbanıń statistikalıq bólistiriliwin tabıń.
 \\
\textbf{A2.} 
Kólemi \(n = 20\) ǵa teń bolǵan tańlanba berilgen: 8,7; 2,7; 1,5; 2,2; 5,7; 1,5; 5,7; 2,7; 1,5; 2,2; 5,7; 8,7; 1,5; 2,2; 1,5; 2,7; 1,5; 5,7; 6,3; 8,7. Bul tańlanbanıń empirikalıq bólistiriw funkciyasın tabıń.
 \\
\textbf{A3.} 
Joqarı matematika páninen 10 dana student test sınaqların tapsırǵan. Hárbir student 10 balǵa shekem toplawı múmkin. Eger test sınaqları nátiyjeleri boyınsha \{9, 8, 6, 9, 5, 4, 5, 7, 8, 9\} tańlanba alınǵan bolsa, onda tańlanba ortasha hám tańlanba dispersiyalardı tabıń.
 \\
\textbf{B1.} 
Eger ortasha kvadratlıq shetleniwi \(\sigma = 4\) bolǵan normal bólistirilgen bas toplamnan alınǵan kólemi \(n = 16\) ǵa teń tańlanba boyınsha \(\overline{x} = 5,8\) tańlanba ortasha mánisi tabılǵan bolsa, onda \(\gamma = 0,90\) isenimlilik penen belgisiz \(\theta\) matematikalıq kútiliwdi qaplaytuǵın isenimlilik intervalın dúziń.
 \\
\textbf{B2.} 
Eger (3,-2,-2,0,-2,-2,-2,0,-2,3,-2,0,3,0,3,-2,0,-2,3,-2,-2,-2,-2,3,3,3,-2,-2,3,3) tańlanba tómende berilgen bólistiriliwden alınǵan bolsa, onda belgisiz \(\theta\) parametr ushın momentler usılı bahasın \(g(x) = |x|\) funkciyası járdeminde tabıń.
\begin{tabular}{|c|c|c|c|}
  \hline
$\xi$ &
$- 2$ &
$0$ &
$3$ \\
\hline
\(P_{\theta}\) & \(3\theta\) & \(1 - 5\theta\) & \(2\theta\) \\
\hline
\end{tabular}
 \\
\textbf{B3.} 
Eger (0,1,2,0) tańlanba tómende berilgen bólistiriliwden alınǵan bolsa, onda belgisiz \(\theta\) parametrdiń shınlıqqa maksimal uqsaslıq bahasın tabıń.
\begin{tabular}{|c|c|c|c|}
  \hline
$\xi$
&
$0$
&
$1$
&
$2$\\
\hline
\(P_{\theta}\) & \(\theta\) & \(2\theta\) & \(1 - 3\theta\) \\
\hline
\end{tabular}
 \\
\textbf{C1.} 
Eger \(X^{(n)} = \left( X_{1},...,X_{n} \right)\) tańlanba \(\ln\theta\) parametrli Puasson bólistiriliwinen alınǵan bolsa, onda belgisiz \(\theta\) parametr ushın \(e^{\overline{x}}\) bahasın jıljımaǵanlıq hám tiykarlılıqqa tekseriń.
 \\
\textbf{C2.} 
Eger \(X^{(n)} = \left( X_{1},...,X_{n} \right)\) tańlanba tıǵızlıq funkciyası
$f(x,\theta) = \frac{2x}{\theta^{2}},x \in \lbrack 0,\theta\rbrack$
bolǵan bólistiriliwden alınǵan bolsa, onda belgisiz \(\theta\) parametr ushın momentler usılı bahasın tabıń.
 \\
\textbf{C3.} 
\(\ f(x,\theta) = \frac{e^{x}}{\sqrt{2\pi}}\exp\left\{ - \frac{\left( e^{x} - \theta \right)^{2}}{2} \right\}\) model ushın \(\theta\) parametri haqıyqatqa maksimal uqsaslıq usılı bahası tabılsın.
 \\

\end{tabular}
\vspace{1cm}


\begin{tabular}{m{17cm}}
\textbf{73-variant}
\newline

\textbf{T1.} 
Gruppalanǵan hám intervallıq variaciyalıq qatarlar.
 \\
\textbf{T2.} 
Haqiyqatqa maksimal uqsaslıq usulı. (haqiyqatqa maksimal uqsaslıq funkciyası, belgisiz parametrlerdi bahalaw).
 \\
\textbf{A1.} 
Kólemi \(n = 20\) ǵa teń bolǵan tańlanba berilgen: 1,8; -1,9; 2,4; 1,8; 2,4; 1,8; 2,4; -0,6; -1,9; 1,8; -0,6; 2,4; -3,3; -1,9; 4,0; -3,3; -3,3; -1,9; -3,3; -1,9. Bul tańlanbanıń statistikalıq bólistiriliwin tabıń.
 \\
\textbf{A2.} 
Kólemi \(n = 20\) ǵa teń bolǵan tańlanba berilgen: 1,4; -1,9; 2,5; 1,4; 2,5; 1,4; 2,5; -0,4; -1,9; 1,4; -0,4; 2,5; -3,7; -1,9; 4,5; -3,7; -3,7; -1,9; -3,7; -1,9. Bul tańlanbanıń empirikalıq bólistiriw funkciyasın tabıń.
 \\
\textbf{A3.} 
Joqarı matematika páninen 10 dana student test sınaqların tapsırǵan. Hárbir student 10 balǵa shekem toplawı múmkin. Eger test sınaqları nátiyjeleri boyınsha \{4, 3, 8, 4, 8, 3, 9, 4, 7, 10\} tańlanba alınǵan bolsa, onda tańlanba ortasha hám tańlanba dispersiyalardı tabıń.
 \\
\textbf{B1.} 
Eger normal bólistirilgen bas toplamnan alınǵan kólemi \(n = 36\) ǵa teń tańlanba boyınsha \(\overline{x} = 20,2\) tańlanba ortasha hám \({\overline{S}}^{2} = 0,64\) dúzetilgen tańlanba dispersiyalar tabılǵan bolsa, onda \(\gamma = 0,90\) isenimlilik penen belgisiz \(\theta\) matematikalıq kútiliw ushın isenimlilik interval dúziń.
 \\
\textbf{B2.} 
Eger \(X^{(n)} = \left( X_{1},...,X_{n} \right)\) tańlanba \(\theta\) parametrli Bernulli bólistiriliwinen alınǵan bolsa, onda belgisiz \(\theta\) parametr ushın momentler usılı bahasın tabıń.
 \\
\textbf{B3.} 
Eger \(X^{(n)} = \left( X_{1},...,X_{n} \right)\) tańlanba \(\left\lbrack - \theta,\theta^{2} \right\rbrack\) aralıqta teń ólshemli bólistiriliwden alınǵan bolsa, onda belgisiz \(\theta > 0\) parametrdiń shınlıqqa maksimal uqsaslıq usılı bahasın tabıń.
 \\
\textbf{C1.} 
Eger \(X^{(n)} = \left( X_{1},...,X_{n} \right)\) tańlanba \((\alpha,\theta)\) parametrli Pareto bólistiriliwinen alınǵan bolsa (\(\alpha -\)belgili), onda belgisiz \(\theta\) parametr ushın \(X_{(1)}\) bahasın jıljımaǵanlıq hám tiykarlılıqqa tekseriń.
 \\
\textbf{C2.} 
Eger \(X^{(n)} = \left( X_{1},...,X_{n} \right)\) tańlanba \({\ \ (a,\theta}^{2})\ \)parametrli normal bólistiriliwden alınǵan bolsa (\(\alpha -\)belgili), onda belgisiz \({\ \theta}^{2}\) parametr ushın momentler usılı bahasın \({\ g(x) = (x - a)}^{2}\) funkciyası járdeminde tabıń.
 \\
\textbf{C3.} 
Eger \(X^{(n)} = \left( X_{1},...,X_{n} \right)\) tańlanba tıǵızlıq funkciyası
$f(x;\theta) = \frac{3x^{2}}{\sqrt{2\pi}}\exp\left\{ - \frac{\left( x^{3} - \theta \right)^{2}}{2} \right\},\ x \in R$
bolǵan bólistiriliwden alınǵan bolsa, onda belgisiz \(\theta\) parametrdiń shınlıqqa maksimal uqsaslıq bahasın tabıń.
 \\

\end{tabular}
\vspace{1cm}


\begin{tabular}{m{17cm}}
\textbf{74-variant}
\newline

\textbf{T1.} Matematikalıq statistikanıń tiykarǵı máseleleri. (Statistikalıq maǵlıwmatlar, gruppalaw)
 \\
\textbf{T2.} 
Statistikalıq gipotezalardı tekseriw (kritikalıq kóplik, 1 hám 2-túr qátelik).
 \\
\textbf{A1.} 
Kólemi \(n = 20\) ǵa teń bolǵan tańlanba berilgen: 2,9; -3,2; 5,3; -4,3; 4,1; 5,3; -1,2; 2,9; -3,2; 4,1; -4,3; 5,3; -3,2; 2,9; -4,3; 4,1; -1,2; 5,3; 2,9; -3,2. Bul tańlanbanıń statistikalıq bólistiriliwin tabıń.
 \\
\textbf{A2.} 
Kólemi \(n = 20\) ǵa teń bolǵan tańlanba berilgen: 2,7; -5,6; 5,2; -8,1; 4,8; 5,2; -1,6; 2,7; -5,6; 4,8; -8,1; 5,2; -5,6; 2,7; -8,1; 4,8; -1,6; 5,2; 2,7; -5,6. Bul tańlanbanıń empirikalıq bólistiriw funkciyasın tabıń.
 \\
\textbf{A3.} 
Joqarı matematika páninen 10 dana student test sınaqların tapsırǵan. Hárbir student 10 balǵa shekem toplawı múmkin. Eger test sınaqları nátiyjeleri boyınsha \{7, 9, 4, 9, 7, 5, 4, 7, 2, 6\} tańlanba alınǵan bolsa, onda tańlanba ortasha hám tańlanba dispersiyalardı tabıń.
 \\
\textbf{B1.} 
Eger normal bólistirilgen bas toplamnan alınǵan kólemi \(n = 11\) ǵa teń bolǵan tańlanba boyınsha \({\overline{S}}^{2} = 0,3\) dúzetilgen tańlanba dispersiya tabılǵan bolsa, onda \(\gamma = 0,95\) isenimlilik penen belgisiz \(\theta_{2}^{2}\) dispersiya ushın isenimlilik interval dúziń.
 \\
\textbf{B2.} 
\(\lbrack\theta_{1},\theta_{2}\rbrack\) aralıqta teń ólshewli bólistiriw parametrleri ushın momentler usulı bahaların tabıń.
 \\
\textbf{B3.} 
Eger \(X^{(n)} = \left( X_{1},...,X_{n} \right)\) tańlanba \(\lbrack - \theta,\theta\rbrack\) aralıqta teń ólshemli bólistiriliwden alınǵan bolsa, onda belgisiz \(\theta > 0\) parametrdiń shınlıqqa maksimal uqsaslıq usılı bahasın tabıń.
 \\
\textbf{C1.} 
Eger \(X^{(n)} = \left( X_{1},...,X_{n} \right)\) tańlanba tıǵızlıq funkciyası: \(f(x;\theta) = e^{- x + \theta}\left( 1 + e^{- x + \theta} \right)^{2},\ x \in R\)
bolǵan bólistiriliwden alınǵan bolsa, onda belgisiz \(\theta\) parametr ushın \(\overline{x}\) bahasın jıljımaǵanlıq hám tiykarlılıqqa tekseriń.
 \\
\textbf{C2.} 
Eger \(X^{(n)} = \left( X_{1},...,X_{n} \right)\) tańlanba \({\lbrack\theta}_{1},\theta_{1}{+ \theta}_{2}\rbrack\) aralıqta teń ólshemli bólistiriliwden alınǵan bolsa, onda belgisiz \(\left( \theta_{1},\theta_{2} \right)\) vektor parametr ushın momentler usılı bahasın tabıń.
 \\
\textbf{C3.} 
Eger \(X^{(n)} = \left( X_{1},...,X_{n} \right)\) tańlanba \(\left\lbrack \theta_{1},\theta_{2} \right\rbrack\) aralıqta teń ólshemli bólistiriliwden alınǵan bolsa, onda belgisiz \(\left( \theta_{1},\theta_{2} \right)\) vektor parametrdiń shınlıqqa maksimal uqsaslıq bahasın tabıń.
 \\

\end{tabular}
\vspace{1cm}


\begin{tabular}{m{17cm}}
\textbf{75-variant}
\newline

\textbf{T1.} 
Poligon hám gistogramma(salıstirmalı jiyilik, intervallıq qatar, grafik)
 \\
\textbf{T2.} 
Sızıqlı korrelyaciya teńlemesi (anıqlaması, regressiya tuwrı sızıǵınıń tańlanba teńlemeleri)
 \\
\textbf{A1.} 
Kólemi \(n = 20\) ǵa teń bolǵan tańlanba berilgen: 14,7; 7,3; 16,6; 9,8; 11,2; 16,6; 6,7; 7,3; 11,2; 14,7; 6,7; 16,6; 7,3; 11,2; 14,7; 16,6; 6,7; 7,3; 11,2; 16,6. Bul tańlanbanıń statistikalıq bólistiriliwin tabıń.
 \\
\textbf{A2.} 
Kólemi \(n = 20\) ǵa teń bolǵan tańlanba berilgen: 14,4; 7,6; 16,7; 9,1; 11,8; 16,7; 6,4; 7,6; 11,8; 14,4; 6,4; 16,7; 7,6; 11,8; 14,4; 16,7; 6,4; 7,6; 11,8; 16,7. Bul tańlanbanıń empirikalıq bólistiriw funkciyasın tabıń.
 \\
\textbf{A3.} 
Joqarı matematika páninen 10 dana student test sınaqların tapsırǵan. Hárbir student 10 balǵa shekem toplawı múmkin. Eger test sınaqları nátiyjeleri boyınsha \{10, 8, 4, 6, 2, 8, 5, 10, 2, 5\} tańlanba alınǵan bolsa, onda tańlanba ortasha hám tańlanba dispersiyalardı tabıń.
 \\
\textbf{B1.} 
Eger ortasha kvadratlıq shetleniwi \(\sigma = 4\) bolǵan normal bólistirilgen bas toplamnan alınǵan kólemi \(n = 49\) ǵa teń tańlanba boyınsha \(\overline{x} = 9,4\) tańlanba ortasha mánisi tabılǵan bolsa, onda \(\gamma = 0,90\) isenimlilik penen belgisiz \(\theta\) matematikalıq kútiliwdi qaplaytuǵın isenimlilik intervalın dúziń.
 \\
\textbf{B2.} 
Eger (3,0,-2,0,-2,3,-2,0,0,3,0,0,0,3,-2,0,0,-2,3,0) tańlanba tómende berilgen bólistiriliwden alınǵan bolsa, onda belgisiz \(\left( \theta_{1},\theta_{2} \right)\) vektor parametr ushın momentler usılı bahalasın tabıń.
\begin{tabular}{|c|c|c|c|}
  \hline
$\xi$ &
$- 2$ &
$0$ &
$3$\\
\hline
\(P_{\theta}\) & \({2\theta}_{1}\) & \(0,5 + \theta_{1} + \theta_{2}\) & \(\theta_{2}\) \\
\hline
\end{tabular}
 \\
\textbf{B3.} 
Eger (0,1,2,0) tańlanba tómende berilgen bólistiriliwden alınǵan bolsa, onda belgisiz \(\theta\) parametrdiń shınlıqqa maksimal uqsaslıq bahasın tabıń.
\begin{tabular}{|c|c|c|c|}
  \hline
$\xi$
&
$0$
&
$1$
&
$2$\\
\hline
\(P_{\theta}\) & \(\theta\) & \(2\theta\) & \(1 - 3\theta\) \\
\hline
\end{tabular}
 \\
\textbf{C1.} 
Eger \(X^{(n)} = \left( X_{1},...,X_{n} \right)\) tańlanba tıǵızlıq funkciyası
$f(x;\theta) = \left\{ \begin{array}{r}
\alpha^{- 1}e^{- \ \frac{x - \theta}{\alpha}},\ \ x \geq \theta, \\
0,\ \ \ \ \ \ \ x < \theta
\end{array} \right.\ $
bolǵan bólistiriliwden alınǵan bolsa (\(\alpha -\)belgili), onda belgisiz \(\theta\) parametr ushın \(X_{(1)}\) bahasın jıljımaǵanlıq hám tiykarlılıqqa tekseriń.
 \\
\textbf{C2.} 
Eger \(X^{(n)} = \left( X_{1},...,X_{n} \right)\) tańlanba\({\ \ (a,\theta}^{2})\) parametrli normal bólistiriliwden alınǵan bolsa (\(\alpha -\)belgili), onda belgisiz\({\ \ \theta}^{2}\) parametr ushın momentler usılı bahasın tabıń.
 \\
\textbf{C3.} 
Eger \(X^{(n)} = \left( X_{1},...,X_{n} \right)\) tańlanba \((\theta,2\theta)\) parametrli normal bólistiriliwden alınǵan bolsa, onda belgisiz \(\theta > 0\) parametrdiń shınlıqqa maksimal uqsaslıq bahasın tabıń.
 \\

\end{tabular}
\vspace{1cm}


\begin{tabular}{m{17cm}}
\textbf{76-variant}
\newline

\textbf{T1.} 
Empirikalıq bólistiriw funkciyası. (Tańlanba, eksperiment)
 \\
\textbf{T2.} 
Isenimlilik intervalların qurıw. Anıq isenimli intervallar
 \\
\textbf{A1.} 
Kólemi \(n = 20\) ǵa teń bolǵan tańlanba berilgen: 4,3; 4,9; 13,4; 13,4; 6,5; 4,9; 4,9; 4,3; 5,1; 6,5; 6,5; 7,0; 4,3; 4,9; 6,5; 6,5; 5,1; 5,1; 4,9; 13,4. Bul tańlanbanıń statistikalıq bólistiriliwin tabıń.
 \\
\textbf{A2.} 
Kólemi \(n = 20\) ǵa teń bolǵan tańlanba berilgen: 4,2; 4,9; 13,8; 13,8; 6,6; 4,9; 4,9; 4,2; 5,3; 6,6; 6,6; 7,5; 4,2; 4,9; 6,6; 6,6; 5,3; 5,3; 4,9; 13,8. Bul tańlanbanıń empirikalıq bólistiriw funkciyasın tabıń.
 \\
\textbf{A3.} 
Joqarı matematika páninen 10 dana student test sınaqların tapsırǵan. Hárbir student 10 balǵa shekem toplawı múmkin. Eger test sınaqları nátiyjeleri boyınsha \{9, 10, 6, 7, 4, 8, 10, 7, 9, 10\} tańlanba alınǵan bolsa, onda tańlanba ortasha hám tańlanba dispersiyalardı tabıń.
 \\
\textbf{B1.} 
Eger ortasha kvadratlıq shetleniwi \(\sigma = 2\) bolǵan normal bólistirilgen bas toplamnan alınǵan kólemi \(n = 10\) ǵa teń tańlanba boyınsha \(\overline{x} = 5,4\) tańlanba ortasha mánisi tabılǵan bolsa, onda \(\gamma = 0,95\) isenimlilik penen belgisiz \(\theta\) matematikalıq kútiliwdi qaplaytuǵın isenimlilik intervalın dúziń.
 \\
\textbf{B2.} 
Kórsetkishli bólistiriw belgisiz \(\theta > 0\) parametri momentlar usulı bahasın tabıń.
 \\
\textbf{B3.} 
Eger \(x_{1} = 1,1;\ x_{2} = 2,7;\ldots;x_{100} = 1,5\) tańlanba \(\theta\) parametrli kórsetkishli bólistiriliwden alınǵan bolıp, \(\sum_{k = 1}^{100}x_{k} = 200\) bolsa, onda belgisiz \(\theta\) parametrdiń shınlıqqa maksimal uqsaslıq bahasın tabıń.
 \\
\textbf{C1.} 
Eger \(X^{(n)} = \left( X_{1},...,X_{n} \right)\) tańlanba \(\lbrack 0,\theta\rbrack\) aralıqta teń ólshemli bólistiriliwden alýnǵan bolsa, onda belgisiz \(\theta\) parametr ushın \((n + 1)X_{(1)}\) bahasın jıljımaǵanlıq hám tiykarlılıqqa tekseriń.
 \\
\textbf{C2.} 
Eger \(X^{(n)} = \left( X_{1},...,X_{n} \right)\) tańlanba \(\theta\) parametrli geometriyalıq bólistiriliwden alınǵan bolsa, onda belgisiz \(\theta\) parametr ushın momentler usılı bahasın tabıń.
 \\
\textbf{C3.} 
Eger \(X^{(n)} = \left( X_{1},...,X_{n} \right)\) tańlanba \(\left( \theta,\theta^{2} \right)\) parametrli normal bólistiriliwden alınǵan bolsa, onda belgisiz \(\theta > 0\) parametrdiń shınlıqqa maksimal uqsaslıq bahasın tabıń.
 \\

\end{tabular}
\vspace{1cm}


\begin{tabular}{m{17cm}}
\textbf{77-variant}
\newline

\textbf{T1.} 
Neyman-Pirson teoreması
 \\
\textbf{T2.} 
Statistikalıq baha qásiyetleri. (Jıljımaytuǵın, tiykarlı, effektiv)
 \\
\textbf{A1.} 
Kólemi \(n = 20\) ǵa teń bolǵan tańlanba berilgen: -2,1; 1,7; 3,3; 3,3; 11,7; 4,7; 1,7; 4,7; -2,1; 4,7; 4,7; 4,7; 8,0; -2,1; 1,7; 4,7; 8,0; 11,7; 1,7; 8,0. Bul tańlanbanıń statistikalıq bólistiriliwin tabıń.
 \\
\textbf{A2.} 
Kólemi \(n = 20\) ǵa teń bolǵan tańlanba berilgen: -2,2; 1,3; 3,8; 3,8; 11,5; 4,1; 1,3; 4,1; -2,2; 4,1; 4,1; 4,1; 8,4; -2,2; 1,3; 4,1; 8,4; 11,5; 1,3; 8,4. Bul tańlanbanıń empirikalıq bólistiriw funkciyasın tabıń.
 \\
\textbf{A3.} 
Joqarı matematika páninen 10 dana student test sınaqların tapsırǵan. Hárbir student 10 balǵa shekem toplawı múmkin. Eger test sınaqları nátiyjeleri boyınsha \{4, 1, 2, 4, 6, 4, 5, 3, 6, 5\} tańlanba alınǵan bolsa, onda tańlanba ortasha hám tańlanba dispersiyalardı tabıń.
 \\
\textbf{B1.} 
Eger normal bólistirilgen bas toplamnan alınǵan kólemi \(n = 16\) ǵa teń tańlanba boyınsha \(\overline{x} = 20,2\) tańlanba ortasha hám \({\overline{S}}^{2} = 0,64\) dúzetilgen tańlanba dispersiyalar tabılǵan bolsa, onda \(\gamma = 0,95\) isenimlilik penen belgisiz \(\theta\) matematikalıq kútiliw ushın isenimlilik interval dúziń.
 \\
\textbf{B2.} 
Eger tıǵızlıq funkciyası \(f(x) = \frac{2x}{\theta}e^{- \frac{x^{2}}{\theta}},\ \ x \geq 0\) kóriniske iye bolsa, onda \(\theta\) parametr momentler usulı bahasın tabıń.
 \\
\textbf{B3.} 
Eger \(X^{(n)} = \left( X_{1},...,X_{n} \right)\) tańlanba \(\left( a,\theta^{2} \right)\) parametrli normal bólistiriliwden alınǵan bolsa (\(\alpha -\)belgili), onda belgisiz \(\theta^{2}\) parametrdiń shınlıqqa maksimal uqsaslıq bahasın tabıń.
 \\
\textbf{C1.} 
Eger \(X^{(n)} = \left( X_{1},...,X_{n} \right)\) tańlanba \(\lbrack 0,\theta\rbrack\) aralıqta teń ólshemli bólistiriliwden alınǵan bolsa, onda belgisiz \(\theta\) parametr ushın \(\frac{n + 1}{n}X_{(n)}\) bahasın jıljımaǵanlıq hám tiykarlılıqqa tekseriń.
 \\
\textbf{C2.} 
Eger \(X^{(n)} = \left( X_{1},...,X_{n} \right)\) tańlanba tıǵızlıq funkciyası
$f(x,\theta) = \left\{ \begin{array}{r}
\theta_{1}^{- 1}e^{- \frac{x - \theta_{2}}{\theta_{1}}},\ \ \ x \geq \theta_{2}, \\
0,\ \ \ x < \theta_{2}
\end{array} \right.\ $
bolǵan bólistiriliwden alınǵan bolsa, onda belgisiz \(\left( \theta_{1},\theta_{2} \right)\) \(\theta_{1} > 0,\) \(\theta_{2} \in R\) vektor parametr ushın momentler usılı bahasın tabıń.
 \\
\textbf{C3.} 
Eger \(X^{(n)} = \left( X_{1},...,X_{n} \right)\) tańlanba tıǵızlıq funkciyası
$f(x;\theta) = \frac{1}{2}e^{- \ |x - \theta|},\ x \in R$
bolǵan Laplas bólistiriliwinen alınǵan bolsa, onda belgisiz \(\theta \in R\) parametrdiń shınlıqqa maksimal uqsaslıq bahasın tabıń.
 \\

\end{tabular}
\vspace{1cm}


\begin{tabular}{m{17cm}}
\textbf{78-variant}
\newline

\textbf{T1.} 
Tańlanba xarakteristikalar. (Variaciyalıq qatar, salıstırmalı jiyilik).
 \\
\textbf{T2.} 
Momentler usulı. (tańlanba momentleri, belgisiz parametrlerdi bahalaw).
 \\
\textbf{A1.} 
Kólemi \(n = 20\) ǵa teń bolǵan tańlanba berilgen: -11,0; -4,1; 0; 2,3; 1,2; 0; 1,2; 2,3; 2,3; 1,2; 2,3; -11,0; 3,4; 1,2; 3,4; 3,4; 0; 3,4; 2,3; 0. Bul tańlanbanıń statistikalıq bólistiriliwin tabıń.
 \\
\textbf{A2.} 
Kólemi \(n = 20\) ǵa teń bolǵan tańlanba berilgen: -11,2; -4,5; 0; 2,9; 1,7; 0; 1,7; 2,9; 2,9; 1,7; 2,9; -11,2; 3,1; 1,7; 3,1; 3,1; 0; 3,1; 2,9; 0. Bul tańlanbanıń empirikalıq bólistiriw funkciyasın tabıń.
 \\
\textbf{A3.} 
Joqarı matematika páninen 10 dana student test sınaqların tapsırǵan. Hárbir student 10 balǵa shekem toplawı múmkin. Eger test sınaqları nátiyjeleri boyınsha \{8, 9, 10, 4, 9, 7, 6, 7, 6, 4\} tańlanba alınǵan bolsa, onda tańlanba ortasha hám tańlanba dispersiyalardı tabıń.
 \\
\textbf{B1.} 
Eger normal bólistirilgen bas toplamnan alınǵan kólemi \(n = 11\) ǵa teń bolǵan tańlanba boyınsha \({\overline{S}}^{2} = 0,5\) dúzetilgen tańlanba dispersiya tabılǵan bolsa, onda \(\gamma = 0,90\) isenimlilik penen belgisiz \(\theta_{2}^{2}\) dispersiya ushın isenimlilik interval dúziń.
 \\
\textbf{B2.} 
Eger (3,0,-2,0,-2,3,-2,0,0,3,0,0,0,3,-2,0,0,-2,3,0) tańlanba tómende berilgen bólistiriliwden alınǵan bolsa, onda belgisiz \(\left( \theta_{1},\theta_{2} \right)\) vektor parametr ushın momentler usılı bahalasın tabıń.
\begin{tabular}{|c|c|c|c|}
  \hline
$\xi$ &
$- 2$ &
$0$ &
$3$\\
\hline
\(P_{\theta}\) & \({2\theta}_{1}\) & \(0,5 + \theta_{1} + \theta_{2}\) & \(\theta_{2}\) \\
\hline
\end{tabular}
 \\
\textbf{B3.} 
Eger \(X^{(n)} = \left( X_{1},...,X_{n} \right)\) tańlanba \(\theta\) parametrli kórsetkishli bólistiriliwden alınǵan bolsa, onda belgisiz \(\theta\) parametrdiń shınlıqqa maksimal uqsaslıq usılı bahasın tabıń.
 \\
\textbf{C1.} 
Eger \(X^{(n)} = \left( X_{1},...,X_{n} \right)\) tańlanba \(M\xi = a\) belgili hám \(M\xi^{2}\) shekli bolǵan bólistiriliwden alınǵan bolsa, onda belgisiz \(D\xi\) dispersiya ushın \({\overline{S}}^{2}\) bahasın jıljımaǵanlıq hám tiykarlılıqqa tekseriń.
 \\
\textbf{C2.} 
Eger \(X^{(n)} = \left( X_{1},...,X_{n} \right)\) tańlanba \(\frac{1}{\sqrt{\theta}}\) parametrli kórsetkishli bólistiriliwden alınǵan bolsa, onda belgisiz \(\theta\) parametr ushın momentler usılı bahasın tabıń.
 \\
\textbf{C3.} 
Eger \(X^{(n)} = \left( X_{1},...,X_{n} \right)\) tańlanba \(\lbrack\theta,\theta + 2\rbrack\) aralıqta teń ólshemli bólistiriliwden alınǵan bolsa, onda belgisiz \(\theta\) parametrdiń shınlıqqa maksimal uqsaslıq usılı bahasın tabıń.
 \\

\end{tabular}
\vspace{1cm}


\begin{tabular}{m{17cm}}
\textbf{79-variant}
\newline

\textbf{T1.} 
Momentler usulı. (tańlanba momentleri, belgisiz parametrlerdi bahalaw).
 \\
\textbf{T2.} 
Kolmogorovtıń kelisimlilik belgisi (Kolmogorov teoreması)
 \\
\textbf{A1.} 
Kólemi \(n = 20\) ǵa teń bolǵan tańlanba berilgen: 2,5; 3,8; 4,3; 2,5; 3,8; 2,5; 3,1; 4,3; 4,3; 5,5; 6,2; 2,5; 3,1; 6,2; 5,5; 6,2; 3,1; 3,1; 6,2; 3,1. Bul tańlanbanıń statistikalıq bólistiriliwin tabıń.
 \\
\textbf{A2.} 
Kólemi \(n = 20\) ǵa teń bolǵan tańlanba berilgen: 2,7; 4,2; 4,8; 2,7; 4,2; 2,7; 3,9; 4,8; 4,8; 5,9; 6,5; 2,7; 3,9; 6,5; 5,9; 6,5; 3,9; 3,9; 6,5; 3,9. Bul tańlanbanıń empirikalıq bólistiriw funkciyasın tabıń.
 \\
\textbf{A3.} 
Joqarı matematika páninen 10 dana student test sınaqların tapsırǵan. Hárbir student 10 balǵa shekem toplawı múmkin. Eger test sınaqları nátiyjeleri boyınsha \{7, 8, 7, 6, 4, 8, 4, 7, 9, 10\} tańlanba alınǵan bolsa, onda tańlanba ortasha hám tańlanba dispersiyalardı tabıń.
 \\
\textbf{B1.} 
Eger ortasha kvadratlıq shetleniwi \(\sigma = 3\) bolǵan normal bólistirilgen bas toplamnan alınǵan kólemi \(n = 9\) ǵa teń tańlanba boyınsha \(\overline{x} = 4,5\) tańlanba ortasha mánisi tabılǵan bolsa, onda \(\gamma = 0,95\) isenimlilik penen belgisiz \(\theta\) matematikalıq kútiliwdi qaplaytuǵın isenimlilik intervalın dúziń.
 \\
\textbf{B2.} 
Puasson bólistiriliwi belgisiz \(\theta > 0\) parametri momentlar usuli bahasin tabıń.
 \\
\textbf{B3.} 
\(\ f(x) = \frac{\theta}{2}e^{- \theta|x|}\) model ushın \(\theta\) parametri haqıyqatqa maksimal uqsaslıq usılı bahası tabılsın.
 \\
\textbf{C1.} 
Eger \(X^{(n)} = \left( X_{1},...,X_{n} \right)\) tańlanba \(M\xi = a\) belgili hám \(M\xi^{2}\) shekli bolǵan bólistiriliwden alınǵan bolsa, onda belgisiz \(D\xi\) dispersiya ushın \(\frac{1}{n - 1}\sum_{i = 1}^{n}\left( X_{i} - a \right)^{2}\) bahasın jıljımaǵanlıq hám tiykarlılıqqa tekseriń.
 \\
\textbf{C2.} 
Eger \(X^{(n)} = \left( X_{1},...,X_{n} \right)\) tańlanba tıǵızlıq funkciyası
${f(x,\theta) = \theta x}^{\theta - 1},x \in \lbrack 0,1\rbrack$
bolǵan bólistiriliwden alınǵan bolsa, onda belgisiz \(\theta\) parametr ushın momentler usılı bahasın tabıń.
 \\
\textbf{C3.} 
Eger \(X^{(n)} = \left( X_{1},...,X_{n} \right)\) tańlanba tıǵızlıq funkciyası
$f(x;\theta) = \left\{ \begin{matrix}
3x^{2}\theta^{- 3}e^{- \ \left( \frac{x}{\theta} \right)^{3}},\ \ x \geq 0, \\
\ \ \ \ \ \ \ \ \ \ \ \ \ \ 0,\ \ \ \ \ \ \ \ \ x < 0
\end{matrix} \right.\ $
bolǵan bólistiriliwden alınǵan bolsa, onda belgisiz \(\theta > 0\) parametrdiń shınlıqqa maksimal uqsaslıq bahasın tabıń.
 \\

\end{tabular}
\vspace{1cm}


\begin{tabular}{m{17cm}}
\textbf{80-variant}
\newline

\textbf{T1.} 
Tańlanba xarakteristikaları.(tańlanba orta, tańlanba dispersiya)
 \\
\textbf{T2.} 
Statistikalıq gipotezalardı tekseriw (kritikalıq kóplik, 1 hám 2-túr qátelik)
 \\
\textbf{A1.} 
Kólemi \(n = 20\) ǵa teń bolǵan tańlanba berilgen: -4,3; 2,6; 0; -2,5; 2,6; 1,9; 2,2; 0; -4,3; -2,5; 1,9; -2,5; 1,9; 2,2; 2,6; 1,9; 2,6; 2,2; 2,2; 1,9. Bul tańlanbanıń statistikalıq bólistiriliwin tabıń.
 \\
\textbf{A2.} 
Kólemi \(n = 20\) ǵa teń bolǵan tańlanba berilgen: -4,9; 2,6; 0,5; -2,6; 2,6; 1,7; 2,3; 0,5; -4,9; -2,6; 1,7; -2,6; 1,7; 2,3; 2,6; 1,7; 2,6; 2,3; 2,3; 1,7. Bul tańlanbanıń empirikalıq bólistiriw funkciyasın tabıń.
 \\
\textbf{A3.} 
Joqarı matematika páninen 10 dana student test sınaqların tapsırǵan. Hárbir student 10 balǵa shekem toplawı múmkin. Eger test sınaqları nátiyjeleri boyınsha \{9, 5, 6, 8, 4, 7, 4, 6, 9, 7\} tańlanba alınǵan bolsa, onda tańlanba ortasha hám tańlanba dispersiyalardı tabıń.
 \\
\textbf{B1.} 
Eger normal bólistirilgen bas toplamnan alınǵan kólemi \(n = 25\) ǵa teń tańlanba boyınsha \(\overline{x} = 18,6\) tańlanba ortasha hám \({\overline{S}}^{2} = 0,49\) dúzetilgen tańlanba dispersiyalar tabılǵan bolsa, onda \(\gamma = 0,95\) isenimlilik penen belgisiz \(\theta\) matematikalıq kútiliw ushın isenimlilik interval dúziń.
 \\
\textbf{B2.} 
Eger (0,-2,0,-2,3,-2,0,0,3,0,0,0,3,-2,0,0,-2,3,0,3) tańlanba tómende berilgen bólistiriliwden alınǵan bolsa, onda belgisiz \(\theta\) parametr ushın momentler usılı bahasın tabıń.
\begin{tabular}{|c|c|c|c|}
  \hline
$\xi$ & $- 2$  & $0$  & $3$ \\
\hline
\(P_{\theta}\) & \(\theta\) & \(1 - 2\theta\) & \(\theta\) \\
\hline
\end{tabular}
 \\
\textbf{B3.} 
Eger \(X^{(n)} = \left( X_{1},...,X_{n} \right)\) tańlanba \(\theta\) parametrli Bernulli bólistiriliwinen alınǵan bolsa, onda belgisiz \(\theta\) parametrdiń shınlıqqa maksimal uqsaslıq usılı bahasın tabıń.
 \\
\textbf{C1.} 
Eger \(X^{(n)} = \left( X_{1},...,X_{n} \right)\) tańlanba \(M\xi = a\) belgili hám \(M\xi^{2}\) shekli bolǵan bólistiriliwden alınǵan bolsa, onda belgisiz \(D\xi\) dispersiya ushın \(\frac{1}{n}\sum_{i = 1}^{n}\left( X_{i} - a \right)^{2}\) bahasın jıljımaǵanlıq hám tiykarlılıqqa tekseriń.
 \\
\textbf{C2.} 
Eger \(X^{(n)} = \left( X_{1},...,X_{n} \right)\) tańlanba \({(\theta,\theta}^{2})\) parametrli normal bólistiriliwden \({\ g(x) = (x)}^{2}\ \)alınǵan bolsa, onda belgisiz \(\theta > 0\) parametr ushın momentler usılı bahasın funkciyası járdeminde tabıń.
 \\
\textbf{C3.} 
Eger \(X^{(n)} = \left( X_{1},...,X_{n} \right)\) tańlanba \(\theta\) parametrli geometriyalıq bólistiriliwden alınǵan bolsa, onda belgisiz \(\theta\) parametrdiń shınlıqqa maksimal uqsaslıq usılı bahasın tabıń.
 \\

\end{tabular}
\vspace{1cm}


\begin{tabular}{m{17cm}}
\textbf{81-variant}
\newline

\textbf{T1.} 
Glivenko-Kantelli teoreması. (empirikalıq bólistiriw funkciyası, 1itimallıq penen jaqınlasıw)
 \\
\textbf{T2.} 
Normal nızamnıń dispersiyası ushın isenimlilik intervalın dúziw. (Isenimlilik itimallıǵı, interval)
 \\
\textbf{A1.} 
Kólemi \(n = 20\) ǵa teń bolǵan tańlanba berilgen: -2,9; -3,8; 2,3; 1,8; 1,8; 0,7; -3,8; -1,5; 2,3; 0,7; -2,9; -1,5; 1,8; -2,9; -1,5; -3,8; 1,8; 1,8; -3,8; 1,8. Bul tańlanbanıń statistikalıq bólistiriliwin tabıń.
 \\
\textbf{A2.} 
Kólemi \(n = 20\) ǵa teń bolǵan tańlanba berilgen: -2,4; -3,5; 2,8; 1,4; 1,4; 0,1; -3,5; -1,9; 2,8; 0,1; -2,4; -1,9; 1,4; -2,4; -1,9; -3,5; 1,4; 1,4; -3,5; 1,4. Bul tańlanbanıń empirikalıq bólistiriw funkciyasın tabıń.
 \\
\textbf{A3.} 
Joqarı matematika páninen 10 dana student test sınaqların tapsırǵan. Hárbir student 10 balǵa shekem toplawı múmkin. Eger test sınaqları nátiyjeleri boyınsha \{8, 9, 7, 10, 6, 8, 10, 3, 10, 9\} tańlanba alınǵan bolsa, onda tańlanba ortasha hám tańlanba dispersiyalardı tabıń.
 \\
\textbf{B1.} 
Eger normal bólistirilgen bas toplamnan alınǵan kólemi \(n = 12\) ǵa teń bolǵan tańlanba boyınsha \({\overline{S}}^{2} = 0,4\) dúzetilgen tańlanba dispersiya tabılǵan bolsa, onda \(\gamma = 0,90\) isenimlilik penen belgisiz \(\theta_{2}^{2}\) dispersiya ushın isenimlilik interval dúziń.
 \\
\textbf{B2.} 
\(\lbrack\theta_{1},\theta_{2}\rbrack\) aralıqta teń ólshewli bólistiriw parametrleri ushın momentler usulı bahaların tabıń.
 \\
\textbf{B3.} 
Eger (4,8,5,3) tańlanba \(\left( a,\theta^{2} \right)\) parametrli normal bólistiriliwden alınǵan bolsa, onda belgisiz \(\theta^{2}\) parametrdiń shınlıqqa maksimal uqsaslıq bahasın tabıń.
 \\
\textbf{C1.} 
Eger \(X^{(n)} = \left( X_{1},...,X_{n} \right)\) tańlanba \(M\xi = a\) belgili hám \(M\xi^{2}\) shekli bolǵan bólistiriliwden alınǵan bolsa, onda belgisiz \(D\xi\) dispersiya ushın \(\overline{x^{2}} - a^{2}\) bahasın jıljımaǵanlıq hám tiykarlılıqqa tekseriń.
 \\
\textbf{C2.} 
Eger \(X^{(n)} = \left( X_{1},...,X_{n} \right)\) tańlanba tıǵızlıq funkciyası
$f(x,\theta) = \frac{2x}{\theta^{2}},x \in \lbrack 0,\theta\rbrack$
bolǵan bólistiriliwden alınǵan bolsa, onda belgisiz \(\theta\) parametr ushın momentler usılı bahasın tabıń.
 \\
\textbf{C3.} 
Eger \(X^{(n)} = \left( X_{1},...,X_{n} \right)\) tańlanba tıǵızlıq funkciyası
$f(x;\theta) = \frac{\theta ln^{\theta - 1}x}{x},\ x \in \lbrack 1,e\rbrack$
bolǵan bólistiriliwden alınǵan bolsa, onda belgisiz \(\theta > 0\) parametr ushın shınlıqqa maksimal uqsaslıq bahasın tabıń.
 \\

\end{tabular}
\vspace{1cm}


\begin{tabular}{m{17cm}}
\textbf{82-variant}
\newline

\textbf{T1.} 
Gruppalanǵan hám intervallıq variaciyalıq qatarlar.
 \\
\textbf{T2.} 
Statistikalıq gipotezalardı tekseriw (kritikalıq kóplik, 1 hám 2-túr qátelik)
 \\
\textbf{A1.} 
Kólemi \(n = 20\) ǵa teń bolǵan tańlanba berilgen: 3,6; 2,9; 3,6; 3,2; 1,1; 0,3; 1,1; 3,6; 1,7; 1,1; 0,3; 1,7; 1,1; 0,3; 2,9; 2,9; 2,9; 1,1; 2,9; 1,7. Bul tańlanbanıń statistikalıq bólistiriliwin tabıń.
 \\
\textbf{A2.} 
Kólemi \(n = 20\) ǵa teń bolǵan tańlanba berilgen: 4,6; 2,5; 4,6; 3,3; 1,8; 0,3; 1,8; 4,6; 2,1; 1,8; 0,3; 2,1; 1,8; 0,3; 2,5; 2,5; 2,5; 1,8; 2,5; 2,1. Bul tańlanbanıń empirikalıq bólistiriw funkciyasın tabıń.
 \\
\textbf{A3.} 
Joqarı matematika páninen 10 dana student test sınaqların tapsırǵan. Hárbir student 10 balǵa shekem toplawı múmkin. Eger test sınaqları nátiyjeleri boyınsha \{5, 7, 5, 9, 5, 8, 10, 6, 7, 8\} tańlanba alınǵan bolsa, onda tańlanba ortasha hám tańlanba dispersiyalardı tabıń.
 \\
\textbf{B1.} 
Eger ortasha kvadratlıq shetleniwi \(\sigma = 1\) bolǵan normal bólistirilgen bas toplamnan alınǵan kólemi \(n = 15\) ǵa teń tańlanba boyınsha \(\overline{x} = 5,8\) tańlanba ortasha mánisi tabılǵan bolsa, onda \(\gamma = 0,90\) isenimlilik penen belgisiz \(\theta\) matematikalıq kútiliwdi qaplaytuǵın isenimlilik intervalın dúziń.
 \\
\textbf{B2.} 
\(\lbrack 0,\theta\rbrack\) aralıqta teń ólshewli bólistirilgen \(\theta\) parametri ushın momentler usulı bahasın tabıń.
 \\
\textbf{B3.} 
Eger \(X^{(n)} = \left( X_{1},...,X_{n} \right)\) tańlanba tıǵızlıq funkciyası \(f(x;\theta) = \frac{2x}{\theta}e^{- \frac{x^{2}}{\theta}},\ x \geq 0\). bolǵan bólistiriliwden alınǵan bolsa, onda belgisiz \(\theta > 0\) parametrdiń shınlıqqa maksimal uqsaslıq bahasın tabıń.
 \\
\textbf{C1.} 
Eger \(X^{(n)} = \left( X_{1},...,X_{n} \right)\) tańlanba tıǵızlıq funkciyası: \(f(x,\theta) = \left\{ \begin{matrix}
e^{\theta - x},\ \ x \geq \theta, \\
\ \ 0,\ \ \ \ \ \ \ x < \theta
\end{matrix} \right.\ \)
bolǵan bólistiriliwden alınǵan bolsa, onda belgisiz \(\theta\) parametr ushın \(X_{(1)}\) bahasın jıljımaǵanlıq hám tiykarlılıqqa tekseriń.
 \\
\textbf{C2.} 
Eger \(X^{(n)} = \left( X_{1},...,X_{n} \right)\) tańlanba \({(\theta,\theta}^{2})\ \) parametrli normal bólistiriliwden alınǵan bolsa, onda belgisiz \(\theta > 0\) parametr ushın momentler usılı bahasın tabıń.
 \\
\textbf{C3.} 
\(\ f(x,\theta) = \frac{4x^{3}}{\theta_{2}\sqrt{2\pi}}\exp\left\{ - \frac{\left( x^{4} - \theta_{1} \right)^{2}}{2{\theta_{2}}^{2}} \right\}\) model ushın \(\theta_{1}\) hám \({\theta_{2}}^{2}\) parametrler haqıyqatqa maksimal uqsaslıq usılı bahaları tabılsın.
 \\

\end{tabular}
\vspace{1cm}


\begin{tabular}{m{17cm}}
\textbf{83-variant}
\newline

\textbf{T1.} 
Tańlanba momentleri (\(k -\)tártipli baslanǵısh, baslanǵısh absolyut, oraylıq hám oraylıq absolyut momentler).
 \\
\textbf{T2.} 
Haqiyqatqa maksimal uqsaslıq usulı. (haqiyqatqa maksimal uqsaslıq funkciyası, belgisiz parametrlerdi bahalaw).
 \\
\textbf{A1.} 
Kólemi \(n = 20\) ǵa teń bolǵan tańlanba berilgen: -1,3; 0; 0,8; 2,3; 1,1; 0,8; 0,8; 2,3; 1,1; 0,8; -1,3; 1,8; 1,1; -1,3; 1,1; 1,8; 1,8; 1,1; 1,8; 1,8. Bul tańlanbanıń statistikalıq bólistiriliwin tabıń.
 \\
\textbf{A2.} 
Kólemi \(n = 20\) ǵa teń bolǵan tańlanba berilgen: -1,9; 0,7; 0,9; 2,8; 1,3; 0,9; 0,9; 2,8; 1,3; 0,9; -1,9; 1,6; 1,3; -1,9; 1,3; 1,6; 1,6; 1,3; 1,6; 1,6. Bul tańlanbanıń empirikalıq bólistiriw funkciyasın tabıń.
 \\
\textbf{A3.} 
Joqarı matematika páninen 10 dana student test sınaqların tapsırǵan. Hárbir student 10 balǵa shekem toplawı múmkin. Eger test sınaqları nátiyjeleri boyınsha \{8, 4, 3, 7, 3, 6, 5, 3, 5, 6\} tańlanba alınǵan bolsa, onda tańlanba ortasha hám tańlanba dispersiyalardı tabıń.
 \\
\textbf{B1.} 
Eger normal bólistirilgen bas toplamnan alınǵan kólemi \(n = 20\) ǵa teń tańlanba boyınsha \(\overline{x} = 16,6\) tańlanba ortasha hám \({\overline{S}}^{2} = 0,64\) dúzetilgen tańlanba dispersiyalar tabılǵan bolsa, onda \(\gamma = 0,95\) isenimlilik penen belgisiz \(\theta\) matematikalıq kútiliw ushın isenimlilik interval dúziń.
 \\
\textbf{B2.} 
Kórsetkishli bólistiriw belgisiz \(\theta > 0\) parametri momentlar usulı bahasın tabıń.
 \\
\textbf{B3.} 
\(\ f(x) = \frac{2x}{\theta}e^{- \frac{x^{2}}{\theta}},\ \ x \geq 0\) model ushın \(\theta\) parametri haqıyqatqa maksimal uqsaslıq usılı bahası tabılsın.
 \\
\textbf{C1.} 
Eger \(X^{(n)} = \left( X_{1},...,X_{n} \right)\) tańlanba tıǵızlıq funkciyası: \(f(x,\theta) = \left\{ \begin{matrix}
e^{\theta - x},\ \ x \geq \theta, \\
\ \ 0,\ \ \ \ \ \ \ x < \theta
\end{matrix} \right.\ \)
bolǵan bólistiriliwden alınǵan bolsa, onda belgisiz \(\theta\) parametr ushın \(\overline{x} - 1\) bahasın jıljımaǵanlıq hám tiykarlılıqqa tekseriń.
 \\
\textbf{C2.} 
Eger \(X^{(n)} = \left( X_{1},...,X_{n} \right)\) tańlanba \(\theta\) parametrli Puasson bólistiriliwinen alınǵan bolsa, onda belgisiz \(\theta\) parametr ushın momentler usılı bahasın tabıń. Eger \(X^{(n)} = \left( X_{1},...,X_{n} \right)\) tańlanba \(\theta\) parametrli Puasson bólistiriliwinen alınǵan bolsa, onda belgisiz \(\theta\) parametr ushın momentler usılı bahasın\({\ g(x) = x}^{2}\) funkciyası járdeminde tabıń.
 \\
\textbf{C3.} 
\(\ f(x,\theta) = \frac{e^{x}}{\sqrt{2\pi}}\exp\left\{ - \frac{\left( e^{x} - \theta \right)^{2}}{2} \right\}\) model ushın \(\theta\) parametri haqıyqatqa maksimal uqsaslıq usılı bahası tabılsın.
 \\

\end{tabular}
\vspace{1cm}


\begin{tabular}{m{17cm}}
\textbf{84-variant}
\newline

\textbf{T1.} Matematikalıq statistikanıń tiykarǵı máseleleri. (Statistikalıq maǵlıwmatlar, gruppalaw)
 \\
\textbf{T2.} 
Momentler usulı. (tańlanba momentleri, belgisiz parametrlerdi bahalaw).
 \\
\textbf{A1.} 
Kólemi \(n = 20\) ǵa teń bolǵan tańlanba berilgen: -2,4; 5,6; 5,6; -5,2; -6,7; 5,1; -5,2; -2,4; 4,3; 5,1; -6,7; 4,3; -2,4; -6,7; 4,3; 5,1; 4,3; 5,6; -6,7; 5,6. Bul tańlanbanıń statistikalıq bólistiriliwin tabıń.
 \\
\textbf{A2.} 
Kólemi \(n = 20\) ǵa teń bolǵan tańlanba berilgen: -2,9; 7,6; 7,6; -5,7; -6,1; 5,5; -5,7; -2,9; 4,2; 5,5; -6,1; 4,2; -2,9; -6,1; 4,2; 5,5; 4,2; 7,6; -6,1; 7,6. Bul tańlanbanıń empirikalıq bólistiriw funkciyasın tabıń.
 \\
\textbf{A3.} 
Joqarı matematika páninen 10 dana student test sınaqların tapsırǵan. Hárbir student 10 balǵa shekem toplawı múmkin. Eger test sınaqları nátiyjeleri boyınsha \{9, 8, 6, 7, 5, 8, 5, 7, 4, 6\} tańlanba alınǵan bolsa, onda tańlanba ortasha hám tańlanba dispersiyalardı tabıń.
 \\
\textbf{B1.} 
Eger normal bólistirilgen bas toplamnan alınǵan kólemi \(n = 13\) ǵa teń bolǵan tańlanba boyınsha \({\overline{S}}^{2} = 1,2\) dúzetilgen tańlanba dispersiya tabılǵan bolsa, onda \(\gamma = 0,90\) isenimlilik penen belgisiz \(\theta_{2}^{2}\) dispersiya ushın isenimlilik interval dúziń.
 \\
\textbf{B2.} 
Eger (-2,0,-2,0,-2,3,-2,0,0,3,0,0,0,3,-2,0,0,-2,3,0) tańlanba tómende berilgen bólistiriliwden alınǵan bolsa, onda belgisiz \(\left( \theta_{1},\theta_{2} \right)\) vektor parametr ushın momentler usılı bahalasın tabıń.
\begin{tabular}{|c|c|c|c|}
  \hline
$\xi$ &
$- 2$ &
$0$ &
$3$\\
\hline
\(P_{\theta}\) & \(\theta_{1}\) & \(1 - \theta_{1} - \theta_{2}\) & \(\theta_{2}\) \\
\hline
\end{tabular}
 \\
\textbf{B3.} 
Eger (-1,-1,0,-1,0,-1,-1,5,-1,0,-1,0,5,-1,-1,-1,5,-1,-1,-1,5,0,-1,-1,5) tańlanba tómende berilgen bólistiriliwden alınǵan bolsa, onda belgisiz \(\theta\) parametrdiń shınlıqqa maksimal uqsaslıq usılı bahasın tabıń.
\begin{tabular}{|c|c|c|c|}
  \hline
$\xi$
&
$- 1$
&
$0$
&
$5$\\
\hline
\(P_{\theta}\) & \(1 - \theta\) & \(\theta/2\) & \(\theta/2\ \) \\
\hline
\end{tabular}
 \\
\textbf{C1.} 
Eger \(X^{(n)} = \left( X_{1},...,X_{n} \right)\) tańlanba \(\lbrack - 3\theta,\theta\rbrack\) aralıqta teń ólshemli bólistiriliwden alınǵan bolsa, onda belgisiz \(\theta\) parametr ushın \(4X_{(n)} + X_{(1)}\) bahasın jıljımaǵanlıq hám tiykarlılıqqa tekseriń.
 \\
\textbf{C2.} 
Eger \(X^{(n)} = \left( X_{1},...,X_{n} \right)\) tańlanba \({\lbrack\theta}_{1},\theta_{2}\rbrack\) aralıqta teń ólshemli bólistiriliwden alınǵan bolsa, onda belgisiz \(\left( \theta_{1},\theta_{2} \right)\) vektor parametr ushın momentler usılı bahasın tabıń.
 \\
\textbf{C3.} 
Eger \(X^{(n)} = \left( X_{1},...,X_{n} \right)\) tańlanba tıǵızlıq funkciyası
$f(x;\theta) = \frac{\theta}{2}e^{- \theta|x|},\ x \in R$
bolǵan bólistiriliwden alınǵan bolsa, onda belgisiz \(\theta > 0\) parametrdiń shınlıqqa maksimal uqsaslıq bahasın tabıń.
 \\

\end{tabular}
\vspace{1cm}


\begin{tabular}{m{17cm}}
\textbf{85-variant}
\newline

\textbf{T1.} 
Neyman-Pirson teoreması
 \\
\textbf{T2.} 
Sızıqlı korrelyaciya teńlemesi (anıqlaması, regressiya tuwrı sızıǵınıń tańlanba teńlemeleri)
 \\
\textbf{A1.} 
Kólemi \(n = 20\) ǵa teń bolǵan tańlanba berilgen:-3,3; 0; 4,4; 2,2; -2,7; 4,4; 2,2; 4,4;-3,3; 2,2; -2,7; 2,2; -3,3; -2,7; 2,2; 3,4; 4,4; 0; -3,3; 0. Bul tańlanbanıń statistikalıq bólistiriliwin tabıń.
 \\
\textbf{A2.} 
Kólemi \(n = 20\) ǵa teń bolǵan tańlanba berilgen:-3,3; 0; 4,9; 2,8; -2,6; 4,9; 2,8; 4,9;-3,3; 2,8; -2,6; 2,8; -3,3; -2,6; 2,8; 3,1; 4,9; 0; -3,3; 0. Bul tańlanbanıń empirikalıq bólistiriw funkciyasın tabıń.
 \\
\textbf{A3.} 
Joqarı matematika páninen 10 dana student test sınaqların tapsırǵan. Hárbir student 10 balǵa shekem toplawı múmkin. Eger test sınaqları nátiyjeleri boyınsha \{4, 7, 6, 9, 3, 8, 3, 7, 4, 9\} tańlanba alınǵan bolsa, onda tańlanba ortasha hám tańlanba dispersiyalardı tabıń.
 \\
\textbf{B1.} 
Eger ortasha kvadratlıq shetleniwi \(\sigma = 4\) bolǵan normal bólistirilgen bas toplamnan alınǵan kólemi \(n = 12\) ǵa teń tańlanba boyınsha \(\overline{x} = 3\) tańlanba ortasha mánisi tabılǵan bolsa, onda \(\gamma = 0,95\) isenimlilik penen belgisiz \(\theta\) matematikalıq kútiliwdi qaplaytuǵın isenimlilik intervalın dúziń.
 \\
\textbf{B2.} 
Eger (3,-2,-2,0,-2,-2,-2,0,-2,3,-2,0,3,0,3,-2,0,-2,3,-2,-2,-2,-2,3,3,3,-2,-2,3,3) tańlanba tómende berilgen bólistiriliwden alınǵan bolsa, onda belgisiz \(\theta\) parametr ushın momentler usılı bahasın \(g(x) = |x|\) funkciyası járdeminde tabıń.
\begin{tabular}{|c|c|c|c|}
  \hline
$\xi$ &
$- 2$ &
$0$ &
$3$ \\
\hline
\(P_{\theta}\) & \(3\theta\) & \(1 - 5\theta\) & \(2\theta\) \\
\hline
\end{tabular}
 \\
\textbf{B3.} 
Eger \(X^{(n)} = \left( X_{1},...,X_{n} \right)\) tańlanba \(\theta\) parametrli kórsetkishli bólistiriliwden alınǵan bolsa, onda belgisiz \(\theta\) parametrdiń shınlıqqa maksimal uqsaslıq usılı bahasın tabıń.
 \\
\textbf{C1.} 
Eger \(X^{(n)} = \left( X_{1},...,X_{n} \right)\) tańlanba bólistiriw funkciyası \(F(x)\) bolǵan bólistiriliwden alınǵan bolsa, onda belgisiz \(F(x)\) ushın \(F_{n}(x)\) empirikalıq bólistiriw funkciyasın jıljımaǵanlıq hám tiykarlılıqqa tekseriń.
 \\
\textbf{C2.} 
Eger \(X^{(n)} = \left( X_{1},...,X_{n} \right)\) tańlanba \(\frac{1}{\theta}\) parametrli kórsetkishli bólistiriliwden alınǵan bolsa, onda belgisiz \(\theta\) parametr ushın momentler usılı bahasın\({\ g(x) = x}^{k},\) \(k \in N\)funkciyası járdeminde tabıń.
 \\
\textbf{C3.} 
Eger \(X^{(n)} = \left( X_{1},...,X_{n} \right)\) tańlanba tıǵızlıq funkciyası
$f(x;\theta) = \frac{4x^{3}}{\sqrt{2\pi}\theta_{2}}\exp\left\{ - \frac{\left( x^{4} - \theta_{1} \right)^{2}}{2{\theta_{2}}^{2}} \right\},\ x \in R$
bolǵan bólistiriliwden alınǵan bolsa, onda belgisiz \(\left( \theta_{1},\theta_{2}^{2} \right)\) vektor parametrdiń shınlıqqa maksimal uqsaslıq usılı bahaların tabıń.
 \\

\end{tabular}
\vspace{1cm}


\begin{tabular}{m{17cm}}
\textbf{86-variant}
\newline

\textbf{T1.} 
Tańlanba xarakteristikaları.(tańlanba orta, tańlanba dispersiya)
 \\
\textbf{T2.} 
Kolmogorovtıń kelisimlilik belgisi (Kolmogorov teoreması)
 \\
\textbf{A1.} 
Kólemi \(n = 20\) ǵa teń bolǵan tańlanba berilgen: 3,7; 3,1; 4,8; 2,8; 3,1; 4,3; 3,7; 4,3; 2,4; 3,1; 2,4; 4,3; 3,1; 3,7; 4,8; 2,8; 2,4; 2,8; 2,4; 3,1. Bul tańlanbanıń statistikalıq bólistiriliwin tabıń.
 \\
\textbf{A2.} 
Kólemi \(n = 20\) ǵa teń bolǵan tańlanba berilgen: 3,8; 3,4; 4,8; 2,9; 3,4; 4,6; 3,8; 4,6; 2,1; 3,4; 2,1; 4,6; 3,4; 3,8; 4,8; 2,9; 2,1; 2,9; 2,1; 3,4. Bul tańlanbanıń empirikalıq bólistiriw funkciyasın tabıń.
 \\
\textbf{A3.} 
Joqarı matematika páninen 10 dana student test sınaqların tapsırǵan. Hárbir student 10 balǵa shekem toplawı múmkin. Eger test sınaqları nátiyjeleri boyınsha \{6, 5, 6, 9, 5, 7, 10, 5, 9, 8\} tańlanba alınǵan bolsa, onda tańlanba ortasha hám tańlanba dispersiyalardı tabıń.
 \\
\textbf{B1.} 
Eger normal bólistirilgen bas toplamnan alınǵan kólemi \(n = 25\) ǵa teń tańlanba boyınsha \(\overline{x} = 9\) tańlanba ortasha hám \({\overline{S}}^{2} = 0,64\) dúzetilgen tańlanba dispersiyalar tabılǵan bolsa, onda \(\gamma = 0,95\) isenimlilik penen belgisiz \(\theta\) matematikalıq kútiliw ushın isenimlilik interval dúziń.
 \\
\textbf{B2.} 
Eger tıǵızlıq funkciyası \(f(x) = \frac{2x}{\theta}e^{- \frac{x^{2}}{\theta}},\ \ x \geq 0\) kóriniske iye bolsa, onda \(\theta\) parametr momentler usulı bahasın tabıń.
 \\
\textbf{B3.} 
\(\ f(x) = \frac{2x}{\theta}e^{- \frac{x^{2}}{\theta}},\ \ x \geq 0\) model ushın \(\theta\) parametri haqıyqatqa maksimal uqsaslıq usılı bahası tabılsın.
 \\
\textbf{C1.} 
Eger \(X^{(n)} = \left( X_{1},...,X_{n} \right)\) tańlanba \(\left( a,\theta^{2} \right)\) parametrli normal bólistiriliwden alınǵan bolsa (\(a -\)belgili), onda belgisiz \(\theta\) parametr ushın \(\sqrt{\frac{\pi}{2}}\left| \overline{x - a} \right|\) bahasın jıljımaǵanlıq hám tiykarlılıqqa tekseriń.
 \\
\textbf{C2.} 
Eger \(X^{(n)} = \left( X_{1},...,X_{n} \right)\) tańlanba\(\ (\theta,2\theta)\ \) parametrli normal bólistiriliwden alınǵan bolsa, onda belgisiz \(\theta > 0\) parametr ushın momentler usılı bahasın tabıń.
 \\
\textbf{C3.} 
\(\ f(x;\theta) = \frac{7x^{6}}{\sqrt{2\pi}}exp\{ - \frac{(x^{7} - \theta)^{2}}{2}\}\) model ushın \(\theta\) parametri haqıyqatqa maksimal uqsaslıq usılı bahası tabılsın.
 \\

\end{tabular}
\vspace{1cm}


\begin{tabular}{m{17cm}}
\textbf{87-variant}
\newline

\textbf{T1.} 
Momentler usulı. (tańlanba momentleri, belgisiz parametrlerdi bahalaw).
 \\
\textbf{T2.} 
Statistikalıq gipotezalardı tekseriw (kritikalıq kóplik, 1 hám 2-túr qátelik).
 \\
\textbf{A1.} 
Kólemi \(n = 20\) ǵa teń bolǵan tańlanba berilgen: 1,5; -0,9; -2,4; -0,9; 0,7; 1,5; -0,9; -0,2; -2,4; 0,7; -2,4; 0,7; -0,9; 1,5; -1,7; -0,9; -0,2; 0,7; -1,7; -0,9. Bul tańlanbanıń statistikalıq bólistiriliwin tabıń.
 \\
\textbf{A2.} 
Kólemi \(n = 20\) ǵa teń bolǵan tańlanba berilgen: 1,9; -0,3; -2,7; -0,3; 0,6; 1,9; -0,3; -0,1; -2,7; 0,6; -2,7; 0,6; -0,3; 1,9; -1,8; -0,3; -0,1; 0,6; -1,8; -0,3. Bul tańlanbanıń empirikalıq bólistiriw funkciyasın tabıń.
 \\
\textbf{A3.} 
Joqarı matematika páninen 10 dana student test sınaqların tapsırǵan. Hárbir student 10 balǵa shekem toplawı múmkin. Eger test sınaqları nátiyjeleri boyınsha \{4, 6, 6, 9, 5, 8, 4, 7, 5, 6\} tańlanba alınǵan bolsa, onda tańlanba ortasha hám tańlanba dispersiyalardı tabıń.
 \\
\textbf{B1.} 
Eger normal bólistirilgen bas toplamnan alınǵan kólemi \(n = 10\) ǵa teń bolǵan tańlanba boyınsha \({\overline{S}}^{2} = 0,6\) dúzetilgen tańlanba dispersiya tabılǵan bolsa, onda \(\gamma = 0,95\) isenimlilik penen belgisiz \(\theta_{2}^{2}\) dispersiya ushın isenimlilik interval dúziń.
 \\
\textbf{B2.} 
Eger \(X^{(n)} = \left( X_{1},...,X_{n} \right)\) tańlanba \(\theta\) parametrli Bernulli bólistiriliwinen alınǵan bolsa, onda belgisiz \(\theta\) parametr ushın momentler usılı bahasın tabıń.
 \\
\textbf{B3.} 
Eger \(x_{1} = 1,1;\ x_{2} = 2,7;\ldots;x_{100} = 1,5\) tańlanba \(\theta\) parametrli kórsetkishli bólistiriliwden alınǵan bolıp, \(\sum_{k = 1}^{100}x_{k} = 200\) bolsa, onda belgisiz \(\theta\) parametrdiń shınlıqqa maksimal uqsaslıq bahasın tabıń.
 \\
\textbf{C1.} 
Eger \(X^{(n)} = \left( X_{1},...,X_{n} \right)\) tańlanba \(\theta\) parametrli kórsetkishli bólistiriliwinen alınǵan bolsa, onda belgisiz \(\theta\) parametr ushın \(\frac{1}{\overline{x}}\) bahasın jıljımaǵanlıq hám tiykarlılıqqa tekseriń.
 \\
\textbf{C2.} 
Eger \(X^{(n)} = \left( X_{1},...,X_{n} \right)\) tańlanba \(\left( \theta_{1},\theta_{2} \right)\) parametrli gamma bólistiriliwden alınǵan bolsa, onda belgisiz \(\left( \theta_{1},\theta_{2} \right)\) vektor parametr ushın momentler usılı bahasın tabıń.
 \\
\textbf{C3.} 
Eger \(X^{(n)} = \left( X_{1},...,X_{n} \right)\) tańlanba tıǵızlıq funkciyası
$f(x;\theta) = \left\{ \begin{matrix}
e^{\theta - x},\ \ x \geq \theta, \\
\ \ 0,\ \ \ \ \ \ \ x < \theta
\end{matrix} \right.\ $
bolǵan bólistiriliwden alınǵan bolsa, onda belgisiz \(\theta\) parametrdiń shınlıqqa maksimal uqsaslıq bahasın tabıń.
 \\

\end{tabular}
\vspace{1cm}


\begin{tabular}{m{17cm}}
\textbf{88-variant}
\newline

\textbf{T1.} 
Empirikalıq bólistiriw funkciyası. (Tańlanba, eksperiment)
 \\
\textbf{T2.} 
Isenimlilik intervalların qurıw. Anıq isenimli intervallar
 \\
\textbf{A1.} 
Kólemi \(n = 20\) ǵa teń bolǵan tańlanba berilgen:9,4; 6,8; -8,5; 9,4; 2,9; 9,4; -8,5; -6,4; 6,8; -8,5; 9,4; -6,4; 6,8; 9,4; 2,9; 9,4; -3,6; -8,5; 2,9; -6,4. Bul tańlanbanıń statistikalıq bólistiriliwin tabıń.
 \\
\textbf{A2.} 
Kólemi \(n = 20\) ǵa teń bolǵan tańlanba berilgen:9,1; 6,4; -8,6; 9,1; 2,3; 9,1; -8,6; -6,2; 6,4; -8,6; 9,1; -6,2; 6,4; 9,1; 2,3; 9,1; -3,9; -8,6; 2,3; -6,2. Bul tańlanbanıń empirikalıq bólistiriw funkciyasın tabıń.
 \\
\textbf{A3.} 
Joqarı matematika páninen 10 dana student test sınaqların tapsırǵan. Hárbir student 10 balǵa shekem toplawı múmkin. Eger test sınaqları nátiyjeleri boyınsha \{3, 7, 6, 4, 5, 4, 3, 7, 8, 3\} tańlanba alınǵan bolsa, onda tańlanba ortasha hám tańlanba dispersiyalardı tabıń.
 \\
\textbf{B1.} 
Eger ortasha kvadratlıq shetleniwi \(\sigma = 5\) bolǵan normal bólistirilgen bas toplamnan alınǵan kólemi \(n = 16\) ǵa teń tańlanba boyınsha \(\overline{x} = 3,6\) tańlanba ortasha mánisi tabılǵan bolsa, onda \(\gamma = 0,90\) isenimlilik penen belgisiz \(\theta\) matematikalıq kútiliwdi qaplaytuǵın isenimlilik intervalın dúziń.
 \\
\textbf{B2.} 
Eger \(X^{(n)} = \left( X_{1},...,X_{n} \right)\) tańlanba \(\theta\) parametrli kórsetkishli bólistiriliwden alınǵan bolsa, onda belgisiz \(\theta\) parametr ushın momentler usılı bahasın tabıń.
 \\
\textbf{B3.} 
Eger \(X^{(n)} = \left( X_{1},...,X_{n} \right)\) tańlanba \(\left\lbrack - \theta,\theta^{2} \right\rbrack\) aralıqta teń ólshemli bólistiriliwden alınǵan bolsa, onda belgisiz \(\theta > 0\) parametrdiń shınlıqqa maksimal uqsaslıq usılı bahasın tabıń.
 \\
\textbf{C1.} 
Eger \(X^{(n)} = \left( X_{1},...,X_{n} \right)\) tańlanba \(\frac{1}{\sqrt{\theta}}\) parametrli kórsetkishli bólistiriliwinen alınǵan bolsa, onda belgisiz \(\theta\) parametr ushın \((\overline{x})^{2}\) bahasın jıljımaǵanlıq hám tiykarlılıqqa tekseriń.
 \\
\textbf{C2.} 
Eger \(X^{(n)} = \left( X_{1},...,X_{n} \right)\) tańlanba {[}\(0,2\theta\rbrack\) aralıqta teń ólshemli bólistiriliwden alınǵan bolsa, onda belgisiz \(\theta > 0\) parametr ushın momentler usılı bahasın tabıń.
 \\
\textbf{C3.} 
Eger \(X^{(n)} = \left( X_{1},...,X_{n} \right)\) tańlanba tıǵızlıq funkciyası
$f(x;\theta) = \frac{\theta}{\sqrt{2\pi x^{3}}}e^{\frac{- \ \theta^{2}}{2x}},\ x \geq 0$
bolǵan bólistiriliwden alınǵan bolsa, onda belgisiz \(\theta > 0\) parametrdiń shınlıqqa maksimal uqsaslıq bahasın tabıń.
 \\

\end{tabular}
\vspace{1cm}


\begin{tabular}{m{17cm}}
\textbf{89-variant}
\newline

\textbf{T1.} 
Tańlanba xarakteristikalar. (Variaciyalıq qatar, salıstırmalı jiyilik).
 \\
\textbf{T2.} 
Pirsonnıń xi-kvadrat kelisimlilik belgisi (Pirson teoreması).
 \\
\textbf{A1.} 
Kólemi \(n = 20\) ǵa teń bolǵan tańlanba berilgen: 6,2; -5,3; 7,2; 3,7; -2,2; 6,2; 3,7; -7,6; 3,7; 7,2; 6,2; -5,3; -7,6; -5,3; -7,6; 6,2; 7,2; -2,2; -7,6; 7,2. Bul tańlanbanıń statistikalıq bólistiriliwin tabıń.
 \\
\textbf{A2.} 
Kólemi \(n = 20\) ǵa teń bolǵan tańlanba berilgen: 6,1; -5,8; 7,9; 3,5; -2,5; 6,1; 3,5; -7,2; 3,5; 7,9; 6,1; -5,8; -7,2; -5,8; -7,2; 6,1; 7,9; -2,5; -7,2; 7,9. Bul tańlanbanıń empirikalıq bólistiriw funkciyasın tabıń.
 \\
\textbf{A3.} 
Joqarı matematika páninen 10 dana student test sınaqların tapsırǵan. Hárbir student 10 balǵa shekem toplawı múmkin. Eger test sınaqları nátiyjeleri boyınsha \{10, 8, 6, 5, 4, 8, 10, 7, 5, 7\} tańlanba alınǵan bolsa, onda tańlanba ortasha hám tańlanba dispersiyalardı tabıń.
 \\
\textbf{B1.} 
Eger normal bólistirilgen bas toplamnan alınǵan kólemi \(n = 16\) ǵa teń tańlanba boyınsha \(\overline{x} = 15,2\) tańlanba ortasha hám \({\overline{S}}^{2} = 0,81\) dúzetilgen tańlanba dispersiyalar tabılǵan bolsa, onda \(\gamma = 0,90\) isenimlilik penen belgisiz \(\theta\) matematikalıq kútiliw ushın isenimlilik interval dúziń.
 \\
\textbf{B2.} 
Eger (0,-2,0,-2,3,-2,0,0,3,0,0,0,3,-2,0,0,-2,3,0,3) tańlanba tómende berilgen bólistiriliwden alınǵan bolsa, onda belgisiz \(\theta\) parametr ushın momentler usılı bahasın tabıń.
\begin{tabular}{|c|c|c|c|}
  \hline
$\xi$ & $- 2$  & $0$  & $3$ \\
\hline
\(P_{\theta}\) & \(\theta\) & \(1 - 2\theta\) & \(\theta\) \\
\hline
\end{tabular}
 \\
\textbf{B3.} 
Eger (-1,-1,0,-1,0,-1,-1,5,-1,0,-1,0,5,-1,-1,-1,5,-1,-1,-1,5,0,-1,-1,5) tańlanba tómende berilgen bólistiriliwden alınǵan bolsa, onda belgisiz \(\theta\) parametrdiń shınlıqqa maksimal uqsaslıq usılı bahasın tabıń.
\begin{tabular}{|c|c|c|c|}
  \hline
$\xi$
&
$- 1$
&
$0$
&
$5$\\
\hline
\(P_{\theta}\) & \(1 - \theta\) & \(\theta/2\) & \(\theta/2\ \) \\
\hline
\end{tabular}
 \\
\textbf{C1.} 
Eger \(X^{(n)} = \left( X_{1},...,X_{n} \right)\) tańlanba \(\sqrt{\theta}\) parametrli Bernulli bólistiriliwinen alınǵan bolsa, onda belgisiz \(\theta\) parametr ushın \((\overline{x})^{2}\) bahasın jıljımaǵanlıq hám tiykarlılıqqa tekseriń.
 \\
\textbf{C2.} 
Eger \(X^{(n)} = \left( X_{1},...,X_{n} \right)\) tańlanba \((\theta,2\theta)\) parametrli normal bólistiriliwden alınǵan bolsa, onda belgisiz \(\theta > 0\) parametr ushın momentler usılı bahasın \({\ g(x) = (x)}^{2}\) funkciyası járdeminde tabıń.
 \\
\textbf{C3.} 
Eger \(X^{(n)} = \left( X_{1},...,X_{n} \right)\) tańlanba tıǵızlıq funkciyası
$f(x;\theta) = \left\{ \begin{matrix}
\theta_{1}^{- 1}e^{\frac{x - \theta_{2}}{\theta_{1}}},\ \ x \geq \theta_{2}, \\
\ \ \ \ \ \ \ \ \ \ \ \ 0,\ \ \ \ \ \ \ x < \theta_{2}
\end{matrix} \right.\ $
bolǵan bólistiriliwden alınǵan bolsa, onda belgisiz \(\left( \theta_{1},\theta_{2} \right),\) \(\theta_{1} > 0,\) \(\theta_{2} \in R\) vektor parametrdiń shınlıqqa maksimal uqsaslıq bahasın tabıń.
 \\

\end{tabular}
\vspace{1cm}


\begin{tabular}{m{17cm}}
\textbf{90-variant}
\newline

\textbf{T1.} 
Poligon hám gistogramma(salıstirmalı jiyilik, intervallıq qatar, grafik)
 \\
\textbf{T2.} 
Statistikalıq baha qásiyetleri. (Jıljımaytuǵın, tiykarlı, effektiv)
 \\
\textbf{A1.} 
Kólemi \(n = 20\) ǵa teń bolǵan tańlanba berilgen: 9,6; 1,5; 7,4; 9,6; 2,8; 1,5; 6,3; 1,5; 9,6; 6,3; 2,8; 4,1; 6,3; 9,6; 1,5; 1,5; 6,3; 7,4; 4,1; 7,4. Bul tańlanbanıń statistikalıq bólistiriliwin tabıń.
 \\
\textbf{A2.} 
Kólemi \(n = 20\) ǵa teń bolǵan tańlanba berilgen: 9,8; 1,2; 7,1; 9,8; 2,9; 1,2; 6,7; 1,2; 9,8; 6,7; 2,9; 4,6; 6,7; 9,8; 1,2; 1,2; 6,7; 7,1; 4,6; 7,1. Bul tańlanbanıń empirikalıq bólistiriw funkciyasın tabıń.
 \\
\textbf{A3.} 
Joqarı matematika páninen 10 dana student test sınaqların tapsırǵan. Hárbir student 10 balǵa shekem toplawı múmkin. Eger test sınaqları nátiyjeleri boyınsha \{9, 10, 5, 6, 4, 8, 4, 6, 10, 8\} tańlanba alınǵan bolsa, onda tańlanba ortasha hám tańlanba dispersiyalardı tabıń.
 \\
\textbf{B1.} 
Eger normal bólistirilgen bas toplamnan alınǵan kólemi \(n = 10\) ǵa teń bolǵan tańlanba boyınsha \({\overline{S}}^{2} = 0,45\) dúzetilgen tańlanba dispersiya tabılǵan bolsa, onda \(\gamma = 0,95\) isenimlilik penen belgisiz \(\theta_{2}^{2}\) dispersiya ushın isenimlilik interval dúziń.
 \\
\textbf{B2.} 
Eger \(X^{(n)} = \left( X_{1},...,X_{n} \right)\) tańlanba \(\theta\) parametrli Bernulli bólistiriliwinen alınǵan bolsa, onda belgisiz \(\theta\) parametr ushın momentler usılı bahasın tabıń.
 \\
\textbf{B3.} 
Eger \(X^{(n)} = \left( X_{1},...,X_{n} \right)\) tańlanba \(\lbrack - \theta,\theta\rbrack\) aralıqta teń ólshemli bólistiriliwden alınǵan bolsa, onda belgisiz \(\theta > 0\) parametrdiń shınlıqqa maksimal uqsaslıq usılı bahasın tabıń.
 \\
\textbf{C1.} 
Eger \(X^{(n)} = \left( X_{1},...,X_{n} \right)\) tańlanba \(\theta\) parametrli Bernulli bólistiriliwinen alınǵan bolsa, onda belgisiz \(\theta\) parametr ushın \(X_{n}\) bahasın jıljımaǵanlıq hám tiykarlılıqqa tekseriń.
 \\
\textbf{C2.} 
Eger \(X^{(n)} = \left( X_{1},...,X_{n} \right)\) tańlanba tıǵızlıq funkciyası
$
{f(x,\theta) = \left\{ \begin{array}{r}
e^{\theta - x},\ \ \ x \geq \theta, \\
0,\ \ \ x < \theta
\end{array} \right.\ }$
bolǵan bólistiriliwden alınǵan bolsa, onda belgisiz \(\theta\) parametr ushın momentler usılı bahasın tabıń.
 \\
\textbf{C3.} 
Eger \(X^{(n)} = \left( X_{1},...,X_{n} \right)\) tańlanba tıǵızlıq funkciyası
$f(x;\theta) = \frac{e^{x}}{\sqrt{2\pi}}\exp\left\{ - \frac{\left( e^{x} - \theta \right)^{2}}{2} \right\},\ x \in R$
bolǵan bólistiriliwden alınǵan bolsa, onda belgisiz \(\theta\) parametrdiń shınlıqqa maksimal uqsaslıq bahasın tabıń.
 \\

\end{tabular}
\vspace{1cm}


\begin{tabular}{m{17cm}}
\textbf{91-variant}
\newline

\textbf{T1.} 
Momentler usulı. (tańlanba momentleri, belgisiz parametrlerdi bahalaw).
 \\
\textbf{T2.} 
Haqiyqatqa maksimal uqsaslıq usulı. (haqiyqatqa maksimal uqsaslıq funkciyası, belgisiz parametrlerdi bahalaw).
 \\
\textbf{A1.} 
Kólemi \(n = 20\) ǵa teń bolǵan tańlanba berilgen:1,8; -8,4; 7,3; 4,7; -3,9; 1,8; 4,7; -10,4; -8,4; 7,3; -10,4; 4,7; -8,4; 1,8; 4,7; -10,4; 7,3; -3,9; 4,7; -8,4. Bul tańlanbanıń statistikalıq bólistiriliwin tabıń.
 \\
\textbf{A2.} 
Kólemi \(n = 20\) ǵa teń bolǵan tańlanba berilgen:1,6; -8,3; 7,6; 4,2; -3,1; 1,6; 4,2; -10,5; -8,3; 7,6; -10,5; 4,2; -8,3; 1,6; 4,2; -10,5; 7,6; -3,1; 4,2; -8,3. Bul tańlanbanıń empirikalıq bólistiriw funkciyasın tabıń.
 \\
\textbf{A3.} 
Joqarı matematika páninen 10 dana student test sınaqların tapsırǵan. Hárbir student 10 balǵa shekem toplawı múmkin. Eger test sınaqları nátiyjeleri boyınsha \{9, 3, 6, 3, 7, 6, 4, 6, 10, 6\} tańlanba alınǵan bolsa, onda tańlanba ortasha hám tańlanba dispersiyalardı tabıń.
 \\
\textbf{B1.} 
Eger ortasha kvadratlıq shetleniwi \(\sigma = 2\) bolǵan normal bólistirilgen bas toplamnan alınǵan kólemi \(n = 18\) ǵa teń tańlanba boyınsha \(\overline{x} = 5,2\) tańlanba ortasha mánisi tabılǵan bolsa, onda \(\gamma = 0,90\) isenimlilik penen belgisiz \(\theta\) matematikalıq kútiliwdi qaplaytuǵın isenimlilik intervalın dúziń.
 \\
\textbf{B2.} 
Eger tıǵızlıq funkciyası \(f(x) = \frac{2x}{\theta}e^{- \frac{x^{2}}{\theta}},\ \ x \geq 0\) kóriniske iye bolsa, onda \(\theta\) parametr momentler usulı bahasın tabıń.
 \\
\textbf{B3.} 
Eger (4,8,5,3) tańlanba \(\left( a,\theta^{2} \right)\) parametrli normal bólistiriliwden alınǵan bolsa, onda belgisiz \(\theta^{2}\) parametrdiń shınlıqqa maksimal uqsaslıq bahasın tabıń.
 \\
\textbf{C1.} 
Eger \(X^{(n)} = \left( X_{1},...,X_{n} \right)\) tańlanba \(\theta\) parametrli Bernulli bólistiriliwinen alınǵan bolsa, onda belgisiz \(\theta(1 - \theta)\) parametr ushın \(X_{1}\left( 1 - X_{n} \right)\) bahasın jıljımaǵanlıq hám tiykarlılıqqa tekseriń.
 \\
\textbf{C2.} 
Eger \(X^{(n)} = \left( X_{1},...,X_{n} \right)\) tańlanba tıǵızlıq funkciyası
${f(x,\theta) = \theta x}^{\theta - 1},x \in \lbrack 0,1\rbrack$
bolǵan bólistiriliwden alınǵan bolsa, onda belgisiz \(\theta\) parametr ushın momentler usılı bahasın tabıń.
 \\
\textbf{C3.} 
Eger \(X^{(n)} = \left( X_{1},...,X_{n} \right)\) tańlanba tıǵızlıq funkciyası
$f(x;\theta) = \frac{1}{2}e^{- \ |x - \theta|},\ x \in R$
bolǵan Laplas bólistiriliwinen alınǵan bolsa, onda belgisiz \(\theta \in R\) parametrdiń shınlıqqa maksimal uqsaslıq bahasın tabıń.
 \\

\end{tabular}
\vspace{1cm}


\begin{tabular}{m{17cm}}
\textbf{92-variant}
\newline

\textbf{T1.} Matematikalıq statistikanıń tiykarǵı máseleleri. (Statistikalıq maǵlıwmatlar, gruppalaw)
 \\
\textbf{T2.} 
Kolmogorovtıń kelisimlilik belgisi (Kolmogorov teoreması)
 \\
\textbf{A1.} 
Kólemi \(n = 20\) ǵa teń bolǵan tańlanba berilgen: 2,7; -13,5; 1,2; 2,7; 1,2; 4,9; -9,5; 1,2; 2,7; 4,9; -9,5; 2,7; -3,5; 1,2; 2,7; 4,9; -3,5; 2,7; 4,9; 1,2;. Bul tańlanbanıń statistikalıq bólistiriliwin tabıń.
 \\
\textbf{A2.} 
Kólemi \(n = 20\) ǵa teń bolǵan tańlanba berilgen: 2,8; -13,9; 1,9; 2,8; 1,9; 4,3; -9,4; 1,9; 2,8; 4,3; -9,4; 2,8; -3,7; 1,9; 2,8; 4,3; -3,7; 2,8; 4,3; 1,9. Bul tańlanbanıń empirikalıq bólistiriw funkciyasın tabıń.
 \\
\textbf{A3.} 
Joqarı matematika páninen 10 dana student test sınaqların tapsırǵan. Hárbir student 10 balǵa shekem toplawı múmkin. Eger test sınaqları nátiyjeleri boyınsha \{10, 7, 5, 9, 3, 8, 10, 7, 8, 3\} tańlanba alınǵan bolsa, onda tańlanba ortasha hám tańlanba dispersiyalardı tabıń.
 \\
\textbf{B1.} 
Eger normal bólistirilgen bas toplamnan alınǵan kólemi \(n = 36\) ǵa teń tańlanba boyınsha \(\overline{x} = 20,2\) tańlanba ortasha hám \({\overline{S}}^{2} = 0,81\) dúzetilgen tańlanba dispersiyalar tabılǵan bolsa, onda \(\gamma = 0,95\) isenimlilik penen belgisiz \(\theta\) matematikalıq kútiliw ushın isenimlilik interval dúziń.
 \\
\textbf{B2.} 
Eger (-2,0,-2,0,-2,3,-2,0,0,3,0,0,0,3,-2,0,0,-2,3,0) tańlanba tómende berilgen bólistiriliwden alınǵan bolsa, onda belgisiz \(\left( \theta_{1},\theta_{2} \right)\) vektor parametr ushın momentler usılı bahalasın tabıń.
\begin{tabular}{|c|c|c|c|}
  \hline
$\xi$ &
$- 2$ &
$0$ &
$3$\\
\hline
\(P_{\theta}\) & \(\theta_{1}\) & \(1 - \theta_{1} - \theta_{2}\) & \(\theta_{2}\) \\
\hline
\end{tabular}
 \\
\textbf{B3.} 
Eger \(X^{(n)} = \left( X_{1},...,X_{n} \right)\) tańlanba tıǵızlıq funkciyası \(f(x;\theta) = \frac{2x}{\theta}e^{- \frac{x^{2}}{\theta}},\ x \geq 0\). bolǵan bólistiriliwden alınǵan bolsa, onda belgisiz \(\theta > 0\) parametrdiń shınlıqqa maksimal uqsaslıq bahasın tabıń.
 \\
\textbf{C1.} 
Eger \(X^{(n)} = \left( X_{1},...,X_{n} \right)\) tańlanba \(\theta\) parametrli Bernulli bólistiriliwinen alınǵan bolsa, onda belgisiz \(\theta^{2}\) parametr ushın \(X_{1}X_{n}\) bahasın jıljımaǵanlıq hám tiykarlılıqqa tekseriń.
 \\
\textbf{C2.} 
Eger \(X^{(n)} = \left( X_{1},...,X_{n} \right)\) tańlanba \(\left( \theta_{1},\theta_{2} \right)\) parametrli gamma bólistiriliwden alınǵan bolsa, onda belgisiz \(\left( \theta_{1},\theta_{2} \right)\) vektor parametr ushın momentler usılı bahasın tabıń.
 \\
\textbf{C3.} 
Eger \(X^{(n)} = \left( X_{1},...,X_{n} \right)\) tańlanba tıǵızlıq funkciyası
$f(x;\theta) = \frac{\theta}{\sqrt{2\pi x^{3}}}e^{\frac{- \ \theta^{2}}{2x}},\ x \geq 0$
bolǵan bólistiriliwden alınǵan bolsa, onda belgisiz \(\theta > 0\) parametrdiń shınlıqqa maksimal uqsaslıq bahasın tabıń.
 \\

\end{tabular}
\vspace{1cm}


\begin{tabular}{m{17cm}}
\textbf{93-variant}
\newline

\textbf{T1.} 
Poligon hám gistogramma(salıstirmalı jiyilik, intervallıq qatar, grafik)
 \\
\textbf{T2.} 
Isenimlilik intervalların qurıw. Anıq isenimli intervallar
 \\
\textbf{A1.} 
Kólemi \(n = 20\) ǵa teń bolǵan tańlanba berilgen: 9,9; 5,7; 3,2; 2,8; 5,7; 9,9; 7,5; 3,7; 9,9; 3,2; 2,8; 3,7; 7,5; 5,7; 3,2; 2,8; 7,5; 3,2; 9,9; 7,5. Bul tańlanbanıń statistikalıq bólistiriliwin tabıń.
 \\
\textbf{A2.} 
Kólemi \(n = 20\) ǵa teń bolǵan tańlanba berilgen: 9,7; 5,2; 3,2; 2,4; 5,2; 9,7; 7,5; 3,7; 9,7; 3,2; 2,4; 3,7; 7,5; 5,2; 3,2; 2,4; 7,5; 3,2; 9,7; 7,5. Bul tańlanbanıń empirikalıq bólistiriw funkciyasın tabıń.
 \\
\textbf{A3.} 
Joqarı matematika páninen 10 dana student test sınaqların tapsırǵan. Hárbir student 10 balǵa shekem toplawı múmkin. Eger test sınaqları nátiyjeleri boyınsha \{1, 6, 2, 6, 3, 6, 4, 6, 10, 6\} tańlanba alınǵan bolsa, onda tańlanba ortasha hám tańlanba dispersiyalardı tabıń.
 \\
\textbf{B1.} 
Eger normal bólistirilgen bas toplamnan alınǵan kólemi \(n = 10\) ǵa teń bolǵan tańlanba boyınsha \({\overline{S}}^{2} = 0,7\) dúzetilgen tańlanba dispersiya tabılǵan bolsa, onda \(\gamma = 0,95\) isenimlilik penen belgisiz \(\theta_{2}^{2}\) dispersiya ushın isenimlilik interval dúziń.
 \\
\textbf{B2.} 
Kórsetkishli bólistiriw belgisiz \(\theta > 0\) parametri momentlar usulı bahasın tabıń.
 \\
\textbf{B3.} 
Eger \(X^{(n)} = \left( X_{1},...,X_{n} \right)\) tańlanba \(\theta\) parametrli Bernulli bólistiriliwinen alınǵan bolsa, onda belgisiz \(\theta\) parametrdiń shınlıqqa maksimal uqsaslıq usılı bahasın tabıń.
 \\
\textbf{C1.} 
Eger \(X^{(n)} = \left( X_{1},...,X_{n} \right)\) tańlanba \((\alpha,\theta)\) parametrli Veybull bólistiriliwinen alınǵan bolsa (\(\alpha -\)belgili), onda belgisiz \(\theta\) parametr ushın \(\frac{1}{\overline{x^{\alpha}}}\) bahasın jıljımaǵanlıq hám tiykarlılıqqa tekseriń.
 \\
\textbf{C2.} 
Eger \(X^{(n)} = \left( X_{1},...,X_{n} \right)\) tańlanba \(\theta\) parametrli Puasson bólistiriliwinen alınǵan bolsa, onda belgisiz \(\theta\) parametr ushın momentler usılı bahasın tabıń. Eger \(X^{(n)} = \left( X_{1},...,X_{n} \right)\) tańlanba \(\theta\) parametrli Puasson bólistiriliwinen alınǵan bolsa, onda belgisiz \(\theta\) parametr ushın momentler usılı bahasın\({\ g(x) = x}^{2}\) funkciyası járdeminde tabıń.
 \\
\textbf{C3.} 
Eger \(X^{(n)} = \left( X_{1},...,X_{n} \right)\) tańlanba tıǵızlıq funkciyası
$f(x;\theta) = \frac{3x^{2}}{\sqrt{2\pi}}\exp\left\{ - \frac{\left( x^{3} - \theta \right)^{2}}{2} \right\},\ x \in R$
bolǵan bólistiriliwden alınǵan bolsa, onda belgisiz \(\theta\) parametrdiń shınlıqqa maksimal uqsaslıq bahasın tabıń.
 \\

\end{tabular}
\vspace{1cm}


\begin{tabular}{m{17cm}}
\textbf{94-variant}
\newline

\textbf{T1.} 
Gruppalanǵan hám intervallıq variaciyalıq qatarlar.
 \\
\textbf{T2.} 
Statistikalıq gipotezalardı tekseriw (kritikalıq kóplik, 1 hám 2-túr qátelik).
 \\
\textbf{A1.} 
Kólemi \(n = 20\) ǵa teń bolǵan tańlanba berilgen: 3,6; 1,1; -1,8; 0,4; 3,6; 0; 5,3; 1,1; 0; -1,8; 3,6; 0,4; 1,1; 0; 0,4; 1,1; 3,6; -1,8; 3,6; 0. Bul tańlanbanıń statistikalıq bólistiriliwin tabıń.
 \\
\textbf{A2.} 
Kólemi \(n = 20\) ǵa teń bolǵan tańlanba berilgen: 3,2; 1,8; -1,1; 0,9; 3,2; 0; 5,6; 1,8; 0; -1,1; 3,2; 0,9; 1,8; 0; 0,9; 1,8; 3,2; -1,1; 3,2; 0. Bul tańlanbanıń empirikalıq bólistiriw funkciyasın tabıń.
 \\
\textbf{A3.} 
Joqarı matematika páninen 10 dana student test sınaqların tapsırǵan. Hárbir student 10 balǵa shekem toplawı múmkin. Eger test sınaqları nátiyjeleri boyınsha \{2, 7, 3, 7, 6, 7, 4, 7, 7, 10\} tańlanba alınǵan bolsa, onda tańlanba ortasha hám tańlanba dispersiyalardı tabıń.
 \\
\textbf{B1.} 
Eger ortasha kvadratlıq shetleniwi \(\sigma = 3\) bolǵan normal bólistirilgen bas toplamnan alınǵan kólemi \(n = 14\) ǵa teń tańlanba boyınsha \(\overline{x} = 5,5\) tańlanba ortasha mánisi tabılǵan bolsa, onda \(\gamma = 0,90\) isenimlilik penen belgisiz \(\theta\) matematikalıq kútiliwdi qaplaytuǵın isenimlilik intervalın dúziń.
 \\
\textbf{B2.} 
\(\lbrack\theta_{1},\theta_{2}\rbrack\) aralıqta teń ólshewli bólistiriw parametrleri ushın momentler usulı bahaların tabıń.
 \\
\textbf{B3.} 
Eger \(X^{(n)} = \left( X_{1},...,X_{n} \right)\) tańlanba \(\left( a,\theta^{2} \right)\) parametrli normal bólistiriliwden alınǵan bolsa (\(\alpha -\)belgili), onda belgisiz \(\theta^{2}\) parametrdiń shınlıqqa maksimal uqsaslıq bahasın tabıń.
 \\
\textbf{C1.} 
Eger \(X^{(n)} = \left( X_{1},...,X_{n} \right)\) tańlanba \(\theta\) parametrli geometriyalıq bólistiriliwden alınǵan bolsa, onda belgisiz \(\theta\) parametr ushın \(\frac{1}{(1 + \overline{x})}\) bahasın jıljımaǵanlıq hám tiykarlılıqqa tekseriń.
 \\
\textbf{C2.} 
Eger \(X^{(n)} = \left( X_{1},...,X_{n} \right)\) tańlanba \((\theta,2\theta)\) parametrli normal bólistiriliwden alınǵan bolsa, onda belgisiz \(\theta > 0\) parametr ushın momentler usılı bahasın \({\ g(x) = (x)}^{2}\) funkciyası járdeminde tabıń.
 \\
\textbf{C3.} 
\(\ f(x,\theta) = \frac{4x^{3}}{\theta_{2}\sqrt{2\pi}}\exp\left\{ - \frac{\left( x^{4} - \theta_{1} \right)^{2}}{2{\theta_{2}}^{2}} \right\}\) model ushın \(\theta_{1}\) hám \({\theta_{2}}^{2}\) parametrler haqıyqatqa maksimal uqsaslıq usılı bahaları tabılsın.
 \\

\end{tabular}
\vspace{1cm}


\begin{tabular}{m{17cm}}
\textbf{95-variant}
\newline

\textbf{T1.} 
Tańlanba xarakteristikaları.(tańlanba orta, tańlanba dispersiya)
 \\
\textbf{T2.} 
Momentler usulı. (tańlanba momentleri, belgisiz parametrlerdi bahalaw).
 \\
\textbf{A1.} 
Kólemi \(n = 20\) ǵa teń bolǵan tańlanba berilgen: 7,1; 3,9; 6,3; 4,6; 7,1; 2,3; 6,3; 3,9; 4,6; 7,1; 2,3; 3,9; 7,6; 2,3; 4,6; 3,9; 2,3; 3,9; 7,6; 4,6. Bul tańlanbanıń statistikalıq bólistiriliwin tabıń.
 \\
\textbf{A2.} 
Kólemi \(n = 20\) ǵa teń bolǵan tańlanba berilgen: 7,9; 3,8; 6,1; 4,2; 7,9; 2,4; 6,1; 3,8; 4,2; 7,9; 2,4; 3,8; 10,2; 2,4; 4,2; 3,8; 2,4; 3,8; 10,2; 4,2. Bul tańlanbanıń empirikalıq bólistiriw funkciyasın tabıń.
 \\
\textbf{A3.} 
Joqarı matematika páninen 10 dana student test sınaqların tapsırǵan. Hárbir student 10 balǵa shekem toplawı múmkin. Eger test sınaqları nátiyjeleri boyınsha \{9, 8, 6, 8, 6, 4, 5, 4, 7, 4\} tańlanba alınǵan bolsa, onda tańlanba ortasha hám tańlanba dispersiyalardı tabıń.
 \\
\textbf{B1.} 
Eger normal bólistirilgen bas toplamnan alınǵan kólemi \(n = 49\) ǵa teń tańlanba boyınsha \(\overline{x} = 14,2\) tańlanba ortasha hám \({\overline{S}}^{2} = 0,64\) dúzetilgen tańlanba dispersiyalar tabılǵan bolsa, onda \(\gamma = 0,95\) isenimlilik penen belgisiz \(\theta\) matematikalıq kútiliw ushın isenimlilik interval dúziń.
 \\
\textbf{B2.} 
Eger (3,-2,-2,0,-2,-2,-2,0,-2,3,-2,0,3,0,3,-2,0,-2,3,-2,-2,-2,-2,3,3,3,-2,-2,3,3) tańlanba tómende berilgen bólistiriliwden alınǵan bolsa, onda belgisiz \(\theta\) parametr ushın momentler usılı bahasın \(g(x) = |x|\) funkciyası járdeminde tabıń.
\begin{tabular}{|c|c|c|c|}
  \hline
$\xi$ &
$- 2$ &
$0$ &
$3$ \\
\hline
\(P_{\theta}\) & \(3\theta\) & \(1 - 5\theta\) & \(2\theta\) \\
\hline
\end{tabular}
 \\
\textbf{B3.} 
\(\ f(x) = \frac{\theta}{2}e^{- \theta|x|}\) model ushın \(\theta\) parametri haqıyqatqa maksimal uqsaslıq usılı bahası tabılsın.
 \\
\textbf{C1.} 
Eger \(X^{(n)} = \left( X_{1},...,X_{n} \right)\) tańlanba \(\theta\) parametrli Puasson bólistiriliwinen alınǵan bolsa, onda belgisiz \(\theta\) parametr ushın \(\frac{n + 3}{n + 4}\overline{x}\) bahasın jıljımaǵanlıq hám tiykarlılıqqa tekseriń.
 \\
\textbf{C2.} 
Eger \(X^{(n)} = \left( X_{1},...,X_{n} \right)\) tańlanba {[}\(0,2\theta\rbrack\) aralıqta teń ólshemli bólistiriliwden alınǵan bolsa, onda belgisiz \(\theta > 0\) parametr ushın momentler usılı bahasın tabıń.
 \\
\textbf{C3.} 
Eger \(X^{(n)} = \left( X_{1},...,X_{n} \right)\) tańlanba tıǵızlıq funkciyası
$f(x;\theta) = \left\{ \begin{matrix}
3x^{2}\theta^{- 3}e^{- \ \left( \frac{x}{\theta} \right)^{3}},\ \ x \geq 0, \\
\ \ \ \ \ \ \ \ \ \ \ \ \ \ 0,\ \ \ \ \ \ \ \ \ x < 0
\end{matrix} \right.\ $
bolǵan bólistiriliwden alınǵan bolsa, onda belgisiz \(\theta > 0\) parametrdiń shınlıqqa maksimal uqsaslıq bahasın tabıń.
 \\

\end{tabular}
\vspace{1cm}


\begin{tabular}{m{17cm}}
\textbf{96-variant}
\newline

\textbf{T1.} 
Neyman-Pirson teoreması
 \\
\textbf{T2.} 
Pirsonnıń xi-kvadrat kelisimlilik belgisi (Pirson teoreması).
 \\
\textbf{A1.} 
Kólemi \(n = 20\) ǵa teń bolǵan tańlanba berilgen: 0,6; -3,8; -2,3; -4,3; 2,8; 4,7; -2,3; 0,6; -3,8; 2,8; -2,3; -4,3; 0,6; -2,3; 2,8; -3,8; -4,3; -2,3; 2,8; -3,8. Bul tańlanbanıń statistikalıq bólistiriliwin tabıń.
 \\
\textbf{A2.} 
Kólemi \(n = 20\) ǵa teń bolǵan tańlanba berilgen: 0,7; -3,1; -2,3; -4,8; 2,6; 4,9; -2,3; 0,7; -3,1; 2,6; -2,3; -4,8; 0,7; -2,3; 2,6; -3,1; -4,8; -2,3; 2,6; -3,1. Bul tańlanbanıń empirikalıq bólistiriw funkciyasın tabıń.
 \\
\textbf{A3.} 
Joqarı matematika páninen 10 dana student test sınaqların tapsırǵan. Hárbir student 10 balǵa shekem toplawı múmkin. Eger test sınaqları nátiyjeleri boyınsha \{10, 4, 6, 5, 5, 4, 10, 7, 9, 10\} tańlanba alınǵan bolsa, onda tańlanba ortasha hám tańlanba dispersiyalardı tabıń.
 \\
\textbf{B1.} 
Eger normal bólistirilgen bas toplamnan alınǵan kólemi \(n = 8\) ǵa teń bolǵan tańlanba boyınsha \({\overline{S}}^{2} = 0,35\) dúzetilgen tańlanba dispersiya tabılǵan bolsa, onda \(\gamma = 0,90\) isenimlilik penen belgisiz \(\theta_{2}^{2}\) dispersiya ushın isenimlilik interval dúziń.
 \\
\textbf{B2.} 
\(\lbrack 0,\theta\rbrack\) aralıqta teń ólshewli bólistirilgen \(\theta\) parametri ushın momentler usulı bahasın tabıń.
 \\
\textbf{B3.} 
Eger (0,1,2,0) tańlanba tómende berilgen bólistiriliwden alınǵan bolsa, onda belgisiz \(\theta\) parametrdiń shınlıqqa maksimal uqsaslıq bahasın tabıń.
\begin{tabular}{|c|c|c|c|}
  \hline
$\xi$
&
$0$
&
$1$
&
$2$\\
\hline
\(P_{\theta}\) & \(\theta\) & \(2\theta\) & \(1 - 3\theta\) \\
\hline
\end{tabular}
 \\
\textbf{C1.} 
Eger \(X^{(n)} = \left( X_{1},...,X_{n} \right)\) tańlanba \(\theta\) parametrli Puasson bólistiriliwinen alınǵan bolsa, onda belgisiz \(\theta\) parametr ushın \(\frac{X_{1} + X_{3}}{2}\) bahasın jıljımaǵanlıq hám tiykarlılıqqa tekseriń.
 \\
\textbf{C2.} 
Eger \(X^{(n)} = \left( X_{1},...,X_{n} \right)\) tańlanba \(\frac{1}{\theta}\) parametrli kórsetkishli bólistiriliwden alınǵan bolsa, onda belgisiz \(\theta\) parametr ushın momentler usılı bahasın\({\ g(x) = x}^{k},\) \(k \in N\)funkciyası járdeminde tabıń.
 \\
\textbf{C3.} 
Eger \(X^{(n)} = \left( X_{1},...,X_{n} \right)\) tańlanba tıǵızlıq funkciyası
$f(x;\theta) = \left\{ \begin{matrix}
e^{\theta - x},\ \ x \geq \theta, \\
\ \ 0,\ \ \ \ \ \ \ x < \theta
\end{matrix} \right.\ $
bolǵan bólistiriliwden alınǵan bolsa, onda belgisiz \(\theta\) parametrdiń shınlıqqa maksimal uqsaslıq bahasın tabıń.
 \\

\end{tabular}
\vspace{1cm}


\begin{tabular}{m{17cm}}
\textbf{97-variant}
\newline

\textbf{T1.} 
Empirikalıq bólistiriw funkciyası. (Tańlanba, eksperiment)
 \\
\textbf{T2.} 
Statistikalıq baha qásiyetleri. (Jıljımaytuǵın, tiykarlı, effektiv)
 \\
\textbf{A1.} 
Kólemi \(n = 20\) ǵa teń bolǵan tańlanba berilgen: 8,9; 2,7; 1,7; 2,2; 5,6; 1,7; 5,6; 2,7; 1,7; 2,2; 5,6; 8,9; 1,7; 2,2; 1,7; 2,7; 1,7; 5,6; 6,1; 8,9. Bul tańlanbanıń statistikalıq bólistiriliwin tabıń.
 \\
\textbf{A2.} 
Kólemi \(n = 20\) ǵa teń bolǵan tańlanba berilgen: 8,7; 2,7; 1,5; 2,2; 5,7; 1,5; 5,7; 2,7; 1,5; 2,2; 5,7; 8,7; 1,5; 2,2; 1,5; 2,7; 1,5; 5,7; 6,3; 8,7. Bul tańlanbanıń empirikalıq bólistiriw funkciyasın tabıń.
 \\
\textbf{A3.} 
Joqarı matematika páninen 10 dana student test sınaqların tapsırǵan. Hárbir student 10 balǵa shekem toplawı múmkin. Eger test sınaqları nátiyjeleri boyınsha \{9, 8, 6, 9, 5, 4, 5, 7, 8, 9\} tańlanba alınǵan bolsa, onda tańlanba ortasha hám tańlanba dispersiyalardı tabıń.
 \\
\textbf{B1.} 
Eger ortasha kvadratlıq shetleniwi \(\sigma = 4\) bolǵan normal bólistirilgen bas toplamnan alınǵan kólemi \(n = 16\) ǵa teń tańlanba boyınsha \(\overline{x} = 5,8\) tańlanba ortasha mánisi tabılǵan bolsa, onda \(\gamma = 0,90\) isenimlilik penen belgisiz \(\theta\) matematikalıq kútiliwdi qaplaytuǵın isenimlilik intervalın dúziń.
 \\
\textbf{B2.} 
Eger \(X^{(n)} = \left( X_{1},...,X_{n} \right)\) tańlanba \(\theta\) parametrli kórsetkishli bólistiriliwden alınǵan bolsa, onda belgisiz \(\theta\) parametr ushın momentler usılı bahasın tabıń.
 \\
\textbf{B3.} 
Eger \(X^{(n)} = \left( X_{1},...,X_{n} \right)\) tańlanba tıǵızlıq funkciyası \(f(x;\theta) = \frac{2x}{\theta}e^{- \frac{x^{2}}{\theta}},\ x \geq 0\). bolǵan bólistiriliwden alınǵan bolsa, onda belgisiz \(\theta > 0\) parametrdiń shınlıqqa maksimal uqsaslıq bahasın tabıń.
 \\
\textbf{C1.} 
Eger \(X^{(n)} = \left( X_{1},...,X_{n} \right)\) tańlanba \(\ln\theta\) parametrli Puasson bólistiriliwinen alınǵan bolsa, onda belgisiz \(\theta\) parametr ushın \(e^{\overline{x}}\) bahasın jıljımaǵanlıq hám tiykarlılıqqa tekseriń.
 \\
\textbf{C2.} 
Eger \(X^{(n)} = \left( X_{1},...,X_{n} \right)\) tańlanba \({(\theta,\theta}^{2})\ \) parametrli normal bólistiriliwden alınǵan bolsa, onda belgisiz \(\theta > 0\) parametr ushın momentler usılı bahasın tabıń.
 \\
\textbf{C3.} 
Eger \(X^{(n)} = \left( X_{1},...,X_{n} \right)\) tańlanba tıǵızlıq funkciyası
$f(x;\theta) = \frac{\theta}{2}e^{- \theta|x|},\ x \in R$
bolǵan bólistiriliwden alınǵan bolsa, onda belgisiz \(\theta > 0\) parametrdiń shınlıqqa maksimal uqsaslıq bahasın tabıń.
 \\

\end{tabular}
\vspace{1cm}


\begin{tabular}{m{17cm}}
\textbf{98-variant}
\newline

\textbf{T1.} 
Glivenko-Kantelli teoreması. (empirikalıq bólistiriw funkciyası, 1itimallıq penen jaqınlasıw)
 \\
\textbf{T2.} 
Statistikalıq gipotezalardı tekseriw (kritikalıq kóplik, 1 hám 2-túr qátelik)
 \\
\textbf{A1.} 
Kólemi \(n = 20\) ǵa teń bolǵan tańlanba berilgen: 1,8; -1,9; 2,4; 1,8; 2,4; 1,8; 2,4; -0,6; -1,9; 1,8; -0,6; 2,4; -3,3; -1,9; 4,0; -3,3; -3,3; -1,9; -3,3; -1,9. Bul tańlanbanıń statistikalıq bólistiriliwin tabıń.
 \\
\textbf{A2.} 
Kólemi \(n = 20\) ǵa teń bolǵan tańlanba berilgen: 1,4; -1,9; 2,5; 1,4; 2,5; 1,4; 2,5; -0,4; -1,9; 1,4; -0,4; 2,5; -3,7; -1,9; 4,5; -3,7; -3,7; -1,9; -3,7; -1,9. Bul tańlanbanıń empirikalıq bólistiriw funkciyasın tabıń.
 \\
\textbf{A3.} 
Joqarı matematika páninen 10 dana student test sınaqların tapsırǵan. Hárbir student 10 balǵa shekem toplawı múmkin. Eger test sınaqları nátiyjeleri boyınsha \{4, 3, 8, 4, 8, 3, 9, 4, 7, 10\} tańlanba alınǵan bolsa, onda tańlanba ortasha hám tańlanba dispersiyalardı tabıń.
 \\
\textbf{B1.} 
Eger normal bólistirilgen bas toplamnan alınǵan kólemi \(n = 36\) ǵa teń tańlanba boyınsha \(\overline{x} = 20,2\) tańlanba ortasha hám \({\overline{S}}^{2} = 0,64\) dúzetilgen tańlanba dispersiyalar tabılǵan bolsa, onda \(\gamma = 0,90\) isenimlilik penen belgisiz \(\theta\) matematikalıq kútiliw ushın isenimlilik interval dúziń.
 \\
\textbf{B2.} 
Puasson bólistiriliwi belgisiz \(\theta > 0\) parametri momentlar usuli bahasin tabıń.
 \\
\textbf{B3.} 
Eger \(X^{(n)} = \left( X_{1},...,X_{n} \right)\) tańlanba \(\theta\) parametrli kórsetkishli bólistiriliwden alınǵan bolsa, onda belgisiz \(\theta\) parametrdiń shınlıqqa maksimal uqsaslıq usılı bahasın tabıń.
 \\
\textbf{C1.} 
Eger \(X^{(n)} = \left( X_{1},...,X_{n} \right)\) tańlanba \((\alpha,\theta)\) parametrli Pareto bólistiriliwinen alınǵan bolsa (\(\alpha -\)belgili), onda belgisiz \(\theta\) parametr ushın \(X_{(1)}\) bahasın jıljımaǵanlıq hám tiykarlılıqqa tekseriń.
 \\
\textbf{C2.} 
Eger \(X^{(n)} = \left( X_{1},...,X_{n} \right)\) tańlanba \({(\theta,\theta}^{2})\) parametrli normal bólistiriliwden \({\ g(x) = (x)}^{2}\ \)alınǵan bolsa, onda belgisiz \(\theta > 0\) parametr ushın momentler usılı bahasın funkciyası járdeminde tabıń.
 \\
\textbf{C3.} 
Eger \(X^{(n)} = \left( X_{1},...,X_{n} \right)\) tańlanba tıǵızlıq funkciyası
$f(x;\theta) = \frac{e^{x}}{\sqrt{2\pi}}\exp\left\{ - \frac{\left( e^{x} - \theta \right)^{2}}{2} \right\},\ x \in R$
bolǵan bólistiriliwden alınǵan bolsa, onda belgisiz \(\theta\) parametrdiń shınlıqqa maksimal uqsaslıq bahasın tabıń.
 \\

\end{tabular}
\vspace{1cm}


\begin{tabular}{m{17cm}}
\textbf{99-variant}
\newline

\textbf{T1.} 
Tańlanba xarakteristikalar. (Variaciyalıq qatar, salıstırmalı jiyilik).
 \\
\textbf{T2.} 
Sızıqlı korrelyaciya teńlemesi (anıqlaması, regressiya tuwrı sızıǵınıń tańlanba teńlemeleri)
 \\
\textbf{A1.} 
Kólemi \(n = 20\) ǵa teń bolǵan tańlanba berilgen: 2,9; -3,2; 5,3; -4,3; 4,1; 5,3; -1,2; 2,9; -3,2; 4,1; -4,3; 5,3; -3,2; 2,9; -4,3; 4,1; -1,2; 5,3; 2,9; -3,2. Bul tańlanbanıń statistikalıq bólistiriliwin tabıń.
 \\
\textbf{A2.} 
Kólemi \(n = 20\) ǵa teń bolǵan tańlanba berilgen: 2,7; -5,6; 5,2; -8,1; 4,8; 5,2; -1,6; 2,7; -5,6; 4,8; -8,1; 5,2; -5,6; 2,7; -8,1; 4,8; -1,6; 5,2; 2,7; -5,6. Bul tańlanbanıń empirikalıq bólistiriw funkciyasın tabıń.
 \\
\textbf{A3.} 
Joqarı matematika páninen 10 dana student test sınaqların tapsırǵan. Hárbir student 10 balǵa shekem toplawı múmkin. Eger test sınaqları nátiyjeleri boyınsha \{7, 9, 4, 9, 7, 5, 4, 7, 2, 6\} tańlanba alınǵan bolsa, onda tańlanba ortasha hám tańlanba dispersiyalardı tabıń.
 \\
\textbf{B1.} 
Eger normal bólistirilgen bas toplamnan alınǵan kólemi \(n = 11\) ǵa teń bolǵan tańlanba boyınsha \({\overline{S}}^{2} = 0,3\) dúzetilgen tańlanba dispersiya tabılǵan bolsa, onda \(\gamma = 0,95\) isenimlilik penen belgisiz \(\theta_{2}^{2}\) dispersiya ushın isenimlilik interval dúziń.
 \\
\textbf{B2.} 
Eger (3,0,-2,0,-2,3,-2,0,0,3,0,0,0,3,-2,0,0,-2,3,0) tańlanba tómende berilgen bólistiriliwden alınǵan bolsa, onda belgisiz \(\left( \theta_{1},\theta_{2} \right)\) vektor parametr ushın momentler usılı bahalasın tabıń.
\begin{tabular}{|c|c|c|c|}
  \hline
$\xi$ &
$- 2$ &
$0$ &
$3$\\
\hline
\(P_{\theta}\) & \({2\theta}_{1}\) & \(0,5 + \theta_{1} + \theta_{2}\) & \(\theta_{2}\) \\
\hline
\end{tabular}
 \\
\textbf{B3.} 
Eger (4,8,5,3) tańlanba \(\left( a,\theta^{2} \right)\) parametrli normal bólistiriliwden alınǵan bolsa, onda belgisiz \(\theta^{2}\) parametrdiń shınlıqqa maksimal uqsaslıq bahasın tabıń.
 \\
\textbf{C1.} 
Eger \(X^{(n)} = \left( X_{1},...,X_{n} \right)\) tańlanba tıǵızlıq funkciyası: \(f(x;\theta) = e^{- x + \theta}\left( 1 + e^{- x + \theta} \right)^{2},\ x \in R\)
bolǵan bólistiriliwden alınǵan bolsa, onda belgisiz \(\theta\) parametr ushın \(\overline{x}\) bahasın jıljımaǵanlıq hám tiykarlılıqqa tekseriń.
 \\
\textbf{C2.} 
Eger \(X^{(n)} = \left( X_{1},...,X_{n} \right)\) tańlanba\(\ (\theta,2\theta)\ \) parametrli normal bólistiriliwden alınǵan bolsa, onda belgisiz \(\theta > 0\) parametr ushın momentler usılı bahasın tabıń.
 \\
\textbf{C3.} 
Eger \(X^{(n)} = \left( X_{1},...,X_{n} \right)\) tańlanba \(\left\lbrack \theta_{1},\theta_{2} \right\rbrack\) aralıqta teń ólshemli bólistiriliwden alınǵan bolsa, onda belgisiz \(\left( \theta_{1},\theta_{2} \right)\) vektor parametrdiń shınlıqqa maksimal uqsaslıq bahasın tabıń.
 \\

\end{tabular}
\vspace{1cm}


\begin{tabular}{m{17cm}}
\textbf{100-variant}
\newline

\textbf{T1.} 
Tańlanba momentleri (\(k -\)tártipli baslanǵısh, baslanǵısh absolyut, oraylıq hám oraylıq absolyut momentler).
 \\
\textbf{T2.} 
Normal nızamnıń dispersiyası ushın isenimlilik intervalın dúziw. (Isenimlilik itimallıǵı, interval)
 \\
\textbf{A1.} 
Kólemi \(n = 20\) ǵa teń bolǵan tańlanba berilgen: 14,7; 7,3; 16,6; 9,8; 11,2; 16,6; 6,7; 7,3; 11,2; 14,7; 6,7; 16,6; 7,3; 11,2; 14,7; 16,6; 6,7; 7,3; 11,2; 16,6. Bul tańlanbanıń statistikalıq bólistiriliwin tabıń.
 \\
\textbf{A2.} 
Kólemi \(n = 20\) ǵa teń bolǵan tańlanba berilgen: 14,4; 7,6; 16,7; 9,1; 11,8; 16,7; 6,4; 7,6; 11,8; 14,4; 6,4; 16,7; 7,6; 11,8; 14,4; 16,7; 6,4; 7,6; 11,8; 16,7. Bul tańlanbanıń empirikalıq bólistiriw funkciyasın tabıń.
 \\
\textbf{A3.} 
Joqarı matematika páninen 10 dana student test sınaqların tapsırǵan. Hárbir student 10 balǵa shekem toplawı múmkin. Eger test sınaqları nátiyjeleri boyınsha \{10, 8, 4, 6, 2, 8, 5, 10, 2, 5\} tańlanba alınǵan bolsa, onda tańlanba ortasha hám tańlanba dispersiyalardı tabıń.
 \\
\textbf{B1.} 
Eger ortasha kvadratlıq shetleniwi \(\sigma = 4\) bolǵan normal bólistirilgen bas toplamnan alınǵan kólemi \(n = 49\) ǵa teń tańlanba boyınsha \(\overline{x} = 9,4\) tańlanba ortasha mánisi tabılǵan bolsa, onda \(\gamma = 0,90\) isenimlilik penen belgisiz \(\theta\) matematikalıq kútiliwdi qaplaytuǵın isenimlilik intervalın dúziń.
 \\
\textbf{B2.} 
Eger \(X^{(n)} = \left( X_{1},...,X_{n} \right)\) tańlanba \(\theta\) parametrli kórsetkishli bólistiriliwden alınǵan bolsa, onda belgisiz \(\theta\) parametr ushın momentler usılı bahasın tabıń.
 \\
\textbf{B3.} 
Eger \(X^{(n)} = \left( X_{1},...,X_{n} \right)\) tańlanba \(\left( a,\theta^{2} \right)\) parametrli normal bólistiriliwden alınǵan bolsa (\(\alpha -\)belgili), onda belgisiz \(\theta^{2}\) parametrdiń shınlıqqa maksimal uqsaslıq bahasın tabıń.
 \\
\textbf{C1.} 
Eger \(X^{(n)} = \left( X_{1},...,X_{n} \right)\) tańlanba tıǵızlıq funkciyası
$f(x;\theta) = \left\{ \begin{array}{r}
\alpha^{- 1}e^{- \ \frac{x - \theta}{\alpha}},\ \ x \geq \theta, \\
0,\ \ \ \ \ \ \ x < \theta
\end{array} \right.\ $
bolǵan bólistiriliwden alınǵan bolsa (\(\alpha -\)belgili), onda belgisiz \(\theta\) parametr ushın \(X_{(1)}\) bahasın jıljımaǵanlıq hám tiykarlılıqqa tekseriń.
 \\
\textbf{C2.} 
Eger \(X^{(n)} = \left( X_{1},...,X_{n} \right)\) tańlanba\({\ \ (a,\theta}^{2})\) parametrli normal bólistiriliwden alınǵan bolsa (\(\alpha -\)belgili), onda belgisiz\({\ \ \theta}^{2}\) parametr ushın momentler usılı bahasın tabıń.
 \\
\textbf{C3.} 
\(\ f(x,\theta) = \frac{e^{x}}{\sqrt{2\pi}}\exp\left\{ - \frac{\left( e^{x} - \theta \right)^{2}}{2} \right\}\) model ushın \(\theta\) parametri haqıyqatqa maksimal uqsaslıq usılı bahası tabılsın.
 \\

\end{tabular}
\vspace{1cm}



\end{document}
