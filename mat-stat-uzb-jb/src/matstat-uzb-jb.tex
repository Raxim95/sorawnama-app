\documentclass{article}
\usepackage[fontsize=14pt]{fontsize}
\usepackage[utf8]{inputenc}

\usepackage{array}
\usepackage[a4paper,
left=7mm,
right=5mm,
top=7mm,]{geometry}
\usepackage{amsmath}
\usepackage{amssymb}
\usepackage{amsfonts}
\usepackage{setspace}



\renewcommand{\baselinestretch}{1} 

\everymath{\displaystyle}
\everydisplay{\displaystyle}
% \linespread{1.25}

\DeclareMathOperator{\sign}{sign}


\begin{document}

\onehalfspacing
\pagenumbering{gobble}


\begin{tabular}{m{17cm}}
\textbf{1-variant}
\newline

\textbf{T1.} 
Tanlanma xarakteristikalari.(tanlanma o'rta, tanlanma dispersiya).
\\
\textbf{T2.} 
Kolmogorovning muvofiqlik kritireyesi (Kolmogorov teoremasi)
\\
\textbf{A1.} 
Hajmi \(n = 20\) ga teng bo'lgan tanlanma berilgan: 4,3; 4,9; 13,4; 13,4; 6,5; 4,9; 4,9; 4,3; 5,1; 6,5; 6,5; 7,0; 4,3; 4,9; 6,5; 6,5; 5,1; 5,1; 4,9; 13,4. Bu tanlanmaning statistik taqsimotin toping.
\\
\textbf{A2.} 
Hajmi \(n = 20\) ga teng bo'lgan tanlanma berilgan: 4,2; 4,9; 13,8; 13,8; 6,6; 4,9; 4,9; 4,2; 5,3; 6,6; 6,6; 7,5; 4,2; 4,9; 6,6; 6,6; 5,3; 5,3; 4,9; 13,8. Bu tanlanmaning empirik taqsimot funksiyasin toping.
\\
\textbf{A3.} 
Oliy matematika fanidan 10 ta talaba test topshiriqlarin topshirdi. Harbir talaba 10 balgacha to'plashi mumkin. Agar test topshiriqlari natijalari bo'yicha \{9, 10, 6, 7, 4, 8, 10, 7, 9, 10\} tanlanma olingan bo'lsa, ushbu tanlanmalarning tanlanma o'rta va tanlanma dispersiyalarin toping.
\\
\textbf{B1.} 
Agar o'rta kvadratik chetlanish \(\sigma = 2\) bo'lgan normal taqsimot bosh to'plamdan olingan hajmi \(n = 10\)ga teng tanlanma bo'yicha \(\overline{x} = 5,4\) tanlanma o'rta qiymati topilgan bo'lsa, u holda \(\gamma = 0,95\) ishonchlilik bilan noma'lum \(\theta\) matematik kutilma uchun ishonchlilik intervalin tuzing .
\\
\textbf{B2.} 
Puasson taqsimoti noma'lum \(\theta > 0\) parametri momentlar usuli bahosini toping.
\\
\textbf{B3.} 
Agar (0,1,2,0) tanlanma quyida berilgan taqsimotdan olingan bo'lsa, u holda noma'lum \(\theta\) parametrning haqiqatga maksimal o'xshashlik bahosini toping.
$\begin{array}{|c|c|c|c|}
    \hline
    \xi & 0 & 1 & 2 \\
    \hline
    P_{\theta} & \theta & 2\theta & 1 - 3\theta \\
    \hline
\end{array}$
\\
\textbf{C1.} 
Agar \(X^{(n)} = \left( X_{1},...,X_{n} \right)\) tanlanma \(\lbrack 0,\theta\rbrack\) oraliqda tekis taqsimotdan olingan bo'lsa, u holda noma'lum \(\theta\) parametr uchun \((n + 1)X_{(1)})\) bahoning siljimaganligi va asosliligini tekshiring.
\\
\textbf{C2.} 
Agar \(X^{(n)} = \left( X_{1},...,X_{n} \right)\) tanlanma \(1\sqrt{\theta}\) parametrli ko'rsatkichli taqsimotdan olingan bo'lsa, u holda noma'lum \(\theta\) parametr uchun momentlar usuli bahosini toping.
\\
\textbf{C3.} 
Agar \(X^{(n)} = \left( X_{1},...,X_{n} \right)\) tanlanma \(\left\lbrack \theta_{1},\theta_{2} \right\rbrack\) oraliqda tekis taqsimotdan olingan bo'lsa, u holda noma'lum \(\left( \theta_{1},\theta_{2} \right)\) vektor parametrning haqiqatga maksimal o'xshashlik bahosini toping.
\\

\end{tabular}
\vspace{1cm}


\begin{tabular}{m{17cm}}
\textbf{2-variant}
\newline

\textbf{T1.} 
Neyman-Pirson teoremasi.
\\
\textbf{T2.} 
Statistik baho xossalari. (Siljimagan, asosliy, effektiv)
\\
\textbf{A1.} 
Hajmi \(n = 20\) ga teng bo'lgan tanlanma berilgan: -2,1; 1,7; 3,3; 3,3; 11,7; 4,7; 1,7; 4,7; -2,1; 4,7; 4,7; 4,7; 8,0; -2,1; 1,7; 4,7; 8,0; 11,7; 1,7; 8,0. Bu tanlanmaning statistik taqsimotin toping.
\\
\textbf{A2.} 
Hajmi \(n = 20\) ga teng bo'lgan tanlanma berilgan: -2,2; 1,3; 3,8; 3,8; 11,5; 4,1; 1,3; 4,1; -2,2; 4,1; 4,1; 4,1; 8,4; -2,2; 1,3; 4,1; 8,4; 11,5; 1,3; 8,4. Bu tanlanmaning empirik taqsimot funksiyasin toping.
\\
\textbf{A3.} 
Oliy matematika fanidan 10 ta talaba test topshiriqlarin topshirdi. Harbir talaba 10 balgacha to'plashi mumkin. Agar test topshiriqlari natijalari bo'yicha \{4, 1, 2, 4, 6, 4, 5, 3, 6, 5\} tanlanma olingan bo'lsa, ushbu tanlanmalarning tanlanma o'rta va tanlanma dispersiyalarin toping.
\\
\textbf{B1.} 
Agar normal taqsimlangan bosh to'plamdan olingan hajmi \(n = 16\)ga teng tanlanma bo'yicha \(\overline{x} = 20,2\) tanlanma o'rta va \({\overline{S}}^{2} = 0,64\) tuzatilgan tanlanma dispersiyalar topilgan bo'lsa, u holda \(\gamma = 0,95\) ishonchlilik bilan noma'lum \(\theta\) matematik kutilma uchun ishonchlilik intervalin tuzing.
\\
\textbf{B2.} 
Agar zichlik funksiyasi \(f(x) = \frac{2x}{\theta}e^{- \frac{x^{2}}{\theta}},\ \ x \geq 0\) ko'rinishga ega bo'lsa, u holda \(\theta\) parametr momentlar usuli bahosini toping.
\\
\textbf{B3.} 
\(f(x) = \frac{\theta}{2}e^{- \theta|x|}\) model uchun \(\theta\) parametri haqiqatga maksimal o'xshashlik usuli bahosi topilsin.
\\
\textbf{C1.} 
Agar \(X^{(n)} = \left( X_{1},...,X_{n} \right)\) tanlanma \(\lbrack 0,\theta\rbrack\) oraliqda tekis taqsimotdan olingan bo'lsa, u holda noma'lum \(\theta\) parametr uchun \(\frac{n + 1}{n}X_{(n)}\) bahoning siljimaganligi va asosliligini tekshiring.
\\
\textbf{C2.} 
Agar \(X^{(n)} = \left( X_{1},...,X_{n} \right)\) tanlanma \(\theta\) parametrli geometrik taqsimotdan olingan bo'lsa, u holda noma'lum \(\theta\) parametr uchun momentlar usuli bahosini toping.
\\
\textbf{C3.} 
Agar \(X^{(n)} = \left( X_{1},...,X_{n} \right)\) tanlanma \(\theta\) parametrli geometrik taqsimotdan olingan bo'lsa, u holda noma'lum \(\theta\) parametrning haqiqatga maksimal o'xshashlik usuli bahosini toping.
\\

\end{tabular}
\vspace{1cm}


\begin{tabular}{m{17cm}}
\textbf{3-variant}
\newline

\textbf{T1.} 
Glivenko-Kantelli teoremasi. (empirik taqsimot funktsiyasi, ehtimollik bilan yaqinlashish).
\\
\textbf{T2.} 
Haqiqatga maksimal o'xshashlik usuli. (haqiqatga maksimal o'xshashlik funktsiyasi, noma'lum parametrlarni baholash).
\\
\textbf{A1.} 
Hajmi \(n = 20\) ga teng bo'lgan tanlanma berilgan: -11,0; -4,1; 0; 2,3; 1,2; 0; 1,2; 2,3; 2,3; 1,2; 2,3; -11,0; 3,4; 1,2; 3,4; 3,4; 0; 3,4; 2,3; 0. Bu tanlanmaning statistik taqsimotin toping.
\\
\textbf{A2.} 
Hajmi \(n = 20\) ga teng bo'lgan tanlanma berilgan: -11,2; -4,5; 0; 2,9; 1,7; 0; 1,7; 2,9; 2,9; 1,7; 2,9; -11,2; 3,1; 1,7; 3,1; 3,1; 0; 3,1; 2,9; 0. Bu tanlanmaning empirik taqsimot funksiyasin toping.
\\
\textbf{A3.} 
Oliy matematika fanidan 10 ta talaba test topshiriqlarin topshirdi. Harbir talaba 10 balgacha to'plashi mumkin. Agar test topshiriqlari natijalari bo'yicha \{8, 9, 10, 4, 9, 7, 6, 7, 6, 4\} tanlanma olingan bo'lsa, ushbu tanlanmalarning tanlanma o'rta va tanlanma dispersiyalarin toping.
\\
\textbf{B1.} 
Agar normal taqsimlangan bosh to'plamdan olingan hajmi \(n = 11\) ga teng bo'lgan tanlanma bo'yicha \({\overline{S}}^{2} = 0,5\) tuzatilgan tanlanma dispersiya topilgan bo'lsa, u holda \(\gamma = 0,90\) ishonchlilik bilan noma'lum \(\theta_{2}^{2}\) dispersiya uchun ishonchlilik intervalin tuzing.
\\
\textbf{B2.} 
\(\left\lbrack \theta_{1},\theta_{2} \right\rbrack\) oraliqda tekis taqsimot parametrlari uchun momentlar usuli baholarini toping.
\\
\textbf{B3.} 
Agar \(X^{(n)} = \left( X_{1},...,X_{n} \right)\) tanlanma \(\theta\) parametrli Bernulli taqsimotidan olingan bo'lsa, u holda noma'lum \(\theta\) parametrning haqiqatga maksimal o'xshashlik usuli bahosini toping.
\\
\textbf{C1.} 
Agar \(X^{(n)} = \left( X_{1},...,X_{n} \right)\) tanlanma \(M\xi = a\) ma'lum va \(M\xi^{2}\) chekli bo'lgan taqsimotdan olingan bo'lsa, u holda noma'lum \(D\xi\) dispersiya uchun \({\overline{S}}^{2}\) bahoning siljimaganligi va asosliligini tekshiring.
\\
\textbf{C2.} 
Agar \(X^{(n)} = \left( X_{1},...,X_{n} \right)\) tanlanma \(\theta\) parametrli Puasson taqsimotidan olingan bo'lsa, u holda noma'lum \(\theta\) parametr uchun momentlar usuli bahosini toping.
\\
\textbf{C3.} 
Agar \(X^{(n)} = \left( X_{1},...,X_{n} \right)\) tanlanma \((\theta,2\theta)\) parametrli normal taqsimotdan olingan bo'lsa, u holda noma'lum \(\theta > 0\) parametrning haqiqatga maksimal o'xshashlik bahosini toping.
\\

\end{tabular}
\vspace{1cm}


\begin{tabular}{m{17cm}}
\textbf{4-variant}
\newline

\textbf{T1.} 
Guruhlangan va interval variatsion qatorlar.
\\
\textbf{T2.} 
Normal qonun dispersiyasi uchun ishonchlilik intervalin tuzish. (Ishonchlilik ehtimolligi, interval)
\\
\textbf{A1.} 
Hajmi \(n = 20\) ga teng bo'lgan tanlanma berilgan: 2,5; 3,8; 4,3; 2,5; 3,8; 2,5; 3,1; 4,3; 4,3; 5,5; 6,2; 2,5; 3,1; 6,2; 5,5; 6,2; 3,1; 3,1; 6,2; 3,1. Bu tanlanmaning statistik taqsimotin toping.
\\
\textbf{A2.} 
Hajmi \(n = 20\) ga teng bo'lgan tanlanma berilgan: 2,7; 4,2; 4,8; 2,7; 4,2; 2,7; 3,9; 4,8; 4,8; 5,9; 6,5; 2,7; 3,9; 6,5; 5,9; 6,5; 3,9; 3,9; 6,5; 3,9. Bu tanlanmaning empirik taqsimot funksiyasin toping.
\\
\textbf{A3.} 
Oliy matematika fanidan 10 ta talaba test topshiriqlarin topshirdi. Harbir talaba 10 balgacha to'plashi mumkin. Agar test topshiriqlari natijalari bo'yicha \{7, 8, 7, 6, 4, 8, 4, 7, 9, 10\} tanlanma olingan bo'lsa, ushbu tanlanmalarning tanlanma o'rta va tanlanma dispersiyalarin toping.
\\
\textbf{B1.} 
Agar o'rta kvadratik chetlanish \(\sigma = 3\) bo'lgan normal taqsimot bosh to'plamdan olingan hajmi \(n = 9\)ga teng tanlanma bo'yicha \(\overline{x} = 4,5\) tanlanma o'rta qiymati topilgan bo'lsa, u holda \(\gamma = 0,95\) ishonchlilik bilan noma'lum \(\theta\) matematik kutilma uchun ishonchlilik intervalin tuzing .
\\
\textbf{B2.} 
Agar (3,-2,-2,0,-2,2,-2,0,-2,3,-2,0,3,0,3,-2,0,-2,3,-2,2,-2,-2,3,3,2,-2,2,3,3) tanlanma quyida berilgan taqsimotdan olingan bo'lsa, u holda noma'lum \(\theta\) parametr uchun momentlar usuli bahosini \(g(x) = |x|\) funksiya yordamida toping.
$\begin{array}{|c|c|c|c|}
    \hline
    \xi & -2 & 0 & 3 \\
    \hline
    P_{\theta} & 3\theta & 1 - 5\theta & 2\theta \\
    \hline
\end{array}$
\\
\textbf{B3.} 
Agar \(X^{(n)} = \left( X_{1},...,X_{n} \right)\) tanlanma \(\lbrack - \theta,\theta\rbrack\) oraliqda tekis taqsimotdan olingan bo'lsa, u holda noma'lum \(\theta > 0\) parametrning haqiqatga maksimal o'xshashlik usuli bahosini toping.
\\
\textbf{C1.} 
Agar \(X^{(n)} = \left( X_{1},...,X_{n} \right)\) tanlanma \(M\xi = a\) ma'lum va \(M\xi^{2}\) chekli bo'lgan taqsimotdan olingan bo'lsa, u holda noma'lum \(D\xi\) dispersiya uchun \(\frac{1}{n}\sum_{i = 1}^{n}{X_{i}a}\) bahoning siljimaganligi va asosliligini tekshiring.
\\
\textbf{C2.} 
Agar \(X^{(n)} = \left( X_{1},...,X_{n} \right)\) tanlanma \({\lbrack\theta}_{1},\theta_{2}\rbrack\) oraliqda tekis taqsimotdan olingan bo'lsa, u holda noma'lum \(\left( \theta_{1},\theta_{2} \right)\) vektor parametr uchun momentlar usuli bahosini toping.
\\
\textbf{C3.} 
\(f(x,\theta) = \frac{e^{x}}{\sqrt{2\pi}}\exp\left\{ - \frac{\left( e^{x} - \theta \right)^{2}}{2} \right\}\) model uchun \(\theta\) parametri haqiqatga maksimal o'xshashlik usuli bahosi topilsin.
\\

\end{tabular}
\vspace{1cm}


\begin{tabular}{m{17cm}}
\textbf{5-variant}
\newline

\textbf{T1.} 
Empirik taqsimot funktsiyasi. (Tanlanma, eksperiment)
\\
\textbf{T2.} 
Ishonchlilik intervallarin tuzish. Aniq ishonchlilik intervallar.
\\
\textbf{A1.} 
Hajmi \(n = 20\) ga teng bo'lgan tanlanma berilgan: -4,3; 2,6; 0; -2,5; 2,6; 1,9; 2,2; 0; -4,3; -2,5; 1,9; -2,5; 1,9; 2,2; 2,6; 1,9; 2,6; 2,2; 2,2; 1,9. Bu tanlanmaning statistik taqsimotin toping.
\\
\textbf{A2.} 
Hajmi \(n = 20\) ga teng bo'lgan tanlanma berilgan: -4,9; 2,6; 0,5; -2,6; 2,6; 1,7; 2,3; 0,5; -4,9; -2,6; 1,7; -2,6; 1,7; 2,3; 2,6; 1,7; 2,6; 2,3; 2,3; 1,7. Bu tanlanmaning empirik taqsimot funksiyasin toping.
\\
\textbf{A3.} 
Oliy matematika fanidan 10 ta talaba test topshiriqlarin topshirdi. Harbir talaba 10 balgacha to'plashi mumkin. Agar test topshiriqlari natijalari bo'yicha \{9, 5, 6, 8, 4, 7, 4, 6, 9, 7\} tanlanma olingan bo'lsa, ushbu tanlanmalarning tanlanma o'rta va tanlanma dispersiyalarin toping.
\\
\textbf{B1.} 
Agar normal taqsimlangan bosh to'plamdan olingan hajmi \(n = 25\)ga teng tanlanma bo'yicha \(\overline{x} = 18,6\) tanlanma o'rta va \({\overline{S}}^{2} = 0,49\) tuzatilgan tanlanma dispersiyalar topilgan bo'lsa, u holda \(\gamma = 0,95\) ishonchlilik bilan noma'lum \(\theta\) matematik kutilma uchun ishonchlilik intervalin tuzing.
\\
\textbf{B2.} 
Agar (0,-2,0,-2,3,-2,0,0,3,0,0,0,0,3,-2,0,0,-2,3,0,3) tanlanma quyida berilgan taqsimotdan olingan bo'lsa, u holda noma'lum \(\theta\) parametr uchun momentlar usuli bahosini toping.
$\begin{array}{|c|c|c|c|}
    \hline
    \xi & - 2 & 0 & 3 \\
    \hline
    P_{\theta} & \theta & 1 - 2\theta & \theta \\
    \hline
\end{array}$
\\
\textbf{B3.} 
Agar \(X^{(n)} = \left( X_{1},...,X_{n} \right)\) tanlanma \(\left( a,\theta^{2} \right)\) parametrli normal taqsimotdan olingan bo'lsa (\(\alpha -\) ma'lum), u holda noma'lum \(\theta^{2}\) parametrning haqiqatga maksimal o'xshashlik bahosini toping.
\\
\textbf{C1.} 
Agar \(X^{(n)} = \left( X_{1},...,X_{n} \right)\) tanlanma \(M\xi = a\) ma'lum va \(M\xi^{2}\) chekli bo'lgan taqsimotdan olingan bo'lsa, u holda noma'lum \(D\xi\) dispersiya uchun \(\overline{x^{2}} - a^{2}\) bahoning siljimaganligi va asosliligini tekshiring.
\\
\textbf{C2.} 
Agar \(X^{(n)} = \left( X_{1},...,X_{n} \right)\) tanlanma zichlik funksiyasi\(f(x,\theta) = \frac{2x}{\theta^{2}},x \in \lbrack 0,\theta\rbrack\)bo'lgan taqsimotdan olingan bo'lsa, u holda noma'lum \(\theta\) parametr uchun momentlar usuli bahosini toping.
\\
\textbf{C3.} 
Agar \(X^{(n)} = \left( X_{1},...,X_{n} \right)\) tanlanma zichlik funksiyasi\(f(x;\theta) = \frac{\theta}{2}e^{- \theta|x|},\ \ x \in R\) bo'lgan taqsimotdan olingan bo'lsa, u holda noma'lum \(\theta > 0\) parametrning haqiqatga maksimal o'xshashlik bahosini toping.
\\

\end{tabular}
\vspace{1cm}


\begin{tabular}{m{17cm}}
\textbf{6-variant}
\newline

\textbf{T1.} 
Momentler usuli. (tanlanma momentleri, noma'lum parametrlarni baholash).
\\
\textbf{T2.} 
Chiziqli korrelyatsiya tenglamasi (ta'rifi, regressiya to'g'ri chiziqning tanlanma tenglamalari)
\\
\textbf{A1.} 
Hajmi \(n = 20\) ga teng bo'lgan tanlanma berilgan: -2,9; -3,8; 2,3; 1,8; 1,8; 0,7; -3,8; -1,5; 2,3; 0,7; -2,9; -1,5; 1,8; -2,9; -1,5; -3,8; 1,8; 1,8; -3,8; 1,8. Bu tanlanmaning statistik taqsimotin toping.
\\
\textbf{A2.} 
Hajmi \(n = 20\) ga teng bo'lgan tanlanma berilgan: -2,4; -3,5; 2,8; 1,4; 1,4; 0,1; -3,5; -1,9; 2,8; 0,1; -2,4; -1,9; 1,4; -2,4; -1,9; -3,5; 1,4; 1,4; -3,5; 1,4. Bu tanlanmaning empirik taqsimot funksiyasin toping.
\\
\textbf{A3.} 
Oliy matematika fanidan 10 ta talaba test topshiriqlarin topshirdi. Harbir talaba 10 balgacha to'plashi mumkin. Agar test topshiriqlari natijalari bo'yicha \{8, 9, 7, 10, 6, 8, 10, 3, 10, 9\} tanlanma olingan bo'lsa, ushbu tanlanmalarning tanlanma o'rta va tanlanma dispersiyalarin toping.
\\
\textbf{B1.} 
Agar normal taqsimlangan bosh to'plamdan olingan hajmi \(n = 12\) ga teng bo'lgan tanlanma bo'yicha \({\overline{S}}^{2} = 0,4\) tuzatilgan tanlanma dispersiya topilgan bo'lsa, u holda \(\gamma = 0,90\) ishonchlilik bilan noma'lum \(\theta_{2}^{2}\) dispersiya uchun ishonchlilik intervalin tuzing.
\\
\textbf{B2.} 
\(\lbrack 0,\theta\rbrack\) oraliqda tekis taqsimlangan \(\theta\) parametri uchun momentlar usuli bahosini toping.
\\
\textbf{B3.} 
\(f(x) = \frac{2x}{\theta}e^{- \frac{x^{2}}{\theta}},\ \ x \geq 0\) model uchun \(\theta\) parametri haqiqatga maksimal o'xshashlik usuli bahosi topilsin.
\\
\textbf{C1.} 
Agar \(X^{(n)} = \left( X_{1},...,X_{n} \right)\) tanlanma zichlik funksiyasi bo'lsa: \(f(x,\theta) = \left\{ \begin{matrix}
e^{\theta - x},\ \ x \geq \theta, \\
\ \ 0,\ \ x < \theta
\end{matrix} \right.\ \) bo'lgan taqsimotdan olingan bo'lsa, u holda noma'lum \(\theta\) parametr uchun \(X_{(1)}\) bahoning siljimaganligi va asosliligini tekshiring.
\\
\textbf{C2.} 
Agar \(X^{(n)} = \left( X_{1},...,X_{n} \right)\) tanlanma \(1/\theta\) parametrli ko'rsatkichli taqsimotdan olingan bo'lsa, u holda noma'lum \(\theta\) parametr uchun momentlar usuli bahosini \(\ \ g(x) = x^{k},\) \(k \in N\) funksiya yordamida toping.
\\
\textbf{C3.} 
Agar \(X^{(n)} = \left( X_{1},...,X_{n} \right)\) tanlanma zichlik funksiyasi \(f(x;\theta) = \left\{ \begin{array}{r}
\begin{matrix}
\theta_{1}^{- 1}e^{\frac{x - \theta_{2}}{\theta_{1}}},\ \ x \geq \theta_{2}
\end{matrix} \\
0,\ \ \ \ x < \theta_{2}
\end{array} \right.\ \) bo'lgan taqsimotdan olingan bo'lsa, u holda noma'lum \(.\left( \theta_{1},\theta_{2} \right),\) \(\theta_{1} > 0,\) \(\theta_{2} \in R\) vektor parametrning haqiqatga maksimal o'xshashlik bahosini toping.
\\

\end{tabular}
\vspace{1cm}


\begin{tabular}{m{17cm}}
\textbf{7-variant}
\newline

\textbf{T1.} Matematik statistikaning asosiy masalalari. (Statistik ma'lumotlar, guruhlash)
\\
\textbf{T2.} 
Pirsonning xi-kvadrat muvofiqlik kriteriysi (Pirson teoremasi).
\\
\textbf{A1.} 
Hajmi \(n = 20\) ga teng bo'lgan tanlanma berilgan: 3,6; 2,9; 3,6; 3,2; 1,1; 0,3; 1,1; 3,6; 1,7; 1,1; 0,3; 1,7; 1,1; 0,3; 2,9; 2,9; 2,9; 1,1; 2,9; 1,7. Bu tanlanmaning statistik taqsimotin toping.
\\
\textbf{A2.} 
Hajmi \(n = 20\) ga teng bo'lgan tanlanma berilgan: 4,6; 2,5; 4,6; 3,3; 1,8; 0,3; 1,8; 4,6; 2,1; 1,8; 0,3; 2,1; 1,8; 0,3; 2,5; 2,5; 2,5; 1,8; 2,5; 2,1. Bu tanlanmaning empirik taqsimot funksiyasin toping.
\\
\textbf{A3.} 
Oliy matematika fanidan 10 ta talaba test topshiriqlarin topshirdi. Harbir talaba 10 balgacha to'plashi mumkin. Agar test topshiriqlari natijalari bo'yicha \{5, 7, 5, 9, 5, 8, 10, 6, 7, 8\} tanlanma olingan bo'lsa, ushbu tanlanmalarning tanlanma o'rta va tanlanma dispersiyalarin toping.
\\
\textbf{B1.} 
Agar o'rta kvadratik chetlanish \(\sigma = 1\) bo'lgan normal taqsimot bosh to'plamdan olingan hajmi \(n = 15\)ga teng tanlanma bo'yicha \(\overline{x} = 5,8\) tanlanma o'rta qiymati topilgan bo'lsa, u holda \(\gamma = 0,90\) ishonchlilik bilan noma'lum \(\theta\) matematik kutilma uchun ishonchlilik intervalin tuzing .
\\
\textbf{B2.} 
Agar (3,0,-2,0,-2,3,-2,0,0,3,0,0,0,0,3,-2,0,0,-2,3,0) tanlanma quyida berilgan taqsimotdan olingan bo'lsa, u holda noma'lum \(\left( \theta_{1},\theta_{2} \right)\) vektor parametr uchun momentlar usuli bahosini toping.
$\begin{array}{|c|c|c|c|}
    \hline
    \xi & - 2 & 0 & 3 \\
    \hline
    P_{\theta} & 2\theta_{1} & 0,5 + \theta_{1} + \theta_{2} & \theta_{2} \\
    \hline
\end{array}$
\\
\textbf{B3.} 
Agar (-1,-1,0,-1,0,-1,-1,5,-1,0,-1,0,5,-1,-1,-1,5,-1,-1,-1,1,-1,5,0,-1,-1,5) tanlanma quyida berilgan taqsimotdan olingan bo'lsa, u holda noma'lum \(\theta\) parametrning haqiqatga maksimal o'xshashlik usuli bahosini toping.
$\begin{array}{|c|c|c|c|}
    \hline
    \xi & - 1 & 0 & 5\\
    \hline
    P_{\theta} & 1 - \theta & \theta/2 & \theta/2 \\
    \hline
\end{array}$
\\
\textbf{C1.} 
Agar \(X^{(n)} = \left( X_{1},...,X_{n} \right)\) tanlanma zichlik funksiyasi bo'lsa: \(f(x,\theta) = \left\{ \begin{matrix}
e^{\theta - x},\ \ x \geq \theta, \\
\ \ 0,\ \ x < \theta
\end{matrix} \right.\ \) bo'lgan taqsimotdan olingan bo'lsa, u holda noma'lum \(\theta\) parametr uchun \(\overline{x} - 1\) bahoning siljimaganligi va asosliligini tekshiring.
\\
\textbf{C2.} 
Agar \(X^{(n)} = \left( X_{1},...,X_{n} \right)\) tanlanma \({\lbrack\theta}_{1},\theta_{1} + \theta_{2}\rbrack\) oraliqda tekis taqsimotdan olingan bo'lsa, u holda noma'lum \(\left( \theta_{1},\theta_{2} \right)\) vektor parametr uchun momentlar usuli bahosini toping.
\\
\textbf{C3.} 
Agar \(X^{(n)} = \left( X_{1},...,X_{n} \right)\) tanlanma zichlik funksiyasi\(f(x;\theta) = \left\{ \begin{matrix}
e^{\theta - x},\ \ x \geq \theta, \\
\ \ 0,\ \ x < \theta
\end{matrix} \right.\ \) bo'lgan taqsimotdan olingan bo'lsa, u holda noma'lum \(\theta\) parametrning haqiqatga maksimal o'xshashlik bahosini toping.
\\

\end{tabular}
\vspace{1cm}


\begin{tabular}{m{17cm}}
\textbf{8-variant}
\newline

\textbf{T1.} 
Tanlanma xarakteristikalar. (Variatsion qator, nisbiy chastota).
\\
\textbf{T2.} 
Statistik gipotezalarni tekshirish (kritik to'plam, 1 va 2-tur xatolik)
\\
\textbf{A1.} 
Hajmi \(n = 20\) ga teng bo'lgan tanlanma berilgan: -1,3; 0; 0,8; 2,3; 1,1; 0,8; 0,8; 2,3; 1,1; 0,8; -1,3; 1,8; 1,1; -1,3; 1,1; 1,8; 1,8; 1,1; 1,8; 1,8. Bu tanlanmaning statistik taqsimotin toping.
\\
\textbf{A2.} 
Hajmi \(n = 20\) ga teng bo'lgan tanlanma berilgan: -1,9; 0,7; 0,9; 2,8; 1,3; 0,9; 0,9; 2,8; 1,3; 0,9; -1,9; 1,6; 1,3; -1,9; 1,3; 1,6; 1,6; 1,3; 1,6; 1,6. Bu tanlanmaning empirik taqsimot funksiyasin toping.
\\
\textbf{A3.} 
Oliy matematika fanidan 10 ta talaba test topshiriqlarin topshirdi. Harbir talaba 10 balgacha to'plashi mumkin. Agar test topshiriqlari natijalari bo'yicha \{8, 4, 3, 7, 3, 6, 5, 3, 5, 6\} tanlanma olingan bo'lsa, ushbu tanlanmalarning tanlanma o'rta va tanlanma dispersiyalarin toping.
\\
\textbf{B1.} 
Agar normal taqsimlangan bosh to'plamdan olingan hajmi \(n = 20\)ga teng tanlanma bo'yicha \(\overline{x} = 16,6\) tanlanma o'rta va \({\overline{S}}^{2} = 0,64\) tuzatilgan tanlanma dispersiyalar topilgan bo'lsa, u holda \(\gamma = 0,95\) ishonchlilik bilan noma'lum \(\theta\) matematik kutilma uchun ishonchlilik intervalin tuzing.
\\
\textbf{B2.} 
Agar (-2,0,-2,0,-2,3,-2,0,0,3,0,0,0,0,3,-2,0,0,-2,3,0) tanlanma quyida berilgan taqsimotdan olingan bo'lsa, u holda noma'lum \(\left( \theta_{1},\theta_{2} \right)\) vektor parametr uchun momentlar usuli bahosini toping.
$\begin{array}{|c|c|c|c|}
    \hline
    \xi & - 2 & 0 & 3\\
    \hline
    P_{\theta} & \theta_{1} & 1 - \theta_{1} - \theta_{2} & \theta_{2} \\
    \hline
\end{array}$
\\
\textbf{B3.} 
Agar (4,8,5,3) tanlanma \((a,\theta^{2}\) parametrli normal taqsimotdan olingan bo'lsa, u holda noma'lum \(\theta^{2}\) parametrning haqiqatga maksimal o'xshashlik bahosini toping.
\\
\textbf{C1.} 
Agar \(X^{(n)} = \left( X_{1},...,X_{n} \right)\) tanlanma \(\lbrack - 3\theta,\theta\rbrack\) oraliqda tekis taqsimotdan olingan bo'lsa, u holda noma'lum \(\theta\) parametr uchun \(4X_{(n)} + X_{(1)}\) bahoni siljimaganligi va asosliligini tekshiring.
\\
\textbf{C2.} 
Agar \(X^{(n)} = \left( X_{1},...,X_{n} \right)\) tanlanma \((\theta,\theta^{2})\ \ \) parametrli normal taqsimotdan olingan bo'lsa, u holda noma'lum \(\theta > 0\) parametr uchun momentlar usuli bahosini toping.
\\
\textbf{C3.} 
Agar \(X^{(n)} = \left( X_{1},...,X_{n} \right)\) tanlanma zichlik funksiyasi \(f(x;\theta) = \frac{\theta}{\sqrt{2\pi x^{3}}}e^{- \theta^{2}/2x},\ \ x \geq 0\) bo'lgan taqsimotdan olingan bo'lsa, u holda noma'lum \(\theta > 0\) parametrning haqiqatga maksimal o'xshashlik bahosini toping.
\\

\end{tabular}
\vspace{1cm}


\begin{tabular}{m{17cm}}
\textbf{9-variant}
\newline

\textbf{T1.} 
Poligon va gistogramma(nisbiy chastota, interval qator, grafik).
\\
\textbf{T2.} 
Momentler usuli. (tanlanma momentleri, noma'lum parametrlarni baholash).
\\
\textbf{A1.} 
Hajmi \(n = 20\) ga teng bo'lgan tanlanma berilgan: -2,4; 5,6; 5,6; -5,2; -6,7; 5,1; -5,2; -2,4; 4,3; 5,1; -6,7; 4,3; -2,4; -6,7; 4,3; 5,1; 4,3; 5,6; -6,7; 5,6. Bu tanlanmaning statistik taqsimotin toping.
\\
\textbf{A2.} 
Hajmi \(n = 20\) ga teng bo'lgan tanlanma berilgan: -2,9; 7,6; 7,6; -5,7; -6,1; 5,5; -5,7; -2,9; 4,2; 5,5; -6,1; 4,2; -2,9; -6,1; 4,2; 5,5; 4,2; 7,6; -6,1; 7,6. Bu tanlanmaning empirik taqsimot funksiyasin toping.
\\
\textbf{A3.} 
Oliy matematika fanidan 10 ta talaba test topshiriqlarin topshirdi. Harbir talaba 10 balgacha to'plashi mumkin. Agar test topshiriqlari natijalari bo'yicha \{9, 8, 6, 7, 5, 8, 5, 7, 4, 6\} tanlanma olingan bo'lsa, ushbu tanlanmalarning tanlanma o'rta va tanlanma dispersiyalarin toping.
\\
\textbf{B1.} 
Agar normal taqsimlangan bosh to'plamdan olingan hajmi \(n = 13\) ga teng bo'lgan tanlanma bo'yicha \({\overline{S}}^{2} = 1,2\) tuzatilgan tanlanma dispersiya topilgan bo'lsa, u holda \(\gamma = 0,90\) ishonchlilik bilan noma'lum \(\theta_{2}^{2}\) dispersiya uchun ishonchlilik intervalin tuzing.
\\
\textbf{B2.} 
Agar \(X^{(n)} = \left( X_{1},...,X_{n} \right)\) tanlanma \(\theta\) parametrli Bernulli taqsimotidan olingan bo'lsa, u holda noma'lum \(\theta\) parametr uchun momentlar usuli bahosini toping.
\\
\textbf{B3.} 
Agar \(x_{1} = 1,1;\ \ x_{2} = 2,7;\ldots;x_{100} = 1,5\) tanlanma \(\theta\) parametrli ko'rsatkichli taqsimotdan olingan bo'lib, \(\sum_{k = 1}^{100}x_{k} = 200\) bo'lsa, u holda noma'lum \(\theta\) parametrning haqiqatga maksimal o'xshashlik bahosini toping.
\\
\textbf{C1.} 
Agar \(X^{(n)} = \left( X_{1},...,X_{n} \right)\) tanlanma taqsimot funksiyasi \(F(x)\) bo'lgan taqsimotdan olingan bo'lsa, u holda noma'lum \(F(x)\) uchun \(F_{n}(x)\) empirik taqsimot funksiyasining siljimaganligi va asosliligini tekshiring.
\\
\textbf{C2.} 
Agar \(X^{(n)} = \left( X_{1},...,X_{n} \right)\) tanlanma\(\ \ (a,\theta^{2})\) parametrli normal taqsimotdan olingan bo'lsa (\(\alpha -\) ma'lum), u holda noma'lum\(\ \ \theta^{2}\) parametr uchun momentlar usuli bahosini toping.
\\
\textbf{C3.} 
Agar \(X^{(n)} = \left( X_{1},...,X_{n} \right)\) tanlanma zichlik funksiyasi \(f(x;\theta) = \frac{e^{x}}{\sqrt{2\pi}}\exp\left\{ - \frac{\left( e^{x} - \theta \right)^{2}}{2} \right\},\ \ x \in R\) bo'lgan taqsimotdan olingan bo'lsa, u holda noma'lum \(\theta\) parametrning haqiqatga maksimal o'xshashlik bahosini toping.
\\

\end{tabular}
\vspace{1cm}


\begin{tabular}{m{17cm}}
\textbf{10-variant}
\newline

\textbf{T1.} 
Tanlanma momentleri (\(k -\)tartibli boshlang'ich, boshlang'ich absolyut, markaziy va markaziy absolyut momentler).
\\
\textbf{T2.} 
Statistik gipotezalarni tekshirish (kritik to'plam, 1 va 2-tur xatolik).
\\
\textbf{A1.} 
Hajmi \(n = 20\) ga teng bo'lgan tanlanma berilgan:-3,3; 0; 4,4; 2,2; -2,7; 4,4; 2,2; 4,4;-3,3; 2,2; -2,7; 2,2; -3,3; -2,7; 2,2; 3,4; 4,4; 0; -3,3; 0. Bu tanlanmaning statistik taqsimotin toping.
\\
\textbf{A2.} 
Hajmi \(n = 20\) ga teng bo'lgan tanlanma berilgan:-3,3; 0; 4,9; 2,8; -2,6; 4,9; 2,8; 4,9;-3,3; 2,8; -2,6; 2,8; -3,3; -2,6; 2,8; 3,1; 4,9; 0; -3,3; 0. Bu tanlanmaning empirik taqsimot funksiyasin toping.
\\
\textbf{A3.} 
Oliy matematika fanidan 10 ta talaba test topshiriqlarin topshirdi. Harbir talaba 10 balgacha to'plashi mumkin. Agar test topshiriqlari natijalari bo'yicha \{4, 7, 6, 9, 3, 8, 3, 7, 4, 9\} tanlanma olingan bo'lsa, ushbu tanlanmalarning tanlanma o'rta va tanlanma dispersiyalarin toping.
\\
\textbf{B1.} 
Agar o'rta kvadratik chetlanish \(\sigma = 4\) bo'lgan normal taqsimot bosh to'plamdan olingan hajmi \(n = 12\)ga teng tanlanma bo'yicha \(\overline{x} = 3\) tanlanma o'rta qiymati topilgan bo'lsa, u holda \(\gamma = 0,95\) ishonchlilik bilan noma'lum \(\theta\) matematik kutilma uchun ishonchlilik intervalin tuzing .
\\
\textbf{B2.} 
Ko'rsatkichli taqsimot noma'lum \(\theta > 0\) parametri momentlar usuli bahosini toping.
\\
\textbf{B3.} 
Agar \(X^{(n)} = \left( X_{1},...,X_{n} \right)\) tanlanma \(\left\lbrack - \theta,\theta^{2} \right\rbrack\) oraliqda tekis taqsimotdan olingan bo'lsa, u holda noma'lum \(\theta > 0\) parametrning haqiqatga maksimal o'xshashlik usuli bahosini toping.
\\
\textbf{C1.} 
Agar \(X^{(n)} = \left( X_{1},...,X_{n} \right)\) tanlanma \(\left( a,\theta^{2} \right)\) parametrli normal taqsimotdan olingan bo'lsa (\(a -\) ma'lum), u holda noma'lum \(\theta\) parametr uchun \(\sqrt{\frac{\pi}{2}}\left| \overline{x - a} \right|\) bahoning siljimaganligi va asosliligini tekshiring.
\\
\textbf{C2.} 
Agar \(X^{(n)} = \left( X_{1},...,X_{n} \right)\) tanlanma \((\theta,2\theta)\) parametrli normal taqsimotdan olingan bo'lsa, u holda noma'lum \(\theta > 0\) parametr uchun momentlar usuli bahosini toping.
\\
\textbf{C3.} 
Agar \(X^{(n)} = \left( X_{1},...,X_{n} \right)\) tanlanma zichlik funksiyasi \(f(x;\theta) = \frac{3x^{2}}{\sqrt{2\pi}}\exp\left\{ - \frac{\left( x^{3} - \theta \right)^{2}}{2} \right\},\ \ x \in R\) bo'lgan taqsimotdan olingan bo'lsa, u holda noma'lum \(\theta\) parametrning haqiqatga maksimal o'xshashlik bahosini toping.
\\

\end{tabular}
\vspace{1cm}


\begin{tabular}{m{17cm}}
\textbf{11-variant}
\newline

\textbf{T1.} 
Momentler usuli. (tanlanma momentleri, noma'lum parametrlarni baholash).
\\
\textbf{T2.} 
Momentler usuli. (tanlanma momentleri, noma'lum parametrlarni baholash).
\\
\textbf{A1.} 
Hajmi \(n = 20\) ga teng bo'lgan tanlanma berilgan: 3,7; 3,1; 4,8; 2,8; 3,1; 4,3; 3,7; 4,3; 2,4; 3,1; 2,4; 4,3; 3,1; 3,7; 4,8; 2,8; 2,4; 2,8; 2,4; 3,1. Bu tanlanmaning statistik taqsimotin toping.
\\
\textbf{A2.} 
Hajmi \(n = 20\) ga teng bo'lgan tanlanma berilgan: 3,8; 3,4; 4,8; 2,9; 3,4; 4,6; 3,8; 4,6; 2,1; 3,4; 2,1; 4,6; 3,4; 3,8; 4,8; 2,9; 2,1; 2,9; 2,1; 3,4. Bu tanlanmaning empirik taqsimot funksiyasin toping.
\\
\textbf{A3.} 
Oliy matematika fanidan 10 ta talaba test topshiriqlarin topshirdi. Harbir talaba 10 balgacha to'plashi mumkin. Agar test topshiriqlari natijalari bo'yicha \{6, 5, 6, 9, 5, 7, 10, 5, 9, 8\} tanlanma olingan bo'lsa, ushbu tanlanmalarning tanlanma o'rta va tanlanma dispersiyalarin toping.
\\
\textbf{B1.} 
Agar normal taqsimlangan bosh to'plamdan olingan hajmi \(n = 25\)ga teng tanlanma bo'yicha \(\overline{x} = 9\) tanlanma o'rta va \({\overline{S}}^{2} = 0,64\) tuzatilgan tanlanma dispersiyalar topilgan bo'lsa, u holda \(\gamma = 0,95\) ishonchlilik bilan noma'lum \(\theta\) matematik kutilma uchun ishonchlilik intervalin tuzing.
\\
\textbf{B2.} 
Agar \(X^{(n)} = \left( X_{1},...,X_{n} \right)\) tanlanma \(\theta\) parametrli ko'rsatkichli taqsimotdan olingan bo'lsa, u holda noma'lum \(\theta\) parametr uchun momentlar usuli bahosini toping.
\\
\textbf{B3.} 
Agar \(X^{(n)} = \left( X_{1},...,X_{n} \right)\) tanlanma zichlik funksiyasi \(f(x;\theta) = \frac{2x}{\theta}e^{- \frac{x^{2}}{\theta}},\ \ x \geq 0\) bo'lgan taqsimotdan olingan bo'lsa, u holda noma'lum \(\theta > 0\) parametrning haqiqatga maksimal usuli bahosini toping.
\\
\textbf{C1.} 
Agar \(X^{(n)} = \left( X_{1},...,X_{n} \right)\) tanlanma \(\theta\) parametrli ko'rsatkichli taqsimotdan olingan bo'lsa, u holda noma'lum \(\theta\) parametr uchun \(1/\overline{x}\) bahoning siljimaganligi va asosliligini tekshiring.
\\
\textbf{C2.} 
Agar \(X^{(n)} = \left( X_{1},...,X_{n} \right)\) tanlanma zichlik funksiyasi\(f(x,\theta) = \left\{ \begin{matrix}
\theta_{1}^{- 1}e^{- \ \frac{x - \theta_{2}}{\theta_{1}}},\ \ x \geq \theta_{2}, \\
0,\ \ x < \theta_{2}
\end{matrix} \right.\ \)bo'lgan taqsimotdan olingan bo'lsa, u holda noma'lum \(\left( \theta_{1},\theta_{2} \right)\) \(\theta_{1} > 0,\) \(\theta_{2} \in R\) vektor parametr uchun momentlar usuli bahosini toping.
\\
\textbf{C3.} 
Agar \(X^{(n)} = \left( X_{1},...,X_{n} \right)\) tanlanma zichlik funksiyasi \(f(x;\theta) = \left\{ \begin{array}{r}
3x^{2}\theta^{- 3}{e^{- \left( \frac{x}{\theta} \right)}}^{3},\ \ \ \ x \geq 0 \\
0,\ \ \ \ \ \ \ \ \ \ \ x < 0
\end{array} \right.\ \) bo'lgan taqsimotdan olingan bo'lsa, u holda noma'lum \(\theta > 0\) parametrning haqiqatga maksimal o'xshashlik bahosini toping.
\\

\end{tabular}
\vspace{1cm}


\begin{tabular}{m{17cm}}
\textbf{12-variant}
\newline

\textbf{T1.} 
Tanlanma xarakteristikalar. (Variatsion qator, nisbiy chastota).
\\
\textbf{T2.} 
Statistik gipotezalarni tekshirish (kritik to'plam, 1 va 2-tur xatolik).
\\
\textbf{A1.} 
Hajmi \(n = 20\) ga teng bo'lgan tanlanma berilgan: 1,5; -0,9; -2,4; -0,9; 0,7; 1,5; -0,9; -0,2; -2,4; 0,7; -2,4; 0,7; -0,9; 1,5; -1,7; -0,9; -0,2; 0,7; -1,7; -0,9. Bu tanlanmaning statistik taqsimotin toping.
\\
\textbf{A2.} 
Hajmi \(n = 20\) ga teng bo'lgan tanlanma berilgan: 1,9; -0,3; -2,7; -0,3; 0,6; 1,9; -0,3; -0,1; -2,7; 0,6; -2,7; 0,6; -0,3; 1,9; -1,8; -0,3; -0,1; 0,6; -1,8; -0,3. Bu tanlanmaning empirik taqsimot funksiyasin toping.
\\
\textbf{A3.} 
Oliy matematika fanidan 10 ta talaba test topshiriqlarin topshirdi. Harbir talaba 10 balgacha to'plashi mumkin. Agar test topshiriqlari natijalari bo'yicha \{4, 6, 6, 9, 5, 8, 4, 7, 5, 6\} tanlanma olingan bo'lsa, ushbu tanlanmalarning tanlanma o'rta va tanlanma dispersiyalarin toping.
\\
\textbf{B1.} 
Agar normal taqsimlangan bosh to'plamdan olingan hajmi \(n = 10\) ga teng bo'lgan tanlanma bo'yicha \({\overline{S}}^{2} = 0,6\) tuzatilgan tanlanma dispersiya topilgan bo'lsa, u holda \(\gamma = 0,95\) ishonchlilik bilan noma'lum \(\theta_{2}^{2}\) dispersiya uchun ishonchlilik intervalin tuzing.
\\
\textbf{B2.} 
Agar zichlik funksiyasi \(f(x) = \frac{2x}{\theta}e^{- \frac{x^{2}}{\theta}},\ \ x \geq 0\) ko'rinishga ega bo'lsa, u holda \(\theta\) parametr momentlar usuli bahosini toping.
\\
\textbf{B3.} 
Agar \(X^{(n)} = \left( X_{1},...,X_{n} \right)\) tanlanma \(\theta\) parametrli ko'rsatkichli taqsimotdan olingan bo'lsa, u holda noma'lum \(\theta\) parametrning haqiqatga maksimal o'xshashlik usuli bahosini toping.
\\
\textbf{C1.} 
Agar \(X^{(n)} = \left( X_{1},...,X_{n} \right)\) tanlanma \(1\sqrt{\theta}\) parametrli ko'rsatkichli taqsimotdan olingan bo'lsa, u holda noma'lum \(\theta\) parametr uchun \((\overline{x})^{2}\) bahoning siljimaganligi va asosliligini tekshiring.
\\
\textbf{C2.} 
Agar \(X^{(n)} = \left( X_{1},...,X_{n} \right)\) tanlanma \(\left( \theta_{1},\theta_{2} \right)\) parametrli gamma taqsimotdan olingan bo'lsa, u holda noma'lum \(\left( \theta_{1},\theta_{2} \right)\) vektor parametr uchun momentlar usuli bahosini toping.
\\
\textbf{C3.} 
Agar \(X^{(n)} = \left( X_{1},...,X_{n} \right)\) tanlanma zichlik funksiyasi\(f(x;\theta) = \frac{1}{2}e^{- |x - \theta|},\ \ x \in R\) bo'lgan Laplas taqsimotidan olingan bo'lsa, u holda noma'lum \(\theta \in R\) parametrning haqiqatga maksimal o'xshashlik bahosini toping.
\\

\end{tabular}
\vspace{1cm}


\begin{tabular}{m{17cm}}
\textbf{13-variant}
\newline

\textbf{T1.} 
Tanlanma xarakteristikalari.(tanlanma o'rta, tanlanma dispersiya).
\\
\textbf{T2.} 
Chiziqli korrelyatsiya tenglamasi (ta'rifi, regressiya to'g'ri chiziqning tanlanma tenglamalari)
\\
\textbf{A1.} 
Hajmi \(n = 20\) ga teng bo'lgan tanlanma berilgan:9,4; 6,8; -8,5; 9,4; 2,9; 9,4; -8,5; -6,4; 6,8; -8,5; 9,4; -6,4; 6,8; 9,4; 2,9; 9,4; -3,6; -8,5; 2,9; -6,4. Bu tanlanmaning statistik taqsimotin toping.
\\
\textbf{A2.} 
Hajmi \(n = 20\) ga teng bo'lgan tanlanma berilgan:9,1; 6,4; -8,6; 9,1; 2,3; 9,1; -8,6; -6,2; 6,4; -8,6; 9,1; -6,2; 6,4; 9,1; 2,3; 9,1; -3,9; -8,6; 2,3; -6,2. Bu tanlanmaning empirik taqsimot funksiyasin toping.
\\
\textbf{A3.} 
Oliy matematika fanidan 10 ta talaba test topshiriqlarin topshirdi. Harbir talaba 10 balgacha to'plashi mumkin. Agar test topshiriqlari natijalari bo'yicha \{3, 7, 6, 4, 5, 4, 3, 7, 8, 3\} tanlanma olingan bo'lsa, ushbu tanlanmalarning tanlanma o'rta va tanlanma dispersiyalarin toping.
\\
\textbf{B1.} 
Agar o'rta kvadratik chetlanish \(\sigma = 5\) bo'lgan normal taqsimot bosh to'plamdan olingan hajmi \(n = 16\)ga teng tanlanma bo'yicha \(\overline{x} = 3,6\) tanlanma o'rta qiymati topilgan bo'lsa, u holda \(\gamma = 0,90\) ishonchlilik bilan noma'lum \(\theta\) matematik kutilma uchun ishonchlilik intervalin tuzing .
\\
\textbf{B2.} 
Agar \(X^{(n)} = \left( X_{1},...,X_{n} \right)\) tanlanma \(\theta\) parametrli ko'rsatkichli taqsimotdan olingan bo'lsa, u holda noma'lum \(\theta\) parametr uchun momentlar usuli bahosini toping.
\\
\textbf{B3.} 
Agar (0,1,2,0) tanlanma quyida berilgan taqsimotdan olingan bo'lsa, u holda noma'lum \(\theta\) parametrning haqiqatga maksimal o'xshashlik bahosini toping.
$\begin{array}{|c|c|c|c|}
    \hline
    \xi & 0 & 1 & 2 \\
    \hline
    P_{\theta} & \theta & 2\theta & 1 - 3\theta \\
    \hline
\end{array}$
\\
\textbf{C1.} 
Agar \(X^{(n)} = \left( X_{1},...,X_{n} \right)\) tanlanma \(\sqrt{\theta}\) parametrli Bernulli taqsimotidan olingan bo'lsa, u holda noma'lum \(\theta\) parametr uchun \((\overline{x})^{2}\) bahoning siljimaganligi va asosliligini tekshiring.
\\
\textbf{C2.} 
Agar \(X^{(n)} = \left( X_{1},...,X_{n} \right)\) tanlanma zichlik funksiyasi\(f(x,\theta) = \left\{ \begin{matrix}
e^{\theta - x},\ \ x \geq \theta, \\
0,\ \ x < \theta
\end{matrix} \right.\ \)bo'lgan taqsimotdan olingan bo'lsa, u holda noma'lum \(\theta\) parametr uchun momentlar usuli bahosini toping.
\\
\textbf{C3.} 
Agar \(X^{(n)} = \left( X_{1},...,X_{n} \right)\) tanlanma zichlik funksiyasi\(f(x;\theta) = \frac{4x^{3}}{\sqrt{2\pi}\theta_{2}}\exp\left\{ - \frac{\left( x^{4} - \theta_{1} \right)^{2}}{2{\theta_{2}}^{2}} \right\},\ \ x \in R\) bo'lgan taqsimotdan olingan bo'lsa, u holda noma'lum \(\left( \theta_{1},\theta_{2}^{2} \right)\) vektor parametrning haqiqatga maksimal o'xshashlik usuli baholarini toping.
\\

\end{tabular}
\vspace{1cm}


\begin{tabular}{m{17cm}}
\textbf{14-variant}
\newline

\textbf{T1.} 
Tanlanma momentleri (\(k -\)tartibli boshlang'ich, boshlang'ich absolyut, markaziy va markaziy absolyut momentler).
\\
\textbf{T2.} 
Kolmogorovning muvofiqlik kritireyesi (Kolmogorov teoremasi)
\\
\textbf{A1.} 
Hajmi \(n = 20\) ga teng bo'lgan tanlanma berilgan: 6,2; -5,3; 7,2; 3,7; -2,2; 6,2; 3,7; -7,6; 3,7; 7,2; 6,2; -5,3; -7,6; -5,3; -7,6; 6,2; 7,2; -2,2; -7,6; 7,2. Bu tanlanmaning statistik taqsimotin toping.
\\
\textbf{A2.} 
Hajmi \(n = 20\) ga teng bo'lgan tanlanma berilgan: 6,1; -5,8; 7,9; 3,5; -2,5; 6,1; 3,5; -7,2; 3,5; 7,9; 6,1; -5,8; -7,2; -5,8; -7,2; 6,1; 7,9; -2,5; -7,2; 7,9. Bu tanlanmaning empirik taqsimot funksiyasin toping.
\\
\textbf{A3.} 
Oliy matematika fanidan 10 ta talaba test topshiriqlarin topshirdi. Harbir talaba 10 balgacha to'plashi mumkin. Agar test topshiriqlari natijalari bo'yicha \{10, 8, 6, 5, 4, 8, 10, 7, 5, 7\} tanlanma olingan bo'lsa, ushbu tanlanmalarning tanlanma o'rta va tanlanma dispersiyalarin toping.
\\
\textbf{B1.} 
Agar normal taqsimlangan bosh to'plamdan olingan hajmi \(n = 16\)ga teng tanlanma bo'yicha \(\overline{x} = 15,2\) tanlanma o'rta va \({\overline{S}}^{2} = 0,81\) tuzatilgan tanlanma dispersiyalar topilgan bo'lsa, u holda \(\gamma = 0,90\) ishonchlilik bilan noma'lum \(\theta\) matematik kutilma uchun ishonchlilik intervalin tuzing.
\\
\textbf{B2.} 
Agar \(X^{(n)} = \left( X_{1},...,X_{n} \right)\) tanlanma \(\theta\) parametrli Bernulli taqsimotidan olingan bo'lsa, u holda noma'lum \(\theta\) parametr uchun momentlar usuli bahosini toping.
\\
\textbf{B3.} 
\(f(x) = \frac{2x}{\theta}e^{- \frac{x^{2}}{\theta}},\ \ x \geq 0\) model uchun \(\theta\) parametri haqiqatga maksimal o'xshashlik usuli bahosi topilsin.
\\
\textbf{C1.} 
Agar \(X^{(n)} = \left( X_{1},...,X_{n} \right)\) tanlanma \(\theta\) parametrli Bernulli taqsimotidan olingan bo'lsa, u holda noma'lum \(\theta\) parametr uchun \(X_{n}\) bahoning siljimaganligi va asosliligini tekshiring.
\\
\textbf{C2.} 
Agar \(X^{(n)} = \left( X_{1},...,X_{n} \right)\) tanlanma zichlik funksiyasi\(f(x,\theta) = \theta x^{\theta - 1},x \in \lbrack 0,1\rbrack\)bo'lgan taqsimotdan olingan bo'lsa, u holda noma'lum \(\theta\) parametr uchun momentlar usuli bahosini toping.
\\
\textbf{C3.} 
Agar \(X^{(n)} = \left( X_{1},...,X_{n} \right)\) tanlanma zichlik funksiyasi \(f(x;\theta) = \frac{\theta ln^{\theta - 1}x}{x},\ \ x \in \lbrack 1,e\rbrack\) bo'lgan taqsimotdan olingan bo'lsa, u holda noma'lum \(\theta > 0\) parametr uchun haqiqatga maksimal o'xshashlik bahosini toping.
\\

\end{tabular}
\vspace{1cm}


\begin{tabular}{m{17cm}}
\textbf{15-variant}
\newline

\textbf{T1.} 
Poligon va gistogramma(nisbiy chastota, interval qator, grafik).
\\
\textbf{T2.} 
Pirsonning xi-kvadrat muvofiqlik kriteriysi (Pirson teoremasi).
\\
\textbf{A1.} 
Hajmi \(n = 20\) ga teng bo'lgan tanlanma berilgan: 9,6; 1,5; 7,4; 9,6; 2,8; 1,5; 6,3; 1,5; 9,6; 6,3; 2,8; 4,1; 6,3; 9,6; 1,5; 1,5; 6,3; 7,4; 4,1; 7,4. Bu tanlanmaning statistik taqsimotin toping.
\\
\textbf{A2.} 
Hajmi \(n = 20\) ga teng bo'lgan tanlanma berilgan: 9,8; 1,2; 7,1; 9,8; 2,9; 1,2; 6,7; 1,2; 9,8; 6,7; 2,9; 4,6; 6,7; 9,8; 1,2; 1,2; 6,7; 7,1; 4,6; 7,1. Bu tanlanmaning empirik taqsimot funksiyasin toping.
\\
\textbf{A3.} 
Oliy matematika fanidan 10 ta talaba test topshiriqlarin topshirdi. Harbir talaba 10 balgacha to'plashi mumkin. Agar test topshiriqlari natijalari bo'yicha \{9, 10, 5, 6, 4, 8, 4, 6, 10, 8\} tanlanma olingan bo'lsa, ushbu tanlanmalarning tanlanma o'rta va tanlanma dispersiyalarin toping.
\\
\textbf{B1.} 
Agar normal taqsimlangan bosh to'plamdan olingan hajmi \(n = 10\) ga teng bo'lgan tanlanma bo'yicha \({\overline{S}}^{2} = 0,45\) tuzatilgan tanlanma dispersiya topilgan bo'lsa, u holda \(\gamma = 0,95\) ishonchlilik bilan noma'lum \(\theta_{2}^{2}\) dispersiya uchun ishonchlilik intervalin tuzing.
\\
\textbf{B2.} 
Agar (3,0,-2,0,-2,3,-2,0,0,3,0,0,0,0,3,-2,0,0,-2,3,0) tanlanma quyida berilgan taqsimotdan olingan bo'lsa, u holda noma'lum \(\left( \theta_{1},\theta_{2} \right)\) vektor parametr uchun momentlar usuli bahosini toping.
$\begin{array}{|c|c|c|c|}
    \hline
    \xi & - 2 & 0 & 3 \\
    \hline
    P_{\theta} & 2\theta_{1} & 0,5 + \theta_{1} + \theta_{2} & \theta_{2} \\
    \hline
\end{array}$
\\
\textbf{B3.} 
Agar \(X^{(n)} = \left( X_{1},...,X_{n} \right)\) tanlanma \(\theta\) parametrli ko'rsatkichli taqsimotdan olingan bo'lsa, u holda noma'lum \(\theta\) parametrning haqiqatga maksimal o'xshashlik usuli bahosini toping.
\\
\textbf{C1.} 
Agar \(X^{(n)} = \left( X_{1},...,X_{n} \right)\) tanlanma \(\theta\) parametrli Bernulli taqsimotidan olingan bo'lsa, u holda noma'lum \(\theta(1 - \theta)\) parametr uchun \(X_{1}\left( 1 - X_{n} \right)\) bahoning siljimaganligi va asosliligini tekshiring.
\\
\textbf{C2.} 
Agar \(X^{(n)} = \left( X_{1},...,X_{n} \right)\) tanlanma \(\ \ (a,\theta^{2})\ \ \) parametrli normal taqsimotdan olingan bo'lsa (\(\alpha -\) ma'lum), u holda noma'lum \(\ \ \theta^{2}\) parametr uchun momentlar usuli bahosini \(\ \ g(x) = (x - a)^{2}\) funksiyasi yordamida toping.
\\
\textbf{C3.} 
\(f(x,\theta) = \frac{4x^{3}}{\theta_{2}\sqrt{2\pi}}\exp\left\{ - \frac{\left( x^{4} - \theta_{1} \right)^{2}}{2{\theta_{2}}^{2}} \right\}\) model uchun \(\theta_{1}\) va \(\theta_{2}\) parametrlarning haqiqatga maksimal o'xshashlik usuli baholari topilsin.
\\

\end{tabular}
\vspace{1cm}


\begin{tabular}{m{17cm}}
\textbf{16-variant}
\newline

\textbf{T1.} 
Empirik taqsimot funktsiyasi. (Tanlanma, eksperiment)
\\
\textbf{T2.} 
Ishonchlilik intervallarin tuzish. Aniq ishonchlilik intervallar.
\\
\textbf{A1.} 
Hajmi \(n = 20\) ga teng bo'lgan tanlanma berilgan:1,8; -8,4; 7,3; 4,7; -3,9; 1,8; 4,7; -10,4; -8,4; 7,3; -10,4; 4,7; -8,4; 1,8; 4,7; -10,4; 7,3; -3,9; 4,7; -8,4. Bu tanlanmaning statistik taqsimotin toping.
\\
\textbf{A2.} 
Hajmi \(n = 20\) ga teng bo'lgan tanlanmaberilgan:1,6; -8,3; 7,6; 4,2; -3,1; 1,6; 4,2; -10,5; -8,3; 7,6; -10,5; 4,2; -8,3; 1,6; 4,2; -10,5; 7,6; -3,1; 4,2; -8,3. Bu tanlanmaning empirik taqsimot funksiyasin toping.
\\
\textbf{A3.} 
Oliy matematika fanidan 10 ta talaba test topshiriqlarin topshirdi. Harbir talaba 10 balgacha to'plashi mumkin. Agar test topshiriqlari natijalari bo'yicha \{9, 3, 6, 3, 7, 6, 4, 6, 10, 6\} tanlanma olingan bo'lsa, ushbu tanlanmalarning tanlanma o'rta va tanlanma dispersiyalarin toping.
\\
\textbf{B1.} 
Agar o'rta kvadratik chetlanish \(\sigma = 2\) bo'lgan normal taqsimot bosh to'plamdan olingan hajmi \(n = 18\)ga teng tanlanma bo'yicha \(\overline{x} = 5,2\) tanlanma o'rta qiymati topilgan bo'lsa, u holda \(\gamma = 0,90\) ishonchlilik bilan noma'lum \(\theta\) matematik kutilma uchun ishonchlilik intervalin tuzing .
\\
\textbf{B2.} 
Puasson taqsimoti noma'lum \(\theta > 0\) parametri momentlar usuli bahosini toping.
\\
\textbf{B3.} 
Agar \(X^{(n)} = \left( X_{1},...,X_{n} \right)\) tanlanma \(\lbrack - \theta,\theta\rbrack\) oraliqda tekis taqsimotdan olingan bo'lsa, u holda noma'lum \(\theta > 0\) parametrning haqiqatga maksimal o'xshashlik usuli bahosini toping.
\\
\textbf{C1.} 
Agar \(X^{(n)} = \left( X_{1},...,X_{n} \right)\) tanlanma \(\theta\) parametrli Bernulli taqsimotidan olingan bo'lsa, u holda noma'lum \(\theta^{2}\) parametr uchun \(X_{1}X_{n}\) bahoning siljimaganligi va asosliligini tekshiring.
\\
\textbf{C2.} 
Agar \(X^{(n)} = \left( X_{1},...,X_{n} \right)\) tanlanma \((\theta,\theta^{2})\) parametrli normal taqsimotdan \(\ \ g(x) = (x)^{2}\ \ \) olingan bo'lsa, u holda noma'lum \(\theta > 0\) parametr uchun momentlar usuli bahosini funksiya yordamida toping.
\\
\textbf{C3.} 
Agar \(X^{(n)} = \left( X_{1},...,X_{n} \right)\) tanlanma \(\lbrack\theta,\theta + 2\rbrack\) oraliqda tekis taqsimotdan olingan bo'lsa, u holda noma'lum \(\theta\) parametrning haqiqatga maksimal o'xshashlik usuli bahosini toping.
\\

\end{tabular}
\vspace{1cm}


\begin{tabular}{m{17cm}}
\textbf{17-variant}
\newline

\textbf{T1.} Matematik statistikaning asosiy masalalari. (Statistik ma'lumotlar, guruhlash)
\\
\textbf{T2.} 
Haqiqatga maksimal o'xshashlik usuli. (haqiqatga maksimal o'xshashlik funktsiyasi, noma'lum parametrlarni baholash).
\\
\textbf{A1.} 
Hajmi \(n = 20\) ga teng bo'lgan tanlanma berilgan: 2,7; -13,5; 1,2; 2,7; 1,2; 4,9; -9,5; 1,2; 2,7; 4,9; -9,5; 2,7; -3,5; 1,2; 2,7; 4,9; -3,5; 2,7; 4,9; 1,2;. Bu tanlanmaning statistik taqsimotin toping.
\\
\textbf{A2.} 
Hajmi \(n = 20\) ga teng bo'lgan tanlanma berilgan: 2,8; -13,9; 1,9; 2,8; 1,9; 4,3; -9,4; 1,9; 2,8; 4,3; -9,4; 2,8; -3,7; 1,9; 2,8; 4,3; -3,7; 2,8; 4,3; 1,9. Bu tanlanmaning empirik taqsimot funksiyasin toping.
\\
\textbf{A3.} 
Oliy matematika fanidan 10 ta talaba test topshiriqlarin topshirdi. Harbir talaba 10 balgacha to'plashi mumkin. Agar test topshiriqlari natijalari bo'yicha \{10, 7, 5, 9, 3, 8, 10, 7, 8, 3\} tanlanma olingan bo'lsa, ushbu tanlanmalarning tanlanma o'rta va tanlanma dispersiyalarin toping.
\\
\textbf{B1.} 
Agar normal taqsimlangan bosh to'plamdan olingan hajmi \(n = 36\)ga teng tanlanma bo'yicha \(\overline{x} = 20,2\) tanlanma o'rta va \({\overline{S}}^{2} = 0,81\) tuzatilgan tanlanma dispersiyalar topilgan bo'lsa, u holda \(\gamma = 0,95\) ishonchlilik bilan noma'lum \(\theta\) matematik kutilma uchun ishonchlilik intervalin tuzing.
\\
\textbf{B2.} 
\(\left\lbrack \theta_{1},\theta_{2} \right\rbrack\) oraliqda tekis taqsimot parametrlari uchun momentlar usuli baholarini toping.
\\
\textbf{B3.} 
Agar \(X^{(n)} = \left( X_{1},...,X_{n} \right)\) tanlanma \(\left( a,\theta^{2} \right)\) parametrli normal taqsimotdan olingan bo'lsa (\(\alpha -\) ma'lum), u holda noma'lum \(\theta^{2}\) parametrning haqiqatga maksimal o'xshashlik bahosini toping.
\\
\textbf{C1.} 
Agar \(X^{(n)} = \left( X_{1},...,X_{n} \right)\) tanlanma \((\alpha,\theta)\) parametrli Veybull taqsimotdan olingan bo'lsa (\(\alpha -\) ma'lum), u holda noma'lum \(\theta\) parametr uchun \(1/\overline{x^{\alpha}}\) bahoning siljimaganligi va asosliligini tekshiring.
\\
\textbf{C2.} 
Agar \(X^{(n)} = \left( X_{1},...,X_{n} \right)\) tanlanma {[}\(0,2\theta\rbrack\) oraliqda tekis taqsimotdan olingan bo'lsa, u holda noma'lum \(\theta > 0\) parametr uchun momentlar usuli bahosini toping.
\\
\textbf{C3.} 
Agar \(X^{(n)} = \left( X_{1},...,X_{n} \right)\) tanlanma \(\left( \theta,\theta^{2} \right)\) parametrli normal taqsimotdan olingan bo'lsa, u holda noma'lum \(\theta > 0\) parametrning haqiqatga maksimal o'xshashlik bahosini toping.
\\

\end{tabular}
\vspace{1cm}


\begin{tabular}{m{17cm}}
\textbf{18-variant}
\newline

\textbf{T1.} 
Guruhlangan va interval variatsion qatorlar.
\\
\textbf{T2.} 
Normal qonun dispersiyasi uchun ishonchlilik intervalin tuzish. (Ishonchlilik ehtimolligi, interval)
\\
\textbf{A1.} 
Hajmi \(n = 20\) ga teng bo'lgan tanlanma berilgan: 9,9; 5,7; 3,2; 2,8; 5,7; 9,9; 7,5; 3,7; 9,9; 3,2; 2,8; 3,7; 7,5; 5,7; 3,2; 2,8; 7,5; 3,2; 9,9; 7,5. Bu tanlanmaning statistik taqsimotin toping.
\\
\textbf{A2.} 
Hajmi \(n = 20\) ga teng bo'lgan tanlanma berilgan: 9,7; 5,2; 3,2; 2,4; 5,2; 9,7; 7,5; 3,7; 9,7; 3,2; 2,4; 3,7; 7,5; 5,2; 3,2; 2,4; 7,5; 3,2; 9,7; 7,5. Bu tanlanmaning empirik taqsimot funksiyasin toping.
\\
\textbf{A3.} 
Oliy matematika fanidan 10 ta talaba test topshiriqlarin topshirdi. Harbir talaba 10 balgacha to'plashi mumkin. Agar test topshiriqlari natijalari bo'yicha \{1, 6, 2, 6, 3, 6, 4, 6, 10, 6\} tanlanma olingan bo'lsa, ushbu tanlanmalarning tanlanma o'rta va tanlanma dispersiyalarin toping.
\\
\textbf{B1.} 
Agar normal taqsimlangan bosh to'plamdan olingan hajmi \(n = 10\) ga teng bo'lgan tanlanma bo'yicha \({\overline{S}}^{2} = 0,7\) tuzatilgan tanlanma dispersiya topilgan bo'lsa, u holda \(\gamma = 0,95\) ishonchlilik bilan noma'lum \(\theta_{2}^{2}\) dispersiya uchun ishonchlilik intervalin tuzing.
\\
\textbf{B2.} 
Agar (3,-2,-2,0,-2,2,-2,0,-2,3,-2,0,3,0,3,-2,0,-2,3,-2,2,-2,-2,3,3,2,-2,2,3,3) tanlanma quyida berilgan taqsimotdan olingan bo'lsa, u holda noma'lum \(\theta\) parametr uchun momentlar usuli bahosini \(g(x) = |x|\) funksiya yordamida toping.
$\begin{array}{|c|c|c|c|}
    \hline
    \xi & -2 & 0 & 3 \\
    \hline
    P_{\theta} & 3\theta & 1 - 5\theta & 2\theta \\
    \hline
\end{array}$
\\
\textbf{B3.} 
Agar \(x_{1} = 1,1;\ \ x_{2} = 2,7;\ldots;x_{100} = 1,5\) tanlanma \(\theta\) parametrli ko'rsatkichli taqsimotdan olingan bo'lib, \(\sum_{k = 1}^{100}x_{k} = 200\) bo'lsa, u holda noma'lum \(\theta\) parametrning haqiqatga maksimal o'xshashlik bahosini toping.
\\
\textbf{C1.} 
Agar \(X^{(n)} = \left( X_{1},...,X_{n} \right)\) tanlanma \(\theta\) parametrli geometrik taqsimotdan olingan bo'lsa, u holda noma'lum \(\theta\) parametr uchun \(t(1 + \overline{x})\) bahoning siljimaganligi va asosliligini tekshiring.
\\
\textbf{C2.} 
Agar \(X^{(n)} = \left( X_{1},...,X_{n} \right)\) tanlanma \((\theta,2\theta)\) parametrli normal taqsimotdan olingan bo'lsa, u holda noma'lum \(\theta > 0\) parametr uchun momentlar usuli bahosini \(\ \ g(x) = (x)^{2}\) funksiya yordamida toping.
\\
\textbf{C3.} 
\(f(x;\theta) = \frac{7x^{6}}{\sqrt{2\pi}}\exp\left\{ - \frac{(x^{7} - \theta)^{2}}{2} \right\}\) model uchun \(\theta\) parametri haqiqatga maksimal o'xshashlik usuli bahosi topilsin.
\\

\end{tabular}
\vspace{1cm}


\begin{tabular}{m{17cm}}
\textbf{19-variant}
\newline

\textbf{T1.} 
Glivenko-Kantelli teoremasi. (empirik taqsimot funktsiyasi, ehtimollik bilan yaqinlashish).
\\
\textbf{T2.} 
Statistik gipotezalarni tekshirish (kritik to'plam, 1 va 2-tur xatolik)
\\
\textbf{A1.} 
Hajmi \(n = 20\) ga teng bo'lgan tanlanma berilgan: 3,6; 1,1; -1,8; 0,4; 3,6; 0; 5,3; 1,1; 0; -1,8; 3,6; 0,4; 1,1; 0; 0,4; 1,1; 3,6; -1,8; 3,6; 0. Bu tanlanmaning statistik taqsimotin toping.
\\
\textbf{A2.} 
Hajmi \(n = 20\) ga teng bo'lgan tanlanma berilgan: 3,2; 1,8; -1,1; 0,9; 3,2; 0; 5,6; 1,8; 0; -1,1; 3,2; 0,9; 1,8; 0; 0,9; 1,8; 3,2; -1,1; 3,2; 0. Bu tanlanmaning empirik taqsimot funksiyasin toping.
\\
\textbf{A3.} 
Oliy matematika fanidan 10 ta talaba test topshiriqlarin topshirdi. Harbir talaba 10 balgacha to'plashi mumkin. Agar test topshiriqlari natijalari bo'yicha \{2, 7, 3, 7, 6, 7, 4, 7, 7, 10\} tanlanma olingan bo'lsa, ushbu tanlanmalarning tanlanma o'rta va tanlanma dispersiyalarin toping.
\\
\textbf{B1.} 
Agar o'rta kvadratik chetlanish \(\sigma = 3\) bo'lgan normal taqsimot bosh to'plamdan olingan hajmi \(n = 14\)ga teng tanlanma bo'yicha \(\overline{x} = 5,5\) tanlanma o'rta qiymati topilgan bo'lsa, u holda \(\gamma = 0,90\) ishonchlilik bilan noma'lum \(\theta\) matematik kutilma uchun ishonchlilik intervalin tuzing .
\\
\textbf{B2.} 
\(\lbrack 0,\theta\rbrack\) oraliqda tekis taqsimlangan \(\theta\) parametri uchun momentlar usuli bahosini toping.
\\
\textbf{B3.} 
Agar \(X^{(n)} = \left( X_{1},...,X_{n} \right)\) tanlanma \(\left\lbrack - \theta,\theta^{2} \right\rbrack\) oraliqda tekis taqsimotdan olingan bo'lsa, u holda noma'lum \(\theta > 0\) parametrning haqiqatga maksimal o'xshashlik usuli bahosini toping.
\\
\textbf{C1.} 
Agar \(X^{(n)} = \left( X_{1},...,X_{n} \right)\) tanlanma \(\theta\) parametrli Puasson taqsimotidan olingan bo'lsa, u holda noma'lum \(\theta\) parametr uchun \(\frac{n + 3}{n + 4}\overline{x}\) bahoning siljimaganligi va asosliligini tekshiring.
\\
\textbf{C2.} 
Agar \(X^{(n)} = \left( X_{1},...,X_{n} \right)\) tanlanma zichlik funksiyasi\(f(x,\theta) = \theta x^{\theta - 1},x \in \lbrack 0,1\rbrack\)bo'lgan taqsimotdan olingan bo'lsa, u holda noma'lum \(\theta\) parametr uchun momentlar usuli bahosini toping.
\\
\textbf{C3.} 
Agar \(X^{(n)} = \left( X_{1},...,X_{n} \right)\) tanlanma \(\left\lbrack \theta_{1},\theta_{2} \right\rbrack\) oraliqda tekis taqsimotdan olingan bo'lsa, u holda noma'lum \(\left( \theta_{1},\theta_{2} \right)\) vektor parametrning haqiqatga maksimal o'xshashlik bahosini toping.
\\

\end{tabular}
\vspace{1cm}


\begin{tabular}{m{17cm}}
\textbf{20-variant}
\newline

\textbf{T1.} 
Neyman-Pirson teoremasi.
\\
\textbf{T2.} 
Statistik baho xossalari. (Siljimagan, asosliy, effektiv)
\\
\textbf{A1.} 
Hajmi \(n = 20\) ga teng bo'lgan tanlanma berilgan: 7,1; 3,9; 6,3; 4,6; 7,1; 2,3; 6,3; 3,9; 4,6; 7,1; 2,3; 3,9; 7,6; 2,3; 4,6; 3,9; 2,3; 3,9; 7,6; 4,6. Bu tanlanmaning statistik taqsimotin toping.
\\
\textbf{A2.} 
Hajmi \(n = 20\) ga teng bo'lgan tanlanma berilgan: 7,9; 3,8; 6,1; 4,2; 7,9; 2,4; 6,1; 3,8; 4,2; 7,9; 2,4; 3,8; 10,2; 2,4; 4,2; 3,8; 2,4; 3,8; 10,2; 4,2. Bu tanlanmaning empirik taqsimot funksiyasin toping.
\\
\textbf{A3.} 
Oliy matematika fanidan 10 ta talaba test topshiriqlarin topshirdi. Harbir talaba 10 balgacha to'plashi mumkin. Agar test topshiriqlari natijalari bo'yicha \{9, 8, 6, 8, 6, 4, 5, 4, 7, 4\} tanlanma olingan bo'lsa, ushbu tanlanmalarning tanlanma o'rta va tanlanma dispersiyalarin toping.
\\
\textbf{B1.} 
Agar normal taqsimlangan bosh to'plamdan olingan hajmi \(n = 49\)ga teng tanlanma bo'yicha \(\overline{x} = 14,2\) tanlanma o'rta va \({\overline{S}}^{2} = 0,64\) tuzatilgan tanlanma dispersiyalar topilgan bo'lsa, u holda \(\gamma = 0,95\) ishonchlilik bilan noma'lum \(\theta\) matematik kutilma uchun ishonchlilik intervalin tuzing.
\\
\textbf{B2.} 
Ko'rsatkichli taqsimot noma'lum \(\theta > 0\) parametri momentlar usuli bahosini toping.
\\
\textbf{B3.} 
Agar (4,8,5,3) tanlanma \((a,\theta^{2}\) parametrli normal taqsimotdan olingan bo'lsa, u holda noma'lum \(\theta^{2}\) parametrning haqiqatga maksimal o'xshashlik bahosini toping.
\\
\textbf{C1.} 
Agar \(X^{(n)} = \left( X_{1},...,X_{n} \right)\) tanlanma \(\theta\) parametrli Puasson taqsimotidan olingan bo'lsa, u holda noma'lum \(\theta\) parametr uchun \(\frac{X_{1} + X_{3}}{2}\) bahoning siljimaganligi va asosliligini tekshiring.
\\
\textbf{C2.} 
Agar \(X^{(n)} = \left( X_{1},...,X_{n} \right)\) tanlanma \(1/\theta\) parametrli ko'rsatkichli taqsimotdan olingan bo'lsa, u holda noma'lum \(\theta\) parametr uchun momentlar usuli bahosini \(\ \ g(x) = x^{k},\) \(k \in N\) funksiya yordamida toping.
\\
\textbf{C3.} 
Agar \(X^{(n)} = \left( X_{1},...,X_{n} \right)\) tanlanma zichlik funksiyasi \(f(x;\theta) = \frac{\theta ln^{\theta - 1}x}{x},\ \ x \in \lbrack 1,e\rbrack\) bo'lgan taqsimotdan olingan bo'lsa, u holda noma'lum \(\theta > 0\) parametr uchun haqiqatga maksimal o'xshashlik bahosini toping.
\\

\end{tabular}
\vspace{1cm}


\begin{tabular}{m{17cm}}
\textbf{21-variant}
\newline

\textbf{T1.} 
Tanlanma momentleri (\(k -\)tartibli boshlang'ich, boshlang'ich absolyut, markaziy va markaziy absolyut momentler).
\\
\textbf{T2.} 
Statistik gipotezalarni tekshirish (kritik to'plam, 1 va 2-tur xatolik)
\\
\textbf{A1.} 
Hajmi \(n = 20\) ga teng bo'lgan tanlanma berilgan: 0,6; -3,8; -2,3; -4,3; 2,8; 4,7; -2,3; 0,6; -3,8; 2,8; -2,3; -4,3; 0,6; -2,3; 2,8; -3,8; -4,3; -2,3; 2,8; -3,8. Bu tanlanmaning statistik taqsimotin toping.
\\
\textbf{A2.} 
Hajmi \(n = 20\) ga teng bo'lgan tanlanma berilgan: 0,7; -3,1; -2,3; -4,8; 2,6; 4,9; -2,3; 0,7; -3,1; 2,6; -2,3; -4,8; 0,7; -2,3; 2,6; -3,1; -4,8; -2,3; 2,6; -3,1. Bu tanlanmaning empirik taqsimot funksiyasin toping.
\\
\textbf{A3.} 
Oliy matematika fanidan 10 ta talaba test topshiriqlarin topshirdi. Harbir talaba 10 balgacha to'plashi mumkin. Agar test topshiriqlari natijalari bo'yicha \{10, 4, 6, 5, 5, 4, 10, 7, 9, 10\} tanlanma olingan bo'lsa, ushbu tanlanmalarning tanlanma o'rta va tanlanma dispersiyalarin toping.
\\
\textbf{B1.} 
Agar normal taqsimlangan bosh to'plamdan olingan hajmi \(n = 8\) ga teng bo'lgan tanlanma bo'yicha \({\overline{S}}^{2} = 0,35\) tuzatilgan tanlanma dispersiya topilgan bo'lsa, u holda \(\gamma = 0,90\) ishonchlilik bilan noma'lum \(\theta_{2}^{2}\) dispersiya uchun ishonchlilik intervalin tuzing.
\\
\textbf{B2.} 
Agar (-2,0,-2,0,-2,3,-2,0,0,3,0,0,0,0,3,-2,0,0,-2,3,0) tanlanma quyida berilgan taqsimotdan olingan bo'lsa, u holda noma'lum \(\left( \theta_{1},\theta_{2} \right)\) vektor parametr uchun momentlar usuli bahosini toping.
$\begin{array}{|c|c|c|c|}
    \hline
    \xi & - 2 & 0 & 3\\
    \hline
    P_{\theta} & \theta_{1} & 1 - \theta_{1} - \theta_{2} & \theta_{2} \\
    \hline
\end{array}$
\\
\textbf{B3.} 
Agar \(X^{(n)} = \left( X_{1},...,X_{n} \right)\) tanlanma \(\theta\) parametrli Bernulli taqsimotidan olingan bo'lsa, u holda noma'lum \(\theta\) parametrning haqiqatga maksimal o'xshashlik usuli bahosini toping.
\\
\textbf{C1.} 
Agar \(X^{(n)} = \left( X_{1},...,X_{n} \right)\) tanlanma \(\ln\theta\) parametrli Puasson taqsimotidan olingan bo'lsa, u holda noma'lum \(\theta\) parametr uchun \(e^{\overline{x}}\) bahoning siljimaganligi va asosliligini tekshiring.
\\
\textbf{C2.} 
Agar \(X^{(n)} = \left( X_{1},...,X_{n} \right)\) tanlanma \(\theta\) parametrli geometrik taqsimotdan olingan bo'lsa, u holda noma'lum \(\theta\) parametr uchun momentlar usuli bahosini toping.
\\
\textbf{C3.} 
Agar \(X^{(n)} = \left( X_{1},...,X_{n} \right)\) tanlanma \(\lbrack\theta,\theta + 2\rbrack\) oraliqda tekis taqsimotdan olingan bo'lsa, u holda noma'lum \(\theta\) parametrning haqiqatga maksimal o'xshashlik usuli bahosini toping.
\\

\end{tabular}
\vspace{1cm}


\begin{tabular}{m{17cm}}
\textbf{22-variant}
\newline

\textbf{T1.} 
Tanlanma xarakteristikalar. (Variatsion qator, nisbiy chastota).
\\
\textbf{T2.} 
Kolmogorovning muvofiqlik kritireyesi (Kolmogorov teoremasi)
\\
\textbf{A1.} 
Hajmi \(n = 20\) ga teng bo'lgan tanlanma berilgan: 8,9; 2,7; 1,7; 2,2; 5,6; 1,7; 5,6; 2,7; 1,7; 2,2; 5,6; 8,9; 1,7; 2,2; 1,7; 2,7; 1,7; 5,6; 6,1; 8,9. Bu tanlanmaning statistik taqsimotin toping.
\\
\textbf{A2.} 
Hajmi \(n = 20\) ga teng bo'lgan tanlanma berilgan: 8,7; 2,7; 1,5; 2,2; 5,7; 1,5; 5,7; 2,7; 1,5; 2,2; 5,7; 8,7; 1,5; 2,2; 1,5; 2,7; 1,5; 5,7; 6,3; 8,7. Bu tanlanmaning empirik taqsimot funksiyasin toping.
\\
\textbf{A3.} 
Oliy matematika fanidan 10 ta talaba test topshiriqlarin topshirdi. Harbir talaba 10 balgacha to'plashi mumkin. Agar test topshiriqlari natijalari bo'yicha \{9, 8, 6, 9, 5, 4, 5, 7, 8, 9\} tanlanma olingan bo'lsa, ushbu tanlanmalarning tanlanma o'rta va tanlanma dispersiyalarin toping.
\\
\textbf{B1.} 
Agar o'rta kvadratik chetlanish \(\sigma = 4\) bo'lgan normal taqsimot bosh to'plamdan olingan hajmi \(n = 16\)ga teng tanlanma bo'yicha \(\overline{x} = 5,8\) tanlanma o'rta qiymati topilgan bo'lsa, u holda \(\gamma = 0,90\) ishonchlilik bilan noma'lum \(\theta\) matematik kutilma uchun ishonchlilik intervalin tuzing .
\\
\textbf{B2.} 
Agar (0,-2,0,-2,3,-2,0,0,3,0,0,0,0,3,-2,0,0,-2,3,0,3) tanlanma quyida berilgan taqsimotdan olingan bo'lsa, u holda noma'lum \(\theta\) parametr uchun momentlar usuli bahosini toping.
$\begin{array}{|c|c|c|c|}
    \hline
    \xi & - 2 & 0 & 3 \\
    \hline
    P_{\theta} & \theta & 1 - 2\theta & \theta \\
    \hline
\end{array}$
\\
\textbf{B3.} 
Agar (-1,-1,0,-1,0,-1,-1,5,-1,0,-1,0,5,-1,-1,-1,5,-1,-1,-1,1,-1,5,0,-1,-1,5) tanlanma quyida berilgan taqsimotdan olingan bo'lsa, u holda noma'lum \(\theta\) parametrning haqiqatga maksimal o'xshashlik usuli bahosini toping.
$\begin{array}{|c|c|c|c|}
    \hline
    \xi & - 1 & 0 & 5\\
    \hline
    P_{\theta} & 1 - \theta & \theta/2 & \theta/2 \\
    \hline
\end{array}$
\\
\textbf{C1.} 
Agar \(X^{(n)} = \left( X_{1},...,X_{n} \right)\) tanlanma \((\alpha,\theta)\) parametrli Pareto taqsimotdan olingan bo'lsa (\(\alpha -\) ma'lum), u holda noma'lum \(\theta\) parametr uchun \(X_{(1)}\) bahoning siljimaganligi va asosliligini tekshiring.
\\
\textbf{C2.} 
Agar \(X^{(n)} = \left( X_{1},...,X_{n} \right)\) tanlanma \({\lbrack\theta}_{1},\theta_{2}\rbrack\) oraliqda tekis taqsimotdan olingan bo'lsa, u holda noma'lum \(\left( \theta_{1},\theta_{2} \right)\) vektor parametr uchun momentlar usuli bahosini toping.
\\
\textbf{C3.} 
\(f(x,\theta) = \frac{4x^{3}}{\theta_{2}\sqrt{2\pi}}\exp\left\{ - \frac{\left( x^{4} - \theta_{1} \right)^{2}}{2{\theta_{2}}^{2}} \right\}\) model uchun \(\theta_{1}\) va \(\theta_{2}\) parametrlarning haqiqatga maksimal o'xshashlik usuli baholari topilsin.
\\

\end{tabular}
\vspace{1cm}


\begin{tabular}{m{17cm}}
\textbf{23-variant}
\newline

\textbf{T1.} 
Tanlanma xarakteristikalari.(tanlanma o'rta, tanlanma dispersiya).
\\
\textbf{T2.} 
Statistik gipotezalarni tekshirish (kritik to'plam, 1 va 2-tur xatolik).
\\
\textbf{A1.} 
Hajmi \(n = 20\) ga teng bo'lgan tanlanma berilgan: 1,8; -1,9; 2,4; 1,8; 2,4; 1,8; 2,4; -0,6; -1,9; 1,8; -0,6; 2,4; -3,3; -1,9; 4,0; -3,3; -3,3; -1,9; -3,3; -1,9. Bu tanlanmaning statistik taqsimotin toping.
\\
\textbf{A2.} 
Hajmi \(n = 20\) ga teng bo'lgan tanlanma berilgan: 1,4; -1,9; 2,5; 1,4; 2,5; 1,4; 2,5; -0,4; -1,9; 1,4; -0,4; 2,5; -3,7; -1,9; 4,5; -3,7; -3,7; -1,9; -3,7; -1,9. Bu tanlanmaning empirik taqsimot funksiyasin toping.
\\
\textbf{A3.} 
Oliy matematika fanidan 10 ta talaba test topshiriqlarin topshirdi. Harbir talaba 10 balgacha to'plashi mumkin. Agar test topshiriqlari natijalari bo'yicha \{4, 3, 8, 4, 8, 3, 9, 4, 7, 10\} tanlanma olingan bo'lsa, ushbu tanlanmalarning tanlanma o'rta va tanlanma dispersiyalarin toping.
\\
\textbf{B1.} 
Agar normal taqsimlangan bosh to'plamdan olingan hajmi \(n = 36\)ga teng tanlanma bo'yicha \(\overline{x} = 20,2\) tanlanma o'rta va \({\overline{S}}^{2} = 0,64\) tuzatilgan tanlanma dispersiyalar topilgan bo'lsa, u holda \(\gamma = 0,90\) ishonchlilik bilan noma'lum \(\theta\) matematik kutilma uchun ishonchlilik intervalin tuzing.
\\
\textbf{B2.} 
Agar zichlik funksiyasi \(f(x) = \frac{2x}{\theta}e^{- \frac{x^{2}}{\theta}},\ \ x \geq 0\) ko'rinishga ega bo'lsa, u holda \(\theta\) parametr momentlar usuli bahosini toping.
\\
\textbf{B3.} 
Agar \(X^{(n)} = \left( X_{1},...,X_{n} \right)\) tanlanma zichlik funksiyasi \(f(x;\theta) = \frac{2x}{\theta}e^{- \frac{x^{2}}{\theta}},\ \ x \geq 0\) bo'lgan taqsimotdan olingan bo'lsa, u holda noma'lum \(\theta > 0\) parametrning haqiqatga maksimal usuli bahosini toping.
\\
\textbf{C1.} 
Agar \(X^{(n)} = \left( X_{1},...,X_{n} \right)\) tanlanma zichlik funksiyasi bo'lsa: \(f(x;\theta) = e^{- x + \theta}\left( 1 + e^{- x + \theta} \right)^{2},\ \ x \in R\)bo'lgan taqsimotdan olingan bo'lsa, u holda noma'lum \(\theta\) parametr uchun \(\overline{x}\) bahoning siljimaganligi va asosliligini tekshiring.
\\
\textbf{C2.} 
Agar \(X^{(n)} = \left( X_{1},...,X_{n} \right)\) tanlanma \((\theta,\theta^{2})\) parametrli normal taqsimotdan \(\ \ g(x) = (x)^{2}\ \ \) olingan bo'lsa, u holda noma'lum \(\theta > 0\) parametr uchun momentlar usuli bahosini funksiya yordamida toping.
\\
\textbf{C3.} 
Agar \(X^{(n)} = \left( X_{1},...,X_{n} \right)\) tanlanma zichlik funksiyasi\(f(x;\theta) = \frac{1}{2}e^{- |x - \theta|},\ \ x \in R\) bo'lgan Laplas taqsimotidan olingan bo'lsa, u holda noma'lum \(\theta \in R\) parametrning haqiqatga maksimal o'xshashlik bahosini toping.
\\

\end{tabular}
\vspace{1cm}


\begin{tabular}{m{17cm}}
\textbf{24-variant}
\newline

\textbf{T1.} 
Glivenko-Kantelli teoremasi. (empirik taqsimot funktsiyasi, ehtimollik bilan yaqinlashish).
\\
\textbf{T2.} 
Statistik baho xossalari. (Siljimagan, asosliy, effektiv)
\\
\textbf{A1.} 
Hajmi \(n = 20\) ga teng bo'lgan tanlanma berilgan: 2,9; -3,2; 5,3; -4,3; 4,1; 5,3; -1,2; 2,9; -3,2; 4,1; -4,3; 5,3; -3,2; 2,9; -4,3; 4,1; -1,2; 5,3; 2,9; -3,2. Bu tanlanmaning statistik taqsimotin toping.
\\
\textbf{A2.} 
Hajmi \(n = 20\) ga teng bo'lgan tanlanma berilgan: 2,7; -5,6; 5,2; -8,1; 4,8; 5,2; -1,6; 2,7; -5,6; 4,8; -8,1; 5,2; -5,6; 2,7; -8,1; 4,8; -1,6; 5,2; 2,7; -5,6. Bu tanlanmaning empirik taqsimot funksiyasin toping.
\\
\textbf{A3.} 
Oliy matematika fanidan 10 ta talaba test topshiriqlarin topshirdi. Harbir talaba 10 balgacha to'plashi mumkin. Agar test topshiriqlari natijalari bo'yicha \{7, 9, 4, 9, 7, 5, 4, 7, 2, 6\} tanlanma olingan bo'lsa, ushbu tanlanmalarning tanlanma o'rta va tanlanma dispersiyalarin toping.
\\
\textbf{B1.} 
Agar normal taqsimlangan bosh to'plamdan olingan hajmi \(n = 11\) ga teng bo'lgan tanlanma bo'yicha \({\overline{S}}^{2} = 0,3\) tuzatilgan tanlanma dispersiya topilgan bo'lsa, u holda \(\gamma = 0,95\) ishonchlilik bilan noma'lum \(\theta_{2}^{2}\) dispersiya uchun ishonchlilik intervalin tuzing.
\\
\textbf{B2.} 
\(\left\lbrack \theta_{1},\theta_{2} \right\rbrack\) oraliqda tekis taqsimot parametrlari uchun momentlar usuli baholarini toping.
\\
\textbf{B3.} 
\(f(x) = \frac{\theta}{2}e^{- \theta|x|}\) model uchun \(\theta\) parametri haqiqatga maksimal o'xshashlik usuli bahosi topilsin.
\\
\textbf{C1.} 
Agar \(X^{(n)} = \left( X_{1},...,X_{n} \right)\) tanlanma zichlik funksiyasi \(f(x;\theta) = \left\{ \begin{matrix}
\alpha^{- 1}e^{- \ \frac{x - \theta}{\alpha}},\ \ x \geq \theta, \\
0,\ \ x < \theta
\end{matrix} \right.\ \)bo'lgan taqsimotdan olingan bo'lsa (\(\alpha -\) ma'lum), u holda noma'lum \(\theta\) parametr uchun \(X_{(1)}\) bahoning siljimaganligi va asosliligini tekshiring.
\\
\textbf{C2.} 
Agar \(X^{(n)} = \left( X_{1},...,X_{n} \right)\) tanlanma \(\left( \theta_{1},\theta_{2} \right)\) parametrli gamma taqsimotdan olingan bo'lsa, u holda noma'lum \(\left( \theta_{1},\theta_{2} \right)\) vektor parametr uchun momentlar usuli bahosini toping.
\\
\textbf{C3.} 
\(f(x;\theta) = \frac{7x^{6}}{\sqrt{2\pi}}\exp\left\{ - \frac{(x^{7} - \theta)^{2}}{2} \right\}\) model uchun \(\theta\) parametri haqiqatga maksimal o'xshashlik usuli bahosi topilsin.
\\

\end{tabular}
\vspace{1cm}


\begin{tabular}{m{17cm}}
\textbf{25-variant}
\newline

\textbf{T1.} 
Neyman-Pirson teoremasi.
\\
\textbf{T2.} 
Pirsonning xi-kvadrat muvofiqlik kriteriysi (Pirson teoremasi).
\\
\textbf{A1.} 
Hajmi \(n = 20\) ga teng bo'lgan tanlanma berilgan: 14,7; 7,3; 16,6; 9,8; 11,2; 16,6; 6,7; 7,3; 11,2; 14,7; 6,7; 16,6; 7,3; 11,2; 14,7; 16,6; 6,7; 7,3; 11,2; 16,6. Bu tanlanmaning statistik taqsimotin toping.
\\
\textbf{A2.} 
Hajmi \(n = 20\) ga teng bo'lgan tanlanma berilgan: 14,4; 7,6; 16,7; 9,1; 11,8; 16,7; 6,4; 7,6; 11,8; 14,4; 6,4; 16,7; 7,6; 11,8; 14,4; 16,7; 6,4; 7,6; 11,8; 16,7. Bu tanlanmaning empirik taqsimot funksiyasin toping.
\\
\textbf{A3.} 
Oliy matematika fanidan 10 ta talaba test topshiriqlarin topshirdi. Harbir talaba 10 balgacha to'plashi mumkin. Agar test topshiriqlari natijalari bo'yicha \{10, 8, 4, 6, 2, 8, 5, 10, 2, 5\} tanlanma olingan bo'lsa, ushbu tanlanmalarning tanlanma o'rta va tanlanma dispersiyalarin toping.
\\
\textbf{B1.} 
Agar o'rta kvadratik chetlanish \(\sigma = 4\) bo'lgan normal taqsimot bosh to'plamdan olingan hajmi \(n = 49\)ga teng tanlanma bo'yicha \(\overline{x} = 9,4\) tanlanma o'rta qiymati topilgan bo'lsa, u holda \(\gamma = 0,9\) ishonchlilik bilan noma'lum \(\theta\) matematik kutilma uchun ishonchlilik intervalin tuzing .
\\
\textbf{B2.} 
Agar (-2,0,-2,0,-2,3,-2,0,0,3,0,0,0,0,3,-2,0,0,-2,3,0) tanlanma quyida berilgan taqsimotdan olingan bo'lsa, u holda noma'lum \(\left( \theta_{1},\theta_{2} \right)\) vektor parametr uchun momentlar usuli bahosini toping.
$\begin{array}{|c|c|c|c|}
    \hline
    \xi & - 2 & 0 & 3\\
    \hline
    P_{\theta} & \theta_{1} & 1 - \theta_{1} - \theta_{2} & \theta_{2} \\
    \hline
\end{array}$
\\
\textbf{B3.} 
Agar \(X^{(n)} = \left( X_{1},...,X_{n} \right)\) tanlanma \(\left( a,\theta^{2} \right)\) parametrli normal taqsimotdan olingan bo'lsa (\(\alpha -\) ma'lum), u holda noma'lum \(\theta^{2}\) parametrning haqiqatga maksimal o'xshashlik bahosini toping.
\\
\textbf{C1.} 
Agar \(X^{(n)} = \left( X_{1},...,X_{n} \right)\) tanlanma \(\lbrack 0,\theta\rbrack\) oraliqda tekis taqsimotdan olingan bo'lsa, u holda noma'lum \(\theta\) parametr uchun \((n + 1)X_{(1)})\) bahoning siljimaganligi va asosliligini tekshiring.
\\
\textbf{C2.} 
Agar \(X^{(n)} = \left( X_{1},...,X_{n} \right)\) tanlanma zichlik funksiyasi\(f(x,\theta) = \left\{ \begin{matrix}
e^{\theta - x},\ \ x \geq \theta, \\
0,\ \ x < \theta
\end{matrix} \right.\ \)bo'lgan taqsimotdan olingan bo'lsa, u holda noma'lum \(\theta\) parametr uchun momentlar usuli bahosini toping.
\\
\textbf{C3.} 
Agar \(X^{(n)} = \left( X_{1},...,X_{n} \right)\) tanlanma \(\left( \theta,\theta^{2} \right)\) parametrli normal taqsimotdan olingan bo'lsa, u holda noma'lum \(\theta > 0\) parametrning haqiqatga maksimal o'xshashlik bahosini toping.
\\

\end{tabular}
\vspace{1cm}


\begin{tabular}{m{17cm}}
\textbf{26-variant}
\newline

\textbf{T1.} 
Empirik taqsimot funktsiyasi. (Tanlanma, eksperiment)
\\
\textbf{T2.} 
Normal qonun dispersiyasi uchun ishonchlilik intervalin tuzish. (Ishonchlilik ehtimolligi, interval)
\\
\textbf{A1.} 
Hajmi \(n = 20\) ga teng bo'lgan tanlanma berilgan: 4,3; 4,9; 13,4; 13,4; 6,5; 4,9; 4,9; 4,3; 5,1; 6,5; 6,5; 7,0; 4,3; 4,9; 6,5; 6,5; 5,1; 5,1; 4,9; 13,4. Bu tanlanmaning statistik taqsimotin toping.
\\
\textbf{A2.} 
Hajmi \(n = 20\) ga teng bo'lgan tanlanma berilgan: 4,2; 4,9; 13,8; 13,8; 6,6; 4,9; 4,9; 4,2; 5,3; 6,6; 6,6; 7,5; 4,2; 4,9; 6,6; 6,6; 5,3; 5,3; 4,9; 13,8. Bu tanlanmaning empirik taqsimot funksiyasin toping.
\\
\textbf{A3.} 
Oliy matematika fanidan 10 ta talaba test topshiriqlarin topshirdi. Harbir talaba 10 balgacha to'plashi mumkin. Agar test topshiriqlari natijalari bo'yicha \{9, 10, 6, 7, 4, 8, 10, 7, 9, 10\} tanlanma olingan bo'lsa, ushbu tanlanmalarning tanlanma o'rta va tanlanma dispersiyalarin toping.
\\
\textbf{B1.} 
Agar o'rta kvadratik chetlanish \(\sigma = 2\) bo'lgan normal taqsimot bosh to'plamdan olingan hajmi \(n = 10\)ga teng tanlanma bo'yicha \(\overline{x} = 5,4\) tanlanma o'rta qiymati topilgan bo'lsa, u holda \(\gamma = 0,95\) ishonchlilik bilan noma'lum \(\theta\) matematik kutilma uchun ishonchlilik intervalin tuzing .
\\
\textbf{B2.} 
Agar \(X^{(n)} = \left( X_{1},...,X_{n} \right)\) tanlanma \(\theta\) parametrli ko'rsatkichli taqsimotdan olingan bo'lsa, u holda noma'lum \(\theta\) parametr uchun momentlar usuli bahosini toping.
\\
\textbf{B3.} 
Agar \(X^{(n)} = \left( X_{1},...,X_{n} \right)\) tanlanma \(\lbrack - \theta,\theta\rbrack\) oraliqda tekis taqsimotdan olingan bo'lsa, u holda noma'lum \(\theta > 0\) parametrning haqiqatga maksimal o'xshashlik usuli bahosini toping.
\\
\textbf{C1.} 
Agar \(X^{(n)} = \left( X_{1},...,X_{n} \right)\) tanlanma \(\lbrack 0,\theta\rbrack\) oraliqda tekis taqsimotdan olingan bo'lsa, u holda noma'lum \(\theta\) parametr uchun \(\frac{n + 1}{n}X_{(n)}\) bahoning siljimaganligi va asosliligini tekshiring.
\\
\textbf{C2.} 
Agar \(X^{(n)} = \left( X_{1},...,X_{n} \right)\) tanlanma \((\theta,2\theta)\) parametrli normal taqsimotdan olingan bo'lsa, u holda noma'lum \(\theta > 0\) parametr uchun momentlar usuli bahosini \(\ \ g(x) = (x)^{2}\) funksiya yordamida toping.
\\
\textbf{C3.} 
Agar \(X^{(n)} = \left( X_{1},...,X_{n} \right)\) tanlanma \((\theta,2\theta)\) parametrli normal taqsimotdan olingan bo'lsa, u holda noma'lum \(\theta > 0\) parametrning haqiqatga maksimal o'xshashlik bahosini toping.
\\

\end{tabular}
\vspace{1cm}


\begin{tabular}{m{17cm}}
\textbf{27-variant}
\newline

\textbf{T1.} 
Momentler usuli. (tanlanma momentleri, noma'lum parametrlarni baholash).
\\
\textbf{T2.} 
Ishonchlilik intervallarin tuzish. Aniq ishonchlilik intervallar.
\\
\textbf{A1.} 
Hajmi \(n = 20\) ga teng bo'lgan tanlanma berilgan: -2,1; 1,7; 3,3; 3,3; 11,7; 4,7; 1,7; 4,7; -2,1; 4,7; 4,7; 4,7; 8,0; -2,1; 1,7; 4,7; 8,0; 11,7; 1,7; 8,0. Bu tanlanmaning statistik taqsimotin toping.
\\
\textbf{A2.} 
Hajmi \(n = 20\) ga teng bo'lgan tanlanma berilgan: -2,2; 1,3; 3,8; 3,8; 11,5; 4,1; 1,3; 4,1; -2,2; 4,1; 4,1; 4,1; 8,4; -2,2; 1,3; 4,1; 8,4; 11,5; 1,3; 8,4. Bu tanlanmaning empirik taqsimot funksiyasin toping.
\\
\textbf{A3.} 
Oliy matematika fanidan 10 ta talaba test topshiriqlarin topshirdi. Harbir talaba 10 balgacha to'plashi mumkin. Agar test topshiriqlari natijalari bo'yicha \{4, 1, 2, 4, 6, 4, 5, 3, 6, 5\} tanlanma olingan bo'lsa, ushbu tanlanmalarning tanlanma o'rta va tanlanma dispersiyalarin toping.
\\
\textbf{B1.} 
Agar normal taqsimlangan bosh to'plamdan olingan hajmi \(n = 16\)ga teng tanlanma bo'yicha \(\overline{x} = 20,2\) tanlanma o'rta va \({\overline{S}}^{2} = 0,64\) tuzatilgan tanlanma dispersiyalar topilgan bo'lsa, u holda \(\gamma = 0,95\) ishonchlilik bilan noma'lum \(\theta\) matematik kutilma uchun ishonchlilik intervalin tuzing.
\\
\textbf{B2.} 
Agar \(X^{(n)} = \left( X_{1},...,X_{n} \right)\) tanlanma \(\theta\) parametrli Bernulli taqsimotidan olingan bo'lsa, u holda noma'lum \(\theta\) parametr uchun momentlar usuli bahosini toping.
\\
\textbf{B3.} 
Agar (0,1,2,0) tanlanma quyida berilgan taqsimotdan olingan bo'lsa, u holda noma'lum \(\theta\) parametrning haqiqatga maksimal o'xshashlik bahosini toping.
$\begin{array}{|c|c|c|c|}
    \hline
    \xi & 0 & 1 & 2 \\
    \hline
    P_{\theta} & \theta & 2\theta & 1 - 3\theta \\
    \hline
\end{array}$
\\
\textbf{C1.} 
Agar \(X^{(n)} = \left( X_{1},...,X_{n} \right)\) tanlanma \(M\xi = a\) ma'lum va \(M\xi^{2}\) chekli bo'lgan taqsimotdan olingan bo'lsa, u holda noma'lum \(D\xi\) dispersiya uchun \({\overline{S}}^{2}\) bahoning siljimaganligi va asosliligini tekshiring.
\\
\textbf{C2.} 
Agar \(X^{(n)} = \left( X_{1},...,X_{n} \right)\) tanlanma \(\theta\) parametrli Puasson taqsimotidan olingan bo'lsa, u holda noma'lum \(\theta\) parametr uchun momentlar usuli bahosini toping.
\\
\textbf{C3.} 
Agar \(X^{(n)} = \left( X_{1},...,X_{n} \right)\) tanlanma zichlik funksiyasi \(f(x;\theta) = \left\{ \begin{array}{r}
3x^{2}\theta^{- 3}{e^{- \left( \frac{x}{\theta} \right)}}^{3},\ \ \ \ x \geq 0 \\
0,\ \ \ \ \ \ \ \ \ \ \ x < 0
\end{array} \right.\ \) bo'lgan taqsimotdan olingan bo'lsa, u holda noma'lum \(\theta > 0\) parametrning haqiqatga maksimal o'xshashlik bahosini toping.
\\

\end{tabular}
\vspace{1cm}


\begin{tabular}{m{17cm}}
\textbf{28-variant}
\newline

\textbf{T1.} Matematik statistikaning asosiy masalalari. (Statistik ma'lumotlar, guruhlash)
\\
\textbf{T2.} 
Chiziqli korrelyatsiya tenglamasi (ta'rifi, regressiya to'g'ri chiziqning tanlanma tenglamalari)
\\
\textbf{A1.} 
Hajmi \(n = 20\) ga teng bo'lgan tanlanma berilgan: -11,0; -4,1; 0; 2,3; 1,2; 0; 1,2; 2,3; 2,3; 1,2; 2,3; -11,0; 3,4; 1,2; 3,4; 3,4; 0; 3,4; 2,3; 0. Bu tanlanmaning statistik taqsimotin toping.
\\
\textbf{A2.} 
Hajmi \(n = 20\) ga teng bo'lgan tanlanma berilgan: -11,2; -4,5; 0; 2,9; 1,7; 0; 1,7; 2,9; 2,9; 1,7; 2,9; -11,2; 3,1; 1,7; 3,1; 3,1; 0; 3,1; 2,9; 0. Bu tanlanmaning empirik taqsimot funksiyasin toping.
\\
\textbf{A3.} 
Oliy matematika fanidan 10 ta talaba test topshiriqlarin topshirdi. Harbir talaba 10 balgacha to'plashi mumkin. Agar test topshiriqlari natijalari bo'yicha \{8, 9, 10, 4, 9, 7, 6, 7, 6, 4\} tanlanma olingan bo'lsa, ushbu tanlanmalarning tanlanma o'rta va tanlanma dispersiyalarin toping.
\\
\textbf{B1.} 
Agar normal taqsimlangan bosh to'plamdan olingan hajmi \(n = 11\) ga teng bo'lgan tanlanma bo'yicha \({\overline{S}}^{2} = 0,5\) tuzatilgan tanlanma dispersiya topilgan bo'lsa, u holda \(\gamma = 0,90\) ishonchlilik bilan noma'lum \(\theta_{2}^{2}\) dispersiya uchun ishonchlilik intervalin tuzing.
\\
\textbf{B2.} 
\(\lbrack 0,\theta\rbrack\) oraliqda tekis taqsimlangan \(\theta\) parametri uchun momentlar usuli bahosini toping.
\\
\textbf{B3.} 
Agar \(X^{(n)} = \left( X_{1},...,X_{n} \right)\) tanlanma \(\theta\) parametrli Bernulli taqsimotidan olingan bo'lsa, u holda noma'lum \(\theta\) parametrning haqiqatga maksimal o'xshashlik usuli bahosini toping.
\\
\textbf{C1.} 
Agar \(X^{(n)} = \left( X_{1},...,X_{n} \right)\) tanlanma \(M\xi = a\) ma'lum va \(M\xi^{2}\) chekli bo'lgan taqsimotdan olingan bo'lsa, u holda noma'lum \(D\xi\) dispersiya uchun \(\frac{1}{n}\sum_{i = 1}^{n}{X_{i}a}\) bahoning siljimaganligi va asosliligini tekshiring.
\\
\textbf{C2.} 
Agar \(X^{(n)} = \left( X_{1},...,X_{n} \right)\) tanlanma {[}\(0,2\theta\rbrack\) oraliqda tekis taqsimotdan olingan bo'lsa, u holda noma'lum \(\theta > 0\) parametr uchun momentlar usuli bahosini toping.
\\
\textbf{C3.} 
Agar \(X^{(n)} = \left( X_{1},...,X_{n} \right)\) tanlanma \(\theta\) parametrli geometrik taqsimotdan olingan bo'lsa, u holda noma'lum \(\theta\) parametrning haqiqatga maksimal o'xshashlik usuli bahosini toping.
\\

\end{tabular}
\vspace{1cm}


\begin{tabular}{m{17cm}}
\textbf{29-variant}
\newline

\textbf{T1.} 
Poligon va gistogramma(nisbiy chastota, interval qator, grafik).
\\
\textbf{T2.} 
Momentler usuli. (tanlanma momentleri, noma'lum parametrlarni baholash).
\\
\textbf{A1.} 
Hajmi \(n = 20\) ga teng bo'lgan tanlanma berilgan: 2,5; 3,8; 4,3; 2,5; 3,8; 2,5; 3,1; 4,3; 4,3; 5,5; 6,2; 2,5; 3,1; 6,2; 5,5; 6,2; 3,1; 3,1; 6,2; 3,1. Bu tanlanmaning statistik taqsimotin toping.
\\
\textbf{A2.} 
Hajmi \(n = 20\) ga teng bo'lgan tanlanma berilgan: 2,7; 4,2; 4,8; 2,7; 4,2; 2,7; 3,9; 4,8; 4,8; 5,9; 6,5; 2,7; 3,9; 6,5; 5,9; 6,5; 3,9; 3,9; 6,5; 3,9. Bu tanlanmaning empirik taqsimot funksiyasin toping.
\\
\textbf{A3.} 
Oliy matematika fanidan 10 ta talaba test topshiriqlarin topshirdi. Harbir talaba 10 balgacha to'plashi mumkin. Agar test topshiriqlari natijalari bo'yicha \{7, 8, 7, 6, 4, 8, 4, 7, 9, 10\} tanlanma olingan bo'lsa, ushbu tanlanmalarning tanlanma o'rta va tanlanma dispersiyalarin toping.
\\
\textbf{B1.} 
Agar o'rta kvadratik chetlanish \(\sigma = 3\) bo'lgan normal taqsimot bosh to'plamdan olingan hajmi \(n = 9\)ga teng tanlanma bo'yicha \(\overline{x} = 4,5\) tanlanma o'rta qiymati topilgan bo'lsa, u holda \(\gamma = 0,95\) ishonchlilik bilan noma'lum \(\theta\) matematik kutilma uchun ishonchlilik intervalin tuzing .
\\
\textbf{B2.} 
Agar (3,0,-2,0,-2,3,-2,0,0,3,0,0,0,0,3,-2,0,0,-2,3,0) tanlanma quyida berilgan taqsimotdan olingan bo'lsa, u holda noma'lum \(\left( \theta_{1},\theta_{2} \right)\) vektor parametr uchun momentlar usuli bahosini toping.
$\begin{array}{|c|c|c|c|}
    \hline
    \xi & - 2 & 0 & 3 \\
    \hline
    P_{\theta} & 2\theta_{1} & 0,5 + \theta_{1} + \theta_{2} & \theta_{2} \\
    \hline
\end{array}$
\\
\textbf{B3.} 
Agar \(X^{(n)} = \left( X_{1},...,X_{n} \right)\) tanlanma \(\left\lbrack - \theta,\theta^{2} \right\rbrack\) oraliqda tekis taqsimotdan olingan bo'lsa, u holda noma'lum \(\theta > 0\) parametrning haqiqatga maksimal o'xshashlik usuli bahosini toping.
\\
\textbf{C1.} 
Agar \(X^{(n)} = \left( X_{1},...,X_{n} \right)\) tanlanma \(M\xi = a\) ma'lum va \(M\xi^{2}\) chekli bo'lgan taqsimotdan olingan bo'lsa, u holda noma'lum \(D\xi\) dispersiya uchun \(\overline{x^{2}} - a^{2}\) bahoning siljimaganligi va asosliligini tekshiring.
\\
\textbf{C2.} 
Agar \(X^{(n)} = \left( X_{1},...,X_{n} \right)\) tanlanma \(1\sqrt{\theta}\) parametrli ko'rsatkichli taqsimotdan olingan bo'lsa, u holda noma'lum \(\theta\) parametr uchun momentlar usuli bahosini toping.
\\
\textbf{C3.} 
\(f(x,\theta) = \frac{e^{x}}{\sqrt{2\pi}}\exp\left\{ - \frac{\left( e^{x} - \theta \right)^{2}}{2} \right\}\) model uchun \(\theta\) parametri haqiqatga maksimal o'xshashlik usuli bahosi topilsin.
\\

\end{tabular}
\vspace{1cm}


\begin{tabular}{m{17cm}}
\textbf{30-variant}
\newline

\textbf{T1.} 
Guruhlangan va interval variatsion qatorlar.
\\
\textbf{T2.} 
Haqiqatga maksimal o'xshashlik usuli. (haqiqatga maksimal o'xshashlik funktsiyasi, noma'lum parametrlarni baholash).
\\
\textbf{A1.} 
Hajmi \(n = 20\) ga teng bo'lgan tanlanma berilgan: -4,3; 2,6; 0; -2,5; 2,6; 1,9; 2,2; 0; -4,3; -2,5; 1,9; -2,5; 1,9; 2,2; 2,6; 1,9; 2,6; 2,2; 2,2; 1,9. Bu tanlanmaning statistik taqsimotin toping.
\\
\textbf{A2.} 
Hajmi \(n = 20\) ga teng bo'lgan tanlanma berilgan: -4,9; 2,6; 0,5; -2,6; 2,6; 1,7; 2,3; 0,5; -4,9; -2,6; 1,7; -2,6; 1,7; 2,3; 2,6; 1,7; 2,6; 2,3; 2,3; 1,7. Bu tanlanmaning empirik taqsimot funksiyasin toping.
\\
\textbf{A3.} 
Oliy matematika fanidan 10 ta talaba test topshiriqlarin topshirdi. Harbir talaba 10 balgacha to'plashi mumkin. Agar test topshiriqlari natijalari bo'yicha \{9, 5, 6, 8, 4, 7, 4, 6, 9, 7\} tanlanma olingan bo'lsa, ushbu tanlanmalarning tanlanma o'rta va tanlanma dispersiyalarin toping.
\\
\textbf{B1.} 
Agar normal taqsimlangan bosh to'plamdan olingan hajmi \(n = 25\)ga teng tanlanma bo'yicha \(\overline{x} = 18,6\) tanlanma o'rta va \({\overline{S}}^{2} = 0,49\) tuzatilgan tanlanma dispersiyalar topilgan bo'lsa, u holda \(\gamma = 0,95\) ishonchlilik bilan noma'lum \(\theta\) matematik kutilma uchun ishonchlilik intervalin tuzing.
\\
\textbf{B2.} 
Agar (3,-2,-2,0,-2,2,-2,0,-2,3,-2,0,3,0,3,-2,0,-2,3,-2,2,-2,-2,3,3,2,-2,2,3,3) tanlanma quyida berilgan taqsimotdan olingan bo'lsa, u holda noma'lum \(\theta\) parametr uchun momentlar usuli bahosini \(g(x) = |x|\) funksiya yordamida toping.
$\begin{array}{|c|c|c|c|}
    \hline
    \xi & -2 & 0 & 3 \\
    \hline
    P_{\theta} & 3\theta & 1 - 5\theta & 2\theta \\
    \hline
\end{array}$
\\
\textbf{B3.} 
Agar \(X^{(n)} = \left( X_{1},...,X_{n} \right)\) tanlanma \(\theta\) parametrli ko'rsatkichli taqsimotdan olingan bo'lsa, u holda noma'lum \(\theta\) parametrning haqiqatga maksimal o'xshashlik usuli bahosini toping.
\\
\textbf{C1.} 
Agar \(X^{(n)} = \left( X_{1},...,X_{n} \right)\) tanlanma zichlik funksiyasi bo'lsa: \(f(x,\theta) = \left\{ \begin{matrix}
e^{\theta - x},\ \ x \geq \theta, \\
\ \ 0,\ \ x < \theta
\end{matrix} \right.\ \) bo'lgan taqsimotdan olingan bo'lsa, u holda noma'lum \(\theta\) parametr uchun \(X_{(1)}\) bahoning siljimaganligi va asosliligini tekshiring.
\\
\textbf{C2.} 
Agar \(X^{(n)} = \left( X_{1},...,X_{n} \right)\) tanlanma zichlik funksiyasi\(f(x,\theta) = \left\{ \begin{matrix}
\theta_{1}^{- 1}e^{- \ \frac{x - \theta_{2}}{\theta_{1}}},\ \ x \geq \theta_{2}, \\
0,\ \ x < \theta_{2}
\end{matrix} \right.\ \)bo'lgan taqsimotdan olingan bo'lsa, u holda noma'lum \(\left( \theta_{1},\theta_{2} \right)\) \(\theta_{1} > 0,\) \(\theta_{2} \in R\) vektor parametr uchun momentlar usuli bahosini toping.
\\
\textbf{C3.} 
Agar \(X^{(n)} = \left( X_{1},...,X_{n} \right)\) tanlanma zichlik funksiyasi \(f(x;\theta) = \left\{ \begin{array}{r}
\begin{matrix}
\theta_{1}^{- 1}e^{\frac{x - \theta_{2}}{\theta_{1}}},\ \ x \geq \theta_{2}
\end{matrix} \\
0,\ \ \ \ x < \theta_{2}
\end{array} \right.\ \) bo'lgan taqsimotdan olingan bo'lsa, u holda noma'lum \(.\left( \theta_{1},\theta_{2} \right),\) \(\theta_{1} > 0,\) \(\theta_{2} \in R\) vektor parametrning haqiqatga maksimal o'xshashlik bahosini toping.
\\

\end{tabular}
\vspace{1cm}


\begin{tabular}{m{17cm}}
\textbf{31-variant}
\newline

\textbf{T1.} 
Tanlanma momentleri (\(k -\)tartibli boshlang'ich, boshlang'ich absolyut, markaziy va markaziy absolyut momentler).
\\
\textbf{T2.} 
Statistik baho xossalari. (Siljimagan, asosliy, effektiv)
\\
\textbf{A1.} 
Hajmi \(n = 20\) ga teng bo'lgan tanlanma berilgan: -2,9; -3,8; 2,3; 1,8; 1,8; 0,7; -3,8; -1,5; 2,3; 0,7; -2,9; -1,5; 1,8; -2,9; -1,5; -3,8; 1,8; 1,8; -3,8; 1,8. Bu tanlanmaning statistik taqsimotin toping.
\\
\textbf{A2.} 
Hajmi \(n = 20\) ga teng bo'lgan tanlanma berilgan: -2,4; -3,5; 2,8; 1,4; 1,4; 0,1; -3,5; -1,9; 2,8; 0,1; -2,4; -1,9; 1,4; -2,4; -1,9; -3,5; 1,4; 1,4; -3,5; 1,4. Bu tanlanmaning empirik taqsimot funksiyasin toping.
\\
\textbf{A3.} 
Oliy matematika fanidan 10 ta talaba test topshiriqlarin topshirdi. Harbir talaba 10 balgacha to'plashi mumkin. Agar test topshiriqlari natijalari bo'yicha \{8, 9, 7, 10, 6, 8, 10, 3, 10, 9\} tanlanma olingan bo'lsa, ushbu tanlanmalarning tanlanma o'rta va tanlanma dispersiyalarin toping.
\\
\textbf{B1.} 
Agar normal taqsimlangan bosh to'plamdan olingan hajmi \(n = 12\) ga teng bo'lgan tanlanma bo'yicha \({\overline{S}}^{2} = 0,4\) tuzatilgan tanlanma dispersiya topilgan bo'lsa, u holda \(\gamma = 0,90\) ishonchlilik bilan noma'lum \(\theta_{2}^{2}\) dispersiya uchun ishonchlilik intervalin tuzing.
\\
\textbf{B2.} 
Puasson taqsimoti noma'lum \(\theta > 0\) parametri momentlar usuli bahosini toping.
\\
\textbf{B3.} 
\(f(x) = \frac{2x}{\theta}e^{- \frac{x^{2}}{\theta}},\ \ x \geq 0\) model uchun \(\theta\) parametri haqiqatga maksimal o'xshashlik usuli bahosi topilsin.
\\
\textbf{C1.} 
Agar \(X^{(n)} = \left( X_{1},...,X_{n} \right)\) tanlanma zichlik funksiyasi bo'lsa: \(f(x,\theta) = \left\{ \begin{matrix}
e^{\theta - x},\ \ x \geq \theta, \\
\ \ 0,\ \ x < \theta
\end{matrix} \right.\ \) bo'lgan taqsimotdan olingan bo'lsa, u holda noma'lum \(\theta\) parametr uchun \(\overline{x} - 1\) bahoning siljimaganligi va asosliligini tekshiring.
\\
\textbf{C2.} 
Agar \(X^{(n)} = \left( X_{1},...,X_{n} \right)\) tanlanma \((\theta,2\theta)\) parametrli normal taqsimotdan olingan bo'lsa, u holda noma'lum \(\theta > 0\) parametr uchun momentlar usuli bahosini toping.
\\
\textbf{C3.} 
Agar \(X^{(n)} = \left( X_{1},...,X_{n} \right)\) tanlanma zichlik funksiyasi\(f(x;\theta) = \frac{\theta}{2}e^{- \theta|x|},\ \ x \in R\) bo'lgan taqsimotdan olingan bo'lsa, u holda noma'lum \(\theta > 0\) parametrning haqiqatga maksimal o'xshashlik bahosini toping.
\\

\end{tabular}
\vspace{1cm}


\begin{tabular}{m{17cm}}
\textbf{32-variant}
\newline

\textbf{T1.} 
Tanlanma xarakteristikalari.(tanlanma o'rta, tanlanma dispersiya).
\\
\textbf{T2.} 
Statistik gipotezalarni tekshirish (kritik to'plam, 1 va 2-tur xatolik)
\\
\textbf{A1.} 
Hajmi \(n = 20\) ga teng bo'lgan tanlanma berilgan: 3,6; 2,9; 3,6; 3,2; 1,1; 0,3; 1,1; 3,6; 1,7; 1,1; 0,3; 1,7; 1,1; 0,3; 2,9; 2,9; 2,9; 1,1; 2,9; 1,7. Bu tanlanmaning statistik taqsimotin toping.
\\
\textbf{A2.} 
Hajmi \(n = 20\) ga teng bo'lgan tanlanma berilgan: 4,6; 2,5; 4,6; 3,3; 1,8; 0,3; 1,8; 4,6; 2,1; 1,8; 0,3; 2,1; 1,8; 0,3; 2,5; 2,5; 2,5; 1,8; 2,5; 2,1. Bu tanlanmaning empirik taqsimot funksiyasin toping.
\\
\textbf{A3.} 
Oliy matematika fanidan 10 ta talaba test topshiriqlarin topshirdi. Harbir talaba 10 balgacha to'plashi mumkin. Agar test topshiriqlari natijalari bo'yicha \{5, 7, 5, 9, 5, 8, 10, 6, 7, 8\} tanlanma olingan bo'lsa, ushbu tanlanmalarning tanlanma o'rta va tanlanma dispersiyalarin toping.
\\
\textbf{B1.} 
Agar o'rta kvadratik chetlanish \(\sigma = 1\) bo'lgan normal taqsimot bosh to'plamdan olingan hajmi \(n = 15\)ga teng tanlanma bo'yicha \(\overline{x} = 5,8\) tanlanma o'rta qiymati topilgan bo'lsa, u holda \(\gamma = 0,90\) ishonchlilik bilan noma'lum \(\theta\) matematik kutilma uchun ishonchlilik intervalin tuzing .
\\
\textbf{B2.} 
Agar (0,-2,0,-2,3,-2,0,0,3,0,0,0,0,3,-2,0,0,-2,3,0,3) tanlanma quyida berilgan taqsimotdan olingan bo'lsa, u holda noma'lum \(\theta\) parametr uchun momentlar usuli bahosini toping.
$\begin{array}{|c|c|c|c|}
    \hline
    \xi & - 2 & 0 & 3 \\
    \hline
    P_{\theta} & \theta & 1 - 2\theta & \theta \\
    \hline
\end{array}$
\\
\textbf{B3.} 
Agar \(X^{(n)} = \left( X_{1},...,X_{n} \right)\) tanlanma zichlik funksiyasi \(f(x;\theta) = \frac{2x}{\theta}e^{- \frac{x^{2}}{\theta}},\ \ x \geq 0\) bo'lgan taqsimotdan olingan bo'lsa, u holda noma'lum \(\theta > 0\) parametrning haqiqatga maksimal usuli bahosini toping.
\\
\textbf{C1.} 
Agar \(X^{(n)} = \left( X_{1},...,X_{n} \right)\) tanlanma \(\lbrack - 3\theta,\theta\rbrack\) oraliqda tekis taqsimotdan olingan bo'lsa, u holda noma'lum \(\theta\) parametr uchun \(4X_{(n)} + X_{(1)}\) bahoni siljimaganligi va asosliligini tekshiring.
\\
\textbf{C2.} 
Agar \(X^{(n)} = \left( X_{1},...,X_{n} \right)\) tanlanma \((\theta,\theta^{2})\ \ \) parametrli normal taqsimotdan olingan bo'lsa, u holda noma'lum \(\theta > 0\) parametr uchun momentlar usuli bahosini toping.
\\
\textbf{C3.} 
Agar \(X^{(n)} = \left( X_{1},...,X_{n} \right)\) tanlanma zichlik funksiyasi\(f(x;\theta) = \left\{ \begin{matrix}
e^{\theta - x},\ \ x \geq \theta, \\
\ \ 0,\ \ x < \theta
\end{matrix} \right.\ \) bo'lgan taqsimotdan olingan bo'lsa, u holda noma'lum \(\theta\) parametrning haqiqatga maksimal o'xshashlik bahosini toping.
\\

\end{tabular}
\vspace{1cm}


\begin{tabular}{m{17cm}}
\textbf{33-variant}
\newline

\textbf{T1.} 
Momentler usuli. (tanlanma momentleri, noma'lum parametrlarni baholash).
\\
\textbf{T2.} 
Chiziqli korrelyatsiya tenglamasi (ta'rifi, regressiya to'g'ri chiziqning tanlanma tenglamalari)
\\
\textbf{A1.} 
Hajmi \(n = 20\) ga teng bo'lgan tanlanma berilgan: -1,3; 0; 0,8; 2,3; 1,1; 0,8; 0,8; 2,3; 1,1; 0,8; -1,3; 1,8; 1,1; -1,3; 1,1; 1,8; 1,8; 1,1; 1,8; 1,8. Bu tanlanmaning statistik taqsimotin toping.
\\
\textbf{A2.} 
Hajmi \(n = 20\) ga teng bo'lgan tanlanma berilgan: -1,9; 0,7; 0,9; 2,8; 1,3; 0,9; 0,9; 2,8; 1,3; 0,9; -1,9; 1,6; 1,3; -1,9; 1,3; 1,6; 1,6; 1,3; 1,6; 1,6. Bu tanlanmaning empirik taqsimot funksiyasin toping.
\\
\textbf{A3.} 
Oliy matematika fanidan 10 ta talaba test topshiriqlarin topshirdi. Harbir talaba 10 balgacha to'plashi mumkin. Agar test topshiriqlari natijalari bo'yicha \{8, 4, 3, 7, 3, 6, 5, 3, 5, 6\} tanlanma olingan bo'lsa, ushbu tanlanmalarning tanlanma o'rta va tanlanma dispersiyalarin toping.
\\
\textbf{B1.} 
Agar normal taqsimlangan bosh to'plamdan olingan hajmi \(n = 20\)ga teng tanlanma bo'yicha \(\overline{x} = 16,6\) tanlanma o'rta va \({\overline{S}}^{2} = 0,64\) tuzatilgan tanlanma dispersiyalar topilgan bo'lsa, u holda \(\gamma = 0,95\) ishonchlilik bilan noma'lum \(\theta\) matematik kutilma uchun ishonchlilik intervalin tuzing.
\\
\textbf{B2.} 
Ko'rsatkichli taqsimot noma'lum \(\theta > 0\) parametri momentlar usuli bahosini toping.
\\
\textbf{B3.} 
Agar (-1,-1,0,-1,0,-1,-1,5,-1,0,-1,0,5,-1,-1,-1,5,-1,-1,-1,1,-1,5,0,-1,-1,5) tanlanma quyida berilgan taqsimotdan olingan bo'lsa, u holda noma'lum \(\theta\) parametrning haqiqatga maksimal o'xshashlik usuli bahosini toping.
$\begin{array}{|c|c|c|c|}
    \hline
    \xi & - 1 & 0 & 5\\
    \hline
    P_{\theta} & 1 - \theta & \theta/2 & \theta/2 \\
    \hline
\end{array}$
\\
\textbf{C1.} 
Agar \(X^{(n)} = \left( X_{1},...,X_{n} \right)\) tanlanma taqsimot funksiyasi \(F(x)\) bo'lgan taqsimotdan olingan bo'lsa, u holda noma'lum \(F(x)\) uchun \(F_{n}(x)\) empirik taqsimot funksiyasining siljimaganligi va asosliligini tekshiring.
\\
\textbf{C2.} 
Agar \(X^{(n)} = \left( X_{1},...,X_{n} \right)\) tanlanma zichlik funksiyasi\(f(x,\theta) = \frac{2x}{\theta^{2}},x \in \lbrack 0,\theta\rbrack\)bo'lgan taqsimotdan olingan bo'lsa, u holda noma'lum \(\theta\) parametr uchun momentlar usuli bahosini toping.
\\
\textbf{C3.} 
Agar \(X^{(n)} = \left( X_{1},...,X_{n} \right)\) tanlanma zichlik funksiyasi \(f(x;\theta) = \frac{\theta}{\sqrt{2\pi x^{3}}}e^{- \theta^{2}/2x},\ \ x \geq 0\) bo'lgan taqsimotdan olingan bo'lsa, u holda noma'lum \(\theta > 0\) parametrning haqiqatga maksimal o'xshashlik bahosini toping.
\\

\end{tabular}
\vspace{1cm}


\begin{tabular}{m{17cm}}
\textbf{34-variant}
\newline

\textbf{T1.} 
Neyman-Pirson teoremasi.
\\
\textbf{T2.} 
Statistik gipotezalarni tekshirish (kritik to'plam, 1 va 2-tur xatolik).
\\
\textbf{A1.} 
Hajmi \(n = 20\) ga teng bo'lgan tanlanma berilgan: -2,4; 5,6; 5,6; -5,2; -6,7; 5,1; -5,2; -2,4; 4,3; 5,1; -6,7; 4,3; -2,4; -6,7; 4,3; 5,1; 4,3; 5,6; -6,7; 5,6. Bu tanlanmaning statistik taqsimotin toping.
\\
\textbf{A2.} 
Hajmi \(n = 20\) ga teng bo'lgan tanlanma berilgan: -2,9; 7,6; 7,6; -5,7; -6,1; 5,5; -5,7; -2,9; 4,2; 5,5; -6,1; 4,2; -2,9; -6,1; 4,2; 5,5; 4,2; 7,6; -6,1; 7,6. Bu tanlanmaning empirik taqsimot funksiyasin toping.
\\
\textbf{A3.} 
Oliy matematika fanidan 10 ta talaba test topshiriqlarin topshirdi. Harbir talaba 10 balgacha to'plashi mumkin. Agar test topshiriqlari natijalari bo'yicha \{9, 8, 6, 7, 5, 8, 5, 7, 4, 6\} tanlanma olingan bo'lsa, ushbu tanlanmalarning tanlanma o'rta va tanlanma dispersiyalarin toping.
\\
\textbf{B1.} 
Agar normal taqsimlangan bosh to'plamdan olingan hajmi \(n = 13\) ga teng bo'lgan tanlanma bo'yicha \({\overline{S}}^{2} = 1,2\) tuzatilgan tanlanma dispersiya topilgan bo'lsa, u holda \(\gamma = 0,90\) ishonchlilik bilan noma'lum \(\theta_{2}^{2}\) dispersiya uchun ishonchlilik intervalin tuzing.
\\
\textbf{B2.} 
Agar (0,-2,0,-2,3,-2,0,0,3,0,0,0,0,3,-2,0,0,-2,3,0,3) tanlanma quyida berilgan taqsimotdan olingan bo'lsa, u holda noma'lum \(\theta\) parametr uchun momentlar usuli bahosini toping.
$\begin{array}{|c|c|c|c|}
    \hline
    \xi & - 2 & 0 & 3 \\
    \hline
    P_{\theta} & \theta & 1 - 2\theta & \theta \\
    \hline
\end{array}$
\\
\textbf{B3.} 
Agar (4,8,5,3) tanlanma \((a,\theta^{2}\) parametrli normal taqsimotdan olingan bo'lsa, u holda noma'lum \(\theta^{2}\) parametrning haqiqatga maksimal o'xshashlik bahosini toping.
\\
\textbf{C1.} 
Agar \(X^{(n)} = \left( X_{1},...,X_{n} \right)\) tanlanma \(\left( a,\theta^{2} \right)\) parametrli normal taqsimotdan olingan bo'lsa (\(a -\) ma'lum), u holda noma'lum \(\theta\) parametr uchun \(\sqrt{\frac{\pi}{2}}\left| \overline{x - a} \right|\) bahoning siljimaganligi va asosliligini tekshiring.
\\
\textbf{C2.} 
Agar \(X^{(n)} = \left( X_{1},...,X_{n} \right)\) tanlanma \({\lbrack\theta}_{1},\theta_{1} + \theta_{2}\rbrack\) oraliqda tekis taqsimotdan olingan bo'lsa, u holda noma'lum \(\left( \theta_{1},\theta_{2} \right)\) vektor parametr uchun momentlar usuli bahosini toping.
\\
\textbf{C3.} 
Agar \(X^{(n)} = \left( X_{1},...,X_{n} \right)\) tanlanma zichlik funksiyasi\(f(x;\theta) = \frac{4x^{3}}{\sqrt{2\pi}\theta_{2}}\exp\left\{ - \frac{\left( x^{4} - \theta_{1} \right)^{2}}{2{\theta_{2}}^{2}} \right\},\ \ x \in R\) bo'lgan taqsimotdan olingan bo'lsa, u holda noma'lum \(\left( \theta_{1},\theta_{2}^{2} \right)\) vektor parametrning haqiqatga maksimal o'xshashlik usuli baholarini toping.
\\

\end{tabular}
\vspace{1cm}


\begin{tabular}{m{17cm}}
\textbf{35-variant}
\newline

\textbf{T1.} 
Poligon va gistogramma(nisbiy chastota, interval qator, grafik).
\\
\textbf{T2.} 
Ishonchlilik intervallarin tuzish. Aniq ishonchlilik intervallar.
\\
\textbf{A1.} 
Hajmi \(n = 20\) ga teng bo'lgan tanlanma berilgan:-3,3; 0; 4,4; 2,2; -2,7; 4,4; 2,2; 4,4;-3,3; 2,2; -2,7; 2,2; -3,3; -2,7; 2,2; 3,4; 4,4; 0; -3,3; 0. Bu tanlanmaning statistik taqsimotin toping.
\\
\textbf{A2.} 
Hajmi \(n = 20\) ga teng bo'lgan tanlanma berilgan:-3,3; 0; 4,9; 2,8; -2,6; 4,9; 2,8; 4,9;-3,3; 2,8; -2,6; 2,8; -3,3; -2,6; 2,8; 3,1; 4,9; 0; -3,3; 0. Bu tanlanmaning empirik taqsimot funksiyasin toping.
\\
\textbf{A3.} 
Oliy matematika fanidan 10 ta talaba test topshiriqlarin topshirdi. Harbir talaba 10 balgacha to'plashi mumkin. Agar test topshiriqlari natijalari bo'yicha \{4, 7, 6, 9, 3, 8, 3, 7, 4, 9\} tanlanma olingan bo'lsa, ushbu tanlanmalarning tanlanma o'rta va tanlanma dispersiyalarin toping.
\\
\textbf{B1.} 
Agar o'rta kvadratik chetlanish \(\sigma = 4\) bo'lgan normal taqsimot bosh to'plamdan olingan hajmi \(n = 12\)ga teng tanlanma bo'yicha \(\overline{x} = 3\) tanlanma o'rta qiymati topilgan bo'lsa, u holda \(\gamma = 0,95\) ishonchlilik bilan noma'lum \(\theta\) matematik kutilma uchun ishonchlilik intervalin tuzing .
\\
\textbf{B2.} 
Agar \(X^{(n)} = \left( X_{1},...,X_{n} \right)\) tanlanma \(\theta\) parametrli ko'rsatkichli taqsimotdan olingan bo'lsa, u holda noma'lum \(\theta\) parametr uchun momentlar usuli bahosini toping.
\\
\textbf{B3.} 
Agar \(x_{1} = 1,1;\ \ x_{2} = 2,7;\ldots;x_{100} = 1,5\) tanlanma \(\theta\) parametrli ko'rsatkichli taqsimotdan olingan bo'lib, \(\sum_{k = 1}^{100}x_{k} = 200\) bo'lsa, u holda noma'lum \(\theta\) parametrning haqiqatga maksimal o'xshashlik bahosini toping.
\\
\textbf{C1.} 
Agar \(X^{(n)} = \left( X_{1},...,X_{n} \right)\) tanlanma \(\theta\) parametrli ko'rsatkichli taqsimotdan olingan bo'lsa, u holda noma'lum \(\theta\) parametr uchun \(1/\overline{x}\) bahoning siljimaganligi va asosliligini tekshiring.
\\
\textbf{C2.} 
Agar \(X^{(n)} = \left( X_{1},...,X_{n} \right)\) tanlanma \(\ \ (a,\theta^{2})\ \ \) parametrli normal taqsimotdan olingan bo'lsa (\(\alpha -\) ma'lum), u holda noma'lum \(\ \ \theta^{2}\) parametr uchun momentlar usuli bahosini \(\ \ g(x) = (x - a)^{2}\) funksiyasi yordamida toping.
\\
\textbf{C3.} 
Agar \(X^{(n)} = \left( X_{1},...,X_{n} \right)\) tanlanma zichlik funksiyasi \(f(x;\theta) = \frac{3x^{2}}{\sqrt{2\pi}}\exp\left\{ - \frac{\left( x^{3} - \theta \right)^{2}}{2} \right\},\ \ x \in R\) bo'lgan taqsimotdan olingan bo'lsa, u holda noma'lum \(\theta\) parametrning haqiqatga maksimal o'xshashlik bahosini toping.
\\

\end{tabular}
\vspace{1cm}


\begin{tabular}{m{17cm}}
\textbf{36-variant}
\newline

\textbf{T1.} 
Empirik taqsimot funktsiyasi. (Tanlanma, eksperiment)
\\
\textbf{T2.} 
Kolmogorovning muvofiqlik kritireyesi (Kolmogorov teoremasi)
\\
\textbf{A1.} 
Hajmi \(n = 20\) ga teng bo'lgan tanlanma berilgan: 3,7; 3,1; 4,8; 2,8; 3,1; 4,3; 3,7; 4,3; 2,4; 3,1; 2,4; 4,3; 3,1; 3,7; 4,8; 2,8; 2,4; 2,8; 2,4; 3,1. Bu tanlanmaning statistik taqsimotin toping.
\\
\textbf{A2.} 
Hajmi \(n = 20\) ga teng bo'lgan tanlanma berilgan: 3,8; 3,4; 4,8; 2,9; 3,4; 4,6; 3,8; 4,6; 2,1; 3,4; 2,1; 4,6; 3,4; 3,8; 4,8; 2,9; 2,1; 2,9; 2,1; 3,4. Bu tanlanmaning empirik taqsimot funksiyasin toping.
\\
\textbf{A3.} 
Oliy matematika fanidan 10 ta talaba test topshiriqlarin topshirdi. Harbir talaba 10 balgacha to'plashi mumkin. Agar test topshiriqlari natijalari bo'yicha \{6, 5, 6, 9, 5, 7, 10, 5, 9, 8\} tanlanma olingan bo'lsa, ushbu tanlanmalarning tanlanma o'rta va tanlanma dispersiyalarin toping.
\\
\textbf{B1.} 
Agar normal taqsimlangan bosh to'plamdan olingan hajmi \(n = 25\)ga teng tanlanma bo'yicha \(\overline{x} = 9\) tanlanma o'rta va \({\overline{S}}^{2} = 0,64\) tuzatilgan tanlanma dispersiyalar topilgan bo'lsa, u holda \(\gamma = 0,95\) ishonchlilik bilan noma'lum \(\theta\) matematik kutilma uchun ishonchlilik intervalin tuzing.
\\
\textbf{B2.} 
\(\left\lbrack \theta_{1},\theta_{2} \right\rbrack\) oraliqda tekis taqsimot parametrlari uchun momentlar usuli baholarini toping.
\\
\textbf{B3.} 
\(f(x) = \frac{\theta}{2}e^{- \theta|x|}\) model uchun \(\theta\) parametri haqiqatga maksimal o'xshashlik usuli bahosi topilsin.
\\
\textbf{C1.} 
Agar \(X^{(n)} = \left( X_{1},...,X_{n} \right)\) tanlanma \(1\sqrt{\theta}\) parametrli ko'rsatkichli taqsimotdan olingan bo'lsa, u holda noma'lum \(\theta\) parametr uchun \((\overline{x})^{2}\) bahoning siljimaganligi va asosliligini tekshiring.
\\
\textbf{C2.} 
Agar \(X^{(n)} = \left( X_{1},...,X_{n} \right)\) tanlanma\(\ \ (a,\theta^{2})\) parametrli normal taqsimotdan olingan bo'lsa (\(\alpha -\) ma'lum), u holda noma'lum\(\ \ \theta^{2}\) parametr uchun momentlar usuli bahosini toping.
\\
\textbf{C3.} 
Agar \(X^{(n)} = \left( X_{1},...,X_{n} \right)\) tanlanma zichlik funksiyasi \(f(x;\theta) = \frac{e^{x}}{\sqrt{2\pi}}\exp\left\{ - \frac{\left( e^{x} - \theta \right)^{2}}{2} \right\},\ \ x \in R\) bo'lgan taqsimotdan olingan bo'lsa, u holda noma'lum \(\theta\) parametrning haqiqatga maksimal o'xshashlik bahosini toping.
\\

\end{tabular}
\vspace{1cm}


\begin{tabular}{m{17cm}}
\textbf{37-variant}
\newline

\textbf{T1.} 
Glivenko-Kantelli teoremasi. (empirik taqsimot funktsiyasi, ehtimollik bilan yaqinlashish).
\\
\textbf{T2.} 
Momentler usuli. (tanlanma momentleri, noma'lum parametrlarni baholash).
\\
\textbf{A1.} 
Hajmi \(n = 20\) ga teng bo'lgan tanlanma berilgan: 1,5; -0,9; -2,4; -0,9; 0,7; 1,5; -0,9; -0,2; -2,4; 0,7; -2,4; 0,7; -0,9; 1,5; -1,7; -0,9; -0,2; 0,7; -1,7; -0,9. Bu tanlanmaning statistik taqsimotin toping.
\\
\textbf{A2.} 
Hajmi \(n = 20\) ga teng bo'lgan tanlanma berilgan: 1,9; -0,3; -2,7; -0,3; 0,6; 1,9; -0,3; -0,1; -2,7; 0,6; -2,7; 0,6; -0,3; 1,9; -1,8; -0,3; -0,1; 0,6; -1,8; -0,3. Bu tanlanmaning empirik taqsimot funksiyasin toping.
\\
\textbf{A3.} 
Oliy matematika fanidan 10 ta talaba test topshiriqlarin topshirdi. Harbir talaba 10 balgacha to'plashi mumkin. Agar test topshiriqlari natijalari bo'yicha \{4, 6, 6, 9, 5, 8, 4, 7, 5, 6\} tanlanma olingan bo'lsa, ushbu tanlanmalarning tanlanma o'rta va tanlanma dispersiyalarin toping.
\\
\textbf{B1.} 
Agar normal taqsimlangan bosh to'plamdan olingan hajmi \(n = 10\) ga teng bo'lgan tanlanma bo'yicha \({\overline{S}}^{2} = 0,6\) tuzatilgan tanlanma dispersiya topilgan bo'lsa, u holda \(\gamma = 0,95\) ishonchlilik bilan noma'lum \(\theta_{2}^{2}\) dispersiya uchun ishonchlilik intervalin tuzing.
\\
\textbf{B2.} 
\(\lbrack 0,\theta\rbrack\) oraliqda tekis taqsimlangan \(\theta\) parametri uchun momentlar usuli bahosini toping.
\\
\textbf{B3.} 
Agar \(X^{(n)} = \left( X_{1},...,X_{n} \right)\) tanlanma \(\theta\) parametrli Bernulli taqsimotidan olingan bo'lsa, u holda noma'lum \(\theta\) parametrning haqiqatga maksimal o'xshashlik usuli bahosini toping.
\\
\textbf{C1.} 
Agar \(X^{(n)} = \left( X_{1},...,X_{n} \right)\) tanlanma \(\sqrt{\theta}\) parametrli Bernulli taqsimotidan olingan bo'lsa, u holda noma'lum \(\theta\) parametr uchun \((\overline{x})^{2}\) bahoning siljimaganligi va asosliligini tekshiring.
\\
\textbf{C2.} 
Agar \(X^{(n)} = \left( X_{1},...,X_{n} \right)\) tanlanma \(\ \ (a,\theta^{2})\ \ \) parametrli normal taqsimotdan olingan bo'lsa (\(\alpha -\) ma'lum), u holda noma'lum \(\ \ \theta^{2}\) parametr uchun momentlar usuli bahosini \(\ \ g(x) = (x - a)^{2}\) funksiyasi yordamida toping.
\\
\textbf{C3.} 
Agar \(X^{(n)} = \left( X_{1},...,X_{n} \right)\) tanlanma zichlik funksiyasi \(f(x;\theta) = \frac{\theta}{\sqrt{2\pi x^{3}}}e^{- \theta^{2}/2x},\ \ x \geq 0\) bo'lgan taqsimotdan olingan bo'lsa, u holda noma'lum \(\theta > 0\) parametrning haqiqatga maksimal o'xshashlik bahosini toping.
\\

\end{tabular}
\vspace{1cm}


\begin{tabular}{m{17cm}}
\textbf{38-variant}
\newline

\textbf{T1.} 
Guruhlangan va interval variatsion qatorlar.
\\
\textbf{T2.} 
Normal qonun dispersiyasi uchun ishonchlilik intervalin tuzish. (Ishonchlilik ehtimolligi, interval)
\\
\textbf{A1.} 
Hajmi \(n = 20\) ga teng bo'lgan tanlanma berilgan:9,4; 6,8; -8,5; 9,4; 2,9; 9,4; -8,5; -6,4; 6,8; -8,5; 9,4; -6,4; 6,8; 9,4; 2,9; 9,4; -3,6; -8,5; 2,9; -6,4. Bu tanlanmaning statistik taqsimotin toping.
\\
\textbf{A2.} 
Hajmi \(n = 20\) ga teng bo'lgan tanlanma berilgan:9,1; 6,4; -8,6; 9,1; 2,3; 9,1; -8,6; -6,2; 6,4; -8,6; 9,1; -6,2; 6,4; 9,1; 2,3; 9,1; -3,9; -8,6; 2,3; -6,2. Bu tanlanmaning empirik taqsimot funksiyasin toping.
\\
\textbf{A3.} 
Oliy matematika fanidan 10 ta talaba test topshiriqlarin topshirdi. Harbir talaba 10 balgacha to'plashi mumkin. Agar test topshiriqlari natijalari bo'yicha \{3, 7, 6, 4, 5, 4, 3, 7, 8, 3\} tanlanma olingan bo'lsa, ushbu tanlanmalarning tanlanma o'rta va tanlanma dispersiyalarin toping.
\\
\textbf{B1.} 
Agar o'rta kvadratik chetlanish \(\sigma = 5\) bo'lgan normal taqsimot bosh to'plamdan olingan hajmi \(n = 16\)ga teng tanlanma bo'yicha \(\overline{x} = 3,6\) tanlanma o'rta qiymati topilgan bo'lsa, u holda \(\gamma = 0,90\) ishonchlilik bilan noma'lum \(\theta\) matematik kutilma uchun ishonchlilik intervalin tuzing .
\\
\textbf{B2.} 
Agar (3,-2,-2,0,-2,2,-2,0,-2,3,-2,0,3,0,3,-2,0,-2,3,-2,2,-2,-2,3,3,2,-2,2,3,3) tanlanma quyida berilgan taqsimotdan olingan bo'lsa, u holda noma'lum \(\theta\) parametr uchun momentlar usuli bahosini \(g(x) = |x|\) funksiya yordamida toping.
$\begin{array}{|c|c|c|c|}
    \hline
    \xi & -2 & 0 & 3 \\
    \hline
    P_{\theta} & 3\theta & 1 - 5\theta & 2\theta \\
    \hline
\end{array}$
\\
\textbf{B3.} 
Agar (0,1,2,0) tanlanma quyida berilgan taqsimotdan olingan bo'lsa, u holda noma'lum \(\theta\) parametrning haqiqatga maksimal o'xshashlik bahosini toping.
$\begin{array}{|c|c|c|c|}
    \hline
    \xi & 0 & 1 & 2 \\
    \hline
    P_{\theta} & \theta & 2\theta & 1 - 3\theta \\
    \hline
\end{array}$
\\
\textbf{C1.} 
Agar \(X^{(n)} = \left( X_{1},...,X_{n} \right)\) tanlanma \(\theta\) parametrli Bernulli taqsimotidan olingan bo'lsa, u holda noma'lum \(\theta\) parametr uchun \(X_{n}\) bahoning siljimaganligi va asosliligini tekshiring.
\\
\textbf{C2.} 
Agar \(X^{(n)} = \left( X_{1},...,X_{n} \right)\) tanlanma zichlik funksiyasi\(f(x,\theta) = \frac{2x}{\theta^{2}},x \in \lbrack 0,\theta\rbrack\)bo'lgan taqsimotdan olingan bo'lsa, u holda noma'lum \(\theta\) parametr uchun momentlar usuli bahosini toping.
\\
\textbf{C3.} 
Agar \(X^{(n)} = \left( X_{1},...,X_{n} \right)\) tanlanma zichlik funksiyasi\(f(x;\theta) = \frac{4x^{3}}{\sqrt{2\pi}\theta_{2}}\exp\left\{ - \frac{\left( x^{4} - \theta_{1} \right)^{2}}{2{\theta_{2}}^{2}} \right\},\ \ x \in R\) bo'lgan taqsimotdan olingan bo'lsa, u holda noma'lum \(\left( \theta_{1},\theta_{2}^{2} \right)\) vektor parametrning haqiqatga maksimal o'xshashlik usuli baholarini toping.
\\

\end{tabular}
\vspace{1cm}


\begin{tabular}{m{17cm}}
\textbf{39-variant}
\newline

\textbf{T1.} Matematik statistikaning asosiy masalalari. (Statistik ma'lumotlar, guruhlash)
\\
\textbf{T2.} 
Haqiqatga maksimal o'xshashlik usuli. (haqiqatga maksimal o'xshashlik funktsiyasi, noma'lum parametrlarni baholash).
\\
\textbf{A1.} 
Hajmi \(n = 20\) ga teng bo'lgan tanlanma berilgan: 6,2; -5,3; 7,2; 3,7; -2,2; 6,2; 3,7; -7,6; 3,7; 7,2; 6,2; -5,3; -7,6; -5,3; -7,6; 6,2; 7,2; -2,2; -7,6; 7,2. Bu tanlanmaning statistik taqsimotin toping.
\\
\textbf{A2.} 
Hajmi \(n = 20\) ga teng bo'lgan tanlanma berilgan: 6,1; -5,8; 7,9; 3,5; -2,5; 6,1; 3,5; -7,2; 3,5; 7,9; 6,1; -5,8; -7,2; -5,8; -7,2; 6,1; 7,9; -2,5; -7,2; 7,9. Bu tanlanmaning empirik taqsimot funksiyasin toping.
\\
\textbf{A3.} 
Oliy matematika fanidan 10 ta talaba test topshiriqlarin topshirdi. Harbir talaba 10 balgacha to'plashi mumkin. Agar test topshiriqlari natijalari bo'yicha \{10, 8, 6, 5, 4, 8, 10, 7, 5, 7\} tanlanma olingan bo'lsa, ushbu tanlanmalarning tanlanma o'rta va tanlanma dispersiyalarin toping.
\\
\textbf{B1.} 
Agar normal taqsimlangan bosh to'plamdan olingan hajmi \(n = 16\)ga teng tanlanma bo'yicha \(\overline{x} = 15,2\) tanlanma o'rta va \({\overline{S}}^{2} = 0,81\) tuzatilgan tanlanma dispersiyalar topilgan bo'lsa, u holda \(\gamma = 0,90\) ishonchlilik bilan noma'lum \(\theta\) matematik kutilma uchun ishonchlilik intervalin tuzing.
\\
\textbf{B2.} 
Agar (3,0,-2,0,-2,3,-2,0,0,3,0,0,0,0,3,-2,0,0,-2,3,0) tanlanma quyida berilgan taqsimotdan olingan bo'lsa, u holda noma'lum \(\left( \theta_{1},\theta_{2} \right)\) vektor parametr uchun momentlar usuli bahosini toping.
$\begin{array}{|c|c|c|c|}
    \hline
    \xi & - 2 & 0 & 3 \\
    \hline
    P_{\theta} & 2\theta_{1} & 0,5 + \theta_{1} + \theta_{2} & \theta_{2} \\
    \hline
\end{array}$
\\
\textbf{B3.} 
\(f(x) = \frac{\theta}{2}e^{- \theta|x|}\) model uchun \(\theta\) parametri haqiqatga maksimal o'xshashlik usuli bahosi topilsin.
\\
\textbf{C1.} 
Agar \(X^{(n)} = \left( X_{1},...,X_{n} \right)\) tanlanma \(\theta\) parametrli Bernulli taqsimotidan olingan bo'lsa, u holda noma'lum \(\theta(1 - \theta)\) parametr uchun \(X_{1}\left( 1 - X_{n} \right)\) bahoning siljimaganligi va asosliligini tekshiring.
\\
\textbf{C2.} 
Agar \(X^{(n)} = \left( X_{1},...,X_{n} \right)\) tanlanma zichlik funksiyasi\(f(x,\theta) = \left\{ \begin{matrix}
\theta_{1}^{- 1}e^{- \ \frac{x - \theta_{2}}{\theta_{1}}},\ \ x \geq \theta_{2}, \\
0,\ \ x < \theta_{2}
\end{matrix} \right.\ \)bo'lgan taqsimotdan olingan bo'lsa, u holda noma'lum \(\left( \theta_{1},\theta_{2} \right)\) \(\theta_{1} > 0,\) \(\theta_{2} \in R\) vektor parametr uchun momentlar usuli bahosini toping.
\\
\textbf{C3.} 
Agar \(X^{(n)} = \left( X_{1},...,X_{n} \right)\) tanlanma zichlik funksiyasi \(f(x;\theta) = \frac{\theta ln^{\theta - 1}x}{x},\ \ x \in \lbrack 1,e\rbrack\) bo'lgan taqsimotdan olingan bo'lsa, u holda noma'lum \(\theta > 0\) parametr uchun haqiqatga maksimal o'xshashlik bahosini toping.
\\

\end{tabular}
\vspace{1cm}


\begin{tabular}{m{17cm}}
\textbf{40-variant}
\newline

\textbf{T1.} 
Tanlanma xarakteristikalar. (Variatsion qator, nisbiy chastota).
\\
\textbf{T2.} 
Pirsonning xi-kvadrat muvofiqlik kriteriysi (Pirson teoremasi).
\\
\textbf{A1.} 
Hajmi \(n = 20\) ga teng bo'lgan tanlanma berilgan: 9,6; 1,5; 7,4; 9,6; 2,8; 1,5; 6,3; 1,5; 9,6; 6,3; 2,8; 4,1; 6,3; 9,6; 1,5; 1,5; 6,3; 7,4; 4,1; 7,4. Bu tanlanmaning statistik taqsimotin toping.
\\
\textbf{A2.} 
Hajmi \(n = 20\) ga teng bo'lgan tanlanma berilgan: 9,8; 1,2; 7,1; 9,8; 2,9; 1,2; 6,7; 1,2; 9,8; 6,7; 2,9; 4,6; 6,7; 9,8; 1,2; 1,2; 6,7; 7,1; 4,6; 7,1. Bu tanlanmaning empirik taqsimot funksiyasin toping.
\\
\textbf{A3.} 
Oliy matematika fanidan 10 ta talaba test topshiriqlarin topshirdi. Harbir talaba 10 balgacha to'plashi mumkin. Agar test topshiriqlari natijalari bo'yicha \{9, 10, 5, 6, 4, 8, 4, 6, 10, 8\} tanlanma olingan bo'lsa, ushbu tanlanmalarning tanlanma o'rta va tanlanma dispersiyalarin toping.
\\
\textbf{B1.} 
Agar normal taqsimlangan bosh to'plamdan olingan hajmi \(n = 10\) ga teng bo'lgan tanlanma bo'yicha \({\overline{S}}^{2} = 0,45\) tuzatilgan tanlanma dispersiya topilgan bo'lsa, u holda \(\gamma = 0,95\) ishonchlilik bilan noma'lum \(\theta_{2}^{2}\) dispersiya uchun ishonchlilik intervalin tuzing.
\\
\textbf{B2.} 
Agar (-2,0,-2,0,-2,3,-2,0,0,3,0,0,0,0,3,-2,0,0,-2,3,0) tanlanma quyida berilgan taqsimotdan olingan bo'lsa, u holda noma'lum \(\left( \theta_{1},\theta_{2} \right)\) vektor parametr uchun momentlar usuli bahosini toping.
$\begin{array}{|c|c|c|c|}
    \hline
    \xi & - 2 & 0 & 3\\
    \hline
    P_{\theta} & \theta_{1} & 1 - \theta_{1} - \theta_{2} & \theta_{2} \\
    \hline
\end{array}$
\\
\textbf{B3.} 
Agar \(X^{(n)} = \left( X_{1},...,X_{n} \right)\) tanlanma \(\lbrack - \theta,\theta\rbrack\) oraliqda tekis taqsimotdan olingan bo'lsa, u holda noma'lum \(\theta > 0\) parametrning haqiqatga maksimal o'xshashlik usuli bahosini toping.
\\
\textbf{C1.} 
Agar \(X^{(n)} = \left( X_{1},...,X_{n} \right)\) tanlanma \(\theta\) parametrli Bernulli taqsimotidan olingan bo'lsa, u holda noma'lum \(\theta^{2}\) parametr uchun \(X_{1}X_{n}\) bahoning siljimaganligi va asosliligini tekshiring.
\\
\textbf{C2.} 
Agar \(X^{(n)} = \left( X_{1},...,X_{n} \right)\) tanlanma \({\lbrack\theta}_{1},\theta_{1} + \theta_{2}\rbrack\) oraliqda tekis taqsimotdan olingan bo'lsa, u holda noma'lum \(\left( \theta_{1},\theta_{2} \right)\) vektor parametr uchun momentlar usuli bahosini toping.
\\
\textbf{C3.} 
Agar \(X^{(n)} = \left( X_{1},...,X_{n} \right)\) tanlanma zichlik funksiyasi \(f(x;\theta) = \left\{ \begin{array}{r}
3x^{2}\theta^{- 3}{e^{- \left( \frac{x}{\theta} \right)}}^{3},\ \ \ \ x \geq 0 \\
0,\ \ \ \ \ \ \ \ \ \ \ x < 0
\end{array} \right.\ \) bo'lgan taqsimotdan olingan bo'lsa, u holda noma'lum \(\theta > 0\) parametrning haqiqatga maksimal o'xshashlik bahosini toping.
\\

\end{tabular}
\vspace{1cm}


\begin{tabular}{m{17cm}}
\textbf{41-variant}
\newline

\textbf{T1.} 
Neyman-Pirson teoremasi.
\\
\textbf{T2.} 
Statistik baho xossalari. (Siljimagan, asosliy, effektiv)
\\
\textbf{A1.} 
Hajmi \(n = 20\) ga teng bo'lgan tanlanma berilgan:1,8; -8,4; 7,3; 4,7; -3,9; 1,8; 4,7; -10,4; -8,4; 7,3; -10,4; 4,7; -8,4; 1,8; 4,7; -10,4; 7,3; -3,9; 4,7; -8,4. Bu tanlanmaning statistik taqsimotin toping.
\\
\textbf{A2.} 
Hajmi \(n = 20\) ga teng bo'lgan tanlanmaberilgan:1,6; -8,3; 7,6; 4,2; -3,1; 1,6; 4,2; -10,5; -8,3; 7,6; -10,5; 4,2; -8,3; 1,6; 4,2; -10,5; 7,6; -3,1; 4,2; -8,3. Bu tanlanmaning empirik taqsimot funksiyasin toping.
\\
\textbf{A3.} 
Oliy matematika fanidan 10 ta talaba test topshiriqlarin topshirdi. Harbir talaba 10 balgacha to'plashi mumkin. Agar test topshiriqlari natijalari bo'yicha \{9, 3, 6, 3, 7, 6, 4, 6, 10, 6\} tanlanma olingan bo'lsa, ushbu tanlanmalarning tanlanma o'rta va tanlanma dispersiyalarin toping.
\\
\textbf{B1.} 
Agar o'rta kvadratik chetlanish \(\sigma = 2\) bo'lgan normal taqsimot bosh to'plamdan olingan hajmi \(n = 18\)ga teng tanlanma bo'yicha \(\overline{x} = 5,2\) tanlanma o'rta qiymati topilgan bo'lsa, u holda \(\gamma = 0,90\) ishonchlilik bilan noma'lum \(\theta\) matematik kutilma uchun ishonchlilik intervalin tuzing .
\\
\textbf{B2.} 
Ko'rsatkichli taqsimot noma'lum \(\theta > 0\) parametri momentlar usuli bahosini toping.
\\
\textbf{B3.} 
\(f(x) = \frac{2x}{\theta}e^{- \frac{x^{2}}{\theta}},\ \ x \geq 0\) model uchun \(\theta\) parametri haqiqatga maksimal o'xshashlik usuli bahosi topilsin.
\\
\textbf{C1.} 
Agar \(X^{(n)} = \left( X_{1},...,X_{n} \right)\) tanlanma \((\alpha,\theta)\) parametrli Veybull taqsimotdan olingan bo'lsa (\(\alpha -\) ma'lum), u holda noma'lum \(\theta\) parametr uchun \(1/\overline{x^{\alpha}}\) bahoning siljimaganligi va asosliligini tekshiring.
\\
\textbf{C2.} 
Agar \(X^{(n)} = \left( X_{1},...,X_{n} \right)\) tanlanma zichlik funksiyasi\(f(x,\theta) = \left\{ \begin{matrix}
e^{\theta - x},\ \ x \geq \theta, \\
0,\ \ x < \theta
\end{matrix} \right.\ \)bo'lgan taqsimotdan olingan bo'lsa, u holda noma'lum \(\theta\) parametr uchun momentlar usuli bahosini toping.
\\
\textbf{C3.} 
\(f(x;\theta) = \frac{7x^{6}}{\sqrt{2\pi}}\exp\left\{ - \frac{(x^{7} - \theta)^{2}}{2} \right\}\) model uchun \(\theta\) parametri haqiqatga maksimal o'xshashlik usuli bahosi topilsin.
\\

\end{tabular}
\vspace{1cm}


\begin{tabular}{m{17cm}}
\textbf{42-variant}
\newline

\textbf{T1.} 
Poligon va gistogramma(nisbiy chastota, interval qator, grafik).
\\
\textbf{T2.} 
Haqiqatga maksimal o'xshashlik usuli. (haqiqatga maksimal o'xshashlik funktsiyasi, noma'lum parametrlarni baholash).
\\
\textbf{A1.} 
Hajmi \(n = 20\) ga teng bo'lgan tanlanma berilgan: 2,7; -13,5; 1,2; 2,7; 1,2; 4,9; -9,5; 1,2; 2,7; 4,9; -9,5; 2,7; -3,5; 1,2; 2,7; 4,9; -3,5; 2,7; 4,9; 1,2;. Bu tanlanmaning statistik taqsimotin toping.
\\
\textbf{A2.} 
Hajmi \(n = 20\) ga teng bo'lgan tanlanma berilgan: 2,8; -13,9; 1,9; 2,8; 1,9; 4,3; -9,4; 1,9; 2,8; 4,3; -9,4; 2,8; -3,7; 1,9; 2,8; 4,3; -3,7; 2,8; 4,3; 1,9. Bu tanlanmaning empirik taqsimot funksiyasin toping.
\\
\textbf{A3.} 
Oliy matematika fanidan 10 ta talaba test topshiriqlarin topshirdi. Harbir talaba 10 balgacha to'plashi mumkin. Agar test topshiriqlari natijalari bo'yicha \{10, 7, 5, 9, 3, 8, 10, 7, 8, 3\} tanlanma olingan bo'lsa, ushbu tanlanmalarning tanlanma o'rta va tanlanma dispersiyalarin toping.
\\
\textbf{B1.} 
Agar normal taqsimlangan bosh to'plamdan olingan hajmi \(n = 36\)ga teng tanlanma bo'yicha \(\overline{x} = 20,2\) tanlanma o'rta va \({\overline{S}}^{2} = 0,81\) tuzatilgan tanlanma dispersiyalar topilgan bo'lsa, u holda \(\gamma = 0,95\) ishonchlilik bilan noma'lum \(\theta\) matematik kutilma uchun ishonchlilik intervalin tuzing.
\\
\textbf{B2.} 
Agar zichlik funksiyasi \(f(x) = \frac{2x}{\theta}e^{- \frac{x^{2}}{\theta}},\ \ x \geq 0\) ko'rinishga ega bo'lsa, u holda \(\theta\) parametr momentlar usuli bahosini toping.
\\
\textbf{B3.} 
Agar (-1,-1,0,-1,0,-1,-1,5,-1,0,-1,0,5,-1,-1,-1,5,-1,-1,-1,1,-1,5,0,-1,-1,5) tanlanma quyida berilgan taqsimotdan olingan bo'lsa, u holda noma'lum \(\theta\) parametrning haqiqatga maksimal o'xshashlik usuli bahosini toping.
$\begin{array}{|c|c|c|c|}
    \hline
    \xi & - 1 & 0 & 5\\
    \hline
    P_{\theta} & 1 - \theta & \theta/2 & \theta/2 \\
    \hline
\end{array}$
\\
\textbf{C1.} 
Agar \(X^{(n)} = \left( X_{1},...,X_{n} \right)\) tanlanma \(\theta\) parametrli geometrik taqsimotdan olingan bo'lsa, u holda noma'lum \(\theta\) parametr uchun \(t(1 + \overline{x})\) bahoning siljimaganligi va asosliligini tekshiring.
\\
\textbf{C2.} 
Agar \(X^{(n)} = \left( X_{1},...,X_{n} \right)\) tanlanma \(\theta\) parametrli Puasson taqsimotidan olingan bo'lsa, u holda noma'lum \(\theta\) parametr uchun momentlar usuli bahosini toping.
\\
\textbf{C3.} 
Agar \(X^{(n)} = \left( X_{1},...,X_{n} \right)\) tanlanma zichlik funksiyasi \(f(x;\theta) = \left\{ \begin{array}{r}
\begin{matrix}
\theta_{1}^{- 1}e^{\frac{x - \theta_{2}}{\theta_{1}}},\ \ x \geq \theta_{2}
\end{matrix} \\
0,\ \ \ \ x < \theta_{2}
\end{array} \right.\ \) bo'lgan taqsimotdan olingan bo'lsa, u holda noma'lum \(.\left( \theta_{1},\theta_{2} \right),\) \(\theta_{1} > 0,\) \(\theta_{2} \in R\) vektor parametrning haqiqatga maksimal o'xshashlik bahosini toping.
\\

\end{tabular}
\vspace{1cm}


\begin{tabular}{m{17cm}}
\textbf{43-variant}
\newline

\textbf{T1.} 
Glivenko-Kantelli teoremasi. (empirik taqsimot funktsiyasi, ehtimollik bilan yaqinlashish).
\\
\textbf{T2.} 
Chiziqli korrelyatsiya tenglamasi (ta'rifi, regressiya to'g'ri chiziqning tanlanma tenglamalari)
\\
\textbf{A1.} 
Hajmi \(n = 20\) ga teng bo'lgan tanlanma berilgan: 9,9; 5,7; 3,2; 2,8; 5,7; 9,9; 7,5; 3,7; 9,9; 3,2; 2,8; 3,7; 7,5; 5,7; 3,2; 2,8; 7,5; 3,2; 9,9; 7,5. Bu tanlanmaning statistik taqsimotin toping.
\\
\textbf{A2.} 
Hajmi \(n = 20\) ga teng bo'lgan tanlanma berilgan: 9,7; 5,2; 3,2; 2,4; 5,2; 9,7; 7,5; 3,7; 9,7; 3,2; 2,4; 3,7; 7,5; 5,2; 3,2; 2,4; 7,5; 3,2; 9,7; 7,5. Bu tanlanmaning empirik taqsimot funksiyasin toping.
\\
\textbf{A3.} 
Oliy matematika fanidan 10 ta talaba test topshiriqlarin topshirdi. Harbir talaba 10 balgacha to'plashi mumkin. Agar test topshiriqlari natijalari bo'yicha \{1, 6, 2, 6, 3, 6, 4, 6, 10, 6\} tanlanma olingan bo'lsa, ushbu tanlanmalarning tanlanma o'rta va tanlanma dispersiyalarin toping.
\\
\textbf{B1.} 
Agar normal taqsimlangan bosh to'plamdan olingan hajmi \(n = 10\) ga teng bo'lgan tanlanma bo'yicha \({\overline{S}}^{2} = 0,7\) tuzatilgan tanlanma dispersiya topilgan bo'lsa, u holda \(\gamma = 0,95\) ishonchlilik bilan noma'lum \(\theta_{2}^{2}\) dispersiya uchun ishonchlilik intervalin tuzing.
\\
\textbf{B2.} 
Agar \(X^{(n)} = \left( X_{1},...,X_{n} \right)\) tanlanma \(\theta\) parametrli Bernulli taqsimotidan olingan bo'lsa, u holda noma'lum \(\theta\) parametr uchun momentlar usuli bahosini toping.
\\
\textbf{B3.} 
Agar \(X^{(n)} = \left( X_{1},...,X_{n} \right)\) tanlanma \(\left( a,\theta^{2} \right)\) parametrli normal taqsimotdan olingan bo'lsa (\(\alpha -\) ma'lum), u holda noma'lum \(\theta^{2}\) parametrning haqiqatga maksimal o'xshashlik bahosini toping.
\\
\textbf{C1.} 
Agar \(X^{(n)} = \left( X_{1},...,X_{n} \right)\) tanlanma \(\theta\) parametrli Puasson taqsimotidan olingan bo'lsa, u holda noma'lum \(\theta\) parametr uchun \(\frac{n + 3}{n + 4}\overline{x}\) bahoning siljimaganligi va asosliligini tekshiring.
\\
\textbf{C2.} 
Agar \(X^{(n)} = \left( X_{1},...,X_{n} \right)\) tanlanma \(\left( \theta_{1},\theta_{2} \right)\) parametrli gamma taqsimotdan olingan bo'lsa, u holda noma'lum \(\left( \theta_{1},\theta_{2} \right)\) vektor parametr uchun momentlar usuli bahosini toping.
\\
\textbf{C3.} 
Agar \(X^{(n)} = \left( X_{1},...,X_{n} \right)\) tanlanma \((\theta,2\theta)\) parametrli normal taqsimotdan olingan bo'lsa, u holda noma'lum \(\theta > 0\) parametrning haqiqatga maksimal o'xshashlik bahosini toping.
\\

\end{tabular}
\vspace{1cm}


\begin{tabular}{m{17cm}}
\textbf{44-variant}
\newline

\textbf{T1.} 
Tanlanma momentleri (\(k -\)tartibli boshlang'ich, boshlang'ich absolyut, markaziy va markaziy absolyut momentler).
\\
\textbf{T2.} 
Statistik gipotezalarni tekshirish (kritik to'plam, 1 va 2-tur xatolik)
\\
\textbf{A1.} 
Hajmi \(n = 20\) ga teng bo'lgan tanlanma berilgan: 3,6; 1,1; -1,8; 0,4; 3,6; 0; 5,3; 1,1; 0; -1,8; 3,6; 0,4; 1,1; 0; 0,4; 1,1; 3,6; -1,8; 3,6; 0. Bu tanlanmaning statistik taqsimotin toping.
\\
\textbf{A2.} 
Hajmi \(n = 20\) ga teng bo'lgan tanlanma berilgan: 3,2; 1,8; -1,1; 0,9; 3,2; 0; 5,6; 1,8; 0; -1,1; 3,2; 0,9; 1,8; 0; 0,9; 1,8; 3,2; -1,1; 3,2; 0. Bu tanlanmaning empirik taqsimot funksiyasin toping.
\\
\textbf{A3.} 
Oliy matematika fanidan 10 ta talaba test topshiriqlarin topshirdi. Harbir talaba 10 balgacha to'plashi mumkin. Agar test topshiriqlari natijalari bo'yicha \{2, 7, 3, 7, 6, 7, 4, 7, 7, 10\} tanlanma olingan bo'lsa, ushbu tanlanmalarning tanlanma o'rta va tanlanma dispersiyalarin toping.
\\
\textbf{B1.} 
Agar o'rta kvadratik chetlanish \(\sigma = 3\) bo'lgan normal taqsimot bosh to'plamdan olingan hajmi \(n = 14\)ga teng tanlanma bo'yicha \(\overline{x} = 5,5\) tanlanma o'rta qiymati topilgan bo'lsa, u holda \(\gamma = 0,90\) ishonchlilik bilan noma'lum \(\theta\) matematik kutilma uchun ishonchlilik intervalin tuzing .
\\
\textbf{B2.} 
Puasson taqsimoti noma'lum \(\theta > 0\) parametri momentlar usuli bahosini toping.
\\
\textbf{B3.} 
Agar \(X^{(n)} = \left( X_{1},...,X_{n} \right)\) tanlanma zichlik funksiyasi \(f(x;\theta) = \frac{2x}{\theta}e^{- \frac{x^{2}}{\theta}},\ \ x \geq 0\) bo'lgan taqsimotdan olingan bo'lsa, u holda noma'lum \(\theta > 0\) parametrning haqiqatga maksimal usuli bahosini toping.
\\
\textbf{C1.} 
Agar \(X^{(n)} = \left( X_{1},...,X_{n} \right)\) tanlanma \(\theta\) parametrli Puasson taqsimotidan olingan bo'lsa, u holda noma'lum \(\theta\) parametr uchun \(\frac{X_{1} + X_{3}}{2}\) bahoning siljimaganligi va asosliligini tekshiring.
\\
\textbf{C2.} 
Agar \(X^{(n)} = \left( X_{1},...,X_{n} \right)\) tanlanma zichlik funksiyasi\(f(x,\theta) = \theta x^{\theta - 1},x \in \lbrack 0,1\rbrack\)bo'lgan taqsimotdan olingan bo'lsa, u holda noma'lum \(\theta\) parametr uchun momentlar usuli bahosini toping.
\\
\textbf{C3.} 
\(f(x,\theta) = \frac{e^{x}}{\sqrt{2\pi}}\exp\left\{ - \frac{\left( e^{x} - \theta \right)^{2}}{2} \right\}\) model uchun \(\theta\) parametri haqiqatga maksimal o'xshashlik usuli bahosi topilsin.
\\

\end{tabular}
\vspace{1cm}


\begin{tabular}{m{17cm}}
\textbf{45-variant}
\newline

\textbf{T1.} 
Tanlanma xarakteristikalar. (Variatsion qator, nisbiy chastota).
\\
\textbf{T2.} 
Kolmogorovning muvofiqlik kritireyesi (Kolmogorov teoremasi)
\\
\textbf{A1.} 
Hajmi \(n = 20\) ga teng bo'lgan tanlanma berilgan: 7,1; 3,9; 6,3; 4,6; 7,1; 2,3; 6,3; 3,9; 4,6; 7,1; 2,3; 3,9; 7,6; 2,3; 4,6; 3,9; 2,3; 3,9; 7,6; 4,6. Bu tanlanmaning statistik taqsimotin toping.
\\
\textbf{A2.} 
Hajmi \(n = 20\) ga teng bo'lgan tanlanma berilgan: 7,9; 3,8; 6,1; 4,2; 7,9; 2,4; 6,1; 3,8; 4,2; 7,9; 2,4; 3,8; 10,2; 2,4; 4,2; 3,8; 2,4; 3,8; 10,2; 4,2. Bu tanlanmaning empirik taqsimot funksiyasin toping.
\\
\textbf{A3.} 
Oliy matematika fanidan 10 ta talaba test topshiriqlarin topshirdi. Harbir talaba 10 balgacha to'plashi mumkin. Agar test topshiriqlari natijalari bo'yicha \{9, 8, 6, 8, 6, 4, 5, 4, 7, 4\} tanlanma olingan bo'lsa, ushbu tanlanmalarning tanlanma o'rta va tanlanma dispersiyalarin toping.
\\
\textbf{B1.} 
Agar normal taqsimlangan bosh to'plamdan olingan hajmi \(n = 49\)ga teng tanlanma bo'yicha \(\overline{x} = 14,2\) tanlanma o'rta va \({\overline{S}}^{2} = 0,64\) tuzatilgan tanlanma dispersiyalar topilgan bo'lsa, u holda \(\gamma = 0,95\) ishonchlilik bilan noma'lum \(\theta\) matematik kutilma uchun ishonchlilik intervalin tuzing.
\\
\textbf{B2.} 
Agar zichlik funksiyasi \(f(x) = \frac{2x}{\theta}e^{- \frac{x^{2}}{\theta}},\ \ x \geq 0\) ko'rinishga ega bo'lsa, u holda \(\theta\) parametr momentlar usuli bahosini toping.
\\
\textbf{B3.} 
Agar \(X^{(n)} = \left( X_{1},...,X_{n} \right)\) tanlanma \(\left\lbrack - \theta,\theta^{2} \right\rbrack\) oraliqda tekis taqsimotdan olingan bo'lsa, u holda noma'lum \(\theta > 0\) parametrning haqiqatga maksimal o'xshashlik usuli bahosini toping.
\\
\textbf{C1.} 
Agar \(X^{(n)} = \left( X_{1},...,X_{n} \right)\) tanlanma \(\ln\theta\) parametrli Puasson taqsimotidan olingan bo'lsa, u holda noma'lum \(\theta\) parametr uchun \(e^{\overline{x}}\) bahoning siljimaganligi va asosliligini tekshiring.
\\
\textbf{C2.} 
Agar \(X^{(n)} = \left( X_{1},...,X_{n} \right)\) tanlanma \((\theta,2\theta)\) parametrli normal taqsimotdan olingan bo'lsa, u holda noma'lum \(\theta > 0\) parametr uchun momentlar usuli bahosini \(\ \ g(x) = (x)^{2}\) funksiya yordamida toping.
\\
\textbf{C3.} 
Agar \(X^{(n)} = \left( X_{1},...,X_{n} \right)\) tanlanma zichlik funksiyasi\(f(x;\theta) = \frac{1}{2}e^{- |x - \theta|},\ \ x \in R\) bo'lgan Laplas taqsimotidan olingan bo'lsa, u holda noma'lum \(\theta \in R\) parametrning haqiqatga maksimal o'xshashlik bahosini toping.
\\

\end{tabular}
\vspace{1cm}


\begin{tabular}{m{17cm}}
\textbf{46-variant}
\newline

\textbf{T1.} 
Guruhlangan va interval variatsion qatorlar.
\\
\textbf{T2.} 
Momentler usuli. (tanlanma momentleri, noma'lum parametrlarni baholash).
\\
\textbf{A1.} 
Hajmi \(n = 20\) ga teng bo'lgan tanlanma berilgan: 0,6; -3,8; -2,3; -4,3; 2,8; 4,7; -2,3; 0,6; -3,8; 2,8; -2,3; -4,3; 0,6; -2,3; 2,8; -3,8; -4,3; -2,3; 2,8; -3,8. Bu tanlanmaning statistik taqsimotin toping.
\\
\textbf{A2.} 
Hajmi \(n = 20\) ga teng bo'lgan tanlanma berilgan: 0,7; -3,1; -2,3; -4,8; 2,6; 4,9; -2,3; 0,7; -3,1; 2,6; -2,3; -4,8; 0,7; -2,3; 2,6; -3,1; -4,8; -2,3; 2,6; -3,1. Bu tanlanmaning empirik taqsimot funksiyasin toping.
\\
\textbf{A3.} 
Oliy matematika fanidan 10 ta talaba test topshiriqlarin topshirdi. Harbir talaba 10 balgacha to'plashi mumkin. Agar test topshiriqlari natijalari bo'yicha \{10, 4, 6, 5, 5, 4, 10, 7, 9, 10\} tanlanma olingan bo'lsa, ushbu tanlanmalarning tanlanma o'rta va tanlanma dispersiyalarin toping.
\\
\textbf{B1.} 
Agar normal taqsimlangan bosh to'plamdan olingan hajmi \(n = 8\) ga teng bo'lgan tanlanma bo'yicha \({\overline{S}}^{2} = 0,35\) tuzatilgan tanlanma dispersiya topilgan bo'lsa, u holda \(\gamma = 0,90\) ishonchlilik bilan noma'lum \(\theta_{2}^{2}\) dispersiya uchun ishonchlilik intervalin tuzing.
\\
\textbf{B2.} 
Ko'rsatkichli taqsimot noma'lum \(\theta > 0\) parametri momentlar usuli bahosini toping.
\\
\textbf{B3.} 
Agar \(x_{1} = 1,1;\ \ x_{2} = 2,7;\ldots;x_{100} = 1,5\) tanlanma \(\theta\) parametrli ko'rsatkichli taqsimotdan olingan bo'lib, \(\sum_{k = 1}^{100}x_{k} = 200\) bo'lsa, u holda noma'lum \(\theta\) parametrning haqiqatga maksimal o'xshashlik bahosini toping.
\\
\textbf{C1.} 
Agar \(X^{(n)} = \left( X_{1},...,X_{n} \right)\) tanlanma \((\alpha,\theta)\) parametrli Pareto taqsimotdan olingan bo'lsa (\(\alpha -\) ma'lum), u holda noma'lum \(\theta\) parametr uchun \(X_{(1)}\) bahoning siljimaganligi va asosliligini tekshiring.
\\
\textbf{C2.} 
Agar \(X^{(n)} = \left( X_{1},...,X_{n} \right)\) tanlanma \({\lbrack\theta}_{1},\theta_{2}\rbrack\) oraliqda tekis taqsimotdan olingan bo'lsa, u holda noma'lum \(\left( \theta_{1},\theta_{2} \right)\) vektor parametr uchun momentlar usuli bahosini toping.
\\
\textbf{C3.} 
Agar \(X^{(n)} = \left( X_{1},...,X_{n} \right)\) tanlanma zichlik funksiyasi \(f(x;\theta) = \frac{3x^{2}}{\sqrt{2\pi}}\exp\left\{ - \frac{\left( x^{3} - \theta \right)^{2}}{2} \right\},\ \ x \in R\) bo'lgan taqsimotdan olingan bo'lsa, u holda noma'lum \(\theta\) parametrning haqiqatga maksimal o'xshashlik bahosini toping.
\\

\end{tabular}
\vspace{1cm}


\begin{tabular}{m{17cm}}
\textbf{47-variant}
\newline

\textbf{T1.} 
Empirik taqsimot funktsiyasi. (Tanlanma, eksperiment)
\\
\textbf{T2.} 
Ishonchlilik intervallarin tuzish. Aniq ishonchlilik intervallar.
\\
\textbf{A1.} 
Hajmi \(n = 20\) ga teng bo'lgan tanlanma berilgan: 8,9; 2,7; 1,7; 2,2; 5,6; 1,7; 5,6; 2,7; 1,7; 2,2; 5,6; 8,9; 1,7; 2,2; 1,7; 2,7; 1,7; 5,6; 6,1; 8,9. Bu tanlanmaning statistik taqsimotin toping.
\\
\textbf{A2.} 
Hajmi \(n = 20\) ga teng bo'lgan tanlanma berilgan: 8,7; 2,7; 1,5; 2,2; 5,7; 1,5; 5,7; 2,7; 1,5; 2,2; 5,7; 8,7; 1,5; 2,2; 1,5; 2,7; 1,5; 5,7; 6,3; 8,7. Bu tanlanmaning empirik taqsimot funksiyasin toping.
\\
\textbf{A3.} 
Oliy matematika fanidan 10 ta talaba test topshiriqlarin topshirdi. Harbir talaba 10 balgacha to'plashi mumkin. Agar test topshiriqlari natijalari bo'yicha \{9, 8, 6, 9, 5, 4, 5, 7, 8, 9\} tanlanma olingan bo'lsa, ushbu tanlanmalarning tanlanma o'rta va tanlanma dispersiyalarin toping.
\\
\textbf{B1.} 
Agar o'rta kvadratik chetlanish \(\sigma = 4\) bo'lgan normal taqsimot bosh to'plamdan olingan hajmi \(n = 16\)ga teng tanlanma bo'yicha \(\overline{x} = 5,8\) tanlanma o'rta qiymati topilgan bo'lsa, u holda \(\gamma = 0,90\) ishonchlilik bilan noma'lum \(\theta\) matematik kutilma uchun ishonchlilik intervalin tuzing .
\\
\textbf{B2.} 
\(\left\lbrack \theta_{1},\theta_{2} \right\rbrack\) oraliqda tekis taqsimot parametrlari uchun momentlar usuli baholarini toping.
\\
\textbf{B3.} 
Agar (4,8,5,3) tanlanma \((a,\theta^{2}\) parametrli normal taqsimotdan olingan bo'lsa, u holda noma'lum \(\theta^{2}\) parametrning haqiqatga maksimal o'xshashlik bahosini toping.
\\
\textbf{C1.} 
Agar \(X^{(n)} = \left( X_{1},...,X_{n} \right)\) tanlanma zichlik funksiyasi bo'lsa: \(f(x;\theta) = e^{- x + \theta}\left( 1 + e^{- x + \theta} \right)^{2},\ \ x \in R\)bo'lgan taqsimotdan olingan bo'lsa, u holda noma'lum \(\theta\) parametr uchun \(\overline{x}\) bahoning siljimaganligi va asosliligini tekshiring.
\\
\textbf{C2.} 
Agar \(X^{(n)} = \left( X_{1},...,X_{n} \right)\) tanlanma \(1\sqrt{\theta}\) parametrli ko'rsatkichli taqsimotdan olingan bo'lsa, u holda noma'lum \(\theta\) parametr uchun momentlar usuli bahosini toping.
\\
\textbf{C3.} 
Agar \(X^{(n)} = \left( X_{1},...,X_{n} \right)\) tanlanma \(\lbrack\theta,\theta + 2\rbrack\) oraliqda tekis taqsimotdan olingan bo'lsa, u holda noma'lum \(\theta\) parametrning haqiqatga maksimal o'xshashlik usuli bahosini toping.
\\

\end{tabular}
\vspace{1cm}


\begin{tabular}{m{17cm}}
\textbf{48-variant}
\newline

\textbf{T1.} Matematik statistikaning asosiy masalalari. (Statistik ma'lumotlar, guruhlash)
\\
\textbf{T2.} 
Statistik gipotezalarni tekshirish (kritik to'plam, 1 va 2-tur xatolik).
\\
\textbf{A1.} 
Hajmi \(n = 20\) ga teng bo'lgan tanlanma berilgan: 1,8; -1,9; 2,4; 1,8; 2,4; 1,8; 2,4; -0,6; -1,9; 1,8; -0,6; 2,4; -3,3; -1,9; 4,0; -3,3; -3,3; -1,9; -3,3; -1,9. Bu tanlanmaning statistik taqsimotin toping.
\\
\textbf{A2.} 
Hajmi \(n = 20\) ga teng bo'lgan tanlanma berilgan: 1,4; -1,9; 2,5; 1,4; 2,5; 1,4; 2,5; -0,4; -1,9; 1,4; -0,4; 2,5; -3,7; -1,9; 4,5; -3,7; -3,7; -1,9; -3,7; -1,9. Bu tanlanmaning empirik taqsimot funksiyasin toping.
\\
\textbf{A3.} 
Oliy matematika fanidan 10 ta talaba test topshiriqlarin topshirdi. Harbir talaba 10 balgacha to'plashi mumkin. Agar test topshiriqlari natijalari bo'yicha \{4, 3, 8, 4, 8, 3, 9, 4, 7, 10\} tanlanma olingan bo'lsa, ushbu tanlanmalarning tanlanma o'rta va tanlanma dispersiyalarin toping.
\\
\textbf{B1.} 
Agar normal taqsimlangan bosh to'plamdan olingan hajmi \(n = 36\)ga teng tanlanma bo'yicha \(\overline{x} = 20,2\) tanlanma o'rta va \({\overline{S}}^{2} = 0,64\) tuzatilgan tanlanma dispersiyalar topilgan bo'lsa, u holda \(\gamma = 0,90\) ishonchlilik bilan noma'lum \(\theta\) matematik kutilma uchun ishonchlilik intervalin tuzing.
\\
\textbf{B2.} 
Agar (3,0,-2,0,-2,3,-2,0,0,3,0,0,0,0,3,-2,0,0,-2,3,0) tanlanma quyida berilgan taqsimotdan olingan bo'lsa, u holda noma'lum \(\left( \theta_{1},\theta_{2} \right)\) vektor parametr uchun momentlar usuli bahosini toping.
$\begin{array}{|c|c|c|c|}
    \hline
    \xi & - 2 & 0 & 3 \\
    \hline
    P_{\theta} & 2\theta_{1} & 0,5 + \theta_{1} + \theta_{2} & \theta_{2} \\
    \hline
\end{array}$
\\
\textbf{B3.} 
Agar \(X^{(n)} = \left( X_{1},...,X_{n} \right)\) tanlanma \(\theta\) parametrli ko'rsatkichli taqsimotdan olingan bo'lsa, u holda noma'lum \(\theta\) parametrning haqiqatga maksimal o'xshashlik usuli bahosini toping.
\\
\textbf{C1.} 
Agar \(X^{(n)} = \left( X_{1},...,X_{n} \right)\) tanlanma zichlik funksiyasi \(f(x;\theta) = \left\{ \begin{matrix}
\alpha^{- 1}e^{- \ \frac{x - \theta}{\alpha}},\ \ x \geq \theta, \\
0,\ \ x < \theta
\end{matrix} \right.\ \)bo'lgan taqsimotdan olingan bo'lsa (\(\alpha -\) ma'lum), u holda noma'lum \(\theta\) parametr uchun \(X_{(1)}\) bahoning siljimaganligi va asosliligini tekshiring.
\\
\textbf{C2.} 
Agar \(X^{(n)} = \left( X_{1},...,X_{n} \right)\) tanlanma \((\theta,\theta^{2})\) parametrli normal taqsimotdan \(\ \ g(x) = (x)^{2}\ \ \) olingan bo'lsa, u holda noma'lum \(\theta > 0\) parametr uchun momentlar usuli bahosini funksiya yordamida toping.
\\
\textbf{C3.} 
Agar \(X^{(n)} = \left( X_{1},...,X_{n} \right)\) tanlanma \(\theta\) parametrli geometrik taqsimotdan olingan bo'lsa, u holda noma'lum \(\theta\) parametrning haqiqatga maksimal o'xshashlik usuli bahosini toping.
\\

\end{tabular}
\vspace{1cm}


\begin{tabular}{m{17cm}}
\textbf{49-variant}
\newline

\textbf{T1.} 
Momentler usuli. (tanlanma momentleri, noma'lum parametrlarni baholash).
\\
\textbf{T2.} 
Pirsonning xi-kvadrat muvofiqlik kriteriysi (Pirson teoremasi).
\\
\textbf{A1.} 
Hajmi \(n = 20\) ga teng bo'lgan tanlanma berilgan: 2,9; -3,2; 5,3; -4,3; 4,1; 5,3; -1,2; 2,9; -3,2; 4,1; -4,3; 5,3; -3,2; 2,9; -4,3; 4,1; -1,2; 5,3; 2,9; -3,2. Bu tanlanmaning statistik taqsimotin toping.
\\
\textbf{A2.} 
Hajmi \(n = 20\) ga teng bo'lgan tanlanma berilgan: 2,7; -5,6; 5,2; -8,1; 4,8; 5,2; -1,6; 2,7; -5,6; 4,8; -8,1; 5,2; -5,6; 2,7; -8,1; 4,8; -1,6; 5,2; 2,7; -5,6. Bu tanlanmaning empirik taqsimot funksiyasin toping.
\\
\textbf{A3.} 
Oliy matematika fanidan 10 ta talaba test topshiriqlarin topshirdi. Harbir talaba 10 balgacha to'plashi mumkin. Agar test topshiriqlari natijalari bo'yicha \{7, 9, 4, 9, 7, 5, 4, 7, 2, 6\} tanlanma olingan bo'lsa, ushbu tanlanmalarning tanlanma o'rta va tanlanma dispersiyalarin toping.
\\
\textbf{B1.} 
Agar normal taqsimlangan bosh to'plamdan olingan hajmi \(n = 11\) ga teng bo'lgan tanlanma bo'yicha \({\overline{S}}^{2} = 0,3\) tuzatilgan tanlanma dispersiya topilgan bo'lsa, u holda \(\gamma = 0,95\) ishonchlilik bilan noma'lum \(\theta_{2}^{2}\) dispersiya uchun ishonchlilik intervalin tuzing.
\\
\textbf{B2.} 
Agar (0,-2,0,-2,3,-2,0,0,3,0,0,0,0,3,-2,0,0,-2,3,0,3) tanlanma quyida berilgan taqsimotdan olingan bo'lsa, u holda noma'lum \(\theta\) parametr uchun momentlar usuli bahosini toping.
$\begin{array}{|c|c|c|c|}
    \hline
    \xi & - 2 & 0 & 3 \\
    \hline
    P_{\theta} & \theta & 1 - 2\theta & \theta \\
    \hline
\end{array}$
\\
\textbf{B3.} 
Agar \(x_{1} = 1,1;\ \ x_{2} = 2,7;\ldots;x_{100} = 1,5\) tanlanma \(\theta\) parametrli ko'rsatkichli taqsimotdan olingan bo'lib, \(\sum_{k = 1}^{100}x_{k} = 200\) bo'lsa, u holda noma'lum \(\theta\) parametrning haqiqatga maksimal o'xshashlik bahosini toping.
\\
\textbf{C1.} 
Agar \(X^{(n)} = \left( X_{1},...,X_{n} \right)\) tanlanma \(\lbrack 0,\theta\rbrack\) oraliqda tekis taqsimotdan olingan bo'lsa, u holda noma'lum \(\theta\) parametr uchun \((n + 1)X_{(1)})\) bahoning siljimaganligi va asosliligini tekshiring.
\\
\textbf{C2.} 
Agar \(X^{(n)} = \left( X_{1},...,X_{n} \right)\) tanlanma \((\theta,2\theta)\) parametrli normal taqsimotdan olingan bo'lsa, u holda noma'lum \(\theta > 0\) parametr uchun momentlar usuli bahosini toping.
\\
\textbf{C3.} 
\(f(x,\theta) = \frac{4x^{3}}{\theta_{2}\sqrt{2\pi}}\exp\left\{ - \frac{\left( x^{4} - \theta_{1} \right)^{2}}{2{\theta_{2}}^{2}} \right\}\) model uchun \(\theta_{1}\) va \(\theta_{2}\) parametrlarning haqiqatga maksimal o'xshashlik usuli baholari topilsin.
\\

\end{tabular}
\vspace{1cm}


\begin{tabular}{m{17cm}}
\textbf{50-variant}
\newline

\textbf{T1.} 
Tanlanma xarakteristikalari.(tanlanma o'rta, tanlanma dispersiya).
\\
\textbf{T2.} 
Normal qonun dispersiyasi uchun ishonchlilik intervalin tuzish. (Ishonchlilik ehtimolligi, interval)
\\
\textbf{A1.} 
Hajmi \(n = 20\) ga teng bo'lgan tanlanma berilgan: 14,7; 7,3; 16,6; 9,8; 11,2; 16,6; 6,7; 7,3; 11,2; 14,7; 6,7; 16,6; 7,3; 11,2; 14,7; 16,6; 6,7; 7,3; 11,2; 16,6. Bu tanlanmaning statistik taqsimotin toping.
\\
\textbf{A2.} 
Hajmi \(n = 20\) ga teng bo'lgan tanlanma berilgan: 14,4; 7,6; 16,7; 9,1; 11,8; 16,7; 6,4; 7,6; 11,8; 14,4; 6,4; 16,7; 7,6; 11,8; 14,4; 16,7; 6,4; 7,6; 11,8; 16,7. Bu tanlanmaning empirik taqsimot funksiyasin toping.
\\
\textbf{A3.} 
Oliy matematika fanidan 10 ta talaba test topshiriqlarin topshirdi. Harbir talaba 10 balgacha to'plashi mumkin. Agar test topshiriqlari natijalari bo'yicha \{10, 8, 4, 6, 2, 8, 5, 10, 2, 5\} tanlanma olingan bo'lsa, ushbu tanlanmalarning tanlanma o'rta va tanlanma dispersiyalarin toping.
\\
\textbf{B1.} 
Agar o'rta kvadratik chetlanish \(\sigma = 4\) bo'lgan normal taqsimot bosh to'plamdan olingan hajmi \(n = 49\)ga teng tanlanma bo'yicha \(\overline{x} = 9,4\) tanlanma o'rta qiymati topilgan bo'lsa, u holda \(\gamma = 0,9\) ishonchlilik bilan noma'lum \(\theta\) matematik kutilma uchun ishonchlilik intervalin tuzing .
\\
\textbf{B2.} 
Agar \(X^{(n)} = \left( X_{1},...,X_{n} \right)\) tanlanma \(\theta\) parametrli ko'rsatkichli taqsimotdan olingan bo'lsa, u holda noma'lum \(\theta\) parametr uchun momentlar usuli bahosini toping.
\\
\textbf{B3.} 
\(f(x) = \frac{\theta}{2}e^{- \theta|x|}\) model uchun \(\theta\) parametri haqiqatga maksimal o'xshashlik usuli bahosi topilsin.
\\
\textbf{C1.} 
Agar \(X^{(n)} = \left( X_{1},...,X_{n} \right)\) tanlanma \(\lbrack 0,\theta\rbrack\) oraliqda tekis taqsimotdan olingan bo'lsa, u holda noma'lum \(\theta\) parametr uchun \(\frac{n + 1}{n}X_{(n)}\) bahoning siljimaganligi va asosliligini tekshiring.
\\
\textbf{C2.} 
Agar \(X^{(n)} = \left( X_{1},...,X_{n} \right)\) tanlanma \(1/\theta\) parametrli ko'rsatkichli taqsimotdan olingan bo'lsa, u holda noma'lum \(\theta\) parametr uchun momentlar usuli bahosini \(\ \ g(x) = x^{k},\) \(k \in N\) funksiya yordamida toping.
\\
\textbf{C3.} 
Agar \(X^{(n)} = \left( X_{1},...,X_{n} \right)\) tanlanma \(\left\lbrack \theta_{1},\theta_{2} \right\rbrack\) oraliqda tekis taqsimotdan olingan bo'lsa, u holda noma'lum \(\left( \theta_{1},\theta_{2} \right)\) vektor parametrning haqiqatga maksimal o'xshashlik bahosini toping.
\\

\end{tabular}
\vspace{1cm}


\begin{tabular}{m{17cm}}
\textbf{51-variant}
\newline

\textbf{T1.} Matematik statistikaning asosiy masalalari. (Statistik ma'lumotlar, guruhlash)
\\
\textbf{T2.} 
Ishonchlilik intervallarin tuzish. Aniq ishonchlilik intervallar.
\\
\textbf{A1.} 
Hajmi \(n = 20\) ga teng bo'lgan tanlanma berilgan: 4,3; 4,9; 13,4; 13,4; 6,5; 4,9; 4,9; 4,3; 5,1; 6,5; 6,5; 7,0; 4,3; 4,9; 6,5; 6,5; 5,1; 5,1; 4,9; 13,4. Bu tanlanmaning statistik taqsimotin toping.
\\
\textbf{A2.} 
Hajmi \(n = 20\) ga teng bo'lgan tanlanma berilgan: 4,2; 4,9; 13,8; 13,8; 6,6; 4,9; 4,9; 4,2; 5,3; 6,6; 6,6; 7,5; 4,2; 4,9; 6,6; 6,6; 5,3; 5,3; 4,9; 13,8. Bu tanlanmaning empirik taqsimot funksiyasin toping.
\\
\textbf{A3.} 
Oliy matematika fanidan 10 ta talaba test topshiriqlarin topshirdi. Harbir talaba 10 balgacha to'plashi mumkin. Agar test topshiriqlari natijalari bo'yicha \{9, 10, 6, 7, 4, 8, 10, 7, 9, 10\} tanlanma olingan bo'lsa, ushbu tanlanmalarning tanlanma o'rta va tanlanma dispersiyalarin toping.
\\
\textbf{B1.} 
Agar o'rta kvadratik chetlanish \(\sigma = 2\) bo'lgan normal taqsimot bosh to'plamdan olingan hajmi \(n = 10\)ga teng tanlanma bo'yicha \(\overline{x} = 5,4\) tanlanma o'rta qiymati topilgan bo'lsa, u holda \(\gamma = 0,95\) ishonchlilik bilan noma'lum \(\theta\) matematik kutilma uchun ishonchlilik intervalin tuzing .
\\
\textbf{B2.} 
Agar (3,-2,-2,0,-2,2,-2,0,-2,3,-2,0,3,0,3,-2,0,-2,3,-2,2,-2,-2,3,3,2,-2,2,3,3) tanlanma quyida berilgan taqsimotdan olingan bo'lsa, u holda noma'lum \(\theta\) parametr uchun momentlar usuli bahosini \(g(x) = |x|\) funksiya yordamida toping.
$\begin{array}{|c|c|c|c|}
    \hline
    \xi & -2 & 0 & 3 \\
    \hline
    P_{\theta} & 3\theta & 1 - 5\theta & 2\theta \\
    \hline
\end{array}$
\\
\textbf{B3.} 
Agar \(X^{(n)} = \left( X_{1},...,X_{n} \right)\) tanlanma \(\lbrack - \theta,\theta\rbrack\) oraliqda tekis taqsimotdan olingan bo'lsa, u holda noma'lum \(\theta > 0\) parametrning haqiqatga maksimal o'xshashlik usuli bahosini toping.
\\
\textbf{C1.} 
Agar \(X^{(n)} = \left( X_{1},...,X_{n} \right)\) tanlanma \(M\xi = a\) ma'lum va \(M\xi^{2}\) chekli bo'lgan taqsimotdan olingan bo'lsa, u holda noma'lum \(D\xi\) dispersiya uchun \({\overline{S}}^{2}\) bahoning siljimaganligi va asosliligini tekshiring.
\\
\textbf{C2.} 
Agar \(X^{(n)} = \left( X_{1},...,X_{n} \right)\) tanlanma {[}\(0,2\theta\rbrack\) oraliqda tekis taqsimotdan olingan bo'lsa, u holda noma'lum \(\theta > 0\) parametr uchun momentlar usuli bahosini toping.
\\
\textbf{C3.} 
Agar \(X^{(n)} = \left( X_{1},...,X_{n} \right)\) tanlanma \(\left( \theta,\theta^{2} \right)\) parametrli normal taqsimotdan olingan bo'lsa, u holda noma'lum \(\theta > 0\) parametrning haqiqatga maksimal o'xshashlik bahosini toping.
\\

\end{tabular}
\vspace{1cm}


\begin{tabular}{m{17cm}}
\textbf{52-variant}
\newline

\textbf{T1.} 
Guruhlangan va interval variatsion qatorlar.
\\
\textbf{T2.} 
Kolmogorovning muvofiqlik kritireyesi (Kolmogorov teoremasi)
\\
\textbf{A1.} 
Hajmi \(n = 20\) ga teng bo'lgan tanlanma berilgan: -2,1; 1,7; 3,3; 3,3; 11,7; 4,7; 1,7; 4,7; -2,1; 4,7; 4,7; 4,7; 8,0; -2,1; 1,7; 4,7; 8,0; 11,7; 1,7; 8,0. Bu tanlanmaning statistik taqsimotin toping.
\\
\textbf{A2.} 
Hajmi \(n = 20\) ga teng bo'lgan tanlanma berilgan: -2,2; 1,3; 3,8; 3,8; 11,5; 4,1; 1,3; 4,1; -2,2; 4,1; 4,1; 4,1; 8,4; -2,2; 1,3; 4,1; 8,4; 11,5; 1,3; 8,4. Bu tanlanmaning empirik taqsimot funksiyasin toping.
\\
\textbf{A3.} 
Oliy matematika fanidan 10 ta talaba test topshiriqlarin topshirdi. Harbir talaba 10 balgacha to'plashi mumkin. Agar test topshiriqlari natijalari bo'yicha \{4, 1, 2, 4, 6, 4, 5, 3, 6, 5\} tanlanma olingan bo'lsa, ushbu tanlanmalarning tanlanma o'rta va tanlanma dispersiyalarin toping.
\\
\textbf{B1.} 
Agar normal taqsimlangan bosh to'plamdan olingan hajmi \(n = 16\)ga teng tanlanma bo'yicha \(\overline{x} = 20,2\) tanlanma o'rta va \({\overline{S}}^{2} = 0,64\) tuzatilgan tanlanma dispersiyalar topilgan bo'lsa, u holda \(\gamma = 0,95\) ishonchlilik bilan noma'lum \(\theta\) matematik kutilma uchun ishonchlilik intervalin tuzing.
\\
\textbf{B2.} 
Puasson taqsimoti noma'lum \(\theta > 0\) parametri momentlar usuli bahosini toping.
\\
\textbf{B3.} 
Agar \(X^{(n)} = \left( X_{1},...,X_{n} \right)\) tanlanma \(\left( a,\theta^{2} \right)\) parametrli normal taqsimotdan olingan bo'lsa (\(\alpha -\) ma'lum), u holda noma'lum \(\theta^{2}\) parametrning haqiqatga maksimal o'xshashlik bahosini toping.
\\
\textbf{C1.} 
Agar \(X^{(n)} = \left( X_{1},...,X_{n} \right)\) tanlanma \(M\xi = a\) ma'lum va \(M\xi^{2}\) chekli bo'lgan taqsimotdan olingan bo'lsa, u holda noma'lum \(D\xi\) dispersiya uchun \(\frac{1}{n}\sum_{i = 1}^{n}{X_{i}a}\) bahoning siljimaganligi va asosliligini tekshiring.
\\
\textbf{C2.} 
Agar \(X^{(n)} = \left( X_{1},...,X_{n} \right)\) tanlanma \((\theta,\theta^{2})\ \ \) parametrli normal taqsimotdan olingan bo'lsa, u holda noma'lum \(\theta > 0\) parametr uchun momentlar usuli bahosini toping.
\\
\textbf{C3.} 
Agar \(X^{(n)} = \left( X_{1},...,X_{n} \right)\) tanlanma zichlik funksiyasi\(f(x;\theta) = \frac{\theta}{2}e^{- \theta|x|},\ \ x \in R\) bo'lgan taqsimotdan olingan bo'lsa, u holda noma'lum \(\theta > 0\) parametrning haqiqatga maksimal o'xshashlik bahosini toping.
\\

\end{tabular}
\vspace{1cm}


\begin{tabular}{m{17cm}}
\textbf{53-variant}
\newline

\textbf{T1.} 
Tanlanma xarakteristikalar. (Variatsion qator, nisbiy chastota).
\\
\textbf{T2.} 
Normal qonun dispersiyasi uchun ishonchlilik intervalin tuzish. (Ishonchlilik ehtimolligi, interval)
\\
\textbf{A1.} 
Hajmi \(n = 20\) ga teng bo'lgan tanlanma berilgan: -11,0; -4,1; 0; 2,3; 1,2; 0; 1,2; 2,3; 2,3; 1,2; 2,3; -11,0; 3,4; 1,2; 3,4; 3,4; 0; 3,4; 2,3; 0. Bu tanlanmaning statistik taqsimotin toping.
\\
\textbf{A2.} 
Hajmi \(n = 20\) ga teng bo'lgan tanlanma berilgan: -11,2; -4,5; 0; 2,9; 1,7; 0; 1,7; 2,9; 2,9; 1,7; 2,9; -11,2; 3,1; 1,7; 3,1; 3,1; 0; 3,1; 2,9; 0. Bu tanlanmaning empirik taqsimot funksiyasin toping.
\\
\textbf{A3.} 
Oliy matematika fanidan 10 ta talaba test topshiriqlarin topshirdi. Harbir talaba 10 balgacha to'plashi mumkin. Agar test topshiriqlari natijalari bo'yicha \{8, 9, 10, 4, 9, 7, 6, 7, 6, 4\} tanlanma olingan bo'lsa, ushbu tanlanmalarning tanlanma o'rta va tanlanma dispersiyalarin toping.
\\
\textbf{B1.} 
Agar normal taqsimlangan bosh to'plamdan olingan hajmi \(n = 11\) ga teng bo'lgan tanlanma bo'yicha \({\overline{S}}^{2} = 0,5\) tuzatilgan tanlanma dispersiya topilgan bo'lsa, u holda \(\gamma = 0,90\) ishonchlilik bilan noma'lum \(\theta_{2}^{2}\) dispersiya uchun ishonchlilik intervalin tuzing.
\\
\textbf{B2.} 
\(\lbrack 0,\theta\rbrack\) oraliqda tekis taqsimlangan \(\theta\) parametri uchun momentlar usuli bahosini toping.
\\
\textbf{B3.} 
Agar \(X^{(n)} = \left( X_{1},...,X_{n} \right)\) tanlanma \(\theta\) parametrli ko'rsatkichli taqsimotdan olingan bo'lsa, u holda noma'lum \(\theta\) parametrning haqiqatga maksimal o'xshashlik usuli bahosini toping.
\\
\textbf{C1.} 
Agar \(X^{(n)} = \left( X_{1},...,X_{n} \right)\) tanlanma \(M\xi = a\) ma'lum va \(M\xi^{2}\) chekli bo'lgan taqsimotdan olingan bo'lsa, u holda noma'lum \(D\xi\) dispersiya uchun \(\overline{x^{2}} - a^{2}\) bahoning siljimaganligi va asosliligini tekshiring.
\\
\textbf{C2.} 
Agar \(X^{(n)} = \left( X_{1},...,X_{n} \right)\) tanlanma \(\theta\) parametrli geometrik taqsimotdan olingan bo'lsa, u holda noma'lum \(\theta\) parametr uchun momentlar usuli bahosini toping.
\\
\textbf{C3.} 
Agar \(X^{(n)} = \left( X_{1},...,X_{n} \right)\) tanlanma zichlik funksiyasi \(f(x;\theta) = \frac{e^{x}}{\sqrt{2\pi}}\exp\left\{ - \frac{\left( e^{x} - \theta \right)^{2}}{2} \right\},\ \ x \in R\) bo'lgan taqsimotdan olingan bo'lsa, u holda noma'lum \(\theta\) parametrning haqiqatga maksimal o'xshashlik bahosini toping.
\\

\end{tabular}
\vspace{1cm}


\begin{tabular}{m{17cm}}
\textbf{54-variant}
\newline

\textbf{T1.} 
Tanlanma xarakteristikalari.(tanlanma o'rta, tanlanma dispersiya).
\\
\textbf{T2.} 
Statistik gipotezalarni tekshirish (kritik to'plam, 1 va 2-tur xatolik)
\\
\textbf{A1.} 
Hajmi \(n = 20\) ga teng bo'lgan tanlanma berilgan: 2,5; 3,8; 4,3; 2,5; 3,8; 2,5; 3,1; 4,3; 4,3; 5,5; 6,2; 2,5; 3,1; 6,2; 5,5; 6,2; 3,1; 3,1; 6,2; 3,1. Bu tanlanmaning statistik taqsimotin toping.
\\
\textbf{A2.} 
Hajmi \(n = 20\) ga teng bo'lgan tanlanma berilgan: 2,7; 4,2; 4,8; 2,7; 4,2; 2,7; 3,9; 4,8; 4,8; 5,9; 6,5; 2,7; 3,9; 6,5; 5,9; 6,5; 3,9; 3,9; 6,5; 3,9. Bu tanlanmaning empirik taqsimot funksiyasin toping.
\\
\textbf{A3.} 
Oliy matematika fanidan 10 ta talaba test topshiriqlarin topshirdi. Harbir talaba 10 balgacha to'plashi mumkin. Agar test topshiriqlari natijalari bo'yicha \{7, 8, 7, 6, 4, 8, 4, 7, 9, 10\} tanlanma olingan bo'lsa, ushbu tanlanmalarning tanlanma o'rta va tanlanma dispersiyalarin toping.
\\
\textbf{B1.} 
Agar o'rta kvadratik chetlanish \(\sigma = 3\) bo'lgan normal taqsimot bosh to'plamdan olingan hajmi \(n = 9\)ga teng tanlanma bo'yicha \(\overline{x} = 4,5\) tanlanma o'rta qiymati topilgan bo'lsa, u holda \(\gamma = 0,95\) ishonchlilik bilan noma'lum \(\theta\) matematik kutilma uchun ishonchlilik intervalin tuzing .
\\
\textbf{B2.} 
Agar \(X^{(n)} = \left( X_{1},...,X_{n} \right)\) tanlanma \(\theta\) parametrli Bernulli taqsimotidan olingan bo'lsa, u holda noma'lum \(\theta\) parametr uchun momentlar usuli bahosini toping.
\\
\textbf{B3.} 
Agar (-1,-1,0,-1,0,-1,-1,5,-1,0,-1,0,5,-1,-1,-1,5,-1,-1,-1,1,-1,5,0,-1,-1,5) tanlanma quyida berilgan taqsimotdan olingan bo'lsa, u holda noma'lum \(\theta\) parametrning haqiqatga maksimal o'xshashlik usuli bahosini toping.
$\begin{array}{|c|c|c|c|}
    \hline
    \xi & - 1 & 0 & 5\\
    \hline
    P_{\theta} & 1 - \theta & \theta/2 & \theta/2 \\
    \hline
\end{array}$
\\
\textbf{C1.} 
Agar \(X^{(n)} = \left( X_{1},...,X_{n} \right)\) tanlanma zichlik funksiyasi bo'lsa: \(f(x,\theta) = \left\{ \begin{matrix}
e^{\theta - x},\ \ x \geq \theta, \\
\ \ 0,\ \ x < \theta
\end{matrix} \right.\ \) bo'lgan taqsimotdan olingan bo'lsa, u holda noma'lum \(\theta\) parametr uchun \(X_{(1)}\) bahoning siljimaganligi va asosliligini tekshiring.
\\
\textbf{C2.} 
Agar \(X^{(n)} = \left( X_{1},...,X_{n} \right)\) tanlanma\(\ \ (a,\theta^{2})\) parametrli normal taqsimotdan olingan bo'lsa (\(\alpha -\) ma'lum), u holda noma'lum\(\ \ \theta^{2}\) parametr uchun momentlar usuli bahosini toping.
\\
\textbf{C3.} 
Agar \(X^{(n)} = \left( X_{1},...,X_{n} \right)\) tanlanma zichlik funksiyasi\(f(x;\theta) = \left\{ \begin{matrix}
e^{\theta - x},\ \ x \geq \theta, \\
\ \ 0,\ \ x < \theta
\end{matrix} \right.\ \) bo'lgan taqsimotdan olingan bo'lsa, u holda noma'lum \(\theta\) parametrning haqiqatga maksimal o'xshashlik bahosini toping.
\\

\end{tabular}
\vspace{1cm}


\begin{tabular}{m{17cm}}
\textbf{55-variant}
\newline

\textbf{T1.} 
Empirik taqsimot funktsiyasi. (Tanlanma, eksperiment)
\\
\textbf{T2.} 
Pirsonning xi-kvadrat muvofiqlik kriteriysi (Pirson teoremasi).
\\
\textbf{A1.} 
Hajmi \(n = 20\) ga teng bo'lgan tanlanma berilgan: -4,3; 2,6; 0; -2,5; 2,6; 1,9; 2,2; 0; -4,3; -2,5; 1,9; -2,5; 1,9; 2,2; 2,6; 1,9; 2,6; 2,2; 2,2; 1,9. Bu tanlanmaning statistik taqsimotin toping.
\\
\textbf{A2.} 
Hajmi \(n = 20\) ga teng bo'lgan tanlanma berilgan: -4,9; 2,6; 0,5; -2,6; 2,6; 1,7; 2,3; 0,5; -4,9; -2,6; 1,7; -2,6; 1,7; 2,3; 2,6; 1,7; 2,6; 2,3; 2,3; 1,7. Bu tanlanmaning empirik taqsimot funksiyasin toping.
\\
\textbf{A3.} 
Oliy matematika fanidan 10 ta talaba test topshiriqlarin topshirdi. Harbir talaba 10 balgacha to'plashi mumkin. Agar test topshiriqlari natijalari bo'yicha \{9, 5, 6, 8, 4, 7, 4, 6, 9, 7\} tanlanma olingan bo'lsa, ushbu tanlanmalarning tanlanma o'rta va tanlanma dispersiyalarin toping.
\\
\textbf{B1.} 
Agar normal taqsimlangan bosh to'plamdan olingan hajmi \(n = 25\)ga teng tanlanma bo'yicha \(\overline{x} = 18,6\) tanlanma o'rta va \({\overline{S}}^{2} = 0,49\) tuzatilgan tanlanma dispersiyalar topilgan bo'lsa, u holda \(\gamma = 0,95\) ishonchlilik bilan noma'lum \(\theta\) matematik kutilma uchun ishonchlilik intervalin tuzing.
\\
\textbf{B2.} 
Agar (-2,0,-2,0,-2,3,-2,0,0,3,0,0,0,0,3,-2,0,0,-2,3,0) tanlanma quyida berilgan taqsimotdan olingan bo'lsa, u holda noma'lum \(\left( \theta_{1},\theta_{2} \right)\) vektor parametr uchun momentlar usuli bahosini toping.
$\begin{array}{|c|c|c|c|}
    \hline
    \xi & - 2 & 0 & 3\\
    \hline
    P_{\theta} & \theta_{1} & 1 - \theta_{1} - \theta_{2} & \theta_{2} \\
    \hline
\end{array}$
\\
\textbf{B3.} 
Agar \(X^{(n)} = \left( X_{1},...,X_{n} \right)\) tanlanma \(\theta\) parametrli Bernulli taqsimotidan olingan bo'lsa, u holda noma'lum \(\theta\) parametrning haqiqatga maksimal o'xshashlik usuli bahosini toping.
\\
\textbf{C1.} 
Agar \(X^{(n)} = \left( X_{1},...,X_{n} \right)\) tanlanma zichlik funksiyasi bo'lsa: \(f(x,\theta) = \left\{ \begin{matrix}
e^{\theta - x},\ \ x \geq \theta, \\
\ \ 0,\ \ x < \theta
\end{matrix} \right.\ \) bo'lgan taqsimotdan olingan bo'lsa, u holda noma'lum \(\theta\) parametr uchun \(\overline{x} - 1\) bahoning siljimaganligi va asosliligini tekshiring.
\\
\textbf{C2.} 
Agar \(X^{(n)} = \left( X_{1},...,X_{n} \right)\) tanlanma \(\theta\) parametrli Puasson taqsimotidan olingan bo'lsa, u holda noma'lum \(\theta\) parametr uchun momentlar usuli bahosini toping.
\\
\textbf{C3.} 
Agar \(X^{(n)} = \left( X_{1},...,X_{n} \right)\) tanlanma zichlik funksiyasi\(f(x;\theta) = \left\{ \begin{matrix}
e^{\theta - x},\ \ x \geq \theta, \\
\ \ 0,\ \ x < \theta
\end{matrix} \right.\ \) bo'lgan taqsimotdan olingan bo'lsa, u holda noma'lum \(\theta\) parametrning haqiqatga maksimal o'xshashlik bahosini toping.
\\

\end{tabular}
\vspace{1cm}


\begin{tabular}{m{17cm}}
\textbf{56-variant}
\newline

\textbf{T1.} 
Neyman-Pirson teoremasi.
\\
\textbf{T2.} 
Statistik gipotezalarni tekshirish (kritik to'plam, 1 va 2-tur xatolik).
\\
\textbf{A1.} 
Hajmi \(n = 20\) ga teng bo'lgan tanlanma berilgan: -2,9; -3,8; 2,3; 1,8; 1,8; 0,7; -3,8; -1,5; 2,3; 0,7; -2,9; -1,5; 1,8; -2,9; -1,5; -3,8; 1,8; 1,8; -3,8; 1,8. Bu tanlanmaning statistik taqsimotin toping.
\\
\textbf{A2.} 
Hajmi \(n = 20\) ga teng bo'lgan tanlanma berilgan: -2,4; -3,5; 2,8; 1,4; 1,4; 0,1; -3,5; -1,9; 2,8; 0,1; -2,4; -1,9; 1,4; -2,4; -1,9; -3,5; 1,4; 1,4; -3,5; 1,4. Bu tanlanmaning empirik taqsimot funksiyasin toping.
\\
\textbf{A3.} 
Oliy matematika fanidan 10 ta talaba test topshiriqlarin topshirdi. Harbir talaba 10 balgacha to'plashi mumkin. Agar test topshiriqlari natijalari bo'yicha \{8, 9, 7, 10, 6, 8, 10, 3, 10, 9\} tanlanma olingan bo'lsa, ushbu tanlanmalarning tanlanma o'rta va tanlanma dispersiyalarin toping.
\\
\textbf{B1.} 
Agar normal taqsimlangan bosh to'plamdan olingan hajmi \(n = 12\) ga teng bo'lgan tanlanma bo'yicha \({\overline{S}}^{2} = 0,4\) tuzatilgan tanlanma dispersiya topilgan bo'lsa, u holda \(\gamma = 0,90\) ishonchlilik bilan noma'lum \(\theta_{2}^{2}\) dispersiya uchun ishonchlilik intervalin tuzing.
\\
\textbf{B2.} 
Agar zichlik funksiyasi \(f(x) = \frac{2x}{\theta}e^{- \frac{x^{2}}{\theta}},\ \ x \geq 0\) ko'rinishga ega bo'lsa, u holda \(\theta\) parametr momentlar usuli bahosini toping.
\\
\textbf{B3.} 
Agar \(X^{(n)} = \left( X_{1},...,X_{n} \right)\) tanlanma \(\left\lbrack - \theta,\theta^{2} \right\rbrack\) oraliqda tekis taqsimotdan olingan bo'lsa, u holda noma'lum \(\theta > 0\) parametrning haqiqatga maksimal o'xshashlik usuli bahosini toping.
\\
\textbf{C1.} 
Agar \(X^{(n)} = \left( X_{1},...,X_{n} \right)\) tanlanma \(\lbrack - 3\theta,\theta\rbrack\) oraliqda tekis taqsimotdan olingan bo'lsa, u holda noma'lum \(\theta\) parametr uchun \(4X_{(n)} + X_{(1)}\) bahoni siljimaganligi va asosliligini tekshiring.
\\
\textbf{C2.} 
Agar \(X^{(n)} = \left( X_{1},...,X_{n} \right)\) tanlanma \(\theta\) parametrli geometrik taqsimotdan olingan bo'lsa, u holda noma'lum \(\theta\) parametr uchun momentlar usuli bahosini toping.
\\
\textbf{C3.} 
\(f(x,\theta) = \frac{4x^{3}}{\theta_{2}\sqrt{2\pi}}\exp\left\{ - \frac{\left( x^{4} - \theta_{1} \right)^{2}}{2{\theta_{2}}^{2}} \right\}\) model uchun \(\theta_{1}\) va \(\theta_{2}\) parametrlarning haqiqatga maksimal o'xshashlik usuli baholari topilsin.
\\

\end{tabular}
\vspace{1cm}


\begin{tabular}{m{17cm}}
\textbf{57-variant}
\newline

\textbf{T1.} 
Poligon va gistogramma(nisbiy chastota, interval qator, grafik).
\\
\textbf{T2.} 
Chiziqli korrelyatsiya tenglamasi (ta'rifi, regressiya to'g'ri chiziqning tanlanma tenglamalari)
\\
\textbf{A1.} 
Hajmi \(n = 20\) ga teng bo'lgan tanlanma berilgan: 3,6; 2,9; 3,6; 3,2; 1,1; 0,3; 1,1; 3,6; 1,7; 1,1; 0,3; 1,7; 1,1; 0,3; 2,9; 2,9; 2,9; 1,1; 2,9; 1,7. Bu tanlanmaning statistik taqsimotin toping.
\\
\textbf{A2.} 
Hajmi \(n = 20\) ga teng bo'lgan tanlanma berilgan: 4,6; 2,5; 4,6; 3,3; 1,8; 0,3; 1,8; 4,6; 2,1; 1,8; 0,3; 2,1; 1,8; 0,3; 2,5; 2,5; 2,5; 1,8; 2,5; 2,1. Bu tanlanmaning empirik taqsimot funksiyasin toping.
\\
\textbf{A3.} 
Oliy matematika fanidan 10 ta talaba test topshiriqlarin topshirdi. Harbir talaba 10 balgacha to'plashi mumkin. Agar test topshiriqlari natijalari bo'yicha \{5, 7, 5, 9, 5, 8, 10, 6, 7, 8\} tanlanma olingan bo'lsa, ushbu tanlanmalarning tanlanma o'rta va tanlanma dispersiyalarin toping.
\\
\textbf{B1.} 
Agar o'rta kvadratik chetlanish \(\sigma = 1\) bo'lgan normal taqsimot bosh to'plamdan olingan hajmi \(n = 15\)ga teng tanlanma bo'yicha \(\overline{x} = 5,8\) tanlanma o'rta qiymati topilgan bo'lsa, u holda \(\gamma = 0,90\) ishonchlilik bilan noma'lum \(\theta\) matematik kutilma uchun ishonchlilik intervalin tuzing .
\\
\textbf{B2.} 
Agar (3,0,-2,0,-2,3,-2,0,0,3,0,0,0,0,3,-2,0,0,-2,3,0) tanlanma quyida berilgan taqsimotdan olingan bo'lsa, u holda noma'lum \(\left( \theta_{1},\theta_{2} \right)\) vektor parametr uchun momentlar usuli bahosini toping.
$\begin{array}{|c|c|c|c|}
    \hline
    \xi & - 2 & 0 & 3 \\
    \hline
    P_{\theta} & 2\theta_{1} & 0,5 + \theta_{1} + \theta_{2} & \theta_{2} \\
    \hline
\end{array}$
\\
\textbf{B3.} 
\(f(x) = \frac{2x}{\theta}e^{- \frac{x^{2}}{\theta}},\ \ x \geq 0\) model uchun \(\theta\) parametri haqiqatga maksimal o'xshashlik usuli bahosi topilsin.
\\
\textbf{C1.} 
Agar \(X^{(n)} = \left( X_{1},...,X_{n} \right)\) tanlanma taqsimot funksiyasi \(F(x)\) bo'lgan taqsimotdan olingan bo'lsa, u holda noma'lum \(F(x)\) uchun \(F_{n}(x)\) empirik taqsimot funksiyasining siljimaganligi va asosliligini tekshiring.
\\
\textbf{C2.} 
Agar \(X^{(n)} = \left( X_{1},...,X_{n} \right)\) tanlanma zichlik funksiyasi\(f(x,\theta) = \left\{ \begin{matrix}
e^{\theta - x},\ \ x \geq \theta, \\
0,\ \ x < \theta
\end{matrix} \right.\ \)bo'lgan taqsimotdan olingan bo'lsa, u holda noma'lum \(\theta\) parametr uchun momentlar usuli bahosini toping.
\\
\textbf{C3.} 
Agar \(X^{(n)} = \left( X_{1},...,X_{n} \right)\) tanlanma \(\left( \theta,\theta^{2} \right)\) parametrli normal taqsimotdan olingan bo'lsa, u holda noma'lum \(\theta > 0\) parametrning haqiqatga maksimal o'xshashlik bahosini toping.
\\

\end{tabular}
\vspace{1cm}


\begin{tabular}{m{17cm}}
\textbf{58-variant}
\newline

\textbf{T1.} 
Tanlanma momentleri (\(k -\)tartibli boshlang'ich, boshlang'ich absolyut, markaziy va markaziy absolyut momentler).
\\
\textbf{T2.} 
Momentler usuli. (tanlanma momentleri, noma'lum parametrlarni baholash).
\\
\textbf{A1.} 
Hajmi \(n = 20\) ga teng bo'lgan tanlanma berilgan: -1,3; 0; 0,8; 2,3; 1,1; 0,8; 0,8; 2,3; 1,1; 0,8; -1,3; 1,8; 1,1; -1,3; 1,1; 1,8; 1,8; 1,1; 1,8; 1,8. Bu tanlanmaning statistik taqsimotin toping.
\\
\textbf{A2.} 
Hajmi \(n = 20\) ga teng bo'lgan tanlanma berilgan: -1,9; 0,7; 0,9; 2,8; 1,3; 0,9; 0,9; 2,8; 1,3; 0,9; -1,9; 1,6; 1,3; -1,9; 1,3; 1,6; 1,6; 1,3; 1,6; 1,6. Bu tanlanmaning empirik taqsimot funksiyasin toping.
\\
\textbf{A3.} 
Oliy matematika fanidan 10 ta talaba test topshiriqlarin topshirdi. Harbir talaba 10 balgacha to'plashi mumkin. Agar test topshiriqlari natijalari bo'yicha \{8, 4, 3, 7, 3, 6, 5, 3, 5, 6\} tanlanma olingan bo'lsa, ushbu tanlanmalarning tanlanma o'rta va tanlanma dispersiyalarin toping.
\\
\textbf{B1.} 
Agar normal taqsimlangan bosh to'plamdan olingan hajmi \(n = 20\)ga teng tanlanma bo'yicha \(\overline{x} = 16,6\) tanlanma o'rta va \({\overline{S}}^{2} = 0,64\) tuzatilgan tanlanma dispersiyalar topilgan bo'lsa, u holda \(\gamma = 0,95\) ishonchlilik bilan noma'lum \(\theta\) matematik kutilma uchun ishonchlilik intervalin tuzing.
\\
\textbf{B2.} 
\(\lbrack 0,\theta\rbrack\) oraliqda tekis taqsimlangan \(\theta\) parametri uchun momentlar usuli bahosini toping.
\\
\textbf{B3.} 
Agar (0,1,2,0) tanlanma quyida berilgan taqsimotdan olingan bo'lsa, u holda noma'lum \(\theta\) parametrning haqiqatga maksimal o'xshashlik bahosini toping.
$\begin{array}{|c|c|c|c|}
    \hline
    \xi & 0 & 1 & 2 \\
    \hline
    P_{\theta} & \theta & 2\theta & 1 - 3\theta \\
    \hline
\end{array}$
\\
\textbf{C1.} 
Agar \(X^{(n)} = \left( X_{1},...,X_{n} \right)\) tanlanma \(\left( a,\theta^{2} \right)\) parametrli normal taqsimotdan olingan bo'lsa (\(a -\) ma'lum), u holda noma'lum \(\theta\) parametr uchun \(\sqrt{\frac{\pi}{2}}\left| \overline{x - a} \right|\) bahoning siljimaganligi va asosliligini tekshiring.
\\
\textbf{C2.} 
Agar \(X^{(n)} = \left( X_{1},...,X_{n} \right)\) tanlanma\(\ \ (a,\theta^{2})\) parametrli normal taqsimotdan olingan bo'lsa (\(\alpha -\) ma'lum), u holda noma'lum\(\ \ \theta^{2}\) parametr uchun momentlar usuli bahosini toping.
\\
\textbf{C3.} 
Agar \(X^{(n)} = \left( X_{1},...,X_{n} \right)\) tanlanma zichlik funksiyasi\(f(x;\theta) = \frac{4x^{3}}{\sqrt{2\pi}\theta_{2}}\exp\left\{ - \frac{\left( x^{4} - \theta_{1} \right)^{2}}{2{\theta_{2}}^{2}} \right\},\ \ x \in R\) bo'lgan taqsimotdan olingan bo'lsa, u holda noma'lum \(\left( \theta_{1},\theta_{2}^{2} \right)\) vektor parametrning haqiqatga maksimal o'xshashlik usuli baholarini toping.
\\

\end{tabular}
\vspace{1cm}


\begin{tabular}{m{17cm}}
\textbf{59-variant}
\newline

\textbf{T1.} 
Glivenko-Kantelli teoremasi. (empirik taqsimot funktsiyasi, ehtimollik bilan yaqinlashish).
\\
\textbf{T2.} 
Statistik baho xossalari. (Siljimagan, asosliy, effektiv)
\\
\textbf{A1.} 
Hajmi \(n = 20\) ga teng bo'lgan tanlanma berilgan: -2,4; 5,6; 5,6; -5,2; -6,7; 5,1; -5,2; -2,4; 4,3; 5,1; -6,7; 4,3; -2,4; -6,7; 4,3; 5,1; 4,3; 5,6; -6,7; 5,6. Bu tanlanmaning statistik taqsimotin toping.
\\
\textbf{A2.} 
Hajmi \(n = 20\) ga teng bo'lgan tanlanma berilgan: -2,9; 7,6; 7,6; -5,7; -6,1; 5,5; -5,7; -2,9; 4,2; 5,5; -6,1; 4,2; -2,9; -6,1; 4,2; 5,5; 4,2; 7,6; -6,1; 7,6. Bu tanlanmaning empirik taqsimot funksiyasin toping.
\\
\textbf{A3.} 
Oliy matematika fanidan 10 ta talaba test topshiriqlarin topshirdi. Harbir talaba 10 balgacha to'plashi mumkin. Agar test topshiriqlari natijalari bo'yicha \{9, 8, 6, 7, 5, 8, 5, 7, 4, 6\} tanlanma olingan bo'lsa, ushbu tanlanmalarning tanlanma o'rta va tanlanma dispersiyalarin toping.
\\
\textbf{B1.} 
Agar normal taqsimlangan bosh to'plamdan olingan hajmi \(n = 13\) ga teng bo'lgan tanlanma bo'yicha \({\overline{S}}^{2} = 1,2\) tuzatilgan tanlanma dispersiya topilgan bo'lsa, u holda \(\gamma = 0,90\) ishonchlilik bilan noma'lum \(\theta_{2}^{2}\) dispersiya uchun ishonchlilik intervalin tuzing.
\\
\textbf{B2.} 
Puasson taqsimoti noma'lum \(\theta > 0\) parametri momentlar usuli bahosini toping.
\\
\textbf{B3.} 
Agar (4,8,5,3) tanlanma \((a,\theta^{2}\) parametrli normal taqsimotdan olingan bo'lsa, u holda noma'lum \(\theta^{2}\) parametrning haqiqatga maksimal o'xshashlik bahosini toping.
\\
\textbf{C1.} 
Agar \(X^{(n)} = \left( X_{1},...,X_{n} \right)\) tanlanma \(\theta\) parametrli ko'rsatkichli taqsimotdan olingan bo'lsa, u holda noma'lum \(\theta\) parametr uchun \(1/\overline{x}\) bahoning siljimaganligi va asosliligini tekshiring.
\\
\textbf{C2.} 
Agar \(X^{(n)} = \left( X_{1},...,X_{n} \right)\) tanlanma \(\left( \theta_{1},\theta_{2} \right)\) parametrli gamma taqsimotdan olingan bo'lsa, u holda noma'lum \(\left( \theta_{1},\theta_{2} \right)\) vektor parametr uchun momentlar usuli bahosini toping.
\\
\textbf{C3.} 
Agar \(X^{(n)} = \left( X_{1},...,X_{n} \right)\) tanlanma zichlik funksiyasi \(f(x;\theta) = \frac{\theta}{\sqrt{2\pi x^{3}}}e^{- \theta^{2}/2x},\ \ x \geq 0\) bo'lgan taqsimotdan olingan bo'lsa, u holda noma'lum \(\theta > 0\) parametrning haqiqatga maksimal o'xshashlik bahosini toping.
\\

\end{tabular}
\vspace{1cm}


\begin{tabular}{m{17cm}}
\textbf{60-variant}
\newline

\textbf{T1.} 
Momentler usuli. (tanlanma momentleri, noma'lum parametrlarni baholash).
\\
\textbf{T2.} 
Haqiqatga maksimal o'xshashlik usuli. (haqiqatga maksimal o'xshashlik funktsiyasi, noma'lum parametrlarni baholash).
\\
\textbf{A1.} 
Hajmi \(n = 20\) ga teng bo'lgan tanlanma berilgan:-3,3; 0; 4,4; 2,2; -2,7; 4,4; 2,2; 4,4;-3,3; 2,2; -2,7; 2,2; -3,3; -2,7; 2,2; 3,4; 4,4; 0; -3,3; 0. Bu tanlanmaning statistik taqsimotin toping.
\\
\textbf{A2.} 
Hajmi \(n = 20\) ga teng bo'lgan tanlanma berilgan:-3,3; 0; 4,9; 2,8; -2,6; 4,9; 2,8; 4,9;-3,3; 2,8; -2,6; 2,8; -3,3; -2,6; 2,8; 3,1; 4,9; 0; -3,3; 0. Bu tanlanmaning empirik taqsimot funksiyasin toping.
\\
\textbf{A3.} 
Oliy matematika fanidan 10 ta talaba test topshiriqlarin topshirdi. Harbir talaba 10 balgacha to'plashi mumkin. Agar test topshiriqlari natijalari bo'yicha \{4, 7, 6, 9, 3, 8, 3, 7, 4, 9\} tanlanma olingan bo'lsa, ushbu tanlanmalarning tanlanma o'rta va tanlanma dispersiyalarin toping.
\\
\textbf{B1.} 
Agar o'rta kvadratik chetlanish \(\sigma = 4\) bo'lgan normal taqsimot bosh to'plamdan olingan hajmi \(n = 12\)ga teng tanlanma bo'yicha \(\overline{x} = 3\) tanlanma o'rta qiymati topilgan bo'lsa, u holda \(\gamma = 0,95\) ishonchlilik bilan noma'lum \(\theta\) matematik kutilma uchun ishonchlilik intervalin tuzing .
\\
\textbf{B2.} 
Agar \(X^{(n)} = \left( X_{1},...,X_{n} \right)\) tanlanma \(\theta\) parametrli Bernulli taqsimotidan olingan bo'lsa, u holda noma'lum \(\theta\) parametr uchun momentlar usuli bahosini toping.
\\
\textbf{B3.} 
Agar \(X^{(n)} = \left( X_{1},...,X_{n} \right)\) tanlanma zichlik funksiyasi \(f(x;\theta) = \frac{2x}{\theta}e^{- \frac{x^{2}}{\theta}},\ \ x \geq 0\) bo'lgan taqsimotdan olingan bo'lsa, u holda noma'lum \(\theta > 0\) parametrning haqiqatga maksimal usuli bahosini toping.
\\
\textbf{C1.} 
Agar \(X^{(n)} = \left( X_{1},...,X_{n} \right)\) tanlanma \(1\sqrt{\theta}\) parametrli ko'rsatkichli taqsimotdan olingan bo'lsa, u holda noma'lum \(\theta\) parametr uchun \((\overline{x})^{2}\) bahoning siljimaganligi va asosliligini tekshiring.
\\
\textbf{C2.} 
Agar \(X^{(n)} = \left( X_{1},...,X_{n} \right)\) tanlanma \(1/\theta\) parametrli ko'rsatkichli taqsimotdan olingan bo'lsa, u holda noma'lum \(\theta\) parametr uchun momentlar usuli bahosini \(\ \ g(x) = x^{k},\) \(k \in N\) funksiya yordamida toping.
\\
\textbf{C3.} 
\(f(x;\theta) = \frac{7x^{6}}{\sqrt{2\pi}}\exp\left\{ - \frac{(x^{7} - \theta)^{2}}{2} \right\}\) model uchun \(\theta\) parametri haqiqatga maksimal o'xshashlik usuli bahosi topilsin.
\\

\end{tabular}
\vspace{1cm}


\begin{tabular}{m{17cm}}
\textbf{61-variant}
\newline

\textbf{T1.} 
Tanlanma momentleri (\(k -\)tartibli boshlang'ich, boshlang'ich absolyut, markaziy va markaziy absolyut momentler).
\\
\textbf{T2.} 
Kolmogorovning muvofiqlik kritireyesi (Kolmogorov teoremasi)
\\
\textbf{A1.} 
Hajmi \(n = 20\) ga teng bo'lgan tanlanma berilgan: 3,7; 3,1; 4,8; 2,8; 3,1; 4,3; 3,7; 4,3; 2,4; 3,1; 2,4; 4,3; 3,1; 3,7; 4,8; 2,8; 2,4; 2,8; 2,4; 3,1. Bu tanlanmaning statistik taqsimotin toping.
\\
\textbf{A2.} 
Hajmi \(n = 20\) ga teng bo'lgan tanlanma berilgan: 3,8; 3,4; 4,8; 2,9; 3,4; 4,6; 3,8; 4,6; 2,1; 3,4; 2,1; 4,6; 3,4; 3,8; 4,8; 2,9; 2,1; 2,9; 2,1; 3,4. Bu tanlanmaning empirik taqsimot funksiyasin toping.
\\
\textbf{A3.} 
Oliy matematika fanidan 10 ta talaba test topshiriqlarin topshirdi. Harbir talaba 10 balgacha to'plashi mumkin. Agar test topshiriqlari natijalari bo'yicha \{6, 5, 6, 9, 5, 7, 10, 5, 9, 8\} tanlanma olingan bo'lsa, ushbu tanlanmalarning tanlanma o'rta va tanlanma dispersiyalarin toping.
\\
\textbf{B1.} 
Agar normal taqsimlangan bosh to'plamdan olingan hajmi \(n = 25\)ga teng tanlanma bo'yicha \(\overline{x} = 9\) tanlanma o'rta va \({\overline{S}}^{2} = 0,64\) tuzatilgan tanlanma dispersiyalar topilgan bo'lsa, u holda \(\gamma = 0,95\) ishonchlilik bilan noma'lum \(\theta\) matematik kutilma uchun ishonchlilik intervalin tuzing.
\\
\textbf{B2.} 
Ko'rsatkichli taqsimot noma'lum \(\theta > 0\) parametri momentlar usuli bahosini toping.
\\
\textbf{B3.} 
Agar \(X^{(n)} = \left( X_{1},...,X_{n} \right)\) tanlanma zichlik funksiyasi \(f(x;\theta) = \frac{2x}{\theta}e^{- \frac{x^{2}}{\theta}},\ \ x \geq 0\) bo'lgan taqsimotdan olingan bo'lsa, u holda noma'lum \(\theta > 0\) parametrning haqiqatga maksimal usuli bahosini toping.
\\
\textbf{C1.} 
Agar \(X^{(n)} = \left( X_{1},...,X_{n} \right)\) tanlanma \(\sqrt{\theta}\) parametrli Bernulli taqsimotidan olingan bo'lsa, u holda noma'lum \(\theta\) parametr uchun \((\overline{x})^{2}\) bahoning siljimaganligi va asosliligini tekshiring.
\\
\textbf{C2.} 
Agar \(X^{(n)} = \left( X_{1},...,X_{n} \right)\) tanlanma \((\theta,\theta^{2})\) parametrli normal taqsimotdan \(\ \ g(x) = (x)^{2}\ \ \) olingan bo'lsa, u holda noma'lum \(\theta > 0\) parametr uchun momentlar usuli bahosini funksiya yordamida toping.
\\
\textbf{C3.} 
Agar \(X^{(n)} = \left( X_{1},...,X_{n} \right)\) tanlanma zichlik funksiyasi \(f(x;\theta) = \frac{\theta ln^{\theta - 1}x}{x},\ \ x \in \lbrack 1,e\rbrack\) bo'lgan taqsimotdan olingan bo'lsa, u holda noma'lum \(\theta > 0\) parametr uchun haqiqatga maksimal o'xshashlik bahosini toping.
\\

\end{tabular}
\vspace{1cm}


\begin{tabular}{m{17cm}}
\textbf{62-variant}
\newline

\textbf{T1.} 
Guruhlangan va interval variatsion qatorlar.
\\
\textbf{T2.} 
Pirsonning xi-kvadrat muvofiqlik kriteriysi (Pirson teoremasi).
\\
\textbf{A1.} 
Hajmi \(n = 20\) ga teng bo'lgan tanlanma berilgan: 1,5; -0,9; -2,4; -0,9; 0,7; 1,5; -0,9; -0,2; -2,4; 0,7; -2,4; 0,7; -0,9; 1,5; -1,7; -0,9; -0,2; 0,7; -1,7; -0,9. Bu tanlanmaning statistik taqsimotin toping.
\\
\textbf{A2.} 
Hajmi \(n = 20\) ga teng bo'lgan tanlanma berilgan: 1,9; -0,3; -2,7; -0,3; 0,6; 1,9; -0,3; -0,1; -2,7; 0,6; -2,7; 0,6; -0,3; 1,9; -1,8; -0,3; -0,1; 0,6; -1,8; -0,3. Bu tanlanmaning empirik taqsimot funksiyasin toping.
\\
\textbf{A3.} 
Oliy matematika fanidan 10 ta talaba test topshiriqlarin topshirdi. Harbir talaba 10 balgacha to'plashi mumkin. Agar test topshiriqlari natijalari bo'yicha \{4, 6, 6, 9, 5, 8, 4, 7, 5, 6\} tanlanma olingan bo'lsa, ushbu tanlanmalarning tanlanma o'rta va tanlanma dispersiyalarin toping.
\\
\textbf{B1.} 
Agar normal taqsimlangan bosh to'plamdan olingan hajmi \(n = 10\) ga teng bo'lgan tanlanma bo'yicha \({\overline{S}}^{2} = 0,6\) tuzatilgan tanlanma dispersiya topilgan bo'lsa, u holda \(\gamma = 0,95\) ishonchlilik bilan noma'lum \(\theta_{2}^{2}\) dispersiya uchun ishonchlilik intervalin tuzing.
\\
\textbf{B2.} 
\(\left\lbrack \theta_{1},\theta_{2} \right\rbrack\) oraliqda tekis taqsimot parametrlari uchun momentlar usuli baholarini toping.
\\
\textbf{B3.} 
Agar (-1,-1,0,-1,0,-1,-1,5,-1,0,-1,0,5,-1,-1,-1,5,-1,-1,-1,1,-1,5,0,-1,-1,5) tanlanma quyida berilgan taqsimotdan olingan bo'lsa, u holda noma'lum \(\theta\) parametrning haqiqatga maksimal o'xshashlik usuli bahosini toping.
$\begin{array}{|c|c|c|c|}
    \hline
    \xi & - 1 & 0 & 5\\
    \hline
    P_{\theta} & 1 - \theta & \theta/2 & \theta/2 \\
    \hline
\end{array}$
\\
\textbf{C1.} 
Agar \(X^{(n)} = \left( X_{1},...,X_{n} \right)\) tanlanma \(\theta\) parametrli Bernulli taqsimotidan olingan bo'lsa, u holda noma'lum \(\theta\) parametr uchun \(X_{n}\) bahoning siljimaganligi va asosliligini tekshiring.
\\
\textbf{C2.} 
Agar \(X^{(n)} = \left( X_{1},...,X_{n} \right)\) tanlanma \((\theta,2\theta)\) parametrli normal taqsimotdan olingan bo'lsa, u holda noma'lum \(\theta > 0\) parametr uchun momentlar usuli bahosini \(\ \ g(x) = (x)^{2}\) funksiya yordamida toping.
\\
\textbf{C3.} 
Agar \(X^{(n)} = \left( X_{1},...,X_{n} \right)\) tanlanma \(\theta\) parametrli geometrik taqsimotdan olingan bo'lsa, u holda noma'lum \(\theta\) parametrning haqiqatga maksimal o'xshashlik usuli bahosini toping.
\\

\end{tabular}
\vspace{1cm}


\begin{tabular}{m{17cm}}
\textbf{63-variant}
\newline

\textbf{T1.} Matematik statistikaning asosiy masalalari. (Statistik ma'lumotlar, guruhlash)
\\
\textbf{T2.} 
Haqiqatga maksimal o'xshashlik usuli. (haqiqatga maksimal o'xshashlik funktsiyasi, noma'lum parametrlarni baholash).
\\
\textbf{A1.} 
Hajmi \(n = 20\) ga teng bo'lgan tanlanma berilgan:9,4; 6,8; -8,5; 9,4; 2,9; 9,4; -8,5; -6,4; 6,8; -8,5; 9,4; -6,4; 6,8; 9,4; 2,9; 9,4; -3,6; -8,5; 2,9; -6,4. Bu tanlanmaning statistik taqsimotin toping.
\\
\textbf{A2.} 
Hajmi \(n = 20\) ga teng bo'lgan tanlanma berilgan:9,1; 6,4; -8,6; 9,1; 2,3; 9,1; -8,6; -6,2; 6,4; -8,6; 9,1; -6,2; 6,4; 9,1; 2,3; 9,1; -3,9; -8,6; 2,3; -6,2. Bu tanlanmaning empirik taqsimot funksiyasin toping.
\\
\textbf{A3.} 
Oliy matematika fanidan 10 ta talaba test topshiriqlarin topshirdi. Harbir talaba 10 balgacha to'plashi mumkin. Agar test topshiriqlari natijalari bo'yicha \{3, 7, 6, 4, 5, 4, 3, 7, 8, 3\} tanlanma olingan bo'lsa, ushbu tanlanmalarning tanlanma o'rta va tanlanma dispersiyalarin toping.
\\
\textbf{B1.} 
Agar o'rta kvadratik chetlanish \(\sigma = 5\) bo'lgan normal taqsimot bosh to'plamdan olingan hajmi \(n = 16\)ga teng tanlanma bo'yicha \(\overline{x} = 3,6\) tanlanma o'rta qiymati topilgan bo'lsa, u holda \(\gamma = 0,90\) ishonchlilik bilan noma'lum \(\theta\) matematik kutilma uchun ishonchlilik intervalin tuzing .
\\
\textbf{B2.} 
Agar (3,-2,-2,0,-2,2,-2,0,-2,3,-2,0,3,0,3,-2,0,-2,3,-2,2,-2,-2,3,3,2,-2,2,3,3) tanlanma quyida berilgan taqsimotdan olingan bo'lsa, u holda noma'lum \(\theta\) parametr uchun momentlar usuli bahosini \(g(x) = |x|\) funksiya yordamida toping.
$\begin{array}{|c|c|c|c|}
    \hline
    \xi & -2 & 0 & 3 \\
    \hline
    P_{\theta} & 3\theta & 1 - 5\theta & 2\theta \\
    \hline
\end{array}$
\\
\textbf{B3.} 
Agar (4,8,5,3) tanlanma \((a,\theta^{2}\) parametrli normal taqsimotdan olingan bo'lsa, u holda noma'lum \(\theta^{2}\) parametrning haqiqatga maksimal o'xshashlik bahosini toping.
\\
\textbf{C1.} 
Agar \(X^{(n)} = \left( X_{1},...,X_{n} \right)\) tanlanma \(\theta\) parametrli Bernulli taqsimotidan olingan bo'lsa, u holda noma'lum \(\theta(1 - \theta)\) parametr uchun \(X_{1}\left( 1 - X_{n} \right)\) bahoning siljimaganligi va asosliligini tekshiring.
\\
\textbf{C2.} 
Agar \(X^{(n)} = \left( X_{1},...,X_{n} \right)\) tanlanma zichlik funksiyasi\(f(x,\theta) = \left\{ \begin{matrix}
\theta_{1}^{- 1}e^{- \ \frac{x - \theta_{2}}{\theta_{1}}},\ \ x \geq \theta_{2}, \\
0,\ \ x < \theta_{2}
\end{matrix} \right.\ \)bo'lgan taqsimotdan olingan bo'lsa, u holda noma'lum \(\left( \theta_{1},\theta_{2} \right)\) \(\theta_{1} > 0,\) \(\theta_{2} \in R\) vektor parametr uchun momentlar usuli bahosini toping.
\\
\textbf{C3.} 
Agar \(X^{(n)} = \left( X_{1},...,X_{n} \right)\) tanlanma zichlik funksiyasi \(f(x;\theta) = \left\{ \begin{array}{r}
3x^{2}\theta^{- 3}{e^{- \left( \frac{x}{\theta} \right)}}^{3},\ \ \ \ x \geq 0 \\
0,\ \ \ \ \ \ \ \ \ \ \ x < 0
\end{array} \right.\ \) bo'lgan taqsimotdan olingan bo'lsa, u holda noma'lum \(\theta > 0\) parametrning haqiqatga maksimal o'xshashlik bahosini toping.
\\

\end{tabular}
\vspace{1cm}


\begin{tabular}{m{17cm}}
\textbf{64-variant}
\newline

\textbf{T1.} 
Empirik taqsimot funktsiyasi. (Tanlanma, eksperiment)
\\
\textbf{T2.} 
Statistik gipotezalarni tekshirish (kritik to'plam, 1 va 2-tur xatolik)
\\
\textbf{A1.} 
Hajmi \(n = 20\) ga teng bo'lgan tanlanma berilgan: 6,2; -5,3; 7,2; 3,7; -2,2; 6,2; 3,7; -7,6; 3,7; 7,2; 6,2; -5,3; -7,6; -5,3; -7,6; 6,2; 7,2; -2,2; -7,6; 7,2. Bu tanlanmaning statistik taqsimotin toping.
\\
\textbf{A2.} 
Hajmi \(n = 20\) ga teng bo'lgan tanlanma berilgan: 6,1; -5,8; 7,9; 3,5; -2,5; 6,1; 3,5; -7,2; 3,5; 7,9; 6,1; -5,8; -7,2; -5,8; -7,2; 6,1; 7,9; -2,5; -7,2; 7,9. Bu tanlanmaning empirik taqsimot funksiyasin toping.
\\
\textbf{A3.} 
Oliy matematika fanidan 10 ta talaba test topshiriqlarin topshirdi. Harbir talaba 10 balgacha to'plashi mumkin. Agar test topshiriqlari natijalari bo'yicha \{10, 8, 6, 5, 4, 8, 10, 7, 5, 7\} tanlanma olingan bo'lsa, ushbu tanlanmalarning tanlanma o'rta va tanlanma dispersiyalarin toping.
\\
\textbf{B1.} 
Agar normal taqsimlangan bosh to'plamdan olingan hajmi \(n = 16\)ga teng tanlanma bo'yicha \(\overline{x} = 15,2\) tanlanma o'rta va \({\overline{S}}^{2} = 0,81\) tuzatilgan tanlanma dispersiyalar topilgan bo'lsa, u holda \(\gamma = 0,90\) ishonchlilik bilan noma'lum \(\theta\) matematik kutilma uchun ishonchlilik intervalin tuzing.
\\
\textbf{B2.} 
Agar (0,-2,0,-2,3,-2,0,0,3,0,0,0,0,3,-2,0,0,-2,3,0,3) tanlanma quyida berilgan taqsimotdan olingan bo'lsa, u holda noma'lum \(\theta\) parametr uchun momentlar usuli bahosini toping.
$\begin{array}{|c|c|c|c|}
    \hline
    \xi & - 2 & 0 & 3 \\
    \hline
    P_{\theta} & \theta & 1 - 2\theta & \theta \\
    \hline
\end{array}$
\\
\textbf{B3.} 
Agar \(X^{(n)} = \left( X_{1},...,X_{n} \right)\) tanlanma \(\theta\) parametrli ko'rsatkichli taqsimotdan olingan bo'lsa, u holda noma'lum \(\theta\) parametrning haqiqatga maksimal o'xshashlik usuli bahosini toping.
\\
\textbf{C1.} 
Agar \(X^{(n)} = \left( X_{1},...,X_{n} \right)\) tanlanma \(\theta\) parametrli Bernulli taqsimotidan olingan bo'lsa, u holda noma'lum \(\theta^{2}\) parametr uchun \(X_{1}X_{n}\) bahoning siljimaganligi va asosliligini tekshiring.
\\
\textbf{C2.} 
Agar \(X^{(n)} = \left( X_{1},...,X_{n} \right)\) tanlanma \(1\sqrt{\theta}\) parametrli ko'rsatkichli taqsimotdan olingan bo'lsa, u holda noma'lum \(\theta\) parametr uchun momentlar usuli bahosini toping.
\\
\textbf{C3.} 
Agar \(X^{(n)} = \left( X_{1},...,X_{n} \right)\) tanlanma zichlik funksiyasi \(f(x;\theta) = \frac{e^{x}}{\sqrt{2\pi}}\exp\left\{ - \frac{\left( e^{x} - \theta \right)^{2}}{2} \right\},\ \ x \in R\) bo'lgan taqsimotdan olingan bo'lsa, u holda noma'lum \(\theta\) parametrning haqiqatga maksimal o'xshashlik bahosini toping.
\\

\end{tabular}
\vspace{1cm}


\begin{tabular}{m{17cm}}
\textbf{65-variant}
\newline

\textbf{T1.} 
Momentler usuli. (tanlanma momentleri, noma'lum parametrlarni baholash).
\\
\textbf{T2.} 
Momentler usuli. (tanlanma momentleri, noma'lum parametrlarni baholash).
\\
\textbf{A1.} 
Hajmi \(n = 20\) ga teng bo'lgan tanlanma berilgan: 9,6; 1,5; 7,4; 9,6; 2,8; 1,5; 6,3; 1,5; 9,6; 6,3; 2,8; 4,1; 6,3; 9,6; 1,5; 1,5; 6,3; 7,4; 4,1; 7,4. Bu tanlanmaning statistik taqsimotin toping.
\\
\textbf{A2.} 
Hajmi \(n = 20\) ga teng bo'lgan tanlanma berilgan: 9,8; 1,2; 7,1; 9,8; 2,9; 1,2; 6,7; 1,2; 9,8; 6,7; 2,9; 4,6; 6,7; 9,8; 1,2; 1,2; 6,7; 7,1; 4,6; 7,1. Bu tanlanmaning empirik taqsimot funksiyasin toping.
\\
\textbf{A3.} 
Oliy matematika fanidan 10 ta talaba test topshiriqlarin topshirdi. Harbir talaba 10 balgacha to'plashi mumkin. Agar test topshiriqlari natijalari bo'yicha \{9, 10, 5, 6, 4, 8, 4, 6, 10, 8\} tanlanma olingan bo'lsa, ushbu tanlanmalarning tanlanma o'rta va tanlanma dispersiyalarin toping.
\\
\textbf{B1.} 
Agar normal taqsimlangan bosh to'plamdan olingan hajmi \(n = 10\) ga teng bo'lgan tanlanma bo'yicha \({\overline{S}}^{2} = 0,45\) tuzatilgan tanlanma dispersiya topilgan bo'lsa, u holda \(\gamma = 0,95\) ishonchlilik bilan noma'lum \(\theta_{2}^{2}\) dispersiya uchun ishonchlilik intervalin tuzing.
\\
\textbf{B2.} 
Agar (-2,0,-2,0,-2,3,-2,0,0,3,0,0,0,0,3,-2,0,0,-2,3,0) tanlanma quyida berilgan taqsimotdan olingan bo'lsa, u holda noma'lum \(\left( \theta_{1},\theta_{2} \right)\) vektor parametr uchun momentlar usuli bahosini toping.
$\begin{array}{|c|c|c|c|}
    \hline
    \xi & - 2 & 0 & 3\\
    \hline
    P_{\theta} & \theta_{1} & 1 - \theta_{1} - \theta_{2} & \theta_{2} \\
    \hline
\end{array}$
\\
\textbf{B3.} 
\(f(x) = \frac{\theta}{2}e^{- \theta|x|}\) model uchun \(\theta\) parametri haqiqatga maksimal o'xshashlik usuli bahosi topilsin.
\\
\textbf{C1.} 
Agar \(X^{(n)} = \left( X_{1},...,X_{n} \right)\) tanlanma \((\alpha,\theta)\) parametrli Veybull taqsimotdan olingan bo'lsa (\(\alpha -\) ma'lum), u holda noma'lum \(\theta\) parametr uchun \(1/\overline{x^{\alpha}}\) bahoning siljimaganligi va asosliligini tekshiring.
\\
\textbf{C2.} 
Agar \(X^{(n)} = \left( X_{1},...,X_{n} \right)\) tanlanma \({\lbrack\theta}_{1},\theta_{2}\rbrack\) oraliqda tekis taqsimotdan olingan bo'lsa, u holda noma'lum \(\left( \theta_{1},\theta_{2} \right)\) vektor parametr uchun momentlar usuli bahosini toping.
\\
\textbf{C3.} 
Agar \(X^{(n)} = \left( X_{1},...,X_{n} \right)\) tanlanma \(\left\lbrack \theta_{1},\theta_{2} \right\rbrack\) oraliqda tekis taqsimotdan olingan bo'lsa, u holda noma'lum \(\left( \theta_{1},\theta_{2} \right)\) vektor parametrning haqiqatga maksimal o'xshashlik bahosini toping.
\\

\end{tabular}
\vspace{1cm}


\begin{tabular}{m{17cm}}
\textbf{66-variant}
\newline

\textbf{T1.} 
Poligon va gistogramma(nisbiy chastota, interval qator, grafik).
\\
\textbf{T2.} 
Normal qonun dispersiyasi uchun ishonchlilik intervalin tuzish. (Ishonchlilik ehtimolligi, interval)
\\
\textbf{A1.} 
Hajmi \(n = 20\) ga teng bo'lgan tanlanma berilgan:1,8; -8,4; 7,3; 4,7; -3,9; 1,8; 4,7; -10,4; -8,4; 7,3; -10,4; 4,7; -8,4; 1,8; 4,7; -10,4; 7,3; -3,9; 4,7; -8,4. Bu tanlanmaning statistik taqsimotin toping.
\\
\textbf{A2.} 
Hajmi \(n = 20\) ga teng bo'lgan tanlanmaberilgan:1,6; -8,3; 7,6; 4,2; -3,1; 1,6; 4,2; -10,5; -8,3; 7,6; -10,5; 4,2; -8,3; 1,6; 4,2; -10,5; 7,6; -3,1; 4,2; -8,3. Bu tanlanmaning empirik taqsimot funksiyasin toping.
\\
\textbf{A3.} 
Oliy matematika fanidan 10 ta talaba test topshiriqlarin topshirdi. Harbir talaba 10 balgacha to'plashi mumkin. Agar test topshiriqlari natijalari bo'yicha \{9, 3, 6, 3, 7, 6, 4, 6, 10, 6\} tanlanma olingan bo'lsa, ushbu tanlanmalarning tanlanma o'rta va tanlanma dispersiyalarin toping.
\\
\textbf{B1.} 
Agar o'rta kvadratik chetlanish \(\sigma = 2\) bo'lgan normal taqsimot bosh to'plamdan olingan hajmi \(n = 18\)ga teng tanlanma bo'yicha \(\overline{x} = 5,2\) tanlanma o'rta qiymati topilgan bo'lsa, u holda \(\gamma = 0,90\) ishonchlilik bilan noma'lum \(\theta\) matematik kutilma uchun ishonchlilik intervalin tuzing .
\\
\textbf{B2.} 
Agar \(X^{(n)} = \left( X_{1},...,X_{n} \right)\) tanlanma \(\theta\) parametrli ko'rsatkichli taqsimotdan olingan bo'lsa, u holda noma'lum \(\theta\) parametr uchun momentlar usuli bahosini toping.
\\
\textbf{B3.} 
Agar \(x_{1} = 1,1;\ \ x_{2} = 2,7;\ldots;x_{100} = 1,5\) tanlanma \(\theta\) parametrli ko'rsatkichli taqsimotdan olingan bo'lib, \(\sum_{k = 1}^{100}x_{k} = 200\) bo'lsa, u holda noma'lum \(\theta\) parametrning haqiqatga maksimal o'xshashlik bahosini toping.
\\
\textbf{C1.} 
Agar \(X^{(n)} = \left( X_{1},...,X_{n} \right)\) tanlanma \(\theta\) parametrli geometrik taqsimotdan olingan bo'lsa, u holda noma'lum \(\theta\) parametr uchun \(t(1 + \overline{x})\) bahoning siljimaganligi va asosliligini tekshiring.
\\
\textbf{C2.} 
Agar \(X^{(n)} = \left( X_{1},...,X_{n} \right)\) tanlanma \({\lbrack\theta}_{1},\theta_{1} + \theta_{2}\rbrack\) oraliqda tekis taqsimotdan olingan bo'lsa, u holda noma'lum \(\left( \theta_{1},\theta_{2} \right)\) vektor parametr uchun momentlar usuli bahosini toping.
\\
\textbf{C3.} 
Agar \(X^{(n)} = \left( X_{1},...,X_{n} \right)\) tanlanma \((\theta,2\theta)\) parametrli normal taqsimotdan olingan bo'lsa, u holda noma'lum \(\theta > 0\) parametrning haqiqatga maksimal o'xshashlik bahosini toping.
\\

\end{tabular}
\vspace{1cm}


\begin{tabular}{m{17cm}}
\textbf{67-variant}
\newline

\textbf{T1.} 
Tanlanma xarakteristikalar. (Variatsion qator, nisbiy chastota).
\\
\textbf{T2.} 
Statistik gipotezalarni tekshirish (kritik to'plam, 1 va 2-tur xatolik).
\\
\textbf{A1.} 
Hajmi \(n = 20\) ga teng bo'lgan tanlanma berilgan: 2,7; -13,5; 1,2; 2,7; 1,2; 4,9; -9,5; 1,2; 2,7; 4,9; -9,5; 2,7; -3,5; 1,2; 2,7; 4,9; -3,5; 2,7; 4,9; 1,2;. Bu tanlanmaning statistik taqsimotin toping.
\\
\textbf{A2.} 
Hajmi \(n = 20\) ga teng bo'lgan tanlanma berilgan: 2,8; -13,9; 1,9; 2,8; 1,9; 4,3; -9,4; 1,9; 2,8; 4,3; -9,4; 2,8; -3,7; 1,9; 2,8; 4,3; -3,7; 2,8; 4,3; 1,9. Bu tanlanmaning empirik taqsimot funksiyasin toping.
\\
\textbf{A3.} 
Oliy matematika fanidan 10 ta talaba test topshiriqlarin topshirdi. Harbir talaba 10 balgacha to'plashi mumkin. Agar test topshiriqlari natijalari bo'yicha \{10, 7, 5, 9, 3, 8, 10, 7, 8, 3\} tanlanma olingan bo'lsa, ushbu tanlanmalarning tanlanma o'rta va tanlanma dispersiyalarin toping.
\\
\textbf{B1.} 
Agar normal taqsimlangan bosh to'plamdan olingan hajmi \(n = 36\)ga teng tanlanma bo'yicha \(\overline{x} = 20,2\) tanlanma o'rta va \({\overline{S}}^{2} = 0,81\) tuzatilgan tanlanma dispersiyalar topilgan bo'lsa, u holda \(\gamma = 0,95\) ishonchlilik bilan noma'lum \(\theta\) matematik kutilma uchun ishonchlilik intervalin tuzing.
\\
\textbf{B2.} 
Agar (-2,0,-2,0,-2,3,-2,0,0,3,0,0,0,0,3,-2,0,0,-2,3,0) tanlanma quyida berilgan taqsimotdan olingan bo'lsa, u holda noma'lum \(\left( \theta_{1},\theta_{2} \right)\) vektor parametr uchun momentlar usuli bahosini toping.
$\begin{array}{|c|c|c|c|}
    \hline
    \xi & - 2 & 0 & 3\\
    \hline
    P_{\theta} & \theta_{1} & 1 - \theta_{1} - \theta_{2} & \theta_{2} \\
    \hline
\end{array}$
\\
\textbf{B3.} 
Agar (0,1,2,0) tanlanma quyida berilgan taqsimotdan olingan bo'lsa, u holda noma'lum \(\theta\) parametrning haqiqatga maksimal o'xshashlik bahosini toping.
$\begin{array}{|c|c|c|c|}
    \hline
    \xi & 0 & 1 & 2 \\
    \hline
    P_{\theta} & \theta & 2\theta & 1 - 3\theta \\
    \hline
\end{array}$
\\
\textbf{C1.} 
Agar \(X^{(n)} = \left( X_{1},...,X_{n} \right)\) tanlanma \(\theta\) parametrli Puasson taqsimotidan olingan bo'lsa, u holda noma'lum \(\theta\) parametr uchun \(\frac{n + 3}{n + 4}\overline{x}\) bahoning siljimaganligi va asosliligini tekshiring.
\\
\textbf{C2.} 
Agar \(X^{(n)} = \left( X_{1},...,X_{n} \right)\) tanlanma zichlik funksiyasi\(f(x,\theta) = \theta x^{\theta - 1},x \in \lbrack 0,1\rbrack\)bo'lgan taqsimotdan olingan bo'lsa, u holda noma'lum \(\theta\) parametr uchun momentlar usuli bahosini toping.
\\
\textbf{C3.} 
Agar \(X^{(n)} = \left( X_{1},...,X_{n} \right)\) tanlanma zichlik funksiyasi \(f(x;\theta) = \left\{ \begin{array}{r}
\begin{matrix}
\theta_{1}^{- 1}e^{\frac{x - \theta_{2}}{\theta_{1}}},\ \ x \geq \theta_{2}
\end{matrix} \\
0,\ \ \ \ x < \theta_{2}
\end{array} \right.\ \) bo'lgan taqsimotdan olingan bo'lsa, u holda noma'lum \(.\left( \theta_{1},\theta_{2} \right),\) \(\theta_{1} > 0,\) \(\theta_{2} \in R\) vektor parametrning haqiqatga maksimal o'xshashlik bahosini toping.
\\

\end{tabular}
\vspace{1cm}


\begin{tabular}{m{17cm}}
\textbf{68-variant}
\newline

\textbf{T1.} 
Neyman-Pirson teoremasi.
\\
\textbf{T2.} 
Statistik baho xossalari. (Siljimagan, asosliy, effektiv)
\\
\textbf{A1.} 
Hajmi \(n = 20\) ga teng bo'lgan tanlanma berilgan: 9,9; 5,7; 3,2; 2,8; 5,7; 9,9; 7,5; 3,7; 9,9; 3,2; 2,8; 3,7; 7,5; 5,7; 3,2; 2,8; 7,5; 3,2; 9,9; 7,5. Bu tanlanmaning statistik taqsimotin toping.
\\
\textbf{A2.} 
Hajmi \(n = 20\) ga teng bo'lgan tanlanma berilgan: 9,7; 5,2; 3,2; 2,4; 5,2; 9,7; 7,5; 3,7; 9,7; 3,2; 2,4; 3,7; 7,5; 5,2; 3,2; 2,4; 7,5; 3,2; 9,7; 7,5. Bu tanlanmaning empirik taqsimot funksiyasin toping.
\\
\textbf{A3.} 
Oliy matematika fanidan 10 ta talaba test topshiriqlarin topshirdi. Harbir talaba 10 balgacha to'plashi mumkin. Agar test topshiriqlari natijalari bo'yicha \{1, 6, 2, 6, 3, 6, 4, 6, 10, 6\} tanlanma olingan bo'lsa, ushbu tanlanmalarning tanlanma o'rta va tanlanma dispersiyalarin toping.
\\
\textbf{B1.} 
Agar normal taqsimlangan bosh to'plamdan olingan hajmi \(n = 10\) ga teng bo'lgan tanlanma bo'yicha \({\overline{S}}^{2} = 0,7\) tuzatilgan tanlanma dispersiya topilgan bo'lsa, u holda \(\gamma = 0,95\) ishonchlilik bilan noma'lum \(\theta_{2}^{2}\) dispersiya uchun ishonchlilik intervalin tuzing.
\\
\textbf{B2.} 
Ko'rsatkichli taqsimot noma'lum \(\theta > 0\) parametri momentlar usuli bahosini toping.
\\
\textbf{B3.} 
Agar \(X^{(n)} = \left( X_{1},...,X_{n} \right)\) tanlanma \(\theta\) parametrli Bernulli taqsimotidan olingan bo'lsa, u holda noma'lum \(\theta\) parametrning haqiqatga maksimal o'xshashlik usuli bahosini toping.
\\
\textbf{C1.} 
Agar \(X^{(n)} = \left( X_{1},...,X_{n} \right)\) tanlanma \(\theta\) parametrli Puasson taqsimotidan olingan bo'lsa, u holda noma'lum \(\theta\) parametr uchun \(\frac{X_{1} + X_{3}}{2}\) bahoning siljimaganligi va asosliligini tekshiring.
\\
\textbf{C2.} 
Agar \(X^{(n)} = \left( X_{1},...,X_{n} \right)\) tanlanma {[}\(0,2\theta\rbrack\) oraliqda tekis taqsimotdan olingan bo'lsa, u holda noma'lum \(\theta > 0\) parametr uchun momentlar usuli bahosini toping.
\\
\textbf{C3.} 
Agar \(X^{(n)} = \left( X_{1},...,X_{n} \right)\) tanlanma \(\lbrack\theta,\theta + 2\rbrack\) oraliqda tekis taqsimotdan olingan bo'lsa, u holda noma'lum \(\theta\) parametrning haqiqatga maksimal o'xshashlik usuli bahosini toping.
\\

\end{tabular}
\vspace{1cm}


\begin{tabular}{m{17cm}}
\textbf{69-variant}
\newline

\textbf{T1.} 
Tanlanma xarakteristikalari.(tanlanma o'rta, tanlanma dispersiya).
\\
\textbf{T2.} 
Ishonchlilik intervallarin tuzish. Aniq ishonchlilik intervallar.
\\
\textbf{A1.} 
Hajmi \(n = 20\) ga teng bo'lgan tanlanma berilgan: 3,6; 1,1; -1,8; 0,4; 3,6; 0; 5,3; 1,1; 0; -1,8; 3,6; 0,4; 1,1; 0; 0,4; 1,1; 3,6; -1,8; 3,6; 0. Bu tanlanmaning statistik taqsimotin toping.
\\
\textbf{A2.} 
Hajmi \(n = 20\) ga teng bo'lgan tanlanma berilgan: 3,2; 1,8; -1,1; 0,9; 3,2; 0; 5,6; 1,8; 0; -1,1; 3,2; 0,9; 1,8; 0; 0,9; 1,8; 3,2; -1,1; 3,2; 0. Bu tanlanmaning empirik taqsimot funksiyasin toping.
\\
\textbf{A3.} 
Oliy matematika fanidan 10 ta talaba test topshiriqlarin topshirdi. Harbir talaba 10 balgacha to'plashi mumkin. Agar test topshiriqlari natijalari bo'yicha \{2, 7, 3, 7, 6, 7, 4, 7, 7, 10\} tanlanma olingan bo'lsa, ushbu tanlanmalarning tanlanma o'rta va tanlanma dispersiyalarin toping.
\\
\textbf{B1.} 
Agar o'rta kvadratik chetlanish \(\sigma = 3\) bo'lgan normal taqsimot bosh to'plamdan olingan hajmi \(n = 14\)ga teng tanlanma bo'yicha \(\overline{x} = 5,5\) tanlanma o'rta qiymati topilgan bo'lsa, u holda \(\gamma = 0,90\) ishonchlilik bilan noma'lum \(\theta\) matematik kutilma uchun ishonchlilik intervalin tuzing .
\\
\textbf{B2.} 
Agar \(X^{(n)} = \left( X_{1},...,X_{n} \right)\) tanlanma \(\theta\) parametrli ko'rsatkichli taqsimotdan olingan bo'lsa, u holda noma'lum \(\theta\) parametr uchun momentlar usuli bahosini toping.
\\
\textbf{B3.} 
Agar \(X^{(n)} = \left( X_{1},...,X_{n} \right)\) tanlanma \(\left( a,\theta^{2} \right)\) parametrli normal taqsimotdan olingan bo'lsa (\(\alpha -\) ma'lum), u holda noma'lum \(\theta^{2}\) parametrning haqiqatga maksimal o'xshashlik bahosini toping.
\\
\textbf{C1.} 
Agar \(X^{(n)} = \left( X_{1},...,X_{n} \right)\) tanlanma \(\ln\theta\) parametrli Puasson taqsimotidan olingan bo'lsa, u holda noma'lum \(\theta\) parametr uchun \(e^{\overline{x}}\) bahoning siljimaganligi va asosliligini tekshiring.
\\
\textbf{C2.} 
Agar \(X^{(n)} = \left( X_{1},...,X_{n} \right)\) tanlanma \((\theta,2\theta)\) parametrli normal taqsimotdan olingan bo'lsa, u holda noma'lum \(\theta > 0\) parametr uchun momentlar usuli bahosini toping.
\\
\textbf{C3.} 
Agar \(X^{(n)} = \left( X_{1},...,X_{n} \right)\) tanlanma zichlik funksiyasi\(f(x;\theta) = \frac{\theta}{2}e^{- \theta|x|},\ \ x \in R\) bo'lgan taqsimotdan olingan bo'lsa, u holda noma'lum \(\theta > 0\) parametrning haqiqatga maksimal o'xshashlik bahosini toping.
\\

\end{tabular}
\vspace{1cm}


\begin{tabular}{m{17cm}}
\textbf{70-variant}
\newline

\textbf{T1.} 
Glivenko-Kantelli teoremasi. (empirik taqsimot funktsiyasi, ehtimollik bilan yaqinlashish).
\\
\textbf{T2.} 
Chiziqli korrelyatsiya tenglamasi (ta'rifi, regressiya to'g'ri chiziqning tanlanma tenglamalari)
\\
\textbf{A1.} 
Hajmi \(n = 20\) ga teng bo'lgan tanlanma berilgan: 7,1; 3,9; 6,3; 4,6; 7,1; 2,3; 6,3; 3,9; 4,6; 7,1; 2,3; 3,9; 7,6; 2,3; 4,6; 3,9; 2,3; 3,9; 7,6; 4,6. Bu tanlanmaning statistik taqsimotin toping.
\\
\textbf{A2.} 
Hajmi \(n = 20\) ga teng bo'lgan tanlanma berilgan: 7,9; 3,8; 6,1; 4,2; 7,9; 2,4; 6,1; 3,8; 4,2; 7,9; 2,4; 3,8; 10,2; 2,4; 4,2; 3,8; 2,4; 3,8; 10,2; 4,2. Bu tanlanmaning empirik taqsimot funksiyasin toping.
\\
\textbf{A3.} 
Oliy matematika fanidan 10 ta talaba test topshiriqlarin topshirdi. Harbir talaba 10 balgacha to'plashi mumkin. Agar test topshiriqlari natijalari bo'yicha \{9, 8, 6, 8, 6, 4, 5, 4, 7, 4\} tanlanma olingan bo'lsa, ushbu tanlanmalarning tanlanma o'rta va tanlanma dispersiyalarin toping.
\\
\textbf{B1.} 
Agar normal taqsimlangan bosh to'plamdan olingan hajmi \(n = 49\)ga teng tanlanma bo'yicha \(\overline{x} = 14,2\) tanlanma o'rta va \({\overline{S}}^{2} = 0,64\) tuzatilgan tanlanma dispersiyalar topilgan bo'lsa, u holda \(\gamma = 0,95\) ishonchlilik bilan noma'lum \(\theta\) matematik kutilma uchun ishonchlilik intervalin tuzing.
\\
\textbf{B2.} 
Agar \(X^{(n)} = \left( X_{1},...,X_{n} \right)\) tanlanma \(\theta\) parametrli Bernulli taqsimotidan olingan bo'lsa, u holda noma'lum \(\theta\) parametr uchun momentlar usuli bahosini toping.
\\
\textbf{B3.} 
Agar \(X^{(n)} = \left( X_{1},...,X_{n} \right)\) tanlanma \(\left\lbrack - \theta,\theta^{2} \right\rbrack\) oraliqda tekis taqsimotdan olingan bo'lsa, u holda noma'lum \(\theta > 0\) parametrning haqiqatga maksimal o'xshashlik usuli bahosini toping.
\\
\textbf{C1.} 
Agar \(X^{(n)} = \left( X_{1},...,X_{n} \right)\) tanlanma \((\alpha,\theta)\) parametrli Pareto taqsimotdan olingan bo'lsa (\(\alpha -\) ma'lum), u holda noma'lum \(\theta\) parametr uchun \(X_{(1)}\) bahoning siljimaganligi va asosliligini tekshiring.
\\
\textbf{C2.} 
Agar \(X^{(n)} = \left( X_{1},...,X_{n} \right)\) tanlanma \(\ \ (a,\theta^{2})\ \ \) parametrli normal taqsimotdan olingan bo'lsa (\(\alpha -\) ma'lum), u holda noma'lum \(\ \ \theta^{2}\) parametr uchun momentlar usuli bahosini \(\ \ g(x) = (x - a)^{2}\) funksiyasi yordamida toping.
\\
\textbf{C3.} 
\(f(x,\theta) = \frac{e^{x}}{\sqrt{2\pi}}\exp\left\{ - \frac{\left( e^{x} - \theta \right)^{2}}{2} \right\}\) model uchun \(\theta\) parametri haqiqatga maksimal o'xshashlik usuli bahosi topilsin.
\\

\end{tabular}
\vspace{1cm}


\begin{tabular}{m{17cm}}
\textbf{71-variant}
\newline

\textbf{T1.} 
Tanlanma xarakteristikalar. (Variatsion qator, nisbiy chastota).
\\
\textbf{T2.} 
Haqiqatga maksimal o'xshashlik usuli. (haqiqatga maksimal o'xshashlik funktsiyasi, noma'lum parametrlarni baholash).
\\
\textbf{A1.} 
Hajmi \(n = 20\) ga teng bo'lgan tanlanma berilgan: 0,6; -3,8; -2,3; -4,3; 2,8; 4,7; -2,3; 0,6; -3,8; 2,8; -2,3; -4,3; 0,6; -2,3; 2,8; -3,8; -4,3; -2,3; 2,8; -3,8. Bu tanlanmaning statistik taqsimotin toping.
\\
\textbf{A2.} 
Hajmi \(n = 20\) ga teng bo'lgan tanlanma berilgan: 0,7; -3,1; -2,3; -4,8; 2,6; 4,9; -2,3; 0,7; -3,1; 2,6; -2,3; -4,8; 0,7; -2,3; 2,6; -3,1; -4,8; -2,3; 2,6; -3,1. Bu tanlanmaning empirik taqsimot funksiyasin toping.
\\
\textbf{A3.} 
Oliy matematika fanidan 10 ta talaba test topshiriqlarin topshirdi. Harbir talaba 10 balgacha to'plashi mumkin. Agar test topshiriqlari natijalari bo'yicha \{10, 4, 6, 5, 5, 4, 10, 7, 9, 10\} tanlanma olingan bo'lsa, ushbu tanlanmalarning tanlanma o'rta va tanlanma dispersiyalarin toping.
\\
\textbf{B1.} 
Agar normal taqsimlangan bosh to'plamdan olingan hajmi \(n = 8\) ga teng bo'lgan tanlanma bo'yicha \({\overline{S}}^{2} = 0,35\) tuzatilgan tanlanma dispersiya topilgan bo'lsa, u holda \(\gamma = 0,90\) ishonchlilik bilan noma'lum \(\theta_{2}^{2}\) dispersiya uchun ishonchlilik intervalin tuzing.
\\
\textbf{B2.} 
Agar (0,-2,0,-2,3,-2,0,0,3,0,0,0,0,3,-2,0,0,-2,3,0,3) tanlanma quyida berilgan taqsimotdan olingan bo'lsa, u holda noma'lum \(\theta\) parametr uchun momentlar usuli bahosini toping.
$\begin{array}{|c|c|c|c|}
    \hline
    \xi & - 2 & 0 & 3 \\
    \hline
    P_{\theta} & \theta & 1 - 2\theta & \theta \\
    \hline
\end{array}$
\\
\textbf{B3.} 
Agar \(X^{(n)} = \left( X_{1},...,X_{n} \right)\) tanlanma \(\lbrack - \theta,\theta\rbrack\) oraliqda tekis taqsimotdan olingan bo'lsa, u holda noma'lum \(\theta > 0\) parametrning haqiqatga maksimal o'xshashlik usuli bahosini toping.
\\
\textbf{C1.} 
Agar \(X^{(n)} = \left( X_{1},...,X_{n} \right)\) tanlanma zichlik funksiyasi bo'lsa: \(f(x;\theta) = e^{- x + \theta}\left( 1 + e^{- x + \theta} \right)^{2},\ \ x \in R\)bo'lgan taqsimotdan olingan bo'lsa, u holda noma'lum \(\theta\) parametr uchun \(\overline{x}\) bahoning siljimaganligi va asosliligini tekshiring.
\\
\textbf{C2.} 
Agar \(X^{(n)} = \left( X_{1},...,X_{n} \right)\) tanlanma \((\theta,\theta^{2})\ \ \) parametrli normal taqsimotdan olingan bo'lsa, u holda noma'lum \(\theta > 0\) parametr uchun momentlar usuli bahosini toping.
\\
\textbf{C3.} 
Agar \(X^{(n)} = \left( X_{1},...,X_{n} \right)\) tanlanma zichlik funksiyasi \(f(x;\theta) = \frac{3x^{2}}{\sqrt{2\pi}}\exp\left\{ - \frac{\left( x^{3} - \theta \right)^{2}}{2} \right\},\ \ x \in R\) bo'lgan taqsimotdan olingan bo'lsa, u holda noma'lum \(\theta\) parametrning haqiqatga maksimal o'xshashlik bahosini toping.
\\

\end{tabular}
\vspace{1cm}


\begin{tabular}{m{17cm}}
\textbf{72-variant}
\newline

\textbf{T1.} 
Tanlanma xarakteristikalari.(tanlanma o'rta, tanlanma dispersiya).
\\
\textbf{T2.} 
Kolmogorovning muvofiqlik kritireyesi (Kolmogorov teoremasi)
\\
\textbf{A1.} 
Hajmi \(n = 20\) ga teng bo'lgan tanlanma berilgan: 8,9; 2,7; 1,7; 2,2; 5,6; 1,7; 5,6; 2,7; 1,7; 2,2; 5,6; 8,9; 1,7; 2,2; 1,7; 2,7; 1,7; 5,6; 6,1; 8,9. Bu tanlanmaning statistik taqsimotin toping.
\\
\textbf{A2.} 
Hajmi \(n = 20\) ga teng bo'lgan tanlanma berilgan: 8,7; 2,7; 1,5; 2,2; 5,7; 1,5; 5,7; 2,7; 1,5; 2,2; 5,7; 8,7; 1,5; 2,2; 1,5; 2,7; 1,5; 5,7; 6,3; 8,7. Bu tanlanmaning empirik taqsimot funksiyasin toping.
\\
\textbf{A3.} 
Oliy matematika fanidan 10 ta talaba test topshiriqlarin topshirdi. Harbir talaba 10 balgacha to'plashi mumkin. Agar test topshiriqlari natijalari bo'yicha \{9, 8, 6, 9, 5, 4, 5, 7, 8, 9\} tanlanma olingan bo'lsa, ushbu tanlanmalarning tanlanma o'rta va tanlanma dispersiyalarin toping.
\\
\textbf{B1.} 
Agar o'rta kvadratik chetlanish \(\sigma = 4\) bo'lgan normal taqsimot bosh to'plamdan olingan hajmi \(n = 16\)ga teng tanlanma bo'yicha \(\overline{x} = 5,8\) tanlanma o'rta qiymati topilgan bo'lsa, u holda \(\gamma = 0,90\) ishonchlilik bilan noma'lum \(\theta\) matematik kutilma uchun ishonchlilik intervalin tuzing .
\\
\textbf{B2.} 
\(\left\lbrack \theta_{1},\theta_{2} \right\rbrack\) oraliqda tekis taqsimot parametrlari uchun momentlar usuli baholarini toping.
\\
\textbf{B3.} 
\(f(x) = \frac{2x}{\theta}e^{- \frac{x^{2}}{\theta}},\ \ x \geq 0\) model uchun \(\theta\) parametri haqiqatga maksimal o'xshashlik usuli bahosi topilsin.
\\
\textbf{C1.} 
Agar \(X^{(n)} = \left( X_{1},...,X_{n} \right)\) tanlanma zichlik funksiyasi \(f(x;\theta) = \left\{ \begin{matrix}
\alpha^{- 1}e^{- \ \frac{x - \theta}{\alpha}},\ \ x \geq \theta, \\
0,\ \ x < \theta
\end{matrix} \right.\ \)bo'lgan taqsimotdan olingan bo'lsa (\(\alpha -\) ma'lum), u holda noma'lum \(\theta\) parametr uchun \(X_{(1)}\) bahoning siljimaganligi va asosliligini tekshiring.
\\
\textbf{C2.} 
Agar \(X^{(n)} = \left( X_{1},...,X_{n} \right)\) tanlanma zichlik funksiyasi\(f(x,\theta) = \frac{2x}{\theta^{2}},x \in \lbrack 0,\theta\rbrack\)bo'lgan taqsimotdan olingan bo'lsa, u holda noma'lum \(\theta\) parametr uchun momentlar usuli bahosini toping.
\\
\textbf{C3.} 
Agar \(X^{(n)} = \left( X_{1},...,X_{n} \right)\) tanlanma zichlik funksiyasi\(f(x;\theta) = \frac{1}{2}e^{- |x - \theta|},\ \ x \in R\) bo'lgan Laplas taqsimotidan olingan bo'lsa, u holda noma'lum \(\theta \in R\) parametrning haqiqatga maksimal o'xshashlik bahosini toping.
\\

\end{tabular}
\vspace{1cm}


\begin{tabular}{m{17cm}}
\textbf{73-variant}
\newline

\textbf{T1.} Matematik statistikaning asosiy masalalari. (Statistik ma'lumotlar, guruhlash)
\\
\textbf{T2.} 
Chiziqli korrelyatsiya tenglamasi (ta'rifi, regressiya to'g'ri chiziqning tanlanma tenglamalari)
\\
\textbf{A1.} 
Hajmi \(n = 20\) ga teng bo'lgan tanlanma berilgan: 1,8; -1,9; 2,4; 1,8; 2,4; 1,8; 2,4; -0,6; -1,9; 1,8; -0,6; 2,4; -3,3; -1,9; 4,0; -3,3; -3,3; -1,9; -3,3; -1,9. Bu tanlanmaning statistik taqsimotin toping.
\\
\textbf{A2.} 
Hajmi \(n = 20\) ga teng bo'lgan tanlanma berilgan: 1,4; -1,9; 2,5; 1,4; 2,5; 1,4; 2,5; -0,4; -1,9; 1,4; -0,4; 2,5; -3,7; -1,9; 4,5; -3,7; -3,7; -1,9; -3,7; -1,9. Bu tanlanmaning empirik taqsimot funksiyasin toping.
\\
\textbf{A3.} 
Oliy matematika fanidan 10 ta talaba test topshiriqlarin topshirdi. Harbir talaba 10 balgacha to'plashi mumkin. Agar test topshiriqlari natijalari bo'yicha \{4, 3, 8, 4, 8, 3, 9, 4, 7, 10\} tanlanma olingan bo'lsa, ushbu tanlanmalarning tanlanma o'rta va tanlanma dispersiyalarin toping.
\\
\textbf{B1.} 
Agar normal taqsimlangan bosh to'plamdan olingan hajmi \(n = 36\)ga teng tanlanma bo'yicha \(\overline{x} = 20,2\) tanlanma o'rta va \({\overline{S}}^{2} = 0,64\) tuzatilgan tanlanma dispersiyalar topilgan bo'lsa, u holda \(\gamma = 0,90\) ishonchlilik bilan noma'lum \(\theta\) matematik kutilma uchun ishonchlilik intervalin tuzing.
\\
\textbf{B2.} 
Puasson taqsimoti noma'lum \(\theta > 0\) parametri momentlar usuli bahosini toping.
\\
\textbf{B3.} 
Agar \(X^{(n)} = \left( X_{1},...,X_{n} \right)\) tanlanma \(\theta\) parametrli ko'rsatkichli taqsimotdan olingan bo'lsa, u holda noma'lum \(\theta\) parametrning haqiqatga maksimal o'xshashlik usuli bahosini toping.
\\
\textbf{C1.} 
Agar \(X^{(n)} = \left( X_{1},...,X_{n} \right)\) tanlanma \(\lbrack 0,\theta\rbrack\) oraliqda tekis taqsimotdan olingan bo'lsa, u holda noma'lum \(\theta\) parametr uchun \((n + 1)X_{(1)})\) bahoning siljimaganligi va asosliligini tekshiring.
\\
\textbf{C2.} 
Agar \(X^{(n)} = \left( X_{1},...,X_{n} \right)\) tanlanma \(\left( \theta_{1},\theta_{2} \right)\) parametrli gamma taqsimotdan olingan bo'lsa, u holda noma'lum \(\left( \theta_{1},\theta_{2} \right)\) vektor parametr uchun momentlar usuli bahosini toping.
\\
\textbf{C3.} 
Agar \(X^{(n)} = \left( X_{1},...,X_{n} \right)\) tanlanma zichlik funksiyasi \(f(x;\theta) = \frac{\theta}{\sqrt{2\pi x^{3}}}e^{- \theta^{2}/2x},\ \ x \geq 0\) bo'lgan taqsimotdan olingan bo'lsa, u holda noma'lum \(\theta > 0\) parametrning haqiqatga maksimal o'xshashlik bahosini toping.
\\

\end{tabular}
\vspace{1cm}


\begin{tabular}{m{17cm}}
\textbf{74-variant}
\newline

\textbf{T1.} 
Tanlanma momentleri (\(k -\)tartibli boshlang'ich, boshlang'ich absolyut, markaziy va markaziy absolyut momentler).
\\
\textbf{T2.} 
Normal qonun dispersiyasi uchun ishonchlilik intervalin tuzish. (Ishonchlilik ehtimolligi, interval)
\\
\textbf{A1.} 
Hajmi \(n = 20\) ga teng bo'lgan tanlanma berilgan: 2,9; -3,2; 5,3; -4,3; 4,1; 5,3; -1,2; 2,9; -3,2; 4,1; -4,3; 5,3; -3,2; 2,9; -4,3; 4,1; -1,2; 5,3; 2,9; -3,2. Bu tanlanmaning statistik taqsimotin toping.
\\
\textbf{A2.} 
Hajmi \(n = 20\) ga teng bo'lgan tanlanma berilgan: 2,7; -5,6; 5,2; -8,1; 4,8; 5,2; -1,6; 2,7; -5,6; 4,8; -8,1; 5,2; -5,6; 2,7; -8,1; 4,8; -1,6; 5,2; 2,7; -5,6. Bu tanlanmaning empirik taqsimot funksiyasin toping.
\\
\textbf{A3.} 
Oliy matematika fanidan 10 ta talaba test topshiriqlarin topshirdi. Harbir talaba 10 balgacha to'plashi mumkin. Agar test topshiriqlari natijalari bo'yicha \{7, 9, 4, 9, 7, 5, 4, 7, 2, 6\} tanlanma olingan bo'lsa, ushbu tanlanmalarning tanlanma o'rta va tanlanma dispersiyalarin toping.
\\
\textbf{B1.} 
Agar normal taqsimlangan bosh to'plamdan olingan hajmi \(n = 11\) ga teng bo'lgan tanlanma bo'yicha \({\overline{S}}^{2} = 0,3\) tuzatilgan tanlanma dispersiya topilgan bo'lsa, u holda \(\gamma = 0,95\) ishonchlilik bilan noma'lum \(\theta_{2}^{2}\) dispersiya uchun ishonchlilik intervalin tuzing.
\\
\textbf{B2.} 
\(\lbrack 0,\theta\rbrack\) oraliqda tekis taqsimlangan \(\theta\) parametri uchun momentlar usuli bahosini toping.
\\
\textbf{B3.} 
Agar \(x_{1} = 1,1;\ \ x_{2} = 2,7;\ldots;x_{100} = 1,5\) tanlanma \(\theta\) parametrli ko'rsatkichli taqsimotdan olingan bo'lib, \(\sum_{k = 1}^{100}x_{k} = 200\) bo'lsa, u holda noma'lum \(\theta\) parametrning haqiqatga maksimal o'xshashlik bahosini toping.
\\
\textbf{C1.} 
Agar \(X^{(n)} = \left( X_{1},...,X_{n} \right)\) tanlanma \(\lbrack 0,\theta\rbrack\) oraliqda tekis taqsimotdan olingan bo'lsa, u holda noma'lum \(\theta\) parametr uchun \(\frac{n + 1}{n}X_{(n)}\) bahoning siljimaganligi va asosliligini tekshiring.
\\
\textbf{C2.} 
Agar \(X^{(n)} = \left( X_{1},...,X_{n} \right)\) tanlanma zichlik funksiyasi\(f(x,\theta) = \frac{2x}{\theta^{2}},x \in \lbrack 0,\theta\rbrack\)bo'lgan taqsimotdan olingan bo'lsa, u holda noma'lum \(\theta\) parametr uchun momentlar usuli bahosini toping.
\\
\textbf{C3.} 
Agar \(X^{(n)} = \left( X_{1},...,X_{n} \right)\) tanlanma \(\theta\) parametrli geometrik taqsimotdan olingan bo'lsa, u holda noma'lum \(\theta\) parametrning haqiqatga maksimal o'xshashlik usuli bahosini toping.
\\

\end{tabular}
\vspace{1cm}


\begin{tabular}{m{17cm}}
\textbf{75-variant}
\newline

\textbf{T1.} 
Empirik taqsimot funktsiyasi. (Tanlanma, eksperiment)
\\
\textbf{T2.} 
Ishonchlilik intervallarin tuzish. Aniq ishonchlilik intervallar.
\\
\textbf{A1.} 
Hajmi \(n = 20\) ga teng bo'lgan tanlanma berilgan: 14,7; 7,3; 16,6; 9,8; 11,2; 16,6; 6,7; 7,3; 11,2; 14,7; 6,7; 16,6; 7,3; 11,2; 14,7; 16,6; 6,7; 7,3; 11,2; 16,6. Bu tanlanmaning statistik taqsimotin toping.
\\
\textbf{A2.} 
Hajmi \(n = 20\) ga teng bo'lgan tanlanma berilgan: 14,4; 7,6; 16,7; 9,1; 11,8; 16,7; 6,4; 7,6; 11,8; 14,4; 6,4; 16,7; 7,6; 11,8; 14,4; 16,7; 6,4; 7,6; 11,8; 16,7. Bu tanlanmaning empirik taqsimot funksiyasin toping.
\\
\textbf{A3.} 
Oliy matematika fanidan 10 ta talaba test topshiriqlarin topshirdi. Harbir talaba 10 balgacha to'plashi mumkin. Agar test topshiriqlari natijalari bo'yicha \{10, 8, 4, 6, 2, 8, 5, 10, 2, 5\} tanlanma olingan bo'lsa, ushbu tanlanmalarning tanlanma o'rta va tanlanma dispersiyalarin toping.
\\
\textbf{B1.} 
Agar o'rta kvadratik chetlanish \(\sigma = 4\) bo'lgan normal taqsimot bosh to'plamdan olingan hajmi \(n = 49\)ga teng tanlanma bo'yicha \(\overline{x} = 9,4\) tanlanma o'rta qiymati topilgan bo'lsa, u holda \(\gamma = 0,9\) ishonchlilik bilan noma'lum \(\theta\) matematik kutilma uchun ishonchlilik intervalin tuzing .
\\
\textbf{B2.} 
Agar (3,0,-2,0,-2,3,-2,0,0,3,0,0,0,0,3,-2,0,0,-2,3,0) tanlanma quyida berilgan taqsimotdan olingan bo'lsa, u holda noma'lum \(\left( \theta_{1},\theta_{2} \right)\) vektor parametr uchun momentlar usuli bahosini toping.
$\begin{array}{|c|c|c|c|}
    \hline
    \xi & - 2 & 0 & 3 \\
    \hline
    P_{\theta} & 2\theta_{1} & 0,5 + \theta_{1} + \theta_{2} & \theta_{2} \\
    \hline
\end{array}$
\\
\textbf{B3.} 
Agar \(X^{(n)} = \left( X_{1},...,X_{n} \right)\) tanlanma zichlik funksiyasi \(f(x;\theta) = \frac{2x}{\theta}e^{- \frac{x^{2}}{\theta}},\ \ x \geq 0\) bo'lgan taqsimotdan olingan bo'lsa, u holda noma'lum \(\theta > 0\) parametrning haqiqatga maksimal usuli bahosini toping.
\\
\textbf{C1.} 
Agar \(X^{(n)} = \left( X_{1},...,X_{n} \right)\) tanlanma \(M\xi = a\) ma'lum va \(M\xi^{2}\) chekli bo'lgan taqsimotdan olingan bo'lsa, u holda noma'lum \(D\xi\) dispersiya uchun \({\overline{S}}^{2}\) bahoning siljimaganligi va asosliligini tekshiring.
\\
\textbf{C2.} 
Agar \(X^{(n)} = \left( X_{1},...,X_{n} \right)\) tanlanma zichlik funksiyasi\(f(x,\theta) = \left\{ \begin{matrix}
\theta_{1}^{- 1}e^{- \ \frac{x - \theta_{2}}{\theta_{1}}},\ \ x \geq \theta_{2}, \\
0,\ \ x < \theta_{2}
\end{matrix} \right.\ \)bo'lgan taqsimotdan olingan bo'lsa, u holda noma'lum \(\left( \theta_{1},\theta_{2} \right)\) \(\theta_{1} > 0,\) \(\theta_{2} \in R\) vektor parametr uchun momentlar usuli bahosini toping.
\\
\textbf{C3.} 
Agar \(X^{(n)} = \left( X_{1},...,X_{n} \right)\) tanlanma zichlik funksiyasi\(f(x;\theta) = \left\{ \begin{matrix}
e^{\theta - x},\ \ x \geq \theta, \\
\ \ 0,\ \ x < \theta
\end{matrix} \right.\ \) bo'lgan taqsimotdan olingan bo'lsa, u holda noma'lum \(\theta\) parametrning haqiqatga maksimal o'xshashlik bahosini toping.
\\

\end{tabular}
\vspace{1cm}


\begin{tabular}{m{17cm}}
\textbf{76-variant}
\newline

\textbf{T1.} 
Glivenko-Kantelli teoremasi. (empirik taqsimot funktsiyasi, ehtimollik bilan yaqinlashish).
\\
\textbf{T2.} 
Statistik baho xossalari. (Siljimagan, asosliy, effektiv)
\\
\textbf{A1.} 
Hajmi \(n = 20\) ga teng bo'lgan tanlanma berilgan: 4,3; 4,9; 13,4; 13,4; 6,5; 4,9; 4,9; 4,3; 5,1; 6,5; 6,5; 7,0; 4,3; 4,9; 6,5; 6,5; 5,1; 5,1; 4,9; 13,4. Bu tanlanmaning statistik taqsimotin toping.
\\
\textbf{A2.} 
Hajmi \(n = 20\) ga teng bo'lgan tanlanma berilgan: 4,2; 4,9; 13,8; 13,8; 6,6; 4,9; 4,9; 4,2; 5,3; 6,6; 6,6; 7,5; 4,2; 4,9; 6,6; 6,6; 5,3; 5,3; 4,9; 13,8. Bu tanlanmaning empirik taqsimot funksiyasin toping.
\\
\textbf{A3.} 
Oliy matematika fanidan 10 ta talaba test topshiriqlarin topshirdi. Harbir talaba 10 balgacha to'plashi mumkin. Agar test topshiriqlari natijalari bo'yicha \{9, 10, 6, 7, 4, 8, 10, 7, 9, 10\} tanlanma olingan bo'lsa, ushbu tanlanmalarning tanlanma o'rta va tanlanma dispersiyalarin toping.
\\
\textbf{B1.} 
Agar o'rta kvadratik chetlanish \(\sigma = 2\) bo'lgan normal taqsimot bosh to'plamdan olingan hajmi \(n = 10\)ga teng tanlanma bo'yicha \(\overline{x} = 5,4\) tanlanma o'rta qiymati topilgan bo'lsa, u holda \(\gamma = 0,95\) ishonchlilik bilan noma'lum \(\theta\) matematik kutilma uchun ishonchlilik intervalin tuzing .
\\
\textbf{B2.} 
Agar zichlik funksiyasi \(f(x) = \frac{2x}{\theta}e^{- \frac{x^{2}}{\theta}},\ \ x \geq 0\) ko'rinishga ega bo'lsa, u holda \(\theta\) parametr momentlar usuli bahosini toping.
\\
\textbf{B3.} 
Agar (0,1,2,0) tanlanma quyida berilgan taqsimotdan olingan bo'lsa, u holda noma'lum \(\theta\) parametrning haqiqatga maksimal o'xshashlik bahosini toping.
$\begin{array}{|c|c|c|c|}
    \hline
    \xi & 0 & 1 & 2 \\
    \hline
    P_{\theta} & \theta & 2\theta & 1 - 3\theta \\
    \hline
\end{array}$
\\
\textbf{C1.} 
Agar \(X^{(n)} = \left( X_{1},...,X_{n} \right)\) tanlanma \(M\xi = a\) ma'lum va \(M\xi^{2}\) chekli bo'lgan taqsimotdan olingan bo'lsa, u holda noma'lum \(D\xi\) dispersiya uchun \(\frac{1}{n}\sum_{i = 1}^{n}{X_{i}a}\) bahoning siljimaganligi va asosliligini tekshiring.
\\
\textbf{C2.} 
Agar \(X^{(n)} = \left( X_{1},...,X_{n} \right)\) tanlanma \((\theta,2\theta)\) parametrli normal taqsimotdan olingan bo'lsa, u holda noma'lum \(\theta > 0\) parametr uchun momentlar usuli bahosini toping.
\\
\textbf{C3.} 
Agar \(X^{(n)} = \left( X_{1},...,X_{n} \right)\) tanlanma zichlik funksiyasi\(f(x;\theta) = \frac{4x^{3}}{\sqrt{2\pi}\theta_{2}}\exp\left\{ - \frac{\left( x^{4} - \theta_{1} \right)^{2}}{2{\theta_{2}}^{2}} \right\},\ \ x \in R\) bo'lgan taqsimotdan olingan bo'lsa, u holda noma'lum \(\left( \theta_{1},\theta_{2}^{2} \right)\) vektor parametrning haqiqatga maksimal o'xshashlik usuli baholarini toping.
\\

\end{tabular}
\vspace{1cm}


\begin{tabular}{m{17cm}}
\textbf{77-variant}
\newline

\textbf{T1.} 
Neyman-Pirson teoremasi.
\\
\textbf{T2.} 
Pirsonning xi-kvadrat muvofiqlik kriteriysi (Pirson teoremasi).
\\
\textbf{A1.} 
Hajmi \(n = 20\) ga teng bo'lgan tanlanma berilgan: -2,1; 1,7; 3,3; 3,3; 11,7; 4,7; 1,7; 4,7; -2,1; 4,7; 4,7; 4,7; 8,0; -2,1; 1,7; 4,7; 8,0; 11,7; 1,7; 8,0. Bu tanlanmaning statistik taqsimotin toping.
\\
\textbf{A2.} 
Hajmi \(n = 20\) ga teng bo'lgan tanlanma berilgan: -2,2; 1,3; 3,8; 3,8; 11,5; 4,1; 1,3; 4,1; -2,2; 4,1; 4,1; 4,1; 8,4; -2,2; 1,3; 4,1; 8,4; 11,5; 1,3; 8,4. Bu tanlanmaning empirik taqsimot funksiyasin toping.
\\
\textbf{A3.} 
Oliy matematika fanidan 10 ta talaba test topshiriqlarin topshirdi. Harbir talaba 10 balgacha to'plashi mumkin. Agar test topshiriqlari natijalari bo'yicha \{4, 1, 2, 4, 6, 4, 5, 3, 6, 5\} tanlanma olingan bo'lsa, ushbu tanlanmalarning tanlanma o'rta va tanlanma dispersiyalarin toping.
\\
\textbf{B1.} 
Agar normal taqsimlangan bosh to'plamdan olingan hajmi \(n = 16\)ga teng tanlanma bo'yicha \(\overline{x} = 20,2\) tanlanma o'rta va \({\overline{S}}^{2} = 0,64\) tuzatilgan tanlanma dispersiyalar topilgan bo'lsa, u holda \(\gamma = 0,95\) ishonchlilik bilan noma'lum \(\theta\) matematik kutilma uchun ishonchlilik intervalin tuzing.
\\
\textbf{B2.} 
Agar (3,-2,-2,0,-2,2,-2,0,-2,3,-2,0,3,0,3,-2,0,-2,3,-2,2,-2,-2,3,3,2,-2,2,3,3) tanlanma quyida berilgan taqsimotdan olingan bo'lsa, u holda noma'lum \(\theta\) parametr uchun momentlar usuli bahosini \(g(x) = |x|\) funksiya yordamida toping.
$\begin{array}{|c|c|c|c|}
    \hline
    \xi & -2 & 0 & 3 \\
    \hline
    P_{\theta} & 3\theta & 1 - 5\theta & 2\theta \\
    \hline
\end{array}$
\\
\textbf{B3.} 
\(f(x) = \frac{2x}{\theta}e^{- \frac{x^{2}}{\theta}},\ \ x \geq 0\) model uchun \(\theta\) parametri haqiqatga maksimal o'xshashlik usuli bahosi topilsin.
\\
\textbf{C1.} 
Agar \(X^{(n)} = \left( X_{1},...,X_{n} \right)\) tanlanma \(M\xi = a\) ma'lum va \(M\xi^{2}\) chekli bo'lgan taqsimotdan olingan bo'lsa, u holda noma'lum \(D\xi\) dispersiya uchun \(\overline{x^{2}} - a^{2}\) bahoning siljimaganligi va asosliligini tekshiring.
\\
\textbf{C2.} 
Agar \(X^{(n)} = \left( X_{1},...,X_{n} \right)\) tanlanma \((\theta,\theta^{2})\ \ \) parametrli normal taqsimotdan olingan bo'lsa, u holda noma'lum \(\theta > 0\) parametr uchun momentlar usuli bahosini toping.
\\
\textbf{C3.} 
Agar \(X^{(n)} = \left( X_{1},...,X_{n} \right)\) tanlanma zichlik funksiyasi \(f(x;\theta) = \frac{3x^{2}}{\sqrt{2\pi}}\exp\left\{ - \frac{\left( x^{3} - \theta \right)^{2}}{2} \right\},\ \ x \in R\) bo'lgan taqsimotdan olingan bo'lsa, u holda noma'lum \(\theta\) parametrning haqiqatga maksimal o'xshashlik bahosini toping.
\\

\end{tabular}
\vspace{1cm}


\begin{tabular}{m{17cm}}
\textbf{78-variant}
\newline

\textbf{T1.} 
Poligon va gistogramma(nisbiy chastota, interval qator, grafik).
\\
\textbf{T2.} 
Statistik gipotezalarni tekshirish (kritik to'plam, 1 va 2-tur xatolik).
\\
\textbf{A1.} 
Hajmi \(n = 20\) ga teng bo'lgan tanlanma berilgan: -11,0; -4,1; 0; 2,3; 1,2; 0; 1,2; 2,3; 2,3; 1,2; 2,3; -11,0; 3,4; 1,2; 3,4; 3,4; 0; 3,4; 2,3; 0. Bu tanlanmaning statistik taqsimotin toping.
\\
\textbf{A2.} 
Hajmi \(n = 20\) ga teng bo'lgan tanlanma berilgan: -11,2; -4,5; 0; 2,9; 1,7; 0; 1,7; 2,9; 2,9; 1,7; 2,9; -11,2; 3,1; 1,7; 3,1; 3,1; 0; 3,1; 2,9; 0. Bu tanlanmaning empirik taqsimot funksiyasin toping.
\\
\textbf{A3.} 
Oliy matematika fanidan 10 ta talaba test topshiriqlarin topshirdi. Harbir talaba 10 balgacha to'plashi mumkin. Agar test topshiriqlari natijalari bo'yicha \{8, 9, 10, 4, 9, 7, 6, 7, 6, 4\} tanlanma olingan bo'lsa, ushbu tanlanmalarning tanlanma o'rta va tanlanma dispersiyalarin toping.
\\
\textbf{B1.} 
Agar normal taqsimlangan bosh to'plamdan olingan hajmi \(n = 11\) ga teng bo'lgan tanlanma bo'yicha \({\overline{S}}^{2} = 0,5\) tuzatilgan tanlanma dispersiya topilgan bo'lsa, u holda \(\gamma = 0,90\) ishonchlilik bilan noma'lum \(\theta_{2}^{2}\) dispersiya uchun ishonchlilik intervalin tuzing.
\\
\textbf{B2.} 
Agar zichlik funksiyasi \(f(x) = \frac{2x}{\theta}e^{- \frac{x^{2}}{\theta}},\ \ x \geq 0\) ko'rinishga ega bo'lsa, u holda \(\theta\) parametr momentlar usuli bahosini toping.
\\
\textbf{B3.} 
Agar (-1,-1,0,-1,0,-1,-1,5,-1,0,-1,0,5,-1,-1,-1,5,-1,-1,-1,1,-1,5,0,-1,-1,5) tanlanma quyida berilgan taqsimotdan olingan bo'lsa, u holda noma'lum \(\theta\) parametrning haqiqatga maksimal o'xshashlik usuli bahosini toping.
$\begin{array}{|c|c|c|c|}
    \hline
    \xi & - 1 & 0 & 5\\
    \hline
    P_{\theta} & 1 - \theta & \theta/2 & \theta/2 \\
    \hline
\end{array}$
\\
\textbf{C1.} 
Agar \(X^{(n)} = \left( X_{1},...,X_{n} \right)\) tanlanma zichlik funksiyasi bo'lsa: \(f(x,\theta) = \left\{ \begin{matrix}
e^{\theta - x},\ \ x \geq \theta, \\
\ \ 0,\ \ x < \theta
\end{matrix} \right.\ \) bo'lgan taqsimotdan olingan bo'lsa, u holda noma'lum \(\theta\) parametr uchun \(X_{(1)}\) bahoning siljimaganligi va asosliligini tekshiring.
\\
\textbf{C2.} 
Agar \(X^{(n)} = \left( X_{1},...,X_{n} \right)\) tanlanma \((\theta,\theta^{2})\) parametrli normal taqsimotdan \(\ \ g(x) = (x)^{2}\ \ \) olingan bo'lsa, u holda noma'lum \(\theta > 0\) parametr uchun momentlar usuli bahosini funksiya yordamida toping.
\\
\textbf{C3.} 
Agar \(X^{(n)} = \left( X_{1},...,X_{n} \right)\) tanlanma zichlik funksiyasi\(f(x;\theta) = \frac{1}{2}e^{- |x - \theta|},\ \ x \in R\) bo'lgan Laplas taqsimotidan olingan bo'lsa, u holda noma'lum \(\theta \in R\) parametrning haqiqatga maksimal o'xshashlik bahosini toping.
\\

\end{tabular}
\vspace{1cm}


\begin{tabular}{m{17cm}}
\textbf{79-variant}
\newline

\textbf{T1.} 
Guruhlangan va interval variatsion qatorlar.
\\
\textbf{T2.} 
Statistik gipotezalarni tekshirish (kritik to'plam, 1 va 2-tur xatolik)
\\
\textbf{A1.} 
Hajmi \(n = 20\) ga teng bo'lgan tanlanma berilgan: 2,5; 3,8; 4,3; 2,5; 3,8; 2,5; 3,1; 4,3; 4,3; 5,5; 6,2; 2,5; 3,1; 6,2; 5,5; 6,2; 3,1; 3,1; 6,2; 3,1. Bu tanlanmaning statistik taqsimotin toping.
\\
\textbf{A2.} 
Hajmi \(n = 20\) ga teng bo'lgan tanlanma berilgan: 2,7; 4,2; 4,8; 2,7; 4,2; 2,7; 3,9; 4,8; 4,8; 5,9; 6,5; 2,7; 3,9; 6,5; 5,9; 6,5; 3,9; 3,9; 6,5; 3,9. Bu tanlanmaning empirik taqsimot funksiyasin toping.
\\
\textbf{A3.} 
Oliy matematika fanidan 10 ta talaba test topshiriqlarin topshirdi. Harbir talaba 10 balgacha to'plashi mumkin. Agar test topshiriqlari natijalari bo'yicha \{7, 8, 7, 6, 4, 8, 4, 7, 9, 10\} tanlanma olingan bo'lsa, ushbu tanlanmalarning tanlanma o'rta va tanlanma dispersiyalarin toping.
\\
\textbf{B1.} 
Agar o'rta kvadratik chetlanish \(\sigma = 3\) bo'lgan normal taqsimot bosh to'plamdan olingan hajmi \(n = 9\)ga teng tanlanma bo'yicha \(\overline{x} = 4,5\) tanlanma o'rta qiymati topilgan bo'lsa, u holda \(\gamma = 0,95\) ishonchlilik bilan noma'lum \(\theta\) matematik kutilma uchun ishonchlilik intervalin tuzing .
\\
\textbf{B2.} 
Agar (3,-2,-2,0,-2,2,-2,0,-2,3,-2,0,3,0,3,-2,0,-2,3,-2,2,-2,-2,3,3,2,-2,2,3,3) tanlanma quyida berilgan taqsimotdan olingan bo'lsa, u holda noma'lum \(\theta\) parametr uchun momentlar usuli bahosini \(g(x) = |x|\) funksiya yordamida toping.
$\begin{array}{|c|c|c|c|}
    \hline
    \xi & -2 & 0 & 3 \\
    \hline
    P_{\theta} & 3\theta & 1 - 5\theta & 2\theta \\
    \hline
\end{array}$
\\
\textbf{B3.} 
\(f(x) = \frac{\theta}{2}e^{- \theta|x|}\) model uchun \(\theta\) parametri haqiqatga maksimal o'xshashlik usuli bahosi topilsin.
\\
\textbf{C1.} 
Agar \(X^{(n)} = \left( X_{1},...,X_{n} \right)\) tanlanma zichlik funksiyasi bo'lsa: \(f(x,\theta) = \left\{ \begin{matrix}
e^{\theta - x},\ \ x \geq \theta, \\
\ \ 0,\ \ x < \theta
\end{matrix} \right.\ \) bo'lgan taqsimotdan olingan bo'lsa, u holda noma'lum \(\theta\) parametr uchun \(\overline{x} - 1\) bahoning siljimaganligi va asosliligini tekshiring.
\\
\textbf{C2.} 
Agar \(X^{(n)} = \left( X_{1},...,X_{n} \right)\) tanlanma {[}\(0,2\theta\rbrack\) oraliqda tekis taqsimotdan olingan bo'lsa, u holda noma'lum \(\theta > 0\) parametr uchun momentlar usuli bahosini toping.
\\
\textbf{C3.} 
Agar \(X^{(n)} = \left( X_{1},...,X_{n} \right)\) tanlanma \((\theta,2\theta)\) parametrli normal taqsimotdan olingan bo'lsa, u holda noma'lum \(\theta > 0\) parametrning haqiqatga maksimal o'xshashlik bahosini toping.
\\

\end{tabular}
\vspace{1cm}


\begin{tabular}{m{17cm}}
\textbf{80-variant}
\newline

\textbf{T1.} 
Momentler usuli. (tanlanma momentleri, noma'lum parametrlarni baholash).
\\
\textbf{T2.} 
Momentler usuli. (tanlanma momentleri, noma'lum parametrlarni baholash).
\\
\textbf{A1.} 
Hajmi \(n = 20\) ga teng bo'lgan tanlanma berilgan: -4,3; 2,6; 0; -2,5; 2,6; 1,9; 2,2; 0; -4,3; -2,5; 1,9; -2,5; 1,9; 2,2; 2,6; 1,9; 2,6; 2,2; 2,2; 1,9. Bu tanlanmaning statistik taqsimotin toping.
\\
\textbf{A2.} 
Hajmi \(n = 20\) ga teng bo'lgan tanlanma berilgan: -4,9; 2,6; 0,5; -2,6; 2,6; 1,7; 2,3; 0,5; -4,9; -2,6; 1,7; -2,6; 1,7; 2,3; 2,6; 1,7; 2,6; 2,3; 2,3; 1,7. Bu tanlanmaning empirik taqsimot funksiyasin toping.
\\
\textbf{A3.} 
Oliy matematika fanidan 10 ta talaba test topshiriqlarin topshirdi. Harbir talaba 10 balgacha to'plashi mumkin. Agar test topshiriqlari natijalari bo'yicha \{9, 5, 6, 8, 4, 7, 4, 6, 9, 7\} tanlanma olingan bo'lsa, ushbu tanlanmalarning tanlanma o'rta va tanlanma dispersiyalarin toping.
\\
\textbf{B1.} 
Agar normal taqsimlangan bosh to'plamdan olingan hajmi \(n = 25\)ga teng tanlanma bo'yicha \(\overline{x} = 18,6\) tanlanma o'rta va \({\overline{S}}^{2} = 0,49\) tuzatilgan tanlanma dispersiyalar topilgan bo'lsa, u holda \(\gamma = 0,95\) ishonchlilik bilan noma'lum \(\theta\) matematik kutilma uchun ishonchlilik intervalin tuzing.
\\
\textbf{B2.} 
\(\left\lbrack \theta_{1},\theta_{2} \right\rbrack\) oraliqda tekis taqsimot parametrlari uchun momentlar usuli baholarini toping.
\\
\textbf{B3.} 
Agar \(X^{(n)} = \left( X_{1},...,X_{n} \right)\) tanlanma \(\left( a,\theta^{2} \right)\) parametrli normal taqsimotdan olingan bo'lsa (\(\alpha -\) ma'lum), u holda noma'lum \(\theta^{2}\) parametrning haqiqatga maksimal o'xshashlik bahosini toping.
\\
\textbf{C1.} 
Agar \(X^{(n)} = \left( X_{1},...,X_{n} \right)\) tanlanma \(\lbrack - 3\theta,\theta\rbrack\) oraliqda tekis taqsimotdan olingan bo'lsa, u holda noma'lum \(\theta\) parametr uchun \(4X_{(n)} + X_{(1)}\) bahoni siljimaganligi va asosliligini tekshiring.
\\
\textbf{C2.} 
Agar \(X^{(n)} = \left( X_{1},...,X_{n} \right)\) tanlanma zichlik funksiyasi\(f(x,\theta) = \theta x^{\theta - 1},x \in \lbrack 0,1\rbrack\)bo'lgan taqsimotdan olingan bo'lsa, u holda noma'lum \(\theta\) parametr uchun momentlar usuli bahosini toping.
\\
\textbf{C3.} 
Agar \(X^{(n)} = \left( X_{1},...,X_{n} \right)\) tanlanma \(\left( \theta,\theta^{2} \right)\) parametrli normal taqsimotdan olingan bo'lsa, u holda noma'lum \(\theta > 0\) parametrning haqiqatga maksimal o'xshashlik bahosini toping.
\\

\end{tabular}
\vspace{1cm}


\begin{tabular}{m{17cm}}
\textbf{81-variant}
\newline

\textbf{T1.} 
Empirik taqsimot funktsiyasi. (Tanlanma, eksperiment)
\\
\textbf{T2.} 
Statistik gipotezalarni tekshirish (kritik to'plam, 1 va 2-tur xatolik).
\\
\textbf{A1.} 
Hajmi \(n = 20\) ga teng bo'lgan tanlanma berilgan: -2,9; -3,8; 2,3; 1,8; 1,8; 0,7; -3,8; -1,5; 2,3; 0,7; -2,9; -1,5; 1,8; -2,9; -1,5; -3,8; 1,8; 1,8; -3,8; 1,8. Bu tanlanmaning statistik taqsimotin toping.
\\
\textbf{A2.} 
Hajmi \(n = 20\) ga teng bo'lgan tanlanma berilgan: -2,4; -3,5; 2,8; 1,4; 1,4; 0,1; -3,5; -1,9; 2,8; 0,1; -2,4; -1,9; 1,4; -2,4; -1,9; -3,5; 1,4; 1,4; -3,5; 1,4. Bu tanlanmaning empirik taqsimot funksiyasin toping.
\\
\textbf{A3.} 
Oliy matematika fanidan 10 ta talaba test topshiriqlarin topshirdi. Harbir talaba 10 balgacha to'plashi mumkin. Agar test topshiriqlari natijalari bo'yicha \{8, 9, 7, 10, 6, 8, 10, 3, 10, 9\} tanlanma olingan bo'lsa, ushbu tanlanmalarning tanlanma o'rta va tanlanma dispersiyalarin toping.
\\
\textbf{B1.} 
Agar normal taqsimlangan bosh to'plamdan olingan hajmi \(n = 12\) ga teng bo'lgan tanlanma bo'yicha \({\overline{S}}^{2} = 0,4\) tuzatilgan tanlanma dispersiya topilgan bo'lsa, u holda \(\gamma = 0,90\) ishonchlilik bilan noma'lum \(\theta_{2}^{2}\) dispersiya uchun ishonchlilik intervalin tuzing.
\\
\textbf{B2.} 
\(\lbrack 0,\theta\rbrack\) oraliqda tekis taqsimlangan \(\theta\) parametri uchun momentlar usuli bahosini toping.
\\
\textbf{B3.} 
Agar \(X^{(n)} = \left( X_{1},...,X_{n} \right)\) tanlanma \(\left\lbrack - \theta,\theta^{2} \right\rbrack\) oraliqda tekis taqsimotdan olingan bo'lsa, u holda noma'lum \(\theta > 0\) parametrning haqiqatga maksimal o'xshashlik usuli bahosini toping.
\\
\textbf{C1.} 
Agar \(X^{(n)} = \left( X_{1},...,X_{n} \right)\) tanlanma taqsimot funksiyasi \(F(x)\) bo'lgan taqsimotdan olingan bo'lsa, u holda noma'lum \(F(x)\) uchun \(F_{n}(x)\) empirik taqsimot funksiyasining siljimaganligi va asosliligini tekshiring.
\\
\textbf{C2.} 
Agar \(X^{(n)} = \left( X_{1},...,X_{n} \right)\) tanlanma \((\theta,2\theta)\) parametrli normal taqsimotdan olingan bo'lsa, u holda noma'lum \(\theta > 0\) parametr uchun momentlar usuli bahosini \(\ \ g(x) = (x)^{2}\) funksiya yordamida toping.
\\
\textbf{C3.} 
Agar \(X^{(n)} = \left( X_{1},...,X_{n} \right)\) tanlanma zichlik funksiyasi \(f(x;\theta) = \frac{\theta ln^{\theta - 1}x}{x},\ \ x \in \lbrack 1,e\rbrack\) bo'lgan taqsimotdan olingan bo'lsa, u holda noma'lum \(\theta > 0\) parametr uchun haqiqatga maksimal o'xshashlik bahosini toping.
\\

\end{tabular}
\vspace{1cm}


\begin{tabular}{m{17cm}}
\textbf{82-variant}
\newline

\textbf{T1.} 
Guruhlangan va interval variatsion qatorlar.
\\
\textbf{T2.} 
Normal qonun dispersiyasi uchun ishonchlilik intervalin tuzish. (Ishonchlilik ehtimolligi, interval)
\\
\textbf{A1.} 
Hajmi \(n = 20\) ga teng bo'lgan tanlanma berilgan: 3,6; 2,9; 3,6; 3,2; 1,1; 0,3; 1,1; 3,6; 1,7; 1,1; 0,3; 1,7; 1,1; 0,3; 2,9; 2,9; 2,9; 1,1; 2,9; 1,7. Bu tanlanmaning statistik taqsimotin toping.
\\
\textbf{A2.} 
Hajmi \(n = 20\) ga teng bo'lgan tanlanma berilgan: 4,6; 2,5; 4,6; 3,3; 1,8; 0,3; 1,8; 4,6; 2,1; 1,8; 0,3; 2,1; 1,8; 0,3; 2,5; 2,5; 2,5; 1,8; 2,5; 2,1. Bu tanlanmaning empirik taqsimot funksiyasin toping.
\\
\textbf{A3.} 
Oliy matematika fanidan 10 ta talaba test topshiriqlarin topshirdi. Harbir talaba 10 balgacha to'plashi mumkin. Agar test topshiriqlari natijalari bo'yicha \{5, 7, 5, 9, 5, 8, 10, 6, 7, 8\} tanlanma olingan bo'lsa, ushbu tanlanmalarning tanlanma o'rta va tanlanma dispersiyalarin toping.
\\
\textbf{B1.} 
Agar o'rta kvadratik chetlanish \(\sigma = 1\) bo'lgan normal taqsimot bosh to'plamdan olingan hajmi \(n = 15\)ga teng tanlanma bo'yicha \(\overline{x} = 5,8\) tanlanma o'rta qiymati topilgan bo'lsa, u holda \(\gamma = 0,90\) ishonchlilik bilan noma'lum \(\theta\) matematik kutilma uchun ishonchlilik intervalin tuzing .
\\
\textbf{B2.} 
Agar \(X^{(n)} = \left( X_{1},...,X_{n} \right)\) tanlanma \(\theta\) parametrli Bernulli taqsimotidan olingan bo'lsa, u holda noma'lum \(\theta\) parametr uchun momentlar usuli bahosini toping.
\\
\textbf{B3.} 
Agar \(X^{(n)} = \left( X_{1},...,X_{n} \right)\) tanlanma \(\lbrack - \theta,\theta\rbrack\) oraliqda tekis taqsimotdan olingan bo'lsa, u holda noma'lum \(\theta > 0\) parametrning haqiqatga maksimal o'xshashlik usuli bahosini toping.
\\
\textbf{C1.} 
Agar \(X^{(n)} = \left( X_{1},...,X_{n} \right)\) tanlanma \(\left( a,\theta^{2} \right)\) parametrli normal taqsimotdan olingan bo'lsa (\(a -\) ma'lum), u holda noma'lum \(\theta\) parametr uchun \(\sqrt{\frac{\pi}{2}}\left| \overline{x - a} \right|\) bahoning siljimaganligi va asosliligini tekshiring.
\\
\textbf{C2.} 
Agar \(X^{(n)} = \left( X_{1},...,X_{n} \right)\) tanlanma \(\ \ (a,\theta^{2})\ \ \) parametrli normal taqsimotdan olingan bo'lsa (\(\alpha -\) ma'lum), u holda noma'lum \(\ \ \theta^{2}\) parametr uchun momentlar usuli bahosini \(\ \ g(x) = (x - a)^{2}\) funksiyasi yordamida toping.
\\
\textbf{C3.} 
Agar \(X^{(n)} = \left( X_{1},...,X_{n} \right)\) tanlanma zichlik funksiyasi \(f(x;\theta) = \left\{ \begin{array}{r}
3x^{2}\theta^{- 3}{e^{- \left( \frac{x}{\theta} \right)}}^{3},\ \ \ \ x \geq 0 \\
0,\ \ \ \ \ \ \ \ \ \ \ x < 0
\end{array} \right.\ \) bo'lgan taqsimotdan olingan bo'lsa, u holda noma'lum \(\theta > 0\) parametrning haqiqatga maksimal o'xshashlik bahosini toping.
\\

\end{tabular}
\vspace{1cm}


\begin{tabular}{m{17cm}}
\textbf{83-variant}
\newline

\textbf{T1.} 
Momentler usuli. (tanlanma momentleri, noma'lum parametrlarni baholash).
\\
\textbf{T2.} 
Ishonchlilik intervallarin tuzish. Aniq ishonchlilik intervallar.
\\
\textbf{A1.} 
Hajmi \(n = 20\) ga teng bo'lgan tanlanma berilgan: -1,3; 0; 0,8; 2,3; 1,1; 0,8; 0,8; 2,3; 1,1; 0,8; -1,3; 1,8; 1,1; -1,3; 1,1; 1,8; 1,8; 1,1; 1,8; 1,8. Bu tanlanmaning statistik taqsimotin toping.
\\
\textbf{A2.} 
Hajmi \(n = 20\) ga teng bo'lgan tanlanma berilgan: -1,9; 0,7; 0,9; 2,8; 1,3; 0,9; 0,9; 2,8; 1,3; 0,9; -1,9; 1,6; 1,3; -1,9; 1,3; 1,6; 1,6; 1,3; 1,6; 1,6. Bu tanlanmaning empirik taqsimot funksiyasin toping.
\\
\textbf{A3.} 
Oliy matematika fanidan 10 ta talaba test topshiriqlarin topshirdi. Harbir talaba 10 balgacha to'plashi mumkin. Agar test topshiriqlari natijalari bo'yicha \{8, 4, 3, 7, 3, 6, 5, 3, 5, 6\} tanlanma olingan bo'lsa, ushbu tanlanmalarning tanlanma o'rta va tanlanma dispersiyalarin toping.
\\
\textbf{B1.} 
Agar normal taqsimlangan bosh to'plamdan olingan hajmi \(n = 20\)ga teng tanlanma bo'yicha \(\overline{x} = 16,6\) tanlanma o'rta va \({\overline{S}}^{2} = 0,64\) tuzatilgan tanlanma dispersiyalar topilgan bo'lsa, u holda \(\gamma = 0,95\) ishonchlilik bilan noma'lum \(\theta\) matematik kutilma uchun ishonchlilik intervalin tuzing.
\\
\textbf{B2.} 
Puasson taqsimoti noma'lum \(\theta > 0\) parametri momentlar usuli bahosini toping.
\\
\textbf{B3.} 
Agar \(X^{(n)} = \left( X_{1},...,X_{n} \right)\) tanlanma \(\theta\) parametrli Bernulli taqsimotidan olingan bo'lsa, u holda noma'lum \(\theta\) parametrning haqiqatga maksimal o'xshashlik usuli bahosini toping.
\\
\textbf{C1.} 
Agar \(X^{(n)} = \left( X_{1},...,X_{n} \right)\) tanlanma \(\theta\) parametrli ko'rsatkichli taqsimotdan olingan bo'lsa, u holda noma'lum \(\theta\) parametr uchun \(1/\overline{x}\) bahoning siljimaganligi va asosliligini tekshiring.
\\
\textbf{C2.} 
Agar \(X^{(n)} = \left( X_{1},...,X_{n} \right)\) tanlanma \({\lbrack\theta}_{1},\theta_{1} + \theta_{2}\rbrack\) oraliqda tekis taqsimotdan olingan bo'lsa, u holda noma'lum \(\left( \theta_{1},\theta_{2} \right)\) vektor parametr uchun momentlar usuli bahosini toping.
\\
\textbf{C3.} 
Agar \(X^{(n)} = \left( X_{1},...,X_{n} \right)\) tanlanma zichlik funksiyasi \(f(x;\theta) = \frac{e^{x}}{\sqrt{2\pi}}\exp\left\{ - \frac{\left( e^{x} - \theta \right)^{2}}{2} \right\},\ \ x \in R\) bo'lgan taqsimotdan olingan bo'lsa, u holda noma'lum \(\theta\) parametrning haqiqatga maksimal o'xshashlik bahosini toping.
\\

\end{tabular}
\vspace{1cm}


\begin{tabular}{m{17cm}}
\textbf{84-variant}
\newline

\textbf{T1.} Matematik statistikaning asosiy masalalari. (Statistik ma'lumotlar, guruhlash)
\\
\textbf{T2.} 
Pirsonning xi-kvadrat muvofiqlik kriteriysi (Pirson teoremasi).
\\
\textbf{A1.} 
Hajmi \(n = 20\) ga teng bo'lgan tanlanma berilgan: -2,4; 5,6; 5,6; -5,2; -6,7; 5,1; -5,2; -2,4; 4,3; 5,1; -6,7; 4,3; -2,4; -6,7; 4,3; 5,1; 4,3; 5,6; -6,7; 5,6. Bu tanlanmaning statistik taqsimotin toping.
\\
\textbf{A2.} 
Hajmi \(n = 20\) ga teng bo'lgan tanlanma berilgan: -2,9; 7,6; 7,6; -5,7; -6,1; 5,5; -5,7; -2,9; 4,2; 5,5; -6,1; 4,2; -2,9; -6,1; 4,2; 5,5; 4,2; 7,6; -6,1; 7,6. Bu tanlanmaning empirik taqsimot funksiyasin toping.
\\
\textbf{A3.} 
Oliy matematika fanidan 10 ta talaba test topshiriqlarin topshirdi. Harbir talaba 10 balgacha to'plashi mumkin. Agar test topshiriqlari natijalari bo'yicha \{9, 8, 6, 7, 5, 8, 5, 7, 4, 6\} tanlanma olingan bo'lsa, ushbu tanlanmalarning tanlanma o'rta va tanlanma dispersiyalarin toping.
\\
\textbf{B1.} 
Agar normal taqsimlangan bosh to'plamdan olingan hajmi \(n = 13\) ga teng bo'lgan tanlanma bo'yicha \({\overline{S}}^{2} = 1,2\) tuzatilgan tanlanma dispersiya topilgan bo'lsa, u holda \(\gamma = 0,90\) ishonchlilik bilan noma'lum \(\theta_{2}^{2}\) dispersiya uchun ishonchlilik intervalin tuzing.
\\
\textbf{B2.} 
Agar (0,-2,0,-2,3,-2,0,0,3,0,0,0,0,3,-2,0,0,-2,3,0,3) tanlanma quyida berilgan taqsimotdan olingan bo'lsa, u holda noma'lum \(\theta\) parametr uchun momentlar usuli bahosini toping.
$\begin{array}{|c|c|c|c|}
    \hline
    \xi & - 2 & 0 & 3 \\
    \hline
    P_{\theta} & \theta & 1 - 2\theta & \theta \\
    \hline
\end{array}$
\\
\textbf{B3.} 
Agar (4,8,5,3) tanlanma \((a,\theta^{2}\) parametrli normal taqsimotdan olingan bo'lsa, u holda noma'lum \(\theta^{2}\) parametrning haqiqatga maksimal o'xshashlik bahosini toping.
\\
\textbf{C1.} 
Agar \(X^{(n)} = \left( X_{1},...,X_{n} \right)\) tanlanma \(1\sqrt{\theta}\) parametrli ko'rsatkichli taqsimotdan olingan bo'lsa, u holda noma'lum \(\theta\) parametr uchun \((\overline{x})^{2}\) bahoning siljimaganligi va asosliligini tekshiring.
\\
\textbf{C2.} 
Agar \(X^{(n)} = \left( X_{1},...,X_{n} \right)\) tanlanma zichlik funksiyasi\(f(x,\theta) = \left\{ \begin{matrix}
e^{\theta - x},\ \ x \geq \theta, \\
0,\ \ x < \theta
\end{matrix} \right.\ \)bo'lgan taqsimotdan olingan bo'lsa, u holda noma'lum \(\theta\) parametr uchun momentlar usuli bahosini toping.
\\
\textbf{C3.} 
\(f(x,\theta) = \frac{e^{x}}{\sqrt{2\pi}}\exp\left\{ - \frac{\left( e^{x} - \theta \right)^{2}}{2} \right\}\) model uchun \(\theta\) parametri haqiqatga maksimal o'xshashlik usuli bahosi topilsin.
\\

\end{tabular}
\vspace{1cm}


\begin{tabular}{m{17cm}}
\textbf{85-variant}
\newline

\textbf{T1.} 
Glivenko-Kantelli teoremasi. (empirik taqsimot funktsiyasi, ehtimollik bilan yaqinlashish).
\\
\textbf{T2.} 
Statistik baho xossalari. (Siljimagan, asosliy, effektiv)
\\
\textbf{A1.} 
Hajmi \(n = 20\) ga teng bo'lgan tanlanma berilgan:-3,3; 0; 4,4; 2,2; -2,7; 4,4; 2,2; 4,4;-3,3; 2,2; -2,7; 2,2; -3,3; -2,7; 2,2; 3,4; 4,4; 0; -3,3; 0. Bu tanlanmaning statistik taqsimotin toping.
\\
\textbf{A2.} 
Hajmi \(n = 20\) ga teng bo'lgan tanlanma berilgan:-3,3; 0; 4,9; 2,8; -2,6; 4,9; 2,8; 4,9;-3,3; 2,8; -2,6; 2,8; -3,3; -2,6; 2,8; 3,1; 4,9; 0; -3,3; 0. Bu tanlanmaning empirik taqsimot funksiyasin toping.
\\
\textbf{A3.} 
Oliy matematika fanidan 10 ta talaba test topshiriqlarin topshirdi. Harbir talaba 10 balgacha to'plashi mumkin. Agar test topshiriqlari natijalari bo'yicha \{4, 7, 6, 9, 3, 8, 3, 7, 4, 9\} tanlanma olingan bo'lsa, ushbu tanlanmalarning tanlanma o'rta va tanlanma dispersiyalarin toping.
\\
\textbf{B1.} 
Agar o'rta kvadratik chetlanish \(\sigma = 4\) bo'lgan normal taqsimot bosh to'plamdan olingan hajmi \(n = 12\)ga teng tanlanma bo'yicha \(\overline{x} = 3\) tanlanma o'rta qiymati topilgan bo'lsa, u holda \(\gamma = 0,95\) ishonchlilik bilan noma'lum \(\theta\) matematik kutilma uchun ishonchlilik intervalin tuzing .
\\
\textbf{B2.} 
Agar \(X^{(n)} = \left( X_{1},...,X_{n} \right)\) tanlanma \(\theta\) parametrli ko'rsatkichli taqsimotdan olingan bo'lsa, u holda noma'lum \(\theta\) parametr uchun momentlar usuli bahosini toping.
\\
\textbf{B3.} 
Agar (0,1,2,0) tanlanma quyida berilgan taqsimotdan olingan bo'lsa, u holda noma'lum \(\theta\) parametrning haqiqatga maksimal o'xshashlik bahosini toping.
$\begin{array}{|c|c|c|c|}
    \hline
    \xi & 0 & 1 & 2 \\
    \hline
    P_{\theta} & \theta & 2\theta & 1 - 3\theta \\
    \hline
\end{array}$
\\
\textbf{C1.} 
Agar \(X^{(n)} = \left( X_{1},...,X_{n} \right)\) tanlanma \(\sqrt{\theta}\) parametrli Bernulli taqsimotidan olingan bo'lsa, u holda noma'lum \(\theta\) parametr uchun \((\overline{x})^{2}\) bahoning siljimaganligi va asosliligini tekshiring.
\\
\textbf{C2.} 
Agar \(X^{(n)} = \left( X_{1},...,X_{n} \right)\) tanlanma \(1/\theta\) parametrli ko'rsatkichli taqsimotdan olingan bo'lsa, u holda noma'lum \(\theta\) parametr uchun momentlar usuli bahosini \(\ \ g(x) = x^{k},\) \(k \in N\) funksiya yordamida toping.
\\
\textbf{C3.} 
Agar \(X^{(n)} = \left( X_{1},...,X_{n} \right)\) tanlanma zichlik funksiyasi \(f(x;\theta) = \left\{ \begin{array}{r}
\begin{matrix}
\theta_{1}^{- 1}e^{\frac{x - \theta_{2}}{\theta_{1}}},\ \ x \geq \theta_{2}
\end{matrix} \\
0,\ \ \ \ x < \theta_{2}
\end{array} \right.\ \) bo'lgan taqsimotdan olingan bo'lsa, u holda noma'lum \(.\left( \theta_{1},\theta_{2} \right),\) \(\theta_{1} > 0,\) \(\theta_{2} \in R\) vektor parametrning haqiqatga maksimal o'xshashlik bahosini toping.
\\

\end{tabular}
\vspace{1cm}


\begin{tabular}{m{17cm}}
\textbf{86-variant}
\newline

\textbf{T1.} 
Tanlanma xarakteristikalar. (Variatsion qator, nisbiy chastota).
\\
\textbf{T2.} 
Haqiqatga maksimal o'xshashlik usuli. (haqiqatga maksimal o'xshashlik funktsiyasi, noma'lum parametrlarni baholash).
\\
\textbf{A1.} 
Hajmi \(n = 20\) ga teng bo'lgan tanlanma berilgan: 3,7; 3,1; 4,8; 2,8; 3,1; 4,3; 3,7; 4,3; 2,4; 3,1; 2,4; 4,3; 3,1; 3,7; 4,8; 2,8; 2,4; 2,8; 2,4; 3,1. Bu tanlanmaning statistik taqsimotin toping.
\\
\textbf{A2.} 
Hajmi \(n = 20\) ga teng bo'lgan tanlanma berilgan: 3,8; 3,4; 4,8; 2,9; 3,4; 4,6; 3,8; 4,6; 2,1; 3,4; 2,1; 4,6; 3,4; 3,8; 4,8; 2,9; 2,1; 2,9; 2,1; 3,4. Bu tanlanmaning empirik taqsimot funksiyasin toping.
\\
\textbf{A3.} 
Oliy matematika fanidan 10 ta talaba test topshiriqlarin topshirdi. Harbir talaba 10 balgacha to'plashi mumkin. Agar test topshiriqlari natijalari bo'yicha \{6, 5, 6, 9, 5, 7, 10, 5, 9, 8\} tanlanma olingan bo'lsa, ushbu tanlanmalarning tanlanma o'rta va tanlanma dispersiyalarin toping.
\\
\textbf{B1.} 
Agar normal taqsimlangan bosh to'plamdan olingan hajmi \(n = 25\)ga teng tanlanma bo'yicha \(\overline{x} = 9\) tanlanma o'rta va \({\overline{S}}^{2} = 0,64\) tuzatilgan tanlanma dispersiyalar topilgan bo'lsa, u holda \(\gamma = 0,95\) ishonchlilik bilan noma'lum \(\theta\) matematik kutilma uchun ishonchlilik intervalin tuzing.
\\
\textbf{B2.} 
Ko'rsatkichli taqsimot noma'lum \(\theta > 0\) parametri momentlar usuli bahosini toping.
\\
\textbf{B3.} 
Agar \(X^{(n)} = \left( X_{1},...,X_{n} \right)\) tanlanma \(\left( a,\theta^{2} \right)\) parametrli normal taqsimotdan olingan bo'lsa (\(\alpha -\) ma'lum), u holda noma'lum \(\theta^{2}\) parametrning haqiqatga maksimal o'xshashlik bahosini toping.
\\
\textbf{C1.} 
Agar \(X^{(n)} = \left( X_{1},...,X_{n} \right)\) tanlanma \(\theta\) parametrli Bernulli taqsimotidan olingan bo'lsa, u holda noma'lum \(\theta\) parametr uchun \(X_{n}\) bahoning siljimaganligi va asosliligini tekshiring.
\\
\textbf{C2.} 
Agar \(X^{(n)} = \left( X_{1},...,X_{n} \right)\) tanlanma\(\ \ (a,\theta^{2})\) parametrli normal taqsimotdan olingan bo'lsa (\(\alpha -\) ma'lum), u holda noma'lum\(\ \ \theta^{2}\) parametr uchun momentlar usuli bahosini toping.
\\
\textbf{C3.} 
Agar \(X^{(n)} = \left( X_{1},...,X_{n} \right)\) tanlanma \(\left\lbrack \theta_{1},\theta_{2} \right\rbrack\) oraliqda tekis taqsimotdan olingan bo'lsa, u holda noma'lum \(\left( \theta_{1},\theta_{2} \right)\) vektor parametrning haqiqatga maksimal o'xshashlik bahosini toping.
\\

\end{tabular}
\vspace{1cm}


\begin{tabular}{m{17cm}}
\textbf{87-variant}
\newline

\textbf{T1.} 
Tanlanma xarakteristikalari.(tanlanma o'rta, tanlanma dispersiya).
\\
\textbf{T2.} 
Kolmogorovning muvofiqlik kritireyesi (Kolmogorov teoremasi)
\\
\textbf{A1.} 
Hajmi \(n = 20\) ga teng bo'lgan tanlanma berilgan: 1,5; -0,9; -2,4; -0,9; 0,7; 1,5; -0,9; -0,2; -2,4; 0,7; -2,4; 0,7; -0,9; 1,5; -1,7; -0,9; -0,2; 0,7; -1,7; -0,9. Bu tanlanmaning statistik taqsimotin toping.
\\
\textbf{A2.} 
Hajmi \(n = 20\) ga teng bo'lgan tanlanma berilgan: 1,9; -0,3; -2,7; -0,3; 0,6; 1,9; -0,3; -0,1; -2,7; 0,6; -2,7; 0,6; -0,3; 1,9; -1,8; -0,3; -0,1; 0,6; -1,8; -0,3. Bu tanlanmaning empirik taqsimot funksiyasin toping.
\\
\textbf{A3.} 
Oliy matematika fanidan 10 ta talaba test topshiriqlarin topshirdi. Harbir talaba 10 balgacha to'plashi mumkin. Agar test topshiriqlari natijalari bo'yicha \{4, 6, 6, 9, 5, 8, 4, 7, 5, 6\} tanlanma olingan bo'lsa, ushbu tanlanmalarning tanlanma o'rta va tanlanma dispersiyalarin toping.
\\
\textbf{B1.} 
Agar normal taqsimlangan bosh to'plamdan olingan hajmi \(n = 10\) ga teng bo'lgan tanlanma bo'yicha \({\overline{S}}^{2} = 0,6\) tuzatilgan tanlanma dispersiya topilgan bo'lsa, u holda \(\gamma = 0,95\) ishonchlilik bilan noma'lum \(\theta_{2}^{2}\) dispersiya uchun ishonchlilik intervalin tuzing.
\\
\textbf{B2.} 
Agar (3,0,-2,0,-2,3,-2,0,0,3,0,0,0,0,3,-2,0,0,-2,3,0) tanlanma quyida berilgan taqsimotdan olingan bo'lsa, u holda noma'lum \(\left( \theta_{1},\theta_{2} \right)\) vektor parametr uchun momentlar usuli bahosini toping.
$\begin{array}{|c|c|c|c|}
    \hline
    \xi & - 2 & 0 & 3 \\
    \hline
    P_{\theta} & 2\theta_{1} & 0,5 + \theta_{1} + \theta_{2} & \theta_{2} \\
    \hline
\end{array}$
\\
\textbf{B3.} 
Agar (-1,-1,0,-1,0,-1,-1,5,-1,0,-1,0,5,-1,-1,-1,5,-1,-1,-1,1,-1,5,0,-1,-1,5) tanlanma quyida berilgan taqsimotdan olingan bo'lsa, u holda noma'lum \(\theta\) parametrning haqiqatga maksimal o'xshashlik usuli bahosini toping.
$\begin{array}{|c|c|c|c|}
    \hline
    \xi & - 1 & 0 & 5\\
    \hline
    P_{\theta} & 1 - \theta & \theta/2 & \theta/2 \\
    \hline
\end{array}$
\\
\textbf{C1.} 
Agar \(X^{(n)} = \left( X_{1},...,X_{n} \right)\) tanlanma \(\theta\) parametrli Bernulli taqsimotidan olingan bo'lsa, u holda noma'lum \(\theta(1 - \theta)\) parametr uchun \(X_{1}\left( 1 - X_{n} \right)\) bahoning siljimaganligi va asosliligini tekshiring.
\\
\textbf{C2.} 
Agar \(X^{(n)} = \left( X_{1},...,X_{n} \right)\) tanlanma \(1\sqrt{\theta}\) parametrli ko'rsatkichli taqsimotdan olingan bo'lsa, u holda noma'lum \(\theta\) parametr uchun momentlar usuli bahosini toping.
\\
\textbf{C3.} 
Agar \(X^{(n)} = \left( X_{1},...,X_{n} \right)\) tanlanma zichlik funksiyasi\(f(x;\theta) = \frac{\theta}{2}e^{- \theta|x|},\ \ x \in R\) bo'lgan taqsimotdan olingan bo'lsa, u holda noma'lum \(\theta > 0\) parametrning haqiqatga maksimal o'xshashlik bahosini toping.
\\

\end{tabular}
\vspace{1cm}


\begin{tabular}{m{17cm}}
\textbf{88-variant}
\newline

\textbf{T1.} 
Neyman-Pirson teoremasi.
\\
\textbf{T2.} 
Momentler usuli. (tanlanma momentleri, noma'lum parametrlarni baholash).
\\
\textbf{A1.} 
Hajmi \(n = 20\) ga teng bo'lgan tanlanma berilgan:9,4; 6,8; -8,5; 9,4; 2,9; 9,4; -8,5; -6,4; 6,8; -8,5; 9,4; -6,4; 6,8; 9,4; 2,9; 9,4; -3,6; -8,5; 2,9; -6,4. Bu tanlanmaning statistik taqsimotin toping.
\\
\textbf{A2.} 
Hajmi \(n = 20\) ga teng bo'lgan tanlanma berilgan:9,1; 6,4; -8,6; 9,1; 2,3; 9,1; -8,6; -6,2; 6,4; -8,6; 9,1; -6,2; 6,4; 9,1; 2,3; 9,1; -3,9; -8,6; 2,3; -6,2. Bu tanlanmaning empirik taqsimot funksiyasin toping.
\\
\textbf{A3.} 
Oliy matematika fanidan 10 ta talaba test topshiriqlarin topshirdi. Harbir talaba 10 balgacha to'plashi mumkin. Agar test topshiriqlari natijalari bo'yicha \{3, 7, 6, 4, 5, 4, 3, 7, 8, 3\} tanlanma olingan bo'lsa, ushbu tanlanmalarning tanlanma o'rta va tanlanma dispersiyalarin toping.
\\
\textbf{B1.} 
Agar o'rta kvadratik chetlanish \(\sigma = 5\) bo'lgan normal taqsimot bosh to'plamdan olingan hajmi \(n = 16\)ga teng tanlanma bo'yicha \(\overline{x} = 3,6\) tanlanma o'rta qiymati topilgan bo'lsa, u holda \(\gamma = 0,90\) ishonchlilik bilan noma'lum \(\theta\) matematik kutilma uchun ishonchlilik intervalin tuzing .
\\
\textbf{B2.} 
Agar (-2,0,-2,0,-2,3,-2,0,0,3,0,0,0,0,3,-2,0,0,-2,3,0) tanlanma quyida berilgan taqsimotdan olingan bo'lsa, u holda noma'lum \(\left( \theta_{1},\theta_{2} \right)\) vektor parametr uchun momentlar usuli bahosini toping.
$\begin{array}{|c|c|c|c|}
    \hline
    \xi & - 2 & 0 & 3\\
    \hline
    P_{\theta} & \theta_{1} & 1 - \theta_{1} - \theta_{2} & \theta_{2} \\
    \hline
\end{array}$
\\
\textbf{B3.} 
Agar \(X^{(n)} = \left( X_{1},...,X_{n} \right)\) tanlanma zichlik funksiyasi \(f(x;\theta) = \frac{2x}{\theta}e^{- \frac{x^{2}}{\theta}},\ \ x \geq 0\) bo'lgan taqsimotdan olingan bo'lsa, u holda noma'lum \(\theta > 0\) parametrning haqiqatga maksimal usuli bahosini toping.
\\
\textbf{C1.} 
Agar \(X^{(n)} = \left( X_{1},...,X_{n} \right)\) tanlanma \(\theta\) parametrli Bernulli taqsimotidan olingan bo'lsa, u holda noma'lum \(\theta^{2}\) parametr uchun \(X_{1}X_{n}\) bahoning siljimaganligi va asosliligini tekshiring.
\\
\textbf{C2.} 
Agar \(X^{(n)} = \left( X_{1},...,X_{n} \right)\) tanlanma \(\theta\) parametrli Puasson taqsimotidan olingan bo'lsa, u holda noma'lum \(\theta\) parametr uchun momentlar usuli bahosini toping.
\\
\textbf{C3.} 
\(f(x;\theta) = \frac{7x^{6}}{\sqrt{2\pi}}\exp\left\{ - \frac{(x^{7} - \theta)^{2}}{2} \right\}\) model uchun \(\theta\) parametri haqiqatga maksimal o'xshashlik usuli bahosi topilsin.
\\

\end{tabular}
\vspace{1cm}


\begin{tabular}{m{17cm}}
\textbf{89-variant}
\newline

\textbf{T1.} 
Poligon va gistogramma(nisbiy chastota, interval qator, grafik).
\\
\textbf{T2.} 
Chiziqli korrelyatsiya tenglamasi (ta'rifi, regressiya to'g'ri chiziqning tanlanma tenglamalari)
\\
\textbf{A1.} 
Hajmi \(n = 20\) ga teng bo'lgan tanlanma berilgan: 6,2; -5,3; 7,2; 3,7; -2,2; 6,2; 3,7; -7,6; 3,7; 7,2; 6,2; -5,3; -7,6; -5,3; -7,6; 6,2; 7,2; -2,2; -7,6; 7,2. Bu tanlanmaning statistik taqsimotin toping.
\\
\textbf{A2.} 
Hajmi \(n = 20\) ga teng bo'lgan tanlanma berilgan: 6,1; -5,8; 7,9; 3,5; -2,5; 6,1; 3,5; -7,2; 3,5; 7,9; 6,1; -5,8; -7,2; -5,8; -7,2; 6,1; 7,9; -2,5; -7,2; 7,9. Bu tanlanmaning empirik taqsimot funksiyasin toping.
\\
\textbf{A3.} 
Oliy matematika fanidan 10 ta talaba test topshiriqlarin topshirdi. Harbir talaba 10 balgacha to'plashi mumkin. Agar test topshiriqlari natijalari bo'yicha \{10, 8, 6, 5, 4, 8, 10, 7, 5, 7\} tanlanma olingan bo'lsa, ushbu tanlanmalarning tanlanma o'rta va tanlanma dispersiyalarin toping.
\\
\textbf{B1.} 
Agar normal taqsimlangan bosh to'plamdan olingan hajmi \(n = 16\)ga teng tanlanma bo'yicha \(\overline{x} = 15,2\) tanlanma o'rta va \({\overline{S}}^{2} = 0,81\) tuzatilgan tanlanma dispersiyalar topilgan bo'lsa, u holda \(\gamma = 0,90\) ishonchlilik bilan noma'lum \(\theta\) matematik kutilma uchun ishonchlilik intervalin tuzing.
\\
\textbf{B2.} 
Agar zichlik funksiyasi \(f(x) = \frac{2x}{\theta}e^{- \frac{x^{2}}{\theta}},\ \ x \geq 0\) ko'rinishga ega bo'lsa, u holda \(\theta\) parametr momentlar usuli bahosini toping.
\\
\textbf{B3.} 
Agar \(X^{(n)} = \left( X_{1},...,X_{n} \right)\) tanlanma \(\left\lbrack - \theta,\theta^{2} \right\rbrack\) oraliqda tekis taqsimotdan olingan bo'lsa, u holda noma'lum \(\theta > 0\) parametrning haqiqatga maksimal o'xshashlik usuli bahosini toping.
\\
\textbf{C1.} 
Agar \(X^{(n)} = \left( X_{1},...,X_{n} \right)\) tanlanma \((\alpha,\theta)\) parametrli Veybull taqsimotdan olingan bo'lsa (\(\alpha -\) ma'lum), u holda noma'lum \(\theta\) parametr uchun \(1/\overline{x^{\alpha}}\) bahoning siljimaganligi va asosliligini tekshiring.
\\
\textbf{C2.} 
Agar \(X^{(n)} = \left( X_{1},...,X_{n} \right)\) tanlanma \(\theta\) parametrli geometrik taqsimotdan olingan bo'lsa, u holda noma'lum \(\theta\) parametr uchun momentlar usuli bahosini toping.
\\
\textbf{C3.} 
\(f(x,\theta) = \frac{4x^{3}}{\theta_{2}\sqrt{2\pi}}\exp\left\{ - \frac{\left( x^{4} - \theta_{1} \right)^{2}}{2{\theta_{2}}^{2}} \right\}\) model uchun \(\theta_{1}\) va \(\theta_{2}\) parametrlarning haqiqatga maksimal o'xshashlik usuli baholari topilsin.
\\

\end{tabular}
\vspace{1cm}


\begin{tabular}{m{17cm}}
\textbf{90-variant}
\newline

\textbf{T1.} 
Tanlanma momentleri (\(k -\)tartibli boshlang'ich, boshlang'ich absolyut, markaziy va markaziy absolyut momentler).
\\
\textbf{T2.} 
Statistik gipotezalarni tekshirish (kritik to'plam, 1 va 2-tur xatolik)
\\
\textbf{A1.} 
Hajmi \(n = 20\) ga teng bo'lgan tanlanma berilgan: 9,6; 1,5; 7,4; 9,6; 2,8; 1,5; 6,3; 1,5; 9,6; 6,3; 2,8; 4,1; 6,3; 9,6; 1,5; 1,5; 6,3; 7,4; 4,1; 7,4. Bu tanlanmaning statistik taqsimotin toping.
\\
\textbf{A2.} 
Hajmi \(n = 20\) ga teng bo'lgan tanlanma berilgan: 9,8; 1,2; 7,1; 9,8; 2,9; 1,2; 6,7; 1,2; 9,8; 6,7; 2,9; 4,6; 6,7; 9,8; 1,2; 1,2; 6,7; 7,1; 4,6; 7,1. Bu tanlanmaning empirik taqsimot funksiyasin toping.
\\
\textbf{A3.} 
Oliy matematika fanidan 10 ta talaba test topshiriqlarin topshirdi. Harbir talaba 10 balgacha to'plashi mumkin. Agar test topshiriqlari natijalari bo'yicha \{9, 10, 5, 6, 4, 8, 4, 6, 10, 8\} tanlanma olingan bo'lsa, ushbu tanlanmalarning tanlanma o'rta va tanlanma dispersiyalarin toping.
\\
\textbf{B1.} 
Agar normal taqsimlangan bosh to'plamdan olingan hajmi \(n = 10\) ga teng bo'lgan tanlanma bo'yicha \({\overline{S}}^{2} = 0,45\) tuzatilgan tanlanma dispersiya topilgan bo'lsa, u holda \(\gamma = 0,95\) ishonchlilik bilan noma'lum \(\theta_{2}^{2}\) dispersiya uchun ishonchlilik intervalin tuzing.
\\
\textbf{B2.} 
\(\lbrack 0,\theta\rbrack\) oraliqda tekis taqsimlangan \(\theta\) parametri uchun momentlar usuli bahosini toping.
\\
\textbf{B3.} 
\(f(x) = \frac{2x}{\theta}e^{- \frac{x^{2}}{\theta}},\ \ x \geq 0\) model uchun \(\theta\) parametri haqiqatga maksimal o'xshashlik usuli bahosi topilsin.
\\
\textbf{C1.} 
Agar \(X^{(n)} = \left( X_{1},...,X_{n} \right)\) tanlanma \(\theta\) parametrli geometrik taqsimotdan olingan bo'lsa, u holda noma'lum \(\theta\) parametr uchun \(t(1 + \overline{x})\) bahoning siljimaganligi va asosliligini tekshiring.
\\
\textbf{C2.} 
Agar \(X^{(n)} = \left( X_{1},...,X_{n} \right)\) tanlanma \({\lbrack\theta}_{1},\theta_{2}\rbrack\) oraliqda tekis taqsimotdan olingan bo'lsa, u holda noma'lum \(\left( \theta_{1},\theta_{2} \right)\) vektor parametr uchun momentlar usuli bahosini toping.
\\
\textbf{C3.} 
Agar \(X^{(n)} = \left( X_{1},...,X_{n} \right)\) tanlanma \(\lbrack\theta,\theta + 2\rbrack\) oraliqda tekis taqsimotdan olingan bo'lsa, u holda noma'lum \(\theta\) parametrning haqiqatga maksimal o'xshashlik usuli bahosini toping.
\\

\end{tabular}
\vspace{1cm}


\begin{tabular}{m{17cm}}
\textbf{91-variant}
\newline

\textbf{T1.} 
Poligon va gistogramma(nisbiy chastota, interval qator, grafik).
\\
\textbf{T2.} 
Statistik baho xossalari. (Siljimagan, asosliy, effektiv)
\\
\textbf{A1.} 
Hajmi \(n = 20\) ga teng bo'lgan tanlanma berilgan:1,8; -8,4; 7,3; 4,7; -3,9; 1,8; 4,7; -10,4; -8,4; 7,3; -10,4; 4,7; -8,4; 1,8; 4,7; -10,4; 7,3; -3,9; 4,7; -8,4. Bu tanlanmaning statistik taqsimotin toping.
\\
\textbf{A2.} 
Hajmi \(n = 20\) ga teng bo'lgan tanlanmaberilgan:1,6; -8,3; 7,6; 4,2; -3,1; 1,6; 4,2; -10,5; -8,3; 7,6; -10,5; 4,2; -8,3; 1,6; 4,2; -10,5; 7,6; -3,1; 4,2; -8,3. Bu tanlanmaning empirik taqsimot funksiyasin toping.
\\
\textbf{A3.} 
Oliy matematika fanidan 10 ta talaba test topshiriqlarin topshirdi. Harbir talaba 10 balgacha to'plashi mumkin. Agar test topshiriqlari natijalari bo'yicha \{9, 3, 6, 3, 7, 6, 4, 6, 10, 6\} tanlanma olingan bo'lsa, ushbu tanlanmalarning tanlanma o'rta va tanlanma dispersiyalarin toping.
\\
\textbf{B1.} 
Agar o'rta kvadratik chetlanish \(\sigma = 2\) bo'lgan normal taqsimot bosh to'plamdan olingan hajmi \(n = 18\)ga teng tanlanma bo'yicha \(\overline{x} = 5,2\) tanlanma o'rta qiymati topilgan bo'lsa, u holda \(\gamma = 0,90\) ishonchlilik bilan noma'lum \(\theta\) matematik kutilma uchun ishonchlilik intervalin tuzing .
\\
\textbf{B2.} 
Agar (0,-2,0,-2,3,-2,0,0,3,0,0,0,0,3,-2,0,0,-2,3,0,3) tanlanma quyida berilgan taqsimotdan olingan bo'lsa, u holda noma'lum \(\theta\) parametr uchun momentlar usuli bahosini toping.
$\begin{array}{|c|c|c|c|}
    \hline
    \xi & - 2 & 0 & 3 \\
    \hline
    P_{\theta} & \theta & 1 - 2\theta & \theta \\
    \hline
\end{array}$
\\
\textbf{B3.} 
Agar \(X^{(n)} = \left( X_{1},...,X_{n} \right)\) tanlanma \(\lbrack - \theta,\theta\rbrack\) oraliqda tekis taqsimotdan olingan bo'lsa, u holda noma'lum \(\theta > 0\) parametrning haqiqatga maksimal o'xshashlik usuli bahosini toping.
\\
\textbf{C1.} 
Agar \(X^{(n)} = \left( X_{1},...,X_{n} \right)\) tanlanma \(\theta\) parametrli Puasson taqsimotidan olingan bo'lsa, u holda noma'lum \(\theta\) parametr uchun \(\frac{n + 3}{n + 4}\overline{x}\) bahoning siljimaganligi va asosliligini tekshiring.
\\
\textbf{C2.} 
Agar \(X^{(n)} = \left( X_{1},...,X_{n} \right)\) tanlanma \(\left( \theta_{1},\theta_{2} \right)\) parametrli gamma taqsimotdan olingan bo'lsa, u holda noma'lum \(\left( \theta_{1},\theta_{2} \right)\) vektor parametr uchun momentlar usuli bahosini toping.
\\
\textbf{C3.} 
Agar \(X^{(n)} = \left( X_{1},...,X_{n} \right)\) tanlanma zichlik funksiyasi\(f(x;\theta) = \frac{1}{2}e^{- |x - \theta|},\ \ x \in R\) bo'lgan Laplas taqsimotidan olingan bo'lsa, u holda noma'lum \(\theta \in R\) parametrning haqiqatga maksimal o'xshashlik bahosini toping.
\\

\end{tabular}
\vspace{1cm}


\begin{tabular}{m{17cm}}
\textbf{92-variant}
\newline

\textbf{T1.} 
Tanlanma xarakteristikalari.(tanlanma o'rta, tanlanma dispersiya).
\\
\textbf{T2.} 
Pirsonning xi-kvadrat muvofiqlik kriteriysi (Pirson teoremasi).
\\
\textbf{A1.} 
Hajmi \(n = 20\) ga teng bo'lgan tanlanma berilgan: 2,7; -13,5; 1,2; 2,7; 1,2; 4,9; -9,5; 1,2; 2,7; 4,9; -9,5; 2,7; -3,5; 1,2; 2,7; 4,9; -3,5; 2,7; 4,9; 1,2;. Bu tanlanmaning statistik taqsimotin toping.
\\
\textbf{A2.} 
Hajmi \(n = 20\) ga teng bo'lgan tanlanma berilgan: 2,8; -13,9; 1,9; 2,8; 1,9; 4,3; -9,4; 1,9; 2,8; 4,3; -9,4; 2,8; -3,7; 1,9; 2,8; 4,3; -3,7; 2,8; 4,3; 1,9. Bu tanlanmaning empirik taqsimot funksiyasin toping.
\\
\textbf{A3.} 
Oliy matematika fanidan 10 ta talaba test topshiriqlarin topshirdi. Harbir talaba 10 balgacha to'plashi mumkin. Agar test topshiriqlari natijalari bo'yicha \{10, 7, 5, 9, 3, 8, 10, 7, 8, 3\} tanlanma olingan bo'lsa, ushbu tanlanmalarning tanlanma o'rta va tanlanma dispersiyalarin toping.
\\
\textbf{B1.} 
Agar normal taqsimlangan bosh to'plamdan olingan hajmi \(n = 36\)ga teng tanlanma bo'yicha \(\overline{x} = 20,2\) tanlanma o'rta va \({\overline{S}}^{2} = 0,81\) tuzatilgan tanlanma dispersiyalar topilgan bo'lsa, u holda \(\gamma = 0,95\) ishonchlilik bilan noma'lum \(\theta\) matematik kutilma uchun ishonchlilik intervalin tuzing.
\\
\textbf{B2.} 
\(\left\lbrack \theta_{1},\theta_{2} \right\rbrack\) oraliqda tekis taqsimot parametrlari uchun momentlar usuli baholarini toping.
\\
\textbf{B3.} 
\(f(x) = \frac{\theta}{2}e^{- \theta|x|}\) model uchun \(\theta\) parametri haqiqatga maksimal o'xshashlik usuli bahosi topilsin.
\\
\textbf{C1.} 
Agar \(X^{(n)} = \left( X_{1},...,X_{n} \right)\) tanlanma \(\theta\) parametrli Puasson taqsimotidan olingan bo'lsa, u holda noma'lum \(\theta\) parametr uchun \(\frac{X_{1} + X_{3}}{2}\) bahoning siljimaganligi va asosliligini tekshiring.
\\
\textbf{C2.} 
Agar \(X^{(n)} = \left( X_{1},...,X_{n} \right)\) tanlanma {[}\(0,2\theta\rbrack\) oraliqda tekis taqsimotdan olingan bo'lsa, u holda noma'lum \(\theta > 0\) parametr uchun momentlar usuli bahosini toping.
\\
\textbf{C3.} 
Agar \(X^{(n)} = \left( X_{1},...,X_{n} \right)\) tanlanma zichlik funksiyasi \(f(x;\theta) = \left\{ \begin{array}{r}
\begin{matrix}
\theta_{1}^{- 1}e^{\frac{x - \theta_{2}}{\theta_{1}}},\ \ x \geq \theta_{2}
\end{matrix} \\
0,\ \ \ \ x < \theta_{2}
\end{array} \right.\ \) bo'lgan taqsimotdan olingan bo'lsa, u holda noma'lum \(.\left( \theta_{1},\theta_{2} \right),\) \(\theta_{1} > 0,\) \(\theta_{2} \in R\) vektor parametrning haqiqatga maksimal o'xshashlik bahosini toping.
\\

\end{tabular}
\vspace{1cm}


\begin{tabular}{m{17cm}}
\textbf{93-variant}
\newline

\textbf{T1.} 
Neyman-Pirson teoremasi.
\\
\textbf{T2.} 
Haqiqatga maksimal o'xshashlik usuli. (haqiqatga maksimal o'xshashlik funktsiyasi, noma'lum parametrlarni baholash).
\\
\textbf{A1.} 
Hajmi \(n = 20\) ga teng bo'lgan tanlanma berilgan: 9,9; 5,7; 3,2; 2,8; 5,7; 9,9; 7,5; 3,7; 9,9; 3,2; 2,8; 3,7; 7,5; 5,7; 3,2; 2,8; 7,5; 3,2; 9,9; 7,5. Bu tanlanmaning statistik taqsimotin toping.
\\
\textbf{A2.} 
Hajmi \(n = 20\) ga teng bo'lgan tanlanma berilgan: 9,7; 5,2; 3,2; 2,4; 5,2; 9,7; 7,5; 3,7; 9,7; 3,2; 2,4; 3,7; 7,5; 5,2; 3,2; 2,4; 7,5; 3,2; 9,7; 7,5. Bu tanlanmaning empirik taqsimot funksiyasin toping.
\\
\textbf{A3.} 
Oliy matematika fanidan 10 ta talaba test topshiriqlarin topshirdi. Harbir talaba 10 balgacha to'plashi mumkin. Agar test topshiriqlari natijalari bo'yicha \{1, 6, 2, 6, 3, 6, 4, 6, 10, 6\} tanlanma olingan bo'lsa, ushbu tanlanmalarning tanlanma o'rta va tanlanma dispersiyalarin toping.
\\
\textbf{B1.} 
Agar normal taqsimlangan bosh to'plamdan olingan hajmi \(n = 10\) ga teng bo'lgan tanlanma bo'yicha \({\overline{S}}^{2} = 0,7\) tuzatilgan tanlanma dispersiya topilgan bo'lsa, u holda \(\gamma = 0,95\) ishonchlilik bilan noma'lum \(\theta_{2}^{2}\) dispersiya uchun ishonchlilik intervalin tuzing.
\\
\textbf{B2.} 
Ko'rsatkichli taqsimot noma'lum \(\theta > 0\) parametri momentlar usuli bahosini toping.
\\
\textbf{B3.} 
Agar (4,8,5,3) tanlanma \((a,\theta^{2}\) parametrli normal taqsimotdan olingan bo'lsa, u holda noma'lum \(\theta^{2}\) parametrning haqiqatga maksimal o'xshashlik bahosini toping.
\\
\textbf{C1.} 
Agar \(X^{(n)} = \left( X_{1},...,X_{n} \right)\) tanlanma \(\ln\theta\) parametrli Puasson taqsimotidan olingan bo'lsa, u holda noma'lum \(\theta\) parametr uchun \(e^{\overline{x}}\) bahoning siljimaganligi va asosliligini tekshiring.
\\
\textbf{C2.} 
Agar \(X^{(n)} = \left( X_{1},...,X_{n} \right)\) tanlanma \((\theta,2\theta)\) parametrli normal taqsimotdan olingan bo'lsa, u holda noma'lum \(\theta > 0\) parametr uchun momentlar usuli bahosini \(\ \ g(x) = (x)^{2}\) funksiya yordamida toping.
\\
\textbf{C3.} 
\(f(x,\theta) = \frac{e^{x}}{\sqrt{2\pi}}\exp\left\{ - \frac{\left( e^{x} - \theta \right)^{2}}{2} \right\}\) model uchun \(\theta\) parametri haqiqatga maksimal o'xshashlik usuli bahosi topilsin.
\\

\end{tabular}
\vspace{1cm}


\begin{tabular}{m{17cm}}
\textbf{94-variant}
\newline

\textbf{T1.} 
Tanlanma xarakteristikalar. (Variatsion qator, nisbiy chastota).
\\
\textbf{T2.} 
Statistik gipotezalarni tekshirish (kritik to'plam, 1 va 2-tur xatolik).
\\
\textbf{A1.} 
Hajmi \(n = 20\) ga teng bo'lgan tanlanma berilgan: 3,6; 1,1; -1,8; 0,4; 3,6; 0; 5,3; 1,1; 0; -1,8; 3,6; 0,4; 1,1; 0; 0,4; 1,1; 3,6; -1,8; 3,6; 0. Bu tanlanmaning statistik taqsimotin toping.
\\
\textbf{A2.} 
Hajmi \(n = 20\) ga teng bo'lgan tanlanma berilgan: 3,2; 1,8; -1,1; 0,9; 3,2; 0; 5,6; 1,8; 0; -1,1; 3,2; 0,9; 1,8; 0; 0,9; 1,8; 3,2; -1,1; 3,2; 0. Bu tanlanmaning empirik taqsimot funksiyasin toping.
\\
\textbf{A3.} 
Oliy matematika fanidan 10 ta talaba test topshiriqlarin topshirdi. Harbir talaba 10 balgacha to'plashi mumkin. Agar test topshiriqlari natijalari bo'yicha \{2, 7, 3, 7, 6, 7, 4, 7, 7, 10\} tanlanma olingan bo'lsa, ushbu tanlanmalarning tanlanma o'rta va tanlanma dispersiyalarin toping.
\\
\textbf{B1.} 
Agar o'rta kvadratik chetlanish \(\sigma = 3\) bo'lgan normal taqsimot bosh to'plamdan olingan hajmi \(n = 14\)ga teng tanlanma bo'yicha \(\overline{x} = 5,5\) tanlanma o'rta qiymati topilgan bo'lsa, u holda \(\gamma = 0,90\) ishonchlilik bilan noma'lum \(\theta\) matematik kutilma uchun ishonchlilik intervalin tuzing .
\\
\textbf{B2.} 
Agar (-2,0,-2,0,-2,3,-2,0,0,3,0,0,0,0,3,-2,0,0,-2,3,0) tanlanma quyida berilgan taqsimotdan olingan bo'lsa, u holda noma'lum \(\left( \theta_{1},\theta_{2} \right)\) vektor parametr uchun momentlar usuli bahosini toping.
$\begin{array}{|c|c|c|c|}
    \hline
    \xi & - 2 & 0 & 3\\
    \hline
    P_{\theta} & \theta_{1} & 1 - \theta_{1} - \theta_{2} & \theta_{2} \\
    \hline
\end{array}$
\\
\textbf{B3.} 
Agar \(X^{(n)} = \left( X_{1},...,X_{n} \right)\) tanlanma \(\theta\) parametrli Bernulli taqsimotidan olingan bo'lsa, u holda noma'lum \(\theta\) parametrning haqiqatga maksimal o'xshashlik usuli bahosini toping.
\\
\textbf{C1.} 
Agar \(X^{(n)} = \left( X_{1},...,X_{n} \right)\) tanlanma \((\alpha,\theta)\) parametrli Pareto taqsimotdan olingan bo'lsa (\(\alpha -\) ma'lum), u holda noma'lum \(\theta\) parametr uchun \(X_{(1)}\) bahoning siljimaganligi va asosliligini tekshiring.
\\
\textbf{C2.} 
Agar \(X^{(n)} = \left( X_{1},...,X_{n} \right)\) tanlanma \((\theta,\theta^{2})\ \ \) parametrli normal taqsimotdan olingan bo'lsa, u holda noma'lum \(\theta > 0\) parametr uchun momentlar usuli bahosini toping.
\\
\textbf{C3.} 
Agar \(X^{(n)} = \left( X_{1},...,X_{n} \right)\) tanlanma zichlik funksiyasi\(f(x;\theta) = \frac{4x^{3}}{\sqrt{2\pi}\theta_{2}}\exp\left\{ - \frac{\left( x^{4} - \theta_{1} \right)^{2}}{2{\theta_{2}}^{2}} \right\},\ \ x \in R\) bo'lgan taqsimotdan olingan bo'lsa, u holda noma'lum \(\left( \theta_{1},\theta_{2}^{2} \right)\) vektor parametrning haqiqatga maksimal o'xshashlik usuli baholarini toping.
\\

\end{tabular}
\vspace{1cm}


\begin{tabular}{m{17cm}}
\textbf{95-variant}
\newline

\textbf{T1.} 
Tanlanma momentleri (\(k -\)tartibli boshlang'ich, boshlang'ich absolyut, markaziy va markaziy absolyut momentler).
\\
\textbf{T2.} 
Ishonchlilik intervallarin tuzish. Aniq ishonchlilik intervallar.
\\
\textbf{A1.} 
Hajmi \(n = 20\) ga teng bo'lgan tanlanma berilgan: 7,1; 3,9; 6,3; 4,6; 7,1; 2,3; 6,3; 3,9; 4,6; 7,1; 2,3; 3,9; 7,6; 2,3; 4,6; 3,9; 2,3; 3,9; 7,6; 4,6. Bu tanlanmaning statistik taqsimotin toping.
\\
\textbf{A2.} 
Hajmi \(n = 20\) ga teng bo'lgan tanlanma berilgan: 7,9; 3,8; 6,1; 4,2; 7,9; 2,4; 6,1; 3,8; 4,2; 7,9; 2,4; 3,8; 10,2; 2,4; 4,2; 3,8; 2,4; 3,8; 10,2; 4,2. Bu tanlanmaning empirik taqsimot funksiyasin toping.
\\
\textbf{A3.} 
Oliy matematika fanidan 10 ta talaba test topshiriqlarin topshirdi. Harbir talaba 10 balgacha to'plashi mumkin. Agar test topshiriqlari natijalari bo'yicha \{9, 8, 6, 8, 6, 4, 5, 4, 7, 4\} tanlanma olingan bo'lsa, ushbu tanlanmalarning tanlanma o'rta va tanlanma dispersiyalarin toping.
\\
\textbf{B1.} 
Agar normal taqsimlangan bosh to'plamdan olingan hajmi \(n = 49\)ga teng tanlanma bo'yicha \(\overline{x} = 14,2\) tanlanma o'rta va \({\overline{S}}^{2} = 0,64\) tuzatilgan tanlanma dispersiyalar topilgan bo'lsa, u holda \(\gamma = 0,95\) ishonchlilik bilan noma'lum \(\theta\) matematik kutilma uchun ishonchlilik intervalin tuzing.
\\
\textbf{B2.} 
Puasson taqsimoti noma'lum \(\theta > 0\) parametri momentlar usuli bahosini toping.
\\
\textbf{B3.} 
Agar \(x_{1} = 1,1;\ \ x_{2} = 2,7;\ldots;x_{100} = 1,5\) tanlanma \(\theta\) parametrli ko'rsatkichli taqsimotdan olingan bo'lib, \(\sum_{k = 1}^{100}x_{k} = 200\) bo'lsa, u holda noma'lum \(\theta\) parametrning haqiqatga maksimal o'xshashlik bahosini toping.
\\
\textbf{C1.} 
Agar \(X^{(n)} = \left( X_{1},...,X_{n} \right)\) tanlanma zichlik funksiyasi bo'lsa: \(f(x;\theta) = e^{- x + \theta}\left( 1 + e^{- x + \theta} \right)^{2},\ \ x \in R\)bo'lgan taqsimotdan olingan bo'lsa, u holda noma'lum \(\theta\) parametr uchun \(\overline{x}\) bahoning siljimaganligi va asosliligini tekshiring.
\\
\textbf{C2.} 
Agar \(X^{(n)} = \left( X_{1},...,X_{n} \right)\) tanlanma\(\ \ (a,\theta^{2})\) parametrli normal taqsimotdan olingan bo'lsa (\(\alpha -\) ma'lum), u holda noma'lum\(\ \ \theta^{2}\) parametr uchun momentlar usuli bahosini toping.
\\
\textbf{C3.} 
Agar \(X^{(n)} = \left( X_{1},...,X_{n} \right)\) tanlanma zichlik funksiyasi \(f(x;\theta) = \frac{e^{x}}{\sqrt{2\pi}}\exp\left\{ - \frac{\left( e^{x} - \theta \right)^{2}}{2} \right\},\ \ x \in R\) bo'lgan taqsimotdan olingan bo'lsa, u holda noma'lum \(\theta\) parametrning haqiqatga maksimal o'xshashlik bahosini toping.
\\

\end{tabular}
\vspace{1cm}


\begin{tabular}{m{17cm}}
\textbf{96-variant}
\newline

\textbf{T1.} 
Glivenko-Kantelli teoremasi. (empirik taqsimot funktsiyasi, ehtimollik bilan yaqinlashish).
\\
\textbf{T2.} 
Statistik gipotezalarni tekshirish (kritik to'plam, 1 va 2-tur xatolik)
\\
\textbf{A1.} 
Hajmi \(n = 20\) ga teng bo'lgan tanlanma berilgan: 0,6; -3,8; -2,3; -4,3; 2,8; 4,7; -2,3; 0,6; -3,8; 2,8; -2,3; -4,3; 0,6; -2,3; 2,8; -3,8; -4,3; -2,3; 2,8; -3,8. Bu tanlanmaning statistik taqsimotin toping.
\\
\textbf{A2.} 
Hajmi \(n = 20\) ga teng bo'lgan tanlanma berilgan: 0,7; -3,1; -2,3; -4,8; 2,6; 4,9; -2,3; 0,7; -3,1; 2,6; -2,3; -4,8; 0,7; -2,3; 2,6; -3,1; -4,8; -2,3; 2,6; -3,1. Bu tanlanmaning empirik taqsimot funksiyasin toping.
\\
\textbf{A3.} 
Oliy matematika fanidan 10 ta talaba test topshiriqlarin topshirdi. Harbir talaba 10 balgacha to'plashi mumkin. Agar test topshiriqlari natijalari bo'yicha \{10, 4, 6, 5, 5, 4, 10, 7, 9, 10\} tanlanma olingan bo'lsa, ushbu tanlanmalarning tanlanma o'rta va tanlanma dispersiyalarin toping.
\\
\textbf{B1.} 
Agar normal taqsimlangan bosh to'plamdan olingan hajmi \(n = 8\) ga teng bo'lgan tanlanma bo'yicha \({\overline{S}}^{2} = 0,35\) tuzatilgan tanlanma dispersiya topilgan bo'lsa, u holda \(\gamma = 0,90\) ishonchlilik bilan noma'lum \(\theta_{2}^{2}\) dispersiya uchun ishonchlilik intervalin tuzing.
\\
\textbf{B2.} 
Agar (3,-2,-2,0,-2,2,-2,0,-2,3,-2,0,3,0,3,-2,0,-2,3,-2,2,-2,-2,3,3,2,-2,2,3,3) tanlanma quyida berilgan taqsimotdan olingan bo'lsa, u holda noma'lum \(\theta\) parametr uchun momentlar usuli bahosini \(g(x) = |x|\) funksiya yordamida toping.
$\begin{array}{|c|c|c|c|}
    \hline
    \xi & -2 & 0 & 3 \\
    \hline
    P_{\theta} & 3\theta & 1 - 5\theta & 2\theta \\
    \hline
\end{array}$
\\
\textbf{B3.} 
Agar \(X^{(n)} = \left( X_{1},...,X_{n} \right)\) tanlanma \(\theta\) parametrli ko'rsatkichli taqsimotdan olingan bo'lsa, u holda noma'lum \(\theta\) parametrning haqiqatga maksimal o'xshashlik usuli bahosini toping.
\\
\textbf{C1.} 
Agar \(X^{(n)} = \left( X_{1},...,X_{n} \right)\) tanlanma zichlik funksiyasi \(f(x;\theta) = \left\{ \begin{matrix}
\alpha^{- 1}e^{- \ \frac{x - \theta}{\alpha}},\ \ x \geq \theta, \\
0,\ \ x < \theta
\end{matrix} \right.\ \)bo'lgan taqsimotdan olingan bo'lsa (\(\alpha -\) ma'lum), u holda noma'lum \(\theta\) parametr uchun \(X_{(1)}\) bahoning siljimaganligi va asosliligini tekshiring.
\\
\textbf{C2.} 
Agar \(X^{(n)} = \left( X_{1},...,X_{n} \right)\) tanlanma \(1\sqrt{\theta}\) parametrli ko'rsatkichli taqsimotdan olingan bo'lsa, u holda noma'lum \(\theta\) parametr uchun momentlar usuli bahosini toping.
\\
\textbf{C3.} 
Agar \(X^{(n)} = \left( X_{1},...,X_{n} \right)\) tanlanma zichlik funksiyasi \(f(x;\theta) = \frac{\theta ln^{\theta - 1}x}{x},\ \ x \in \lbrack 1,e\rbrack\) bo'lgan taqsimotdan olingan bo'lsa, u holda noma'lum \(\theta > 0\) parametr uchun haqiqatga maksimal o'xshashlik bahosini toping.
\\

\end{tabular}
\vspace{1cm}


\begin{tabular}{m{17cm}}
\textbf{97-variant}
\newline

\textbf{T1.} 
Empirik taqsimot funktsiyasi. (Tanlanma, eksperiment)
\\
\textbf{T2.} 
Kolmogorovning muvofiqlik kritireyesi (Kolmogorov teoremasi)
\\
\textbf{A1.} 
Hajmi \(n = 20\) ga teng bo'lgan tanlanma berilgan: 8,9; 2,7; 1,7; 2,2; 5,6; 1,7; 5,6; 2,7; 1,7; 2,2; 5,6; 8,9; 1,7; 2,2; 1,7; 2,7; 1,7; 5,6; 6,1; 8,9. Bu tanlanmaning statistik taqsimotin toping.
\\
\textbf{A2.} 
Hajmi \(n = 20\) ga teng bo'lgan tanlanma berilgan: 8,7; 2,7; 1,5; 2,2; 5,7; 1,5; 5,7; 2,7; 1,5; 2,2; 5,7; 8,7; 1,5; 2,2; 1,5; 2,7; 1,5; 5,7; 6,3; 8,7. Bu tanlanmaning empirik taqsimot funksiyasin toping.
\\
\textbf{A3.} 
Oliy matematika fanidan 10 ta talaba test topshiriqlarin topshirdi. Harbir talaba 10 balgacha to'plashi mumkin. Agar test topshiriqlari natijalari bo'yicha \{9, 8, 6, 9, 5, 4, 5, 7, 8, 9\} tanlanma olingan bo'lsa, ushbu tanlanmalarning tanlanma o'rta va tanlanma dispersiyalarin toping.
\\
\textbf{B1.} 
Agar o'rta kvadratik chetlanish \(\sigma = 4\) bo'lgan normal taqsimot bosh to'plamdan olingan hajmi \(n = 16\)ga teng tanlanma bo'yicha \(\overline{x} = 5,8\) tanlanma o'rta qiymati topilgan bo'lsa, u holda \(\gamma = 0,90\) ishonchlilik bilan noma'lum \(\theta\) matematik kutilma uchun ishonchlilik intervalin tuzing .
\\
\textbf{B2.} 
Agar \(X^{(n)} = \left( X_{1},...,X_{n} \right)\) tanlanma \(\theta\) parametrli ko'rsatkichli taqsimotdan olingan bo'lsa, u holda noma'lum \(\theta\) parametr uchun momentlar usuli bahosini toping.
\\
\textbf{B3.} 
Agar (-1,-1,0,-1,0,-1,-1,5,-1,0,-1,0,5,-1,-1,-1,5,-1,-1,-1,1,-1,5,0,-1,-1,5) tanlanma quyida berilgan taqsimotdan olingan bo'lsa, u holda noma'lum \(\theta\) parametrning haqiqatga maksimal o'xshashlik usuli bahosini toping.
$\begin{array}{|c|c|c|c|}
    \hline
    \xi & - 1 & 0 & 5\\
    \hline
    P_{\theta} & 1 - \theta & \theta/2 & \theta/2 \\
    \hline
\end{array}$
\\
\textbf{C1.} 
Agar \(X^{(n)} = \left( X_{1},...,X_{n} \right)\) tanlanma \(\lbrack 0,\theta\rbrack\) oraliqda tekis taqsimotdan olingan bo'lsa, u holda noma'lum \(\theta\) parametr uchun \((n + 1)X_{(1)})\) bahoning siljimaganligi va asosliligini tekshiring.
\\
\textbf{C2.} 
Agar \(X^{(n)} = \left( X_{1},...,X_{n} \right)\) tanlanma zichlik funksiyasi\(f(x,\theta) = \frac{2x}{\theta^{2}},x \in \lbrack 0,\theta\rbrack\)bo'lgan taqsimotdan olingan bo'lsa, u holda noma'lum \(\theta\) parametr uchun momentlar usuli bahosini toping.
\\
\textbf{C3.} 
Agar \(X^{(n)} = \left( X_{1},...,X_{n} \right)\) tanlanma zichlik funksiyasi\(f(x;\theta) = \left\{ \begin{matrix}
e^{\theta - x},\ \ x \geq \theta, \\
\ \ 0,\ \ x < \theta
\end{matrix} \right.\ \) bo'lgan taqsimotdan olingan bo'lsa, u holda noma'lum \(\theta\) parametrning haqiqatga maksimal o'xshashlik bahosini toping.
\\

\end{tabular}
\vspace{1cm}


\begin{tabular}{m{17cm}}
\textbf{98-variant}
\newline

\textbf{T1.} 
Momentler usuli. (tanlanma momentleri, noma'lum parametrlarni baholash).
\\
\textbf{T2.} 
Momentler usuli. (tanlanma momentleri, noma'lum parametrlarni baholash).
\\
\textbf{A1.} 
Hajmi \(n = 20\) ga teng bo'lgan tanlanma berilgan: 1,8; -1,9; 2,4; 1,8; 2,4; 1,8; 2,4; -0,6; -1,9; 1,8; -0,6; 2,4; -3,3; -1,9; 4,0; -3,3; -3,3; -1,9; -3,3; -1,9. Bu tanlanmaning statistik taqsimotin toping.
\\
\textbf{A2.} 
Hajmi \(n = 20\) ga teng bo'lgan tanlanma berilgan: 1,4; -1,9; 2,5; 1,4; 2,5; 1,4; 2,5; -0,4; -1,9; 1,4; -0,4; 2,5; -3,7; -1,9; 4,5; -3,7; -3,7; -1,9; -3,7; -1,9. Bu tanlanmaning empirik taqsimot funksiyasin toping.
\\
\textbf{A3.} 
Oliy matematika fanidan 10 ta talaba test topshiriqlarin topshirdi. Harbir talaba 10 balgacha to'plashi mumkin. Agar test topshiriqlari natijalari bo'yicha \{4, 3, 8, 4, 8, 3, 9, 4, 7, 10\} tanlanma olingan bo'lsa, ushbu tanlanmalarning tanlanma o'rta va tanlanma dispersiyalarin toping.
\\
\textbf{B1.} 
Agar normal taqsimlangan bosh to'plamdan olingan hajmi \(n = 36\)ga teng tanlanma bo'yicha \(\overline{x} = 20,2\) tanlanma o'rta va \({\overline{S}}^{2} = 0,64\) tuzatilgan tanlanma dispersiyalar topilgan bo'lsa, u holda \(\gamma = 0,90\) ishonchlilik bilan noma'lum \(\theta\) matematik kutilma uchun ishonchlilik intervalin tuzing.
\\
\textbf{B2.} 
Agar \(X^{(n)} = \left( X_{1},...,X_{n} \right)\) tanlanma \(\theta\) parametrli Bernulli taqsimotidan olingan bo'lsa, u holda noma'lum \(\theta\) parametr uchun momentlar usuli bahosini toping.
\\
\textbf{B3.} 
Agar \(X^{(n)} = \left( X_{1},...,X_{n} \right)\) tanlanma \(\theta\) parametrli Bernulli taqsimotidan olingan bo'lsa, u holda noma'lum \(\theta\) parametrning haqiqatga maksimal o'xshashlik usuli bahosini toping.
\\
\textbf{C1.} 
Agar \(X^{(n)} = \left( X_{1},...,X_{n} \right)\) tanlanma \(\lbrack 0,\theta\rbrack\) oraliqda tekis taqsimotdan olingan bo'lsa, u holda noma'lum \(\theta\) parametr uchun \(\frac{n + 1}{n}X_{(n)}\) bahoning siljimaganligi va asosliligini tekshiring.
\\
\textbf{C2.} 
Agar \(X^{(n)} = \left( X_{1},...,X_{n} \right)\) tanlanma \({\lbrack\theta}_{1},\theta_{2}\rbrack\) oraliqda tekis taqsimotdan olingan bo'lsa, u holda noma'lum \(\left( \theta_{1},\theta_{2} \right)\) vektor parametr uchun momentlar usuli bahosini toping.
\\
\textbf{C3.} 
Agar \(X^{(n)} = \left( X_{1},...,X_{n} \right)\) tanlanma \(\theta\) parametrli geometrik taqsimotdan olingan bo'lsa, u holda noma'lum \(\theta\) parametrning haqiqatga maksimal o'xshashlik usuli bahosini toping.
\\

\end{tabular}
\vspace{1cm}


\begin{tabular}{m{17cm}}
\textbf{99-variant}
\newline

\textbf{T1.} Matematik statistikaning asosiy masalalari. (Statistik ma'lumotlar, guruhlash)
\\
\textbf{T2.} 
Chiziqli korrelyatsiya tenglamasi (ta'rifi, regressiya to'g'ri chiziqning tanlanma tenglamalari)
\\
\textbf{A1.} 
Hajmi \(n = 20\) ga teng bo'lgan tanlanma berilgan: 2,9; -3,2; 5,3; -4,3; 4,1; 5,3; -1,2; 2,9; -3,2; 4,1; -4,3; 5,3; -3,2; 2,9; -4,3; 4,1; -1,2; 5,3; 2,9; -3,2. Bu tanlanmaning statistik taqsimotin toping.
\\
\textbf{A2.} 
Hajmi \(n = 20\) ga teng bo'lgan tanlanma berilgan: 2,7; -5,6; 5,2; -8,1; 4,8; 5,2; -1,6; 2,7; -5,6; 4,8; -8,1; 5,2; -5,6; 2,7; -8,1; 4,8; -1,6; 5,2; 2,7; -5,6. Bu tanlanmaning empirik taqsimot funksiyasin toping.
\\
\textbf{A3.} 
Oliy matematika fanidan 10 ta talaba test topshiriqlarin topshirdi. Harbir talaba 10 balgacha to'plashi mumkin. Agar test topshiriqlari natijalari bo'yicha \{7, 9, 4, 9, 7, 5, 4, 7, 2, 6\} tanlanma olingan bo'lsa, ushbu tanlanmalarning tanlanma o'rta va tanlanma dispersiyalarin toping.
\\
\textbf{B1.} 
Agar normal taqsimlangan bosh to'plamdan olingan hajmi \(n = 11\) ga teng bo'lgan tanlanma bo'yicha \({\overline{S}}^{2} = 0,3\) tuzatilgan tanlanma dispersiya topilgan bo'lsa, u holda \(\gamma = 0,95\) ishonchlilik bilan noma'lum \(\theta_{2}^{2}\) dispersiya uchun ishonchlilik intervalin tuzing.
\\
\textbf{B2.} 
Agar (3,0,-2,0,-2,3,-2,0,0,3,0,0,0,0,3,-2,0,0,-2,3,0) tanlanma quyida berilgan taqsimotdan olingan bo'lsa, u holda noma'lum \(\left( \theta_{1},\theta_{2} \right)\) vektor parametr uchun momentlar usuli bahosini toping.
$\begin{array}{|c|c|c|c|}
    \hline
    \xi & - 2 & 0 & 3 \\
    \hline
    P_{\theta} & 2\theta_{1} & 0,5 + \theta_{1} + \theta_{2} & \theta_{2} \\
    \hline
\end{array}$
\\
\textbf{B3.} 
Agar \(X^{(n)} = \left( X_{1},...,X_{n} \right)\) tanlanma zichlik funksiyasi \(f(x;\theta) = \frac{2x}{\theta}e^{- \frac{x^{2}}{\theta}},\ \ x \geq 0\) bo'lgan taqsimotdan olingan bo'lsa, u holda noma'lum \(\theta > 0\) parametrning haqiqatga maksimal usuli bahosini toping.
\\
\textbf{C1.} 
Agar \(X^{(n)} = \left( X_{1},...,X_{n} \right)\) tanlanma \(M\xi = a\) ma'lum va \(M\xi^{2}\) chekli bo'lgan taqsimotdan olingan bo'lsa, u holda noma'lum \(D\xi\) dispersiya uchun \({\overline{S}}^{2}\) bahoning siljimaganligi va asosliligini tekshiring.
\\
\textbf{C2.} 
Agar \(X^{(n)} = \left( X_{1},...,X_{n} \right)\) tanlanma \((\theta,2\theta)\) parametrli normal taqsimotdan olingan bo'lsa, u holda noma'lum \(\theta > 0\) parametr uchun momentlar usuli bahosini toping.
\\
\textbf{C3.} 
Agar \(X^{(n)} = \left( X_{1},...,X_{n} \right)\) tanlanma zichlik funksiyasi \(f(x;\theta) = \frac{\theta}{\sqrt{2\pi x^{3}}}e^{- \theta^{2}/2x},\ \ x \geq 0\) bo'lgan taqsimotdan olingan bo'lsa, u holda noma'lum \(\theta > 0\) parametrning haqiqatga maksimal o'xshashlik bahosini toping.
\\

\end{tabular}
\vspace{1cm}


\begin{tabular}{m{17cm}}
\textbf{100-variant}
\newline

\textbf{T1.} 
Guruhlangan va interval variatsion qatorlar.
\\
\textbf{T2.} 
Normal qonun dispersiyasi uchun ishonchlilik intervalin tuzish. (Ishonchlilik ehtimolligi, interval)
\\
\textbf{A1.} 
Hajmi \(n = 20\) ga teng bo'lgan tanlanma berilgan: 14,7; 7,3; 16,6; 9,8; 11,2; 16,6; 6,7; 7,3; 11,2; 14,7; 6,7; 16,6; 7,3; 11,2; 14,7; 16,6; 6,7; 7,3; 11,2; 16,6. Bu tanlanmaning statistik taqsimotin toping.
\\
\textbf{A2.} 
Hajmi \(n = 20\) ga teng bo'lgan tanlanma berilgan: 14,4; 7,6; 16,7; 9,1; 11,8; 16,7; 6,4; 7,6; 11,8; 14,4; 6,4; 16,7; 7,6; 11,8; 14,4; 16,7; 6,4; 7,6; 11,8; 16,7. Bu tanlanmaning empirik taqsimot funksiyasin toping.
\\
\textbf{A3.} 
Oliy matematika fanidan 10 ta talaba test topshiriqlarin topshirdi. Harbir talaba 10 balgacha to'plashi mumkin. Agar test topshiriqlari natijalari bo'yicha \{10, 8, 4, 6, 2, 8, 5, 10, 2, 5\} tanlanma olingan bo'lsa, ushbu tanlanmalarning tanlanma o'rta va tanlanma dispersiyalarin toping.
\\
\textbf{B1.} 
Agar o'rta kvadratik chetlanish \(\sigma = 4\) bo'lgan normal taqsimot bosh to'plamdan olingan hajmi \(n = 49\)ga teng tanlanma bo'yicha \(\overline{x} = 9,4\) tanlanma o'rta qiymati topilgan bo'lsa, u holda \(\gamma = 0,9\) ishonchlilik bilan noma'lum \(\theta\) matematik kutilma uchun ishonchlilik intervalin tuzing .
\\
\textbf{B2.} 
Agar zichlik funksiyasi \(f(x) = \frac{2x}{\theta}e^{- \frac{x^{2}}{\theta}},\ \ x \geq 0\) ko'rinishga ega bo'lsa, u holda \(\theta\) parametr momentlar usuli bahosini toping.
\\
\textbf{B3.} 
Agar \(X^{(n)} = \left( X_{1},...,X_{n} \right)\) tanlanma \(\left( a,\theta^{2} \right)\) parametrli normal taqsimotdan olingan bo'lsa (\(\alpha -\) ma'lum), u holda noma'lum \(\theta^{2}\) parametrning haqiqatga maksimal o'xshashlik bahosini toping.
\\
\textbf{C1.} 
Agar \(X^{(n)} = \left( X_{1},...,X_{n} \right)\) tanlanma \(M\xi = a\) ma'lum va \(M\xi^{2}\) chekli bo'lgan taqsimotdan olingan bo'lsa, u holda noma'lum \(D\xi\) dispersiya uchun \(\frac{1}{n}\sum_{i = 1}^{n}{X_{i}a}\) bahoning siljimaganligi va asosliligini tekshiring.
\\
\textbf{C2.} 
Agar \(X^{(n)} = \left( X_{1},...,X_{n} \right)\) tanlanma \((\theta,\theta^{2})\) parametrli normal taqsimotdan \(\ \ g(x) = (x)^{2}\ \ \) olingan bo'lsa, u holda noma'lum \(\theta > 0\) parametr uchun momentlar usuli bahosini funksiya yordamida toping.
\\
\textbf{C3.} 
Agar \(X^{(n)} = \left( X_{1},...,X_{n} \right)\) tanlanma zichlik funksiyasi \(f(x;\theta) = \left\{ \begin{array}{r}
3x^{2}\theta^{- 3}{e^{- \left( \frac{x}{\theta} \right)}}^{3},\ \ \ \ x \geq 0 \\
0,\ \ \ \ \ \ \ \ \ \ \ x < 0
\end{array} \right.\ \) bo'lgan taqsimotdan olingan bo'lsa, u holda noma'lum \(\theta > 0\) parametrning haqiqatga maksimal o'xshashlik bahosini toping.
\\

\end{tabular}
\vspace{1cm}



\end{document}
