1. Алгебра высказываний и понятие формула алгебры высказываний \emph{О.с. Равносильные формулы алгебры логики и их свойтсва}

2. Понятие выводимости формулы из совокупности формулы \emph{О.с. Теорема дедукции. Обобщенная теорема дедукции.}

3. Нормальные формы \emph{О.с. Дизъюнктивная, конъюнктивная и совершенная дизъюнктивная, совершенна конъюнктивная нормальная форма}

4. Исчисления высказываний и понятие формулы исчиления высказываний \emph{О.с. Определения доказуемой формулы. Аксиомы исчисления высказываний. Правыла выводимости}

5. Функции алгебры логики \emph{О.с. Алгебра Буля. Полная система функции}

6. Производные правила вывода \emph{О.с. Правило силлогизма, правило контрпозиции,правило снятия двойного отрицания}

7. Некоторые приложения алгебры логики \emph{О.с. Реле контактные схемы, применения алгебры логики в решении логических задач}

8. Логика предикатов. \emph{О.с. Понятие предиката, понятие формулы логики предикатов, равносильные формулы логики предикатов}

9. Алгебра высказываний и понятие формула алгебры высказываний \emph{О.с. Равносильные формулы алгебры логики и их свойтсва}

10. Производные правила вывода \emph{О.с. Правило силлогизма, правило контрпозиции,правило снятия двойного отрицания}

++++

1. Некоторые приложения алгебры логики \emph{О.с. Реле контактные схемы, применения алгебры логики в решении логических задач}

2. Исчисления высказываний и понятие формулы исчиления высказываний \emph{О.с. Определения доказуемой формулы. Аксиомы исчисления высказываний. Правыла выводимости}

3. Функции алгебры логики \emph{О.с. Алгебра Буля. Полная система функции}

4. Понятие выводимости формулы из совокупности формулы \emph{О.с. Теорема дедукции. Обобщенная теорема дедукции.}

5. Нормальные формы \emph{О.с. Дизъюнктивная, конъюнктивная и совершенная дизъюнктивная, совершенна конъюнктивная нормальная форма}

6. Логика предикатов. \emph{О.с. Понятие предиката, понятие формулы логики предикатов, равносильные формулы логики предикатов}

7. Алгебра высказываний и понятие формула алгебры высказываний \emph{О.с. Равносильные формулы алгебры логики и их свойтсва}

8. Понятие выводимости формулы из совокупности формулы \emph{О.с. Теорема дедукции. Обобщенная теорема дедукции.}

9. Нормальные формы \emph{О.с. Дизъюнктивная, конъюнктивная и совершенная дизъюнктивная, совершенна конъюнктивная нормальная форма}

10. Исчисления высказываний и понятие формулы исчиления высказываний \emph{О.с. Определения доказуемой формулы. Аксиомы исчисления высказываний. Правыла выводимости}

++++

1. Доказать тоджество. \(A\bigcap(\overline{A}) = \varnothing\)

2. Известно, что высказывание \(A \rightarrow B\) ложно. Что можно сказать обистинности \(A\) и \(B\) ?

3. Доказать тоджество. \(A\backslash(B\bigcup C) = (A\backslash B)\bigcup(A\backslash C)\)

4. Доказать тоджество. \(A\bigcap(B\bigcup C) = (A\bigcap B)\bigcup(B\bigcap C)\)

5. Доказать тоджество. \(A\bigcup(B\bigcap C) = (A\bigcup B)\bigcap(A\bigcup C)\)

6. Доказать тождество. \(\overline{A\bigcap B} = \overline{A}\bigcup\overline{B}\)

7. Доказать тоджество. \((A\bigcup B)\bigcap A = (A\bigcap B)\bigcup A = A\)

8. Доказать тоджество. \(A\bigcap(B\backslash C) = (A\bigcap B)\backslash(A\bigcap C) = (A\bigcap B)\backslash C\)

++++

1. Найти все подформулы. \(((a_{0} \rightarrow a_{1}) \land ((a_{1} \rightarrow a_{2}) \rightarrow (\overline{a_{0}} \vee a_{2})).\)

2. Найти все подформулы. \(\left( (р \land q) \vee q \right) \vee (q \rightarrow p)\)

3. Найти все подформулы. \(х \land у \rightarrow (z \vee у \rightarrow z)\)

4. Найти все подформулы. \((x \rightarrow y) \land (x \rightarrow \overline{y}) \rightarrow \overline{x}\)

5. Найти все подформулы. \((х \vee у) \rightarrow \left( х \land \overline{у \vee х \rightarrow у} \right)\)

6. Найти все подформулы. \(\overline{(a \rightarrow c)} \rightarrow \left( (b \rightarrow c) \rightarrow (a \vee b \rightarrow c) \right)\)

7. Найти все подформулы. \((x \rightarrow z) \rightarrow (y \rightarrow z) \rightarrow ((x \vee y) \rightarrow z)\)

8. Найти все подформулы. \((х \rightarrow у) \rightarrow (\overline{х} \rightarrow \overline{у)}\).

9. Найти все подформулы. \((x \rightarrow y) \land (x \rightarrow \overline{y}) \rightarrow \overline{x} \vee y\)

10. Найти все подформулы. \((x \rightarrow y) \land (x \rightarrow \overline{y}) \rightarrow \overline{x} \leftrightarrow \overline{y}\)

++++

1. Составить таблица истинности. \(\left( (р \land q) \vee q \right) \vee (q \rightarrow p)\)

2. Составить таблица истинности. \((x \rightarrow y) \land (x \rightarrow \overline{y}) \rightarrow \overline{x}\)

3. Составить таблица истинности. \((x \rightarrow z) \rightarrow (y \rightarrow z) \rightarrow ((x \vee y) \rightarrow z)\)

4. Составить таблица истинности. \((x \rightarrow y) \land (x \rightarrow \overline{y}) \rightarrow \overline{x} \vee y\)

5. Составить таблица истинности. \((x \rightarrow \overline{y}) \rightarrow \overline{x}\)

6. Составить таблица истинности. \((x \rightarrow y) \land (x \rightarrow \overline{y}) \rightarrow \overline{x} \leftrightarrow \overline{y}\)

7. Составить таблица истинности. \((x \rightarrow z) \rightarrow (y \rightarrow z) \rightarrow ((x \vee y) \rightarrow z)\)

8. Составить таблица истинности. \((x \rightarrow z) \rightarrow (y \rightarrow z)\)

9. Составить таблица истинности. \((x \rightarrow y) \land (x \rightarrow \overline{y}) \rightarrow \overline{x}\)

++++

1. Для следующих формул найти ДНФ и КНФ. \(х \land у \rightarrow (z \vee у \rightarrow z)\)

2. Для следующих формул найти ДНФ и КНФ. \(\ x_{1} \land (x_{2} \vee (x_{1} \rightarrow x_{3}))\)

3. Для следующих формул найти СКНФ \(\overline{(a \rightarrow c)} \rightarrow \left( (b \rightarrow c) \rightarrow (a \vee b \rightarrow c) \right)\)

4. Для следующих формул найти КНФ. \((х \rightarrow у) \rightarrow (\overline{х} \rightarrow \overline{у)}\).

5. Для следующих формул найти СДНФ \(\overline{\overline{x \vee y} \rightarrow \overline{x \land y}}\)

6. Для следующих формул найти ДНФ и КНФ. \(\ x_{1} \land (x_{2} \vee (x_{1} \rightarrow x_{3}))\)

7. Для следующих формул найти ДНФ и КНФ. \(\left( х_{1} \land \overline{х_{2}} \right) \vee х_{3}\)

8. Для следующих формул найти ДНФ и КНФ. \(х \land у \rightarrow (z \vee у \rightarrow z)\)

9. Для следующих формул найти СКНФ \((z \rightarrow x \land y))\)

10. Для следующих формул найти ДНФ и КНФ. \(\overline{x \vee (x_{2} \rightarrow x_{1})}\)

++++

1. Доказать тождественно ложность формул. \(F\left( р_{1},р_{2} \right) = \overline{p_{1} \rightarrow (p_{2} \rightarrow p_{1})}\)

2. Упростить формулу алгебра логики. \(А(х,у,z) = (x \rightarrow y) \land z \rightarrow (x \rightarrow z)\)

3. Упростить формулу алгебра логики. \(A(x,\ \ y) = (x \leftrightarrow y) \land (x \vee y)\)

4. Упростить формулу алгебра логики. \((х \rightarrow у) \rightarrow (\overline{х} \rightarrow \overline{у)}\)

5. Упростить формулу алгебра логики. \((x \rightarrow y) \land (x \rightarrow \overline{y}) \rightarrow \overline{x}\)

6. Упростить формул. \(A(a,\ \ b,\ \ c) = (a \rightarrow b) \rightarrow (\overline{a}\  \land (b \vee c))\)

7. Упростить формулу алгебра логики. \(А(х,у,z) = (x \rightarrow y) \land z \rightarrow (x \rightarrow z)\)

8. Упростить формулу алгебра логики. \(А(х,у,z) = (x \rightarrow y) \land z \rightarrow (x \rightarrow z)\)

9. Упростить формул. \(A(a,\ \ b,\ \ c) = (a \rightarrow b) \rightarrow (\overline{a}\  \land (b \vee c))\)

++++

1. Найти \(x\), если \(\left( \overline{x \vee a} \right) \vee \left( \overline{x \vee \overline{a}} \right) \equiv b\)

2. Доказать равносильность.\(x_{1} \land x_{2} \land \ \ \ ...\ \  \land x_{n} \rightarrow y \equiv x_{1} \rightarrow (x_{2} \rightarrow (\ \ ...\ \  \rightarrow (x_{n} \rightarrow y)\ \ ,,.\ \ ))\)

3. Доказать, что двойственные формулы.\(f(х,\ \ y) = x \land y\) и \(f^{*}(х,\ \ y) = x \vee y.\)

4. Доказать равносильность. \((x \vee y) \land (x \vee \overline{y}) \equiv x\)

5. Найти двойственную формулу к формуле \(f(х,\ \ y,\ \ z) = xy \vee yz \vee xz\)

6. Доказать, что двойственные формулы. \(f(х,\ \ y) = x \land y\) и \(f^{*}(х) = x \vee y.\)

7. Доказать равносильность. \(x \vee \left( \overline{x\ } \land \overline{y} \right) \equiv x \vee \overline{y}\)

8. Доказать равносильность. \(x \rightarrow \left( y \rightarrow \overline{z} \right) \equiv x \land y \rightarrow \overline{z}\)

9. Доказать, что двойственные формулы. \(f(х,\ \ y) = x \leftrightarrow y\) и \(f^{*}(х) = \overline{x \leftrightarrow y}.\)

++++

1. Составить РКС для формулы \(F(a,\ \ b,\ \ c) = (a \vee b) \rightarrow (a \land b) \vee c.\)

2. Составить РКС для формулы \(A(a,\ \ b,\ \ c) = (a \rightarrow b) \rightarrow (\overline{a}\  \land (b \vee c))\)

3. Составить РКС для формулы \((x \rightarrow z) \rightarrow (y \rightarrow z) \rightarrow ((x \vee y) \rightarrow z)\)

4. Составить РКС для формулы \((х \rightarrow у) \rightarrow (\overline{х} \rightarrow \overline{у)}\)

5. Составить РКС для формулы \((x \land y \rightarrow z) \rightarrow (x \rightarrow (y \rightarrow z))\)

6. Составить РКС для формулы \((x \rightarrow y) \land (x \rightarrow \overline{y}) \rightarrow \overline{x} \leftrightarrow \overline{y}\)

7. Составить РКС для формулы \(А(х,у,z) = (x \rightarrow y) \land z \rightarrow (x \rightarrow z)\)

8. Составить РКС для формулы \(A(a,\ \ b,\ \ c) = (a \rightarrow b) \rightarrow (\overline{a}\  \land (b \vee c))\)

9. Составить РКС для формулы \(x \rightarrow (x \rightarrow y) \rightarrow (\overline{x} \rightarrow y)\ \)

10. Составить РКС для формулы \(\overline{(a \rightarrow c)} \rightarrow \left( (b \rightarrow c) \rightarrow (a \vee b \rightarrow c) \right)\)

++++

1. Булевы функции заданы последовательностью их значений при лексикографическом упорядочении аргументов. Найти все базисы, которые можно составить из следующих функций.\(f = (00111100)\)

2. Булевы функции заданы последовательностью их значений при лексикографическом упорядочении аргументов. Найти все базисы, которые можно составить из следующих функций. \(f = (11101000)\)

3. Булевы функции заданы последовательностью их значений при лексикографическом упорядочении аргументов. Найти все базисы, которые можно составить из следующих функций. \(f = (01010111)\)

4. Булевы функции заданы последовательностью их значений при лексикографическом упорядочении аргументов. Найти все базисы, которые можно составить из следующих функций. \(f = (10100101)\)

5. Булевы функции заданы последовательностью их значений при лексикографическом упорядочении аргументов. Найти все базисы, которые можно составить из следующих функций. \(f = (01110001)\)

6. Булевы функции заданы последовательностью их значений при лексикографическом упорядочении аргументов. Найти все базисы, которые можно составить из следующих функций. \(f = (00010101)\)

7. Булевы функции заданы последовательностью их значений при лексикографическом упорядочении аргументов. Найти все базисы, которые можно составить из следующих функций. \(f = (01101001)\)

8. Булевы функции заданы последовательностью их значений при лексикографическом упорядочении аргументов. Найти все базисы, которые можно составить из следующих функций. \(f = (10011011)\)

9. Булевы функции заданы последовательностью их значений при лексикографическом упорядочении аргументов. Найти все базисы, которые можно составить из следующих функций. \(f = (10010110)\)

10 Булевы функции заданы последовательностью их значений при лексикографическом упорядочении аргументов. Найти все базисы, которые можно составить из следующих функций. \(f = (11001000)\)

++++

1. Пусть \(A,\ \ B\)- некоторые формулы исчисления предикатов, причем переменная\(x\) не входит в \(B\). Доказать следующие соотношения. \(\forall xAЂ\);

2. Пусть \(A,\ \ B\)- некоторые формулы исчисления предикатов, причем переменная\(x\) не входит в \(B\). Доказать следующие соотношения. \(\forall xA \vee B \rightleftarrows \forall x(A \vee B)\);

3. Пусть \(A,\ \ B\)- некоторые формулы исчисления предикатов, причем переменная\(x\) не входит в \(B\). Доказать следующие соотношения. \(\exists xA \vee B \rightleftarrows \exists x(A \vee B)\)

4. Привести к предваренной нормальной форме \(\forall xB(x) \supset \exists y(A(y) \supset B(x))\)

5. Привести к предваренной нормальной форме \(\forall x(\neg A(x) \supset \exists yB(y)) \supset B(y) \vee A(x)\)

6. Привести к предваренной нормальной форме \(\forall x(A(x) \supset B(y))Ђ\)

7. Пусть \(A,\ \ B\)- некоторые формулы исчисления предикатов, причем переменная\(x\) не входит в \(B\). Доказать следующие соотношения. \(\exists xA \vee B \rightleftarrows \exists x(A \vee B)\);

8. Пусть \(A,\ \ B\)- некоторые формулы исчисления предикатов, причем переменная\(x\) не входит в \(B\). Доказать следующие соотношения. \(\forall xB \rightleftarrows B\);

9. Пусть \(A,\ \ B\)- некоторые формулы исчисления предикатов, причем переменная\(x\) не входит в \(B\). Доказать следующие соотношения. \(\exists xB \rightleftarrows B\)
