\documentclass{article}
\usepackage[fontsize=14pt]{fontsize}
\usepackage[utf8]{inputenc}
\usepackage[T2A]{fontenc}
% \usepackage{unicode-math}

\usepackage{array}
\usepackage[a4paper,
left=7mm,
right=5mm,
top=7mm,]{geometry}
\usepackage{amsmath}
\usepackage{amssymb}
\usepackage{amsfonts}
\usepackage{setspace}



\renewcommand{\baselinestretch}{1} 

\everymath{\displaystyle}
\everydisplay{\displaystyle}
% \linespread{1.25}

\DeclareMathOperator{\sign}{sign}


\begin{document}

\pagenumbering{gobble}


\begin{tabular}{m{17cm}}
\textbf{1-вариант}
\newline

T1. 5. Функции алгебры логики \emph{О.с. Алгебра Буля. Полная система функции} \\
T2. 4. Понятие выводимости формулы из совокупности формулы \emph{О.с. Теорема дедукции. Обобщенная теорема дедукции.} \\
A1. 7. Доказать тоджество. \((A\bigcup B)\bigcap A = (A\bigcap B)\bigcup A = A\) \\
A2. 1. Найти все подформулы. \(((a_{0} \rightarrow a_{1}) \land ((a_{1} \rightarrow a_{2}) \rightarrow (\overline{a_{0}} \vee a_{2})).\) \\
A3. 8. Составить таблица истинности. \((x \rightarrow z) \rightarrow (y \rightarrow z)\) \\
B1. 9. Для следующих формул найти СКНФ \((z \rightarrow x \land y))\) \\
B2. 2. Упростить формулу алгебра логики. \(А(х,у,z) = (x \rightarrow y) \land z \rightarrow (x \rightarrow z)\) \\
B3. 4. Доказать равносильность. \((x \vee y) \land (x \vee \overline{y}) \equiv x\) \\
C1. 2. Составить РКС для формулы \(A(a,\ \ b,\ \ c) = (a \rightarrow b) \rightarrow (\overline{a}\  \land (b \vee c))\) \\
C2. 7. Булевы функции заданы последовательностью их значений при лексикографическом упорядочении аргументов. Найти все базисы, которые можно составить из следующих функций. \(f = (01101001)\) \\
C3. 8. Пусть \(A,\ \ B\)- некоторые формулы исчисления предикатов, причем переменная\(x\) не входит в \(B\). Доказать следующие соотношения. \(\forall xB \rightleftarrows B\); \\

\end{tabular}
\vspace{1cm}


\begin{tabular}{m{17cm}}
\textbf{2-вариант}
\newline

T1. 9. Алгебра высказываний и понятие формула алгебры высказываний \emph{О.с. Равносильные формулы алгебры логики и их свойтсва} \\
T2. 6. Логика предикатов. \emph{О.с. Понятие предиката, понятие формулы логики предикатов, равносильные формулы логики предикатов} \\
A1. 3. Доказать тоджество. \(A\backslash(B\bigcup C) = (A\backslash B)\bigcup(A\backslash C)\) \\
A2. 4. Найти все подформулы. \((x \rightarrow y) \land (x \rightarrow \overline{y}) \rightarrow \overline{x}\) \\
A3. 5. Составить таблица истинности. \((x \rightarrow \overline{y}) \rightarrow \overline{x}\) \\
B1. 8. Для следующих формул найти ДНФ и КНФ. \(х \land у \rightarrow (z \vee у \rightarrow z)\) \\
B2. 5. Упростить формулу алгебра логики. \((x \rightarrow y) \land (x \rightarrow \overline{y}) \rightarrow \overline{x}\) \\
B3. 5. Найти двойственную формулу к формуле \(f(х,\ \ y,\ \ z) = xy \vee yz \vee xz\) \\
C1. 1. Составить РКС для формулы \(F(a,\ \ b,\ \ c) = (a \vee b) \rightarrow (a \land b) \vee c.\) \\
C2. 6. Булевы функции заданы последовательностью их значений при лексикографическом упорядочении аргументов. Найти все базисы, которые можно составить из следующих функций. \(f = (00010101)\) \\
C3. 9. Пусть \(A,\ \ B\)- некоторые формулы исчисления предикатов, причем переменная\(x\) не входит в \(B\). Доказать следующие соотношения. \(\exists xB \rightleftarrows B\)
 \\

\end{tabular}
\vspace{1cm}


\begin{tabular}{m{17cm}}
\textbf{3-вариант}
\newline

T1. 8. Логика предикатов. \emph{О.с. Понятие предиката, понятие формулы логики предикатов, равносильные формулы логики предикатов} \\
T2. 9. Нормальные формы \emph{О.с. Дизъюнктивная, конъюнктивная и совершенная дизъюнктивная, совершенна конъюнктивная нормальная форма} \\
A1. 8. Доказать тоджество. \(A\bigcap(B\backslash C) = (A\bigcap B)\backslash(A\bigcap C) = (A\bigcap B)\backslash C\) \\
A2. 2. Найти все подформулы. \(\left( (р \land q) \vee q \right) \vee (q \rightarrow p)\) \\
A3. 6. Составить таблица истинности. \((x \rightarrow y) \land (x \rightarrow \overline{y}) \rightarrow \overline{x} \leftrightarrow \overline{y}\) \\
B1. 10. Для следующих формул найти ДНФ и КНФ. \(\overline{x \vee (x_{2} \rightarrow x_{1})}\) \\
B2. 6. Упростить формул. \(A(a,\ \ b,\ \ c) = (a \rightarrow b) \rightarrow (\overline{a}\  \land (b \vee c))\) \\
B3. 9. Доказать, что двойственные формулы. \(f(х,\ \ y) = x \leftrightarrow y\) и \(f^{*}(х) = \overline{x \leftrightarrow y}.\) \\
C1. 7. Составить РКС для формулы \(А(х,у,z) = (x \rightarrow y) \land z \rightarrow (x \rightarrow z)\) \\
C2. 1. Булевы функции заданы последовательностью их значений при лексикографическом упорядочении аргументов. Найти все базисы, которые можно составить из следующих функций.\(f = (00111100)\) \\
C3. 5. Привести к предваренной нормальной форме \(\forall x(\neg A(x) \supset \exists yB(y)) \supset B(y) \vee A(x)\) \\

\end{tabular}
\vspace{1cm}


\begin{tabular}{m{17cm}}
\textbf{4-вариант}
\newline

T1. 10. Производные правила вывода \emph{О.с. Правило силлогизма, правило контрпозиции,правило снятия двойного отрицания} \\
T2. 2. Исчисления высказываний и понятие формулы исчиления высказываний \emph{О.с. Определения доказуемой формулы. Аксиомы исчисления высказываний. Правыла выводимости} \\
A1. 5. Доказать тоджество. \(A\bigcup(B\bigcap C) = (A\bigcup B)\bigcap(A\bigcup C)\) \\
A2. 6. Найти все подформулы. \(\overline{(a \rightarrow c)} \rightarrow \left( (b \rightarrow c) \rightarrow (a \vee b \rightarrow c) \right)\) \\
A3. 1. Составить таблица истинности. \(\left( (р \land q) \vee q \right) \vee (q \rightarrow p)\) \\
B1. 7. Для следующих формул найти ДНФ и КНФ. \(\left( х_{1} \land \overline{х_{2}} \right) \vee х_{3}\) \\
B2. 9. Упростить формул. \(A(a,\ \ b,\ \ c) = (a \rightarrow b) \rightarrow (\overline{a}\  \land (b \vee c))\) \\
B3. 3. Доказать, что двойственные формулы.\(f(х,\ \ y) = x \land y\) и \(f^{*}(х,\ \ y) = x \vee y.\) \\
C1. 5. Составить РКС для формулы \((x \land y \rightarrow z) \rightarrow (x \rightarrow (y \rightarrow z))\) \\
C2. 4. Булевы функции заданы последовательностью их значений при лексикографическом упорядочении аргументов. Найти все базисы, которые можно составить из следующих функций. \(f = (10100101)\) \\
C3. 7. Пусть \(A,\ \ B\)- некоторые формулы исчисления предикатов, причем переменная\(x\) не входит в \(B\). Доказать следующие соотношения. \(\exists xA \vee B \rightleftarrows \exists x(A \vee B)\); \\

\end{tabular}
\vspace{1cm}


\begin{tabular}{m{17cm}}
\textbf{5-вариант}
\newline

T1. 3. Нормальные формы \emph{О.с. Дизъюнктивная, конъюнктивная и совершенная дизъюнктивная, совершенна конъюнктивная нормальная форма} \\
T2. 7. Алгебра высказываний и понятие формула алгебры высказываний \emph{О.с. Равносильные формулы алгебры логики и их свойтсва} \\
A1. 4. Доказать тоджество. \(A\bigcap(B\bigcup C) = (A\bigcap B)\bigcup(B\bigcap C)\) \\
A2. 9. Найти все подформулы. \((x \rightarrow y) \land (x \rightarrow \overline{y}) \rightarrow \overline{x} \vee y\) \\
A3. 7. Составить таблица истинности. \((x \rightarrow z) \rightarrow (y \rightarrow z) \rightarrow ((x \vee y) \rightarrow z)\) \\
B1. 4. Для следующих формул найти КНФ. \((х \rightarrow у) \rightarrow (\overline{х} \rightarrow \overline{у)}\). \\
B2. 3. Упростить формулу алгебра логики. \(A(x,\ \ y) = (x \leftrightarrow y) \land (x \vee y)\) \\
B3. 7. Доказать равносильность. \(x \vee \left( \overline{x\ } \land \overline{y} \right) \equiv x \vee \overline{y}\) \\
C1. 3. Составить РКС для формулы \((x \rightarrow z) \rightarrow (y \rightarrow z) \rightarrow ((x \vee y) \rightarrow z)\) \\
C2. 10 Булевы функции заданы последовательностью их значений при лексикографическом упорядочении аргументов. Найти все базисы, которые можно составить из следующих функций. \(f = (11001000)\) \\
C3. 4. Привести к предваренной нормальной форме \(\forall xB(x) \supset \exists y(A(y) \supset B(x))\) \\

\end{tabular}
\vspace{1cm}


\begin{tabular}{m{17cm}}
\textbf{6-вариант}
\newline

T1. 6. Производные правила вывода \emph{О.с. Правило силлогизма, правило контрпозиции,правило снятия двойного отрицания} \\
T2. 3. Функции алгебры логики \emph{О.с. Алгебра Буля. Полная система функции} \\
A1. 6. Доказать тождество. \(\overline{A\bigcap B} = \overline{A}\bigcup\overline{B}\) \\
A2. 8. Найти все подформулы. \((х \rightarrow у) \rightarrow (\overline{х} \rightarrow \overline{у)}\). \\
A3. 4. Составить таблица истинности. \((x \rightarrow y) \land (x \rightarrow \overline{y}) \rightarrow \overline{x} \vee y\) \\
B1. 3. Для следующих формул найти СКНФ \(\overline{(a \rightarrow c)} \rightarrow \left( (b \rightarrow c) \rightarrow (a \vee b \rightarrow c) \right)\) \\
B2. 8. Упростить формулу алгебра логики. \(А(х,у,z) = (x \rightarrow y) \land z \rightarrow (x \rightarrow z)\) \\
B3. 8. Доказать равносильность. \(x \rightarrow \left( y \rightarrow \overline{z} \right) \equiv x \land y \rightarrow \overline{z}\) \\
C1. 9. Составить РКС для формулы \(x \rightarrow (x \rightarrow y) \rightarrow (\overline{x} \rightarrow y)\ \) \\
C2. 9. Булевы функции заданы последовательностью их значений при лексикографическом упорядочении аргументов. Найти все базисы, которые можно составить из следующих функций. \(f = (10010110)\) \\
C3. 2. Пусть \(A,\ \ B\)- некоторые формулы исчисления предикатов, причем переменная\(x\) не входит в \(B\). Доказать следующие соотношения. \(\forall xA \vee B \rightleftarrows \forall x(A \vee B)\); \\

\end{tabular}
\vspace{1cm}


\begin{tabular}{m{17cm}}
\textbf{7-вариант}
\newline

T1. 2. Понятие выводимости формулы из совокупности формулы \emph{О.с. Теорема дедукции. Обобщенная теорема дедукции.} \\
T2. 5. Нормальные формы \emph{О.с. Дизъюнктивная, конъюнктивная и совершенная дизъюнктивная, совершенна конъюнктивная нормальная форма} \\
A1. 1. Доказать тоджество. \(A\bigcap(\overline{A}) = \varnothing\) \\
A2. 10. Найти все подформулы. \((x \rightarrow y) \land (x \rightarrow \overline{y}) \rightarrow \overline{x} \leftrightarrow \overline{y}\) \\
A3. 3. Составить таблица истинности. \((x \rightarrow z) \rightarrow (y \rightarrow z) \rightarrow ((x \vee y) \rightarrow z)\) \\
B1. 2. Для следующих формул найти ДНФ и КНФ. \(\ x_{1} \land (x_{2} \vee (x_{1} \rightarrow x_{3}))\) \\
B2. 1. Доказать тождественно ложность формул. \(F\left( р_{1},р_{2} \right) = \overline{p_{1} \rightarrow (p_{2} \rightarrow p_{1})}\) \\
B3. 1. Найти \(x\), если \(\left( \overline{x \vee a} \right) \vee \left( \overline{x \vee \overline{a}} \right) \equiv b\) \\
C1. 8. Составить РКС для формулы \(A(a,\ \ b,\ \ c) = (a \rightarrow b) \rightarrow (\overline{a}\  \land (b \vee c))\) \\
C2. 3. Булевы функции заданы последовательностью их значений при лексикографическом упорядочении аргументов. Найти все базисы, которые можно составить из следующих функций. \(f = (01010111)\) \\
C3. 1. Пусть \(A,\ \ B\)- некоторые формулы исчисления предикатов, причем переменная\(x\) не входит в \(B\). Доказать следующие соотношения. \(\forall xAЂ\); \\

\end{tabular}
\vspace{1cm}


\begin{tabular}{m{17cm}}
\textbf{8-вариант}
\newline

T1. 1. Алгебра высказываний и понятие формула алгебры высказываний \emph{О.с. Равносильные формулы алгебры логики и их свойтсва} \\
T2. 8. Понятие выводимости формулы из совокупности формулы \emph{О.с. Теорема дедукции. Обобщенная теорема дедукции.} \\
A1. 2. Известно, что высказывание \(A \rightarrow B\) ложно. Что можно сказать обистинности \(A\) и \(B\) ? \\
A2. 5. Найти все подформулы. \((х \vee у) \rightarrow \left( х \land \overline{у \vee х \rightarrow у} \right)\) \\
A3. 2. Составить таблица истинности. \((x \rightarrow y) \land (x \rightarrow \overline{y}) \rightarrow \overline{x}\) \\
B1. 6. Для следующих формул найти ДНФ и КНФ. \(\ x_{1} \land (x_{2} \vee (x_{1} \rightarrow x_{3}))\) \\
B2. 4. Упростить формулу алгебра логики. \((х \rightarrow у) \rightarrow (\overline{х} \rightarrow \overline{у)}\) \\
B3. 6. Доказать, что двойственные формулы. \(f(х,\ \ y) = x \land y\) и \(f^{*}(х) = x \vee y.\) \\
C1. 6. Составить РКС для формулы \((x \rightarrow y) \land (x \rightarrow \overline{y}) \rightarrow \overline{x} \leftrightarrow \overline{y}\) \\
C2. 2. Булевы функции заданы последовательностью их значений при лексикографическом упорядочении аргументов. Найти все базисы, которые можно составить из следующих функций. \(f = (11101000)\) \\
C3. 6. Привести к предваренной нормальной форме \(\forall x(A(x) \supset B(y))Ђ\) \\

\end{tabular}
\vspace{1cm}


\begin{tabular}{m{17cm}}
\textbf{9-вариант}
\newline

T1. 7. Некоторые приложения алгебры логики \emph{О.с. Реле контактные схемы, применения алгебры логики в решении логических задач} \\
T2. 10. Исчисления высказываний и понятие формулы исчиления высказываний \emph{О.с. Определения доказуемой формулы. Аксиомы исчисления высказываний. Правыла выводимости} \\
A1. 2. Известно, что высказывание \(A \rightarrow B\) ложно. Что можно сказать обистинности \(A\) и \(B\) ? \\
A2. 3. Найти все подформулы. \(х \land у \rightarrow (z \vee у \rightarrow z)\) \\
A3. 9. Составить таблица истинности. \((x \rightarrow y) \land (x \rightarrow \overline{y}) \rightarrow \overline{x}\) \\
B1. 1. Для следующих формул найти ДНФ и КНФ. \(х \land у \rightarrow (z \vee у \rightarrow z)\) \\
B2. 7. Упростить формулу алгебра логики. \(А(х,у,z) = (x \rightarrow y) \land z \rightarrow (x \rightarrow z)\) \\
B3. 2. Доказать равносильность.\(x_{1} \land x_{2} \land \ \ \ ...\ \  \land x_{n} \rightarrow y \equiv x_{1} \rightarrow (x_{2} \rightarrow (\ \ ...\ \  \rightarrow (x_{n} \rightarrow y)\ \ ,,.\ \ ))\) \\
C1. 4. Составить РКС для формулы \((х \rightarrow у) \rightarrow (\overline{х} \rightarrow \overline{у)}\) \\
C2. 5. Булевы функции заданы последовательностью их значений при лексикографическом упорядочении аргументов. Найти все базисы, которые можно составить из следующих функций. \(f = (01110001)\) \\
C3. 3. Пусть \(A,\ \ B\)- некоторые формулы исчисления предикатов, причем переменная\(x\) не входит в \(B\). Доказать следующие соотношения. \(\exists xA \vee B \rightleftarrows \exists x(A \vee B)\) \\

\end{tabular}
\vspace{1cm}


\begin{tabular}{m{17cm}}
\textbf{10-вариант}
\newline

T1. 4. Исчисления высказываний и понятие формулы исчиления высказываний \emph{О.с. Определения доказуемой формулы. Аксиомы исчисления высказываний. Правыла выводимости} \\
T2. 1. Некоторые приложения алгебры логики \emph{О.с. Реле контактные схемы, применения алгебры логики в решении логических задач} \\
A1. 7. Доказать тоджество. \((A\bigcup B)\bigcap A = (A\bigcap B)\bigcup A = A\) \\
A2. 7. Найти все подформулы. \((x \rightarrow z) \rightarrow (y \rightarrow z) \rightarrow ((x \vee y) \rightarrow z)\) \\
A3. 1. Составить таблица истинности. \(\left( (р \land q) \vee q \right) \vee (q \rightarrow p)\) \\
B1. 5. Для следующих формул найти СДНФ \(\overline{\overline{x \vee y} \rightarrow \overline{x \land y}}\) \\
B2. 4. Упростить формулу алгебра логики. \((х \rightarrow у) \rightarrow (\overline{х} \rightarrow \overline{у)}\) \\
B3. 5. Найти двойственную формулу к формуле \(f(х,\ \ y,\ \ z) = xy \vee yz \vee xz\) \\
C1. 10. Составить РКС для формулы \(\overline{(a \rightarrow c)} \rightarrow \left( (b \rightarrow c) \rightarrow (a \vee b \rightarrow c) \right)\) \\
C2. 8. Булевы функции заданы последовательностью их значений при лексикографическом упорядочении аргументов. Найти все базисы, которые можно составить из следующих функций. \(f = (10011011)\) \\
C3. 7. Пусть \(A,\ \ B\)- некоторые формулы исчисления предикатов, причем переменная\(x\) не входит в \(B\). Доказать следующие соотношения. \(\exists xA \vee B \rightleftarrows \exists x(A \vee B)\); \\

\end{tabular}
\vspace{1cm}


\begin{tabular}{m{17cm}}
\textbf{11-вариант}
\newline

T1. 1. Алгебра высказываний и понятие формула алгебры высказываний \emph{О.с. Равносильные формулы алгебры логики и их свойтсва} \\
T2. 1. Некоторые приложения алгебры логики \emph{О.с. Реле контактные схемы, применения алгебры логики в решении логических задач} \\
A1. 1. Доказать тоджество. \(A\bigcap(\overline{A}) = \varnothing\) \\
A2. 7. Найти все подформулы. \((x \rightarrow z) \rightarrow (y \rightarrow z) \rightarrow ((x \vee y) \rightarrow z)\) \\
A3. 5. Составить таблица истинности. \((x \rightarrow \overline{y}) \rightarrow \overline{x}\) \\
B1. 5. Для следующих формул найти СДНФ \(\overline{\overline{x \vee y} \rightarrow \overline{x \land y}}\) \\
B2. 2. Упростить формулу алгебра логики. \(А(х,у,z) = (x \rightarrow y) \land z \rightarrow (x \rightarrow z)\) \\
B3. 2. Доказать равносильность.\(x_{1} \land x_{2} \land \ \ \ ...\ \  \land x_{n} \rightarrow y \equiv x_{1} \rightarrow (x_{2} \rightarrow (\ \ ...\ \  \rightarrow (x_{n} \rightarrow y)\ \ ,,.\ \ ))\) \\
C1. 5. Составить РКС для формулы \((x \land y \rightarrow z) \rightarrow (x \rightarrow (y \rightarrow z))\) \\
C2. 3. Булевы функции заданы последовательностью их значений при лексикографическом упорядочении аргументов. Найти все базисы, которые можно составить из следующих функций. \(f = (01010111)\) \\
C3. 2. Пусть \(A,\ \ B\)- некоторые формулы исчисления предикатов, причем переменная\(x\) не входит в \(B\). Доказать следующие соотношения. \(\forall xA \vee B \rightleftarrows \forall x(A \vee B)\); \\

\end{tabular}
\vspace{1cm}


\begin{tabular}{m{17cm}}
\textbf{12-вариант}
\newline

T1. 4. Исчисления высказываний и понятие формулы исчиления высказываний \emph{О.с. Определения доказуемой формулы. Аксиомы исчисления высказываний. Правыла выводимости} \\
T2. 2. Исчисления высказываний и понятие формулы исчиления высказываний \emph{О.с. Определения доказуемой формулы. Аксиомы исчисления высказываний. Правыла выводимости} \\
A1. 3. Доказать тоджество. \(A\backslash(B\bigcup C) = (A\backslash B)\bigcup(A\backslash C)\) \\
A2. 2. Найти все подформулы. \(\left( (р \land q) \vee q \right) \vee (q \rightarrow p)\) \\
A3. 8. Составить таблица истинности. \((x \rightarrow z) \rightarrow (y \rightarrow z)\) \\
B1. 10. Для следующих формул найти ДНФ и КНФ. \(\overline{x \vee (x_{2} \rightarrow x_{1})}\) \\
B2. 8. Упростить формулу алгебра логики. \(А(х,у,z) = (x \rightarrow y) \land z \rightarrow (x \rightarrow z)\) \\
B3. 6. Доказать, что двойственные формулы. \(f(х,\ \ y) = x \land y\) и \(f^{*}(х) = x \vee y.\) \\
C1. 3. Составить РКС для формулы \((x \rightarrow z) \rightarrow (y \rightarrow z) \rightarrow ((x \vee y) \rightarrow z)\) \\
C2. 9. Булевы функции заданы последовательностью их значений при лексикографическом упорядочении аргументов. Найти все базисы, которые можно составить из следующих функций. \(f = (10010110)\) \\
C3. 1. Пусть \(A,\ \ B\)- некоторые формулы исчисления предикатов, причем переменная\(x\) не входит в \(B\). Доказать следующие соотношения. \(\forall xAЂ\); \\

\end{tabular}
\vspace{1cm}


\begin{tabular}{m{17cm}}
\textbf{13-вариант}
\newline

T1. 3. Нормальные формы \emph{О.с. Дизъюнктивная, конъюнктивная и совершенная дизъюнктивная, совершенна конъюнктивная нормальная форма} \\
T2. 8. Понятие выводимости формулы из совокупности формулы \emph{О.с. Теорема дедукции. Обобщенная теорема дедукции.} \\
A1. 4. Доказать тоджество. \(A\bigcap(B\bigcup C) = (A\bigcap B)\bigcup(B\bigcap C)\) \\
A2. 10. Найти все подформулы. \((x \rightarrow y) \land (x \rightarrow \overline{y}) \rightarrow \overline{x} \leftrightarrow \overline{y}\) \\
A3. 7. Составить таблица истинности. \((x \rightarrow z) \rightarrow (y \rightarrow z) \rightarrow ((x \vee y) \rightarrow z)\) \\
B1. 4. Для следующих формул найти КНФ. \((х \rightarrow у) \rightarrow (\overline{х} \rightarrow \overline{у)}\). \\
B2. 6. Упростить формул. \(A(a,\ \ b,\ \ c) = (a \rightarrow b) \rightarrow (\overline{a}\  \land (b \vee c))\) \\
B3. 7. Доказать равносильность. \(x \vee \left( \overline{x\ } \land \overline{y} \right) \equiv x \vee \overline{y}\) \\
C1. 2. Составить РКС для формулы \(A(a,\ \ b,\ \ c) = (a \rightarrow b) \rightarrow (\overline{a}\  \land (b \vee c))\) \\
C2. 1. Булевы функции заданы последовательностью их значений при лексикографическом упорядочении аргументов. Найти все базисы, которые можно составить из следующих функций.\(f = (00111100)\) \\
C3. 4. Привести к предваренной нормальной форме \(\forall xB(x) \supset \exists y(A(y) \supset B(x))\) \\

\end{tabular}
\vspace{1cm}


\begin{tabular}{m{17cm}}
\textbf{14-вариант}
\newline

T1. 6. Производные правила вывода \emph{О.с. Правило силлогизма, правило контрпозиции,правило снятия двойного отрицания} \\
T2. 6. Логика предикатов. \emph{О.с. Понятие предиката, понятие формулы логики предикатов, равносильные формулы логики предикатов} \\
A1. 6. Доказать тождество. \(\overline{A\bigcap B} = \overline{A}\bigcup\overline{B}\) \\
A2. 8. Найти все подформулы. \((х \rightarrow у) \rightarrow (\overline{х} \rightarrow \overline{у)}\). \\
A3. 9. Составить таблица истинности. \((x \rightarrow y) \land (x \rightarrow \overline{y}) \rightarrow \overline{x}\) \\
B1. 3. Для следующих формул найти СКНФ \(\overline{(a \rightarrow c)} \rightarrow \left( (b \rightarrow c) \rightarrow (a \vee b \rightarrow c) \right)\) \\
B2. 9. Упростить формул. \(A(a,\ \ b,\ \ c) = (a \rightarrow b) \rightarrow (\overline{a}\  \land (b \vee c))\) \\
B3. 3. Доказать, что двойственные формулы.\(f(х,\ \ y) = x \land y\) и \(f^{*}(х,\ \ y) = x \vee y.\) \\
C1. 4. Составить РКС для формулы \((х \rightarrow у) \rightarrow (\overline{х} \rightarrow \overline{у)}\) \\
C2. 5. Булевы функции заданы последовательностью их значений при лексикографическом упорядочении аргументов. Найти все базисы, которые можно составить из следующих функций. \(f = (01110001)\) \\
C3. 3. Пусть \(A,\ \ B\)- некоторые формулы исчисления предикатов, причем переменная\(x\) не входит в \(B\). Доказать следующие соотношения. \(\exists xA \vee B \rightleftarrows \exists x(A \vee B)\) \\

\end{tabular}
\vspace{1cm}


\begin{tabular}{m{17cm}}
\textbf{15-вариант}
\newline

T1. 8. Логика предикатов. \emph{О.с. Понятие предиката, понятие формулы логики предикатов, равносильные формулы логики предикатов} \\
T2. 5. Нормальные формы \emph{О.с. Дизъюнктивная, конъюнктивная и совершенная дизъюнктивная, совершенна конъюнктивная нормальная форма} \\
A1. 8. Доказать тоджество. \(A\bigcap(B\backslash C) = (A\bigcap B)\backslash(A\bigcap C) = (A\bigcap B)\backslash C\) \\
A2. 9. Найти все подформулы. \((x \rightarrow y) \land (x \rightarrow \overline{y}) \rightarrow \overline{x} \vee y\) \\
A3. 3. Составить таблица истинности. \((x \rightarrow z) \rightarrow (y \rightarrow z) \rightarrow ((x \vee y) \rightarrow z)\) \\
B1. 7. Для следующих формул найти ДНФ и КНФ. \(\left( х_{1} \land \overline{х_{2}} \right) \vee х_{3}\) \\
B2. 3. Упростить формулу алгебра логики. \(A(x,\ \ y) = (x \leftrightarrow y) \land (x \vee y)\) \\
B3. 9. Доказать, что двойственные формулы. \(f(х,\ \ y) = x \leftrightarrow y\) и \(f^{*}(х) = \overline{x \leftrightarrow y}.\) \\
C1. 7. Составить РКС для формулы \(А(х,у,z) = (x \rightarrow y) \land z \rightarrow (x \rightarrow z)\) \\
C2. 4. Булевы функции заданы последовательностью их значений при лексикографическом упорядочении аргументов. Найти все базисы, которые можно составить из следующих функций. \(f = (10100101)\) \\
C3. 8. Пусть \(A,\ \ B\)- некоторые формулы исчисления предикатов, причем переменная\(x\) не входит в \(B\). Доказать следующие соотношения. \(\forall xB \rightleftarrows B\); \\

\end{tabular}
\vspace{1cm}


\begin{tabular}{m{17cm}}
\textbf{16-вариант}
\newline

T1. 10. Производные правила вывода \emph{О.с. Правило силлогизма, правило контрпозиции,правило снятия двойного отрицания} \\
T2. 4. Понятие выводимости формулы из совокупности формулы \emph{О.с. Теорема дедукции. Обобщенная теорема дедукции.} \\
A1. 5. Доказать тоджество. \(A\bigcup(B\bigcap C) = (A\bigcup B)\bigcap(A\bigcup C)\) \\
A2. 3. Найти все подформулы. \(х \land у \rightarrow (z \vee у \rightarrow z)\) \\
A3. 2. Составить таблица истинности. \((x \rightarrow y) \land (x \rightarrow \overline{y}) \rightarrow \overline{x}\) \\
B1. 2. Для следующих формул найти ДНФ и КНФ. \(\ x_{1} \land (x_{2} \vee (x_{1} \rightarrow x_{3}))\) \\
B2. 1. Доказать тождественно ложность формул. \(F\left( р_{1},р_{2} \right) = \overline{p_{1} \rightarrow (p_{2} \rightarrow p_{1})}\) \\
B3. 8. Доказать равносильность. \(x \rightarrow \left( y \rightarrow \overline{z} \right) \equiv x \land y \rightarrow \overline{z}\) \\
C1. 6. Составить РКС для формулы \((x \rightarrow y) \land (x \rightarrow \overline{y}) \rightarrow \overline{x} \leftrightarrow \overline{y}\) \\
C2. 10 Булевы функции заданы последовательностью их значений при лексикографическом упорядочении аргументов. Найти все базисы, которые можно составить из следующих функций. \(f = (11001000)\) \\
C3. 9. Пусть \(A,\ \ B\)- некоторые формулы исчисления предикатов, причем переменная\(x\) не входит в \(B\). Доказать следующие соотношения. \(\exists xB \rightleftarrows B\)
 \\

\end{tabular}
\vspace{1cm}


\begin{tabular}{m{17cm}}
\textbf{17-вариант}
\newline

T1. 2. Понятие выводимости формулы из совокупности формулы \emph{О.с. Теорема дедукции. Обобщенная теорема дедукции.} \\
T2. 7. Алгебра высказываний и понятие формула алгебры высказываний \emph{О.с. Равносильные формулы алгебры логики и их свойтсва} \\
A1. 6. Доказать тождество. \(\overline{A\bigcap B} = \overline{A}\bigcup\overline{B}\) \\
A2. 4. Найти все подформулы. \((x \rightarrow y) \land (x \rightarrow \overline{y}) \rightarrow \overline{x}\) \\
A3. 4. Составить таблица истинности. \((x \rightarrow y) \land (x \rightarrow \overline{y}) \rightarrow \overline{x} \vee y\) \\
B1. 1. Для следующих формул найти ДНФ и КНФ. \(х \land у \rightarrow (z \vee у \rightarrow z)\) \\
B2. 7. Упростить формулу алгебра логики. \(А(х,у,z) = (x \rightarrow y) \land z \rightarrow (x \rightarrow z)\) \\
B3. 1. Найти \(x\), если \(\left( \overline{x \vee a} \right) \vee \left( \overline{x \vee \overline{a}} \right) \equiv b\) \\
C1. 9. Составить РКС для формулы \(x \rightarrow (x \rightarrow y) \rightarrow (\overline{x} \rightarrow y)\ \) \\
C2. 7. Булевы функции заданы последовательностью их значений при лексикографическом упорядочении аргументов. Найти все базисы, которые можно составить из следующих функций. \(f = (01101001)\) \\
C3. 6. Привести к предваренной нормальной форме \(\forall x(A(x) \supset B(y))Ђ\) \\

\end{tabular}
\vspace{1cm}


\begin{tabular}{m{17cm}}
\textbf{18-вариант}
\newline

T1. 9. Алгебра высказываний и понятие формула алгебры высказываний \emph{О.с. Равносильные формулы алгебры логики и их свойтсва} \\
T2. 9. Нормальные формы \emph{О.с. Дизъюнктивная, конъюнктивная и совершенная дизъюнктивная, совершенна конъюнктивная нормальная форма} \\
A1. 4. Доказать тоджество. \(A\bigcap(B\bigcup C) = (A\bigcap B)\bigcup(B\bigcap C)\) \\
A2. 5. Найти все подформулы. \((х \vee у) \rightarrow \left( х \land \overline{у \vee х \rightarrow у} \right)\) \\
A3. 6. Составить таблица истинности. \((x \rightarrow y) \land (x \rightarrow \overline{y}) \rightarrow \overline{x} \leftrightarrow \overline{y}\) \\
B1. 8. Для следующих формул найти ДНФ и КНФ. \(х \land у \rightarrow (z \vee у \rightarrow z)\) \\
B2. 5. Упростить формулу алгебра логики. \((x \rightarrow y) \land (x \rightarrow \overline{y}) \rightarrow \overline{x}\) \\
B3. 4. Доказать равносильность. \((x \vee y) \land (x \vee \overline{y}) \equiv x\) \\
C1. 1. Составить РКС для формулы \(F(a,\ \ b,\ \ c) = (a \vee b) \rightarrow (a \land b) \vee c.\) \\
C2. 6. Булевы функции заданы последовательностью их значений при лексикографическом упорядочении аргументов. Найти все базисы, которые можно составить из следующих функций. \(f = (00010101)\) \\
C3. 5. Привести к предваренной нормальной форме \(\forall x(\neg A(x) \supset \exists yB(y)) \supset B(y) \vee A(x)\) \\

\end{tabular}
\vspace{1cm}


\begin{tabular}{m{17cm}}
\textbf{19-вариант}
\newline

T1. 7. Некоторые приложения алгебры логики \emph{О.с. Реле контактные схемы, применения алгебры логики в решении логических задач} \\
T2. 3. Функции алгебры логики \emph{О.с. Алгебра Буля. Полная система функции} \\
A1. 5. Доказать тоджество. \(A\bigcup(B\bigcap C) = (A\bigcup B)\bigcap(A\bigcup C)\) \\
A2. 1. Найти все подформулы. \(((a_{0} \rightarrow a_{1}) \land ((a_{1} \rightarrow a_{2}) \rightarrow (\overline{a_{0}} \vee a_{2})).\) \\
A3. 5. Составить таблица истинности. \((x \rightarrow \overline{y}) \rightarrow \overline{x}\) \\
B1. 9. Для следующих формул найти СКНФ \((z \rightarrow x \land y))\) \\
B2. 2. Упростить формулу алгебра логики. \(А(х,у,z) = (x \rightarrow y) \land z \rightarrow (x \rightarrow z)\) \\
B3. 5. Найти двойственную формулу к формуле \(f(х,\ \ y,\ \ z) = xy \vee yz \vee xz\) \\
C1. 10. Составить РКС для формулы \(\overline{(a \rightarrow c)} \rightarrow \left( (b \rightarrow c) \rightarrow (a \vee b \rightarrow c) \right)\) \\
C2. 2. Булевы функции заданы последовательностью их значений при лексикографическом упорядочении аргументов. Найти все базисы, которые можно составить из следующих функций. \(f = (11101000)\) \\
C3. 4. Привести к предваренной нормальной форме \(\forall xB(x) \supset \exists y(A(y) \supset B(x))\) \\

\end{tabular}
\vspace{1cm}


\begin{tabular}{m{17cm}}
\textbf{20-вариант}
\newline

T1. 5. Функции алгебры логики \emph{О.с. Алгебра Буля. Полная система функции} \\
T2. 10. Исчисления высказываний и понятие формулы исчиления высказываний \emph{О.с. Определения доказуемой формулы. Аксиомы исчисления высказываний. Правыла выводимости} \\
A1. 2. Известно, что высказывание \(A \rightarrow B\) ложно. Что можно сказать обистинности \(A\) и \(B\) ? \\
A2. 6. Найти все подформулы. \(\overline{(a \rightarrow c)} \rightarrow \left( (b \rightarrow c) \rightarrow (a \vee b \rightarrow c) \right)\) \\
A3. 3. Составить таблица истинности. \((x \rightarrow z) \rightarrow (y \rightarrow z) \rightarrow ((x \vee y) \rightarrow z)\) \\
B1. 6. Для следующих формул найти ДНФ и КНФ. \(\ x_{1} \land (x_{2} \vee (x_{1} \rightarrow x_{3}))\) \\
B2. 4. Упростить формулу алгебра логики. \((х \rightarrow у) \rightarrow (\overline{х} \rightarrow \overline{у)}\) \\
B3. 6. Доказать, что двойственные формулы. \(f(х,\ \ y) = x \land y\) и \(f^{*}(х) = x \vee y.\) \\
C1. 8. Составить РКС для формулы \(A(a,\ \ b,\ \ c) = (a \rightarrow b) \rightarrow (\overline{a}\  \land (b \vee c))\) \\
C2. 8. Булевы функции заданы последовательностью их значений при лексикографическом упорядочении аргументов. Найти все базисы, которые можно составить из следующих функций. \(f = (10011011)\) \\
C3. 7. Пусть \(A,\ \ B\)- некоторые формулы исчисления предикатов, причем переменная\(x\) не входит в \(B\). Доказать следующие соотношения. \(\exists xA \vee B \rightleftarrows \exists x(A \vee B)\); \\

\end{tabular}
\vspace{1cm}


\begin{tabular}{m{17cm}}
\textbf{21-вариант}
\newline

T1. 3. Нормальные формы \emph{О.с. Дизъюнктивная, конъюнктивная и совершенная дизъюнктивная, совершенна конъюнктивная нормальная форма} \\
T2. 4. Понятие выводимости формулы из совокупности формулы \emph{О.с. Теорема дедукции. Обобщенная теорема дедукции.} \\
A1. 7. Доказать тоджество. \((A\bigcup B)\bigcap A = (A\bigcap B)\bigcup A = A\) \\
A2. 3. Найти все подформулы. \(х \land у \rightarrow (z \vee у \rightarrow z)\) \\
A3. 7. Составить таблица истинности. \((x \rightarrow z) \rightarrow (y \rightarrow z) \rightarrow ((x \vee y) \rightarrow z)\) \\
B1. 2. Для следующих формул найти ДНФ и КНФ. \(\ x_{1} \land (x_{2} \vee (x_{1} \rightarrow x_{3}))\) \\
B2. 8. Упростить формулу алгебра логики. \(А(х,у,z) = (x \rightarrow y) \land z \rightarrow (x \rightarrow z)\) \\
B3. 7. Доказать равносильность. \(x \vee \left( \overline{x\ } \land \overline{y} \right) \equiv x \vee \overline{y}\) \\
C1. 1. Составить РКС для формулы \(F(a,\ \ b,\ \ c) = (a \vee b) \rightarrow (a \land b) \vee c.\) \\
C2. 1. Булевы функции заданы последовательностью их значений при лексикографическом упорядочении аргументов. Найти все базисы, которые можно составить из следующих функций.\(f = (00111100)\) \\
C3. 8. Пусть \(A,\ \ B\)- некоторые формулы исчисления предикатов, причем переменная\(x\) не входит в \(B\). Доказать следующие соотношения. \(\forall xB \rightleftarrows B\); \\

\end{tabular}
\vspace{1cm}


\begin{tabular}{m{17cm}}
\textbf{22-вариант}
\newline

T1. 8. Логика предикатов. \emph{О.с. Понятие предиката, понятие формулы логики предикатов, равносильные формулы логики предикатов} \\
T2. 10. Исчисления высказываний и понятие формулы исчиления высказываний \emph{О.с. Определения доказуемой формулы. Аксиомы исчисления высказываний. Правыла выводимости} \\
A1. 3. Доказать тоджество. \(A\backslash(B\bigcup C) = (A\backslash B)\bigcup(A\backslash C)\) \\
A2. 4. Найти все подформулы. \((x \rightarrow y) \land (x \rightarrow \overline{y}) \rightarrow \overline{x}\) \\
A3. 9. Составить таблица истинности. \((x \rightarrow y) \land (x \rightarrow \overline{y}) \rightarrow \overline{x}\) \\
B1. 8. Для следующих формул найти ДНФ и КНФ. \(х \land у \rightarrow (z \vee у \rightarrow z)\) \\
B2. 3. Упростить формулу алгебра логики. \(A(x,\ \ y) = (x \leftrightarrow y) \land (x \vee y)\) \\
B3. 3. Доказать, что двойственные формулы.\(f(х,\ \ y) = x \land y\) и \(f^{*}(х,\ \ y) = x \vee y.\) \\
C1. 3. Составить РКС для формулы \((x \rightarrow z) \rightarrow (y \rightarrow z) \rightarrow ((x \vee y) \rightarrow z)\) \\
C2. 4. Булевы функции заданы последовательностью их значений при лексикографическом упорядочении аргументов. Найти все базисы, которые можно составить из следующих функций. \(f = (10100101)\) \\
C3. 2. Пусть \(A,\ \ B\)- некоторые формулы исчисления предикатов, причем переменная\(x\) не входит в \(B\). Доказать следующие соотношения. \(\forall xA \vee B \rightleftarrows \forall x(A \vee B)\); \\

\end{tabular}
\vspace{1cm}


\begin{tabular}{m{17cm}}
\textbf{23-вариант}
\newline

T1. 9. Алгебра высказываний и понятие формула алгебры высказываний \emph{О.с. Равносильные формулы алгебры логики и их свойтсва} \\
T2. 6. Логика предикатов. \emph{О.с. Понятие предиката, понятие формулы логики предикатов, равносильные формулы логики предикатов} \\
A1. 1. Доказать тоджество. \(A\bigcap(\overline{A}) = \varnothing\) \\
A2. 1. Найти все подформулы. \(((a_{0} \rightarrow a_{1}) \land ((a_{1} \rightarrow a_{2}) \rightarrow (\overline{a_{0}} \vee a_{2})).\) \\
A3. 8. Составить таблица истинности. \((x \rightarrow z) \rightarrow (y \rightarrow z)\) \\
B1. 10. Для следующих формул найти ДНФ и КНФ. \(\overline{x \vee (x_{2} \rightarrow x_{1})}\) \\
B2. 1. Доказать тождественно ложность формул. \(F\left( р_{1},р_{2} \right) = \overline{p_{1} \rightarrow (p_{2} \rightarrow p_{1})}\) \\
B3. 9. Доказать, что двойственные формулы. \(f(х,\ \ y) = x \leftrightarrow y\) и \(f^{*}(х) = \overline{x \leftrightarrow y}.\) \\
C1. 9. Составить РКС для формулы \(x \rightarrow (x \rightarrow y) \rightarrow (\overline{x} \rightarrow y)\ \) \\
C2. 6. Булевы функции заданы последовательностью их значений при лексикографическом упорядочении аргументов. Найти все базисы, которые можно составить из следующих функций. \(f = (00010101)\) \\
C3. 3. Пусть \(A,\ \ B\)- некоторые формулы исчисления предикатов, причем переменная\(x\) не входит в \(B\). Доказать следующие соотношения. \(\exists xA \vee B \rightleftarrows \exists x(A \vee B)\) \\

\end{tabular}
\vspace{1cm}


\begin{tabular}{m{17cm}}
\textbf{24-вариант}
\newline

T1. 5. Функции алгебры логики \emph{О.с. Алгебра Буля. Полная система функции} \\
T2. 5. Нормальные формы \emph{О.с. Дизъюнктивная, конъюнктивная и совершенная дизъюнктивная, совершенна конъюнктивная нормальная форма} \\
A1. 8. Доказать тоджество. \(A\bigcap(B\backslash C) = (A\bigcap B)\backslash(A\bigcap C) = (A\bigcap B)\backslash C\) \\
A2. 2. Найти все подформулы. \(\left( (р \land q) \vee q \right) \vee (q \rightarrow p)\) \\
A3. 6. Составить таблица истинности. \((x \rightarrow y) \land (x \rightarrow \overline{y}) \rightarrow \overline{x} \leftrightarrow \overline{y}\) \\
B1. 3. Для следующих формул найти СКНФ \(\overline{(a \rightarrow c)} \rightarrow \left( (b \rightarrow c) \rightarrow (a \vee b \rightarrow c) \right)\) \\
B2. 5. Упростить формулу алгебра логики. \((x \rightarrow y) \land (x \rightarrow \overline{y}) \rightarrow \overline{x}\) \\
B3. 1. Найти \(x\), если \(\left( \overline{x \vee a} \right) \vee \left( \overline{x \vee \overline{a}} \right) \equiv b\) \\
C1. 6. Составить РКС для формулы \((x \rightarrow y) \land (x \rightarrow \overline{y}) \rightarrow \overline{x} \leftrightarrow \overline{y}\) \\
C2. 8. Булевы функции заданы последовательностью их значений при лексикографическом упорядочении аргументов. Найти все базисы, которые можно составить из следующих функций. \(f = (10011011)\) \\
C3. 5. Привести к предваренной нормальной форме \(\forall x(\neg A(x) \supset \exists yB(y)) \supset B(y) \vee A(x)\) \\

\end{tabular}
\vspace{1cm}


\begin{tabular}{m{17cm}}
\textbf{25-вариант}
\newline

T1. 2. Понятие выводимости формулы из совокупности формулы \emph{О.с. Теорема дедукции. Обобщенная теорема дедукции.} \\
T2. 8. Понятие выводимости формулы из совокупности формулы \emph{О.с. Теорема дедукции. Обобщенная теорема дедукции.} \\
A1. 8. Доказать тоджество. \(A\bigcap(B\backslash C) = (A\bigcap B)\backslash(A\bigcap C) = (A\bigcap B)\backslash C\) \\
A2. 5. Найти все подформулы. \((х \vee у) \rightarrow \left( х \land \overline{у \vee х \rightarrow у} \right)\) \\
A3. 4. Составить таблица истинности. \((x \rightarrow y) \land (x \rightarrow \overline{y}) \rightarrow \overline{x} \vee y\) \\
B1. 5. Для следующих формул найти СДНФ \(\overline{\overline{x \vee y} \rightarrow \overline{x \land y}}\) \\
B2. 9. Упростить формул. \(A(a,\ \ b,\ \ c) = (a \rightarrow b) \rightarrow (\overline{a}\  \land (b \vee c))\) \\
B3. 8. Доказать равносильность. \(x \rightarrow \left( y \rightarrow \overline{z} \right) \equiv x \land y \rightarrow \overline{z}\) \\
C1. 8. Составить РКС для формулы \(A(a,\ \ b,\ \ c) = (a \rightarrow b) \rightarrow (\overline{a}\  \land (b \vee c))\) \\
C2. 2. Булевы функции заданы последовательностью их значений при лексикографическом упорядочении аргументов. Найти все базисы, которые можно составить из следующих функций. \(f = (11101000)\) \\
C3. 1. Пусть \(A,\ \ B\)- некоторые формулы исчисления предикатов, причем переменная\(x\) не входит в \(B\). Доказать следующие соотношения. \(\forall xAЂ\); \\

\end{tabular}
\vspace{1cm}


\begin{tabular}{m{17cm}}
\textbf{26-вариант}
\newline

T1. 4. Исчисления высказываний и понятие формулы исчиления высказываний \emph{О.с. Определения доказуемой формулы. Аксиомы исчисления высказываний. Правыла выводимости} \\
T2. 9. Нормальные формы \emph{О.с. Дизъюнктивная, конъюнктивная и совершенная дизъюнктивная, совершенна конъюнктивная нормальная форма} \\
A1. 3. Доказать тоджество. \(A\backslash(B\bigcup C) = (A\backslash B)\bigcup(A\backslash C)\) \\
A2. 7. Найти все подформулы. \((x \rightarrow z) \rightarrow (y \rightarrow z) \rightarrow ((x \vee y) \rightarrow z)\) \\
A3. 1. Составить таблица истинности. \(\left( (р \land q) \vee q \right) \vee (q \rightarrow p)\) \\
B1. 1. Для следующих формул найти ДНФ и КНФ. \(х \land у \rightarrow (z \vee у \rightarrow z)\) \\
B2. 7. Упростить формулу алгебра логики. \(А(х,у,z) = (x \rightarrow y) \land z \rightarrow (x \rightarrow z)\) \\
B3. 2. Доказать равносильность.\(x_{1} \land x_{2} \land \ \ \ ...\ \  \land x_{n} \rightarrow y \equiv x_{1} \rightarrow (x_{2} \rightarrow (\ \ ...\ \  \rightarrow (x_{n} \rightarrow y)\ \ ,,.\ \ ))\) \\
C1. 7. Составить РКС для формулы \(А(х,у,z) = (x \rightarrow y) \land z \rightarrow (x \rightarrow z)\) \\
C2. 10 Булевы функции заданы последовательностью их значений при лексикографическом упорядочении аргументов. Найти все базисы, которые можно составить из следующих функций. \(f = (11001000)\) \\
C3. 9. Пусть \(A,\ \ B\)- некоторые формулы исчисления предикатов, причем переменная\(x\) не входит в \(B\). Доказать следующие соотношения. \(\exists xB \rightleftarrows B\)
 \\

\end{tabular}
\vspace{1cm}


\begin{tabular}{m{17cm}}
\textbf{27-вариант}
\newline

T1. 6. Производные правила вывода \emph{О.с. Правило силлогизма, правило контрпозиции,правило снятия двойного отрицания} \\
T2. 7. Алгебра высказываний и понятие формула алгебры высказываний \emph{О.с. Равносильные формулы алгебры логики и их свойтсва} \\
A1. 1. Доказать тоджество. \(A\bigcap(\overline{A}) = \varnothing\) \\
A2. 9. Найти все подформулы. \((x \rightarrow y) \land (x \rightarrow \overline{y}) \rightarrow \overline{x} \vee y\) \\
A3. 2. Составить таблица истинности. \((x \rightarrow y) \land (x \rightarrow \overline{y}) \rightarrow \overline{x}\) \\
B1. 7. Для следующих формул найти ДНФ и КНФ. \(\left( х_{1} \land \overline{х_{2}} \right) \vee х_{3}\) \\
B2. 6. Упростить формул. \(A(a,\ \ b,\ \ c) = (a \rightarrow b) \rightarrow (\overline{a}\  \land (b \vee c))\) \\
B3. 4. Доказать равносильность. \((x \vee y) \land (x \vee \overline{y}) \equiv x\) \\
C1. 4. Составить РКС для формулы \((х \rightarrow у) \rightarrow (\overline{х} \rightarrow \overline{у)}\) \\
C2. 9. Булевы функции заданы последовательностью их значений при лексикографическом упорядочении аргументов. Найти все базисы, которые можно составить из следующих функций. \(f = (10010110)\) \\
C3. 6. Привести к предваренной нормальной форме \(\forall x(A(x) \supset B(y))Ђ\) \\

\end{tabular}
\vspace{1cm}


\begin{tabular}{m{17cm}}
\textbf{28-вариант}
\newline

T1. 7. Некоторые приложения алгебры логики \emph{О.с. Реле контактные схемы, применения алгебры логики в решении логических задач} \\
T2. 2. Исчисления высказываний и понятие формулы исчиления высказываний \emph{О.с. Определения доказуемой формулы. Аксиомы исчисления высказываний. Правыла выводимости} \\
A1. 5. Доказать тоджество. \(A\bigcup(B\bigcap C) = (A\bigcup B)\bigcap(A\bigcup C)\) \\
A2. 6. Найти все подформулы. \(\overline{(a \rightarrow c)} \rightarrow \left( (b \rightarrow c) \rightarrow (a \vee b \rightarrow c) \right)\) \\
A3. 4. Составить таблица истинности. \((x \rightarrow y) \land (x \rightarrow \overline{y}) \rightarrow \overline{x} \vee y\) \\
B1. 9. Для следующих формул найти СКНФ \((z \rightarrow x \land y))\) \\
B2. 3. Упростить формулу алгебра логики. \(A(x,\ \ y) = (x \leftrightarrow y) \land (x \vee y)\) \\
B3. 4. Доказать равносильность. \((x \vee y) \land (x \vee \overline{y}) \equiv x\) \\
C1. 10. Составить РКС для формулы \(\overline{(a \rightarrow c)} \rightarrow \left( (b \rightarrow c) \rightarrow (a \vee b \rightarrow c) \right)\) \\
C2. 7. Булевы функции заданы последовательностью их значений при лексикографическом упорядочении аргументов. Найти все базисы, которые можно составить из следующих функций. \(f = (01101001)\) \\
C3. 7. Пусть \(A,\ \ B\)- некоторые формулы исчисления предикатов, причем переменная\(x\) не входит в \(B\). Доказать следующие соотношения. \(\exists xA \vee B \rightleftarrows \exists x(A \vee B)\); \\

\end{tabular}
\vspace{1cm}


\begin{tabular}{m{17cm}}
\textbf{29-вариант}
\newline

T1. 1. Алгебра высказываний и понятие формула алгебры высказываний \emph{О.с. Равносильные формулы алгебры логики и их свойтсва} \\
T2. 3. Функции алгебры логики \emph{О.с. Алгебра Буля. Полная система функции} \\
A1. 6. Доказать тождество. \(\overline{A\bigcap B} = \overline{A}\bigcup\overline{B}\) \\
A2. 8. Найти все подформулы. \((х \rightarrow у) \rightarrow (\overline{х} \rightarrow \overline{у)}\). \\
A3. 9. Составить таблица истинности. \((x \rightarrow y) \land (x \rightarrow \overline{y}) \rightarrow \overline{x}\) \\
B1. 4. Для следующих формул найти КНФ. \((х \rightarrow у) \rightarrow (\overline{х} \rightarrow \overline{у)}\). \\
B2. 5. Упростить формулу алгебра логики. \((x \rightarrow y) \land (x \rightarrow \overline{y}) \rightarrow \overline{x}\) \\
B3. 3. Доказать, что двойственные формулы.\(f(х,\ \ y) = x \land y\) и \(f^{*}(х,\ \ y) = x \vee y.\) \\
C1. 2. Составить РКС для формулы \(A(a,\ \ b,\ \ c) = (a \rightarrow b) \rightarrow (\overline{a}\  \land (b \vee c))\) \\
C2. 5. Булевы функции заданы последовательностью их значений при лексикографическом упорядочении аргументов. Найти все базисы, которые можно составить из следующих функций. \(f = (01110001)\) \\
C3. 1. Пусть \(A,\ \ B\)- некоторые формулы исчисления предикатов, причем переменная\(x\) не входит в \(B\). Доказать следующие соотношения. \(\forall xAЂ\); \\

\end{tabular}
\vspace{1cm}


\begin{tabular}{m{17cm}}
\textbf{30-вариант}
\newline

T1. 10. Производные правила вывода \emph{О.с. Правило силлогизма, правило контрпозиции,правило снятия двойного отрицания} \\
T2. 1. Некоторые приложения алгебры логики \emph{О.с. Реле контактные схемы, применения алгебры логики в решении логических задач} \\
A1. 7. Доказать тоджество. \((A\bigcup B)\bigcap A = (A\bigcap B)\bigcup A = A\) \\
A2. 10. Найти все подформулы. \((x \rightarrow y) \land (x \rightarrow \overline{y}) \rightarrow \overline{x} \leftrightarrow \overline{y}\) \\
A3. 8. Составить таблица истинности. \((x \rightarrow z) \rightarrow (y \rightarrow z)\) \\
B1. 6. Для следующих формул найти ДНФ и КНФ. \(\ x_{1} \land (x_{2} \vee (x_{1} \rightarrow x_{3}))\) \\
B2. 1. Доказать тождественно ложность формул. \(F\left( р_{1},р_{2} \right) = \overline{p_{1} \rightarrow (p_{2} \rightarrow p_{1})}\) \\
B3. 7. Доказать равносильность. \(x \vee \left( \overline{x\ } \land \overline{y} \right) \equiv x \vee \overline{y}\) \\
C1. 5. Составить РКС для формулы \((x \land y \rightarrow z) \rightarrow (x \rightarrow (y \rightarrow z))\) \\
C2. 3. Булевы функции заданы последовательностью их значений при лексикографическом упорядочении аргументов. Найти все базисы, которые можно составить из следующих функций. \(f = (01010111)\) \\
C3. 9. Пусть \(A,\ \ B\)- некоторые формулы исчисления предикатов, причем переменная\(x\) не входит в \(B\). Доказать следующие соотношения. \(\exists xB \rightleftarrows B\)
 \\

\end{tabular}
\vspace{1cm}


\begin{tabular}{m{17cm}}
\textbf{31-вариант}
\newline

T1. 8. Логика предикатов. \emph{О.с. Понятие предиката, понятие формулы логики предикатов, равносильные формулы логики предикатов} \\
T2. 2. Исчисления высказываний и понятие формулы исчиления высказываний \emph{О.с. Определения доказуемой формулы. Аксиомы исчисления высказываний. Правыла выводимости} \\
A1. 4. Доказать тоджество. \(A\bigcap(B\bigcup C) = (A\bigcap B)\bigcup(B\bigcap C)\) \\
A2. 6. Найти все подформулы. \(\overline{(a \rightarrow c)} \rightarrow \left( (b \rightarrow c) \rightarrow (a \vee b \rightarrow c) \right)\) \\
A3. 7. Составить таблица истинности. \((x \rightarrow z) \rightarrow (y \rightarrow z) \rightarrow ((x \vee y) \rightarrow z)\) \\
B1. 7. Для следующих формул найти ДНФ и КНФ. \(\left( х_{1} \land \overline{х_{2}} \right) \vee х_{3}\) \\
B2. 4. Упростить формулу алгебра логики. \((х \rightarrow у) \rightarrow (\overline{х} \rightarrow \overline{у)}\) \\
B3. 5. Найти двойственную формулу к формуле \(f(х,\ \ y,\ \ z) = xy \vee yz \vee xz\) \\
C1. 2. Составить РКС для формулы \(A(a,\ \ b,\ \ c) = (a \rightarrow b) \rightarrow (\overline{a}\  \land (b \vee c))\) \\
C2. 1. Булевы функции заданы последовательностью их значений при лексикографическом упорядочении аргументов. Найти все базисы, которые можно составить из следующих функций.\(f = (00111100)\) \\
C3. 8. Пусть \(A,\ \ B\)- некоторые формулы исчисления предикатов, причем переменная\(x\) не входит в \(B\). Доказать следующие соотношения. \(\forall xB \rightleftarrows B\); \\

\end{tabular}
\vspace{1cm}


\begin{tabular}{m{17cm}}
\textbf{32-вариант}
\newline

T1. 1. Алгебра высказываний и понятие формула алгебры высказываний \emph{О.с. Равносильные формулы алгебры логики и их свойтсва} \\
T2. 7. Алгебра высказываний и понятие формула алгебры высказываний \emph{О.с. Равносильные формулы алгебры логики и их свойтсва} \\
A1. 2. Известно, что высказывание \(A \rightarrow B\) ложно. Что можно сказать обистинности \(A\) и \(B\) ? \\
A2. 1. Найти все подформулы. \(((a_{0} \rightarrow a_{1}) \land ((a_{1} \rightarrow a_{2}) \rightarrow (\overline{a_{0}} \vee a_{2})).\) \\
A3. 5. Составить таблица истинности. \((x \rightarrow \overline{y}) \rightarrow \overline{x}\) \\
B1. 8. Для следующих формул найти ДНФ и КНФ. \(х \land у \rightarrow (z \vee у \rightarrow z)\) \\
B2. 9. Упростить формул. \(A(a,\ \ b,\ \ c) = (a \rightarrow b) \rightarrow (\overline{a}\  \land (b \vee c))\) \\
B3. 8. Доказать равносильность. \(x \rightarrow \left( y \rightarrow \overline{z} \right) \equiv x \land y \rightarrow \overline{z}\) \\
C1. 10. Составить РКС для формулы \(\overline{(a \rightarrow c)} \rightarrow \left( (b \rightarrow c) \rightarrow (a \vee b \rightarrow c) \right)\) \\
C2. 3. Булевы функции заданы последовательностью их значений при лексикографическом упорядочении аргументов. Найти все базисы, которые можно составить из следующих функций. \(f = (01010111)\) \\
C3. 6. Привести к предваренной нормальной форме \(\forall x(A(x) \supset B(y))Ђ\) \\

\end{tabular}
\vspace{1cm}


\begin{tabular}{m{17cm}}
\textbf{33-вариант}
\newline

T1. 6. Производные правила вывода \emph{О.с. Правило силлогизма, правило контрпозиции,правило снятия двойного отрицания} \\
T2. 10. Исчисления высказываний и понятие формулы исчиления высказываний \emph{О.с. Определения доказуемой формулы. Аксиомы исчисления высказываний. Правыла выводимости} \\
A1. 3. Доказать тоджество. \(A\backslash(B\bigcup C) = (A\backslash B)\bigcup(A\backslash C)\) \\
A2. 4. Найти все подформулы. \((x \rightarrow y) \land (x \rightarrow \overline{y}) \rightarrow \overline{x}\) \\
A3. 2. Составить таблица истинности. \((x \rightarrow y) \land (x \rightarrow \overline{y}) \rightarrow \overline{x}\) \\
B1. 5. Для следующих формул найти СДНФ \(\overline{\overline{x \vee y} \rightarrow \overline{x \land y}}\) \\
B2. 6. Упростить формул. \(A(a,\ \ b,\ \ c) = (a \rightarrow b) \rightarrow (\overline{a}\  \land (b \vee c))\) \\
B3. 9. Доказать, что двойственные формулы. \(f(х,\ \ y) = x \leftrightarrow y\) и \(f^{*}(х) = \overline{x \leftrightarrow y}.\) \\
C1. 5. Составить РКС для формулы \((x \land y \rightarrow z) \rightarrow (x \rightarrow (y \rightarrow z))\) \\
C2. 6. Булевы функции заданы последовательностью их значений при лексикографическом упорядочении аргументов. Найти все базисы, которые можно составить из следующих функций. \(f = (00010101)\) \\
C3. 5. Привести к предваренной нормальной форме \(\forall x(\neg A(x) \supset \exists yB(y)) \supset B(y) \vee A(x)\) \\

\end{tabular}
\vspace{1cm}


\begin{tabular}{m{17cm}}
\textbf{34-вариант}
\newline

T1. 10. Производные правила вывода \emph{О.с. Правило силлогизма, правило контрпозиции,правило снятия двойного отрицания} \\
T2. 9. Нормальные формы \emph{О.с. Дизъюнктивная, конъюнктивная и совершенная дизъюнктивная, совершенна конъюнктивная нормальная форма} \\
A1. 2. Известно, что высказывание \(A \rightarrow B\) ложно. Что можно сказать обистинности \(A\) и \(B\) ? \\
A2. 7. Найти все подформулы. \((x \rightarrow z) \rightarrow (y \rightarrow z) \rightarrow ((x \vee y) \rightarrow z)\) \\
A3. 3. Составить таблица истинности. \((x \rightarrow z) \rightarrow (y \rightarrow z) \rightarrow ((x \vee y) \rightarrow z)\) \\
B1. 10. Для следующих формул найти ДНФ и КНФ. \(\overline{x \vee (x_{2} \rightarrow x_{1})}\) \\
B2. 7. Упростить формулу алгебра логики. \(А(х,у,z) = (x \rightarrow y) \land z \rightarrow (x \rightarrow z)\) \\
B3. 2. Доказать равносильность.\(x_{1} \land x_{2} \land \ \ \ ...\ \  \land x_{n} \rightarrow y \equiv x_{1} \rightarrow (x_{2} \rightarrow (\ \ ...\ \  \rightarrow (x_{n} \rightarrow y)\ \ ,,.\ \ ))\) \\
C1. 1. Составить РКС для формулы \(F(a,\ \ b,\ \ c) = (a \vee b) \rightarrow (a \land b) \vee c.\) \\
C2. 2. Булевы функции заданы последовательностью их значений при лексикографическом упорядочении аргументов. Найти все базисы, которые можно составить из следующих функций. \(f = (11101000)\) \\
C3. 3. Пусть \(A,\ \ B\)- некоторые формулы исчисления предикатов, причем переменная\(x\) не входит в \(B\). Доказать следующие соотношения. \(\exists xA \vee B \rightleftarrows \exists x(A \vee B)\) \\

\end{tabular}
\vspace{1cm}


\begin{tabular}{m{17cm}}
\textbf{35-вариант}
\newline

T1. 4. Исчисления высказываний и понятие формулы исчиления высказываний \emph{О.с. Определения доказуемой формулы. Аксиомы исчисления высказываний. Правыла выводимости} \\
T2. 6. Логика предикатов. \emph{О.с. Понятие предиката, понятие формулы логики предикатов, равносильные формулы логики предикатов} \\
A1. 7. Доказать тоджество. \((A\bigcup B)\bigcap A = (A\bigcap B)\bigcup A = A\) \\
A2. 9. Найти все подформулы. \((x \rightarrow y) \land (x \rightarrow \overline{y}) \rightarrow \overline{x} \vee y\) \\
A3. 6. Составить таблица истинности. \((x \rightarrow y) \land (x \rightarrow \overline{y}) \rightarrow \overline{x} \leftrightarrow \overline{y}\) \\
B1. 4. Для следующих формул найти КНФ. \((х \rightarrow у) \rightarrow (\overline{х} \rightarrow \overline{у)}\). \\
B2. 2. Упростить формулу алгебра логики. \(А(х,у,z) = (x \rightarrow y) \land z \rightarrow (x \rightarrow z)\) \\
B3. 1. Найти \(x\), если \(\left( \overline{x \vee a} \right) \vee \left( \overline{x \vee \overline{a}} \right) \equiv b\) \\
C1. 9. Составить РКС для формулы \(x \rightarrow (x \rightarrow y) \rightarrow (\overline{x} \rightarrow y)\ \) \\
C2. 5. Булевы функции заданы последовательностью их значений при лексикографическом упорядочении аргументов. Найти все базисы, которые можно составить из следующих функций. \(f = (01110001)\) \\
C3. 2. Пусть \(A,\ \ B\)- некоторые формулы исчисления предикатов, причем переменная\(x\) не входит в \(B\). Доказать следующие соотношения. \(\forall xA \vee B \rightleftarrows \forall x(A \vee B)\); \\

\end{tabular}
\vspace{1cm}


\begin{tabular}{m{17cm}}
\textbf{36-вариант}
\newline

T1. 9. Алгебра высказываний и понятие формула алгебры высказываний \emph{О.с. Равносильные формулы алгебры логики и их свойтсва} \\
T2. 5. Нормальные формы \emph{О.с. Дизъюнктивная, конъюнктивная и совершенная дизъюнктивная, совершенна конъюнктивная нормальная форма} \\
A1. 8. Доказать тоджество. \(A\bigcap(B\backslash C) = (A\bigcap B)\backslash(A\bigcap C) = (A\bigcap B)\backslash C\) \\
A2. 10. Найти все подформулы. \((x \rightarrow y) \land (x \rightarrow \overline{y}) \rightarrow \overline{x} \leftrightarrow \overline{y}\) \\
A3. 1. Составить таблица истинности. \(\left( (р \land q) \vee q \right) \vee (q \rightarrow p)\) \\
B1. 6. Для следующих формул найти ДНФ и КНФ. \(\ x_{1} \land (x_{2} \vee (x_{1} \rightarrow x_{3}))\) \\
B2. 8. Упростить формулу алгебра логики. \(А(х,у,z) = (x \rightarrow y) \land z \rightarrow (x \rightarrow z)\) \\
B3. 6. Доказать, что двойственные формулы. \(f(х,\ \ y) = x \land y\) и \(f^{*}(х) = x \vee y.\) \\
C1. 6. Составить РКС для формулы \((x \rightarrow y) \land (x \rightarrow \overline{y}) \rightarrow \overline{x} \leftrightarrow \overline{y}\) \\
C2. 4. Булевы функции заданы последовательностью их значений при лексикографическом упорядочении аргументов. Найти все базисы, которые можно составить из следующих функций. \(f = (10100101)\) \\
C3. 4. Привести к предваренной нормальной форме \(\forall xB(x) \supset \exists y(A(y) \supset B(x))\) \\

\end{tabular}
\vspace{1cm}


\begin{tabular}{m{17cm}}
\textbf{37-вариант}
\newline

T1. 7. Некоторые приложения алгебры логики \emph{О.с. Реле контактные схемы, применения алгебры логики в решении логических задач} \\
T2. 1. Некоторые приложения алгебры логики \emph{О.с. Реле контактные схемы, применения алгебры логики в решении логических задач} \\
A1. 5. Доказать тоджество. \(A\bigcup(B\bigcap C) = (A\bigcup B)\bigcap(A\bigcup C)\) \\
A2. 3. Найти все подформулы. \(х \land у \rightarrow (z \vee у \rightarrow z)\) \\
A3. 7. Составить таблица истинности. \((x \rightarrow z) \rightarrow (y \rightarrow z) \rightarrow ((x \vee y) \rightarrow z)\) \\
B1. 2. Для следующих формул найти ДНФ и КНФ. \(\ x_{1} \land (x_{2} \vee (x_{1} \rightarrow x_{3}))\) \\
B2. 1. Доказать тождественно ложность формул. \(F\left( р_{1},р_{2} \right) = \overline{p_{1} \rightarrow (p_{2} \rightarrow p_{1})}\) \\
B3. 3. Доказать, что двойственные формулы.\(f(х,\ \ y) = x \land y\) и \(f^{*}(х,\ \ y) = x \vee y.\) \\
C1. 8. Составить РКС для формулы \(A(a,\ \ b,\ \ c) = (a \rightarrow b) \rightarrow (\overline{a}\  \land (b \vee c))\) \\
C2. 7. Булевы функции заданы последовательностью их значений при лексикографическом упорядочении аргументов. Найти все базисы, которые можно составить из следующих функций. \(f = (01101001)\) \\
C3. 9. Пусть \(A,\ \ B\)- некоторые формулы исчисления предикатов, причем переменная\(x\) не входит в \(B\). Доказать следующие соотношения. \(\exists xB \rightleftarrows B\)
 \\

\end{tabular}
\vspace{1cm}


\begin{tabular}{m{17cm}}
\textbf{38-вариант}
\newline

T1. 2. Понятие выводимости формулы из совокупности формулы \emph{О.с. Теорема дедукции. Обобщенная теорема дедукции.} \\
T2. 4. Понятие выводимости формулы из совокупности формулы \emph{О.с. Теорема дедукции. Обобщенная теорема дедукции.} \\
A1. 4. Доказать тоджество. \(A\bigcap(B\bigcup C) = (A\bigcap B)\bigcup(B\bigcap C)\) \\
A2. 2. Найти все подформулы. \(\left( (р \land q) \vee q \right) \vee (q \rightarrow p)\) \\
A3. 2. Составить таблица истинности. \((x \rightarrow y) \land (x \rightarrow \overline{y}) \rightarrow \overline{x}\) \\
B1. 9. Для следующих формул найти СКНФ \((z \rightarrow x \land y))\) \\
B2. 3. Упростить формулу алгебра логики. \(A(x,\ \ y) = (x \leftrightarrow y) \land (x \vee y)\) \\
B3. 8. Доказать равносильность. \(x \rightarrow \left( y \rightarrow \overline{z} \right) \equiv x \land y \rightarrow \overline{z}\) \\
C1. 3. Составить РКС для формулы \((x \rightarrow z) \rightarrow (y \rightarrow z) \rightarrow ((x \vee y) \rightarrow z)\) \\
C2. 9. Булевы функции заданы последовательностью их значений при лексикографическом упорядочении аргументов. Найти все базисы, которые можно составить из следующих функций. \(f = (10010110)\) \\
C3. 1. Пусть \(A,\ \ B\)- некоторые формулы исчисления предикатов, причем переменная\(x\) не входит в \(B\). Доказать следующие соотношения. \(\forall xAЂ\); \\

\end{tabular}
\vspace{1cm}


\begin{tabular}{m{17cm}}
\textbf{39-вариант}
\newline

T1. 3. Нормальные формы \emph{О.с. Дизъюнктивная, конъюнктивная и совершенная дизъюнктивная, совершенна конъюнктивная нормальная форма} \\
T2. 8. Понятие выводимости формулы из совокупности формулы \emph{О.с. Теорема дедукции. Обобщенная теорема дедукции.} \\
A1. 6. Доказать тождество. \(\overline{A\bigcap B} = \overline{A}\bigcup\overline{B}\) \\
A2. 8. Найти все подформулы. \((х \rightarrow у) \rightarrow (\overline{х} \rightarrow \overline{у)}\). \\
A3. 5. Составить таблица истинности. \((x \rightarrow \overline{y}) \rightarrow \overline{x}\) \\
B1. 3. Для следующих формул найти СКНФ \(\overline{(a \rightarrow c)} \rightarrow \left( (b \rightarrow c) \rightarrow (a \vee b \rightarrow c) \right)\) \\
B2. 5. Упростить формулу алгебра логики. \((x \rightarrow y) \land (x \rightarrow \overline{y}) \rightarrow \overline{x}\) \\
B3. 1. Найти \(x\), если \(\left( \overline{x \vee a} \right) \vee \left( \overline{x \vee \overline{a}} \right) \equiv b\) \\
C1. 7. Составить РКС для формулы \(А(х,у,z) = (x \rightarrow y) \land z \rightarrow (x \rightarrow z)\) \\
C2. 10 Булевы функции заданы последовательностью их значений при лексикографическом упорядочении аргументов. Найти все базисы, которые можно составить из следующих функций. \(f = (11001000)\) \\
C3. 5. Привести к предваренной нормальной форме \(\forall x(\neg A(x) \supset \exists yB(y)) \supset B(y) \vee A(x)\) \\

\end{tabular}
\vspace{1cm}


\begin{tabular}{m{17cm}}
\textbf{40-вариант}
\newline

T1. 5. Функции алгебры логики \emph{О.с. Алгебра Буля. Полная система функции} \\
T2. 3. Функции алгебры логики \emph{О.с. Алгебра Буля. Полная система функции} \\
A1. 1. Доказать тоджество. \(A\bigcap(\overline{A}) = \varnothing\) \\
A2. 5. Найти все подформулы. \((х \vee у) \rightarrow \left( х \land \overline{у \vee х \rightarrow у} \right)\) \\
A3. 9. Составить таблица истинности. \((x \rightarrow y) \land (x \rightarrow \overline{y}) \rightarrow \overline{x}\) \\
B1. 1. Для следующих формул найти ДНФ и КНФ. \(х \land у \rightarrow (z \vee у \rightarrow z)\) \\
B2. 6. Упростить формул. \(A(a,\ \ b,\ \ c) = (a \rightarrow b) \rightarrow (\overline{a}\  \land (b \vee c))\) \\
B3. 5. Найти двойственную формулу к формуле \(f(х,\ \ y,\ \ z) = xy \vee yz \vee xz\) \\
C1. 4. Составить РКС для формулы \((х \rightarrow у) \rightarrow (\overline{х} \rightarrow \overline{у)}\) \\
C2. 8. Булевы функции заданы последовательностью их значений при лексикографическом упорядочении аргументов. Найти все базисы, которые можно составить из следующих функций. \(f = (10011011)\) \\
C3. 8. Пусть \(A,\ \ B\)- некоторые формулы исчисления предикатов, причем переменная\(x\) не входит в \(B\). Доказать следующие соотношения. \(\forall xB \rightleftarrows B\); \\

\end{tabular}
\vspace{1cm}


\begin{tabular}{m{17cm}}
\textbf{41-вариант}
\newline

T1. 6. Производные правила вывода \emph{О.с. Правило силлогизма, правило контрпозиции,правило снятия двойного отрицания} \\
T2. 2. Исчисления высказываний и понятие формулы исчиления высказываний \emph{О.с. Определения доказуемой формулы. Аксиомы исчисления высказываний. Правыла выводимости} \\
A1. 7. Доказать тоджество. \((A\bigcup B)\bigcap A = (A\bigcap B)\bigcup A = A\) \\
A2. 6. Найти все подформулы. \(\overline{(a \rightarrow c)} \rightarrow \left( (b \rightarrow c) \rightarrow (a \vee b \rightarrow c) \right)\) \\
A3. 1. Составить таблица истинности. \(\left( (р \land q) \vee q \right) \vee (q \rightarrow p)\) \\
B1. 5. Для следующих формул найти СДНФ \(\overline{\overline{x \vee y} \rightarrow \overline{x \land y}}\) \\
B2. 4. Упростить формулу алгебра логики. \((х \rightarrow у) \rightarrow (\overline{х} \rightarrow \overline{у)}\) \\
B3. 2. Доказать равносильность.\(x_{1} \land x_{2} \land \ \ \ ...\ \  \land x_{n} \rightarrow y \equiv x_{1} \rightarrow (x_{2} \rightarrow (\ \ ...\ \  \rightarrow (x_{n} \rightarrow y)\ \ ,,.\ \ ))\) \\
C1. 2. Составить РКС для формулы \(A(a,\ \ b,\ \ c) = (a \rightarrow b) \rightarrow (\overline{a}\  \land (b \vee c))\) \\
C2. 2. Булевы функции заданы последовательностью их значений при лексикографическом упорядочении аргументов. Найти все базисы, которые можно составить из следующих функций. \(f = (11101000)\) \\
C3. 7. Пусть \(A,\ \ B\)- некоторые формулы исчисления предикатов, причем переменная\(x\) не входит в \(B\). Доказать следующие соотношения. \(\exists xA \vee B \rightleftarrows \exists x(A \vee B)\); \\

\end{tabular}
\vspace{1cm}


\begin{tabular}{m{17cm}}
\textbf{42-вариант}
\newline

T1. 7. Некоторые приложения алгебры логики \emph{О.с. Реле контактные схемы, применения алгебры логики в решении логических задач} \\
T2. 4. Понятие выводимости формулы из совокупности формулы \emph{О.с. Теорема дедукции. Обобщенная теорема дедукции.} \\
A1. 8. Доказать тоджество. \(A\bigcap(B\backslash C) = (A\bigcap B)\backslash(A\bigcap C) = (A\bigcap B)\backslash C\) \\
A2. 4. Найти все подформулы. \((x \rightarrow y) \land (x \rightarrow \overline{y}) \rightarrow \overline{x}\) \\
A3. 3. Составить таблица истинности. \((x \rightarrow z) \rightarrow (y \rightarrow z) \rightarrow ((x \vee y) \rightarrow z)\) \\
B1. 6. Для следующих формул найти ДНФ и КНФ. \(\ x_{1} \land (x_{2} \vee (x_{1} \rightarrow x_{3}))\) \\
B2. 7. Упростить формулу алгебра логики. \(А(х,у,z) = (x \rightarrow y) \land z \rightarrow (x \rightarrow z)\) \\
B3. 9. Доказать, что двойственные формулы. \(f(х,\ \ y) = x \leftrightarrow y\) и \(f^{*}(х) = \overline{x \leftrightarrow y}.\) \\
C1. 7. Составить РКС для формулы \(А(х,у,z) = (x \rightarrow y) \land z \rightarrow (x \rightarrow z)\) \\
C2. 1. Булевы функции заданы последовательностью их значений при лексикографическом упорядочении аргументов. Найти все базисы, которые можно составить из следующих функций.\(f = (00111100)\) \\
C3. 6. Привести к предваренной нормальной форме \(\forall x(A(x) \supset B(y))Ђ\) \\

\end{tabular}
\vspace{1cm}


\begin{tabular}{m{17cm}}
\textbf{43-вариант}
\newline

T1. 8. Логика предикатов. \emph{О.с. Понятие предиката, понятие формулы логики предикатов, равносильные формулы логики предикатов} \\
T2. 8. Понятие выводимости формулы из совокупности формулы \emph{О.с. Теорема дедукции. Обобщенная теорема дедукции.} \\
A1. 3. Доказать тоджество. \(A\backslash(B\bigcup C) = (A\backslash B)\bigcup(A\backslash C)\) \\
A2. 1. Найти все подформулы. \(((a_{0} \rightarrow a_{1}) \land ((a_{1} \rightarrow a_{2}) \rightarrow (\overline{a_{0}} \vee a_{2})).\) \\
A3. 6. Составить таблица истинности. \((x \rightarrow y) \land (x \rightarrow \overline{y}) \rightarrow \overline{x} \leftrightarrow \overline{y}\) \\
B1. 9. Для следующих формул найти СКНФ \((z \rightarrow x \land y))\) \\
B2. 8. Упростить формулу алгебра логики. \(А(х,у,z) = (x \rightarrow y) \land z \rightarrow (x \rightarrow z)\) \\
B3. 6. Доказать, что двойственные формулы. \(f(х,\ \ y) = x \land y\) и \(f^{*}(х) = x \vee y.\) \\
C1. 3. Составить РКС для формулы \((x \rightarrow z) \rightarrow (y \rightarrow z) \rightarrow ((x \vee y) \rightarrow z)\) \\
C2. 9. Булевы функции заданы последовательностью их значений при лексикографическом упорядочении аргументов. Найти все базисы, которые можно составить из следующих функций. \(f = (10010110)\) \\
C3. 4. Привести к предваренной нормальной форме \(\forall xB(x) \supset \exists y(A(y) \supset B(x))\) \\

\end{tabular}
\vspace{1cm}


\begin{tabular}{m{17cm}}
\textbf{44-вариант}
\newline

T1. 9. Алгебра высказываний и понятие формула алгебры высказываний \emph{О.с. Равносильные формулы алгебры логики и их свойтсва} \\
T2. 7. Алгебра высказываний и понятие формула алгебры высказываний \emph{О.с. Равносильные формулы алгебры логики и их свойтсва} \\
A1. 5. Доказать тоджество. \(A\bigcup(B\bigcap C) = (A\bigcup B)\bigcap(A\bigcup C)\) \\
A2. 8. Найти все подформулы. \((х \rightarrow у) \rightarrow (\overline{х} \rightarrow \overline{у)}\). \\
A3. 8. Составить таблица истинности. \((x \rightarrow z) \rightarrow (y \rightarrow z)\) \\
B1. 2. Для следующих формул найти ДНФ и КНФ. \(\ x_{1} \land (x_{2} \vee (x_{1} \rightarrow x_{3}))\) \\
B2. 2. Упростить формулу алгебра логики. \(А(х,у,z) = (x \rightarrow y) \land z \rightarrow (x \rightarrow z)\) \\
B3. 7. Доказать равносильность. \(x \vee \left( \overline{x\ } \land \overline{y} \right) \equiv x \vee \overline{y}\) \\
C1. 1. Составить РКС для формулы \(F(a,\ \ b,\ \ c) = (a \vee b) \rightarrow (a \land b) \vee c.\) \\
C2. 5. Булевы функции заданы последовательностью их значений при лексикографическом упорядочении аргументов. Найти все базисы, которые можно составить из следующих функций. \(f = (01110001)\) \\
C3. 2. Пусть \(A,\ \ B\)- некоторые формулы исчисления предикатов, причем переменная\(x\) не входит в \(B\). Доказать следующие соотношения. \(\forall xA \vee B \rightleftarrows \forall x(A \vee B)\); \\

\end{tabular}
\vspace{1cm}


\begin{tabular}{m{17cm}}
\textbf{45-вариант}
\newline

T1. 4. Исчисления высказываний и понятие формулы исчиления высказываний \emph{О.с. Определения доказуемой формулы. Аксиомы исчисления высказываний. Правыла выводимости} \\
T2. 9. Нормальные формы \emph{О.с. Дизъюнктивная, конъюнктивная и совершенная дизъюнктивная, совершенна конъюнктивная нормальная форма} \\
A1. 4. Доказать тоджество. \(A\bigcap(B\bigcup C) = (A\bigcap B)\bigcup(B\bigcap C)\) \\
A2. 5. Найти все подформулы. \((х \vee у) \rightarrow \left( х \land \overline{у \vee х \rightarrow у} \right)\) \\
A3. 4. Составить таблица истинности. \((x \rightarrow y) \land (x \rightarrow \overline{y}) \rightarrow \overline{x} \vee y\) \\
B1. 8. Для следующих формул найти ДНФ и КНФ. \(х \land у \rightarrow (z \vee у \rightarrow z)\) \\
B2. 9. Упростить формул. \(A(a,\ \ b,\ \ c) = (a \rightarrow b) \rightarrow (\overline{a}\  \land (b \vee c))\) \\
B3. 4. Доказать равносильность. \((x \vee y) \land (x \vee \overline{y}) \equiv x\) \\
C1. 10. Составить РКС для формулы \(\overline{(a \rightarrow c)} \rightarrow \left( (b \rightarrow c) \rightarrow (a \vee b \rightarrow c) \right)\) \\
C2. 4. Булевы функции заданы последовательностью их значений при лексикографическом упорядочении аргументов. Найти все базисы, которые можно составить из следующих функций. \(f = (10100101)\) \\
C3. 3. Пусть \(A,\ \ B\)- некоторые формулы исчисления предикатов, причем переменная\(x\) не входит в \(B\). Доказать следующие соотношения. \(\exists xA \vee B \rightleftarrows \exists x(A \vee B)\) \\

\end{tabular}
\vspace{1cm}


\begin{tabular}{m{17cm}}
\textbf{46-вариант}
\newline

T1. 10. Производные правила вывода \emph{О.с. Правило силлогизма, правило контрпозиции,правило снятия двойного отрицания} \\
T2. 6. Логика предикатов. \emph{О.с. Понятие предиката, понятие формулы логики предикатов, равносильные формулы логики предикатов} \\
A1. 2. Известно, что высказывание \(A \rightarrow B\) ложно. Что можно сказать обистинности \(A\) и \(B\) ? \\
A2. 2. Найти все подформулы. \(\left( (р \land q) \vee q \right) \vee (q \rightarrow p)\) \\
A3. 3. Составить таблица истинности. \((x \rightarrow z) \rightarrow (y \rightarrow z) \rightarrow ((x \vee y) \rightarrow z)\) \\
B1. 3. Для следующих формул найти СКНФ \(\overline{(a \rightarrow c)} \rightarrow \left( (b \rightarrow c) \rightarrow (a \vee b \rightarrow c) \right)\) \\
B2. 5. Упростить формулу алгебра логики. \((x \rightarrow y) \land (x \rightarrow \overline{y}) \rightarrow \overline{x}\) \\
B3. 9. Доказать, что двойственные формулы. \(f(х,\ \ y) = x \leftrightarrow y\) и \(f^{*}(х) = \overline{x \leftrightarrow y}.\) \\
C1. 8. Составить РКС для формулы \(A(a,\ \ b,\ \ c) = (a \rightarrow b) \rightarrow (\overline{a}\  \land (b \vee c))\) \\
C2. 6. Булевы функции заданы последовательностью их значений при лексикографическом упорядочении аргументов. Найти все базисы, которые можно составить из следующих функций. \(f = (00010101)\) \\
C3. 4. Привести к предваренной нормальной форме \(\forall xB(x) \supset \exists y(A(y) \supset B(x))\) \\

\end{tabular}
\vspace{1cm}


\begin{tabular}{m{17cm}}
\textbf{47-вариант}
\newline

T1. 3. Нормальные формы \emph{О.с. Дизъюнктивная, конъюнктивная и совершенная дизъюнктивная, совершенна конъюнктивная нормальная форма} \\
T2. 1. Некоторые приложения алгебры логики \emph{О.с. Реле контактные схемы, применения алгебры логики в решении логических задач} \\
A1. 6. Доказать тождество. \(\overline{A\bigcap B} = \overline{A}\bigcup\overline{B}\) \\
A2. 9. Найти все подформулы. \((x \rightarrow y) \land (x \rightarrow \overline{y}) \rightarrow \overline{x} \vee y\) \\
A3. 1. Составить таблица истинности. \(\left( (р \land q) \vee q \right) \vee (q \rightarrow p)\) \\
B1. 1. Для следующих формул найти ДНФ и КНФ. \(х \land у \rightarrow (z \vee у \rightarrow z)\) \\
B2. 8. Упростить формулу алгебра логики. \(А(х,у,z) = (x \rightarrow y) \land z \rightarrow (x \rightarrow z)\) \\
B3. 5. Найти двойственную формулу к формуле \(f(х,\ \ y,\ \ z) = xy \vee yz \vee xz\) \\
C1. 4. Составить РКС для формулы \((х \rightarrow у) \rightarrow (\overline{х} \rightarrow \overline{у)}\) \\
C2. 3. Булевы функции заданы последовательностью их значений при лексикографическом упорядочении аргументов. Найти все базисы, которые можно составить из следующих функций. \(f = (01010111)\) \\
C3. 1. Пусть \(A,\ \ B\)- некоторые формулы исчисления предикатов, причем переменная\(x\) не входит в \(B\). Доказать следующие соотношения. \(\forall xAЂ\); \\

\end{tabular}
\vspace{1cm}


\begin{tabular}{m{17cm}}
\textbf{48-вариант}
\newline

T1. 5. Функции алгебры логики \emph{О.с. Алгебра Буля. Полная система функции} \\
T2. 3. Функции алгебры логики \emph{О.с. Алгебра Буля. Полная система функции} \\
A1. 1. Доказать тоджество. \(A\bigcap(\overline{A}) = \varnothing\) \\
A2. 10. Найти все подформулы. \((x \rightarrow y) \land (x \rightarrow \overline{y}) \rightarrow \overline{x} \leftrightarrow \overline{y}\) \\
A3. 5. Составить таблица истинности. \((x \rightarrow \overline{y}) \rightarrow \overline{x}\) \\
B1. 4. Для следующих формул найти КНФ. \((х \rightarrow у) \rightarrow (\overline{х} \rightarrow \overline{у)}\). \\
B2. 2. Упростить формулу алгебра логики. \(А(х,у,z) = (x \rightarrow y) \land z \rightarrow (x \rightarrow z)\) \\
B3. 1. Найти \(x\), если \(\left( \overline{x \vee a} \right) \vee \left( \overline{x \vee \overline{a}} \right) \equiv b\) \\
C1. 6. Составить РКС для формулы \((x \rightarrow y) \land (x \rightarrow \overline{y}) \rightarrow \overline{x} \leftrightarrow \overline{y}\) \\
C2. 10 Булевы функции заданы последовательностью их значений при лексикографическом упорядочении аргументов. Найти все базисы, которые можно составить из следующих функций. \(f = (11001000)\) \\
C3. 6. Привести к предваренной нормальной форме \(\forall x(A(x) \supset B(y))Ђ\) \\

\end{tabular}
\vspace{1cm}


\begin{tabular}{m{17cm}}
\textbf{49-вариант}
\newline

T1. 1. Алгебра высказываний и понятие формула алгебры высказываний \emph{О.с. Равносильные формулы алгебры логики и их свойтсва} \\
T2. 10. Исчисления высказываний и понятие формулы исчиления высказываний \emph{О.с. Определения доказуемой формулы. Аксиомы исчисления высказываний. Правыла выводимости} \\
A1. 2. Известно, что высказывание \(A \rightarrow B\) ложно. Что можно сказать обистинности \(A\) и \(B\) ? \\
A2. 3. Найти все подформулы. \(х \land у \rightarrow (z \vee у \rightarrow z)\) \\
A3. 2. Составить таблица истинности. \((x \rightarrow y) \land (x \rightarrow \overline{y}) \rightarrow \overline{x}\) \\
B1. 7. Для следующих формул найти ДНФ и КНФ. \(\left( х_{1} \land \overline{х_{2}} \right) \vee х_{3}\) \\
B2. 9. Упростить формул. \(A(a,\ \ b,\ \ c) = (a \rightarrow b) \rightarrow (\overline{a}\  \land (b \vee c))\) \\
B3. 2. Доказать равносильность.\(x_{1} \land x_{2} \land \ \ \ ...\ \  \land x_{n} \rightarrow y \equiv x_{1} \rightarrow (x_{2} \rightarrow (\ \ ...\ \  \rightarrow (x_{n} \rightarrow y)\ \ ,,.\ \ ))\) \\
C1. 5. Составить РКС для формулы \((x \land y \rightarrow z) \rightarrow (x \rightarrow (y \rightarrow z))\) \\
C2. 7. Булевы функции заданы последовательностью их значений при лексикографическом упорядочении аргументов. Найти все базисы, которые можно составить из следующих функций. \(f = (01101001)\) \\
C3. 3. Пусть \(A,\ \ B\)- некоторые формулы исчисления предикатов, причем переменная\(x\) не входит в \(B\). Доказать следующие соотношения. \(\exists xA \vee B \rightleftarrows \exists x(A \vee B)\) \\

\end{tabular}
\vspace{1cm}


\begin{tabular}{m{17cm}}
\textbf{50-вариант}
\newline

T1. 2. Понятие выводимости формулы из совокупности формулы \emph{О.с. Теорема дедукции. Обобщенная теорема дедукции.} \\
T2. 5. Нормальные формы \emph{О.с. Дизъюнктивная, конъюнктивная и совершенная дизъюнктивная, совершенна конъюнктивная нормальная форма} \\
A1. 6. Доказать тождество. \(\overline{A\bigcap B} = \overline{A}\bigcup\overline{B}\) \\
A2. 7. Найти все подформулы. \((x \rightarrow z) \rightarrow (y \rightarrow z) \rightarrow ((x \vee y) \rightarrow z)\) \\
A3. 4. Составить таблица истинности. \((x \rightarrow y) \land (x \rightarrow \overline{y}) \rightarrow \overline{x} \vee y\) \\
B1. 10. Для следующих формул найти ДНФ и КНФ. \(\overline{x \vee (x_{2} \rightarrow x_{1})}\) \\
B2. 7. Упростить формулу алгебра логики. \(А(х,у,z) = (x \rightarrow y) \land z \rightarrow (x \rightarrow z)\) \\
B3. 3. Доказать, что двойственные формулы.\(f(х,\ \ y) = x \land y\) и \(f^{*}(х,\ \ y) = x \vee y.\) \\
C1. 9. Составить РКС для формулы \(x \rightarrow (x \rightarrow y) \rightarrow (\overline{x} \rightarrow y)\ \) \\
C2. 8. Булевы функции заданы последовательностью их значений при лексикографическом упорядочении аргументов. Найти все базисы, которые можно составить из следующих функций. \(f = (10011011)\) \\
C3. 9. Пусть \(A,\ \ B\)- некоторые формулы исчисления предикатов, причем переменная\(x\) не входит в \(B\). Доказать следующие соотношения. \(\exists xB \rightleftarrows B\)
 \\

\end{tabular}
\vspace{1cm}


\begin{tabular}{m{17cm}}
\textbf{51-вариант}
\newline

T1. 4. Исчисления высказываний и понятие формулы исчиления высказываний \emph{О.с. Определения доказуемой формулы. Аксиомы исчисления высказываний. Правыла выводимости} \\
T2. 6. Логика предикатов. \emph{О.с. Понятие предиката, понятие формулы логики предикатов, равносильные формулы логики предикатов} \\
A1. 7. Доказать тоджество. \((A\bigcup B)\bigcap A = (A\bigcap B)\bigcup A = A\) \\
A2. 2. Найти все подформулы. \(\left( (р \land q) \vee q \right) \vee (q \rightarrow p)\) \\
A3. 9. Составить таблица истинности. \((x \rightarrow y) \land (x \rightarrow \overline{y}) \rightarrow \overline{x}\) \\
B1. 6. Для следующих формул найти ДНФ и КНФ. \(\ x_{1} \land (x_{2} \vee (x_{1} \rightarrow x_{3}))\) \\
B2. 1. Доказать тождественно ложность формул. \(F\left( р_{1},р_{2} \right) = \overline{p_{1} \rightarrow (p_{2} \rightarrow p_{1})}\) \\
B3. 6. Доказать, что двойственные формулы. \(f(х,\ \ y) = x \land y\) и \(f^{*}(х) = x \vee y.\) \\
C1. 4. Составить РКС для формулы \((х \rightarrow у) \rightarrow (\overline{х} \rightarrow \overline{у)}\) \\
C2. 6. Булевы функции заданы последовательностью их значений при лексикографическом упорядочении аргументов. Найти все базисы, которые можно составить из следующих функций. \(f = (00010101)\) \\
C3. 7. Пусть \(A,\ \ B\)- некоторые формулы исчисления предикатов, причем переменная\(x\) не входит в \(B\). Доказать следующие соотношения. \(\exists xA \vee B \rightleftarrows \exists x(A \vee B)\); \\

\end{tabular}
\vspace{1cm}


\begin{tabular}{m{17cm}}
\textbf{52-вариант}
\newline

T1. 8. Логика предикатов. \emph{О.с. Понятие предиката, понятие формулы логики предикатов, равносильные формулы логики предикатов} \\
T2. 10. Исчисления высказываний и понятие формулы исчиления высказываний \emph{О.с. Определения доказуемой формулы. Аксиомы исчисления высказываний. Правыла выводимости} \\
A1. 1. Доказать тоджество. \(A\bigcap(\overline{A}) = \varnothing\) \\
A2. 5. Найти все подформулы. \((х \vee у) \rightarrow \left( х \land \overline{у \vee х \rightarrow у} \right)\) \\
A3. 6. Составить таблица истинности. \((x \rightarrow y) \land (x \rightarrow \overline{y}) \rightarrow \overline{x} \leftrightarrow \overline{y}\) \\
B1. 5. Для следующих формул найти СДНФ \(\overline{\overline{x \vee y} \rightarrow \overline{x \land y}}\) \\
B2. 6. Упростить формул. \(A(a,\ \ b,\ \ c) = (a \rightarrow b) \rightarrow (\overline{a}\  \land (b \vee c))\) \\
B3. 7. Доказать равносильность. \(x \vee \left( \overline{x\ } \land \overline{y} \right) \equiv x \vee \overline{y}\) \\
C1. 7. Составить РКС для формулы \(А(х,у,z) = (x \rightarrow y) \land z \rightarrow (x \rightarrow z)\) \\
C2. 2. Булевы функции заданы последовательностью их значений при лексикографическом упорядочении аргументов. Найти все базисы, которые можно составить из следующих функций. \(f = (11101000)\) \\
C3. 5. Привести к предваренной нормальной форме \(\forall x(\neg A(x) \supset \exists yB(y)) \supset B(y) \vee A(x)\) \\

\end{tabular}
\vspace{1cm}


\begin{tabular}{m{17cm}}
\textbf{53-вариант}
\newline

T1. 2. Понятие выводимости формулы из совокупности формулы \emph{О.с. Теорема дедукции. Обобщенная теорема дедукции.} \\
T2. 8. Понятие выводимости формулы из совокупности формулы \emph{О.с. Теорема дедукции. Обобщенная теорема дедукции.} \\
A1. 5. Доказать тоджество. \(A\bigcup(B\bigcap C) = (A\bigcup B)\bigcap(A\bigcup C)\) \\
A2. 8. Найти все подформулы. \((х \rightarrow у) \rightarrow (\overline{х} \rightarrow \overline{у)}\). \\
A3. 7. Составить таблица истинности. \((x \rightarrow z) \rightarrow (y \rightarrow z) \rightarrow ((x \vee y) \rightarrow z)\) \\
B1. 4. Для следующих формул найти КНФ. \((х \rightarrow у) \rightarrow (\overline{х} \rightarrow \overline{у)}\). \\
B2. 4. Упростить формулу алгебра логики. \((х \rightarrow у) \rightarrow (\overline{х} \rightarrow \overline{у)}\) \\
B3. 8. Доказать равносильность. \(x \rightarrow \left( y \rightarrow \overline{z} \right) \equiv x \land y \rightarrow \overline{z}\) \\
C1. 9. Составить РКС для формулы \(x \rightarrow (x \rightarrow y) \rightarrow (\overline{x} \rightarrow y)\ \) \\
C2. 8. Булевы функции заданы последовательностью их значений при лексикографическом упорядочении аргументов. Найти все базисы, которые можно составить из следующих функций. \(f = (10011011)\) \\
C3. 8. Пусть \(A,\ \ B\)- некоторые формулы исчисления предикатов, причем переменная\(x\) не входит в \(B\). Доказать следующие соотношения. \(\forall xB \rightleftarrows B\); \\

\end{tabular}
\vspace{1cm}


\begin{tabular}{m{17cm}}
\textbf{54-вариант}
\newline

T1. 7. Некоторые приложения алгебры логики \emph{О.с. Реле контактные схемы, применения алгебры логики в решении логических задач} \\
T2. 3. Функции алгебры логики \emph{О.с. Алгебра Буля. Полная система функции} \\
A1. 3. Доказать тоджество. \(A\backslash(B\bigcup C) = (A\backslash B)\bigcup(A\backslash C)\) \\
A2. 4. Найти все подформулы. \((x \rightarrow y) \land (x \rightarrow \overline{y}) \rightarrow \overline{x}\) \\
A3. 8. Составить таблица истинности. \((x \rightarrow z) \rightarrow (y \rightarrow z)\) \\
B1. 1. Для следующих формул найти ДНФ и КНФ. \(х \land у \rightarrow (z \vee у \rightarrow z)\) \\
B2. 3. Упростить формулу алгебра логики. \(A(x,\ \ y) = (x \leftrightarrow y) \land (x \vee y)\) \\
B3. 4. Доказать равносильность. \((x \vee y) \land (x \vee \overline{y}) \equiv x\) \\
C1. 1. Составить РКС для формулы \(F(a,\ \ b,\ \ c) = (a \vee b) \rightarrow (a \land b) \vee c.\) \\
C2. 5. Булевы функции заданы последовательностью их значений при лексикографическом упорядочении аргументов. Найти все базисы, которые можно составить из следующих функций. \(f = (01110001)\) \\
C3. 2. Пусть \(A,\ \ B\)- некоторые формулы исчисления предикатов, причем переменная\(x\) не входит в \(B\). Доказать следующие соотношения. \(\forall xA \vee B \rightleftarrows \forall x(A \vee B)\); \\

\end{tabular}
\vspace{1cm}


\begin{tabular}{m{17cm}}
\textbf{55-вариант}
\newline

T1. 3. Нормальные формы \emph{О.с. Дизъюнктивная, конъюнктивная и совершенная дизъюнктивная, совершенна конъюнктивная нормальная форма} \\
T2. 9. Нормальные формы \emph{О.с. Дизъюнктивная, конъюнктивная и совершенная дизъюнктивная, совершенна конъюнктивная нормальная форма} \\
A1. 8. Доказать тоджество. \(A\bigcap(B\backslash C) = (A\bigcap B)\backslash(A\bigcap C) = (A\bigcap B)\backslash C\) \\
A2. 10. Найти все подформулы. \((x \rightarrow y) \land (x \rightarrow \overline{y}) \rightarrow \overline{x} \leftrightarrow \overline{y}\) \\
A3. 9. Составить таблица истинности. \((x \rightarrow y) \land (x \rightarrow \overline{y}) \rightarrow \overline{x}\) \\
B1. 10. Для следующих формул найти ДНФ и КНФ. \(\overline{x \vee (x_{2} \rightarrow x_{1})}\) \\
B2. 6. Упростить формул. \(A(a,\ \ b,\ \ c) = (a \rightarrow b) \rightarrow (\overline{a}\  \land (b \vee c))\) \\
B3. 1. Найти \(x\), если \(\left( \overline{x \vee a} \right) \vee \left( \overline{x \vee \overline{a}} \right) \equiv b\) \\
C1. 10. Составить РКС для формулы \(\overline{(a \rightarrow c)} \rightarrow \left( (b \rightarrow c) \rightarrow (a \vee b \rightarrow c) \right)\) \\
C2. 4. Булевы функции заданы последовательностью их значений при лексикографическом упорядочении аргументов. Найти все базисы, которые можно составить из следующих функций. \(f = (10100101)\) \\
C3. 7. Пусть \(A,\ \ B\)- некоторые формулы исчисления предикатов, причем переменная\(x\) не входит в \(B\). Доказать следующие соотношения. \(\exists xA \vee B \rightleftarrows \exists x(A \vee B)\); \\

\end{tabular}
\vspace{1cm}


\begin{tabular}{m{17cm}}
\textbf{56-вариант}
\newline

T1. 6. Производные правила вывода \emph{О.с. Правило силлогизма, правило контрпозиции,правило снятия двойного отрицания} \\
T2. 4. Понятие выводимости формулы из совокупности формулы \emph{О.с. Теорема дедукции. Обобщенная теорема дедукции.} \\
A1. 4. Доказать тоджество. \(A\bigcap(B\bigcup C) = (A\bigcap B)\bigcup(B\bigcap C)\) \\
A2. 9. Найти все подформулы. \((x \rightarrow y) \land (x \rightarrow \overline{y}) \rightarrow \overline{x} \vee y\) \\
A3. 8. Составить таблица истинности. \((x \rightarrow z) \rightarrow (y \rightarrow z)\) \\
B1. 2. Для следующих формул найти ДНФ и КНФ. \(\ x_{1} \land (x_{2} \vee (x_{1} \rightarrow x_{3}))\) \\
B2. 3. Упростить формулу алгебра логики. \(A(x,\ \ y) = (x \leftrightarrow y) \land (x \vee y)\) \\
B3. 5. Найти двойственную формулу к формуле \(f(х,\ \ y,\ \ z) = xy \vee yz \vee xz\) \\
C1. 5. Составить РКС для формулы \((x \land y \rightarrow z) \rightarrow (x \rightarrow (y \rightarrow z))\) \\
C2. 3. Булевы функции заданы последовательностью их значений при лексикографическом упорядочении аргументов. Найти все базисы, которые можно составить из следующих функций. \(f = (01010111)\) \\
C3. 6. Привести к предваренной нормальной форме \(\forall x(A(x) \supset B(y))Ђ\) \\

\end{tabular}
\vspace{1cm}


\begin{tabular}{m{17cm}}
\textbf{57-вариант}
\newline

T1. 9. Алгебра высказываний и понятие формула алгебры высказываний \emph{О.с. Равносильные формулы алгебры логики и их свойтсва} \\
T2. 7. Алгебра высказываний и понятие формула алгебры высказываний \emph{О.с. Равносильные формулы алгебры логики и их свойтсва} \\
A1. 4. Доказать тоджество. \(A\bigcap(B\bigcup C) = (A\bigcap B)\bigcup(B\bigcap C)\) \\
A2. 7. Найти все подформулы. \((x \rightarrow z) \rightarrow (y \rightarrow z) \rightarrow ((x \vee y) \rightarrow z)\) \\
A3. 4. Составить таблица истинности. \((x \rightarrow y) \land (x \rightarrow \overline{y}) \rightarrow \overline{x} \vee y\) \\
B1. 9. Для следующих формул найти СКНФ \((z \rightarrow x \land y))\) \\
B2. 8. Упростить формулу алгебра логики. \(А(х,у,z) = (x \rightarrow y) \land z \rightarrow (x \rightarrow z)\) \\
B3. 7. Доказать равносильность. \(x \vee \left( \overline{x\ } \land \overline{y} \right) \equiv x \vee \overline{y}\) \\
C1. 2. Составить РКС для формулы \(A(a,\ \ b,\ \ c) = (a \rightarrow b) \rightarrow (\overline{a}\  \land (b \vee c))\) \\
C2. 1. Булевы функции заданы последовательностью их значений при лексикографическом упорядочении аргументов. Найти все базисы, которые можно составить из следующих функций.\(f = (00111100)\) \\
C3. 3. Пусть \(A,\ \ B\)- некоторые формулы исчисления предикатов, причем переменная\(x\) не входит в \(B\). Доказать следующие соотношения. \(\exists xA \vee B \rightleftarrows \exists x(A \vee B)\) \\

\end{tabular}
\vspace{1cm}


\begin{tabular}{m{17cm}}
\textbf{58-вариант}
\newline

T1. 1. Алгебра высказываний и понятие формула алгебры высказываний \emph{О.с. Равносильные формулы алгебры логики и их свойтсва} \\
T2. 1. Некоторые приложения алгебры логики \emph{О.с. Реле контактные схемы, применения алгебры логики в решении логических задач} \\
A1. 1. Доказать тоджество. \(A\bigcap(\overline{A}) = \varnothing\) \\
A2. 6. Найти все подформулы. \(\overline{(a \rightarrow c)} \rightarrow \left( (b \rightarrow c) \rightarrow (a \vee b \rightarrow c) \right)\) \\
A3. 5. Составить таблица истинности. \((x \rightarrow \overline{y}) \rightarrow \overline{x}\) \\
B1. 3. Для следующих формул найти СКНФ \(\overline{(a \rightarrow c)} \rightarrow \left( (b \rightarrow c) \rightarrow (a \vee b \rightarrow c) \right)\) \\
B2. 2. Упростить формулу алгебра логики. \(А(х,у,z) = (x \rightarrow y) \land z \rightarrow (x \rightarrow z)\) \\
B3. 2. Доказать равносильность.\(x_{1} \land x_{2} \land \ \ \ ...\ \  \land x_{n} \rightarrow y \equiv x_{1} \rightarrow (x_{2} \rightarrow (\ \ ...\ \  \rightarrow (x_{n} \rightarrow y)\ \ ,,.\ \ ))\) \\
C1. 6. Составить РКС для формулы \((x \rightarrow y) \land (x \rightarrow \overline{y}) \rightarrow \overline{x} \leftrightarrow \overline{y}\) \\
C2. 10 Булевы функции заданы последовательностью их значений при лексикографическом упорядочении аргументов. Найти все базисы, которые можно составить из следующих функций. \(f = (11001000)\) \\
C3. 8. Пусть \(A,\ \ B\)- некоторые формулы исчисления предикатов, причем переменная\(x\) не входит в \(B\). Доказать следующие соотношения. \(\forall xB \rightleftarrows B\); \\

\end{tabular}
\vspace{1cm}


\begin{tabular}{m{17cm}}
\textbf{59-вариант}
\newline

T1. 5. Функции алгебры логики \emph{О.с. Алгебра Буля. Полная система функции} \\
T2. 5. Нормальные формы \emph{О.с. Дизъюнктивная, конъюнктивная и совершенная дизъюнктивная, совершенна конъюнктивная нормальная форма} \\
A1. 7. Доказать тоджество. \((A\bigcup B)\bigcap A = (A\bigcap B)\bigcup A = A\) \\
A2. 1. Найти все подформулы. \(((a_{0} \rightarrow a_{1}) \land ((a_{1} \rightarrow a_{2}) \rightarrow (\overline{a_{0}} \vee a_{2})).\) \\
A3. 6. Составить таблица истинности. \((x \rightarrow y) \land (x \rightarrow \overline{y}) \rightarrow \overline{x} \leftrightarrow \overline{y}\) \\
B1. 8. Для следующих формул найти ДНФ и КНФ. \(х \land у \rightarrow (z \vee у \rightarrow z)\) \\
B2. 7. Упростить формулу алгебра логики. \(А(х,у,z) = (x \rightarrow y) \land z \rightarrow (x \rightarrow z)\) \\
B3. 8. Доказать равносильность. \(x \rightarrow \left( y \rightarrow \overline{z} \right) \equiv x \land y \rightarrow \overline{z}\) \\
C1. 8. Составить РКС для формулы \(A(a,\ \ b,\ \ c) = (a \rightarrow b) \rightarrow (\overline{a}\  \land (b \vee c))\) \\
C2. 7. Булевы функции заданы последовательностью их значений при лексикографическом упорядочении аргументов. Найти все базисы, которые можно составить из следующих функций. \(f = (01101001)\) \\
C3. 9. Пусть \(A,\ \ B\)- некоторые формулы исчисления предикатов, причем переменная\(x\) не входит в \(B\). Доказать следующие соотношения. \(\exists xB \rightleftarrows B\)
 \\

\end{tabular}
\vspace{1cm}


\begin{tabular}{m{17cm}}
\textbf{60-вариант}
\newline

T1. 10. Производные правила вывода \emph{О.с. Правило силлогизма, правило контрпозиции,правило снятия двойного отрицания} \\
T2. 2. Исчисления высказываний и понятие формулы исчиления высказываний \emph{О.с. Определения доказуемой формулы. Аксиомы исчисления высказываний. Правыла выводимости} \\
A1. 8. Доказать тоджество. \(A\bigcap(B\backslash C) = (A\bigcap B)\backslash(A\bigcap C) = (A\bigcap B)\backslash C\) \\
A2. 3. Найти все подформулы. \(х \land у \rightarrow (z \vee у \rightarrow z)\) \\
A3. 3. Составить таблица истинности. \((x \rightarrow z) \rightarrow (y \rightarrow z) \rightarrow ((x \vee y) \rightarrow z)\) \\
B1. 7. Для следующих формул найти ДНФ и КНФ. \(\left( х_{1} \land \overline{х_{2}} \right) \vee х_{3}\) \\
B2. 4. Упростить формулу алгебра логики. \((х \rightarrow у) \rightarrow (\overline{х} \rightarrow \overline{у)}\) \\
B3. 6. Доказать, что двойственные формулы. \(f(х,\ \ y) = x \land y\) и \(f^{*}(х) = x \vee y.\) \\
C1. 3. Составить РКС для формулы \((x \rightarrow z) \rightarrow (y \rightarrow z) \rightarrow ((x \vee y) \rightarrow z)\) \\
C2. 9. Булевы функции заданы последовательностью их значений при лексикографическом упорядочении аргументов. Найти все базисы, которые можно составить из следующих функций. \(f = (10010110)\) \\
C3. 5. Привести к предваренной нормальной форме \(\forall x(\neg A(x) \supset \exists yB(y)) \supset B(y) \vee A(x)\) \\

\end{tabular}
\vspace{1cm}


\begin{tabular}{m{17cm}}
\textbf{61-вариант}
\newline

T1. 10. Производные правила вывода \emph{О.с. Правило силлогизма, правило контрпозиции,правило снятия двойного отрицания} \\
T2. 4. Понятие выводимости формулы из совокупности формулы \emph{О.с. Теорема дедукции. Обобщенная теорема дедукции.} \\
A1. 5. Доказать тоджество. \(A\bigcup(B\bigcap C) = (A\bigcup B)\bigcap(A\bigcup C)\) \\
A2. 6. Найти все подформулы. \(\overline{(a \rightarrow c)} \rightarrow \left( (b \rightarrow c) \rightarrow (a \vee b \rightarrow c) \right)\) \\
A3. 7. Составить таблица истинности. \((x \rightarrow z) \rightarrow (y \rightarrow z) \rightarrow ((x \vee y) \rightarrow z)\) \\
B1. 9. Для следующих формул найти СКНФ \((z \rightarrow x \land y))\) \\
B2. 1. Доказать тождественно ложность формул. \(F\left( р_{1},р_{2} \right) = \overline{p_{1} \rightarrow (p_{2} \rightarrow p_{1})}\) \\
B3. 3. Доказать, что двойственные формулы.\(f(х,\ \ y) = x \land y\) и \(f^{*}(х,\ \ y) = x \vee y.\) \\
C1. 10. Составить РКС для формулы \(\overline{(a \rightarrow c)} \rightarrow \left( (b \rightarrow c) \rightarrow (a \vee b \rightarrow c) \right)\) \\
C2. 6. Булевы функции заданы последовательностью их значений при лексикографическом упорядочении аргументов. Найти все базисы, которые можно составить из следующих функций. \(f = (00010101)\) \\
C3. 1. Пусть \(A,\ \ B\)- некоторые формулы исчисления предикатов, причем переменная\(x\) не входит в \(B\). Доказать следующие соотношения. \(\forall xAЂ\); \\

\end{tabular}
\vspace{1cm}


\begin{tabular}{m{17cm}}
\textbf{62-вариант}
\newline

T1. 7. Некоторые приложения алгебры логики \emph{О.с. Реле контактные схемы, применения алгебры логики в решении логических задач} \\
T2. 5. Нормальные формы \emph{О.с. Дизъюнктивная, конъюнктивная и совершенная дизъюнктивная, совершенна конъюнктивная нормальная форма} \\
A1. 2. Известно, что высказывание \(A \rightarrow B\) ложно. Что можно сказать обистинности \(A\) и \(B\) ? \\
A2. 3. Найти все подформулы. \(х \land у \rightarrow (z \vee у \rightarrow z)\) \\
A3. 1. Составить таблица истинности. \(\left( (р \land q) \vee q \right) \vee (q \rightarrow p)\) \\
B1. 6. Для следующих формул найти ДНФ и КНФ. \(\ x_{1} \land (x_{2} \vee (x_{1} \rightarrow x_{3}))\) \\
B2. 5. Упростить формулу алгебра логики. \((x \rightarrow y) \land (x \rightarrow \overline{y}) \rightarrow \overline{x}\) \\
B3. 4. Доказать равносильность. \((x \vee y) \land (x \vee \overline{y}) \equiv x\) \\
C1. 2. Составить РКС для формулы \(A(a,\ \ b,\ \ c) = (a \rightarrow b) \rightarrow (\overline{a}\  \land (b \vee c))\) \\
C2. 3. Булевы функции заданы последовательностью их значений при лексикографическом упорядочении аргументов. Найти все базисы, которые можно составить из следующих функций. \(f = (01010111)\) \\
C3. 2. Пусть \(A,\ \ B\)- некоторые формулы исчисления предикатов, причем переменная\(x\) не входит в \(B\). Доказать следующие соотношения. \(\forall xA \vee B \rightleftarrows \forall x(A \vee B)\); \\

\end{tabular}
\vspace{1cm}


\begin{tabular}{m{17cm}}
\textbf{63-вариант}
\newline

T1. 8. Логика предикатов. \emph{О.с. Понятие предиката, понятие формулы логики предикатов, равносильные формулы логики предикатов} \\
T2. 9. Нормальные формы \emph{О.с. Дизъюнктивная, конъюнктивная и совершенная дизъюнктивная, совершенна конъюнктивная нормальная форма} \\
A1. 6. Доказать тождество. \(\overline{A\bigcap B} = \overline{A}\bigcup\overline{B}\) \\
A2. 1. Найти все подформулы. \(((a_{0} \rightarrow a_{1}) \land ((a_{1} \rightarrow a_{2}) \rightarrow (\overline{a_{0}} \vee a_{2})).\) \\
A3. 2. Составить таблица истинности. \((x \rightarrow y) \land (x \rightarrow \overline{y}) \rightarrow \overline{x}\) \\
B1. 10. Для следующих формул найти ДНФ и КНФ. \(\overline{x \vee (x_{2} \rightarrow x_{1})}\) \\
B2. 9. Упростить формул. \(A(a,\ \ b,\ \ c) = (a \rightarrow b) \rightarrow (\overline{a}\  \land (b \vee c))\) \\
B3. 9. Доказать, что двойственные формулы. \(f(х,\ \ y) = x \leftrightarrow y\) и \(f^{*}(х) = \overline{x \leftrightarrow y}.\) \\
C1. 5. Составить РКС для формулы \((x \land y \rightarrow z) \rightarrow (x \rightarrow (y \rightarrow z))\) \\
C2. 9. Булевы функции заданы последовательностью их значений при лексикографическом упорядочении аргументов. Найти все базисы, которые можно составить из следующих функций. \(f = (10010110)\) \\
C3. 4. Привести к предваренной нормальной форме \(\forall xB(x) \supset \exists y(A(y) \supset B(x))\) \\

\end{tabular}
\vspace{1cm}


\begin{tabular}{m{17cm}}
\textbf{64-вариант}
\newline

T1. 4. Исчисления высказываний и понятие формулы исчиления высказываний \emph{О.с. Определения доказуемой формулы. Аксиомы исчисления высказываний. Правыла выводимости} \\
T2. 6. Логика предикатов. \emph{О.с. Понятие предиката, понятие формулы логики предикатов, равносильные формулы логики предикатов} \\
A1. 3. Доказать тоджество. \(A\backslash(B\bigcup C) = (A\backslash B)\bigcup(A\backslash C)\) \\
A2. 5. Найти все подформулы. \((х \vee у) \rightarrow \left( х \land \overline{у \vee х \rightarrow у} \right)\) \\
A3. 9. Составить таблица истинности. \((x \rightarrow y) \land (x \rightarrow \overline{y}) \rightarrow \overline{x}\) \\
B1. 1. Для следующих формул найти ДНФ и КНФ. \(х \land у \rightarrow (z \vee у \rightarrow z)\) \\
B2. 4. Упростить формулу алгебра логики. \((х \rightarrow у) \rightarrow (\overline{х} \rightarrow \overline{у)}\) \\
B3. 8. Доказать равносильность. \(x \rightarrow \left( y \rightarrow \overline{z} \right) \equiv x \land y \rightarrow \overline{z}\) \\
C1. 3. Составить РКС для формулы \((x \rightarrow z) \rightarrow (y \rightarrow z) \rightarrow ((x \vee y) \rightarrow z)\) \\
C2. 2. Булевы функции заданы последовательностью их значений при лексикографическом упорядочении аргументов. Найти все базисы, которые можно составить из следующих функций. \(f = (11101000)\) \\
C3. 8. Пусть \(A,\ \ B\)- некоторые формулы исчисления предикатов, причем переменная\(x\) не входит в \(B\). Доказать следующие соотношения. \(\forall xB \rightleftarrows B\); \\

\end{tabular}
\vspace{1cm}


\begin{tabular}{m{17cm}}
\textbf{65-вариант}
\newline

T1. 5. Функции алгебры логики \emph{О.с. Алгебра Буля. Полная система функции} \\
T2. 3. Функции алгебры логики \emph{О.с. Алгебра Буля. Полная система функции} \\
A1. 5. Доказать тоджество. \(A\bigcup(B\bigcap C) = (A\bigcup B)\bigcap(A\bigcup C)\) \\
A2. 4. Найти все подформулы. \((x \rightarrow y) \land (x \rightarrow \overline{y}) \rightarrow \overline{x}\) \\
A3. 8. Составить таблица истинности. \((x \rightarrow z) \rightarrow (y \rightarrow z)\) \\
B1. 5. Для следующих формул найти СДНФ \(\overline{\overline{x \vee y} \rightarrow \overline{x \land y}}\) \\
B2. 9. Упростить формул. \(A(a,\ \ b,\ \ c) = (a \rightarrow b) \rightarrow (\overline{a}\  \land (b \vee c))\) \\
B3. 5. Найти двойственную формулу к формуле \(f(х,\ \ y,\ \ z) = xy \vee yz \vee xz\) \\
C1. 1. Составить РКС для формулы \(F(a,\ \ b,\ \ c) = (a \vee b) \rightarrow (a \land b) \vee c.\) \\
C2. 1. Булевы функции заданы последовательностью их значений при лексикографическом упорядочении аргументов. Найти все базисы, которые можно составить из следующих функций.\(f = (00111100)\) \\
C3. 9. Пусть \(A,\ \ B\)- некоторые формулы исчисления предикатов, причем переменная\(x\) не входит в \(B\). Доказать следующие соотношения. \(\exists xB \rightleftarrows B\)
 \\

\end{tabular}
\vspace{1cm}


\begin{tabular}{m{17cm}}
\textbf{66-вариант}
\newline

T1. 3. Нормальные формы \emph{О.с. Дизъюнктивная, конъюнктивная и совершенная дизъюнктивная, совершенна конъюнктивная нормальная форма} \\
T2. 1. Некоторые приложения алгебры логики \emph{О.с. Реле контактные схемы, применения алгебры логики в решении логических задач} \\
A1. 4. Доказать тоджество. \(A\bigcap(B\bigcup C) = (A\bigcap B)\bigcup(B\bigcap C)\) \\
A2. 8. Найти все подформулы. \((х \rightarrow у) \rightarrow (\overline{х} \rightarrow \overline{у)}\). \\
A3. 3. Составить таблица истинности. \((x \rightarrow z) \rightarrow (y \rightarrow z) \rightarrow ((x \vee y) \rightarrow z)\) \\
B1. 7. Для следующих формул найти ДНФ и КНФ. \(\left( х_{1} \land \overline{х_{2}} \right) \vee х_{3}\) \\
B2. 6. Упростить формул. \(A(a,\ \ b,\ \ c) = (a \rightarrow b) \rightarrow (\overline{a}\  \land (b \vee c))\) \\
B3. 2. Доказать равносильность.\(x_{1} \land x_{2} \land \ \ \ ...\ \  \land x_{n} \rightarrow y \equiv x_{1} \rightarrow (x_{2} \rightarrow (\ \ ...\ \  \rightarrow (x_{n} \rightarrow y)\ \ ,,.\ \ ))\) \\
C1. 9. Составить РКС для формулы \(x \rightarrow (x \rightarrow y) \rightarrow (\overline{x} \rightarrow y)\ \) \\
C2. 4. Булевы функции заданы последовательностью их значений при лексикографическом упорядочении аргументов. Найти все базисы, которые можно составить из следующих функций. \(f = (10100101)\) \\
C3. 6. Привести к предваренной нормальной форме \(\forall x(A(x) \supset B(y))Ђ\) \\

\end{tabular}
\vspace{1cm}


\begin{tabular}{m{17cm}}
\textbf{67-вариант}
\newline

T1. 6. Производные правила вывода \emph{О.с. Правило силлогизма, правило контрпозиции,правило снятия двойного отрицания} \\
T2. 2. Исчисления высказываний и понятие формулы исчиления высказываний \emph{О.с. Определения доказуемой формулы. Аксиомы исчисления высказываний. Правыла выводимости} \\
A1. 8. Доказать тоджество. \(A\bigcap(B\backslash C) = (A\bigcap B)\backslash(A\bigcap C) = (A\bigcap B)\backslash C\) \\
A2. 10. Найти все подформулы. \((x \rightarrow y) \land (x \rightarrow \overline{y}) \rightarrow \overline{x} \leftrightarrow \overline{y}\) \\
A3. 6. Составить таблица истинности. \((x \rightarrow y) \land (x \rightarrow \overline{y}) \rightarrow \overline{x} \leftrightarrow \overline{y}\) \\
B1. 2. Для следующих формул найти ДНФ и КНФ. \(\ x_{1} \land (x_{2} \vee (x_{1} \rightarrow x_{3}))\) \\
B2. 3. Упростить формулу алгебра логики. \(A(x,\ \ y) = (x \leftrightarrow y) \land (x \vee y)\) \\
B3. 7. Доказать равносильность. \(x \vee \left( \overline{x\ } \land \overline{y} \right) \equiv x \vee \overline{y}\) \\
C1. 6. Составить РКС для формулы \((x \rightarrow y) \land (x \rightarrow \overline{y}) \rightarrow \overline{x} \leftrightarrow \overline{y}\) \\
C2. 10 Булевы функции заданы последовательностью их значений при лексикографическом упорядочении аргументов. Найти все базисы, которые можно составить из следующих функций. \(f = (11001000)\) \\
C3. 2. Пусть \(A,\ \ B\)- некоторые формулы исчисления предикатов, причем переменная\(x\) не входит в \(B\). Доказать следующие соотношения. \(\forall xA \vee B \rightleftarrows \forall x(A \vee B)\); \\

\end{tabular}
\vspace{1cm}


\begin{tabular}{m{17cm}}
\textbf{68-вариант}
\newline

T1. 1. Алгебра высказываний и понятие формула алгебры высказываний \emph{О.с. Равносильные формулы алгебры логики и их свойтсва} \\
T2. 7. Алгебра высказываний и понятие формула алгебры высказываний \emph{О.с. Равносильные формулы алгебры логики и их свойтсва} \\
A1. 3. Доказать тоджество. \(A\backslash(B\bigcup C) = (A\backslash B)\bigcup(A\backslash C)\) \\
A2. 2. Найти все подформулы. \(\left( (р \land q) \vee q \right) \vee (q \rightarrow p)\) \\
A3. 2. Составить таблица истинности. \((x \rightarrow y) \land (x \rightarrow \overline{y}) \rightarrow \overline{x}\) \\
B1. 3. Для следующих формул найти СКНФ \(\overline{(a \rightarrow c)} \rightarrow \left( (b \rightarrow c) \rightarrow (a \vee b \rightarrow c) \right)\) \\
B2. 5. Упростить формулу алгебра логики. \((x \rightarrow y) \land (x \rightarrow \overline{y}) \rightarrow \overline{x}\) \\
B3. 4. Доказать равносильность. \((x \vee y) \land (x \vee \overline{y}) \equiv x\) \\
C1. 7. Составить РКС для формулы \(А(х,у,z) = (x \rightarrow y) \land z \rightarrow (x \rightarrow z)\) \\
C2. 7. Булевы функции заданы последовательностью их значений при лексикографическом упорядочении аргументов. Найти все базисы, которые можно составить из следующих функций. \(f = (01101001)\) \\
C3. 5. Привести к предваренной нормальной форме \(\forall x(\neg A(x) \supset \exists yB(y)) \supset B(y) \vee A(x)\) \\

\end{tabular}
\vspace{1cm}


\begin{tabular}{m{17cm}}
\textbf{69-вариант}
\newline

T1. 2. Понятие выводимости формулы из совокупности формулы \emph{О.с. Теорема дедукции. Обобщенная теорема дедукции.} \\
T2. 10. Исчисления высказываний и понятие формулы исчиления высказываний \emph{О.с. Определения доказуемой формулы. Аксиомы исчисления высказываний. Правыла выводимости} \\
A1. 1. Доказать тоджество. \(A\bigcap(\overline{A}) = \varnothing\) \\
A2. 9. Найти все подформулы. \((x \rightarrow y) \land (x \rightarrow \overline{y}) \rightarrow \overline{x} \vee y\) \\
A3. 1. Составить таблица истинности. \(\left( (р \land q) \vee q \right) \vee (q \rightarrow p)\) \\
B1. 4. Для следующих формул найти КНФ. \((х \rightarrow у) \rightarrow (\overline{х} \rightarrow \overline{у)}\). \\
B2. 1. Доказать тождественно ложность формул. \(F\left( р_{1},р_{2} \right) = \overline{p_{1} \rightarrow (p_{2} \rightarrow p_{1})}\) \\
B3. 3. Доказать, что двойственные формулы.\(f(х,\ \ y) = x \land y\) и \(f^{*}(х,\ \ y) = x \vee y.\) \\
C1. 4. Составить РКС для формулы \((х \rightarrow у) \rightarrow (\overline{х} \rightarrow \overline{у)}\) \\
C2. 8. Булевы функции заданы последовательностью их значений при лексикографическом упорядочении аргументов. Найти все базисы, которые можно составить из следующих функций. \(f = (10011011)\) \\
C3. 7. Пусть \(A,\ \ B\)- некоторые формулы исчисления предикатов, причем переменная\(x\) не входит в \(B\). Доказать следующие соотношения. \(\exists xA \vee B \rightleftarrows \exists x(A \vee B)\); \\

\end{tabular}
\vspace{1cm}


\begin{tabular}{m{17cm}}
\textbf{70-вариант}
\newline

T1. 9. Алгебра высказываний и понятие формула алгебры высказываний \emph{О.с. Равносильные формулы алгебры логики и их свойтсва} \\
T2. 8. Понятие выводимости формулы из совокупности формулы \emph{О.с. Теорема дедукции. Обобщенная теорема дедукции.} \\
A1. 7. Доказать тоджество. \((A\bigcup B)\bigcap A = (A\bigcap B)\bigcup A = A\) \\
A2. 7. Найти все подформулы. \((x \rightarrow z) \rightarrow (y \rightarrow z) \rightarrow ((x \vee y) \rightarrow z)\) \\
A3. 4. Составить таблица истинности. \((x \rightarrow y) \land (x \rightarrow \overline{y}) \rightarrow \overline{x} \vee y\) \\
B1. 8. Для следующих формул найти ДНФ и КНФ. \(х \land у \rightarrow (z \vee у \rightarrow z)\) \\
B2. 2. Упростить формулу алгебра логики. \(А(х,у,z) = (x \rightarrow y) \land z \rightarrow (x \rightarrow z)\) \\
B3. 9. Доказать, что двойственные формулы. \(f(х,\ \ y) = x \leftrightarrow y\) и \(f^{*}(х) = \overline{x \leftrightarrow y}.\) \\
C1. 8. Составить РКС для формулы \(A(a,\ \ b,\ \ c) = (a \rightarrow b) \rightarrow (\overline{a}\  \land (b \vee c))\) \\
C2. 5. Булевы функции заданы последовательностью их значений при лексикографическом упорядочении аргументов. Найти все базисы, которые можно составить из следующих функций. \(f = (01110001)\) \\
C3. 1. Пусть \(A,\ \ B\)- некоторые формулы исчисления предикатов, причем переменная\(x\) не входит в \(B\). Доказать следующие соотношения. \(\forall xAЂ\); \\

\end{tabular}
\vspace{1cm}


\begin{tabular}{m{17cm}}
\textbf{71-вариант}
\newline

T1. 10. Производные правила вывода \emph{О.с. Правило силлогизма, правило контрпозиции,правило снятия двойного отрицания} \\
T2. 7. Алгебра высказываний и понятие формула алгебры высказываний \emph{О.с. Равносильные формулы алгебры логики и их свойтсва} \\
A1. 2. Известно, что высказывание \(A \rightarrow B\) ложно. Что можно сказать обистинности \(A\) и \(B\) ? \\
A2. 10. Найти все подформулы. \((x \rightarrow y) \land (x \rightarrow \overline{y}) \rightarrow \overline{x} \leftrightarrow \overline{y}\) \\
A3. 5. Составить таблица истинности. \((x \rightarrow \overline{y}) \rightarrow \overline{x}\) \\
B1. 3. Для следующих формул найти СКНФ \(\overline{(a \rightarrow c)} \rightarrow \left( (b \rightarrow c) \rightarrow (a \vee b \rightarrow c) \right)\) \\
B2. 7. Упростить формулу алгебра логики. \(А(х,у,z) = (x \rightarrow y) \land z \rightarrow (x \rightarrow z)\) \\
B3. 1. Найти \(x\), если \(\left( \overline{x \vee a} \right) \vee \left( \overline{x \vee \overline{a}} \right) \equiv b\) \\
C1. 6. Составить РКС для формулы \((x \rightarrow y) \land (x \rightarrow \overline{y}) \rightarrow \overline{x} \leftrightarrow \overline{y}\) \\
C2. 6. Булевы функции заданы последовательностью их значений при лексикографическом упорядочении аргументов. Найти все базисы, которые можно составить из следующих функций. \(f = (00010101)\) \\
C3. 3. Пусть \(A,\ \ B\)- некоторые формулы исчисления предикатов, причем переменная\(x\) не входит в \(B\). Доказать следующие соотношения. \(\exists xA \vee B \rightleftarrows \exists x(A \vee B)\) \\

\end{tabular}
\vspace{1cm}


\begin{tabular}{m{17cm}}
\textbf{72-вариант}
\newline

T1. 8. Логика предикатов. \emph{О.с. Понятие предиката, понятие формулы логики предикатов, равносильные формулы логики предикатов} \\
T2. 2. Исчисления высказываний и понятие формулы исчиления высказываний \emph{О.с. Определения доказуемой формулы. Аксиомы исчисления высказываний. Правыла выводимости} \\
A1. 6. Доказать тождество. \(\overline{A\bigcap B} = \overline{A}\bigcup\overline{B}\) \\
A2. 9. Найти все подформулы. \((x \rightarrow y) \land (x \rightarrow \overline{y}) \rightarrow \overline{x} \vee y\) \\
A3. 7. Составить таблица истинности. \((x \rightarrow z) \rightarrow (y \rightarrow z) \rightarrow ((x \vee y) \rightarrow z)\) \\
B1. 6. Для следующих формул найти ДНФ и КНФ. \(\ x_{1} \land (x_{2} \vee (x_{1} \rightarrow x_{3}))\) \\
B2. 8. Упростить формулу алгебра логики. \(А(х,у,z) = (x \rightarrow y) \land z \rightarrow (x \rightarrow z)\) \\
B3. 6. Доказать, что двойственные формулы. \(f(х,\ \ y) = x \land y\) и \(f^{*}(х) = x \vee y.\) \\
C1. 7. Составить РКС для формулы \(А(х,у,z) = (x \rightarrow y) \land z \rightarrow (x \rightarrow z)\) \\
C2. 5. Булевы функции заданы последовательностью их значений при лексикографическом упорядочении аргументов. Найти все базисы, которые можно составить из следующих функций. \(f = (01110001)\) \\
C3. 4. Привести к предваренной нормальной форме \(\forall xB(x) \supset \exists y(A(y) \supset B(x))\) \\

\end{tabular}
\vspace{1cm}


\begin{tabular}{m{17cm}}
\textbf{73-вариант}
\newline

T1. 2. Понятие выводимости формулы из совокупности формулы \emph{О.с. Теорема дедукции. Обобщенная теорема дедукции.} \\
T2. 5. Нормальные формы \emph{О.с. Дизъюнктивная, конъюнктивная и совершенная дизъюнктивная, совершенна конъюнктивная нормальная форма} \\
A1. 4. Доказать тоджество. \(A\bigcap(B\bigcup C) = (A\bigcap B)\bigcup(B\bigcap C)\) \\
A2. 6. Найти все подформулы. \(\overline{(a \rightarrow c)} \rightarrow \left( (b \rightarrow c) \rightarrow (a \vee b \rightarrow c) \right)\) \\
A3. 1. Составить таблица истинности. \(\left( (р \land q) \vee q \right) \vee (q \rightarrow p)\) \\
B1. 8. Для следующих формул найти ДНФ и КНФ. \(х \land у \rightarrow (z \vee у \rightarrow z)\) \\
B2. 4. Упростить формулу алгебра логики. \((х \rightarrow у) \rightarrow (\overline{х} \rightarrow \overline{у)}\) \\
B3. 9. Доказать, что двойственные формулы. \(f(х,\ \ y) = x \leftrightarrow y\) и \(f^{*}(х) = \overline{x \leftrightarrow y}.\) \\
C1. 2. Составить РКС для формулы \(A(a,\ \ b,\ \ c) = (a \rightarrow b) \rightarrow (\overline{a}\  \land (b \vee c))\) \\
C2. 9. Булевы функции заданы последовательностью их значений при лексикографическом упорядочении аргументов. Найти все базисы, которые можно составить из следующих функций. \(f = (10010110)\) \\
C3. 6. Привести к предваренной нормальной форме \(\forall x(A(x) \supset B(y))Ђ\) \\

\end{tabular}
\vspace{1cm}


\begin{tabular}{m{17cm}}
\textbf{74-вариант}
\newline

T1. 9. Алгебра высказываний и понятие формула алгебры высказываний \emph{О.с. Равносильные формулы алгебры логики и их свойтсва} \\
T2. 6. Логика предикатов. \emph{О.с. Понятие предиката, понятие формулы логики предикатов, равносильные формулы логики предикатов} \\
A1. 1. Доказать тоджество. \(A\bigcap(\overline{A}) = \varnothing\) \\
A2. 5. Найти все подформулы. \((х \vee у) \rightarrow \left( х \land \overline{у \vee х \rightarrow у} \right)\) \\
A3. 6. Составить таблица истинности. \((x \rightarrow y) \land (x \rightarrow \overline{y}) \rightarrow \overline{x} \leftrightarrow \overline{y}\) \\
B1. 1. Для следующих формул найти ДНФ и КНФ. \(х \land у \rightarrow (z \vee у \rightarrow z)\) \\
B2. 5. Упростить формулу алгебра логики. \((x \rightarrow y) \land (x \rightarrow \overline{y}) \rightarrow \overline{x}\) \\
B3. 3. Доказать, что двойственные формулы.\(f(х,\ \ y) = x \land y\) и \(f^{*}(х,\ \ y) = x \vee y.\) \\
C1. 5. Составить РКС для формулы \((x \land y \rightarrow z) \rightarrow (x \rightarrow (y \rightarrow z))\) \\
C2. 7. Булевы функции заданы последовательностью их значений при лексикографическом упорядочении аргументов. Найти все базисы, которые можно составить из следующих функций. \(f = (01101001)\) \\
C3. 2. Пусть \(A,\ \ B\)- некоторые формулы исчисления предикатов, причем переменная\(x\) не входит в \(B\). Доказать следующие соотношения. \(\forall xA \vee B \rightleftarrows \forall x(A \vee B)\); \\

\end{tabular}
\vspace{1cm}


\begin{tabular}{m{17cm}}
\textbf{75-вариант}
\newline

T1. 7. Некоторые приложения алгебры логики \emph{О.с. Реле контактные схемы, применения алгебры логики в решении логических задач} \\
T2. 8. Понятие выводимости формулы из совокупности формулы \emph{О.с. Теорема дедукции. Обобщенная теорема дедукции.} \\
A1. 8. Доказать тоджество. \(A\bigcap(B\backslash C) = (A\bigcap B)\backslash(A\bigcap C) = (A\bigcap B)\backslash C\) \\
A2. 8. Найти все подформулы. \((х \rightarrow у) \rightarrow (\overline{х} \rightarrow \overline{у)}\). \\
A3. 3. Составить таблица истинности. \((x \rightarrow z) \rightarrow (y \rightarrow z) \rightarrow ((x \vee y) \rightarrow z)\) \\
B1. 9. Для следующих формул найти СКНФ \((z \rightarrow x \land y))\) \\
B2. 9. Упростить формул. \(A(a,\ \ b,\ \ c) = (a \rightarrow b) \rightarrow (\overline{a}\  \land (b \vee c))\) \\
B3. 4. Доказать равносильность. \((x \vee y) \land (x \vee \overline{y}) \equiv x\) \\
C1. 10. Составить РКС для формулы \(\overline{(a \rightarrow c)} \rightarrow \left( (b \rightarrow c) \rightarrow (a \vee b \rightarrow c) \right)\) \\
C2. 4. Булевы функции заданы последовательностью их значений при лексикографическом упорядочении аргументов. Найти все базисы, которые можно составить из следующих функций. \(f = (10100101)\) \\
C3. 4. Привести к предваренной нормальной форме \(\forall xB(x) \supset \exists y(A(y) \supset B(x))\) \\

\end{tabular}
\vspace{1cm}


\begin{tabular}{m{17cm}}
\textbf{76-вариант}
\newline

T1. 5. Функции алгебры логики \emph{О.с. Алгебра Буля. Полная система функции} \\
T2. 4. Понятие выводимости формулы из совокупности формулы \emph{О.с. Теорема дедукции. Обобщенная теорема дедукции.} \\
A1. 7. Доказать тоджество. \((A\bigcup B)\bigcap A = (A\bigcap B)\bigcup A = A\) \\
A2. 4. Найти все подформулы. \((x \rightarrow y) \land (x \rightarrow \overline{y}) \rightarrow \overline{x}\) \\
A3. 8. Составить таблица истинности. \((x \rightarrow z) \rightarrow (y \rightarrow z)\) \\
B1. 7. Для следующих формул найти ДНФ и КНФ. \(\left( х_{1} \land \overline{х_{2}} \right) \vee х_{3}\) \\
B2. 2. Упростить формулу алгебра логики. \(А(х,у,z) = (x \rightarrow y) \land z \rightarrow (x \rightarrow z)\) \\
B3. 5. Найти двойственную формулу к формуле \(f(х,\ \ y,\ \ z) = xy \vee yz \vee xz\) \\
C1. 9. Составить РКС для формулы \(x \rightarrow (x \rightarrow y) \rightarrow (\overline{x} \rightarrow y)\ \) \\
C2. 8. Булевы функции заданы последовательностью их значений при лексикографическом упорядочении аргументов. Найти все базисы, которые можно составить из следующих функций. \(f = (10011011)\) \\
C3. 9. Пусть \(A,\ \ B\)- некоторые формулы исчисления предикатов, причем переменная\(x\) не входит в \(B\). Доказать следующие соотношения. \(\exists xB \rightleftarrows B\)
 \\

\end{tabular}
\vspace{1cm}


\begin{tabular}{m{17cm}}
\textbf{77-вариант}
\newline

T1. 1. Алгебра высказываний и понятие формула алгебры высказываний \emph{О.с. Равносильные формулы алгебры логики и их свойтсва} \\
T2. 1. Некоторые приложения алгебры логики \emph{О.с. Реле контактные схемы, применения алгебры логики в решении логических задач} \\
A1. 5. Доказать тоджество. \(A\bigcup(B\bigcap C) = (A\bigcup B)\bigcap(A\bigcup C)\) \\
A2. 7. Найти все подформулы. \((x \rightarrow z) \rightarrow (y \rightarrow z) \rightarrow ((x \vee y) \rightarrow z)\) \\
A3. 5. Составить таблица истинности. \((x \rightarrow \overline{y}) \rightarrow \overline{x}\) \\
B1. 10. Для следующих формул найти ДНФ и КНФ. \(\overline{x \vee (x_{2} \rightarrow x_{1})}\) \\
B2. 8. Упростить формулу алгебра логики. \(А(х,у,z) = (x \rightarrow y) \land z \rightarrow (x \rightarrow z)\) \\
B3. 7. Доказать равносильность. \(x \vee \left( \overline{x\ } \land \overline{y} \right) \equiv x \vee \overline{y}\) \\
C1. 8. Составить РКС для формулы \(A(a,\ \ b,\ \ c) = (a \rightarrow b) \rightarrow (\overline{a}\  \land (b \vee c))\) \\
C2. 10 Булевы функции заданы последовательностью их значений при лексикографическом упорядочении аргументов. Найти все базисы, которые можно составить из следующих функций. \(f = (11001000)\) \\
C3. 3. Пусть \(A,\ \ B\)- некоторые формулы исчисления предикатов, причем переменная\(x\) не входит в \(B\). Доказать следующие соотношения. \(\exists xA \vee B \rightleftarrows \exists x(A \vee B)\) \\

\end{tabular}
\vspace{1cm}


\begin{tabular}{m{17cm}}
\textbf{78-вариант}
\newline

T1. 3. Нормальные формы \emph{О.с. Дизъюнктивная, конъюнктивная и совершенная дизъюнктивная, совершенна конъюнктивная нормальная форма} \\
T2. 10. Исчисления высказываний и понятие формулы исчиления высказываний \emph{О.с. Определения доказуемой формулы. Аксиомы исчисления высказываний. Правыла выводимости} \\
A1. 2. Известно, что высказывание \(A \rightarrow B\) ложно. Что можно сказать обистинности \(A\) и \(B\) ? \\
A2. 3. Найти все подформулы. \(х \land у \rightarrow (z \vee у \rightarrow z)\) \\
A3. 7. Составить таблица истинности. \((x \rightarrow z) \rightarrow (y \rightarrow z) \rightarrow ((x \vee y) \rightarrow z)\) \\
B1. 5. Для следующих формул найти СДНФ \(\overline{\overline{x \vee y} \rightarrow \overline{x \land y}}\) \\
B2. 7. Упростить формулу алгебра логики. \(А(х,у,z) = (x \rightarrow y) \land z \rightarrow (x \rightarrow z)\) \\
B3. 6. Доказать, что двойственные формулы. \(f(х,\ \ y) = x \land y\) и \(f^{*}(х) = x \vee y.\) \\
C1. 3. Составить РКС для формулы \((x \rightarrow z) \rightarrow (y \rightarrow z) \rightarrow ((x \vee y) \rightarrow z)\) \\
C2. 2. Булевы функции заданы последовательностью их значений при лексикографическом упорядочении аргументов. Найти все базисы, которые можно составить из следующих функций. \(f = (11101000)\) \\
C3. 1. Пусть \(A,\ \ B\)- некоторые формулы исчисления предикатов, причем переменная\(x\) не входит в \(B\). Доказать следующие соотношения. \(\forall xAЂ\); \\

\end{tabular}
\vspace{1cm}


\begin{tabular}{m{17cm}}
\textbf{79-вариант}
\newline

T1. 4. Исчисления высказываний и понятие формулы исчиления высказываний \emph{О.с. Определения доказуемой формулы. Аксиомы исчисления высказываний. Правыла выводимости} \\
T2. 9. Нормальные формы \emph{О.с. Дизъюнктивная, конъюнктивная и совершенная дизъюнктивная, совершенна конъюнктивная нормальная форма} \\
A1. 3. Доказать тоджество. \(A\backslash(B\bigcup C) = (A\backslash B)\bigcup(A\backslash C)\) \\
A2. 1. Найти все подформулы. \(((a_{0} \rightarrow a_{1}) \land ((a_{1} \rightarrow a_{2}) \rightarrow (\overline{a_{0}} \vee a_{2})).\) \\
A3. 4. Составить таблица истинности. \((x \rightarrow y) \land (x \rightarrow \overline{y}) \rightarrow \overline{x} \vee y\) \\
B1. 2. Для следующих формул найти ДНФ и КНФ. \(\ x_{1} \land (x_{2} \vee (x_{1} \rightarrow x_{3}))\) \\
B2. 3. Упростить формулу алгебра логики. \(A(x,\ \ y) = (x \leftrightarrow y) \land (x \vee y)\) \\
B3. 1. Найти \(x\), если \(\left( \overline{x \vee a} \right) \vee \left( \overline{x \vee \overline{a}} \right) \equiv b\) \\
C1. 1. Составить РКС для формулы \(F(a,\ \ b,\ \ c) = (a \vee b) \rightarrow (a \land b) \vee c.\) \\
C2. 1. Булевы функции заданы последовательностью их значений при лексикографическом упорядочении аргументов. Найти все базисы, которые можно составить из следующих функций.\(f = (00111100)\) \\
C3. 7. Пусть \(A,\ \ B\)- некоторые формулы исчисления предикатов, причем переменная\(x\) не входит в \(B\). Доказать следующие соотношения. \(\exists xA \vee B \rightleftarrows \exists x(A \vee B)\); \\

\end{tabular}
\vspace{1cm}


\begin{tabular}{m{17cm}}
\textbf{80-вариант}
\newline

T1. 6. Производные правила вывода \emph{О.с. Правило силлогизма, правило контрпозиции,правило снятия двойного отрицания} \\
T2. 3. Функции алгебры логики \emph{О.с. Алгебра Буля. Полная система функции} \\
A1. 6. Доказать тождество. \(\overline{A\bigcap B} = \overline{A}\bigcup\overline{B}\) \\
A2. 2. Найти все подформулы. \(\left( (р \land q) \vee q \right) \vee (q \rightarrow p)\) \\
A3. 2. Составить таблица истинности. \((x \rightarrow y) \land (x \rightarrow \overline{y}) \rightarrow \overline{x}\) \\
B1. 4. Для следующих формул найти КНФ. \((х \rightarrow у) \rightarrow (\overline{х} \rightarrow \overline{у)}\). \\
B2. 1. Доказать тождественно ложность формул. \(F\left( р_{1},р_{2} \right) = \overline{p_{1} \rightarrow (p_{2} \rightarrow p_{1})}\) \\
B3. 2. Доказать равносильность.\(x_{1} \land x_{2} \land \ \ \ ...\ \  \land x_{n} \rightarrow y \equiv x_{1} \rightarrow (x_{2} \rightarrow (\ \ ...\ \  \rightarrow (x_{n} \rightarrow y)\ \ ,,.\ \ ))\) \\
C1. 4. Составить РКС для формулы \((х \rightarrow у) \rightarrow (\overline{х} \rightarrow \overline{у)}\) \\
C2. 3. Булевы функции заданы последовательностью их значений при лексикографическом упорядочении аргументов. Найти все базисы, которые можно составить из следующих функций. \(f = (01010111)\) \\
C3. 5. Привести к предваренной нормальной форме \(\forall x(\neg A(x) \supset \exists yB(y)) \supset B(y) \vee A(x)\) \\

\end{tabular}
\vspace{1cm}


\begin{tabular}{m{17cm}}
\textbf{81-вариант}
\newline

T1. 4. Исчисления высказываний и понятие формулы исчиления высказываний \emph{О.с. Определения доказуемой формулы. Аксиомы исчисления высказываний. Правыла выводимости} \\
T2. 4. Понятие выводимости формулы из совокупности формулы \emph{О.с. Теорема дедукции. Обобщенная теорема дедукции.} \\
A1. 5. Доказать тоджество. \(A\bigcup(B\bigcap C) = (A\bigcup B)\bigcap(A\bigcup C)\) \\
A2. 6. Найти все подформулы. \(\overline{(a \rightarrow c)} \rightarrow \left( (b \rightarrow c) \rightarrow (a \vee b \rightarrow c) \right)\) \\
A3. 9. Составить таблица истинности. \((x \rightarrow y) \land (x \rightarrow \overline{y}) \rightarrow \overline{x}\) \\
B1. 4. Для следующих формул найти КНФ. \((х \rightarrow у) \rightarrow (\overline{х} \rightarrow \overline{у)}\). \\
B2. 6. Упростить формул. \(A(a,\ \ b,\ \ c) = (a \rightarrow b) \rightarrow (\overline{a}\  \land (b \vee c))\) \\
B3. 8. Доказать равносильность. \(x \rightarrow \left( y \rightarrow \overline{z} \right) \equiv x \land y \rightarrow \overline{z}\) \\
C1. 2. Составить РКС для формулы \(A(a,\ \ b,\ \ c) = (a \rightarrow b) \rightarrow (\overline{a}\  \land (b \vee c))\) \\
C2. 5. Булевы функции заданы последовательностью их значений при лексикографическом упорядочении аргументов. Найти все базисы, которые можно составить из следующих функций. \(f = (01110001)\) \\
C3. 8. Пусть \(A,\ \ B\)- некоторые формулы исчисления предикатов, причем переменная\(x\) не входит в \(B\). Доказать следующие соотношения. \(\forall xB \rightleftarrows B\); \\

\end{tabular}
\vspace{1cm}


\begin{tabular}{m{17cm}}
\textbf{82-вариант}
\newline

T1. 8. Логика предикатов. \emph{О.с. Понятие предиката, понятие формулы логики предикатов, равносильные формулы логики предикатов} \\
T2. 2. Исчисления высказываний и понятие формулы исчиления высказываний \emph{О.с. Определения доказуемой формулы. Аксиомы исчисления высказываний. Правыла выводимости} \\
A1. 4. Доказать тоджество. \(A\bigcap(B\bigcup C) = (A\bigcap B)\bigcup(B\bigcap C)\) \\
A2. 2. Найти все подформулы. \(\left( (р \land q) \vee q \right) \vee (q \rightarrow p)\) \\
A3. 3. Составить таблица истинности. \((x \rightarrow z) \rightarrow (y \rightarrow z) \rightarrow ((x \vee y) \rightarrow z)\) \\
B1. 8. Для следующих формул найти ДНФ и КНФ. \(х \land у \rightarrow (z \vee у \rightarrow z)\) \\
B2. 4. Упростить формулу алгебра логики. \((х \rightarrow у) \rightarrow (\overline{х} \rightarrow \overline{у)}\) \\
B3. 4. Доказать равносильность. \((x \vee y) \land (x \vee \overline{y}) \equiv x\) \\
C1. 6. Составить РКС для формулы \((x \rightarrow y) \land (x \rightarrow \overline{y}) \rightarrow \overline{x} \leftrightarrow \overline{y}\) \\
C2. 2. Булевы функции заданы последовательностью их значений при лексикографическом упорядочении аргументов. Найти все базисы, которые можно составить из следующих функций. \(f = (11101000)\) \\
C3. 6. Привести к предваренной нормальной форме \(\forall x(A(x) \supset B(y))Ђ\) \\

\end{tabular}
\vspace{1cm}


\begin{tabular}{m{17cm}}
\textbf{83-вариант}
\newline

T1. 7. Некоторые приложения алгебры логики \emph{О.с. Реле контактные схемы, применения алгебры логики в решении логических задач} \\
T2. 5. Нормальные формы \emph{О.с. Дизъюнктивная, конъюнктивная и совершенная дизъюнктивная, совершенна конъюнктивная нормальная форма} \\
A1. 8. Доказать тоджество. \(A\bigcap(B\backslash C) = (A\bigcap B)\backslash(A\bigcap C) = (A\bigcap B)\backslash C\) \\
A2. 10. Найти все подформулы. \((x \rightarrow y) \land (x \rightarrow \overline{y}) \rightarrow \overline{x} \leftrightarrow \overline{y}\) \\
A3. 5. Составить таблица истинности. \((x \rightarrow \overline{y}) \rightarrow \overline{x}\) \\
B1. 10. Для следующих формул найти ДНФ и КНФ. \(\overline{x \vee (x_{2} \rightarrow x_{1})}\) \\
B2. 3. Упростить формулу алгебра логики. \(A(x,\ \ y) = (x \leftrightarrow y) \land (x \vee y)\) \\
B3. 2. Доказать равносильность.\(x_{1} \land x_{2} \land \ \ \ ...\ \  \land x_{n} \rightarrow y \equiv x_{1} \rightarrow (x_{2} \rightarrow (\ \ ...\ \  \rightarrow (x_{n} \rightarrow y)\ \ ,,.\ \ ))\) \\
C1. 1. Составить РКС для формулы \(F(a,\ \ b,\ \ c) = (a \vee b) \rightarrow (a \land b) \vee c.\) \\
C2. 8. Булевы функции заданы последовательностью их значений при лексикографическом упорядочении аргументов. Найти все базисы, которые можно составить из следующих функций. \(f = (10011011)\) \\
C3. 2. Пусть \(A,\ \ B\)- некоторые формулы исчисления предикатов, причем переменная\(x\) не входит в \(B\). Доказать следующие соотношения. \(\forall xA \vee B \rightleftarrows \forall x(A \vee B)\); \\

\end{tabular}
\vspace{1cm}


\begin{tabular}{m{17cm}}
\textbf{84-вариант}
\newline

T1. 10. Производные правила вывода \emph{О.с. Правило силлогизма, правило контрпозиции,правило снятия двойного отрицания} \\
T2. 3. Функции алгебры логики \emph{О.с. Алгебра Буля. Полная система функции} \\
A1. 6. Доказать тождество. \(\overline{A\bigcap B} = \overline{A}\bigcup\overline{B}\) \\
A2. 7. Найти все подформулы. \((x \rightarrow z) \rightarrow (y \rightarrow z) \rightarrow ((x \vee y) \rightarrow z)\) \\
A3. 9. Составить таблица истинности. \((x \rightarrow y) \land (x \rightarrow \overline{y}) \rightarrow \overline{x}\) \\
B1. 9. Для следующих формул найти СКНФ \((z \rightarrow x \land y))\) \\
B2. 5. Упростить формулу алгебра логики. \((x \rightarrow y) \land (x \rightarrow \overline{y}) \rightarrow \overline{x}\) \\
B3. 9. Доказать, что двойственные формулы. \(f(х,\ \ y) = x \leftrightarrow y\) и \(f^{*}(х) = \overline{x \leftrightarrow y}.\) \\
C1. 8. Составить РКС для формулы \(A(a,\ \ b,\ \ c) = (a \rightarrow b) \rightarrow (\overline{a}\  \land (b \vee c))\) \\
C2. 1. Булевы функции заданы последовательностью их значений при лексикографическом упорядочении аргументов. Найти все базисы, которые можно составить из следующих функций.\(f = (00111100)\) \\
C3. 9. Пусть \(A,\ \ B\)- некоторые формулы исчисления предикатов, причем переменная\(x\) не входит в \(B\). Доказать следующие соотношения. \(\exists xB \rightleftarrows B\)
 \\

\end{tabular}
\vspace{1cm}


\begin{tabular}{m{17cm}}
\textbf{85-вариант}
\newline

T1. 3. Нормальные формы \emph{О.с. Дизъюнктивная, конъюнктивная и совершенная дизъюнктивная, совершенна конъюнктивная нормальная форма} \\
T2. 10. Исчисления высказываний и понятие формулы исчиления высказываний \emph{О.с. Определения доказуемой формулы. Аксиомы исчисления высказываний. Правыла выводимости} \\
A1. 2. Известно, что высказывание \(A \rightarrow B\) ложно. Что можно сказать обистинности \(A\) и \(B\) ? \\
A2. 3. Найти все подформулы. \(х \land у \rightarrow (z \vee у \rightarrow z)\) \\
A3. 7. Составить таблица истинности. \((x \rightarrow z) \rightarrow (y \rightarrow z) \rightarrow ((x \vee y) \rightarrow z)\) \\
B1. 5. Для следующих формул найти СДНФ \(\overline{\overline{x \vee y} \rightarrow \overline{x \land y}}\) \\
B2. 1. Доказать тождественно ложность формул. \(F\left( р_{1},р_{2} \right) = \overline{p_{1} \rightarrow (p_{2} \rightarrow p_{1})}\) \\
B3. 8. Доказать равносильность. \(x \rightarrow \left( y \rightarrow \overline{z} \right) \equiv x \land y \rightarrow \overline{z}\) \\
C1. 10. Составить РКС для формулы \(\overline{(a \rightarrow c)} \rightarrow \left( (b \rightarrow c) \rightarrow (a \vee b \rightarrow c) \right)\) \\
C2. 9. Булевы функции заданы последовательностью их значений при лексикографическом упорядочении аргументов. Найти все базисы, которые можно составить из следующих функций. \(f = (10010110)\) \\
C3. 1. Пусть \(A,\ \ B\)- некоторые формулы исчисления предикатов, причем переменная\(x\) не входит в \(B\). Доказать следующие соотношения. \(\forall xAЂ\); \\

\end{tabular}
\vspace{1cm}


\begin{tabular}{m{17cm}}
\textbf{86-вариант}
\newline

T1. 6. Производные правила вывода \emph{О.с. Правило силлогизма, правило контрпозиции,правило снятия двойного отрицания} \\
T2. 1. Некоторые приложения алгебры логики \emph{О.с. Реле контактные схемы, применения алгебры логики в решении логических задач} \\
A1. 3. Доказать тоджество. \(A\backslash(B\bigcup C) = (A\backslash B)\bigcup(A\backslash C)\) \\
A2. 1. Найти все подформулы. \(((a_{0} \rightarrow a_{1}) \land ((a_{1} \rightarrow a_{2}) \rightarrow (\overline{a_{0}} \vee a_{2})).\) \\
A3. 2. Составить таблица истинности. \((x \rightarrow y) \land (x \rightarrow \overline{y}) \rightarrow \overline{x}\) \\
B1. 2. Для следующих формул найти ДНФ и КНФ. \(\ x_{1} \land (x_{2} \vee (x_{1} \rightarrow x_{3}))\) \\
B2. 2. Упростить формулу алгебра логики. \(А(х,у,z) = (x \rightarrow y) \land z \rightarrow (x \rightarrow z)\) \\
B3. 1. Найти \(x\), если \(\left( \overline{x \vee a} \right) \vee \left( \overline{x \vee \overline{a}} \right) \equiv b\) \\
C1. 3. Составить РКС для формулы \((x \rightarrow z) \rightarrow (y \rightarrow z) \rightarrow ((x \vee y) \rightarrow z)\) \\
C2. 3. Булевы функции заданы последовательностью их значений при лексикографическом упорядочении аргументов. Найти все базисы, которые можно составить из следующих функций. \(f = (01010111)\) \\
C3. 7. Пусть \(A,\ \ B\)- некоторые формулы исчисления предикатов, причем переменная\(x\) не входит в \(B\). Доказать следующие соотношения. \(\exists xA \vee B \rightleftarrows \exists x(A \vee B)\); \\

\end{tabular}
\vspace{1cm}


\begin{tabular}{m{17cm}}
\textbf{87-вариант}
\newline

T1. 2. Понятие выводимости формулы из совокупности формулы \emph{О.с. Теорема дедукции. Обобщенная теорема дедукции.} \\
T2. 6. Логика предикатов. \emph{О.с. Понятие предиката, понятие формулы логики предикатов, равносильные формулы логики предикатов} \\
A1. 1. Доказать тоджество. \(A\bigcap(\overline{A}) = \varnothing\) \\
A2. 5. Найти все подформулы. \((х \vee у) \rightarrow \left( х \land \overline{у \vee х \rightarrow у} \right)\) \\
A3. 1. Составить таблица истинности. \(\left( (р \land q) \vee q \right) \vee (q \rightarrow p)\) \\
B1. 1. Для следующих формул найти ДНФ и КНФ. \(х \land у \rightarrow (z \vee у \rightarrow z)\) \\
B2. 8. Упростить формулу алгебра логики. \(А(х,у,z) = (x \rightarrow y) \land z \rightarrow (x \rightarrow z)\) \\
B3. 7. Доказать равносильность. \(x \vee \left( \overline{x\ } \land \overline{y} \right) \equiv x \vee \overline{y}\) \\
C1. 7. Составить РКС для формулы \(А(х,у,z) = (x \rightarrow y) \land z \rightarrow (x \rightarrow z)\) \\
C2. 6. Булевы функции заданы последовательностью их значений при лексикографическом упорядочении аргументов. Найти все базисы, которые можно составить из следующих функций. \(f = (00010101)\) \\
C3. 4. Привести к предваренной нормальной форме \(\forall xB(x) \supset \exists y(A(y) \supset B(x))\) \\

\end{tabular}
\vspace{1cm}


\begin{tabular}{m{17cm}}
\textbf{88-вариант}
\newline

T1. 1. Алгебра высказываний и понятие формула алгебры высказываний \emph{О.с. Равносильные формулы алгебры логики и их свойтсва} \\
T2. 9. Нормальные формы \emph{О.с. Дизъюнктивная, конъюнктивная и совершенная дизъюнктивная, совершенна конъюнктивная нормальная форма} \\
A1. 7. Доказать тоджество. \((A\bigcup B)\bigcap A = (A\bigcap B)\bigcup A = A\) \\
A2. 8. Найти все подформулы. \((х \rightarrow у) \rightarrow (\overline{х} \rightarrow \overline{у)}\). \\
A3. 8. Составить таблица истинности. \((x \rightarrow z) \rightarrow (y \rightarrow z)\) \\
B1. 7. Для следующих формул найти ДНФ и КНФ. \(\left( х_{1} \land \overline{х_{2}} \right) \vee х_{3}\) \\
B2. 6. Упростить формул. \(A(a,\ \ b,\ \ c) = (a \rightarrow b) \rightarrow (\overline{a}\  \land (b \vee c))\) \\
B3. 6. Доказать, что двойственные формулы. \(f(х,\ \ y) = x \land y\) и \(f^{*}(х) = x \vee y.\) \\
C1. 4. Составить РКС для формулы \((х \rightarrow у) \rightarrow (\overline{х} \rightarrow \overline{у)}\) \\
C2. 10 Булевы функции заданы последовательностью их значений при лексикографическом упорядочении аргументов. Найти все базисы, которые можно составить из следующих функций. \(f = (11001000)\) \\
C3. 8. Пусть \(A,\ \ B\)- некоторые формулы исчисления предикатов, причем переменная\(x\) не входит в \(B\). Доказать следующие соотношения. \(\forall xB \rightleftarrows B\); \\

\end{tabular}
\vspace{1cm}


\begin{tabular}{m{17cm}}
\textbf{89-вариант}
\newline

T1. 5. Функции алгебры логики \emph{О.с. Алгебра Буля. Полная система функции} \\
T2. 7. Алгебра высказываний и понятие формула алгебры высказываний \emph{О.с. Равносильные формулы алгебры логики и их свойтсва} \\
A1. 4. Доказать тоджество. \(A\bigcap(B\bigcup C) = (A\bigcap B)\bigcup(B\bigcap C)\) \\
A2. 9. Найти все подформулы. \((x \rightarrow y) \land (x \rightarrow \overline{y}) \rightarrow \overline{x} \vee y\) \\
A3. 4. Составить таблица истинности. \((x \rightarrow y) \land (x \rightarrow \overline{y}) \rightarrow \overline{x} \vee y\) \\
B1. 3. Для следующих формул найти СКНФ \(\overline{(a \rightarrow c)} \rightarrow \left( (b \rightarrow c) \rightarrow (a \vee b \rightarrow c) \right)\) \\
B2. 9. Упростить формул. \(A(a,\ \ b,\ \ c) = (a \rightarrow b) \rightarrow (\overline{a}\  \land (b \vee c))\) \\
B3. 5. Найти двойственную формулу к формуле \(f(х,\ \ y,\ \ z) = xy \vee yz \vee xz\) \\
C1. 5. Составить РКС для формулы \((x \land y \rightarrow z) \rightarrow (x \rightarrow (y \rightarrow z))\) \\
C2. 4. Булевы функции заданы последовательностью их значений при лексикографическом упорядочении аргументов. Найти все базисы, которые можно составить из следующих функций. \(f = (10100101)\) \\
C3. 3. Пусть \(A,\ \ B\)- некоторые формулы исчисления предикатов, причем переменная\(x\) не входит в \(B\). Доказать следующие соотношения. \(\exists xA \vee B \rightleftarrows \exists x(A \vee B)\) \\

\end{tabular}
\vspace{1cm}


\begin{tabular}{m{17cm}}
\textbf{90-вариант}
\newline

T1. 9. Алгебра высказываний и понятие формула алгебры высказываний \emph{О.с. Равносильные формулы алгебры логики и их свойтсва} \\
T2. 8. Понятие выводимости формулы из совокупности формулы \emph{О.с. Теорема дедукции. Обобщенная теорема дедукции.} \\
A1. 1. Доказать тоджество. \(A\bigcap(\overline{A}) = \varnothing\) \\
A2. 4. Найти все подформулы. \((x \rightarrow y) \land (x \rightarrow \overline{y}) \rightarrow \overline{x}\) \\
A3. 6. Составить таблица истинности. \((x \rightarrow y) \land (x \rightarrow \overline{y}) \rightarrow \overline{x} \leftrightarrow \overline{y}\) \\
B1. 6. Для следующих формул найти ДНФ и КНФ. \(\ x_{1} \land (x_{2} \vee (x_{1} \rightarrow x_{3}))\) \\
B2. 7. Упростить формулу алгебра логики. \(А(х,у,z) = (x \rightarrow y) \land z \rightarrow (x \rightarrow z)\) \\
B3. 3. Доказать, что двойственные формулы.\(f(х,\ \ y) = x \land y\) и \(f^{*}(х,\ \ y) = x \vee y.\) \\
C1. 9. Составить РКС для формулы \(x \rightarrow (x \rightarrow y) \rightarrow (\overline{x} \rightarrow y)\ \) \\
C2. 7. Булевы функции заданы последовательностью их значений при лексикографическом упорядочении аргументов. Найти все базисы, которые можно составить из следующих функций. \(f = (01101001)\) \\
C3. 5. Привести к предваренной нормальной форме \(\forall x(\neg A(x) \supset \exists yB(y)) \supset B(y) \vee A(x)\) \\

\end{tabular}
\vspace{1cm}


\begin{tabular}{m{17cm}}
\textbf{91-вариант}
\newline

T1. 1. Алгебра высказываний и понятие формула алгебры высказываний \emph{О.с. Равносильные формулы алгебры логики и их свойтсва} \\
T2. 10. Исчисления высказываний и понятие формулы исчиления высказываний \emph{О.с. Определения доказуемой формулы. Аксиомы исчисления высказываний. Правыла выводимости} \\
A1. 6. Доказать тождество. \(\overline{A\bigcap B} = \overline{A}\bigcup\overline{B}\) \\
A2. 3. Найти все подформулы. \(х \land у \rightarrow (z \vee у \rightarrow z)\) \\
A3. 6. Составить таблица истинности. \((x \rightarrow y) \land (x \rightarrow \overline{y}) \rightarrow \overline{x} \leftrightarrow \overline{y}\) \\
B1. 6. Для следующих формул найти ДНФ и КНФ. \(\ x_{1} \land (x_{2} \vee (x_{1} \rightarrow x_{3}))\) \\
B2. 2. Упростить формулу алгебра логики. \(А(х,у,z) = (x \rightarrow y) \land z \rightarrow (x \rightarrow z)\) \\
B3. 7. Доказать равносильность. \(x \vee \left( \overline{x\ } \land \overline{y} \right) \equiv x \vee \overline{y}\) \\
C1. 9. Составить РКС для формулы \(x \rightarrow (x \rightarrow y) \rightarrow (\overline{x} \rightarrow y)\ \) \\
C2. 4. Булевы функции заданы последовательностью их значений при лексикографическом упорядочении аргументов. Найти все базисы, которые можно составить из следующих функций. \(f = (10100101)\) \\
C3. 1. Пусть \(A,\ \ B\)- некоторые формулы исчисления предикатов, причем переменная\(x\) не входит в \(B\). Доказать следующие соотношения. \(\forall xAЂ\); \\

\end{tabular}
\vspace{1cm}


\begin{tabular}{m{17cm}}
\textbf{92-вариант}
\newline

T1. 7. Некоторые приложения алгебры логики \emph{О.с. Реле контактные схемы, применения алгебры логики в решении логических задач} \\
T2. 1. Некоторые приложения алгебры логики \emph{О.с. Реле контактные схемы, применения алгебры логики в решении логических задач} \\
A1. 8. Доказать тоджество. \(A\bigcap(B\backslash C) = (A\bigcap B)\backslash(A\bigcap C) = (A\bigcap B)\backslash C\) \\
A2. 2. Найти все подформулы. \(\left( (р \land q) \vee q \right) \vee (q \rightarrow p)\) \\
A3. 9. Составить таблица истинности. \((x \rightarrow y) \land (x \rightarrow \overline{y}) \rightarrow \overline{x}\) \\
B1. 2. Для следующих формул найти ДНФ и КНФ. \(\ x_{1} \land (x_{2} \vee (x_{1} \rightarrow x_{3}))\) \\
B2. 8. Упростить формулу алгебра логики. \(А(х,у,z) = (x \rightarrow y) \land z \rightarrow (x \rightarrow z)\) \\
B3. 9. Доказать, что двойственные формулы. \(f(х,\ \ y) = x \leftrightarrow y\) и \(f^{*}(х) = \overline{x \leftrightarrow y}.\) \\
C1. 7. Составить РКС для формулы \(А(х,у,z) = (x \rightarrow y) \land z \rightarrow (x \rightarrow z)\) \\
C2. 5. Булевы функции заданы последовательностью их значений при лексикографическом упорядочении аргументов. Найти все базисы, которые можно составить из следующих функций. \(f = (01110001)\) \\
C3. 6. Привести к предваренной нормальной форме \(\forall x(A(x) \supset B(y))Ђ\) \\

\end{tabular}
\vspace{1cm}


\begin{tabular}{m{17cm}}
\textbf{93-вариант}
\newline

T1. 8. Логика предикатов. \emph{О.с. Понятие предиката, понятие формулы логики предикатов, равносильные формулы логики предикатов} \\
T2. 2. Исчисления высказываний и понятие формулы исчиления высказываний \emph{О.с. Определения доказуемой формулы. Аксиомы исчисления высказываний. Правыла выводимости} \\
A1. 3. Доказать тоджество. \(A\backslash(B\bigcup C) = (A\backslash B)\bigcup(A\backslash C)\) \\
A2. 4. Найти все подформулы. \((x \rightarrow y) \land (x \rightarrow \overline{y}) \rightarrow \overline{x}\) \\
A3. 7. Составить таблица истинности. \((x \rightarrow z) \rightarrow (y \rightarrow z) \rightarrow ((x \vee y) \rightarrow z)\) \\
B1. 8. Для следующих формул найти ДНФ и КНФ. \(х \land у \rightarrow (z \vee у \rightarrow z)\) \\
B2. 1. Доказать тождественно ложность формул. \(F\left( р_{1},р_{2} \right) = \overline{p_{1} \rightarrow (p_{2} \rightarrow p_{1})}\) \\
B3. 8. Доказать равносильность. \(x \rightarrow \left( y \rightarrow \overline{z} \right) \equiv x \land y \rightarrow \overline{z}\) \\
C1. 4. Составить РКС для формулы \((х \rightarrow у) \rightarrow (\overline{х} \rightarrow \overline{у)}\) \\
C2. 7. Булевы функции заданы последовательностью их значений при лексикографическом упорядочении аргументов. Найти все базисы, которые можно составить из следующих функций. \(f = (01101001)\) \\
C3. 3. Пусть \(A,\ \ B\)- некоторые формулы исчисления предикатов, причем переменная\(x\) не входит в \(B\). Доказать следующие соотношения. \(\exists xA \vee B \rightleftarrows \exists x(A \vee B)\) \\

\end{tabular}
\vspace{1cm}


\begin{tabular}{m{17cm}}
\textbf{94-вариант}
\newline

T1. 3. Нормальные формы \emph{О.с. Дизъюнктивная, конъюнктивная и совершенная дизъюнктивная, совершенна конъюнктивная нормальная форма} \\
T2. 9. Нормальные формы \emph{О.с. Дизъюнктивная, конъюнктивная и совершенная дизъюнктивная, совершенна конъюнктивная нормальная форма} \\
A1. 5. Доказать тоджество. \(A\bigcup(B\bigcap C) = (A\bigcup B)\bigcap(A\bigcup C)\) \\
A2. 9. Найти все подформулы. \((x \rightarrow y) \land (x \rightarrow \overline{y}) \rightarrow \overline{x} \vee y\) \\
A3. 8. Составить таблица истинности. \((x \rightarrow z) \rightarrow (y \rightarrow z)\) \\
B1. 5. Для следующих формул найти СДНФ \(\overline{\overline{x \vee y} \rightarrow \overline{x \land y}}\) \\
B2. 3. Упростить формулу алгебра логики. \(A(x,\ \ y) = (x \leftrightarrow y) \land (x \vee y)\) \\
B3. 6. Доказать, что двойственные формулы. \(f(х,\ \ y) = x \land y\) и \(f^{*}(х) = x \vee y.\) \\
C1. 6. Составить РКС для формулы \((x \rightarrow y) \land (x \rightarrow \overline{y}) \rightarrow \overline{x} \leftrightarrow \overline{y}\) \\
C2. 8. Булевы функции заданы последовательностью их значений при лексикографическом упорядочении аргументов. Найти все базисы, которые можно составить из следующих функций. \(f = (10011011)\) \\
C3. 5. Привести к предваренной нормальной форме \(\forall x(\neg A(x) \supset \exists yB(y)) \supset B(y) \vee A(x)\) \\

\end{tabular}
\vspace{1cm}


\begin{tabular}{m{17cm}}
\textbf{95-вариант}
\newline

T1. 2. Понятие выводимости формулы из совокупности формулы \emph{О.с. Теорема дедукции. Обобщенная теорема дедукции.} \\
T2. 7. Алгебра высказываний и понятие формула алгебры высказываний \emph{О.с. Равносильные формулы алгебры логики и их свойтсва} \\
A1. 2. Известно, что высказывание \(A \rightarrow B\) ложно. Что можно сказать обистинности \(A\) и \(B\) ? \\
A2. 7. Найти все подформулы. \((x \rightarrow z) \rightarrow (y \rightarrow z) \rightarrow ((x \vee y) \rightarrow z)\) \\
A3. 4. Составить таблица истинности. \((x \rightarrow y) \land (x \rightarrow \overline{y}) \rightarrow \overline{x} \vee y\) \\
B1. 10. Для следующих формул найти ДНФ и КНФ. \(\overline{x \vee (x_{2} \rightarrow x_{1})}\) \\
B2. 9. Упростить формул. \(A(a,\ \ b,\ \ c) = (a \rightarrow b) \rightarrow (\overline{a}\  \land (b \vee c))\) \\
B3. 1. Найти \(x\), если \(\left( \overline{x \vee a} \right) \vee \left( \overline{x \vee \overline{a}} \right) \equiv b\) \\
C1. 2. Составить РКС для формулы \(A(a,\ \ b,\ \ c) = (a \rightarrow b) \rightarrow (\overline{a}\  \land (b \vee c))\) \\
C2. 2. Булевы функции заданы последовательностью их значений при лексикографическом упорядочении аргументов. Найти все базисы, которые можно составить из следующих функций. \(f = (11101000)\) \\
C3. 2. Пусть \(A,\ \ B\)- некоторые формулы исчисления предикатов, причем переменная\(x\) не входит в \(B\). Доказать следующие соотношения. \(\forall xA \vee B \rightleftarrows \forall x(A \vee B)\); \\

\end{tabular}
\vspace{1cm}


\begin{tabular}{m{17cm}}
\textbf{96-вариант}
\newline

T1. 5. Функции алгебры логики \emph{О.с. Алгебра Буля. Полная система функции} \\
T2. 6. Логика предикатов. \emph{О.с. Понятие предиката, понятие формулы логики предикатов, равносильные формулы логики предикатов} \\
A1. 7. Доказать тоджество. \((A\bigcup B)\bigcap A = (A\bigcap B)\bigcup A = A\) \\
A2. 10. Найти все подформулы. \((x \rightarrow y) \land (x \rightarrow \overline{y}) \rightarrow \overline{x} \leftrightarrow \overline{y}\) \\
A3. 2. Составить таблица истинности. \((x \rightarrow y) \land (x \rightarrow \overline{y}) \rightarrow \overline{x}\) \\
B1. 1. Для следующих формул найти ДНФ и КНФ. \(х \land у \rightarrow (z \vee у \rightarrow z)\) \\
B2. 6. Упростить формул. \(A(a,\ \ b,\ \ c) = (a \rightarrow b) \rightarrow (\overline{a}\  \land (b \vee c))\) \\
B3. 2. Доказать равносильность.\(x_{1} \land x_{2} \land \ \ \ ...\ \  \land x_{n} \rightarrow y \equiv x_{1} \rightarrow (x_{2} \rightarrow (\ \ ...\ \  \rightarrow (x_{n} \rightarrow y)\ \ ,,.\ \ ))\) \\
C1. 3. Составить РКС для формулы \((x \rightarrow z) \rightarrow (y \rightarrow z) \rightarrow ((x \vee y) \rightarrow z)\) \\
C2. 1. Булевы функции заданы последовательностью их значений при лексикографическом упорядочении аргументов. Найти все базисы, которые можно составить из следующих функций.\(f = (00111100)\) \\
C3. 7. Пусть \(A,\ \ B\)- некоторые формулы исчисления предикатов, причем переменная\(x\) не входит в \(B\). Доказать следующие соотношения. \(\exists xA \vee B \rightleftarrows \exists x(A \vee B)\); \\

\end{tabular}
\vspace{1cm}


\begin{tabular}{m{17cm}}
\textbf{97-вариант}
\newline

T1. 6. Производные правила вывода \emph{О.с. Правило силлогизма, правило контрпозиции,правило снятия двойного отрицания} \\
T2. 3. Функции алгебры логики \emph{О.с. Алгебра Буля. Полная система функции} \\
A1. 4. Доказать тоджество. \(A\bigcap(B\bigcup C) = (A\bigcap B)\bigcup(B\bigcap C)\) \\
A2. 8. Найти все подформулы. \((х \rightarrow у) \rightarrow (\overline{х} \rightarrow \overline{у)}\). \\
A3. 5. Составить таблица истинности. \((x \rightarrow \overline{y}) \rightarrow \overline{x}\) \\
B1. 9. Для следующих формул найти СКНФ \((z \rightarrow x \land y))\) \\
B2. 7. Упростить формулу алгебра логики. \(А(х,у,z) = (x \rightarrow y) \land z \rightarrow (x \rightarrow z)\) \\
B3. 4. Доказать равносильность. \((x \vee y) \land (x \vee \overline{y}) \equiv x\) \\
C1. 5. Составить РКС для формулы \((x \land y \rightarrow z) \rightarrow (x \rightarrow (y \rightarrow z))\) \\
C2. 6. Булевы функции заданы последовательностью их значений при лексикографическом упорядочении аргументов. Найти все базисы, которые можно составить из следующих функций. \(f = (00010101)\) \\
C3. 8. Пусть \(A,\ \ B\)- некоторые формулы исчисления предикатов, причем переменная\(x\) не входит в \(B\). Доказать следующие соотношения. \(\forall xB \rightleftarrows B\); \\

\end{tabular}
\vspace{1cm}


\begin{tabular}{m{17cm}}
\textbf{98-вариант}
\newline

T1. 4. Исчисления высказываний и понятие формулы исчиления высказываний \emph{О.с. Определения доказуемой формулы. Аксиомы исчисления высказываний. Правыла выводимости} \\
T2. 4. Понятие выводимости формулы из совокупности формулы \emph{О.с. Теорема дедукции. Обобщенная теорема дедукции.} \\
A1. 7. Доказать тоджество. \((A\bigcup B)\bigcap A = (A\bigcap B)\bigcup A = A\) \\
A2. 6. Найти все подформулы. \(\overline{(a \rightarrow c)} \rightarrow \left( (b \rightarrow c) \rightarrow (a \vee b \rightarrow c) \right)\) \\
A3. 1. Составить таблица истинности. \(\left( (р \land q) \vee q \right) \vee (q \rightarrow p)\) \\
B1. 4. Для следующих формул найти КНФ. \((х \rightarrow у) \rightarrow (\overline{х} \rightarrow \overline{у)}\). \\
B2. 4. Упростить формулу алгебра логики. \((х \rightarrow у) \rightarrow (\overline{х} \rightarrow \overline{у)}\) \\
B3. 5. Найти двойственную формулу к формуле \(f(х,\ \ y,\ \ z) = xy \vee yz \vee xz\) \\
C1. 10. Составить РКС для формулы \(\overline{(a \rightarrow c)} \rightarrow \left( (b \rightarrow c) \rightarrow (a \vee b \rightarrow c) \right)\) \\
C2. 10 Булевы функции заданы последовательностью их значений при лексикографическом упорядочении аргументов. Найти все базисы, которые можно составить из следующих функций. \(f = (11001000)\) \\
C3. 9. Пусть \(A,\ \ B\)- некоторые формулы исчисления предикатов, причем переменная\(x\) не входит в \(B\). Доказать следующие соотношения. \(\exists xB \rightleftarrows B\)
 \\

\end{tabular}
\vspace{1cm}


\begin{tabular}{m{17cm}}
\textbf{99-вариант}
\newline

T1. 9. Алгебра высказываний и понятие формула алгебры высказываний \emph{О.с. Равносильные формулы алгебры логики и их свойтсва} \\
T2. 8. Понятие выводимости формулы из совокупности формулы \emph{О.с. Теорема дедукции. Обобщенная теорема дедукции.} \\
A1. 6. Доказать тождество. \(\overline{A\bigcap B} = \overline{A}\bigcup\overline{B}\) \\
A2. 1. Найти все подформулы. \(((a_{0} \rightarrow a_{1}) \land ((a_{1} \rightarrow a_{2}) \rightarrow (\overline{a_{0}} \vee a_{2})).\) \\
A3. 3. Составить таблица истинности. \((x \rightarrow z) \rightarrow (y \rightarrow z) \rightarrow ((x \vee y) \rightarrow z)\) \\
B1. 7. Для следующих формул найти ДНФ и КНФ. \(\left( х_{1} \land \overline{х_{2}} \right) \vee х_{3}\) \\
B2. 5. Упростить формулу алгебра логики. \((x \rightarrow y) \land (x \rightarrow \overline{y}) \rightarrow \overline{x}\) \\
B3. 3. Доказать, что двойственные формулы.\(f(х,\ \ y) = x \land y\) и \(f^{*}(х,\ \ y) = x \vee y.\) \\
C1. 8. Составить РКС для формулы \(A(a,\ \ b,\ \ c) = (a \rightarrow b) \rightarrow (\overline{a}\  \land (b \vee c))\) \\
C2. 3. Булевы функции заданы последовательностью их значений при лексикографическом упорядочении аргументов. Найти все базисы, которые можно составить из следующих функций. \(f = (01010111)\) \\
C3. 4. Привести к предваренной нормальной форме \(\forall xB(x) \supset \exists y(A(y) \supset B(x))\) \\

\end{tabular}
\vspace{1cm}


\begin{tabular}{m{17cm}}
\textbf{100-вариант}
\newline

T1. 10. Производные правила вывода \emph{О.с. Правило силлогизма, правило контрпозиции,правило снятия двойного отрицания} \\
T2. 5. Нормальные формы \emph{О.с. Дизъюнктивная, конъюнктивная и совершенная дизъюнктивная, совершенна конъюнктивная нормальная форма} \\
A1. 8. Доказать тоджество. \(A\bigcap(B\backslash C) = (A\bigcap B)\backslash(A\bigcap C) = (A\bigcap B)\backslash C\) \\
A2. 5. Найти все подформулы. \((х \vee у) \rightarrow \left( х \land \overline{у \vee х \rightarrow у} \right)\) \\
A3. 4. Составить таблица истинности. \((x \rightarrow y) \land (x \rightarrow \overline{y}) \rightarrow \overline{x} \vee y\) \\
B1. 3. Для следующих формул найти СКНФ \(\overline{(a \rightarrow c)} \rightarrow \left( (b \rightarrow c) \rightarrow (a \vee b \rightarrow c) \right)\) \\
B2. 2. Упростить формулу алгебра логики. \(А(х,у,z) = (x \rightarrow y) \land z \rightarrow (x \rightarrow z)\) \\
B3. 1. Найти \(x\), если \(\left( \overline{x \vee a} \right) \vee \left( \overline{x \vee \overline{a}} \right) \equiv b\) \\
C1. 1. Составить РКС для формулы \(F(a,\ \ b,\ \ c) = (a \vee b) \rightarrow (a \land b) \vee c.\) \\
C2. 9. Булевы функции заданы последовательностью их значений при лексикографическом упорядочении аргументов. Найти все базисы, которые можно составить из следующих функций. \(f = (10010110)\) \\
C3. 8. Пусть \(A,\ \ B\)- некоторые формулы исчисления предикатов, причем переменная\(x\) не входит в \(B\). Доказать следующие соотношения. \(\forall xB \rightleftarrows B\); \\

\end{tabular}
\vspace{1cm}



\end{document}
