1. Orta mektepte matematika pánin oqitiw maqsetleri (uliwma bilimlendiriw maqsetleri, tárbiyalıq maqsetler, ámeliy maqsetler).
====
2. Matematikalıq pikirlew hám juwmaq shıǵarıw (súwretlew túrleri, aksiomalar, postulatlar, teoremalar).
====
A1. Esaplań: \(\frac{1 \cdot 2 \cdot 3 + 3 \cdot 6 \cdot 9 + 5 \cdot 10 \cdot 15 + 7 \cdot 14 \cdot 21}{2 \cdot 4 \cdot 6 + 6 \cdot 12 \cdot 18 + 10 \cdot 20 \cdot 30 + 28}\)
====
A2. Aybek 36 minutta eki máseleni sheshti. Birinshi máseleni sheshiwge ekinshi máseleni sheshiwge qaraganda 6 minut kóp waqıt sarpladı. Ol ekinshi máseleni neshe minutta sheshti?
====
A3. Teń qaptallı úshmúyeshliktiń tóbesindegi múyeshi 16° qa teń. Múyeshtin qaptal tárepi hám ultanındaǵı bissektrisası menen payda bolgan úńsiz múyesh tabılsın.
====
B1. A kóplikte 16 element, B kóplikte 18 element bar. AՍB kóplikte minimal muģdardaģı elementler qansha boliwi múmkin?
====
B2. Teńsizlikti sheshin: \(\frac{ (x + 1) (3 - 2x) }{x - 5} \leq 0\).
====
B3. Orayları A, B, C, D hám E noqatlarda bolgan bes sheńber sırtqı tárizde urınadı. Eger AB=46, BC=44, CD=47, DE=43, AE=44 bolsa, en úlken radiuslı sheńberdiń orayı qaysı noqatta boladı?
====
C1. Esaplań:\(\frac{2sin^{2}70^{0} - 1}{2ctg115^{0}cos155^{0}}\)
====
C2. Funkciyanıń anıqlanıw oblastınan en kishi sandı tabiń \(f (x) = \sqrt{log_{12,5}x - log_{x}12,5}\)
====
C3. Tuwri múyeshli ABC úshmúyeshlikte tuwri múyeshtiń S tóbesinen CM mediana, SL bissektrisa hám SN biyiklik ótkerilgen.
====
1. Matematikanıń pán sıpatında hám oqiw páni sıpatındaǵı xarakteristikası (matematikanıń rawajlanıw dáwirleri, matematika boyınsha dáslepki sabaqlıqlar).
====
2. Matematika pánin oqitiwda analogiya (analogiya túrleri, analogiyanı qollanıw)
====
A1. 10,14,18, ... arifmetikalıq progressiyada 110 sanınıń tártip nomerin tabıń
====
A2. Esaplań: \(\arcsin (\sin3 ) \).
====
A3. 160 sanınıń neshe bólegin 32 sanı quraydı?
====
B1. Esaplań: \(\frac{\sqrt{2} - 15}{\sqrt{\sqrt{2} + 1} - 4} - \sqrt{\sqrt{2} + 1} + 1\)
====
B2. $(\frac{|x| + x}{x-2})^{2} - \frac{12x}{x-2} + 5 = 0$ teńleme neshe haqıyqıy korenge iye
====
B3. 2,3,4,...,13 natural sanlar izbe-izliginen bir san tasodiy túrde tańlansa, tańlangan sannıń ápiwayı bolıwı itimallıǵı qanday?
====
C1. Esaplań: \(\sqrt{0,5 (13 + 3\sqrt{17}) } + \sqrt{0,5 (13 - 3\sqrt{17}) }\)
====
C2. Teńsizliktiń sheshimler oblastına neshe pútin san kiredi:\(\log_{\frac{x}{20}} (log_{x}\sqrt{20 - x}) > 0\)
====
C3. Teń qaptallı ABC úshmúyeshlikte (AB=BC) O noqat radiusı 4,2 ge teń bolgan sızılgan sheńberdin orayı hám ABC úshmúyeshlikke sırtlay sızılgan sheńberdin radiusı 14.
====
1. Matematika hám oni oqitiwda ilimiy metodlar (baqlaw, tájiriybe, salıstırıw hám t.b.)
====
2. Matematikalıq túsinikler (qáliplesiw procesi, uliwmalastırıw, mazmun hám kólemi, túsiniklerdi klassifikaciyalaw).
====
A1. Esaplań: \(tg\left(arctg2 - arctg\frac{1}{2} \right) \).
====
A2. Eger \(\log_{\sqrt{3}}\sqrt[6]{a}\), eger \(\log_{a}27 = b\) bolsa, oni tabiń.
====
A3. Aybek 36 minutta eki máseleni sheshti. Birinshi máseleni sheshiwge ekinshi máseleni sheshiwge qaraganda 6 minut kóp waqıt sarpladı. Ol ekinshi máseleni neshe minutta sheshti?
====
B1. 3,8,13,18,... arifmetikalıq progressiyanın neshe aģzası 520 ten kishi?
====
B2. Funkciyanıń anıqlanıw oblastın tabiń: \(y = \frac{\sqrt{3^{x} - 27}}{8 - 2^{x}}\)
====
B3. Piramida ultanınıń barlıq diagonalları sanı piramidanıń barlıq qabırǵaları sanına teń. Berilgen piramidanıń barlıq jaqları hám tóbeleri sanlarının qosındısın tabıń.
====
C1. Esaplań:\(\frac{2sin^{2}70^{0} - 1}{2ctg115^{0}cos155^{0}}\)
====
C2. Funkciyanıń anıqlanıw oblastınan en kishi sandı tabiń \(f (x) = \sqrt{log_{12,5}x - log_{x}12,5}\)
====
C3. Tuwri múyeshli trapeciyada úlken diagonal trapeciyanıń ótkir múyeshinin bissektrisası bolip esaplanadı.Bul diagonaldıń uzınlıǵı 8 ge teń, topıraq múyeshtin tóbesinen usı diagonalǵa shekemgi aralıq 2 ge teń.
====
1. Matematika pánin oqitiwda analiz hám sintez (analiz formaları, dálillewde, másele sheshiwde qollanılıwı)
====
2. Matematikalıq pikirlew formaları (konkret, abstrakt, intuitiv, funkcional, dialektikalıq, dóretiwshilik).
====
A1. Anlatpanı ápiwayılastırıń: \(\frac{2^{3n - 4} \cdot 2^{5 + 6n}}{2^{1 + 3n}}\).
====
A2. x tiń qanday mánisinde tómendegi anlatpa duris boladı: (3 (2 - x) - 8 = 10)?
====
A3. Teńlemeniń haqiqiy korenlerinin qosındısın tabiń: \((x^2 + 14x + 14) x^2 + x + 14 = 14x^2\)
====
B1. Integraldı esaplań: \(\int_{}^{}{\frac{1}{\cos^{2} (2x - 3) }dx}\)
====
B2. \(\overline{a} (- 12,13, - 15) \) vektorınıń Oxy tegisligindegi proekciyası bolgan vektordı tabiń.
====
B3. Cilindr sferanın átirapına sızılgan. Sfera betiniń maydanı 32π ge teń. Tolıq bettiń maydanın tabiń.
====
C1. Teńlemeniń úlken korenin tabiń: \(\sqrt[3]{22 + 6x} + \sqrt[3]{15 - 6x} = 1\)
====
C2. Teńlemeniń eń kishi sheshimin tabiń: \(\left| 2x^2 - 10x + 8 \right| = 13x - 22\).
====
C3. Teń qaptallı ABC úshmúyeshlikke (AB=BC) sızılgan sheńber bul úshmúyeshlik biyikliklerinin kesilisiw noqatınan ótedi.
====
1. Matematika pánin oqitiwda uliwmalastırıw hám abstraktlastırıw (ara baylanıslılıq, abstraktlastırıw túrleri).
====
2. Evristik metod - matematika pánin oqitiw (teoremalardı úyreniwde, másele sheshiwde paydalanıw).
====
A1. Eger \(2^{x} = 152\) bolsa, \(|x - 8| + |x - 6|\) anlatpanı ápiwayılastırıń.
====
A2. Teńlemeniń sheshimlerinin qosındısın tabıń: \(x2 - 4|x| - a + 3 = 0\) bunda \(a = 3\).
====
A3. Funkciyanın tiykarģı dáwirin tabiń: \(y = \frac{1}{2}\sin{\frac{x}{2}\cos\frac{x}{2}}\).
====
Ápiwayılastırıń: \(\frac{\cos^{2}a \cdot {ctg}^{2}a}{\sin^{2}a}\)
====
B2. Algebralıq ańlatpanıń mánisin tabiń: 0,125xy-0,5x+1, bunda x=10, y=8.
====
B3. Teńlemeniń haqıyqıy korenleriniń qosındısın tabiń: \((x-3) \sqrt{x^{2} - 3x + 6} = 2x - 6\)
====
C1. Teńlemeniń kishi korenin tabiń: \(log_{6}x \cdot log_{4}x = log_{6}4\)
====
C2. Ápiwayılastırıń hám \(a = \frac{2\pi}{15}:\frac{4 (cos3a - cos8a) }{\sqrt{31} (sin3a + sin8a) }\) da esaplań
====
C3. Teń qaptallı ABC úshmúyeshlikte (AB=BC) O noqat sızılgan sheńber orayı, ABC úshmúyeshliktiń maydanı 300 ge, AOC úshmúyeshliktiń maydanı 112,5 ge teń. AB táreptiń uzınlıǵın tabıń.
====
1. Metodika mektep matematika kursına matematikalıq túsiniklerdi kirgiziw (konkret induktiv metod, abstrakt-deduktiv metod).
====
2. Matematikalıq pikirlew usılı (moslasıwshılıq, belsendilik, maqsetke bagdarlanganlıq, tayarlıq, pikirlewdiń keńligi, tereńligi, sın kózqaraslılıǵı hám ózin-ózi sın kózqaraslılıǵı).
====
A1. Teńlemeniń haqiqiy korenlerinin qosındısın tabiń: \((x^2 + 14x + 14) (x^2 + x + 14) = 14x^2\)
====
A2. Teńsizlikti sheshin: \(\frac{arccos (- \frac{3}{\pi}) \cdot \log_{\frac{3}{\pi}}\frac{\pi}{4}}{1 - 2\log_{\log_{2}x}2} \geq 0\).
====
A3. Teń qaptallı úshmúyeshliktiń tóbesindegi múyeshi 16° qa teń. Múyeshtin qaptal tárepi hám ultanındaǵı bissektrisası menen payda bolgan úńsiz múyesh tabılsın.
====
B1. Teńsizlikti sheshin: \(25^{\log_{5}{ (1 - 2x) }} + { (2x - 1) }^{2} < 50\)
====
B2. y=-8x+3 tuwri sızıqtıń x=1 ge salıstırganda simmetriyalı tuwri sızıqtıń teńlemesin kórsetiń.
====
B3. 6 tańbalı 553n52 sanı 18 ke qaldıqsız bólinetuģın bolsa, n niń mánisin anıqlań.
====
C1. Teńlemeni sheshiń: \((\sqrt{13}) ^{x + 20} = (\sqrt[3]{15}) ^{x + 20}\)
====
C2. Teńlemeniń en úlken teris korenin graduslarda tabiń: \(2sin11xsin5x = \frac{\sqrt{3}}{2} - cos16x\).
====
C3. 30 radiuslı sheńber tuwri múyeshli hám eki katet uzınlıqlarınıń gipotenuzasına tiyedi, eger gipotenuza uzınlıǵı 26 ge teń bolsa, kishi katet uzınlıǵın tabıń.
====
1. Matematika pánin oqitiwda máseleniń orni hám roli (tekstli máseleler, logikalıq máseleler, házil máseleleri).
====
2. Matematika pánin oqitiwda kórgizbeli qurallar hám texnikalıq qurallar (kórgizbeli, model, keste, kompyuter hám olarga tiyisli didaktikalıq materiallar).
====
A1. Teńsizliktiń barlıq pútin sheshimlerinin qosındısın tabiń: \(x^2 + 5x + 3 \leq 0\).
====
A2. Funkciyanın tiykarģı dáwirin tabiń: \(y = \frac{1}{2}\sin{\frac{x}{2}\cos\frac{x}{2}}\).
====
A3. x tiń qanday mánisinde tómendegi anlatpa duris boladı: \(3 (2 - x) - 8 = 10\)?
====
B1. \(0,003 \cdot 0,004 \cdot 10^{8}\) kóbeymeni standart kóriniske keltiriń.
====
B2. Eger a=16-\(x^2, b=x^2 - 4\) hám a,b natural sanlar bolsa, en kishi mánisin tabiń: \(a^2 + b^2\).
====
B3. Alti tańbalı 553n52 sanı 18 ke qaldıqsız bólinetuģın bolsa, n niń mánisin anıqlań.
====
C1. \(tg30xtg97^{0} + tg97^{0}tg38^{0} + tg38^{0}tg30x = 1\) teńlemeniń eń kishi on korenin graduslarda tabıń.
====
C2. Teńlemeniń eń kishi sheshimin tabiń: \(\left| 2x2 - 10x + 8 \right| = 13x - 22\).
====
ABC tuwri múyeshli trayektoriya átirapında sheńber sızılgan.VAS ótkir múyeshtin bissektrisası BC katetti M noqatta, sheńber bolsa K noqatta kesip ótedi. Tabiw kerek: \(\cos\angle BAC\).
====
1. Matematika pánin oqitiwda tiykargı didaktikalıq principler (ilimiylik principi, tárbiya principi, kórgizbelilik principi, bilimniń bekkemligi principi hám basqalar).
====
2. Matematika páninen sabaqtan tısqarı jumislar (dóńgelekler, toparlar, artta qalıwshılar menen islesiw, ózlestiriwshiler menen islesiw).
====
A1. Funkciyanıń tuwındısın tabiń: \(y = 5 sin 9x + 3 sin 15x\).
====
A2. Teńlemeni sheshin: \(\frac{x - 3}{x - 1} + \frac{x + 3}{x + 1} = \frac{x + 6}{x + 2} + \frac{x - 6}{x - 2}\).
====
A3. Esaplań: \(tg\left(arctg2 - arctg\frac{1}{2} \right) \).
====
B1. Altı tańbalı 553n52 sanı 18 ke qaldıqsız bólinetuģın bolsa, n niń mánisin anıqlań.
====
B2. Piramida ultanınıń barlıq diagonalları sanı piramidanıń barlıq qabırǵaları sanına teń. Berilgen piramidanıń barlıq jaqları hám tóbeleri sanlarının qosındısın tabıń.
====
B3. Teńsizlikti sheshin: \(\frac{ (x + 1) (3 - 2x) }{x - 5} \leq 0\).
====
C1. Esaplań:\(\frac{2sin^{2}70^{0} - 1}{2ctg115^{0}cos155^{0}}\)
====
C2. Ápiwayılastırıń hám \(a = \frac{2\pi}{15}:\frac{4 (cos3a - cos8a) }{\sqrt{31} (sin3a + sin8a) }\) da esaplań
====
C3. Teńlemeni sheshin: \((\sqrt{13}) ^{x + 20} = (\sqrt[3]{15}) ^{x + 20}\)
====
1. Ózbekstanda bilimlendiriw, bilimlendiriw haqqındaǵı nızam (bilimlendiriw reformaları, bilimlendiriw sisteması, bilimlendiriw baǵdarlamaları).
====
2. Aktiv oqitiw metodi (modeller, aldınģi pedagogikalıq metodlar).
====
A1. Dáslepki funkciyanı tabiń: \(y=cos{3x}\cos{12x}\).
====
A2. Teńlemeni sheshin: \(3,6x - 7,4x = 1,3 - 4,8x\).
====
A3. Teńsizliktiń barlıq pútin sheshimlerinin qosındısın tabiń: \(x2 + 5x + 3 \leq 0\).
====
B. 3,8,13,18,... arifmetikalıq progressiyanın neshe aģzası 520 ten kishi?
====
B2. Ápiwayılastırıń: \(\frac{\cos^{2}a \cdot {ctg}^{2}a}{\sin^{2}a}\)
====
B3. A kóplikte 16 element, B kóplikte 18 element bar. AՍB kóplikte minimal muģdardaģı elementler qansha boliwi múmkin?
====
C1. Teńlemeni sheshiń: \((\sqrt{13}) ^{x + 20} = (\sqrt[3]{15}) ^{x + 20}\)
====
C2. Teńlemeniń en úlken teris korenin graduslarda tabiń: \(2sin11xsin5x = \frac{\sqrt{3}}{2} - cos16x\).
====
C3. ABC úshmúyeshlikte AB hám BC táreplerge júrgizilgen medianalar óz ara perpendikulyar.
====
1. Oqıtıwshını sabaqqa tayarlaw sisteması (jańa oqiw jılına tayarlaw, oqıtıwshını gezektegi sabaqqa tayarlaw).
====
2. Matematika pánin oqitiw metodları hám formaları (matematika pánin oqitiw metodları, metodları hám formalarının túrleri).
====
A1. 32 sanı 160 sanınıń neshe bólegin quraydı?
====
A2. Sanlar kósherinde -7,5 hám 4,2 sanları arasında neshe pútin san jaylasqan?
====
A3. 10,14,18, ... arifmetikalıq progressiyada 110 sanınıń tártip nomerin tabıń
====
B1. Funkciyanıń anıqlanıw oblastın tabiń: \(y = \frac{\sqrt{3^{x} - 27}}{8 - 2^{x}}\)
====
B2. Orayları A, B, C, D hám E noqatlarda bolgan bes sheńber sırtqı tárizde urınadı. Eger AB=46, BC=44, CD=47, DE=43, AE=44 bolsa, en úlken radiuslı sheńber orayı qaysı noqatta jaylasqan?
====
B3. Teńlemeniń haqıyqıy korenleriniń qosındısın tabiń: \((x-3) \sqrt{x^{2} - 3x + 6} = 2x - 6\)
====
C1. Teńlemeniń kishi korenin tabiń: \(log_{6}x \cdot log_{4}x = log_{6}4\)
====
C2. Esaplań: \(\sqrt{0,5 (13 + 3\sqrt{17}) } + \sqrt{0,5 (13 - 3\sqrt{17}) }\)
====
C3. Tuwri múyeshli trapeciyada úlken diagonal trapeciyanıń ótkir múyeshinin bissektrisası bolip esaplanadı.Bul diagonaldıń uzınlıǵı 8 ge teń, tómengi múyeshtin tóbesinen bul diagonalģa shekemgi aralıq 2 ge teń.