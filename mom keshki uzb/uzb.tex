1. O‘rta maktabda matematika fanini o‘qitish maqsadlari (umumta’lim maqsadlari, tarbiyaviy maqsadlar, amaliy maqsadlar).
====
2. Matematik fikrlash va xulosa chiqarish (tasvirlash turlari, aksiomalar, postulatlar, teoremalar).
====
A1. Hisoblang: \(\frac{1 \cdot 2 \cdot 3 + 3 \cdot 6 \cdot 9 + 5 \cdot 10 \cdot 15 + 7 \cdot 14 \cdot 21}{2 \cdot 4 \cdot 6 + 6 \cdot 12 \cdot 18 + 10 \cdot 20 \cdot 30 + 28}\)
====
A2. Oybek 36 minutda ikkita masalani yechdi. Birinchi masalani yechishga ikkinchi masalani yechishga qaraganda 6 daqiqa ko‘proq vaqt sarfladi. U ikkinchi masalani necha minutda yechdi?
====
A3. Teng yonli uchburchakning uchidagi burchagi 16° ga teng. Burchakning yon tomoni va asosidagi bissektrisasi bilan hosil bo‘lgan botiq burchak topilsin.
====
B1. A to‘plamda 16 ta element, B to‘plamda 18 ta element bor. AЅB to‘plamda minimal miqdordagi elementlar qancha bo‘lishi mumkin?
====
B2. Tengsizlikni yeching: \(\frac{ (x + 1) (3 - 2x) }{x - 5} \leq 0\).
====
B3. Markazlari A, B, C, D va E nuqtalarda bo‘lgan beshta aylana tashqi urinadi. Agar AB=46, BC=44, CD=47, DE=43, AE=44 bo‘lsa, eng katta radiusli aylananing markazi qaysi nuqtada bo‘ladi?
====
C1. Hisoblang:\(\frac{2sin^{2}70^{0} - 1}{2ctg115^{0}cos155^{0}}\)
====
C2. Funksiyaning aniqlanish sohasidan eng kichik sonni toping \(f (x) = \sqrt{log_{12,5}x - log_{x}12,5}\)
====
C3. To‘g‘ri burchakli ABC uchburchakda to‘g‘ri burchakning S uchidan CM mediana, SL bissektrisa va SN balandlik o‘tkazilgan.
====
1. Matematikaning fan sifatida va o‘quv fani sifatidagi tavsifi (matematikaning rivojlanish davrlari, matematika bo‘yicha dastlabki darsliklar).
====
2. Matematika fanini o‘qitishda analogiya (analogiya turlari, analogiyani qo‘llash)
====
A1. 10,14,18,... arifmetik progressiyada 110 sonining tartib raqamini toping
====
A2. Hisoblang: \(\arcsin (\sin3) \).
====
A3. 160 sonining necha qismini 32 soni tashkil qiladi?
====
B1. Hisoblang: \(\frac{\sqrt{2} - 15}{\sqrt{\sqrt{2} + 1} - 4} - \sqrt{\sqrt{2} + 1} + 1\)
====
B2. $ (\frac{|x| + x}{x-2}) ^{2} - \frac{12x}{x-2} + 5 = 0$ tenglama nechta haqiqiy ildizga ega
====
B3. 2,3,4,...,13 natural sonlar ketma-ketligidan bitta son tasodifiy tanlansa, tanlangan sonning tub bo‘lish ehtimoli qanday?
====
C1. Hisoblang: \(\sqrt{0,5 (13 + 3\sqrt{17}) } + \sqrt{0,5 (13 - 3\sqrt{17}) }\)
====
C2. Tengsizlikning yechimlar sohasiga nechta butun son kiradi: \(\log_{\frac{x}{20}} (log_{x}\sqrt{20 - x}) > 0\)
====
C3. Teng yonli ABC uchburchakda (AB=BC) O nuqta radiusi 4,2 ga teng bo‘lgan chizilgan aylananing markazi va ABC uchburchakka tashqi chizilgan aylananing radiusi 14.
====
1. Matematika va uni o‘qitishda ilmiy metodlar (kuzatish, tajriba, taqqoslash va h.k.)
====
2. Matematik tushunchalar (shakllanish jarayoni, umumlashtirish, mazmun va ko‘lam, tushunchalarni tasniflash).
====
A1. Hisoblang: \(tg\left (arctg2 - arctg\frac{1}{2} \right) \).
====
A2. Agar \(\log_{\sqrt{3}}\sqrt[6]{a}\), agar \(\log_{a}27 = b\) bo‘lsa, uni toping.
====
A3. Oybek 36 minutda ikkita masalani yechdi. Birinchi masalani yechishga ikkinchi masalani yechishga qaraganda 6 daqiqa ko‘proq vaqt sarfladi. U ikkinchi masalani necha minutda yechdi?
====
B1. 3,8,13,18,... arifmetik progressiyaning nechta hadi 520 dan kichik?
====
B2. Funksiyaning aniqlanish sohasini toping: \(y = \frac{\sqrt{3^{x} - 27}}{8 - 2^{x}}\)
====
B3. Piramida asosining barcha diagonallari soni piramidaning barcha qirralari soniga teng. Berilgan piramidaning barcha yoqlari va uchlari sonlarining yig‘indisini toping.
====
C1. Hisoblang:\(\frac{2sin^{2}70^{0} - 1}{2ctg115^{0}cos155^{0}}\)
====
C2. Funksiyaning aniqlanish sohasidan eng kichik sonni toping \(f (x) = \sqrt{log_{12,5}x - log_{x}12,5}\)
====
C3. To‘g‘ri burchakli trapetsiyada katta diagonal trapetsiyaning o‘tkir burchagining bissektrisasi hisoblanadi.Bu diagonalning uzunligi 8 ga, burchak uchidan shu diagonalgacha bo‘lgan masofa 2 ga teng.
====
1. Matematika fanini o‘qitishda analiz va sintez (analiz shakllari, isbotlashda, masala yechishda qo‘llanishi)
====
2. Matematik tafakkur shakllari (konkret, abstrakt, intuitiv, funksional, dialektik, ijodiy).
====
A1. Ifodani soddalashtiring: \(\frac{2^{3n - 4} \cdot 2^{5 + 6n}}{2^{1 + 3n}}\).
====
A2. x ning qanday qiymatida quyidagi ifoda to‘g‘ri bo‘ladi: (3 (2 - x) - 8 = 10)?
====
A3. Tenglamaning haqiqiy ildizlari yig‘indisini toping: \((x^2 + 14x + 14) (x^2 + x + 14) = 14x^2\)
====
B1. Integralni hisoblang: \(\int_{}^{}{\frac{1}{\cos^{2} (2x - 3) }dx}\)
====
B2. \(\overline{a} (- 12,13, - 15) \) vektorning Oxy tekisligidagi proyeksiyasi bo‘lgan vektorni toping.
====
B3. Silindr sferaning atrofiga chizilgan. Sfera sirtining yuzi 32π ga teng. To‘la sirtning yuzini toping.
====
C1. Tenglamaning katta ildizini toping: \(\sqrt[3]{22 + 6x} + \sqrt[3]{15 - 6x} = 1\)
====
C2. Tenglamaning eng kichik yechimini toping: \(\left| 2x^2 - 10x + 8 \right| = 13x - 22\).
====
C3. Teng yonli ABC uchburchakka (AB=BC) chizilgan aylana bu uchburchak balandliklarining kesishish nuqtasidan o‘tadi.
====
1. Matematika fanini o‘qitishda umumlashtirish va abstraktlashtirish (o‘zaro bog‘liqlik, abstraktlashtirish turlari).
====
2. Evristik metod - matematika fanini o‘qitish (teoremalarni o‘rganishda, masala yechishda foydalanish).
====
A1. Agar \(2^{x} = 152\) bo‘lsa, \(|x - 8| + |x - 6|\) ifodani soddalashtiring.
====
A2. Tenglamaning yechimlari yig‘indisini toping: \(x^2 - 4|x| - a + 3 = 0\) bunda \(a = 3\).
====
A3. Funksiyaning asosiy davrini toping: \(y = \frac{1}{2}\sin{\frac{x}{2}\cos\frac{x}{2}}\).
====
Soddalashtiring: \(\frac{\cos^{2}a \cdot {ctg}^{2}a}{\sin^{2}a}\)
====
B2. Algebraik ifodaning qiymatini toping: 0,125xy-0,5x+1, bunda x=10, y=8.
====
B3. Tenglamaning haqiqiy ildizlari yig‘indisini toping: \((x-3) \sqrt{x^{2} - 3x + 6} = 2x - 6\)
====
C1. Tenglamaning kichik ildizini toping: \(log_{6}x \cdot log_{4}x = log_{6}4\)
====
C2. Soddalashtiring va \(a = \frac{2\pi}{15}:\frac{4 (cos3a - cos8a) }{\sqrt{31} (sin3a + sin8a) }\) da hisoblang
====
C3. Teng yonli ABC uchburchakda (AB=BC) O nuqta chizilgan aylana markazi, ABC uchburchakning yuzi 300 ga, AOC uchburchakning yuzi 112,5 ga teng. AB tomonning uzunligini toping.
====
1. Metodika maktab matematika kursiga matematik tushunchalarni kiritish (konkret induktiv metod, abstrakt-deduktiv metod).
====
2. Matematik fikrlash usuli (moslashuvchanlik, faollik, maqsadga yo‘nalganlik, tayyorgarlik, fikrlashning kengligi, chuqurligi, tanqidiy va o‘z-o‘zini tanqidiy yondashuv).
====
A1. Tenglamaning haqiqiy ildizlari yig‘indisini toping: \((x^2 + 14x + 14) (x^2 + x + 14) = 14x^2\)
====
A2. Tengsizlikni yeching: \(\frac{arccos (- \frac{3}{\pi}) \cdot \log_{\frac{3}{\pi}}\frac{\pi}{4}}{1 - 2\log_{\log_{2}x}2} \geq 0\).
====
A3. Teng yonli uchburchakning uchidagi burchagi 16° ga teng. Burchakning yon tomoni va asosidagi bissektrisasi bilan hosil bo‘lgan botiq burchak topilsin.
====
B1. Tengsizlikni yeching: \(25^{\log_{5}{ (1 - 2x) }} + { (2x - 1) }^{2} < 50\)
====
B2. y=-8x+3 to‘g‘ri chiziqning x=1 ga nisbatan simmetrik to‘g‘ri chiziq tenglamasini ko‘rsating.
====
B3. 6 xonali 553n52 soni 18 ga qoldiqsiz bo‘linadigan bo‘lsa, n ning qiymatini aniqlang.
====
C1. Tenglamani yeching: \((\sqrt{13}) ^{x + 20} = (\sqrt[3]{15}) ^{x + 20}\)
====
C2. Tenglamaning eng katta manfiy ildizini graduslarda toping: \(2sin11xsin5x = \frac{\sqrt{3}}{2} - cos16x\).
====
C3. 30 radiusli aylana to‘g‘ri burchakli va ikkita katet uzunliklarining gipotenuzasiga urinadi, agar gipotenuza uzunligi 26 ga teng bo‘lsa, kichik katet uzunligini toping.
====
1. Matematika fanini o‘qitishda masalaning o‘rni va roli (matnli masalalar, mantiqiy masalalar, hazil masalalari).
====
2. Matematika fanini o‘qitishda ko‘rgazmali qurollar va texnik vositalar (ko‘rgazmali, model, jadval, kompyuter va ularga tegishli didaktik materiallar).
====
A1. Tengsizlikning barcha butun yechimlari yig‘indisini toping: \(x^2 + 5x + 3 \leq 0\).
====
A2. Funksiyaning asosiy davrini toping: \(y = \frac{1}{2}\sin{\frac{x}{2}\cos\frac{x}{2}}\).
====
A3. x ning qanday qiymatida quyidagi ifoda to‘g‘ri bo‘ladi: \(3 (2 - x) - 8 = 10\)?
====
B1. \(0,003 \cdot 0,004 \cdot 10^{8}\) ko‘paytmani standart shaklga keltiring.
====
B2. Agar a=16-\(x^2, b=x^2-4\) va a,b natural sonlar bo‘lsa, eng kichik qiymatini toping: \(a^2 + b^2\).
====
B3. Olti xonali 553n52 soni 18 ga qoldiqsiz bo‘linadigan bo‘lsa, n ning qiymatini aniqlang.
====
C1. \(tg30xtg97^{0} + tg97^{0}tg38^{0} + tg38^{0}tg30x = 1\) tenglamaning eng kichik musbat ildizini graduslarda toping.
====
C2. Tenglamaning eng kichik yechimini toping: \(\left| 2x^2 - 10x + 8 \right| = 13x - 22\).
====
C3. ABC to‘g‘ri burchakli trayektoriya atrofida aylana chizilgan.VAS o‘tkir burchakning bissektrisasi BC katetni M nuqtada, aylana esa K nuqtada kesib o‘tadi. Topish kerak: \(\cos\angle BAC\).
====
1. Matematika fanini o‘qitishda asosiy didaktik tamoyillar (ilmiylik tamoyili, tarbiya tamoyili, ko‘rgazmalilik tamoyili, bilimning mustahkamligi tamoyili va boshqalar).
====
2. Matematika fanidan darsdan tashqari ishlar (davralar, guruhlar, ortda qoluvchilar bilan ishlash, o‘zlashtiruvchilar bilan ishlash).
====
A1. Funksiyaning hosilasini toping: (y = 5 sin 9x + 3 sin 15x).
====
A2. Tenglamani yeching: \(\frac{x - 3}{x - 1} + \frac{x + 3}{x + 1} = \frac{x + 6}{x + 2} + \frac{x - 6}{x - 2}\).
====
A3. Hisoblang: \(tg\left(arctg2 - arctg\frac{1}{2} \right) \).
====
B1. Olti xonali 553n52 soni 18 ga qoldiqsiz bo‘linadigan bo‘lsa, n ning qiymatini aniqlang.
====
B2. Piramida asosining barcha diagonallari soni piramidaning barcha qirralari soniga teng. Berilgan piramidaning barcha yoqlari va uchlari sonlarining yig‘indisini toping.
====
B3. Tengsizlikni yeching: \(\frac{ (x + 1) (3 - 2x) }{x - 5} \leq 0\).
====
C1. Hisoblang:\(\frac{2sin^{2}70^{0} - 1}{2ctg115^{0}cos155^{0}}\)
====
C2. Soddalashtiring va \(a = \frac{2\pi}{15}:\frac{4 (cos3a - cos8a) }{\sqrt{31} (sin3a + sin8a) }\) da hisoblang
====
C3. Tenglamani yeching: \((\sqrt{13}) ^{x + 20} = (\sqrt[3]{15}) ^{x + 20}\)
====
1. O‘zbekistonda ta’lim, ta’lim to‘g‘risidagi qonun (ta’lim islohotlari, ta’lim tizimi, ta’lim dasturlari).
====
2. Faol o‘qitish metodi (modellar, ilg‘or pedagogik metodlar).
====
A1. Dastlabki funksiyani toping: \(y=cos{3x}\cos{12x}\).
====
A2. Tenglamani yeching: (3,6x - 7,4x = 1,3 - 4,8x).
====
A3. Tengsizlikning barcha butun yechimlari yig‘indisini toping: \(x^2 + 5x + 3 \leq 0\).
====
B. 3,8,13,18,... arifmetik progressiyaning nechta hadi 520 dan kichik?
====
B2. Soddalashtiring: \(\frac{\cos^{2}a \cdot {ctg}^{2}a}{\sin^{2}a}\)
====
B3. A to‘plamda 16 ta element, B to‘plamda 18 ta element bor. AЅB to‘plamda minimal miqdordagi elementlar qancha bo‘lishi mumkin?
====
C1. Tenglamani yeching: \((\sqrt{13}) ^{x + 20} = (\sqrt[3]{15}) ^{x + 20}\)
====
C2. Tenglamaning eng katta manfiy ildizini graduslarda toping: \(2sin11xsin5x = \frac{\sqrt{3}}{2} - cos16x\).
====
C3. ABC uchburchakda AB va BC tomonlarga o‘tkazilgan medianalar o‘zaro perpendikulyar.
====
1. O‘qituvchini darsga tayyorlash tizimi (yangi o‘quv yiliga tayyorlash, o‘qituvchini navbatdagi darsga tayyorlash).
====
2. Matematika fanini o‘qitish metodlari va shakllari (matematika fanini o‘qitish metodlari, metodlari va shakllarining turlari).
====
A1. 32 soni 160 sonining necha qismini tashkil qiladi?
====
A2. Sonlar o‘qida -7,5 va 4,2 sonlari orasida nechta butun son joylashgan?
====
A3. 10,14,18,... arifmetik progressiyada 110 sonining tartib raqamini toping
====
B1. Funksiyaning aniqlanish sohasini toping: \(y = \frac{\sqrt{3^{x} - 27}}{8 - 2^{x}}\)
====
B2. Markazlari A, B, C, D va E nuqtalarda bo‘lgan beshta aylana tashqi urinadi. Agar AB=46, BC=44, CD=47, DE=43, AE=44 bo‘lsa, eng katta radiusli aylana markazi qaysi nuqtada joylashgan?
====
B3. Tenglamaning haqiqiy ildizlari yig‘indisini toping: \((x-3) \sqrt{x^{2} - 3x + 6} = 2x - 6\)
====
C1. Tenglamaning kichik ildizini toping: \(log_{6}x \cdot log_{4}x = log_{6}4\)
====
C2. Hisoblang: \(\sqrt{0,5 (13 + 3\sqrt{17}) } + \sqrt{0,5 (13 - 3\sqrt{17}) }\)
====
C3. To‘g‘ri burchakli trapetsiyada katta diagonal trapetsiyaning o‘tkir burchagining bissektrisasi hisoblanadi.Bu diagonalning uzunligi 8 ga, pastki burchak uchidan bu diagonalgacha bo‘lgan masofa 2 ga teng.