1. Цели преподавания математика в средней школе (общеобразовательные цели, воспитательные цели, практические цели).
====
2. Математические суждение и умозаключение (виды суждение, аксиомы, постулаты, теоремы).
====
A1. Вычислите: \(\frac{1\  \cdot \ 2\  \cdot \ 3\  + \ 3\  \cdot \ 6\  \cdot \ 9\  + \ 5\  \cdot \ 10\  \cdot \ 15\  + \ 7\  \cdot \ 14\  \cdot \ 21}{2\  \cdot \ 4\  \cdot \ 6\  + \ 6\  \cdot \ 12\  \cdot \ 18\  + \ 10\  \cdot \ 20\  \cdot \ 30\  + \ 14\  \cdot \ 28\  \cdot \ 42}\).
====
A2. Айбек решил две задачи за 36 минут. На решение первой задачи он потратил 6 минут больше, чем на решение второй задачи. За сколько минут он решил вторую задачу?
====
A3. Угол при вершине равнобедренного треугольника равен 16º. Найти тупой угол, образованный боковой стороной и биссектрисой угла при основании.
====
B1. Множество A содержит 16 элементов, а множество B содержит 18 элементов. Какое минимальное количество элементов может содержать множество AՍB?
====
B2. Решите неравенство: \(\frac{(x + 1)(3 - 2x)}{x - 5} \leq 0\).
====
B3. Пять окружностей с центрами в точках A, B, C, D и E касаются внешним образом. В какой точке находится центр окружности наибольшего радиуса, если AB=46, BC=44, CD=47, DE=43, AE=44?
====
C1. Вычислите:\(\frac{2sin^{2}70^{0} - 1}{2ctg115^{0}cos155^{0}}\)
====
C2. Найти найменьшее число из области определения функции \(f(x) = \sqrt{log_{12,5}x - log_{x}12,5}\)
====
C3. В прямоугольном тр-ке АВС из вершины прямого угла С проведены медиана СМ,биссектриса СЛ и высота СН.Найти длину СЛ, если СМ=50,СН=14.
====
1. Характеристика математика как наука и как учебного предмета (периоды развитие математики, первые учебники по математике).
====
2. Аналогия в преподавание математики (виды аналогии, применение аналогии)
====
A1. Найти порядковый номер числа 110 в арифметической прогрессии 10,14,18, \ldots{}
====
A2. Вычислите: \(arcsin(sin3)\).
====
A3. Какую часть числа 160 составляет число 32?
====
B1. Вычислить: \(\frac{\sqrt{2} - 15}{\sqrt{\sqrt{2} + 1} - 4} - \sqrt{\sqrt{2} + 1} + 1\)
====
B2. Сколько действительных корней имеет уравнение: \({(\frac{|x| + x}{x - 2})}^{2} - \frac{12x}{x - 2} + 5 = 0\)
====
B3. Если из последовательности натуральных чисел 2,3,4,...,13 случайным образом выбрать одно число, то какова вероятностьтого, что выбранное число будет простым?
====
C1. Вычислить:\(\sqrt{0,5(13 + 3\sqrt{17})} + \sqrt{0,5(13 - 3\sqrt{17})}\)
====
C2. Сколько целых чисел входит в область решений неравенства:\(log_{\frac{x}{20}}(log_{x}\sqrt{20 - x}) \succ 0\)
====
C3. В равнобедренном треугольнике АВС(АВ=ВС) точка О-центр вписанный окружности радиус которого равен 4,2 ,а радиус окружности описанный около тр-ка АОС равен 14.Найти радиус окружности описанный около тр-ка АВС.
====
1. Научные методы в математике и ее преподавании (наблюдение, опыт, сравнение и т.д.)
====
2. Математические понятия (процесс формирование, обобщение, содержание и объем, классификация понятий).
====
A1. Вычислите: \(tg\left( arctg2 - arctg\frac{1}{2} \right)\).
====
A2. Найдите \(\log_{\sqrt{3}}\sqrt[6]{a}\) , если \(\log_{a}27 = b\).
====
A3. Айбек решил две задачи за 36 минут. На решение первой задачи он потратил 6 минут больше, чем на решение второй задачи. За сколько минут он решил вторую задачу?
====
B1. Сколько членов арифметической прогрессии 3,8,13,18,... меньше 520?
====
B2. Найдите область определения функции: \(y = \frac{\sqrt{3^{x} - 27}}{8 - 2^{x}}\)
====
B3. В пирамиде количество всех диагоналей основания равно количеству всех рёбер пирамиды. Найдите сумму количеств всех граней и вершин данной пирамиды.
====
C1. Вычислите:\(\frac{2sin^{2}70^{0} - 1}{2ctg115^{0}cos155^{0}}\)
====
C2. Найти найменьшее число из области определения функции \(f(x) = \sqrt{log_{12,5}x - log_{x}12,5}\)
====
C3. В прямоугольной трапеции большая диагональ является биссектрисой острого угла трареции.Длина этой диагонали равна 8,расстояние от вершины тупого угла до этой диагонали равно 2.Найти площадь трапеции.
====
1. Анализ и синтез в преподавании математики (формы анализа, применение при доказательствах, при решениях задач)
====
2. Формы математических мышлении (конкретное, абстрактное, интуитивное, функциональное, диалектическое, творческое).
====
A1. Упростите выражение: \(\frac{2^{3n - 4} \bullet 2^{5 + 6n}}{2^{1 + 3n}}\).
====
A2. При каком значении x справедливо выражение: \(3(2 - x) - 8 = 10\)?
====
A3. Найти сумму действительных корней уравнения: \((x^{2} + 14x + 14)(x^{2} + x + 14) = 14x^{2}\)
====
B1. Вычислить интешрал: \(\int_{}^{}{\frac{1}{\cos^{2}(2x - 3)}dx}\)
====
B2. Найдите вектор, являющийся проекцией вектора \(\overline{a}( - 12,13, - 15)\ \)на плоскости Oxy.
====
B3. Цилиндр описан около сферы. Площадь поверхности сферы равна 32π. Найдите площадь полной поверхности.
====
C1. Найти больший корень уравнения:\(\sqrt[3]{22 + 6x} + \sqrt[3]{15 - 6x} = 1\)
====
C2. Найти найменьшее решение уравнения:\(\left| 2x^{2} - 10x + 8 \right| = 13x - 22\).
====
C3. Окружность вписанная в равнобедренный тр-к АВС(АВ=ВС) проходит через точку пересечения высот этого тр-ка,Найти длину АВ,если АС=6.
====
1. Обобщение и абстрагирование в преподавание математики (взаимосвязи, формы абстракции).
====
2. Эвристический метод обучение математике (использование при изучении теорем, при решении задач).
====
A1. Упростите выражение \(|x - 8| + |x - 6|\) , если \(2^{x} = 152\).
====
A2. Найдите сумму корней уравнения: \(x^{2} - 4|x| - a + 3 = 0\) при \(a \geq 3\).
====
A3. Найти основной период функции: \(y = \frac{1}{2}\sin{\frac{x}{2}\cos\frac{x}{2}}\) .
====
B1. Упростите: \(\frac{\cos^{2}a \bullet {ctg}^{2}a}{\sin^{2}a}\)
====
B2. Найдите значение алгебраического выражения: 0,125xy-0.5x+1, при x=10, y=8.
====
B3. Найдите сумму действительных корней уравнения: \((x - 3)\sqrt{x^{2} - 3x + 6} = 2x - 6\)
====
C1. Найти меньший корень уравнения:\(log_{6}x \cdot log_{4}x = log_{6}4\)
====
C2. Упростить и вычислить при \(a = \frac{2\pi}{15}:\frac{4(cos3a - cos8a)}{\sqrt{31}(sin3a + sin8a)}\)
====
C3. В равнобедренном тр-ке АВС(АВ=ВС) точка О-центр вписанной окружности,Площадь тр-ка АВС равна 300,площадь тр-ка АОС равна 112,5. Найти длину стораны АВ.
====
1. Методика введение математических понятий в школьном курсе математики (конкретно-индуктивный метод, абстрактно-дедуктивный метод).
====
2. Математический стиль мышления (гибкость, активность, целенаправленность, готовность, широта, глубина, критичность и самокритичность мышления).
====
A1. Найти сумму действительных корней уравнения: \((x^{2} + 14x + 14)(x^{2} + x + 14) = 14x^{2}\)
====
A2. Решите неравенство: \(\frac{arccos( - \frac{3}{\pi}) \bullet \log_{\frac{3}{\pi}}\frac{\pi}{4}}{1 - 2\log_{\log_{2}x}2} \geq 0\).
====
A3. Угол при вершине равнобедренного треугольника равен 16º. Найти тупой угол, образованный боковой стороной и биссектрисой угла при основании.
====
B1. Решите неравенство: \(25^{\log_{5}{(1 - 2x)}} + {(2x - 1)}^{2} < 50\)
====
B2. Укажите уравнение прямой, симметричной прямой y=-8x+3 относительно x=1.
====
B3. Определите значение n? если шестизначное число 553n52 делится на 18 без остатка.
====
C1. Решить уравнение:\((\sqrt{13})^{x + 20} = (\sqrt[3]{15})^{x + 20}\)
====
C2. Найти в градусах найбольший отрицательный корень уравнения: \(2sin11xsin5x = \frac{\sqrt{3}}{2} - cos16x\).
====
C3. Окружность радиуса 30 касается гипотенузы прямоугольного тр-ка и продолжений обоих катетов,Найти длину меньшего катета, если длина гипотенузы равна 26.
====
1. Роль и место задачи в обучении математике (текстовые задачи, логические задачи, задачи шутки).
====
2. Наглядные пособия и технические средства в преподавание математике (наглядность, модели, таблицы, компьютеры и дидактические материалы к ним).
====
A1. Найти сумму всех целых решений неравенства: \(x^{2} + 5x + 3 \leq 0\).
====
A2. Найти основной период функции: \(y = \frac{1}{2}\sin{\frac{x}{2}\cos\frac{x}{2}}\) .
====
A3. При каком значении x справедливо выражение: \(3(2 - x) - 8 = 10\)?
====
B1. Приведите произведение \(0,003 \bullet 0,004 \bullet 10^{8}\) к стандартному виду.
====
B2. Если a=16-\(x^{2}\) , b=\(x^{2} - 4\) и a,b натуральные числа, то найдите наименьшее значение \(a^{2} + b^{2}\).
====
B3. Определите значение n если шестизначное число 553n52 делится на 18 без остатка.
====
C1. Найти в градусах наименьший положительный корень уравнения: \(tg30xtg97^{0} + tg97^{0}tg38^{0} + tg38^{0}tg30x = 1\).
====
C2. Найти найменьшее решение уравнения:\(\left| 2x^{2} - 10x + 8 \right| = 13x - 22\).
====
C3. Около прямоугольного тр-ка АВС описана окружность.Биссектриса острого угла ВАС пересекает катет ВС в точке М, а окружность- в точке К.Известно, что АМ=14,МК=3. Найти \(\cos\angle BAC\).
====
1. Основные дидактические принципы в обучение математике (принцип научности, принцип воспитания, принцип наглядности, принцип прочности знании и т.д.).
====
2. Внеклассные работы по математике (кружки, группы, работа с отстающими, работа с успевающими).
====
A1. Найти производную функции: \(y = 5\sin{9x} + 3\sin{15x}\) .
====
A2. Решите уравнение: \(\frac{x - 3}{x - 1} + \frac{x + 3}{x + 1} = \frac{x + 6}{x + 2} + \frac{x - 6}{x - 2}\) .
====
A3. Вычислите: \(tg\left( arctg2 - arctg\frac{1}{2} \right)\).
====
B1. Определите значение n если шестизначное число 553n52 делится на 18 без остатка.
====
B2. В пирамиде количество всех диагоналей основания равно количеству всех рёбер пирамиды. Найдите сумму количеств всех граней и вершин данной пирамиды.
====
B3. Решите неравенство: \(\frac{(x + 1)(3 - 2x)}{x - 5} \leq 0\).
====
C1. Вычислите:\(\frac{2sin^{2}70^{0} - 1}{2ctg115^{0}cos155^{0}}\)
====
C2. Упростить и вычислить при \(a = \frac{2\pi}{15}:\frac{4(cos3a - cos8a)}{\sqrt{31}(sin3a + sin8a)}\)
====
C3. Решить уравнение:\((\sqrt{13})^{x + 20} = (\sqrt[3]{15})^{x + 20}\)
====
1. Образование в Узбекистане, закон об образования (реформы образования, система образования, программы образования).
====
2. Метод активного обучения (модели, передовые педагогические методы).
====
A1. Найти первообразную функции: \(y = \cos{3x}\cos{12x}\) .
====
A2. Решите уравнение: \(3,6x - 7,4x = 1,3 - 4,8x\) .
====
A3. Найти сумму всех целых решений неравенства: \(x^{2} + 5x + 3 \leq 0\).
====
B1. Сколько членов арифметической прогрессии 3,8,13,18,... меньше 520?
====
B2. Упростите: \(\frac{\cos^{2}a \bullet {ctg}^{2}a}{\sin^{2}a}\)
====
B3. Множество A содержит 16 элементов, а множество B содержит 18 элементов. Какое минимальное количество элементов может содержать множество AՍB?
====
C1. Решить уравнение:\((\sqrt{13})^{x + 20} = (\sqrt[3]{15})^{x + 20}\)
====
C2. Найти в градусах найбольший отрицательный корень уравнения: \(2sin11xsin5x = \frac{\sqrt{3}}{2} - cos16x\).
====
C3. В тр-ке АВС медианы, проведенные к сторонам АВ и ВС взаимно перпендикулярны.Найти длину АС,если АВ=\(\sqrt{14}\)и ВС=\(\sqrt{31}\).
====
1. Система подготовка учителя к урокам (подготовка к новому учебному году, подготовка учителя к очередному уроку).
====
2. Методы и формы обучения математике (методы обучения , типы методов и форм обучения математике).
====
A1. Какую часть числа 160 составляет число 32?
====
A2. Сколько целых чисел расположено между числами -7,5 и 4,2 на числовой оси?
====
A3. Найти порядковый номер числа 110 в арифметической прогрессии 10,14,18, \ldots{}
====
B1. Найдите область определения функции: \(y = \frac{\sqrt{3^{x} - 27}}{8 - 2^{x}}\)
====
B2. Пять окружностей с центрами в точках A, B, C, D и E касаются внешним образом. В какой точке находится центр окружности наибольшего радиуса, если AB=46, BC=44, CD=47, DE=43, AE=44?
====
B3. Найдите сумму действительных корней уравнения: \((x - 3)\sqrt{x^{2} - 3x + 6} = 2x - 6\)
====
C1. Найти меньший корень уравнения:\(log_{6}x \cdot log_{4}x = log_{6}4\)
====
C2. Вычислить:\(\sqrt{0,5(13 + 3\sqrt{17})} + \sqrt{0,5(13 - 3\sqrt{17})}\)
====
C3. В прямоугольной трапеции большая диагональ является биссектрисой острого угла трареции.Длина этой диагонали равна 8,расстояние от вершины тупого угла до этой диагонали равно 2.Найти площадь трапеции.