\documentclass[10pt]{article}
\usepackage[turkish]{babel}
\usepackage[utf8]{inputenc}
\usepackage[T1]{fontenc}
\usepackage{amsmath}
\usepackage{amsfonts}
\usepackage{amssymb}
\usepackage[version=4]{mhchem}
\usepackage{stmaryrd}

\begin{document}
\textbf{2.3.26.} Berilgan $\vec{a}(n ; 2 n+1 ; 1-n), \vec{b}(n+1 ; n-1 ; \lambda)$ va $\vec{c}(n-1 ; 3 n ; 1)$ vektorlar $\lambda$ parametrning qanday qiymatida komplanar bo'ladi?\\
\textbf{2.3.27.} $\lambda$ parametrning qanday qiymatida $\vec{a}(\lambda n ; n-2 ; n+1)$ va $\vec{b}(n-3 ; \lambda n ; n-1)$ vektorlar ortogonal bo' lishini aniqlang.\\
\textbf{2.3.28.} $\vec{x}(n ; n+4 ; n-1)$ vektorni $\vec{e}_{1}(1 ; 1 ; 0), \quad \vec{e}_{2}(1 ; 0 ; 1)$ va $\vec{e}_{3}(0 ; 1 ; 1)$ bazisdagi yoyilmasini toping .\\
\textbf{2.3.29.} $\vec{a}(2 n ; n+3 ; n-1), \vec{b}(n ; 2 n-13 ; 4 n)$ va $\vec{c}(2 n ; 13-5 n ;-13n)$ vektorlar chiziqli bog'liq ekanligini ko'rsating va bu bog'lanishni toping.\\
\textbf{4.1.16.} $\vec{a}(3 ; 4), \vec{b}(6 ;-7)$ vektorlar berilgan. Bir vaqtning o'zida ikkita $\vec{a} \vec{x}=2, \vec{b} \vec{x}=19$ tenglamani qanoatlantiradigan $\vec{x}$ vektor topilsin.\\
\textbf{4.1.17.} $\vec{a}(3 ;-2 ; 4), \vec{b}(5 ; 1 ; 6), \vec{c}(-3 ; 0 ; 2)$ vektorlar berilgan. Bir vaqtning o'zida $\vec{a} \cdot \vec{x}=4, \vec{b} \cdot \vec{x}=35, \vec{c} \cdot \vec{x}=0 \quad$ tenglamalarni qanoatlantiradigan $\vec{x}$ vektor topilsin.\\
\textbf{4.1.18.} $\vec{a}(2 ; 1 ;-1)$ vektorga kollinear va $\vec{x} \vec{a}=3$ shartni qanoatlantiruvchi $\vec{x}$ vektorni toping.\\
\textbf{4.1.19.} $\vec{x}$ vektor $\vec{a}=3 \vec{\imath}+2 \vec{\jmath}+2 \vec{k}$ va $\vec{b}=18 \vec{\imath}-22 \vec{\jmath}-5 \vec{k}$ vektorlarga perpendikulyar, Oy o'qi bilan o'tmas burchak hosil qiladi. $|\vec{x}|=14$ bo'lsa, uning koordinatalarini toping.\\
\textbf{4.1.20.} Uchta $\vec{a}=2 \vec{\imath}-\vec{\jmath}+3 \vec{k}, \quad \vec{b}=\vec{\imath}-3 \vec{\jmath}+2 \vec{k}$ va $\vec{c}=3 \vec{\imath}+2 \vec{\jmath}-$ $-4 \vec{k}$ vektorlar berilgan. $\vec{x} \vec{a}=-5, \vec{x} \vec{b}=-11$ va $\vec{x} \vec{c}=20$ shartlarni qanoatlantiruvchi $\vec{x}$ vektorni toping.\\
\textbf{4.1.21.} Tomonlari birga teng bo'lgan teng tomonli $A B C$ uchburchak berilgan. $\overrightarrow{B C}=\vec{a}, \quad \overrightarrow{C A}=\vec{b}, \quad \overrightarrow{A B}=\vec{c}$ deb $\vec{a} \vec{b}+\vec{b} \vec{c}+\vec{a} \vec{c} \quad$ ifoda hisoblansin.\\
\textbf{4.2.5.} $\vec{a}$ va $\vec{b}$ vektorlar o'zaro $\varphi=\frac{\pi}{6}$ burchak hosil qiladi. Agar $|\vec{a}|=6$, $|\vec{b}|=5$ bo'lsa, $|[\vec{a} \vec{b}]|$ ni hisoblang.\\
\textbf{4.2.6.} $|\vec{a}|=10,|\vec{b}|=2$ va $\vec{a} \vec{b}=12$ berilgan bo'lsa, $|[\vec{a} \vec{b}]| \mathrm{ni}$ hisoblang.\\
\textbf{4.2.7.} $|\vec{a}|=3,|\vec{b}|=26$ va $|[\vec{a} \vec{b}]|=72$ bo'lsa, $\vec{a} \vec{b}$ ni toping.\\
\textbf{4.2.8.} $\vec{a}$ va $\vec{b}$ vektorlar o'zaro perpendikulyar. $|\vec{a}|=3$ va $|\vec{b}|=4 \mathrm{ni}$ bilgan holda, quyidagilarni hisoblang:

\begin{enumerate}
  \item $|[(\vec{a}+\vec{b})(\vec{a}-\vec{b})]|$;
  \item $|[(3 \vec{a}-\vec{b})(\vec{a}-2 \vec{b})]|$.
\end{enumerate}\\
\textbf{4.2.9.} $\vec{a}$ va $\vec{b}$ vektorlar o'zaro $\varphi=\frac{2 \pi}{3}$ burchak hosil qiladi. $|\vec{a}|=1$, $|\vec{b}|=2$ ni bilgan holda, quyidagilarni hisoblang:\\
1. $[\vec{a}, \vec{b}]^{2}$; \\
2. $[(2 \vec{a}+\vec{b})(\vec{a}+2 \vec{b})]^{2}$ \\
3. $[(\vec{a}+3 \vec{b})(3 \vec{a}-\vec{b})]^{2}$. \\


\end{document}