\(\vec{a} + \vec{b} + \vec{c} = 0\) shartni qanoatlantiruvchi birlik \(\vec{a},\ \vec{b}\) va \(\vec{c}\) vektorlar berilgan. Hisoblang: \(\left(\vec{a},\vec{b} \right) + \left(\vec{b},\vec{c} \right) + \left(\vec{c},\vec{a} \right) \).
====
\(\vec{a} + \vec{b}\) vektor \(\vec{a} - \vec{b}\) vektorga perpendikulyar bo‘lishi uchun \(\vec{a}\) va \(\vec{b}\) vektorlar qanday shartlarni qanoatlantirishi kerak?
====
\(\vec{p} = \vec{b} (\vec{a},\vec{c}) - \vec{c} (\vec{a},\vec{b}) \) vektor \(\vec{a}\) vektorga perpendikulyar ekanini isbotlang.
====
\(\vec{p} = \vec{b} - \frac{\vec{a} (\vec{a},\vec{b}) }{{\vec{a}}^{2}}\) vektor \(\vec{a}\) vektorga perpendikulyar ekanini isbotlang.
====
\(ABC\) uchburchakning tomonlari bilan mos keluvchi \(\vec{AB} = \vec{b}\) va \(\vec{AC} = \vec{c}\) vektorlar berilgan. Bu uchburchakning \(B\) uchidan tushirilgan \(BD\) balandligining \(\vec{b},\ \vec{c}\) bazis bo‘yicha yoyilmasini toping.
====
\(\vec{a}+\vec{b}\) va \(\vec{a} - \vec{b}\) vektorlar kollinear bo‘lishi uchun \(\vec{a},\vec{b}\) vektorlar qanday shartni qanoatlantirishi kerak?
====
Ayniyatni isbotlang: \(\lbrack\vec{a},\vec{b}\rbrack^{2} + (\vec{a},\vec{b}) ^{2} = {\vec{a}}^{2}{\vec{b}}^{2}\).
====
\(\lbrack\vec{a},\vec{b}\rbrack^{2} < {\vec{a}}^{2}{\vec{b}}^{2}\) ekanini isbotlang; qanday holda bu yerda tenglik ishorasi bo‘ladi?
====
\(\vec{a},\ \vec{b}\) va \(\vec{c}\) vektorlar \(\vec{a} + \vec{b} + \vec{c} = 0\) shartni qanoatlantiradi. \(\lbrack\vec{a},\vec{b}\rbrack = \lbrack\vec{b},\vec{c}\rbrack = \lbrack\vec{c},\vec{a}\rbrack\) ekanini isbotlang.
====
Ayniyatni isbotlang: \((\lbrack\vec{a} + \vec{b},\vec{b} + \vec{c}\rbrack,\vec{c} + \vec{a}) = 2 (\lbrack\vec{a},\vec{b}\rbrack,\vec{c}) \).
====
Ayniyatni isbotlang: \((\lbrack\vec{a},\vec{b}\rbrack,\vec{c} + \lambda\vec{a} + \mu\vec{b}) = (\lbrack\vec{a},\vec{b}\rbrack,\vec{c}) \), bunda \(\lambda\) va \(\mu\) - ixtiyoriy sonlar.