Berilgan $M_1 (3; 1) $, $M_2 (2; 3) $, $M_3 (6; 3) $,
$M_4 (-3;-3) $. $M_5 (3;-1) $, $M_6 (-2; 1) $ nuqtalarning qaysilari
$2x-3y-3 = 0$ to‘g‘ri chiziqqa tegishli va qaysilari tegishli
emas.
====
$P1$, $P2$, $P3$, $P4$, $P5$ nuqtalar
3x-2y-6=0 to‘g‘ri chiziqqa tegishli va abssissalari mos ravishda
4, 0, 2, -2, -6 ga teng. Ularning ordinatalarini toping.
====
$Q_1$, $Q_2$, $Q_3$, $Q_4$, $Q_5$ nuqtalar
$x-3y+2=0$ to‘g‘ri chiziqqa tegishli va ordinatalari mos ravishda
1, 0, 2, -1, 3 ga teng. Ularning abssissalarini toping.
====
$5x-y+3=0$ to‘g‘ri chiziqning $k$ burchagi
koeffitsiyentini va $Oy$ o‘qidan kesib olgan kesmaning algebraik
qiymati $b$ ni aniqlang.
====
$2x+3y-6=0$ to‘g‘ri chiziqning $k$ burchagi
koeffitsiyentini va $Oy$ o‘qidan kesib olgan kesmaning algebraik
qiymati $b$ ni aniqlang.
====
$5x+3y+2=0$ to‘g‘ri chiziqning $k$ burchagi
koeffitsiyentini va $Oy$ o‘qidan kesib olgan kesmaning algebraik
qiymati $b$ ni aniqlang.
====
$3x+2y=0$ to‘g‘ri chiziqning $k$ burchagi
koeffitsiyentini va $Oy$ o‘qidan kesib olgan kesmaning algebraik
qiymati $b$ ni aniqlang.
====
$y-3=0$ to‘g‘ri chiziqning $k$ burchagi
koeffitsiyentini va $Oy$ o‘qidan kesib olgan kesmaning algebraik
qiymati $b$ ni aniqlang.
====
Umumiy tenglama bilan berilgan to‘g‘ri chiziqlarning
o‘zaro joylashuvini aniqlang, agar kesishadigan bo‘lsa kesishish nuqtasini
toping: $12x+15y-39=0, 16x-9y-23=0$.
====
Umumiy tenglama bilan berilgan to‘g‘ri chiziqlarning
o‘zaro joylashuvini aniqlang, agar kesishadigan bo‘lsa kesishish nuqtasini
toping: $3x+2y-27=0, x+5y-35=0$.
====
Umumiy tenglama bilan berilgan to‘g‘ri chiziqlarning
o‘zaro joylashuvini aniqlang, agar kesishadigan bo‘lsa kesishish nuqtasini
toping: $12x+59y-19=0, 8x+33y-19=0$.
====
Umumiy tenglama bilan berilgan to‘g‘ri chiziqlarning
o‘zaro joylashuvini aniqlang, agar kesishadigan bo‘lsa kesishish nuqtasini
toping: $6x+10y+9=0, 3x+5y-6=0$.
====
Umumiy tenglama bilan berilgan to‘g‘ri chiziqlarning
o‘zaro joylashuvini aniqlang, agar kesishadigan bo‘lsa kesishish nuqtasini
toping: $14x-9y-24=0, 7x-2y-17=0$.
====
Umumiy tenglama bilan berilgan to‘g‘ri chiziqlarning
o‘zaro joylashuvini aniqlang, agar kesishadigan bo‘lsa kesishish nuqtasini
toping: $2x-3y+12=0, 4x-6y-21=0$.
====
Umumiy tenglama bilan berilgan to‘g‘ri chiziqlarning
o‘zaro joylashuvini aniqlang, agar kesishadigan bo‘lsa kesishish nuqtasini
toping: $2y+9=0, y-5=0$.
====
Umumiy tenglama bilan berilgan to‘g‘ri chiziqlarning
o‘zaro joylashuvini aniqlang, agar kesishadigan bo‘lsa kesishish nuqtasini
toping: $4x-7=0, 3x+8=0$.
====
Umumiy tenglama bilan berilgan to‘g‘ri chiziqlarning
o‘zaro joylashuvini aniqlang, agar kesishadigan bo‘lsa kesishish nuqtasini
toping: $2x-5y+1=0, 6x-15y+3=0$.
====
Umumiy tenglama bilan berilgan to‘g‘ri chiziqlarning
o‘zaro joylashuvini aniqlang, agar kesishadigan bo‘lsa kesishish nuqtasini
toping: $x-5=0, y+12=0$.
====
Umumiy tenglama bilan berilgan to‘g‘ri chiziqlarning
o‘zaro joylashuvini aniqlang, agar kesishadigan bo‘lsa kesishish nuqtasini
toping: $x\sqrt{2}+12=0, 4x+24\sqrt{2}=0$.
====
Umumiy tenglama bilan berilgan to‘g‘ri chiziqlarning
o‘zaro joylashuvini aniqlang, agar kesishadigan bo‘lsa kesishish nuqtasini
toping: $3x+y\sqrt{3}=0, x\sqrt{3}+3y-6=0$.
====
$a$ va $b$ parametrlarining qanday qiymatlarida
$ax-2y-1=0$, $6x-4y-b=0$ to‘g‘ri chiziqlar umumiy nuqtaga ega bo‘ladi?
====
$a$ va $b$ parametrlarining qanday qiymatlarida
$ax-2y-1=0$, $6x-4y-b=0$ to‘g‘ri chiziqlar parallel bo‘ladi?
====
$a$ va $b$ parametrlarining qanday qiymatlarida
$ax-2y-1=0$, $6x-4y-b=0$ to‘g‘ri chiziqlar kesishadimi?
====
$m$ va $n$ parametrlarining qanday qiymatlarida
$mx+8y+n=0$, $2x+my-1=0$ to‘g‘ri chiziqlar parallel bo‘ladi?
====
$m$ parametrining qanday qiymatlarida
$ (m-1) x+my-5=0$, $mx+ (2m-1) y+7=0$ to‘g‘ri chiziqlar abssissa
o‘qida yotuvchi nuqtada kesishadi.
====
$m$ parametrining qanday qiymatlarida
$mx+ (2m+3) y+m+6=0$, $ (2m+1) x+ (m-1) y+m-2=0$ to‘g‘ri chiziqlar ordinata
o‘qida yotuvchi nuqtada kesishadi.
====
$3x-y+2=0$, $4x-5y+5=0$, $2x+3y-1=0$
to‘g‘ri chiziqlar bir nuqtada kesishishadimi?
====
$5x+3y-7=0$, $x-2y-4=0$, $3x-y+3=0$
to‘g‘ri chiziqlar bir nuqtada kesishishadimi?
====
$x+2y-17=0$, $2x-y+1=0$, $x+2y-3=0$
to‘g‘ri chiziqlar bir nuqtada kesishishadimi?
====
$2x-y+2=0$, $4x-2y+4=0$, $6x-3y+6=0$
to‘g‘ri chiziqlar bir nuqtada kesishishadimi?
====
5x-3y+15=0 to‘g‘ri chiziqning koordinata burchagidan
kesib olgan uchburchakning yuzini hisoblang.
====
$M (-3;8) $ nuqtadan o‘tib, koordinata o‘qlaridan
teng kesmalarni kesib oladigan to‘g‘ri chiziqlarning tenglamasini tuzing.
====
$M (3;3)$ nuqtadan o‘tib, koordinata o‘qlaridan teng
kesmalarni kesib oladigan to‘g‘ri chiziqlarning tenglamasini tuzing.
====
$P (2;2)$ nuqtadan o‘tib, koordinata burchagidan
yuzi 1 ga teng uchburchak kesib oladigan to‘g‘ri chiziqlarning
tenglamasini tuzing.
====
$B (-5;5)$ nuqtadan o‘tib, koordinata burchagidan
yuzi 50 ga teng uchburchak kesib oladigan to‘g‘ri chiziqlarning tenglamasini
tuzing.
====
$P (8;6) $ nuqtadan o‘tib, koordinata burchagidan
yuzi 12 ga teng uchburchak kesib oladigan to‘g‘ri chiziqlarning tenglamasini
tuzing.
====
$P (12;6)$ nuqtadan o‘tib, koordinata burchagidan
yuzi 150 ga teng uchburchak kesib oladigan to‘g‘ri chiziqlarning
tenglamasini tuzing.
====
$M (4;3) $ nuqtadan, koordinata burchagidan
yuzi 3 ga teng uchburchak kesib oladigan to‘g‘ri chiziq o‘tkazildi.
Shu to‘g‘ri chiziqning koordinata o‘qlari bilan kesishish nuqtalari
koordinatalarini aniqlang.
====
$A (3;-2) $ nuqtadan $3x+4y-15=0$ to‘g‘ri chiziqqa
gacha siljishni va masofani hisoblang.