Ikkita qarama-qarshi uchlari \(P (4;9) \) va \(Q (-2; 1) \) nuqtalarida joylashgan romning tomon uzunligi \(5\sqrt{10}\). Shu
romba yuzini hisoblang.
====
Ikkita qarama-qarshi uchlari $P (3; -4) $ va $Q (l;2) $ nuqtalarda joylashgan rombaning tomon uzunligi \(5\sqrt{2}\). Shu romb balandligining uzunligini hisoblang.
====
Uchlari $A_1 (1; 1), A_2 (2; 3) $ va $A (5;-1) $
nuqtalarida joylashgan uchburchakning to‘g‘ri burchakli ekanini isbotlang.
====
Uchlari \(M_{1} (1;1), M_{2} (0,2) \) va
\(M_{3} (2;-1) \) nuqtalarda joylashgan uchburchakning ichki 
burchaklari orasida o‘tmas burchak bor yoki yo‘qligini aniqlang.
====
Uchlari \(M (-1;3),\ N (1,2) \ \) va \(P (0;4) \)
nuqtalarida joylashgan uchburchakning ichki burchaklari o‘tkir burchak
ekanligini isbotlang.
====
Uchburchakning uchlari \(A (5;0),\ B (0;1) \) va \(C (3;3) \)
nuqtalarida. Uning ichki burchaklarini toping.
====
Uchburchakning uchlari
\(A\left(-\sqrt{3};1 \right),\ B (0;2) \) va
\(C\left(-2\sqrt{3};2 \right) \) nuqtalarda. Uning $A$
uchidagi tashqi burchakni toping.
====
Abssissa o‘qida shunday $M$ nuqtani topingki,
\(N (2;-3) \) nuqtadan uzoqligi 5 ga teng bo‘lgan.
====
Ordinata o‘qida shunday $M$ nuqtani toping.
\(N (-8;13) \) nuqtadan uzoqligi 17 ga teng bo‘lgan.
====
Ikkita nuqta berilgan \(M (2;2) \) va \(N (5;-2) \); abssissa o‘qida shunday $P$ nuqtani topingki, $MPN$ burchak to‘g‘ri burchak bo‘lsin.
====
\(M_{1} (1;2) \) nuqtaga, \(A (1;0) \) va \(B (-1;-2) \)
nuqtalaridan o‘tuvchi to‘g‘ri chiziqqa nisbatan simmetrik bo‘lgan \(M_{2}\) nuqtaning koordinatalarini toping.
====
Uchburchakning uchlari \(A (2;-5),\ B (1;-2),\ C (4;7) \)
berilgan. $AC$ tomoni bilan $B$ uchining ichki burchagi
bissektrisasining kesishish nuqtasini toping.
====
Uchburchakning uchlari
\(A (3;-5),\ B (-3;3),\ C (-1;-2) \) berilgan. $A$ uchining ichki qismi
burchakli bessektrisaning uzunligini aniqlang.
====
Bir to‘g‘ri chiziqqa tegishli \(A (1;-1),\ B (3;3) \) va
\(C (4;5) \) nuqtalar berilgan. Har bir nuqtaning, qolgan ikki nuqta orqali aniqlanuvchi kesmani bo‘lish nisbati $\lambda$ ni aniqlang.
====
\(P (2;2) \) va \(Q (1;5) \) nuqtalar bilan teng uchta
bo‘lingan kesmaning uchlari $A$ va $B$ nuqtalarning
koordinatalarini aniqlang.
====
To‘g‘ri \(M_{1} (-12;-13) \) va \(M_{2} (-2;-5) \)
nuqtalaridan o‘tadi. Shu to‘g‘ri chiziqda abssissasi 3 ga teng nuqtani toping.
====
To‘g‘ri chiziq \(M (2;-3) \) va \(N (-6;5) \) nuqtalardan o‘tadi.
Shu to‘g‘ri chiziqda ordinatasi $-5$ ga teng nuqtani toping.
====
To‘g‘ri chiziq \(A (7;-3) \) va \(B (23;-6) \) nuqtalardan o‘tadi.
Shu to‘g‘ri chiziqning abssissa o‘qi bilan kesishish nuqtasini toping.
====
To‘g‘ri \(A (5;2) \) va \(B (-4; -7) \) nuqtalaridan o‘tadi.
Shu to‘g‘ri chiziqning ordinata o‘qi bilan kesishish nuqtasini toping.
====
To‘rtburchakning uchlari
\(A (-3;12),\ B (3;-4),\ C (5;-4) \) va \(D (5;8) \) berilgan. Shu
to‘rtburchakning $AC$ diagonali $BD$ diagonali qanday
nisbatda bo'lishini aniqlang.
====
To‘rtburchakning uchlari
\(A (-2;14),\ B (4;-2),\ C (6;-2) \) va \(D (6;10) \) berilgan. Shu
to‘rtburchakning $AC$ va $BD$ diagonallarining kesishishi
nuqtani toping.
====
Uchburchakning uchlari \(A (3;6),\ B (-1;3) \) va
\(C (2:-1) \) nuqtalarda joylashgan. $C$ uchidan tushirilgan balandlik uzunligini hisoblang.
====
Parallelogrammning uchta uchi \(A (3;7),\ B (2;-3) \) va
\(C (-1;4) \) nuqtalarda joylashgan. $B$ uchidan $AC$
tomonidan tushirilgan balandlik uzunligini hisoblang.
====
Ikkala uchi \(A (3;1) \) va \(B (1;-3) \) nuqtalarda, va
og‘irlik markazi $Ox$ o‘qiga tegishli uchburchakning yuzi
\(S=3\) ga teng. Uchinchi $C$ uchining koordinatalarini aniqlang.