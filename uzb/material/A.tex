$M_1 (1; -2) $, $M_2 (2; 1) $ nuqtalar berilgan.
Quyidagi kesmalarning koordinata o‘qlariga proyeksiyalarini toping: $\overline{M_1M_2}$ \\
====
Kvadratning ikkita qo‘shni uchlari $A (3; -7)$ va
$B (-1;4) $ berilgan. Uning yuzini hisoblang.
====
Kvadratning ikkita qarama-qarshi uchlari $P (3; 5) $ va
$Q (1; -3) $ berilgan. Uning yuzini hisoblang.
====
Ikkita uchi $A (-3; 2) $ va $B (1; 6) $ nuqtalarda
joylashgan muntazam uchburchakning yuzini hisoblang.
====
$ABCD$ parallelogrammning uchta uchi $A (3; -7) $,
$B (5; -7) $, $C (-2; 5) $ berilgan, to‘rtinchi uchi $D$,
$B$ uchiga qarama-qarshi. Shu parallelogrammning diagonallari
uzunliklarini aniqlang.
====
Berilgan $A (3; -5) $, $B (-2; -7)$ va
$C (18; 1) $ nuqtalar bir to‘g‘ri chiziqda yotishini isbotlang.
====
$A (2;2) $, $B (-1;6) $, $C (-5;3) $ va $D (-2;-1) $
nuqtalari kvadrat uchlari ekanini isbotlang.
====
Bir jinsli elementdan yasalgan qatorning uchlari
$A (3;-5) $ va $B (-1;1) $ nuqtalarda joylashgan. Uning og‘irligi
markazi koordinatasini aniqlang.
====
Bir jinsli elementdan yasalgan qatorning og‘irlik markazi
$M (1;4) $ nuqtada, bir uchi $P (-2;2) $ nuqtada joylashgan. Shu
qatorning ikkinchi uchi $Q$ ning koordinatalarini aniqlang.
====
Uchburchak uchlarining koordinatalari berilgan
$A (1;-3) $, $B (3;-5) $ va $C (-5;7) $. Tomonlarining o‘rtalarini
aniqlang.
====
$M (2;-1) $, $N (-1;4) $ va $P (-2;2) $ nuqtalar
uchburchak tomonlarining o‘rtalari. Uchlarining koordinatalarini
aniqlang.
====
Parallelogrammning uchlari
$A (3;-5) $, $B (5;-3) $, $C (-1;3) $ berilgan. $B$ tepasiga
qarama-qarshi joylashgan $D$ uchini aniqlang.
====
Parallelogrammning ikkita qo‘shni uchlari $A (-3;5) $, $B (1;7) $
va dioganallarining kesishish nuqtasi $M (1;1)$ berilgan. Qolgan ikki
cho‘qqisini aniqlang.
====
$ABCD$-parallelogrammning uchta uchi
$A (2;3) $, $B (4;-1) $ va $C (0;5) $ berilgan. To‘rtinchi $D$
cho‘qqisini toping.
====
Uchburchakning uchlari $A (1;4) $, $B (3;-9) $, $C (-5;2) $
berilgan. $B$ uchidan o‘tkazilgan mediana uzunligini aniqlang.
====
$A (1;-3) $ va $B (4;3) $ nuqtalarni tutashtiruvchi
kesma teng uch bo‘lakka bo‘lindi. Bo‘luvchi nuqtalarning koordinatalarini
aniqlang.
====
$A (4;2) $, $B (7;-2) $ va $C (1;6) $ nuqtalar bir jinsli
simdan yasalgan uchburchak uchlari. Shu uchburchakning og‘irligi
====
Uchlari $A (2;-3) $, $B (3;2) $ va $C (-2;5) $
nuqtalarida joylashgan uchburchaklarning yuzini hisoblang.
====
Uchlari $M_1 (-3;2) $, $M_2 (5;-2) $ va $M_3 (1;3) $
nuqtalarida joylashgan uchburchaklarning yuzini hisoblang.
====
Uchlari $M (3;-4) $, $N (-2;3) $ va $P (4;5) $
nuqtalarida joylashgan uchburchaklarning yuzini hisoblang.
====
Uch uchi $A (-2;3), \ B (4;-5) $ va
$C (-3;1)$ nuqtalarda joylashgan parallelogrammning yuzini aniqlang.
====
Bir jinsli to‘rtburchakli plastinkaning uchlari berilgan:
$A (2;1), \ B (5;3), \ C (-1;7) $ va $D (-7;5) $. Uning og‘irlik markazi
koordinatalarini aniqlang.
====
Bir jinsli beshburchakli plastinkaning uchlari berilgan:
$A (2;3), \ B (0;6), \ C (-1;5), \ D (0;1) $ va $E (1;1) $. Uning og‘irligi
markazi koordinatalarini aniqlang.
====
Ikkala uchi $A (3;1) $ va $B (1;-3) $ nuqtalarda, a
uchinchi $C$ uchi $Oy$ o‘qiga tegishli uchburchakning
yuzi $S=3$ ga teng. $C$ uchining koordinatalarini aniqlang.
====
Ikkala uchi $A (2;1) $ va $B (3;-2) $ nuqtalarda, va
uchinchi $C$ uchi $Ox$ o‘qiga tegishli bo‘lgan uchburchakning
yuzi $S=4$ ga teng. $C$ uchining koordinatalarini aniqlang.
====
Berilgan $M_1 (3; 1) $, $M_2 (2; 3) $, $M_3 (6; 3) $,
$M_4 (-3;-3) $. $M_5 (3;-1) $, $M_6 (-2; 1) $ nuqtalarning qaysilari
$2x-3y-3 = 0$ to‘g‘ri chiziqqa tegishli va qaysilari tegishli
emas.
====
$P1$, $P2$, $P3$, $P4$, $P5$ nuqtalar
3x-2y-6=0 to‘g‘ri chiziqqa tegishli va abssissalari mos ravishda
4, 0, 2, -2, -6 ga teng. Ularning ordinatalarini toping.
====
$Q_1$, $Q_2$, $Q_3$, $Q_4$, $Q_5$ nuqtalar
$x-3y+2=0$ to‘g‘ri chiziqqa tegishli va ordinatalari mos ravishda
1, 0, 2, -1, 3 ga teng. Ularning abssissalarini toping.
====
$5x-y+3=0$ to‘g‘ri chiziqning $k$ burchagi
koeffitsiyentini va $Oy$ o‘qidan kesib olgan kesmaning algebraik
qiymati $b$ ni aniqlang.
====
$2x+3y-6=0$ to‘g‘ri chiziqning $k$ burchagi
koeffitsiyentini va $Oy$ o‘qidan kesib olgan kesmaning algebraik
qiymati $b$ ni aniqlang.
====
$5x+3y+2=0$ to‘g‘ri chiziqning $k$ burchagi
koeffitsiyentini va $Oy$ o‘qidan kesib olgan kesmaning algebraik
qiymati $b$ ni aniqlang.
====
$3x+2y=0$ to‘g‘ri chiziqning $k$ burchagi
koeffitsiyentini va $Oy$ o‘qidan kesib olgan kesmaning algebraik
qiymati $b$ ni aniqlang.
====
$y-3=0$ to‘g‘ri chiziqning $k$ burchagi
koeffitsiyentini va $Oy$ o‘qidan kesib olgan kesmaning algebraik
qiymati $b$ ni aniqlang.
====
Umumiy tenglama bilan berilgan to‘g‘ri chiziqlarning
o‘zaro joylashuvini aniqlang, agar kesishadigan bo‘lsa kesishish nuqtasini
toping: $12x+15y-39=0, 16x-9y-23=0$.
====
Umumiy tenglama bilan berilgan to‘g‘ri chiziqlarning
o‘zaro joylashuvini aniqlang, agar kesishadigan bo‘lsa kesishish nuqtasini
toping: $3x+2y-27=0, x+5y-35=0$.
====
Umumiy tenglama bilan berilgan to‘g‘ri chiziqlarning
o‘zaro joylashuvini aniqlang, agar kesishadigan bo‘lsa kesishish nuqtasini
toping: $12x+59y-19=0, 8x+33y-19=0$.
====
Umumiy tenglama bilan berilgan to‘g‘ri chiziqlarning
o‘zaro joylashuvini aniqlang, agar kesishadigan bo‘lsa kesishish nuqtasini
toping: $6x+10y+9=0, 3x+5y-6=0$.
====
Umumiy tenglama bilan berilgan to‘g‘ri chiziqlarning
o‘zaro joylashuvini aniqlang, agar kesishadigan bo‘lsa kesishish nuqtasini
toping: $14x-9y-24=0, 7x-2y-17=0$.
====
Umumiy tenglama bilan berilgan to‘g‘ri chiziqlarning
o‘zaro joylashuvini aniqlang, agar kesishadigan bo‘lsa kesishish nuqtasini
toping: $2x-3y+12=0, 4x-6y-21=0$.
====
Umumiy tenglama bilan berilgan to‘g‘ri chiziqlarning
o‘zaro joylashuvini aniqlang, agar kesishadigan bo‘lsa kesishish nuqtasini
toping: $2y+9=0, y-5=0$.
====
Umumiy tenglama bilan berilgan to‘g‘ri chiziqlarning
o‘zaro joylashuvini aniqlang, agar kesishadigan bo‘lsa kesishish nuqtasini
toping: $4x-7=0, 3x+8=0$.
====
Umumiy tenglama bilan berilgan to‘g‘ri chiziqlarning
o‘zaro joylashuvini aniqlang, agar kesishadigan bo‘lsa kesishish nuqtasini
toping: $2x-5y+1=0, 6x-15y+3=0$.
====
Umumiy tenglama bilan berilgan to‘g‘ri chiziqlarning
o‘zaro joylashuvini aniqlang, agar kesishadigan bo‘lsa kesishish nuqtasini
toping: $x-5=0, y+12=0$.
====
Umumiy tenglama bilan berilgan to‘g‘ri chiziqlarning
o‘zaro joylashuvini aniqlang, agar kesishadigan bo‘lsa kesishish nuqtasini
toping: $x\sqrt{2}+12=0, 4x+24\sqrt{2}=0$.
====
Umumiy tenglama bilan berilgan to‘g‘ri chiziqlarning
o‘zaro joylashuvini aniqlang, agar kesishadigan bo‘lsa kesishish nuqtasini
toping: $3x+y\sqrt{3}=0, x\sqrt{3}+3y-6=0$.
====
$a$ va $b$ parametrlarining qanday qiymatlarida
$ax-2y-1=0$, $6x-4y-b=0$ to‘g‘ri chiziqlar umumiy nuqtaga ega bo‘ladi?
====
$a$ va $b$ parametrlarining qanday qiymatlarida
$ax-2y-1=0$, $6x-4y-b=0$ to‘g‘ri chiziqlar parallel bo‘ladi?
====
$a$ va $b$ parametrlarining qanday qiymatlarida
$ax-2y-1=0$, $6x-4y-b=0$ to‘g‘ri chiziqlar kesishadimi?
====
$m$ va $n$ parametrlarining qanday qiymatlarida
$mx+8y+n=0$, $2x+my-1=0$ to‘g‘ri chiziqlar parallel bo‘ladi?
====
$m$ parametrining qanday qiymatlarida
$ (m-1) x+my-5=0$, $mx+ (2m-1) y+7=0$ to‘g‘ri chiziqlar abssissa
o‘qida yotuvchi nuqtada kesishadi.
====
$m$ parametrining qanday qiymatlarida
$mx+ (2m+3) y+m+6=0$, $ (2m+1) x+ (m-1) y+m-2=0$ to‘g‘ri chiziqlar ordinata
o‘qida yotuvchi nuqtada kesishadi.
====
$3x-y+2=0$, $4x-5y+5=0$, $2x+3y-1=0$
to‘g‘ri chiziqlar bir nuqtada kesishishadimi?
====
$5x+3y-7=0$, $x-2y-4=0$, $3x-y+3=0$
to‘g‘ri chiziqlar bir nuqtada kesishishadimi?
====
$x+2y-17=0$, $2x-y+1=0$, $x+2y-3=0$
to‘g‘ri chiziqlar bir nuqtada kesishishadimi?
====
$2x-y+2=0$, $4x-2y+4=0$, $6x-3y+6=0$
to‘g‘ri chiziqlar bir nuqtada kesishishadimi?
====
5x-3y+15=0 to‘g‘ri chiziqning koordinata burchagidan
kesib olgan uchburchakning yuzini hisoblang.
====
$M (-3;8) $ nuqtadan o‘tib, koordinata o‘qlaridan
teng kesmalarni kesib oladigan to‘g‘ri chiziqlarning tenglamasini tuzing.
====
$M (3;3)$ nuqtadan o‘tib, koordinata o‘qlaridan teng
kesmalarni kesib oladigan to‘g‘ri chiziqlarning tenglamasini tuzing.
====
$P (2;2)$ nuqtadan o‘tib, koordinata burchagidan
yuzi 1 ga teng uchburchak kesib oladigan to‘g‘ri chiziqlarning
tenglamasini tuzing.
====
$B (-5;5)$ nuqtadan o‘tib, koordinata burchagidan
yuzi 50 ga teng uchburchak kesib oladigan to‘g‘ri chiziqlarning tenglamasini
tuzing.
====
$P (8;6) $ nuqtadan o‘tib, koordinata burchagidan
yuzi 12 ga teng uchburchak kesib oladigan to‘g‘ri chiziqlarning tenglamasini
tuzing.
====
$P (12;6)$ nuqtadan o‘tib, koordinata burchagidan
yuzi 150 ga teng uchburchak kesib oladigan to‘g‘ri chiziqlarning
tenglamasini tuzing.
====
$M (4;3) $ nuqtadan, koordinata burchagidan
yuzi 3 ga teng uchburchak kesib oladigan to‘g‘ri chiziq o‘tkazildi.
Shu to‘g‘ri chiziqning koordinata o‘qlari bilan kesishish nuqtalari
koordinatalarini aniqlang.
====
$A (3;-2) $ nuqtadan $3x+4y-15=0$ to‘g‘ri chiziqqa
gacha siljishni va masofani hisoblang.
====
Vektor koordinata o‘qlari bilan quyidagi burchaklarni hosil qila oladimi:
$\alpha = 45^{{^\circ}},\beta = 60^{{^\circ}},\gamma = 120^{{^\circ}}$.
====
Vektor koordinata o‘qlari bilan quyidagi burchaklarni hosil qila oladimi:
$\alpha = 45^{{^\circ}},\ \ \ \ \beta = 135^{{^\circ}},\ \gamma = 60^{{^\circ}}$.
====
Vektor koordinata o‘qlari bilan quyidagi burchaklarni hosil qilishi
mumkinmi: $\alpha = 90^{{^\circ}},\ \beta = 150^{{^\circ}}$,
$\gamma = 60^{{^\circ}}?$
====
Tekislikda ikkita vektor
$\overrightarrow{p} = \{ 2; - 3\}$, $\overrightarrow{q} = \{ 1;2\}$.
$\overrightarrow{a} = \{9;4\}$ vektorning
$\overrightarrow{p},\ \overrightarrow{q}$ bazis bo‘yicha yoyilmasi topilsin.
====
To‘rtburchakning uchlari berilgan:
$A (1; - 2;2) $, $B (1;4;0),C (- 4;1;1) $ va $D (- 5; -5;3) $. Uning diagonallari $AC$ va $BD$ o‘zaro
perpendikulyarligini isbotlang.
====
$\alpha$
qanday qiymatlarida 
$\overrightarrow{a} = \alpha\overrightarrow{i} - 3\overrightarrow{j} + 2\overrightarrow{k}$
va
$\overrightarrow{b} = \overrightarrow{i} + 2\overrightarrow{j} - \alpha\overrightarrow{k}$
vektorlar o‘zaro perpendikulyar bo‘lishini aniqlang.
====
$\overrightarrow{a} = \{ 2; - 4;4\}$ va $\overrightarrow{b} = \{ - 3;2;6\}$
vektorlar hosil qilgan burchak kosinusini hisoblang.
====
Uchburchakning uchlari
$A (- 1; - 2;4) $, $B (- 4; - 2;0) $ va $C (3; -2;1) $. Uning $B$ uchidagi
ichki burchakni aniqlang.
====
Uchburchakning uchlari
$A (3;2; 3) $, $B (5;1; - 1) $ va $C (1; -2;1) $. Uning $A$ uchidagi tashqi burchagi aniqlansin.
====
Uchlari $A (1;2;1), B (3;-1;7) $ va $C (7;4;-2) $ bo‘lgan uchburchakning
ichki burchaklarini hisoblab toping. Bu uchburchakning teng yonli ekanligini isbotlang.
====
$\overrightarrow{a}$ va $\overrightarrow{b}$ vektorlar
$\varphi = \pi/6$ burchak hosil qiladi.
$|\overrightarrow{a}| = 6,|\overrightarrow{b}| = 5$ ekanini bilib,
$\left| \left\lbrack \overrightarrow{a},\overrightarrow{b} \right\rbrack \right|$ kattalikni hisoblang.
====
Berilgan: $\overrightarrow{a}| = 10,|\overrightarrow{b}| = 2$ va
$\left(\overrightarrow{a},\overrightarrow{b} \right) = 12$. Hisoblang
$\left| \left\lbrack \overrightarrow{a},\overrightarrow{b} \right\rbrack \right|$.
====
Berilgan: $\overrightarrow{a}| = 3,|\overrightarrow{b}| = 26$ va
$\lbrack\overrightarrow{a},\overrightarrow{b}\rbrack| = 72$. Hisoblang
$\left(\overrightarrow{a},\overrightarrow{b} \right) $.
====
$\overrightarrow{a}
= \{ 1; - 1;3\}, \ \ \ \ \ \overrightarrow{b} = \{ - 2;1\}$, $\overrightarrow{c} = \{3; -2;5\}$ vektorlar berilgan. Hisoblang:
$ (\lbrack\overrightarrow{a},\overrightarrow{b}\rbrack,\overrightarrow{c}) $.
====
Agar \(a = \{ 2;3; - 1\}, \ \ \ \ b = \{ 1; - 1;3\}, \ \ \ \ c = \{ 1;9; - 11\}\) bo‘lsa, $\overrightarrow{a}, \overrightarrow{b}, \overrightarrow{c}$ vektorlar komplanar bo‘lishini tekshiring.
====
Agar \(a = \{ 3; - 2;1\},\ \ \ \ \ b = \{ 2;1;2\},\ \ \ \ c = \{ 3; - 1; - 2\}\) bo‘lsa, $\overrightarrow{a}, \overrightarrow{b}, \overrightarrow{c}$ vektorlar komplanar bo‘lishini tekshiring.
====
Agar \(a = \{ 2; - 1;2\}, \ \ \ \ b = \{ 1;2; - 3\}, \ \ \ \ c = \{ 3; - 4;7\}\) bo‘lsa, $\overrightarrow{a}, \overrightarrow{b}, \overrightarrow{c}$ vektorlar komplanar bo‘lishini tekshiring.