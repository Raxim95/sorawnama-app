Vektor koordinata o‘qlari bilan quyidagi burchaklarni hosil qila oladimi:
$\alpha = 45^{{^\circ}},\beta = 60^{{^\circ}},\gamma = 120^{{^\circ}}$.
====
Vektor koordinata o‘qlari bilan quyidagi burchaklarni hosil qila oladimi:
$\alpha = 45^{{^\circ}},\ \ \ \ \beta = 135^{{^\circ}},\ \gamma = 60^{{^\circ}}$.
====
Vektor koordinata o‘qlari bilan quyidagi burchaklarni hosil qilishi
mumkinmi: $\alpha = 90^{{^\circ}},\ \beta = 150^{{^\circ}}$,
$\gamma = 60^{{^\circ}}?$
====
Tekislikda ikkita vektor
$\overrightarrow{p} = \{ 2; - 3\}$, $\overrightarrow{q} = \{ 1;2\}$.
$\overrightarrow{a} = \{9;4\}$ vektorning
$\overrightarrow{p},\ \overrightarrow{q}$ bazis bo‘yicha yoyilmasi topilsin.
====
To‘rtburchakning uchlari berilgan:
$A (1; - 2;2) $, $B (1;4;0),C (- 4;1;1) $ va $D (- 5; -5;3) $. Uning diagonallari $AC$ va $BD$ o‘zaro
perpendikulyarligini isbotlang.
====
$\alpha$
qanday qiymatlarida 
$\overrightarrow{a} = \alpha\overrightarrow{i} - 3\overrightarrow{j} + 2\overrightarrow{k}$
va
$\overrightarrow{b} = \overrightarrow{i} + 2\overrightarrow{j} - \alpha\overrightarrow{k}$
vektorlar o‘zaro perpendikulyar bo‘lishini aniqlang.
====
$\overrightarrow{a} = \{ 2; - 4;4\}$ va $\overrightarrow{b} = \{ - 3;2;6\}$
vektorlar hosil qilgan burchak kosinusini hisoblang.
====
Uchburchakning uchlari
$A (- 1; - 2;4) $, $B (- 4; - 2;0) $ va $C (3; -2;1) $. Uning $B$ uchidagi
ichki burchakni aniqlang.
====
Uchburchakning uchlari
$A (3;2; 3) $, $B (5;1; - 1) $ va $C (1; -2;1) $. Uning $A$ uchidagi tashqi burchagi aniqlansin.
====
Uchlari $A (1;2;1), B (3;-1;7) $ va $C (7;4;-2) $ bo‘lgan uchburchakning
ichki burchaklarini hisoblab toping. Bu uchburchakning teng yonli ekanligini isbotlang.
====
$\overrightarrow{a}$ va $\overrightarrow{b}$ vektorlar
$\varphi = \pi/6$ burchak hosil qiladi.
$|\overrightarrow{a}| = 6,|\overrightarrow{b}| = 5$ ekanini bilib,
$\left| \left\lbrack \overrightarrow{a},\overrightarrow{b} \right\rbrack \right|$ kattalikni hisoblang.
====
Berilgan: $\overrightarrow{a}| = 10,|\overrightarrow{b}| = 2$ va
$\left(\overrightarrow{a},\overrightarrow{b} \right) = 12$. Hisoblang
$\left| \left\lbrack \overrightarrow{a},\overrightarrow{b} \right\rbrack \right|$.
====
Berilgan: $\overrightarrow{a}| = 3,|\overrightarrow{b}| = 26$ va
$\lbrack\overrightarrow{a},\overrightarrow{b}\rbrack| = 72$. Hisoblang
$\left(\overrightarrow{a},\overrightarrow{b} \right) $.
====
$\overrightarrow{a}
= \{ 1; - 1;3\}, \ \ \ \ \ \overrightarrow{b} = \{ - 2;1\}$, $\overrightarrow{c} = \{3; -2;5\}$ vektorlar berilgan. Hisoblang:
$ (\lbrack\overrightarrow{a},\overrightarrow{b}\rbrack,\overrightarrow{c}) $.
====
Agar \(a = \{ 2;3; - 1\}, \ \ \ \ b = \{ 1; - 1;3\}, \ \ \ \ c = \{ 1;9; - 11\}\) bo‘lsa, $\overrightarrow{a}, \overrightarrow{b}, \overrightarrow{c}$ vektorlar komplanar bo‘lishini tekshiring.
====
Agar \(a = \{ 3; - 2;1\},\ \ \ \ \ b = \{ 2;1;2\},\ \ \ \ c = \{ 3; - 1; - 2\}\) bo‘lsa, $\overrightarrow{a}, \overrightarrow{b}, \overrightarrow{c}$ vektorlar komplanar bo‘lishini tekshiring.
====
Agar \(a = \{ 2; - 1;2\}, \ \ \ \ b = \{ 1;2; - 3\}, \ \ \ \ c = \{ 3; - 4;7\}\) bo‘lsa, $\overrightarrow{a}, \overrightarrow{b}, \overrightarrow{c}$ vektorlar komplanar bo‘lishini tekshiring.