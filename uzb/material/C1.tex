\(A (4;2) \) nuqta orqali, ikkita koordinata o‘qlariga
urinma doira o‘tkazildi. Uning markazi $C$ ni va radiusi
$R$ ni toping.
====
\(M{1} (1; 2) \) nuqta orqali, radiusi 5 ga teng,
$Ox$ o‘qiga urinma aylana o‘tkazildi. Shu aylananing markazi
$S$ ni aniqlang.
====
Uchburchakning uchlari \(M_{1} (- 3;6),\ M_{2} (9; - 10) \)
va \(M_{3} (-5;4) \) berilgan. Shu uchburchakka tashqi chizilgan
aylana markazi $C$ va radiusi $R$ ni aniqlang.
====
Uchburchakning uchlari
\(A (- 1; - 1),\ B (3;5),\ C (- 4;1) \) berilgan. $A$ uchi tashqi
burchak bissektrisasining, $BC$ tomonining davomi bilan kesishish
nuqtani toping.
====
Uchburchakning uchlari
\(A (3; - 5),\ B (1; - 3),\ C (2; - 2) \) berilgan. $B$ uchi tashqi
burchagi bessektrisa uzunligini aniqlang.
====
Ikki uchi \(A (2; - 3) \) va \(B (-5;1) \) nuqtalarda,
uchinchi uchi $C$ ordinata o‘qiga tegishli uchburchakning
medianalarining kesishish nuqtasi $M$ abssissa o‘qida yotadi.
$M$ va $C$ nuqtalarning koordinatalarini aniqlang.
====
Ikkita uchi \(A (2;1) \) va \(B (5; 3) \) nuqtalarida, va
diagonallarining kesishish nuqtasi ordinata o‘qiga tegishli
parallelogrammning yuzi \(S = 17\) ga teng. Qolgan ikki uchining
koordinatalarini aniqlang.