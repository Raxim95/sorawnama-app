Ikkita qarama-qarshi uchlari \(P (4;9) \) va \(Q (-2; 1) \) nuqtalarida joylashgan romning tomon uzunligi \(5\sqrt{10}\). Shu
romba yuzini hisoblang.
====
Ikkita qarama-qarshi uchlari $P (3; -4) $ va $Q (l;2) $ nuqtalarda joylashgan rombaning tomon uzunligi \(5\sqrt{2}\). Shu romb balandligining uzunligini hisoblang.
====
Uchlari $A_1 (1; 1), A_2 (2; 3) $ va $A (5;-1) $
nuqtalarida joylashgan uchburchakning to‘g‘ri burchakli ekanini isbotlang.
====
Uchlari \(M_{1} (1;1), M_{2} (0,2) \) va
\(M_{3} (2;-1) \) nuqtalarda joylashgan uchburchakning ichki 
burchaklari orasida o‘tmas burchak bor yoki yo‘qligini aniqlang.
====
Uchlari \(M (-1;3),\ N (1,2) \ \) va \(P (0;4) \)
nuqtalarida joylashgan uchburchakning ichki burchaklari o‘tkir burchak
ekanligini isbotlang.
====
Uchburchakning uchlari \(A (5;0),\ B (0;1) \) va \(C (3;3) \)
nuqtalarida. Uning ichki burchaklarini toping.
====
Uchburchakning uchlari
\(A\left(-\sqrt{3};1 \right),\ B (0;2) \) va
\(C\left(-2\sqrt{3};2 \right) \) nuqtalarda. Uning $A$
uchidagi tashqi burchakni toping.
====
Abssissa o‘qida shunday $M$ nuqtani topingki,
\(N (2;-3) \) nuqtadan uzoqligi 5 ga teng bo‘lgan.
====
Ordinata o‘qida shunday $M$ nuqtani toping.
\(N (-8;13) \) nuqtadan uzoqligi 17 ga teng bo‘lgan.
====
Ikkita nuqta berilgan \(M (2;2) \) va \(N (5;-2) \); abssissa o‘qida shunday $P$ nuqtani topingki, $MPN$ burchak to‘g‘ri burchak bo‘lsin.
====
\(M_{1} (1;2) \) nuqtaga, \(A (1;0) \) va \(B (-1;-2) \)
nuqtalaridan o‘tuvchi to‘g‘ri chiziqqa nisbatan simmetrik bo‘lgan \(M_{2}\) nuqtaning koordinatalarini toping.
====
Uchburchakning uchlari \(A (2;-5),\ B (1;-2),\ C (4;7) \)
berilgan. $AC$ tomoni bilan $B$ uchining ichki burchagi
bissektrisasining kesishish nuqtasini toping.
====
Uchburchakning uchlari
\(A (3;-5),\ B (-3;3),\ C (-1;-2) \) berilgan. $A$ uchining ichki qismi
burchakli bessektrisaning uzunligini aniqlang.
====
Bir to‘g‘ri chiziqqa tegishli \(A (1;-1),\ B (3;3) \) va
\(C (4;5) \) nuqtalar berilgan. Har bir nuqtaning, qolgan ikki nuqta orqali aniqlanuvchi kesmani bo‘lish nisbati $\lambda$ ni aniqlang.
====
\(P (2;2) \) va \(Q (1;5) \) nuqtalar bilan teng uchta
bo‘lingan kesmaning uchlari $A$ va $B$ nuqtalarning
koordinatalarini aniqlang.
====
To‘g‘ri \(M_{1} (-12;-13) \) va \(M_{2} (-2;-5) \)
nuqtalaridan o‘tadi. Shu to‘g‘ri chiziqda abssissasi 3 ga teng nuqtani toping.
====
To‘g‘ri chiziq \(M (2;-3) \) va \(N (-6;5) \) nuqtalardan o‘tadi.
Shu to‘g‘ri chiziqda ordinatasi $-5$ ga teng nuqtani toping.
====
To‘g‘ri chiziq \(A (7;-3) \) va \(B (23;-6) \) nuqtalardan o‘tadi.
Shu to‘g‘ri chiziqning abssissa o‘qi bilan kesishish nuqtasini toping.
====
To‘g‘ri \(A (5;2) \) va \(B (-4; -7) \) nuqtalaridan o‘tadi.
Shu to‘g‘ri chiziqning ordinata o‘qi bilan kesishish nuqtasini toping.
====
To‘rtburchakning uchlari
\(A (-3;12),\ B (3;-4),\ C (5;-4) \) va \(D (5;8) \) berilgan. Shu
to‘rtburchakning $AC$ diagonali $BD$ diagonali qanday
nisbatda bo'lishini aniqlang.
====
To‘rtburchakning uchlari
\(A (-2;14),\ B (4;-2),\ C (6;-2) \) va \(D (6;10) \) berilgan. Shu
to‘rtburchakning $AC$ va $BD$ diagonallarining kesishishi
nuqtani toping.
====
Uchburchakning uchlari \(A (3;6),\ B (-1;3) \) va
\(C (2:-1) \) nuqtalarda joylashgan. $C$ uchidan tushirilgan balandlik uzunligini hisoblang.
====
Parallelogrammning uchta uchi \(A (3;7),\ B (2;-3) \) va
\(C (-1;4) \) nuqtalarda joylashgan. $B$ uchidan $AC$
tomonidan tushirilgan balandlik uzunligini hisoblang.
====
Ikkala uchi \(A (3;1) \) va \(B (1;-3) \) nuqtalarda, va
og‘irlik markazi $Ox$ o‘qiga tegishli uchburchakning yuzi
\(S=3\) ga teng. Uchinchi $C$ uchining koordinatalarini aniqlang.
====
Berilgan to‘g‘ri chiziqlarning kesishish nuqtasini toping:
$(3x-4y-29=0, 2x+5y+19=0)$.
====
$ABC$ uchburchakning tomonlari:
\(AB:4x+3y-5=0,\ BC:x-3y+10=0,\ AC:x-2=0\) 
tenglamalari bilan berilgan. Uchlarining koordinatalarini aniqlang.
====
Parallelogrammning ikki tomoni tenglamalari
\(8x+3y+1=0,\ 2x+y-1=0\) va bir diagonali tenglamasi
\(3x+2y+3=0\) berilgan. Parallelogramm uchlari koordinatalarini
aniqlang
====
Uchburchakning tomonlari \(x+5y-7=0\),
\(3x-2y-4=0\), \(7x+y+19=0\) to‘g‘ri chiziqlarda yotadi. Uning
yuzini hisoblang.
====
Umumiy tenglamasi \(2x-5y+4=0\) bo‘lgan to‘g‘ri
berilgan. \(M (-3,5) \) nuqtadan o‘tib, berilgan to‘g‘ri chiziqqa: a) parallel;
b) perpendikular bo‘lgan to‘g‘ri chiziqlar tenglamasini tuzing.
====
To‘g‘ri to‘rtburchakning bir uchi \(A (2;-3) \), va ikkita tarafining
ning tenglamalari \(2x+3y+9=0,\ 3x-2y-7=0\)
berilgan. Qolgan ikki tomonning tenglamalarini tuzing.
====
\(N (5;8) \) nuqtaning, \(5x-11y-43=0\) to‘g‘ri chizig‘idagi
proyeksiyasini toping.
====
Quyidagi har bir to‘g‘ri chiziqlar jufti uchun, ularga parallel
bo‘lib, aynan o‘rtasidan o‘tuvchi to‘g‘ri tenglamani tuzing: $3x-2y-3=0$, $3x-2y-17=0$.
====
Berilgan ikki nuqtadan o‘tuvchi to‘g‘ri chiziqning burchagi
koeffitsiyenti $k$ ni hisoblang: $A (-4;3) $, $B (1;8) $.
====
Uchburchak uchlari \(A (1;0),\ B (5;-2),\ C (3;2) \)
koordinatalari bilan berilgan. Uchburchaklar tomonlarining va
medianalarining tenglamalarini tuzing.
====
\(P (3;8) \) va \(Q (-1;-6) \) nuqtalardan o‘tgan
to‘g‘ri chiziqning koordinata o‘qlari bilan kesishish nuqtalarini toping.
====
Doiraviy to‘rtburchakning uchlari
\(A (-2;-6),\ B (7;6),\ C (3;9) \) va \(D (-3;1) \) nuqtalarda
joylashgan. Diagonallarining kesishish nuqtasi topilsin.
====
$ABCD$ parallelogrammning ikkita qo‘shni uchlari
\(A (3,3),\ B (-1;7) \) va diagonallarining kesishish nuqtasi
\(E (2;-4) \) berilgan. Shu parallelogramm tomonlarining tenglamalarini
tuzing.
====
To‘g‘ri to‘rtburchakning ikki tomoni
\(5x+2y-7=0,\ 5x+2y-36=0\) va diagonali
\(3x+7y-10=0\) tenglamalar bilan berilgan. Qolgan ikki tomoni
tenglamalarni tuzing.
====
Berilgan to‘g‘ri chiziqlar orasidagi burchakni aniqlang: $3x+2y+4=0, 5x-y+1=0$.
====
Qirralari
\(7x+y+31=0,\ 3x+4y-1=0,\ x-7y-17=0\) tenglamalar
bilan berilgan uchburchakning teng yonli ekanini isbotlang.
Masalani uchburchakning
burchaklarini topish orqali yeching.
====
\(N (4;-5) \) nuqtadan o‘tib, $2x+5y-7=0$
to‘g‘ri chiziqlariga parallel to‘g‘ri chiziqlarning tenglamasini tuzing. Masalani burchaklik
koeffitsiyentni hisoblamasdan yeching.
====
Quyida berilgan to‘g‘ri chiziqlar juftlarining qaysilari
perpendikular ekanini aniqlang: $4x+y+6=0, 2x-8y-13=0$.
====
Ikki to‘g‘ri chiziqning chetidagi burchakni toping: $2x+y-9=0, 3x-y+11=0$.
====
Parallel to‘g‘ri chiziqlar orasidagi masofani hisoblang: $5x-12y+13=0, 5x-12y-26=0$.
====
Kvadratning ikki tomoni
\(5x-12y+65=0,\ 5x-12y-26=0\) to‘g‘ri chiziqlarda
yotishini bilgan holda, yuzini hisoblang.
====
\(P (2;7) \) nuqtadan o‘tib, \(Q (1;2) \) nuqtagacha
masofasi 5 ga teng bo‘lgan to‘g‘ri chiziqlarning tenglamasini tuzing.
====
\(M (7;-2) \) nuqtadan o‘tib, \(N (4;-6) \) nuqtaga
gacha bo‘lgan masofasi 5 ga teng bo‘lgan to‘g‘ri chiziqlarning tenglamasini tuzing.
====
\(A (4;-5) \) nuqtadan o‘tib, \(B (-2;3) \) nuqtaga
gacha masofasi 12 ga teng bo‘lgan to‘g‘ri chiziqlarning tenglamasini tuzing.
====
Berilgan \(8x-15y-25=0\) to‘g‘ri chiziqdan og‘ishi -2 ga teng
teng bo‘lgan nuqtalarning geometrik o‘rni tenglamasini tuzing.
====
Berilgan \(3x-4y-10=0\) to‘g‘ri chiziqqa parallel va undan
$d=3$ masofada yotuvchi to‘g‘ri chiziqlarning tenglamasini tuzing.
====
Berilgan parallel to‘g‘ri chiziqlardan teng masofada yotuvchi
nuqtalarning geometrik o‘rni tenglamasini tuzing: $2x+y+7=0, 2x+y-3=0$.
====
\(P (1;-2) \) nuqta va koordinatalar boshi, berilgan ikkita
to‘g‘ri yozing: $12x-5y-7=0, 3x+4y-8=0$.
kesishishidan hosil bo‘lgan bir xil burchakdami, qo‘shni burchakdami yoki vertikal
burchaklarda yotadimi?
====
\(P (2;3) \) va \(Q (5;-1) \) nuqtalar, berilgan ikkita
to‘g‘ri: $12x-y-7=0,\ 13x+4y-5=0$.
kesishishidan hosil bo‘lgan bir xil burchakdami, qo‘shni burchakdami yoki vertikal
burchaklarda yotadimi?
====
Koordinata boshi, tomonlarining tenglamalari
\(8x+3y+31=0,\ x+8y-19=0,\ 7x-5y-11=0\) bilan
berilgan uchburchakning tashqarisida yoki ichida yotishini aniqlang.
====
\(P (-3;2) \) nuqta, tomonlarining tenglamalari
\(x+y-4=0,\ 3x-7y+8=0,\ 4x-y-31=0\) bilan
berilgan uchburchakning tashqarisida yoki ichida yotishini aniqlang.
====
Koordinata boshi, berilgan to‘g‘ri chiziqlarning:
\(3x+y-4=0\) va \(3x-2y+6=0\) kesishmasida hosil bo‘ladi
bo‘lgan o‘tkir yoki o‘tmas burchakka tegishli bo‘lishini aniqlang.
====
\(M (2;-5) \) nuqta, berilgan to‘g‘ri chiziqlarning:
\(3x+5y-4=0\) va \(x-2y+3=0\) kesishmasida hosil bo‘ladi
bo‘lgan o‘tkir yoki o‘tmas burchakka tegishli bo‘lishini aniqlang.
====
\(4x+3y-1=0\) va \(3x-2y+5=0\)
to‘g‘ri chiziqlarning kesishish nuqtasidan o‘tib (bu nuqtani aniqlamay), ordinata
o‘qidan \(b=4\) kesmani kesib oladigan to‘g‘ri chiziq tenglamasini tuzing.
====
\(2x+y-2=0\) va \(x-5y-3=0\)
to‘g‘ri chiziqlarning kesishish nuqtasidan o‘tib (bu nuqtani aniqlamay), uchlari
\(A (-1;-4) \) va \(B (5;-6) \) nuqtalarda joylashgan kesmaning
to‘g‘ri o‘rtasidan o‘tuvchi to‘g‘ri chiziqning tenglamasini tuzing.
====
Uchlari \(A (4;-4),\ B (6;-1) \) va \(C (-1;2) \)
nuqtalarida joylashgan bir jinsli plastinkadan yasalgan uchburchakning
og‘irlik markazidan o‘tib, quyida berilgan
\(\alpha (2x+3y-1) +\beta (3x-4y-3) =0\) to‘g‘ri chiziqlar dasturiga
tegishli to‘g‘ri chiziqning tenglamasini tuzing.
====
Tekislikda uchta vektor $\vec{a} = \{ 3; - 2\}$, $\vec{b} = \{ - 2;1\}$ va $\vec{c} = \{ 7; - 4\}$ berilgan. Bu uchta vektorning har birining qolgan ikkitasini bazis sifatida qabul qilib yoyilmasini toping.
====
$\vec{a}$ va $\vec{b}$ vektorlar $\varphi = 2\pi/3$ burchak hosil qiladi. $|\vec{a}| = 3,|\vec{b}| = 4$ ekani ma’lum. Hisoblang:
$\left(\vec{a},\vec{b} \right) $.
====
$\vec{a}$ va $\vec{b}$ vektorlar $\varphi = 2\pi/3$ burchak hosil qiladi. $|\vec{a}| = 3,|\vec{b}| = 4$ ekani ma’lum. Hisoblang:
${\vec{a}}^{2}$.
====
$\vec{a}$ va $\vec{b}$ vektorlar $\varphi = 2\pi/3$ burchak hosil qiladi. $|\vec{a}| = 3,|\vec{b}| = 4$ ekani ma’lum. Hisoblang:
${\vec{b}}^{2}$.
====
$\vec{a}$ va $\vec{b}$ vektorlar $\varphi = 2\pi/3$ burchak hosil qiladi. $|\vec{a}| = 3,|\vec{b}| = 4$ ekani ma’lum. Hisoblang:
$ (\vec{a} + \vec{b}) ^{2}$.
====
$\vec{a}$ va $\vec{b}$ vektorlar $\varphi = 2\pi/3$ burchak hosil qiladi. $|\vec{a}| = 3,|\vec{b}| = 4$ ekani ma’lum. Hisoblang:
$\left(3\vec{a} - 2\vec{b},\vec{a} + 2\vec{b} \right) $.
====
$\vec{a}$ va $\vec{b}$ vektorlar $\varphi = 2\pi/3$ burchak hosil qiladi. $|\vec{a}| = 3,|\vec{b}| = 4$ ekani ma’lum. Hisoblang:
$ (\vec{a} - \vec{b}) ^{2};$ 7) $ (3\vec{a} + 2\vec{b}) ^{2}$.
====
$\vec{a}$ va $\vec{b}$ vektorlar o‘zaro perpendikulyar; $\vec{c}$ vektor ular bilan $\pi/3$ ga teng bo‘lgan burchaklar hosil qiladi; $|\vec{a}| = 3$, $|\vec{b}| = 5,\ |\vec{c}| = 8$ ekani ma’lum, quyidagilarni hisoblang:
$\left(3\vec{a} - 2\vec{b},\vec{b} + 3\vec{c} \right) $.
====
$\vec{a}$ va $\vec{b}$ vektorlar o‘zaro perpendikulyar; $\vec{c}$ vektor ular bilan $\pi/3$ ga teng bo‘lgan burchaklar hosil qiladi; $|\vec{a}| = 3$, $|\vec{b}| = 5,\ |\vec{c}| = 8$ ekani ma’lum, quyidagilarni hisoblang:
$ (\vec{a} + \vec{b} + \vec{c}) ^{2}$.
====
$\vec{a}$ va $\vec{b}$ vektorlar o‘zaro perpendikulyar; $\vec{c}$ vektor ular bilan $\pi/3$ ga teng bo‘lgan burchaklar hosil qiladi; $|\vec{a}| = 3$, $|\vec{b}| = 5,\ |\vec{c}| = 8$ ekani ma’lum, quyidagilarni hisoblang:
$ (\vec{a} + 2\vec{b} - 3\vec{c}) ^{2}$.
====
$\vec{a} + \vec{b} + \vec{c} = 0$ shartni qanoatlantiruvchi $\vec{a},\ \vec{b}$ va $\vec{c}$ vektorlar berilgan. $|\vec{a}| = 3,\ |\vec{b}| = 1$ va $|\vec{c}| = 4$ ekani ma’lum, $\left(\vec{a},\vec{b} \right) + \left(\vec{b},\vec{c} \right) + (\vec{c}) $ ifodani hisoblang.
====
$|\vec{a}| = 3,|\vec{b}| = 5$ berilgan. $\alpha$ ning qanday qiymatida $\vec{a} + \alpha\vec{b}$, $\vec{a} - \alpha\vec{b}$ vektorlar o‘zaro perpendikulyar bo‘lishini aniqlang.
====
$a$ va $b$ vektorlar $\varphi = \pi/6$ burchak hosil qiladi; $|a| = \sqrt{3},|b| = 1$ ekani ma’lum. $p = a + b$ va $q = a - b$ vektorlar orasidagi $\alpha$ burchakni hisoblang.
====
$\vec{a} = \{ 6; - 8; - 7,5\}$ vektorga kollinear bo‘lgan $\vec{x}$ vektor $Oz$ o‘qi bilan o‘tkir burchak hosil qiladi. $|\vec{x}| = 50$ ekanini bilgan holda uning koordinatalarini toping.
====
$\vec{a} = \{ 2;1; - 1\}$ vektorga kollinear bo‘lgan va $\left(\vec{x},\vec{a} \right) = 3$ shartni qanoatlantiruvchi $\vec{x}$ vektorni toping.
====
$\vec{a}$ va $\vec{b}$ vektorlar o‘zaro perpendikulyar. $|\vec{a}| = 3,|\vec{b}| = 4$ ekani ma’lum, quyidagilarni hisoblang:
$|\lbrack\vec{a} + \vec{b},\vec{a} - \vec{b}\rbrack|$.
====
$\vec{a}$ va $\vec{b}$ vektorlar o‘zaro perpendikulyar. $|\vec{a}| = 3,|\vec{b}| = 4$ ekani ma’lum, quyidagilarni hisoblang:
$|\lbrack 3\vec{a} - \vec{b},\vec{a}-2\vec{b}\rbrack|$.
====
$\vec{a}$ va $\vec{b}$ vektorlar $\varphi = 2\pi/3$ burchak hosil qiladi. $|\vec{a}| = 1,|\vec{b}| = 2$ ekanini bilib, quyidagilarni hisoblang:
$\lbrack\vec{a},\vec{b}\rbrack^{2}$.
====
$\vec{a}$ va $\vec{b}$ vektorlar $\varphi = 2\pi/3$ burchak hosil qiladi. $|\vec{a}| = 1,|\vec{b}| = 2$ ekanini bilib, quyidagilarni hisoblang:
$\lbrack 2\overrightarrow{a} + \overrightarrow{b},\overrightarrow{a} + 2\overrightarrow{b}\rbrack^{2}$.
====
$\vec{a}$ va $\vec{b}$ vektorlar $\varphi = 2\pi/3$ burchak hosil qiladi. $|\vec{a}| = 1,|\vec{b}| = 2$ ekanini bilib, quyidagilarni hisoblang:
$\lbrack\overrightarrow{a} + 3\overrightarrow{b},3\overrightarrow{a} - \overrightarrow{b}\rbrack^{2}$
====
$\vec{a} = \{ 3; - 1; - 2\}$ va $\vec{b} = \{ 1;2; - 1\}$ vektorlar berilgan. Quyidagi vektor ko‘paytmalarning koordinatalarini toping:
$\left\lbrack \vec{a},\vec{b} \right\rbrack$.
====
$\vec{a} = \{ 3; - 1; - 2\}$ va $\vec{b} = \{ 1;2; - 1\}$ vektorlar berilgan. Quyidagi vektor ko‘paytmalarning koordinatalarini toping:
$\left\lbrack 2\vec{a} + \vec{b},\vec{b} \right\rbrack$.
====
$\vec{a} = \{ 3; - 1; - 2\}$ va $\vec{b} = \{ 1;2; - 1\}$ vektorlar berilgan. Quyidagi vektor ko‘paytmalarning koordinatalarini toping:
$\left\lbrack 2\vec{a} - \vec{b},2\vec{a} + \vec{b} \right\rbrack$.
====
$A (2; -1;2),B (1;2; 1) $ va $C (3;2;1)$ nuqtalar berilgan. Quyidagi vektor ko‘paytmalarning koordinatalarini toping:
$\lbrack\overline{AB},\overline{BC}\rbrack$.
====
$A (2; -1;2),B (1;2; 1) $ va $C (3;2;1) $ nuqtalar berilgan. Quyidagi vektor ko‘paytmalarning koordinatalarini toping:
$\lbrack\overline{BC} - 2\overline{CA},\overline{CB}\rbrack$. 