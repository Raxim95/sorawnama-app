\documentclass{article}
\usepackage{amsmath}
\usepackage{amssymb}
\usepackage[utf8]{inputenc}
% \usepackage[T1,T2A]{fontenc}
\usepackage[russian]{babel}
\usepackage[a4paper, total={6in, 8in}]{geometry}

\begin{document}

\section*{\S 4. Kesindiniń baǵıtı. Kesindiniń qálegen kósherge proekciyası. 
Kesindiniń koordinata kósherlerine proekciyası. 
Kesindiniń uzınlıǵı hám polyar múyeshi. 
Eki noqat arasındaǵı aralıq}


\textbf{68 (68).} Eki qarama-qarsı tóbeleri \(P(4;9)\) hám \(Q(-2;
1)\) noqatlarında jaylasqan rombanıń tárepi uzınlıǵı \(5\sqrt{10}\). Usı
romba maydanın esaplań.

\textbf{69 (69).} Eki qarama-qarsı tóbeleri $P(3; -4)$ hám $Q(l;2)$ 
noqatlarında jaylasqan rombanıń tárepi uzınlıǵı \(5\sqrt{2}\). Usı
romba biyikliginiń uzınlıǵın esaplań.

\textbf{71 (71).} Tóbeleri $A_1(1; 1), A_2(2; 3)$ hám $A(5;-1)$
noqatlarında jaylasqan úshmúyeshliktiń tuwrımúyeshli ekenligin dálilleń.

\textbf{73 (73).} Tóbeleri \(M_{1}(1;1), M_{2}(0,2)\) hám
\(M_{3}(2;-1)\) noqatlarında jaylasqan úshmúyeshliktiń ishki
múyeshleri arasında doǵal múyesh bar yáki joqlıǵın anıqlań.

\textbf{74 (74).} Tóbeleri \(M(-1;3),\ N(1,2)\ \)hám \(P(0;4)\)
noqatlarında jaylasqan úshmúyeshliktiń ishki múyeshleri súyir múyesh
ekenligin dálilleń.

\textbf{75 (75).} Úshmúyeshliktiń tóbeleri \(A(5;0),\ B(0;1)\) hám \(C(3;3)\)
noqatlarında. Onıń ishki múyeshlerin tabıń.

\textbf{76 (76).} Úshmúyeshliktiń tóbeleri
\(A\left(-\sqrt{3};1 \right),\ B(0;2)\) hám
\(C\left(-2\sqrt{3};2 \right)\) noqatlarında. Onıń $A$
tóbesindegi sırtqı múyeshin tabıń.

\textbf{77 (77).} Abcsissa kósherinde sonday $M$ noqatın tabıń,
\(N(2;-3)\) noqatınan qashıqlıǵı 5 ke teń bolatuǵın.

\textbf{78 (78).} Ordinata kósherinde sonday $M$ noqatın tabıń,
\(N(-8;13)\) noqatınan qashıqlıǵı 17 ge teń bolatuǵın.

\textbf{79 (79).} Eki noqat berilgen \(M(2;2)\) hám \(N(5;-2)\); abscissa kósherinde sonday $P$ noqatın tabıń, $MPN$ múyeshi tuwrı múyesh bolsın.

\textbf{82 (82).} \(M_{1}(1;2)\) noqatına, \(A(1;0)\) hám \(B(-1;-2)\)
noqatlarınan ótetuǵın tuwrıǵa qarata simmetriyalı bolǵan \(M_{2}\) noqatınıń koordinataların tabıń.

\section*{\S 5. Kesindini berilgen qatnasta bóliw}

\textbf{96 (96).} Úshmúyeshliktiń tóbeleri \(A(2;-5),\ B(1;-2),\ C(4;7)\)
berilgen. $AC$ tárepi menen $B$ tóbesiniń ishki múyeshi
bissektrisasınıń kesilisiw noqatın tabıń.

\textbf{97 (97).} Úshmúyeshliktiń tóbeleri
\(A(3;-5),\ B(-3;3),\ C(-1;-2)\) berilgen. $A$ tóbesiniń ishki
múyeshi bessektrisanıń uzınlıǵın anıqlań.

\textbf{100 (100).} Bir tuwrıǵa tiyisli \(A(1;-1),\ B(3;3)\) hám
\(C(4;5)\) noqatları berilgen. Hár-bir noqattıń, qalǵan eki noqat arqalı anıqlanıwshı kesindini bóliw qatnası $\lambda$ nı anıqlań.

\textbf{101 (101).} \(P(2;2)\) hám \(Q(1;5)\) noqatları menen teńdey úsh
bólekke bólingen kesindiniń úshları $A$ hám $B$ noqatlarınıń
koordinataların anıqlań.

\textbf{102 (102).} Tuwrı \(M_{1}(-12;-13)\) hám \(M_{2}(-2;-5)\)
noqatlarınan ótedi. Usı tuwrıda abscissası 3 ke teń noqattı tabıń.

\textbf{103 (103).} Tuwrı \(M(2;-3)\) hám \(N(-6;5)\) noqatlarınan ótedi.
Usı tuwrıda ordinatası $-5$ ke teń noqattı tabıń.

\textbf{104 (104).} Tuwrı \(A(7;-3)\) hám \(B(23;-6)\) noqatlarınan ótedi.
Usı tuwrınıń abscissa kósheri menen kesilisiw noqatın tabıń.

\textbf{105 (105).} Tuwrı \(A(5;2)\) hám \(B( -4; -7)\) noqatlarınan ótedi.
Usı tuwrınıń ordinata kósheri menen kesilisiw noqatın tabıń.

\textbf{106 (106).} Tórtmúyeshliktiń tóbeleri
\(A(-3;12),\ B(3;-4),\ C(5;-4)\) hám \(D(5;8)\) berilgen. Usı
tórtmúyeshliktiń $AC$ diagonalı $BD$ dioganalın qanday
qatnasta bóliwin anıqlań.

\textbf{107 (107).} Tórtmúyeshliktiń tóbeleri
\(A(-2;14),\ B(4;-2),\ C(6;-2)\) hám \(D(6;10)\) berilgen. Usı
tórtmúyeshliktiń $AC$ hám $BD$ dioganallarınıń kesilisiw
noqatın tabıń.

\section*{\S 6. Úshmúyeshliktiń maydanı}

\textbf{117 (117).} Úshmúyeshliktiń tóbeleri \(A(3;6),\ B(-1;3)\) hám
\(C(2:-1)\) noqatlarında jaylasqan. $C$ tóbesinen túsirilgen biyikliktiń uzınlıǵın esaplań.

\textbf{119 (119).} Parallelogrammnıń úsh tóbesi \(A(3;7),\ B(2;-3)\) hám
\(C(-1;4)\) noqatlarında jaylasqan. $B$ tóbesinen $AC$
tárepine túsirilgen biyikliktiń uzınlıǵın esaplań.

\textbf{124 (124).} Eki tóbesi \(A(3;1)\) hám \(B(1;-3)\) noqatlarında, hám
awırlıq orayı $Ox$ kósherine tiyisli úshmúyeshliktiń maydanı
\(S=3\) ke teń. Úshinshi $C$ tóbesiniń koordinataların anıqlań.

\section*{\S 12. Tuwrınıń ulıwma teńlemesi. Tuwrınıń múyeshlik koefficientlik 
teńlemesi. Eki tuwrı arasındaǵı múyesh. Eki tuwrınıń parallellik hám 
perpendikuliyarlıq shártleri}

\textbf{177 (214).} Berilgen tuwrılardıń kesilisiw noqatın tabıń: 
\(3x-4y-29=0, 2x+5y+19=0\).

\textbf{178 (215).} $ABC$ úshmúyeshliginiń tárepleri: 
\(AB:4x+3y-5=0,\ BC:x-3y+10=0,\ AC:x-2=0
\) teńlemeleri menen berilgen. Tóbeleriniń koordinataların anıqlań.

\textbf{179 (216).} Parallellogramnıń eki tárepiniń teńlemeleri
\(8x+3y+1=0,\ 2x+y-1=0\) hám bir diagonalı teńlemesi
\(3x+2y+3=0\) berilgen. Parallellogram tóbeleri koordinataların
anıqlań

\textbf{180 (217).} Úshmúyeshliktiń tárepleri \(x+5y-7=0\),
\(3x-2y-4=0\), \(7x+y+19=0\) tuwrılarında jatadı. Onıń
maydanın esaplań.

\textbf{186 (223*).} Ulıwma teńlemesi \(2x-5y+4=0\) bolǵan tuwrı
berilgen. \(M(-3;5)\) noqatınan ótip, berilgen tuwrıǵa: a) parallel;
b) perpendikuliyar bolǵan tuwrılar teńlemesin dúziń.

\textbf{187 (224).} Tuwrımúyeshliktiń bir tóbesi \(A(2;-3)\), hám eki
tárepiniń teńlemeleri \(2x+3y+9=0,\ 3x-2y-7=0\)
berilgen. Qalǵan eki tárepiniń teńlemelerin dúziń.

\textbf{188 (226*).} \(N(5;8)\) noqatınıń, \(5x-11y-43=0\) tuwrısındaǵı
proekciyasın tabıń.

\textbf{190 (228*).} Tómendegi hár-bir tuwrılar jubı ushın, olarǵa parallel
bolıp, dál ortasınan ótetuǵın tuwrı teńlemesin dúziń: $3x-2y-3=0$, $3x-2y-17=0$.

\textbf{191 (229*).} Berilgen eki noqattan ótetuǵın tuwrınıń múyeshlik
koefficienti $k$ nı esaplań: $A(-4;3)$, $B(1;8)$.

\textbf{200 (238*).} Úshmúyeshliktiń tóbeleri \(A(1;0),\ B(5;-2),\ C(3;2)\)
koordinataları menen berilgen. Úshmúyeshliklerdiń tárepleriniń hám
medianalarınıń teńlemelerin dúziń.

\textbf{201 (239*).} \(P(3;8)\) hám \(Q(-1;-6)\) noqatlarınan ótken
tuwrınıń koordinatalıq kósherler menen kesilisiw noqatların tabıń.

\textbf{202 (242*).} Dóńes tórtmúyeshliktiń tóbeleri
\(A(-2;-6),\ B(7;6),\ C(3;9)\) hám \(D(-3;1)\) noqatlarda
jaylasqan. Diagonallarınıń kesilisiw noqatı tabılsın.

\textbf{203 (243*).} $ABCD$ parallelogrammınıń eki qońsı tóbeleri
\(A(3,3),\ B(-1;7)\) hám diagonallarınıń kesilisiw noqatı
\(E(2;-4)\) berilgen. Usı parallelogram tárepleriniń teńlemelerin
dúziń.

\textbf{204 (244).} Tuwrımúyeshliktiń eki tárepi
\(5x+2y-7=0,\ 5x+2y-36=0\) hám diagonalı
\(3x+7y-10=0\) teńlemeleri menen berilgen. Qalǵan eki tárepi
teńlemelerin dúziń.

\textbf{213.} Berilgen tuwrılar arasındaǵı múyeshti anıqlań:
a) $3x+2y+4=0,\ 5x-y+1=0$; \\
b) $3x-2y+7=0,\ 2x+3y+5=0$; \\
c) $3x+2y-3=0,\ 5x-2y+5=0$; \\
d) $2x+y-5=0,\ 4x+2y+7=0$.

\textbf{218 (260)}. Tárepleri
\(7x+y+31=0,\ 3x+4y-1=0,\ x-7y-17=0\) teńlemeleri
menen berilgen úshmúyeshliktiń teń qaptallı ekenligin dálilleń. 
Máseleni úshmúyeshliktiń
múyeshlerin tabıw arqalı sheshiń.

\textbf{219.} \(N(4;-5)\) noqatınan ótip,

a) $2x+5y-7=0$; \\
b) $x-3y+8=0$; \\
c) $12x+15y-13=0$; \\
d) $3x+8=0;e)4y-5=0$;

tuwrılarına parallel tuwrılardıń teńlemesin dúziń. Máseleni múyeshlik
koefficientti esaplamay sheshiń.

\textbf{220.} Tómende berilgen tuwrılar jubınıń qaysıları
perpendikuliyar ekenligin anıqlań:

a) $4x+y+6=0,\ 2x-8y-13=0$; \\
b) $5x-2y+2=0,\ 2x+5y-12=0$; \\
c) $2x-y+5=0,\ 2x+y-5=0$; \\
d) $6x-2y-15=0,\ x+3y=0$; \\
e) $x+y+2=0,\ x-y+2=0$; \\
f) $9x+4y-1=0,\ 4x-9y+2=0$.

\textbf{221.} Eki tuwrı aqrasındaǵı múyeshti tabıń:

a) $2x+y-9=0,\ 3x-y+11=0$;
b) $\sqrt{6}x-3y+12=0,\ \sqrt{3}x+\sqrt{2}y-1=0$;
c) $4x+3y-5=0,\ 3x-4y+5=0$;
d) $3x-2y-7=0,\ 6x-4y-1=0$;
e) $\sqrt{2}x-\sqrt{3}y-9=0,\ \left( 3+\sqrt{2} \right)x+\left( \sqrt{6}-\sqrt{3} \right)y+11=0$.

\section*{\S 14. Tuwrınıń normal teńlemesi. Noqattan tuwrıǵa shekemgi aralıqtı 
tabıw máselesi}

\textbf{273 (322).} Parallel tuwrılar arasındaǵı aralıqtı esaplań:

1) $5x-12y+13=0,\ 5x-12y-26=0$;
2) $6x-8y-5=0,\ 3x-4y+10=0$;
3) $x+y\sqrt{3}+6=0,\ 2x+2\sqrt{3}y-15=0$;
4) $4x+3y+20=0,\ 8x+6y-25=0$.

\textbf{274.} Kvadrattıń eki tárepi
\(5x-12y+65=0,\ 5x-12y-26=0\) tuwrılarında
jatatuǵının bilgen jaǵdayda, maydanın esaplań.


\textbf{276 (326).} \(P(2;7)\) noqatınan ótip, \(Q(1;2)\) noqatına shekemgi
qashıqlıǵı 5 ke teń bolǵan tuwrılardıń teńlemesin dúziń.

\textbf{277.} \(M(7;-2)\) noqatınan ótip, \(N(4;-6)\) noqatına
shekemgi qashıqlıǵı 5 ke teń bolǵan tuwrılardıń teńlemesin dúziń.

\textbf{278.} \(A(4;-5)\) noqatınan ótip, \(B(-2;3)\) noqatına
shekemgi qashıqlıǵı 12 ge teń bolǵan tuwrılardıń teńlemesin dúziń.

\textbf{279 (330).} Berilgen \(8x-15y-25=0\) tuwrısınan awısıwı -2 ge
teń bolǵan noqatlardıń geometriyalıq ornı teńlemesin dúziń.

\textbf{280.} Berilgen \(3x-4y-10=0\) tuwrısına parallel hám onnan
$d=3$ qashıqlıqta jatatuǵın tuwrılardıń teńlemesin dúziń.

\textbf{287 (338).} Berilgen parallel tuwrılardan teńdey aralıqta jatatuǵın
noqatlardıń geometriyalıq ornı teńlemesin dúziń:\\

1) $2x+y+7=0,\ 2x+y-3=0$; \\
2) $x-4y+10=0,\ x-4y+8=0$; \\
3) $4x+3y-12=0,\ 8x+6y+11=0$; \\
4) $x+y-1=0,\ 2x+2y-3=0$. \\

\textbf{290 (841).} \(P(1;-2)\) noqatı hám koordintalar bası, berilgen eki
tuwrınıń:\\

1) $12x-5y-7=0,\ 3x+4y-8=0$; \\
2) $2x-y-5=0,\ 3x+y+10=0$; \\
3) $7x-24y+15=0,\ 6x+8y-15=0$; \\
4) $x+4y+1=0,\ 2x-y+1=0$. \\

kesilisiwinen payda bolǵan birdey múyeshte me, qońsılas múyeshlerde me yáki vertikal
múyeshlerde jatama?.

\textbf{291.} \(P(2;3)\) hám \(Q(5;-1)\) noqatları, berilgen eki
tuwrınıń:

1) $12x-y-7=0,\ 13x+4y-5=0$; \\
2) $2x-y+5=0,\ 5x+y+13=0$; \\
3) $7x+24y+5=0,\ 6x+8y-5=0;$  \\
4) $x+4y+10=0,\ 2x-y+10=0$. \\

kesilisiwinen payda bolǵan birdey múyeshte me, qońsılas múyeshlerde me yáki vertikal
múyeshlerde jatama?.

\textbf{292 (343).} Koordinata bası, tárepleriniń teńlemeleri
\(8x+3y+31=0,\ x+8y-19=0,\ 7x-5y-11=0\) menen
berilgen úshmúyeshliktiń sırtında yamasa ishinde jatatuǵınlıǵın anıqlań.

\textbf{293.} \(P(-3;2)\) noqatı, tárepleriniń teńlemeleri
\(x+y-4=0,\ 3x-7y+8=0,\ 4x-y-31=0\) menen
berilgen úshmúyeshliktiń sırtanda yamasa ishinde jatatuǵınlıǵın anıqlań.

\textbf{294.} Koordinata bası, berilgen tuwrılardıń:
\(3x+y-4=0\) hám \(3x-2y+6=0\) kesilisiwinde payda
bolǵan súyir yamasa doǵal múyeshke tiyisli bolıwın anıqlań.

\textbf{295.} \(M(2;-5)\)noqatı, berilgen tuwrılardıń:
\(3x+5y-4=0\) hám \(x-2y+3=0\) kesilisiwinde payda
bolǵan súyir yamasa doǵal múyeshke tiyisli bolıwın anıqlań.

\section*{\S 15. Tuwrılar dástesiniń teńlemesi}

\textbf{304.} \(4x+3y-1=0\) hám \(3x-2y+5=0\)
tuwrılarınıń kesilisiw noqatınan ótip (bul noqattı anıqlamay), ordinata
kósherinen \(b=4\) kesindi kesip alatuǵın tuwrınıń teńlemesin dúziń.

\textbf{305.} \(2x+y-2=0\) hám \(x-5y-3=0\)
tuwrılarınıń kesilisiw noqatınan ótip (bul noqattı anıqlamay), ushları
\(A(-1;-4)\) hám \(B(5;-6)\) noqatlarinda jaylasqan kesindiniń
dál ortasınan ótiwshi tuwrınıń teńlemesin dúziń.

\textbf{306.} Tóbeleri \(A(4;-4),\ B(6;-1)\) hám \(C(-1;2)\)
noqatlarında jaylasqan bir tekli plastinkadan jasalǵan úshmúyeshliktiń
awırlıq orayınan ótip, tómende berilgen
\(\alpha(2x+3y-1)+\beta(3x-4y-3)=0\) tuwrılar dástesine
tiyisli tuwrınıń teńlemesin dúziń.

\textbf{307 (358).} \(\alpha(3x-2y-1)+\beta(4x-5y+8)=0\) tuwrılar
dástesi berilgen. Usı tuwrılar dástesine tiyisli hám \(x+2y+4=0\)
tuwrınıń \(2x+3y+5=0\) hám \(x+7y-1=0\) tuwrıları menen
kesilisiwinde payda bolǵan kesindi ortasınan ótken tuwrınıń teńlemesin
dúziń.

\textbf{313.}
\(\alpha(3x+y-1)+\beta(2x-y-9)=0\) tuwrılar dástesi
berilgen. \(x+3y+13=0\) tuwrınıń usı tuwrılar dástesine tiyisli
yamasa tiyisli emesligin anıqlań.

\textbf{314.}
\(\alpha(5x+3y+6)+\beta(3x-4y-37)=0\) tuwrılar
dástesi berilgen. \(7x+2y-15=0\) tuwrınıń usı tuwrılar dástesine
tiyisli yamasa tiyisli emesligin anıqlań.

\textbf{315 (366).} \(\alpha(3x+2y-9)+\beta(2x+5y+5)=0\)
tuwrılar dástesi berilgen. $K$ nıń qanday mánisinde
\(4x-3y+K=0\) tuwrı usı tuwrılar dástesine tiyisli boladı.

\textbf{316.}
\(\alpha(5x+3y-7)+\beta(3x+10y+4)=0\) tuwrılar
dástesi berilgen. $a$ nıń qanday mánisinde \(ax+5y+9=0\)
tuwrı usı tuwrılar dástesine tiyisli bolmaydı.

\end{document}