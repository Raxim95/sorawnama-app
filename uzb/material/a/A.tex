\documentclass{article}
\usepackage{amsmath}
\usepackage{amssymb}
\usepackage[utf8]{inputenc}
% \usepackage[T1,T2A]{fontenc}
\usepackage[russian]{babel}
\usepackage[a4paper, total={6in, 8in}]{geometry}

\begin{document}

% Túsindirme:
% \emph{10 (20*) jazıwı bul sorawnamadaǵı 10-esap D.B. Kletenik 
% avtorlıǵındaǵı "Сборник задач по аналитической геометрии" 
% kitabındaǵı 20-esapqa uqsas ekenligin ańlatadı, eger "*" 
% simvolı bolmasa 20-esaptıń ózi ekenligin ańlatadı.}

\section*{\S 4. Kesindiniń baǵıtı. Kesindiniń qálegen kósherge proekciyası. 
Kesindiniń koordinata kósherlerine proekciyası. 
Kesindiniń uzınlıǵı hám polyar múyeshi. 
Eki noqat arasındaǵı aralıq}

\textbf{47 (47)} $M_1(1; -2)$, $M_2(2; 1)$, $M_3(5; 0)$, 
$M_4(-1; 4)$, $M_5(0; -3)$ noqatları berilgen. 
Tómendegi kesindilerdiń koordinata kósherlerine proekciyaların tabıń:

1) $\overline{M_1M_2}$ \\
2) $\overline{M_3M_1}$ \\
3) $\overline{M_4M_5}$ \\
4) $\overline{M_5M_3}$ \\

\textbf{64 (64).1} Kvadrattıń eki qońsı tóbeleri $A(3; -7)$ hám
$B(-1;4)$ berilgen. Onıń maydanın esaplań.

\textbf{65 (65).2} Kvadrattıń eki qarama-qarsı tóbeleri $P(3; 5)$ hám
$Q(1; -3)$ berilgen. Onıń maydanın esaplań.

\textbf{66 (66).3} Eki tóbesi $A(-3; 2)$ hám $B(1; 6)$ noqatlarında
jaylasqan durıs úshmúyeshliktiń maydanın esaplań.

\textbf{67 (67).4} $ABCD$ parallelogrammınıń úsh tóbesi $A(3; -7)$, 
$B(5; -7)$, $C(-2; 5)$ berilgen, tórtinshi tóbesi $D$, 
$B$ tóbesine qarama-qarsı. Usı parallelogrammnıń diagonalları
uzınlıqların anıqlań.

\textbf{70 (70)} Berilgen $A(3; -5)$, $B(-2; -7)$ hám
$C(18; 1)$ noqatları bir tuwrıda jatatuǵınlıǵın dálilleń.

\textbf{72 (72)} $A(2;2)$, $B(-1;6)$, $C(-5;3)$ hám $D(-2;-1)$
noqatları kvadrat tóbeleri ekenligin dálilleń.

\emph{A tiptan alıp taslaymız}
\textbf{83 (83)} Kvadrattıń eki qarama-qarsı tóbeleri $A(3;0)$ hám
$C(-4;1)$ berilgen. Qalǵan eki tóbesiniń koordinataların tabıń.

\textbf{84 (84)} Kvadrattıń eki qońsı tóbeleri $A(2;-1)$ hám
$B(-1;3)$berilgen. Qalǵan eki tóbesiniń koordinataların tabıń.

\section*{\S 5. Kesindini berilgen qatnasta bóliw}

\textbf{86 (86)} Birtekli elementten islengen saptıń ushları
$A(3;-5)$hám $B(-1;1)$ noqatlarında jaylasqan. Onıń awırlıq
orayı koordinatasın anıqlań.

\textbf{87 (87)} Birtekli elementten islengen saptıń awırlıq orayı
$M(1;4)$ noqatında, bir ushı $P(-2;2)$noqatında jaylasqan. Usı
saptıń ekinshi ushı $Q$-dıń koordinataların anıqlań.

\textbf{88 (88)} Úshmúyeshliktiń tóbeleriniń koordinataları berilgen
$A(1;-3)$, $B(3;-5)$ hám $C(-5;7)$. Tárepleriniń ortaların
anıqlań.

\textbf{90 (90)} $M(2;-1)$, $N(-1;4)$ hám $P(-2;2)$ noqatları
úshmúyeshliktiń tárepleriniń ortaları. Tóbeleriniń koordinataların
anıqlań.

\textbf{91 (91)} Parallelogrammnıń úsh tóbesi
$A(3;-5)$, $B(5;-3)$, $C(-1;3)$ berilgen. $B$ tóbesine
qaraqma-qarsı jaylasqan $D$ tóbesin anıqlań.

\textbf{92 (92)} Parallelogrammnıń eki qońsı tóbeleri $A(-3;5)$, $B(1;7)$
hám dioganallarınıń kesilisiw noqatı $M(1;1)$ berilgen. Qalǵan eki
tóbesin anıqlań.

\textbf{93 (93)} $ABCD$-parallelogrammınıń úsh tóbesi
$A(2;3)$, $B(4;-1)$ hám $C(0;5)$ berilgen. Tórtinshi $D$
tóbesin tabıń.

\textbf{94 (94)} Úshmúyeshliktiń tóbeleri $A(1;4)$, $B(3;-9)$, $C(-5;2)$
berilgen. $B$ tóbesinen júrgizilgen mediana uzınlıǵın anıqlań.

\textbf{95 (95)} $A(1;-3)$ hám $B(4;3)$ noqatların tutastırıwshı
kesindi teńdey úsh bólekke bólindi. Bóliwshi noqatlardıń koordinataların
anıqlań.

\textbf{115 (115)} $A(4;2)$, $B(7;-2)$ hám $C(1;6)$ noqatları birtekli
sımnan islengen úshmúyeshlik tóbeleri. Usı úshmúyeshliktiń awırlıq
orayın anıqlań.

\section*{\S 6. Úshmúyeshliktiń maydanı}

\textbf{116 (116).1} Tóbeleri $A(2;-3)$, $B(3;2)$ hám $C(-2;5)$ 
noqatlarında jaylasqan úshmúyeshliklerdiń maydanın esaplań.

\textbf{116 (116).2} Tóbeleri $M_1(-3;2)$, $M_2(5;-2)$ hám $M_3(1;3)$ 
noqatlarında jaylasqan úshmúyeshliklerdiń maydanın esaplań.

\textbf{116 (116).3} Tóbeleri $M(3;-4)$, $N(-2;3)$ hám $P(4;5)$ 
noqatlarında jaylasqan úshmúyeshliklerdiń maydanın esaplań.

\textbf{118 (118)} Úsh tóbesi $A(-2;3), \ B(4;-5)$ hám
$C(-3;1)$ noqatlarda jaylasqan parallelogrammnıń maydanın anıqlań.

\textbf{120 (120)} Birtekli tórtmúyeshli plastinkanıń tóbeleri berilgen:
$A(2;1), \ B(5;3), \ C(-1;7)$ hám $D(-7;5)$. Onıń awırlıq orayı
koordinataların anıqlań.

\textbf{121 (121)} Birtekli besmúyeshli plastinkanıń tóbeleri berilgen:
$A(2;3), \ B(0;6), \ C(-1;5), \ D(0;1)$ hám $E(1;1)$. Onıń awırlıq
orayı koordinataların anıqlań.

\textbf{122 (122)} Eki tóbesi $A(3;1)$ hám $B(1;-3)$ noqatlarında, al
úshinshi $C$ tóbesi $Oy$ kósherine tiyisli bolǵan úshmúyeshliktiń
maydanı $S=3$ ke teń. $C$ tóbesiniń koordinataların anıqlań.

\textbf{123 (123)} Eki tóbesi $A(2;1)$ hám $B(3;-2)$ noqatlarında, al
úshinshi $C$ tóbesi $Ox$ kósherine tiyisli bolǵan úshmúyeshliktiń
maydanı $S=4$ ke teń. $C$ tóbesiniń koordinataların anıqlań.

\section*{\S 12. Tuwrınıń ulıwma teńlemesi. 
Tuwrınıń múyeshlik koefficientlik teńlemesi. 
Eki tuwrı arasındaǵı múyesh. Eki tuwrınıń parallellik hám 
perpendikuliyarlıq shártleri}

\textbf{173 (210)} Berilgen $M_1 (3; 1) $, $M_2 (2; 3) $, $M_3 (6; 3) $,
$M_4 (-3;-3) $. $M_5 (3;-1) $, $M_6 (-2; 1) $ noqatlardıń qaysıları
$2x-3y-3 = 0$ tuwrısına tiyisli hám qaysıları tiyisli
emes.

\textbf{174 (211)} $P_1$, $P_2$, $P_3$, $P_4$, $P_5$  noqatları 
$3x-2y-6=0$ tuwrısına tiyisli hám abscissaları sáykes túrde 
4, 0, 2, -2, -6 ǵa teń. Olardıń ordinataların tabıń.

\textbf{175 (212)} $Q_1$, $Q_2$, $Q_3$, $Q_4$, $Q_5$ noqatları 
$x-3y+2=0$ tuwrısına tiyisli hám ordinataları sáykes túrde 
1, 0, 2, -1, 3 ke teń. Olardıń abscissaların tabıń.

\textbf{184 (221).1} $5x-y+3=0$ tuwrısınıń $k$ múyeshlik
koefficientin hám $Oy$ kósherinen kesip alǵan kesindiniń algebralıq
mánisi $b$ nı anıqlań.

\textbf{184 (221).2} $2x+3y-6=0$ tuwrısınıń $k$ múyeshlik
koefficientin hám $Oy$ kósherinen kesip alǵan kesindiniń algebralıq
mánisi $b$ nı anıqlań.

\textbf{184 (221).3} $5x+3y+2=0$ tuwrısınıń $k$ múyeshlik
koefficientin hám $Oy$ kósherinen kesip alǵan kesindiniń algebralıq
mánisi $b$ nı anıqlań.

\textbf{184 (221).4} $3x+2y=0$ tuwrısınıń $k$ múyeshlik
koefficientin hám $Oy$ kósherinen kesip alǵan kesindiniń algebralıq
mánisi $b$ nı anıqlań.

\textbf{184 (221).5} $y-3=0$ tuwrısınıń $k$ múyeshlik
koefficientin hám $Oy$ kósherinen kesip alǵan kesindiniń algebralıq
mánisi $b$ nı anıqlań.

\textbf{243.1} Ulıwma teńlemesi menen berilgen tuwrılardıń
óz-ara jaylasıwın anıqlań, eger kesilisetuǵın bolsa kesilisiw noqatın 
tabıń: $12x+15y-39=0, 16x-9y-23=0$.

\textbf{243.2} Ulıwma teńlemesi menen berilgen tuwrılardıń      
óz-ara jaylasıwın anıqlań, eger kesilisetuǵın bolsa kesilisiw noqatın 
tabıń: $3x+2y-27=0, x+5y-35=0$.

\textbf{243.3} Ulıwma teńlemesi menen berilgen tuwrılardıń      
óz-ara jaylasıwın anıqlań, eger kesilisetuǵın bolsa kesilisiw noqatın 
tabıń: $12x+59y-19=0, 8x+33y-19=0$.

\textbf{243.4} Ulıwma teńlemesi menen berilgen tuwrılardıń      
óz-ara jaylasıwın anıqlań, eger kesilisetuǵın bolsa kesilisiw noqatın 
tabıń: $6x+10y+9=0, 3x+5y-6=0$.

\textbf{243.5} Ulıwma teńlemesi menen berilgen tuwrılardıń      
óz-ara jaylasıwın anıqlań, eger kesilisetuǵın bolsa kesilisiw noqatın 
tabıń: $14x-9y-24=0, 7x-2y-17=0$.

\textbf{243.6} Ulıwma teńlemesi menen berilgen tuwrılardıń      
óz-ara jaylasıwın anıqlań, eger kesilisetuǵın bolsa kesilisiw noqatın
tabıń: $2x-3y+12=0, 4x-6y-21=0$.

\textbf{243.7} Ulıwma teńlemesi menen berilgen tuwrılardıń      
óz-ara jaylasıwın anıqlań, eger kesilisetuǵın bolsa kesilisiw noqatın
tabıń: $2y+9=0, y-5=0$.

\textbf{243.8} Ulıwma teńlemesi menen berilgen tuwrılardıń      
óz-ara jaylasıwın anıqlań, eger kesilisetuǵın bolsa kesilisiw noqatın
tabıń: $4x-7=0, 3x+8=0$.

\textbf{243.9} Ulıwma teńlemesi menen berilgen tuwrılardıń
óz-ara jaylasıwın anıqlań, eger kesilisetuǵın bolsa kesilisiw noqatın
tabıń: $2x-5y+1=0, 6x-15y+3=0$.

\textbf{243.10} Ulıwma teńlemesi menen berilgen tuwrılardıń
óz-ara jaylasıwın anıqlań, eger kesilisetuǵın bolsa kesilisiw noqatın
tabıń: $x-5=0, y+12=0$.

\textbf{243.11} Ulıwma teńlemesi menen berilgen tuwrılardıń
óz-ara jaylasıwın anıqlań, eger kesilisetuǵın bolsa kesilisiw noqatın
tabıń: $x\sqrt{2}+12=0, 4x+24\sqrt{2}=0$.

\textbf{243.12} Ulıwma teńlemesi menen berilgen tuwrılardıń
óz-ara jaylasıwın anıqlań, eger kesilisetuǵın bolsa kesilisiw noqatın
tabıń: $3x+y\sqrt{3}=0, x\sqrt{3}+3y-6=0$.

\textbf{244 (291).1} $a$ hám $b$ parametrleriniń qanday mánislerinde
$ax-2y-1=0$, $6x-4y-b=0$ tuwrıları ulıwma noqatqa iye boladı?

\textbf{244 (291).2} $a$ hám $b$ parametrleriniń qanday mánislerinde
$ax-2y-1=0$, $6x-4y-b=0$ tuwrıları parallel boladı?

\textbf{244 (291).3} $a$ hám $b$ parametrleriniń qanday mánislerinde
$ax-2y-1=0$, $6x-4y-b=0$ tuwrıları betlesedi?

\textbf{245 (292)} $m$ hám $n$ parametrleriniń qanday mánislerinde
$mx+8y+n=0$, $2x+my-1=0$ tuwrıları parallel boladı?

\textbf{246 (293*)} $m$ parametriniń qanday mánislerinde 
$(m-1)x+my-5=0$, $mx+(2m-1)y+7=0$ tuwrıları abscissa
kósherinde jatıwshı noqatta kesilisedi.

\textbf{247 (294)} $m$ parametriniń qanday mánislerinde 
$mx+(2m+3)y+m+6=0$, $(2m+1)x+(m-1)y+m-2=0$ tuwrıları ordinata
kósherinde jatıwshı noqatta kesilisedi.

\textbf{248 (295).1}. $3x-y+2=0$, $4x-5y+5=0$, $2x+3y-1=0$ 
tuwrıları bir noqatta kesilise me?

\textbf{248 (295).2}. $5x+3y-7=0$, $x-2y-4=0$, $3x-y+3=0$ 
tuwrıları bir noqatta kesilise me?

\textbf{248 (295).3}. $x+2y-17=0$, $2x-y+1=0$, $x+2y-3=0$ 
tuwrıları bir noqatta kesilise me?

\textbf{248 (295).4}. $2x-y+2=0$, $4x-2y+4=0$, $6x-3y+6=0$ 
tuwrıları bir noqatta kesilise me?

\textbf{253 (300*).} $5x-3y+15=0$ tuwrısınıń koordinata múyeshinen
kesip alǵan úshmúyeshliktiń maydanın esaplań.

\textbf{254 (301*).} $M(-3;8)$ noqatınan ótip, koordinata kósherlerinen
teńdey kesindilerdi kesip alatuǵın tuwrılardıń teńlemesin dúziń.

\textbf{255 (301*).} $M(3;3)$ noqatınan ótip, koordinata kósherlerinen teńdey
kesindilerdi kesip alatuǵın tuwrılardıń teńlemesin dúziń.

\textbf{256 (303*).} $P(2;2)$ noqatınan ótip, koordinata múyeshinen 
maydanı 1 ge teń úshmúyeshlik kesip alatuǵın tuwrılardıń 
teńlemesin dúziń.

\textbf{257 (304).} $B(-5;5)$ noqatınan ótip, koordinata múyeshinen
maydanı 50 ge teń úshmúyeshlik kesip alatuǵın tuwrılardıń teńlemesin
dúziń.

\textbf{258 (305)}. $P(8;6)$ noqatınan ótip, koordinata múyeshinen 
maydanı 12 ge teń úshmúyeshlik kesip alatuǵın tuwrılardıń teńlemesin 
dúziń.

\textbf{259 (306).} $P(12;6)$ noqatınan ótip, koordinata múyeshinen 
maydanı 150 ge teń úshmúyeshlik kesip alatuǵın tuwrılardıń 
teńlemesin dúziń.

\textbf{260 (307).} $M(4;3)$ noqatınan, koordinata múyeshinen 
maydanı 3 ke teń úshmúyeshlik kesip alatuǵın tuwrı júrgizildi. 
Usı tuwrınıń koordinata kósherleri menen kesilisiw noqatları 
koordinataların anıqlań.

\section*{\S 14. Tuwrınıń normal teńlemesi. Noqattan tuwrıǵa shekemgi aralıqtı 
tabıw máselesi}

\textbf{264 (312*).1} $A(3;-2)$ noqatınan $3x+4y-15=0$ tuwrısına 
shekemgi awısıwdı hám aralıqtı esaplań.

\textbf{264 (312*).2} $B(-1;5)$ noqatınan $5x+12y-26=0$ tuwrısına 
shekemgi awısıwdı hám aralıqtı esaplań.

\textbf{264 (312*).3} $C(0;7)$ noqatınan $2x+3y-13=0$ tuwrısına 
shekemgi awısıwdı hám aralıqtı esaplań.

\textbf{264 (312*).4} $D(-3;-5)$ noqatınan $4x-3y+20=0$ tuwrısına 
shekemgi awısıwdı hám aralıqtı esaplań.

\textbf{265 (314*).} $M(4;-5)$ noqatı kvadrattıń bir tóbesi. 
Kvadrattıń bir tárepi $5x-4y+1=0$ tuwrısında jatadı. 
Kvadrattıń maydanın esaplań.

\section*{\S 15. Tuwrılar dástesiniń teńlemesi}

\textbf{302 (353).} $\alpha(2x+3y-1)+\beta(x-2y-4)=0$ teńlemesi
menen berilgen tuwrılar dástesiniń orayınıń koordinataların anıqlań.

\textbf{303 (354).1} $\alpha(x+2y-5)+\beta(3x-2y+1)=0$ tuwrılar
dástesi arasınan, tómendegi tuwrılardıń teńlemesin tabıń:
koordinata basınan ótetuǵın.

\textbf{303 (354).2} $\alpha(x+2y-5)+\beta(3x-2y+1)=0$ tuwrılar
dástesi arasınan, tómendegi tuwrılardıń teńlemesin tabıń:
$M(4;-1)$ noqatınan ótetuǵın.

\textbf{303 (354).3} $\alpha(x+2y-5)+\beta(3x-2y+1)=0$ tuwrılar
dástesi arasınan, tómendegi tuwrılardıń teńlemesin tabıń:
$Ox$ kósherine parallel.

\textbf{303 (354).4} $\alpha(x+2y-5)+\beta(3x-2y+1)=0$ tuwrılar
dástesi arasınan, tómendegi tuwrılardıń teńlemesin tabıń:
$Oy$ kósherine parallel.

\textbf{303 (354*).5} $\alpha(x+2y-5)+\beta(3x-2y+1)=0$ tuwrılar
dástesi arasınan, tómendegi tuwrılardıń teńlemesin tabıń:
$3x+4y-10=0$ tuwrısına parallel.

\textbf{303 (354*).6} $\alpha(x+2y-5)+\beta(3x-2y+1)=0$ tuwrılar
dástesi arasınan, tómendegi tuwrılardıń teńlemesin tabıń:
$2x+3y+7=0$ tuwrısına perpendikuliyar.

\textbf{303 (354*).7} $\alpha(x+2y-5)+\beta(3x-2y+1)=0$ tuwrılar
dástesi arasınan, tómendegi tuwrılardıń teńlemesin tabıń:
$Ox$ kósherine perpendikuliyar.

\textbf{303 (354*).8} $\alpha(x+2y-5)+\beta(3x-2y+1)=0$ tuwrılar
dástesi arasınan, tómendegi tuwrılardıń teńlemesin tabıń:
$Oy$ kósherine perpendikuliyar.
\end{document}