\documentclass{article}
\usepackage{amsmath}
\usepackage{amssymb}
\usepackage[utf8]{inputenc}
% \usepackage[T1,T2A]{fontenc}
\usepackage[russian]{babel}
\usepackage[a4paper, total={6in, 8in}]{geometry}

\begin{document}

% Túsindirme:
% \emph{10 (20*) jazıwı bul sorawnamadaǵı 10-esap D.B. Kletenik 
% avtorlıǵındaǵı "Сборник задач по аналитической геометрии" 
% kitabındaǵı 20-esapqa uqsas ekenligin ańlatadı, eger "*" 
% simvolı bolmasa 20-esaptıń ózi ekenligin ańlatadı.}

\section*{\S 4. Kesindiniń baǵıtı. Kesindiniń qálegen kósherge proekciyası. 
Kesindiniń koordinata kósherlerine proekciyası. 
Kesindiniń uzınlıǵı hám polyar múyeshi. 
Eki noqat arasındaǵı aralıq}

\textbf{80.} \(A(4;2)\) noqatı arqalı, eki koordinata kósherlerine
urınıwshı sheńber ótkerildi. Onıń orayı $C$ nı hám radiusı
$R$ di tabıń.

\textbf{81.} \(M_{1}(1; - 2)\) noqatı arqalı, padiusı 5 ke teń,
$Ox$ kósherine urınıwshı sheńber ótkerildi. Usı sheńberdiń orayı
$С$ nı anıqlań.

\textbf{85.} Úshmúyeshliktiń tóbeleri \(M_{1}( - 3;6),\ M_{2}(9; - 10)\) 
hám \(M_{3}( - 5;4)\) berilgen. Usı úshmúyeshlikke sırtlay sızılǵan
sheńber orayı $C$ nı hám radiusı $R$ di anıqlań.

\section*{\S 5. Kesindini berilgen qatnasta bóliw}

\textbf{98.} Úshmúyeshliktiń tóbeleri
\(A( - 1; - 1),\ B(3;5),\ C( - 4;1)\) berilgen. $A$ tóbesi sırtqı
múyeshi bssektrisasınıń, $BC$ tárepiniń dawamı menen kesilisiw
noqatın tabıń.

\textbf{99.} Úshmúyeshliktiń tóbeleri
\(A(3; - 5),\ B(1; - 3),\ C(2; - 2)\) berilgen. $B$ tóbesi sırtqı
múyeshi bessektrisa uzınlıǵın anıqlań.

\textbf{109.} Eki tóbesi \(A(2; - 3)\) hám \(B( - 5;1)\) noqatlarında,
úshinshi tóbesi $C$ ordinata kósherine tiyisli úshmúyeshliktiń
medianalarınıń kesilisiw noqatı $M$ abscissa kósherinde jatadı.
$M$ hám $C$ noqatlarınıń koordinataların anıqlań.

\section*{\S 6. Úshmúyeshliktiń maydanı}

\textbf{126.} Eki tóbesi \(A(2;1)\) hám \(B(5; - 3)\) noqatlarında, hám
diagonallarınıń kesilisiw noqatı ordinata kósherine tiyisli
parallelogrammnıń maydanı \(S = 17\) ke teń. Qalǵan eki tóbesiniń
koordinataların anıqlań.

\section*{\S 12. Tuwrınıń ulıwma teńlemesi. 
Tuwrınıń múyeshlik koefficientlik teńlemesi. 
Eki tuwrı arasındaǵı múyesh. Eki tuwrınıń parallellik hám 
perpendikuliyarlıq shártleri}

\textbf{181 (218).} Eki tóbesi \(A(1; - 2),\ B(2;3)\) noqatlarda jaylasqan,
maydanı \(S = 8\) ge teń bolǵan úshmúyeshliktiń úshinshi tóbesi
$C$ \(2x + y - 2 = 0\) tuwrısına tiyisli. Usı $C$ tóbesiniń
koordinatasın anıqlań.

\textbf{182.} Eki tóbesi \(A(2; - 3),\ B(3; - 2)\) noqatlarda
jaylasqan, maydanı \(S = 1,5\) ke teń bolǵan úshmúyeshliktiń,
awırlıq orayı \(3x - y - 8 = 0\) tuwrısına tiyisli. Úshinshi $C$
tóbesiniń koordinatasın anıqlań.

\textbf{189 (227*).} \(N(2; - 5)\) noqatınıń \(9x - 7y + 30 = 0\) tuwrısına
qarata simmetriyalı noqatın tabıń.

\textbf{192.} Úshmúyeshliktiń \(A( - 3; - 2),\ B(5; - 4),\ C( - 1;3)\) 
tóbelerinen ótip, qarama-qarsı tárepke parallel tuwrılardıń teńlemelerin
dúziń.

\textbf{193.} Úshmúyeshlik tárepleriniń ortaları
\(M(5;3),\ N(3; - 4),\ E(2;1)\) noqatlarda jaylasqan. Tárepleriniń
teńlemelerin dúziń.

\textbf{194.} Eki noqat \(A(3; - 5)\) hám \(B( - 2;3)\) berilgen.
$B$ noqattan ótip, $AB$ kesindige perpendikuliyar tuwrı
teńlemesin dúziń.

\textbf{195.} Egerde \(M(4;5)\) noqatı, koordinata basınan tuwrıǵa
júrgizilgen perpendikuliyardıń ultanı bolsa, usı tuwrınıń teńlemesin
dúziń.

\textbf{196.} Úshmúyeshliktiń tóbeleri
\(A(3;2),\ B( - 4;4),\ C( - 2; - 5)\) koordinataları menen berilgen.
Biyiklikleriniń teńlemesin dúziń.

\textbf{197.} Úshmúyeshliktiń tárepleri \(x + 5y - 7 = 0\) ,
\(4x - y - 7 = 0\) , \(x + 3y - 31 = 0\) teńlemeleri menen berilgen.
Biyiklikleriniń kesilisiw noqatın tabıń.

\textbf{198 (236).} Úshmúyeshliklerdiń tóbeleri
\(A(1; - 1),\ B( - 2;1),\ C(3;5)\) noqatlarında jaylasqan. $A$
tóbesinen ótip, $B$ tóbesinen júrgizilgen medianaǵa
perpendikuliyar tuwrınıń teńlemesin dúziń.

\textbf{199.} Úshmúyeshliklerdiń tóbeleri
\(A(2; - 2),\ B(3; - 5),\ C(5;7)\) noqatlarında jaylasqan. $C$
tóbesinen ótip, $A$ tóbesinen júrgizilgen bissektrisaǵa
perpendikuliyar tuwrınıń teńlemesin dúziń.

\textbf{205 (245).} Úshmúyeshliktiń tóbeleri \(A(1;-2),\ B(5; 4)\) hám
\(C(-2;0)\) noqatlarda jaylasqan. $A$ tóbesindegi ishki hám sırtqı
múyeshleri bissektrisalarınıń teńlemelerin dúziń.

\textbf{206.} \(A(11; - 15)\) hám \(B( - 7;3)\) noqatlarınan
teńdey aralıqta ham \(C(3; 5)\) noqatınan ótetuǵın tuwrınıń teńlemesin
dúziń.

\textbf{207.} \(Q(5; - 6)\) noqatınıń, \(A(3;8)\) hám \(B(7;5)\) 
noqatlardan ótken tuwrıdaǵı proekciyasın tabıń.

\textbf{208.} \(N( - 4; - 7)\) noqatınıń, \(A(2;0)\) hám \(B( - 3;5)\) 
noqatlardan ótken tuwrıǵa qarata simmetriyalı noqattı tabıń.

\textbf{209.} \(P(2; - 3)\) hám \(Q( - 8; - 2)\) noqatlardan
aralıqlarınıń qosındısı eń kishi bolǵan, abscissa kósherinde jaylasqan
noqattı tabıń.

\textbf{210.} \(P(2;5)\) hám \(Q( - 3;2)\) noqatlardan aralıqlarınıń
ayırması eń úlken bolǵan, ordinata kósherinde jaylasqan noqattı tabıń.

\textbf{211.} \(A( - 5;5)\) hám \(B( - 7;1)\) noqatlardan
aralıqlarınıń qosındısı eń kishi bolǵan, \(2x - y - 5 = 0\) tuwrısında
jaylasqan noqattı tabıń.

\textbf{212.} \(A(0;5)\) hám \(B(5;2)\) noqatlardan aralıqlarınıń
ayırması eń úlken bolǵan, \(3x - y - 2 = 0\) tuwrısında jaylasqan
noqattı tabıń.

\textbf{214}. \(P(3;5)\) noqatınan ótip, \(4x + 6y - 7 = 0\) tuwrısı
menen \(45^{0}\) múyesh jasap kesilisetuǵın tuwrı teńlemesin dúziń.

\textbf{215.} \(A(4;5)\) noqatı, diagonalı \(7x - y - 8 = 0\) teńlemesi
menen berilgen kvadrattıń bir tóbesi. Usı kvadrattıń tárepleriniń hám
ekinshi diagonalınıń teńlemesin dúziń.

\textbf{216.} \(A(3;7)\) hám \(C(6; - 5)\) noqatları kvadrattıń
qarama-qarsı tóbeleri. Onıń tárepleriniń teńlemesin dúziń.

\textbf{217.} Bir tárepi \(x - 4y - 8 = 0\) tuwrısında jatatuǵın
kvadrattıń awırlıq orayı \(M(1;1)\) noqatında jaylasqan. Usı kvadrattıń
qalǵan tárepleri jatatuǵın tuwrılardıń teńlemelerin dúziń.

\textbf{222.} Úshmúyeshliktiń eki tóbesi \(A(6;4),\ B( - 10;2)\) , hám
biyiklikleriniń kesilisiw noqatı \(N(5;2)\) berilgen. Úshinshi $C$
tóbesiniń koordinataların tabıń.

\textbf{223.} $ABC$ úshmúyeshliginiń eki tóbesi
\(A(6; - 2),\ B(10;14)\) , hám biyiklikleriniń kesilisiw noqatı
\(N(4; - 1)\) berilgen. Bul úshmúyeshliktiń tárepleri teńlemesin dúziń.

\textbf{224.} $ABC$ úshmúyeshliginde \(AB:5x - 3y + 2 = 0\) 
tárepiniń, hám de \(AN:4x - 3y + 1 = 0,\ BN:7x + 2y - 22 = 0\) 
biyiklikleriniń tenlemeleri berilgen. Usı úshmúyeshliktiń qalǵan eki
tárepiniń hám úshinshi biyikliginiń teńlemelerin dúziń.

\textbf{225.} $ABC$ úshmúyeshliginiń bir tóbesi \(A(1;3)\) noqatta,
hám eki medianası \(x - 2y + 1 = 0\ ,\ y - 1 = 0\) tuwrılarında
jaylasqan. Tárepleriniń teńlemelerin dúziń.

\textbf{226.} $ABC$ úshmúyeshliginiń bir tóbesi \(B( - 4; - 5)\) ,
hám eki biyikliginiń teńlemeri:
\(3x + 8y + 13 = 0\ ,\ 5x + 3y - 4 = 0\) berilgen. Tárepleriniń
teńlemelerin dúziń.

\textbf{227.} $ABC$ úshmúyeshliginiń bir tóbesi \(C(4; - 1)\) , hám
eki bissektrisasıniń teńlemeri: \(x - 1 = 0\ ,\ x - y - 1 = 0\) 
berilgen. Tárepleriniń teńlemelerin dúziń.

\textbf{228.} $ABC$ úshmúyeshliginiń bir tóbesin \(B(2;6)\) , hám
bir tóbesinen júrgizilgen biyikliginiń: \(x - 7y + 15 = 0\) , hám
bissektrisasınıń: \(7x + y + 5 = 0\) teńlemelerin bilgen jaǵdayda,
tárepleriniń teńlemelerin dúziń.

\textbf{229.} $ABC$ úshmúyeshliginiń bir tóbesin \(A(2; - 1)\) , hám
de basqa-basqa tóbelerinen júrgizilgen biyikliginiń:
\(3x - 4y + 27 = 0\) , hám bissektrisasınıń: \(x + 2y - 5 = 0\) 
teńlemelerin bilgen jaǵdayda, tárepleriniń teńlemelerin dúziń.

\textbf{230.} $ABC$ úshmúyeshliginiń bir tóbesin \(C(4; - 1)\) , hám
bir tóbesinen júrgizilgen biyikliginiń: \(2x - 3y + 12 = 0\) , hám
medianasınıń: \(2x + 3y = 0\) teńlemelerin bilgen jaǵdayda, tárepleriniń
teńlemelerin dúziń.

\textbf{231.} $ABC$ úshmúyeshliginiń bir tóbesin \(A(2; - 7)\) , hám
de basqa-basqa tóbelerinen júrgizilgen biyikliginiń:
\(3x + y + 11 = 0\) , hám medianasınıń: \(x + 2y + 7 = 0\) teńlemelerin
bilgen jaǵdayda, tárepleriniń teńlemelerin dúziń.

\textbf{232.} $ABC$ úshmúyeshliginiń bir tóbesin \(C(4;3)\) , hám de
bir tóbesinen júrgizilgen medianasınıń: \(4x + 13y - 10 = 0\) , hám
bissektrisasınıń: \(x + 2y - 5 = 0\) teńlemelerin bilgen jaǵdayda,
tárepleriniń teńlemelerin dúziń.

\textbf{233.} $ABC$ úshmúyeshliginiń bir tóbesin \(C(4;3)\) , hám de
basqa-basqa tóbelerinen júrgizilgen medianasınıń:
\(6x + 10y - 59 = 0\) , hám bissektrisasınıń: \(x - 4y + 10 = 0\) 
teńlemelerin bilgen jaǵdayda, tárepleriniń teńlemelerin dúziń.

\textbf{234.} Koordinata basınan ótip,
\(2x + y + 9 = 0,\ x - y + 12 = 0\) tuwrıları menen birge, maydanı
1,5 kv.birlikke teń úshmúyeshlik jasaytuǵın tuwrınıń teńlemesin dúziń.

\textbf{235.} \(2x - y - 2 = 0,\ x + y + 3 = 0\) tuwrıları
arasındaǵı kesindi, berilgen \(P(3;0)\) noqatta teń ekige bólinetuǵın
tuwrınıń teńlemesin dúziń.

\textbf{236.} \(x - 3y - 4 = 0,\ x - 3y + 4 = 0\) tuwrıları
arasındaǵı kesindi, berilgen \(P(6;2)\) noqatta teń ekige bólinetuǵın
tuwrınıń teńlemesin dúziń.

\textbf{237.} \(x - 4y - 5 = 0,\ x - 4y + 3 = 0\) tuwrıları
arasındaǵı kesindi, berilgen \(P(1;1)\) noqatta teń ekige bólinetuǵın
tuwrınıń teńlemesin dúziń.

\textbf{238.} Koordinata basınan ótip,
\(2x - y + 5 = 0,\ 2x - y + 10 = 0\) tuwrıları arasındaǵı kesindi
uzınlıǵı \(\sqrt{10}\) ǵa teń bolǵan tuwrılar teńlemesin dúziń.

\textbf{239.} \(P( - 5;4)\) noqatınan ótip,
\(x + 2y + 1 = 0,\ x + 2y - 1 = 0\) tuwrıları arasındaǵı kesindi
uzınlıǵı 5-ke teń bolǵan tuwrılar teńlemesin dúziń.

\section*{\S 14. Tuwrınıń normal teńlemesi. Noqattan tuwrıǵa shekemgi aralıqtı 
tabıw máselesi}

\textbf{272 (321*).} $ABC$ úshmúyeshliginiń tárepleri:
\(AB:\ 5x - 12y + 7 = 0;\ BC:\ x + 21y - 22 = 0;\ AC:\ 4x - 33y + 146 = 0\) 
teńlemeleri menen berilgen. Úshmúyeshliktiń awırlıq orayınan $AB$
tárepine shekemgi aralıqtı tabıń.

\textbf{281.} Kvadrattıń eki qońsı tóbeleri \(A( - 2;0),\ B(1; - 4)\) 
noqatlarında jatadı. Tárepleriniń teńlemesin dúziń.

\textbf{282.} \(A(2; - 3)\) noqatı, bir tárepi \(4x - 3y + 11 = 0\) 
tuwrısında jatatuǵın kvadrattıń bir tóbesi. Qalǵan tárepleri tiyisli
tuwrılardıń teńlemesin dúziń.

\textbf{283.} Kvadrattıń eki tárepiniń teńlemesi:
\(4x + 3y + 7 = 0,\ 4x + 3y - 15 = 0\) , hám bir tóbesi \(A(3;1)\) 
berilgen. Qalǵan eki tárepiniń teńlemelerin dúziń.

\textbf{284.} Kvadrattıń eki tárepiniń teńlemeleri berilgen:
\(5x + 12y - 15 = 0,\ 5x + 12y + 25 = 0.\) \(M( - 3;4)\) noqatı
kvadrattıń tárepine tiyisli ekenligin bilgen jaǵdayda, qalǵan
tárepleriniń teńlemelerin dúziń.

\textbf{285.} $M$ noqatınıń \(12x - 5y + 49 = 0\) hám
\(\ 3x + 4y - 20 = 0\) tuwrılarınan awısıwları sáykes $-4$ hám
$-7$. $M$ noqatınıń koordinataların tabıń.

\textbf{286.} \(P(4; - 5)\) noqatınan ótip,
\(A(5; - 2)\) hám \(B(3;9)\) noqatlarınan teńdey aralıqta jaylasqan
tuwrınıń teńlemesin dúziń.

\textbf{288.} Kesiliwshi tuwrılar arasındaǵı múyesh bissektrisalarınıń
teńlemesin dúziń:\\
1) $x + 4y + 9 = 0$, $4x - y + 10 = 0$; \\
2) $4x + 3y - 5 = 0$, $5x + 12y + 2 = 0$; \\
3) $2x - 5y + 7 = 0$, $5x - 2y + 12 = 0$; \\
4) $7x - 24y + 2 = 0$, $3x - 4y - 1 = 0$.

\textbf{289.} \(P(2; - 1)\) noqatınan ótip,
\(2x - y + 5 = 0,\ 3x + 6y - 1 = 0\) tuwrıları menen teń qaptallı
úshmúyeshlik payda etetuǵın tuwrınıń teńlemesin dúziń.

\textbf{296.} Berilgen tuwrılardıń:
\(3x + y + 10 = 0\) hám \(2x - 6y - 5 = 0\) kesilisiwinde payda
bolǵan, koordinata bası jatatuǵın múyesh bissektrisasınıń teńlemesin
dúziń.

\textbf{297.} Berilgen tuwrılardıń: \(4x + 4y + 1 = 0\) hám
\(2x - 2y - 7 = 0\) kesilisiwinde payda bolǵan, koordinata bası
jatatuǵın múyeshke qońsı múyesh bissektrisasınıń teńlemesin dúziń.

\textbf{298.} Berilgen tuwrılardıń:
\(x + 2y - 10 = 0\) hám \(3x - 6y - 5 = 0\) kesilisiwinde payda
bolǵan, \(M(1; - 3)\) noqatı jatatuǵın múyesh bissektrisasınıń
teńlemesin dúziń.

\textbf{299.} Berilgen tuwrılardıń:
\(2x + 3y - 8 = 0\) hám \(3x - 2y - 5 = 0\) kesilisiwinde payda
bolǵan, \(M(2; - 3)\) noqatı tiyisli múyeshke qońsı múyesh
bissektrisasınıń teńlemesin dúziń.

\textbf{300.} Berilgen tuwrılardıń:
\(3x + 4y - 10 = 0\) hám \(12x - 5y - 13 = 0\) kesilisiwinde payda
bolǵan súyir múyesh bissektrisasınıń teńlemesin dúziń.

\textbf{301.} Berilgen tuwrılardıń: \(3x - y - 10 = 0\) hám
\(2x - 6y - 1 = 0\) kesilisiwinde payda bolǵan doǵal múyesh
bissektrisasınıń teńlemesin dúziń.

\section*{Tuwrılar dástesiniń teńlemesi}

\textbf{308 (359).} Úshmúyeshliktiń tárepleriniń teńlemeleri berilgen:
\(x - 4y + 11 = 0,\ 5x + 4y - 17 = 0,\ x + 2y - 1 = 0.\) 
Úshmúyeshliktiń tóbeleriniń koordinataların anıqlamay, onıń
biyiklikleriniń teńlemelerin dúziń.

\textbf{309.} \(3x + 2y + 5 = 0\) hám \(2x + 7y - 8 = 0\) tuwrılarınıń
kesilisiw noqatınan ótip, \(2x + 3y - 7 = 0\) tuwrısı menen
$45^0$ múyesh jasawshı tuwrınıń teńlemesin dúziń.
Máseleni berilgen tuwrılardıń kesilisiw noqatınıń koordinataların anıqlamay
sheshiń.

\textbf{310.} $ABC$ úshmúyeshliginde \(AB:2x + 3y + 5 = 0\) 
tárepi, \(AN:x + 5y - 3 = 0\) hám \(\ BN:x + y - 1 = 0\) 
biyiklikleri teńlemeleri berilgen. Tóbeleriniń hám biyiklikleriniń
kesilisiw noqatınıń koordinataların anıqlamay, qalǵan eki tárepiniń hám
úshinshi biyikliginiń teńlemesin dúziń.

\textbf{311.} $ABC$ úshmúyeshliginiń bir tóbesi koordinataları
\(A(2; - 1)\) hám bir tóbeden túsirilgen biyiklik
\( 7x - 10y + 1 = 0\) , bissektrisa \(3x - 2y + 5 = 0\) 
teńlemeleri berilgen. $B$ hám $C$ tóbeleri koordinataların
anıqlamay, $ABC$ úshmúyeshliginiń tárepleri teńlemelerin dúziń.

\textbf{312.} \(\alpha\(\ 2x + 3y + 5) + \beta(\ x + y + 3) = 0\) 
tuwrılar dástesi berilgen. Usı tuwrılar dástesinen,
\(x - y - 5 = 0\) hám \(x - y - 2 = 0\) tuwrıları arasındaǵı
kesindi uzınlıǵı \(\sqrt{5}\) ke teń bolǵan tuwrılardıń teńlemelerin
tabıń.

\textbf{317.}
\(\alpha\(\ 2x - 3y + 20) + \beta(3\ x + 5y - 27) = 0\) tuwrılar
dástesiniń orayı, diagonalı \(x + 7y - 16 = 0\) tuwrısında jatatuǵın
kvadrattıń bir tóbesi. Usı kvadrattıń tárepleriniń hám ekinshi diagonali
teńlemelerin dúziń.

\textbf{318.}
\(\alpha\(\ 2x + 5y + 4) + \beta(3\ x - 2y + 25) = 0\) tuwrılar
dástesi berilgen. Usı tuwrılar dástesinen, koordinata kósherlerinen
nolge teń emes, teńdey ólshemdegi (koordinata basınan baslap)
kesindilerdi kesip alıwshı tuwrı teńlemesin tabıń.

\textbf{319.} \(\alpha\(\ 2x + y + 1) + \beta(\ x - 3y - 10) = 0\) 
tuwrılar dástesi berilgen. Usı tuwrılar dástesinen, koordinata
kósherlerinen nolge teń emes, teńdey ólshemdegi (koordinata basınan
baslap) kesindilerdi kesip alıwshı tuwrılar teńlemesin tabıń.

\textbf{320.}
\(\alpha\(\ 21x + 8y - 18) + \beta(11x + 3y + 12) = 0\) tuwrılar
dástesi berilgen. Usı tuwrılar dástesinen, koordinata kósherleri menen
birge, maydanı 9 kv.birlikke teń úshmúyeshlikler kesip alıwshı tuwrılar
teńlemesin tabıń.

\textbf{321.} \(\alpha\(\ 2x + y + 4) + \beta(\ x - 2y - 3) = 0\) 
tuwrılar dástesi berilgen. Usı tuwrılar dástesinen, berilgen
\(P(2; - 3)\) noqatınan aralıǵı \(d = \sqrt{10}\) -ǵa teń tuwrılar
teńlemesin tabıń.

\textbf{322.} \(\alpha\(\ 2x - y - 4) + \beta(\ x - y - 4) = 0\) 
tuwrılar dástesi berilgen. Usı tuwrılar dástesinen, berilgen
\(Q(3; - 1)\) noqatınan aralıǵı \(d = 3\) -ke teń tuwrılar teńlemesin
tabıń.

\textbf{323.} \(3x + y - 5 = 0\ ,\ x - 2y + 10 = 0\) tuwrılarınıń
kesilisiw noqatınan ótip, berilgen \(A( - 1; - 2)\) noqatınan aralıǵı
\(d = 5\) ke teń tuwrılar teńlemesin dúziń. Máseleni berilgen
tuwrılardıń kesilisiw noqatın anıqlamay sheshiń.

\textbf{324.} \(\alpha\(\ 5x + 2y + 4) + \beta(x + 9y - 25) = 0\) 
tuwrılar dástesi berilgen. Usı tuwrılar dástesinen,
\(12x + 8y - 7 = 0,\ 2x - 3y + 5 = 0\) tuwrıları birge, teń
qaptallı úshmúyeshlikler jasawshı tuwrılar teńlemesin tabıń.

\textbf{325.} \(11x + 3y - 7 = 0\ ,\ 12x + y - 19 = 0\) tuwrılarınıń
kesilisiw noqatınan ótip, berilgen
\(A(3; - 2)\) hám \(B( - 1;\ 6)\) noqatlarınan teńdey aralıqtan
ótken tuwrılar teńlemesin dúziń. Máseleni berilgen tuwrılardıń kesilisiw
noqatın anıqlamay sheshiń.

\textbf{326.}
\(\alpha_{1}\(\ 5x + 3y - 2) + \beta_{1}(3x - y - 4) = 0\) ,
\(\alpha_{2}\(x - y + 1) + \beta_{2}(2x - y - 2) = 0\) eki tuwrılar
dástesi teńlemeleri berilgen. Usı tuwrılar dásteleriniń orayın
anıqlamay, olardıń ekewinede tiyisli bolǵan tuwrınıń teńlemesin dúziń.

\textbf{327.} \(ABCD\) - tórtmúyeshliginiń tárepleri
\(AB:\ 5x + y + 13 = 0\) , \(BC:2x - 7y - 17 = 0\) ,
\(CD:\ 3x + 2y - 13 = 0\) , \(DA:\ 3x - 4y + 17 = 0\) teńlemeleri menen
berilgen. Tóbeleriniń koordinataların anıqlamay $AC$ hám $BD$
diagonallarınıń teńlemelerin dúziń.

\textbf{328.}
\(\alpha\(\ 2x - 3y + 20) + \beta(3\ x + 5y - 27) = 0\) tuwrılar
dástesiniń orayı, eki biyikliginiń teńlemeleri
\(x - 4y + 1 = 0,\ 2x + y + 1 = 0\) menen berilgen úshmúyeshliktiń bir
tóbesi. Usı úshmúyeshliktiń tárepleriniń hám úshinshi biyikliginiń
teńlemesin dúziń.

\end{document}