Tekislikda uchta vektor $\vec{a} = \{ 3; - 2\}$, $\vec{b} = \{ - 2;1\}$ va $\vec{c} = \{ 7; - 4\}$ berilgan. Bu uchta vektorning har birining qolgan ikkitasini bazis sifatida qabul qilib yoyilmasini toping.
====
$\vec{a}$ va $\vec{b}$ vektorlar $\varphi = 2\pi/3$ burchak hosil qiladi. $|\vec{a}| = 3,|\vec{b}| = 4$ ekani ma’lum. Hisoblang:
$\left(\vec{a},\vec{b} \right) $.
====
$\vec{a}$ va $\vec{b}$ vektorlar $\varphi = 2\pi/3$ burchak hosil qiladi. $|\vec{a}| = 3,|\vec{b}| = 4$ ekani ma’lum. Hisoblang:
${\vec{a}}^{2}$.
====
$\vec{a}$ va $\vec{b}$ vektorlar $\varphi = 2\pi/3$ burchak hosil qiladi. $|\vec{a}| = 3,|\vec{b}| = 4$ ekani ma’lum. Hisoblang:
${\vec{b}}^{2}$.
====
$\vec{a}$ va $\vec{b}$ vektorlar $\varphi = 2\pi/3$ burchak hosil qiladi. $|\vec{a}| = 3,|\vec{b}| = 4$ ekani ma’lum. Hisoblang:
$ (\vec{a} + \vec{b}) ^{2}$.
====
$\vec{a}$ va $\vec{b}$ vektorlar $\varphi = 2\pi/3$ burchak hosil qiladi. $|\vec{a}| = 3,|\vec{b}| = 4$ ekani ma’lum. Hisoblang:
$\left(3\vec{a} - 2\vec{b},\vec{a} + 2\vec{b} \right) $.
====
$\vec{a}$ va $\vec{b}$ vektorlar $\varphi = 2\pi/3$ burchak hosil qiladi. $|\vec{a}| = 3,|\vec{b}| = 4$ ekani ma’lum. Hisoblang:
$ (\vec{a} - \vec{b}) ^{2};$ 7) $ (3\vec{a} + 2\vec{b}) ^{2}$.
====
$\vec{a}$ va $\vec{b}$ vektorlar o‘zaro perpendikulyar; $\vec{c}$ vektor ular bilan $\pi/3$ ga teng bo‘lgan burchaklar hosil qiladi; $|\vec{a}| = 3$, $|\vec{b}| = 5,\ |\vec{c}| = 8$ ekani ma’lum, quyidagilarni hisoblang:
$\left(3\vec{a} - 2\vec{b},\vec{b} + 3\vec{c} \right) $.
====
$\vec{a}$ va $\vec{b}$ vektorlar o‘zaro perpendikulyar; $\vec{c}$ vektor ular bilan $\pi/3$ ga teng bo‘lgan burchaklar hosil qiladi; $|\vec{a}| = 3$, $|\vec{b}| = 5,\ |\vec{c}| = 8$ ekani ma’lum, quyidagilarni hisoblang:
$ (\vec{a} + \vec{b} + \vec{c}) ^{2}$.
====
$\vec{a}$ va $\vec{b}$ vektorlar o‘zaro perpendikulyar; $\vec{c}$ vektor ular bilan $\pi/3$ ga teng bo‘lgan burchaklar hosil qiladi; $|\vec{a}| = 3$, $|\vec{b}| = 5,\ |\vec{c}| = 8$ ekani ma’lum, quyidagilarni hisoblang:
$ (\vec{a} + 2\vec{b} - 3\vec{c}) ^{2}$.
====
$\vec{a} + \vec{b} + \vec{c} = 0$ shartni qanoatlantiruvchi $\vec{a},\ \vec{b}$ va $\vec{c}$ vektorlar berilgan. $|\vec{a}| = 3,\ |\vec{b}| = 1$ va $|\vec{c}| = 4$ ekani ma’lum, $\left(\vec{a},\vec{b} \right) + \left(\vec{b},\vec{c} \right) + (\vec{c}) $ ifodani hisoblang.
====
$|\vec{a}| = 3,|\vec{b}| = 5$ berilgan. $\alpha$ ning qanday qiymatida $\vec{a} + \alpha\vec{b}$, $\vec{a} - \alpha\vec{b}$ vektorlar o‘zaro perpendikulyar bo‘lishini aniqlang.
====
$a$ va $b$ vektorlar $\varphi = \pi/6$ burchak hosil qiladi; $|a| = \sqrt{3},|b| = 1$ ekani ma’lum. $p = a + b$ va $q = a - b$ vektorlar orasidagi $\alpha$ burchakni hisoblang.
====
$\vec{a} = \{ 6; - 8; - 7,5\}$ vektorga kollinear bo‘lgan $\vec{x}$ vektor $Oz$ o‘qi bilan o‘tkir burchak hosil qiladi. $|\vec{x}| = 50$ ekanini bilgan holda uning koordinatalarini toping.
====
$\vec{a} = \{ 2;1; - 1\}$ vektorga kollinear bo‘lgan va $\left(\vec{x},\vec{a} \right) = 3$ shartni qanoatlantiruvchi $\vec{x}$ vektorni toping.
====
$\vec{a}$ va $\vec{b}$ vektorlar o‘zaro perpendikulyar. $|\vec{a}| = 3,|\vec{b}| = 4$ ekani ma’lum, quyidagilarni hisoblang:
$|\lbrack\vec{a} + \vec{b},\vec{a} - \vec{b}\rbrack|$.
====
$\vec{a}$ va $\vec{b}$ vektorlar o‘zaro perpendikulyar. $|\vec{a}| = 3,|\vec{b}| = 4$ ekani ma’lum, quyidagilarni hisoblang:
$|\lbrack 3\vec{a} - \vec{b},\vec{a}-2\vec{b}\rbrack|$.
====
$\vec{a}$ va $\vec{b}$ vektorlar $\varphi = 2\pi/3$ burchak hosil qiladi. $|\vec{a}| = 1,|\vec{b}| = 2$ ekanini bilib, quyidagilarni hisoblang:
$\lbrack\vec{a},\vec{b}\rbrack^{2}$.
====
$\vec{a}$ va $\vec{b}$ vektorlar $\varphi = 2\pi/3$ burchak hosil qiladi. $|\vec{a}| = 1,|\vec{b}| = 2$ ekanini bilib, quyidagilarni hisoblang:
$\lbrack 2\overrightarrow{a} + \overrightarrow{b},\overrightarrow{a} + 2\overrightarrow{b}\rbrack^{2}$.
====
$\vec{a}$ va $\vec{b}$ vektorlar $\varphi = 2\pi/3$ burchak hosil qiladi. $|\vec{a}| = 1,|\vec{b}| = 2$ ekanini bilib, quyidagilarni hisoblang:
$\lbrack\overrightarrow{a} + 3\overrightarrow{b},3\overrightarrow{a} - \overrightarrow{b}\rbrack^{2}$
====
$\vec{a} = \{ 3; - 1; - 2\}$ va $\vec{b} = \{ 1;2; - 1\}$ vektorlar berilgan. Quyidagi vektor ko‘paytmalarning koordinatalarini toping:
$\left\lbrack \vec{a},\vec{b} \right\rbrack$.
====
$\vec{a} = \{ 3; - 1; - 2\}$ va $\vec{b} = \{ 1;2; - 1\}$ vektorlar berilgan. Quyidagi vektor ko‘paytmalarning koordinatalarini toping:
$\left\lbrack 2\vec{a} + \vec{b},\vec{b} \right\rbrack$.
====
$\vec{a} = \{ 3; - 1; - 2\}$ va $\vec{b} = \{ 1;2; - 1\}$ vektorlar berilgan. Quyidagi vektor ko‘paytmalarning koordinatalarini toping:
$\left\lbrack 2\vec{a} - \vec{b},2\vec{a} + \vec{b} \right\rbrack$.
====
$A (2; -1;2),B (1;2; 1) $ va $C (3;2;1)$ nuqtalar berilgan. Quyidagi vektor ko‘paytmalarning koordinatalarini toping:
$\lbrack\overline{AB},\overline{BC}\rbrack$.
====
$A (2; -1;2),B (1;2; 1) $ va $C (3;2;1) $ nuqtalar berilgan. Quyidagi vektor ko‘paytmalarning koordinatalarini toping:
$\lbrack\overline{BC} - 2\overline{CA},\overline{CB}\rbrack$.