\(A (4;2) \) nuqta orqali, ikkita koordinata o‘qlariga
urinma doira o‘tkazildi. Uning markazi $C$ ni va radiusi
$R$ ni toping.
====
\(M{1} (1; 2) \) nuqta orqali, radiusi 5 ga teng,
$Ox$ o‘qiga urinma aylana o‘tkazildi. Shu aylananing markazi
$S$ ni aniqlang.
====
Uchburchakning uchlari \(M_{1} (- 3;6),\ M_{2} (9; - 10) \)
va \(M_{3} (-5;4) \) berilgan. Shu uchburchakka tashqi chizilgan
aylana markazi $C$ va radiusi $R$ ni aniqlang.
====
Uchburchakning uchlari
\(A (- 1; - 1),\ B (3;5),\ C (- 4;1) \) berilgan. $A$ uchi tashqi
burchak bissektrisasining, $BC$ tomonining davomi bilan kesishish
nuqtani toping.
====
Uchburchakning uchlari
\(A (3; - 5),\ B (1; - 3),\ C (2; - 2) \) berilgan. $B$ uchi tashqi
burchagi bessektrisa uzunligini aniqlang.
====
Ikki uchi \(A (2; - 3) \) va \(B (-5;1) \) nuqtalarda,
uchinchi uchi $C$ ordinata o‘qiga tegishli uchburchakning
medianalarining kesishish nuqtasi $M$ abssissa o‘qida yotadi.
$M$ va $C$ nuqtalarning koordinatalarini aniqlang.
====
Ikkita uchi \(A (2;1) \) va \(B (5; 3) \) nuqtalarida, va
diagonallarining kesishish nuqtasi ordinata o‘qiga tegishli
parallelogrammning yuzi \(S = 17\) ga teng. Qolgan ikki uchining
koordinatalarini aniqlang.
====
Ikkita uchi \(A (1; - 2),\ B (2;3) \) nuqtalarda joylashgan,
yuzi \(S = 8\) ga teng bo‘lgan uchburchakning uchinchi uchi
$C$ \(2x + y - 2 = 0\) to‘g‘ri chiziqqa tegishli. Shu $C$ uchining
koordinatasini aniqlang.
====
Ikki uchi \(A (2; - 3),\ B (3; - 2) \) nuqtalarda
joylashgan, yuzi \(S = 1,5\) ga teng bo‘lgan uchburchakning,
og‘irlik markazi \(3x - y - 8 = 0\) to‘g‘ri chiziqqa tegishli. Uchinchi $C$
uchining koordinatasini aniqlang.
====
\(N (2; 5) \) nuqtaning \(9x - 7y + 30 = 0\) to‘g‘ri chizig‘iga
nisbatan simmetrik nuqtani toping.
====
Uchburchakning \(A (- 3; - 2),\ B (5; - 4),\ C (- 1;3) \)
uchlaridan o‘tib, qarama-qarshi tomonga parallel to‘g‘ri chiziqlarning tenglamalarini
tuzing.
====
Uchburchak tomonlarining o‘rtalari
\(M (5;3),\ N (3; - 4),\ E (2;1) \) nuqtalarda joylashgan. Tomonlarning
tenglamalarni tuzing.
====
Ikki nuqta \(A (3; - 5) \) va \(B (- 2;3) \) berilgan.
$B$ nuqtadan o‘tib, $AB$ kesmaga perpendikular to‘g‘ri chiziq
tenglamasini tuzing.
====
Agarda \(M (4;5) \) nuqta, koordinata boshidan to‘g‘ri chiziqqa
o‘tkazilgan perpendikulyarning asosi bo‘lsa, shu to‘g‘ri chiziq tenglamasini
tuzing.
====
Uchburchakning uchlari
\(A (3;2),\ B (- 4;4),\ C (- 2; 5) \) koordinatalari bilan berilgan.
Balandliklarining tenglamasini tuzing.
====
Uchburchakning tomonlari \(x + 5y - 7 = 0\),
\(4x - y - 7 = 0\), \(x + 3y - 31 = 0\) tenglamalar bilan berilgan.
Balandliklarining kesishish nuqtasini toping.
====
Uchburchaklarning uchlari
\(A (1; - 1),\ B (- 2;1),\ C (3;5) \) nuqtalarda joylashgan. $A$
uchidan o‘tib, $B$ uchidan o‘tkazilgan medianaga
perpendikular to‘g‘ri chiziq tenglamasini tuzing.
====
Uchburchaklarning uchlari
\(A (2; - 2),\ B (3; - 5),\ C (5;7) \) nuqtalarda joylashgan. $C$
uchidan o‘tib, $A$ uchidan o‘tkazilgan bissektrisaga
perpendikular to‘g‘ri chiziqning tenglamasini tuzing.
====
Uchburchakning uchlari \(A (1;-2),\ B (5; 4) \) va
\(C (-2;0) \) nuqtalarda joylashgan. $A$ uchidagi ichki va tashqi
burchaklari bissektrisalarining tenglamalarini tuzing.
====
\(A (11; - 15) \) va \(B (-7;3) \) nuqtalardan
teng masofada va \(C (3; 5) \) nuqtadan o‘tuvchi to‘g‘ri chiziq tenglamasini
tuzing.
====
\(Q (5; - 6) \) nuqtaning, \(A (3;8) \) va \(B (7;5) \)
nuqtalardan o‘tgan to‘g‘ri chiziqdagi proyeksiyasini toping.
====
\(N (- 4; 7) \) nuqtaning, \(A (2;0) \) va \(B (- 3;5) \)
nuqtalardan o‘tgan to‘g‘ri chiziqqa nisbatan simmetrik nuqtani toping.
====
\(P (2; - 3) \) va \(Q (- 8; - 2) \) nuqtalardan
oraliqlarining yig‘indisi eng kichik bo‘lgan, abssissa o‘qida joylashgan
nuqtani toping.
====
\(P (2;5) \) va \(Q (- 3;2) \) nuqtalardan masofalarning
farqi eng katta bo‘lgan, ordinata o‘qida joylashgan nuqtani toping.
====
\(A (-5;5) \) va \(B (-7;1) \) nuqtalardan
masofalarining yig‘indisi eng kichik bo‘lgan \(2x - y - 5 = 0\) to‘g‘ri chiziqda
joylashgan nuqtani toping.
====
\(A (0;5) \) va \(B (5;2) \) nuqtalardan masofalarning
farqi eng katta bo‘lgan, \(3x - y - 2 = 0\) to‘g‘ri chiziqda joylashgan
nuqtani toping.
====
\(P (3;5) \) nuqtadan o‘tib, \(4x + 6y - 7 = 0\) to‘g‘ri chiziq
bilan \(45^{0}\) burchak yasab kesishuvchi to‘g‘ri chiziq tenglamasini tuzing.
====
\(A (4;5) \) nuqta, diagonali \(7x - y - 8 = 0\) tenglama
bilan berilgan kvadratning bir uchi. Shu kvadratning tomonlari va
ikkinchi diagonalining tenglamasini tuzing.
====
\(A (3;7) \) va \(C (6; 5) \) nuqtalar kvadratning
qarama-qarshi uchlari. Uning tomonlari tenglamasini tuzing.
====
Bir tomoni \(x-4y - 8 = 0\) to‘g‘ri chiziqda yotuvchi
kvadratning og‘irlik markazi \(M (1;1) \) nuqtada joylashgan. Shu kvadratning
qolgan tomonlari yotgan to‘g‘ri chiziqlarning tenglamalarini tuzing.
====
Uchburchakning ikki uchi \(A (6;4),\ B (- 10;2) \), va
balandliklarining kesishish nuqtasi \(N (5;2) \) berilgan. Uchinchi $C$
uchining koordinatalarini toping.
====
$ABC$ uchburchakning ikki uchi
\(A (6; - 2),\ B (10;14) \), va balandliklarining kesishish nuqtasi
\(N (4; - 1) \) berilgan. Bu uchburchakning tomonlari tenglamasini tuzing.
====
$ABC$ uchburchakda \(AB:5x-3y+2=0\)
tomonining, shuningdek \(AN:4x - 3y + 1 = 0,\ BN:7x + 2y - 22 = 0\)
balandliklarining tenglamalari berilgan. Shu uchburchakning qolgan ikkita
tomonining va uchinchi balandligining tenglamalarini tuzing.
====
$ABC$ uchburchakning bir uchi \(A (1;3) \) nuqtada,
va ikkita medianasi \(x - 2y + 1 = 0\,\ y - 1 = 0\) to‘g‘ri chiziqlarda
joylashgan. Tomonlarining tenglamalarini tuzing.
====
$ABC$ uchburchakning bir uchi \(B (- 4; - 5) \),
va ikki balandligining tenglamasi:
\(3x + 8y + 13 = 0\,\ 5x + 3y - 4 = 0\) berilgan. Tomonlarning
tenglamalarni tuzing.
====
$ABC$ uchburchakning bir uchi \(C (4; - 1) \), va
ikkita bissektrisasining tenglamasi: \(x - 1 = 0\,\ x - y - 1 = 0\)
berilgan. Tomonlarining tenglamalarini tuzing.
====
$ABC$ uchburchakning bir uchini \(B (2;6) \), va
bir uchidan o‘tkazilgan balandlikning: \(x - 7y + 15 = 0\), va
bissektrisasining: \(7x + y + 5 = 0\) tenglamalarini bilgan holda,
tomonlarining tenglamalarini tuzing.
====
\(\vec{a} + \vec{b} + \vec{c} = 0\) shartni qanoatlantiruvchi birlik \(\vec{a},\ \vec{b}\) va \(\vec{c}\) vektorlar berilgan. Hisoblang: \(\left(\vec{a},\vec{b} \right) + \left(\vec{b},\vec{c} \right) + \left(\vec{c},\vec{a} \right) \).
====
\(\vec{a} + \vec{b}\) vektor \(\vec{a} - \vec{b}\) vektorga perpendikulyar bo‘lishi uchun \(\vec{a}\) va \(\vec{b}\) vektorlar qanday shartlarni qanoatlantirishi kerak?
====
\(\vec{p} = \vec{b} (\vec{a},\vec{c}) - \vec{c} (\vec{a},\vec{b}) \) vektor \(\vec{a}\) vektorga perpendikulyar ekanini isbotlang.
====
\(\vec{p} = \vec{b} - \frac{\vec{a} (\vec{a},\vec{b}) }{{\vec{a}}^{2}}\) vektor \(\vec{a}\) vektorga perpendikulyar ekanini isbotlang.
====
\(ABC\) uchburchakning tomonlari bilan mos keluvchi \(\vec{AB} = \vec{b}\) va \(\vec{AC} = \vec{c}\) vektorlar berilgan. Bu uchburchakning \(B\) uchidan tushirilgan \(BD\) balandligining \(\vec{b},\ \vec{c}\) bazis bo‘yicha yoyilmasini toping.
====
\(\vec{a}+\vec{b}\) va \(\vec{a} - \vec{b}\) vektorlar kollinear bo‘lishi uchun \(\vec{a},\vec{b}\) vektorlar qanday shartni qanoatlantirishi kerak?
====
Ayniyatni isbotlang: \(\lbrack\vec{a},\vec{b}\rbrack^{2} + (\vec{a},\vec{b}) ^{2} = {\vec{a}}^{2}{\vec{b}}^{2}\).
====
\(\lbrack\vec{a},\vec{b}\rbrack^{2} < {\vec{a}}^{2}{\vec{b}}^{2}\) ekanini isbotlang; qanday holda bu yerda tenglik ishorasi bo‘ladi?
====
\(\vec{a},\ \vec{b}\) va \(\vec{c}\) vektorlar \(\vec{a} + \vec{b} + \vec{c} = 0\) shartni qanoatlantiradi. \(\lbrack\vec{a},\vec{b}\rbrack = \lbrack\vec{b},\vec{c}\rbrack = \lbrack\vec{c},\vec{a}\rbrack\) ekanini isbotlang.
====
Ayniyatni isbotlang: \((\lbrack\vec{a} + \vec{b},\vec{b} + \vec{c}\rbrack,\vec{c} + \vec{a}) = 2 (\lbrack\vec{a},\vec{b}\rbrack,\vec{c}) \).
====
Ayniyatni isbotlang: \((\lbrack\vec{a},\vec{b}\rbrack,\vec{c} + \lambda\vec{a} + \mu\vec{b}) = (\lbrack\vec{a},\vec{b}\rbrack,\vec{c}) \), bunda \(\lambda\) va \(\mu\) - ixtiyoriy sonlar.