$M_1 (1; -2) $, $M_2 (2; 1) $ nuqtalar berilgan.
Quyidagi kesmalarning koordinata o‘qlariga proyeksiyalarini toping: $\overline{M_1M_2}$ \\
====
Kvadratning ikkita qo‘shni uchlari $A (3; -7)$ va
$B (-1;4) $ berilgan. Uning yuzini hisoblang.
====
Kvadratning ikkita qarama-qarshi uchlari $P (3; 5) $ va
$Q (1; -3) $ berilgan. Uning yuzini hisoblang.
====
Ikkita uchi $A (-3; 2) $ va $B (1; 6) $ nuqtalarda
joylashgan muntazam uchburchakning yuzini hisoblang.
====
$ABCD$ parallelogrammning uchta uchi $A (3; -7) $,
$B (5; -7) $, $C (-2; 5) $ berilgan, to‘rtinchi uchi $D$,
$B$ uchiga qarama-qarshi. Shu parallelogrammning diagonallari
uzunliklarini aniqlang.
====
Berilgan $A (3; -5) $, $B (-2; -7)$ va
$C (18; 1) $ nuqtalar bir to‘g‘ri chiziqda yotishini isbotlang.
====
$A (2;2) $, $B (-1;6) $, $C (-5;3) $ va $D (-2;-1) $
nuqtalari kvadrat uchlari ekanini isbotlang.
====
Bir jinsli elementdan yasalgan qatorning uchlari
$A (3;-5) $ va $B (-1;1) $ nuqtalarda joylashgan. Uning og‘irligi
markazi koordinatasini aniqlang.
====
Bir jinsli elementdan yasalgan qatorning og‘irlik markazi
$M (1;4) $ nuqtada, bir uchi $P (-2;2) $ nuqtada joylashgan. Shu
qatorning ikkinchi uchi $Q$ ning koordinatalarini aniqlang.
====
Uchburchak uchlarining koordinatalari berilgan
$A (1;-3) $, $B (3;-5) $ va $C (-5;7) $. Tomonlarining o‘rtalarini
aniqlang.
====
$M (2;-1) $, $N (-1;4) $ va $P (-2;2) $ nuqtalar
uchburchak tomonlarining o‘rtalari. Uchlarining koordinatalarini
aniqlang.
====
Parallelogrammning uchlari
$A (3;-5) $, $B (5;-3) $, $C (-1;3) $ berilgan. $B$ tepasiga
qarama-qarshi joylashgan $D$ uchini aniqlang.
====
Parallelogrammning ikkita qo‘shni uchlari $A (-3;5) $, $B (1;7) $
va dioganallarining kesishish nuqtasi $M (1;1)$ berilgan. Qolgan ikki
cho‘qqisini aniqlang.
====
$ABCD$-parallelogrammning uchta uchi
$A (2;3) $, $B (4;-1) $ va $C (0;5) $ berilgan. To‘rtinchi $D$
cho‘qqisini toping.
====
Uchburchakning uchlari $A (1;4) $, $B (3;-9) $, $C (-5;2) $
berilgan. $B$ uchidan o‘tkazilgan mediana uzunligini aniqlang.
====
$A (1;-3) $ va $B (4;3) $ nuqtalarni tutashtiruvchi
kesma teng uch bo‘lakka bo‘lindi. Bo‘luvchi nuqtalarning koordinatalarini
aniqlang.
====
$A (4;2) $, $B (7;-2) $ va $C (1;6) $ nuqtalar bir jinsli
simdan yasalgan uchburchak uchlari. Shu uchburchakning og‘irligi
====
Uchlari $A (2;-3) $, $B (3;2) $ va $C (-2;5) $
nuqtalarida joylashgan uchburchaklarning yuzini hisoblang.
====
Uchlari $M_1 (-3;2) $, $M_2 (5;-2) $ va $M_3 (1;3) $
nuqtalarida joylashgan uchburchaklarning yuzini hisoblang.
====
Uchlari $M (3;-4) $, $N (-2;3) $ va $P (4;5) $
nuqtalarida joylashgan uchburchaklarning yuzini hisoblang.
====
Uch uchi $A (-2;3), \ B (4;-5) $ va
$C (-3;1)$ nuqtalarda joylashgan parallelogrammning yuzini aniqlang.
====
Bir jinsli to‘rtburchakli plastinkaning uchlari berilgan:
$A (2;1), \ B (5;3), \ C (-1;7) $ va $D (-7;5) $. Uning og‘irlik markazi
koordinatalarini aniqlang.
====
Bir jinsli beshburchakli plastinkaning uchlari berilgan:
$A (2;3), \ B (0;6), \ C (-1;5), \ D (0;1) $ va $E (1;1) $. Uning og‘irligi
markazi koordinatalarini aniqlang.
====
Ikkala uchi $A (3;1) $ va $B (1;-3) $ nuqtalarda, a
uchinchi $C$ uchi $Oy$ o‘qiga tegishli uchburchakning
yuzi $S=3$ ga teng. $C$ uchining koordinatalarini aniqlang.
====
Ikkala uchi $A (2;1) $ va $B (3;-2) $ nuqtalarda, va
uchinchi $C$ uchi $Ox$ o‘qiga tegishli bo‘lgan uchburchakning
yuzi $S=4$ ga teng. $C$ uchining koordinatalarini aniqlang.