Berilgan to‘g‘ri chiziqlarning kesishish nuqtasini toping:
$(3x-4y-29=0, 2x+5y+19=0)$.
====
$ABC$ uchburchakning tomonlari:
\(AB:4x+3y-5=0,\ BC:x-3y+10=0,\ AC:x-2=0\) 
tenglamalari bilan berilgan. Uchlarining koordinatalarini aniqlang.
====
Parallelogrammning ikki tomoni tenglamalari
\(8x+3y+1=0,\ 2x+y-1=0\) va bir diagonali tenglamasi
\(3x+2y+3=0\) berilgan. Parallelogramm uchlari koordinatalarini
aniqlang
====
Uchburchakning tomonlari \(x+5y-7=0\),
\(3x-2y-4=0\), \(7x+y+19=0\) to‘g‘ri chiziqlarda yotadi. Uning
yuzini hisoblang.
====
Umumiy tenglamasi \(2x-5y+4=0\) bo‘lgan to‘g‘ri
berilgan. \(M (-3,5) \) nuqtadan o‘tib, berilgan to‘g‘ri chiziqqa: a) parallel;
b) perpendikular bo‘lgan to‘g‘ri chiziqlar tenglamasini tuzing.
====
To‘g‘ri to‘rtburchakning bir uchi \(A (2;-3) \), va ikkita tarafining
ning tenglamalari \(2x+3y+9=0,\ 3x-2y-7=0\)
berilgan. Qolgan ikki tomonning tenglamalarini tuzing.
====
\(N (5;8) \) nuqtaning, \(5x-11y-43=0\) to‘g‘ri chizig‘idagi
proyeksiyasini toping.
====
Quyidagi har bir to‘g‘ri chiziqlar jufti uchun, ularga parallel
bo‘lib, aynan o‘rtasidan o‘tuvchi to‘g‘ri tenglamani tuzing: $3x-2y-3=0$, $3x-2y-17=0$.
====
Berilgan ikki nuqtadan o‘tuvchi to‘g‘ri chiziqning burchagi
koeffitsiyenti $k$ ni hisoblang: $A (-4;3) $, $B (1;8) $.
====
Uchburchak uchlari \(A (1;0),\ B (5;-2),\ C (3;2) \)
koordinatalari bilan berilgan. Uchburchaklar tomonlarining va
medianalarining tenglamalarini tuzing.
====
\(P (3;8) \) va \(Q (-1;-6) \) nuqtalardan o‘tgan
to‘g‘ri chiziqning koordinata o‘qlari bilan kesishish nuqtalarini toping.
====
Doiraviy to‘rtburchakning uchlari
\(A (-2;-6),\ B (7;6),\ C (3;9) \) va \(D (-3;1) \) nuqtalarda
joylashgan. Diagonallarining kesishish nuqtasi topilsin.
====
$ABCD$ parallelogrammning ikkita qo‘shni uchlari
\(A (3,3),\ B (-1;7) \) va diagonallarining kesishish nuqtasi
\(E (2;-4) \) berilgan. Shu parallelogramm tomonlarining tenglamalarini
tuzing.
====
To‘g‘ri to‘rtburchakning ikki tomoni
\(5x+2y-7=0,\ 5x+2y-36=0\) va diagonali
\(3x+7y-10=0\) tenglamalar bilan berilgan. Qolgan ikki tomoni
tenglamalarni tuzing.
====
Berilgan to‘g‘ri chiziqlar orasidagi burchakni aniqlang: $3x+2y+4=0, 5x-y+1=0$.
====
Qirralari
\(7x+y+31=0,\ 3x+4y-1=0,\ x-7y-17=0\) tenglamalar
bilan berilgan uchburchakning teng yonli ekanini isbotlang.
Masalani uchburchakning
burchaklarini topish orqali yeching.
====
\(N (4;-5) \) nuqtadan o‘tib, $2x+5y-7=0$
to‘g‘ri chiziqlariga parallel to‘g‘ri chiziqlarning tenglamasini tuzing. Masalani burchaklik
koeffitsiyentni hisoblamasdan yeching.
====
Quyida berilgan to‘g‘ri chiziqlar juftlarining qaysilari
perpendikular ekanini aniqlang: $4x+y+6=0, 2x-8y-13=0$.
====
Ikki to‘g‘ri chiziqning chetidagi burchakni toping: $2x+y-9=0, 3x-y+11=0$.
====
Parallel to‘g‘ri chiziqlar orasidagi masofani hisoblang: $5x-12y+13=0, 5x-12y-26=0$.
====
Kvadratning ikki tomoni
\(5x-12y+65=0,\ 5x-12y-26=0\) to‘g‘ri chiziqlarda
yotishini bilgan holda, yuzini hisoblang.
====
\(P (2;7) \) nuqtadan o‘tib, \(Q (1;2) \) nuqtagacha
masofasi 5 ga teng bo‘lgan to‘g‘ri chiziqlarning tenglamasini tuzing.
====
\(M (7;-2) \) nuqtadan o‘tib, \(N (4;-6) \) nuqtaga
gacha bo‘lgan masofasi 5 ga teng bo‘lgan to‘g‘ri chiziqlarning tenglamasini tuzing.
====
\(A (4;-5) \) nuqtadan o‘tib, \(B (-2;3) \) nuqtaga
gacha masofasi 12 ga teng bo‘lgan to‘g‘ri chiziqlarning tenglamasini tuzing.
====
Berilgan \(8x-15y-25=0\) to‘g‘ri chiziqdan og‘ishi -2 ga teng
teng bo‘lgan nuqtalarning geometrik o‘rni tenglamasini tuzing.
====
Berilgan \(3x-4y-10=0\) to‘g‘ri chiziqqa parallel va undan
$d=3$ masofada yotuvchi to‘g‘ri chiziqlarning tenglamasini tuzing.
====
Berilgan parallel to‘g‘ri chiziqlardan teng masofada yotuvchi
nuqtalarning geometrik o‘rni tenglamasini tuzing: $2x+y+7=0, 2x+y-3=0$.
====
\(P (1;-2) \) nuqta va koordinatalar boshi, berilgan ikkita
to‘g‘ri yozing: $12x-5y-7=0, 3x+4y-8=0$.
kesishishidan hosil bo‘lgan bir xil burchakdami, qo‘shni burchakdami yoki vertikal
burchaklarda yotadimi?
====
\(P (2;3) \) va \(Q (5;-1) \) nuqtalar, berilgan ikkita
to‘g‘ri: $12x-y-7=0,\ 13x+4y-5=0$.
kesishishidan hosil bo‘lgan bir xil burchakdami, qo‘shni burchakdami yoki vertikal
burchaklarda yotadimi?
====
Koordinata boshi, tomonlarining tenglamalari
\(8x+3y+31=0,\ x+8y-19=0,\ 7x-5y-11=0\) bilan
berilgan uchburchakning tashqarisida yoki ichida yotishini aniqlang.
====
\(P (-3;2) \) nuqta, tomonlarining tenglamalari
\(x+y-4=0,\ 3x-7y+8=0,\ 4x-y-31=0\) bilan
berilgan uchburchakning tashqarisida yoki ichida yotishini aniqlang.
====
Koordinata boshi, berilgan to‘g‘ri chiziqlarning:
\(3x+y-4=0\) va \(3x-2y+6=0\) kesishmasida hosil bo‘ladi
bo‘lgan o‘tkir yoki o‘tmas burchakka tegishli bo‘lishini aniqlang.
====
\(M (2;-5) \) nuqta, berilgan to‘g‘ri chiziqlarning:
\(3x+5y-4=0\) va \(x-2y+3=0\) kesishmasida hosil bo‘ladi
bo‘lgan o‘tkir yoki o‘tmas burchakka tegishli bo‘lishini aniqlang.
====
\(4x+3y-1=0\) va \(3x-2y+5=0\)
to‘g‘ri chiziqlarning kesishish nuqtasidan o‘tib (bu nuqtani aniqlamay), ordinata
o‘qidan \(b=4\) kesmani kesib oladigan to‘g‘ri chiziq tenglamasini tuzing.
====
\(2x+y-2=0\) va \(x-5y-3=0\)
to‘g‘ri chiziqlarning kesishish nuqtasidan o‘tib (bu nuqtani aniqlamay), uchlari
\(A (-1;-4) \) va \(B (5;-6) \) nuqtalarda joylashgan kesmaning
to‘g‘ri o‘rtasidan o‘tuvchi to‘g‘ri chiziqning tenglamasini tuzing.
====
Uchlari \(A (4;-4),\ B (6;-1) \) va \(C (-1;2) \)
nuqtalarida joylashgan bir jinsli plastinkadan yasalgan uchburchakning
og‘irlik markazidan o‘tib, quyida berilgan
\(\alpha (2x+3y-1) +\beta (3x-4y-3) =0\) to‘g‘ri chiziqlar dasturiga
tegishli to‘g‘ri chiziqning tenglamasini tuzing.