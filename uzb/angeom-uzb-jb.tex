\documentclass{article}
\usepackage[fontsize=12pt]{fontsize}
\usepackage[utf8]{inputenc}
\usepackage[T2A]{fontenc}
% \usepackage{unicode-math}

\usepackage{array}
\usepackage[a4paper,
left=7mm,
right=5mm,
top=7mm,]{geometry}
\usepackage{amsmath}
\usepackage{amsfonts}
\usepackage{setspace}



\renewcommand{\baselinestretch}{1} 

\everymath{\displaystyle}
\everydisplay{\displaystyle}
% \linespread{1.25}

\DeclareMathOperator{\sign}{sign}


\begin{document}

\pagenumbering{gobble}


\begin{tabular}{m{17cm}}
\textbf{1-variant}
\newline

T1. 
Vektorning koordinatalari.
 \\
T2. 
Tekislikning tenglamalari. Tekisliklarning o‘zaro joylashishi.
 \\
A1. 
$A (1;-3) $ va $B (4;3) $ nuqtalarni tutashtiruvchi
kesma teng uch bo‘lakka bo‘lindi. Bo‘luvchi nuqtalarning koordinatalarini
aniqlang.
 \\
A2. 
$B (-5;5)$ nuqtadan o‘tib, koordinata burchagidan
yuzi 50 ga teng uchburchak kesib oladigan to‘g‘ri chiziqlarning tenglamasini
tuzing.
 \\
A3. 
Vektor koordinata o‘qlari bilan quyidagi burchaklarni hosil qila oladimi:
$\alpha = 45^{{^\circ}},\ \ \ \ \beta = 135^{{^\circ}},\ \gamma = 60^{{^\circ}}$.
 \\
B1. 
Uchlari \(M_{1} (1;1), M_{2} (0,2) \) va
\(M_{3} (2;-1) \) nuqtalarda joylashgan uchburchakning ichki 
burchaklari orasida o‘tmas burchak bor yoki yo‘qligini aniqlang.
 \\
B2. 
\(M (7;-2) \) nuqtadan o‘tib, \(N (4;-6) \) nuqtaga
gacha bo‘lgan masofasi 5 ga teng bo‘lgan to‘g‘ri chiziqlarning tenglamasini tuzing.
 \\
B3. 
$\vec{a}$ va $\vec{b}$ vektorlar $\varphi = 2\pi/3$ burchak hosil qiladi. $|\vec{a}| = 1,|\vec{b}| = 2$ ekanini bilib, quyidagilarni hisoblang:
$\lbrack 2\overrightarrow{a} + \overrightarrow{b},\overrightarrow{a} + 2\overrightarrow{b}\rbrack^{2}$.
 \\
C1. 
Ikki uchi \(A (2; - 3) \) va \(B (-5;1) \) nuqtalarda,
uchinchi uchi $C$ ordinata o‘qiga tegishli uchburchakning
medianalarining kesishish nuqtasi $M$ abssissa o‘qida yotadi.
$M$ va $C$ nuqtalarning koordinatalarini aniqlang.
 \\
C2. 
Uchburchakning ikki uchi \(A (6;4),\ B (- 10;2) \), va
balandliklarining kesishish nuqtasi \(N (5;2) \) berilgan. Uchinchi $C$
uchining koordinatalarini toping.
 \\
C3. 
\(\vec{p} = \vec{b} (\vec{a},\vec{c}) - \vec{c} (\vec{a},\vec{b}) \) vektor \(\vec{a}\) vektorga perpendikulyar ekanini isbotlang.
 \\

\end{tabular}
\vspace{1cm}


\begin{tabular}{m{17cm}}
\textbf{2-variant}
\newline

T1. 
Vektorlarning skalyar ko‘paytmasi.
 \\
T2. 
Tekislikdagi to‘g‘ri chiziqlarning o‘zaro joylashishi.
 \\
A1. 
Berilgan $A (3; -5) $, $B (-2; -7)$ va
$C (18; 1) $ nuqtalar bir to‘g‘ri chiziqda yotishini isbotlang.
 \\
A2. 
$2x-y+2=0$, $4x-2y+4=0$, $6x-3y+6=0$
to‘g‘ri chiziqlar bir nuqtada kesishishadimi?
 \\
A3. 
Tekislikda ikkita vektor
$\overrightarrow{p} = \{ 2; - 3\}$, $\overrightarrow{q} = \{ 1;2\}$.
$\overrightarrow{a} = \{9;4\}$ vektorning
$\overrightarrow{p},\ \overrightarrow{q}$ bazis bo‘yicha yoyilmasi topilsin.
 \\
B1. 
Uchlari $A_1 (1; 1), A_2 (2; 3) $ va $A (5;-1) $
nuqtalarida joylashgan uchburchakning to‘g‘ri burchakli ekanini isbotlang.
 \\
B2. 
\(P (-3;2) \) nuqta, tomonlarining tenglamalari
\(x+y-4=0,\ 3x-7y+8=0,\ 4x-y-31=0\) bilan
berilgan uchburchakning tashqarisida yoki ichida yotishini aniqlang.
 \\
B3. 
$\vec{a}$ va $\vec{b}$ vektorlar $\varphi = 2\pi/3$ burchak hosil qiladi. $|\vec{a}| = 3,|\vec{b}| = 4$ ekani ma’lum. Hisoblang:
${\vec{a}}^{2}$.
 \\
C1. 
\(M{1} (1; 2) \) nuqta orqali, radiusi 5 ga teng,
$Ox$ o‘qiga urinma aylana o‘tkazildi. Shu aylananing markazi
$S$ ni aniqlang.
 \\
C2. 
$ABC$ uchburchakning bir uchi \(B (- 4; - 5) \),
va ikki balandligining tenglamasi:
\(3x + 8y + 13 = 0\,\ 5x + 3y - 4 = 0\) berilgan. Tomonlarning
tenglamalarni tuzing.
 \\
C3. 
Ayniyatni isbotlang: \((\lbrack\vec{a} + \vec{b},\vec{b} + \vec{c}\rbrack,\vec{c} + \vec{a}) = 2 (\lbrack\vec{a},\vec{b}\rbrack,\vec{c}) \).
 \\

\end{tabular}
\vspace{1cm}


\begin{tabular}{m{17cm}}
\textbf{3-variant}
\newline

T1. 
Vektorlarning vektor ko‘paytmasi va aralash ko‘paytmasi.
 \\
T2. 
Nuqtadan tekislikkacha, fazoda nuqtadan to‘g‘ri chiziqqacha va ayqash to‘g‘ri chiziqlar orasidagi masofa. \\
A1. 
Uchlari $M_1 (-3;2) $, $M_2 (5;-2) $ va $M_3 (1;3) $
nuqtalarida joylashgan uchburchaklarning yuzini hisoblang.
 \\
A2. 
$A (3;-2) $ nuqtadan $3x+4y-15=0$ to‘g‘ri chiziqqa
gacha siljishni va masofani hisoblang. \\
A3. 
$\overrightarrow{a} = \{ 2; - 4;4\}$ va $\overrightarrow{b} = \{ - 3;2;6\}$
vektorlar hosil qilgan burchak kosinusini hisoblang.
 \\
B1. 
Bir to‘g‘ri chiziqqa tegishli \(A (1;-1),\ B (3;3) \) va
\(C (4;5) \) nuqtalar berilgan. Har bir nuqtaning, qolgan ikki nuqta orqali aniqlanuvchi kesmani bo‘lish nisbati $\lambda$ ni aniqlang.
 \\
B2. 
\(P (2;3) \) va \(Q (5;-1) \) nuqtalar, berilgan ikkita
to‘g‘ri: $12x-y-7=0,\ 13x+4y-5=0$.
kesishishidan hosil bo‘lgan bir xil burchakdami, qo‘shni burchakdami yoki vertikal
burchaklarda yotadimi?
 \\
B3. 
$\vec{a}$ va $\vec{b}$ vektorlar o‘zaro perpendikulyar. $|\vec{a}| = 3,|\vec{b}| = 4$ ekani ma’lum, quyidagilarni hisoblang:
$|\lbrack 3\vec{a} - \vec{b},\vec{a}-2\vec{b}\rbrack|$.
 \\
C1. 
Uchburchakning uchlari
\(A (3; - 5),\ B (1; - 3),\ C (2; - 2) \) berilgan. $B$ uchi tashqi
burchagi bessektrisa uzunligini aniqlang.
 \\
C2. 
\(A (4;5) \) nuqta, diagonali \(7x - y - 8 = 0\) tenglama
bilan berilgan kvadratning bir uchi. Shu kvadratning tomonlari va
ikkinchi diagonalining tenglamasini tuzing.
 \\
C3. 
Ayniyatni isbotlang: \((\lbrack\vec{a},\vec{b}\rbrack,\vec{c} + \lambda\vec{a} + \mu\vec{b}) = (\lbrack\vec{a},\vec{b}\rbrack,\vec{c}) \), bunda \(\lambda\) va \(\mu\) - ixtiyoriy sonlar. \\

\end{tabular}
\vspace{1cm}


\begin{tabular}{m{17cm}}
\textbf{4-variant}
\newline

T1. 
Koordinatalari bilan berilgan vektorlarning skalyar, vektor va aralash ko‘paytmalari. \\
T2. 
Tekislik va to‘g‘ri chiziqlarning o‘zaro joylashishi.
 \\
A1. 
$ABCD$ parallelogrammning uchta uchi $A (3; -7) $,
$B (5; -7) $, $C (-2; 5) $ berilgan, to‘rtinchi uchi $D$,
$B$ uchiga qarama-qarshi. Shu parallelogrammning diagonallari
uzunliklarini aniqlang.
 \\
A2. 
$5x-y+3=0$ to‘g‘ri chiziqning $k$ burchagi
koeffitsiyentini va $Oy$ o‘qidan kesib olgan kesmaning algebraik
qiymati $b$ ni aniqlang.
 \\
A3. 
$\alpha$
qanday qiymatlarida 
$\overrightarrow{a} = \alpha\overrightarrow{i} - 3\overrightarrow{j} + 2\overrightarrow{k}$
va
$\overrightarrow{b} = \overrightarrow{i} + 2\overrightarrow{j} - \alpha\overrightarrow{k}$
vektorlar o‘zaro perpendikulyar bo‘lishini aniqlang.
 \\
B1. 
Uchburchakning uchlari \(A (2;-5),\ B (1;-2),\ C (4;7) \)
berilgan. $AC$ tomoni bilan $B$ uchining ichki burchagi
bissektrisasining kesishish nuqtasini toping.
 \\
B2. 
\(N (5;8) \) nuqtaning, \(5x-11y-43=0\) to‘g‘ri chizig‘idagi
proyeksiyasini toping.
 \\
B3. 
$A (2; -1;2),B (1;2; 1) $ va $C (3;2;1)$ nuqtalar berilgan. Quyidagi vektor ko‘paytmalarning koordinatalarini toping:
$\lbrack\overline{AB},\overline{BC}\rbrack$.
 \\
C1. 
Uchburchakning uchlari
\(A (- 1; - 1),\ B (3;5),\ C (- 4;1) \) berilgan. $A$ uchi tashqi
burchak bissektrisasining, $BC$ tomonining davomi bilan kesishish
nuqtani toping.
 \\
C2. 
Uchburchak tomonlarining o‘rtalari
\(M (5;3),\ N (3; - 4),\ E (2;1) \) nuqtalarda joylashgan. Tomonlarning
tenglamalarni tuzing.
 \\
C3. 
\(\vec{a},\ \vec{b}\) va \(\vec{c}\) vektorlar \(\vec{a} + \vec{b} + \vec{c} = 0\) shartni qanoatlantiradi. \(\lbrack\vec{a},\vec{b}\rbrack = \lbrack\vec{b},\vec{c}\rbrack = \lbrack\vec{c},\vec{a}\rbrack\) ekanini isbotlang.
 \\

\end{tabular}
\vspace{1cm}


\begin{tabular}{m{17cm}}
\textbf{5-variant}
\newline

T1. 
Chiziqli bog‘liq va chiziqli bog‘lanmagan vektorlar.
 \\
T2. 
Tekislikda to‘g‘ri chiziqning tenglamalari.
 \\
A1. 
Ikkala uchi $A (3;1) $ va $B (1;-3) $ nuqtalarda, a
uchinchi $C$ uchi $Oy$ o‘qiga tegishli uchburchakning
yuzi $S=3$ ga teng. $C$ uchining koordinatalarini aniqlang.
 \\
A2. 
$a$ va $b$ parametrlarining qanday qiymatlarida
$ax-2y-1=0$, $6x-4y-b=0$ to‘g‘ri chiziqlar kesishadimi?
 \\
A3. 
Berilgan: $\overrightarrow{a}| = 10,|\overrightarrow{b}| = 2$ va
$\left(\overrightarrow{a},\overrightarrow{b} \right) = 12$. Hisoblang
$\left| \left\lbrack \overrightarrow{a},\overrightarrow{b} \right\rbrack \right|$.
 \\
B1. 
\(P (2;2) \) va \(Q (1;5) \) nuqtalar bilan teng uchta
bo‘lingan kesmaning uchlari $A$ va $B$ nuqtalarning
koordinatalarini aniqlang.
 \\
B2. 
Quyida berilgan to‘g‘ri chiziqlar juftlarining qaysilari
perpendikular ekanini aniqlang: $4x+y+6=0, 2x-8y-13=0$.
 \\
B3. 
$\vec{a}$ va $\vec{b}$ vektorlar o‘zaro perpendikulyar; $\vec{c}$ vektor ular bilan $\pi/3$ ga teng bo‘lgan burchaklar hosil qiladi; $|\vec{a}| = 3$, $|\vec{b}| = 5,\ |\vec{c}| = 8$ ekani ma’lum, quyidagilarni hisoblang:
$\left(3\vec{a} - 2\vec{b},\vec{b} + 3\vec{c} \right) $.
 \\
C1. 
Ikkita uchi \(A (2;1) \) va \(B (5; 3) \) nuqtalarida, va
diagonallarining kesishish nuqtasi ordinata o‘qiga tegishli
parallelogrammning yuzi \(S = 17\) ga teng. Qolgan ikki uchining
koordinatalarini aniqlang. \\
C2. 
Uchburchaklarning uchlari
\(A (1; - 1),\ B (- 2;1),\ C (3;5) \) nuqtalarda joylashgan. $A$
uchidan o‘tib, $B$ uchidan o‘tkazilgan medianaga
perpendikular to‘g‘ri chiziq tenglamasini tuzing.
 \\
C3. 
\(\lbrack\vec{a},\vec{b}\rbrack^{2} < {\vec{a}}^{2}{\vec{b}}^{2}\) ekanini isbotlang; qanday holda bu yerda tenglik ishorasi bo‘ladi?
 \\

\end{tabular}
\vspace{1cm}


\begin{tabular}{m{17cm}}
\textbf{6-variant}
\newline

T1. Analitik geometriya fanining predmeti va metodlari.
 \\
T2. Tekislikda va fazoda dekart koordinatalar sistemasini almashtirish.
 \\
A1. 
Parallelogrammning ikkita qo‘shni uchlari $A (-3;5) $, $B (1;7) $
va dioganallarining kesishish nuqtasi $M (1;1)$ berilgan. Qolgan ikki
cho‘qqisini aniqlang.
 \\
A2. 
Umumiy tenglama bilan berilgan to‘g‘ri chiziqlarning
o‘zaro joylashuvini aniqlang, agar kesishadigan bo‘lsa kesishish nuqtasini
toping: $12x+59y-19=0, 8x+33y-19=0$.
 \\
A3. 
Uchlari $A (1;2;1), B (3;-1;7) $ va $C (7;4;-2) $ bo‘lgan uchburchakning
ichki burchaklarini hisoblab toping. Bu uchburchakning teng yonli ekanligini isbotlang.
 \\
B1. 
Abssissa o‘qida shunday $M$ nuqtani topingki,
\(N (2;-3) \) nuqtadan uzoqligi 5 ga teng bo‘lgan.
 \\
B2. 
\(A (4;-5) \) nuqtadan o‘tib, \(B (-2;3) \) nuqtaga
gacha masofasi 12 ga teng bo‘lgan to‘g‘ri chiziqlarning tenglamasini tuzing.
 \\
B3. 
$\vec{a}$ va $\vec{b}$ vektorlar $\varphi = 2\pi/3$ burchak hosil qiladi. $|\vec{a}| = 3,|\vec{b}| = 4$ ekani ma’lum. Hisoblang:
$ (\vec{a} - \vec{b}) ^{2};$ 7) $ (3\vec{a} + 2\vec{b}) ^{2}$.
 \\
C1. \(A (4;2) \) nuqta orqali, ikkita koordinata o‘qlariga
urinma doira o‘tkazildi. Uning markazi $C$ ni va radiusi
$R$ ni toping.
 \\
C2. 
\(P (2;5) \) va \(Q (- 3;2) \) nuqtalardan masofalarning
farqi eng katta bo‘lgan, ordinata o‘qida joylashgan nuqtani toping.
 \\
C3. 
Ayniyatni isbotlang: \(\lbrack\vec{a},\vec{b}\rbrack^{2} + (\vec{a},\vec{b}) ^{2} = {\vec{a}}^{2}{\vec{b}}^{2}\).
 \\

\end{tabular}
\vspace{1cm}


\begin{tabular}{m{17cm}}
\textbf{7-variant}
\newline

T1. 
Vektor tushunchasi. Vektorlar ustida chiziqli amallar.
 \\
T2. 
Nuqtadan to‘g‘ri chiziqqacha bo‘lgan masofa. To‘g‘rilar dastasi.
 \\
A1. 
Bir jinsli beshburchakli plastinkaning uchlari berilgan:
$A (2;3), \ B (0;6), \ C (-1;5), \ D (0;1) $ va $E (1;1) $. Uning og‘irligi
markazi koordinatalarini aniqlang.
 \\
A2. 
Umumiy tenglama bilan berilgan to‘g‘ri chiziqlarning
o‘zaro joylashuvini aniqlang, agar kesishadigan bo‘lsa kesishish nuqtasini
toping: $4x-7=0, 3x+8=0$.
 \\
A3. 
Uchburchakning uchlari
$A (- 1; - 2;4) $, $B (- 4; - 2;0) $ va $C (3; -2;1) $. Uning $B$ uchidagi
ichki burchakni aniqlang.
 \\
B1. 
Uchburchakning uchlari
\(A (3;-5),\ B (-3;3),\ C (-1;-2) \) berilgan. $A$ uchining ichki qismi
burchakli bessektrisaning uzunligini aniqlang.
 \\
B2. 
\(P (3;8) \) va \(Q (-1;-6) \) nuqtalardan o‘tgan
to‘g‘ri chiziqning koordinata o‘qlari bilan kesishish nuqtalarini toping.
 \\
B3. 
$\vec{a} = \{ 3; - 1; - 2\}$ va $\vec{b} = \{ 1;2; - 1\}$ vektorlar berilgan. Quyidagi vektor ko‘paytmalarning koordinatalarini toping:
$\left\lbrack 2\vec{a} - \vec{b},2\vec{a} + \vec{b} \right\rbrack$.
 \\
C1. 
Uchburchakning uchlari \(M_{1} (- 3;6),\ M_{2} (9; - 10) \)
va \(M_{3} (-5;4) \) berilgan. Shu uchburchakka tashqi chizilgan
aylana markazi $C$ va radiusi $R$ ni aniqlang.
 \\
C2. 
$ABC$ uchburchakning bir uchi \(C (4; - 1) \), va
ikkita bissektrisasining tenglamasi: \(x - 1 = 0\,\ x - y - 1 = 0\)
berilgan. Tomonlarining tenglamalarini tuzing.
 \\
C3. 
\(\vec{a} + \vec{b}\) vektor \(\vec{a} - \vec{b}\) vektorga perpendikulyar bo‘lishi uchun \(\vec{a}\) va \(\vec{b}\) vektorlar qanday shartlarni qanoatlantirishi kerak?
 \\

\end{tabular}
\vspace{1cm}


\begin{tabular}{m{17cm}}
\textbf{8-variant}
\newline

T1. 
Vektorlarning vektor ko‘paytmasi va aralash ko‘paytmasi.
 \\
T2. 
Fazoviy to‘g‘ri chiziqning tenglamalari. To‘g‘ri chiziqlarning o‘zaro joylashishi.
 \\
A1. 
Uchlari $M (3;-4) $, $N (-2;3) $ va $P (4;5) $
nuqtalarida joylashgan uchburchaklarning yuzini hisoblang.
 \\
A2. Berilgan $M_1 (3; 1) $, $M_2 (2; 3) $, $M_3 (6; 3) $,
$M_4 (-3;-3) $. $M_5 (3;-1) $, $M_6 (-2; 1) $ nuqtalarning qaysilari
$2x-3y-3 = 0$ to‘g‘ri chiziqqa tegishli va qaysilari tegishli
emas.
 \\
A3. Vektor koordinata o‘qlari bilan quyidagi burchaklarni hosil qila oladimi:
$\alpha = 45^{{^\circ}},\beta = 60^{{^\circ}},\gamma = 120^{{^\circ}}$.
 \\
B1. Ikkita qarama-qarshi uchlari \(P (4;9) \) va \(Q (-2; 1) \) nuqtalarida joylashgan romning tomon uzunligi \(5\sqrt{10}\). Shu
romba yuzini hisoblang.
 \\
B2. 
Berilgan parallel to‘g‘ri chiziqlardan teng masofada yotuvchi
nuqtalarning geometrik o‘rni tenglamasini tuzing: $2x+y+7=0, 2x+y-3=0$.
 \\
B3. 
$A (2; -1;2),B (1;2; 1) $ va $C (3;2;1) $ nuqtalar berilgan. Quyidagi vektor ko‘paytmalarning koordinatalarini toping:
$\lbrack\overline{BC} - 2\overline{CA},\overline{CB}\rbrack$. \\
C1. \(A (4;2) \) nuqta orqali, ikkita koordinata o‘qlariga
urinma doira o‘tkazildi. Uning markazi $C$ ni va radiusi
$R$ ni toping.
 \\
C2. 
$ABC$ uchburchakda \(AB:5x-3y+2=0\)
tomonining, shuningdek \(AN:4x - 3y + 1 = 0,\ BN:7x + 2y - 22 = 0\)
balandliklarining tenglamalari berilgan. Shu uchburchakning qolgan ikkita
tomonining va uchinchi balandligining tenglamalarini tuzing.
 \\
C3. 
\(\vec{a}+\vec{b}\) va \(\vec{a} - \vec{b}\) vektorlar kollinear bo‘lishi uchun \(\vec{a},\vec{b}\) vektorlar qanday shartni qanoatlantirishi kerak?
 \\

\end{tabular}
\vspace{1cm}


\begin{tabular}{m{17cm}}
\textbf{9-variant}
\newline

T1. 
Vektor tushunchasi. Vektorlar ustida chiziqli amallar.
 \\
T2. 
Tekislikda to‘g‘ri chiziqning tenglamalari.
 \\
A1. $M_1 (1; -2) $, $M_2 (2; 1) $ nuqtalar berilgan.
Quyidagi kesmalarning koordinata o‘qlariga proyeksiyalarini toping: $\overline{M_1M_2}$ \\
 \\
A2. 
Umumiy tenglama bilan berilgan to‘g‘ri chiziqlarning
o‘zaro joylashuvini aniqlang, agar kesishadigan bo‘lsa kesishish nuqtasini
toping: $x\sqrt{2}+12=0, 4x+24\sqrt{2}=0$.
 \\
A3. 
Vektor koordinata o‘qlari bilan quyidagi burchaklarni hosil qilishi
mumkinmi: $\alpha = 90^{{^\circ}},\ \beta = 150^{{^\circ}}$,
$\gamma = 60^{{^\circ}}?$
 \\
B1. 
Uchburchakning uchlari \(A (5;0),\ B (0;1) \) va \(C (3;3) \)
nuqtalarida. Uning ichki burchaklarini toping.
 \\
B2. 
Uchburchakning tomonlari \(x+5y-7=0\),
\(3x-2y-4=0\), \(7x+y+19=0\) to‘g‘ri chiziqlarda yotadi. Uning
yuzini hisoblang.
 \\
B3. 
$\vec{a} = \{ 3; - 1; - 2\}$ va $\vec{b} = \{ 1;2; - 1\}$ vektorlar berilgan. Quyidagi vektor ko‘paytmalarning koordinatalarini toping:
$\left\lbrack \vec{a},\vec{b} \right\rbrack$.
 \\
C1. 
Uchburchakning uchlari
\(A (3; - 5),\ B (1; - 3),\ C (2; - 2) \) berilgan. $B$ uchi tashqi
burchagi bessektrisa uzunligini aniqlang.
 \\
C2. 
Uchburchakning uchlari
\(A (3;2),\ B (- 4;4),\ C (- 2; 5) \) koordinatalari bilan berilgan.
Balandliklarining tenglamasini tuzing.
 \\
C3. \(\vec{a} + \vec{b} + \vec{c} = 0\) shartni qanoatlantiruvchi birlik \(\vec{a},\ \vec{b}\) va \(\vec{c}\) vektorlar berilgan. Hisoblang: \(\left(\vec{a},\vec{b} \right) + \left(\vec{b},\vec{c} \right) + \left(\vec{c},\vec{a} \right) \).
 \\

\end{tabular}
\vspace{1cm}


\begin{tabular}{m{17cm}}
\textbf{10-variant}
\newline

T1. 
Vektorlarning skalyar ko‘paytmasi.
 \\
T2. 
Fazoviy to‘g‘ri chiziqning tenglamalari. To‘g‘ri chiziqlarning o‘zaro joylashishi.
 \\
A1. 
$ABCD$-parallelogrammning uchta uchi
$A (2;3) $, $B (4;-1) $ va $C (0;5) $ berilgan. To‘rtinchi $D$
cho‘qqisini toping.
 \\
A2. 
$2x+3y-6=0$ to‘g‘ri chiziqning $k$ burchagi
koeffitsiyentini va $Oy$ o‘qidan kesib olgan kesmaning algebraik
qiymati $b$ ni aniqlang.
 \\
A3. 
$\overrightarrow{a}$ va $\overrightarrow{b}$ vektorlar
$\varphi = \pi/6$ burchak hosil qiladi.
$|\overrightarrow{a}| = 6,|\overrightarrow{b}| = 5$ ekanini bilib,
$\left| \left\lbrack \overrightarrow{a},\overrightarrow{b} \right\rbrack \right|$ kattalikni hisoblang.
 \\
B1. 
\(M_{1} (1;2) \) nuqtaga, \(A (1;0) \) va \(B (-1;-2) \)
nuqtalaridan o‘tuvchi to‘g‘ri chiziqqa nisbatan simmetrik bo‘lgan \(M_{2}\) nuqtaning koordinatalarini toping.
 \\
B2. 
Kvadratning ikki tomoni
\(5x-12y+65=0,\ 5x-12y-26=0\) to‘g‘ri chiziqlarda
yotishini bilgan holda, yuzini hisoblang.
 \\
B3. 
$\vec{a}$ va $\vec{b}$ vektorlar $\varphi = 2\pi/3$ burchak hosil qiladi. $|\vec{a}| = 1,|\vec{b}| = 2$ ekanini bilib, quyidagilarni hisoblang:
$\lbrack\vec{a},\vec{b}\rbrack^{2}$.
 \\
C1. 
Uchburchakning uchlari \(M_{1} (- 3;6),\ M_{2} (9; - 10) \)
va \(M_{3} (-5;4) \) berilgan. Shu uchburchakka tashqi chizilgan
aylana markazi $C$ va radiusi $R$ ni aniqlang.
 \\
C2. 
Uchburchakning tomonlari \(x + 5y - 7 = 0\),
\(4x - y - 7 = 0\), \(x + 3y - 31 = 0\) tenglamalar bilan berilgan.
Balandliklarining kesishish nuqtasini toping.
 \\
C3. 
\(ABC\) uchburchakning tomonlari bilan mos keluvchi \(\vec{AB} = \vec{b}\) va \(\vec{AC} = \vec{c}\) vektorlar berilgan. Bu uchburchakning \(B\) uchidan tushirilgan \(BD\) balandligining \(\vec{b},\ \vec{c}\) bazis bo‘yicha yoyilmasini toping.
 \\

\end{tabular}
\vspace{1cm}


\begin{tabular}{m{17cm}}
\textbf{11-variant}
\newline

T1. 
Koordinatalari bilan berilgan vektorlarning skalyar, vektor va aralash ko‘paytmalari. \\
T2. 
Nuqtadan tekislikkacha, fazoda nuqtadan to‘g‘ri chiziqqacha va ayqash to‘g‘ri chiziqlar orasidagi masofa. \\
A1. 
Ikkita uchi $A (-3; 2) $ va $B (1; 6) $ nuqtalarda
joylashgan muntazam uchburchakning yuzini hisoblang.
 \\
A2. 
$3x-y+2=0$, $4x-5y+5=0$, $2x+3y-1=0$
to‘g‘ri chiziqlar bir nuqtada kesishishadimi?
 \\
A3. 
To‘rtburchakning uchlari berilgan:
$A (1; - 2;2) $, $B (1;4;0),C (- 4;1;1) $ va $D (- 5; -5;3) $. Uning diagonallari $AC$ va $BD$ o‘zaro
perpendikulyarligini isbotlang.
 \\
B1. 
To‘g‘ri chiziq \(A (7;-3) \) va \(B (23;-6) \) nuqtalardan o‘tadi.
Shu to‘g‘ri chiziqning abssissa o‘qi bilan kesishish nuqtasini toping.
 \\
B2. Berilgan to‘g‘ri chiziqlarning kesishish nuqtasini toping:
$(3x-4y-29=0, 2x+5y+19=0)$.
 \\
B3. 
$\vec{a}$ va $\vec{b}$ vektorlar o‘zaro perpendikulyar. $|\vec{a}| = 3,|\vec{b}| = 4$ ekani ma’lum, quyidagilarni hisoblang:
$|\lbrack\vec{a} + \vec{b},\vec{a} - \vec{b}\rbrack|$.
 \\
C1. 
Ikki uchi \(A (2; - 3) \) va \(B (-5;1) \) nuqtalarda,
uchinchi uchi $C$ ordinata o‘qiga tegishli uchburchakning
medianalarining kesishish nuqtasi $M$ abssissa o‘qida yotadi.
$M$ va $C$ nuqtalarning koordinatalarini aniqlang.
 \\
C2. 
$ABC$ uchburchakning bir uchi \(A (1;3) \) nuqtada,
va ikkita medianasi \(x - 2y + 1 = 0\,\ y - 1 = 0\) to‘g‘ri chiziqlarda
joylashgan. Tomonlarining tenglamalarini tuzing.
 \\
C3. 
\(\vec{p} = \vec{b} - \frac{\vec{a} (\vec{a},\vec{b}) }{{\vec{a}}^{2}}\) vektor \(\vec{a}\) vektorga perpendikulyar ekanini isbotlang.
 \\

\end{tabular}
\vspace{1cm}


\begin{tabular}{m{17cm}}
\textbf{12-variant}
\newline

T1. 
Chiziqli bog‘liq va chiziqli bog‘lanmagan vektorlar.
 \\
T2. Tekislikda va fazoda dekart koordinatalar sistemasini almashtirish.
 \\
A1. 
Bir jinsli to‘rtburchakli plastinkaning uchlari berilgan:
$A (2;1), \ B (5;3), \ C (-1;7) $ va $D (-7;5) $. Uning og‘irlik markazi
koordinatalarini aniqlang.
 \\
A2. 
Umumiy tenglama bilan berilgan to‘g‘ri chiziqlarning
o‘zaro joylashuvini aniqlang, agar kesishadigan bo‘lsa kesishish nuqtasini
toping: $x-5=0, y+12=0$.
 \\
A3. 
Uchburchakning uchlari
$A (3;2; 3) $, $B (5;1; - 1) $ va $C (1; -2;1) $. Uning $A$ uchidagi tashqi burchagi aniqlansin.
 \\
B1. 
Ikkita nuqta berilgan \(M (2;2) \) va \(N (5;-2) \); abssissa o‘qida shunday $P$ nuqtani topingki, $MPN$ burchak to‘g‘ri burchak bo‘lsin.
 \\
B2. 
\(M (2;-5) \) nuqta, berilgan to‘g‘ri chiziqlarning:
\(3x+5y-4=0\) va \(x-2y+3=0\) kesishmasida hosil bo‘ladi
bo‘lgan o‘tkir yoki o‘tmas burchakka tegishli bo‘lishini aniqlang.
 \\
B3. 
$\vec{a} + \vec{b} + \vec{c} = 0$ shartni qanoatlantiruvchi $\vec{a},\ \vec{b}$ va $\vec{c}$ vektorlar berilgan. $|\vec{a}| = 3,\ |\vec{b}| = 1$ va $|\vec{c}| = 4$ ekani ma’lum, $\left(\vec{a},\vec{b} \right) + \left(\vec{b},\vec{c} \right) + (\vec{c}) $ ifodani hisoblang.
 \\
C1. 
Ikkita uchi \(A (2;1) \) va \(B (5; 3) \) nuqtalarida, va
diagonallarining kesishish nuqtasi ordinata o‘qiga tegishli
parallelogrammning yuzi \(S = 17\) ga teng. Qolgan ikki uchining
koordinatalarini aniqlang. \\
C2. 
\(A (11; - 15) \) va \(B (-7;3) \) nuqtalardan
teng masofada va \(C (3; 5) \) nuqtadan o‘tuvchi to‘g‘ri chiziq tenglamasini
tuzing.
 \\
C3. 
\(\vec{a},\ \vec{b}\) va \(\vec{c}\) vektorlar \(\vec{a} + \vec{b} + \vec{c} = 0\) shartni qanoatlantiradi. \(\lbrack\vec{a},\vec{b}\rbrack = \lbrack\vec{b},\vec{c}\rbrack = \lbrack\vec{c},\vec{a}\rbrack\) ekanini isbotlang.
 \\

\end{tabular}
\vspace{1cm}


\begin{tabular}{m{17cm}}
\textbf{13-variant}
\newline

T1. Analitik geometriya fanining predmeti va metodlari.
 \\
T2. 
Tekislikning tenglamalari. Tekisliklarning o‘zaro joylashishi.
 \\
A1. 
$A (2;2) $, $B (-1;6) $, $C (-5;3) $ va $D (-2;-1) $
nuqtalari kvadrat uchlari ekanini isbotlang.
 \\
A2. 
Umumiy tenglama bilan berilgan to‘g‘ri chiziqlarning
o‘zaro joylashuvini aniqlang, agar kesishadigan bo‘lsa kesishish nuqtasini
toping: $6x+10y+9=0, 3x+5y-6=0$.
 \\
A3. 
$\overrightarrow{a}
= \{ 1; - 1;3\}, \ \ \ \ \ \overrightarrow{b} = \{ - 2;1\}$, $\overrightarrow{c} = \{3; -2;5\}$ vektorlar berilgan. Hisoblang:
$ (\lbrack\overrightarrow{a},\overrightarrow{b}\rbrack,\overrightarrow{c}) $.
 \\
B1. 
Ikkala uchi \(A (3;1) \) va \(B (1;-3) \) nuqtalarda, va
og‘irlik markazi $Ox$ o‘qiga tegishli uchburchakning yuzi
\(S=3\) ga teng. Uchinchi $C$ uchining koordinatalarini aniqlang. \\
B2. 
Berilgan to‘g‘ri chiziqlar orasidagi burchakni aniqlang: $3x+2y+4=0, 5x-y+1=0$.
 \\
B3. 
$|\vec{a}| = 3,|\vec{b}| = 5$ berilgan. $\alpha$ ning qanday qiymatida $\vec{a} + \alpha\vec{b}$, $\vec{a} - \alpha\vec{b}$ vektorlar o‘zaro perpendikulyar bo‘lishini aniqlang.
 \\
C1. 
Uchburchakning uchlari
\(A (- 1; - 1),\ B (3;5),\ C (- 4;1) \) berilgan. $A$ uchi tashqi
burchak bissektrisasining, $BC$ tomonining davomi bilan kesishish
nuqtani toping.
 \\
C2. 
\(P (3;5) \) nuqtadan o‘tib, \(4x + 6y - 7 = 0\) to‘g‘ri chiziq
bilan \(45^{0}\) burchak yasab kesishuvchi to‘g‘ri chiziq tenglamasini tuzing.
 \\
C3. 
\(\lbrack\vec{a},\vec{b}\rbrack^{2} < {\vec{a}}^{2}{\vec{b}}^{2}\) ekanini isbotlang; qanday holda bu yerda tenglik ishorasi bo‘ladi?
 \\

\end{tabular}
\vspace{1cm}


\begin{tabular}{m{17cm}}
\textbf{14-variant}
\newline

T1. 
Vektorning koordinatalari.
 \\
T2. 
Tekislikdagi to‘g‘ri chiziqlarning o‘zaro joylashishi.
 \\
A1. 
Uchlari $A (2;-3) $, $B (3;2) $ va $C (-2;5) $
nuqtalarida joylashgan uchburchaklarning yuzini hisoblang.
 \\
A2. 
$Q_1$, $Q_2$, $Q_3$, $Q_4$, $Q_5$ nuqtalar
$x-3y+2=0$ to‘g‘ri chiziqqa tegishli va ordinatalari mos ravishda
1, 0, 2, -1, 3 ga teng. Ularning abssissalarini toping.
 \\
A3. 
Agar \(a = \{ 2; - 1;2\}, \ \ \ \ b = \{ 1;2; - 3\}, \ \ \ \ c = \{ 3; - 4;7\}\) bo‘lsa, $\overrightarrow{a}, \overrightarrow{b}, \overrightarrow{c}$ vektorlar komplanar bo‘lishini tekshiring. \\
B1. 
To‘g‘ri chiziq \(M (2;-3) \) va \(N (-6;5) \) nuqtalardan o‘tadi.
Shu to‘g‘ri chiziqda ordinatasi $-5$ ga teng nuqtani toping.
 \\
B2. 
Ikki to‘g‘ri chiziqning chetidagi burchakni toping: $2x+y-9=0, 3x-y+11=0$.
 \\
B3. 
$\vec{a}$ va $\vec{b}$ vektorlar o‘zaro perpendikulyar; $\vec{c}$ vektor ular bilan $\pi/3$ ga teng bo‘lgan burchaklar hosil qiladi; $|\vec{a}| = 3$, $|\vec{b}| = 5,\ |\vec{c}| = 8$ ekani ma’lum, quyidagilarni hisoblang:
$ (\vec{a} + 2\vec{b} - 3\vec{c}) ^{2}$.
 \\
C1. 
\(M{1} (1; 2) \) nuqta orqali, radiusi 5 ga teng,
$Ox$ o‘qiga urinma aylana o‘tkazildi. Shu aylananing markazi
$S$ ni aniqlang.
 \\
C2. 
\(A (-5;5) \) va \(B (-7;1) \) nuqtalardan
masofalarining yig‘indisi eng kichik bo‘lgan \(2x - y - 5 = 0\) to‘g‘ri chiziqda
joylashgan nuqtani toping.
 \\
C3. 
Ayniyatni isbotlang: \((\lbrack\vec{a},\vec{b}\rbrack,\vec{c} + \lambda\vec{a} + \mu\vec{b}) = (\lbrack\vec{a},\vec{b}\rbrack,\vec{c}) \), bunda \(\lambda\) va \(\mu\) - ixtiyoriy sonlar. \\

\end{tabular}
\vspace{1cm}


\begin{tabular}{m{17cm}}
\textbf{15-variant}
\newline

T1. 
Vektorning koordinatalari.
 \\
T2. 
Tekislik va to‘g‘ri chiziqlarning o‘zaro joylashishi.
 \\
A1. 
Bir jinsli elementdan yasalgan qatorning og‘irlik markazi
$M (1;4) $ nuqtada, bir uchi $P (-2;2) $ nuqtada joylashgan. Shu
qatorning ikkinchi uchi $Q$ ning koordinatalarini aniqlang.
 \\
A2. 
$5x+3y+2=0$ to‘g‘ri chiziqning $k$ burchagi
koeffitsiyentini va $Oy$ o‘qidan kesib olgan kesmaning algebraik
qiymati $b$ ni aniqlang.
 \\
A3. 
Agar \(a = \{ 3; - 2;1\},\ \ \ \ \ b = \{ 2;1;2\},\ \ \ \ c = \{ 3; - 1; - 2\}\) bo‘lsa, $\overrightarrow{a}, \overrightarrow{b}, \overrightarrow{c}$ vektorlar komplanar bo‘lishini tekshiring.
 \\
B1. 
To‘rtburchakning uchlari
\(A (-3;12),\ B (3;-4),\ C (5;-4) \) va \(D (5;8) \) berilgan. Shu
to‘rtburchakning $AC$ diagonali $BD$ diagonali qanday
nisbatda bo'lishini aniqlang.
 \\
B2. 
Berilgan \(3x-4y-10=0\) to‘g‘ri chiziqqa parallel va undan
$d=3$ masofada yotuvchi to‘g‘ri chiziqlarning tenglamasini tuzing.
 \\
B3. 
$a$ va $b$ vektorlar $\varphi = \pi/6$ burchak hosil qiladi; $|a| = \sqrt{3},|b| = 1$ ekani ma’lum. $p = a + b$ va $q = a - b$ vektorlar orasidagi $\alpha$ burchakni hisoblang.
 \\
C1. 
Uchburchakning uchlari
\(A (3; - 5),\ B (1; - 3),\ C (2; - 2) \) berilgan. $B$ uchi tashqi
burchagi bessektrisa uzunligini aniqlang.
 \\
C2. 
$ABC$ uchburchakning bir uchini \(B (2;6) \), va
bir uchidan o‘tkazilgan balandlikning: \(x - 7y + 15 = 0\), va
bissektrisasining: \(7x + y + 5 = 0\) tenglamalarini bilgan holda,
tomonlarining tenglamalarini tuzing. \\
C3. \(\vec{a} + \vec{b} + \vec{c} = 0\) shartni qanoatlantiruvchi birlik \(\vec{a},\ \vec{b}\) va \(\vec{c}\) vektorlar berilgan. Hisoblang: \(\left(\vec{a},\vec{b} \right) + \left(\vec{b},\vec{c} \right) + \left(\vec{c},\vec{a} \right) \).
 \\

\end{tabular}
\vspace{1cm}


\begin{tabular}{m{17cm}}
\textbf{16-variant}
\newline

T1. Analitik geometriya fanining predmeti va metodlari.
 \\
T2. 
Nuqtadan to‘g‘ri chiziqqacha bo‘lgan masofa. To‘g‘rilar dastasi.
 \\
A1. 
Uchburchak uchlarining koordinatalari berilgan
$A (1;-3) $, $B (3;-5) $ va $C (-5;7) $. Tomonlarining o‘rtalarini
aniqlang.
 \\
A2. 
Umumiy tenglama bilan berilgan to‘g‘ri chiziqlarning
o‘zaro joylashuvini aniqlang, agar kesishadigan bo‘lsa kesishish nuqtasini
toping: $3x+2y-27=0, x+5y-35=0$.
 \\
A3. 
Berilgan: $\overrightarrow{a}| = 3,|\overrightarrow{b}| = 26$ va
$\lbrack\overrightarrow{a},\overrightarrow{b}\rbrack| = 72$. Hisoblang
$\left(\overrightarrow{a},\overrightarrow{b} \right) $.
 \\
B1. 
Uchlari \(M (-1;3),\ N (1,2) \ \) va \(P (0;4) \)
nuqtalarida joylashgan uchburchakning ichki burchaklari o‘tkir burchak
ekanligini isbotlang.
 \\
B2. 
Berilgan ikki nuqtadan o‘tuvchi to‘g‘ri chiziqning burchagi
koeffitsiyenti $k$ ni hisoblang: $A (-4;3) $, $B (1;8) $.
 \\
B3. 
$\vec{a}$ va $\vec{b}$ vektorlar $\varphi = 2\pi/3$ burchak hosil qiladi. $|\vec{a}| = 3,|\vec{b}| = 4$ ekani ma’lum. Hisoblang:
$\left(\vec{a},\vec{b} \right) $.
 \\
C1. 
\(M{1} (1; 2) \) nuqta orqali, radiusi 5 ga teng,
$Ox$ o‘qiga urinma aylana o‘tkazildi. Shu aylananing markazi
$S$ ni aniqlang.
 \\
C2. 
\(A (3;7) \) va \(C (6; 5) \) nuqtalar kvadratning
qarama-qarshi uchlari. Uning tomonlari tenglamasini tuzing.
 \\
C3. 
Ayniyatni isbotlang: \(\lbrack\vec{a},\vec{b}\rbrack^{2} + (\vec{a},\vec{b}) ^{2} = {\vec{a}}^{2}{\vec{b}}^{2}\).
 \\

\end{tabular}
\vspace{1cm}


\begin{tabular}{m{17cm}}
\textbf{17-variant}
\newline

T1. 
Vektorlarning vektor ko‘paytmasi va aralash ko‘paytmasi.
 \\
T2. 
Nuqtadan tekislikkacha, fazoda nuqtadan to‘g‘ri chiziqqacha va ayqash to‘g‘ri chiziqlar orasidagi masofa. \\
A1. 
Kvadratning ikkita qo‘shni uchlari $A (3; -7)$ va
$B (-1;4) $ berilgan. Uning yuzini hisoblang.
 \\
A2. 
$M (3;3)$ nuqtadan o‘tib, koordinata o‘qlaridan teng
kesmalarni kesib oladigan to‘g‘ri chiziqlarning tenglamasini tuzing.
 \\
A3. 
Agar \(a = \{ 2;3; - 1\}, \ \ \ \ b = \{ 1; - 1;3\}, \ \ \ \ c = \{ 1;9; - 11\}\) bo‘lsa, $\overrightarrow{a}, \overrightarrow{b}, \overrightarrow{c}$ vektorlar komplanar bo‘lishini tekshiring.
 \\
B1. 
Uchburchakning uchlari
\(A\left(-\sqrt{3};1 \right),\ B (0;2) \) va
\(C\left(-2\sqrt{3};2 \right) \) nuqtalarda. Uning $A$
uchidagi tashqi burchakni toping.
 \\
B2. 
Qirralari
\(7x+y+31=0,\ 3x+4y-1=0,\ x-7y-17=0\) tenglamalar
bilan berilgan uchburchakning teng yonli ekanini isbotlang.
Masalani uchburchakning
burchaklarini topish orqali yeching.
 \\
B3. 
$\vec{a}$ va $\vec{b}$ vektorlar $\varphi = 2\pi/3$ burchak hosil qiladi. $|\vec{a}| = 3,|\vec{b}| = 4$ ekani ma’lum. Hisoblang:
$ (\vec{a} + \vec{b}) ^{2}$.
 \\
C1. 
Ikki uchi \(A (2; - 3) \) va \(B (-5;1) \) nuqtalarda,
uchinchi uchi $C$ ordinata o‘qiga tegishli uchburchakning
medianalarining kesishish nuqtasi $M$ abssissa o‘qida yotadi.
$M$ va $C$ nuqtalarning koordinatalarini aniqlang.
 \\
C2. 
Agarda \(M (4;5) \) nuqta, koordinata boshidan to‘g‘ri chiziqqa
o‘tkazilgan perpendikulyarning asosi bo‘lsa, shu to‘g‘ri chiziq tenglamasini
tuzing.
 \\
C3. 
\(\vec{p} = \vec{b} - \frac{\vec{a} (\vec{a},\vec{b}) }{{\vec{a}}^{2}}\) vektor \(\vec{a}\) vektorga perpendikulyar ekanini isbotlang.
 \\

\end{tabular}
\vspace{1cm}


\begin{tabular}{m{17cm}}
\textbf{18-variant}
\newline

T1. 
Chiziqli bog‘liq va chiziqli bog‘lanmagan vektorlar.
 \\
T2. 
Tekislikdagi to‘g‘ri chiziqlarning o‘zaro joylashishi.
 \\
A1. 
Bir jinsli elementdan yasalgan qatorning uchlari
$A (3;-5) $ va $B (-1;1) $ nuqtalarda joylashgan. Uning og‘irligi
markazi koordinatasini aniqlang.
 \\
A2. 
$m$ va $n$ parametrlarining qanday qiymatlarida
$mx+8y+n=0$, $2x+my-1=0$ to‘g‘ri chiziqlar parallel bo‘ladi?
 \\
A3. 
Vektor koordinata o‘qlari bilan quyidagi burchaklarni hosil qilishi
mumkinmi: $\alpha = 90^{{^\circ}},\ \beta = 150^{{^\circ}}$,
$\gamma = 60^{{^\circ}}?$
 \\
B1. 
To‘g‘ri \(A (5;2) \) va \(B (-4; -7) \) nuqtalaridan o‘tadi.
Shu to‘g‘ri chiziqning ordinata o‘qi bilan kesishish nuqtasini toping.
 \\
B2. 
Berilgan \(8x-15y-25=0\) to‘g‘ri chiziqdan og‘ishi -2 ga teng
teng bo‘lgan nuqtalarning geometrik o‘rni tenglamasini tuzing.
 \\
B3. 
$\vec{a} = \{ 2;1; - 1\}$ vektorga kollinear bo‘lgan va $\left(\vec{x},\vec{a} \right) = 3$ shartni qanoatlantiruvchi $\vec{x}$ vektorni toping.
 \\
C1. 
Ikkita uchi \(A (2;1) \) va \(B (5; 3) \) nuqtalarida, va
diagonallarining kesishish nuqtasi ordinata o‘qiga tegishli
parallelogrammning yuzi \(S = 17\) ga teng. Qolgan ikki uchining
koordinatalarini aniqlang. \\
C2. 
Bir tomoni \(x-4y - 8 = 0\) to‘g‘ri chiziqda yotuvchi
kvadratning og‘irlik markazi \(M (1;1) \) nuqtada joylashgan. Shu kvadratning
qolgan tomonlari yotgan to‘g‘ri chiziqlarning tenglamalarini tuzing.
 \\
C3. 
\(\vec{a}+\vec{b}\) va \(\vec{a} - \vec{b}\) vektorlar kollinear bo‘lishi uchun \(\vec{a},\vec{b}\) vektorlar qanday shartni qanoatlantirishi kerak?
 \\

\end{tabular}
\vspace{1cm}


\begin{tabular}{m{17cm}}
\textbf{19-variant}
\newline

T1. 
Vektor tushunchasi. Vektorlar ustida chiziqli amallar.
 \\
T2. 
Nuqtadan to‘g‘ri chiziqqacha bo‘lgan masofa. To‘g‘rilar dastasi.
 \\
A1. 
Uchburchakning uchlari $A (1;4) $, $B (3;-9) $, $C (-5;2) $
berilgan. $B$ uchidan o‘tkazilgan mediana uzunligini aniqlang.
 \\
A2. 
$x+2y-17=0$, $2x-y+1=0$, $x+2y-3=0$
to‘g‘ri chiziqlar bir nuqtada kesishishadimi?
 \\
A3. 
Agar \(a = \{ 2; - 1;2\}, \ \ \ \ b = \{ 1;2; - 3\}, \ \ \ \ c = \{ 3; - 4;7\}\) bo‘lsa, $\overrightarrow{a}, \overrightarrow{b}, \overrightarrow{c}$ vektorlar komplanar bo‘lishini tekshiring. \\
B1. 
Ordinata o‘qida shunday $M$ nuqtani toping.
\(N (-8;13) \) nuqtadan uzoqligi 17 ga teng bo‘lgan.
 \\
B2. 
\(4x+3y-1=0\) va \(3x-2y+5=0\)
to‘g‘ri chiziqlarning kesishish nuqtasidan o‘tib (bu nuqtani aniqlamay), ordinata
o‘qidan \(b=4\) kesmani kesib oladigan to‘g‘ri chiziq tenglamasini tuzing.
 \\
B3. 
$\vec{a} = \{ 3; - 1; - 2\}$ va $\vec{b} = \{ 1;2; - 1\}$ vektorlar berilgan. Quyidagi vektor ko‘paytmalarning koordinatalarini toping:
$\left\lbrack 2\vec{a} + \vec{b},\vec{b} \right\rbrack$.
 \\
C1. 
Uchburchakning uchlari
\(A (- 1; - 1),\ B (3;5),\ C (- 4;1) \) berilgan. $A$ uchi tashqi
burchak bissektrisasining, $BC$ tomonining davomi bilan kesishish
nuqtani toping.
 \\
C2. 
\(Q (5; - 6) \) nuqtaning, \(A (3;8) \) va \(B (7;5) \)
nuqtalardan o‘tgan to‘g‘ri chiziqdagi proyeksiyasini toping.
 \\
C3. 
\(\vec{p} = \vec{b} (\vec{a},\vec{c}) - \vec{c} (\vec{a},\vec{b}) \) vektor \(\vec{a}\) vektorga perpendikulyar ekanini isbotlang.
 \\

\end{tabular}
\vspace{1cm}


\begin{tabular}{m{17cm}}
\textbf{20-variant}
\newline

T1. 
Vektorlarning skalyar ko‘paytmasi.
 \\
T2. 
Tekislikda to‘g‘ri chiziqning tenglamalari.
 \\
A1. 
$M (2;-1) $, $N (-1;4) $ va $P (-2;2) $ nuqtalar
uchburchak tomonlarining o‘rtalari. Uchlarining koordinatalarini
aniqlang.
 \\
A2. 
Umumiy tenglama bilan berilgan to‘g‘ri chiziqlarning
o‘zaro joylashuvini aniqlang, agar kesishadigan bo‘lsa kesishish nuqtasini
toping: $12x+15y-39=0, 16x-9y-23=0$.
 \\
A3. 
Uchburchakning uchlari
$A (3;2; 3) $, $B (5;1; - 1) $ va $C (1; -2;1) $. Uning $A$ uchidagi tashqi burchagi aniqlansin.
 \\
B1. 
Uchburchakning uchlari \(A (3;6),\ B (-1;3) \) va
\(C (2:-1) \) nuqtalarda joylashgan. $C$ uchidan tushirilgan balandlik uzunligini hisoblang.
 \\
B2. 
Parallelogrammning ikki tomoni tenglamalari
\(8x+3y+1=0,\ 2x+y-1=0\) va bir diagonali tenglamasi
\(3x+2y+3=0\) berilgan. Parallelogramm uchlari koordinatalarini
aniqlang
 \\
B3. 
$\vec{a}$ va $\vec{b}$ vektorlar $\varphi = 2\pi/3$ burchak hosil qiladi. $|\vec{a}| = 1,|\vec{b}| = 2$ ekanini bilib, quyidagilarni hisoblang:
$\lbrack\overrightarrow{a} + 3\overrightarrow{b},3\overrightarrow{a} - \overrightarrow{b}\rbrack^{2}$
 \\
C1. 
Uchburchakning uchlari \(M_{1} (- 3;6),\ M_{2} (9; - 10) \)
va \(M_{3} (-5;4) \) berilgan. Shu uchburchakka tashqi chizilgan
aylana markazi $C$ va radiusi $R$ ni aniqlang.
 \\
C2. 
Uchburchaklarning uchlari
\(A (2; - 2),\ B (3; - 5),\ C (5;7) \) nuqtalarda joylashgan. $C$
uchidan o‘tib, $A$ uchidan o‘tkazilgan bissektrisaga
perpendikular to‘g‘ri chiziqning tenglamasini tuzing.
 \\
C3. 
Ayniyatni isbotlang: \((\lbrack\vec{a} + \vec{b},\vec{b} + \vec{c}\rbrack,\vec{c} + \vec{a}) = 2 (\lbrack\vec{a},\vec{b}\rbrack,\vec{c}) \).
 \\

\end{tabular}
\vspace{1cm}


\begin{tabular}{m{17cm}}
\textbf{21-variant}
\newline

T1. 
Koordinatalari bilan berilgan vektorlarning skalyar, vektor va aralash ko‘paytmalari. \\
T2. 
Tekislikning tenglamalari. Tekisliklarning o‘zaro joylashishi.
 \\
A1. 
$A (4;2) $, $B (7;-2) $ va $C (1;6) $ nuqtalar bir jinsli
simdan yasalgan uchburchak uchlari. Shu uchburchakning og‘irligi
 \\
A2. 
Umumiy tenglama bilan berilgan to‘g‘ri chiziqlarning
o‘zaro joylashuvini aniqlang, agar kesishadigan bo‘lsa kesishish nuqtasini
toping: $2x-3y+12=0, 4x-6y-21=0$.
 \\
A3. 
Berilgan: $\overrightarrow{a}| = 10,|\overrightarrow{b}| = 2$ va
$\left(\overrightarrow{a},\overrightarrow{b} \right) = 12$. Hisoblang
$\left| \left\lbrack \overrightarrow{a},\overrightarrow{b} \right\rbrack \right|$.
 \\
B1. 
To‘g‘ri \(M_{1} (-12;-13) \) va \(M_{2} (-2;-5) \)
nuqtalaridan o‘tadi. Shu to‘g‘ri chiziqda abssissasi 3 ga teng nuqtani toping.
 \\
B2. 
\(P (2;7) \) nuqtadan o‘tib, \(Q (1;2) \) nuqtagacha
masofasi 5 ga teng bo‘lgan to‘g‘ri chiziqlarning tenglamasini tuzing.
 \\
B3. Tekislikda uchta vektor $\vec{a} = \{ 3; - 2\}$, $\vec{b} = \{ - 2;1\}$ va $\vec{c} = \{ 7; - 4\}$ berilgan. Bu uchta vektorning har birining qolgan ikkitasini bazis sifatida qabul qilib yoyilmasini toping.
 \\
C1. \(A (4;2) \) nuqta orqali, ikkita koordinata o‘qlariga
urinma doira o‘tkazildi. Uning markazi $C$ ni va radiusi
$R$ ni toping.
 \\
C2. Ikkita uchi \(A (1; - 2),\ B (2;3) \) nuqtalarda joylashgan,
yuzi \(S = 8\) ga teng bo‘lgan uchburchakning uchinchi uchi
$C$ \(2x + y - 2 = 0\) to‘g‘ri chiziqqa tegishli. Shu $C$ uchining
koordinatasini aniqlang.
 \\
C3. 
\(ABC\) uchburchakning tomonlari bilan mos keluvchi \(\vec{AB} = \vec{b}\) va \(\vec{AC} = \vec{c}\) vektorlar berilgan. Bu uchburchakning \(B\) uchidan tushirilgan \(BD\) balandligining \(\vec{b},\ \vec{c}\) bazis bo‘yicha yoyilmasini toping.
 \\

\end{tabular}
\vspace{1cm}


\begin{tabular}{m{17cm}}
\textbf{22-variant}
\newline

T1. 
Chiziqli bog‘liq va chiziqli bog‘lanmagan vektorlar.
 \\
T2. 
Fazoviy to‘g‘ri chiziqning tenglamalari. To‘g‘ri chiziqlarning o‘zaro joylashishi.
 \\
A1. 
Ikkala uchi $A (2;1) $ va $B (3;-2) $ nuqtalarda, va
uchinchi $C$ uchi $Ox$ o‘qiga tegishli bo‘lgan uchburchakning
yuzi $S=4$ ga teng. $C$ uchining koordinatalarini aniqlang. \\
A2. 
$M (4;3) $ nuqtadan, koordinata burchagidan
yuzi 3 ga teng uchburchak kesib oladigan to‘g‘ri chiziq o‘tkazildi.
Shu to‘g‘ri chiziqning koordinata o‘qlari bilan kesishish nuqtalari
koordinatalarini aniqlang.
 \\
A3. 
Agar \(a = \{ 3; - 2;1\},\ \ \ \ \ b = \{ 2;1;2\},\ \ \ \ c = \{ 3; - 1; - 2\}\) bo‘lsa, $\overrightarrow{a}, \overrightarrow{b}, \overrightarrow{c}$ vektorlar komplanar bo‘lishini tekshiring.
 \\
B1. 
Parallelogrammning uchta uchi \(A (3;7),\ B (2;-3) \) va
\(C (-1;4) \) nuqtalarda joylashgan. $B$ uchidan $AC$
tomonidan tushirilgan balandlik uzunligini hisoblang.
 \\
B2. 
Koordinata boshi, tomonlarining tenglamalari
\(8x+3y+31=0,\ x+8y-19=0,\ 7x-5y-11=0\) bilan
berilgan uchburchakning tashqarisida yoki ichida yotishini aniqlang.
 \\
B3. 
$\vec{a} = \{ 6; - 8; - 7,5\}$ vektorga kollinear bo‘lgan $\vec{x}$ vektor $Oz$ o‘qi bilan o‘tkir burchak hosil qiladi. $|\vec{x}| = 50$ ekanini bilgan holda uning koordinatalarini toping.
 \\
C1. 
Uchburchakning uchlari
\(A (- 1; - 1),\ B (3;5),\ C (- 4;1) \) berilgan. $A$ uchi tashqi
burchak bissektrisasining, $BC$ tomonining davomi bilan kesishish
nuqtani toping.
 \\
C2. 
\(N (2; 5) \) nuqtaning \(9x - 7y + 30 = 0\) to‘g‘ri chizig‘iga
nisbatan simmetrik nuqtani toping.
 \\
C3. 
\(\vec{a} + \vec{b}\) vektor \(\vec{a} - \vec{b}\) vektorga perpendikulyar bo‘lishi uchun \(\vec{a}\) va \(\vec{b}\) vektorlar qanday shartlarni qanoatlantirishi kerak?
 \\

\end{tabular}
\vspace{1cm}


\begin{tabular}{m{17cm}}
\textbf{23-variant}
\newline

T1. 
Vektorlarning vektor ko‘paytmasi va aralash ko‘paytmasi.
 \\
T2. Tekislikda va fazoda dekart koordinatalar sistemasini almashtirish.
 \\
A1. 
Kvadratning ikkita qarama-qarshi uchlari $P (3; 5) $ va
$Q (1; -3) $ berilgan. Uning yuzini hisoblang.
 \\
A2. 
Umumiy tenglama bilan berilgan to‘g‘ri chiziqlarning
o‘zaro joylashuvini aniqlang, agar kesishadigan bo‘lsa kesishish nuqtasini
toping: $2y+9=0, y-5=0$.
 \\
A3. 
To‘rtburchakning uchlari berilgan:
$A (1; - 2;2) $, $B (1;4;0),C (- 4;1;1) $ va $D (- 5; -5;3) $. Uning diagonallari $AC$ va $BD$ o‘zaro
perpendikulyarligini isbotlang.
 \\
B1. 
Ikkita qarama-qarshi uchlari $P (3; -4) $ va $Q (l;2) $ nuqtalarda joylashgan rombaning tomon uzunligi \(5\sqrt{2}\). Shu romb balandligining uzunligini hisoblang.
 \\
B2. 
To‘g‘ri to‘rtburchakning ikki tomoni
\(5x+2y-7=0,\ 5x+2y-36=0\) va diagonali
\(3x+7y-10=0\) tenglamalar bilan berilgan. Qolgan ikki tomoni
tenglamalarni tuzing.
 \\
B3. 
$\vec{a}$ va $\vec{b}$ vektorlar $\varphi = 2\pi/3$ burchak hosil qiladi. $|\vec{a}| = 3,|\vec{b}| = 4$ ekani ma’lum. Hisoblang:
$\left(3\vec{a} - 2\vec{b},\vec{a} + 2\vec{b} \right) $.
 \\
C1. 
\(M{1} (1; 2) \) nuqta orqali, radiusi 5 ga teng,
$Ox$ o‘qiga urinma aylana o‘tkazildi. Shu aylananing markazi
$S$ ni aniqlang.
 \\
C2. 
Ikki nuqta \(A (3; - 5) \) va \(B (- 2;3) \) berilgan.
$B$ nuqtadan o‘tib, $AB$ kesmaga perpendikular to‘g‘ri chiziq
tenglamasini tuzing.
 \\
C3. 
Ayniyatni isbotlang: \(\lbrack\vec{a},\vec{b}\rbrack^{2} + (\vec{a},\vec{b}) ^{2} = {\vec{a}}^{2}{\vec{b}}^{2}\).
 \\

\end{tabular}
\vspace{1cm}


\begin{tabular}{m{17cm}}
\textbf{24-variant}
\newline

T1. 
Vektor tushunchasi. Vektorlar ustida chiziqli amallar.
 \\
T2. 
Tekislik va to‘g‘ri chiziqlarning o‘zaro joylashishi.
 \\
A1. 
Uch uchi $A (-2;3), \ B (4;-5) $ va
$C (-3;1)$ nuqtalarda joylashgan parallelogrammning yuzini aniqlang.
 \\
A2. 
$m$ parametrining qanday qiymatlarida
$ (m-1) x+my-5=0$, $mx+ (2m-1) y+7=0$ to‘g‘ri chiziqlar abssissa
o‘qida yotuvchi nuqtada kesishadi.
 \\
A3. 
Tekislikda ikkita vektor
$\overrightarrow{p} = \{ 2; - 3\}$, $\overrightarrow{q} = \{ 1;2\}$.
$\overrightarrow{a} = \{9;4\}$ vektorning
$\overrightarrow{p},\ \overrightarrow{q}$ bazis bo‘yicha yoyilmasi topilsin.
 \\
B1. 
To‘rtburchakning uchlari
\(A (-2;14),\ B (4;-2),\ C (6;-2) \) va \(D (6;10) \) berilgan. Shu
to‘rtburchakning $AC$ va $BD$ diagonallarining kesishishi
nuqtani toping.
 \\
B2. 
To‘g‘ri to‘rtburchakning bir uchi \(A (2;-3) \), va ikkita tarafining
ning tenglamalari \(2x+3y+9=0,\ 3x-2y-7=0\)
berilgan. Qolgan ikki tomonning tenglamalarini tuzing.
 \\
B3. 
$\vec{a}$ va $\vec{b}$ vektorlar o‘zaro perpendikulyar; $\vec{c}$ vektor ular bilan $\pi/3$ ga teng bo‘lgan burchaklar hosil qiladi; $|\vec{a}| = 3$, $|\vec{b}| = 5,\ |\vec{c}| = 8$ ekani ma’lum, quyidagilarni hisoblang:
$ (\vec{a} + \vec{b} + \vec{c}) ^{2}$.
 \\
C1. 
Uchburchakning uchlari \(M_{1} (- 3;6),\ M_{2} (9; - 10) \)
va \(M_{3} (-5;4) \) berilgan. Shu uchburchakka tashqi chizilgan
aylana markazi $C$ va radiusi $R$ ni aniqlang.
 \\
C2. 
Ikki uchi \(A (2; - 3),\ B (3; - 2) \) nuqtalarda
joylashgan, yuzi \(S = 1,5\) ga teng bo‘lgan uchburchakning,
og‘irlik markazi \(3x - y - 8 = 0\) to‘g‘ri chiziqqa tegishli. Uchinchi $C$
uchining koordinatasini aniqlang.
 \\
C3. 
\(\vec{a} + \vec{b}\) vektor \(\vec{a} - \vec{b}\) vektorga perpendikulyar bo‘lishi uchun \(\vec{a}\) va \(\vec{b}\) vektorlar qanday shartlarni qanoatlantirishi kerak?
 \\

\end{tabular}
\vspace{1cm}


\begin{tabular}{m{17cm}}
\textbf{25-variant}
\newline

T1. Analitik geometriya fanining predmeti va metodlari.
 \\
T2. Tekislikda va fazoda dekart koordinatalar sistemasini almashtirish.
 \\
A1. 
Parallelogrammning uchlari
$A (3;-5) $, $B (5;-3) $, $C (-1;3) $ berilgan. $B$ tepasiga
qarama-qarshi joylashgan $D$ uchini aniqlang.
 \\
A2. 
$a$ va $b$ parametrlarining qanday qiymatlarida
$ax-2y-1=0$, $6x-4y-b=0$ to‘g‘ri chiziqlar umumiy nuqtaga ega bo‘ladi?
 \\
A3. 
Agar \(a = \{ 2;3; - 1\}, \ \ \ \ b = \{ 1; - 1;3\}, \ \ \ \ c = \{ 1;9; - 11\}\) bo‘lsa, $\overrightarrow{a}, \overrightarrow{b}, \overrightarrow{c}$ vektorlar komplanar bo‘lishini tekshiring.
 \\
B1. 
Uchburchakning uchlari \(A (3;6),\ B (-1;3) \) va
\(C (2:-1) \) nuqtalarda joylashgan. $C$ uchidan tushirilgan balandlik uzunligini hisoblang.
 \\
B2. 
Parallel to‘g‘ri chiziqlar orasidagi masofani hisoblang: $5x-12y+13=0, 5x-12y-26=0$.
 \\
B3. 
$\vec{a}$ va $\vec{b}$ vektorlar $\varphi = 2\pi/3$ burchak hosil qiladi. $|\vec{a}| = 3,|\vec{b}| = 4$ ekani ma’lum. Hisoblang:
${\vec{b}}^{2}$.
 \\
C1. 
Ikki uchi \(A (2; - 3) \) va \(B (-5;1) \) nuqtalarda,
uchinchi uchi $C$ ordinata o‘qiga tegishli uchburchakning
medianalarining kesishish nuqtasi $M$ abssissa o‘qida yotadi.
$M$ va $C$ nuqtalarning koordinatalarini aniqlang.
 \\
C2. 
$ABC$ uchburchakning ikki uchi
\(A (6; - 2),\ B (10;14) \), va balandliklarining kesishish nuqtasi
\(N (4; - 1) \) berilgan. Bu uchburchakning tomonlari tenglamasini tuzing.
 \\
C3. 
\(\vec{p} = \vec{b} - \frac{\vec{a} (\vec{a},\vec{b}) }{{\vec{a}}^{2}}\) vektor \(\vec{a}\) vektorga perpendikulyar ekanini isbotlang.
 \\

\end{tabular}
\vspace{1cm}


\begin{tabular}{m{17cm}}
\textbf{26-variant}
\newline

T1. 
Vektorlarning skalyar ko‘paytmasi.
 \\
T2. 
Nuqtadan to‘g‘ri chiziqqacha bo‘lgan masofa. To‘g‘rilar dastasi.
 \\
A1. 
Parallelogrammning uchlari
$A (3;-5) $, $B (5;-3) $, $C (-1;3) $ berilgan. $B$ tepasiga
qarama-qarshi joylashgan $D$ uchini aniqlang.
 \\
A2. 
$P (2;2)$ nuqtadan o‘tib, koordinata burchagidan
yuzi 1 ga teng uchburchak kesib oladigan to‘g‘ri chiziqlarning
tenglamasini tuzing.
 \\
A3. 
Berilgan: $\overrightarrow{a}| = 3,|\overrightarrow{b}| = 26$ va
$\lbrack\overrightarrow{a},\overrightarrow{b}\rbrack| = 72$. Hisoblang
$\left(\overrightarrow{a},\overrightarrow{b} \right) $.
 \\
B1. 
\(M_{1} (1;2) \) nuqtaga, \(A (1;0) \) va \(B (-1;-2) \)
nuqtalaridan o‘tuvchi to‘g‘ri chiziqqa nisbatan simmetrik bo‘lgan \(M_{2}\) nuqtaning koordinatalarini toping.
 \\
B2. 
\(2x+y-2=0\) va \(x-5y-3=0\)
to‘g‘ri chiziqlarning kesishish nuqtasidan o‘tib (bu nuqtani aniqlamay), uchlari
\(A (-1;-4) \) va \(B (5;-6) \) nuqtalarda joylashgan kesmaning
to‘g‘ri o‘rtasidan o‘tuvchi to‘g‘ri chiziqning tenglamasini tuzing.
 \\
B3. 
$\vec{a} + \vec{b} + \vec{c} = 0$ shartni qanoatlantiruvchi $\vec{a},\ \vec{b}$ va $\vec{c}$ vektorlar berilgan. $|\vec{a}| = 3,\ |\vec{b}| = 1$ va $|\vec{c}| = 4$ ekani ma’lum, $\left(\vec{a},\vec{b} \right) + \left(\vec{b},\vec{c} \right) + (\vec{c}) $ ifodani hisoblang.
 \\
C1. 
Uchburchakning uchlari
\(A (3; - 5),\ B (1; - 3),\ C (2; - 2) \) berilgan. $B$ uchi tashqi
burchagi bessektrisa uzunligini aniqlang.
 \\
C2. 
Uchburchakning \(A (- 3; - 2),\ B (5; - 4),\ C (- 1;3) \)
uchlaridan o‘tib, qarama-qarshi tomonga parallel to‘g‘ri chiziqlarning tenglamalarini
tuzing.
 \\
C3. 
\(\vec{a},\ \vec{b}\) va \(\vec{c}\) vektorlar \(\vec{a} + \vec{b} + \vec{c} = 0\) shartni qanoatlantiradi. \(\lbrack\vec{a},\vec{b}\rbrack = \lbrack\vec{b},\vec{c}\rbrack = \lbrack\vec{c},\vec{a}\rbrack\) ekanini isbotlang.
 \\

\end{tabular}
\vspace{1cm}


\begin{tabular}{m{17cm}}
\textbf{27-variant}
\newline

T1. 
Koordinatalari bilan berilgan vektorlarning skalyar, vektor va aralash ko‘paytmalari. \\
T2. 
Nuqtadan tekislikkacha, fazoda nuqtadan to‘g‘ri chiziqqacha va ayqash to‘g‘ri chiziqlar orasidagi masofa. \\
A1. 
Bir jinsli elementdan yasalgan qatorning uchlari
$A (3;-5) $ va $B (-1;1) $ nuqtalarda joylashgan. Uning og‘irligi
markazi koordinatasini aniqlang.
 \\
A2. 
$5x+3y-7=0$, $x-2y-4=0$, $3x-y+3=0$
to‘g‘ri chiziqlar bir nuqtada kesishishadimi?
 \\
A3. 
$\overrightarrow{a} = \{ 2; - 4;4\}$ va $\overrightarrow{b} = \{ - 3;2;6\}$
vektorlar hosil qilgan burchak kosinusini hisoblang.
 \\
B1. 
Parallelogrammning uchta uchi \(A (3;7),\ B (2;-3) \) va
\(C (-1;4) \) nuqtalarda joylashgan. $B$ uchidan $AC$
tomonidan tushirilgan balandlik uzunligini hisoblang.
 \\
B2. 
Quyidagi har bir to‘g‘ri chiziqlar jufti uchun, ularga parallel
bo‘lib, aynan o‘rtasidan o‘tuvchi to‘g‘ri tenglamani tuzing: $3x-2y-3=0$, $3x-2y-17=0$.
 \\
B3. 
$\vec{a}$ va $\vec{b}$ vektorlar $\varphi = 2\pi/3$ burchak hosil qiladi. $|\vec{a}| = 3,|\vec{b}| = 4$ ekani ma’lum. Hisoblang:
$\left(3\vec{a} - 2\vec{b},\vec{a} + 2\vec{b} \right) $.
 \\
C1. \(A (4;2) \) nuqta orqali, ikkita koordinata o‘qlariga
urinma doira o‘tkazildi. Uning markazi $C$ ni va radiusi
$R$ ni toping.
 \\
C2. 
\(P (2; - 3) \) va \(Q (- 8; - 2) \) nuqtalardan
oraliqlarining yig‘indisi eng kichik bo‘lgan, abssissa o‘qida joylashgan
nuqtani toping.
 \\
C3. 
Ayniyatni isbotlang: \((\lbrack\vec{a} + \vec{b},\vec{b} + \vec{c}\rbrack,\vec{c} + \vec{a}) = 2 (\lbrack\vec{a},\vec{b}\rbrack,\vec{c}) \).
 \\

\end{tabular}
\vspace{1cm}


\begin{tabular}{m{17cm}}
\textbf{28-variant}
\newline

T1. 
Vektorning koordinatalari.
 \\
T2. 
Fazoviy to‘g‘ri chiziqning tenglamalari. To‘g‘ri chiziqlarning o‘zaro joylashishi.
 \\
A1. 
$A (2;2) $, $B (-1;6) $, $C (-5;3) $ va $D (-2;-1) $
nuqtalari kvadrat uchlari ekanini isbotlang.
 \\
A2. 
$P (12;6)$ nuqtadan o‘tib, koordinata burchagidan
yuzi 150 ga teng uchburchak kesib oladigan to‘g‘ri chiziqlarning
tenglamasini tuzing.
 \\
A3. 
$\overrightarrow{a}
= \{ 1; - 1;3\}, \ \ \ \ \ \overrightarrow{b} = \{ - 2;1\}$, $\overrightarrow{c} = \{3; -2;5\}$ vektorlar berilgan. Hisoblang:
$ (\lbrack\overrightarrow{a},\overrightarrow{b}\rbrack,\overrightarrow{c}) $.
 \\
B1. 
To‘rtburchakning uchlari
\(A (-3;12),\ B (3;-4),\ C (5;-4) \) va \(D (5;8) \) berilgan. Shu
to‘rtburchakning $AC$ diagonali $BD$ diagonali qanday
nisbatda bo'lishini aniqlang.
 \\
B2. 
Uchlari \(A (4;-4),\ B (6;-1) \) va \(C (-1;2) \)
nuqtalarida joylashgan bir jinsli plastinkadan yasalgan uchburchakning
og‘irlik markazidan o‘tib, quyida berilgan
\(\alpha (2x+3y-1) +\beta (3x-4y-3) =0\) to‘g‘ri chiziqlar dasturiga
tegishli to‘g‘ri chiziqning tenglamasini tuzing. \\
B3. 
$|\vec{a}| = 3,|\vec{b}| = 5$ berilgan. $\alpha$ ning qanday qiymatida $\vec{a} + \alpha\vec{b}$, $\vec{a} - \alpha\vec{b}$ vektorlar o‘zaro perpendikulyar bo‘lishini aniqlang.
 \\
C1. 
Ikkita uchi \(A (2;1) \) va \(B (5; 3) \) nuqtalarida, va
diagonallarining kesishish nuqtasi ordinata o‘qiga tegishli
parallelogrammning yuzi \(S = 17\) ga teng. Qolgan ikki uchining
koordinatalarini aniqlang. \\
C2. 
\(N (- 4; 7) \) nuqtaning, \(A (2;0) \) va \(B (- 3;5) \)
nuqtalardan o‘tgan to‘g‘ri chiziqqa nisbatan simmetrik nuqtani toping.
 \\
C3. 
\(\lbrack\vec{a},\vec{b}\rbrack^{2} < {\vec{a}}^{2}{\vec{b}}^{2}\) ekanini isbotlang; qanday holda bu yerda tenglik ishorasi bo‘ladi?
 \\

\end{tabular}
\vspace{1cm}


\begin{tabular}{m{17cm}}
\textbf{29-variant}
\newline

T1. 
Vektor tushunchasi. Vektorlar ustida chiziqli amallar.
 \\
T2. 
Tekislikda to‘g‘ri chiziqning tenglamalari.
 \\
A1. 
$ABCD$-parallelogrammning uchta uchi
$A (2;3) $, $B (4;-1) $ va $C (0;5) $ berilgan. To‘rtinchi $D$
cho‘qqisini toping.
 \\
A2. 
Umumiy tenglama bilan berilgan to‘g‘ri chiziqlarning
o‘zaro joylashuvini aniqlang, agar kesishadigan bo‘lsa kesishish nuqtasini
toping: $2x-5y+1=0, 6x-15y+3=0$.
 \\
A3. 
$\alpha$
qanday qiymatlarida 
$\overrightarrow{a} = \alpha\overrightarrow{i} - 3\overrightarrow{j} + 2\overrightarrow{k}$
va
$\overrightarrow{b} = \overrightarrow{i} + 2\overrightarrow{j} - \alpha\overrightarrow{k}$
vektorlar o‘zaro perpendikulyar bo‘lishini aniqlang.
 \\
B1. 
Ikkita qarama-qarshi uchlari $P (3; -4) $ va $Q (l;2) $ nuqtalarda joylashgan rombaning tomon uzunligi \(5\sqrt{2}\). Shu romb balandligining uzunligini hisoblang.
 \\
B2. 
Koordinata boshi, berilgan to‘g‘ri chiziqlarning:
\(3x+y-4=0\) va \(3x-2y+6=0\) kesishmasida hosil bo‘ladi
bo‘lgan o‘tkir yoki o‘tmas burchakka tegishli bo‘lishini aniqlang.
 \\
B3. 
$\vec{a} = \{ 2;1; - 1\}$ vektorga kollinear bo‘lgan va $\left(\vec{x},\vec{a} \right) = 3$ shartni qanoatlantiruvchi $\vec{x}$ vektorni toping.
 \\
C1. 
Uchburchakning uchlari
\(A (3; - 5),\ B (1; - 3),\ C (2; - 2) \) berilgan. $B$ uchi tashqi
burchagi bessektrisa uzunligini aniqlang.
 \\
C2. 
\(A (0;5) \) va \(B (5;2) \) nuqtalardan masofalarning
farqi eng katta bo‘lgan, \(3x - y - 2 = 0\) to‘g‘ri chiziqda joylashgan
nuqtani toping.
 \\
C3. 
Ayniyatni isbotlang: \((\lbrack\vec{a},\vec{b}\rbrack,\vec{c} + \lambda\vec{a} + \mu\vec{b}) = (\lbrack\vec{a},\vec{b}\rbrack,\vec{c}) \), bunda \(\lambda\) va \(\mu\) - ixtiyoriy sonlar. \\

\end{tabular}
\vspace{1cm}


\begin{tabular}{m{17cm}}
\textbf{30-variant}
\newline

T1. 
Vektorlarning vektor ko‘paytmasi va aralash ko‘paytmasi.
 \\
T2. 
Tekislikning tenglamalari. Tekisliklarning o‘zaro joylashishi.
 \\
A1. 
Ikkala uchi $A (3;1) $ va $B (1;-3) $ nuqtalarda, a
uchinchi $C$ uchi $Oy$ o‘qiga tegishli uchburchakning
yuzi $S=3$ ga teng. $C$ uchining koordinatalarini aniqlang.
 \\
A2. 
Umumiy tenglama bilan berilgan to‘g‘ri chiziqlarning
o‘zaro joylashuvini aniqlang, agar kesishadigan bo‘lsa kesishish nuqtasini
toping: $3x+y\sqrt{3}=0, x\sqrt{3}+3y-6=0$.
 \\
A3. 
Uchlari $A (1;2;1), B (3;-1;7) $ va $C (7;4;-2) $ bo‘lgan uchburchakning
ichki burchaklarini hisoblab toping. Bu uchburchakning teng yonli ekanligini isbotlang.
 \\
B1. 
Ikkala uchi \(A (3;1) \) va \(B (1;-3) \) nuqtalarda, va
og‘irlik markazi $Ox$ o‘qiga tegishli uchburchakning yuzi
\(S=3\) ga teng. Uchinchi $C$ uchining koordinatalarini aniqlang. \\
B2. 
Uchburchak uchlari \(A (1;0),\ B (5;-2),\ C (3;2) \)
koordinatalari bilan berilgan. Uchburchaklar tomonlarining va
medianalarining tenglamalarini tuzing.
 \\
B3. 
$\vec{a} = \{ 3; - 1; - 2\}$ va $\vec{b} = \{ 1;2; - 1\}$ vektorlar berilgan. Quyidagi vektor ko‘paytmalarning koordinatalarini toping:
$\left\lbrack \vec{a},\vec{b} \right\rbrack$.
 \\
C1. 
Uchburchakning uchlari
\(A (- 1; - 1),\ B (3;5),\ C (- 4;1) \) berilgan. $A$ uchi tashqi
burchak bissektrisasining, $BC$ tomonining davomi bilan kesishish
nuqtani toping.
 \\
C2. 
Uchburchakning uchlari \(A (1;-2),\ B (5; 4) \) va
\(C (-2;0) \) nuqtalarda joylashgan. $A$ uchidagi ichki va tashqi
burchaklari bissektrisalarining tenglamalarini tuzing.
 \\
C3. 
\(\vec{p} = \vec{b} (\vec{a},\vec{c}) - \vec{c} (\vec{a},\vec{b}) \) vektor \(\vec{a}\) vektorga perpendikulyar ekanini isbotlang.
 \\

\end{tabular}
\vspace{1cm}


\begin{tabular}{m{17cm}}
\textbf{31-variant}
\newline

T1. 
Chiziqli bog‘liq va chiziqli bog‘lanmagan vektorlar.
 \\
T2. 
Tekislik va to‘g‘ri chiziqlarning o‘zaro joylashishi.
 \\
A1. $M_1 (1; -2) $, $M_2 (2; 1) $ nuqtalar berilgan.
Quyidagi kesmalarning koordinata o‘qlariga proyeksiyalarini toping: $\overline{M_1M_2}$ \\
 \\
A2. 
$a$ va $b$ parametrlarining qanday qiymatlarida
$ax-2y-1=0$, $6x-4y-b=0$ to‘g‘ri chiziqlar parallel bo‘ladi?
 \\
A3. Vektor koordinata o‘qlari bilan quyidagi burchaklarni hosil qila oladimi:
$\alpha = 45^{{^\circ}},\beta = 60^{{^\circ}},\gamma = 120^{{^\circ}}$.
 \\
B1. 
Uchlari \(M (-1;3),\ N (1,2) \ \) va \(P (0;4) \)
nuqtalarida joylashgan uchburchakning ichki burchaklari o‘tkir burchak
ekanligini isbotlang.
 \\
B2. 
\(N (4;-5) \) nuqtadan o‘tib, $2x+5y-7=0$
to‘g‘ri chiziqlariga parallel to‘g‘ri chiziqlarning tenglamasini tuzing. Masalani burchaklik
koeffitsiyentni hisoblamasdan yeching.
 \\
B3. 
$\vec{a}$ va $\vec{b}$ vektorlar $\varphi = 2\pi/3$ burchak hosil qiladi. $|\vec{a}| = 3,|\vec{b}| = 4$ ekani ma’lum. Hisoblang:
$ (\vec{a} - \vec{b}) ^{2};$ 7) $ (3\vec{a} + 2\vec{b}) ^{2}$.
 \\
C1. 
Uchburchakning uchlari \(M_{1} (- 3;6),\ M_{2} (9; - 10) \)
va \(M_{3} (-5;4) \) berilgan. Shu uchburchakka tashqi chizilgan
aylana markazi $C$ va radiusi $R$ ni aniqlang.
 \\
C2. 
\(A (4;5) \) nuqta, diagonali \(7x - y - 8 = 0\) tenglama
bilan berilgan kvadratning bir uchi. Shu kvadratning tomonlari va
ikkinchi diagonalining tenglamasini tuzing.
 \\
C3. \(\vec{a} + \vec{b} + \vec{c} = 0\) shartni qanoatlantiruvchi birlik \(\vec{a},\ \vec{b}\) va \(\vec{c}\) vektorlar berilgan. Hisoblang: \(\left(\vec{a},\vec{b} \right) + \left(\vec{b},\vec{c} \right) + \left(\vec{c},\vec{a} \right) \).
 \\

\end{tabular}
\vspace{1cm}


\begin{tabular}{m{17cm}}
\textbf{32-variant}
\newline

T1. 
Vektorlarning skalyar ko‘paytmasi.
 \\
T2. 
Tekislikdagi to‘g‘ri chiziqlarning o‘zaro joylashishi.
 \\
A1. 
Kvadratning ikkita qarama-qarshi uchlari $P (3; 5) $ va
$Q (1; -3) $ berilgan. Uning yuzini hisoblang.
 \\
A2. 
5x-3y+15=0 to‘g‘ri chiziqning koordinata burchagidan
kesib olgan uchburchakning yuzini hisoblang.
 \\
A3. 
$\overrightarrow{a}$ va $\overrightarrow{b}$ vektorlar
$\varphi = \pi/6$ burchak hosil qiladi.
$|\overrightarrow{a}| = 6,|\overrightarrow{b}| = 5$ ekanini bilib,
$\left| \left\lbrack \overrightarrow{a},\overrightarrow{b} \right\rbrack \right|$ kattalikni hisoblang.
 \\
B1. 
To‘rtburchakning uchlari
\(A (-2;14),\ B (4;-2),\ C (6;-2) \) va \(D (6;10) \) berilgan. Shu
to‘rtburchakning $AC$ va $BD$ diagonallarining kesishishi
nuqtani toping.
 \\
B2. 
\(P (1;-2) \) nuqta va koordinatalar boshi, berilgan ikkita
to‘g‘ri yozing: $12x-5y-7=0, 3x+4y-8=0$.
kesishishidan hosil bo‘lgan bir xil burchakdami, qo‘shni burchakdami yoki vertikal
burchaklarda yotadimi?
 \\
B3. 
$\vec{a}$ va $\vec{b}$ vektorlar $\varphi = 2\pi/3$ burchak hosil qiladi. $|\vec{a}| = 3,|\vec{b}| = 4$ ekani ma’lum. Hisoblang:
${\vec{b}}^{2}$.
 \\
C1. \(A (4;2) \) nuqta orqali, ikkita koordinata o‘qlariga
urinma doira o‘tkazildi. Uning markazi $C$ ni va radiusi
$R$ ni toping.
 \\
C2. Ikkita uchi \(A (1; - 2),\ B (2;3) \) nuqtalarda joylashgan,
yuzi \(S = 8\) ga teng bo‘lgan uchburchakning uchinchi uchi
$C$ \(2x + y - 2 = 0\) to‘g‘ri chiziqqa tegishli. Shu $C$ uchining
koordinatasini aniqlang.
 \\
C3. 
\(\vec{a}+\vec{b}\) va \(\vec{a} - \vec{b}\) vektorlar kollinear bo‘lishi uchun \(\vec{a},\vec{b}\) vektorlar qanday shartni qanoatlantirishi kerak?
 \\

\end{tabular}
\vspace{1cm}


\begin{tabular}{m{17cm}}
\textbf{33-variant}
\newline

T1. 
Koordinatalari bilan berilgan vektorlarning skalyar, vektor va aralash ko‘paytmalari. \\
T2. Tekislikda va fazoda dekart koordinatalar sistemasini almashtirish.
 \\
A1. 
Uchburchakning uchlari $A (1;4) $, $B (3;-9) $, $C (-5;2) $
berilgan. $B$ uchidan o‘tkazilgan mediana uzunligini aniqlang.
 \\
A2. 
$m$ parametrining qanday qiymatlarida
$mx+ (2m+3) y+m+6=0$, $ (2m+1) x+ (m-1) y+m-2=0$ to‘g‘ri chiziqlar ordinata
o‘qida yotuvchi nuqtada kesishadi.
 \\
A3. 
Uchburchakning uchlari
$A (- 1; - 2;4) $, $B (- 4; - 2;0) $ va $C (3; -2;1) $. Uning $B$ uchidagi
ichki burchakni aniqlang.
 \\
B1. 
\(P (2;2) \) va \(Q (1;5) \) nuqtalar bilan teng uchta
bo‘lingan kesmaning uchlari $A$ va $B$ nuqtalarning
koordinatalarini aniqlang.
 \\
B2. 
$ABCD$ parallelogrammning ikkita qo‘shni uchlari
\(A (3,3),\ B (-1;7) \) va diagonallarining kesishish nuqtasi
\(E (2;-4) \) berilgan. Shu parallelogramm tomonlarining tenglamalarini
tuzing.
 \\
B3. 
$A (2; -1;2),B (1;2; 1) $ va $C (3;2;1)$ nuqtalar berilgan. Quyidagi vektor ko‘paytmalarning koordinatalarini toping:
$\lbrack\overline{AB},\overline{BC}\rbrack$.
 \\
C1. 
Ikkita uchi \(A (2;1) \) va \(B (5; 3) \) nuqtalarida, va
diagonallarining kesishish nuqtasi ordinata o‘qiga tegishli
parallelogrammning yuzi \(S = 17\) ga teng. Qolgan ikki uchining
koordinatalarini aniqlang. \\
C2. 
$ABC$ uchburchakning bir uchi \(C (4; - 1) \), va
ikkita bissektrisasining tenglamasi: \(x - 1 = 0\,\ x - y - 1 = 0\)
berilgan. Tomonlarining tenglamalarini tuzing.
 \\
C3. 
\(ABC\) uchburchakning tomonlari bilan mos keluvchi \(\vec{AB} = \vec{b}\) va \(\vec{AC} = \vec{c}\) vektorlar berilgan. Bu uchburchakning \(B\) uchidan tushirilgan \(BD\) balandligining \(\vec{b},\ \vec{c}\) bazis bo‘yicha yoyilmasini toping.
 \\

\end{tabular}
\vspace{1cm}


\begin{tabular}{m{17cm}}
\textbf{34-variant}
\newline

T1. Analitik geometriya fanining predmeti va metodlari.
 \\
T2. 
Tekislikda to‘g‘ri chiziqning tenglamalari.
 \\
A1. 
$ABCD$ parallelogrammning uchta uchi $A (3; -7) $,
$B (5; -7) $, $C (-2; 5) $ berilgan, to‘rtinchi uchi $D$,
$B$ uchiga qarama-qarshi. Shu parallelogrammning diagonallari
uzunliklarini aniqlang.
 \\
A2. 
Umumiy tenglama bilan berilgan to‘g‘ri chiziqlarning
o‘zaro joylashuvini aniqlang, agar kesishadigan bo‘lsa kesishish nuqtasini
toping: $14x-9y-24=0, 7x-2y-17=0$.
 \\
A3. 
Vektor koordinata o‘qlari bilan quyidagi burchaklarni hosil qila oladimi:
$\alpha = 45^{{^\circ}},\ \ \ \ \beta = 135^{{^\circ}},\ \gamma = 60^{{^\circ}}$.
 \\
B1. 
Ikkita nuqta berilgan \(M (2;2) \) va \(N (5;-2) \); abssissa o‘qida shunday $P$ nuqtani topingki, $MPN$ burchak to‘g‘ri burchak bo‘lsin.
 \\
B2. 
Umumiy tenglamasi \(2x-5y+4=0\) bo‘lgan to‘g‘ri
berilgan. \(M (-3,5) \) nuqtadan o‘tib, berilgan to‘g‘ri chiziqqa: a) parallel;
b) perpendikular bo‘lgan to‘g‘ri chiziqlar tenglamasini tuzing.
 \\
B3. 
$\vec{a}$ va $\vec{b}$ vektorlar $\varphi = 2\pi/3$ burchak hosil qiladi. $|\vec{a}| = 3,|\vec{b}| = 4$ ekani ma’lum. Hisoblang:
$ (\vec{a} + \vec{b}) ^{2}$.
 \\
C1. 
Ikki uchi \(A (2; - 3) \) va \(B (-5;1) \) nuqtalarda,
uchinchi uchi $C$ ordinata o‘qiga tegishli uchburchakning
medianalarining kesishish nuqtasi $M$ abssissa o‘qida yotadi.
$M$ va $C$ nuqtalarning koordinatalarini aniqlang.
 \\
C2. 
\(P (2;5) \) va \(Q (- 3;2) \) nuqtalardan masofalarning
farqi eng katta bo‘lgan, ordinata o‘qida joylashgan nuqtani toping.
 \\
C3. 
Ayniyatni isbotlang: \(\lbrack\vec{a},\vec{b}\rbrack^{2} + (\vec{a},\vec{b}) ^{2} = {\vec{a}}^{2}{\vec{b}}^{2}\).
 \\

\end{tabular}
\vspace{1cm}


\begin{tabular}{m{17cm}}
\textbf{35-variant}
\newline

T1. 
Vektorning koordinatalari.
 \\
T2. 
Fazoviy to‘g‘ri chiziqning tenglamalari. To‘g‘ri chiziqlarning o‘zaro joylashishi.
 \\
A1. 
Ikkala uchi $A (2;1) $ va $B (3;-2) $ nuqtalarda, va
uchinchi $C$ uchi $Ox$ o‘qiga tegishli bo‘lgan uchburchakning
yuzi $S=4$ ga teng. $C$ uchining koordinatalarini aniqlang. \\
A2. 
$P1$, $P2$, $P3$, $P4$, $P5$ nuqtalar
3x-2y-6=0 to‘g‘ri chiziqqa tegishli va abssissalari mos ravishda
4, 0, 2, -2, -6 ga teng. Ularning ordinatalarini toping.
 \\
A3. 
Tekislikda ikkita vektor
$\overrightarrow{p} = \{ 2; - 3\}$, $\overrightarrow{q} = \{ 1;2\}$.
$\overrightarrow{a} = \{9;4\}$ vektorning
$\overrightarrow{p},\ \overrightarrow{q}$ bazis bo‘yicha yoyilmasi topilsin.
 \\
B1. 
Uchburchakning uchlari
\(A\left(-\sqrt{3};1 \right),\ B (0;2) \) va
\(C\left(-2\sqrt{3};2 \right) \) nuqtalarda. Uning $A$
uchidagi tashqi burchakni toping.
 \\
B2. 
$ABC$ uchburchakning tomonlari:
\(AB:4x+3y-5=0,\ BC:x-3y+10=0,\ AC:x-2=0\) 
tenglamalari bilan berilgan. Uchlarining koordinatalarini aniqlang.
 \\
B3. 
$A (2; -1;2),B (1;2; 1) $ va $C (3;2;1) $ nuqtalar berilgan. Quyidagi vektor ko‘paytmalarning koordinatalarini toping:
$\lbrack\overline{BC} - 2\overline{CA},\overline{CB}\rbrack$. \\
C1. 
\(M{1} (1; 2) \) nuqta orqali, radiusi 5 ga teng,
$Ox$ o‘qiga urinma aylana o‘tkazildi. Shu aylananing markazi
$S$ ni aniqlang.
 \\
C2. 
\(Q (5; - 6) \) nuqtaning, \(A (3;8) \) va \(B (7;5) \)
nuqtalardan o‘tgan to‘g‘ri chiziqdagi proyeksiyasini toping.
 \\
C3. 
\(\vec{a}+\vec{b}\) va \(\vec{a} - \vec{b}\) vektorlar kollinear bo‘lishi uchun \(\vec{a},\vec{b}\) vektorlar qanday shartni qanoatlantirishi kerak?
 \\

\end{tabular}
\vspace{1cm}


\begin{tabular}{m{17cm}}
\textbf{36-variant}
\newline

T1. 
Chiziqli bog‘liq va chiziqli bog‘lanmagan vektorlar.
 \\
T2. 
Tekislikdagi to‘g‘ri chiziqlarning o‘zaro joylashishi.
 \\
A1. 
Uchburchak uchlarining koordinatalari berilgan
$A (1;-3) $, $B (3;-5) $ va $C (-5;7) $. Tomonlarining o‘rtalarini
aniqlang.
 \\
A2. 
$3x+2y=0$ to‘g‘ri chiziqning $k$ burchagi
koeffitsiyentini va $Oy$ o‘qidan kesib olgan kesmaning algebraik
qiymati $b$ ni aniqlang.
 \\
A3. Vektor koordinata o‘qlari bilan quyidagi burchaklarni hosil qila oladimi:
$\alpha = 45^{{^\circ}},\beta = 60^{{^\circ}},\gamma = 120^{{^\circ}}$.
 \\
B1. 
Uchburchakning uchlari \(A (5;0),\ B (0;1) \) va \(C (3;3) \)
nuqtalarida. Uning ichki burchaklarini toping.
 \\
B2. 
Doiraviy to‘rtburchakning uchlari
\(A (-2;-6),\ B (7;6),\ C (3;9) \) va \(D (-3;1) \) nuqtalarda
joylashgan. Diagonallarining kesishish nuqtasi topilsin.
 \\
B3. 
$\vec{a} = \{ 3; - 1; - 2\}$ va $\vec{b} = \{ 1;2; - 1\}$ vektorlar berilgan. Quyidagi vektor ko‘paytmalarning koordinatalarini toping:
$\left\lbrack 2\vec{a} + \vec{b},\vec{b} \right\rbrack$.
 \\
C1. 
Ikki uchi \(A (2; - 3) \) va \(B (-5;1) \) nuqtalarda,
uchinchi uchi $C$ ordinata o‘qiga tegishli uchburchakning
medianalarining kesishish nuqtasi $M$ abssissa o‘qida yotadi.
$M$ va $C$ nuqtalarning koordinatalarini aniqlang.
 \\
C2. 
\(A (0;5) \) va \(B (5;2) \) nuqtalardan masofalarning
farqi eng katta bo‘lgan, \(3x - y - 2 = 0\) to‘g‘ri chiziqda joylashgan
nuqtani toping.
 \\
C3. 
Ayniyatni isbotlang: \((\lbrack\vec{a} + \vec{b},\vec{b} + \vec{c}\rbrack,\vec{c} + \vec{a}) = 2 (\lbrack\vec{a},\vec{b}\rbrack,\vec{c}) \).
 \\

\end{tabular}
\vspace{1cm}


\begin{tabular}{m{17cm}}
\textbf{37-variant}
\newline

T1. Analitik geometriya fanining predmeti va metodlari.
 \\
T2. 
Tekislik va to‘g‘ri chiziqlarning o‘zaro joylashishi.
 \\
A1. 
Bir jinsli beshburchakli plastinkaning uchlari berilgan:
$A (2;3), \ B (0;6), \ C (-1;5), \ D (0;1) $ va $E (1;1) $. Uning og‘irligi
markazi koordinatalarini aniqlang.
 \\
A2. 
$y-3=0$ to‘g‘ri chiziqning $k$ burchagi
koeffitsiyentini va $Oy$ o‘qidan kesib olgan kesmaning algebraik
qiymati $b$ ni aniqlang.
 \\
A3. 
Uchburchakning uchlari
$A (3;2; 3) $, $B (5;1; - 1) $ va $C (1; -2;1) $. Uning $A$ uchidagi tashqi burchagi aniqlansin.
 \\
B1. 
Uchlari $A_1 (1; 1), A_2 (2; 3) $ va $A (5;-1) $
nuqtalarida joylashgan uchburchakning to‘g‘ri burchakli ekanini isbotlang.
 \\
B2. 
\(2x+y-2=0\) va \(x-5y-3=0\)
to‘g‘ri chiziqlarning kesishish nuqtasidan o‘tib (bu nuqtani aniqlamay), uchlari
\(A (-1;-4) \) va \(B (5;-6) \) nuqtalarda joylashgan kesmaning
to‘g‘ri o‘rtasidan o‘tuvchi to‘g‘ri chiziqning tenglamasini tuzing.
 \\
B3. 
$\vec{a}$ va $\vec{b}$ vektorlar $\varphi = 2\pi/3$ burchak hosil qiladi. $|\vec{a}| = 3,|\vec{b}| = 4$ ekani ma’lum. Hisoblang:
$\left(\vec{a},\vec{b} \right) $.
 \\
C1. 
\(M{1} (1; 2) \) nuqta orqali, radiusi 5 ga teng,
$Ox$ o‘qiga urinma aylana o‘tkazildi. Shu aylananing markazi
$S$ ni aniqlang.
 \\
C2. 
\(A (-5;5) \) va \(B (-7;1) \) nuqtalardan
masofalarining yig‘indisi eng kichik bo‘lgan \(2x - y - 5 = 0\) to‘g‘ri chiziqda
joylashgan nuqtani toping.
 \\
C3. 
\(\vec{p} = \vec{b} - \frac{\vec{a} (\vec{a},\vec{b}) }{{\vec{a}}^{2}}\) vektor \(\vec{a}\) vektorga perpendikulyar ekanini isbotlang.
 \\

\end{tabular}
\vspace{1cm}


\begin{tabular}{m{17cm}}
\textbf{38-variant}
\newline

T1. 
Vektor tushunchasi. Vektorlar ustida chiziqli amallar.
 \\
T2. 
Nuqtadan to‘g‘ri chiziqqacha bo‘lgan masofa. To‘g‘rilar dastasi.
 \\
A1. 
Bir jinsli to‘rtburchakli plastinkaning uchlari berilgan:
$A (2;1), \ B (5;3), \ C (-1;7) $ va $D (-7;5) $. Uning og‘irlik markazi
koordinatalarini aniqlang.
 \\
A2. 
$P (8;6) $ nuqtadan o‘tib, koordinata burchagidan
yuzi 12 ga teng uchburchak kesib oladigan to‘g‘ri chiziqlarning tenglamasini
tuzing.
 \\
A3. 
Vektor koordinata o‘qlari bilan quyidagi burchaklarni hosil qila oladimi:
$\alpha = 45^{{^\circ}},\ \ \ \ \beta = 135^{{^\circ}},\ \gamma = 60^{{^\circ}}$.
 \\
B1. 
Uchburchakning uchlari
\(A (3;-5),\ B (-3;3),\ C (-1;-2) \) berilgan. $A$ uchining ichki qismi
burchakli bessektrisaning uzunligini aniqlang.
 \\
B2. 
Berilgan parallel to‘g‘ri chiziqlardan teng masofada yotuvchi
nuqtalarning geometrik o‘rni tenglamasini tuzing: $2x+y+7=0, 2x+y-3=0$.
 \\
B3. 
$\vec{a}$ va $\vec{b}$ vektorlar o‘zaro perpendikulyar; $\vec{c}$ vektor ular bilan $\pi/3$ ga teng bo‘lgan burchaklar hosil qiladi; $|\vec{a}| = 3$, $|\vec{b}| = 5,\ |\vec{c}| = 8$ ekani ma’lum, quyidagilarni hisoblang:
$ (\vec{a} + \vec{b} + \vec{c}) ^{2}$.
 \\
C1. 
Uchburchakning uchlari
\(A (- 1; - 1),\ B (3;5),\ C (- 4;1) \) berilgan. $A$ uchi tashqi
burchak bissektrisasining, $BC$ tomonining davomi bilan kesishish
nuqtani toping.
 \\
C2. 
$ABC$ uchburchakning bir uchi \(B (- 4; - 5) \),
va ikki balandligining tenglamasi:
\(3x + 8y + 13 = 0\,\ 5x + 3y - 4 = 0\) berilgan. Tomonlarning
tenglamalarni tuzing.
 \\
C3. 
\(\vec{p} = \vec{b} (\vec{a},\vec{c}) - \vec{c} (\vec{a},\vec{b}) \) vektor \(\vec{a}\) vektorga perpendikulyar ekanini isbotlang.
 \\

\end{tabular}
\vspace{1cm}


\begin{tabular}{m{17cm}}
\textbf{39-variant}
\newline

T1. 
Koordinatalari bilan berilgan vektorlarning skalyar, vektor va aralash ko‘paytmalari. \\
T2. 
Nuqtadan tekislikkacha, fazoda nuqtadan to‘g‘ri chiziqqacha va ayqash to‘g‘ri chiziqlar orasidagi masofa. \\
A1. 
Uch uchi $A (-2;3), \ B (4;-5) $ va
$C (-3;1)$ nuqtalarda joylashgan parallelogrammning yuzini aniqlang.
 \\
A2. 
$M (-3;8) $ nuqtadan o‘tib, koordinata o‘qlaridan
teng kesmalarni kesib oladigan to‘g‘ri chiziqlarning tenglamasini tuzing.
 \\
A3. 
To‘rtburchakning uchlari berilgan:
$A (1; - 2;2) $, $B (1;4;0),C (- 4;1;1) $ va $D (- 5; -5;3) $. Uning diagonallari $AC$ va $BD$ o‘zaro
perpendikulyarligini isbotlang.
 \\
B1. 
To‘g‘ri \(M_{1} (-12;-13) \) va \(M_{2} (-2;-5) \)
nuqtalaridan o‘tadi. Shu to‘g‘ri chiziqda abssissasi 3 ga teng nuqtani toping.
 \\
B2. 
Qirralari
\(7x+y+31=0,\ 3x+4y-1=0,\ x-7y-17=0\) tenglamalar
bilan berilgan uchburchakning teng yonli ekanini isbotlang.
Masalani uchburchakning
burchaklarini topish orqali yeching.
 \\
B3. 
$\vec{a}$ va $\vec{b}$ vektorlar $\varphi = 2\pi/3$ burchak hosil qiladi. $|\vec{a}| = 3,|\vec{b}| = 4$ ekani ma’lum. Hisoblang:
${\vec{a}}^{2}$.
 \\
C1. \(A (4;2) \) nuqta orqali, ikkita koordinata o‘qlariga
urinma doira o‘tkazildi. Uning markazi $C$ ni va radiusi
$R$ ni toping.
 \\
C2. 
Uchburchakning \(A (- 3; - 2),\ B (5; - 4),\ C (- 1;3) \)
uchlaridan o‘tib, qarama-qarshi tomonga parallel to‘g‘ri chiziqlarning tenglamalarini
tuzing.
 \\
C3. 
\(\vec{a},\ \vec{b}\) va \(\vec{c}\) vektorlar \(\vec{a} + \vec{b} + \vec{c} = 0\) shartni qanoatlantiradi. \(\lbrack\vec{a},\vec{b}\rbrack = \lbrack\vec{b},\vec{c}\rbrack = \lbrack\vec{c},\vec{a}\rbrack\) ekanini isbotlang.
 \\

\end{tabular}
\vspace{1cm}


\begin{tabular}{m{17cm}}
\textbf{40-variant}
\newline

T1. 
Vektorlarning vektor ko‘paytmasi va aralash ko‘paytmasi.
 \\
T2. 
Tekislikning tenglamalari. Tekisliklarning o‘zaro joylashishi.
 \\
A1. 
Uchlari $M_1 (-3;2) $, $M_2 (5;-2) $ va $M_3 (1;3) $
nuqtalarida joylashgan uchburchaklarning yuzini hisoblang.
 \\
A2. 
Umumiy tenglama bilan berilgan to‘g‘ri chiziqlarning
o‘zaro joylashuvini aniqlang, agar kesishadigan bo‘lsa kesishish nuqtasini
toping: $3x+y\sqrt{3}=0, x\sqrt{3}+3y-6=0$.
 \\
A3. 
Berilgan: $\overrightarrow{a}| = 10,|\overrightarrow{b}| = 2$ va
$\left(\overrightarrow{a},\overrightarrow{b} \right) = 12$. Hisoblang
$\left| \left\lbrack \overrightarrow{a},\overrightarrow{b} \right\rbrack \right|$.
 \\
B1. Ikkita qarama-qarshi uchlari \(P (4;9) \) va \(Q (-2; 1) \) nuqtalarida joylashgan romning tomon uzunligi \(5\sqrt{10}\). Shu
romba yuzini hisoblang.
 \\
B2. 
Quyida berilgan to‘g‘ri chiziqlar juftlarining qaysilari
perpendikular ekanini aniqlang: $4x+y+6=0, 2x-8y-13=0$.
 \\
B3. 
$\vec{a} = \{ 6; - 8; - 7,5\}$ vektorga kollinear bo‘lgan $\vec{x}$ vektor $Oz$ o‘qi bilan o‘tkir burchak hosil qiladi. $|\vec{x}| = 50$ ekanini bilgan holda uning koordinatalarini toping.
 \\
C1. 
Ikkita uchi \(A (2;1) \) va \(B (5; 3) \) nuqtalarida, va
diagonallarining kesishish nuqtasi ordinata o‘qiga tegishli
parallelogrammning yuzi \(S = 17\) ga teng. Qolgan ikki uchining
koordinatalarini aniqlang. \\
C2. 
\(N (2; 5) \) nuqtaning \(9x - 7y + 30 = 0\) to‘g‘ri chizig‘iga
nisbatan simmetrik nuqtani toping.
 \\
C3. 
Ayniyatni isbotlang: \((\lbrack\vec{a},\vec{b}\rbrack,\vec{c} + \lambda\vec{a} + \mu\vec{b}) = (\lbrack\vec{a},\vec{b}\rbrack,\vec{c}) \), bunda \(\lambda\) va \(\mu\) - ixtiyoriy sonlar. \\

\end{tabular}
\vspace{1cm}


\begin{tabular}{m{17cm}}
\textbf{41-variant}
\newline

T1. 
Vektorlarning skalyar ko‘paytmasi.
 \\
T2. 
Fazoviy to‘g‘ri chiziqning tenglamalari. To‘g‘ri chiziqlarning o‘zaro joylashishi.
 \\
A1. 
Ikkita uchi $A (-3; 2) $ va $B (1; 6) $ nuqtalarda
joylashgan muntazam uchburchakning yuzini hisoblang.
 \\
A2. 
$5x-y+3=0$ to‘g‘ri chiziqning $k$ burchagi
koeffitsiyentini va $Oy$ o‘qidan kesib olgan kesmaning algebraik
qiymati $b$ ni aniqlang.
 \\
A3. 
$\overrightarrow{a} = \{ 2; - 4;4\}$ va $\overrightarrow{b} = \{ - 3;2;6\}$
vektorlar hosil qilgan burchak kosinusini hisoblang.
 \\
B1. 
Abssissa o‘qida shunday $M$ nuqtani topingki,
\(N (2;-3) \) nuqtadan uzoqligi 5 ga teng bo‘lgan.
 \\
B2. 
$ABCD$ parallelogrammning ikkita qo‘shni uchlari
\(A (3,3),\ B (-1;7) \) va diagonallarining kesishish nuqtasi
\(E (2;-4) \) berilgan. Shu parallelogramm tomonlarining tenglamalarini
tuzing.
 \\
B3. 
$\vec{a}$ va $\vec{b}$ vektorlar o‘zaro perpendikulyar. $|\vec{a}| = 3,|\vec{b}| = 4$ ekani ma’lum, quyidagilarni hisoblang:
$|\lbrack\vec{a} + \vec{b},\vec{a} - \vec{b}\rbrack|$.
 \\
C1. 
Uchburchakning uchlari
\(A (3; - 5),\ B (1; - 3),\ C (2; - 2) \) berilgan. $B$ uchi tashqi
burchagi bessektrisa uzunligini aniqlang.
 \\
C2. 
$ABC$ uchburchakda \(AB:5x-3y+2=0\)
tomonining, shuningdek \(AN:4x - 3y + 1 = 0,\ BN:7x + 2y - 22 = 0\)
balandliklarining tenglamalari berilgan. Shu uchburchakning qolgan ikkita
tomonining va uchinchi balandligining tenglamalarini tuzing.
 \\
C3. 
\(ABC\) uchburchakning tomonlari bilan mos keluvchi \(\vec{AB} = \vec{b}\) va \(\vec{AC} = \vec{c}\) vektorlar berilgan. Bu uchburchakning \(B\) uchidan tushirilgan \(BD\) balandligining \(\vec{b},\ \vec{c}\) bazis bo‘yicha yoyilmasini toping.
 \\

\end{tabular}
\vspace{1cm}


\begin{tabular}{m{17cm}}
\textbf{42-variant}
\newline

T1. 
Vektorning koordinatalari.
 \\
T2. Tekislikda va fazoda dekart koordinatalar sistemasini almashtirish.
 \\
A1. 
Parallelogrammning ikkita qo‘shni uchlari $A (-3;5) $, $B (1;7) $
va dioganallarining kesishish nuqtasi $M (1;1)$ berilgan. Qolgan ikki
cho‘qqisini aniqlang.
 \\
A2. 
$x+2y-17=0$, $2x-y+1=0$, $x+2y-3=0$
to‘g‘ri chiziqlar bir nuqtada kesishishadimi?
 \\
A3. 
Vektor koordinata o‘qlari bilan quyidagi burchaklarni hosil qilishi
mumkinmi: $\alpha = 90^{{^\circ}},\ \beta = 150^{{^\circ}}$,
$\gamma = 60^{{^\circ}}?$
 \\
B1. 
Uchburchakning uchlari \(A (2;-5),\ B (1;-2),\ C (4;7) \)
berilgan. $AC$ tomoni bilan $B$ uchining ichki burchagi
bissektrisasining kesishish nuqtasini toping.
 \\
B2. 
To‘g‘ri to‘rtburchakning bir uchi \(A (2;-3) \), va ikkita tarafining
ning tenglamalari \(2x+3y+9=0,\ 3x-2y-7=0\)
berilgan. Qolgan ikki tomonning tenglamalarini tuzing.
 \\
B3. 
$\vec{a}$ va $\vec{b}$ vektorlar o‘zaro perpendikulyar; $\vec{c}$ vektor ular bilan $\pi/3$ ga teng bo‘lgan burchaklar hosil qiladi; $|\vec{a}| = 3$, $|\vec{b}| = 5,\ |\vec{c}| = 8$ ekani ma’lum, quyidagilarni hisoblang:
$\left(3\vec{a} - 2\vec{b},\vec{b} + 3\vec{c} \right) $.
 \\
C1. 
Uchburchakning uchlari \(M_{1} (- 3;6),\ M_{2} (9; - 10) \)
va \(M_{3} (-5;4) \) berilgan. Shu uchburchakka tashqi chizilgan
aylana markazi $C$ va radiusi $R$ ni aniqlang.
 \\
C2. 
Uchburchakning uchlari \(A (1;-2),\ B (5; 4) \) va
\(C (-2;0) \) nuqtalarda joylashgan. $A$ uchidagi ichki va tashqi
burchaklari bissektrisalarining tenglamalarini tuzing.
 \\
C3. 
\(\vec{a} + \vec{b}\) vektor \(\vec{a} - \vec{b}\) vektorga perpendikulyar bo‘lishi uchun \(\vec{a}\) va \(\vec{b}\) vektorlar qanday shartlarni qanoatlantirishi kerak?
 \\

\end{tabular}
\vspace{1cm}


\begin{tabular}{m{17cm}}
\textbf{43-variant}
\newline

T1. 
Vektor tushunchasi. Vektorlar ustida chiziqli amallar.
 \\
T2. 
Tekislik va to‘g‘ri chiziqlarning o‘zaro joylashishi.
 \\
A1. 
$A (1;-3) $ va $B (4;3) $ nuqtalarni tutashtiruvchi
kesma teng uch bo‘lakka bo‘lindi. Bo‘luvchi nuqtalarning koordinatalarini
aniqlang.
 \\
A2. 
$3x+2y=0$ to‘g‘ri chiziqning $k$ burchagi
koeffitsiyentini va $Oy$ o‘qidan kesib olgan kesmaning algebraik
qiymati $b$ ni aniqlang.
 \\
A3. 
Berilgan: $\overrightarrow{a}| = 3,|\overrightarrow{b}| = 26$ va
$\lbrack\overrightarrow{a},\overrightarrow{b}\rbrack| = 72$. Hisoblang
$\left(\overrightarrow{a},\overrightarrow{b} \right) $.
 \\
B1. 
To‘g‘ri \(A (5;2) \) va \(B (-4; -7) \) nuqtalaridan o‘tadi.
Shu to‘g‘ri chiziqning ordinata o‘qi bilan kesishish nuqtasini toping.
 \\
B2. 
$ABC$ uchburchakning tomonlari:
\(AB:4x+3y-5=0,\ BC:x-3y+10=0,\ AC:x-2=0\) 
tenglamalari bilan berilgan. Uchlarining koordinatalarini aniqlang.
 \\
B3. 
$a$ va $b$ vektorlar $\varphi = \pi/6$ burchak hosil qiladi; $|a| = \sqrt{3},|b| = 1$ ekani ma’lum. $p = a + b$ va $q = a - b$ vektorlar orasidagi $\alpha$ burchakni hisoblang.
 \\
C1. 
Ikki uchi \(A (2; - 3) \) va \(B (-5;1) \) nuqtalarda,
uchinchi uchi $C$ ordinata o‘qiga tegishli uchburchakning
medianalarining kesishish nuqtasi $M$ abssissa o‘qida yotadi.
$M$ va $C$ nuqtalarning koordinatalarini aniqlang.
 \\
C2. 
Uchburchak tomonlarining o‘rtalari
\(M (5;3),\ N (3; - 4),\ E (2;1) \) nuqtalarda joylashgan. Tomonlarning
tenglamalarni tuzing.
 \\
C3. \(\vec{a} + \vec{b} + \vec{c} = 0\) shartni qanoatlantiruvchi birlik \(\vec{a},\ \vec{b}\) va \(\vec{c}\) vektorlar berilgan. Hisoblang: \(\left(\vec{a},\vec{b} \right) + \left(\vec{b},\vec{c} \right) + \left(\vec{c},\vec{a} \right) \).
 \\

\end{tabular}
\vspace{1cm}


\begin{tabular}{m{17cm}}
\textbf{44-variant}
\newline

T1. Analitik geometriya fanining predmeti va metodlari.
 \\
T2. 
Tekislikning tenglamalari. Tekisliklarning o‘zaro joylashishi.
 \\
A1. 
$A (4;2) $, $B (7;-2) $ va $C (1;6) $ nuqtalar bir jinsli
simdan yasalgan uchburchak uchlari. Shu uchburchakning og‘irligi
 \\
A2. 
Umumiy tenglama bilan berilgan to‘g‘ri chiziqlarning
o‘zaro joylashuvini aniqlang, agar kesishadigan bo‘lsa kesishish nuqtasini
toping: $x\sqrt{2}+12=0, 4x+24\sqrt{2}=0$.
 \\
A3. 
$\overrightarrow{a}$ va $\overrightarrow{b}$ vektorlar
$\varphi = \pi/6$ burchak hosil qiladi.
$|\overrightarrow{a}| = 6,|\overrightarrow{b}| = 5$ ekanini bilib,
$\left| \left\lbrack \overrightarrow{a},\overrightarrow{b} \right\rbrack \right|$ kattalikni hisoblang.
 \\
B1. 
Ordinata o‘qida shunday $M$ nuqtani toping.
\(N (-8;13) \) nuqtadan uzoqligi 17 ga teng bo‘lgan.
 \\
B2. 
Quyidagi har bir to‘g‘ri chiziqlar jufti uchun, ularga parallel
bo‘lib, aynan o‘rtasidan o‘tuvchi to‘g‘ri tenglamani tuzing: $3x-2y-3=0$, $3x-2y-17=0$.
 \\
B3. 
$\vec{a}$ va $\vec{b}$ vektorlar $\varphi = 2\pi/3$ burchak hosil qiladi. $|\vec{a}| = 1,|\vec{b}| = 2$ ekanini bilib, quyidagilarni hisoblang:
$\lbrack 2\overrightarrow{a} + \overrightarrow{b},\overrightarrow{a} + 2\overrightarrow{b}\rbrack^{2}$.
 \\
C1. 
\(M{1} (1; 2) \) nuqta orqali, radiusi 5 ga teng,
$Ox$ o‘qiga urinma aylana o‘tkazildi. Shu aylananing markazi
$S$ ni aniqlang.
 \\
C2. 
\(P (2; - 3) \) va \(Q (- 8; - 2) \) nuqtalardan
oraliqlarining yig‘indisi eng kichik bo‘lgan, abssissa o‘qida joylashgan
nuqtani toping.
 \\
C3. 
\(\lbrack\vec{a},\vec{b}\rbrack^{2} < {\vec{a}}^{2}{\vec{b}}^{2}\) ekanini isbotlang; qanday holda bu yerda tenglik ishorasi bo‘ladi?
 \\

\end{tabular}
\vspace{1cm}


\begin{tabular}{m{17cm}}
\textbf{45-variant}
\newline

T1. 
Koordinatalari bilan berilgan vektorlarning skalyar, vektor va aralash ko‘paytmalari. \\
T2. 
Tekislikdagi to‘g‘ri chiziqlarning o‘zaro joylashishi.
 \\
A1. 
Uchlari $A (2;-3) $, $B (3;2) $ va $C (-2;5) $
nuqtalarida joylashgan uchburchaklarning yuzini hisoblang.
 \\
A2. 
$3x-y+2=0$, $4x-5y+5=0$, $2x+3y-1=0$
to‘g‘ri chiziqlar bir nuqtada kesishishadimi?
 \\
A3. 
Agar \(a = \{ 3; - 2;1\},\ \ \ \ \ b = \{ 2;1;2\},\ \ \ \ c = \{ 3; - 1; - 2\}\) bo‘lsa, $\overrightarrow{a}, \overrightarrow{b}, \overrightarrow{c}$ vektorlar komplanar bo‘lishini tekshiring.
 \\
B1. 
To‘g‘ri chiziq \(A (7;-3) \) va \(B (23;-6) \) nuqtalardan o‘tadi.
Shu to‘g‘ri chiziqning abssissa o‘qi bilan kesishish nuqtasini toping.
 \\
B2. 
\(A (4;-5) \) nuqtadan o‘tib, \(B (-2;3) \) nuqtaga
gacha masofasi 12 ga teng bo‘lgan to‘g‘ri chiziqlarning tenglamasini tuzing.
 \\
B3. 
$\vec{a}$ va $\vec{b}$ vektorlar $\varphi = 2\pi/3$ burchak hosil qiladi. $|\vec{a}| = 1,|\vec{b}| = 2$ ekanini bilib, quyidagilarni hisoblang:
$\lbrack\vec{a},\vec{b}\rbrack^{2}$.
 \\
C1. 
Ikkita uchi \(A (2;1) \) va \(B (5; 3) \) nuqtalarida, va
diagonallarining kesishish nuqtasi ordinata o‘qiga tegishli
parallelogrammning yuzi \(S = 17\) ga teng. Qolgan ikki uchining
koordinatalarini aniqlang. \\
C2. 
Uchburchakning ikki uchi \(A (6;4),\ B (- 10;2) \), va
balandliklarining kesishish nuqtasi \(N (5;2) \) berilgan. Uchinchi $C$
uchining koordinatalarini toping.
 \\
C3. 
\(ABC\) uchburchakning tomonlari bilan mos keluvchi \(\vec{AB} = \vec{b}\) va \(\vec{AC} = \vec{c}\) vektorlar berilgan. Bu uchburchakning \(B\) uchidan tushirilgan \(BD\) balandligining \(\vec{b},\ \vec{c}\) bazis bo‘yicha yoyilmasini toping.
 \\

\end{tabular}
\vspace{1cm}


\begin{tabular}{m{17cm}}
\textbf{46-variant}
\newline

T1. 
Vektorlarning vektor ko‘paytmasi va aralash ko‘paytmasi.
 \\
T2. 
Nuqtadan to‘g‘ri chiziqqacha bo‘lgan masofa. To‘g‘rilar dastasi.
 \\
A1. 
$M (2;-1) $, $N (-1;4) $ va $P (-2;2) $ nuqtalar
uchburchak tomonlarining o‘rtalari. Uchlarining koordinatalarini
aniqlang.
 \\
A2. Berilgan $M_1 (3; 1) $, $M_2 (2; 3) $, $M_3 (6; 3) $,
$M_4 (-3;-3) $. $M_5 (3;-1) $, $M_6 (-2; 1) $ nuqtalarning qaysilari
$2x-3y-3 = 0$ to‘g‘ri chiziqqa tegishli va qaysilari tegishli
emas.
 \\
A3. 
Uchburchakning uchlari
$A (- 1; - 2;4) $, $B (- 4; - 2;0) $ va $C (3; -2;1) $. Uning $B$ uchidagi
ichki burchakni aniqlang.
 \\
B1. 
Bir to‘g‘ri chiziqqa tegishli \(A (1;-1),\ B (3;3) \) va
\(C (4;5) \) nuqtalar berilgan. Har bir nuqtaning, qolgan ikki nuqta orqali aniqlanuvchi kesmani bo‘lish nisbati $\lambda$ ni aniqlang.
 \\
B2. 
Berilgan \(3x-4y-10=0\) to‘g‘ri chiziqqa parallel va undan
$d=3$ masofada yotuvchi to‘g‘ri chiziqlarning tenglamasini tuzing.
 \\
B3. 
$\vec{a}$ va $\vec{b}$ vektorlar o‘zaro perpendikulyar. $|\vec{a}| = 3,|\vec{b}| = 4$ ekani ma’lum, quyidagilarni hisoblang:
$|\lbrack 3\vec{a} - \vec{b},\vec{a}-2\vec{b}\rbrack|$.
 \\
C1. 
Uchburchakning uchlari
\(A (- 1; - 1),\ B (3;5),\ C (- 4;1) \) berilgan. $A$ uchi tashqi
burchak bissektrisasining, $BC$ tomonining davomi bilan kesishish
nuqtani toping.
 \\
C2. 
Uchburchakning uchlari
\(A (3;2),\ B (- 4;4),\ C (- 2; 5) \) koordinatalari bilan berilgan.
Balandliklarining tenglamasini tuzing.
 \\
C3. 
\(\vec{a} + \vec{b}\) vektor \(\vec{a} - \vec{b}\) vektorga perpendikulyar bo‘lishi uchun \(\vec{a}\) va \(\vec{b}\) vektorlar qanday shartlarni qanoatlantirishi kerak?
 \\

\end{tabular}
\vspace{1cm}


\begin{tabular}{m{17cm}}
\textbf{47-variant}
\newline

T1. 
Chiziqli bog‘liq va chiziqli bog‘lanmagan vektorlar.
 \\
T2. 
Nuqtadan tekislikkacha, fazoda nuqtadan to‘g‘ri chiziqqacha va ayqash to‘g‘ri chiziqlar orasidagi masofa. \\
A1. 
Bir jinsli elementdan yasalgan qatorning og‘irlik markazi
$M (1;4) $ nuqtada, bir uchi $P (-2;2) $ nuqtada joylashgan. Shu
qatorning ikkinchi uchi $Q$ ning koordinatalarini aniqlang.
 \\
A2. 
$2x-y+2=0$, $4x-2y+4=0$, $6x-3y+6=0$
to‘g‘ri chiziqlar bir nuqtada kesishishadimi?
 \\
A3. 
Agar \(a = \{ 2;3; - 1\}, \ \ \ \ b = \{ 1; - 1;3\}, \ \ \ \ c = \{ 1;9; - 11\}\) bo‘lsa, $\overrightarrow{a}, \overrightarrow{b}, \overrightarrow{c}$ vektorlar komplanar bo‘lishini tekshiring.
 \\
B1. 
Uchlari \(M_{1} (1;1), M_{2} (0,2) \) va
\(M_{3} (2;-1) \) nuqtalarda joylashgan uchburchakning ichki 
burchaklari orasida o‘tmas burchak bor yoki yo‘qligini aniqlang.
 \\
B2. 
Parallelogrammning ikki tomoni tenglamalari
\(8x+3y+1=0,\ 2x+y-1=0\) va bir diagonali tenglamasi
\(3x+2y+3=0\) berilgan. Parallelogramm uchlari koordinatalarini
aniqlang
 \\
B3. Tekislikda uchta vektor $\vec{a} = \{ 3; - 2\}$, $\vec{b} = \{ - 2;1\}$ va $\vec{c} = \{ 7; - 4\}$ berilgan. Bu uchta vektorning har birining qolgan ikkitasini bazis sifatida qabul qilib yoyilmasini toping.
 \\
C1. \(A (4;2) \) nuqta orqali, ikkita koordinata o‘qlariga
urinma doira o‘tkazildi. Uning markazi $C$ ni va radiusi
$R$ ni toping.
 \\
C2. 
Uchburchaklarning uchlari
\(A (1; - 1),\ B (- 2;1),\ C (3;5) \) nuqtalarda joylashgan. $A$
uchidan o‘tib, $B$ uchidan o‘tkazilgan medianaga
perpendikular to‘g‘ri chiziq tenglamasini tuzing.
 \\
C3. 
Ayniyatni isbotlang: \((\lbrack\vec{a} + \vec{b},\vec{b} + \vec{c}\rbrack,\vec{c} + \vec{a}) = 2 (\lbrack\vec{a},\vec{b}\rbrack,\vec{c}) \).
 \\

\end{tabular}
\vspace{1cm}


\begin{tabular}{m{17cm}}
\textbf{48-variant}
\newline

T1. 
Vektorning koordinatalari.
 \\
T2. 
Tekislikda to‘g‘ri chiziqning tenglamalari.
 \\
A1. 
Uchlari $M (3;-4) $, $N (-2;3) $ va $P (4;5) $
nuqtalarida joylashgan uchburchaklarning yuzini hisoblang.
 \\
A2. 
Umumiy tenglama bilan berilgan to‘g‘ri chiziqlarning
o‘zaro joylashuvini aniqlang, agar kesishadigan bo‘lsa kesishish nuqtasini
toping: $4x-7=0, 3x+8=0$.
 \\
A3. 
Agar \(a = \{ 2; - 1;2\}, \ \ \ \ b = \{ 1;2; - 3\}, \ \ \ \ c = \{ 3; - 4;7\}\) bo‘lsa, $\overrightarrow{a}, \overrightarrow{b}, \overrightarrow{c}$ vektorlar komplanar bo‘lishini tekshiring. \\
B1. 
To‘g‘ri chiziq \(M (2;-3) \) va \(N (-6;5) \) nuqtalardan o‘tadi.
Shu to‘g‘ri chiziqda ordinatasi $-5$ ga teng nuqtani toping.
 \\
B2. 
Doiraviy to‘rtburchakning uchlari
\(A (-2;-6),\ B (7;6),\ C (3;9) \) va \(D (-3;1) \) nuqtalarda
joylashgan. Diagonallarining kesishish nuqtasi topilsin.
 \\
B3. 
$\vec{a} = \{ 3; - 1; - 2\}$ va $\vec{b} = \{ 1;2; - 1\}$ vektorlar berilgan. Quyidagi vektor ko‘paytmalarning koordinatalarini toping:
$\left\lbrack 2\vec{a} - \vec{b},2\vec{a} + \vec{b} \right\rbrack$.
 \\
C1. 
Uchburchakning uchlari \(M_{1} (- 3;6),\ M_{2} (9; - 10) \)
va \(M_{3} (-5;4) \) berilgan. Shu uchburchakka tashqi chizilgan
aylana markazi $C$ va radiusi $R$ ni aniqlang.
 \\
C2. 
\(A (11; - 15) \) va \(B (-7;3) \) nuqtalardan
teng masofada va \(C (3; 5) \) nuqtadan o‘tuvchi to‘g‘ri chiziq tenglamasini
tuzing.
 \\
C3. 
\(\lbrack\vec{a},\vec{b}\rbrack^{2} < {\vec{a}}^{2}{\vec{b}}^{2}\) ekanini isbotlang; qanday holda bu yerda tenglik ishorasi bo‘ladi?
 \\

\end{tabular}
\vspace{1cm}


\begin{tabular}{m{17cm}}
\textbf{49-variant}
\newline

T1. 
Vektorlarning skalyar ko‘paytmasi.
 \\
T2. 
Tekislik va to‘g‘ri chiziqlarning o‘zaro joylashishi.
 \\
A1. 
Berilgan $A (3; -5) $, $B (-2; -7)$ va
$C (18; 1) $ nuqtalar bir to‘g‘ri chiziqda yotishini isbotlang.
 \\
A2. 
Umumiy tenglama bilan berilgan to‘g‘ri chiziqlarning
o‘zaro joylashuvini aniqlang, agar kesishadigan bo‘lsa kesishish nuqtasini
toping: $2x-5y+1=0, 6x-15y+3=0$.
 \\
A3. 
$\alpha$
qanday qiymatlarida 
$\overrightarrow{a} = \alpha\overrightarrow{i} - 3\overrightarrow{j} + 2\overrightarrow{k}$
va
$\overrightarrow{b} = \overrightarrow{i} + 2\overrightarrow{j} - \alpha\overrightarrow{k}$
vektorlar o‘zaro perpendikulyar bo‘lishini aniqlang.
 \\
B1. 
Ikkita qarama-qarshi uchlari $P (3; -4) $ va $Q (l;2) $ nuqtalarda joylashgan rombaning tomon uzunligi \(5\sqrt{2}\). Shu romb balandligining uzunligini hisoblang.
 \\
B2. 
\(P (3;8) \) va \(Q (-1;-6) \) nuqtalardan o‘tgan
to‘g‘ri chiziqning koordinata o‘qlari bilan kesishish nuqtalarini toping.
 \\
B3. 
$\vec{a}$ va $\vec{b}$ vektorlar o‘zaro perpendikulyar; $\vec{c}$ vektor ular bilan $\pi/3$ ga teng bo‘lgan burchaklar hosil qiladi; $|\vec{a}| = 3$, $|\vec{b}| = 5,\ |\vec{c}| = 8$ ekani ma’lum, quyidagilarni hisoblang:
$ (\vec{a} + 2\vec{b} - 3\vec{c}) ^{2}$.
 \\
C1. 
Uchburchakning uchlari
\(A (3; - 5),\ B (1; - 3),\ C (2; - 2) \) berilgan. $B$ uchi tashqi
burchagi bessektrisa uzunligini aniqlang.
 \\
C2. 
Uchburchakning tomonlari \(x + 5y - 7 = 0\),
\(4x - y - 7 = 0\), \(x + 3y - 31 = 0\) tenglamalar bilan berilgan.
Balandliklarining kesishish nuqtasini toping.
 \\
C3. 
\(\vec{p} = \vec{b} - \frac{\vec{a} (\vec{a},\vec{b}) }{{\vec{a}}^{2}}\) vektor \(\vec{a}\) vektorga perpendikulyar ekanini isbotlang.
 \\

\end{tabular}
\vspace{1cm}


\begin{tabular}{m{17cm}}
\textbf{50-variant}
\newline

T1. Analitik geometriya fanining predmeti va metodlari.
 \\
T2. 
Nuqtadan tekislikkacha, fazoda nuqtadan to‘g‘ri chiziqqacha va ayqash to‘g‘ri chiziqlar orasidagi masofa. \\
A1. 
Kvadratning ikkita qo‘shni uchlari $A (3; -7)$ va
$B (-1;4) $ berilgan. Uning yuzini hisoblang.
 \\
A2. 
$y-3=0$ to‘g‘ri chiziqning $k$ burchagi
koeffitsiyentini va $Oy$ o‘qidan kesib olgan kesmaning algebraik
qiymati $b$ ni aniqlang.
 \\
A3. 
Uchlari $A (1;2;1), B (3;-1;7) $ va $C (7;4;-2) $ bo‘lgan uchburchakning
ichki burchaklarini hisoblab toping. Bu uchburchakning teng yonli ekanligini isbotlang.
 \\
B1. 
Uchburchakning uchlari
\(A\left(-\sqrt{3};1 \right),\ B (0;2) \) va
\(C\left(-2\sqrt{3};2 \right) \) nuqtalarda. Uning $A$
uchidagi tashqi burchakni toping.
 \\
B2. 
\(M (2;-5) \) nuqta, berilgan to‘g‘ri chiziqlarning:
\(3x+5y-4=0\) va \(x-2y+3=0\) kesishmasida hosil bo‘ladi
bo‘lgan o‘tkir yoki o‘tmas burchakka tegishli bo‘lishini aniqlang.
 \\
B3. 
$\vec{a}$ va $\vec{b}$ vektorlar $\varphi = 2\pi/3$ burchak hosil qiladi. $|\vec{a}| = 1,|\vec{b}| = 2$ ekanini bilib, quyidagilarni hisoblang:
$\lbrack\overrightarrow{a} + 3\overrightarrow{b},3\overrightarrow{a} - \overrightarrow{b}\rbrack^{2}$
 \\
C1. 
\(M{1} (1; 2) \) nuqta orqali, radiusi 5 ga teng,
$Ox$ o‘qiga urinma aylana o‘tkazildi. Shu aylananing markazi
$S$ ni aniqlang.
 \\
C2. 
Uchburchaklarning uchlari
\(A (2; - 2),\ B (3; - 5),\ C (5;7) \) nuqtalarda joylashgan. $C$
uchidan o‘tib, $A$ uchidan o‘tkazilgan bissektrisaga
perpendikular to‘g‘ri chiziqning tenglamasini tuzing.
 \\
C3. 
\(\vec{a}+\vec{b}\) va \(\vec{a} - \vec{b}\) vektorlar kollinear bo‘lishi uchun \(\vec{a},\vec{b}\) vektorlar qanday shartni qanoatlantirishi kerak?
 \\

\end{tabular}
\vspace{1cm}


\begin{tabular}{m{17cm}}
\textbf{51-variant}
\newline

T1. 
Vektor tushunchasi. Vektorlar ustida chiziqli amallar.
 \\
T2. 
Tekislikda to‘g‘ri chiziqning tenglamalari.
 \\
A1. 
Kvadratning ikkita qarama-qarshi uchlari $P (3; 5) $ va
$Q (1; -3) $ berilgan. Uning yuzini hisoblang.
 \\
A2. 
5x-3y+15=0 to‘g‘ri chiziqning koordinata burchagidan
kesib olgan uchburchakning yuzini hisoblang.
 \\
A3. 
$\overrightarrow{a}
= \{ 1; - 1;3\}, \ \ \ \ \ \overrightarrow{b} = \{ - 2;1\}$, $\overrightarrow{c} = \{3; -2;5\}$ vektorlar berilgan. Hisoblang:
$ (\lbrack\overrightarrow{a},\overrightarrow{b}\rbrack,\overrightarrow{c}) $.
 \\
B1. 
Ordinata o‘qida shunday $M$ nuqtani toping.
\(N (-8;13) \) nuqtadan uzoqligi 17 ga teng bo‘lgan.
 \\
B2. 
\(P (1;-2) \) nuqta va koordinatalar boshi, berilgan ikkita
to‘g‘ri yozing: $12x-5y-7=0, 3x+4y-8=0$.
kesishishidan hosil bo‘lgan bir xil burchakdami, qo‘shni burchakdami yoki vertikal
burchaklarda yotadimi?
 \\
B3. 
$a$ va $b$ vektorlar $\varphi = \pi/6$ burchak hosil qiladi; $|a| = \sqrt{3},|b| = 1$ ekani ma’lum. $p = a + b$ va $q = a - b$ vektorlar orasidagi $\alpha$ burchakni hisoblang.
 \\
C1. 
Uchburchakning uchlari \(M_{1} (- 3;6),\ M_{2} (9; - 10) \)
va \(M_{3} (-5;4) \) berilgan. Shu uchburchakka tashqi chizilgan
aylana markazi $C$ va radiusi $R$ ni aniqlang.
 \\
C2. 
$ABC$ uchburchakning bir uchini \(B (2;6) \), va
bir uchidan o‘tkazilgan balandlikning: \(x - 7y + 15 = 0\), va
bissektrisasining: \(7x + y + 5 = 0\) tenglamalarini bilgan holda,
tomonlarining tenglamalarini tuzing. \\
C3. \(\vec{a} + \vec{b} + \vec{c} = 0\) shartni qanoatlantiruvchi birlik \(\vec{a},\ \vec{b}\) va \(\vec{c}\) vektorlar berilgan. Hisoblang: \(\left(\vec{a},\vec{b} \right) + \left(\vec{b},\vec{c} \right) + \left(\vec{c},\vec{a} \right) \).
 \\

\end{tabular}
\vspace{1cm}


\begin{tabular}{m{17cm}}
\textbf{52-variant}
\newline

T1. 
Koordinatalari bilan berilgan vektorlarning skalyar, vektor va aralash ko‘paytmalari. \\
T2. 
Nuqtadan to‘g‘ri chiziqqacha bo‘lgan masofa. To‘g‘rilar dastasi.
 \\
A1. 
Uch uchi $A (-2;3), \ B (4;-5) $ va
$C (-3;1)$ nuqtalarda joylashgan parallelogrammning yuzini aniqlang.
 \\
A2. 
$Q_1$, $Q_2$, $Q_3$, $Q_4$, $Q_5$ nuqtalar
$x-3y+2=0$ to‘g‘ri chiziqqa tegishli va ordinatalari mos ravishda
1, 0, 2, -1, 3 ga teng. Ularning abssissalarini toping.
 \\
A3. 
Uchburchakning uchlari
$A (- 1; - 2;4) $, $B (- 4; - 2;0) $ va $C (3; -2;1) $. Uning $B$ uchidagi
ichki burchakni aniqlang.
 \\
B1. 
Uchburchakning uchlari \(A (5;0),\ B (0;1) \) va \(C (3;3) \)
nuqtalarida. Uning ichki burchaklarini toping.
 \\
B2. 
\(M (7;-2) \) nuqtadan o‘tib, \(N (4;-6) \) nuqtaga
gacha bo‘lgan masofasi 5 ga teng bo‘lgan to‘g‘ri chiziqlarning tenglamasini tuzing.
 \\
B3. Tekislikda uchta vektor $\vec{a} = \{ 3; - 2\}$, $\vec{b} = \{ - 2;1\}$ va $\vec{c} = \{ 7; - 4\}$ berilgan. Bu uchta vektorning har birining qolgan ikkitasini bazis sifatida qabul qilib yoyilmasini toping.
 \\
C1. \(A (4;2) \) nuqta orqali, ikkita koordinata o‘qlariga
urinma doira o‘tkazildi. Uning markazi $C$ ni va radiusi
$R$ ni toping.
 \\
C2. 
\(A (3;7) \) va \(C (6; 5) \) nuqtalar kvadratning
qarama-qarshi uchlari. Uning tomonlari tenglamasini tuzing.
 \\
C3. 
\(\vec{a},\ \vec{b}\) va \(\vec{c}\) vektorlar \(\vec{a} + \vec{b} + \vec{c} = 0\) shartni qanoatlantiradi. \(\lbrack\vec{a},\vec{b}\rbrack = \lbrack\vec{b},\vec{c}\rbrack = \lbrack\vec{c},\vec{a}\rbrack\) ekanini isbotlang.
 \\

\end{tabular}
\vspace{1cm}


\begin{tabular}{m{17cm}}
\textbf{53-variant}
\newline

T1. 
Vektorlarning skalyar ko‘paytmasi.
 \\
T2. Tekislikda va fazoda dekart koordinatalar sistemasini almashtirish.
 \\
A1. 
$ABCD$ parallelogrammning uchta uchi $A (3; -7) $,
$B (5; -7) $, $C (-2; 5) $ berilgan, to‘rtinchi uchi $D$,
$B$ uchiga qarama-qarshi. Shu parallelogrammning diagonallari
uzunliklarini aniqlang.
 \\
A2. 
Umumiy tenglama bilan berilgan to‘g‘ri chiziqlarning
o‘zaro joylashuvini aniqlang, agar kesishadigan bo‘lsa kesishish nuqtasini
toping: $12x+59y-19=0, 8x+33y-19=0$.
 \\
A3. 
Vektor koordinata o‘qlari bilan quyidagi burchaklarni hosil qilishi
mumkinmi: $\alpha = 90^{{^\circ}},\ \beta = 150^{{^\circ}}$,
$\gamma = 60^{{^\circ}}?$
 \\
B1. 
Uchburchakning uchlari
\(A (3;-5),\ B (-3;3),\ C (-1;-2) \) berilgan. $A$ uchining ichki qismi
burchakli bessektrisaning uzunligini aniqlang.
 \\
B2. 
Uchlari \(A (4;-4),\ B (6;-1) \) va \(C (-1;2) \)
nuqtalarida joylashgan bir jinsli plastinkadan yasalgan uchburchakning
og‘irlik markazidan o‘tib, quyida berilgan
\(\alpha (2x+3y-1) +\beta (3x-4y-3) =0\) to‘g‘ri chiziqlar dasturiga
tegishli to‘g‘ri chiziqning tenglamasini tuzing. \\
B3. 
$\vec{a}$ va $\vec{b}$ vektorlar $\varphi = 2\pi/3$ burchak hosil qiladi. $|\vec{a}| = 3,|\vec{b}| = 4$ ekani ma’lum. Hisoblang:
$ (\vec{a} + \vec{b}) ^{2}$.
 \\
C1. 
Uchburchakning uchlari
\(A (3; - 5),\ B (1; - 3),\ C (2; - 2) \) berilgan. $B$ uchi tashqi
burchagi bessektrisa uzunligini aniqlang.
 \\
C2. 
\(N (- 4; 7) \) nuqtaning, \(A (2;0) \) va \(B (- 3;5) \)
nuqtalardan o‘tgan to‘g‘ri chiziqqa nisbatan simmetrik nuqtani toping.
 \\
C3. 
Ayniyatni isbotlang: \((\lbrack\vec{a},\vec{b}\rbrack,\vec{c} + \lambda\vec{a} + \mu\vec{b}) = (\lbrack\vec{a},\vec{b}\rbrack,\vec{c}) \), bunda \(\lambda\) va \(\mu\) - ixtiyoriy sonlar. \\

\end{tabular}
\vspace{1cm}


\begin{tabular}{m{17cm}}
\textbf{54-variant}
\newline

T1. 
Vektorning koordinatalari.
 \\
T2. 
Fazoviy to‘g‘ri chiziqning tenglamalari. To‘g‘ri chiziqlarning o‘zaro joylashishi.
 \\
A1. 
$A (2;2) $, $B (-1;6) $, $C (-5;3) $ va $D (-2;-1) $
nuqtalari kvadrat uchlari ekanini isbotlang.
 \\
A2. 
Umumiy tenglama bilan berilgan to‘g‘ri chiziqlarning
o‘zaro joylashuvini aniqlang, agar kesishadigan bo‘lsa kesishish nuqtasini
toping: $3x+2y-27=0, x+5y-35=0$.
 \\
A3. 
Agar \(a = \{ 2;3; - 1\}, \ \ \ \ b = \{ 1; - 1;3\}, \ \ \ \ c = \{ 1;9; - 11\}\) bo‘lsa, $\overrightarrow{a}, \overrightarrow{b}, \overrightarrow{c}$ vektorlar komplanar bo‘lishini tekshiring.
 \\
B1. 
Uchburchakning uchlari \(A (2;-5),\ B (1;-2),\ C (4;7) \)
berilgan. $AC$ tomoni bilan $B$ uchining ichki burchagi
bissektrisasining kesishish nuqtasini toping.
 \\
B2. 
\(P (-3;2) \) nuqta, tomonlarining tenglamalari
\(x+y-4=0,\ 3x-7y+8=0,\ 4x-y-31=0\) bilan
berilgan uchburchakning tashqarisida yoki ichida yotishini aniqlang.
 \\
B3. 
$\vec{a} = \{ 3; - 1; - 2\}$ va $\vec{b} = \{ 1;2; - 1\}$ vektorlar berilgan. Quyidagi vektor ko‘paytmalarning koordinatalarini toping:
$\left\lbrack 2\vec{a} - \vec{b},2\vec{a} + \vec{b} \right\rbrack$.
 \\
C1. 
Ikkita uchi \(A (2;1) \) va \(B (5; 3) \) nuqtalarida, va
diagonallarining kesishish nuqtasi ordinata o‘qiga tegishli
parallelogrammning yuzi \(S = 17\) ga teng. Qolgan ikki uchining
koordinatalarini aniqlang. \\
C2. 
$ABC$ uchburchakning ikki uchi
\(A (6; - 2),\ B (10;14) \), va balandliklarining kesishish nuqtasi
\(N (4; - 1) \) berilgan. Bu uchburchakning tomonlari tenglamasini tuzing.
 \\
C3. 
\(\vec{p} = \vec{b} (\vec{a},\vec{c}) - \vec{c} (\vec{a},\vec{b}) \) vektor \(\vec{a}\) vektorga perpendikulyar ekanini isbotlang.
 \\

\end{tabular}
\vspace{1cm}


\begin{tabular}{m{17cm}}
\textbf{55-variant}
\newline

T1. 
Vektorlarning vektor ko‘paytmasi va aralash ko‘paytmasi.
 \\
T2. 
Tekislikdagi to‘g‘ri chiziqlarning o‘zaro joylashishi.
 \\
A1. 
Bir jinsli beshburchakli plastinkaning uchlari berilgan:
$A (2;3), \ B (0;6), \ C (-1;5), \ D (0;1) $ va $E (1;1) $. Uning og‘irligi
markazi koordinatalarini aniqlang.
 \\
A2. 
Umumiy tenglama bilan berilgan to‘g‘ri chiziqlarning
o‘zaro joylashuvini aniqlang, agar kesishadigan bo‘lsa kesishish nuqtasini
toping: $12x+15y-39=0, 16x-9y-23=0$.
 \\
A3. 
Vektor koordinata o‘qlari bilan quyidagi burchaklarni hosil qila oladimi:
$\alpha = 45^{{^\circ}},\ \ \ \ \beta = 135^{{^\circ}},\ \gamma = 60^{{^\circ}}$.
 \\
B1. 
Uchlari \(M (-1;3),\ N (1,2) \ \) va \(P (0;4) \)
nuqtalarida joylashgan uchburchakning ichki burchaklari o‘tkir burchak
ekanligini isbotlang.
 \\
B2. 
To‘g‘ri to‘rtburchakning ikki tomoni
\(5x+2y-7=0,\ 5x+2y-36=0\) va diagonali
\(3x+7y-10=0\) tenglamalar bilan berilgan. Qolgan ikki tomoni
tenglamalarni tuzing.
 \\
B3. 
$\vec{a} = \{ 3; - 1; - 2\}$ va $\vec{b} = \{ 1;2; - 1\}$ vektorlar berilgan. Quyidagi vektor ko‘paytmalarning koordinatalarini toping:
$\left\lbrack 2\vec{a} + \vec{b},\vec{b} \right\rbrack$.
 \\
C1. 
Ikki uchi \(A (2; - 3) \) va \(B (-5;1) \) nuqtalarda,
uchinchi uchi $C$ ordinata o‘qiga tegishli uchburchakning
medianalarining kesishish nuqtasi $M$ abssissa o‘qida yotadi.
$M$ va $C$ nuqtalarning koordinatalarini aniqlang.
 \\
C2. 
\(P (3;5) \) nuqtadan o‘tib, \(4x + 6y - 7 = 0\) to‘g‘ri chiziq
bilan \(45^{0}\) burchak yasab kesishuvchi to‘g‘ri chiziq tenglamasini tuzing.
 \\
C3. 
Ayniyatni isbotlang: \(\lbrack\vec{a},\vec{b}\rbrack^{2} + (\vec{a},\vec{b}) ^{2} = {\vec{a}}^{2}{\vec{b}}^{2}\).
 \\

\end{tabular}
\vspace{1cm}


\begin{tabular}{m{17cm}}
\textbf{56-variant}
\newline

T1. 
Chiziqli bog‘liq va chiziqli bog‘lanmagan vektorlar.
 \\
T2. 
Tekislikning tenglamalari. Tekisliklarning o‘zaro joylashishi.
 \\
A1. 
Uchlari $M (3;-4) $, $N (-2;3) $ va $P (4;5) $
nuqtalarida joylashgan uchburchaklarning yuzini hisoblang.
 \\
A2. 
$a$ va $b$ parametrlarining qanday qiymatlarida
$ax-2y-1=0$, $6x-4y-b=0$ to‘g‘ri chiziqlar umumiy nuqtaga ega bo‘ladi?
 \\
A3. 
$\overrightarrow{a}$ va $\overrightarrow{b}$ vektorlar
$\varphi = \pi/6$ burchak hosil qiladi.
$|\overrightarrow{a}| = 6,|\overrightarrow{b}| = 5$ ekanini bilib,
$\left| \left\lbrack \overrightarrow{a},\overrightarrow{b} \right\rbrack \right|$ kattalikni hisoblang.
 \\
B1. 
To‘rtburchakning uchlari
\(A (-2;14),\ B (4;-2),\ C (6;-2) \) va \(D (6;10) \) berilgan. Shu
to‘rtburchakning $AC$ va $BD$ diagonallarining kesishishi
nuqtani toping.
 \\
B2. 
Uchburchak uchlari \(A (1;0),\ B (5;-2),\ C (3;2) \)
koordinatalari bilan berilgan. Uchburchaklar tomonlarining va
medianalarining tenglamalarini tuzing.
 \\
B3. 
$A (2; -1;2),B (1;2; 1) $ va $C (3;2;1) $ nuqtalar berilgan. Quyidagi vektor ko‘paytmalarning koordinatalarini toping:
$\lbrack\overline{BC} - 2\overline{CA},\overline{CB}\rbrack$. \\
C1. 
Uchburchakning uchlari
\(A (- 1; - 1),\ B (3;5),\ C (- 4;1) \) berilgan. $A$ uchi tashqi
burchak bissektrisasining, $BC$ tomonining davomi bilan kesishish
nuqtani toping.
 \\
C2. 
Bir tomoni \(x-4y - 8 = 0\) to‘g‘ri chiziqda yotuvchi
kvadratning og‘irlik markazi \(M (1;1) \) nuqtada joylashgan. Shu kvadratning
qolgan tomonlari yotgan to‘g‘ri chiziqlarning tenglamalarini tuzing.
 \\
C3. 
\(\vec{a}+\vec{b}\) va \(\vec{a} - \vec{b}\) vektorlar kollinear bo‘lishi uchun \(\vec{a},\vec{b}\) vektorlar qanday shartni qanoatlantirishi kerak?
 \\

\end{tabular}
\vspace{1cm}


\begin{tabular}{m{17cm}}
\textbf{57-variant}
\newline

T1. 
Vektor tushunchasi. Vektorlar ustida chiziqli amallar.
 \\
T2. 
Nuqtadan to‘g‘ri chiziqqacha bo‘lgan masofa. To‘g‘rilar dastasi.
 \\
A1. 
Uchburchakning uchlari $A (1;4) $, $B (3;-9) $, $C (-5;2) $
berilgan. $B$ uchidan o‘tkazilgan mediana uzunligini aniqlang.
 \\
A2. 
Umumiy tenglama bilan berilgan to‘g‘ri chiziqlarning
o‘zaro joylashuvini aniqlang, agar kesishadigan bo‘lsa kesishish nuqtasini
toping: $2y+9=0, y-5=0$.
 \\
A3. 
Tekislikda ikkita vektor
$\overrightarrow{p} = \{ 2; - 3\}$, $\overrightarrow{q} = \{ 1;2\}$.
$\overrightarrow{a} = \{9;4\}$ vektorning
$\overrightarrow{p},\ \overrightarrow{q}$ bazis bo‘yicha yoyilmasi topilsin.
 \\
B1. 
Uchlari \(M_{1} (1;1), M_{2} (0,2) \) va
\(M_{3} (2;-1) \) nuqtalarda joylashgan uchburchakning ichki 
burchaklari orasida o‘tmas burchak bor yoki yo‘qligini aniqlang.
 \\
B2. 
\(N (4;-5) \) nuqtadan o‘tib, $2x+5y-7=0$
to‘g‘ri chiziqlariga parallel to‘g‘ri chiziqlarning tenglamasini tuzing. Masalani burchaklik
koeffitsiyentni hisoblamasdan yeching.
 \\
B3. 
$\vec{a} = \{ 6; - 8; - 7,5\}$ vektorga kollinear bo‘lgan $\vec{x}$ vektor $Oz$ o‘qi bilan o‘tkir burchak hosil qiladi. $|\vec{x}| = 50$ ekanini bilgan holda uning koordinatalarini toping.
 \\
C1. 
Uchburchakning uchlari
\(A (3; - 5),\ B (1; - 3),\ C (2; - 2) \) berilgan. $B$ uchi tashqi
burchagi bessektrisa uzunligini aniqlang.
 \\
C2. 
$ABC$ uchburchakning bir uchi \(A (1;3) \) nuqtada,
va ikkita medianasi \(x - 2y + 1 = 0\,\ y - 1 = 0\) to‘g‘ri chiziqlarda
joylashgan. Tomonlarining tenglamalarini tuzing.
 \\
C3. 
\(ABC\) uchburchakning tomonlari bilan mos keluvchi \(\vec{AB} = \vec{b}\) va \(\vec{AC} = \vec{c}\) vektorlar berilgan. Bu uchburchakning \(B\) uchidan tushirilgan \(BD\) balandligining \(\vec{b},\ \vec{c}\) bazis bo‘yicha yoyilmasini toping.
 \\

\end{tabular}
\vspace{1cm}


\begin{tabular}{m{17cm}}
\textbf{58-variant}
\newline

T1. 
Vektorlarning skalyar ko‘paytmasi.
 \\
T2. 
Tekislik va to‘g‘ri chiziqlarning o‘zaro joylashishi.
 \\
A1. 
Ikkala uchi $A (3;1) $ va $B (1;-3) $ nuqtalarda, a
uchinchi $C$ uchi $Oy$ o‘qiga tegishli uchburchakning
yuzi $S=3$ ga teng. $C$ uchining koordinatalarini aniqlang.
 \\
A2. 
$P1$, $P2$, $P3$, $P4$, $P5$ nuqtalar
3x-2y-6=0 to‘g‘ri chiziqqa tegishli va abssissalari mos ravishda
4, 0, 2, -2, -6 ga teng. Ularning ordinatalarini toping.
 \\
A3. 
Uchlari $A (1;2;1), B (3;-1;7) $ va $C (7;4;-2) $ bo‘lgan uchburchakning
ichki burchaklarini hisoblab toping. Bu uchburchakning teng yonli ekanligini isbotlang.
 \\
B1. 
Parallelogrammning uchta uchi \(A (3;7),\ B (2;-3) \) va
\(C (-1;4) \) nuqtalarda joylashgan. $B$ uchidan $AC$
tomonidan tushirilgan balandlik uzunligini hisoblang.
 \\
B2. 
Koordinata boshi, berilgan to‘g‘ri chiziqlarning:
\(3x+y-4=0\) va \(3x-2y+6=0\) kesishmasida hosil bo‘ladi
bo‘lgan o‘tkir yoki o‘tmas burchakka tegishli bo‘lishini aniqlang.
 \\
B3. 
$A (2; -1;2),B (1;2; 1) $ va $C (3;2;1)$ nuqtalar berilgan. Quyidagi vektor ko‘paytmalarning koordinatalarini toping:
$\lbrack\overline{AB},\overline{BC}\rbrack$.
 \\
C1. 
Ikkita uchi \(A (2;1) \) va \(B (5; 3) \) nuqtalarida, va
diagonallarining kesishish nuqtasi ordinata o‘qiga tegishli
parallelogrammning yuzi \(S = 17\) ga teng. Qolgan ikki uchining
koordinatalarini aniqlang. \\
C2. 
Ikki uchi \(A (2; - 3),\ B (3; - 2) \) nuqtalarda
joylashgan, yuzi \(S = 1,5\) ga teng bo‘lgan uchburchakning,
og‘irlik markazi \(3x - y - 8 = 0\) to‘g‘ri chiziqqa tegishli. Uchinchi $C$
uchining koordinatasini aniqlang.
 \\
C3. 
Ayniyatni isbotlang: \((\lbrack\vec{a},\vec{b}\rbrack,\vec{c} + \lambda\vec{a} + \mu\vec{b}) = (\lbrack\vec{a},\vec{b}\rbrack,\vec{c}) \), bunda \(\lambda\) va \(\mu\) - ixtiyoriy sonlar. \\

\end{tabular}
\vspace{1cm}


\begin{tabular}{m{17cm}}
\textbf{59-variant}
\newline

T1. 
Vektorning koordinatalari.
 \\
T2. 
Tekislikdagi to‘g‘ri chiziqlarning o‘zaro joylashishi.
 \\
A1. 
$ABCD$-parallelogrammning uchta uchi
$A (2;3) $, $B (4;-1) $ va $C (0;5) $ berilgan. To‘rtinchi $D$
cho‘qqisini toping.
 \\
A2. 
$P (2;2)$ nuqtadan o‘tib, koordinata burchagidan
yuzi 1 ga teng uchburchak kesib oladigan to‘g‘ri chiziqlarning
tenglamasini tuzing.
 \\
A3. 
Berilgan: $\overrightarrow{a}| = 3,|\overrightarrow{b}| = 26$ va
$\lbrack\overrightarrow{a},\overrightarrow{b}\rbrack| = 72$. Hisoblang
$\left(\overrightarrow{a},\overrightarrow{b} \right) $.
 \\
B1. 
\(P (2;2) \) va \(Q (1;5) \) nuqtalar bilan teng uchta
bo‘lingan kesmaning uchlari $A$ va $B$ nuqtalarning
koordinatalarini aniqlang.
 \\
B2. 
Parallel to‘g‘ri chiziqlar orasidagi masofani hisoblang: $5x-12y+13=0, 5x-12y-26=0$.
 \\
B3. 
$\vec{a} = \{ 2;1; - 1\}$ vektorga kollinear bo‘lgan va $\left(\vec{x},\vec{a} \right) = 3$ shartni qanoatlantiruvchi $\vec{x}$ vektorni toping.
 \\
C1. 
Uchburchakning uchlari \(M_{1} (- 3;6),\ M_{2} (9; - 10) \)
va \(M_{3} (-5;4) \) berilgan. Shu uchburchakka tashqi chizilgan
aylana markazi $C$ va radiusi $R$ ni aniqlang.
 \\
C2. 
Agarda \(M (4;5) \) nuqta, koordinata boshidan to‘g‘ri chiziqqa
o‘tkazilgan perpendikulyarning asosi bo‘lsa, shu to‘g‘ri chiziq tenglamasini
tuzing.
 \\
C3. 
\(\vec{p} = \vec{b} (\vec{a},\vec{c}) - \vec{c} (\vec{a},\vec{b}) \) vektor \(\vec{a}\) vektorga perpendikulyar ekanini isbotlang.
 \\

\end{tabular}
\vspace{1cm}


\begin{tabular}{m{17cm}}
\textbf{60-variant}
\newline

T1. 
Vektorlarning vektor ko‘paytmasi va aralash ko‘paytmasi.
 \\
T2. 
Tekislikning tenglamalari. Tekisliklarning o‘zaro joylashishi.
 \\
A1. 
Ikkita uchi $A (-3; 2) $ va $B (1; 6) $ nuqtalarda
joylashgan muntazam uchburchakning yuzini hisoblang.
 \\
A2. 
$a$ va $b$ parametrlarining qanday qiymatlarida
$ax-2y-1=0$, $6x-4y-b=0$ to‘g‘ri chiziqlar parallel bo‘ladi?
 \\
A3. 
Uchburchakning uchlari
$A (3;2; 3) $, $B (5;1; - 1) $ va $C (1; -2;1) $. Uning $A$ uchidagi tashqi burchagi aniqlansin.
 \\
B1. 
To‘g‘ri \(M_{1} (-12;-13) \) va \(M_{2} (-2;-5) \)
nuqtalaridan o‘tadi. Shu to‘g‘ri chiziqda abssissasi 3 ga teng nuqtani toping.
 \\
B2. 
Berilgan to‘g‘ri chiziqlar orasidagi burchakni aniqlang: $3x+2y+4=0, 5x-y+1=0$.
 \\
B3. 
$\vec{a}$ va $\vec{b}$ vektorlar $\varphi = 2\pi/3$ burchak hosil qiladi. $|\vec{a}| = 1,|\vec{b}| = 2$ ekanini bilib, quyidagilarni hisoblang:
$\lbrack\vec{a},\vec{b}\rbrack^{2}$.
 \\
C1. \(A (4;2) \) nuqta orqali, ikkita koordinata o‘qlariga
urinma doira o‘tkazildi. Uning markazi $C$ ni va radiusi
$R$ ni toping.
 \\
C2. 
Ikki nuqta \(A (3; - 5) \) va \(B (- 2;3) \) berilgan.
$B$ nuqtadan o‘tib, $AB$ kesmaga perpendikular to‘g‘ri chiziq
tenglamasini tuzing.
 \\
C3. \(\vec{a} + \vec{b} + \vec{c} = 0\) shartni qanoatlantiruvchi birlik \(\vec{a},\ \vec{b}\) va \(\vec{c}\) vektorlar berilgan. Hisoblang: \(\left(\vec{a},\vec{b} \right) + \left(\vec{b},\vec{c} \right) + \left(\vec{c},\vec{a} \right) \).
 \\

\end{tabular}
\vspace{1cm}


\begin{tabular}{m{17cm}}
\textbf{61-variant}
\newline

T1. 
Chiziqli bog‘liq va chiziqli bog‘lanmagan vektorlar.
 \\
T2. Tekislikda va fazoda dekart koordinatalar sistemasini almashtirish.
 \\
A1. 
Parallelogrammning uchlari
$A (3;-5) $, $B (5;-3) $, $C (-1;3) $ berilgan. $B$ tepasiga
qarama-qarshi joylashgan $D$ uchini aniqlang.
 \\
A2. 
$M (-3;8) $ nuqtadan o‘tib, koordinata o‘qlaridan
teng kesmalarni kesib oladigan to‘g‘ri chiziqlarning tenglamasini tuzing.
 \\
A3. Vektor koordinata o‘qlari bilan quyidagi burchaklarni hosil qila oladimi:
$\alpha = 45^{{^\circ}},\beta = 60^{{^\circ}},\gamma = 120^{{^\circ}}$.
 \\
B1. 
Uchburchakning uchlari \(A (3;6),\ B (-1;3) \) va
\(C (2:-1) \) nuqtalarda joylashgan. $C$ uchidan tushirilgan balandlik uzunligini hisoblang.
 \\
B2. 
Koordinata boshi, tomonlarining tenglamalari
\(8x+3y+31=0,\ x+8y-19=0,\ 7x-5y-11=0\) bilan
berilgan uchburchakning tashqarisida yoki ichida yotishini aniqlang.
 \\
B3. 
$\vec{a}$ va $\vec{b}$ vektorlar $\varphi = 2\pi/3$ burchak hosil qiladi. $|\vec{a}| = 3,|\vec{b}| = 4$ ekani ma’lum. Hisoblang:
$\left(3\vec{a} - 2\vec{b},\vec{a} + 2\vec{b} \right) $.
 \\
C1. 
Uchburchakning uchlari
\(A (- 1; - 1),\ B (3;5),\ C (- 4;1) \) berilgan. $A$ uchi tashqi
burchak bissektrisasining, $BC$ tomonining davomi bilan kesishish
nuqtani toping.
 \\
C2. 
\(Q (5; - 6) \) nuqtaning, \(A (3;8) \) va \(B (7;5) \)
nuqtalardan o‘tgan to‘g‘ri chiziqdagi proyeksiyasini toping.
 \\
C3. 
Ayniyatni isbotlang: \((\lbrack\vec{a} + \vec{b},\vec{b} + \vec{c}\rbrack,\vec{c} + \vec{a}) = 2 (\lbrack\vec{a},\vec{b}\rbrack,\vec{c}) \).
 \\

\end{tabular}
\vspace{1cm}


\begin{tabular}{m{17cm}}
\textbf{62-variant}
\newline

T1. 
Koordinatalari bilan berilgan vektorlarning skalyar, vektor va aralash ko‘paytmalari. \\
T2. 
Nuqtadan tekislikkacha, fazoda nuqtadan to‘g‘ri chiziqqacha va ayqash to‘g‘ri chiziqlar orasidagi masofa. \\
A1. 
$A (4;2) $, $B (7;-2) $ va $C (1;6) $ nuqtalar bir jinsli
simdan yasalgan uchburchak uchlari. Shu uchburchakning og‘irligi
 \\
A2. 
$m$ va $n$ parametrlarining qanday qiymatlarida
$mx+8y+n=0$, $2x+my-1=0$ to‘g‘ri chiziqlar parallel bo‘ladi?
 \\
A3. 
Agar \(a = \{ 2; - 1;2\}, \ \ \ \ b = \{ 1;2; - 3\}, \ \ \ \ c = \{ 3; - 4;7\}\) bo‘lsa, $\overrightarrow{a}, \overrightarrow{b}, \overrightarrow{c}$ vektorlar komplanar bo‘lishini tekshiring. \\
B1. 
To‘rtburchakning uchlari
\(A (-3;12),\ B (3;-4),\ C (5;-4) \) va \(D (5;8) \) berilgan. Shu
to‘rtburchakning $AC$ diagonali $BD$ diagonali qanday
nisbatda bo'lishini aniqlang.
 \\
B2. 
Berilgan ikki nuqtadan o‘tuvchi to‘g‘ri chiziqning burchagi
koeffitsiyenti $k$ ni hisoblang: $A (-4;3) $, $B (1;8) $.
 \\
B3. 
$\vec{a}$ va $\vec{b}$ vektorlar $\varphi = 2\pi/3$ burchak hosil qiladi. $|\vec{a}| = 1,|\vec{b}| = 2$ ekanini bilib, quyidagilarni hisoblang:
$\lbrack 2\overrightarrow{a} + \overrightarrow{b},\overrightarrow{a} + 2\overrightarrow{b}\rbrack^{2}$.
 \\
C1. 
\(M{1} (1; 2) \) nuqta orqali, radiusi 5 ga teng,
$Ox$ o‘qiga urinma aylana o‘tkazildi. Shu aylananing markazi
$S$ ni aniqlang.
 \\
C2. 
Ikki uchi \(A (2; - 3),\ B (3; - 2) \) nuqtalarda
joylashgan, yuzi \(S = 1,5\) ga teng bo‘lgan uchburchakning,
og‘irlik markazi \(3x - y - 8 = 0\) to‘g‘ri chiziqqa tegishli. Uchinchi $C$
uchining koordinatasini aniqlang.
 \\
C3. 
\(\vec{p} = \vec{b} - \frac{\vec{a} (\vec{a},\vec{b}) }{{\vec{a}}^{2}}\) vektor \(\vec{a}\) vektorga perpendikulyar ekanini isbotlang.
 \\

\end{tabular}
\vspace{1cm}


\begin{tabular}{m{17cm}}
\textbf{63-variant}
\newline

T1. Analitik geometriya fanining predmeti va metodlari.
 \\
T2. 
Tekislikda to‘g‘ri chiziqning tenglamalari.
 \\
A1. 
Uchlari $A (2;-3) $, $B (3;2) $ va $C (-2;5) $
nuqtalarida joylashgan uchburchaklarning yuzini hisoblang.
 \\
A2. 
$P (12;6)$ nuqtadan o‘tib, koordinata burchagidan
yuzi 150 ga teng uchburchak kesib oladigan to‘g‘ri chiziqlarning
tenglamasini tuzing.
 \\
A3. 
$\alpha$
qanday qiymatlarida 
$\overrightarrow{a} = \alpha\overrightarrow{i} - 3\overrightarrow{j} + 2\overrightarrow{k}$
va
$\overrightarrow{b} = \overrightarrow{i} + 2\overrightarrow{j} - \alpha\overrightarrow{k}$
vektorlar o‘zaro perpendikulyar bo‘lishini aniqlang.
 \\
B1. 
To‘g‘ri chiziq \(A (7;-3) \) va \(B (23;-6) \) nuqtalardan o‘tadi.
Shu to‘g‘ri chiziqning abssissa o‘qi bilan kesishish nuqtasini toping.
 \\
B2. 
Berilgan \(8x-15y-25=0\) to‘g‘ri chiziqdan og‘ishi -2 ga teng
teng bo‘lgan nuqtalarning geometrik o‘rni tenglamasini tuzing.
 \\
B3. 
$\vec{a}$ va $\vec{b}$ vektorlar $\varphi = 2\pi/3$ burchak hosil qiladi. $|\vec{a}| = 3,|\vec{b}| = 4$ ekani ma’lum. Hisoblang:
${\vec{b}}^{2}$.
 \\
C1. 
Ikki uchi \(A (2; - 3) \) va \(B (-5;1) \) nuqtalarda,
uchinchi uchi $C$ ordinata o‘qiga tegishli uchburchakning
medianalarining kesishish nuqtasi $M$ abssissa o‘qida yotadi.
$M$ va $C$ nuqtalarning koordinatalarini aniqlang.
 \\
C2. 
Uchburchakning tomonlari \(x + 5y - 7 = 0\),
\(4x - y - 7 = 0\), \(x + 3y - 31 = 0\) tenglamalar bilan berilgan.
Balandliklarining kesishish nuqtasini toping.
 \\
C3. 
\(\vec{a},\ \vec{b}\) va \(\vec{c}\) vektorlar \(\vec{a} + \vec{b} + \vec{c} = 0\) shartni qanoatlantiradi. \(\lbrack\vec{a},\vec{b}\rbrack = \lbrack\vec{b},\vec{c}\rbrack = \lbrack\vec{c},\vec{a}\rbrack\) ekanini isbotlang.
 \\

\end{tabular}
\vspace{1cm}


\begin{tabular}{m{17cm}}
\textbf{64-variant}
\newline

T1. 
Chiziqli bog‘liq va chiziqli bog‘lanmagan vektorlar.
 \\
T2. 
Fazoviy to‘g‘ri chiziqning tenglamalari. To‘g‘ri chiziqlarning o‘zaro joylashishi.
 \\
A1. 
Ikkala uchi $A (2;1) $ va $B (3;-2) $ nuqtalarda, va
uchinchi $C$ uchi $Ox$ o‘qiga tegishli bo‘lgan uchburchakning
yuzi $S=4$ ga teng. $C$ uchining koordinatalarini aniqlang. \\
A2. 
$P (8;6) $ nuqtadan o‘tib, koordinata burchagidan
yuzi 12 ga teng uchburchak kesib oladigan to‘g‘ri chiziqlarning tenglamasini
tuzing.
 \\
A3. 
To‘rtburchakning uchlari berilgan:
$A (1; - 2;2) $, $B (1;4;0),C (- 4;1;1) $ va $D (- 5; -5;3) $. Uning diagonallari $AC$ va $BD$ o‘zaro
perpendikulyarligini isbotlang.
 \\
B1. 
To‘g‘ri chiziq \(M (2;-3) \) va \(N (-6;5) \) nuqtalardan o‘tadi.
Shu to‘g‘ri chiziqda ordinatasi $-5$ ga teng nuqtani toping.
 \\
B2. Berilgan to‘g‘ri chiziqlarning kesishish nuqtasini toping:
$(3x-4y-29=0, 2x+5y+19=0)$.
 \\
B3. 
$\vec{a}$ va $\vec{b}$ vektorlar o‘zaro perpendikulyar. $|\vec{a}| = 3,|\vec{b}| = 4$ ekani ma’lum, quyidagilarni hisoblang:
$|\lbrack 3\vec{a} - \vec{b},\vec{a}-2\vec{b}\rbrack|$.
 \\
C1. 
Uchburchakning uchlari \(M_{1} (- 3;6),\ M_{2} (9; - 10) \)
va \(M_{3} (-5;4) \) berilgan. Shu uchburchakka tashqi chizilgan
aylana markazi $C$ va radiusi $R$ ni aniqlang.
 \\
C2. 
\(A (11; - 15) \) va \(B (-7;3) \) nuqtalardan
teng masofada va \(C (3; 5) \) nuqtadan o‘tuvchi to‘g‘ri chiziq tenglamasini
tuzing.
 \\
C3. 
Ayniyatni isbotlang: \(\lbrack\vec{a},\vec{b}\rbrack^{2} + (\vec{a},\vec{b}) ^{2} = {\vec{a}}^{2}{\vec{b}}^{2}\).
 \\

\end{tabular}
\vspace{1cm}


\begin{tabular}{m{17cm}}
\textbf{65-variant}
\newline

T1. 
Vektor tushunchasi. Vektorlar ustida chiziqli amallar.
 \\
T2. 
Fazoviy to‘g‘ri chiziqning tenglamalari. To‘g‘ri chiziqlarning o‘zaro joylashishi.
 \\
A1. 
Bir jinsli to‘rtburchakli plastinkaning uchlari berilgan:
$A (2;1), \ B (5;3), \ C (-1;7) $ va $D (-7;5) $. Uning og‘irlik markazi
koordinatalarini aniqlang.
 \\
A2. 
Umumiy tenglama bilan berilgan to‘g‘ri chiziqlarning
o‘zaro joylashuvini aniqlang, agar kesishadigan bo‘lsa kesishish nuqtasini
toping: $2x-3y+12=0, 4x-6y-21=0$.
 \\
A3. 
$\overrightarrow{a}
= \{ 1; - 1;3\}, \ \ \ \ \ \overrightarrow{b} = \{ - 2;1\}$, $\overrightarrow{c} = \{3; -2;5\}$ vektorlar berilgan. Hisoblang:
$ (\lbrack\overrightarrow{a},\overrightarrow{b}\rbrack,\overrightarrow{c}) $.
 \\
B1. 
Ikkala uchi \(A (3;1) \) va \(B (1;-3) \) nuqtalarda, va
og‘irlik markazi $Ox$ o‘qiga tegishli uchburchakning yuzi
\(S=3\) ga teng. Uchinchi $C$ uchining koordinatalarini aniqlang. \\
B2. 
Kvadratning ikki tomoni
\(5x-12y+65=0,\ 5x-12y-26=0\) to‘g‘ri chiziqlarda
yotishini bilgan holda, yuzini hisoblang.
 \\
B3. 
$\vec{a}$ va $\vec{b}$ vektorlar o‘zaro perpendikulyar; $\vec{c}$ vektor ular bilan $\pi/3$ ga teng bo‘lgan burchaklar hosil qiladi; $|\vec{a}| = 3$, $|\vec{b}| = 5,\ |\vec{c}| = 8$ ekani ma’lum, quyidagilarni hisoblang:
$ (\vec{a} + \vec{b} + \vec{c}) ^{2}$.
 \\
C1. 
\(M{1} (1; 2) \) nuqta orqali, radiusi 5 ga teng,
$Ox$ o‘qiga urinma aylana o‘tkazildi. Shu aylananing markazi
$S$ ni aniqlang.
 \\
C2. 
Uchburchaklarning uchlari
\(A (2; - 2),\ B (3; - 5),\ C (5;7) \) nuqtalarda joylashgan. $C$
uchidan o‘tib, $A$ uchidan o‘tkazilgan bissektrisaga
perpendikular to‘g‘ri chiziqning tenglamasini tuzing.
 \\
C3. 
\(\lbrack\vec{a},\vec{b}\rbrack^{2} < {\vec{a}}^{2}{\vec{b}}^{2}\) ekanini isbotlang; qanday holda bu yerda tenglik ishorasi bo‘ladi?
 \\

\end{tabular}
\vspace{1cm}


\begin{tabular}{m{17cm}}
\textbf{66-variant}
\newline

T1. 
Vektorlarning skalyar ko‘paytmasi.
 \\
T2. Tekislikda va fazoda dekart koordinatalar sistemasini almashtirish.
 \\
A1. 
Bir jinsli elementdan yasalgan qatorning og‘irlik markazi
$M (1;4) $ nuqtada, bir uchi $P (-2;2) $ nuqtada joylashgan. Shu
qatorning ikkinchi uchi $Q$ ning koordinatalarini aniqlang.
 \\
A2. 
Umumiy tenglama bilan berilgan to‘g‘ri chiziqlarning
o‘zaro joylashuvini aniqlang, agar kesishadigan bo‘lsa kesishish nuqtasini
toping: $14x-9y-24=0, 7x-2y-17=0$.
 \\
A3. 
Berilgan: $\overrightarrow{a}| = 10,|\overrightarrow{b}| = 2$ va
$\left(\overrightarrow{a},\overrightarrow{b} \right) = 12$. Hisoblang
$\left| \left\lbrack \overrightarrow{a},\overrightarrow{b} \right\rbrack \right|$.
 \\
B1. 
Ikkita nuqta berilgan \(M (2;2) \) va \(N (5;-2) \); abssissa o‘qida shunday $P$ nuqtani topingki, $MPN$ burchak to‘g‘ri burchak bo‘lsin.
 \\
B2. 
\(P (2;7) \) nuqtadan o‘tib, \(Q (1;2) \) nuqtagacha
masofasi 5 ga teng bo‘lgan to‘g‘ri chiziqlarning tenglamasini tuzing.
 \\
B3. 
$\vec{a} + \vec{b} + \vec{c} = 0$ shartni qanoatlantiruvchi $\vec{a},\ \vec{b}$ va $\vec{c}$ vektorlar berilgan. $|\vec{a}| = 3,\ |\vec{b}| = 1$ va $|\vec{c}| = 4$ ekani ma’lum, $\left(\vec{a},\vec{b} \right) + \left(\vec{b},\vec{c} \right) + (\vec{c}) $ ifodani hisoblang.
 \\
C1. \(A (4;2) \) nuqta orqali, ikkita koordinata o‘qlariga
urinma doira o‘tkazildi. Uning markazi $C$ ni va radiusi
$R$ ni toping.
 \\
C2. 
Bir tomoni \(x-4y - 8 = 0\) to‘g‘ri chiziqda yotuvchi
kvadratning og‘irlik markazi \(M (1;1) \) nuqtada joylashgan. Shu kvadratning
qolgan tomonlari yotgan to‘g‘ri chiziqlarning tenglamalarini tuzing.
 \\
C3. 
\(\vec{a} + \vec{b}\) vektor \(\vec{a} - \vec{b}\) vektorga perpendikulyar bo‘lishi uchun \(\vec{a}\) va \(\vec{b}\) vektorlar qanday shartlarni qanoatlantirishi kerak?
 \\

\end{tabular}
\vspace{1cm}


\begin{tabular}{m{17cm}}
\textbf{67-variant}
\newline

T1. 
Vektorning koordinatalari.
 \\
T2. 
Tekislikdagi to‘g‘ri chiziqlarning o‘zaro joylashishi.
 \\
A1. 
Uchlari $M_1 (-3;2) $, $M_2 (5;-2) $ va $M_3 (1;3) $
nuqtalarida joylashgan uchburchaklarning yuzini hisoblang.
 \\
A2. 
$B (-5;5)$ nuqtadan o‘tib, koordinata burchagidan
yuzi 50 ga teng uchburchak kesib oladigan to‘g‘ri chiziqlarning tenglamasini
tuzing.
 \\
A3. 
Agar \(a = \{ 3; - 2;1\},\ \ \ \ \ b = \{ 2;1;2\},\ \ \ \ c = \{ 3; - 1; - 2\}\) bo‘lsa, $\overrightarrow{a}, \overrightarrow{b}, \overrightarrow{c}$ vektorlar komplanar bo‘lishini tekshiring.
 \\
B1. 
Bir to‘g‘ri chiziqqa tegishli \(A (1;-1),\ B (3;3) \) va
\(C (4;5) \) nuqtalar berilgan. Har bir nuqtaning, qolgan ikki nuqta orqali aniqlanuvchi kesmani bo‘lish nisbati $\lambda$ ni aniqlang.
 \\
B2. 
Umumiy tenglamasi \(2x-5y+4=0\) bo‘lgan to‘g‘ri
berilgan. \(M (-3,5) \) nuqtadan o‘tib, berilgan to‘g‘ri chiziqqa: a) parallel;
b) perpendikular bo‘lgan to‘g‘ri chiziqlar tenglamasini tuzing.
 \\
B3. 
$\vec{a} = \{ 3; - 1; - 2\}$ va $\vec{b} = \{ 1;2; - 1\}$ vektorlar berilgan. Quyidagi vektor ko‘paytmalarning koordinatalarini toping:
$\left\lbrack \vec{a},\vec{b} \right\rbrack$.
 \\
C1. 
Uchburchakning uchlari
\(A (3; - 5),\ B (1; - 3),\ C (2; - 2) \) berilgan. $B$ uchi tashqi
burchagi bessektrisa uzunligini aniqlang.
 \\
C2. 
\(A (4;5) \) nuqta, diagonali \(7x - y - 8 = 0\) tenglama
bilan berilgan kvadratning bir uchi. Shu kvadratning tomonlari va
ikkinchi diagonalining tenglamasini tuzing.
 \\
C3. 
Ayniyatni isbotlang: \((\lbrack\vec{a} + \vec{b},\vec{b} + \vec{c}\rbrack,\vec{c} + \vec{a}) = 2 (\lbrack\vec{a},\vec{b}\rbrack,\vec{c}) \).
 \\

\end{tabular}
\vspace{1cm}


\begin{tabular}{m{17cm}}
\textbf{68-variant}
\newline

T1. 
Koordinatalari bilan berilgan vektorlarning skalyar, vektor va aralash ko‘paytmalari. \\
T2. 
Nuqtadan to‘g‘ri chiziqqacha bo‘lgan masofa. To‘g‘rilar dastasi.
 \\
A1. 
Bir jinsli elementdan yasalgan qatorning uchlari
$A (3;-5) $ va $B (-1;1) $ nuqtalarda joylashgan. Uning og‘irligi
markazi koordinatasini aniqlang.
 \\
A2. 
$m$ parametrining qanday qiymatlarida
$mx+ (2m+3) y+m+6=0$, $ (2m+1) x+ (m-1) y+m-2=0$ to‘g‘ri chiziqlar ordinata
o‘qida yotuvchi nuqtada kesishadi.
 \\
A3. 
$\overrightarrow{a} = \{ 2; - 4;4\}$ va $\overrightarrow{b} = \{ - 3;2;6\}$
vektorlar hosil qilgan burchak kosinusini hisoblang.
 \\
B1. 
Abssissa o‘qida shunday $M$ nuqtani topingki,
\(N (2;-3) \) nuqtadan uzoqligi 5 ga teng bo‘lgan.
 \\
B2. 
\(P (2;3) \) va \(Q (5;-1) \) nuqtalar, berilgan ikkita
to‘g‘ri: $12x-y-7=0,\ 13x+4y-5=0$.
kesishishidan hosil bo‘lgan bir xil burchakdami, qo‘shni burchakdami yoki vertikal
burchaklarda yotadimi?
 \\
B3. 
$\vec{a}$ va $\vec{b}$ vektorlar $\varphi = 2\pi/3$ burchak hosil qiladi. $|\vec{a}| = 3,|\vec{b}| = 4$ ekani ma’lum. Hisoblang:
$\left(\vec{a},\vec{b} \right) $.
 \\
C1. 
Uchburchakning uchlari
\(A (- 1; - 1),\ B (3;5),\ C (- 4;1) \) berilgan. $A$ uchi tashqi
burchak bissektrisasining, $BC$ tomonining davomi bilan kesishish
nuqtani toping.
 \\
C2. 
Uchburchakning uchlari
\(A (3;2),\ B (- 4;4),\ C (- 2; 5) \) koordinatalari bilan berilgan.
Balandliklarining tenglamasini tuzing.
 \\
C3. 
\(ABC\) uchburchakning tomonlari bilan mos keluvchi \(\vec{AB} = \vec{b}\) va \(\vec{AC} = \vec{c}\) vektorlar berilgan. Bu uchburchakning \(B\) uchidan tushirilgan \(BD\) balandligining \(\vec{b},\ \vec{c}\) bazis bo‘yicha yoyilmasini toping.
 \\

\end{tabular}
\vspace{1cm}


\begin{tabular}{m{17cm}}
\textbf{69-variant}
\newline

T1. Analitik geometriya fanining predmeti va metodlari.
 \\
T2. 
Tekislikning tenglamalari. Tekisliklarning o‘zaro joylashishi.
 \\
A1. 
Uchburchak uchlarining koordinatalari berilgan
$A (1;-3) $, $B (3;-5) $ va $C (-5;7) $. Tomonlarining o‘rtalarini
aniqlang.
 \\
A2. 
$5x+3y-7=0$, $x-2y-4=0$, $3x-y+3=0$
to‘g‘ri chiziqlar bir nuqtada kesishishadimi?
 \\
A3. 
Agar \(a = \{ 2; - 1;2\}, \ \ \ \ b = \{ 1;2; - 3\}, \ \ \ \ c = \{ 3; - 4;7\}\) bo‘lsa, $\overrightarrow{a}, \overrightarrow{b}, \overrightarrow{c}$ vektorlar komplanar bo‘lishini tekshiring. \\
B1. Ikkita qarama-qarshi uchlari \(P (4;9) \) va \(Q (-2; 1) \) nuqtalarida joylashgan romning tomon uzunligi \(5\sqrt{10}\). Shu
romba yuzini hisoblang.
 \\
B2. 
\(N (5;8) \) nuqtaning, \(5x-11y-43=0\) to‘g‘ri chizig‘idagi
proyeksiyasini toping.
 \\
B3. 
$\vec{a}$ va $\vec{b}$ vektorlar $\varphi = 2\pi/3$ burchak hosil qiladi. $|\vec{a}| = 1,|\vec{b}| = 2$ ekanini bilib, quyidagilarni hisoblang:
$\lbrack\overrightarrow{a} + 3\overrightarrow{b},3\overrightarrow{a} - \overrightarrow{b}\rbrack^{2}$
 \\
C1. 
Ikki uchi \(A (2; - 3) \) va \(B (-5;1) \) nuqtalarda,
uchinchi uchi $C$ ordinata o‘qiga tegishli uchburchakning
medianalarining kesishish nuqtasi $M$ abssissa o‘qida yotadi.
$M$ va $C$ nuqtalarning koordinatalarini aniqlang.
 \\
C2. 
\(A (-5;5) \) va \(B (-7;1) \) nuqtalardan
masofalarining yig‘indisi eng kichik bo‘lgan \(2x - y - 5 = 0\) to‘g‘ri chiziqda
joylashgan nuqtani toping.
 \\
C3. 
Ayniyatni isbotlang: \((\lbrack\vec{a},\vec{b}\rbrack,\vec{c} + \lambda\vec{a} + \mu\vec{b}) = (\lbrack\vec{a},\vec{b}\rbrack,\vec{c}) \), bunda \(\lambda\) va \(\mu\) - ixtiyoriy sonlar. \\

\end{tabular}
\vspace{1cm}


\begin{tabular}{m{17cm}}
\textbf{70-variant}
\newline

T1. 
Vektorlarning vektor ko‘paytmasi va aralash ko‘paytmasi.
 \\
T2. 
Tekislik va to‘g‘ri chiziqlarning o‘zaro joylashishi.
 \\
A1. 
Kvadratning ikkita qo‘shni uchlari $A (3; -7)$ va
$B (-1;4) $ berilgan. Uning yuzini hisoblang.
 \\
A2. 
$M (4;3) $ nuqtadan, koordinata burchagidan
yuzi 3 ga teng uchburchak kesib oladigan to‘g‘ri chiziq o‘tkazildi.
Shu to‘g‘ri chiziqning koordinata o‘qlari bilan kesishish nuqtalari
koordinatalarini aniqlang.
 \\
A3. 
Berilgan: $\overrightarrow{a}| = 10,|\overrightarrow{b}| = 2$ va
$\left(\overrightarrow{a},\overrightarrow{b} \right) = 12$. Hisoblang
$\left| \left\lbrack \overrightarrow{a},\overrightarrow{b} \right\rbrack \right|$.
 \\
B1. 
Uchlari $A_1 (1; 1), A_2 (2; 3) $ va $A (5;-1) $
nuqtalarida joylashgan uchburchakning to‘g‘ri burchakli ekanini isbotlang.
 \\
B2. 
Ikki to‘g‘ri chiziqning chetidagi burchakni toping: $2x+y-9=0, 3x-y+11=0$.
 \\
B3. 
$|\vec{a}| = 3,|\vec{b}| = 5$ berilgan. $\alpha$ ning qanday qiymatida $\vec{a} + \alpha\vec{b}$, $\vec{a} - \alpha\vec{b}$ vektorlar o‘zaro perpendikulyar bo‘lishini aniqlang.
 \\
C1. 
Ikkita uchi \(A (2;1) \) va \(B (5; 3) \) nuqtalarida, va
diagonallarining kesishish nuqtasi ordinata o‘qiga tegishli
parallelogrammning yuzi \(S = 17\) ga teng. Qolgan ikki uchining
koordinatalarini aniqlang. \\
C2. 
\(N (- 4; 7) \) nuqtaning, \(A (2;0) \) va \(B (- 3;5) \)
nuqtalardan o‘tgan to‘g‘ri chiziqqa nisbatan simmetrik nuqtani toping.
 \\
C3. \(\vec{a} + \vec{b} + \vec{c} = 0\) shartni qanoatlantiruvchi birlik \(\vec{a},\ \vec{b}\) va \(\vec{c}\) vektorlar berilgan. Hisoblang: \(\left(\vec{a},\vec{b} \right) + \left(\vec{b},\vec{c} \right) + \left(\vec{c},\vec{a} \right) \).
 \\

\end{tabular}
\vspace{1cm}



\end{document}
