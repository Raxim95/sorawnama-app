\documentclass[10pt]{article}
% \usepackage[turkish]{babel}
\usepackage[utf8]{inputenc}
\usepackage[T1]{fontenc}
\usepackage{amsmath}
\usepackage{amsfonts}
\usepackage{amssymb}
% \usepackage[version=4]{mhchem}
\usepackage{enumitem}

\begin{document}
\textbf{2.1.5.} $\overrightarrow{A B}(-2 ; 3 ;-7)$ vektorning boshi $A(-3 ; 5 ; 6)$ ekanligi ma'lum bo'lsa, vektorning oxiri $B(x ; y ; z)$ nuqtani toping.\\
\textbf{2.1.6.} $\overrightarrow{A B}(5 ; 4 ;-2)$ vektorning boshi $A(2 ;-5 ; 6)$ ekanligi ma'lum bo'lsa, vektorning oxiri $B(x ; y ; z)$ nuqtani toping.\\
\textbf{2.1.7.} $\overrightarrow{A B}(-5 ; 7)$ vektorning oxiri $B(4 ;-1)$ ekanligi ma'lum bo'lsa, vektorning boshi $A(x ; y)$ nuqtani toping.\\
\textbf{2.1.8.} $\overrightarrow{A B}(-2 ;-1 ; 4)$ vektorning oxiri $B(-6 ; 7 ;-3)$ ekanligi ma'lum bo'lsa, vektorning boshi $A(x ; y ; z)$ nuqtani toping.\\
\textbf{2.1.9.} $\overrightarrow{A B}(1 ; 7 ;-9)$ vektorning oxiri $B(1 ;-3 ;-2)$ ekanligi ma'lum bo'lsa, vektorning boshi $A(x ; y ; z)$ nuqtani toping.\\
\textbf{2.1.10.} $\vec{a}(-4 ; 3)$ vektorga yo'nalishdosh bo'lgan birlik vektorni toping.\\
\textbf{2.1.11.} $\vec{b}(-8 ;-6)$ vektorga yo'nalishdosh bo'lgan birlik vektorni toping.\\
\textbf{2.1.12.} $\vec{c}(9 ;-12)$ vektorga qarama-qarshi yo'nalgan birlik vektorni toping.\\
\textbf{2.1.13.} $\vec{d}(6 ;-2 ;-3)$ vektorga yo'nalishdosh bo'lgan birlik vektorni toping.\\
\textbf{2.2.3.} $\vec{a}(12 ;-3)$ va $\vec{b}(-3 ; 6)$ vektorlar berilgan. Quyidagi vektorlarning koordinata o'qlaridagi proyeksiyalarini aniqlang:

\begin{enumerate}[label=\alph*)]
  \item $2 \vec{a}+\vec{b}$;
  \item $\vec{a}-2 \vec{b}$;
  \item $-3 \vec{a}$;
  \item $-\frac{1}{3} \vec{b}$;
  \item $4 \vec{a}+3 \vec{b}$;
  \item $\frac{1}{3} \vec{a}-2 \vec{b}$.\\
\end{enumerate}

\textbf{2.2.4.} $\vec{a}(-4 ; 1)$ va $\vec{b}(6 ;-8)$ vektorlar berilgan. Quyidagi vektorlarning koordinata o'qlaridagi proyeksiyalarini aniqlang:
\begin{enumerate}
  \item $\vec{a}+\vec{b}$;
  \item $\vec{a}-\vec{b}$;
  \item $2 \vec{a}$;
  \item $-\frac{1}{2} \vec{b}$;
  \item $2 \vec{a}+3 \vec{b}$;
  \item $\frac{1}{4} \vec{a}-\vec{b}$.
\end{enumerate}\\

\textbf{2.2.5.} $\vec{a}(8 ;-4)$ va $\vec{b}(-9 ;-3)$ vektorlar berilgan. Quyidagi vektorlarning koordinata o'qlaridagi proyeksiyalarini aniqlang:
\begin{enumerate}
  \item $3 \vec{a}-2 \vec{b}$;
  \item $\vec{a}+2 \vec{b}$;
  \item $-2 \vec{a}$;
  \item $\frac{1}{3} \vec{b}$;
  \item $-3 \vec{a}+2 \vec{b}$
  \item $\frac{1}{4} \vec{a}-2 \vec{b}$.
\end{enumerate}\\

\textbf{2.2.6.} $\vec{a}(-2 ; 3 ;-4)$ va $\vec{b}(0 ;-2 ; 6)$ vektorlar berilgan. Quyidagi vektorlarning koordinata o'qlaridagi proyeksiyalarini aniqlang:
\begin{enumerate}
  \item $-\vec{a}+2 \vec{b}$;
  \item $\vec{a}-3 \vec{b}$;
  \item $-4 \vec{a}$;
  \item $-\frac{2}{3} \vec{b}$;
  \item $4 \vec{a}+\vec{b}$;
  \item $\frac{1}{2} \vec{a}-3 \vec{b}$.
\end{enumerate}\\

\textbf{3.1.1.} $\vec{b}(8 ;-6)$ vektorning modulini toping.\\
\textbf{3.1.2.} $\vec{d}(-2 ; 3 ;-6)$ vektorning modulini toping.\\
\textbf{3.1.3.} $\vec{a}(9 ;-2 ; 6)$ vektorning modulini toping.\\
\textbf{3.1.4.} $\vec{c}(-4 ; 12 ;-3)$ vektorning modulini toping.\\
\textbf{3.1.5.} $\vec{d}(12 ;-1 ; 12)$ vektorning modulini toping.\\
\textbf{3.1.6.} $\vec{c}(12 ;-9)$ vektorning yo'naltiruvchi kosinuslarini aniqlang.\\
\textbf{3.1.7.} $\vec{b}(-10 ; 2 ; 11)$ vektorning yo'naltiruvchi kosinuslarini aniqlang.\\
\textbf{3.1.8.} $\vec{a}(12 ;-15 ; 16)$ vektorning yo'naltiruvchi kosinuslarini aniqlang.\\
\textbf{3.1.9.} $\vec{c}(1 ;-12 ; 12)$ vektorning yo'naltiruvchi kosinuslarini aniqlang.\\
\textbf{3.1.10.} $\overrightarrow{O P}(3 ;-6 ; 2)$ vektorning yo'naltiruvchi kosinuslarini toping.\\
\textbf{3.1.11.} $\vec{a}(12 ;-15 ;-16)$ vektorning yo'naltiruvchi kosinuslarini toping.\\
\textbf{3.1.12.} Boshi $A(-3 ; 5)$ oxiri $B(5 ;-1)$ nuqtalarda bo'lgan $\overrightarrow{A B}$ vektorning yo'naltiruvchi kosinuslari va uzunligi topilsin.\\
\textbf{4.1.2.} $\vec{a}(3 ; 5 ; 7), \vec{b}(-2 ; 6 ; 1)$ va $\vec{c}(2 ;-4 ; 0)$ vektorlar uchun
\begin{enumerate}
  \item $\vec{a} \vec{b}$;
  \item $\vec{a} \vec{c}$;
  \item $\vec{b} \vec{c}$;
  \item $(2 \vec{a}-\vec{b})(3 \vec{b}+\vec{c})$;
  \item $(3 \vec{a}+2 \vec{c})(2 \vec{b}-\vec{c})$ skalyar ko'paytmasini hisoblang.
\end{enumerate}\\

\textbf{4.1.3.} Koordinatalari bilan berilgan $\vec{a}(6 ;-8), \vec{b}(12 ; 9), \vec{c}(2 ;-5)$, $\vec{d}(3 ; 7), \vec{m}(-2 ; 6)$ va $\vec{n}(3 ;-9)$ vektorlar orasidagi
\begin{enumerate}
  \item $\vec{a}^{\wedge} \vec{b}$;
  \item $\vec{c} \wedge \vec{d}$;
  \item $\vec{m} \wedge \vec{n}$ ni toping.
\end{enumerate}\\

\textbf{4.1.4.} Koordinatalari bilan berilgan $\vec{a}(8 ; 4 ; 1), \vec{b}(2 ;-2 ; 1), \vec{c}(2 ; 5 ; 4)$ va $\vec{d}(6 ; 0 ;-3)$ vektorlar orasidagi
  1) $\vec{a}^{\wedge} \vec{b}$; 2) $\vec{c}^{\wedge} \vec{d}$ ni toping.\\
\textbf{4.1.5.} $|\vec{a}|=8,|\vec{b}|=5, \quad\left(\vec{a}^{\wedge} \vec{b}\right)=60^{\circ}$ berilgan bo'lsa, $\vec{a}$ va $\vec{b}$ vektorlarning skalyar ko'paytmasini toping.\\
\textbf{4.1.6.} $\vec{c}$ va $\vec{d}$ birlik vektor va $\left(\vec{c}^{\wedge} \vec{d}\right)=135^{\circ}$ berilgan bo'lsa, $\vec{c}$ va $\vec{d}$ vektorlarning skalyar ko'paytmasini toping.\\
\textbf{4.1.7.} $|\vec{a}|=3,|\vec{b}|=6, \vec{a} \downarrow \downarrow \vec{b}$ berilgan bo'lsa, $\vec{a}$ va $\vec{b}$ vektorlarning skalyar ko'paytmasini toping.

\end{document}