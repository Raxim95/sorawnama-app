\documentclass[10pt]{article}
\usepackage[turkish]{babel}
\usepackage[utf8]{inputenc}
\usepackage[T1]{fontenc}
\usepackage{amsmath}
\usepackage{amsfonts}
\usepackage{amssymb}
\usepackage[version=4]{mhchem}
\usepackage{stmaryrd}

\begin{document}
\textbf{2.3.1.} $\vec{\imath}, \vec{\jmath}, \vec{k}$ bazis bo'yicha vektorlar yoyilmasi berilgan: $\vec{c}=16 \vec{\imath}-15 \vec{\jmath}+12 \vec{k}$. Shu bazis bo'yicha $\vec{c}$ vektorga parallel va qarama-qarshi $\vec{d}$ vektorning yoyilmasini aniqlang, bunda $|\vec{d}|=75$ ga teng.\\
\textbf{2.3.2.} Tekislikda $\vec{p}(2 ;-3), \vec{q}(1 ; 2)$ vektorlar berilgan bo'lsin. $\vec{a}(9 ; 4)$ vektorni $\vec{p}, \vec{q}$ bazis bo'yicha yoyilmasini toping.\\
\textbf{2.3.3.} Tekislikda $\vec{p}(-4 ; 1), \vec{q}(3 ;-5)$ vektorlar berilgan bo'lsin. $\vec{a}(11 ;-7)$ vektorni $\vec{p}, \vec{q}$ bazis bo'yicha yoyilmasini toping.\\
\textbf{2.3.4.} Tekislikda $\vec{p}(3 ;-2), \vec{q}(-4 ; 1)$ vektorlar berilgan bo'lsin. $\vec{a}(17 ;-8)$ vektorni $\vec{p}, \vec{q}$ bazis bo'yicha yoyilmasini toping.\\
\textbf{2.3.5.} Tekislikda $\vec{a}(3 ;-2), \vec{b}(-2 ; 1)$ va $\vec{c}(7 ;-4)$ vektorlar berilgan. Har bir vektorni, qolgan ikki vektorni bazis sifatida qabul qilib, yoyilmasini aniqlang.\\
\textbf{2.3.6.} $\vec{p}(3 ;-2 ; 1), \vec{q}(-1 ; 1 ;-2), \vec{r}(2 ; 1 ;-3)$ va $\vec{c}(11 ;-6 ; 5)$ vektorlar berilgan. $\vec{p}, \vec{q}, \vec{r}$ bazis bo'yicha $\vec{c}=\alpha \vec{p}+\beta \vec{q}+\gamma \vec{r}$ vektorning yoyilmasini toping.\\
\textbf{2.3.11.} $\vec{a}(3 ;-1), \vec{b}(1 ;-2), \vec{c}(-1 ; 7)$ vektorlar berilgan. $\vec{a}, \vec{b}$ bazis bo‘ yicha $\vec{p}=\vec{a}+\vec{b}+\vec{c}$ vektorning yoyilmasini aniqlang.\\
\textbf{2.3.12.} $\vec{a}(2 ; 1 ; 0), \vec{b}(1 ;-1 ; 2), \vec{c}(2 ; 2 ;-1)$ va $\vec{d}(3 ; 7 ;-7)$ vektorlar berilgan bo'lsin. Har bir vektorning yoyilmasini qolgan uchta vektorni bazis sifatida qabul qilib aniqlang.\\
\textbf{2.3.13.} $\vec{a}(2 ;-1 ; 3)$ va $\vec{b}(-6 ; 3 ;-9)$ vektorlar kollinearligini tekshiring. Ularning qaysi biri necha marta uzunligini, qanday yo'nalganligini, bir tomonga yoki qarama-qarshi ekanligini ko'rsating. \textbf{2.3.14.} $\alpha, \beta$ ning qanday qiymatida $\vec{a}=-2 \vec{\imath}+3 \vec{\jmath}+\beta \vec{k}$ va $\vec{b}=\alpha \vec{\imath}-$ $-6 \vec{\jmath}+2 \vec{k}$ vektorlar kollinear bo'ladi?\\
\textbf{2.3.15.} $\vec{a}=-\vec{\imath}+2 \vec{\jmath}+\beta \vec{k}$ va $\vec{b}=\alpha \vec{\imath}+6 \vec{\jmath}-2 \vec{k}$ vektorlar kollinear bo'lsa, $\alpha$ va $\beta$ ni toping.\\
\textbf{2.3.16.} $\vec{a}(2 ;-1 ; 3), \vec{b}(-6 ; 3 ;-9), \vec{c}(1 ; 2 ; 3), \vec{d}(-6 ; 12 ; 18)$ vektorlar berilgan. Ulardan qaysilari o'zaro kollinear?\\
\textbf{2.3.17.} $\vec{a}(\lambda \mathrm{n} ; \mathrm{n}-2 ; \mathrm{n}+1)$ va $\vec{b}(\mathrm{n}-3 ; \mu \mathrm{n} ; \mathrm{n}-1)$ vektorlar $\lambda$ va $\mu$ parametrlarning qanday qiymatlarida kollinear bo'lishini aniqlang.\\
\textbf{3.1.20.} Vektor $O x$ va $O z$ o'qlari bilan $\alpha=120^{\circ}, \gamma=45^{\circ}$ burchaklar tashkil qiladi. Shu vektor $O y$ o'qi bilan qanday burchak hosil qiladi?\\
\textbf{3.1.21.} Vektor $O y$ va $O z$ o'qlari bilan $\beta=45^{\circ}, \gamma=60^{\circ}$ burchaklar tashkil qiladi. Shu vektor $O x$ o'qi bilan qanday burchak hosil qiladi?\\
\textbf{3.1.22.} Vektorning 2 ta koordinatasi $x=4, y=-12$ berilgan. $|\vec{a}|=13$ bo'lgan holda vektorning uchinchi $z$ o'qining koordinatasini aniqlang.\\
\textbf{3.1.23.} Vektorning 2 ta koordinatasi $x=-16, z=15$ berilgan. $|\vec{a}|=25$ bo'lgan holda vektorning uchinchi $y$ o'qining koordinatasini aniqlang.\\
\textbf{3.1.24.} Birinchi koordinatalari mos ravishda $x=7, y=6$ ga teng bo'lib, uzunligi 11 ga teng vektorning boshi $A(2 ;-1 ; 5)$ nuqtada joylashgan bo'lsa, bu vektor oxirining koordinatalari topilsin.\\
\textbf{3.1.25.} Birinchi koordinatalari mos ravishda $y=-3, z=4$ ga teng bo'lib, uzunligi 13 ga teng vektorning oxiri $B(-5 ; 3 ;-2)$ nuqtada joylashgan bo'lsa, bu vektor boshining koordinatalari topilsin.\\
\textbf{3.1.26.} Birinchi koordinatalari mos ravishda $x=4, z=12$ ga teng bo'lib, uzunligi 13 ga teng vektorning boshi $A(4 ;-2 ;-3)$ nuqtada iovlashgan bo'lsa. hu vektor oxirining koordinatalari tonilsin.\\
\textbf{4.1.9.} $\vec{a}$ va $\vec{b}$ vektorlar o'zaro $\varphi=\frac{2 \pi}{3}$ burchak tashkil qiladi. $|\vec{a}|=3$ va $|\vec{b}|=4$ bo'lsa, quyidagilarni hisoblang:

\begin{enumerate}
  \item $\vec{a} \vec{b}$;
  \item $\vec{a}^{2}$;
  \item $\vec{b}^{2}$;
  \item $(\vec{a}+\vec{b})^{2}$;
  \item $(\vec{a}-\vec{b})^{2}$;
  \item $(3 \vec{a}+2 \vec{b})^{2}$; 7) $(2 \vec{a}-3 \vec{b})^{2}$;
  \item $(3 \vec{a}-2 \vec{b})(\vec{a}+2 \vec{b})$.
\end{enumerate}\\
\textbf{4.1.10.} $\vec{a}$ va $\vec{b}$ vektorlar o'zaro perpendikulyar, $\vec{c}$ vektor ularning har biri bilan $\varphi=\frac{\pi}{3}$ burchak hosil qilib, $|\vec{a}|=3,|\vec{b}|=5,|\vec{c}|=8$ ga teng bo'lsa, quyidagilarni hisoblang:
\begin{enumerate}
  \item $(3 \vec{a}-2 \vec{b})(\vec{b}+3 \vec{c})$;
  \item $(\vec{a}+\vec{b}+\vec{c})^{2}$;
  \item $(\vec{a}+2 \vec{b}-3 \vec{c})^{2}$;
  \item $(\vec{a}+\vec{b}-\vec{c})(\vec{a}+\vec{b}+\vec{c})$;
  \item $(2 \vec{a}-\vec{b}+3 \vec{c})(2 \vec{a}+\vec{b}-3 \vec{c})$
\end{enumerate}\\
\textbf{4.1.11.} $\vec{a}(5 ;-6 ; 1), \vec{b}(-4 ; 3 ; 0), \vec{c}(5 ;-8 ; 10)$ vektorlar berilgan bo'lsa,
\begin{enumerate}
  \item $3 \vec{a}^{2}-4 \vec{a} \vec{b}+2 \vec{c}^{2}$;
  \item $3 \vec{a} \vec{b}-4 \vec{b} \vec{c}-5 \vec{a} \vec{c}$;
  \item $2 \vec{a}^{2}+4 \vec{b}^{2}-5 \vec{c}^{2}$ ifodalarni hisoblang.
\end{enumerate}

\end{document}