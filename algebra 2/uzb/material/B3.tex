\(< Z,\ \  + >\) gruppasining \(nZ\) qism gruppasi boyisha qo\textquotesingle shni sinflarin toping.

\(A_{3}\) Juft orniga qoyishlar gruppasining \(S_{3}\) boyisha o\textquotesingle ng qo\textquotesingle shni sinflarin toping.

Noldan pariqli haqiyqiy sonlar multiplikativ gruppasi \(R\backslash\{ 0\}\) ning o\textquotesingle ng haqiyqiy sonlar qism gruppasi \(R_{+}\) boyisha faktor gruppasin toping.

\(Z\) Butun sonlarning additiv gruppasining \(n\) natural soniga karrali qism gruppasi boyisha qo\textquotesingle shni sinflarin toping.

\(\frac{3Z}{15Z}\) boyisha faktor halqasin toping.

Faktor gruppasin toping. \(\frac{3Z}{9Z}\),

Faktor gruppasin toping. \(\frac{5Z}{25Z}\)

\(Z\) Butun sonlarning additiv gruppasining \(nZ\ \ (n \in N)\) qism gruppasi boyisha qo\textquotesingle shni sinflarin toping.

\(M_{2}(Z)\) Halqada \(I = \left\{ \begin{bmatrix}
a & 0 \\
b & 0
\end{bmatrix}|a,b\mathbb{\in Z} \right\}\) ideal bo`ladimi?

Quyidagi \(G\) gruppaning \(H\) qism gruppasi boyisha o\textquotesingle ng qo\textquotesingle shni gruppalarni toping. \(G = S_{3}\) va \(H = \{ e,(1\ 2\ 3),(1\ 3\ 2)\}\)

Quyidagi \(G\) gruppaning \(H\) qism gruppasi boyisha o\textquotesingle ng qo\textquotesingle shni sinflarin toping. \(G = S_{3}\) va \(H = \{ e,(1\ 2\ 3),(1\ 3\ 2)\}\)

\(S_{3}\) simmetriyalik gruppa. \(H = \left\{ e,\begin{pmatrix}
1 & 2 & 3 \\
2 & 3 & 1
\end{pmatrix},\begin{pmatrix}
1 & 2 & 3 \\
3 & 1 & 2
\end{pmatrix} \right\}\) \(S_{3}\) ning qism gruppasi bola\textquotesingle di. \(S_{3}\) ning \(H\) qism gruppasi yordaminda barcha chap qo\textquotesingle shni sinflarin tuzing.

\(f:\ \ C^{*} \rightarrow R^{*}\) akslantirish gomomorfizm bo`ladimi: \(f(z) = |z|;\)

\(f:\ \ C^{*} \rightarrow R^{*}\) akslantirish gomomorfizm bo`ladimi: \(f(z) = 3 + |z|;\)

\(f:\ \ C^{*} \rightarrow R^{*}\) akslantirish gomomorfizm bo`ladimi: \(f(z) = 5|z|;\)

\(f:\ \ C^{*} \rightarrow R^{*}\) akslantirish gomomorfizm bo`ladimi: \(f(z) = |z|^{2};\)

\(f:\ \ C^{*} \rightarrow R^{*}\) akslantirish gomomorfizm bo`ladimi: \(f(z) = 1;\)

\(f:\ \ C^{*} \rightarrow R^{*}\) akslantirish gomomorfizm bo`ladimi: \(f(z) = 2.\)