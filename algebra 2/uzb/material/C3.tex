Tartibi \(n\) ga teng bo`lgan \(< a >\) sikl gruppasining barcha endomorfizmlarin toping.

\(f(n) = n^{2}\) akslantirishi \(Z\) gruppasining endomorfizmlarini bo`ladimi ?

Aytaylik gruppalarning \(f:G_{1} \rightarrow G_{2}\) epimorfizmi berilgan bo\textquotesingle lsin. \(G_{1}/Kerf\underline{\sim}\ G_{2}\) bo`lishini isbotlang.

\(f:nZ \rightarrow nZ\) gruppaning o\textquotesingle z-o\textquotesingle ziga izomorf bo`lishini isbotlang.

Butun sonlar gruppasi \(Z\) ning o\textquotesingle z-o\textquotesingle ziga izomorfizm bo`lishini ko`rsating.

Agar \(|G:H| = 2\) bolsa, Unda \(H\underline{\vartriangleleft}\ G\) bo`lishini isbotlang.

\(C\) kompleks sonlarning additiv gruppasining \(R\) haqiyqiy sonlarning qism gruppasi boyisha qo\textquotesingle shni sinflarin toping.

Bir o`zgariwshili ko`phadlar to`plami \(f(x)\) ko`phadlardi qo`shish va ko`paytirish amallariga nisbatan halqa tuzishini ko`rsating.

Kolsoning Ixtiyoriy sondagi ideallarining keshishmasi da uchbu halqaning ideali bo\textquotesingle lishin isbotlang.

Aytaylik \(K\) halqaning \(K'\) halqasiga \(f:K \rightarrow K'\) gomomorfizmi berilgan bo\textquotesingle lsin . \(Kerf\) qism halqasi \(K\) halqaning ideali bo\textquotesingle lishin va \(K/Kerf\) faktor halqaning \(f(K)\) halqasiga izomorf bo`lishini ko`rsating.

Aytaylik \(R\) va \(C\) xos haqiyqiy va kompleks sonlar halqalari va\(M = \left\{ \left. \ \begin{pmatrix}
\ \ \ a\ \ \ \ \ \ \ \ b \\
 - b\ \ \ \ \ \ \ \ a
\end{pmatrix}\ \  \right|a,\ b \in R \right\}\)bo\textquotesingle lsin . \(M\underline{\sim}\ C\) bo`lishini isbotlang.

Aytaylik, \(R\) va \(C\) xos rasional va haqiyqiy sonlar halqalari va\(M = \left\{ \left. \ \begin{pmatrix}
\ a\ \ \ \ \ \ \ \ b \\
\ 0\ \ \ \ \ \ \ \ a
\end{pmatrix}\ \  \right|\ \ \ \ \ \ a,\ \ b \in R\  \right\}\)bo\textquotesingle lsin . \(M\underline{\sim}\ C\) bo`lishini isbotlang.

\(M_{2}(R)\) halqa regulyar halqa bo`lishini ko`rsating.

\(\{\ a + b\sqrt{3}\left| \ \ \ \ \ a,\ \ b\  \in \ Q\ \ \} \right.\ \) to`plami maydon bo\textquotesingle lishin ko`rsating.

\(\mathbb{Z}\) Butun sonlar to`plamida \(x \oplus y = x + y - 1\) ko`rinishida aniqlangan. \((\mathbb{Z},\ \  \oplus )\)-- gruppa tashkil qiluvchi va uning \((\mathbb{Z}, + )\) gruppasina izomorf bo`lishinii isbotlang.

Berilgan \(f\) akslantirish \(G\) gruppani \(G_{1}\) gruppaga o\textquotesingle tkazuvchi gomomorfizm bo`ladimi? Agar gomomorfizm bolsa, Unda uning yadrosin toping. \(G = \left( \mathbb{R}^{+}, \cdot \right),G_{1} = \left( \mathbb{R}^{+}, \cdot \right),f(a) = a^{2}.\)

Berilgan \(f\) akslantirish \(G\) gruppani \(G_{1}\) gruppaga o\textquotesingle tkazuvchi gomomorfizm bo`ladimi? Agar gomomorfizm bolsa, Unda uning yadrosin toping.\(G = (\mathbb{C}\backslash\{ 0\}, \cdot ),G_{1} = \left( \mathbb{R}^{+}, \cdot \right),f(z) = |z|.\)

\(\frac{GL_{n}(\mathbb{C})}{SL_{n}(\mathbb{C})} \cong \mathbb{C}^{*}\) bo`lishini isbotlang.

Siklli gruppaning qism gruppasi siklli bo`lishini isbotlang.

Berilgan \(f\) akslantirish \(G\) gruppani \(G_{1}\) gruppaga o\textquotesingle tkazuvchi gomomorfizm bo`ladimi? Agar gomomorfizm bolsa, Unda uning yadrosin toping.\(G = (\mathbb{R}, + ),G_{1} = \left( \mathbb{R}^{+}, \cdot \right),f(a) = 2^{a}.\)

Aytaylik \(R = \left\{ \left. \ a + b\sqrt{2} \right|\ \ a,b \in Z \right\}\) va \(R' = \left\{ \left. \ \begin{pmatrix}
a & 2b \\
b & a
\end{pmatrix} \right|\ \ a,b \in Z \right\}\) halqalar berilgan bo\textquotesingle lsin . \(\varphi:R \rightarrow R'\) akslantirish izomorfizm bo`lishini isbotlang.

Aytaylik \(S_{n}\)- simmetrik gruppa va \(\varphi:S_{n} \rightarrow \mathbb{Z}_{2}\) akslantirish quyidagisha aniqlansa.\(\varphi(\sigma) = \left\{ \begin{matrix}
0,\ \ \ eger\ \ \sigma\ juft\ orniga\ \ qoy\imath sh\ \ bolsa, \\
1,\ \ eger\ \ \sigma\ toq\ orniga\ \ qoy\imath sh\ bolsa
\end{matrix} \right.\ \) unda \(\varphi\) akslantirish gomomorfizm bo`lishini isbotlang.

Har qanday siklli gruppa abellik(kommutativ) gruppa bo`lishini isbotlang.

Tartibi \emph{n} ga teng bo`lgan ixtiyoriy siklli gruppa \((\mathbb{Z}_{n},\ \  +_{n})\) gruppaga, ixtiyoriy sheksiz siklli gruppa \((\mathbb{Z},\ \  + )\) gruppaga izomorf boladi.

\(S_{3}\) gruppaning \(H = \left\{ e,\ \ (123),\ (132) \right\}\) qism gruppasi normal qism gruppa bo`ladimi, Agar bolsa \(\frac{S_{3}}{H}\) faktor gruppasin aniqlang.

\(\frac{GL_{n}(\mathbb{C})}{SL_{n}(\mathbb{C})} \cong \mathbb{C}^{*}\) bo`lishini isbotlang.
