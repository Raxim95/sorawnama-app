Aytaylik \(f:\ G \rightarrow G_{1}\) akslantirishshi epimorfizm bo\textquotesingle lsin . Agar \(H\) \(G\) ning normal qism gruppasi bolsa, unda \(f(H)\) ta \(G_{1}\) ning normal qism gruppasi bolishin isbotlang.

Aytaylik \(G_{1}\) va \(G_{2}\) gruppalarining \(f:G_{1} \rightarrow G_{2}\) gomomorfizmi berilgan bo\textquotesingle lsin . Agar \(H \leq G_{1}\) bolsa, \(f(H) = H \leq G_{2}\) bo`lishini isbotlang.

Tartibi \(n\) ga teng \(< a >\) elementidan hosil bo\textquotesingle lgan siklli gruppaning o\textquotesingle z-o\textquotesingle ziga bo`lgan barcha gomomorf akslantirishlarin toping.

\(f:a^{n} \rightarrow a^{n}\) \((a \neq 0,\  \pm 1 \in R,\ \ \ n \in Z)\) gruppaning o\textquotesingle z-o\textquotesingle ziga izomorf bo`lishini isbotlang.

Tartibi \(n\) ga teng bo`lgan \(< a >\) sikl gruppasining o\textquotesingle z-o\textquotesingle ziga gomomorfizm bo`lishini ko`rsating.

Tartibi 24 ga teng bo`lgan \(< a >\) sikl gruppasining tartibi 4 ga teng bo`lgan barcha elementlarin ko`rsating.

Tartibi 6 ga teng \(< a >\) elementidan hosil bo\textquotesingle lgan siklli gruppaning tartibi 18 ga teng \(< b >\) elementidan hosil bo\textquotesingle lgan siklli gruppaga bo`lgan barcha gomomorf akslantirishlarin toping.

Tartibi 12 ga teng \(< a >\) elementidan hosil bo\textquotesingle lgan siklli gruppaning Tartibi 15 ga teng \(< b >\) elementidan hosil bo\textquotesingle lgan siklli gruppaga bo`lgan barcha gomomorf akslantirishlarin toping.

Butun sonlar juftlarining to`plami \(K = \{(a,\ \ b)\left| \ \ \ a,\ \ b \in Z \right.\ \}\) quyidagi \(\begin{matrix}
(a_{1},\ \ \ b_{1}) + (a_{2},\ \ b_{2}) = (a_{1} + a_{2},\ \ \ \ \ \ b_{1} + b_{2}), \\
(a_{1},\ \ b_{1}) \cdot (a_{2},\ \ b_{2}) = (a_{1} \cdot a_{2},\ \ \ \ \ b_{1} \cdot b_{2})
\end{matrix}\)berilgan qo`shish va ko`paytirish amallariga nisbatan halqa tuzishini ko`rsating va uchbu halqadagi barcha nolning bo`luvchilarin toping.

{[}-1; 1{]} kesmasinda uzliksiz bo`lgan funksiyalarning halqasinda nolning bo`luvchilariga misollar keltiring.

Tartibi \(n \geq 2\ \ \) bo`lgan haqiyqiy elementli diogonal matrisalar, matrisalarni qo`shish va ko`paytirish amallariga nisbatan kommutativ halqa bolishini isbotlang va uchbu halqadaǵi nolning bo`luvchilarin toping:\(\begin{pmatrix}
\ a_{11} & 0\ \  & 0 & ... & 0\ \  \\
0\ \  & a_{22} & 0 & ... & 0\ \  \\
... & ... & ... & ... & ... \\
0\ \  & 0\ \  & 0 & ... & a_{nn}
\end{pmatrix}.\)

\(\mathbb{Z}\) butun sonlar to`plamida qo`shish va ko`paytirish amallari \(x \oplus y = x + y - 1\) va \(x \otimes y = x + y - xy\) ko`rinishida aniqlangan. \((\mathbb{Z},\ \  \oplus , \otimes )\) -- halqa bo`lishini va uning \((\mathbb{Z}, + , \cdot )\) halqasina izomorf bo`lishini isbotlang.

Quyidagi matricalar to`plami \((GL_{2}^{\ }\ (R), \cdot )\) gruppaning qism gruppasi bo`lishini isbotlang. \(S = \left\{ \begin{pmatrix}
a & 0 \\
0 & a
\end{pmatrix},a \neq 0 \right\}\)

\(GL(2,\mathbb{\ \ R})\) gruppasining \(\begin{pmatrix}
1 & 1 \\
0 & 1
\end{pmatrix}\) elementi bilan tuzilgan siklli qism gruppasining barcha elelmentlarin toping.

\(GL(2,\mathbb{\ \ R})\) gruppasining \(\begin{pmatrix}
0 & - 1 \\
 - 1 & 0
\end{pmatrix}\) elementi bilan tuzilgan siklli qism gruppasining barcha elelmentlarin toping.

\(GL(2,\mathbb{\ \ R})\) gruppasining \(\begin{pmatrix}
3 & 0 \\
0 & 2
\end{pmatrix}\) elementi bilan tuzilgan siklli qism gruppasining barcha elelmentlarin toping.

\emph{G} gruppa va uning \emph{H} normal qism gruppasi uchun faktor gruppa elementlarin toping.\(G = (\mathbb{Z}_{12}, + )\) hám \(H = \left\langle \overline{4} \right\rangle\)

\(GL(2,\mathbb{\ \ R})\) gruppasining \(\begin{pmatrix}
0 & - 2 \\
 - 2 & 0
\end{pmatrix}\) elementi bilan tuzilgan siklli qism gruppasining barcha elelmentlarin toping.