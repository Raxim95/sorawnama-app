Quyidagi gruppalarning barcha qism gruppalarin toping: \(S_{3},\)

\(Z_{6}\) gruppasining barcha qism gruppalarin toping.

\(Z_{12}\) gruppasining barcha qism gruppalarin toping.

\(Z_{12}\) siklli gruppani o`zining qism gruppalarining tog\textquotesingle ri kopaytmaga yoying.

\(A_{3}\) Juft orniga qoyishlar gruppasining \(S_{3}\) normal qism gruppa ekenin isbotlang.

\(Z_{3}\) maydoninda \(f(x) = 5x^{3} + 3x^{2} - x + 1\) va \(g(x) = 5x^{2} + 3x + 1\) ko`phadlarining eng katta uminiy bo`liwshisin toping.

\(\mathbf{Z}_{\mathbf{5}}\) maydoninda \(x^{4} + 3x^{3} + 2x^{2} + x + 4\) ko`phadsin keltirilmas ko`phadlarga yoying.

\(S_{3}\) gruppasining \(H = \left\{ e,\ \ (12) \right\}\) qism gruppasi normal qism gruppa bo`ladimi.

\(S_{3}\) gruppaning \(T = \left\{ x \in S_{3}|x^{2} = e \right\}\)qism to`plami qism gruppa bo`ladimi bo`ladimi?

Quyidagi to`plamning \(M_{2}(\mathbb{R})\)matricalar halqaning qism halqasi bo`lishini isbotlang. \(A = \left\{ \left. \ \begin{pmatrix}
a & b \\
0 & c
\end{pmatrix} \right|a,b,c\mathbb{\in R} \right\}\)

Quyidagi to`plamning \(M_{2}(\mathbb{R})\)matricalar halqaning qism halqasi bo`lishini isbotlang. \(A = \left\{ \left. \ \begin{pmatrix}
a & b \\
 - b & a
\end{pmatrix} \right|a,b\mathbb{\in R} \right\}\)

Quyidagi to`plamning \(M_{2}(\mathbb{R})\)matricalar halqaning qism halqasi bo`lishini isbotlang. \(A = \left\{ \left. \ \begin{pmatrix}
a & b \\
0 & a
\end{pmatrix} \right|a,b\mathbb{\in R} \right\}\)

Quyidagi to`plamning \(M_{2}(\mathbb{R})\)matricalar halqaning qism halqasi bo`lishini isbotlang. \(A = \left\{ \left. \ \begin{pmatrix}
a & b\sqrt{3} \\
 - b\sqrt{3} & a
\end{pmatrix} \right|a,b\mathbb{\in Q} \right\}\)

Quyidagi to`plamning \(M_{2}(\mathbb{R})\)matricalar halqaning qism halqasi bo`lishini isbotlang. \(A = \left\{ \left. \ \begin{pmatrix}
a + b & b \\
 - b & a
\end{pmatrix} \right|a,b\mathbb{\in Z} \right\}\)