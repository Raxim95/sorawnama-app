\documentclass{article}
\usepackage[fontsize=14pt]{fontsize}
\usepackage[utf8]{inputenc}
\usepackage[T2A]{fontenc}
% \usepackage{unicode-math}

\usepackage{array}
\usepackage[a4paper,
left=7mm,
right=5mm,
top=7mm,]{geometry}
\usepackage{amsmath}
\usepackage{amssymb}
\usepackage{amsfonts}
\usepackage{setspace}



\renewcommand{\baselinestretch}{1} 

\everymath{\displaystyle}
\everydisplay{\displaystyle}
% \linespread{1.25}

\DeclareMathOperator{\sign}{sign}


\begin{document}

\pagenumbering{gobble}


\begin{tabular}{m{17cm}}
\textbf{1-variant}
\newline

\textbf{1.} Kompleks sanlar hám olar ústinde ámeller. Muavr formulası.  \\
\textbf{2.} Kópaǵzalılardıń EÚUB si. Keltirilmeytuǵın   kópaǵzalılar. \\
\textbf{3.} Algebralıq formada esaplań: $\frac{5+i}{(1-2 i)(5-i)}$; \\
\textbf{4.} Qaldıqlı bóliwdi orınlań:  $x^3-3 x^2+4 x-1$ di $x^2-x+1$ ge; \\
\textbf{5.} Haqıyqıy sanlar maydanında ápiwayı bólsheklerge jayıń:  $\frac{x^2}{(x-1)(x+2)(x+3)}$; \\

\end{tabular}
\vspace{1cm}


\begin{tabular}{m{17cm}}
\textbf{2-variant}
\newline

\textbf{1.} Bir belgisizli kópaǵzalılar. Gorner sxeması. Bezu teoreması.  \\
\textbf{2.} Qaldıqlı bóliw.  \\
\textbf{3.} Kompleks sannan koren shıǵarıń: $\sqrt{\frac{1+i}{\sqrt{3}-i}}$; \\
\textbf{4.} Qaldıqlı bóliwdi orınlań: $2 x^4-3 x^3+4 x^2-5 x+6$ nı $x^2-3 x+1$ ge; \\
\textbf{5.} Kópaǵzalılardıń eń úlken ulıwma bóliwshisin tabıń:  $f(x)=2 x^6-5 x^5+14 x^4+36 x^3+86 x^2+12 x-31$ hám $g(x)=2 x^5-9 x^4+2 x^3+37 x^2+10 x-14$ \\

\end{tabular}
\vspace{1cm}


\begin{tabular}{m{17cm}}
\textbf{3-variant}
\newline

\textbf{1.} Kompleks sannan koren shıǵarıw. \\
\textbf{2.} Ratsional bólshekler hám olardı eń ápiwayı bólsheklerge ajıratıw. \\
\textbf{3.} Kompleks sannan koren shıǵarıń:  $\sqrt[3]{i}$; \\
\textbf{4.} Gorner sxemasınan paydalanıp, $f(x)$ kópaǵzalısın $x-x_0$ dárejeleri boyınsha jayıń hám $f\left(x_0\right)$ di esaplań: $f(x)=3 x^5+x^4-19 x^2-13 x-10 ; \quad x_0=2$ \\
\textbf{5.} Kópaǵzalılardıń eń úlken ulıwma bóliwshisin tabıń:  $f(x)=x^5+3 x^2-2 x+2$ hám $g(x)=x^6+x^5+x^4-3 x^2+2 x-6$ \\

\end{tabular}
\vspace{1cm}


\begin{tabular}{m{17cm}}
\textbf{4-variant}
\newline

\textbf{1.} Kompleks sanlar hám olar ústinde ámeller. Muavr formulası.  \\
\textbf{2.} Kompleks sannan koren shıǵarıw. \\
\textbf{3.} Esaplań:  $\frac{(-1+i \sqrt{3})^{15}}{(1-i)^{20}}$; \\
\textbf{4.} Gorner sxemasınan paydalanıp, $f(x)$ kópaǵzalısın $x-x_0$ dárejeleri boyınsha jayıń hám $f\left(x_0\right)$ di esaplań: $f(x)=x^4-3 x^3-10 x^2+2 x+5 ; \quad x_0=-2$. \\
\textbf{5.} Haqıyqıy sanlar maydanında ápiwayı bólsheklerge jayıń:  $\frac{1}{x^3-1}$; \\

\end{tabular}
\vspace{1cm}


\begin{tabular}{m{17cm}}
\textbf{5-variant}
\newline

\textbf{1.} Qaldıqlı bóliw.  \\
\textbf{2.} Kópaǵzalılardıń EÚUB si. Keltirilmeytuǵın   kópaǵzalılar. \\
\textbf{3.} Trigonometriyalıq kórinisin tabıń: $-2+i$; \\
\textbf{4.} Qanday shártler orınlanǵanda $x^4+p x^2-q$ kópaǵzalısı $x^2+m q+1$ kópaǵzalısına qaldıqsız bólinedi?. \\
\textbf{5.} Kompleks sanlar maydanında ápiwayı bólsheklerge jayıń: $\frac{3+x}{(x-1)\left(x^2+1\right)}$; \\

\end{tabular}
\vspace{1cm}


\begin{tabular}{m{17cm}}
\textbf{6-variant}
\newline

\textbf{1.} Bir belgisizli kópaǵzalılar. Gorner sxeması. Bezu teoreması.  \\
\textbf{2.} Ratsional bólshekler hám olardı eń ápiwayı bólsheklerge ajıratıw. \\
\textbf{3.} Algebralıq formada esaplań: $\frac{(5+2 i)(4-3 i)}{(1-2 i)(1+3 i)}$; \\
\textbf{4.} Gorner sxemasınan paydalanıp, $f(x)$ kópaǵzalısın $x-x_0$ dárejeleri boyınsha jayıń hám $f\left(x_0\right)$ di esaplań: $f(x)=x^5-3 x^4+3 x^2-2 x+1 ; \quad x_0=-1$ \\
\textbf{5.} Kompleks sanlar maydanında ápiwayı bólsheklerge jayıń: $\frac{3 x^2+6 x-23}{(x-1)^3(x+1)^2(x-2)}$ \\

\end{tabular}
\vspace{1cm}


\begin{tabular}{m{17cm}}
\textbf{7-variant}
\newline

\textbf{1.} Bir belgisizli kópaǵzalılar. Gorner sxeması. Bezu teoreması.  \\
\textbf{2.} Qaldıqlı bóliw.  \\
\textbf{3.} Algebralıq formada esaplań: $\frac{1+i \sqrt{3}}{1-i \sqrt{3}}-(1-i)^2$; \\
\textbf{4.} Gorner sxemasınan paydalanıp, $f(x)$ kópaǵzalısın $x-x_0$ dárejeleri boyınsha jayıń hám $f\left(x_0\right)$ di esaplań: $f(x)=3 x^4+4 x^3+5 x^2+x+33 ; \quad x_0=-2$ \\
\textbf{5.} Kópaǵzalılardıń eń úlken ulıwma bóliwshisin tabıń:  $f(x)=x^4-10 x^2+1$ hám $g(x)=x^4-4 \sqrt{2} x^3+6 x^2+4 \sqrt{2} x+1$ \\

\end{tabular}
\vspace{1cm}


\begin{tabular}{m{17cm}}
\textbf{8-variant}
\newline

\textbf{1.} Kompleks sannan koren shıǵarıw. \\
\textbf{2.} Kópaǵzalılardıń EÚUB si. Keltirilmeytuǵın   kópaǵzalılar. \\
\textbf{3.} $x, y \in R$, tabıń, egerde:  $\frac{8 i}{x}+i y-2=7 i-\frac{10}{x}+y$. \\
\textbf{4.} Gorner sxemasınan paydalanıp, $f(x)$ kópaǵzalısın $x-x_0$ dárejeleri boyınsha jayıń hám $f\left(x_0\right)$ di esaplań: $f(x)=x^4+2 x^3-3 x^2-4 x+1 ; \quad x_0=-1$ \\
\textbf{5.} Haqıyqıy sanlar maydanında ápiwayı bólsheklerge jayıń:  $\frac{x}{(x+1)\left(x^2+1\right)^2}$; \\

\end{tabular}
\vspace{1cm}


\begin{tabular}{m{17cm}}
\textbf{9-variant}
\newline

\textbf{1.} Kompleks sanlar hám olar ústinde ámeller. Muavr formulası.  \\
\textbf{2.} Ratsional bólshekler hám olardı eń ápiwayı bólsheklerge ajıratıw. \\
\textbf{3.} Algebralıq formada esaplań: $\sqrt{-15+8 i}$; \\
\textbf{4.} Gorner sxemasınan paydalanıp, $f(x)$ kópaǵzalısın $x-x_0$ dárejeleri boyınsha jayıń hám $f\left(x_0\right)$ di esaplań: $f(x)=x^4-3 x^3+6 x^2-10 x+16 ; x_0=4$ \\
\textbf{5.} Kópaǵzalılardıń eń úlken ulıwma bóliwshisin tabıń: $f(x)=x^5+x^4-x^3-3 x^2-3 x-1$ hám $g(x)=x^4-2 x^3-x^2-2 x+1$ \\

\end{tabular}
\vspace{1cm}


\begin{tabular}{m{17cm}}
\textbf{10-variant}
\newline

\textbf{1.} Kópaǵzalılardıń EÚUB si. Keltirilmeytuǵın   kópaǵzalılar. \\
\textbf{2.} Kompleks sanlar hám olar ústinde ámeller. Muavr formulası.  \\
\textbf{3.} Algebralıq formada esaplań: $\frac{(2-3 i)(4-i)}{5-i}$; \\
\textbf{4.} Qaldıqlı bóliwdi orınlań: $x^4-3 x^2+2 x-5$ ti $x^2+2 x-1$ ge; \\
\textbf{5.} Evklid algoritminen paydalanıp, $f(x)$ hám $g(x)$ kópaǵzalıları ushın $f(x) \cdot u(x)+g(x) \cdot v(x)=d(x)$ teńligin qanaatlandırıwshı $u(x)$ hám $v(x)$ kópaǵzalıların tabıń, bunda $d(x)=(f(x), g(x))$:  $f(x)=x^5+3 x^4+x^3+x^2+3 x+1$ hám $g(x)=x^4+2 x^3+x+2$ \\

\end{tabular}
\vspace{1cm}


\begin{tabular}{m{17cm}}
\textbf{11-variant}
\newline

\textbf{1.} Ratsional bólshekler hám olardı eń ápiwayı bólsheklerge ajıratıw. \\
\textbf{2.} Kompleks sannan koren shıǵarıw. \\
\textbf{3.}  $x, y \in R$, tabıń, egerde: $(2-3 i) x+(3+2 i) y=2-5 i$; \\
\textbf{4.} Qaldıqlı bóliwdi orınlań: $2 x^5-5 x^3-8 x$ ti $x+3$ ge; \\
\textbf{5.} Evklid algoritminen paydalanıp, $f(x)$ hám $g(x)$ kópaǵzalıları ushın $f(x) \cdot u(x)+g(x) \cdot v(x)=d(x)$ teńligin qanaatlandırıwshı $u(x)$ hám $v(x)$ kópaǵzalıların tabıń, bunda $d(x)=(f(x), g(x))$:  $f(x)=x^4-x^3-4 x^2+4 x+1$ hám $g(x)=x^2-x-1$ \\

\end{tabular}
\vspace{1cm}


\begin{tabular}{m{17cm}}
\textbf{12-variant}
\newline

\textbf{1.} Qaldıqlı bóliw.  \\
\textbf{2.} Bir belgisizli kópaǵzalılar. Gorner sxeması. Bezu teoreması.  \\
\textbf{3.} Algebralıq formada esaplań: $\frac{(1+i)^n}{(1-i)^{n-2}}$; \\
\textbf{4.} Qaldıqlı bóliwdi orınlań: $4 x^3+x^2$ tı $x+1+i$ ge; \\
\textbf{5.} Evklid algoritminen paydalanıp, $f(x)$ hám $g(x)$ kópaǵzalıları ushın $f(x) \cdot u(x)+g(x) \cdot v(x)=d(x)$ teńligin qanaatlandırıwshı $u(x)$ hám $v(x)$ kópaǵzalıların tabıń, bunda $d(x)=(f(x), g(x))$:  $f(x)=x^4-2 x^3-16 x^2+5 x+9$ hám $g(x)=2 x^3-x^2-5 x+4$ \\

\end{tabular}
\vspace{1cm}


\begin{tabular}{m{17cm}}
\textbf{13-variant}
\newline

\textbf{1.} Ratsional bólshekler hám olardı eń ápiwayı bólsheklerge ajıratıw. \\
\textbf{2.} Qaldıqlı bóliw.  \\
\textbf{3.} Teńlemeni sheshiń:  $x^2-(5-3 i) x+(4-7 i)=0$. \\
\textbf{4.} Gorner sxemasınan paydalanıp, $f(x)$ kópaǵzalısın $x-x_0$ dárejeleri boyınsha jayıń hám $f\left(x_0\right)$ di esaplań:  $f(x)=2 x^5+x^3-3 x^2+1 ; \quad x_0=1$ \\
\textbf{5.} Haqıyqıy sanlar maydanında ápiwayı bólsheklerge jayıń:  $\frac{1}{\left(x^4-4\right)^2}$; \\

\end{tabular}
\vspace{1cm}


\begin{tabular}{m{17cm}}
\textbf{14-variant}
\newline

\textbf{1.} Kópaǵzalılardıń EÚUB si. Keltirilmeytuǵın   kópaǵzalılar. \\
\textbf{2.} Bir belgisizli kópaǵzalılar. Gorner sxeması. Bezu teoreması.  \\
\textbf{3.} Algebralıq formada esaplań: $\sqrt{8+6 i}$; г) $\sqrt{2-3 i}$. \\
\textbf{4.} Gorner sxemasınan paydalanıp, $f(x)$ kópaǵzalısın $x-x_0$ dárejeleri boyınsha jayıń hám $f\left(x_0\right)$ di esaplań: $f(x)=2 x^5-5 x^3-8 x ; \quad x_0=-3$ \\
\textbf{5.} Evklid algoritminen paydalanıp, $f(x)$ hám $g(x)$ kópaǵzalıları ushın $f(x) \cdot u(x)+g(x) \cdot v(x)=d(x)$ teńligin qanaatlandırıwshı $u(x)$ hám $v(x)$ kópaǵzalıların tabıń, bunda $d(x)=(f(x), g(x))$:  $f(x)=3 x^3-2 x^2+x+2$ hám $g(x)=x^2-x+1$ \\

\end{tabular}
\vspace{1cm}


\begin{tabular}{m{17cm}}
\textbf{15-variant}
\newline

\textbf{1.} Kompleks sanlar hám olar ústinde ámeller. Muavr formulası.  \\
\textbf{2.} Kompleks sannan koren shıǵarıw. \\
\textbf{3.} Teńlemeni sheshiń:  $x^2-(3-2 i) x+(5-5 i)=0$; \\
\textbf{4.} Gorner sxemasınan paydalanıp, $f(x)$ kópaǵzalısın $x-x_0$ dárejeleri boyınsha jayıń hám $f\left(x_0\right)$ di esaplań: $f(x)=5 x^5-19 x^3-7 x^2+9 x+3 ; \quad x_0=2$ \\
\textbf{5.} Haqıyqıy sanlar maydanında ápiwayı bólsheklerge jayıń:  $\frac{x}{\left(x^2-1\right)^2}$; \\

\end{tabular}
\vspace{1cm}


\begin{tabular}{m{17cm}}
\textbf{16-variant}
\newline

\textbf{1.} Ratsional bólshekler hám olardı eń ápiwayı bólsheklerge ajıratıw. \\
\textbf{2.} Kompleks sannan koren shıǵarıw. \\
\textbf{3.} Trigonometriyalıq kórinisin tabıń: $3 i$; \\
\textbf{4.} Qaldıqlı bóliwdi orınlań: $x^4-2 x^3+4 x^2-6 x+8$ di $x-1$ ge; \\
\textbf{5.} Haqıyqıy sanlar maydanında ápiwayı bólsheklerge jayıń:  $\frac{x^2}{x^6+27}$; \\

\end{tabular}
\vspace{1cm}


\begin{tabular}{m{17cm}}
\textbf{17-variant}
\newline

\textbf{1.} Kompleks sanlar hám olar ústinde ámeller. Muavr formulası.  \\
\textbf{2.} Qaldıqlı bóliw.  \\
\textbf{3.} Trigonometriyalıq kórinisin tabıń: -5 ;  \\
\textbf{4.} Gorner sxemasınan paydalanıp, $f(x)$ kópaǵzalısın $x-x_0$ dárejeleri boyınsha jayıń hám $f\left(x_0\right)$ di esaplań: $f(x)=x^4-2 x^3+4 x^2-6 x+8 ; \quad x_0=1$ \\
\textbf{5.} Kompleks sanlar maydanında ápiwayı bólsheklerge jayıń: $\frac{x^2}{x^4-1}$; \\

\end{tabular}
\vspace{1cm}


\begin{tabular}{m{17cm}}
\textbf{18-variant}
\newline

\textbf{1.} Bir belgisizli kópaǵzalılar. Gorner sxeması. Bezu teoreması.  \\
\textbf{2.} Kópaǵzalılardıń EÚUB si. Keltirilmeytuǵın   kópaǵzalılar. \\
\textbf{3.} Algebralıq formada esaplań: $(1+2 i)^5-(1-2 i)^5$; \\
\textbf{4.} Qanday shártler orınlanǵanda $x^3+p x+q$ kópaǵzalısı $x^2+m x-1$ kópaǵzalısına qaldıqsız bólinedi?. \\
\textbf{5.} Kópaǵzalılardıń eń úlken ulıwma bóliwshisin tabıń:  $f(x)=x^4+7 x^3+19 x^2+23 x+10$ hám $g(x)=x^4+7 x^3+18 x^2+22 x+12$ \\

\end{tabular}
\vspace{1cm}


\begin{tabular}{m{17cm}}
\textbf{19-variant}
\newline

\textbf{1.} Ratsional bólshekler hám olardı eń ápiwayı bólsheklerge ajıratıw. \\
\textbf{2.} Kópaǵzalılardıń EÚUB si. Keltirilmeytuǵın   kópaǵzalılar. \\
\textbf{3.} Teńlemeni sheshiń:  $x^2+3 x-10 i=0$; \\
\textbf{4.} Qaldıqlı bóliwdi orınlań: $x^3-3 x^2-x-1$ di $3 x^2-2 x+1$ ge; \\
\textbf{5.} Kópaǵzalılardıń eń úlken ulıwma bóliwshisin tabıń:  $f(x)=x^4+x^3-4 x+5$ hám $g(x)=2 x^3-x^2-2 x+2$ \\

\end{tabular}
\vspace{1cm}


\begin{tabular}{m{17cm}}
\textbf{20-variant}
\newline

\textbf{1.} Kompleks sannan koren shıǵarıw. \\
\textbf{2.} Bir belgisizli kópaǵzalılar. Gorner sxeması. Bezu teoreması.  \\
\textbf{3.} Kompleks sannan koren shıǵarıń: $\sqrt[4]{(2+2 i)(-1+i \sqrt{3})}$ \\
\textbf{4.} Qaldıqlı bóliwdi orınlań:  $x^3-x^2-x$ t1 $x-1+2 i$ ge; \\
\textbf{5.} Evklid algoritminen paydalanıp, $f(x)$ hám $g(x)$ kópaǵzalıları ushın $f(x) \cdot u(x)+g(x) \cdot v(x)=d(x)$ teńligin qanaatlandırıwshı $u(x)$ hám $v(x)$ kópaǵzalıların tabıń, bunda $d(x)=(f(x), g(x))$:  $f(x)=x^5+5 x^4+9 x^3+7 x^2+5 x+3$ hám $g(x)=x^4+2 x^3+2 x^2+x+1$ \\

\end{tabular}
\vspace{1cm}


\begin{tabular}{m{17cm}}
\textbf{21-variant}
\newline

\textbf{1.} Kompleks sanlar hám olar ústinde ámeller. Muavr formulası.  \\
\textbf{2.} Qaldıqlı bóliw.  \\
\textbf{3.} Esaplań:  $(1+i)^{25}$; \\
\textbf{4.} Gorner sxemasınan paydalanıp, $f(x)$ kópaǵzalısın $x-x_0$ dárejeleri boyınsha jayıń hám $f\left(x_0\right)$ di esaplań: $f(x)=x^4+2 x^3-3 x^2-4 x+1 ; \quad x_0=-1$ \\
\textbf{5.} Haqıyqıy sanlar maydanında ápiwayı bólsheklerge jayıń:  $\frac{1}{x^4+4}$; \\

\end{tabular}
\vspace{1cm}


\begin{tabular}{m{17cm}}
\textbf{22-variant}
\newline

\textbf{1.} Kompleks sanlar hám olar ústinde ámeller. Muavr formulası.  \\
\textbf{2.} Ratsional bólshekler hám olardı eń ápiwayı bólsheklerge ajıratıw. \\
\textbf{3.} Esaplań:  $\frac{(-1-i \sqrt{3})^{10}}{(-1+i)^{16}}$. \\
\textbf{4.} Gorner sxemasınan paydalanıp, $f(x)$ kópaǵzalısın $x-x_0$ dárejeleri boyınsha jayıń hám $f\left(x_0\right)$ di esaplań: $f(x)=x^4-3 x^3+6 x^2-10 x+16 ; x_0=4$ \\
\textbf{5.} Evklid algoritminen paydalanıp, $f(x)$ hám $g(x)$ kópaǵzalıları ushın $f(x) \cdot u(x)+g(x) \cdot v(x)=d(x)$ teńligin qanaatlandırıwshı $u(x)$ hám $v(x)$ kópaǵzalıların tabıń, bunda $d(x)=(f(x), g(x))$:  $f(x)=x^5-5 x^4-2 x^3+12 x^2-2 x+12$ hám $g(x)=x^3-5 x^2-3 x+17$ \\

\end{tabular}
\vspace{1cm}


\begin{tabular}{m{17cm}}
\textbf{23-variant}
\newline

\textbf{1.} Kompleks sannan koren shıǵarıw. \\
\textbf{2.} Qaldıqlı bóliw.  \\
\textbf{3.} Trigonometriyalıq kórinisin tabıń: $1-i$; \\
\textbf{4.} Gorner sxemasınan paydalanıp, $f(x)$ kópaǵzalısın $x-x_0$ dárejeleri boyınsha jayıń hám $f\left(x_0\right)$ di esaplań: $f(x)=x^4-2 x^3+4 x^2-6 x+8 ; \quad x_0=1$ \\
\textbf{5.} Kompleks sanlar maydanında ápiwayı bólsheklerge jayıń: $\frac{1}{x\left(x^2+1\right)}$; \\

\end{tabular}
\vspace{1cm}


\begin{tabular}{m{17cm}}
\textbf{24-variant}
\newline

\textbf{1.} Bir belgisizli kópaǵzalılar. Gorner sxeması. Bezu teoreması.  \\
\textbf{2.} Kópaǵzalılardıń EÚUB si. Keltirilmeytuǵın   kópaǵzalılar. \\
\textbf{3.} Algebralıq formada esaplań: $(5+4 i)+(3-7 i)-(2+5 i)$; \\
\textbf{4.} Gorner sxemasınan paydalanıp, $f(x)$ kópaǵzalısın $x-x_0$ dárejeleri boyınsha jayıń hám $f\left(x_0\right)$ di esaplań: $f(x)=3 x^4+4 x^3+5 x^2+x+33 ; \quad x_0=-2$ \\
\textbf{5.} Kópaǵzalılardıń eń úlken ulıwma bóliwshisin tabıń:  $f(x)=x^4+x^3-3 x^2-4 x-1$ hám $g(x)=x^3+x^2-x-1$ \\

\end{tabular}
\vspace{1cm}


\begin{tabular}{m{17cm}}
\textbf{25-variant}
\newline

\textbf{1.} Ratsional bólshekler hám olardı eń ápiwayı bólsheklerge ajıratıw. \\
\textbf{2.} Kópaǵzalılardıń EÚUB si. Keltirilmeytuǵın   kópaǵzalılar. \\
\textbf{3.}  $x, y \in R$, tabıń, egerde: $2+5 i x-3 i y=14 i+3 x-5 y$; \\
\textbf{4.} Gorner sxemasınan paydalanıp, $f(x)$ kópaǵzalısın $x-x_0$ dárejeleri boyınsha jayıń hám $f\left(x_0\right)$ di esaplań: $f(x)=2 x^5-5 x^3-8 x ; \quad x_0=-3$ \\
\textbf{5.} Kópaǵzalılardıń eń úlken ulıwma bóliwshisin tabıń:  $f(x)=x^4-4 x^3+1$ hám $g(x)=x^3-3 x^2+1$ \\

\end{tabular}
\vspace{1cm}


\begin{tabular}{m{17cm}}
\textbf{26-variant}
\newline

\textbf{1.} Kompleks sannan koren shıǵarıw. \\
\textbf{2.} Qaldıqlı bóliw.  \\
\textbf{3.} Algebralıq formada esaplań: $(2+i)^7+(2-i)^7$. \\
\textbf{4.} Qaldıqlı bóliwdi orınlań: $2 x^4-3 x^3+4 x^2-5 x+6$ nı $x^2-3 x+1$ ge; \\
\textbf{5.} Evklid algoritminen paydalanıp, $f(x)$ hám $g(x)$ kópaǵzalıları ushın $f(x) \cdot u(x)+g(x) \cdot v(x)=d(x)$ teńligin qanaatlandırıwshı $u(x)$ hám $v(x)$ kópaǵzalıların tabıń, bunda $d(x)=(f(x), g(x))$:  $f(x)=3 x^5+5 x^4-16 x^3-6 x^2-5 x-6$ hám $g(x)=3 x^4-4 x^3-x^2-x-2$ \\

\end{tabular}
\vspace{1cm}


\begin{tabular}{m{17cm}}
\textbf{27-variant}
\newline

\textbf{1.} Kompleks sanlar hám olar ústinde ámeller. Muavr formulası.  \\
\textbf{2.} Bir belgisizli kópaǵzalılar. Gorner sxeması. Bezu teoreması.  \\
\textbf{3.} Algebralıq formada esaplań:  $(1+i)(2+i)+\frac{5}{1+2 i}$; \\
\textbf{4.} Gorner sxemasınan paydalanıp, $f(x)$ kópaǵzalısın $x-x_0$ dárejeleri boyınsha jayıń hám $f\left(x_0\right)$ di esaplań:  $f(x)=2 x^5+x^3-3 x^2+1 ; \quad x_0=1$ \\
\textbf{5.} Kompleks sanlar maydanında ápiwayı bólsheklerge jayıń: $\frac{1}{x^2-1}$; \\

\end{tabular}
\vspace{1cm}


\begin{tabular}{m{17cm}}
\textbf{28-variant}
\newline

\textbf{1.} Kompleks sannan koren shıǵarıw. \\
\textbf{2.} Qaldıqlı bóliw.  \\
\textbf{3.} Trigonometriyalıq kórinisin tabıń: $-1-i \sqrt{3}$; \\
\textbf{4.} Gorner sxemasınan paydalanıp, $f(x)$ kópaǵzalısın $x-x_0$ dárejeleri boyınsha jayıń hám $f\left(x_0\right)$ di esaplań: $f(x)=3 x^5+x^4-19 x^2-13 x-10 ; \quad x_0=2$ \\
\textbf{5.} Kompleks sanlar maydanında ápiwayı bólsheklerge jayıń: $\frac{x}{\left(x^2-1\right)^2}$; \\

\end{tabular}
\vspace{1cm}


\begin{tabular}{m{17cm}}
\textbf{29-variant}
\newline

\textbf{1.} Ratsional bólshekler hám olardı eń ápiwayı bólsheklerge ajıratıw. \\
\textbf{2.} Bir belgisizli kópaǵzalılar. Gorner sxeması. Bezu teoreması.  \\
\textbf{3.}  $x, y \in R$, tabıń, egerde:  $(1+i) x+(1-i) y=3-i$; \\
\textbf{4.} Qanday shártler orınlanǵanda $x^3+p x+q$ kópaǵzalısı $x^2+m x-1$ kópaǵzalısına qaldıqsız bólinedi?. \\
\textbf{5.} Kompleks sanlar maydanında ápiwayı bólsheklerge jayıń: $\frac{1}{\left(x^2-1\right)^2}$; \\

\end{tabular}
\vspace{1cm}


\begin{tabular}{m{17cm}}
\textbf{30-variant}
\newline

\textbf{1.} Kompleks sanlar hám olar ústinde ámeller. Muavr formulası.  \\
\textbf{2.} Kópaǵzalılardıń EÚUB si. Keltirilmeytuǵın   kópaǵzalılar. \\
\textbf{3.} Trigonometriyalıq kórinisin tabıń: $-1-2 i$. \\
\textbf{4.} Gorner sxemasınan paydalanıp, $f(x)$ kópaǵzalısın $x-x_0$ dárejeleri boyınsha jayıń hám $f\left(x_0\right)$ di esaplań: $f(x)=5 x^5-19 x^3-7 x^2+9 x+3 ; \quad x_0=2$ \\
\textbf{5.} Kópaǵzalılardıń eń úlken ulıwma bóliwshisin tabıń:  $f(x)=x^5+3 x^4-12 x^3-52 x^2-52 x-12$ hám $g(x)=x^4+3 x^3-6 x^2-22 x-12$ \\

\end{tabular}
\vspace{1cm}


\begin{tabular}{m{17cm}}
\textbf{31-variant}
\newline

\textbf{1.} Bir belgisizli kópaǵzalılar. Gorner sxeması. Bezu teoreması.  \\
\textbf{2.} Qaldıqlı bóliw.  \\
\textbf{3.} Esaplań:  $\left(\frac{1+i \sqrt{3}}{1-i}\right)^{20}$; \\
\textbf{4.} Gorner sxemasınan paydalanıp, $f(x)$ kópaǵzalısın $x-x_0$ dárejeleri boyınsha jayıń hám $f\left(x_0\right)$ di esaplań: $f(x)=x^4-3 x^3-10 x^2+2 x+5 ; \quad x_0=-2$. \\
\textbf{5.} Haqıyqıy sanlar maydanında ápiwayı bólsheklerge jayıń:  $\frac{x^2}{x^4-16}$; \\

\end{tabular}
\vspace{1cm}


\begin{tabular}{m{17cm}}
\textbf{32-variant}
\newline

\textbf{1.} Kompleks sanlar hám olar ústinde ámeller. Muavr formulası.  \\
\textbf{2.} Ratsional bólshekler hám olardı eń ápiwayı bólsheklerge ajıratıw. \\
\textbf{3.} Algebralıq formada esaplań:  $\frac{(1+2 i)^2-(1-i)^2}{(3+2 i)^3-(2+i)^2}$; \\
\textbf{4.} Qaldıqlı bóliwdi orınlań: $x^4-2 x^3+4 x^2-6 x+8$ di $x-1$ ge; \\
\textbf{5.} Kompleks sanlar maydanında ápiwayı bólsheklerge jayıń: $\frac{1}{(x-1)(x-2)(x-3)(x-4)}$; \\

\end{tabular}
\vspace{1cm}


\begin{tabular}{m{17cm}}
\textbf{33-variant}
\newline

\textbf{1.} Kópaǵzalılardıń EÚUB si. Keltirilmeytuǵın   kópaǵzalılar. \\
\textbf{2.} Kompleks sannan koren shıǵarıw. \\
\textbf{3.} Kompleks sannan koren shıǵarıń: $\sqrt[6]{\frac{1-i}{\sqrt{3}+i}}$; \\
\textbf{4.} Gorner sxemasınan paydalanıp, $f(x)$ kópaǵzalısın $x-x_0$ dárejeleri boyınsha jayıń hám $f\left(x_0\right)$ di esaplań: $f(x)=x^5-3 x^4+3 x^2-2 x+1 ; \quad x_0=-1$ \\
\textbf{5.} Kompleks sanlar maydanında ápiwayı bólsheklerge jayıń: $\frac{1}{x^4+4}$; \\

\end{tabular}
\vspace{1cm}


\begin{tabular}{m{17cm}}
\textbf{34-variant}
\newline

\textbf{1.} Bir belgisizli kópaǵzalılar. Gorner sxeması. Bezu teoreması.  \\
\textbf{2.} Kompleks sannan koren shıǵarıw. \\
\textbf{3.} Kompleks sannan koren shıǵarıń: $\sqrt[6]{-2 i}$; \\
\textbf{4.} Qaldıqlı bóliwdi orınlań:  $x^3-3 x^2+4 x-1$ di $x^2-x+1$ ge; \\
\textbf{5.} Haqıyqıy sanlar maydanında ápiwayı bólsheklerge jayıń:  $\frac{1}{(x+1)(x-3)(x+5)}$. \\

\end{tabular}
\vspace{1cm}


\begin{tabular}{m{17cm}}
\textbf{35-variant}
\newline

\textbf{1.} Kópaǵzalılardıń EÚUB si. Keltirilmeytuǵın   kópaǵzalılar. \\
\textbf{2.} Ratsional bólshekler hám olardı eń ápiwayı bólsheklerge ajıratıw. \\
\textbf{3.} Kompleks sannan koren shıǵarıń: $\sqrt[4]{-4}$; \\
\textbf{4.} Qaldıqlı bóliwdi orınlań: $x^3-3 x^2-x-1$ di $3 x^2-2 x+1$ ge; \\
\textbf{5.} Haqıyqıy sanlar maydanında ápiwayı bólsheklerge jayıń:  $\frac{2 x-1}{x(x+1)^2\left(x^2+x+1\right)^2}$; \\

\end{tabular}
\vspace{1cm}


\begin{tabular}{m{17cm}}
\textbf{36-variant}
\newline

\textbf{1.} Qaldıqlı bóliw.  \\
\textbf{2.} Kompleks sanlar hám olar ústinde ámeller. Muavr formulası.  \\
\textbf{3.} Kompleks sannan koren shıǵarıń: $\sqrt[6]{1}$; \\
\textbf{4.} Qaldıqlı bóliwdi orınlań:  $x^3-x^2-x$ t1 $x-1+2 i$ ge; \\
\textbf{5.} Evklid algoritminen paydalanıp, $f(x)$ hám $g(x)$ kópaǵzalıları ushın $f(x) \cdot u(x)+g(x) \cdot v(x)=d(x)$ teńligin qanaatlandırıwshı $u(x)$ hám $v(x)$ kópaǵzalıların tabıń, bunda $d(x)=(f(x), g(x))$:  $f(x)=3 x^4-5 x^3+4 x^2-2 x+1$ hám $g(x)=3 x^3-2 x^2+x-1$ \\

\end{tabular}
\vspace{1cm}


\begin{tabular}{m{17cm}}
\textbf{37-variant}
\newline

\textbf{1.} Kompleks sanlar hám olar ústinde ámeller. Muavr formulası.  \\
\textbf{2.} Bir belgisizli kópaǵzalılar. Gorner sxeması. Bezu teoreması.  \\
\textbf{3.} Algebralıq formada esaplań:  $\left(-\frac{1}{2}+i \frac{\sqrt{3}}{2}\right)^3$; \\
\textbf{4.} Qaldıqlı bóliwdi orınlań: $2 x^5-5 x^3-8 x$ ti $x+3$ ge; \\
\textbf{5.} Kópaǵzalılardıń eń úlken ulıwma bóliwshisin tabıń:  $f(x)=x^5+2 x^4-4 x^3-3 x^2+8 x-5$ hám $g(x)=x^5+x^2-x-1$ \\

\end{tabular}
\vspace{1cm}


\begin{tabular}{m{17cm}}
\textbf{38-variant}
\newline

\textbf{1.} Qaldıqlı bóliw.  \\
\textbf{2.} Kópaǵzalılardıń EÚUB si. Keltirilmeytuǵın   kópaǵzalılar. \\
\textbf{3.} Algebralıq formada esaplań:  $\sqrt{3+4 i}$; \\
\textbf{4.} Qaldıqlı bóliwdi orınlań: $x^4-3 x^2+2 x-5$ ti $x^2+2 x-1$ ge; \\
\textbf{5.} Evklid algoritminen paydalanıp, $f(x)$ hám $g(x)$ kópaǵzalıları ushın $f(x) \cdot u(x)+g(x) \cdot v(x)=d(x)$ teńligin qanaatlandırıwshı $u(x)$ hám $v(x)$ kópaǵzalıların tabıń, bunda $d(x)=(f(x), g(x))$:  $f(x)=2 x^4+3 x^3-3 x^2-5 x+2$ hám $g(x)=2 x^3+x^2-x-1$ \\

\end{tabular}
\vspace{1cm}


\begin{tabular}{m{17cm}}
\textbf{39-variant}
\newline

\textbf{1.} Ratsional bólshekler hám olardı eń ápiwayı bólsheklerge ajıratıw. \\
\textbf{2.} Kompleks sannan koren shıǵarıw. \\
\textbf{3.} Trigonometriyalıq kórinisin tabıń: $4-i$; \\
\textbf{4.} Qanday shártler orınlanǵanda $x^4+p x^2-q$ kópaǵzalısı $x^2+m q+1$ kópaǵzalısına qaldıqsız bólinedi?. \\
\textbf{5.} Kópaǵzalılardıń eń úlken ulıwma bóliwshisin tabıń:  $f(x)=x^5+3 x^2-2 x+2$ hám $g(x)=x^6+x^5+x^4-3 x^2+2 x-6$ \\

\end{tabular}
\vspace{1cm}


\begin{tabular}{m{17cm}}
\textbf{40-variant}
\newline

\textbf{1.} Kópaǵzalılardıń EÚUB si. Keltirilmeytuǵın   kópaǵzalılar. \\
\textbf{2.} Bir belgisizli kópaǵzalılar. Gorner sxeması. Bezu teoreması.  \\
\textbf{3.} Kompleks sannan koren shıǵarıń: $\sqrt[3]{2-2 i}$; \\
\textbf{4.} Qaldıqlı bóliwdi orınlań: $4 x^3+x^2$ tı $x+1+i$ ge; \\
\textbf{5.} Kompleks sanlar maydanında ápiwayı bólsheklerge jayıń: $\frac{1}{(x-1)(x-2)(x-3)(x-4)}$; \\

\end{tabular}
\vspace{1cm}


\begin{tabular}{m{17cm}}
\textbf{41-variant}
\newline

\textbf{1.} Kompleks sannan koren shıǵarıw. \\
\textbf{2.} Kompleks sanlar hám olar ústinde ámeller. Muavr formulası.  \\
\textbf{3.} Teńlemeni sheshiń:  $(2+i) x^2-(5-i) x+(2-2 i)=0$; \\
\textbf{4.} Gorner sxemasınan paydalanıp, $f(x)$ kópaǵzalısın $x-x_0$ dárejeleri boyınsha jayıń hám $f\left(x_0\right)$ di esaplań: $f(x)=2 x^5-5 x^3-8 x ; \quad x_0=-3$ \\
\textbf{5.} Evklid algoritminen paydalanıp, $f(x)$ hám $g(x)$ kópaǵzalıları ushın $f(x) \cdot u(x)+g(x) \cdot v(x)=d(x)$ teńligin qanaatlandırıwshı $u(x)$ hám $v(x)$ kópaǵzalıların tabıń, bunda $d(x)=(f(x), g(x))$:  $f(x)=x^5+5 x^4+9 x^3+7 x^2+5 x+3$ hám $g(x)=x^4+2 x^3+2 x^2+x+1$ \\

\end{tabular}
\vspace{1cm}


\begin{tabular}{m{17cm}}
\textbf{42-variant}
\newline

\textbf{1.} Qaldıqlı bóliw.  \\
\textbf{2.} Ratsional bólshekler hám olardı eń ápiwayı bólsheklerge ajıratıw. \\
\textbf{3.} Algebralıq formada esaplań: $(1+2 i)^6$; \\
\textbf{4.} Gorner sxemasınan paydalanıp, $f(x)$ kópaǵzalısın $x-x_0$ dárejeleri boyınsha jayıń hám $f\left(x_0\right)$ di esaplań: $f(x)=x^5-3 x^4+3 x^2-2 x+1 ; \quad x_0=-1$ \\
\textbf{5.} Kompleks sanlar maydanında ápiwayı bólsheklerge jayıń: $\frac{x}{\left(x^2-1\right)^2}$; \\

\end{tabular}
\vspace{1cm}


\begin{tabular}{m{17cm}}
\textbf{43-variant}
\newline

\textbf{1.} Ratsional bólshekler hám olardı eń ápiwayı bólsheklerge ajıratıw. \\
\textbf{2.} Bir belgisizli kópaǵzalılar. Gorner sxeması. Bezu teoreması.  \\
\textbf{3.} Algebralıq formada esaplań: $\sqrt{8+6 i}$; г) $\sqrt{2-3 i}$. \\
\textbf{4.} Qaldıqlı bóliwdi orınlań:  $x^3-x^2-x$ t1 $x-1+2 i$ ge; \\
\textbf{5.} Evklid algoritminen paydalanıp, $f(x)$ hám $g(x)$ kópaǵzalıları ushın $f(x) \cdot u(x)+g(x) \cdot v(x)=d(x)$ teńligin qanaatlandırıwshı $u(x)$ hám $v(x)$ kópaǵzalıların tabıń, bunda $d(x)=(f(x), g(x))$:  $f(x)=x^4-x^3-4 x^2+4 x+1$ hám $g(x)=x^2-x-1$ \\

\end{tabular}
\vspace{1cm}


\begin{tabular}{m{17cm}}
\textbf{44-variant}
\newline

\textbf{1.} Kópaǵzalılardıń EÚUB si. Keltirilmeytuǵın   kópaǵzalılar. \\
\textbf{2.} Kompleks sannan koren shıǵarıw. \\
\textbf{3.}  $x, y \in R$, tabıń, egerde: $2+5 i x-3 i y=14 i+3 x-5 y$; \\
\textbf{4.} Qaldıqlı bóliwdi orınlań: $2 x^5-5 x^3-8 x$ ti $x+3$ ge; \\
\textbf{5.} Haqıyqıy sanlar maydanında ápiwayı bólsheklerge jayıń:  $\frac{1}{x^4+4}$; \\

\end{tabular}
\vspace{1cm}


\begin{tabular}{m{17cm}}
\textbf{45-variant}
\newline

\textbf{1.} Qaldıqlı bóliw.  \\
\textbf{2.} Kompleks sanlar hám olar ústinde ámeller. Muavr formulası.  \\
\textbf{3.} Trigonometriyalıq kórinisin tabıń: $-1-2 i$. \\
\textbf{4.} Gorner sxemasınan paydalanıp, $f(x)$ kópaǵzalısın $x-x_0$ dárejeleri boyınsha jayıń hám $f\left(x_0\right)$ di esaplań: $f(x)=x^4-3 x^3-10 x^2+2 x+5 ; \quad x_0=-2$. \\
\textbf{5.} Kópaǵzalılardıń eń úlken ulıwma bóliwshisin tabıń:  $f(x)=x^5+2 x^4-4 x^3-3 x^2+8 x-5$ hám $g(x)=x^5+x^2-x-1$ \\

\end{tabular}
\vspace{1cm}


\begin{tabular}{m{17cm}}
\textbf{46-variant}
\newline

\textbf{1.} Bir belgisizli kópaǵzalılar. Gorner sxeması. Bezu teoreması.  \\
\textbf{2.} Kompleks sanlar hám olar ústinde ámeller. Muavr formulası.  \\
\textbf{3.} Esaplań:  $\left(\frac{1+i \sqrt{3}}{1-i}\right)^{20}$; \\
\textbf{4.} Gorner sxemasınan paydalanıp, $f(x)$ kópaǵzalısın $x-x_0$ dárejeleri boyınsha jayıń hám $f\left(x_0\right)$ di esaplań: $f(x)=x^4-3 x^3+6 x^2-10 x+16 ; x_0=4$ \\
\textbf{5.} Kópaǵzalılardıń eń úlken ulıwma bóliwshisin tabıń:  $f(x)=x^4+x^3-4 x+5$ hám $g(x)=2 x^3-x^2-2 x+2$ \\

\end{tabular}
\vspace{1cm}


\begin{tabular}{m{17cm}}
\textbf{47-variant}
\newline

\textbf{1.} Qaldıqlı bóliw.  \\
\textbf{2.} Kompleks sannan koren shıǵarıw. \\
\textbf{3.} Algebralıq formada esaplań: $(2+i)^7+(2-i)^7$. \\
\textbf{4.} Qaldıqlı bóliwdi orınlań: $2 x^4-3 x^3+4 x^2-5 x+6$ nı $x^2-3 x+1$ ge; \\
\textbf{5.} Haqıyqıy sanlar maydanında ápiwayı bólsheklerge jayıń:  $\frac{x^2}{(x-1)(x+2)(x+3)}$; \\

\end{tabular}
\vspace{1cm}


\begin{tabular}{m{17cm}}
\textbf{48-variant}
\newline

\textbf{1.} Ratsional bólshekler hám olardı eń ápiwayı bólsheklerge ajıratıw. \\
\textbf{2.} Kópaǵzalılardıń EÚUB si. Keltirilmeytuǵın   kópaǵzalılar. \\
\textbf{3.} Algebralıq formada esaplań: $(1+2 i)^6$; \\
\textbf{4.} Qaldıqlı bóliwdi orınlań: $x^4-3 x^2+2 x-5$ ti $x^2+2 x-1$ ge; \\
\textbf{5.} Evklid algoritminen paydalanıp, $f(x)$ hám $g(x)$ kópaǵzalıları ushın $f(x) \cdot u(x)+g(x) \cdot v(x)=d(x)$ teńligin qanaatlandırıwshı $u(x)$ hám $v(x)$ kópaǵzalıların tabıń, bunda $d(x)=(f(x), g(x))$:  $f(x)=x^5-5 x^4-2 x^3+12 x^2-2 x+12$ hám $g(x)=x^3-5 x^2-3 x+17$ \\

\end{tabular}
\vspace{1cm}


\begin{tabular}{m{17cm}}
\textbf{49-variant}
\newline

\textbf{1.} Bir belgisizli kópaǵzalılar. Gorner sxeması. Bezu teoreması.  \\
\textbf{2.} Kompleks sannan koren shıǵarıw. \\
\textbf{3.}  $x, y \in R$, tabıń, egerde:  $(1+i) x+(1-i) y=3-i$; \\
\textbf{4.} Gorner sxemasınan paydalanıp, $f(x)$ kópaǵzalısın $x-x_0$ dárejeleri boyınsha jayıń hám $f\left(x_0\right)$ di esaplań: $f(x)=5 x^5-19 x^3-7 x^2+9 x+3 ; \quad x_0=2$ \\
\textbf{5.} Kópaǵzalılardıń eń úlken ulıwma bóliwshisin tabıń:  $f(x)=x^4+7 x^3+19 x^2+23 x+10$ hám $g(x)=x^4+7 x^3+18 x^2+22 x+12$ \\

\end{tabular}
\vspace{1cm}


\begin{tabular}{m{17cm}}
\textbf{50-variant}
\newline

\textbf{1.} Kompleks sanlar hám olar ústinde ámeller. Muavr formulası.  \\
\textbf{2.} Ratsional bólshekler hám olardı eń ápiwayı bólsheklerge ajıratıw. \\
\textbf{3.} Algebralıq formada esaplań:  $\sqrt{3+4 i}$; \\
\textbf{4.} Gorner sxemasınan paydalanıp, $f(x)$ kópaǵzalısın $x-x_0$ dárejeleri boyınsha jayıń hám $f\left(x_0\right)$ di esaplań: $f(x)=3 x^4+4 x^3+5 x^2+x+33 ; \quad x_0=-2$ \\
\textbf{5.} Kompleks sanlar maydanında ápiwayı bólsheklerge jayıń: $\frac{x^2}{x^4-1}$; \\

\end{tabular}
\vspace{1cm}


\begin{tabular}{m{17cm}}
\textbf{51-variant}
\newline

\textbf{1.} Qaldıqlı bóliw.  \\
\textbf{2.} Kópaǵzalılardıń EÚUB si. Keltirilmeytuǵın   kópaǵzalılar. \\
\textbf{3.} Trigonometriyalıq kórinisin tabıń: $-2+i$; \\
\textbf{4.} Qanday shártler orınlanǵanda $x^4+p x^2-q$ kópaǵzalısı $x^2+m q+1$ kópaǵzalısına qaldıqsız bólinedi?. \\
\textbf{5.} Evklid algoritminen paydalanıp, $f(x)$ hám $g(x)$ kópaǵzalıları ushın $f(x) \cdot u(x)+g(x) \cdot v(x)=d(x)$ teńligin qanaatlandırıwshı $u(x)$ hám $v(x)$ kópaǵzalıların tabıń, bunda $d(x)=(f(x), g(x))$:  $f(x)=2 x^4+3 x^3-3 x^2-5 x+2$ hám $g(x)=2 x^3+x^2-x-1$ \\

\end{tabular}
\vspace{1cm}


\begin{tabular}{m{17cm}}
\textbf{52-variant}
\newline

\textbf{1.} Kompleks sannan koren shıǵarıw. \\
\textbf{2.} Kópaǵzalılardıń EÚUB si. Keltirilmeytuǵın   kópaǵzalılar. \\
\textbf{3.} $x, y \in R$, tabıń, egerde:  $\frac{8 i}{x}+i y-2=7 i-\frac{10}{x}+y$. \\
\textbf{4.} Qaldıqlı bóliwdi orınlań: $x^3-3 x^2-x-1$ di $3 x^2-2 x+1$ ge; \\
\textbf{5.} Kópaǵzalılardıń eń úlken ulıwma bóliwshisin tabıń: $f(x)=x^5+x^4-x^3-3 x^2-3 x-1$ hám $g(x)=x^4-2 x^3-x^2-2 x+1$ \\

\end{tabular}
\vspace{1cm}


\begin{tabular}{m{17cm}}
\textbf{53-variant}
\newline

\textbf{1.} Bir belgisizli kópaǵzalılar. Gorner sxeması. Bezu teoreması.  \\
\textbf{2.} Ratsional bólshekler hám olardı eń ápiwayı bólsheklerge ajıratıw. \\
\textbf{3.} Trigonometriyalıq kórinisin tabıń: -5 ;  \\
\textbf{4.} Qaldıqlı bóliwdi orınlań:  $x^3-3 x^2+4 x-1$ di $x^2-x+1$ ge; \\
\textbf{5.} Kompleks sanlar maydanında ápiwayı bólsheklerge jayıń: $\frac{1}{x^4+4}$; \\

\end{tabular}
\vspace{1cm}


\begin{tabular}{m{17cm}}
\textbf{54-variant}
\newline

\textbf{1.} Qaldıqlı bóliw.  \\
\textbf{2.} Kompleks sanlar hám olar ústinde ámeller. Muavr formulası.  \\
\textbf{3.} Teńlemeni sheshiń:  $(2+i) x^2-(5-i) x+(2-2 i)=0$; \\
\textbf{4.} Qaldıqlı bóliwdi orınlań: $4 x^3+x^2$ tı $x+1+i$ ge; \\
\textbf{5.} Kompleks sanlar maydanında ápiwayı bólsheklerge jayıń: $\frac{1}{x^2-1}$; \\

\end{tabular}
\vspace{1cm}


\begin{tabular}{m{17cm}}
\textbf{55-variant}
\newline

\textbf{1.} Qaldıqlı bóliw.  \\
\textbf{2.} Kompleks sanlar hám olar ústinde ámeller. Muavr formulası.  \\
\textbf{3.} Teńlemeni sheshiń:  $x^2-(3-2 i) x+(5-5 i)=0$; \\
\textbf{4.} Qanday shártler orınlanǵanda $x^3+p x+q$ kópaǵzalısı $x^2+m x-1$ kópaǵzalısına qaldıqsız bólinedi?. \\
\textbf{5.} Haqıyqıy sanlar maydanında ápiwayı bólsheklerge jayıń:  $\frac{x^2}{x^6+27}$; \\

\end{tabular}
\vspace{1cm}


\begin{tabular}{m{17cm}}
\textbf{56-variant}
\newline

\textbf{1.} Ratsional bólshekler hám olardı eń ápiwayı bólsheklerge ajıratıw. \\
\textbf{2.} Bir belgisizli kópaǵzalılar. Gorner sxeması. Bezu teoreması.  \\
\textbf{3.} Esaplań:  $\frac{(-1-i \sqrt{3})^{10}}{(-1+i)^{16}}$. \\
\textbf{4.} Qaldıqlı bóliwdi orınlań: $x^4-2 x^3+4 x^2-6 x+8$ di $x-1$ ge; \\
\textbf{5.} Kompleks sanlar maydanında ápiwayı bólsheklerge jayıń: $\frac{3+x}{(x-1)\left(x^2+1\right)}$; \\

\end{tabular}
\vspace{1cm}


\begin{tabular}{m{17cm}}
\textbf{57-variant}
\newline

\textbf{1.} Kompleks sannan koren shıǵarıw. \\
\textbf{2.} Kópaǵzalılardıń EÚUB si. Keltirilmeytuǵın   kópaǵzalılar. \\
\textbf{3.} Esaplań:  $\frac{(-1+i \sqrt{3})^{15}}{(1-i)^{20}}$; \\
\textbf{4.} Gorner sxemasınan paydalanıp, $f(x)$ kópaǵzalısın $x-x_0$ dárejeleri boyınsha jayıń hám $f\left(x_0\right)$ di esaplań: $f(x)=3 x^5+x^4-19 x^2-13 x-10 ; \quad x_0=2$ \\
\textbf{5.} Haqıyqıy sanlar maydanında ápiwayı bólsheklerge jayıń:  $\frac{1}{x^3-1}$; \\

\end{tabular}
\vspace{1cm}


\begin{tabular}{m{17cm}}
\textbf{58-variant}
\newline

\textbf{1.} Bir belgisizli kópaǵzalılar. Gorner sxeması. Bezu teoreması.  \\
\textbf{2.} Kópaǵzalılardıń EÚUB si. Keltirilmeytuǵın   kópaǵzalılar. \\
\textbf{3.} Kompleks sannan koren shıǵarıń: $\sqrt[6]{\frac{1-i}{\sqrt{3}+i}}$; \\
\textbf{4.} Gorner sxemasınan paydalanıp, $f(x)$ kópaǵzalısın $x-x_0$ dárejeleri boyınsha jayıń hám $f\left(x_0\right)$ di esaplań: $f(x)=x^4+2 x^3-3 x^2-4 x+1 ; \quad x_0=-1$ \\
\textbf{5.} Evklid algoritminen paydalanıp, $f(x)$ hám $g(x)$ kópaǵzalıları ushın $f(x) \cdot u(x)+g(x) \cdot v(x)=d(x)$ teńligin qanaatlandırıwshı $u(x)$ hám $v(x)$ kópaǵzalıların tabıń, bunda $d(x)=(f(x), g(x))$:  $f(x)=3 x^3-2 x^2+x+2$ hám $g(x)=x^2-x+1$ \\

\end{tabular}
\vspace{1cm}


\begin{tabular}{m{17cm}}
\textbf{59-variant}
\newline

\textbf{1.} Ratsional bólshekler hám olardı eń ápiwayı bólsheklerge ajıratıw. \\
\textbf{2.} Kompleks sannan koren shıǵarıw. \\
\textbf{3.} Trigonometriyalıq kórinisin tabıń: $-1-i \sqrt{3}$; \\
\textbf{4.} Gorner sxemasınan paydalanıp, $f(x)$ kópaǵzalısın $x-x_0$ dárejeleri boyınsha jayıń hám $f\left(x_0\right)$ di esaplań: $f(x)=x^4-2 x^3+4 x^2-6 x+8 ; \quad x_0=1$ \\
\textbf{5.} Kópaǵzalılardıń eń úlken ulıwma bóliwshisin tabıń:  $f(x)=x^4+x^3-3 x^2-4 x-1$ hám $g(x)=x^3+x^2-x-1$ \\

\end{tabular}
\vspace{1cm}


\begin{tabular}{m{17cm}}
\textbf{60-variant}
\newline

\textbf{1.} Qaldıqlı bóliw.  \\
\textbf{2.} Kompleks sanlar hám olar ústinde ámeller. Muavr formulası.  \\
\textbf{3.} Algebralıq formada esaplań: $\frac{(1+i)^n}{(1-i)^{n-2}}$; \\
\textbf{4.} Gorner sxemasınan paydalanıp, $f(x)$ kópaǵzalısın $x-x_0$ dárejeleri boyınsha jayıń hám $f\left(x_0\right)$ di esaplań:  $f(x)=2 x^5+x^3-3 x^2+1 ; \quad x_0=1$ \\
\textbf{5.} Haqıyqıy sanlar maydanında ápiwayı bólsheklerge jayıń:  $\frac{x}{\left(x^2-1\right)^2}$; \\

\end{tabular}
\vspace{1cm}


\begin{tabular}{m{17cm}}
\textbf{61-variant}
\newline

\textbf{1.} Kompleks sannan koren shıǵarıw. \\
\textbf{2.} Qaldıqlı bóliw.  \\
\textbf{3.} Kompleks sannan koren shıǵarıń: $\sqrt[6]{1}$; \\
\textbf{4.} Gorner sxemasınan paydalanıp, $f(x)$ kópaǵzalısın $x-x_0$ dárejeleri boyınsha jayıń hám $f\left(x_0\right)$ di esaplań: $f(x)=3 x^5+x^4-19 x^2-13 x-10 ; \quad x_0=2$ \\
\textbf{5.} Kompleks sanlar maydanında ápiwayı bólsheklerge jayıń: $\frac{1}{x\left(x^2+1\right)}$; \\

\end{tabular}
\vspace{1cm}


\begin{tabular}{m{17cm}}
\textbf{62-variant}
\newline

\textbf{1.} Kópaǵzalılardıń EÚUB si. Keltirilmeytuǵın   kópaǵzalılar. \\
\textbf{2.} Bir belgisizli kópaǵzalılar. Gorner sxeması. Bezu teoreması.  \\
\textbf{3.} Teńlemeni sheshiń:  $x^2+3 x-10 i=0$; \\
\textbf{4.} Qanday shártler orınlanǵanda $x^3+p x+q$ kópaǵzalısı $x^2+m x-1$ kópaǵzalısına qaldıqsız bólinedi?. \\
\textbf{5.} Evklid algoritminen paydalanıp, $f(x)$ hám $g(x)$ kópaǵzalıları ushın $f(x) \cdot u(x)+g(x) \cdot v(x)=d(x)$ teńligin qanaatlandırıwshı $u(x)$ hám $v(x)$ kópaǵzalıların tabıń, bunda $d(x)=(f(x), g(x))$:  $f(x)=x^4-2 x^3-16 x^2+5 x+9$ hám $g(x)=2 x^3-x^2-5 x+4$ \\

\end{tabular}
\vspace{1cm}


\begin{tabular}{m{17cm}}
\textbf{63-variant}
\newline

\textbf{1.} Ratsional bólshekler hám olardı eń ápiwayı bólsheklerge ajıratıw. \\
\textbf{2.} Kompleks sanlar hám olar ústinde ámeller. Muavr formulası.  \\
\textbf{3.} Algebralıq formada esaplań: $\frac{5+i}{(1-2 i)(5-i)}$; \\
\textbf{4.} Gorner sxemasınan paydalanıp, $f(x)$ kópaǵzalısın $x-x_0$ dárejeleri boyınsha jayıń hám $f\left(x_0\right)$ di esaplań: $f(x)=5 x^5-19 x^3-7 x^2+9 x+3 ; \quad x_0=2$ \\
\textbf{5.} Kópaǵzalılardıń eń úlken ulıwma bóliwshisin tabıń:  $f(x)=x^4-4 x^3+1$ hám $g(x)=x^3-3 x^2+1$ \\

\end{tabular}
\vspace{1cm}


\begin{tabular}{m{17cm}}
\textbf{64-variant}
\newline

\textbf{1.} Kompleks sannan koren shıǵarıw. \\
\textbf{2.} Qaldıqlı bóliw.  \\
\textbf{3.} Teńlemeni sheshiń:  $x^2-(5-3 i) x+(4-7 i)=0$. \\
\textbf{4.} Gorner sxemasınan paydalanıp, $f(x)$ kópaǵzalısın $x-x_0$ dárejeleri boyınsha jayıń hám $f\left(x_0\right)$ di esaplań: $f(x)=x^4-3 x^3+6 x^2-10 x+16 ; x_0=4$ \\
\textbf{5.} Kópaǵzalılardıń eń úlken ulıwma bóliwshisin tabıń:  $f(x)=x^4-10 x^2+1$ hám $g(x)=x^4-4 \sqrt{2} x^3+6 x^2+4 \sqrt{2} x+1$ \\

\end{tabular}
\vspace{1cm}


\begin{tabular}{m{17cm}}
\textbf{65-variant}
\newline

\textbf{1.} Kompleks sanlar hám olar ústinde ámeller. Muavr formulası.  \\
\textbf{2.} Kópaǵzalılardıń EÚUB si. Keltirilmeytuǵın   kópaǵzalılar. \\
\textbf{3.} Algebralıq formada esaplań: $\frac{1+i \sqrt{3}}{1-i \sqrt{3}}-(1-i)^2$; \\
\textbf{4.} Qaldıqlı bóliwdi orınlań: $x^4-3 x^2+2 x-5$ ti $x^2+2 x-1$ ge; \\
\textbf{5.} Haqıyqıy sanlar maydanında ápiwayı bólsheklerge jayıń:  $\frac{x^2}{x^4-16}$; \\

\end{tabular}
\vspace{1cm}


\begin{tabular}{m{17cm}}
\textbf{66-variant}
\newline

\textbf{1.} Bir belgisizli kópaǵzalılar. Gorner sxeması. Bezu teoreması.  \\
\textbf{2.} Ratsional bólshekler hám olardı eń ápiwayı bólsheklerge ajıratıw. \\
\textbf{3.}  $x, y \in R$, tabıń, egerde: $(2-3 i) x+(3+2 i) y=2-5 i$; \\
\textbf{4.} Qaldıqlı bóliwdi orınlań: $2 x^4-3 x^3+4 x^2-5 x+6$ nı $x^2-3 x+1$ ge; \\
\textbf{5.} Evklid algoritminen paydalanıp, $f(x)$ hám $g(x)$ kópaǵzalıları ushın $f(x) \cdot u(x)+g(x) \cdot v(x)=d(x)$ teńligin qanaatlandırıwshı $u(x)$ hám $v(x)$ kópaǵzalıların tabıń, bunda $d(x)=(f(x), g(x))$:  $f(x)=3 x^5+5 x^4-16 x^3-6 x^2-5 x-6$ hám $g(x)=3 x^4-4 x^3-x^2-x-2$ \\

\end{tabular}
\vspace{1cm}


\begin{tabular}{m{17cm}}
\textbf{67-variant}
\newline

\textbf{1.} Bir belgisizli kópaǵzalılar. Gorner sxeması. Bezu teoreması.  \\
\textbf{2.} Kompleks sannan koren shıǵarıw. \\
\textbf{3.} Trigonometriyalıq kórinisin tabıń: $1-i$; \\
\textbf{4.} Gorner sxemasınan paydalanıp, $f(x)$ kópaǵzalısın $x-x_0$ dárejeleri boyınsha jayıń hám $f\left(x_0\right)$ di esaplań: $f(x)=x^4-2 x^3+4 x^2-6 x+8 ; \quad x_0=1$ \\
\textbf{5.} Evklid algoritminen paydalanıp, $f(x)$ hám $g(x)$ kópaǵzalıları ushın $f(x) \cdot u(x)+g(x) \cdot v(x)=d(x)$ teńligin qanaatlandırıwshı $u(x)$ hám $v(x)$ kópaǵzalıların tabıń, bunda $d(x)=(f(x), g(x))$:  $f(x)=x^5+3 x^4+x^3+x^2+3 x+1$ hám $g(x)=x^4+2 x^3+x+2$ \\

\end{tabular}
\vspace{1cm}


\begin{tabular}{m{17cm}}
\textbf{68-variant}
\newline

\textbf{1.} Qaldıqlı bóliw.  \\
\textbf{2.} Kompleks sanlar hám olar ústinde ámeller. Muavr formulası.  \\
\textbf{3.} Esaplań:  $(1+i)^{25}$; \\
\textbf{4.} Qaldıqlı bóliwdi orınlań: $x^3-3 x^2-x-1$ di $3 x^2-2 x+1$ ge; \\
\textbf{5.} Kópaǵzalılardıń eń úlken ulıwma bóliwshisin tabıń:  $f(x)=2 x^6-5 x^5+14 x^4+36 x^3+86 x^2+12 x-31$ hám $g(x)=2 x^5-9 x^4+2 x^3+37 x^2+10 x-14$ \\

\end{tabular}
\vspace{1cm}


\begin{tabular}{m{17cm}}
\textbf{69-variant}
\newline

\textbf{1.} Kópaǵzalılardıń EÚUB si. Keltirilmeytuǵın   kópaǵzalılar. \\
\textbf{2.} Ratsional bólshekler hám olardı eń ápiwayı bólsheklerge ajıratıw. \\
\textbf{3.} Algebralıq formada esaplań: $\frac{(2-3 i)(4-i)}{5-i}$; \\
\textbf{4.} Gorner sxemasınan paydalanıp, $f(x)$ kópaǵzalısın $x-x_0$ dárejeleri boyınsha jayıń hám $f\left(x_0\right)$ di esaplań: $f(x)=2 x^5-5 x^3-8 x ; \quad x_0=-3$ \\
\textbf{5.} Evklid algoritminen paydalanıp, $f(x)$ hám $g(x)$ kópaǵzalıları ushın $f(x) \cdot u(x)+g(x) \cdot v(x)=d(x)$ teńligin qanaatlandırıwshı $u(x)$ hám $v(x)$ kópaǵzalıların tabıń, bunda $d(x)=(f(x), g(x))$:  $f(x)=3 x^4-5 x^3+4 x^2-2 x+1$ hám $g(x)=3 x^3-2 x^2+x-1$ \\

\end{tabular}
\vspace{1cm}


\begin{tabular}{m{17cm}}
\textbf{70-variant}
\newline

\textbf{1.} Kópaǵzalılardıń EÚUB si. Keltirilmeytuǵın   kópaǵzalılar. \\
\textbf{2.} Kompleks sanlar hám olar ústinde ámeller. Muavr formulası.  \\
\textbf{3.} Kompleks sannan koren shıǵarıń: $\sqrt[6]{-2 i}$; \\
\textbf{4.} Qanday shártler orınlanǵanda $x^4+p x^2-q$ kópaǵzalısı $x^2+m q+1$ kópaǵzalısına qaldıqsız bólinedi?. \\
\textbf{5.} Kópaǵzalılardıń eń úlken ulıwma bóliwshisin tabıń:  $f(x)=x^5+3 x^4-12 x^3-52 x^2-52 x-12$ hám $g(x)=x^4+3 x^3-6 x^2-22 x-12$ \\

\end{tabular}
\vspace{1cm}


\begin{tabular}{m{17cm}}
\textbf{71-variant}
\newline

\textbf{1.} Qaldıqlı bóliw.  \\
\textbf{2.} Kompleks sannan koren shıǵarıw. \\
\textbf{3.} Algebralıq formada esaplań: $(1+2 i)^5-(1-2 i)^5$; \\
\textbf{4.} Gorner sxemasınan paydalanıp, $f(x)$ kópaǵzalısın $x-x_0$ dárejeleri boyınsha jayıń hám $f\left(x_0\right)$ di esaplań:  $f(x)=2 x^5+x^3-3 x^2+1 ; \quad x_0=1$ \\
\textbf{5.} Kompleks sanlar maydanında ápiwayı bólsheklerge jayıń: $\frac{3 x^2+6 x-23}{(x-1)^3(x+1)^2(x-2)}$ \\

\end{tabular}
\vspace{1cm}


\begin{tabular}{m{17cm}}
\textbf{72-variant}
\newline

\textbf{1.} Bir belgisizli kópaǵzalılar. Gorner sxeması. Bezu teoreması.  \\
\textbf{2.} Ratsional bólshekler hám olardı eń ápiwayı bólsheklerge ajıratıw. \\
\textbf{3.} Kompleks sannan koren shıǵarıń: $\sqrt[4]{-4}$; \\
\textbf{4.} Gorner sxemasınan paydalanıp, $f(x)$ kópaǵzalısın $x-x_0$ dárejeleri boyınsha jayıń hám $f\left(x_0\right)$ di esaplań: $f(x)=x^5-3 x^4+3 x^2-2 x+1 ; \quad x_0=-1$ \\
\textbf{5.} Haqıyqıy sanlar maydanında ápiwayı bólsheklerge jayıń:  $\frac{2 x-1}{x(x+1)^2\left(x^2+x+1\right)^2}$; \\

\end{tabular}
\vspace{1cm}


\begin{tabular}{m{17cm}}
\textbf{73-variant}
\newline

\textbf{1.} Kompleks sannan koren shıǵarıw. \\
\textbf{2.} Kópaǵzalılardıń EÚUB si. Keltirilmeytuǵın   kópaǵzalılar. \\
\textbf{3.} Kompleks sannan koren shıǵarıń: $\sqrt[4]{(2+2 i)(-1+i \sqrt{3})}$ \\
\textbf{4.} Qaldıqlı bóliwdi orınlań:  $x^3-3 x^2+4 x-1$ di $x^2-x+1$ ge; \\
\textbf{5.} Haqıyqıy sanlar maydanında ápiwayı bólsheklerge jayıń:  $\frac{x}{(x+1)\left(x^2+1\right)^2}$; \\

\end{tabular}
\vspace{1cm}


\begin{tabular}{m{17cm}}
\textbf{74-variant}
\newline

\textbf{1.} Kompleks sanlar hám olar ústinde ámeller. Muavr formulası.  \\
\textbf{2.} Ratsional bólshekler hám olardı eń ápiwayı bólsheklerge ajıratıw. \\
\textbf{3.} Algebralıq formada esaplań:  $\left(-\frac{1}{2}+i \frac{\sqrt{3}}{2}\right)^3$; \\
\textbf{4.} Qaldıqlı bóliwdi orınlań: $x^4-2 x^3+4 x^2-6 x+8$ di $x-1$ ge; \\
\textbf{5.} Haqıyqıy sanlar maydanında ápiwayı bólsheklerge jayıń:  $\frac{1}{(x+1)(x-3)(x+5)}$. \\

\end{tabular}
\vspace{1cm}


\begin{tabular}{m{17cm}}
\textbf{75-variant}
\newline

\textbf{1.} Bir belgisizli kópaǵzalılar. Gorner sxeması. Bezu teoreması.  \\
\textbf{2.} Qaldıqlı bóliw.  \\
\textbf{3.} Trigonometriyalıq kórinisin tabıń: $3 i$; \\
\textbf{4.} Qaldıqlı bóliwdi orınlań: $2 x^5-5 x^3-8 x$ ti $x+3$ ge; \\
\textbf{5.} Haqıyqıy sanlar maydanında ápiwayı bólsheklerge jayıń:  $\frac{1}{\left(x^4-4\right)^2}$; \\

\end{tabular}
\vspace{1cm}


\begin{tabular}{m{17cm}}
\textbf{76-variant}
\newline

\textbf{1.} Qaldıqlı bóliw.  \\
\textbf{2.} Kompleks sanlar hám olar ústinde ámeller. Muavr formulası.  \\
\textbf{3.} Kompleks sannan koren shıǵarıń: $\sqrt[3]{2-2 i}$; \\
\textbf{4.} Qaldıqlı bóliwdi orınlań: $4 x^3+x^2$ tı $x+1+i$ ge; \\
\textbf{5.} Kompleks sanlar maydanında ápiwayı bólsheklerge jayıń: $\frac{1}{\left(x^2-1\right)^2}$; \\

\end{tabular}
\vspace{1cm}


\begin{tabular}{m{17cm}}
\textbf{77-variant}
\newline

\textbf{1.} Kópaǵzalılardıń EÚUB si. Keltirilmeytuǵın   kópaǵzalılar. \\
\textbf{2.} Bir belgisizli kópaǵzalılar. Gorner sxeması. Bezu teoreması.  \\
\textbf{3.} Algebralıq formada esaplań:  $\frac{(1+2 i)^2-(1-i)^2}{(3+2 i)^3-(2+i)^2}$; \\
\textbf{4.} Gorner sxemasınan paydalanıp, $f(x)$ kópaǵzalısın $x-x_0$ dárejeleri boyınsha jayıń hám $f\left(x_0\right)$ di esaplań: $f(x)=x^4-3 x^3-10 x^2+2 x+5 ; \quad x_0=-2$. \\
\textbf{5.} Kópaǵzalılardıń eń úlken ulıwma bóliwshisin tabıń:  $f(x)=x^4+7 x^3+19 x^2+23 x+10$ hám $g(x)=x^4+7 x^3+18 x^2+22 x+12$ \\

\end{tabular}
\vspace{1cm}


\begin{tabular}{m{17cm}}
\textbf{78-variant}
\newline

\textbf{1.} Ratsional bólshekler hám olardı eń ápiwayı bólsheklerge ajıratıw. \\
\textbf{2.} Kompleks sannan koren shıǵarıw. \\
\textbf{3.} Algebralıq formada esaplań: $\frac{(5+2 i)(4-3 i)}{(1-2 i)(1+3 i)}$; \\
\textbf{4.} Gorner sxemasınan paydalanıp, $f(x)$ kópaǵzalısın $x-x_0$ dárejeleri boyınsha jayıń hám $f\left(x_0\right)$ di esaplań: $f(x)=3 x^4+4 x^3+5 x^2+x+33 ; \quad x_0=-2$ \\
\textbf{5.} Haqıyqıy sanlar maydanında ápiwayı bólsheklerge jayıń:  $\frac{x}{\left(x^2-1\right)^2}$; \\

\end{tabular}
\vspace{1cm}


\begin{tabular}{m{17cm}}
\textbf{79-variant}
\newline

\textbf{1.} Kópaǵzalılardıń EÚUB si. Keltirilmeytuǵın   kópaǵzalılar. \\
\textbf{2.} Kompleks sannan koren shıǵarıw. \\
\textbf{3.} Algebralıq formada esaplań: $\sqrt{-15+8 i}$; \\
\textbf{4.} Qaldıqlı bóliwdi orınlań:  $x^3-x^2-x$ t1 $x-1+2 i$ ge; \\
\textbf{5.} Haqıyqıy sanlar maydanında ápiwayı bólsheklerge jayıń:  $\frac{1}{x^3-1}$; \\

\end{tabular}
\vspace{1cm}


\begin{tabular}{m{17cm}}
\textbf{80-variant}
\newline

\textbf{1.} Bir belgisizli kópaǵzalılar. Gorner sxeması. Bezu teoreması.  \\
\textbf{2.} Ratsional bólshekler hám olardı eń ápiwayı bólsheklerge ajıratıw. \\
\textbf{3.} Algebralıq formada esaplań: $(5+4 i)+(3-7 i)-(2+5 i)$; \\
\textbf{4.} Gorner sxemasınan paydalanıp, $f(x)$ kópaǵzalısın $x-x_0$ dárejeleri boyınsha jayıń hám $f\left(x_0\right)$ di esaplań: $f(x)=x^4+2 x^3-3 x^2-4 x+1 ; \quad x_0=-1$ \\
\textbf{5.} Kompleks sanlar maydanında ápiwayı bólsheklerge jayıń: $\frac{1}{x^4+4}$; \\

\end{tabular}
\vspace{1cm}


\begin{tabular}{m{17cm}}
\textbf{81-variant}
\newline

\textbf{1.} Kompleks sanlar hám olar ústinde ámeller. Muavr formulası.  \\
\textbf{2.} Qaldıqlı bóliw.  \\
\textbf{3.} Kompleks sannan koren shıǵarıń: $\sqrt{\frac{1+i}{\sqrt{3}-i}}$; \\
\textbf{4.} Gorner sxemasınan paydalanıp, $f(x)$ kópaǵzalısın $x-x_0$ dárejeleri boyınsha jayıń hám $f\left(x_0\right)$ di esaplań: $f(x)=2 x^5-5 x^3-8 x ; \quad x_0=-3$ \\
\textbf{5.} Kompleks sanlar maydanında ápiwayı bólsheklerge jayıń: $\frac{1}{(x-1)(x-2)(x-3)(x-4)}$; \\

\end{tabular}
\vspace{1cm}


\begin{tabular}{m{17cm}}
\textbf{82-variant}
\newline

\textbf{1.} Bir belgisizli kópaǵzalılar. Gorner sxeması. Bezu teoreması.  \\
\textbf{2.} Qaldıqlı bóliw.  \\
\textbf{3.} Trigonometriyalıq kórinisin tabıń: $4-i$; \\
\textbf{4.} Gorner sxemasınan paydalanıp, $f(x)$ kópaǵzalısın $x-x_0$ dárejeleri boyınsha jayıń hám $f\left(x_0\right)$ di esaplań: $f(x)=x^4-3 x^3+6 x^2-10 x+16 ; x_0=4$ \\
\textbf{5.} Kompleks sanlar maydanında ápiwayı bólsheklerge jayıń: $\frac{1}{\left(x^2-1\right)^2}$; \\

\end{tabular}
\vspace{1cm}


\begin{tabular}{m{17cm}}
\textbf{83-variant}
\newline

\textbf{1.} Kópaǵzalılardıń EÚUB si. Keltirilmeytuǵın   kópaǵzalılar. \\
\textbf{2.} Kompleks sannan koren shıǵarıw. \\
\textbf{3.} Kompleks sannan koren shıǵarıń:  $\sqrt[3]{i}$; \\
\textbf{4.} Qaldıqlı bóliwdi orınlań: $x^3-3 x^2-x-1$ di $3 x^2-2 x+1$ ge; \\
\textbf{5.} Kópaǵzalılardıń eń úlken ulıwma bóliwshisin tabıń: $f(x)=x^5+x^4-x^3-3 x^2-3 x-1$ hám $g(x)=x^4-2 x^3-x^2-2 x+1$ \\

\end{tabular}
\vspace{1cm}


\begin{tabular}{m{17cm}}
\textbf{84-variant}
\newline

\textbf{1.} Kompleks sanlar hám olar ústinde ámeller. Muavr formulası.  \\
\textbf{2.} Ratsional bólshekler hám olardı eń ápiwayı bólsheklerge ajıratıw. \\
\textbf{3.} Algebralıq formada esaplań:  $(1+i)(2+i)+\frac{5}{1+2 i}$; \\
\textbf{4.} Qaldıqlı bóliwdi orınlań: $2 x^4-3 x^3+4 x^2-5 x+6$ nı $x^2-3 x+1$ ge; \\
\textbf{5.} Haqıyqıy sanlar maydanında ápiwayı bólsheklerge jayıń:  $\frac{x}{(x+1)\left(x^2+1\right)^2}$; \\

\end{tabular}
\vspace{1cm}


\begin{tabular}{m{17cm}}
\textbf{85-variant}
\newline

\textbf{1.} Qaldıqlı bóliw.  \\
\textbf{2.} Bir belgisizli kópaǵzalılar. Gorner sxeması. Bezu teoreması.  \\
\textbf{3.} Algebralıq formada esaplań: $\frac{5+i}{(1-2 i)(5-i)}$; \\
\textbf{4.} Gorner sxemasınan paydalanıp, $f(x)$ kópaǵzalısın $x-x_0$ dárejeleri boyınsha jayıń hám $f\left(x_0\right)$ di esaplań: $f(x)=x^4-3 x^3-10 x^2+2 x+5 ; \quad x_0=-2$. \\
\textbf{5.} Evklid algoritminen paydalanıp, $f(x)$ hám $g(x)$ kópaǵzalıları ushın $f(x) \cdot u(x)+g(x) \cdot v(x)=d(x)$ teńligin qanaatlandırıwshı $u(x)$ hám $v(x)$ kópaǵzalıların tabıń, bunda $d(x)=(f(x), g(x))$:  $f(x)=x^5+3 x^4+x^3+x^2+3 x+1$ hám $g(x)=x^4+2 x^3+x+2$ \\

\end{tabular}
\vspace{1cm}


\begin{tabular}{m{17cm}}
\textbf{86-variant}
\newline

\textbf{1.} Kompleks sannan koren shıǵarıw. \\
\textbf{2.} Kompleks sanlar hám olar ústinde ámeller. Muavr formulası.  \\
\textbf{3.}  $x, y \in R$, tabıń, egerde: $(2-3 i) x+(3+2 i) y=2-5 i$; \\
\textbf{4.} Qaldıqlı bóliwdi orınlań:  $x^3-x^2-x$ t1 $x-1+2 i$ ge; \\
\textbf{5.} Evklid algoritminen paydalanıp, $f(x)$ hám $g(x)$ kópaǵzalıları ushın $f(x) \cdot u(x)+g(x) \cdot v(x)=d(x)$ teńligin qanaatlandırıwshı $u(x)$ hám $v(x)$ kópaǵzalıların tabıń, bunda $d(x)=(f(x), g(x))$:  $f(x)=3 x^4-5 x^3+4 x^2-2 x+1$ hám $g(x)=3 x^3-2 x^2+x-1$ \\

\end{tabular}
\vspace{1cm}


\begin{tabular}{m{17cm}}
\textbf{87-variant}
\newline

\textbf{1.} Kópaǵzalılardıń EÚUB si. Keltirilmeytuǵın   kópaǵzalılar. \\
\textbf{2.} Ratsional bólshekler hám olardı eń ápiwayı bólsheklerge ajıratıw. \\
\textbf{3.} Trigonometriyalıq kórinisin tabıń: $-2+i$; \\
\textbf{4.} Gorner sxemasınan paydalanıp, $f(x)$ kópaǵzalısın $x-x_0$ dárejeleri boyınsha jayıń hám $f\left(x_0\right)$ di esaplań: $f(x)=3 x^5+x^4-19 x^2-13 x-10 ; \quad x_0=2$ \\
\textbf{5.} Haqıyqıy sanlar maydanında ápiwayı bólsheklerge jayıń:  $\frac{2 x-1}{x(x+1)^2\left(x^2+x+1\right)^2}$; \\

\end{tabular}
\vspace{1cm}


\begin{tabular}{m{17cm}}
\textbf{88-variant}
\newline

\textbf{1.} Kópaǵzalılardıń EÚUB si. Keltirilmeytuǵın   kópaǵzalılar. \\
\textbf{2.} Kompleks sanlar hám olar ústinde ámeller. Muavr formulası.  \\
\textbf{3.} Algebralıq formada esaplań: $\frac{(2-3 i)(4-i)}{5-i}$; \\
\textbf{4.} Gorner sxemasınan paydalanıp, $f(x)$ kópaǵzalısın $x-x_0$ dárejeleri boyınsha jayıń hám $f\left(x_0\right)$ di esaplań: $f(x)=x^4+2 x^3-3 x^2-4 x+1 ; \quad x_0=-1$ \\
\textbf{5.} Kompleks sanlar maydanında ápiwayı bólsheklerge jayıń: $\frac{1}{x^2-1}$; \\

\end{tabular}
\vspace{1cm}


\begin{tabular}{m{17cm}}
\textbf{89-variant}
\newline

\textbf{1.} Bir belgisizli kópaǵzalılar. Gorner sxeması. Bezu teoreması.  \\
\textbf{2.} Qaldıqlı bóliw.  \\
\textbf{3.} Teńlemeni sheshiń:  $(2+i) x^2-(5-i) x+(2-2 i)=0$; \\
\textbf{4.} Qaldıqlı bóliwdi orınlań:  $x^3-3 x^2+4 x-1$ di $x^2-x+1$ ge; \\
\textbf{5.} Haqıyqıy sanlar maydanında ápiwayı bólsheklerge jayıń:  $\frac{x^2}{(x-1)(x+2)(x+3)}$; \\

\end{tabular}
\vspace{1cm}


\begin{tabular}{m{17cm}}
\textbf{90-variant}
\newline

\textbf{1.} Ratsional bólshekler hám olardı eń ápiwayı bólsheklerge ajıratıw. \\
\textbf{2.} Kompleks sannan koren shıǵarıw. \\
\textbf{3.} Algebralıq formada esaplań: $\sqrt{8+6 i}$; г) $\sqrt{2-3 i}$. \\
\textbf{4.} Qaldıqlı bóliwdi orınlań: $x^4-3 x^2+2 x-5$ ti $x^2+2 x-1$ ge; \\
\textbf{5.} Haqıyqıy sanlar maydanında ápiwayı bólsheklerge jayıń:  $\frac{1}{(x+1)(x-3)(x+5)}$. \\

\end{tabular}
\vspace{1cm}


\begin{tabular}{m{17cm}}
\textbf{91-variant}
\newline

\textbf{1.} Bir belgisizli kópaǵzalılar. Gorner sxeması. Bezu teoreması.  \\
\textbf{2.} Kompleks sanlar hám olar ústinde ámeller. Muavr formulası.  \\
\textbf{3.} Algebralıq formada esaplań: $(2+i)^7+(2-i)^7$. \\
\textbf{4.} Qaldıqlı bóliwdi orınlań: $x^4-2 x^3+4 x^2-6 x+8$ di $x-1$ ge; \\
\textbf{5.} Kópaǵzalılardıń eń úlken ulıwma bóliwshisin tabıń:  $f(x)=2 x^6-5 x^5+14 x^4+36 x^3+86 x^2+12 x-31$ hám $g(x)=2 x^5-9 x^4+2 x^3+37 x^2+10 x-14$ \\

\end{tabular}
\vspace{1cm}


\begin{tabular}{m{17cm}}
\textbf{92-variant}
\newline

\textbf{1.} Kompleks sannan koren shıǵarıw. \\
\textbf{2.} Kópaǵzalılardıń EÚUB si. Keltirilmeytuǵın   kópaǵzalılar. \\
\textbf{3.} Esaplań:  $\left(\frac{1+i \sqrt{3}}{1-i}\right)^{20}$; \\
\textbf{4.} Qaldıqlı bóliwdi orınlań: $2 x^5-5 x^3-8 x$ ti $x+3$ ge; \\
\textbf{5.} Haqıyqıy sanlar maydanında ápiwayı bólsheklerge jayıń:  $\frac{1}{\left(x^4-4\right)^2}$; \\

\end{tabular}
\vspace{1cm}


\begin{tabular}{m{17cm}}
\textbf{93-variant}
\newline

\textbf{1.} Qaldıqlı bóliw.  \\
\textbf{2.} Ratsional bólshekler hám olardı eń ápiwayı bólsheklerge ajıratıw. \\
\textbf{3.} Algebralıq formada esaplań: $(5+4 i)+(3-7 i)-(2+5 i)$; \\
\textbf{4.} Qanday shártler orınlanǵanda $x^4+p x^2-q$ kópaǵzalısı $x^2+m q+1$ kópaǵzalısına qaldıqsız bólinedi?. \\
\textbf{5.} Kompleks sanlar maydanında ápiwayı bólsheklerge jayıń: $\frac{x}{\left(x^2-1\right)^2}$; \\

\end{tabular}
\vspace{1cm}


\begin{tabular}{m{17cm}}
\textbf{94-variant}
\newline

\textbf{1.} Ratsional bólshekler hám olardı eń ápiwayı bólsheklerge ajıratıw. \\
\textbf{2.} Kópaǵzalılardıń EÚUB si. Keltirilmeytuǵın   kópaǵzalılar. \\
\textbf{3.} Kompleks sannan koren shıǵarıń: $\sqrt[4]{(2+2 i)(-1+i \sqrt{3})}$ \\
\textbf{4.} Gorner sxemasınan paydalanıp, $f(x)$ kópaǵzalısın $x-x_0$ dárejeleri boyınsha jayıń hám $f\left(x_0\right)$ di esaplań: $f(x)=5 x^5-19 x^3-7 x^2+9 x+3 ; \quad x_0=2$ \\
\textbf{5.} Evklid algoritminen paydalanıp, $f(x)$ hám $g(x)$ kópaǵzalıları ushın $f(x) \cdot u(x)+g(x) \cdot v(x)=d(x)$ teńligin qanaatlandırıwshı $u(x)$ hám $v(x)$ kópaǵzalıların tabıń, bunda $d(x)=(f(x), g(x))$:  $f(x)=2 x^4+3 x^3-3 x^2-5 x+2$ hám $g(x)=2 x^3+x^2-x-1$ \\

\end{tabular}
\vspace{1cm}


\begin{tabular}{m{17cm}}
\textbf{95-variant}
\newline

\textbf{1.} Qaldıqlı bóliw.  \\
\textbf{2.} Kompleks sanlar hám olar ústinde ámeller. Muavr formulası.  \\
\textbf{3.} Teńlemeni sheshiń:  $x^2+3 x-10 i=0$; \\
\textbf{4.} Gorner sxemasınan paydalanıp, $f(x)$ kópaǵzalısın $x-x_0$ dárejeleri boyınsha jayıń hám $f\left(x_0\right)$ di esaplań:  $f(x)=2 x^5+x^3-3 x^2+1 ; \quad x_0=1$ \\
\textbf{5.} Evklid algoritminen paydalanıp, $f(x)$ hám $g(x)$ kópaǵzalıları ushın $f(x) \cdot u(x)+g(x) \cdot v(x)=d(x)$ teńligin qanaatlandırıwshı $u(x)$ hám $v(x)$ kópaǵzalıların tabıń, bunda $d(x)=(f(x), g(x))$:  $f(x)=3 x^3-2 x^2+x+2$ hám $g(x)=x^2-x+1$ \\

\end{tabular}
\vspace{1cm}


\begin{tabular}{m{17cm}}
\textbf{96-variant}
\newline

\textbf{1.} Kompleks sannan koren shıǵarıw. \\
\textbf{2.} Bir belgisizli kópaǵzalılar. Gorner sxeması. Bezu teoreması.  \\
\textbf{3.} Esaplań:  $(1+i)^{25}$; \\
\textbf{4.} Gorner sxemasınan paydalanıp, $f(x)$ kópaǵzalısın $x-x_0$ dárejeleri boyınsha jayıń hám $f\left(x_0\right)$ di esaplań: $f(x)=x^5-3 x^4+3 x^2-2 x+1 ; \quad x_0=-1$ \\
\textbf{5.} Kompleks sanlar maydanında ápiwayı bólsheklerge jayıń: $\frac{1}{x\left(x^2+1\right)}$; \\

\end{tabular}
\vspace{1cm}


\begin{tabular}{m{17cm}}
\textbf{97-variant}
\newline

\textbf{1.} Ratsional bólshekler hám olardı eń ápiwayı bólsheklerge ajıratıw. \\
\textbf{2.} Kompleks sannan koren shıǵarıw. \\
\textbf{3.} Algebralıq formada esaplań:  $\left(-\frac{1}{2}+i \frac{\sqrt{3}}{2}\right)^3$; \\
\textbf{4.} Gorner sxemasınan paydalanıp, $f(x)$ kópaǵzalısın $x-x_0$ dárejeleri boyınsha jayıń hám $f\left(x_0\right)$ di esaplań: $f(x)=3 x^4+4 x^3+5 x^2+x+33 ; \quad x_0=-2$ \\
\textbf{5.} Evklid algoritminen paydalanıp, $f(x)$ hám $g(x)$ kópaǵzalıları ushın $f(x) \cdot u(x)+g(x) \cdot v(x)=d(x)$ teńligin qanaatlandırıwshı $u(x)$ hám $v(x)$ kópaǵzalıların tabıń, bunda $d(x)=(f(x), g(x))$:  $f(x)=x^4-x^3-4 x^2+4 x+1$ hám $g(x)=x^2-x-1$ \\

\end{tabular}
\vspace{1cm}


\begin{tabular}{m{17cm}}
\textbf{98-variant}
\newline

\textbf{1.} Qaldıqlı bóliw.  \\
\textbf{2.} Kompleks sanlar hám olar ústinde ámeller. Muavr formulası.  \\
\textbf{3.} Algebralıq formada esaplań:  $\sqrt{3+4 i}$; \\
\textbf{4.} Qaldıqlı bóliwdi orınlań: $4 x^3+x^2$ tı $x+1+i$ ge; \\
\textbf{5.} Kópaǵzalılardıń eń úlken ulıwma bóliwshisin tabıń:  $f(x)=x^4+x^3-4 x+5$ hám $g(x)=2 x^3-x^2-2 x+2$ \\

\end{tabular}
\vspace{1cm}


\begin{tabular}{m{17cm}}
\textbf{99-variant}
\newline

\textbf{1.} Bir belgisizli kópaǵzalılar. Gorner sxeması. Bezu teoreması.  \\
\textbf{2.} Kópaǵzalılardıń EÚUB si. Keltirilmeytuǵın   kópaǵzalılar. \\
\textbf{3.} Algebralıq formada esaplań: $\frac{(5+2 i)(4-3 i)}{(1-2 i)(1+3 i)}$; \\
\textbf{4.} Gorner sxemasınan paydalanıp, $f(x)$ kópaǵzalısın $x-x_0$ dárejeleri boyınsha jayıń hám $f\left(x_0\right)$ di esaplań: $f(x)=x^4-2 x^3+4 x^2-6 x+8 ; \quad x_0=1$ \\
\textbf{5.} Haqıyqıy sanlar maydanında ápiwayı bólsheklerge jayıń:  $\frac{x^2}{x^4-16}$; \\

\end{tabular}
\vspace{1cm}


\begin{tabular}{m{17cm}}
\textbf{100-variant}
\newline

\textbf{1.} Kópaǵzalılardıń EÚUB si. Keltirilmeytuǵın   kópaǵzalılar. \\
\textbf{2.} Kompleks sanlar hám olar ústinde ámeller. Muavr formulası.  \\
\textbf{3.} Kompleks sannan koren shıǵarıń:  $\sqrt[3]{i}$; \\
\textbf{4.} Qanday shártler orınlanǵanda $x^3+p x+q$ kópaǵzalısı $x^2+m x-1$ kópaǵzalısına qaldıqsız bólinedi?. \\
\textbf{5.} Kópaǵzalılardıń eń úlken ulıwma bóliwshisin tabıń:  $f(x)=x^5+3 x^2-2 x+2$ hám $g(x)=x^6+x^5+x^4-3 x^2+2 x-6$ \\

\end{tabular}
\vspace{1cm}



\end{document}
