Qoldiqli bo’lishni bajaring:  $x^3-3 x^2+4 x-1$ di $x^2-x+1$ ge;
Qoldiqli bo’lishni bajaring: $x^4-3 x^2+2 x-5$ ti $x^2+2 x-1$ ge;
Qoldiqli bo’lishni bajaring: $x^4-2 x^3+4 x^2-6 x+8$ di $x-1$ ge;
Qoldiqli bo’lishni bajaring: $2 x^5-5 x^3-8 x$ ti $x+3$ ge;
Qoldiqli bo’lishni bajaring: $2 x^4-3 x^3+4 x^2-5 x+6$ nı $x^2-3 x+1$ ge;
Qoldiqli bo’lishni bajaring: $x^3-3 x^2-x-1$ di $3 x^2-2 x+1$ ge;
Qoldiqli bo’lishni bajaring: $4 x^3+x^2$ tı $x+1+i$ ge;
Qanday shartlar bajarilganda $x^3+p x+q$ ko’phadi $x^2+m x-1$ ko’phadga qoldiqsiz bo’linadi?.
Qoldiqli bo’lishni bajaring:  $x^3-x^2-x$ t1 $x-1+2 i$ ge;
Qanday shartlar bajarilganda $x^4+p x^2-q$ ko’phadi $x^2+m q+1$ ko’phadga qoldiqsiz bo’linadi?.
Gorner sxemasidan foydalanib, $f(x)$ ko’phadni $x-x_0$ darajalari bo’yicha yoying va $f\left(x_0\right)$ ni hisoblang:  $f(x)=2 x^5+x^3-3 x^2+1 ; \quad x_0=1$
Gorner sxemasidan foydalanib, $f(x)$ ko’phadni $x-x_0$ darajalari bo’yicha yoying va $f\left(x_0\right)$ ni hisoblang: $f(x)=x^5-3 x^4+3 x^2-2 x+1 ; \quad x_0=-1$
Gorner sxemasidan foydalanib, $f(x)$ ko’phadni $x-x_0$ darajalari bo’yicha yoying va $f\left(x_0\right)$ ni hisoblang: $f(x)=3 x^4+4 x^3+5 x^2+x+33 ; \quad x_0=-2$
Gorner sxemasidan foydalanib, $f(x)$ ko’phadni $x-x_0$ darajalari bo’yicha yoying va $f\left(x_0\right)$ ni hisoblang: $f(x)=5 x^5-19 x^3-7 x^2+9 x+3 ; \quad x_0=2$
Gorner sxemasidan foydalanib, $f(x)$ ko’phadni $x-x_0$ darajalari bo’yicha yoying va $f\left(x_0\right)$ ni hisoblang: $f(x)=x^4+2 x^3-3 x^2-4 x+1 ; \quad x_0=-1$
Gorner sxemasidan foydalanib, $f(x)$ ko’phadni $x-x_0$ darajalari bo’yicha yoying va $f\left(x_0\right)$ ni hisoblang: $f(x)=x^4-3 x^3+6 x^2-10 x+16 ; x_0=4$
Gorner sxemasidan foydalanib, $f(x)$ ko’phadni $x-x_0$ darajalari bo’yicha yoying va $f\left(x_0\right)$ ni hisoblang: $f(x)=x^4-2 x^3+4 x^2-6 x+8 ; \quad x_0=1$
Gorner sxemasidan foydalanib, $f(x)$ ko’phadni $x-x_0$ darajalari bo’yicha yoying va $f\left(x_0\right)$ ni hisoblang: $f(x)=2 x^5-5 x^3-8 x ; \quad x_0=-3$
Gorner sxemasidan foydalanib, $f(x)$ ko’phadni $x-x_0$ darajalari bo’yicha yoying va $f\left(x_0\right)$ ni hisoblang: $f(x)=3 x^5+x^4-19 x^2-13 x-10 ; \quad x_0=2$
Gorner sxemasidan foydalanib, $f(x)$ ko’phadni $x-x_0$ darajalari bo’yicha yoying va $f\left(x_0\right)$ ni hisoblang: $f(x)=x^4-3 x^3-10 x^2+2 x+5 ; \quad x_0=-2$.