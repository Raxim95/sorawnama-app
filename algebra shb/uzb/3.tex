Ko’phadlarning EKUB toping:  $f(x)=x^4+x^3-3 x^2-4 x-1$ hám $g(x)=x^3+x^2-x-1$
Ko’phadlarning EKUB toping:  $f(x)=x^5+2 x^4-4 x^3-3 x^2+8 x-5$ hám $g(x)=x^5+x^2-x-1$
Ko’phadlarning EKUB toping:  $f(x)=x^5+3 x^2-2 x+2$ hám $g(x)=x^6+x^5+x^4-3 x^2+2 x-6$
Ko’phadlarning EKUB toping:  $f(x)=x^4+x^3-4 x+5$ hám $g(x)=2 x^3-x^2-2 x+2$
Ko’phadlarning EKUB toping:  $f(x)=x^4-4 x^3+1$ hám $g(x)=x^3-3 x^2+1$
Ko’phadlarning EKUB toping: $f(x)=x^5+x^4-x^3-3 x^2-3 x-1$ hám $g(x)=x^4-2 x^3-x^2-2 x+1$
Ko’phadlarning EKUB toping:  $f(x)=x^4-10 x^2+1$ hám $g(x)=x^4-4 \sqrt{2} x^3+6 x^2+4 \sqrt{2} x+1$
Ko’phadlarning EKUB toping:  $f(x)=x^4+7 x^3+19 x^2+23 x+10$ hám $g(x)=x^4+7 x^3+18 x^2+22 x+12$
Ko’phadlarning EKUB toping:  $f(x)=x^5+3 x^4-12 x^3-52 x^2-52 x-12$ hám $g(x)=x^4+3 x^3-6 x^2-22 x-12$
Ko’phadlarning EKUB toping:  $f(x)=2 x^6-5 x^5+14 x^4+36 x^3+86 x^2+12 x-31$ hám $g(x)=2 x^5-9 x^4+2 x^3+37 x^2+10 x-14$
Evklid algoritmidan foydalanib, $f(x)$ va $g(x)$ ko’phadlar uchun $f(x) \cdot u(x)+g(x) \cdot v(x)=d(x)$ tengligini qanoatlantiradigan $u(x)$ va $v(x)$ ko’phadlarini toping, bu yerda $d(x)=(f(x), g(x))$:  $f(x)=x^4-2 x^3-16 x^2+5 x+9$ hám $g(x)=2 x^3-x^2-5 x+4$
Evklid algoritmidan foydalanib, $f(x)$ va $g(x)$ ko’phadlar uchun $f(x) \cdot u(x)+g(x) \cdot v(x)=d(x)$ tengligini qanoatlantiradigan $u(x)$ va $v(x)$ ko’phadlarini toping, bu yerda $d(x)=(f(x), g(x))$:  $f(x)=3 x^5+5 x^4-16 x^3-6 x^2-5 x-6$ hám $g(x)=3 x^4-4 x^3-x^2-x-2$
Evklid algoritmidan foydalanib, $f(x)$ va $g(x)$ ko’phadlar uchun $f(x) \cdot u(x)+g(x) \cdot v(x)=d(x)$ tengligini qanoatlantiradigan $u(x)$ va $v(x)$ ko’phadlarini toping, bu yerda $d(x)=(f(x), g(x))$:  $f(x)=x^5+3 x^4+x^3+x^2+3 x+1$ hám $g(x)=x^4+2 x^3+x+2$
Evklid algoritmidan foydalanib, $f(x)$ va $g(x)$ ko’phadlar uchun $f(x) \cdot u(x)+g(x) \cdot v(x)=d(x)$ tengligini qanoatlantiradigan $u(x)$ va $v(x)$ ko’phadlarini toping, bu yerda $d(x)=(f(x), g(x))$:  $f(x)=3 x^3-2 x^2+x+2$ hám $g(x)=x^2-x+1$
Evklid algoritmidan foydalanib, $f(x)$ va $g(x)$ ko’phadlar uchun $f(x) \cdot u(x)+g(x) \cdot v(x)=d(x)$ tengligini qanoatlantiradigan $u(x)$ va $v(x)$ ko’phadlarini toping, bu yerda $d(x)=(f(x), g(x))$:  $f(x)=x^4-x^3-4 x^2+4 x+1$ hám $g(x)=x^2-x-1$
Evklid algoritmidan foydalanib, $f(x)$ va $g(x)$ ko’phadlar uchun $f(x) \cdot u(x)+g(x) \cdot v(x)=d(x)$ tengligini qanoatlantiradigan $u(x)$ va $v(x)$ ko’phadlarini toping, bu yerda $d(x)=(f(x), g(x))$:  $f(x)=x^5-5 x^4-2 x^3+12 x^2-2 x+12$ hám $g(x)=x^3-5 x^2-3 x+17$
Evklid algoritmidan foydalanib, $f(x)$ va $g(x)$ ko’phadlar uchun $f(x) \cdot u(x)+g(x) \cdot v(x)=d(x)$ tengligini qanoatlantiradigan $u(x)$ va $v(x)$ ko’phadlarini toping, bu yerda $d(x)=(f(x), g(x))$:  $f(x)=2 x^4+3 x^3-3 x^2-5 x+2$ hám $g(x)=2 x^3+x^2-x-1$
Evklid algoritmidan foydalanib, $f(x)$ va $g(x)$ ko’phadlar uchun $f(x) \cdot u(x)+g(x) \cdot v(x)=d(x)$ tengligini qanoatlantiradigan $u(x)$ va $v(x)$ ko’phadlarini toping, bu yerda $d(x)=(f(x), g(x))$:  $f(x)=3 x^4-5 x^3+4 x^2-2 x+1$ hám $g(x)=3 x^3-2 x^2+x-1$
Evklid algoritmidan foydalanib, $f(x)$ va $g(x)$ ko’phadlar uchun $f(x) \cdot u(x)+g(x) \cdot v(x)=d(x)$ tengligini qanoatlantiradigan $u(x)$ va $v(x)$ ko’phadlarini toping, bu yerda $d(x)=(f(x), g(x))$:  $f(x)=x^5+5 x^4+9 x^3+7 x^2+5 x+3$ hám $g(x)=x^4+2 x^3+2 x^2+x+1$
Haqiqiy sonlar maydoni ustida sodda kasrlarga yoying:  $\frac{x^2}{(x-1)(x+2)(x+3)}$;
Haqiqiy sonlar maydoni ustida sodda kasrlarga yoying:  $\frac{1}{x^3-1}$;
Haqiqiy sonlar maydoni ustida sodda kasrlarga yoying:  $\frac{x^2}{x^4-16}$;
Haqiqiy sonlar maydoni ustida sodda kasrlarga yoying:  $\frac{1}{x^4+4}$;
Haqiqiy sonlar maydoni ustida sodda kasrlarga yoying:  $\frac{x^2}{x^6+27}$;
Haqiqiy sonlar maydoni ustida sodda kasrlarga yoying:  $\frac{x}{(x+1)\left(x^2+1\right)^2}$;
Haqiqiy sonlar maydoni ustida sodda kasrlarga yoying:  $\frac{2 x-1}{x(x+1)^2\left(x^2+x+1\right)^2}$;
Haqiqiy sonlar maydoni ustida sodda kasrlarga yoying:  $\frac{x}{\left(x^2-1\right)^2}$;
Haqiqiy sonlar maydoni ustida sodda kasrlarga yoying:  $\frac{1}{\left(x^4-4\right)^2}$;
Haqiqiy sonlar maydoni ustida sodda kasrlarga yoying:  $\frac{1}{(x+1)(x-3)(x+5)}$.
Kompleks sonlar maydoni ustida sodda kasrlarga yoying:$\frac{3+x}{(x-1)\left(x^2+1\right)}$;
Kompleks sonlar maydoni ustida sodda kasrlarga yoying:$\frac{x^2}{x^4-1}$;
Kompleks sonlar maydoni ustida sodda kasrlarga yoying:$\frac{1}{x^2-1}$;
Kompleks sonlar maydoni ustida sodda kasrlarga yoying:$\frac{1}{x^4+4}$;
Kompleks sonlar maydoni ustida sodda kasrlarga yoying:$\frac{x}{\left(x^2-1\right)^2}$;
Kompleks sonlar maydoni ustida sodda kasrlarga yoying:$\frac{1}{(x-1)(x-2)(x-3)(x-4)}$;
Kompleks sonlar maydoni ustida sodda kasrlarga yoying:$\frac{1}{\left(x^2-1\right)^2}$;
Kompleks sonlar maydoni ustida sodda kasrlarga yoying:$\frac{1}{x\left(x^2+1\right)}$;
Kompleks sonlar maydoni ustida sodda kasrlarga yoying:$\frac{3 x^2+6 x-23}{(x-1)^3(x+1)^2(x-2)}$