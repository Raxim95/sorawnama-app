T1.Kolcolar (anıqlaması, qásiyetleri, mısallar)
T2.p-adikalıq sanlar keńisligi (anıqlaması, qásiyetleri, mısallar)
A1. \(\mathbf{Q}\)da \(\sqrt{\frac{1}{4} + \sqrt{5}}\) tiń minimal kόpaǵzalısın tabıń.
A2. Ámellerdi orınlań: \(\mathbb{Z}_{12}\) de \(\left( 5x^{2} + 3x - 4 \right) + \left( 4x^{2} - x + 9 \right)\)
A3. Tómendegi sannıń \(p\)-adikalıq normasın tabıń. \(|15|_{3} =\)
B1. Tόmendegi kόplik kolco dúzedi ma? \(\mathbb{Q(}i\sqrt{n}) = \{ x + iy\sqrt{n}\ |\ x,y \in \mathbb{Q}\}\)
B2. Tόmendegi kόplik\(M_{2 \times 2}\left( \mathbb{R} \right)\)kolcosınıń úles kolcosı ekenliginkórsetiń.
\begin{quote}
\[T = \left\{ \begin{pmatrix}
a + b & b \\
 - b & a
\end{pmatrix}\left| \ \ a,b\mathbb{\in Z} \right.\  \right\}\]
\end{quote}
B3. \(x^{2} - 7\) kόpaǵzalı\(\mathbb{Q}(\sqrt{3})\)de keltirilmeytuǵın kόpaǵzalı ekenligin kórsetiń.
C1. Tόmendegi kόplikti maydan shártlerine tekseriń. \(Z_{p}\)
C2. Tómendegi kolconıń barlıq ideallarıń tabıń. Bul ideallardan qaysı-biri maksimal boladı? \(\mathbb{Z}_{25}\)
C3. Tόmendegi sáwlelendiriwdi gomomorfizm shártlerine tekseriń.
\[f:\begin{pmatrix}
a & b \\
 - b & a
\end{pmatrix} \rightarrow a + bi\]
T1.Pútinlik oblastı hám maydan (anıqlaması, qásiyetleri, mısallar)
T2. p-Adikalıq norma, p-Adikalıq norma (anıqlaması, qásiyetleri, mısallar)
A1. \(\mathbf{Q}\)da \(\sqrt{2} + \sqrt[3]{7}\)tiń minimal kόpaǵzalısın tabıń.
A2. Ámellerdi orınlań: \(\mathbb{Z}_{5}\) te \(\left( 3x^{2} + 3x - 4 \right)\left( x^{2} + 2 \right)\)
A3. Tómendegi sannıń \(p\)-adikalıq normasın tabıń. \(|6|_{3} =\)
B1. Tόmendegi kόplik kolco dúzedi ma? \(7\mathbb{Z}\)
B2. \(Z_{16}\) kolconıń barlıq úles kolcoların anıqlań.
B3. \(x^{2} + x + 1\) kόpaǵzalı \(\mathbb{Z}_{5}\) de keltirilmeytuǵın kόpaǵzalı ekenligin kórsetiń.
C1. Tόmendegi kόplikti maydan shártlerine tekseriń. \(\mathbb{Z}\left\lbrack \sqrt{n} \right\rbrack = \left\{ x + y\sqrt{n}\ \ \left| \right.\ x,y \in \mathbb{Z} \right\}\)
C2. Tómendegi kolconıń barlıq ideallarıń tabıń. Bul ideallardan qaysı-biri maksimal boladı? \(\mathbb{Z}_{27}\)
C3. Tόmendegi sáwlelendiriwdi gomomorfizm shártlerine tekseriń. \(f\left( \begin{pmatrix}
a & 0 \\
0 & a
\end{pmatrix} \right) = a\)
T1.Kolco gomomorfizmleri hám ideallar (anıqlaması, qásiyetleri, mısallar)
T2. Teńlemelerdiń radikallarda sheshiliwi (anıqlaması, qásiyetleri, mısallar)
A1. \(\mathbf{Q}\)da \(\sqrt{2} + \sqrt{3}i\)tiń minimal kόpaǵzalısın tabıń.
A2. Ámellerdi orınlań: \(\mathbb{Z}_{10}\) da \(\left( 7x^{3} + 3x^{2} - x \right) + \left( 6x^{2} - 8x + 4 \right)\)
A3. Tómendegi sannıń \(p\)-adikalıq normasın tabıń.
B1. Tόmendegi kόplik kolco dúzedi ma? \(\mathbb{Q}\left( \sqrt{2} \right) = \left\{ a + b\sqrt{2}:a,b \in \mathbb{Q} \right\}\)
B2. Tόmendegi kόplik\(M_{2 \times 2}\left( \mathbb{R} \right)\) niń úles maydanı ekenligin kórsetiń. \(A = \left\{ \left. \ \begin{pmatrix}
a & 0 \\
2b & a
\end{pmatrix} \right|a,b\mathbb{\in R},a \neq 0 \right\}\)
B3. \(x^{2} + 1\) kόpaǵzalı \(\mathbb{Z}_{3}\)de keltirilmeytuǵın kόpaǵzalı ekenligin kórsetiń.
C1. Tόmendegi kόplikti maydan shártlerine tekseriń. \(\mathbf{Q} = \left\{ \frac{m}{n},\ \ m \in Z,\ n \in N \right\}\)
C2. Tómendegi kolconıń barlıq ideallarıń tabıń. Bul ideallardan qaysı-biri maksimal boladı? \(\mathbb{M}_{2}\left( \mathbb{R} \right)\), elementleri \(\mathbb{R}\) bol\(g'\)an \(2 \times 2\) matrica
C3. Tόmendegi sáwlelendiriwdi gomomorfizm shártlerine tekseriń. \(f(a + ib) = \begin{pmatrix}
a & b \\
 - b & a
\end{pmatrix}\)
T1.Maksimal hám ápiwayı ideallar (anıqlaması, qásiyetleri, mısallar)
T2.Fundamental teoremalar (anıqlaması, qásiyetleri, mısallar)
A1. \(\mathbf{Q}\)da \(\sqrt{1/3 + \sqrt{7}}\)tiń minimal kόpaǵzalısın tabıń.
A2. Ámellerdi orınlań: \(\mathbb{Z}_{10}\) da \(\left( 5x^{2} + 3x - 4 \right)\left( 4x^{2} - x + 9 \right)\)
A3. Tómendegi sannıń \(p\)-adikalıq normasın tabıń. \(|\frac{3}{4}|_{2} =\)
B1. Tόmendegi kόplik kolco dúzedi ma? \(\mathbb{Z}\left( \sqrt{3} \right) = \left\{ a + b\sqrt{3}:a,b \in \mathbb{Z} \right\}\)
B2. Tόmendegi kόplik\(M_{2 \times 2}\left( \mathbb{R} \right)\) niń úles maydanı ekenliginkórsetiń.
\[A = \left\{ \left. \ \begin{pmatrix}
a & b \\
0 & a
\end{pmatrix} \right|a,b\mathbb{\in Z},a \neq 0 \right\}\]
B3. \(x^{4} - 5x^{2} + 6\) kόpaǵzalı \(\mathbb{Q}\) de keltirilmeytuǵın kόpaǵzalı ekenligin kórsetiń.
C1. \(Z_{18}\) kolconıń barlıq úles kolcoların anıqlań.
C2. Tómendegi kolconıń barlıq ideallarıń tabıń. Bul ideallardan qaysı-biri maksimal boladı? \(\mathbb{M}_{2}\left( \mathbb{Z} \right)\), elementleri \(\mathbb{Z}\) bol\(g'\)an \(2 \times 2\) matrica
C3. Tόmendegi sáwlelendiriwdi gomomorfizm shártlerine tekseriń. \(f(x + iy) = x \cdot y\)
T1.Kópagzalılar kolcosı (anıqlaması, qásiyetleri, mısallar)
T2.Maydanlar avtomorfizmleri (anıqlaması, qásiyetleri, mısallar)
A1. \(\mathbf{Q}\)da \(\sqrt{3} + \sqrt[3]{5}\)tiń minimal kόpaǵzalısın tabıń.
A2. Ámellerdi orınlań: \(\mathbb{Z}_{5}\) te \(\left( x^{2} + 3x - 1 \right)^{2}\)
A3. Tómendegi sannıń \(p\)-adikalıq normasın tabıń. \(|\frac{1}{4}|_{2} =\)
B1. Tόmendegi kόplik kolco dúzedi ma? \(\mathbb{Q(}\sqrt[3]{2}) = \left\{ a + b\sqrt[3]{2}:a,b \in \mathbb{Q} \right\}\)
B2. Tόmendegi kόplik\(M_{2 \times 2}\left( \mathbb{R} \right)\) niń úles maydanı ekenliginkórsetiń. \(A = \left\{ \left. \ \begin{pmatrix}
a & 0 \\
0 & a
\end{pmatrix} \right|a\mathbb{\in R},a \neq 0 \right\}\)
B3. \(x^{2} + x + 1\) kόpaǵzalı \(\mathbb{Z}_{3}\) de keltirilmeytuǵın kόpaǵzalı ekenligin kórsetiń.
C1. Tόmendegi kόplikti maydan shártlerine tekseriń. \(R = \left\{ \begin{pmatrix}
a & b \\
2b & a
\end{pmatrix}\ \ :\ \ a,b \in \mathbf{Q} \right\}\)
C2. \(Z_{12}\) niń barlıq idealların tabıń.
C3. Tόmendegi sáwlelendiriwdi gomomorfizm shártlerine tekseriń. \(f(x) = \sqrt[3]{x}\)
T1.Keltirilmeytuǵın kópaǵzalılar (anıqlaması, qásiyetleri, mısallar)
T2.Shekli maydannıń strukturası (anıqlaması, qásiyetleri, mısallar)
A1. \(\mathbf{Q}\)da \(\sqrt{3} + \sqrt{2}i\) tiń minimal kόpaǵzalısın tabıń.
A2. Ámellerdi orınlań: \(\mathbb{Z}_{12}\) de \((3x - 2)^{3}\)
A3. Tómendegi sannıń \(p\)-adikalıq normasın tabıń. \(|\frac{25}{36}|_{2} =\)
B1. Tόmendegi kόplik kolco dúzedi ma? \(R = \left\{ a + b\sqrt{2}\ :\ \ \ \ a,b \in \mathbf{Z} \right\}\)
B2. Tόmendegi kόplik\(M_{2 \times 2}\left( \mathbb{R} \right)\) niń úles maydanı ekenliginkórsetiń. \(A = \left\{ \left. \ \begin{pmatrix}
a & b \\
 - b & a
\end{pmatrix} \right|a,b\mathbb{\in Z},a^{2} + b^{2} \neq 0 \right\}\)
B3. Tómendegi maydanlardıń berilgen kópa\(g'\)zalı arqalı ajıralıw maydanın tabıń. \(\mathbb{Q}\) da \(x^{4} - 2\)
C1. Tόmendegi kόplikti maydan shártlerine tekseriń. \(R = M_{2 \times 2}\left( \mathbf{Q} \right)\)
C2. \(M_{2 \times 2}\left( \mathbf{Z} \right)\)kolcoda\(I = \left\{ \begin{bmatrix}
a & 0 \\
b & 0
\end{bmatrix}|a,b\mathbb{\in Z} \right\}\) ideal boladı ma?
C3. Tόmendegi sáwlelendiriwdi gomomorfizm shártlerine tekseriń. \(f(x) = x^{2} + x\)
T1.Bólshekler maydanı (anıqlaması, qásiyetleri, mısallar)
T2. Ajıralatuǵın maydanlar (anıqlaması, qásiyetleri, mısallar)
A1. \(\mathbf{Q}\)da \(\sqrt{\sqrt[3]{2} - i}\)tiń minimal kόpaǵzalısın tabıń.
A2. Tómendegi kópa\(g'\)zalınıń barlıq nollerin tabıń: \(\mathbb{Z}_{2}\) de \(x^{3} + x + 1\)
A3. Tómendegi sannıń \(p\)-adikalıq normasın tabıń. \(|\frac{9}{12}|_{7} =\)
B1. Tόmendegi kόplik kolco dúzedi ma? \(\left\{ a + b\sqrt{7}|a,b \in R \right\}\)
B2. Tόmendegi kόplik\(M_{2 \times 2}\left( \mathbb{R} \right)\) niń úles maydanı ekenliginkórsetiń. \(A = \left\{ \left. \ \begin{pmatrix}
a & b\sqrt{7} \\
 - b\sqrt{7} & a
\end{pmatrix} \right|a,b\mathbb{\in Q},a^{2} + 7b^{2} \neq 0 \right\}\)
B3. Tómendegi maydanlardıń berilgen kópa\(g'\)zalı arqalı ajıralıw maydanın tabıń.
\(\mathbb{Q}\) da \(x^{4} - 5x^{2} + 21\).
C1. Tόmendegi kόplikti maydan shártlerine tekseriń. \(R = \left\{ \begin{pmatrix}
a & b \\
2b & a
\end{pmatrix}\ \ :\ \ a,b \in \mathbf{Q} \right\}\)
C2. \(M_{2 \times 2}\left( \mathbf{Z} \right)\)kolcoda\(I = \left\{ \begin{bmatrix}
a & 0 \\
b & 0
\end{bmatrix}|a,b\mathbb{\in Z} \right\}\) ideal boladı ma?
C3. Tόmendegi sáwlelendiriwdi gomomorfizm shártlerine tekseriń. \(f\left( a + \sqrt{2}b \right) = a + bi\)
T1.Maydanlar keńeytpesi (anıqlaması, qásiyetleri, mısallar)
T2.Kolcolar (anıqlaması, qásiyetleri, mısallar)
A1. \(\mathbf{Q}\)da \(\sqrt{6 + 3\sqrt{2}}\)tiń minimal kόpaǵzalısın tabıń.
A2. Tómendegi kópa\(g'\)zalınıń barlıq nollerin tabıń: \(\mathbb{Z}_{5}\) de \(3x^{3} - 4x^{2} - x + 4\)
A3. Tómendegi sannıń \(p\)-adikalıq normasın tabıń. \(|\frac{8}{18}|_{5} =\)
B1. Tόmendegi kόplik kolco dúzedi ma? \(\mathbf{Q} = \left\{ \frac{m}{n},\ \ m \in Z,\ n \in N \right\}\)
B2. \(Z_{12}\) kolconıń barlıq úles kolcoların anıqlań.
B3. Tómendegi maydanlardıń berilgen kópa\(g'\)zalı arqalı ajıralıw maydanın tabıń. \(\mathbb{Q}\) da \(x^{4} - 10x^{2} + 21\).
C1. Tόmendegi kόplikti maydan shártlerine tekseriń. \(\mathbb{Q(}i\sqrt{n}) = \{ x + iy\sqrt{n}\ |\ x,y \in \mathbb{Q}\}\)
C2. \(M_{2 \times 2}\left( \mathbf{Z} \right)\)kolcoda\(I = \left\{ \begin{bmatrix}
a & 0 \\
0 & 0
\end{bmatrix}|a\mathbb{\in Z} \right\}\) ideal boladı ma?
C3. Tόmendegi sáwlelendiriwdi gomomorfizm shártlerine tekseriń. \(f\left( \begin{pmatrix}
a & 0 \\
0 & a
\end{pmatrix} \right) = a\)
T1.Ajıralatuǵın maydanlar (anıqlaması, qásiyetleri, mısallar)
T2.Kolco gomomorfizmleri hám ideallar (anıqlaması, qásiyetleri, mısallar)
A1. \(\mathbf{Q}\)da \(\sqrt{3 - \sqrt{3}}\)tiń minimal kόpaǵzalısın tabıń.
A2. Tómendegi kópa\(g'\)zalınıń barlıq nollerin tabıń: \(\mathbb{Z}_{12}\) de \(5x^{3} + 4x^{2} - x + 9\)
A3. Tómendegi sannıń \(p\)-adikalıq normasın tabıń. \(|\frac{7}{15}|_{3} =\)
B1. Tόmendegi kόplik kolco dúzedi ma? \(R = \left\{ \begin{pmatrix}
a & b \\
2b & a
\end{pmatrix}\ \ :\ \ a,b \in \mathbf{Q} \right\}\)
B2. Tόmendegi kόplik\(M_{2 \times 2}\left( \mathbb{R} \right)\) niń úles maydanı ekenliginkórsetiń. \(A = \left\{ \left. \ \begin{pmatrix}
a & b\sqrt{2} \\
b\sqrt{2} & a
\end{pmatrix} \right|a,b\mathbb{\in Q},a^{2} - 2b^{2} \neq 0 \right\}\)
B3. Tómendegi maydanlardıń berilgen kópa\(g'\)zalı arqalı ajıralıw maydanın tabıń. \(\mathbb{Q}\) da \(x^{4} + 1\).
C1. Tόmendegi kόplikti maydan shártlerine tekseriń. \(\mathbb{Z}\left( \sqrt{3} \right) = \left\{ a + b\sqrt{3}:a,b \in \mathbb{Z} \right\}\)
C2. Tómendegi kolconıń úles kóplikleri ideal bolıwın kórsetiń:
\(R = \mathbb{Z}_{28}\), \(I = \left\{ \overline{0},\overline{7},\overline{14},\overline{21} \right\}\)
C3. Tόmendegi sáwlelendiriwdi gomomorfizm shártlerine tekseriń. \(f(x) = 5^{x}\)
T1.Shekli maydannıń strukturası (anıqlaması, qásiyetleri, mısallar)
T2.Maydanlar avtomorfizmleri (anıqlaması, qásiyetleri, mısallar)
A1. \(\mathbf{Q}\)da \(\sqrt{2 + \sqrt{2}}\)tiń minimal kόpaǵzalısın tabıń.
A2. Ámellerdi orınlań: \(\mathbb{Z}_{12}\) de \(\left( 5x^{2} + 3x - 2 \right)^{2}\)
A3. Tómendegi sannıń \(p\)-adikalıq normasın tabıń. \(|124|_{2} =\)
B1. Tόmendegi kόplik kolco dúzedi ma? \(G = \left\{ a^{n},a \neq 0, \pm 1,n \in \mathbb{Z} \right\}\)
B2. Tόmendegi kόplik\(M_{2 \times 2}\left( \mathbb{R} \right)\)kolcosınıń úles kolcosı ekenliginkórsetiń. \(A = \left\{ \left. \ \begin{pmatrix}
a & 0 \\
0 & 0
\end{pmatrix} \right|a \in Z \right\}\)
B3. Tómendegi maydanlardıń berilgen kópa\(g'\)zalı arqalı ajıralıw maydanın tabıń. \(\mathbb{Q}\) da \(x^{3} - 3\).
C1. Tόmendegi kόplikti maydan shártlerine tekseriń. \(R = M_{2 \times 2}\left( \mathbf{Z} \right)\)
C2. \(\mathbb{Z}_{24}\)tiń barlıq idealların tabıń
C3. Tόmendegi sáwlelendiriwdi gomomorfizm shártlerine tekseriń. \(f(a) = a^{n}\)
T1.Maydanlar avtomorfizmleri (anıqlaması, qásiyetleri, mısallar)
T2. Fundamental teoremalar (anıqlaması, qásiyetleri, mısallar)
A1. \(\mathbf{Q}\)da \(\sqrt{2} + \sqrt{3}\)tiń minimal kόpaǵzalısın tabıń.
A2. Ámellerdi orınlań: \(\mathbb{Z}_{5}\) te \(\left( x^{2} + 3x - 4 \right)\left( x^{2} - 3 \right)\)
A3. Tómendegi sannıń \(p\)-adikalıq normasın tabıń. \(|48|_{3} =\)
B1. Tόmendegi kόplik kolco dúzedi ma? \(R = \left\{ \begin{pmatrix}
a & b \\
2b & a
\end{pmatrix}\ \ :\ \ a,b \in \mathbf{Q} \right\}\)
B2. Tόmendegi kόplik\(M_{2 \times 2}\left( \mathbb{R} \right)\)kolcosınıń úles kolcosı ekenliginkórsetiń. \(A = \left\{ \left. \ \begin{pmatrix}
a & b \\
0 & a
\end{pmatrix} \right|a,b\mathbb{\in Q} \right\}\)
B3. Tómendegi kópa\(g'\)zalı \(\mathbb{Z}_{3}\) da keltirilmeytu\(g'\)ın kópa\(g'\)zalı boladıma? \(x^{3} + 2x + 2\)
C1. Tόmendegi kόplikti maydan shártlerine tekseriń. \(G = \left\{ a^{n},a \neq 0, \pm 1,n \in \mathbb{Z} \right\}\)
C2. Tómendegi kolconıń úles kóplikleri ideal bolıwın kórsetiń:
\(R = \left\{ \begin{pmatrix}
a & b \\
0 & c
\end{pmatrix}\ |\ a,b,c \in \mathbb{Z} \right\}\), \(I = \left\{ \begin{pmatrix}
0 & a \\
0 & 0
\end{pmatrix}\ |\ a \in \mathbb{Z} \right\}\)
C3. Tόmendegi sáwlelendiriwdi gomomorfizm shártlerine tekseriń. \(f(x) = \sqrt{x}\)
T1.Fundamental teoremalar (anıqlaması, qásiyetleri, mısallar)
T2.Maksimal hám ápiwayı ideallar (anıqlaması, qásiyetleri, mısallar)
A1. \(\mathbf{Q}\)da \(\sqrt{2} + \sqrt{5}\)tiń minimal kόpaǵzalısın tabıń.
A2. Ámellerdi orınlań: \(\mathbb{Z}_{5}\) te \(\left( 3x^{2} + 3x - 4 \right)\left( 4x^{2} + 2 \right)\)
A3. Tómendegi sannıń \(p\)-adikalıq normasın tabıń. \(|729|_{3} =\)
B1. Tόmendegi kόplik kolco dúzedi ma? \(R = M_{2 \times 2}\left( \mathbf{Q} \right)\)
B2. \(Z_{20}\) kolconıń barlıq úles kolcoların anıqlań.
B3. Tómendegi kópa\(g'\)zalı \(\mathbb{Q\lbrack}x\rbrack\) da keltirilmeytu\(g'\)ın kópa\(g'\)zalı boladıma? \(x^{4} - 5x^{3} + 3x - 2\)
C1. Tόmendegi kόplikti maydan shártlerine tekseriń. \(\left\{ a + b\sqrt{7}|a,b \in R \right\}\)
C2. Tómendegi kolconıń úles kóplikleri ideal bolıwın kórsetiń:
\(R = \left\{ \begin{pmatrix}
a & b \\
0 & c
\end{pmatrix}\ |\ a,b,c \in \mathbb{Z} \right\}\), \(I = \left\{ \begin{pmatrix}
0 & b \\
0 & c
\end{pmatrix}\ |\ a \in \mathbb{Z} \right\}\).
C3. Tόmendegi sáwlelendiriwdi gomomorfizm shártlerine tekseriń.
\[f\left( a - \sqrt{2}b \right) = a + \sqrt{2}b\]
T1.Teńlemelerdiń radikallarda sheshiliwi (anıqlaması, qásiyetleri, mısallar)
T2.Keltirilmeytuǵın kópaǵzalılar (anıqlaması, qásiyetleri, mısallar)
A1. \(\mathbf{Q}\)da \(\sqrt{2 + \sqrt{5}}\)tiń minimal kόpaǵzalısın tabıń.
A2. Ámellerdi orınlań: \(\mathbb{Z}_{5}\) te \(\left( 3x^{2} + 2x - 4 \right) + \left( 4x^{2} + 2 \right)\)
A3. Tómendegi sannıń \(p\)-adikalıq normasın tabıń. \(|625|_{5} =\)
B1. Tόmendegi kόplik kolco dúzedi ma? \(R = M_{2 \times 2}\left( \mathbf{Z} \right)\)
B2. Tόmendegi kόplik\(M_{2 \times 2}\left( \mathbb{R} \right)\)kolcosınıń úles kolcosı ekenliginkórsetiń. \(A = \left\{ \left. \ \begin{pmatrix}
a & b \\
7b & a
\end{pmatrix} \right|a,b \in Z \right\}\)
B3. Tómendegi kópa\(g'\)zalı \(\mathbb{Q\lbrack}x\rbrack\) da keltirilmeytu\(g'\)ın kópa\(g'\)zalı boladıma? \(3x^{5} - 4x^{3} - 6x^{2} + 6\)
C1. Tόmendegi kόplikti maydan shártlerine tekseriń. \(\mathbb{Q}\left\lfloor i \right\rfloor = \left\{ a + bi\ \ :\ \ a,b\mathbb{\in Q} \right\}\)
C2. Tómendegi kolconıń úles kóplikleri ideal bolıwın kórsetiń:
\(R\mathbb{= Z\lbrack}\sqrt{7}\rbrack\), \(I = \{ a + b\sqrt{7}\ |\ \ a,b \in \mathbb{Z,}a - b\ \ jupsan\}\).
C3. Tόmendegi sáwlelendiriwdi gomomorfizm shártlerine tekseriń. \(f(x) = e^{x}\)
T1.p-Adikalıq norma, p-Adikalıq norma (anıqlaması, qásiyetleri, mısallar)
T2.Kolcolar (anıqlaması, qásiyetleri, mısallar)
A1. \(\mathbf{Q}\)da \(\sqrt{2 + 2\sqrt{2}}\)tiń minimal kόpaǵzalısın tabıń.
A2. Ámellerdi orınlań: \(\mathbb{Z}_{9}\) da \(\left( 7x^{3} + 3x^{2} - x \right) + \left( 6x^{2} - 8x + 4 \right)\)
A3. Tómendegi sannıń \(p\)-adikalıq normasın tabıń. \(|35|_{7} =\)
B1. Tόmendegi kόplik kolco dúzedi ma? \(\mathbb{Q}\left\lfloor i \right\rfloor = \left\{ a + bi\ \ :\ \ a,b\mathbb{\in Q} \right\}\)
B2. Tόmendegi kόplik\(M_{2 \times 2}\left( \mathbb{R} \right)\)kolcosınıń úles kolcosı ekenliginkórsetiń. \(A = \left\{ \left. \ \begin{pmatrix}
a & b \\
0 & a
\end{pmatrix} \right|a,b\mathbb{\in R} \right\}\)
B3. \(x^{2} - 3\)kόpaǵzalı \(\mathbb{Q}(\sqrt{2})\)de keltirilmeytuǵınkόpaǵzalı ekenliginkórsetiń.
C1. Tόmendegi kόplikti maydan shártlerine tekseriń. \(7\mathbb{Z}\)
C2. Tómendegi kolconıń úles kóplikleri ideal bolıwın kórsetiń:
\(R = \mathbb{Z}_{28}\), \(I = \{\overline{0},\overline{7},\overline{14},\overline{21}\}\).
C3. Tόmendegi sáwlelendiriwdi gomomorfizm shártlerine tekseriń. \(f:\begin{pmatrix}
a & b \\
0 & c
\end{pmatrix} \rightarrow a\)
T1.p-Adikalıq sanlardıń keńisligi (anıqlaması, qásiyetleri, mısallar)
T2.Kópagzalılar kolcosı (anıqlaması, qásiyetleri, mısallar)
A1. \(\mathbf{Q}\)da \(\sqrt{6 + 3\sqrt{2}}\)tiń minimal kόpaǵzalısın tabıń.
A2. Ámellerdi orınlań: \(\mathbb{Z}_{12}\) de \(\left( 5x^{2} + 3x - 4 \right)\left( 4x^{2} - x + 9 \right)\)
A3. Tómendegi sannıń \(p\)-adikalıq normasın tabıń. \(|256|_{2} =\)
B1. Tόmendegi kόplik kolco dúzedi ma? \(\mathbb{Z}\left\lbrack \sqrt{n} \right\rbrack = \left\{ x + y\sqrt{n}\ \ \left| \right.\ x,y \in \mathbb{Z} \right\}\)
B2. Tόmendegi kόplik\(M_{2 \times 2}\left( \mathbb{R} \right)\)kolcosınıń úles kolcosı ekenliginkórsetiń.
\[A = \left\{ \left. \ \begin{pmatrix}
a & b \\
 - b & a
\end{pmatrix} \right|a,b\mathbb{\in R} \right\}\]
B3. Tόmendegi kόpaǵzalı\(\mathbf{Z}_{2}\lbrack x\rbrack\)da keltirilmeytuǵınkόpaǵzalıboladıma? \(x^{3} + x + 1\)
C1. Tόmendegi kόplikti maydan shártlerine tekseriń. \(G = \left\{ a^{n},a \neq 0, \pm 1,n \in \mathbb{Z} \right\}\)
C2. Tómendegi kolconıń úles kóplikleri ideal bolıwın kórsetiń:
\(R = \mathbb{Z}_{24}\), \(I = \{\overline{0},\overline{8},\overline{16}\}\).
C3. Tόmendegi sáwlelendiriwdi gomomorfizm shártlerine tekseriń.
\[f:x \rightarrow x^{p}\]
