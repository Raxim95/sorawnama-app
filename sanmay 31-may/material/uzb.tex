T1. Halqalar.
T2. Integral sohalar va maydonlar.
A1.Quydagi halqaning qism to'plamlari ideal bo'lishini ko'rsating.
\(R = \mathbb{Z}_{24}\), \(I = \{\overline{0},\overline{8},\overline{16}\}\).
A2.Quydagini hisoblang:
\(\mathbb{Z}_{12}\) de \(\left( 5x^{2} + 3x - 4 \right) + \left( 4x^{2} - x + 9 \right)\).
A3. \(\mathbb{Q}\) Ratsianal sonlar maydoni ustida minimal ko'phadni toping.
\[\sqrt{2} + \sqrt{3}\]
B1. Quydagi halqa bo'ladimi:
\[\mathbb{Q}\left( \sqrt{2},\sqrt{3} \right) = \left\{ a + b\sqrt{2} + c\sqrt{3} + d\sqrt{6}:a,b,c,d \in \mathbb{Q} \right\}\]
B2. \(R\) halqani \(R'\) halqaga o'tkazuvchi gomomorfizmini aniqlang.
\(R = (\mathbb{Z}_{4}, +_{4}, \cdot_{4})\) hám \(R' = (\mathbb{Z}_{6}, +_{6}, \cdot_{6})\).
B3. . Quydagi maydonning berilgan ko'phadlar orqali ajralish maydonini toping.
\(\mathbb{Q}\) da \(x^{4} - 10x^{2} + 21\).
C1. Quydagi maydon kengaytmasining har birining bazisini toping. Har bir kengaytmaning darajasi qanday?
\(\mathbb{Q}\) da \(\mathbb{Q}\left( \sqrt{3},\sqrt{6} \right)\).
C2. Quydagi halqaning barcha ideallarini toping. Bul ideallardan qaysi-biri maksimal bo'ladi?
\[\mathbb{Z}_{18}\]
T1. Halqalarning gomomorfizmi va ideallar.
T2. Maksimal va sodda ideallar.
A1.Quydagi halqaning qism to'plamlari ideal bo'lishini ko'rsating.
\(R = \mathbb{Z}_{28}\), \(I = \{\overline{0},\overline{7},\overline{14},\overline{21}\}\).
A2.Quydagini hisoblang:
\(\mathbb{Z}_{12}\) de \(\left( 5x^{2} + 3x - 4 \right)\left( 4x^{2} - x + 9 \right)\).
A3. \(\mathbb{Q}\) Ratsianal sonlar maydoni ustida minimal ko'phadni toping.
\(\sqrt{3} + \sqrt{5}\).
B1. Quydagi halqa bo'ladimi:
\(7\mathbb{Z}\).
B2. \(R\) halqani \(R'\) halqaga o'tkazuvchi gomomorfizmini aniqlang.
\(R = (\mathbb{Z}_{6}, +_{6}, \cdot_{6})\) hám \(R' = (\mathbb{Z}_{10}, +_{10}, \cdot_{10})\).
B3. . Quydagi maydonning berilgan ko'phadlar orqali ajralish maydonini toping.
\(\mathbb{Q}\) da \(x^{4} + 1\).
C1. Quydagi maydon kengaytmasining har birining bazisini toping. Har bir kengaytmaning darajasi qanday?
\(\mathbb{Q}\) da \(\mathbb{Q}\left( \sqrt{2},i \right)\)
C2. Quydagi halqaning barcha ideallarini toping. Bul ideallardan qaysi-biri maksimal bo'ladi?
\[\mathbb{Z}_{25}\]
T1. Keltirilmaydigan ko'phadlar.
T2. Maydonlarning kengaytmasi. Algebrik element. Algebraik yopilma.
A1. Quydagi ko'phadlarning barcha nollarini toping:
\(\mathbb{Z}_{12}\) de \(5x^{3} + 4x^{2} - x + 9\);
A2.Quydagini hisoblang:
\(\mathbb{Z}_{9}\) da \(\left( 7x^{3} + 3x^{2} - x \right) + \left( 6x^{2} - 8x + 4 \right)\)
A3. \(\mathbb{Z}_{4}\lbrack x\rbrack\) da \(\deg\ p(x) > 1\) bo'ladigan \(p(x)\) birligini toping. Quydagi ko'phadlarning qaysilari \(\mathbb{Q\lbrack}x\rbrack\) da keltirilmaydigan?
\[x^{4} - 2x^{3} + 2x^{2} + x + 4\]
B1. Quydagi halqa bo'ladimi:
\(\mathbb{Q}\left( \sqrt{2} \right) = \left\{ a + b\sqrt{2}:a,b \in \mathbb{Q} \right\}\).
B2. \(R\) halqani \(R'\) halqaga o'tkazuvchi gomomorfizmini aniqlang.
\(R\mathbb{= (Z,} + , \cdot )\) va \(R' = (\mathbb{Z}_{6}, +_{6}, \cdot_{6})\).
B3. . Quydagi maydonning berilgan ko'phadlar orqali ajralish maydonini toping.
\(\mathbb{Q}\) da \(x^{4} - 5x^{2} + 21\).
C1. Quydagi maydon kengaytmasining har birining bazisini toping. Har bir kengaytmaning darajasi qanday?
\(\mathbb{Q}\left( \sqrt{3} + \sqrt{5} \right)\) da \(\mathbb{Q}\left( \sqrt{2},\sqrt{6} + \sqrt{10} \right)\)
C2. Quydagi halqaning barcha ideallarini toping. Bul ideallardan qaysi-biri maksimal bo'ladi?
\(\mathbb{M}_{2}\left( \mathbb{Z} \right)\), elementleri \(\mathbb{Z}\) bol\(g'\)an \(2 \times 2\) matrica
T1 Ko'phadlarning halqasi.
T2. Bo'lish algoritmi.
A1.Quydagi halqaning qism to'plamlari ideal bo'lishini ko'rsating.
\(R\mathbb{= Z\lbrack}\sqrt{7}\rbrack\), \(I = \{ a + b\sqrt{7}\ |\ \ a,b \in \mathbb{Z,}a - b\ \ juft\ son\}\).
A2.Quydagini hisoblang:
\(\mathbb{Z}_{9}\) da \(\left( 7x^{3} + 3x^{2} - x \right) + \left( 6x^{2} - 8x + 4 \right)\)
A3. \(\mathbb{Q}\) Ratsianal sonlar maydoni ustida minimal ko'phadni toping.
\[\sqrt{5} + \sqrt{7}\]
B1. Quydagi halqa bo'ladimi:
\(\mathbb{Z}_{18}\).
B2. \(R\) halqani \(R'\) halqaga o'tkazuvchi gomomorfizmini aniqlang.
\(R\mathbb{= (Z,} + , \cdot )\) va \(R' = (\mathbb{Z}_{6}, +_{6}, \cdot_{6})\).
B3. . Quydagi maydonning berilgan ko'phadlar orqali ajralish maydonini toping.
\(\mathbb{Q}\) da \(x^{3} - 3\).
C1. Quydagi maydon kengaytmasining har birining bazisini toping. Har bir kengaytmaning darajasi qanday?
\(\mathbb{Q}\) da \(\mathbb{Q}\left( \sqrt{3},\sqrt{5},\sqrt{7} \right)\)
C2. Quydagi halqaning barcha ideallarini toping. Bul ideallardan qaysi-biri maksimal bo'ladi?
\(\mathbb{M}_{2}\left( \mathbb{R} \right)\), elementleri \(\mathbb{R}\) bol\(g'\)an \(2 \times 2\) matrica
T1. Keltirilmaydigan ko'phadlar.
T2. Maydonlarning kengaytmasi. Algebrik element. Algebraik yopilma.
A1. Quydagi ko'phadlarning barcha nollarini toping:
\(\mathbb{Z}_{5}\) da \(3x^{3} - 4x^{2} - x + 4\);
A2.Quydagini hisoblang:
\(\mathbb{Z}_{5}\) te \(\left( 3x^{2} + 2x - 4 \right) + \left( 4x^{2} + 2 \right)\)
A3. \(\mathbb{Z}_{4}\lbrack x\rbrack\) da \(\deg\ p(x) > 1\) bo'ladigan \(p(x)\) birligini toping. Quydagi ko'phadlarning qaysilari \(\mathbb{Q\lbrack}x\rbrack\) da keltirilmaydigan?
\[x^{4} - 5x^{3} + 3x - 2\]
B1. Quydagi halqa bo'ladimi:
\[\mathbb{Z}\left( \sqrt{3} \right) = \left\{ a + b\sqrt{3}:a,b \in \mathbb{Z} \right\}\]
B2. . Quydagilar halqa bo'ladimi bunda \(i^{2} = - 1\):
\(\mathbb{Q(}i) = \{ x + iy\ |\ x,y \in \mathbb{Q\}}\).
B3. . Quydagi maydonning berilgan ko'phadlar orqali ajralish maydonini toping.
\(\mathbb{Z}_{3}\) te \(x^{3} + 2x + 2\).
C1. Quydagi maydon kengaytmasining har birining bazisini toping. Har bir kengaytmaning darajasi qanday?
\(\mathbb{Q}\left( \sqrt{3} + \sqrt{5} \right)\) da \(\mathbb{Q}\left( \sqrt{2},\sqrt{6} + \sqrt{10} \right)\)
C2. Quydagi halqaning barcha ideallarini toping. Bul ideallardan qaysi-biri maksimal bo'ladi?
\(\mathbb{M}_{2}\left( \mathbb{Z} \right)\), elementleri \(\mathbb{Z}\) bol\(g'\)an \(2 \times 2\) matrica
T1. Maydonlarning ajralishi.
T2. Maydonlarning kengaytmasi. Algebrik element. Algebraik yopilma.
A1. Quydagi ko'phadlarning barcha nollarini toping:
\(\mathbb{Z}_{7}\) de \(5x^{4} + 2x^{2} - 3\);
A2. \(\mathbb{Q}\) Ratsianal sonlar maydoni ustida minimal ko'phadalrini toping.
\(\sqrt{2} + \sqrt{5}\).
A3. \(\mathbb{Z}_{4}\lbrack x\rbrack\) da \(\deg\ p(x) > 1\) bo'ladigan \(p(x)\) birligini toping. Quydagi ko'phadlarning qaysilari \(\mathbb{Q\lbrack}x\rbrack\) da keltirilmaydigan?
\[3x^{5} - 4x^{3} - 6x^{2} + 6\]
B1. . Quydagilarning qaysi biri maydon bo'ladi:
\[5\mathbb{Z}\]
B2. . Quydagilar halqa bo'ladimi bunda \(i^{2} = - 1\):
\(\mathbb{Z(}i\sqrt{n}) = \{ x + iy\sqrt{n}\ |\ x,y \in \mathbb{Z\}}\).
B3.Quydagi matritsalar to'plamining qaysi biri matritsalarni qo'shish va ko'paytirish amallarga qarata halqa bo'ladi.
\(M_{2 \times 2}\mathbb{(R) =}\left\{ \begin{pmatrix}
a & b \\
b & d
\end{pmatrix}\ :\ a,b,c,d \in \mathbb{R} \right\}\).
C1. Quydagi maydon kengaytmasining har birining bazisini toping. Har bir kengaytmaning darajasi qanday?
\(\mathbb{Q}\left( \sqrt{3} + \sqrt{5} \right)\) da \(\mathbb{Q}\left( \sqrt{2},\sqrt{6} + \sqrt{10} \right)\)
C2. Quydagi halqaning barcha ideallarini toping. Bul ideallardan qaysi-biri maksimal bo'ladi?
\[\mathbb{Q}\]
T1. Geometrik konstruksiyasi.
T2. Chekli maydonlarning strukturasi
A1. \(\mathbb{Q}\) Ratsianal sonlar maydoni ustida minimal ko'phadalrini toping.
\(\sqrt{2 + \sqrt{5}}\).
A2.Quydagini hisoblang:
\(\mathbb{Z}_{12}\) de \(\left( 5x^{2} + 3x - 2 \right)^{2}\)
A3. \(\mathbb{Z}_{4}\lbrack x\rbrack\) da \(\deg\ p(x) > 1\) bo'ladigan \(p(x)\) birligini toping. Quydagi ko'phadlarning qaysilari \(\mathbb{Q\lbrack}x\rbrack\) da keltirilmaydigan?
\[x^{4} - 5x^{3} + 3x - 2\]
B1. Quydagi halqa bo'ladimi:
\(\mathbb{Q(}\sqrt[3]{2}) = \left\{ a + b\sqrt[3]{2}:a,b \in \mathbb{Q} \right\}\).
B2 Quydagilarning qaysi biri maydon bo'ladi, bunda \(i^{2} = - 1\):
\(\mathbb{Q\lbrack}i\rbrack = \{ x + iy\ |\ x,y \in \mathbb{Q\}}\)..
B3.Quydagi matritsalar to'plamining qaysi biri matritsalarni qo'shish va ko'paytirish amallarga qarata halqa bo'ladi .
\(M_{2 \times 2}\mathbb{(R) =}\left\{ \begin{pmatrix}
a & b \\
0 & c
\end{pmatrix}\ :\ a,b,c \in \mathbb{R} \right\}\).
C1. Quydagi maydon kengaytmasining har birining bazisini toping. Har bir kengaytmaning darajasi qanday?
\(\mathbb{Q}\left( \sqrt{3} + \sqrt{5} \right)\) da \(\mathbb{Q}\left( \sqrt{2},\sqrt{6} + \sqrt{10} \right)\)
C2. Quydagi halqaning barcha ideallarini toping. Bul ideallardan qaysi-biri maksimal bo'ladi?
\(\mathbb{M}_{2}\left( \mathbb{Z} \right)\), elementleri \(\mathbb{Z}\) bol\(g'\)an \(2 \times 2\) matrica
T1. Radikallarda yechilishi.
T2. Ratsional sonlar maydonini haqiqiy sonlar maydonigacha to'ldirish.
A1. \(\mathbb{Q}\) Ratsianal sonlar maydoni ustida minimal ko'phadalrini toping.
\(\sqrt{3 + \sqrt{3}}\).
A2.Quydagini hisoblang:
\(\mathbb{Z}_{5}\) te \(\left( 3x^{2} + 2x - 4 \right) + \left( 4x^{2} + 2 \right)\)
A3.Quydagi halqalarning qism to'plamlari ideal bo'lishini ko'rsating: \(R = \left\{ \begin{pmatrix}
a & b \\
0 & c \\
 & 
\end{pmatrix}\ |\ a,b,c \in \mathbb{Z} \right\}\), \(I = \left\{ \begin{pmatrix}
0 & b \\
0 & c
\end{pmatrix}\ |\ a \in \mathbb{Z} \right\}\).
B1. Quydagi halqa bo'ladimi:
\(\mathbb{Q}\left( \sqrt[3]{3} \right) = \left\{ a + b\sqrt[3]{3} + c\sqrt[3]{9}:a,b,c \in \mathbb{Q} \right\}\).
B2. . Quydagilar halqa bo'ladimi bunda \(i^{2} = - 1\):
\(\mathbb{Q(}i) = \{ x + iy\ |\ x,y \in \mathbb{Q\}}\).
B3.Quydagi matritsalar to'plamining qaysi biri matritsalarni qo'shish va ko'paytirish amallarga qarata halqa bo'ladi
\[M_{2 \times 2}\mathbb{(R) =}\left\{ \begin{pmatrix}
a & b \\
0 & c
\end{pmatrix}\ :\ a,b,c \in \mathbb{R} \right\}\]
C1. Quydagi maydon kengaytmasining har birining bazisini toping. Har bir kengaytmaning darajasi qanday?
\(\mathbb{Q}\left( \sqrt{2} \right)\) da \(\mathbb{Q}\left( \sqrt{8} \right)\)
C2. Quydagi halqaning barcha ideallarini toping. Bul ideallardan qaysi-biri maksimal bo'ladi?
\(\mathbb{M}_{2}\left( \mathbb{Z} \right)\), elementleri \(\mathbb{Z}\) bol\(g'\)an \(2 \times 2\) matrica
T1. p-adik sonlar maydoni va ular ustida amallar.
T2. Ko'phadlarning halqasi.
A1. \(\mathbb{Q}\) Ratsianal sonlar maydoni ustida minimal ko'phadalrini toping.
\(\sqrt{2 + \sqrt{2}}\).
A2.Quydagini hisoblang:
\(\mathbb{Z}_{12}\) da \(\left( 5x^{2} + 3x - 4 \right) + \left( 4x^{2} - x + 9 \right)\)
A3.Quydagi halqalarning qism to'plamlari ideal bo'lishini ko'rsating: \(R = \left\{ \begin{pmatrix}
a & b \\
0 & c
\end{pmatrix}\ |\ a,b,c \in \mathbb{Z} \right\}\), \(I = \left\{ \begin{pmatrix}
0 & a \\
0 & 0
\end{pmatrix}\ |\ a \in \mathbb{Z} \right\}\).
B1. Quydagi maydon bo'ladimi?
\(\mathbb{Q}\left( \sqrt{5},\sqrt{7} \right) = \left\{ a + b\sqrt{5} + c\sqrt{7} + d\sqrt{35}:a,b,c,d \in \mathbb{Q} \right\}\).
B2. Quydagilar halqa bo'ladimi bunda \(i^{2} = - 1\):
\(\mathbb{Q(}i) = \{ x + iy\ |\ x,y \in \mathbb{Q\}}\).
B3.Quydagi matritsalar to'plamining qaysi biri matritsalarni qo'shish va ko'paytirish amallarga qarata halqa bo'ladi
\(M_{2 \times 2}\mathbb{(R) =}\left\{ \begin{pmatrix}
a & 0 \\
b & c
\end{pmatrix}\ :\ a,b,c \in \mathbb{R} \right\}\).
C1. Quydagi maydon kengaytmasining har birining bazisini toping. Har bir kengaytmaning darajasi qanday?
\(\mathbb{Q}\) da \(\mathbb{Q}\left( i,\sqrt{2} + i,\sqrt{3} + i \right)\)
C2. Quydagi halqaning barcha ideallarini toping. Bul ideallardan qaysi-biri maksimal bo'ladi?
\(\mathbb{M}_{2}\left( \mathbb{Z} \right)\), elementleri \(\mathbb{Z}\) bol\(g'\)an \(2 \times 2\) matrica
T1. p-adik kvadrat tenglamalar.
T2. Maydonlarning avtomorfizmlari
A1. \(\mathbb{Q}\) Ratsianal sonlar maydoni ustida minimal ko'phadalrini toping.
\(\sqrt{2 + \sqrt{3}}\).
A2.Quydagini hisoblang:
\(\mathbb{Z}_{9}\) da \(\left( 7x^{3} + 3x^{2} - x \right) + \left( 6x^{2} - 8x + 4 \right)\)
A3. \(\mathbb{Z}_{4}\lbrack x\rbrack\) da \(\deg\ p(x) > 1\) bo'ladigan \(p(x)\) birligini toping. Quydagi ko'phadlarning qaysilari \(\mathbb{Q\lbrack}x\rbrack\) da keltirilmaydigan?
\[x^{4} - 2x^{3} + 2x^{2} + x + 4\]
B1. Quydagi maydon bo'ladimi:
\(\mathbb{Z}\left( \sqrt{5} \right) = \left\{ a + b\sqrt{5}:a,b \in \mathbb{Z} \right\}\).
B2. \(R\) halqani \(R'\) halqaga o'tkazuvchi gomomorfizmini aniqlang.
\(R\mathbb{= (R,} + , \cdot )\) hám \(R'\mathbb{= (R,} + , \cdot )\).
B3. . Quydagi maydonning berilgan ko'phadlar orqali ajralish maydonini toping.
\(\mathbb{Z}_{3}\) te \(x^{2} + x + 1\).
C1. Quydagi maydon kengaytmasining har birining bazisini toping. Har bir kengaytmaning darajasi qanday?
\(\mathbb{Q}\left( \sqrt{5} \right)\) da \(\mathbb{Q}\left( \sqrt{2} + \sqrt{5} \right)\)
C2. Quydagi halqaning barcha ideallarini toping. Bul ideallardan qaysi-biri maksimal bo'ladi?
\(\mathbb{M}_{2}\left( \mathbb{Z} \right)\), elementleri \(\mathbb{Z}\) bol\(g'\)an \(2 \times 2\) matrica
T1. Maydonlarning avtomorfizmlari.
T2. Maydonlarning kengaytmasi. Algebrik element. Algebraik yopilma.
A1.\(\mathbb{Q}\) Ratsianal sonlar maydoni ustida minimal ko'phadalrini toping.
\(\sqrt{3} + \sqrt{7}\).
A2.Quydagini hisoblang:
\(\mathbb{Z}_{5}\) te \(\left( 3x^{2} + 3x - 4 \right)\left( 4x^{2} + 2 \right)\)
A3. \(\mathbb{Z}_{4}\lbrack x\rbrack\) da \(\deg\ p(x) > 1\) bo'ladigan \(p(x)\) birligini toping. Quydagi ko'phadlarning qaysilari \(\mathbb{Q\lbrack}x\rbrack\) da keltirilmaydigan?
\[5x^{5} - 6x^{4} - 3x^{2} + 9x - 15\]
B1.Quydagilarning qaysi biri maydon bo'ladi:
\(\mathbb{Q}\left( \sqrt{11} \right) = \left\{ a + b\sqrt{11}:a,b \in \mathbb{Q} \right\}\).
B2. Quydagilarning qaysi biri maydon bo'ladi, bunda \(i^{2} = - 1\):
\(\mathbb{Z\lbrack}i\rbrack = \{ x + iy\ |\ x,y \in \mathbb{Z\}}\).
B3. Quydagi maydonning berilgan ko'phadlar orqali ajralish maydonini toping.
\(\mathbb{Q}\) da \(x^{4} - 2\).
C1. Quydagi maydon kengaytmasining har birining bazisini toping. Har bir kengaytmaning darajasi qanday?
\(\mathbb{Q}\left( \sqrt{3} + \sqrt{5} \right)\) da \(\mathbb{Q}\left( \sqrt{2},\sqrt{6} + \sqrt{10} \right)\)
C2. Quydagi halqaning barcha ideallarini toping. Bul ideallardan qaysi-biri maksimal bo'ladi?
\(\mathbb{M}_{2}\left( \mathbb{Z} \right)\), elementleri \(\mathbb{Z}\) bol\(g'\)an \(2 \times 2\) matrica
T1. Fundamental teoremalar.
T2. Radikallarda yechilishi.
A1. Quydagi ko'phadlarning barcha nollarini toping:
\(\mathbb{Z}_{2}\) de \(x^{3} + x + 1\).
A2. \(\mathbb{Q}\) Ratsianal sonlar maydoni ustida minimal ko'phadalrini toping.
\[\sqrt{2} + \sqrt{5}\]
A3. \(\mathbb{Z}_{4}\lbrack x\rbrack\) da \(\deg\ p(x) > 1\) bo'ladigan \(p(x)\) birligini toping. Quydagi ko'phadlarning qaysilari \(\mathbb{Q\lbrack}x\rbrack\) da keltirilmaydigan?
\[5x^{5} - 6x^{4} - 3x^{2} + 9x - 15\]
B1. . Quydagilarning qaysi biri maydon bo'ladi:
\(\mathbb{Q}\left( \sqrt{11} \right) = \left\{ a + b\sqrt{11}:a,b \in \mathbb{Q} \right\}\).
B2. . Quydagilar halqa bo'ladimi bunda \(i^{2} = - 1\):
\(\mathbb{Q(}i\sqrt{n}) = \{ x + iy\sqrt{n}\ |\ x,y \in \mathbb{Q\}}\).
B3.Quydagi matritsalar to'plamining qaysi biri matritsalarni qo'shish va ko'paytirish amallarga qarata halqa bo'ladi.
\(M_{2 \times 2}\mathbb{(R) =}\left\{ \begin{pmatrix}
a & b \\
 - b & d
\end{pmatrix}\ :\ a,b,c,d \in \mathbb{R} \right\}\).
C1. Quydagi maydon kengaytmasining har birining bazisini toping. Har bir kengaytmaning darajasi qanday?
\(\mathbb{Q}\) da \(\mathbb{Q}\left( \sqrt{2},\sqrt[3]{2} \right)\)
C2. Quydagi halqaning barcha ideallarini toping. Bul ideallardan qaysi-biri maksimal bo'ladi?
\[\mathbb{Q}\]
T1. p-adik kvadrat tenglamalar.
T2. Maydonlarning avtomorfizmlari
A1. \(\mathbb{Q}\) Ratsianal sonlar maydoni ustida minimal ko'phadalrini toping.
\(\sqrt{2 + \sqrt{3}}\).
A2.Quydagini hisoblang:
\(\mathbb{Z}_{5}\) te \(\left( 3x^{2} + 3x - 4 \right)\left( 4x^{2} + 2 \right)\)
A3. \(\mathbb{Z}_{4}\lbrack x\rbrack\) da \(\deg\ p(x) > 1\) bo'ladigan \(p(x)\) birligini toping. Quydagi ko'phadlarning qaysilari \(\mathbb{Q\lbrack}x\rbrack\) da keltirilmaydigan?
\[3x^{5} - 4x^{3} - 6x^{2} + 6\]
B1. Quydagi maydon bo'ladimi:
\(\mathbb{Q(}\sqrt{3}) = \left\{ a + b\sqrt[3]{3}:a,b \in \mathbb{Q} \right\}\).
B2. \(R\) halqani \(R'\) halqaga o'tkazuvchi gomomorfizmini aniqlang.
\(R\mathbb{= (R,} + , \cdot )\) hám \(R'\mathbb{= (R,} + , \cdot )\).
B3. Quydagi matritsalar to'plamining qaysi biri matritsalarni qo'shish va ko'paytirish amallarga qarata halqa bo'ladi.
\(M_{2 \times 2}\mathbb{(R) =}\left\{ \begin{pmatrix}
a & 0 \\
b & c
\end{pmatrix}\ :\ a,b,c \in \mathbb{R} \right\}\).
C1. Quydagi maydon kengaytmasining har birining bazisini toping. Har bir kengaytmaning darajasi qanday?
\(\mathbb{Q}\left( \sqrt{3} + \sqrt{5} \right)\) da \(\mathbb{Q}\left( \sqrt{2},\sqrt{6} + \sqrt{10} \right)\)
C2. Quydagi halqaning barcha ideallarini toping. Bul ideallardan qaysi-biri maksimal bo'ladi?
\(\mathbb{M}_{2}\left( \mathbb{Z} \right)\), elementleri \(\mathbb{Z}\) bol\(g'\)an \(2 \times 2\) matrica
T1. p-adik kvadrat tenglamalar.
T2. p-adik sonlar maydoni va ular ustida amallar.
A1. \(\mathbb{Q}\) Ratsianal sonlar maydoni ustida minimal ko'phadalrini toping.
\(\sqrt{2 + \sqrt{3}}\).
A2.Quydagini hisoblang:
\(\mathbb{Z}_{5}\) te \(\left( 3x^{2} + 3x - 4 \right)\left( 4x^{2} + 2 \right)\)
A3. \(\mathbb{Z}_{4}\lbrack x\rbrack\) da \(\deg\ p(x) > 1\) bo'ladigan \(p(x)\) birligini toping. Quydagi ko'phadlarning qaysilari \(\mathbb{Q\lbrack}x\rbrack\) da keltirilmaydigan?
\[3x^{5} - 4x^{3} - 6x^{2} + 6\]
B1. Quydagi maydon bo'ladimi:
\[\mathbb{Q}\left( \sqrt[3]{5} \right) = \left\{ a + b\sqrt[3]{5}:a,b,c \in \mathbb{Q} \right\}\]
B2. Quydagi maydon bo'ladimi, bunda \(i^{2} = - 1\):
\[\mathbb{Z\lbrack}i\sqrt{n}\rbrack = \{ x + iy\sqrt{n}\ |\ x,y \in \mathbb{Z\}}\]
B3. Quydagi matritsalar to'plamining qaysi biri matritsalarni qo'shish va ko'paytirish amallarga qarata halqa bo'ladi.
\(M_{2 \times 2}\mathbb{(R) =}\left\{ \begin{pmatrix}
a & 0 \\
0 & 0
\end{pmatrix}\ :\ a \in \mathbb{R} \right\}\).
C1. Quydagi maydon kengaytmasining har birining bazisini toping. Har bir kengaytmaning darajasi qanday?
\(\mathbb{Q}\) da \(\mathbb{Q}\left( \sqrt{3},\sqrt{6} \right)\)
C2. Quydagi halqaning barcha ideallarini toping. Bul ideallardan qaysi-biri maksimal bo'ladi?
\[\mathbb{Z}_{18}\]
T1. Halqalar.
T2. Halqalarning gomomorfizmi va ideallar.
A1. \(\mathbb{Q}\) Ratsianal sonlar maydoni ustida minimal ko'phadalrini toping.
\(\sqrt{2 + \sqrt{7}}\).
A2.Quydagini hisoblang:
\(\mathbb{Z}_{12}\) de \(\left( 5x^{2} + 3x - 2 \right)^{2}\)
A3. Quydagi ko'phadlarning barcha nollarini toping:
\(\mathbb{Z}_{5}\) de \(3x^{3} - 4x^{2} - x + 4\);
B1. Quydagi maydon bo'ladimi:
\[\mathbb{Q}\left( \sqrt[3]{5} \right) = \left\{ a + b\sqrt[3]{5}:a,b,c \in \mathbb{Q} \right\}\]
B2. Quydagi maydon bo'ladimi, bunda \(i^{2} = - 1\):
\(\mathbb{Q\lbrack}i\sqrt{n}\rbrack = \{ x + iy\sqrt{n}\ |\ x,y \in \mathbb{Q\}}\).
B3. Quydagi matritsalar to'plamining qaysi biri matritsalarni qo'shish va ko'paytirish amallarga qarata halqa bo'ladi.
\(M_{2 \times 2}\mathbb{(R) =}\left\{ \begin{pmatrix}
a & 0 \\
0 & b
\end{pmatrix}\ :\ a,b \in \mathbb{R} \right\}\).
C1. Quydagi maydon kengaytmasining har birining bazisini toping. Har bir kengaytmaning darajasi qanday?
\(\mathbb{Q}\left( \sqrt{3} + \sqrt{5} \right)\) da \(\mathbb{Q}\left( \sqrt{2},\sqrt{6} + \sqrt{10} \right)\)
C2. Quydagi halqaning barcha ideallarini toping. Bul ideallardan qaysi-biri maksimal bo'ladi?
\(\mathbb{M}_{2}\left( \mathbb{Z} \right)\), elementleri \(\mathbb{Z}\) bol\(g'\)an \(2 \times 2\) matrica
T1. Maksimal va sodda ideallar.
T2. Halqalarning gomomorfizmi va ideallar.
A1. \(\mathbb{Q}\) Ratsianal sonlar maydoni ustida minimal ko'phadalrini toping.
\(\sqrt{2 - \sqrt{2}}\).
A2.Quydagini hisoblang:
\(\mathbb{Z}_{12}\) de \(\left( 5x^{2} + 3x - 2 \right)^{2}\)
A3. Quydagi ko'phadlarning barcha nollarini toping:
\(\mathbb{Z}_{5}\) de \(3x^{3} - 4x^{2} - x + 4\);
B1. Quydagi maydonlarning berilgan ko'phadlar orqali ajralish maydonini toping.
\(\mathbb{Q}\) da \(x^{4} - 5x^{2} + 6\).
B2. Quydagi maydon bo'ladimi, bunda \(i^{2} = - 1\):
\(\mathbb{Q\lbrack}i\sqrt{n}\rbrack = \{ x + iy\sqrt{n}\ |\ x,y \in \mathbb{Q\}}\).
B3. Quydagi matritsalar to'plamining qaysi biri matritsalarni qo'shish va ko'paytirish amallarga qarata halqa bo'ladi.
\[M_{2 \times 2}\mathbb{(R) =}\left\{ \begin{pmatrix}
a & 0 \\
0 & a
\end{pmatrix}\ :\ a \in \mathbb{R} \right\}\]
C1. Quydagi maydon kengaytmasining har birining bazisini toping. Har bir kengaytmaning darajasi qanday?
\(\mathbb{Q}\) da \(\mathbb{Q}\left( \sqrt{2},\sqrt[3]{2} \right)\)
C2. Quydagi halqaning barcha ideallarini toping. Bul ideallardan qaysi-biri maksimal bo'ladi?
\(\mathbb{M}_{2}\left( \mathbb{Z} \right)\), elementleri \(\mathbb{Z}\) bol\(g'\)an \(2 \times 2\) matrica
T1. Bo'lish algoritmi.
T2. Maydonlarning ajralishi.
A1. \(\mathbb{Q}\) Ratsianal sonlar maydoni ustida minimal ko'phadalrini toping.
\(\sqrt{3 - \sqrt{3}}\).
A2.Quydagini hisoblang:
\(\mathbb{Z}_{12}\) de \(\left( 5x^{2} + 3x - 4 \right)\left( 4x^{2} - x + 9 \right)\)
A3. Quydagi ko'phadlarning barcha nollarini toping:
\(\mathbb{Z}_{12}\) de \(5x^{3} + 4x^{2} - x + 9\);
B1. Quydagi maydonlarning berilgan ko'phadlar orqali ajralish maydonini toping.
\(\mathbb{Q}\) da \(x^{4} - 5x^{2} + 6\).
B2. Quydagi halqa bo'ladimi:
\(\mathbb{R\lbrack}\omega\rbrack = \{ x + \omega \cdot y\ |\ \omega^{2} = 1\}\).
B3. Quydagi matritsalar to'plamining qaysi biri matritsalarni qo'shish va ko'paytirish amallarga qarata halqa bo'ladi.
\[M_{2 \times 2}\mathbb{(R) =}\left\{ \begin{pmatrix}
a & 0 \\
0 & - a
\end{pmatrix}\ :\ a \in \mathbb{R} \right\}\]
C1. Quydagi maydon kengaytmasining har birining bazisini toping. Har bir kengaytmaning darajasi qanday?
\(\mathbb{Q}\left( \sqrt{5} \right)\) da \(\mathbb{Q}\left( \sqrt{2} + \sqrt{5} \right)\)
C2. Quydagi halqaning barcha ideallarini toping. Bul ideallardan qaysi-biri maksimal bo'ladi?
\[\mathbb{Z}_{25}\]
T1. Geometrik konstruksiyasi.
T2. Ratsional sonlar maydonini haqiqiy sonlar maydonigacha to'ldirish.
A1. \(\mathbb{Q}\) Ratsianal sonlar maydoni ustida minimal ko'phadalrini toping.
\(\sqrt{3 - \sqrt{3}}\).
A2.Quydagini hisoblang:
\(\mathbb{Z}_{5}\) te \(\left( 3x^{2} + 2x - 4 \right) + \left( 4x^{2} + 2 \right)\)
A3. Quydagi ko'phadlarning barcha nollarini toping:
\(\mathbb{Z}_{2}\) de \(x^{3} + x + 1\).
B1. \(R\) halqani \(R'\) halqaga o'tkazuvchi gomomorfizmini aniqlang.
\(R = (\mathbb{Z}_{4}, +_{4}, \cdot_{4})\) hám \(R' = (\mathbb{Z}_{6}, +_{6}, \cdot_{6})\).
B2. Quydagi halqa bo'ladimi:
\(\mathbb{R\lbrack}\sigma\rbrack = \{ x + \sigma \cdot y\ |\ \sigma^{2} = 0\}\).
B3. Quydagi maydonlarning berilgan ko'phadlar orqali ajralish maydonini toping.
\(\mathbb{Z}_{2}\) da \(x^{2} + 1\).
C1. Quydagi maydon kengaytmasining har birining bazisini toping. Har bir kengaytmaning darajasi qanday?
\(\mathbb{Q}\) da \(\mathbb{Q}\left( \sqrt{3},\sqrt{5},\sqrt{7} \right)\)
C2. Quydagi halqaning barcha ideallarini toping. Bul ideallardan qaysi-biri maksimal bo'ladi?
\[\mathbb{Z}_{18}\]
T1. Ko'phadlarning halqasi.
T2. Keltirilmaydigan ko'phadlar.
A1. \(\mathbb{Q}\) Ratsianal sonlar maydoni ustida minimal ko'phadalrini toping.
\(\sqrt{6 + 3\sqrt{2}}\).
A2.Quydagini hisoblang:
\(\mathbb{Z}_{5}\) te \(\left( 3x^{2} + 2x - 4 \right) + \left( 4x^{2} + 2 \right)\)
A3. Quydagi ko'phadlarning barcha nollarini toping:
\(\mathbb{Z}_{2}\) de \(x^{3} + x + 1\).
B1. \(R\) halqani \(R'\) halqaga o'tkazuvchi gomomorfizmini aniqlang.
\(R = (\mathbb{Z}_{6}, +_{6}, \cdot_{6})\) hám \(R' = (\mathbb{Z}_{10}, +_{10}, \cdot_{10})\).
B2. Quydagi halqa bo'ladimi:
\(\mathbb{R\lbrack}\sigma\rbrack = \{ x + \sigma \cdot y\ |\ \sigma^{2} = 0\}\).
B3. Quydagi maydonlarning berilgan ko'phadlar orqali ajralish maydonini toping.
\(\mathbb{Z}_{5}\) da \(x^{2} + x + 1\).
C1. Quydagi maydon kengaytmasining har birining bazisini toping. Har bir kengaytmaning darajasi qanday?
\(\mathbb{Q}\) da \(\mathbb{Q}\left( \sqrt{2},i \right)\)
C2. Quydagi halqaning barcha ideallarini toping. Bul ideallardan qaysi-biri maksimal bo'ladi?
\[\mathbb{Q}\]
T1. p-adik sonlar maydoni va ular ustida amallar.
T2. p-adik kvadrat tenglamalar.
A1. \(\mathbb{Q}\) Ratsianal sonlar maydoni ustida minimal ko'phadalrini toping.
\(\sqrt{2 + 2\sqrt{2}}\).
A2.Quydagini hisoblang:
\(\mathbb{Z}_{5}\) te \(\left( 3x^{2} + 2x - 4 \right) + \left( 4x^{2} + 2 \right)\)
A3. Quydagi ko'phadlarning barcha nollarini toping:
\(\mathbb{Z}_{2}\) de \(x^{3} + x + 1\).
B1. Quydagilar halqa bo'ladimi:
\[\mathbb{Z}_{18}\]
B2. Quydagi maydon bo'ladimi:
\(5\mathbb{Z}\).
B3. Quydagi maydonlarning berilgan ko'phadlar orqali ajralish maydonini toping.
\(\mathbb{Q}\) da \(x^{4} - 10x^{2} + 21\).
C1. Quydagi maydon kengaytmasining har birining bazisini toping. Har bir kengaytmaning darajasi qanday?
\(\mathbb{Q}\) da \(\mathbb{Q}\left( \sqrt{3},\sqrt{6} \right)\)
C2. Quydagi halqaning barcha ideallarini toping. Bul ideallardan qaysi-biri maksimal bo'ladi? $\mathbb{Q}$