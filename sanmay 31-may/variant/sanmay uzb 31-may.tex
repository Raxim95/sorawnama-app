\documentclass{article}
\usepackage[fontsize=12pt]{fontsize}
\usepackage[utf8]{inputenc}
\usepackage[T2A]{fontenc}
% \usepackage{unicode-math}

\usepackage{array}
\usepackage[a4paper,
left=7mm,
right=5mm,
top=7mm,]{geometry}
\usepackage{amsmath}
% \usepackage{amssymbol}
\usepackage{amsfonts}
\usepackage{setspace}
\onehalfspacing  % 1.5 line spacing



\renewcommand{\baselinestretch}{1} 

\everymath{\displaystyle}
\everydisplay{\displaystyle}
% \linespread{1.25}

\DeclareMathOperator{\sign}{sign}


\begin{document}

\pagenumbering{gobble}


\begin{tabular}{m{17cm}}
\textbf{1-variant}
\newline

\textbf{T1.} Maydonlarning avtomorfizmlari. \\
\textbf{T2.} Chekli maydonlarning strukturasi \\
\textbf{A1.} \(\mathbb{Q}\) Ratsianal sonlar maydoni ustida minimal ko'phadalrini toping.
\(\sqrt{2 + \sqrt{7}}\). \\
\textbf{A2.} Quydagini hisoblang:
\(\mathbb{Z}_{9}\) da \(\left( 7x^{3} + 3x^{2} - x \right) + \left( 6x^{2} - 8x + 4 \right)\) \\
\textbf{A3.} \(\mathbb{Z}_{4}\lbrack x\rbrack\) da \(\deg\ p(x) > 1\) bo'ladigan \(p(x)\) birligini toping. Quydagi ko'phadlarning qaysilari \(\mathbb{Q\lbrack}x\rbrack\) da keltirilmaydigan?
\[3x^{5} - 4x^{3} - 6x^{2} + 6\] \\
\textbf{B1.} Quydagi halqa bo'ladimi:
\(\mathbb{Q}\left( \sqrt[3]{3} \right) = \left\{ a + b\sqrt[3]{3} + c\sqrt[3]{9}:a,b,c \in \mathbb{Q} \right\}\). \\
\textbf{B2.} Quydagilar halqa bo'ladimi bunda \(i^{2} = - 1\):
\(\mathbb{Q(}i) = \{ x + iy\ |\ x,y \in \mathbb{Q\}}\). \\
\textbf{B3.} Quydagi matritsalar to'plamining qaysi biri matritsalarni qo'shish va ko'paytirish amallarga qarata halqa bo'ladi.
\(M_{2 \times 2}\mathbb{(R) =}\left\{ \begin{pmatrix}
a & b \\
 - b & d
\end{pmatrix}\ :\ a,b,c,d \in \mathbb{R} \right\}\). \\
\textbf{C1.} Quydagi maydon kengaytmasining har birining bazisini toping. Har bir kengaytmaning darajasi qanday?
\(\mathbb{Q}\) da \(\mathbb{Q}\left( \sqrt{2},i \right)\) \\
\textbf{C2.} Quydagi halqaning barcha ideallarini toping. Bul ideallardan qaysi-biri maksimal bo'ladi?
\[\mathbb{Q}\] \\
\textbf{C3.} Quyidagi akslantirishni gomomorfizm shartlariga tekshiring.
\[f:\begin{pmatrix}
a & b \\
 - b & a
\end{pmatrix} \rightarrow a + bi\] \\

\end{tabular}
\vspace{1cm}


\begin{tabular}{m{17cm}}
\textbf{2-variant}
\newline

\textbf{T1.} Radikallarda yechilishi. \\
\textbf{T2.} Halqalar. \\
\textbf{A1.} Quydagi halqaning qism to'plamlari ideal bo'lishini ko'rsating.
\(R = \mathbb{Z}_{28}\), \(I = \{\overline{0},\overline{7},\overline{14},\overline{21}\}\). \\
\textbf{A2.} Quydagini hisoblang:
\(\mathbb{Z}_{12}\) de \(\left( 5x^{2} + 3x - 4 \right)\left( 4x^{2} - x + 9 \right)\). \\
\textbf{A3.} \(\mathbb{Q}\) Ratsianal sonlar maydoni ustida minimal ko'phadni toping.
\[\sqrt{5} + \sqrt{7}\] \\
\textbf{B1.} Quydagi maydon bo'ladimi:
\[\mathbb{Q}\left( \sqrt[3]{5} \right) = \left\{ a + b\sqrt[3]{5}:a,b,c \in \mathbb{Q} \right\}\] \\
\textbf{B2.} Quydagi halqa bo'ladimi:
\(\mathbb{R\lbrack}\sigma\rbrack = \{ x + \sigma \cdot y\ |\ \sigma^{2} = 0\}\). \\
\textbf{B3.} Quydagi maydonlarning berilgan ko'phadlar orqali ajralish maydonini toping.
\(\mathbb{Z}_{5}\) da \(x^{2} + x + 1\). \\
\textbf{C1.} Quydagi maydon kengaytmasining har birining bazisini toping. Har bir kengaytmaning darajasi qanday?
\(\mathbb{Q}\left( \sqrt{3} + \sqrt{5} \right)\) da \(\mathbb{Q}\left( \sqrt{2},\sqrt{6} + \sqrt{10} \right)\) \\
\textbf{C2.} Quydagi halqaning barcha ideallarini toping. Bul ideallardan qaysi-biri maksimal bo'ladi?
\[\mathbb{Z}_{25}\] \\
\textbf{C3.} Quyidagi akslantirishni gomomorfizm shartlariga tekshiring. \(f:\begin{pmatrix}
a & b \\
0 & c
\end{pmatrix} \rightarrow a\) \\

\end{tabular}
\vspace{1cm}


\begin{tabular}{m{17cm}}
\textbf{3-variant}
\newline

\textbf{T1.} Bo'lish algoritmi. \\
\textbf{T2.} Maydonlarning kengaytmasi. Algebrik element. Algebraik yopilma. \\
\textbf{A1.} \(\mathbb{Q}\) Ratsianal sonlar maydoni ustida minimal ko'phadalrini toping.
\(\sqrt{3} + \sqrt{7}\). \\
\textbf{A2.} Quydagini hisoblang:
\(\mathbb{Z}_{5}\) te \(\left( 3x^{2} + 3x - 4 \right)\left( 4x^{2} + 2 \right)\) \\
\textbf{A3.} Quydagi ko'phadlarning barcha nollarini toping:
\(\mathbb{Z}_{2}\) de \(x^{3} + x + 1\). \\
\textbf{B1.} Quydagilarning qaysi biri maydon bo'ladi:
\(\mathbb{Q}\left( \sqrt{11} \right) = \left\{ a + b\sqrt{11}:a,b \in \mathbb{Q} \right\}\). \\
\textbf{B2.} \(R\) halqani \(R'\) halqaga o'tkazuvchi gomomorfizmini aniqlang.
\(R\mathbb{= (Z,} + , \cdot )\) va \(R' = (\mathbb{Z}_{6}, +_{6}, \cdot_{6})\). \\
\textbf{B3.} Quydagi matritsalar to'plamining qaysi biri matritsalarni qo'shish va ko'paytirish amallarga qarata halqa bo'ladi .
\(M_{2 \times 2}\mathbb{(R) =}\left\{ \begin{pmatrix}
a & b \\
0 & c
\end{pmatrix}\ :\ a,b,c \in \mathbb{R} \right\}\). \\
\textbf{C1.} Quydagi maydon kengaytmasining har birining bazisini toping. Har bir kengaytmaning darajasi qanday?
\(\mathbb{Q}\) da \(\mathbb{Q}\left( \sqrt{3},\sqrt{5},\sqrt{7} \right)\) \\
\textbf{C2.} Quydagi halqaning barcha ideallarini toping. Bul ideallardan qaysi-biri maksimal bo'ladi?
\[\mathbb{Q}\] \\
\textbf{C3.} Quyidagi akslantirishni gomomorfizm shartlariga tekshiring.
\[f:x \rightarrow x^{p}\] \\

\end{tabular}
\vspace{1cm}


\begin{tabular}{m{17cm}}
\textbf{4-variant}
\newline

\textbf{T1.} Keltirilmaydigan ko'phadlar. \\
\textbf{T2.} p-adik kvadrat tenglamalar. \\
\textbf{A1.} Quydagi ko'phadlarning barcha nollarini toping:
\(\mathbb{Z}_{2}\) de \(x^{3} + x + 1\). \\
\textbf{A2.} \(\mathbb{Q}\) Ratsianal sonlar maydoni ustida minimal ko'phadalrini toping.
\[\sqrt{2} + \sqrt{5}\] \\
\textbf{A3.} \(\mathbb{Z}_{4}\lbrack x\rbrack\) da \(\deg\ p(x) > 1\) bo'ladigan \(p(x)\) birligini toping. Quydagi ko'phadlarning qaysilari \(\mathbb{Q\lbrack}x\rbrack\) da keltirilmaydigan?
\[x^{4} - 2x^{3} + 2x^{2} + x + 4\] \\
\textbf{B1.} Quydagilarning qaysi biri maydon bo'ladi:
\[5\mathbb{Z}\] \\
\textbf{B2.} Quydagilar halqa bo'ladimi bunda \(i^{2} = - 1\):
\(\mathbb{Q(}i) = \{ x + iy\ |\ x,y \in \mathbb{Q\}}\). \\
\textbf{B3.} Quydagi maydonning berilgan ko'phadlar orqali ajralish maydonini toping.
\(\mathbb{Q}\) da \(x^{4} - 5x^{2} + 21\). \\
\textbf{C1.} Quydagi maydon kengaytmasining har birining bazisini toping. Har bir kengaytmaning darajasi qanday?
\(\mathbb{Q}\) da \(\mathbb{Q}\left( \sqrt{2},i \right)\) \\
\textbf{C2.} Quydagi halqaning barcha ideallarini toping. Bul ideallardan qaysi-biri maksimal bo'ladi?
\(\mathbb{M}_{2}\left( \mathbb{Z} \right)\), elementleri \(\mathbb{Z}\) bol\(g'\)an \(2 \times 2\) matrica \\
\textbf{C3.} Quyidagi akslantirishni gomomorfizm shartlariga tekshiring. \(f\left( \begin{pmatrix}
a & 0 \\
0 & a
\end{pmatrix} \right) = a\) \\

\end{tabular}
\vspace{1cm}


\begin{tabular}{m{17cm}}
\textbf{5-variant}
\newline

\textbf{T1.} Geometrik konstruksiyasi. \\
\textbf{T2.} Maksimal va sodda ideallar. \\
\textbf{A1.} Quydagi halqaning qism to'plamlari ideal bo'lishini ko'rsating.
\(R = \mathbb{Z}_{24}\), \(I = \{\overline{0},\overline{8},\overline{16}\}\). \\
\textbf{A2.} Quydagini hisoblang:
\(\mathbb{Z}_{12}\) da \(\left( 5x^{2} + 3x - 4 \right) + \left( 4x^{2} - x + 9 \right)\) \\
\textbf{A3.} \(\mathbb{Z}_{4}\lbrack x\rbrack\) da \(\deg\ p(x) > 1\) bo'ladigan \(p(x)\) birligini toping. Quydagi ko'phadlarning qaysilari \(\mathbb{Q\lbrack}x\rbrack\) da keltirilmaydigan?
\[5x^{5} - 6x^{4} - 3x^{2} + 9x - 15\] \\
\textbf{B1.} Quydagi halqa bo'ladimi:
\[\mathbb{Q}\left( \sqrt{2},\sqrt{3} \right) = \left\{ a + b\sqrt{2} + c\sqrt{3} + d\sqrt{6}:a,b,c,d \in \mathbb{Q} \right\}\] \\
\textbf{B2.} Quydagi halqa bo'ladimi:
\(\mathbb{R\lbrack}\omega\rbrack = \{ x + \omega \cdot y\ |\ \omega^{2} = 1\}\). \\
\textbf{B3.} Quydagi maydonning berilgan ko'phadlar orqali ajralish maydonini toping.
\(\mathbb{Q}\) da \(x^{3} - 3\). \\
\textbf{C1.} Quydagi maydon kengaytmasining har birining bazisini toping. Har bir kengaytmaning darajasi qanday?
\(\mathbb{Q}\) da \(\mathbb{Q}\left( \sqrt{3},\sqrt{6} \right)\) \\
\textbf{C2.} Quydagi halqaning barcha ideallarini toping. Bul ideallardan qaysi-biri maksimal bo'ladi? $\mathbb{Q}$ \\
\textbf{C3.} Quyidagi akslantirishni gomomorfizm shartlariga tekshiring. \(f(a) = a^{n}\) \\

\end{tabular}
\vspace{1cm}


\begin{tabular}{m{17cm}}
\textbf{6-variant}
\newline

\textbf{T1.} Halqalarning gomomorfizmi va ideallar. \\
\textbf{T2.} Fundamental teoremalar. \\
\textbf{A1.} Quydagi ko'phadlarning barcha nollarini toping:
\(\mathbb{Z}_{5}\) da \(3x^{3} - 4x^{2} - x + 4\); \\
\textbf{A2.} Quydagini hisoblang:
\(\mathbb{Z}_{5}\) te \(\left( 3x^{2} + 2x - 4 \right) + \left( 4x^{2} + 2 \right)\) \\
\textbf{A3.} \(\mathbb{Q}\) Ratsianal sonlar maydoni ustida minimal ko'phadni toping.
\[\sqrt{2} + \sqrt{3}\] \\
\textbf{B1.} Quydagi halqa bo'ladimi:
\(\mathbb{Z}_{18}\). \\
\textbf{B2.} \(R\) halqani \(R'\) halqaga o'tkazuvchi gomomorfizmini aniqlang.
\(R\mathbb{= (Z,} + , \cdot )\) va \(R' = (\mathbb{Z}_{6}, +_{6}, \cdot_{6})\). \\
\textbf{B3.} Quydagi matritsalar to'plamining qaysi biri matritsalarni qo'shish va ko'paytirish amallarga qarata halqa bo'ladi
\(M_{2 \times 2}\mathbb{(R) =}\left\{ \begin{pmatrix}
a & 0 \\
b & c
\end{pmatrix}\ :\ a,b,c \in \mathbb{R} \right\}\). \\
\textbf{C1.} Quydagi maydon kengaytmasining har birining bazisini toping. Har bir kengaytmaning darajasi qanday?
\(\mathbb{Q}\) da \(\mathbb{Q}\left( \sqrt{2},\sqrt[3]{2} \right)\) \\
\textbf{C2.} Quydagi halqaning barcha ideallarini toping. Bul ideallardan qaysi-biri maksimal bo'ladi?
\(\mathbb{M}_{2}\left( \mathbb{Z} \right)\), elementleri \(\mathbb{Z}\) bol\(g'\)an \(2 \times 2\) matrica \\
\textbf{C3.} Quyidagi akslantirishni gomomorfizm shartlariga tekshiring. \(f(x + iy) = x \cdot y\) \\

\end{tabular}
\vspace{1cm}


\begin{tabular}{m{17cm}}
\textbf{7-variant}
\newline

\textbf{T1.} Ko'phadlarning halqasi. \\
\textbf{T2.} Maydonlarning ajralishi. \\
\textbf{A1.} \(\mathbb{Q}\) Ratsianal sonlar maydoni ustida minimal ko'phadalrini toping.
\(\sqrt{2 + \sqrt{5}}\). \\
\textbf{A2.} Quydagini hisoblang:
\(\mathbb{Z}_{5}\) te \(\left( 3x^{2} + 2x - 4 \right) + \left( 4x^{2} + 2 \right)\) \\
\textbf{A3.} Quydagi ko'phadlarning barcha nollarini toping:
\(\mathbb{Z}_{12}\) de \(5x^{3} + 4x^{2} - x + 9\); \\
\textbf{B1.} Quydagi maydon bo'ladimi:
\(\mathbb{Q(}\sqrt{3}) = \left\{ a + b\sqrt[3]{3}:a,b \in \mathbb{Q} \right\}\). \\
\textbf{B2.} \(R\) halqani \(R'\) halqaga o'tkazuvchi gomomorfizmini aniqlang.
\(R = (\mathbb{Z}_{6}, +_{6}, \cdot_{6})\) hám \(R' = (\mathbb{Z}_{10}, +_{10}, \cdot_{10})\). \\
\textbf{B3.} Quydagi matritsalar to'plamining qaysi biri matritsalarni qo'shish va ko'paytirish amallarga qarata halqa bo'ladi.
\(M_{2 \times 2}\mathbb{(R) =}\left\{ \begin{pmatrix}
a & b \\
b & d
\end{pmatrix}\ :\ a,b,c,d \in \mathbb{R} \right\}\). \\
\textbf{C1.} Quydagi maydon kengaytmasining har birining bazisini toping. Har bir kengaytmaning darajasi qanday?
\(\mathbb{Q}\) da \(\mathbb{Q}\left( \sqrt{3},\sqrt{6} \right)\). \\
\textbf{C2.} Quydagi halqaning barcha ideallarini toping. Bul ideallardan qaysi-biri maksimal bo'ladi?
\(\mathbb{M}_{2}\left( \mathbb{Z} \right)\), elementleri \(\mathbb{Z}\) bol\(g'\)an \(2 \times 2\) matrica \\
\textbf{C3.} Quyidagi akslantirishni gomomorfizm shartlariga tekshiring. \(f(x) = \sqrt[3]{x}\) \\

\end{tabular}
\vspace{1cm}


\begin{tabular}{m{17cm}}
\textbf{8-variant}
\newline

\textbf{T1.} p-adik sonlar maydoni va ular ustida amallar. \\
\textbf{T2.} Ratsional sonlar maydonini haqiqiy sonlar maydonigacha to'ldirish. \\
\textbf{A1.} \(\mathbb{Q}\) Ratsianal sonlar maydoni ustida minimal ko'phadalrini toping.
\(\sqrt{6 + 3\sqrt{2}}\). \\
\textbf{A2.} Quydagini hisoblang:
\(\mathbb{Z}_{12}\) de \(\left( 5x^{2} + 3x - 4 \right)\left( 4x^{2} - x + 9 \right)\) \\
\textbf{A3.} \(\mathbb{Z}_{4}\lbrack x\rbrack\) da \(\deg\ p(x) > 1\) bo'ladigan \(p(x)\) birligini toping. Quydagi ko'phadlarning qaysilari \(\mathbb{Q\lbrack}x\rbrack\) da keltirilmaydigan?
\[x^{4} - 2x^{3} + 2x^{2} + x + 4\] \\
\textbf{B1.} Quydagi maydonlarning berilgan ko'phadlar orqali ajralish maydonini toping.
\(\mathbb{Q}\) da \(x^{4} - 5x^{2} + 6\). \\
\textbf{B2.} Quydagilar halqa bo'ladimi bunda \(i^{2} = - 1\):
\(\mathbb{Q(}i\sqrt{n}) = \{ x + iy\sqrt{n}\ |\ x,y \in \mathbb{Q\}}\). \\
\textbf{B3.} Quydagi matritsalar to'plamining qaysi biri matritsalarni qo'shish va ko'paytirish amallarga qarata halqa bo'ladi.
\[M_{2 \times 2}\mathbb{(R) =}\left\{ \begin{pmatrix}
a & 0 \\
0 & - a
\end{pmatrix}\ :\ a \in \mathbb{R} \right\}\] \\
\textbf{C1.} Quydagi maydon kengaytmasining har birining bazisini toping. Har bir kengaytmaning darajasi qanday?
\(\mathbb{Q}\) da \(\mathbb{Q}\left( \sqrt{2},\sqrt[3]{2} \right)\) \\
\textbf{C2.} Quydagi halqaning barcha ideallarini toping. Bul ideallardan qaysi-biri maksimal bo'ladi?
\[\mathbb{Z}_{18}\] \\
\textbf{C3.} Quyidagi akslantirishni gomomorfizm shartlariga tekshiring. \(f\left( \begin{pmatrix}
a & 0 \\
0 & a
\end{pmatrix} \right) = a\) \\

\end{tabular}
\vspace{1cm}


\begin{tabular}{m{17cm}}
\textbf{9-variant}
\newline

\textbf{T1.} Integral sohalar va maydonlar. \\
\textbf{T2.} Geometrik konstruksiyasi. \\
\textbf{A1.} \(\mathbb{Q}\) Ratsianal sonlar maydoni ustida minimal ko'phadalrini toping.
\(\sqrt{2 - \sqrt{2}}\). \\
\textbf{A2.} Quydagini hisoblang:
\(\mathbb{Z}_{12}\) de \(\left( 5x^{2} + 3x - 2 \right)^{2}\) \\
\textbf{A3.} Quydagi ko'phadlarning barcha nollarini toping:
\(\mathbb{Z}_{5}\) de \(3x^{3} - 4x^{2} - x + 4\); \\
\textbf{B1.} Quydagilarning qaysi biri maydon bo'ladi:
\(\mathbb{Q}\left( \sqrt{11} \right) = \left\{ a + b\sqrt{11}:a,b \in \mathbb{Q} \right\}\). \\
\textbf{B2.} \(R\) halqani \(R'\) halqaga o'tkazuvchi gomomorfizmini aniqlang.
\(R = (\mathbb{Z}_{4}, +_{4}, \cdot_{4})\) hám \(R' = (\mathbb{Z}_{6}, +_{6}, \cdot_{6})\). \\
\textbf{B3.} Quydagi maydonlarning berilgan ko'phadlar orqali ajralish maydonini toping.
\(\mathbb{Q}\) da \(x^{4} - 10x^{2} + 21\). \\
\textbf{C1.} Quydagi maydon kengaytmasining har birining bazisini toping. Har bir kengaytmaning darajasi qanday?
\(\mathbb{Q}\left( \sqrt{2} \right)\) da \(\mathbb{Q}\left( \sqrt{8} \right)\) \\
\textbf{C2.} Quydagi halqaning barcha ideallarini toping. Bul ideallardan qaysi-biri maksimal bo'ladi?
\(\mathbb{M}_{2}\left( \mathbb{R} \right)\), elementleri \(\mathbb{R}\) bol\(g'\)an \(2 \times 2\) matrica \\
\textbf{C3.} Quyidagi akslantirishni gomomorfizm shartlariga tekshiring.
\[f\left( a - \sqrt{2}b \right) = a + \sqrt{2}b\] \\

\end{tabular}
\vspace{1cm}


\begin{tabular}{m{17cm}}
\textbf{10-variant}
\newline

\textbf{T1.} Maydonlarning kengaytmasi. Algebrik element. Algebraik yopilma. \\
\textbf{T2.} Chekli maydonlarning strukturasi \\
\textbf{A1.} \(\mathbb{Q}\) Ratsianal sonlar maydoni ustida minimal ko'phadalrini toping.
\(\sqrt{2 + \sqrt{2}}\). \\
\textbf{A2.} Quydagini hisoblang:
\(\mathbb{Z}_{12}\) de \(\left( 5x^{2} + 3x - 4 \right) + \left( 4x^{2} - x + 9 \right)\). \\
\textbf{A3.} Quydagi ko'phadlarning barcha nollarini toping:
\(\mathbb{Z}_{2}\) de \(x^{3} + x + 1\). \\
\textbf{B1.} Quydagi maydon bo'ladimi:
\[\mathbb{Q}\left( \sqrt[3]{5} \right) = \left\{ a + b\sqrt[3]{5}:a,b,c \in \mathbb{Q} \right\}\] \\
\textbf{B2.} Quydagilar halqa bo'ladimi bunda \(i^{2} = - 1\):
\(\mathbb{Q(}i) = \{ x + iy\ |\ x,y \in \mathbb{Q\}}\). \\
\textbf{B3.} Quydagi matritsalar to'plamining qaysi biri matritsalarni qo'shish va ko'paytirish amallarga qarata halqa bo'ladi.
\[M_{2 \times 2}\mathbb{(R) =}\left\{ \begin{pmatrix}
a & 0 \\
0 & a
\end{pmatrix}\ :\ a \in \mathbb{R} \right\}\] \\
\textbf{C1.} Quydagi maydon kengaytmasining har birining bazisini toping. Har bir kengaytmaning darajasi qanday?
\(\mathbb{Q}\left( \sqrt{3} + \sqrt{5} \right)\) da \(\mathbb{Q}\left( \sqrt{2},\sqrt{6} + \sqrt{10} \right)\) \\
\textbf{C2.} Quydagi halqaning barcha ideallarini toping. Bul ideallardan qaysi-biri maksimal bo'ladi?
\(\mathbb{M}_{2}\left( \mathbb{Z} \right)\), elementleri \(\mathbb{Z}\) bol\(g'\)an \(2 \times 2\) matrica \\
\textbf{C3.} Quyidagi akslantirishni gomomorfizm shartlariga tekshiring. \(f(x) = \sqrt{x}\) \\

\end{tabular}
\vspace{1cm}


\begin{tabular}{m{17cm}}
\textbf{11-variant}
\newline

\textbf{T1.} Halqalar. \\
\textbf{T2.} Keltirilmaydigan ko'phadlar. \\
\textbf{A1.} \(\mathbb{Q}\) Ratsianal sonlar maydoni ustida minimal ko'phadalrini toping.
\(\sqrt{2 + \sqrt{3}}\). \\
\textbf{A2.} Quydagini hisoblang:
\(\mathbb{Z}_{12}\) de \(\left( 5x^{2} + 3x - 2 \right)^{2}\) \\
\textbf{A3.} \(\mathbb{Z}_{4}\lbrack x\rbrack\) da \(\deg\ p(x) > 1\) bo'ladigan \(p(x)\) birligini toping. Quydagi ko'phadlarning qaysilari \(\mathbb{Q\lbrack}x\rbrack\) da keltirilmaydigan?
\[3x^{5} - 4x^{3} - 6x^{2} + 6\] \\
\textbf{B1.} \(R\) halqani \(R'\) halqaga o'tkazuvchi gomomorfizmini aniqlang.
\(R = (\mathbb{Z}_{4}, +_{4}, \cdot_{4})\) hám \(R' = (\mathbb{Z}_{6}, +_{6}, \cdot_{6})\). \\
\textbf{B2.} Quydagilarning qaysi biri maydon bo'ladi, bunda \(i^{2} = - 1\):
\(\mathbb{Z\lbrack}i\rbrack = \{ x + iy\ |\ x,y \in \mathbb{Z\}}\). \\
\textbf{B3.} Quydagi matritsalar to'plamining qaysi biri matritsalarni qo'shish va ko'paytirish amallarga qarata halqa bo'ladi.
\(M_{2 \times 2}\mathbb{(R) =}\left\{ \begin{pmatrix}
a & 0 \\
b & c
\end{pmatrix}\ :\ a,b,c \in \mathbb{R} \right\}\). \\
\textbf{C1.} Quydagi maydon kengaytmasining har birining bazisini toping. Har bir kengaytmaning darajasi qanday?
\(\mathbb{Q}\) da \(\mathbb{Q}\left( i,\sqrt{2} + i,\sqrt{3} + i \right)\) \\
\textbf{C2.} Quydagi halqaning barcha ideallarini toping. Bul ideallardan qaysi-biri maksimal bo'ladi?
\[\mathbb{Z}_{18}\] \\
\textbf{C3.} Quyidagi akslantirishni gomomorfizm shartlariga tekshiring. \(f(x) = e^{x}\) \\

\end{tabular}
\vspace{1cm}


\begin{tabular}{m{17cm}}
\textbf{12-variant}
\newline

\textbf{T1.} p-adik kvadrat tenglamalar. \\
\textbf{T2.} Integral sohalar va maydonlar. \\
\textbf{A1.} \(\mathbb{Q}\) Ratsianal sonlar maydoni ustida minimal ko'phadalrini toping.
\(\sqrt{3 - \sqrt{3}}\). \\
\textbf{A2.} Quydagini hisoblang:
\(\mathbb{Z}_{9}\) da \(\left( 7x^{3} + 3x^{2} - x \right) + \left( 6x^{2} - 8x + 4 \right)\) \\
\textbf{A3.} \(\mathbb{Z}_{4}\lbrack x\rbrack\) da \(\deg\ p(x) > 1\) bo'ladigan \(p(x)\) birligini toping. Quydagi ko'phadlarning qaysilari \(\mathbb{Q\lbrack}x\rbrack\) da keltirilmaydigan?
\[x^{4} - 5x^{3} + 3x - 2\] \\
\textbf{B1.} Quydagi halqa bo'ladimi:
\(\mathbb{Q}\left( \sqrt{2} \right) = \left\{ a + b\sqrt{2}:a,b \in \mathbb{Q} \right\}\). \\
\textbf{B2.} \(R\) halqani \(R'\) halqaga o'tkazuvchi gomomorfizmini aniqlang.
\(R\mathbb{= (R,} + , \cdot )\) hám \(R'\mathbb{= (R,} + , \cdot )\). \\
\textbf{B3.} Quydagi maydonning berilgan ko'phadlar orqali ajralish maydonini toping.
\(\mathbb{Q}\) da \(x^{4} - 2\). \\
\textbf{C1.} Quydagi maydon kengaytmasining har birining bazisini toping. Har bir kengaytmaning darajasi qanday?
\(\mathbb{Q}\left( \sqrt{3} + \sqrt{5} \right)\) da \(\mathbb{Q}\left( \sqrt{2},\sqrt{6} + \sqrt{10} \right)\) \\
\textbf{C2.} Quydagi halqaning barcha ideallarini toping. Bul ideallardan qaysi-biri maksimal bo'ladi?
\(\mathbb{M}_{2}\left( \mathbb{Z} \right)\), elementleri \(\mathbb{Z}\) bol\(g'\)an \(2 \times 2\) matrica \\
\textbf{C3.} Quyidagi akslantirishni gomomorfizm shartlariga tekshiring. \(f(a + ib) = \begin{pmatrix}
a & b \\
 - b & a
\end{pmatrix}\) \\

\end{tabular}
\vspace{1cm}


\begin{tabular}{m{17cm}}
\textbf{13-variant}
\newline

\textbf{T1.} p-adik sonlar maydoni va ular ustida amallar. \\
\textbf{T2.} Radikallarda yechilishi. \\
\textbf{A1.} \(\mathbb{Q}\) Ratsianal sonlar maydoni ustida minimal ko'phadalrini toping.
\(\sqrt{3 - \sqrt{3}}\). \\
\textbf{A2.} Quydagini hisoblang:
\(\mathbb{Z}_{9}\) da \(\left( 7x^{3} + 3x^{2} - x \right) + \left( 6x^{2} - 8x + 4 \right)\) \\
\textbf{A3.} \(\mathbb{Z}_{4}\lbrack x\rbrack\) da \(\deg\ p(x) > 1\) bo'ladigan \(p(x)\) birligini toping. Quydagi ko'phadlarning qaysilari \(\mathbb{Q\lbrack}x\rbrack\) da keltirilmaydigan?
\[5x^{5} - 6x^{4} - 3x^{2} + 9x - 15\] \\
\textbf{B1.} Quydagilar halqa bo'ladimi:
\[\mathbb{Z}_{18}\] \\
\textbf{B2.} Quydagilar halqa bo'ladimi bunda \(i^{2} = - 1\):
\(\mathbb{Z(}i\sqrt{n}) = \{ x + iy\sqrt{n}\ |\ x,y \in \mathbb{Z\}}\). \\
\textbf{B3.} Quydagi maydonning berilgan ko'phadlar orqali ajralish maydonini toping.
\(\mathbb{Q}\) da \(x^{4} + 1\). \\
\textbf{C1.} Quydagi maydon kengaytmasining har birining bazisini toping. Har bir kengaytmaning darajasi qanday?
\(\mathbb{Q}\left( \sqrt{3} + \sqrt{5} \right)\) da \(\mathbb{Q}\left( \sqrt{2},\sqrt{6} + \sqrt{10} \right)\) \\
\textbf{C2.} Quydagi halqaning barcha ideallarini toping. Bul ideallardan qaysi-biri maksimal bo'ladi?
\[\mathbb{Q}\] \\
\textbf{C3.} Quyidagi akslantirishni gomomorfizm shartlariga tekshiring. \(f\left( a + \sqrt{2}b \right) = a + bi\) \\

\end{tabular}
\vspace{1cm}


\begin{tabular}{m{17cm}}
\textbf{14-variant}
\newline

\textbf{T1.} Maydonlarning avtomorfizmlari. \\
\textbf{T2.} Ko'phadlarning halqasi. \\
\textbf{A1.} \(\mathbb{Q}\) Ratsianal sonlar maydoni ustida minimal ko'phadalrini toping.
\(\sqrt{3 + \sqrt{3}}\). \\
\textbf{A2.} \(\mathbb{Q}\) Ratsianal sonlar maydoni ustida minimal ko'phadalrini toping.
\(\sqrt{2} + \sqrt{5}\). \\
\textbf{A3.} \(\mathbb{Q}\) Ratsianal sonlar maydoni ustida minimal ko'phadni toping.
\(\sqrt{3} + \sqrt{5}\). \\
\textbf{B1.} \(R\) halqani \(R'\) halqaga o'tkazuvchi gomomorfizmini aniqlang.
\(R = (\mathbb{Z}_{6}, +_{6}, \cdot_{6})\) hám \(R' = (\mathbb{Z}_{10}, +_{10}, \cdot_{10})\). \\
\textbf{B2.} \(R\) halqani \(R'\) halqaga o'tkazuvchi gomomorfizmini aniqlang.
\(R\mathbb{= (R,} + , \cdot )\) hám \(R'\mathbb{= (R,} + , \cdot )\). \\
\textbf{B3.} Quydagi matritsalar to'plamining qaysi biri matritsalarni qo'shish va ko'paytirish amallarga qarata halqa bo'ladi.
\(M_{2 \times 2}\mathbb{(R) =}\left\{ \begin{pmatrix}
a & 0 \\
0 & b
\end{pmatrix}\ :\ a,b \in \mathbb{R} \right\}\). \\
\textbf{C1.} Quydagi maydon kengaytmasining har birining bazisini toping. Har bir kengaytmaning darajasi qanday?
\(\mathbb{Q}\left( \sqrt{3} + \sqrt{5} \right)\) da \(\mathbb{Q}\left( \sqrt{2},\sqrt{6} + \sqrt{10} \right)\) \\
\textbf{C2.} Quydagi halqaning barcha ideallarini toping. Bul ideallardan qaysi-biri maksimal bo'ladi?
\(\mathbb{M}_{2}\left( \mathbb{Z} \right)\), elementleri \(\mathbb{Z}\) bol\(g'\)an \(2 \times 2\) matrica
T1 Ko'phadlarning halqasi. \\
\textbf{C3.} Quyidagi akslantirishni gomomorfizm shartlariga tekshiring. \(f(x) = x^{2} + x\) \\

\end{tabular}
\vspace{1cm}


\begin{tabular}{m{17cm}}
\textbf{15-variant}
\newline

\textbf{T1.} Halqalarning gomomorfizmi va ideallar. \\
\textbf{T2.} Maksimal va sodda ideallar. \\
\textbf{A1.} \(\mathbb{Q}\) Ratsianal sonlar maydoni ustida minimal ko'phadalrini toping.
\(\sqrt{2 + \sqrt{3}}\). \\
\textbf{A2.} Quydagini hisoblang:
\(\mathbb{Z}_{5}\) te \(\left( 3x^{2} + 3x - 4 \right)\left( 4x^{2} + 2 \right)\) \\
\textbf{A3.} Quydagi ko'phadlarning barcha nollarini toping:
\(\mathbb{Z}_{2}\) de \(x^{3} + x + 1\). \\
\textbf{B1.} Quydagi maydon bo'ladimi:
\(\mathbb{Z}\left( \sqrt{5} \right) = \left\{ a + b\sqrt{5}:a,b \in \mathbb{Z} \right\}\). \\
\textbf{B2.} Quydagi maydon bo'ladimi, bunda \(i^{2} = - 1\):
\[\mathbb{Z\lbrack}i\sqrt{n}\rbrack = \{ x + iy\sqrt{n}\ |\ x,y \in \mathbb{Z\}}\] \\
\textbf{B3.} Quydagi maydonning berilgan ko'phadlar orqali ajralish maydonini toping.
\(\mathbb{Z}_{3}\) te \(x^{3} + 2x + 2\). \\
\textbf{C1.} Quydagi maydon kengaytmasining har birining bazisini toping. Har bir kengaytmaning darajasi qanday?
\(\mathbb{Q}\) da \(\mathbb{Q}\left( \sqrt{3},\sqrt{6} \right)\) \\
\textbf{C2.} Quydagi halqaning barcha ideallarini toping. Bul ideallardan qaysi-biri maksimal bo'ladi?
\(\mathbb{M}_{2}\left( \mathbb{Z} \right)\), elementleri \(\mathbb{Z}\) bol\(g'\)an \(2 \times 2\) matrica \\
\textbf{C3.} Quyidagi akslantirishni gomomorfizm shartlariga tekshiring. \(f(x) = 5^{x}\) \\

\end{tabular}
\vspace{1cm}


\begin{tabular}{m{17cm}}
\textbf{16-variant}
\newline

\textbf{T1.} Fundamental teoremalar. \\
\textbf{T2.} Bo'lish algoritmi. \\
\textbf{A1.} Quydagi halqaning qism to'plamlari ideal bo'lishini ko'rsating.
\(R\mathbb{= Z\lbrack}\sqrt{7}\rbrack\), \(I = \{ a + b\sqrt{7}\ |\ \ a,b \in \mathbb{Z,}a - b\ \ juft\ son\}\). \\
\textbf{A2.} Quydagini hisoblang:
\(\mathbb{Z}_{5}\) te \(\left( 3x^{2} + 2x - 4 \right) + \left( 4x^{2} + 2 \right)\) \\
\textbf{A3.} \(\mathbb{Z}_{4}\lbrack x\rbrack\) da \(\deg\ p(x) > 1\) bo'ladigan \(p(x)\) birligini toping. Quydagi ko'phadlarning qaysilari \(\mathbb{Q\lbrack}x\rbrack\) da keltirilmaydigan?
\[3x^{5} - 4x^{3} - 6x^{2} + 6\] \\
\textbf{B1.} Quydagi halqa bo'ladimi:
\(7\mathbb{Z}\). \\
\textbf{B2.} Quydagi halqa bo'ladimi:
\(\mathbb{R\lbrack}\sigma\rbrack = \{ x + \sigma \cdot y\ |\ \sigma^{2} = 0\}\). \\
\textbf{B3.} Quydagi matritsalar to'plamining qaysi biri matritsalarni qo'shish va ko'paytirish amallarga qarata halqa bo'ladi
\[M_{2 \times 2}\mathbb{(R) =}\left\{ \begin{pmatrix}
a & b \\
0 & c
\end{pmatrix}\ :\ a,b,c \in \mathbb{R} \right\}\] \\
\textbf{C1.} Quydagi maydon kengaytmasining har birining bazisini toping. Har bir kengaytmaning darajasi qanday?
\(\mathbb{Q}\) da \(\mathbb{Q}\left( \sqrt{3},\sqrt{5},\sqrt{7} \right)\) \\
\textbf{C2.} Quydagi halqaning barcha ideallarini toping. Bul ideallardan qaysi-biri maksimal bo'ladi?
\(\mathbb{M}_{2}\left( \mathbb{Z} \right)\), elementleri \(\mathbb{Z}\) bol\(g'\)an \(2 \times 2\) matrica \\
\textbf{C3.} Quyidagi akslantirishni gomomorfizm shartlariga tekshiring. \(f(x + iy) = x \cdot y\) \\

\end{tabular}
\vspace{1cm}


\begin{tabular}{m{17cm}}
\textbf{17-variant}
\newline

\textbf{T1.} Maydonlarning ajralishi. \\
\textbf{T2.} Ratsional sonlar maydonini haqiqiy sonlar maydonigacha to'ldirish. \\
\textbf{A1.} \(\mathbb{Q}\) Ratsianal sonlar maydoni ustida minimal ko'phadalrini toping.
\(\sqrt{2 + \sqrt{3}}\). \\
\textbf{A2.} Quydagini hisoblang:
\(\mathbb{Z}_{5}\) te \(\left( 3x^{2} + 2x - 4 \right) + \left( 4x^{2} + 2 \right)\) \\
\textbf{A3.} \(\mathbb{Z}_{4}\lbrack x\rbrack\) da \(\deg\ p(x) > 1\) bo'ladigan \(p(x)\) birligini toping. Quydagi ko'phadlarning qaysilari \(\mathbb{Q\lbrack}x\rbrack\) da keltirilmaydigan?
\[x^{4} - 5x^{3} + 3x - 2\] \\
\textbf{B1.} Quydagi maydon bo'ladimi?
\(\mathbb{Q}\left( \sqrt{5},\sqrt{7} \right) = \left\{ a + b\sqrt{5} + c\sqrt{7} + d\sqrt{35}:a,b,c,d \in \mathbb{Q} \right\}\). \\
\textbf{B2.} Quydagi maydon bo'ladimi, bunda \(i^{2} = - 1\):
\(\mathbb{Q\lbrack}i\sqrt{n}\rbrack = \{ x + iy\sqrt{n}\ |\ x,y \in \mathbb{Q\}}\). \\
\textbf{B3.} Quydagi maydonlarning berilgan ko'phadlar orqali ajralish maydonini toping.
\(\mathbb{Z}_{2}\) da \(x^{2} + 1\). \\
\textbf{C1.} Quydagi maydon kengaytmasining har birining bazisini toping. Har bir kengaytmaning darajasi qanday?
\(\mathbb{Q}\left( \sqrt{5} \right)\) da \(\mathbb{Q}\left( \sqrt{2} + \sqrt{5} \right)\) \\
\textbf{C2.} Quydagi halqaning barcha ideallarini toping. Bul ideallardan qaysi-biri maksimal bo'ladi?
\[\mathbb{Z}_{18}\] \\
\textbf{C3.} Quyidagi akslantirishni gomomorfizm shartlariga tekshiring. \(f\left( \begin{pmatrix}
a & 0 \\
0 & a
\end{pmatrix} \right) = a\) \\

\end{tabular}
\vspace{1cm}


\begin{tabular}{m{17cm}}
\textbf{18-variant}
\newline

\textbf{T1.} Halqalar. \\
\textbf{T2.} Fundamental teoremalar. \\
\textbf{A1.} \(\mathbb{Q}\) Ratsianal sonlar maydoni ustida minimal ko'phadalrini toping.
\(\sqrt{2 + 2\sqrt{2}}\). \\
\textbf{A2.} Quydagini hisoblang:
\(\mathbb{Z}_{12}\) de \(\left( 5x^{2} + 3x - 2 \right)^{2}\) \\
\textbf{A3.} Quydagi halqalarning qism to'plamlari ideal bo'lishini ko'rsating: \(R = \left\{ \begin{pmatrix}
a & b \\
0 & c
\end{pmatrix}\ |\ a,b,c \in \mathbb{Z} \right\}\), \(I = \left\{ \begin{pmatrix}
0 & a \\
0 & 0
\end{pmatrix}\ |\ a \in \mathbb{Z} \right\}\). \\
\textbf{B1.} Quydagi halqa bo'ladimi:
\(\mathbb{Q(}\sqrt[3]{2}) = \left\{ a + b\sqrt[3]{2}:a,b \in \mathbb{Q} \right\}\).
B2 Quydagilarning qaysi biri maydon bo'ladi, bunda \(i^{2} = - 1\):
\(\mathbb{Q\lbrack}i\rbrack = \{ x + iy\ |\ x,y \in \mathbb{Q\}}\).. \\
\textbf{B2.} Quydagi maydon bo'ladimi:
\(5\mathbb{Z}\). \\
\textbf{B3.} Quydagi maydonning berilgan ko'phadlar orqali ajralish maydonini toping.
\(\mathbb{Q}\) da \(x^{4} - 10x^{2} + 21\). \\
\textbf{C1.} Quydagi maydon kengaytmasining har birining bazisini toping. Har bir kengaytmaning darajasi qanday?
\(\mathbb{Q}\left( \sqrt{3} + \sqrt{5} \right)\) da \(\mathbb{Q}\left( \sqrt{2},\sqrt{6} + \sqrt{10} \right)\) \\
\textbf{C2.} Quydagi halqaning barcha ideallarini toping. Bul ideallardan qaysi-biri maksimal bo'ladi?
\(\mathbb{M}_{2}\left( \mathbb{Z} \right)\), elementleri \(\mathbb{Z}\) bol\(g'\)an \(2 \times 2\) matrica \\
\textbf{C3.} Quyidagi akslantirishni gomomorfizm shartlariga tekshiring. \(f\left( a + \sqrt{2}b \right) = a + bi\) \\

\end{tabular}
\vspace{1cm}


\begin{tabular}{m{17cm}}
\textbf{19-variant}
\newline

\textbf{T1.} Maydonlarning kengaytmasi. Algebrik element. Algebraik yopilma. \\
\textbf{T2.} Halqalarning gomomorfizmi va ideallar. \\
\textbf{A1.} Quydagi ko'phadlarning barcha nollarini toping:
\(\mathbb{Z}_{12}\) de \(5x^{3} + 4x^{2} - x + 9\); \\
\textbf{A2.} Quydagini hisoblang:
\(\mathbb{Z}_{5}\) te \(\left( 3x^{2} + 3x - 4 \right)\left( 4x^{2} + 2 \right)\) \\
\textbf{A3.} Quydagi ko'phadlarning barcha nollarini toping:
\(\mathbb{Z}_{5}\) de \(3x^{3} - 4x^{2} - x + 4\); \\
\textbf{B1.} Quydagi halqa bo'ladimi:
\[\mathbb{Z}\left( \sqrt{3} \right) = \left\{ a + b\sqrt{3}:a,b \in \mathbb{Z} \right\}\] \\
\textbf{B2.} Quydagi maydon bo'ladimi, bunda \(i^{2} = - 1\):
\(\mathbb{Q\lbrack}i\sqrt{n}\rbrack = \{ x + iy\sqrt{n}\ |\ x,y \in \mathbb{Q\}}\). \\
\textbf{B3.} Quydagi matritsalar to'plamining qaysi biri matritsalarni qo'shish va ko'paytirish amallarga qarata halqa bo'ladi.
\(M_{2 \times 2}\mathbb{(R) =}\left\{ \begin{pmatrix}
a & 0 \\
0 & 0
\end{pmatrix}\ :\ a \in \mathbb{R} \right\}\). \\
\textbf{C1.} Quydagi maydon kengaytmasining har birining bazisini toping. Har bir kengaytmaning darajasi qanday?
\(\mathbb{Q}\left( \sqrt{5} \right)\) da \(\mathbb{Q}\left( \sqrt{2} + \sqrt{5} \right)\) \\
\textbf{C2.} Quydagi halqaning barcha ideallarini toping. Bul ideallardan qaysi-biri maksimal bo'ladi?
\(\mathbb{M}_{2}\left( \mathbb{Z} \right)\), elementleri \(\mathbb{Z}\) bol\(g'\)an \(2 \times 2\) matrica \\
\textbf{C3.} Quyidagi akslantirishni gomomorfizm shartlariga tekshiring. \(f(x) = \sqrt{x}\) \\

\end{tabular}
\vspace{1cm}


\begin{tabular}{m{17cm}}
\textbf{20-variant}
\newline

\textbf{T1.} p-adik kvadrat tenglamalar. \\
\textbf{T2.} p-adik sonlar maydoni va ular ustida amallar. \\
\textbf{A1.} Quydagi ko'phadlarning barcha nollarini toping:
\(\mathbb{Z}_{7}\) de \(5x^{4} + 2x^{2} - 3\); \\
\textbf{A2.} Quydagini hisoblang:
\(\mathbb{Z}_{5}\) te \(\left( 3x^{2} + 2x - 4 \right) + \left( 4x^{2} + 2 \right)\) \\
\textbf{A3.} Quydagi halqalarning qism to'plamlari ideal bo'lishini ko'rsating: \(R = \left\{ \begin{pmatrix}
a & b \\
0 & c \\
 & 
\end{pmatrix}\ |\ a,b,c \in \mathbb{Z} \right\}\), \(I = \left\{ \begin{pmatrix}
0 & b \\
0 & c
\end{pmatrix}\ |\ a \in \mathbb{Z} \right\}\). \\
\textbf{B1.} Quydagi maydonlarning berilgan ko'phadlar orqali ajralish maydonini toping.
\(\mathbb{Q}\) da \(x^{4} - 5x^{2} + 6\). \\
\textbf{B2.} Quydagilar halqa bo'ladimi bunda \(i^{2} = - 1\):
\(\mathbb{Q(}i) = \{ x + iy\ |\ x,y \in \mathbb{Q\}}\). \\
\textbf{B3.} Quydagi maydonning berilgan ko'phadlar orqali ajralish maydonini toping.
\(\mathbb{Z}_{3}\) te \(x^{2} + x + 1\). \\
\textbf{C1.} Quydagi maydon kengaytmasining har birining bazisini toping. Har bir kengaytmaning darajasi qanday?
\(\mathbb{Q}\left( \sqrt{3} + \sqrt{5} \right)\) da \(\mathbb{Q}\left( \sqrt{2},\sqrt{6} + \sqrt{10} \right)\) \\
\textbf{C2.} Quydagi halqaning barcha ideallarini toping. Bul ideallardan qaysi-biri maksimal bo'ladi?
\[\mathbb{Z}_{25}\] \\
\textbf{C3.} Quyidagi akslantirishni gomomorfizm shartlariga tekshiring. \(f(x) = x^{2} + x\) \\

\end{tabular}
\vspace{1cm}


\begin{tabular}{m{17cm}}
\textbf{21-variant}
\newline

\textbf{T1.} Bo'lish algoritmi. \\
\textbf{T2.} Chekli maydonlarning strukturasi \\
\textbf{A1.} \(\mathbb{Q}\) Ratsianal sonlar maydoni ustida minimal ko'phadalrini toping.
\(\sqrt{3} + \sqrt{7}\). \\
\textbf{A2.} Quydagini hisoblang:
\(\mathbb{Z}_{5}\) te \(\left( 3x^{2} + 2x - 4 \right) + \left( 4x^{2} + 2 \right)\) \\
\textbf{A3.} Quydagi ko'phadlarning barcha nollarini toping:
\(\mathbb{Z}_{2}\) de \(x^{3} + x + 1\). \\
\textbf{B1.} \(R\) halqani \(R'\) halqaga o'tkazuvchi gomomorfizmini aniqlang.
\(R = (\mathbb{Z}_{6}, +_{6}, \cdot_{6})\) hám \(R' = (\mathbb{Z}_{10}, +_{10}, \cdot_{10})\). \\
\textbf{B2.} \(R\) halqani \(R'\) halqaga o'tkazuvchi gomomorfizmini aniqlang.
\(R\mathbb{= (Z,} + , \cdot )\) va \(R' = (\mathbb{Z}_{6}, +_{6}, \cdot_{6})\). \\
\textbf{B3.} Quydagi matritsalar to'plamining qaysi biri matritsalarni qo'shish va ko'paytirish amallarga qarata halqa bo'ladi.
\(M_{2 \times 2}\mathbb{(R) =}\left\{ \begin{pmatrix}
a & b \\
 - b & d
\end{pmatrix}\ :\ a,b,c,d \in \mathbb{R} \right\}\). \\
\textbf{C1.} Quydagi maydon kengaytmasining har birining bazisini toping. Har bir kengaytmaning darajasi qanday?
\(\mathbb{Q}\left( \sqrt{2} \right)\) da \(\mathbb{Q}\left( \sqrt{8} \right)\) \\
\textbf{C2.} Quydagi halqaning barcha ideallarini toping. Bul ideallardan qaysi-biri maksimal bo'ladi?
\(\mathbb{M}_{2}\left( \mathbb{Z} \right)\), elementleri \(\mathbb{Z}\) bol\(g'\)an \(2 \times 2\) matrica \\
\textbf{C3.} Quyidagi akslantirishni gomomorfizm shartlariga tekshiring.
\[f:\begin{pmatrix}
a & b \\
 - b & a
\end{pmatrix} \rightarrow a + bi\] \\

\end{tabular}
\vspace{1cm}


\begin{tabular}{m{17cm}}
\textbf{22-variant}
\newline

\textbf{T1.} Ratsional sonlar maydonini haqiqiy sonlar maydonigacha to'ldirish. \\
\textbf{T2.} Geometrik konstruksiyasi. \\
\textbf{A1.} Quydagi halqaning qism to'plamlari ideal bo'lishini ko'rsating.
\(R = \mathbb{Z}_{28}\), \(I = \{\overline{0},\overline{7},\overline{14},\overline{21}\}\). \\
\textbf{A2.} Quydagini hisoblang:
\(\mathbb{Z}_{12}\) de \(\left( 5x^{2} + 3x - 2 \right)^{2}\) \\
\textbf{A3.} Quydagi halqalarning qism to'plamlari ideal bo'lishini ko'rsating: \(R = \left\{ \begin{pmatrix}
a & b \\
0 & c
\end{pmatrix}\ |\ a,b,c \in \mathbb{Z} \right\}\), \(I = \left\{ \begin{pmatrix}
0 & a \\
0 & 0
\end{pmatrix}\ |\ a \in \mathbb{Z} \right\}\). \\
\textbf{B1.} Quydagi maydon bo'ladimi:
\(\mathbb{Q(}\sqrt{3}) = \left\{ a + b\sqrt[3]{3}:a,b \in \mathbb{Q} \right\}\). \\
\textbf{B2.} \(R\) halqani \(R'\) halqaga o'tkazuvchi gomomorfizmini aniqlang.
\(R = (\mathbb{Z}_{4}, +_{4}, \cdot_{4})\) hám \(R' = (\mathbb{Z}_{6}, +_{6}, \cdot_{6})\). \\
\textbf{B3.} Quydagi maydonning berilgan ko'phadlar orqali ajralish maydonini toping.
\(\mathbb{Q}\) da \(x^{4} - 2\). \\
\textbf{C1.} Quydagi maydon kengaytmasining har birining bazisini toping. Har bir kengaytmaning darajasi qanday?
\(\mathbb{Q}\) da \(\mathbb{Q}\left( \sqrt{2},i \right)\) \\
\textbf{C2.} Quydagi halqaning barcha ideallarini toping. Bul ideallardan qaysi-biri maksimal bo'ladi?
\(\mathbb{M}_{2}\left( \mathbb{Z} \right)\), elementleri \(\mathbb{Z}\) bol\(g'\)an \(2 \times 2\) matrica \\
\textbf{C3.} Quyidagi akslantirishni gomomorfizm shartlariga tekshiring. \(f:\begin{pmatrix}
a & b \\
0 & c
\end{pmatrix} \rightarrow a\) \\

\end{tabular}
\vspace{1cm}


\begin{tabular}{m{17cm}}
\textbf{23-variant}
\newline

\textbf{T1.} Maksimal va sodda ideallar. \\
\textbf{T2.} Keltirilmaydigan ko'phadlar. \\
\textbf{A1.} Quydagi ko'phadlarning barcha nollarini toping:
\(\mathbb{Z}_{12}\) de \(5x^{3} + 4x^{2} - x + 9\); \\
\textbf{A2.} Quydagini hisoblang:
\(\mathbb{Z}_{5}\) te \(\left( 3x^{2} + 2x - 4 \right) + \left( 4x^{2} + 2 \right)\) \\
\textbf{A3.} Quydagi ko'phadlarning barcha nollarini toping:
\(\mathbb{Z}_{5}\) de \(3x^{3} - 4x^{2} - x + 4\); \\
\textbf{B1.} Quydagi halqa bo'ladimi:
\(\mathbb{Z}_{18}\). \\
\textbf{B2.} \(R\) halqani \(R'\) halqaga o'tkazuvchi gomomorfizmini aniqlang.
\(R\mathbb{= (R,} + , \cdot )\) hám \(R'\mathbb{= (R,} + , \cdot )\). \\
\textbf{B3.} Quydagi maydonning berilgan ko'phadlar orqali ajralish maydonini toping.
\(\mathbb{Q}\) da \(x^{4} - 5x^{2} + 21\). \\
\textbf{C1.} Quydagi maydon kengaytmasining har birining bazisini toping. Har bir kengaytmaning darajasi qanday?
\(\mathbb{Q}\left( \sqrt{3} + \sqrt{5} \right)\) da \(\mathbb{Q}\left( \sqrt{2},\sqrt{6} + \sqrt{10} \right)\) \\
\textbf{C2.} Quydagi halqaning barcha ideallarini toping. Bul ideallardan qaysi-biri maksimal bo'ladi?
\(\mathbb{M}_{2}\left( \mathbb{Z} \right)\), elementleri \(\mathbb{Z}\) bol\(g'\)an \(2 \times 2\) matrica \\
\textbf{C3.} Quyidagi akslantirishni gomomorfizm shartlariga tekshiring. \(f(x) = 5^{x}\) \\

\end{tabular}
\vspace{1cm}


\begin{tabular}{m{17cm}}
\textbf{24-variant}
\newline

\textbf{T1.} Integral sohalar va maydonlar. \\
\textbf{T2.} Maydonlarning ajralishi. \\
\textbf{A1.} \(\mathbb{Q}\) Ratsianal sonlar maydoni ustida minimal ko'phadalrini toping.
\(\sqrt{2 + \sqrt{3}}\). \\
\textbf{A2.} Quydagini hisoblang:
\(\mathbb{Z}_{12}\) de \(\left( 5x^{2} + 3x - 4 \right)\left( 4x^{2} - x + 9 \right)\). \\
\textbf{A3.} \(\mathbb{Q}\) Ratsianal sonlar maydoni ustida minimal ko'phadni toping.
\[\sqrt{2} + \sqrt{3}\] \\
\textbf{B1.} Quydagi halqa bo'ladimi:
\[\mathbb{Q}\left( \sqrt{2},\sqrt{3} \right) = \left\{ a + b\sqrt{2} + c\sqrt{3} + d\sqrt{6}:a,b,c,d \in \mathbb{Q} \right\}\] \\
\textbf{B2.} Quydagilar halqa bo'ladimi bunda \(i^{2} = - 1\):
\(\mathbb{Q(}i\sqrt{n}) = \{ x + iy\sqrt{n}\ |\ x,y \in \mathbb{Q\}}\). \\
\textbf{B3.} Quydagi matritsalar to'plamining qaysi biri matritsalarni qo'shish va ko'paytirish amallarga qarata halqa bo'ladi.
\(M_{2 \times 2}\mathbb{(R) =}\left\{ \begin{pmatrix}
a & b \\
b & d
\end{pmatrix}\ :\ a,b,c,d \in \mathbb{R} \right\}\). \\
\textbf{C1.} Quydagi maydon kengaytmasining har birining bazisini toping. Har bir kengaytmaning darajasi qanday?
\(\mathbb{Q}\) da \(\mathbb{Q}\left( \sqrt{3},\sqrt{6} \right)\). \\
\textbf{C2.} Quydagi halqaning barcha ideallarini toping. Bul ideallardan qaysi-biri maksimal bo'ladi? $\mathbb{Q}$ \\
\textbf{C3.} Quyidagi akslantirishni gomomorfizm shartlariga tekshiring. \(f(x) = e^{x}\) \\

\end{tabular}
\vspace{1cm}


\begin{tabular}{m{17cm}}
\textbf{25-variant}
\newline

\textbf{T1.} Radikallarda yechilishi. \\
\textbf{T2.} Maydonlarning avtomorfizmlari. \\
\textbf{A1.} \(\mathbb{Q}\) Ratsianal sonlar maydoni ustida minimal ko'phadalrini toping.
\(\sqrt{3 - \sqrt{3}}\). \\
\textbf{A2.} Quydagini hisoblang:
\(\mathbb{Z}_{5}\) te \(\left( 3x^{2} + 2x - 4 \right) + \left( 4x^{2} + 2 \right)\) \\
\textbf{A3.} Quydagi halqalarning qism to'plamlari ideal bo'lishini ko'rsating: \(R = \left\{ \begin{pmatrix}
a & b \\
0 & c \\
 & 
\end{pmatrix}\ |\ a,b,c \in \mathbb{Z} \right\}\), \(I = \left\{ \begin{pmatrix}
0 & b \\
0 & c
\end{pmatrix}\ |\ a \in \mathbb{Z} \right\}\). \\
\textbf{B1.} \(R\) halqani \(R'\) halqaga o'tkazuvchi gomomorfizmini aniqlang.
\(R = (\mathbb{Z}_{4}, +_{4}, \cdot_{4})\) hám \(R' = (\mathbb{Z}_{6}, +_{6}, \cdot_{6})\). \\
\textbf{B2.} \(R\) halqani \(R'\) halqaga o'tkazuvchi gomomorfizmini aniqlang.
\(R\mathbb{= (Z,} + , \cdot )\) va \(R' = (\mathbb{Z}_{6}, +_{6}, \cdot_{6})\). \\
\textbf{B3.} Quydagi matritsalar to'plamining qaysi biri matritsalarni qo'shish va ko'paytirish amallarga qarata halqa bo'ladi
\(M_{2 \times 2}\mathbb{(R) =}\left\{ \begin{pmatrix}
a & 0 \\
b & c
\end{pmatrix}\ :\ a,b,c \in \mathbb{R} \right\}\). \\
\textbf{C1.} Quydagi maydon kengaytmasining har birining bazisini toping. Har bir kengaytmaning darajasi qanday?
\(\mathbb{Q}\) da \(\mathbb{Q}\left( \sqrt{2},i \right)\) \\
\textbf{C2.} Quydagi halqaning barcha ideallarini toping. Bul ideallardan qaysi-biri maksimal bo'ladi?
\[\mathbb{Q}\] \\
\textbf{C3.} Quyidagi akslantirishni gomomorfizm shartlariga tekshiring. \(f(a) = a^{n}\) \\

\end{tabular}
\vspace{1cm}


\begin{tabular}{m{17cm}}
\textbf{26-variant}
\newline

\textbf{T1.} Ko'phadlarning halqasi. \\
\textbf{T2.} Maydonlarning avtomorfizmlari. \\
\textbf{A1.} \(\mathbb{Q}\) Ratsianal sonlar maydoni ustida minimal ko'phadalrini toping.
\(\sqrt{2 + \sqrt{2}}\). \\
\textbf{A2.} Quydagini hisoblang:
\(\mathbb{Z}_{5}\) te \(\left( 3x^{2} + 3x - 4 \right)\left( 4x^{2} + 2 \right)\) \\
\textbf{A3.} Quydagi ko'phadlarning barcha nollarini toping:
\(\mathbb{Z}_{2}\) de \(x^{3} + x + 1\). \\
\textbf{B1.} Quydagi maydon bo'ladimi:
\[\mathbb{Q}\left( \sqrt[3]{5} \right) = \left\{ a + b\sqrt[3]{5}:a,b,c \in \mathbb{Q} \right\}\] \\
\textbf{B2.} Quydagi maydon bo'ladimi, bunda \(i^{2} = - 1\):
\[\mathbb{Z\lbrack}i\sqrt{n}\rbrack = \{ x + iy\sqrt{n}\ |\ x,y \in \mathbb{Z\}}\] \\
\textbf{B3.} Quydagi matritsalar to'plamining qaysi biri matritsalarni qo'shish va ko'paytirish amallarga qarata halqa bo'ladi.
\(M_{2 \times 2}\mathbb{(R) =}\left\{ \begin{pmatrix}
a & 0 \\
b & c
\end{pmatrix}\ :\ a,b,c \in \mathbb{R} \right\}\). \\
\textbf{C1.} Quydagi maydon kengaytmasining har birining bazisini toping. Har bir kengaytmaning darajasi qanday?
\(\mathbb{Q}\left( \sqrt{3} + \sqrt{5} \right)\) da \(\mathbb{Q}\left( \sqrt{2},\sqrt{6} + \sqrt{10} \right)\) \\
\textbf{C2.} Quydagi halqaning barcha ideallarini toping. Bul ideallardan qaysi-biri maksimal bo'ladi?
\[\mathbb{Z}_{18}\] \\
\textbf{C3.} Quyidagi akslantirishni gomomorfizm shartlariga tekshiring.
\[f:x \rightarrow x^{p}\] \\

\end{tabular}
\vspace{1cm}


\begin{tabular}{m{17cm}}
\textbf{27-variant}
\newline

\textbf{T1.} Maydonlarning kengaytmasi. Algebrik element. Algebraik yopilma. \\
\textbf{T2.} Radikallarda yechilishi. \\
\textbf{A1.} \(\mathbb{Q}\) Ratsianal sonlar maydoni ustida minimal ko'phadalrini toping.
\(\sqrt{6 + 3\sqrt{2}}\). \\
\textbf{A2.} Quydagini hisoblang:
\(\mathbb{Z}_{9}\) da \(\left( 7x^{3} + 3x^{2} - x \right) + \left( 6x^{2} - 8x + 4 \right)\) \\
\textbf{A3.} \(\mathbb{Z}_{4}\lbrack x\rbrack\) da \(\deg\ p(x) > 1\) bo'ladigan \(p(x)\) birligini toping. Quydagi ko'phadlarning qaysilari \(\mathbb{Q\lbrack}x\rbrack\) da keltirilmaydigan?
\[x^{4} - 2x^{3} + 2x^{2} + x + 4\] \\
\textbf{B1.} Quydagi maydon bo'ladimi:
\(\mathbb{Z}\left( \sqrt{5} \right) = \left\{ a + b\sqrt{5}:a,b \in \mathbb{Z} \right\}\). \\
\textbf{B2.} Quydagilarning qaysi biri maydon bo'ladi, bunda \(i^{2} = - 1\):
\(\mathbb{Z\lbrack}i\rbrack = \{ x + iy\ |\ x,y \in \mathbb{Z\}}\). \\
\textbf{B3.} Quydagi maydonning berilgan ko'phadlar orqali ajralish maydonini toping.
\(\mathbb{Z}_{3}\) te \(x^{3} + 2x + 2\). \\
\textbf{C1.} Quydagi maydon kengaytmasining har birining bazisini toping. Har bir kengaytmaning darajasi qanday?
\(\mathbb{Q}\) da \(\mathbb{Q}\left( \sqrt{3},\sqrt{6} \right)\) \\
\textbf{C2.} Quydagi halqaning barcha ideallarini toping. Bul ideallardan qaysi-biri maksimal bo'ladi?
\[\mathbb{Q}\] \\
\textbf{C3.} Quyidagi akslantirishni gomomorfizm shartlariga tekshiring. \(f(x) = \sqrt[3]{x}\) \\

\end{tabular}
\vspace{1cm}


\begin{tabular}{m{17cm}}
\textbf{28-variant}
\newline

\textbf{T1.} Chekli maydonlarning strukturasi \\
\textbf{T2.} Keltirilmaydigan ko'phadlar. \\
\textbf{A1.} \(\mathbb{Q}\) Ratsianal sonlar maydoni ustida minimal ko'phadalrini toping.
\(\sqrt{3 + \sqrt{3}}\). \\
\textbf{A2.} Quydagini hisoblang:
\(\mathbb{Z}_{12}\) de \(\left( 5x^{2} + 3x - 2 \right)^{2}\) \\
\textbf{A3.} \(\mathbb{Z}_{4}\lbrack x\rbrack\) da \(\deg\ p(x) > 1\) bo'ladigan \(p(x)\) birligini toping. Quydagi ko'phadlarning qaysilari \(\mathbb{Q\lbrack}x\rbrack\) da keltirilmaydigan?
\[x^{4} - 5x^{3} + 3x - 2\] \\
\textbf{B1.} Quydagi maydonlarning berilgan ko'phadlar orqali ajralish maydonini toping.
\(\mathbb{Q}\) da \(x^{4} - 5x^{2} + 6\). \\
\textbf{B2.} Quydagi halqa bo'ladimi:
\(\mathbb{R\lbrack}\sigma\rbrack = \{ x + \sigma \cdot y\ |\ \sigma^{2} = 0\}\). \\
\textbf{B3.} Quydagi matritsalar to'plamining qaysi biri matritsalarni qo'shish va ko'paytirish amallarga qarata halqa bo'ladi.
\(M_{2 \times 2}\mathbb{(R) =}\left\{ \begin{pmatrix}
a & 0 \\
0 & 0
\end{pmatrix}\ :\ a \in \mathbb{R} \right\}\). \\
\textbf{C1.} Quydagi maydon kengaytmasining har birining bazisini toping. Har bir kengaytmaning darajasi qanday?
\(\mathbb{Q}\) da \(\mathbb{Q}\left( \sqrt{3},\sqrt{5},\sqrt{7} \right)\) \\
\textbf{C2.} Quydagi halqaning barcha ideallarini toping. Bul ideallardan qaysi-biri maksimal bo'ladi?
\(\mathbb{M}_{2}\left( \mathbb{Z} \right)\), elementleri \(\mathbb{Z}\) bol\(g'\)an \(2 \times 2\) matrica \\
\textbf{C3.} Quyidagi akslantirishni gomomorfizm shartlariga tekshiring. \(f\left( \begin{pmatrix}
a & 0 \\
0 & a
\end{pmatrix} \right) = a\) \\

\end{tabular}
\vspace{1cm}


\begin{tabular}{m{17cm}}
\textbf{29-variant}
\newline

\textbf{T1.} Bo'lish algoritmi. \\
\textbf{T2.} Fundamental teoremalar. \\
\textbf{A1.} Quydagi ko'phadlarning barcha nollarini toping:
\(\mathbb{Z}_{2}\) de \(x^{3} + x + 1\). \\
\textbf{A2.} Quydagini hisoblang:
\(\mathbb{Z}_{5}\) te \(\left( 3x^{2} + 3x - 4 \right)\left( 4x^{2} + 2 \right)\) \\
\textbf{A3.} Quydagi ko'phadlarning barcha nollarini toping:
\(\mathbb{Z}_{2}\) de \(x^{3} + x + 1\). \\
\textbf{B1.} Quydagi halqa bo'ladimi:
\(\mathbb{Q}\left( \sqrt[3]{3} \right) = \left\{ a + b\sqrt[3]{3} + c\sqrt[3]{9}:a,b,c \in \mathbb{Q} \right\}\). \\
\textbf{B2.} Quydagi halqa bo'ladimi:
\(\mathbb{R\lbrack}\sigma\rbrack = \{ x + \sigma \cdot y\ |\ \sigma^{2} = 0\}\). \\
\textbf{B3.} Quydagi matritsalar to'plamining qaysi biri matritsalarni qo'shish va ko'paytirish amallarga qarata halqa bo'ladi .
\(M_{2 \times 2}\mathbb{(R) =}\left\{ \begin{pmatrix}
a & b \\
0 & c
\end{pmatrix}\ :\ a,b,c \in \mathbb{R} \right\}\). \\
\textbf{C1.} Quydagi maydon kengaytmasining har birining bazisini toping. Har bir kengaytmaning darajasi qanday?
\(\mathbb{Q}\left( \sqrt{3} + \sqrt{5} \right)\) da \(\mathbb{Q}\left( \sqrt{2},\sqrt{6} + \sqrt{10} \right)\) \\
\textbf{C2.} Quydagi halqaning barcha ideallarini toping. Bul ideallardan qaysi-biri maksimal bo'ladi?
\(\mathbb{M}_{2}\left( \mathbb{Z} \right)\), elementleri \(\mathbb{Z}\) bol\(g'\)an \(2 \times 2\) matrica \\
\textbf{C3.} Quyidagi akslantirishni gomomorfizm shartlariga tekshiring. \(f(a + ib) = \begin{pmatrix}
a & b \\
 - b & a
\end{pmatrix}\) \\

\end{tabular}
\vspace{1cm}


\begin{tabular}{m{17cm}}
\textbf{30-variant}
\newline

\textbf{T1.} Ratsional sonlar maydonini haqiqiy sonlar maydonigacha to'ldirish. \\
\textbf{T2.} Geometrik konstruksiyasi. \\
\textbf{A1.} \(\mathbb{Q}\) Ratsianal sonlar maydoni ustida minimal ko'phadalrini toping.
\(\sqrt{3 - \sqrt{3}}\). \\
\textbf{A2.} Quydagini hisoblang:
\(\mathbb{Z}_{12}\) de \(\left( 5x^{2} + 3x - 4 \right) + \left( 4x^{2} - x + 9 \right)\). \\
\textbf{A3.} \(\mathbb{Z}_{4}\lbrack x\rbrack\) da \(\deg\ p(x) > 1\) bo'ladigan \(p(x)\) birligini toping. Quydagi ko'phadlarning qaysilari \(\mathbb{Q\lbrack}x\rbrack\) da keltirilmaydigan?
\[3x^{5} - 4x^{3} - 6x^{2} + 6\] \\
\textbf{B1.} Quydagi maydon bo'ladimi?
\(\mathbb{Q}\left( \sqrt{5},\sqrt{7} \right) = \left\{ a + b\sqrt{5} + c\sqrt{7} + d\sqrt{35}:a,b,c,d \in \mathbb{Q} \right\}\). \\
\textbf{B2.} Quydagi maydon bo'ladimi:
\(5\mathbb{Z}\). \\
\textbf{B3.} Quydagi matritsalar to'plamining qaysi biri matritsalarni qo'shish va ko'paytirish amallarga qarata halqa bo'ladi.
\[M_{2 \times 2}\mathbb{(R) =}\left\{ \begin{pmatrix}
a & 0 \\
0 & a
\end{pmatrix}\ :\ a \in \mathbb{R} \right\}\] \\
\textbf{C1.} Quydagi maydon kengaytmasining har birining bazisini toping. Har bir kengaytmaning darajasi qanday?
\(\mathbb{Q}\left( \sqrt{3} + \sqrt{5} \right)\) da \(\mathbb{Q}\left( \sqrt{2},\sqrt{6} + \sqrt{10} \right)\) \\
\textbf{C2.} Quydagi halqaning barcha ideallarini toping. Bul ideallardan qaysi-biri maksimal bo'ladi?
\[\mathbb{Z}_{25}\] \\
\textbf{C3.} Quyidagi akslantirishni gomomorfizm shartlariga tekshiring.
\[f\left( a - \sqrt{2}b \right) = a + \sqrt{2}b\] \\

\end{tabular}
\vspace{1cm}


\begin{tabular}{m{17cm}}
\textbf{31-variant}
\newline

\textbf{T1.} Integral sohalar va maydonlar. \\
\textbf{T2.} Maydonlarning ajralishi. \\
\textbf{A1.} Quydagi halqaning qism to'plamlari ideal bo'lishini ko'rsating.
\(R\mathbb{= Z\lbrack}\sqrt{7}\rbrack\), \(I = \{ a + b\sqrt{7}\ |\ \ a,b \in \mathbb{Z,}a - b\ \ juft\ son\}\). \\
\textbf{A2.} Quydagini hisoblang:
\(\mathbb{Z}_{5}\) te \(\left( 3x^{2} + 2x - 4 \right) + \left( 4x^{2} + 2 \right)\) \\
\textbf{A3.} \(\mathbb{Z}_{4}\lbrack x\rbrack\) da \(\deg\ p(x) > 1\) bo'ladigan \(p(x)\) birligini toping. Quydagi ko'phadlarning qaysilari \(\mathbb{Q\lbrack}x\rbrack\) da keltirilmaydigan?
\[5x^{5} - 6x^{4} - 3x^{2} + 9x - 15\] \\
\textbf{B1.} Quydagi halqa bo'ladimi:
\[\mathbb{Z}\left( \sqrt{3} \right) = \left\{ a + b\sqrt{3}:a,b \in \mathbb{Z} \right\}\] \\
\textbf{B2.} Quydagi maydon bo'ladimi, bunda \(i^{2} = - 1\):
\(\mathbb{Q\lbrack}i\sqrt{n}\rbrack = \{ x + iy\sqrt{n}\ |\ x,y \in \mathbb{Q\}}\). \\
\textbf{B3.} Quydagi maydonlarning berilgan ko'phadlar orqali ajralish maydonini toping.
\(\mathbb{Z}_{2}\) da \(x^{2} + 1\). \\
\textbf{C1.} Quydagi maydon kengaytmasining har birining bazisini toping. Har bir kengaytmaning darajasi qanday?
\(\mathbb{Q}\) da \(\mathbb{Q}\left( \sqrt{2},\sqrt[3]{2} \right)\) \\
\textbf{C2.} Quydagi halqaning barcha ideallarini toping. Bul ideallardan qaysi-biri maksimal bo'ladi?
\[\mathbb{Z}_{18}\] \\
\textbf{C3.} Quyidagi akslantirishni gomomorfizm shartlariga tekshiring. \(f\left( \begin{pmatrix}
a & 0 \\
0 & a
\end{pmatrix} \right) = a\) \\

\end{tabular}
\vspace{1cm}


\begin{tabular}{m{17cm}}
\textbf{32-variant}
\newline

\textbf{T1.} Halqalarning gomomorfizmi va ideallar. \\
\textbf{T2.} p-adik sonlar maydoni va ular ustida amallar. \\
\textbf{A1.} Quydagi halqaning qism to'plamlari ideal bo'lishini ko'rsating.
\(R = \mathbb{Z}_{24}\), \(I = \{\overline{0},\overline{8},\overline{16}\}\). \\
\textbf{A2.} Quydagini hisoblang:
\(\mathbb{Z}_{9}\) da \(\left( 7x^{3} + 3x^{2} - x \right) + \left( 6x^{2} - 8x + 4 \right)\) \\
\textbf{A3.} \(\mathbb{Z}_{4}\lbrack x\rbrack\) da \(\deg\ p(x) > 1\) bo'ladigan \(p(x)\) birligini toping. Quydagi ko'phadlarning qaysilari \(\mathbb{Q\lbrack}x\rbrack\) da keltirilmaydigan?
\[x^{4} - 2x^{3} + 2x^{2} + x + 4\] \\
\textbf{B1.} Quydagilarning qaysi biri maydon bo'ladi:
\[5\mathbb{Z}\] \\
\textbf{B2.} Quydagi maydon bo'ladimi, bunda \(i^{2} = - 1\):
\(\mathbb{Q\lbrack}i\sqrt{n}\rbrack = \{ x + iy\sqrt{n}\ |\ x,y \in \mathbb{Q\}}\). \\
\textbf{B3.} Quydagi matritsalar to'plamining qaysi biri matritsalarni qo'shish va ko'paytirish amallarga qarata halqa bo'ladi.
\(M_{2 \times 2}\mathbb{(R) =}\left\{ \begin{pmatrix}
a & 0 \\
0 & b
\end{pmatrix}\ :\ a,b \in \mathbb{R} \right\}\). \\
\textbf{C1.} Quydagi maydon kengaytmasining har birining bazisini toping. Har bir kengaytmaning darajasi qanday?
\(\mathbb{Q}\) da \(\mathbb{Q}\left( \sqrt{2},\sqrt[3]{2} \right)\) \\
\textbf{C2.} Quydagi halqaning barcha ideallarini toping. Bul ideallardan qaysi-biri maksimal bo'ladi?
\[\mathbb{Z}_{25}\] \\
\textbf{C3.} Quyidagi akslantirishni gomomorfizm shartlariga tekshiring.
\[f:\begin{pmatrix}
a & b \\
 - b & a
\end{pmatrix} \rightarrow a + bi\] \\

\end{tabular}
\vspace{1cm}


\begin{tabular}{m{17cm}}
\textbf{33-variant}
\newline

\textbf{T1.} Maksimal va sodda ideallar. \\
\textbf{T2.} p-adik kvadrat tenglamalar. \\
\textbf{A1.} \(\mathbb{Q}\) Ratsianal sonlar maydoni ustida minimal ko'phadalrini toping.
\(\sqrt{2 + \sqrt{3}}\). \\
\textbf{A2.} Quydagini hisoblang:
\(\mathbb{Z}_{12}\) da \(\left( 5x^{2} + 3x - 4 \right) + \left( 4x^{2} - x + 9 \right)\) \\
\textbf{A3.} Quydagi ko'phadlarning barcha nollarini toping:
\(\mathbb{Z}_{12}\) de \(5x^{3} + 4x^{2} - x + 9\); \\
\textbf{B1.} Quydagi maydonlarning berilgan ko'phadlar orqali ajralish maydonini toping.
\(\mathbb{Q}\) da \(x^{4} - 5x^{2} + 6\). \\
\textbf{B2.} Quydagilar halqa bo'ladimi bunda \(i^{2} = - 1\):
\(\mathbb{Q(}i) = \{ x + iy\ |\ x,y \in \mathbb{Q\}}\). \\
\textbf{B3.} Quydagi maydonning berilgan ko'phadlar orqali ajralish maydonini toping.
\(\mathbb{Z}_{3}\) te \(x^{2} + x + 1\). \\
\textbf{C1.} Quydagi maydon kengaytmasining har birining bazisini toping. Har bir kengaytmaning darajasi qanday?
\(\mathbb{Q}\) da \(\mathbb{Q}\left( \sqrt{3},\sqrt{5},\sqrt{7} \right)\) \\
\textbf{C2.} Quydagi halqaning barcha ideallarini toping. Bul ideallardan qaysi-biri maksimal bo'ladi?
\[\mathbb{Q}\] \\
\textbf{C3.} Quyidagi akslantirishni gomomorfizm shartlariga tekshiring. \(f(x) = \sqrt{x}\) \\

\end{tabular}
\vspace{1cm}


\begin{tabular}{m{17cm}}
\textbf{34-variant}
\newline

\textbf{T1.} Halqalar. \\
\textbf{T2.} Ko'phadlarning halqasi. \\
\textbf{A1.} \(\mathbb{Q}\) Ratsianal sonlar maydoni ustida minimal ko'phadalrini toping.
\(\sqrt{2 - \sqrt{2}}\). \\
\textbf{A2.} \(\mathbb{Q}\) Ratsianal sonlar maydoni ustida minimal ko'phadalrini toping.
\(\sqrt{2} + \sqrt{5}\). \\
\textbf{A3.} \(\mathbb{Q}\) Ratsianal sonlar maydoni ustida minimal ko'phadni toping.
\(\sqrt{3} + \sqrt{5}\). \\
\textbf{B1.} Quydagi halqa bo'ladimi:
\(\mathbb{Q}\left( \sqrt{2} \right) = \left\{ a + b\sqrt{2}:a,b \in \mathbb{Q} \right\}\). \\
\textbf{B2.} Quydagilar halqa bo'ladimi bunda \(i^{2} = - 1\):
\(\mathbb{Z(}i\sqrt{n}) = \{ x + iy\sqrt{n}\ |\ x,y \in \mathbb{Z\}}\). \\
\textbf{B3.} Quydagi matritsalar to'plamining qaysi biri matritsalarni qo'shish va ko'paytirish amallarga qarata halqa bo'ladi.
\[M_{2 \times 2}\mathbb{(R) =}\left\{ \begin{pmatrix}
a & 0 \\
0 & - a
\end{pmatrix}\ :\ a \in \mathbb{R} \right\}\] \\
\textbf{C1.} Quydagi maydon kengaytmasining har birining bazisini toping. Har bir kengaytmaning darajasi qanday?
\(\mathbb{Q}\left( \sqrt{5} \right)\) da \(\mathbb{Q}\left( \sqrt{2} + \sqrt{5} \right)\) \\
\textbf{C2.} Quydagi halqaning barcha ideallarini toping. Bul ideallardan qaysi-biri maksimal bo'ladi?
\(\mathbb{M}_{2}\left( \mathbb{Z} \right)\), elementleri \(\mathbb{Z}\) bol\(g'\)an \(2 \times 2\) matrica \\
\textbf{C3.} Quyidagi akslantirishni gomomorfizm shartlariga tekshiring. \(f\left( a + \sqrt{2}b \right) = a + bi\) \\

\end{tabular}
\vspace{1cm}


\begin{tabular}{m{17cm}}
\textbf{35-variant}
\newline

\textbf{T1.} Maydonlarning kengaytmasi. Algebrik element. Algebraik yopilma. \\
\textbf{T2.} Radikallarda yechilishi. \\
\textbf{A1.} \(\mathbb{Q}\) Ratsianal sonlar maydoni ustida minimal ko'phadalrini toping.
\(\sqrt{2 + \sqrt{7}}\). \\
\textbf{A2.} Quydagini hisoblang:
\(\mathbb{Z}_{12}\) de \(\left( 5x^{2} + 3x - 4 \right)\left( 4x^{2} - x + 9 \right)\) \\
\textbf{A3.} \(\mathbb{Q}\) Ratsianal sonlar maydoni ustida minimal ko'phadni toping.
\[\sqrt{5} + \sqrt{7}\] \\
\textbf{B1.} Quydagilarning qaysi biri maydon bo'ladi:
\(\mathbb{Q}\left( \sqrt{11} \right) = \left\{ a + b\sqrt{11}:a,b \in \mathbb{Q} \right\}\). \\
\textbf{B2.} Quydagi halqa bo'ladimi:
\(\mathbb{R\lbrack}\omega\rbrack = \{ x + \omega \cdot y\ |\ \omega^{2} = 1\}\). \\
\textbf{B3.} Quydagi maydonning berilgan ko'phadlar orqali ajralish maydonini toping.
\(\mathbb{Q}\) da \(x^{3} - 3\). \\
\textbf{C1.} Quydagi maydon kengaytmasining har birining bazisini toping. Har bir kengaytmaning darajasi qanday?
\(\mathbb{Q}\left( \sqrt{3} + \sqrt{5} \right)\) da \(\mathbb{Q}\left( \sqrt{2},\sqrt{6} + \sqrt{10} \right)\) \\
\textbf{C2.} Quydagi halqaning barcha ideallarini toping. Bul ideallardan qaysi-biri maksimal bo'ladi?
\(\mathbb{M}_{2}\left( \mathbb{Z} \right)\), elementleri \(\mathbb{Z}\) bol\(g'\)an \(2 \times 2\) matrica \\
\textbf{C3.} Quyidagi akslantirishni gomomorfizm shartlariga tekshiring. \(f(a + ib) = \begin{pmatrix}
a & b \\
 - b & a
\end{pmatrix}\) \\

\end{tabular}
\vspace{1cm}


\begin{tabular}{m{17cm}}
\textbf{36-variant}
\newline

\textbf{T1.} Ratsional sonlar maydonini haqiqiy sonlar maydonigacha to'ldirish. \\
\textbf{T2.} Halqalar. \\
\textbf{A1.} Quydagi ko'phadlarning barcha nollarini toping:
\(\mathbb{Z}_{7}\) de \(5x^{4} + 2x^{2} - 3\); \\
\textbf{A2.} \(\mathbb{Q}\) Ratsianal sonlar maydoni ustida minimal ko'phadalrini toping.
\[\sqrt{2} + \sqrt{5}\] \\
\textbf{A3.} \(\mathbb{Z}_{4}\lbrack x\rbrack\) da \(\deg\ p(x) > 1\) bo'ladigan \(p(x)\) birligini toping. Quydagi ko'phadlarning qaysilari \(\mathbb{Q\lbrack}x\rbrack\) da keltirilmaydigan?
\[3x^{5} - 4x^{3} - 6x^{2} + 6\] \\
\textbf{B1.} Quydagi halqa bo'ladimi:
\(7\mathbb{Z}\). \\
\textbf{B2.} Quydagilar halqa bo'ladimi bunda \(i^{2} = - 1\):
\(\mathbb{Q(}i) = \{ x + iy\ |\ x,y \in \mathbb{Q\}}\). \\
\textbf{B3.} Quydagi matritsalar to'plamining qaysi biri matritsalarni qo'shish va ko'paytirish amallarga qarata halqa bo'ladi
\[M_{2 \times 2}\mathbb{(R) =}\left\{ \begin{pmatrix}
a & b \\
0 & c
\end{pmatrix}\ :\ a,b,c \in \mathbb{R} \right\}\] \\
\textbf{C1.} Quydagi maydon kengaytmasining har birining bazisini toping. Har bir kengaytmaning darajasi qanday?
\(\mathbb{Q}\) da \(\mathbb{Q}\left( \sqrt{3},\sqrt{6} \right)\) \\
\textbf{C2.} Quydagi halqaning barcha ideallarini toping. Bul ideallardan qaysi-biri maksimal bo'ladi?
\(\mathbb{M}_{2}\left( \mathbb{Z} \right)\), elementleri \(\mathbb{Z}\) bol\(g'\)an \(2 \times 2\) matrica \\
\textbf{C3.} Quyidagi akslantirishni gomomorfizm shartlariga tekshiring. \(f(x) = \sqrt[3]{x}\) \\

\end{tabular}
\vspace{1cm}


\begin{tabular}{m{17cm}}
\textbf{37-variant}
\newline

\textbf{T1.} Halqalarning gomomorfizmi va ideallar. \\
\textbf{T2.} Bo'lish algoritmi. \\
\textbf{A1.} Quydagi ko'phadlarning barcha nollarini toping:
\(\mathbb{Z}_{5}\) da \(3x^{3} - 4x^{2} - x + 4\); \\
\textbf{A2.} Quydagini hisoblang:
\(\mathbb{Z}_{5}\) te \(\left( 3x^{2} + 2x - 4 \right) + \left( 4x^{2} + 2 \right)\) \\
\textbf{A3.} \(\mathbb{Z}_{4}\lbrack x\rbrack\) da \(\deg\ p(x) > 1\) bo'ladigan \(p(x)\) birligini toping. Quydagi ko'phadlarning qaysilari \(\mathbb{Q\lbrack}x\rbrack\) da keltirilmaydigan?
\[x^{4} - 5x^{3} + 3x - 2\] \\
\textbf{B1.} Quydagi halqa bo'ladimi:
\(\mathbb{Q(}\sqrt[3]{2}) = \left\{ a + b\sqrt[3]{2}:a,b \in \mathbb{Q} \right\}\).
B2 Quydagilarning qaysi biri maydon bo'ladi, bunda \(i^{2} = - 1\):
\(\mathbb{Q\lbrack}i\rbrack = \{ x + iy\ |\ x,y \in \mathbb{Q\}}\).. \\
\textbf{B2.} \(R\) halqani \(R'\) halqaga o'tkazuvchi gomomorfizmini aniqlang.
\(R\mathbb{= (R,} + , \cdot )\) hám \(R'\mathbb{= (R,} + , \cdot )\). \\
\textbf{B3.} Quydagi maydonning berilgan ko'phadlar orqali ajralish maydonini toping.
\(\mathbb{Q}\) da \(x^{4} + 1\). \\
\textbf{C1.} Quydagi maydon kengaytmasining har birining bazisini toping. Har bir kengaytmaning darajasi qanday?
\(\mathbb{Q}\left( \sqrt{3} + \sqrt{5} \right)\) da \(\mathbb{Q}\left( \sqrt{2},\sqrt{6} + \sqrt{10} \right)\) \\
\textbf{C2.} Quydagi halqaning barcha ideallarini toping. Bul ideallardan qaysi-biri maksimal bo'ladi?
\(\mathbb{M}_{2}\left( \mathbb{Z} \right)\), elementleri \(\mathbb{Z}\) bol\(g'\)an \(2 \times 2\) matrica \\
\textbf{C3.} Quyidagi akslantirishni gomomorfizm shartlariga tekshiring. \(f(a) = a^{n}\) \\

\end{tabular}
\vspace{1cm}


\begin{tabular}{m{17cm}}
\textbf{38-variant}
\newline

\textbf{T1.} p-adik sonlar maydoni va ular ustida amallar. \\
\textbf{T2.} Fundamental teoremalar. \\
\textbf{A1.} \(\mathbb{Q}\) Ratsianal sonlar maydoni ustida minimal ko'phadalrini toping.
\(\sqrt{2 + \sqrt{5}}\). \\
\textbf{A2.} Quydagini hisoblang:
\(\mathbb{Z}_{5}\) te \(\left( 3x^{2} + 3x - 4 \right)\left( 4x^{2} + 2 \right)\) \\
\textbf{A3.} \(\mathbb{Z}_{4}\lbrack x\rbrack\) da \(\deg\ p(x) > 1\) bo'ladigan \(p(x)\) birligini toping. Quydagi ko'phadlarning qaysilari \(\mathbb{Q\lbrack}x\rbrack\) da keltirilmaydigan?
\[5x^{5} - 6x^{4} - 3x^{2} + 9x - 15\] \\
\textbf{B1.} Quydagilar halqa bo'ladimi:
\[\mathbb{Z}_{18}\] \\
\textbf{B2.} \(R\) halqani \(R'\) halqaga o'tkazuvchi gomomorfizmini aniqlang.
\(R = (\mathbb{Z}_{6}, +_{6}, \cdot_{6})\) hám \(R' = (\mathbb{Z}_{10}, +_{10}, \cdot_{10})\). \\
\textbf{B3.} Quydagi maydonlarning berilgan ko'phadlar orqali ajralish maydonini toping.
\(\mathbb{Q}\) da \(x^{4} - 10x^{2} + 21\). \\
\textbf{C1.} Quydagi maydon kengaytmasining har birining bazisini toping. Har bir kengaytmaning darajasi qanday?
\(\mathbb{Q}\left( \sqrt{3} + \sqrt{5} \right)\) da \(\mathbb{Q}\left( \sqrt{2},\sqrt{6} + \sqrt{10} \right)\) \\
\textbf{C2.} Quydagi halqaning barcha ideallarini toping. Bul ideallardan qaysi-biri maksimal bo'ladi?
\(\mathbb{M}_{2}\left( \mathbb{R} \right)\), elementleri \(\mathbb{R}\) bol\(g'\)an \(2 \times 2\) matrica \\
\textbf{C3.} Quyidagi akslantirishni gomomorfizm shartlariga tekshiring.
\[f:x \rightarrow x^{p}\] \\

\end{tabular}
\vspace{1cm}


\begin{tabular}{m{17cm}}
\textbf{39-variant}
\newline

\textbf{T1.} Maksimal va sodda ideallar. \\
\textbf{T2.} Chekli maydonlarning strukturasi \\
\textbf{A1.} \(\mathbb{Q}\) Ratsianal sonlar maydoni ustida minimal ko'phadalrini toping.
\(\sqrt{2 + \sqrt{3}}\). \\
\textbf{A2.} Quydagini hisoblang:
\(\mathbb{Z}_{9}\) da \(\left( 7x^{3} + 3x^{2} - x \right) + \left( 6x^{2} - 8x + 4 \right)\) \\
\textbf{A3.} Quydagi ko'phadlarning barcha nollarini toping:
\(\mathbb{Z}_{5}\) de \(3x^{3} - 4x^{2} - x + 4\); \\
\textbf{B1.} Quydagilarning qaysi biri maydon bo'ladi:
\(\mathbb{Q}\left( \sqrt{11} \right) = \left\{ a + b\sqrt{11}:a,b \in \mathbb{Q} \right\}\). \\
\textbf{B2.} Quydagi maydon bo'ladimi, bunda \(i^{2} = - 1\):
\[\mathbb{Z\lbrack}i\sqrt{n}\rbrack = \{ x + iy\sqrt{n}\ |\ x,y \in \mathbb{Z\}}\] \\
\textbf{B3.} Quydagi maydonlarning berilgan ko'phadlar orqali ajralish maydonini toping.
\(\mathbb{Z}_{5}\) da \(x^{2} + x + 1\). \\
\textbf{C1.} Quydagi maydon kengaytmasining har birining bazisini toping. Har bir kengaytmaning darajasi qanday?
\(\mathbb{Q}\left( \sqrt{5} \right)\) da \(\mathbb{Q}\left( \sqrt{2} + \sqrt{5} \right)\) \\
\textbf{C2.} Quydagi halqaning barcha ideallarini toping. Bul ideallardan qaysi-biri maksimal bo'ladi?
\(\mathbb{M}_{2}\left( \mathbb{Z} \right)\), elementleri \(\mathbb{Z}\) bol\(g'\)an \(2 \times 2\) matrica
T1 Ko'phadlarning halqasi. \\
\textbf{C3.} Quyidagi akslantirishni gomomorfizm shartlariga tekshiring. \(f(x) = e^{x}\) \\

\end{tabular}
\vspace{1cm}


\begin{tabular}{m{17cm}}
\textbf{40-variant}
\newline

\textbf{T1.} p-adik kvadrat tenglamalar. \\
\textbf{T2.} Maydonlarning ajralishi. \\
\textbf{A1.} \(\mathbb{Q}\) Ratsianal sonlar maydoni ustida minimal ko'phadalrini toping.
\(\sqrt{2 + 2\sqrt{2}}\). \\
\textbf{A2.} Quydagini hisoblang:
\(\mathbb{Z}_{12}\) de \(\left( 5x^{2} + 3x - 2 \right)^{2}\) \\
\textbf{A3.} \(\mathbb{Z}_{4}\lbrack x\rbrack\) da \(\deg\ p(x) > 1\) bo'ladigan \(p(x)\) birligini toping. Quydagi ko'phadlarning qaysilari \(\mathbb{Q\lbrack}x\rbrack\) da keltirilmaydigan?
\[3x^{5} - 4x^{3} - 6x^{2} + 6\] \\
\textbf{B1.} Quydagi maydon bo'ladimi:
\[\mathbb{Q}\left( \sqrt[3]{5} \right) = \left\{ a + b\sqrt[3]{5}:a,b,c \in \mathbb{Q} \right\}\] \\
\textbf{B2.} Quydagi maydon bo'ladimi, bunda \(i^{2} = - 1\):
\(\mathbb{Q\lbrack}i\sqrt{n}\rbrack = \{ x + iy\sqrt{n}\ |\ x,y \in \mathbb{Q\}}\). \\
\textbf{B3.} Quydagi maydonning berilgan ko'phadlar orqali ajralish maydonini toping.
\(\mathbb{Q}\) da \(x^{4} - 10x^{2} + 21\). \\
\textbf{C1.} Quydagi maydon kengaytmasining har birining bazisini toping. Har bir kengaytmaning darajasi qanday?
\(\mathbb{Q}\) da \(\mathbb{Q}\left( i,\sqrt{2} + i,\sqrt{3} + i \right)\) \\
\textbf{C2.} Quydagi halqaning barcha ideallarini toping. Bul ideallardan qaysi-biri maksimal bo'ladi?
\[\mathbb{Z}_{18}\] \\
\textbf{C3.} Quyidagi akslantirishni gomomorfizm shartlariga tekshiring.
\[f\left( a - \sqrt{2}b \right) = a + \sqrt{2}b\] \\

\end{tabular}
\vspace{1cm}


\begin{tabular}{m{17cm}}
\textbf{41-variant}
\newline

\textbf{T1.} Ko'phadlarning halqasi. \\
\textbf{T2.} Geometrik konstruksiyasi. \\
\textbf{A1.} \(\mathbb{Q}\) Ratsianal sonlar maydoni ustida minimal ko'phadalrini toping.
\(\sqrt{2 + 2\sqrt{2}}\). \\
\textbf{A2.} Quydagini hisoblang:
\(\mathbb{Z}_{9}\) da \(\left( 7x^{3} + 3x^{2} - x \right) + \left( 6x^{2} - 8x + 4 \right)\) \\
\textbf{A3.} Quydagi ko'phadlarning barcha nollarini toping:
\(\mathbb{Z}_{12}\) de \(5x^{3} + 4x^{2} - x + 9\); \\
\textbf{B1.} Quydagi maydonlarning berilgan ko'phadlar orqali ajralish maydonini toping.
\(\mathbb{Q}\) da \(x^{4} - 5x^{2} + 6\). \\
\textbf{B2.} Quydagi halqa bo'ladimi:
\(\mathbb{R\lbrack}\omega\rbrack = \{ x + \omega \cdot y\ |\ \omega^{2} = 1\}\). \\
\textbf{B3.} Quydagi matritsalar to'plamining qaysi biri matritsalarni qo'shish va ko'paytirish amallarga qarata halqa bo'ladi.
\(M_{2 \times 2}\mathbb{(R) =}\left\{ \begin{pmatrix}
a & 0 \\
0 & b
\end{pmatrix}\ :\ a,b \in \mathbb{R} \right\}\). \\
\textbf{C1.} Quydagi maydon kengaytmasining har birining bazisini toping. Har bir kengaytmaning darajasi qanday?
\(\mathbb{Q}\) da \(\mathbb{Q}\left( \sqrt{2},\sqrt[3]{2} \right)\) \\
\textbf{C2.} Quydagi halqaning barcha ideallarini toping. Bul ideallardan qaysi-biri maksimal bo'ladi?
\(\mathbb{M}_{2}\left( \mathbb{Z} \right)\), elementleri \(\mathbb{Z}\) bol\(g'\)an \(2 \times 2\) matrica \\
\textbf{C3.} Quyidagi akslantirishni gomomorfizm shartlariga tekshiring. \(f(x) = 5^{x}\) \\

\end{tabular}
\vspace{1cm}


\begin{tabular}{m{17cm}}
\textbf{42-variant}
\newline

\textbf{T1.} Keltirilmaydigan ko'phadlar. \\
\textbf{T2.} Maydonlarning avtomorfizmlari. \\
\textbf{A1.} \(\mathbb{Q}\) Ratsianal sonlar maydoni ustida minimal ko'phadalrini toping.
\(\sqrt{2 + \sqrt{3}}\). \\
\textbf{A2.} Quydagini hisoblang:
\(\mathbb{Z}_{9}\) da \(\left( 7x^{3} + 3x^{2} - x \right) + \left( 6x^{2} - 8x + 4 \right)\) \\
\textbf{A3.} \(\mathbb{Z}_{4}\lbrack x\rbrack\) da \(\deg\ p(x) > 1\) bo'ladigan \(p(x)\) birligini toping. Quydagi ko'phadlarning qaysilari \(\mathbb{Q\lbrack}x\rbrack\) da keltirilmaydigan?
\[x^{4} - 5x^{3} + 3x - 2\] \\
\textbf{B1.} \(R\) halqani \(R'\) halqaga o'tkazuvchi gomomorfizmini aniqlang.
\(R = (\mathbb{Z}_{6}, +_{6}, \cdot_{6})\) hám \(R' = (\mathbb{Z}_{10}, +_{10}, \cdot_{10})\). \\
\textbf{B2.} Quydagilarning qaysi biri maydon bo'ladi, bunda \(i^{2} = - 1\):
\(\mathbb{Z\lbrack}i\rbrack = \{ x + iy\ |\ x,y \in \mathbb{Z\}}\). \\
\textbf{B3.} Quydagi matritsalar to'plamining qaysi biri matritsalarni qo'shish va ko'paytirish amallarga qarata halqa bo'ladi.
\[M_{2 \times 2}\mathbb{(R) =}\left\{ \begin{pmatrix}
a & 0 \\
0 & a
\end{pmatrix}\ :\ a \in \mathbb{R} \right\}\] \\
\textbf{C1.} Quydagi maydon kengaytmasining har birining bazisini toping. Har bir kengaytmaning darajasi qanday?
\(\mathbb{Q}\) da \(\mathbb{Q}\left( \sqrt{2},\sqrt[3]{2} \right)\) \\
\textbf{C2.} Quydagi halqaning barcha ideallarini toping. Bul ideallardan qaysi-biri maksimal bo'ladi?
\(\mathbb{M}_{2}\left( \mathbb{Z} \right)\), elementleri \(\mathbb{Z}\) bol\(g'\)an \(2 \times 2\) matrica \\
\textbf{C3.} Quyidagi akslantirishni gomomorfizm shartlariga tekshiring. \(f\left( \begin{pmatrix}
a & 0 \\
0 & a
\end{pmatrix} \right) = a\) \\

\end{tabular}
\vspace{1cm}


\begin{tabular}{m{17cm}}
\textbf{43-variant}
\newline

\textbf{T1.} Integral sohalar va maydonlar. \\
\textbf{T2.} Geometrik konstruksiyasi. \\
\textbf{A1.} Quydagi halqaning qism to'plamlari ideal bo'lishini ko'rsating.
\(R = \mathbb{Z}_{24}\), \(I = \{\overline{0},\overline{8},\overline{16}\}\). \\
\textbf{A2.} Quydagini hisoblang:
\(\mathbb{Z}_{12}\) de \(\left( 5x^{2} + 3x - 2 \right)^{2}\) \\
\textbf{A3.} Quydagi ko'phadlarning barcha nollarini toping:
\(\mathbb{Z}_{2}\) de \(x^{3} + x + 1\). \\
\textbf{B1.} Quydagi maydon bo'ladimi:
\[\mathbb{Q}\left( \sqrt[3]{5} \right) = \left\{ a + b\sqrt[3]{5}:a,b,c \in \mathbb{Q} \right\}\] \\
\textbf{B2.} Quydagi maydon bo'ladimi:
\(5\mathbb{Z}\). \\
\textbf{B3.} Quydagi matritsalar to'plamining qaysi biri matritsalarni qo'shish va ko'paytirish amallarga qarata halqa bo'ladi.
\(M_{2 \times 2}\mathbb{(R) =}\left\{ \begin{pmatrix}
a & 0 \\
b & c
\end{pmatrix}\ :\ a,b,c \in \mathbb{R} \right\}\). \\
\textbf{C1.} Quydagi maydon kengaytmasining har birining bazisini toping. Har bir kengaytmaning darajasi qanday?
\(\mathbb{Q}\) da \(\mathbb{Q}\left( i,\sqrt{2} + i,\sqrt{3} + i \right)\) \\
\textbf{C2.} Quydagi halqaning barcha ideallarini toping. Bul ideallardan qaysi-biri maksimal bo'ladi?
\[\mathbb{Z}_{18}\] \\
\textbf{C3.} Quyidagi akslantirishni gomomorfizm shartlariga tekshiring. \(f(x) = x^{2} + x\) \\

\end{tabular}
\vspace{1cm}


\begin{tabular}{m{17cm}}
\textbf{44-variant}
\newline

\textbf{T1.} Ratsional sonlar maydonini haqiqiy sonlar maydonigacha to'ldirish. \\
\textbf{T2.} Maydonlarning avtomorfizmlari. \\
\textbf{A1.} \(\mathbb{Q}\) Ratsianal sonlar maydoni ustida minimal ko'phadalrini toping.
\(\sqrt{2 - \sqrt{2}}\). \\
\textbf{A2.} Quydagini hisoblang:
\(\mathbb{Z}_{12}\) da \(\left( 5x^{2} + 3x - 4 \right) + \left( 4x^{2} - x + 9 \right)\) \\
\textbf{A3.} \(\mathbb{Z}_{4}\lbrack x\rbrack\) da \(\deg\ p(x) > 1\) bo'ladigan \(p(x)\) birligini toping. Quydagi ko'phadlarning qaysilari \(\mathbb{Q\lbrack}x\rbrack\) da keltirilmaydigan?
\[x^{4} - 5x^{3} + 3x - 2\] \\
\textbf{B1.} Quydagi maydon bo'ladimi:
\(\mathbb{Q(}\sqrt{3}) = \left\{ a + b\sqrt[3]{3}:a,b \in \mathbb{Q} \right\}\). \\
\textbf{B2.} Quydagilar halqa bo'ladimi bunda \(i^{2} = - 1\):
\(\mathbb{Q(}i) = \{ x + iy\ |\ x,y \in \mathbb{Q\}}\). \\
\textbf{B3.} Quydagi maydonlarning berilgan ko'phadlar orqali ajralish maydonini toping.
\(\mathbb{Z}_{2}\) da \(x^{2} + 1\). \\
\textbf{C1.} Quydagi maydon kengaytmasining har birining bazisini toping. Har bir kengaytmaning darajasi qanday?
\(\mathbb{Q}\left( \sqrt{3} + \sqrt{5} \right)\) da \(\mathbb{Q}\left( \sqrt{2},\sqrt{6} + \sqrt{10} \right)\) \\
\textbf{C2.} Quydagi halqaning barcha ideallarini toping. Bul ideallardan qaysi-biri maksimal bo'ladi?
\(\mathbb{M}_{2}\left( \mathbb{Z} \right)\), elementleri \(\mathbb{Z}\) bol\(g'\)an \(2 \times 2\) matrica \\
\textbf{C3.} Quyidagi akslantirishni gomomorfizm shartlariga tekshiring. \(f(x + iy) = x \cdot y\) \\

\end{tabular}
\vspace{1cm}


\begin{tabular}{m{17cm}}
\textbf{45-variant}
\newline

\textbf{T1.} Integral sohalar va maydonlar. \\
\textbf{T2.} Ko'phadlarning halqasi. \\
\textbf{A1.} \(\mathbb{Q}\) Ratsianal sonlar maydoni ustida minimal ko'phadalrini toping.
\(\sqrt{6 + 3\sqrt{2}}\). \\
\textbf{A2.} Quydagini hisoblang:
\(\mathbb{Z}_{12}\) de \(\left( 5x^{2} + 3x - 4 \right)\left( 4x^{2} - x + 9 \right)\) \\
\textbf{A3.} Quydagi ko'phadlarning barcha nollarini toping:
\(\mathbb{Z}_{5}\) de \(3x^{3} - 4x^{2} - x + 4\); \\
\textbf{B1.} Quydagilarning qaysi biri maydon bo'ladi:
\(\mathbb{Q}\left( \sqrt{11} \right) = \left\{ a + b\sqrt{11}:a,b \in \mathbb{Q} \right\}\). \\
\textbf{B2.} Quydagi halqa bo'ladimi:
\(\mathbb{R\lbrack}\sigma\rbrack = \{ x + \sigma \cdot y\ |\ \sigma^{2} = 0\}\). \\
\textbf{B3.} Quydagi matritsalar to'plamining qaysi biri matritsalarni qo'shish va ko'paytirish amallarga qarata halqa bo'ladi
\(M_{2 \times 2}\mathbb{(R) =}\left\{ \begin{pmatrix}
a & 0 \\
b & c
\end{pmatrix}\ :\ a,b,c \in \mathbb{R} \right\}\). \\
\textbf{C1.} Quydagi maydon kengaytmasining har birining bazisini toping. Har bir kengaytmaning darajasi qanday?
\(\mathbb{Q}\) da \(\mathbb{Q}\left( \sqrt{2},i \right)\) \\
\textbf{C2.} Quydagi halqaning barcha ideallarini toping. Bul ideallardan qaysi-biri maksimal bo'ladi?
\[\mathbb{Z}_{25}\] \\
\textbf{C3.} Quyidagi akslantirishni gomomorfizm shartlariga tekshiring. \(f:\begin{pmatrix}
a & b \\
0 & c
\end{pmatrix} \rightarrow a\) \\

\end{tabular}
\vspace{1cm}


\begin{tabular}{m{17cm}}
\textbf{46-variant}
\newline

\textbf{T1.} Radikallarda yechilishi. \\
\textbf{T2.} Halqalar. \\
\textbf{A1.} \(\mathbb{Q}\) Ratsianal sonlar maydoni ustida minimal ko'phadalrini toping.
\(\sqrt{3 - \sqrt{3}}\). \\
\textbf{A2.} Quydagini hisoblang:
\(\mathbb{Z}_{5}\) te \(\left( 3x^{2} + 2x - 4 \right) + \left( 4x^{2} + 2 \right)\) \\
\textbf{A3.} Quydagi halqalarning qism to'plamlari ideal bo'lishini ko'rsating: \(R = \left\{ \begin{pmatrix}
a & b \\
0 & c
\end{pmatrix}\ |\ a,b,c \in \mathbb{Z} \right\}\), \(I = \left\{ \begin{pmatrix}
0 & a \\
0 & 0
\end{pmatrix}\ |\ a \in \mathbb{Z} \right\}\). \\
\textbf{B1.} Quydagi maydon bo'ladimi:
\(\mathbb{Z}\left( \sqrt{5} \right) = \left\{ a + b\sqrt{5}:a,b \in \mathbb{Z} \right\}\). \\
\textbf{B2.} Quydagi halqa bo'ladimi:
\(\mathbb{R\lbrack}\sigma\rbrack = \{ x + \sigma \cdot y\ |\ \sigma^{2} = 0\}\). \\
\textbf{B3.} Quydagi matritsalar to'plamining qaysi biri matritsalarni qo'shish va ko'paytirish amallarga qarata halqa bo'ladi.
\(M_{2 \times 2}\mathbb{(R) =}\left\{ \begin{pmatrix}
a & 0 \\
0 & 0
\end{pmatrix}\ :\ a \in \mathbb{R} \right\}\). \\
\textbf{C1.} Quydagi maydon kengaytmasining har birining bazisini toping. Har bir kengaytmaning darajasi qanday?
\(\mathbb{Q}\left( \sqrt{5} \right)\) da \(\mathbb{Q}\left( \sqrt{2} + \sqrt{5} \right)\) \\
\textbf{C2.} Quydagi halqaning barcha ideallarini toping. Bul ideallardan qaysi-biri maksimal bo'ladi?
\[\mathbb{Q}\] \\
\textbf{C3.} Quyidagi akslantirishni gomomorfizm shartlariga tekshiring. \(f\left( \begin{pmatrix}
a & 0 \\
0 & a
\end{pmatrix} \right) = a\) \\

\end{tabular}
\vspace{1cm}


\begin{tabular}{m{17cm}}
\textbf{47-variant}
\newline

\textbf{T1.} Chekli maydonlarning strukturasi \\
\textbf{T2.} p-adik kvadrat tenglamalar. \\
\textbf{A1.} \(\mathbb{Q}\) Ratsianal sonlar maydoni ustida minimal ko'phadalrini toping.
\(\sqrt{2 + \sqrt{3}}\). \\
\textbf{A2.} Quydagini hisoblang:
\(\mathbb{Z}_{5}\) te \(\left( 3x^{2} + 2x - 4 \right) + \left( 4x^{2} + 2 \right)\) \\
\textbf{A3.} Quydagi ko'phadlarning barcha nollarini toping:
\(\mathbb{Z}_{2}\) de \(x^{3} + x + 1\). \\
\textbf{B1.} Quydagi halqa bo'ladimi:
\(7\mathbb{Z}\). \\
\textbf{B2.} \(R\) halqani \(R'\) halqaga o'tkazuvchi gomomorfizmini aniqlang.
\(R = (\mathbb{Z}_{4}, +_{4}, \cdot_{4})\) hám \(R' = (\mathbb{Z}_{6}, +_{6}, \cdot_{6})\). \\
\textbf{B3.} Quydagi maydonlarning berilgan ko'phadlar orqali ajralish maydonini toping.
\(\mathbb{Z}_{5}\) da \(x^{2} + x + 1\). \\
\textbf{C1.} Quydagi maydon kengaytmasining har birining bazisini toping. Har bir kengaytmaning darajasi qanday?
\(\mathbb{Q}\) da \(\mathbb{Q}\left( \sqrt{3},\sqrt{6} \right)\) \\
\textbf{C2.} Quydagi halqaning barcha ideallarini toping. Bul ideallardan qaysi-biri maksimal bo'ladi?
\(\mathbb{M}_{2}\left( \mathbb{Z} \right)\), elementleri \(\mathbb{Z}\) bol\(g'\)an \(2 \times 2\) matrica \\
\textbf{C3.} Quyidagi akslantirishni gomomorfizm shartlariga tekshiring. \(f(a) = a^{n}\) \\

\end{tabular}
\vspace{1cm}


\begin{tabular}{m{17cm}}
\textbf{48-variant}
\newline

\textbf{T1.} Halqalarning gomomorfizmi va ideallar. \\
\textbf{T2.} Bo'lish algoritmi. \\
\textbf{A1.} Quydagi ko'phadlarning barcha nollarini toping:
\(\mathbb{Z}_{5}\) da \(3x^{3} - 4x^{2} - x + 4\); \\
\textbf{A2.} \(\mathbb{Q}\) Ratsianal sonlar maydoni ustida minimal ko'phadalrini toping.
\(\sqrt{2} + \sqrt{5}\). \\
\textbf{A3.} \(\mathbb{Z}_{4}\lbrack x\rbrack\) da \(\deg\ p(x) > 1\) bo'ladigan \(p(x)\) birligini toping. Quydagi ko'phadlarning qaysilari \(\mathbb{Q\lbrack}x\rbrack\) da keltirilmaydigan?
\[x^{4} - 2x^{3} + 2x^{2} + x + 4\] \\
\textbf{B1.} Quydagi halqa bo'ladimi:
\(\mathbb{Q(}\sqrt[3]{2}) = \left\{ a + b\sqrt[3]{2}:a,b \in \mathbb{Q} \right\}\).
B2 Quydagilarning qaysi biri maydon bo'ladi, bunda \(i^{2} = - 1\):
\(\mathbb{Q\lbrack}i\rbrack = \{ x + iy\ |\ x,y \in \mathbb{Q\}}\).. \\
\textbf{B2.} \(R\) halqani \(R'\) halqaga o'tkazuvchi gomomorfizmini aniqlang.
\(R\mathbb{= (Z,} + , \cdot )\) va \(R' = (\mathbb{Z}_{6}, +_{6}, \cdot_{6})\). \\
\textbf{B3.} Quydagi matritsalar to'plamining qaysi biri matritsalarni qo'shish va ko'paytirish amallarga qarata halqa bo'ladi .
\(M_{2 \times 2}\mathbb{(R) =}\left\{ \begin{pmatrix}
a & b \\
0 & c
\end{pmatrix}\ :\ a,b,c \in \mathbb{R} \right\}\). \\
\textbf{C1.} Quydagi maydon kengaytmasining har birining bazisini toping. Har bir kengaytmaning darajasi qanday?
\(\mathbb{Q}\left( \sqrt{2} \right)\) da \(\mathbb{Q}\left( \sqrt{8} \right)\) \\
\textbf{C2.} Quydagi halqaning barcha ideallarini toping. Bul ideallardan qaysi-biri maksimal bo'ladi?
\(\mathbb{M}_{2}\left( \mathbb{R} \right)\), elementleri \(\mathbb{R}\) bol\(g'\)an \(2 \times 2\) matrica \\
\textbf{C3.} Quyidagi akslantirishni gomomorfizm shartlariga tekshiring. \(f(x) = x^{2} + x\) \\

\end{tabular}
\vspace{1cm}


\begin{tabular}{m{17cm}}
\textbf{49-variant}
\newline

\textbf{T1.} Keltirilmaydigan ko'phadlar. \\
\textbf{T2.} Maksimal va sodda ideallar. \\
\textbf{A1.} Quydagi ko'phadlarning barcha nollarini toping:
\(\mathbb{Z}_{7}\) de \(5x^{4} + 2x^{2} - 3\); \\
\textbf{A2.} Quydagini hisoblang:
\(\mathbb{Z}_{12}\) de \(\left( 5x^{2} + 3x - 2 \right)^{2}\) \\
\textbf{A3.} Quydagi halqalarning qism to'plamlari ideal bo'lishini ko'rsating: \(R = \left\{ \begin{pmatrix}
a & b \\
0 & c \\
 & 
\end{pmatrix}\ |\ a,b,c \in \mathbb{Z} \right\}\), \(I = \left\{ \begin{pmatrix}
0 & b \\
0 & c
\end{pmatrix}\ |\ a \in \mathbb{Z} \right\}\). \\
\textbf{B1.} Quydagi halqa bo'ladimi:
\(\mathbb{Q}\left( \sqrt{2} \right) = \left\{ a + b\sqrt{2}:a,b \in \mathbb{Q} \right\}\). \\
\textbf{B2.} Quydagilar halqa bo'ladimi bunda \(i^{2} = - 1\):
\(\mathbb{Q(}i) = \{ x + iy\ |\ x,y \in \mathbb{Q\}}\). \\
\textbf{B3.} Quydagi maydonning berilgan ko'phadlar orqali ajralish maydonini toping.
\(\mathbb{Q}\) da \(x^{4} - 2\). \\
\textbf{C1.} Quydagi maydon kengaytmasining har birining bazisini toping. Har bir kengaytmaning darajasi qanday?
\(\mathbb{Q}\) da \(\mathbb{Q}\left( \sqrt{3},\sqrt{5},\sqrt{7} \right)\) \\
\textbf{C2.} Quydagi halqaning barcha ideallarini toping. Bul ideallardan qaysi-biri maksimal bo'ladi?
\(\mathbb{M}_{2}\left( \mathbb{Z} \right)\), elementleri \(\mathbb{Z}\) bol\(g'\)an \(2 \times 2\) matrica \\
\textbf{C3.} Quyidagi akslantirishni gomomorfizm shartlariga tekshiring.
\[f:\begin{pmatrix}
a & b \\
 - b & a
\end{pmatrix} \rightarrow a + bi\] \\

\end{tabular}
\vspace{1cm}


\begin{tabular}{m{17cm}}
\textbf{50-variant}
\newline

\textbf{T1.} Fundamental teoremalar. \\
\textbf{T2.} Maydonlarning ajralishi. \\
\textbf{A1.} Quydagi halqaning qism to'plamlari ideal bo'lishini ko'rsating.
\(R = \mathbb{Z}_{28}\), \(I = \{\overline{0},\overline{7},\overline{14},\overline{21}\}\). \\
\textbf{A2.} Quydagini hisoblang:
\(\mathbb{Z}_{12}\) de \(\left( 5x^{2} + 3x - 4 \right) + \left( 4x^{2} - x + 9 \right)\). \\
\textbf{A3.} Quydagi ko'phadlarning barcha nollarini toping:
\(\mathbb{Z}_{2}\) de \(x^{3} + x + 1\). \\
\textbf{B1.} \(R\) halqani \(R'\) halqaga o'tkazuvchi gomomorfizmini aniqlang.
\(R = (\mathbb{Z}_{4}, +_{4}, \cdot_{4})\) hám \(R' = (\mathbb{Z}_{6}, +_{6}, \cdot_{6})\). \\
\textbf{B2.} Quydagilar halqa bo'ladimi bunda \(i^{2} = - 1\):
\(\mathbb{Z(}i\sqrt{n}) = \{ x + iy\sqrt{n}\ |\ x,y \in \mathbb{Z\}}\). \\
\textbf{B3.} Quydagi maydonning berilgan ko'phadlar orqali ajralish maydonini toping.
\(\mathbb{Q}\) da \(x^{4} - 10x^{2} + 21\). \\
\textbf{C1.} Quydagi maydon kengaytmasining har birining bazisini toping. Har bir kengaytmaning darajasi qanday?
\(\mathbb{Q}\left( \sqrt{3} + \sqrt{5} \right)\) da \(\mathbb{Q}\left( \sqrt{2},\sqrt{6} + \sqrt{10} \right)\) \\
\textbf{C2.} Quydagi halqaning barcha ideallarini toping. Bul ideallardan qaysi-biri maksimal bo'ladi?
\[\mathbb{Z}_{25}\] \\
\textbf{C3.} Quyidagi akslantirishni gomomorfizm shartlariga tekshiring. \(f(x) = e^{x}\) \\

\end{tabular}
\vspace{1cm}


\begin{tabular}{m{17cm}}
\textbf{51-variant}
\newline

\textbf{T1.} p-adik sonlar maydoni va ular ustida amallar. \\
\textbf{T2.} Maydonlarning kengaytmasi. Algebrik element. Algebraik yopilma. \\
\textbf{A1.} \(\mathbb{Q}\) Ratsianal sonlar maydoni ustida minimal ko'phadalrini toping.
\(\sqrt{2 + \sqrt{2}}\). \\
\textbf{A2.} Quydagini hisoblang:
\(\mathbb{Z}_{5}\) te \(\left( 3x^{2} + 2x - 4 \right) + \left( 4x^{2} + 2 \right)\) \\
\textbf{A3.} \(\mathbb{Z}_{4}\lbrack x\rbrack\) da \(\deg\ p(x) > 1\) bo'ladigan \(p(x)\) birligini toping. Quydagi ko'phadlarning qaysilari \(\mathbb{Q\lbrack}x\rbrack\) da keltirilmaydigan?
\[3x^{5} - 4x^{3} - 6x^{2} + 6\] \\
\textbf{B1.} Quydagilar halqa bo'ladimi:
\[\mathbb{Z}_{18}\] \\
\textbf{B2.} Quydagi maydon bo'ladimi, bunda \(i^{2} = - 1\):
\(\mathbb{Q\lbrack}i\sqrt{n}\rbrack = \{ x + iy\sqrt{n}\ |\ x,y \in \mathbb{Q\}}\). \\
\textbf{B3.} Quydagi matritsalar to'plamining qaysi biri matritsalarni qo'shish va ko'paytirish amallarga qarata halqa bo'ladi
\[M_{2 \times 2}\mathbb{(R) =}\left\{ \begin{pmatrix}
a & b \\
0 & c
\end{pmatrix}\ :\ a,b,c \in \mathbb{R} \right\}\] \\
\textbf{C1.} Quydagi maydon kengaytmasining har birining bazisini toping. Har bir kengaytmaning darajasi qanday?
\(\mathbb{Q}\left( \sqrt{3} + \sqrt{5} \right)\) da \(\mathbb{Q}\left( \sqrt{2},\sqrt{6} + \sqrt{10} \right)\) \\
\textbf{C2.} Quydagi halqaning barcha ideallarini toping. Bul ideallardan qaysi-biri maksimal bo'ladi?
\[\mathbb{Q}\] \\
\textbf{C3.} Quyidagi akslantirishni gomomorfizm shartlariga tekshiring. \(f:\begin{pmatrix}
a & b \\
0 & c
\end{pmatrix} \rightarrow a\) \\

\end{tabular}
\vspace{1cm}


\begin{tabular}{m{17cm}}
\textbf{52-variant}
\newline

\textbf{T1.} Integral sohalar va maydonlar. \\
\textbf{T2.} p-adik kvadrat tenglamalar. \\
\textbf{A1.} \(\mathbb{Q}\) Ratsianal sonlar maydoni ustida minimal ko'phadalrini toping.
\(\sqrt{2 + \sqrt{5}}\). \\
\textbf{A2.} Quydagini hisoblang:
\(\mathbb{Z}_{12}\) de \(\left( 5x^{2} + 3x - 2 \right)^{2}\) \\
\textbf{A3.} \(\mathbb{Z}_{4}\lbrack x\rbrack\) da \(\deg\ p(x) > 1\) bo'ladigan \(p(x)\) birligini toping. Quydagi ko'phadlarning qaysilari \(\mathbb{Q\lbrack}x\rbrack\) da keltirilmaydigan?
\[5x^{5} - 6x^{4} - 3x^{2} + 9x - 15\] \\
\textbf{B1.} Quydagi halqa bo'ladimi:
\[\mathbb{Q}\left( \sqrt{2},\sqrt{3} \right) = \left\{ a + b\sqrt{2} + c\sqrt{3} + d\sqrt{6}:a,b,c,d \in \mathbb{Q} \right\}\] \\
\textbf{B2.} \(R\) halqani \(R'\) halqaga o'tkazuvchi gomomorfizmini aniqlang.
\(R\mathbb{= (Z,} + , \cdot )\) va \(R' = (\mathbb{Z}_{6}, +_{6}, \cdot_{6})\). \\
\textbf{B3.} Quydagi matritsalar to'plamining qaysi biri matritsalarni qo'shish va ko'paytirish amallarga qarata halqa bo'ladi.
\(M_{2 \times 2}\mathbb{(R) =}\left\{ \begin{pmatrix}
a & b \\
b & d
\end{pmatrix}\ :\ a,b,c,d \in \mathbb{R} \right\}\). \\
\textbf{C1.} Quydagi maydon kengaytmasining har birining bazisini toping. Har bir kengaytmaning darajasi qanday?
\(\mathbb{Q}\left( \sqrt{3} + \sqrt{5} \right)\) da \(\mathbb{Q}\left( \sqrt{2},\sqrt{6} + \sqrt{10} \right)\) \\
\textbf{C2.} Quydagi halqaning barcha ideallarini toping. Bul ideallardan qaysi-biri maksimal bo'ladi?
\(\mathbb{M}_{2}\left( \mathbb{Z} \right)\), elementleri \(\mathbb{Z}\) bol\(g'\)an \(2 \times 2\) matrica \\
\textbf{C3.} Quyidagi akslantirishni gomomorfizm shartlariga tekshiring.
\[f:x \rightarrow x^{p}\] \\

\end{tabular}
\vspace{1cm}


\begin{tabular}{m{17cm}}
\textbf{53-variant}
\newline

\textbf{T1.} Keltirilmaydigan ko'phadlar. \\
\textbf{T2.} Bo'lish algoritmi. \\
\textbf{A1.} \(\mathbb{Q}\) Ratsianal sonlar maydoni ustida minimal ko'phadalrini toping.
\(\sqrt{2 + \sqrt{7}}\). \\
\textbf{A2.} \(\mathbb{Q}\) Ratsianal sonlar maydoni ustida minimal ko'phadalrini toping.
\[\sqrt{2} + \sqrt{5}\] \\
\textbf{A3.} \(\mathbb{Z}_{4}\lbrack x\rbrack\) da \(\deg\ p(x) > 1\) bo'ladigan \(p(x)\) birligini toping. Quydagi ko'phadlarning qaysilari \(\mathbb{Q\lbrack}x\rbrack\) da keltirilmaydigan?
\[x^{4} - 2x^{3} + 2x^{2} + x + 4\] \\
\textbf{B1.} Quydagi halqa bo'ladimi:
\(\mathbb{Z}_{18}\). \\
\textbf{B2.} \(R\) halqani \(R'\) halqaga o'tkazuvchi gomomorfizmini aniqlang.
\(R\mathbb{= (R,} + , \cdot )\) hám \(R'\mathbb{= (R,} + , \cdot )\). \\
\textbf{B3.} Quydagi maydonning berilgan ko'phadlar orqali ajralish maydonini toping.
\(\mathbb{Q}\) da \(x^{4} - 5x^{2} + 21\). \\
\textbf{C1.} Quydagi maydon kengaytmasining har birining bazisini toping. Har bir kengaytmaning darajasi qanday?
\(\mathbb{Q}\left( \sqrt{3} + \sqrt{5} \right)\) da \(\mathbb{Q}\left( \sqrt{2},\sqrt{6} + \sqrt{10} \right)\) \\
\textbf{C2.} Quydagi halqaning barcha ideallarini toping. Bul ideallardan qaysi-biri maksimal bo'ladi?
\[\mathbb{Z}_{18}\] \\
\textbf{C3.} Quyidagi akslantirishni gomomorfizm shartlariga tekshiring. \(f(a + ib) = \begin{pmatrix}
a & b \\
 - b & a
\end{pmatrix}\) \\

\end{tabular}
\vspace{1cm}


\begin{tabular}{m{17cm}}
\textbf{54-variant}
\newline

\textbf{T1.} Halqalar. \\
\textbf{T2.} Fundamental teoremalar. \\
\textbf{A1.} Quydagi halqaning qism to'plamlari ideal bo'lishini ko'rsating.
\(R\mathbb{= Z\lbrack}\sqrt{7}\rbrack\), \(I = \{ a + b\sqrt{7}\ |\ \ a,b \in \mathbb{Z,}a - b\ \ juft\ son\}\). \\
\textbf{A2.} Quydagini hisoblang:
\(\mathbb{Z}_{12}\) de \(\left( 5x^{2} + 3x - 4 \right)\left( 4x^{2} - x + 9 \right)\). \\
\textbf{A3.} Quydagi ko'phadlarning barcha nollarini toping:
\(\mathbb{Z}_{5}\) de \(3x^{3} - 4x^{2} - x + 4\); \\
\textbf{B1.} Quydagi maydon bo'ladimi:
\[\mathbb{Q}\left( \sqrt[3]{5} \right) = \left\{ a + b\sqrt[3]{5}:a,b,c \in \mathbb{Q} \right\}\] \\
\textbf{B2.} Quydagilar halqa bo'ladimi bunda \(i^{2} = - 1\):
\(\mathbb{Q(}i\sqrt{n}) = \{ x + iy\sqrt{n}\ |\ x,y \in \mathbb{Q\}}\). \\
\textbf{B3.} Quydagi maydonning berilgan ko'phadlar orqali ajralish maydonini toping.
\(\mathbb{Q}\) da \(x^{4} + 1\). \\
\textbf{C1.} Quydagi maydon kengaytmasining har birining bazisini toping. Har bir kengaytmaning darajasi qanday?
\(\mathbb{Q}\left( \sqrt{3} + \sqrt{5} \right)\) da \(\mathbb{Q}\left( \sqrt{2},\sqrt{6} + \sqrt{10} \right)\) \\
\textbf{C2.} Quydagi halqaning barcha ideallarini toping. Bul ideallardan qaysi-biri maksimal bo'ladi?
\[\mathbb{Z}_{18}\] \\
\textbf{C3.} Quyidagi akslantirishni gomomorfizm shartlariga tekshiring. \(f(x) = \sqrt{x}\) \\

\end{tabular}
\vspace{1cm}


\begin{tabular}{m{17cm}}
\textbf{55-variant}
\newline

\textbf{T1.} Maydonlarning avtomorfizmlari. \\
\textbf{T2.} Maksimal va sodda ideallar. \\
\textbf{A1.} Quydagi ko'phadlarning barcha nollarini toping:
\(\mathbb{Z}_{2}\) de \(x^{3} + x + 1\). \\
\textbf{A2.} Quydagini hisoblang:
\(\mathbb{Z}_{5}\) te \(\left( 3x^{2} + 3x - 4 \right)\left( 4x^{2} + 2 \right)\) \\
\textbf{A3.} \(\mathbb{Z}_{4}\lbrack x\rbrack\) da \(\deg\ p(x) > 1\) bo'ladigan \(p(x)\) birligini toping. Quydagi ko'phadlarning qaysilari \(\mathbb{Q\lbrack}x\rbrack\) da keltirilmaydigan?
\[5x^{5} - 6x^{4} - 3x^{2} + 9x - 15\] \\
\textbf{B1.} Quydagilarning qaysi biri maydon bo'ladi:
\[5\mathbb{Z}\] \\
\textbf{B2.} Quydagilar halqa bo'ladimi bunda \(i^{2} = - 1\):
\(\mathbb{Q(}i) = \{ x + iy\ |\ x,y \in \mathbb{Q\}}\). \\
\textbf{B3.} Quydagi matritsalar to'plamining qaysi biri matritsalarni qo'shish va ko'paytirish amallarga qarata halqa bo'ladi.
\[M_{2 \times 2}\mathbb{(R) =}\left\{ \begin{pmatrix}
a & 0 \\
0 & - a
\end{pmatrix}\ :\ a \in \mathbb{R} \right\}\] \\
\textbf{C1.} Quydagi maydon kengaytmasining har birining bazisini toping. Har bir kengaytmaning darajasi qanday?
\(\mathbb{Q}\left( \sqrt{3} + \sqrt{5} \right)\) da \(\mathbb{Q}\left( \sqrt{2},\sqrt{6} + \sqrt{10} \right)\) \\
\textbf{C2.} Quydagi halqaning barcha ideallarini toping. Bul ideallardan qaysi-biri maksimal bo'ladi? $\mathbb{Q}$ \\
\textbf{C3.} Quyidagi akslantirishni gomomorfizm shartlariga tekshiring. \(f\left( a + \sqrt{2}b \right) = a + bi\) \\

\end{tabular}
\vspace{1cm}


\begin{tabular}{m{17cm}}
\textbf{56-variant}
\newline

\textbf{T1.} p-adik sonlar maydoni va ular ustida amallar. \\
\textbf{T2.} Halqalarning gomomorfizmi va ideallar. \\
\textbf{A1.} \(\mathbb{Q}\) Ratsianal sonlar maydoni ustida minimal ko'phadalrini toping.
\(\sqrt{2 + \sqrt{3}}\). \\
\textbf{A2.} Quydagini hisoblang:
\(\mathbb{Z}_{5}\) te \(\left( 3x^{2} + 3x - 4 \right)\left( 4x^{2} + 2 \right)\) \\
\textbf{A3.} \(\mathbb{Q}\) Ratsianal sonlar maydoni ustida minimal ko'phadni toping.
\(\sqrt{3} + \sqrt{5}\). \\
\textbf{B1.} Quydagi maydon bo'ladimi?
\(\mathbb{Q}\left( \sqrt{5},\sqrt{7} \right) = \left\{ a + b\sqrt{5} + c\sqrt{7} + d\sqrt{35}:a,b,c,d \in \mathbb{Q} \right\}\). \\
\textbf{B2.} \(R\) halqani \(R'\) halqaga o'tkazuvchi gomomorfizmini aniqlang.
\(R = (\mathbb{Z}_{6}, +_{6}, \cdot_{6})\) hám \(R' = (\mathbb{Z}_{10}, +_{10}, \cdot_{10})\). \\
\textbf{B3.} Quydagi matritsalar to'plamining qaysi biri matritsalarni qo'shish va ko'paytirish amallarga qarata halqa bo'ladi.
\(M_{2 \times 2}\mathbb{(R) =}\left\{ \begin{pmatrix}
a & b \\
 - b & d
\end{pmatrix}\ :\ a,b,c,d \in \mathbb{R} \right\}\). \\
\textbf{C1.} Quydagi maydon kengaytmasining har birining bazisini toping. Har bir kengaytmaning darajasi qanday?
\(\mathbb{Q}\) da \(\mathbb{Q}\left( \sqrt{3},\sqrt{5},\sqrt{7} \right)\) \\
\textbf{C2.} Quydagi halqaning barcha ideallarini toping. Bul ideallardan qaysi-biri maksimal bo'ladi?
\(\mathbb{M}_{2}\left( \mathbb{Z} \right)\), elementleri \(\mathbb{Z}\) bol\(g'\)an \(2 \times 2\) matrica \\
\textbf{C3.} Quyidagi akslantirishni gomomorfizm shartlariga tekshiring. \(f(x) = \sqrt[3]{x}\) \\

\end{tabular}
\vspace{1cm}


\begin{tabular}{m{17cm}}
\textbf{57-variant}
\newline

\textbf{T1.} Maydonlarning ajralishi. \\
\textbf{T2.} Chekli maydonlarning strukturasi \\
\textbf{A1.} \(\mathbb{Q}\) Ratsianal sonlar maydoni ustida minimal ko'phadalrini toping.
\(\sqrt{3 - \sqrt{3}}\). \\
\textbf{A2.} Quydagini hisoblang:
\(\mathbb{Z}_{5}\) te \(\left( 3x^{2} + 3x - 4 \right)\left( 4x^{2} + 2 \right)\) \\
\textbf{A3.} \(\mathbb{Z}_{4}\lbrack x\rbrack\) da \(\deg\ p(x) > 1\) bo'ladigan \(p(x)\) birligini toping. Quydagi ko'phadlarning qaysilari \(\mathbb{Q\lbrack}x\rbrack\) da keltirilmaydigan?
\[3x^{5} - 4x^{3} - 6x^{2} + 6\] \\
\textbf{B1.} Quydagilarning qaysi biri maydon bo'ladi:
\(\mathbb{Q}\left( \sqrt{11} \right) = \left\{ a + b\sqrt{11}:a,b \in \mathbb{Q} \right\}\). \\
\textbf{B2.} \(R\) halqani \(R'\) halqaga o'tkazuvchi gomomorfizmini aniqlang.
\(R\mathbb{= (R,} + , \cdot )\) hám \(R'\mathbb{= (R,} + , \cdot )\). \\
\textbf{B3.} Quydagi maydonlarning berilgan ko'phadlar orqali ajralish maydonini toping.
\(\mathbb{Q}\) da \(x^{4} - 10x^{2} + 21\). \\
\textbf{C1.} Quydagi maydon kengaytmasining har birining bazisini toping. Har bir kengaytmaning darajasi qanday?
\(\mathbb{Q}\) da \(\mathbb{Q}\left( \sqrt{3},\sqrt{6} \right)\) \\
\textbf{C2.} Quydagi halqaning barcha ideallarini toping. Bul ideallardan qaysi-biri maksimal bo'ladi?
\[\mathbb{Q}\] \\
\textbf{C3.} Quyidagi akslantirishni gomomorfizm shartlariga tekshiring.
\[f\left( a - \sqrt{2}b \right) = a + \sqrt{2}b\] \\

\end{tabular}
\vspace{1cm}


\begin{tabular}{m{17cm}}
\textbf{58-variant}
\newline

\textbf{T1.} Maydonlarning kengaytmasi. Algebrik element. Algebraik yopilma. \\
\textbf{T2.} Radikallarda yechilishi. \\
\textbf{A1.} \(\mathbb{Q}\) Ratsianal sonlar maydoni ustida minimal ko'phadalrini toping.
\(\sqrt{3 + \sqrt{3}}\). \\
\textbf{A2.} Quydagini hisoblang:
\(\mathbb{Z}_{5}\) te \(\left( 3x^{2} + 2x - 4 \right) + \left( 4x^{2} + 2 \right)\) \\
\textbf{A3.} \(\mathbb{Q}\) Ratsianal sonlar maydoni ustida minimal ko'phadni toping.
\[\sqrt{5} + \sqrt{7}\] \\
\textbf{B1.} Quydagi maydonlarning berilgan ko'phadlar orqali ajralish maydonini toping.
\(\mathbb{Q}\) da \(x^{4} - 5x^{2} + 6\). \\
\textbf{B2.} Quydagi maydon bo'ladimi, bunda \(i^{2} = - 1\):
\(\mathbb{Q\lbrack}i\sqrt{n}\rbrack = \{ x + iy\sqrt{n}\ |\ x,y \in \mathbb{Q\}}\). \\
\textbf{B3.} Quydagi maydonning berilgan ko'phadlar orqali ajralish maydonini toping.
\(\mathbb{Q}\) da \(x^{3} - 3\). \\
\textbf{C1.} Quydagi maydon kengaytmasining har birining bazisini toping. Har bir kengaytmaning darajasi qanday?
\(\mathbb{Q}\left( \sqrt{5} \right)\) da \(\mathbb{Q}\left( \sqrt{2} + \sqrt{5} \right)\) \\
\textbf{C2.} Quydagi halqaning barcha ideallarini toping. Bul ideallardan qaysi-biri maksimal bo'ladi?
\(\mathbb{M}_{2}\left( \mathbb{Z} \right)\), elementleri \(\mathbb{Z}\) bol\(g'\)an \(2 \times 2\) matrica \\
\textbf{C3.} Quyidagi akslantirishni gomomorfizm shartlariga tekshiring. \(f(x) = 5^{x}\) \\

\end{tabular}
\vspace{1cm}


\begin{tabular}{m{17cm}}
\textbf{59-variant}
\newline

\textbf{T1.} Ratsional sonlar maydonini haqiqiy sonlar maydonigacha to'ldirish. \\
\textbf{T2.} Ko'phadlarning halqasi. \\
\textbf{A1.} \(\mathbb{Q}\) Ratsianal sonlar maydoni ustida minimal ko'phadalrini toping.
\(\sqrt{3} + \sqrt{7}\). \\
\textbf{A2.} Quydagini hisoblang:
\(\mathbb{Z}_{9}\) da \(\left( 7x^{3} + 3x^{2} - x \right) + \left( 6x^{2} - 8x + 4 \right)\) \\
\textbf{A3.} \(\mathbb{Z}_{4}\lbrack x\rbrack\) da \(\deg\ p(x) > 1\) bo'ladigan \(p(x)\) birligini toping. Quydagi ko'phadlarning qaysilari \(\mathbb{Q\lbrack}x\rbrack\) da keltirilmaydigan?
\[3x^{5} - 4x^{3} - 6x^{2} + 6\] \\
\textbf{B1.} Quydagi halqa bo'ladimi:
\(\mathbb{Q}\left( \sqrt[3]{3} \right) = \left\{ a + b\sqrt[3]{3} + c\sqrt[3]{9}:a,b,c \in \mathbb{Q} \right\}\). \\
\textbf{B2.} \(R\) halqani \(R'\) halqaga o'tkazuvchi gomomorfizmini aniqlang.
\(R = (\mathbb{Z}_{4}, +_{4}, \cdot_{4})\) hám \(R' = (\mathbb{Z}_{6}, +_{6}, \cdot_{6})\). \\
\textbf{B3.} Quydagi maydonning berilgan ko'phadlar orqali ajralish maydonini toping.
\(\mathbb{Z}_{3}\) te \(x^{2} + x + 1\). \\
\textbf{C1.} Quydagi maydon kengaytmasining har birining bazisini toping. Har bir kengaytmaning darajasi qanday?
\(\mathbb{Q}\) da \(\mathbb{Q}\left( \sqrt{2},i \right)\) \\
\textbf{C2.} Quydagi halqaning barcha ideallarini toping. Bul ideallardan qaysi-biri maksimal bo'ladi?
\(\mathbb{M}_{2}\left( \mathbb{Z} \right)\), elementleri \(\mathbb{Z}\) bol\(g'\)an \(2 \times 2\) matrica
T1 Ko'phadlarning halqasi. \\
\textbf{C3.} Quyidagi akslantirishni gomomorfizm shartlariga tekshiring. \(f\left( \begin{pmatrix}
a & 0 \\
0 & a
\end{pmatrix} \right) = a\) \\

\end{tabular}
\vspace{1cm}


\begin{tabular}{m{17cm}}
\textbf{60-variant}
\newline

\textbf{T1.} Geometrik konstruksiyasi. \\
\textbf{T2.} p-adik sonlar maydoni va ular ustida amallar. \\
\textbf{A1.} Quydagi ko'phadlarning barcha nollarini toping:
\(\mathbb{Z}_{12}\) de \(5x^{3} + 4x^{2} - x + 9\); \\
\textbf{A2.} Quydagini hisoblang:
\(\mathbb{Z}_{5}\) te \(\left( 3x^{2} + 2x - 4 \right) + \left( 4x^{2} + 2 \right)\) \\
\textbf{A3.} \(\mathbb{Q}\) Ratsianal sonlar maydoni ustida minimal ko'phadni toping.
\[\sqrt{2} + \sqrt{3}\] \\
\textbf{B1.} Quydagi halqa bo'ladimi:
\[\mathbb{Z}\left( \sqrt{3} \right) = \left\{ a + b\sqrt{3}:a,b \in \mathbb{Z} \right\}\] \\
\textbf{B2.} Quydagilar halqa bo'ladimi bunda \(i^{2} = - 1\):
\(\mathbb{Q(}i) = \{ x + iy\ |\ x,y \in \mathbb{Q\}}\). \\
\textbf{B3.} Quydagi maydonning berilgan ko'phadlar orqali ajralish maydonini toping.
\(\mathbb{Z}_{3}\) te \(x^{3} + 2x + 2\). \\
\textbf{C1.} Quydagi maydon kengaytmasining har birining bazisini toping. Har bir kengaytmaning darajasi qanday?
\(\mathbb{Q}\) da \(\mathbb{Q}\left( \sqrt{3},\sqrt{6} \right)\). \\
\textbf{C2.} Quydagi halqaning barcha ideallarini toping. Bul ideallardan qaysi-biri maksimal bo'ladi?
\(\mathbb{M}_{2}\left( \mathbb{Z} \right)\), elementleri \(\mathbb{Z}\) bol\(g'\)an \(2 \times 2\) matrica \\
\textbf{C3.} Quyidagi akslantirishni gomomorfizm shartlariga tekshiring. \(f(x + iy) = x \cdot y\) \\

\end{tabular}
\vspace{1cm}


\begin{tabular}{m{17cm}}
\textbf{61-variant}
\newline

\textbf{T1.} Ko'phadlarning halqasi. \\
\textbf{T2.} Halqalarning gomomorfizmi va ideallar. \\
\textbf{A1.} \(\mathbb{Q}\) Ratsianal sonlar maydoni ustida minimal ko'phadalrini toping.
\(\sqrt{6 + 3\sqrt{2}}\). \\
\textbf{A2.} Quydagini hisoblang:
\(\mathbb{Z}_{9}\) da \(\left( 7x^{3} + 3x^{2} - x \right) + \left( 6x^{2} - 8x + 4 \right)\) \\
\textbf{A3.} Quydagi ko'phadlarning barcha nollarini toping:
\(\mathbb{Z}_{5}\) de \(3x^{3} - 4x^{2} - x + 4\); \\
\textbf{B1.} Quydagi halqa bo'ladimi:
\(\mathbb{Z}_{18}\). \\
\textbf{B2.} Quydagi maydon bo'ladimi:
\(5\mathbb{Z}\). \\
\textbf{B3.} Quydagi maydonning berilgan ko'phadlar orqali ajralish maydonini toping.
\(\mathbb{Q}\) da \(x^{4} - 10x^{2} + 21\). \\
\textbf{C1.} Quydagi maydon kengaytmasining har birining bazisini toping. Har bir kengaytmaning darajasi qanday?
\(\mathbb{Q}\left( \sqrt{5} \right)\) da \(\mathbb{Q}\left( \sqrt{2} + \sqrt{5} \right)\) \\
\textbf{C2.} Quydagi halqaning barcha ideallarini toping. Bul ideallardan qaysi-biri maksimal bo'ladi?
\[\mathbb{Z}_{25}\] \\
\textbf{C3.} Quyidagi akslantirishni gomomorfizm shartlariga tekshiring. \(f(x) = x^{2} + x\) \\

\end{tabular}
\vspace{1cm}


\begin{tabular}{m{17cm}}
\textbf{62-variant}
\newline

\textbf{T1.} Geometrik konstruksiyasi. \\
\textbf{T2.} Integral sohalar va maydonlar. \\
\textbf{A1.} Quydagi ko'phadlarning barcha nollarini toping:
\(\mathbb{Z}_{2}\) de \(x^{3} + x + 1\). \\
\textbf{A2.} Quydagini hisoblang:
\(\mathbb{Z}_{12}\) de \(\left( 5x^{2} + 3x - 2 \right)^{2}\) \\
\textbf{A3.} \(\mathbb{Z}_{4}\lbrack x\rbrack\) da \(\deg\ p(x) > 1\) bo'ladigan \(p(x)\) birligini toping. Quydagi ko'phadlarning qaysilari \(\mathbb{Q\lbrack}x\rbrack\) da keltirilmaydigan?
\[x^{4} - 2x^{3} + 2x^{2} + x + 4\] \\
\textbf{B1.} Quydagi maydon bo'ladimi:
\(\mathbb{Q(}\sqrt{3}) = \left\{ a + b\sqrt[3]{3}:a,b \in \mathbb{Q} \right\}\). \\
\textbf{B2.} Quydagilarning qaysi biri maydon bo'ladi, bunda \(i^{2} = - 1\):
\(\mathbb{Z\lbrack}i\rbrack = \{ x + iy\ |\ x,y \in \mathbb{Z\}}\). \\
\textbf{B3.} Quydagi matritsalar to'plamining qaysi biri matritsalarni qo'shish va ko'paytirish amallarga qarata halqa bo'ladi
\[M_{2 \times 2}\mathbb{(R) =}\left\{ \begin{pmatrix}
a & b \\
0 & c
\end{pmatrix}\ :\ a,b,c \in \mathbb{R} \right\}\] \\
\textbf{C1.} Quydagi maydon kengaytmasining har birining bazisini toping. Har bir kengaytmaning darajasi qanday?
\(\mathbb{Q}\) da \(\mathbb{Q}\left( \sqrt{2},\sqrt[3]{2} \right)\) \\
\textbf{C2.} Quydagi halqaning barcha ideallarini toping. Bul ideallardan qaysi-biri maksimal bo'ladi?
\(\mathbb{M}_{2}\left( \mathbb{Z} \right)\), elementleri \(\mathbb{Z}\) bol\(g'\)an \(2 \times 2\) matrica \\
\textbf{C3.} Quyidagi akslantirishni gomomorfizm shartlariga tekshiring. \(f(a) = a^{n}\) \\

\end{tabular}
\vspace{1cm}


\begin{tabular}{m{17cm}}
\textbf{63-variant}
\newline

\textbf{T1.} Chekli maydonlarning strukturasi \\
\textbf{T2.} Bo'lish algoritmi. \\
\textbf{A1.} \(\mathbb{Q}\) Ratsianal sonlar maydoni ustida minimal ko'phadalrini toping.
\(\sqrt{2 + \sqrt{3}}\). \\
\textbf{A2.} Quydagini hisoblang:
\(\mathbb{Z}_{12}\) de \(\left( 5x^{2} + 3x - 4 \right)\left( 4x^{2} - x + 9 \right)\). \\
\textbf{A3.} Quydagi ko'phadlarning barcha nollarini toping:
\(\mathbb{Z}_{5}\) de \(3x^{3} - 4x^{2} - x + 4\); \\
\textbf{B1.} Quydagi maydon bo'ladimi:
\(\mathbb{Z}\left( \sqrt{5} \right) = \left\{ a + b\sqrt{5}:a,b \in \mathbb{Z} \right\}\). \\
\textbf{B2.} Quydagilar halqa bo'ladimi bunda \(i^{2} = - 1\):
\(\mathbb{Z(}i\sqrt{n}) = \{ x + iy\sqrt{n}\ |\ x,y \in \mathbb{Z\}}\). \\
\textbf{B3.} Quydagi maydonning berilgan ko'phadlar orqali ajralish maydonini toping.
\(\mathbb{Q}\) da \(x^{4} - 2\). \\
\textbf{C1.} Quydagi maydon kengaytmasining har birining bazisini toping. Har bir kengaytmaning darajasi qanday?
\(\mathbb{Q}\left( \sqrt{3} + \sqrt{5} \right)\) da \(\mathbb{Q}\left( \sqrt{2},\sqrt{6} + \sqrt{10} \right)\) \\
\textbf{C2.} Quydagi halqaning barcha ideallarini toping. Bul ideallardan qaysi-biri maksimal bo'ladi?
\[\mathbb{Q}\] \\
\textbf{C3.} Quyidagi akslantirishni gomomorfizm shartlariga tekshiring. \(f(x) = 5^{x}\) \\

\end{tabular}
\vspace{1cm}


\begin{tabular}{m{17cm}}
\textbf{64-variant}
\newline

\textbf{T1.} Halqalar. \\
\textbf{T2.} Maydonlarning kengaytmasi. Algebrik element. Algebraik yopilma. \\
\textbf{A1.} Quydagi ko'phadlarning barcha nollarini toping:
\(\mathbb{Z}_{5}\) da \(3x^{3} - 4x^{2} - x + 4\); \\
\textbf{A2.} Quydagini hisoblang:
\(\mathbb{Z}_{12}\) de \(\left( 5x^{2} + 3x - 4 \right)\left( 4x^{2} - x + 9 \right)\) \\
\textbf{A3.} \(\mathbb{Q}\) Ratsianal sonlar maydoni ustida minimal ko'phadni toping.
\(\sqrt{3} + \sqrt{5}\). \\
\textbf{B1.} Quydagi halqa bo'ladimi:
\(\mathbb{Q}\left( \sqrt[3]{3} \right) = \left\{ a + b\sqrt[3]{3} + c\sqrt[3]{9}:a,b,c \in \mathbb{Q} \right\}\). \\
\textbf{B2.} Quydagi halqa bo'ladimi:
\(\mathbb{R\lbrack}\omega\rbrack = \{ x + \omega \cdot y\ |\ \omega^{2} = 1\}\). \\
\textbf{B3.} Quydagi matritsalar to'plamining qaysi biri matritsalarni qo'shish va ko'paytirish amallarga qarata halqa bo'ladi
\(M_{2 \times 2}\mathbb{(R) =}\left\{ \begin{pmatrix}
a & 0 \\
b & c
\end{pmatrix}\ :\ a,b,c \in \mathbb{R} \right\}\). \\
\textbf{C1.} Quydagi maydon kengaytmasining har birining bazisini toping. Har bir kengaytmaning darajasi qanday?
\(\mathbb{Q}\) da \(\mathbb{Q}\left( \sqrt{3},\sqrt{5},\sqrt{7} \right)\) \\
\textbf{C2.} Quydagi halqaning barcha ideallarini toping. Bul ideallardan qaysi-biri maksimal bo'ladi?
\[\mathbb{Z}_{18}\] \\
\textbf{C3.} Quyidagi akslantirishni gomomorfizm shartlariga tekshiring. \(f\left( \begin{pmatrix}
a & 0 \\
0 & a
\end{pmatrix} \right) = a\) \\

\end{tabular}
\vspace{1cm}


\begin{tabular}{m{17cm}}
\textbf{65-variant}
\newline

\textbf{T1.} Radikallarda yechilishi. \\
\textbf{T2.} Keltirilmaydigan ko'phadlar. \\
\textbf{A1.} Quydagi halqaning qism to'plamlari ideal bo'lishini ko'rsating.
\(R = \mathbb{Z}_{24}\), \(I = \{\overline{0},\overline{8},\overline{16}\}\). \\
\textbf{A2.} Quydagini hisoblang:
\(\mathbb{Z}_{5}\) te \(\left( 3x^{2} + 2x - 4 \right) + \left( 4x^{2} + 2 \right)\) \\
\textbf{A3.} Quydagi halqalarning qism to'plamlari ideal bo'lishini ko'rsating: \(R = \left\{ \begin{pmatrix}
a & b \\
0 & c
\end{pmatrix}\ |\ a,b,c \in \mathbb{Z} \right\}\), \(I = \left\{ \begin{pmatrix}
0 & a \\
0 & 0
\end{pmatrix}\ |\ a \in \mathbb{Z} \right\}\). \\
\textbf{B1.} \(R\) halqani \(R'\) halqaga o'tkazuvchi gomomorfizmini aniqlang.
\(R = (\mathbb{Z}_{4}, +_{4}, \cdot_{4})\) hám \(R' = (\mathbb{Z}_{6}, +_{6}, \cdot_{6})\). \\
\textbf{B2.} \(R\) halqani \(R'\) halqaga o'tkazuvchi gomomorfizmini aniqlang.
\(R\mathbb{= (Z,} + , \cdot )\) va \(R' = (\mathbb{Z}_{6}, +_{6}, \cdot_{6})\). \\
\textbf{B3.} Quydagi maydonning berilgan ko'phadlar orqali ajralish maydonini toping.
\(\mathbb{Q}\) da \(x^{4} + 1\). \\
\textbf{C1.} Quydagi maydon kengaytmasining har birining bazisini toping. Har bir kengaytmaning darajasi qanday?
\(\mathbb{Q}\left( \sqrt{3} + \sqrt{5} \right)\) da \(\mathbb{Q}\left( \sqrt{2},\sqrt{6} + \sqrt{10} \right)\) \\
\textbf{C2.} Quydagi halqaning barcha ideallarini toping. Bul ideallardan qaysi-biri maksimal bo'ladi?
\(\mathbb{M}_{2}\left( \mathbb{Z} \right)\), elementleri \(\mathbb{Z}\) bol\(g'\)an \(2 \times 2\) matrica \\
\textbf{C3.} Quyidagi akslantirishni gomomorfizm shartlariga tekshiring. \(f\left( \begin{pmatrix}
a & 0 \\
0 & a
\end{pmatrix} \right) = a\) \\

\end{tabular}
\vspace{1cm}


\begin{tabular}{m{17cm}}
\textbf{66-variant}
\newline

\textbf{T1.} Ratsional sonlar maydonini haqiqiy sonlar maydonigacha to'ldirish. \\
\textbf{T2.} Maydonlarning ajralishi. \\
\textbf{A1.} Quydagi halqaning qism to'plamlari ideal bo'lishini ko'rsating.
\(R\mathbb{= Z\lbrack}\sqrt{7}\rbrack\), \(I = \{ a + b\sqrt{7}\ |\ \ a,b \in \mathbb{Z,}a - b\ \ juft\ son\}\). \\
\textbf{A2.} Quydagini hisoblang:
\(\mathbb{Z}_{9}\) da \(\left( 7x^{3} + 3x^{2} - x \right) + \left( 6x^{2} - 8x + 4 \right)\) \\
\textbf{A3.} \(\mathbb{Z}_{4}\lbrack x\rbrack\) da \(\deg\ p(x) > 1\) bo'ladigan \(p(x)\) birligini toping. Quydagi ko'phadlarning qaysilari \(\mathbb{Q\lbrack}x\rbrack\) da keltirilmaydigan?
\[x^{4} - 2x^{3} + 2x^{2} + x + 4\] \\
\textbf{B1.} Quydagilarning qaysi biri maydon bo'ladi:
\[5\mathbb{Z}\] \\
\textbf{B2.} \(R\) halqani \(R'\) halqaga o'tkazuvchi gomomorfizmini aniqlang.
\(R = (\mathbb{Z}_{6}, +_{6}, \cdot_{6})\) hám \(R' = (\mathbb{Z}_{10}, +_{10}, \cdot_{10})\). \\
\textbf{B3.} Quydagi matritsalar to'plamining qaysi biri matritsalarni qo'shish va ko'paytirish amallarga qarata halqa bo'ladi.
\[M_{2 \times 2}\mathbb{(R) =}\left\{ \begin{pmatrix}
a & 0 \\
0 & - a
\end{pmatrix}\ :\ a \in \mathbb{R} \right\}\] \\
\textbf{C1.} Quydagi maydon kengaytmasining har birining bazisini toping. Har bir kengaytmaning darajasi qanday?
\(\mathbb{Q}\) da \(\mathbb{Q}\left( \sqrt{3},\sqrt{6} \right)\) \\
\textbf{C2.} Quydagi halqaning barcha ideallarini toping. Bul ideallardan qaysi-biri maksimal bo'ladi?
\[\mathbb{Q}\] \\
\textbf{C3.} Quyidagi akslantirishni gomomorfizm shartlariga tekshiring. \(f\left( a + \sqrt{2}b \right) = a + bi\) \\

\end{tabular}
\vspace{1cm}


\begin{tabular}{m{17cm}}
\textbf{67-variant}
\newline

\textbf{T1.} Maydonlarning avtomorfizmlari. \\
\textbf{T2.} Fundamental teoremalar. \\
\textbf{A1.} \(\mathbb{Q}\) Ratsianal sonlar maydoni ustida minimal ko'phadalrini toping.
\(\sqrt{3 - \sqrt{3}}\). \\
\textbf{A2.} Quydagini hisoblang:
\(\mathbb{Z}_{12}\) da \(\left( 5x^{2} + 3x - 4 \right) + \left( 4x^{2} - x + 9 \right)\) \\
\textbf{A3.} \(\mathbb{Z}_{4}\lbrack x\rbrack\) da \(\deg\ p(x) > 1\) bo'ladigan \(p(x)\) birligini toping. Quydagi ko'phadlarning qaysilari \(\mathbb{Q\lbrack}x\rbrack\) da keltirilmaydigan?
\[x^{4} - 5x^{3} + 3x - 2\] \\
\textbf{B1.} Quydagi maydon bo'ladimi:
\[\mathbb{Q}\left( \sqrt[3]{5} \right) = \left\{ a + b\sqrt[3]{5}:a,b,c \in \mathbb{Q} \right\}\] \\
\textbf{B2.} Quydagi maydon bo'ladimi, bunda \(i^{2} = - 1\):
\[\mathbb{Z\lbrack}i\sqrt{n}\rbrack = \{ x + iy\sqrt{n}\ |\ x,y \in \mathbb{Z\}}\] \\
\textbf{B3.} Quydagi maydonning berilgan ko'phadlar orqali ajralish maydonini toping.
\(\mathbb{Q}\) da \(x^{4} - 5x^{2} + 21\). \\
\textbf{C1.} Quydagi maydon kengaytmasining har birining bazisini toping. Har bir kengaytmaning darajasi qanday?
\(\mathbb{Q}\left( \sqrt{3} + \sqrt{5} \right)\) da \(\mathbb{Q}\left( \sqrt{2},\sqrt{6} + \sqrt{10} \right)\) \\
\textbf{C2.} Quydagi halqaning barcha ideallarini toping. Bul ideallardan qaysi-biri maksimal bo'ladi?
\(\mathbb{M}_{2}\left( \mathbb{R} \right)\), elementleri \(\mathbb{R}\) bol\(g'\)an \(2 \times 2\) matrica \\
\textbf{C3.} Quyidagi akslantirishni gomomorfizm shartlariga tekshiring.
\[f:x \rightarrow x^{p}\] \\

\end{tabular}
\vspace{1cm}


\begin{tabular}{m{17cm}}
\textbf{68-variant}
\newline

\textbf{T1.} Maksimal va sodda ideallar. \\
\textbf{T2.} p-adik kvadrat tenglamalar. \\
\textbf{A1.} \(\mathbb{Q}\) Ratsianal sonlar maydoni ustida minimal ko'phadalrini toping.
\(\sqrt{2 + \sqrt{3}}\). \\
\textbf{A2.} Quydagini hisoblang:
\(\mathbb{Z}_{5}\) te \(\left( 3x^{2} + 2x - 4 \right) + \left( 4x^{2} + 2 \right)\) \\
\textbf{A3.} Quydagi ko'phadlarning barcha nollarini toping:
\(\mathbb{Z}_{2}\) de \(x^{3} + x + 1\). \\
\textbf{B1.} Quydagi halqa bo'ladimi:
\(7\mathbb{Z}\). \\
\textbf{B2.} Quydagilar halqa bo'ladimi bunda \(i^{2} = - 1\):
\(\mathbb{Q(}i\sqrt{n}) = \{ x + iy\sqrt{n}\ |\ x,y \in \mathbb{Q\}}\). \\
\textbf{B3.} Quydagi maydonning berilgan ko'phadlar orqali ajralish maydonini toping.
\(\mathbb{Z}_{3}\) te \(x^{3} + 2x + 2\). \\
\textbf{C1.} Quydagi maydon kengaytmasining har birining bazisini toping. Har bir kengaytmaning darajasi qanday?
\(\mathbb{Q}\left( \sqrt{3} + \sqrt{5} \right)\) da \(\mathbb{Q}\left( \sqrt{2},\sqrt{6} + \sqrt{10} \right)\) \\
\textbf{C2.} Quydagi halqaning barcha ideallarini toping. Bul ideallardan qaysi-biri maksimal bo'ladi?
\(\mathbb{M}_{2}\left( \mathbb{Z} \right)\), elementleri \(\mathbb{Z}\) bol\(g'\)an \(2 \times 2\) matrica \\
\textbf{C3.} Quyidagi akslantirishni gomomorfizm shartlariga tekshiring. \(f(x) = \sqrt[3]{x}\) \\

\end{tabular}
\vspace{1cm}


\begin{tabular}{m{17cm}}
\textbf{69-variant}
\newline

\textbf{T1.} Keltirilmaydigan ko'phadlar. \\
\textbf{T2.} Maydonlarning kengaytmasi. Algebrik element. Algebraik yopilma. \\
\textbf{A1.} \(\mathbb{Q}\) Ratsianal sonlar maydoni ustida minimal ko'phadalrini toping.
\(\sqrt{2 + \sqrt{7}}\). \\
\textbf{A2.} Quydagini hisoblang:
\(\mathbb{Z}_{5}\) te \(\left( 3x^{2} + 2x - 4 \right) + \left( 4x^{2} + 2 \right)\) \\
\textbf{A3.} \(\mathbb{Z}_{4}\lbrack x\rbrack\) da \(\deg\ p(x) > 1\) bo'ladigan \(p(x)\) birligini toping. Quydagi ko'phadlarning qaysilari \(\mathbb{Q\lbrack}x\rbrack\) da keltirilmaydigan?
\[x^{4} - 5x^{3} + 3x - 2\] \\
\textbf{B1.} Quydagi halqa bo'ladimi:
\(\mathbb{Q}\left( \sqrt{2} \right) = \left\{ a + b\sqrt{2}:a,b \in \mathbb{Q} \right\}\). \\
\textbf{B2.} \(R\) halqani \(R'\) halqaga o'tkazuvchi gomomorfizmini aniqlang.
\(R\mathbb{= (Z,} + , \cdot )\) va \(R' = (\mathbb{Z}_{6}, +_{6}, \cdot_{6})\). \\
\textbf{B3.} Quydagi maydonning berilgan ko'phadlar orqali ajralish maydonini toping.
\(\mathbb{Z}_{3}\) te \(x^{2} + x + 1\). \\
\textbf{C1.} Quydagi maydon kengaytmasining har birining bazisini toping. Har bir kengaytmaning darajasi qanday?
\(\mathbb{Q}\left( \sqrt{3} + \sqrt{5} \right)\) da \(\mathbb{Q}\left( \sqrt{2},\sqrt{6} + \sqrt{10} \right)\) \\
\textbf{C2.} Quydagi halqaning barcha ideallarini toping. Bul ideallardan qaysi-biri maksimal bo'ladi?
\[\mathbb{Z}_{18}\] \\
\textbf{C3.} Quyidagi akslantirishni gomomorfizm shartlariga tekshiring. \(f(x + iy) = x \cdot y\) \\

\end{tabular}
\vspace{1cm}


\begin{tabular}{m{17cm}}
\textbf{70-variant}
\newline

\textbf{T1.} p-adik kvadrat tenglamalar. \\
\textbf{T2.} Bo'lish algoritmi. \\
\textbf{A1.} Quydagi halqaning qism to'plamlari ideal bo'lishini ko'rsating.
\(R = \mathbb{Z}_{28}\), \(I = \{\overline{0},\overline{7},\overline{14},\overline{21}\}\). \\
\textbf{A2.} Quydagini hisoblang:
\(\mathbb{Z}_{5}\) te \(\left( 3x^{2} + 3x - 4 \right)\left( 4x^{2} + 2 \right)\) \\
\textbf{A3.} \(\mathbb{Z}_{4}\lbrack x\rbrack\) da \(\deg\ p(x) > 1\) bo'ladigan \(p(x)\) birligini toping. Quydagi ko'phadlarning qaysilari \(\mathbb{Q\lbrack}x\rbrack\) da keltirilmaydigan?
\[3x^{5} - 4x^{3} - 6x^{2} + 6\] \\
\textbf{B1.} Quydagi maydonlarning berilgan ko'phadlar orqali ajralish maydonini toping.
\(\mathbb{Q}\) da \(x^{4} - 5x^{2} + 6\). \\
\textbf{B2.} Quydagilar halqa bo'ladimi bunda \(i^{2} = - 1\):
\(\mathbb{Q(}i) = \{ x + iy\ |\ x,y \in \mathbb{Q\}}\). \\
\textbf{B3.} Quydagi maydonlarning berilgan ko'phadlar orqali ajralish maydonini toping.
\(\mathbb{Z}_{5}\) da \(x^{2} + x + 1\). \\
\textbf{C1.} Quydagi maydon kengaytmasining har birining bazisini toping. Har bir kengaytmaning darajasi qanday?
\(\mathbb{Q}\left( \sqrt{3} + \sqrt{5} \right)\) da \(\mathbb{Q}\left( \sqrt{2},\sqrt{6} + \sqrt{10} \right)\) \\
\textbf{C2.} Quydagi halqaning barcha ideallarini toping. Bul ideallardan qaysi-biri maksimal bo'ladi?
\(\mathbb{M}_{2}\left( \mathbb{Z} \right)\), elementleri \(\mathbb{Z}\) bol\(g'\)an \(2 \times 2\) matrica \\
\textbf{C3.} Quyidagi akslantirishni gomomorfizm shartlariga tekshiring.
\[f\left( a - \sqrt{2}b \right) = a + \sqrt{2}b\] \\

\end{tabular}
\vspace{1cm}


\begin{tabular}{m{17cm}}
\textbf{71-variant}
\newline

\textbf{T1.} p-adik sonlar maydoni va ular ustida amallar. \\
\textbf{T2.} Maksimal va sodda ideallar. \\
\textbf{A1.} \(\mathbb{Q}\) Ratsianal sonlar maydoni ustida minimal ko'phadalrini toping.
\(\sqrt{3 + \sqrt{3}}\). \\
\textbf{A2.} \(\mathbb{Q}\) Ratsianal sonlar maydoni ustida minimal ko'phadalrini toping.
\(\sqrt{2} + \sqrt{5}\). \\
\textbf{A3.} Quydagi ko'phadlarning barcha nollarini toping:
\(\mathbb{Z}_{2}\) de \(x^{3} + x + 1\). \\
\textbf{B1.} Quydagi maydon bo'ladimi:
\[\mathbb{Q}\left( \sqrt[3]{5} \right) = \left\{ a + b\sqrt[3]{5}:a,b,c \in \mathbb{Q} \right\}\] \\
\textbf{B2.} Quydagilar halqa bo'ladimi bunda \(i^{2} = - 1\):
\(\mathbb{Q(}i) = \{ x + iy\ |\ x,y \in \mathbb{Q\}}\). \\
\textbf{B3.} Quydagi matritsalar to'plamining qaysi biri matritsalarni qo'shish va ko'paytirish amallarga qarata halqa bo'ladi.
\(M_{2 \times 2}\mathbb{(R) =}\left\{ \begin{pmatrix}
a & b \\
b & d
\end{pmatrix}\ :\ a,b,c,d \in \mathbb{R} \right\}\). \\
\textbf{C1.} Quydagi maydon kengaytmasining har birining bazisini toping. Har bir kengaytmaning darajasi qanday?
\(\mathbb{Q}\) da \(\mathbb{Q}\left( \sqrt{2},i \right)\) \\
\textbf{C2.} Quydagi halqaning barcha ideallarini toping. Bul ideallardan qaysi-biri maksimal bo'ladi?
\[\mathbb{Z}_{25}\] \\
\textbf{C3.} Quyidagi akslantirishni gomomorfizm shartlariga tekshiring. \(f(x) = e^{x}\) \\

\end{tabular}
\vspace{1cm}


\begin{tabular}{m{17cm}}
\textbf{72-variant}
\newline

\textbf{T1.} Radikallarda yechilishi. \\
\textbf{T2.} Ko'phadlarning halqasi. \\
\textbf{A1.} Quydagi ko'phadlarning barcha nollarini toping:
\(\mathbb{Z}_{7}\) de \(5x^{4} + 2x^{2} - 3\); \\
\textbf{A2.} Quydagini hisoblang:
\(\mathbb{Z}_{9}\) da \(\left( 7x^{3} + 3x^{2} - x \right) + \left( 6x^{2} - 8x + 4 \right)\) \\
\textbf{A3.} Quydagi ko'phadlarning barcha nollarini toping:
\(\mathbb{Z}_{12}\) de \(5x^{3} + 4x^{2} - x + 9\); \\
\textbf{B1.} Quydagi halqa bo'ladimi:
\[\mathbb{Q}\left( \sqrt{2},\sqrt{3} \right) = \left\{ a + b\sqrt{2} + c\sqrt{3} + d\sqrt{6}:a,b,c,d \in \mathbb{Q} \right\}\] \\
\textbf{B2.} \(R\) halqani \(R'\) halqaga o'tkazuvchi gomomorfizmini aniqlang.
\(R\mathbb{= (R,} + , \cdot )\) hám \(R'\mathbb{= (R,} + , \cdot )\). \\
\textbf{B3.} Quydagi matritsalar to'plamining qaysi biri matritsalarni qo'shish va ko'paytirish amallarga qarata halqa bo'ladi.
\(M_{2 \times 2}\mathbb{(R) =}\left\{ \begin{pmatrix}
a & 0 \\
b & c
\end{pmatrix}\ :\ a,b,c \in \mathbb{R} \right\}\). \\
\textbf{C1.} Quydagi maydon kengaytmasining har birining bazisini toping. Har bir kengaytmaning darajasi qanday?
\(\mathbb{Q}\left( \sqrt{3} + \sqrt{5} \right)\) da \(\mathbb{Q}\left( \sqrt{2},\sqrt{6} + \sqrt{10} \right)\) \\
\textbf{C2.} Quydagi halqaning barcha ideallarini toping. Bul ideallardan qaysi-biri maksimal bo'ladi?
\(\mathbb{M}_{2}\left( \mathbb{Z} \right)\), elementleri \(\mathbb{Z}\) bol\(g'\)an \(2 \times 2\) matrica \\
\textbf{C3.} Quyidagi akslantirishni gomomorfizm shartlariga tekshiring.
\[f:\begin{pmatrix}
a & b \\
 - b & a
\end{pmatrix} \rightarrow a + bi\] \\

\end{tabular}
\vspace{1cm}


\begin{tabular}{m{17cm}}
\textbf{73-variant}
\newline

\textbf{T1.} Fundamental teoremalar. \\
\textbf{T2.} Ratsional sonlar maydonini haqiqiy sonlar maydonigacha to'ldirish. \\
\textbf{A1.} \(\mathbb{Q}\) Ratsianal sonlar maydoni ustida minimal ko'phadalrini toping.
\(\sqrt{3} + \sqrt{7}\). \\
\textbf{A2.} Quydagini hisoblang:
\(\mathbb{Z}_{12}\) de \(\left( 5x^{2} + 3x - 4 \right) + \left( 4x^{2} - x + 9 \right)\). \\
\textbf{A3.} \(\mathbb{Q}\) Ratsianal sonlar maydoni ustida minimal ko'phadni toping.
\[\sqrt{5} + \sqrt{7}\] \\
\textbf{B1.} Quydagilarning qaysi biri maydon bo'ladi:
\(\mathbb{Q}\left( \sqrt{11} \right) = \left\{ a + b\sqrt{11}:a,b \in \mathbb{Q} \right\}\). \\
\textbf{B2.} Quydagi halqa bo'ladimi:
\(\mathbb{R\lbrack}\sigma\rbrack = \{ x + \sigma \cdot y\ |\ \sigma^{2} = 0\}\). \\
\textbf{B3.} Quydagi matritsalar to'plamining qaysi biri matritsalarni qo'shish va ko'paytirish amallarga qarata halqa bo'ladi.
\(M_{2 \times 2}\mathbb{(R) =}\left\{ \begin{pmatrix}
a & 0 \\
0 & b
\end{pmatrix}\ :\ a,b \in \mathbb{R} \right\}\). \\
\textbf{C1.} Quydagi maydon kengaytmasining har birining bazisini toping. Har bir kengaytmaning darajasi qanday?
\(\mathbb{Q}\left( \sqrt{5} \right)\) da \(\mathbb{Q}\left( \sqrt{2} + \sqrt{5} \right)\) \\
\textbf{C2.} Quydagi halqaning barcha ideallarini toping. Bul ideallardan qaysi-biri maksimal bo'ladi?
\(\mathbb{M}_{2}\left( \mathbb{Z} \right)\), elementleri \(\mathbb{Z}\) bol\(g'\)an \(2 \times 2\) matrica \\
\textbf{C3.} Quyidagi akslantirishni gomomorfizm shartlariga tekshiring. \(f:\begin{pmatrix}
a & b \\
0 & c
\end{pmatrix} \rightarrow a\) \\

\end{tabular}
\vspace{1cm}


\begin{tabular}{m{17cm}}
\textbf{74-variant}
\newline

\textbf{T1.} Halqalarning gomomorfizmi va ideallar. \\
\textbf{T2.} Maydonlarning avtomorfizmlari. \\
\textbf{A1.} \(\mathbb{Q}\) Ratsianal sonlar maydoni ustida minimal ko'phadalrini toping.
\(\sqrt{2 + \sqrt{2}}\). \\
\textbf{A2.} Quydagini hisoblang:
\(\mathbb{Z}_{5}\) te \(\left( 3x^{2} + 2x - 4 \right) + \left( 4x^{2} + 2 \right)\) \\
\textbf{A3.} \(\mathbb{Z}_{4}\lbrack x\rbrack\) da \(\deg\ p(x) > 1\) bo'ladigan \(p(x)\) birligini toping. Quydagi ko'phadlarning qaysilari \(\mathbb{Q\lbrack}x\rbrack\) da keltirilmaydigan?
\[3x^{5} - 4x^{3} - 6x^{2} + 6\] \\
\textbf{B1.} Quydagi halqa bo'ladimi:
\[\mathbb{Z}\left( \sqrt{3} \right) = \left\{ a + b\sqrt{3}:a,b \in \mathbb{Z} \right\}\] \\
\textbf{B2.} Quydagi maydon bo'ladimi, bunda \(i^{2} = - 1\):
\(\mathbb{Q\lbrack}i\sqrt{n}\rbrack = \{ x + iy\sqrt{n}\ |\ x,y \in \mathbb{Q\}}\). \\
\textbf{B3.} Quydagi maydonning berilgan ko'phadlar orqali ajralish maydonini toping.
\(\mathbb{Q}\) da \(x^{3} - 3\). \\
\textbf{C1.} Quydagi maydon kengaytmasining har birining bazisini toping. Har bir kengaytmaning darajasi qanday?
\(\mathbb{Q}\left( \sqrt{2} \right)\) da \(\mathbb{Q}\left( \sqrt{8} \right)\) \\
\textbf{C2.} Quydagi halqaning barcha ideallarini toping. Bul ideallardan qaysi-biri maksimal bo'ladi?
\(\mathbb{M}_{2}\left( \mathbb{Z} \right)\), elementleri \(\mathbb{Z}\) bol\(g'\)an \(2 \times 2\) matrica
T1 Ko'phadlarning halqasi. \\
\textbf{C3.} Quyidagi akslantirishni gomomorfizm shartlariga tekshiring. \(f(a + ib) = \begin{pmatrix}
a & b \\
 - b & a
\end{pmatrix}\) \\

\end{tabular}
\vspace{1cm}


\begin{tabular}{m{17cm}}
\textbf{75-variant}
\newline

\textbf{T1.} Geometrik konstruksiyasi. \\
\textbf{T2.} Halqalar. \\
\textbf{A1.} \(\mathbb{Q}\) Ratsianal sonlar maydoni ustida minimal ko'phadalrini toping.
\(\sqrt{2 + \sqrt{5}}\). \\
\textbf{A2.} \(\mathbb{Q}\) Ratsianal sonlar maydoni ustida minimal ko'phadalrini toping.
\[\sqrt{2} + \sqrt{5}\] \\
\textbf{A3.} \(\mathbb{Z}_{4}\lbrack x\rbrack\) da \(\deg\ p(x) > 1\) bo'ladigan \(p(x)\) birligini toping. Quydagi ko'phadlarning qaysilari \(\mathbb{Q\lbrack}x\rbrack\) da keltirilmaydigan?
\[5x^{5} - 6x^{4} - 3x^{2} + 9x - 15\] \\
\textbf{B1.} Quydagi maydonlarning berilgan ko'phadlar orqali ajralish maydonini toping.
\(\mathbb{Q}\) da \(x^{4} - 5x^{2} + 6\). \\
\textbf{B2.} Quydagi halqa bo'ladimi:
\(\mathbb{R\lbrack}\sigma\rbrack = \{ x + \sigma \cdot y\ |\ \sigma^{2} = 0\}\). \\
\textbf{B3.} Quydagi matritsalar to'plamining qaysi biri matritsalarni qo'shish va ko'paytirish amallarga qarata halqa bo'ladi .
\(M_{2 \times 2}\mathbb{(R) =}\left\{ \begin{pmatrix}
a & b \\
0 & c
\end{pmatrix}\ :\ a,b,c \in \mathbb{R} \right\}\). \\
\textbf{C1.} Quydagi maydon kengaytmasining har birining bazisini toping. Har bir kengaytmaning darajasi qanday?
\(\mathbb{Q}\) da \(\mathbb{Q}\left( i,\sqrt{2} + i,\sqrt{3} + i \right)\) \\
\textbf{C2.} Quydagi halqaning barcha ideallarini toping. Bul ideallardan qaysi-biri maksimal bo'ladi?
\(\mathbb{M}_{2}\left( \mathbb{Z} \right)\), elementleri \(\mathbb{Z}\) bol\(g'\)an \(2 \times 2\) matrica \\
\textbf{C3.} Quyidagi akslantirishni gomomorfizm shartlariga tekshiring. \(f(x) = \sqrt{x}\) \\

\end{tabular}
\vspace{1cm}


\begin{tabular}{m{17cm}}
\textbf{76-variant}
\newline

\textbf{T1.} Maydonlarning ajralishi. \\
\textbf{T2.} Integral sohalar va maydonlar. \\
\textbf{A1.} Quydagi ko'phadlarning barcha nollarini toping:
\(\mathbb{Z}_{12}\) de \(5x^{3} + 4x^{2} - x + 9\); \\
\textbf{A2.} Quydagini hisoblang:
\(\mathbb{Z}_{12}\) de \(\left( 5x^{2} + 3x - 2 \right)^{2}\) \\
\textbf{A3.} \(\mathbb{Z}_{4}\lbrack x\rbrack\) da \(\deg\ p(x) > 1\) bo'ladigan \(p(x)\) birligini toping. Quydagi ko'phadlarning qaysilari \(\mathbb{Q\lbrack}x\rbrack\) da keltirilmaydigan?
\[3x^{5} - 4x^{3} - 6x^{2} + 6\] \\
\textbf{B1.} \(R\) halqani \(R'\) halqaga o'tkazuvchi gomomorfizmini aniqlang.
\(R = (\mathbb{Z}_{6}, +_{6}, \cdot_{6})\) hám \(R' = (\mathbb{Z}_{10}, +_{10}, \cdot_{10})\). \\
\textbf{B2.} \(R\) halqani \(R'\) halqaga o'tkazuvchi gomomorfizmini aniqlang.
\(R\mathbb{= (R,} + , \cdot )\) hám \(R'\mathbb{= (R,} + , \cdot )\). \\
\textbf{B3.} Quydagi matritsalar to'plamining qaysi biri matritsalarni qo'shish va ko'paytirish amallarga qarata halqa bo'ladi.
\(M_{2 \times 2}\mathbb{(R) =}\left\{ \begin{pmatrix}
a & 0 \\
0 & 0
\end{pmatrix}\ :\ a \in \mathbb{R} \right\}\). \\
\textbf{C1.} Quydagi maydon kengaytmasining har birining bazisini toping. Har bir kengaytmaning darajasi qanday?
\(\mathbb{Q}\) da \(\mathbb{Q}\left( \sqrt{3},\sqrt{6} \right)\) \\
\textbf{C2.} Quydagi halqaning barcha ideallarini toping. Bul ideallardan qaysi-biri maksimal bo'ladi?
\(\mathbb{M}_{2}\left( \mathbb{Z} \right)\), elementleri \(\mathbb{Z}\) bol\(g'\)an \(2 \times 2\) matrica \\
\textbf{C3.} Quyidagi akslantirishni gomomorfizm shartlariga tekshiring. \(f(x) = x^{2} + x\) \\

\end{tabular}
\vspace{1cm}


\begin{tabular}{m{17cm}}
\textbf{77-variant}
\newline

\textbf{T1.} Chekli maydonlarning strukturasi \\
\textbf{T2.} Maydonlarning avtomorfizmlari. \\
\textbf{A1.} \(\mathbb{Q}\) Ratsianal sonlar maydoni ustida minimal ko'phadalrini toping.
\(\sqrt{3 - \sqrt{3}}\). \\
\textbf{A2.} Quydagini hisoblang:
\(\mathbb{Z}_{5}\) te \(\left( 3x^{2} + 3x - 4 \right)\left( 4x^{2} + 2 \right)\) \\
\textbf{A3.} Quydagi halqalarning qism to'plamlari ideal bo'lishini ko'rsating: \(R = \left\{ \begin{pmatrix}
a & b \\
0 & c \\
 & 
\end{pmatrix}\ |\ a,b,c \in \mathbb{Z} \right\}\), \(I = \left\{ \begin{pmatrix}
0 & b \\
0 & c
\end{pmatrix}\ |\ a \in \mathbb{Z} \right\}\). \\
\textbf{B1.} Quydagilarning qaysi biri maydon bo'ladi:
\(\mathbb{Q}\left( \sqrt{11} \right) = \left\{ a + b\sqrt{11}:a,b \in \mathbb{Q} \right\}\). \\
\textbf{B2.} Quydagi maydon bo'ladimi, bunda \(i^{2} = - 1\):
\(\mathbb{Q\lbrack}i\sqrt{n}\rbrack = \{ x + iy\sqrt{n}\ |\ x,y \in \mathbb{Q\}}\). \\
\textbf{B3.} Quydagi maydonlarning berilgan ko'phadlar orqali ajralish maydonini toping.
\(\mathbb{Q}\) da \(x^{4} - 10x^{2} + 21\). \\
\textbf{C1.} Quydagi maydon kengaytmasining har birining bazisini toping. Har bir kengaytmaning darajasi qanday?
\(\mathbb{Q}\) da \(\mathbb{Q}\left( \sqrt{3},\sqrt{6} \right)\). \\
\textbf{C2.} Quydagi halqaning barcha ideallarini toping. Bul ideallardan qaysi-biri maksimal bo'ladi? $\mathbb{Q}$ \\
\textbf{C3.} Quyidagi akslantirishni gomomorfizm shartlariga tekshiring. \(f(x + iy) = x \cdot y\) \\

\end{tabular}
\vspace{1cm}


\begin{tabular}{m{17cm}}
\textbf{78-variant}
\newline

\textbf{T1.} Chekli maydonlarning strukturasi \\
\textbf{T2.} p-adik sonlar maydoni va ular ustida amallar. \\
\textbf{A1.} \(\mathbb{Q}\) Ratsianal sonlar maydoni ustida minimal ko'phadalrini toping.
\(\sqrt{2 - \sqrt{2}}\). \\
\textbf{A2.} Quydagini hisoblang:
\(\mathbb{Z}_{5}\) te \(\left( 3x^{2} + 3x - 4 \right)\left( 4x^{2} + 2 \right)\) \\
\textbf{A3.} \(\mathbb{Q}\) Ratsianal sonlar maydoni ustida minimal ko'phadni toping.
\[\sqrt{2} + \sqrt{3}\] \\
\textbf{B1.} Quydagi maydon bo'ladimi?
\(\mathbb{Q}\left( \sqrt{5},\sqrt{7} \right) = \left\{ a + b\sqrt{5} + c\sqrt{7} + d\sqrt{35}:a,b,c,d \in \mathbb{Q} \right\}\). \\
\textbf{B2.} Quydagi halqa bo'ladimi:
\(\mathbb{R\lbrack}\sigma\rbrack = \{ x + \sigma \cdot y\ |\ \sigma^{2} = 0\}\). \\
\textbf{B3.} Quydagi maydonlarning berilgan ko'phadlar orqali ajralish maydonini toping.
\(\mathbb{Z}_{2}\) da \(x^{2} + 1\). \\
\textbf{C1.} Quydagi maydon kengaytmasining har birining bazisini toping. Har bir kengaytmaning darajasi qanday?
\(\mathbb{Q}\) da \(\mathbb{Q}\left( \sqrt{3},\sqrt{5},\sqrt{7} \right)\) \\
\textbf{C2.} Quydagi halqaning barcha ideallarini toping. Bul ideallardan qaysi-biri maksimal bo'ladi?
\[\mathbb{Q}\] \\
\textbf{C3.} Quyidagi akslantirishni gomomorfizm shartlariga tekshiring.
\[f\left( a - \sqrt{2}b \right) = a + \sqrt{2}b\] \\

\end{tabular}
\vspace{1cm}


\begin{tabular}{m{17cm}}
\textbf{79-variant}
\newline

\textbf{T1.} Bo'lish algoritmi. \\
\textbf{T2.} p-adik kvadrat tenglamalar. \\
\textbf{A1.} \(\mathbb{Q}\) Ratsianal sonlar maydoni ustida minimal ko'phadalrini toping.
\(\sqrt{2 + 2\sqrt{2}}\). \\
\textbf{A2.} Quydagini hisoblang:
\(\mathbb{Z}_{12}\) de \(\left( 5x^{2} + 3x - 2 \right)^{2}\) \\
\textbf{A3.} Quydagi ko'phadlarning barcha nollarini toping:
\(\mathbb{Z}_{2}\) de \(x^{3} + x + 1\). \\
\textbf{B1.} Quydagi halqa bo'ladimi:
\(\mathbb{Q(}\sqrt[3]{2}) = \left\{ a + b\sqrt[3]{2}:a,b \in \mathbb{Q} \right\}\).
B2 Quydagilarning qaysi biri maydon bo'ladi, bunda \(i^{2} = - 1\):
\(\mathbb{Q\lbrack}i\rbrack = \{ x + iy\ |\ x,y \in \mathbb{Q\}}\).. \\
\textbf{B2.} Quydagi maydon bo'ladimi:
\(5\mathbb{Z}\). \\
\textbf{B3.} Quydagi matritsalar to'plamining qaysi biri matritsalarni qo'shish va ko'paytirish amallarga qarata halqa bo'ladi.
\(M_{2 \times 2}\mathbb{(R) =}\left\{ \begin{pmatrix}
a & b \\
 - b & d
\end{pmatrix}\ :\ a,b,c,d \in \mathbb{R} \right\}\). \\
\textbf{C1.} Quydagi maydon kengaytmasining har birining bazisini toping. Har bir kengaytmaning darajasi qanday?
\(\mathbb{Q}\) da \(\mathbb{Q}\left( \sqrt{2},i \right)\) \\
\textbf{C2.} Quydagi halqaning barcha ideallarini toping. Bul ideallardan qaysi-biri maksimal bo'ladi?
\[\mathbb{Z}_{18}\] \\
\textbf{C3.} Quyidagi akslantirishni gomomorfizm shartlariga tekshiring. \(f(x) = \sqrt[3]{x}\) \\

\end{tabular}
\vspace{1cm}


\begin{tabular}{m{17cm}}
\textbf{80-variant}
\newline

\textbf{T1.} Integral sohalar va maydonlar. \\
\textbf{T2.} Maydonlarning ajralishi. \\
\textbf{A1.} \(\mathbb{Q}\) Ratsianal sonlar maydoni ustida minimal ko'phadalrini toping.
\(\sqrt{2 + \sqrt{3}}\). \\
\textbf{A2.} Quydagini hisoblang:
\(\mathbb{Z}_{5}\) te \(\left( 3x^{2} + 2x - 4 \right) + \left( 4x^{2} + 2 \right)\) \\
\textbf{A3.} \(\mathbb{Z}_{4}\lbrack x\rbrack\) da \(\deg\ p(x) > 1\) bo'ladigan \(p(x)\) birligini toping. Quydagi ko'phadlarning qaysilari \(\mathbb{Q\lbrack}x\rbrack\) da keltirilmaydigan?
\[5x^{5} - 6x^{4} - 3x^{2} + 9x - 15\] \\
\textbf{B1.} Quydagilar halqa bo'ladimi:
\[\mathbb{Z}_{18}\] \\
\textbf{B2.} Quydagi halqa bo'ladimi:
\(\mathbb{R\lbrack}\sigma\rbrack = \{ x + \sigma \cdot y\ |\ \sigma^{2} = 0\}\). \\
\textbf{B3.} Quydagi matritsalar to'plamining qaysi biri matritsalarni qo'shish va ko'paytirish amallarga qarata halqa bo'ladi.
\[M_{2 \times 2}\mathbb{(R) =}\left\{ \begin{pmatrix}
a & 0 \\
0 & a
\end{pmatrix}\ :\ a \in \mathbb{R} \right\}\] \\
\textbf{C1.} Quydagi maydon kengaytmasining har birining bazisini toping. Har bir kengaytmaning darajasi qanday?
\(\mathbb{Q}\) da \(\mathbb{Q}\left( \sqrt{2},\sqrt[3]{2} \right)\) \\
\textbf{C2.} Quydagi halqaning barcha ideallarini toping. Bul ideallardan qaysi-biri maksimal bo'ladi?
\(\mathbb{M}_{2}\left( \mathbb{Z} \right)\), elementleri \(\mathbb{Z}\) bol\(g'\)an \(2 \times 2\) matrica \\
\textbf{C3.} Quyidagi akslantirishni gomomorfizm shartlariga tekshiring. \(f(x) = e^{x}\) \\

\end{tabular}
\vspace{1cm}


\begin{tabular}{m{17cm}}
\textbf{81-variant}
\newline

\textbf{T1.} Maydonlarning kengaytmasi. Algebrik element. Algebraik yopilma. \\
\textbf{T2.} Halqalar. \\
\textbf{A1.} Quydagi ko'phadlarning barcha nollarini toping:
\(\mathbb{Z}_{2}\) de \(x^{3} + x + 1\). \\
\textbf{A2.} Quydagini hisoblang:
\(\mathbb{Z}_{5}\) te \(\left( 3x^{2} + 2x - 4 \right) + \left( 4x^{2} + 2 \right)\) \\
\textbf{A3.} \(\mathbb{Q}\) Ratsianal sonlar maydoni ustida minimal ko'phadni toping.
\[\sqrt{5} + \sqrt{7}\] \\
\textbf{B1.} Quydagi halqa bo'ladimi:
\(\mathbb{Q}\left( \sqrt[3]{3} \right) = \left\{ a + b\sqrt[3]{3} + c\sqrt[3]{9}:a,b,c \in \mathbb{Q} \right\}\). \\
\textbf{B2.} Quydagilar halqa bo'ladimi bunda \(i^{2} = - 1\):
\(\mathbb{Q(}i) = \{ x + iy\ |\ x,y \in \mathbb{Q\}}\). \\
\textbf{B3.} Quydagi matritsalar to'plamining qaysi biri matritsalarni qo'shish va ko'paytirish amallarga qarata halqa bo'ladi.
\(M_{2 \times 2}\mathbb{(R) =}\left\{ \begin{pmatrix}
a & 0 \\
0 & 0
\end{pmatrix}\ :\ a \in \mathbb{R} \right\}\). \\
\textbf{C1.} Quydagi maydon kengaytmasining har birining bazisini toping. Har bir kengaytmaning darajasi qanday?
\(\mathbb{Q}\) da \(\mathbb{Q}\left( \sqrt{2},i \right)\) \\
\textbf{C2.} Quydagi halqaning barcha ideallarini toping. Bul ideallardan qaysi-biri maksimal bo'ladi?
\[\mathbb{Q}\] \\
\textbf{C3.} Quyidagi akslantirishni gomomorfizm shartlariga tekshiring. \(f\left( \begin{pmatrix}
a & 0 \\
0 & a
\end{pmatrix} \right) = a\) \\

\end{tabular}
\vspace{1cm}


\begin{tabular}{m{17cm}}
\textbf{82-variant}
\newline

\textbf{T1.} Maksimal va sodda ideallar. \\
\textbf{T2.} Ko'phadlarning halqasi. \\
\textbf{A1.} Quydagi ko'phadlarning barcha nollarini toping:
\(\mathbb{Z}_{12}\) de \(5x^{3} + 4x^{2} - x + 9\); \\
\textbf{A2.} Quydagini hisoblang:
\(\mathbb{Z}_{5}\) te \(\left( 3x^{2} + 3x - 4 \right)\left( 4x^{2} + 2 \right)\) \\
\textbf{A3.} Quydagi ko'phadlarning barcha nollarini toping:
\(\mathbb{Z}_{2}\) de \(x^{3} + x + 1\). \\
\textbf{B1.} Quydagi halqa bo'ladimi:
\(\mathbb{Q(}\sqrt[3]{2}) = \left\{ a + b\sqrt[3]{2}:a,b \in \mathbb{Q} \right\}\).
B2 Quydagilarning qaysi biri maydon bo'ladi, bunda \(i^{2} = - 1\):
\(\mathbb{Q\lbrack}i\rbrack = \{ x + iy\ |\ x,y \in \mathbb{Q\}}\).. \\
\textbf{B2.} Quydagi maydon bo'ladimi, bunda \(i^{2} = - 1\):
\[\mathbb{Z\lbrack}i\sqrt{n}\rbrack = \{ x + iy\sqrt{n}\ |\ x,y \in \mathbb{Z\}}\] \\
\textbf{B3.} Quydagi maydonlarning berilgan ko'phadlar orqali ajralish maydonini toping.
\(\mathbb{Z}_{5}\) da \(x^{2} + x + 1\). \\
\textbf{C1.} Quydagi maydon kengaytmasining har birining bazisini toping. Har bir kengaytmaning darajasi qanday?
\(\mathbb{Q}\left( \sqrt{3} + \sqrt{5} \right)\) da \(\mathbb{Q}\left( \sqrt{2},\sqrt{6} + \sqrt{10} \right)\) \\
\textbf{C2.} Quydagi halqaning barcha ideallarini toping. Bul ideallardan qaysi-biri maksimal bo'ladi?
\(\mathbb{M}_{2}\left( \mathbb{Z} \right)\), elementleri \(\mathbb{Z}\) bol\(g'\)an \(2 \times 2\) matrica \\
\textbf{C3.} Quyidagi akslantirishni gomomorfizm shartlariga tekshiring.
\[f:x \rightarrow x^{p}\] \\

\end{tabular}
\vspace{1cm}


\begin{tabular}{m{17cm}}
\textbf{83-variant}
\newline

\textbf{T1.} Ratsional sonlar maydonini haqiqiy sonlar maydonigacha to'ldirish. \\
\textbf{T2.} Geometrik konstruksiyasi. \\
\textbf{A1.} Quydagi ko'phadlarning barcha nollarini toping:
\(\mathbb{Z}_{5}\) da \(3x^{3} - 4x^{2} - x + 4\); \\
\textbf{A2.} \(\mathbb{Q}\) Ratsianal sonlar maydoni ustida minimal ko'phadalrini toping.
\(\sqrt{2} + \sqrt{5}\). \\
\textbf{A3.} \(\mathbb{Z}_{4}\lbrack x\rbrack\) da \(\deg\ p(x) > 1\) bo'ladigan \(p(x)\) birligini toping. Quydagi ko'phadlarning qaysilari \(\mathbb{Q\lbrack}x\rbrack\) da keltirilmaydigan?
\[5x^{5} - 6x^{4} - 3x^{2} + 9x - 15\] \\
\textbf{B1.} Quydagilar halqa bo'ladimi:
\[\mathbb{Z}_{18}\] \\
\textbf{B2.} \(R\) halqani \(R'\) halqaga o'tkazuvchi gomomorfizmini aniqlang.
\(R\mathbb{= (R,} + , \cdot )\) hám \(R'\mathbb{= (R,} + , \cdot )\). \\
\textbf{B3.} Quydagi maydonning berilgan ko'phadlar orqali ajralish maydonini toping.
\(\mathbb{Q}\) da \(x^{4} - 2\). \\
\textbf{C1.} Quydagi maydon kengaytmasining har birining bazisini toping. Har bir kengaytmaning darajasi qanday?
\(\mathbb{Q}\) da \(\mathbb{Q}\left( \sqrt{2},i \right)\) \\
\textbf{C2.} Quydagi halqaning barcha ideallarini toping. Bul ideallardan qaysi-biri maksimal bo'ladi?
\[\mathbb{Z}_{18}\] \\
\textbf{C3.} Quyidagi akslantirishni gomomorfizm shartlariga tekshiring. \(f(a + ib) = \begin{pmatrix}
a & b \\
 - b & a
\end{pmatrix}\) \\

\end{tabular}
\vspace{1cm}


\begin{tabular}{m{17cm}}
\textbf{84-variant}
\newline

\textbf{T1.} Fundamental teoremalar. \\
\textbf{T2.} Halqalarning gomomorfizmi va ideallar. \\
\textbf{A1.} \(\mathbb{Q}\) Ratsianal sonlar maydoni ustida minimal ko'phadalrini toping.
\(\sqrt{3 + \sqrt{3}}\). \\
\textbf{A2.} Quydagini hisoblang:
\(\mathbb{Z}_{12}\) da \(\left( 5x^{2} + 3x - 4 \right) + \left( 4x^{2} - x + 9 \right)\) \\
\textbf{A3.} Quydagi ko'phadlarning barcha nollarini toping:
\(\mathbb{Z}_{2}\) de \(x^{3} + x + 1\). \\
\textbf{B1.} Quydagi halqa bo'ladimi:
\(\mathbb{Z}_{18}\). \\
\textbf{B2.} Quydagilar halqa bo'ladimi bunda \(i^{2} = - 1\):
\(\mathbb{Q(}i\sqrt{n}) = \{ x + iy\sqrt{n}\ |\ x,y \in \mathbb{Q\}}\). \\
\textbf{B3.} Quydagi matritsalar to'plamining qaysi biri matritsalarni qo'shish va ko'paytirish amallarga qarata halqa bo'ladi
\(M_{2 \times 2}\mathbb{(R) =}\left\{ \begin{pmatrix}
a & 0 \\
b & c
\end{pmatrix}\ :\ a,b,c \in \mathbb{R} \right\}\). \\
\textbf{C1.} Quydagi maydon kengaytmasining har birining bazisini toping. Har bir kengaytmaning darajasi qanday?
\(\mathbb{Q}\) da \(\mathbb{Q}\left( \sqrt{3},\sqrt{6} \right)\) \\
\textbf{C2.} Quydagi halqaning barcha ideallarini toping. Bul ideallardan qaysi-biri maksimal bo'ladi?
\(\mathbb{M}_{2}\left( \mathbb{Z} \right)\), elementleri \(\mathbb{Z}\) bol\(g'\)an \(2 \times 2\) matrica \\
\textbf{C3.} Quyidagi akslantirishni gomomorfizm shartlariga tekshiring. \(f\left( a + \sqrt{2}b \right) = a + bi\) \\

\end{tabular}
\vspace{1cm}


\begin{tabular}{m{17cm}}
\textbf{85-variant}
\newline

\textbf{T1.} Radikallarda yechilishi. \\
\textbf{T2.} Keltirilmaydigan ko'phadlar. \\
\textbf{A1.} Quydagi halqaning qism to'plamlari ideal bo'lishini ko'rsating.
\(R\mathbb{= Z\lbrack}\sqrt{7}\rbrack\), \(I = \{ a + b\sqrt{7}\ |\ \ a,b \in \mathbb{Z,}a - b\ \ juft\ son\}\). \\
\textbf{A2.} Quydagini hisoblang:
\(\mathbb{Z}_{12}\) de \(\left( 5x^{2} + 3x - 2 \right)^{2}\) \\
\textbf{A3.} \(\mathbb{Z}_{4}\lbrack x\rbrack\) da \(\deg\ p(x) > 1\) bo'ladigan \(p(x)\) birligini toping. Quydagi ko'phadlarning qaysilari \(\mathbb{Q\lbrack}x\rbrack\) da keltirilmaydigan?
\[3x^{5} - 4x^{3} - 6x^{2} + 6\] \\
\textbf{B1.} Quydagi maydon bo'ladimi:
\[\mathbb{Q}\left( \sqrt[3]{5} \right) = \left\{ a + b\sqrt[3]{5}:a,b,c \in \mathbb{Q} \right\}\] \\
\textbf{B2.} Quydagi halqa bo'ladimi:
\(\mathbb{R\lbrack}\omega\rbrack = \{ x + \omega \cdot y\ |\ \omega^{2} = 1\}\). \\
\textbf{B3.} Quydagi matritsalar to'plamining qaysi biri matritsalarni qo'shish va ko'paytirish amallarga qarata halqa bo'ladi.
\(M_{2 \times 2}\mathbb{(R) =}\left\{ \begin{pmatrix}
a & b \\
b & d
\end{pmatrix}\ :\ a,b,c,d \in \mathbb{R} \right\}\). \\
\textbf{C1.} Quydagi maydon kengaytmasining har birining bazisini toping. Har bir kengaytmaning darajasi qanday?
\(\mathbb{Q}\left( \sqrt{3} + \sqrt{5} \right)\) da \(\mathbb{Q}\left( \sqrt{2},\sqrt{6} + \sqrt{10} \right)\) \\
\textbf{C2.} Quydagi halqaning barcha ideallarini toping. Bul ideallardan qaysi-biri maksimal bo'ladi?
\[\mathbb{Z}_{25}\] \\
\textbf{C3.} Quyidagi akslantirishni gomomorfizm shartlariga tekshiring. \(f(x) = \sqrt{x}\) \\

\end{tabular}
\vspace{1cm}


\begin{tabular}{m{17cm}}
\textbf{86-variant}
\newline

\textbf{T1.} Maydonlarning avtomorfizmlari. \\
\textbf{T2.} Integral sohalar va maydonlar. \\
\textbf{A1.} \(\mathbb{Q}\) Ratsianal sonlar maydoni ustida minimal ko'phadalrini toping.
\(\sqrt{2 + \sqrt{2}}\). \\
\textbf{A2.} Quydagini hisoblang:
\(\mathbb{Z}_{12}\) de \(\left( 5x^{2} + 3x - 4 \right)\left( 4x^{2} - x + 9 \right)\). \\
\textbf{A3.} Quydagi ko'phadlarning barcha nollarini toping:
\(\mathbb{Z}_{2}\) de \(x^{3} + x + 1\). \\
\textbf{B1.} \(R\) halqani \(R'\) halqaga o'tkazuvchi gomomorfizmini aniqlang.
\(R = (\mathbb{Z}_{4}, +_{4}, \cdot_{4})\) hám \(R' = (\mathbb{Z}_{6}, +_{6}, \cdot_{6})\). \\
\textbf{B2.} \(R\) halqani \(R'\) halqaga o'tkazuvchi gomomorfizmini aniqlang.
\(R\mathbb{= (Z,} + , \cdot )\) va \(R' = (\mathbb{Z}_{6}, +_{6}, \cdot_{6})\). \\
\textbf{B3.} Quydagi matritsalar to'plamining qaysi biri matritsalarni qo'shish va ko'paytirish amallarga qarata halqa bo'ladi.
\[M_{2 \times 2}\mathbb{(R) =}\left\{ \begin{pmatrix}
a & 0 \\
0 & a
\end{pmatrix}\ :\ a \in \mathbb{R} \right\}\] \\
\textbf{C1.} Quydagi maydon kengaytmasining har birining bazisini toping. Har bir kengaytmaning darajasi qanday?
\(\mathbb{Q}\) da \(\mathbb{Q}\left( \sqrt{2},\sqrt[3]{2} \right)\) \\
\textbf{C2.} Quydagi halqaning barcha ideallarini toping. Bul ideallardan qaysi-biri maksimal bo'ladi?
\(\mathbb{M}_{2}\left( \mathbb{Z} \right)\), elementleri \(\mathbb{Z}\) bol\(g'\)an \(2 \times 2\) matrica \\
\textbf{C3.} Quyidagi akslantirishni gomomorfizm shartlariga tekshiring. \(f(a) = a^{n}\) \\

\end{tabular}
\vspace{1cm}


\begin{tabular}{m{17cm}}
\textbf{87-variant}
\newline

\textbf{T1.} Bo'lish algoritmi. \\
\textbf{T2.} Geometrik konstruksiyasi. \\
\textbf{A1.} \(\mathbb{Q}\) Ratsianal sonlar maydoni ustida minimal ko'phadalrini toping.
\(\sqrt{2 + \sqrt{5}}\). \\
\textbf{A2.} Quydagini hisoblang:
\(\mathbb{Z}_{9}\) da \(\left( 7x^{3} + 3x^{2} - x \right) + \left( 6x^{2} - 8x + 4 \right)\) \\
\textbf{A3.} \(\mathbb{Z}_{4}\lbrack x\rbrack\) da \(\deg\ p(x) > 1\) bo'ladigan \(p(x)\) birligini toping. Quydagi ko'phadlarning qaysilari \(\mathbb{Q\lbrack}x\rbrack\) da keltirilmaydigan?
\[3x^{5} - 4x^{3} - 6x^{2} + 6\] \\
\textbf{B1.} Quydagi maydonlarning berilgan ko'phadlar orqali ajralish maydonini toping.
\(\mathbb{Q}\) da \(x^{4} - 5x^{2} + 6\). \\
\textbf{B2.} Quydagilar halqa bo'ladimi bunda \(i^{2} = - 1\):
\(\mathbb{Q(}i) = \{ x + iy\ |\ x,y \in \mathbb{Q\}}\). \\
\textbf{B3.} Quydagi matritsalar to'plamining qaysi biri matritsalarni qo'shish va ko'paytirish amallarga qarata halqa bo'ladi
\[M_{2 \times 2}\mathbb{(R) =}\left\{ \begin{pmatrix}
a & b \\
0 & c
\end{pmatrix}\ :\ a,b,c \in \mathbb{R} \right\}\] \\
\textbf{C1.} Quydagi maydon kengaytmasining har birining bazisini toping. Har bir kengaytmaning darajasi qanday?
\(\mathbb{Q}\) da \(\mathbb{Q}\left( \sqrt{3},\sqrt{6} \right)\) \\
\textbf{C2.} Quydagi halqaning barcha ideallarini toping. Bul ideallardan qaysi-biri maksimal bo'ladi?
\(\mathbb{M}_{2}\left( \mathbb{Z} \right)\), elementleri \(\mathbb{Z}\) bol\(g'\)an \(2 \times 2\) matrica \\
\textbf{C3.} Quyidagi akslantirishni gomomorfizm shartlariga tekshiring. \(f(x) = 5^{x}\) \\

\end{tabular}
\vspace{1cm}


\begin{tabular}{m{17cm}}
\textbf{88-variant}
\newline

\textbf{T1.} Ko'phadlarning halqasi. \\
\textbf{T2.} Halqalar. \\
\textbf{A1.} \(\mathbb{Q}\) Ratsianal sonlar maydoni ustida minimal ko'phadalrini toping.
\(\sqrt{2 + \sqrt{3}}\). \\
\textbf{A2.} Quydagini hisoblang:
\(\mathbb{Z}_{5}\) te \(\left( 3x^{2} + 2x - 4 \right) + \left( 4x^{2} + 2 \right)\) \\
\textbf{A3.} Quydagi halqalarning qism to'plamlari ideal bo'lishini ko'rsating: \(R = \left\{ \begin{pmatrix}
a & b \\
0 & c
\end{pmatrix}\ |\ a,b,c \in \mathbb{Z} \right\}\), \(I = \left\{ \begin{pmatrix}
0 & a \\
0 & 0
\end{pmatrix}\ |\ a \in \mathbb{Z} \right\}\). \\
\textbf{B1.} Quydagilarning qaysi biri maydon bo'ladi:
\(\mathbb{Q}\left( \sqrt{11} \right) = \left\{ a + b\sqrt{11}:a,b \in \mathbb{Q} \right\}\). \\
\textbf{B2.} \(R\) halqani \(R'\) halqaga o'tkazuvchi gomomorfizmini aniqlang.
\(R\mathbb{= (R,} + , \cdot )\) hám \(R'\mathbb{= (R,} + , \cdot )\). \\
\textbf{B3.} Quydagi maydonning berilgan ko'phadlar orqali ajralish maydonini toping.
\(\mathbb{Q}\) da \(x^{4} + 1\). \\
\textbf{C1.} Quydagi maydon kengaytmasining har birining bazisini toping. Har bir kengaytmaning darajasi qanday?
\(\mathbb{Q}\left( \sqrt{5} \right)\) da \(\mathbb{Q}\left( \sqrt{2} + \sqrt{5} \right)\) \\
\textbf{C2.} Quydagi halqaning barcha ideallarini toping. Bul ideallardan qaysi-biri maksimal bo'ladi?
\[\mathbb{Q}\] \\
\textbf{C3.} Quyidagi akslantirishni gomomorfizm shartlariga tekshiring. \(f:\begin{pmatrix}
a & b \\
0 & c
\end{pmatrix} \rightarrow a\) \\

\end{tabular}
\vspace{1cm}


\begin{tabular}{m{17cm}}
\textbf{89-variant}
\newline

\textbf{T1.} Halqalarning gomomorfizmi va ideallar. \\
\textbf{T2.} p-adik kvadrat tenglamalar. \\
\textbf{A1.} Quydagi halqaning qism to'plamlari ideal bo'lishini ko'rsating.
\(R = \mathbb{Z}_{28}\), \(I = \{\overline{0},\overline{7},\overline{14},\overline{21}\}\). \\
\textbf{A2.} Quydagini hisoblang:
\(\mathbb{Z}_{12}\) de \(\left( 5x^{2} + 3x - 4 \right) + \left( 4x^{2} - x + 9 \right)\). \\
\textbf{A3.} Quydagi ko'phadlarning barcha nollarini toping:
\(\mathbb{Z}_{12}\) de \(5x^{3} + 4x^{2} - x + 9\); \\
\textbf{B1.} Quydagi maydonlarning berilgan ko'phadlar orqali ajralish maydonini toping.
\(\mathbb{Q}\) da \(x^{4} - 5x^{2} + 6\). \\
\textbf{B2.} Quydagilarning qaysi biri maydon bo'ladi, bunda \(i^{2} = - 1\):
\(\mathbb{Z\lbrack}i\rbrack = \{ x + iy\ |\ x,y \in \mathbb{Z\}}\). \\
\textbf{B3.} Quydagi matritsalar to'plamining qaysi biri matritsalarni qo'shish va ko'paytirish amallarga qarata halqa bo'ladi.
\(M_{2 \times 2}\mathbb{(R) =}\left\{ \begin{pmatrix}
a & 0 \\
b & c
\end{pmatrix}\ :\ a,b,c \in \mathbb{R} \right\}\). \\
\textbf{C1.} Quydagi maydon kengaytmasining har birining bazisini toping. Har bir kengaytmaning darajasi qanday?
\(\mathbb{Q}\left( \sqrt{3} + \sqrt{5} \right)\) da \(\mathbb{Q}\left( \sqrt{2},\sqrt{6} + \sqrt{10} \right)\) \\
\textbf{C2.} Quydagi halqaning barcha ideallarini toping. Bul ideallardan qaysi-biri maksimal bo'ladi?
\(\mathbb{M}_{2}\left( \mathbb{Z} \right)\), elementleri \(\mathbb{Z}\) bol\(g'\)an \(2 \times 2\) matrica
T1 Ko'phadlarning halqasi. \\
\textbf{C3.} Quyidagi akslantirishni gomomorfizm shartlariga tekshiring.
\[f:\begin{pmatrix}
a & b \\
 - b & a
\end{pmatrix} \rightarrow a + bi\] \\

\end{tabular}
\vspace{1cm}


\begin{tabular}{m{17cm}}
\textbf{90-variant}
\newline

\textbf{T1.} Maksimal va sodda ideallar. \\
\textbf{T2.} Keltirilmaydigan ko'phadlar. \\
\textbf{A1.} \(\mathbb{Q}\) Ratsianal sonlar maydoni ustida minimal ko'phadalrini toping.
\(\sqrt{3 - \sqrt{3}}\). \\
\textbf{A2.} Quydagini hisoblang:
\(\mathbb{Z}_{5}\) te \(\left( 3x^{2} + 2x - 4 \right) + \left( 4x^{2} + 2 \right)\) \\
\textbf{A3.} \(\mathbb{Z}_{4}\lbrack x\rbrack\) da \(\deg\ p(x) > 1\) bo'ladigan \(p(x)\) birligini toping. Quydagi ko'phadlarning qaysilari \(\mathbb{Q\lbrack}x\rbrack\) da keltirilmaydigan?
\[x^{4} - 5x^{3} + 3x - 2\] \\
\textbf{B1.} \(R\) halqani \(R'\) halqaga o'tkazuvchi gomomorfizmini aniqlang.
\(R = (\mathbb{Z}_{6}, +_{6}, \cdot_{6})\) hám \(R' = (\mathbb{Z}_{10}, +_{10}, \cdot_{10})\). \\
\textbf{B2.} Quydagilar halqa bo'ladimi bunda \(i^{2} = - 1\):
\(\mathbb{Z(}i\sqrt{n}) = \{ x + iy\sqrt{n}\ |\ x,y \in \mathbb{Z\}}\). \\
\textbf{B3.} Quydagi matritsalar to'plamining qaysi biri matritsalarni qo'shish va ko'paytirish amallarga qarata halqa bo'ladi.
\(M_{2 \times 2}\mathbb{(R) =}\left\{ \begin{pmatrix}
a & 0 \\
0 & b
\end{pmatrix}\ :\ a,b \in \mathbb{R} \right\}\). \\
\textbf{C1.} Quydagi maydon kengaytmasining har birining bazisini toping. Har bir kengaytmaning darajasi qanday?
\(\mathbb{Q}\left( \sqrt{3} + \sqrt{5} \right)\) da \(\mathbb{Q}\left( \sqrt{2},\sqrt{6} + \sqrt{10} \right)\) \\
\textbf{C2.} Quydagi halqaning barcha ideallarini toping. Bul ideallardan qaysi-biri maksimal bo'ladi? $\mathbb{Q}$ \\
\textbf{C3.} Quyidagi akslantirishni gomomorfizm shartlariga tekshiring. \(f\left( \begin{pmatrix}
a & 0 \\
0 & a
\end{pmatrix} \right) = a\) \\

\end{tabular}
\vspace{1cm}


\begin{tabular}{m{17cm}}
\textbf{91-variant}
\newline

\textbf{T1.} Maydonlarning kengaytmasi. Algebrik element. Algebraik yopilma. \\
\textbf{T2.} Ratsional sonlar maydonini haqiqiy sonlar maydonigacha to'ldirish. \\
\textbf{A1.} \(\mathbb{Q}\) Ratsianal sonlar maydoni ustida minimal ko'phadalrini toping.
\(\sqrt{6 + 3\sqrt{2}}\). \\
\textbf{A2.} Quydagini hisoblang:
\(\mathbb{Z}_{9}\) da \(\left( 7x^{3} + 3x^{2} - x \right) + \left( 6x^{2} - 8x + 4 \right)\) \\
\textbf{A3.} Quydagi ko'phadlarning barcha nollarini toping:
\(\mathbb{Z}_{5}\) de \(3x^{3} - 4x^{2} - x + 4\); \\
\textbf{B1.} Quydagi halqa bo'ladimi:
\(7\mathbb{Z}\). \\
\textbf{B2.} Quydagilar halqa bo'ladimi bunda \(i^{2} = - 1\):
\(\mathbb{Q(}i) = \{ x + iy\ |\ x,y \in \mathbb{Q\}}\). \\
\textbf{B3.} Quydagi maydonning berilgan ko'phadlar orqali ajralish maydonini toping.
\(\mathbb{Q}\) da \(x^{3} - 3\). \\
\textbf{C1.} Quydagi maydon kengaytmasining har birining bazisini toping. Har bir kengaytmaning darajasi qanday?
\(\mathbb{Q}\) da \(\mathbb{Q}\left( \sqrt{3},\sqrt{5},\sqrt{7} \right)\) \\
\textbf{C2.} Quydagi halqaning barcha ideallarini toping. Bul ideallardan qaysi-biri maksimal bo'ladi?
\[\mathbb{Z}_{25}\] \\
\textbf{C3.} Quyidagi akslantirishni gomomorfizm shartlariga tekshiring. \(f\left( \begin{pmatrix}
a & 0 \\
0 & a
\end{pmatrix} \right) = a\) \\

\end{tabular}
\vspace{1cm}


\begin{tabular}{m{17cm}}
\textbf{92-variant}
\newline

\textbf{T1.} Maydonlarning ajralishi. \\
\textbf{T2.} Radikallarda yechilishi. \\
\textbf{A1.} \(\mathbb{Q}\) Ratsianal sonlar maydoni ustida minimal ko'phadalrini toping.
\(\sqrt{2 + \sqrt{3}}\). \\
\textbf{A2.} Quydagini hisoblang:
\(\mathbb{Z}_{12}\) de \(\left( 5x^{2} + 3x - 2 \right)^{2}\) \\
\textbf{A3.} \(\mathbb{Z}_{4}\lbrack x\rbrack\) da \(\deg\ p(x) > 1\) bo'ladigan \(p(x)\) birligini toping. Quydagi ko'phadlarning qaysilari \(\mathbb{Q\lbrack}x\rbrack\) da keltirilmaydigan?
\[3x^{5} - 4x^{3} - 6x^{2} + 6\] \\
\textbf{B1.} Quydagi halqa bo'ladimi:
\[\mathbb{Z}\left( \sqrt{3} \right) = \left\{ a + b\sqrt{3}:a,b \in \mathbb{Z} \right\}\] \\
\textbf{B2.} \(R\) halqani \(R'\) halqaga o'tkazuvchi gomomorfizmini aniqlang.
\(R = (\mathbb{Z}_{6}, +_{6}, \cdot_{6})\) hám \(R' = (\mathbb{Z}_{10}, +_{10}, \cdot_{10})\). \\
\textbf{B3.} Quydagi matritsalar to'plamining qaysi biri matritsalarni qo'shish va ko'paytirish amallarga qarata halqa bo'ladi.
\[M_{2 \times 2}\mathbb{(R) =}\left\{ \begin{pmatrix}
a & 0 \\
0 & - a
\end{pmatrix}\ :\ a \in \mathbb{R} \right\}\] \\
\textbf{C1.} Quydagi maydon kengaytmasining har birining bazisini toping. Har bir kengaytmaning darajasi qanday?
\(\mathbb{Q}\left( \sqrt{5} \right)\) da \(\mathbb{Q}\left( \sqrt{2} + \sqrt{5} \right)\) \\
\textbf{C2.} Quydagi halqaning barcha ideallarini toping. Bul ideallardan qaysi-biri maksimal bo'ladi?
\[\mathbb{Z}_{18}\] \\
\textbf{C3.} Quyidagi akslantirishni gomomorfizm shartlariga tekshiring. \(f\left( a + \sqrt{2}b \right) = a + bi\) \\

\end{tabular}
\vspace{1cm}


\begin{tabular}{m{17cm}}
\textbf{93-variant}
\newline

\textbf{T1.} Chekli maydonlarning strukturasi \\
\textbf{T2.} p-adik sonlar maydoni va ular ustida amallar. \\
\textbf{A1.} \(\mathbb{Q}\) Ratsianal sonlar maydoni ustida minimal ko'phadalrini toping.
\(\sqrt{2 + \sqrt{7}}\). \\
\textbf{A2.} Quydagini hisoblang:
\(\mathbb{Z}_{5}\) te \(\left( 3x^{2} + 2x - 4 \right) + \left( 4x^{2} + 2 \right)\) \\
\textbf{A3.} Quydagi ko'phadlarning barcha nollarini toping:
\(\mathbb{Z}_{5}\) de \(3x^{3} - 4x^{2} - x + 4\); \\
\textbf{B1.} Quydagi maydon bo'ladimi:
\(\mathbb{Q(}\sqrt{3}) = \left\{ a + b\sqrt[3]{3}:a,b \in \mathbb{Q} \right\}\). \\
\textbf{B2.} Quydagi maydon bo'ladimi, bunda \(i^{2} = - 1\):
\(\mathbb{Q\lbrack}i\sqrt{n}\rbrack = \{ x + iy\sqrt{n}\ |\ x,y \in \mathbb{Q\}}\). \\
\textbf{B3.} Quydagi maydonning berilgan ko'phadlar orqali ajralish maydonini toping.
\(\mathbb{Z}_{3}\) te \(x^{2} + x + 1\). \\
\textbf{C1.} Quydagi maydon kengaytmasining har birining bazisini toping. Har bir kengaytmaning darajasi qanday?
\(\mathbb{Q}\) da \(\mathbb{Q}\left( i,\sqrt{2} + i,\sqrt{3} + i \right)\) \\
\textbf{C2.} Quydagi halqaning barcha ideallarini toping. Bul ideallardan qaysi-biri maksimal bo'ladi?
\(\mathbb{M}_{2}\left( \mathbb{R} \right)\), elementleri \(\mathbb{R}\) bol\(g'\)an \(2 \times 2\) matrica \\
\textbf{C3.} Quyidagi akslantirishni gomomorfizm shartlariga tekshiring. \(f(x) = \sqrt[3]{x}\) \\

\end{tabular}
\vspace{1cm}


\begin{tabular}{m{17cm}}
\textbf{94-variant}
\newline

\textbf{T1.} Fundamental teoremalar. \\
\textbf{T2.} Ratsional sonlar maydonini haqiqiy sonlar maydonigacha to'ldirish. \\
\textbf{A1.} \(\mathbb{Q}\) Ratsianal sonlar maydoni ustida minimal ko'phadalrini toping.
\(\sqrt{2 + 2\sqrt{2}}\). \\
\textbf{A2.} \(\mathbb{Q}\) Ratsianal sonlar maydoni ustida minimal ko'phadalrini toping.
\[\sqrt{2} + \sqrt{5}\] \\
\textbf{A3.} \(\mathbb{Q}\) Ratsianal sonlar maydoni ustida minimal ko'phadni toping.
\[\sqrt{2} + \sqrt{3}\] \\
\textbf{B1.} Quydagi maydon bo'ladimi?
\(\mathbb{Q}\left( \sqrt{5},\sqrt{7} \right) = \left\{ a + b\sqrt{5} + c\sqrt{7} + d\sqrt{35}:a,b,c,d \in \mathbb{Q} \right\}\). \\
\textbf{B2.} \(R\) halqani \(R'\) halqaga o'tkazuvchi gomomorfizmini aniqlang.
\(R = (\mathbb{Z}_{4}, +_{4}, \cdot_{4})\) hám \(R' = (\mathbb{Z}_{6}, +_{6}, \cdot_{6})\). \\
\textbf{B3.} Quydagi maydonning berilgan ko'phadlar orqali ajralish maydonini toping.
\(\mathbb{Q}\) da \(x^{4} - 5x^{2} + 21\). \\
\textbf{C1.} Quydagi maydon kengaytmasining har birining bazisini toping. Har bir kengaytmaning darajasi qanday?
\(\mathbb{Q}\left( \sqrt{3} + \sqrt{5} \right)\) da \(\mathbb{Q}\left( \sqrt{2},\sqrt{6} + \sqrt{10} \right)\) \\
\textbf{C2.} Quydagi halqaning barcha ideallarini toping. Bul ideallardan qaysi-biri maksimal bo'ladi?
\(\mathbb{M}_{2}\left( \mathbb{Z} \right)\), elementleri \(\mathbb{Z}\) bol\(g'\)an \(2 \times 2\) matrica \\
\textbf{C3.} Quyidagi akslantirishni gomomorfizm shartlariga tekshiring.
\[f:x \rightarrow x^{p}\] \\

\end{tabular}
\vspace{1cm}


\begin{tabular}{m{17cm}}
\textbf{95-variant}
\newline

\textbf{T1.} p-adik sonlar maydoni va ular ustida amallar. \\
\textbf{T2.} Geometrik konstruksiyasi. \\
\textbf{A1.} \(\mathbb{Q}\) Ratsianal sonlar maydoni ustida minimal ko'phadalrini toping.
\(\sqrt{3} + \sqrt{7}\). \\
\textbf{A2.} Quydagini hisoblang:
\(\mathbb{Z}_{12}\) de \(\left( 5x^{2} + 3x - 2 \right)^{2}\) \\
\textbf{A3.} \(\mathbb{Z}_{4}\lbrack x\rbrack\) da \(\deg\ p(x) > 1\) bo'ladigan \(p(x)\) birligini toping. Quydagi ko'phadlarning qaysilari \(\mathbb{Q\lbrack}x\rbrack\) da keltirilmaydigan?
\[x^{4} - 2x^{3} + 2x^{2} + x + 4\] \\
\textbf{B1.} Quydagi maydon bo'ladimi:
\[\mathbb{Q}\left( \sqrt[3]{5} \right) = \left\{ a + b\sqrt[3]{5}:a,b,c \in \mathbb{Q} \right\}\] \\
\textbf{B2.} \(R\) halqani \(R'\) halqaga o'tkazuvchi gomomorfizmini aniqlang.
\(R\mathbb{= (Z,} + , \cdot )\) va \(R' = (\mathbb{Z}_{6}, +_{6}, \cdot_{6})\). \\
\textbf{B3.} Quydagi matritsalar to'plamining qaysi biri matritsalarni qo'shish va ko'paytirish amallarga qarata halqa bo'ladi.
\(M_{2 \times 2}\mathbb{(R) =}\left\{ \begin{pmatrix}
a & b \\
 - b & d
\end{pmatrix}\ :\ a,b,c,d \in \mathbb{R} \right\}\). \\
\textbf{C1.} Quydagi maydon kengaytmasining har birining bazisini toping. Har bir kengaytmaning darajasi qanday?
\(\mathbb{Q}\) da \(\mathbb{Q}\left( \sqrt{3},\sqrt{6} \right)\). \\
\textbf{C2.} Quydagi halqaning barcha ideallarini toping. Bul ideallardan qaysi-biri maksimal bo'ladi?
\(\mathbb{M}_{2}\left( \mathbb{Z} \right)\), elementleri \(\mathbb{Z}\) bol\(g'\)an \(2 \times 2\) matrica \\
\textbf{C3.} Quyidagi akslantirishni gomomorfizm shartlariga tekshiring.
\[f\left( a - \sqrt{2}b \right) = a + \sqrt{2}b\] \\

\end{tabular}
\vspace{1cm}


\begin{tabular}{m{17cm}}
\textbf{96-variant}
\newline

\textbf{T1.} Keltirilmaydigan ko'phadlar. \\
\textbf{T2.} Bo'lish algoritmi. \\
\textbf{A1.} \(\mathbb{Q}\) Ratsianal sonlar maydoni ustida minimal ko'phadalrini toping.
\(\sqrt{2 + \sqrt{3}}\). \\
\textbf{A2.} Quydagini hisoblang:
\(\mathbb{Z}_{5}\) te \(\left( 3x^{2} + 2x - 4 \right) + \left( 4x^{2} + 2 \right)\) \\
\textbf{A3.} \(\mathbb{Z}_{4}\lbrack x\rbrack\) da \(\deg\ p(x) > 1\) bo'ladigan \(p(x)\) birligini toping. Quydagi ko'phadlarning qaysilari \(\mathbb{Q\lbrack}x\rbrack\) da keltirilmaydigan?
\[x^{4} - 2x^{3} + 2x^{2} + x + 4\] \\
\textbf{B1.} Quydagilarning qaysi biri maydon bo'ladi:
\[5\mathbb{Z}\] \\
\textbf{B2.} Quydagilar halqa bo'ladimi bunda \(i^{2} = - 1\):
\(\mathbb{Z(}i\sqrt{n}) = \{ x + iy\sqrt{n}\ |\ x,y \in \mathbb{Z\}}\). \\
\textbf{B3.} Quydagi maydonning berilgan ko'phadlar orqali ajralish maydonini toping.
\(\mathbb{Q}\) da \(x^{4} - 10x^{2} + 21\). \\
\textbf{C1.} Quydagi maydon kengaytmasining har birining bazisini toping. Har bir kengaytmaning darajasi qanday?
\(\mathbb{Q}\left( \sqrt{3} + \sqrt{5} \right)\) da \(\mathbb{Q}\left( \sqrt{2},\sqrt{6} + \sqrt{10} \right)\) \\
\textbf{C2.} Quydagi halqaning barcha ideallarini toping. Bul ideallardan qaysi-biri maksimal bo'ladi?
\(\mathbb{M}_{2}\left( \mathbb{Z} \right)\), elementleri \(\mathbb{Z}\) bol\(g'\)an \(2 \times 2\) matrica \\
\textbf{C3.} Quyidagi akslantirishni gomomorfizm shartlariga tekshiring. \(f(x) = 5^{x}\) \\

\end{tabular}
\vspace{1cm}


\begin{tabular}{m{17cm}}
\textbf{97-variant}
\newline

\textbf{T1.} Maydonlarning kengaytmasi. Algebrik element. Algebraik yopilma. \\
\textbf{T2.} Chekli maydonlarning strukturasi \\
\textbf{A1.} \(\mathbb{Q}\) Ratsianal sonlar maydoni ustida minimal ko'phadalrini toping.
\(\sqrt{2 - \sqrt{2}}\). \\
\textbf{A2.} Quydagini hisoblang:
\(\mathbb{Z}_{5}\) te \(\left( 3x^{2} + 3x - 4 \right)\left( 4x^{2} + 2 \right)\) \\
\textbf{A3.} \(\mathbb{Q}\) Ratsianal sonlar maydoni ustida minimal ko'phadni toping.
\(\sqrt{3} + \sqrt{5}\). \\
\textbf{B1.} Quydagi maydon bo'ladimi:
\(\mathbb{Z}\left( \sqrt{5} \right) = \left\{ a + b\sqrt{5}:a,b \in \mathbb{Z} \right\}\). \\
\textbf{B2.} Quydagi maydon bo'ladimi:
\(5\mathbb{Z}\). \\
\textbf{B3.} Quydagi maydonlarning berilgan ko'phadlar orqali ajralish maydonini toping.
\(\mathbb{Q}\) da \(x^{4} - 10x^{2} + 21\). \\
\textbf{C1.} Quydagi maydon kengaytmasining har birining bazisini toping. Har bir kengaytmaning darajasi qanday?
\(\mathbb{Q}\left( \sqrt{3} + \sqrt{5} \right)\) da \(\mathbb{Q}\left( \sqrt{2},\sqrt{6} + \sqrt{10} \right)\) \\
\textbf{C2.} Quydagi halqaning barcha ideallarini toping. Bul ideallardan qaysi-biri maksimal bo'ladi?
\(\mathbb{M}_{2}\left( \mathbb{Z} \right)\), elementleri \(\mathbb{Z}\) bol\(g'\)an \(2 \times 2\) matrica \\
\textbf{C3.} Quyidagi akslantirishni gomomorfizm shartlariga tekshiring.
\[f:\begin{pmatrix}
a & b \\
 - b & a
\end{pmatrix} \rightarrow a + bi\] \\

\end{tabular}
\vspace{1cm}


\begin{tabular}{m{17cm}}
\textbf{98-variant}
\newline

\textbf{T1.} Integral sohalar va maydonlar. \\
\textbf{T2.} Fundamental teoremalar. \\
\textbf{A1.} \(\mathbb{Q}\) Ratsianal sonlar maydoni ustida minimal ko'phadalrini toping.
\(\sqrt{3 - \sqrt{3}}\). \\
\textbf{A2.} Quydagini hisoblang:
\(\mathbb{Z}_{9}\) da \(\left( 7x^{3} + 3x^{2} - x \right) + \left( 6x^{2} - 8x + 4 \right)\) \\
\textbf{A3.} Quydagi halqalarning qism to'plamlari ideal bo'lishini ko'rsating: \(R = \left\{ \begin{pmatrix}
a & b \\
0 & c \\
 & 
\end{pmatrix}\ |\ a,b,c \in \mathbb{Z} \right\}\), \(I = \left\{ \begin{pmatrix}
0 & b \\
0 & c
\end{pmatrix}\ |\ a \in \mathbb{Z} \right\}\). \\
\textbf{B1.} Quydagi halqa bo'ladimi:
\(\mathbb{Q}\left( \sqrt{2} \right) = \left\{ a + b\sqrt{2}:a,b \in \mathbb{Q} \right\}\). \\
\textbf{B2.} Quydagilarning qaysi biri maydon bo'ladi, bunda \(i^{2} = - 1\):
\(\mathbb{Z\lbrack}i\rbrack = \{ x + iy\ |\ x,y \in \mathbb{Z\}}\). \\
\textbf{B3.} Quydagi maydonlarning berilgan ko'phadlar orqali ajralish maydonini toping.
\(\mathbb{Z}_{2}\) da \(x^{2} + 1\). \\
\textbf{C1.} Quydagi maydon kengaytmasining har birining bazisini toping. Har bir kengaytmaning darajasi qanday?
\(\mathbb{Q}\) da \(\mathbb{Q}\left( \sqrt{2},\sqrt[3]{2} \right)\) \\
\textbf{C2.} Quydagi halqaning barcha ideallarini toping. Bul ideallardan qaysi-biri maksimal bo'ladi?
\[\mathbb{Q}\] \\
\textbf{C3.} Quyidagi akslantirishni gomomorfizm shartlariga tekshiring. \(f\left( \begin{pmatrix}
a & 0 \\
0 & a
\end{pmatrix} \right) = a\) \\

\end{tabular}
\vspace{1cm}


\begin{tabular}{m{17cm}}
\textbf{99-variant}
\newline

\textbf{T1.} Maydonlarning ajralishi. \\
\textbf{T2.} Ko'phadlarning halqasi. \\
\textbf{A1.} Quydagi halqaning qism to'plamlari ideal bo'lishini ko'rsating.
\(R = \mathbb{Z}_{24}\), \(I = \{\overline{0},\overline{8},\overline{16}\}\). \\
\textbf{A2.} Quydagini hisoblang:
\(\mathbb{Z}_{12}\) de \(\left( 5x^{2} + 3x - 4 \right)\left( 4x^{2} - x + 9 \right)\) \\
\textbf{A3.} \(\mathbb{Z}_{4}\lbrack x\rbrack\) da \(\deg\ p(x) > 1\) bo'ladigan \(p(x)\) birligini toping. Quydagi ko'phadlarning qaysilari \(\mathbb{Q\lbrack}x\rbrack\) da keltirilmaydigan?
\[x^{4} - 5x^{3} + 3x - 2\] \\
\textbf{B1.} Quydagi halqa bo'ladimi:
\[\mathbb{Q}\left( \sqrt{2},\sqrt{3} \right) = \left\{ a + b\sqrt{2} + c\sqrt{3} + d\sqrt{6}:a,b,c,d \in \mathbb{Q} \right\}\] \\
\textbf{B2.} Quydagilar halqa bo'ladimi bunda \(i^{2} = - 1\):
\(\mathbb{Q(}i) = \{ x + iy\ |\ x,y \in \mathbb{Q\}}\). \\
\textbf{B3.} Quydagi maydonning berilgan ko'phadlar orqali ajralish maydonini toping.
\(\mathbb{Z}_{3}\) te \(x^{3} + 2x + 2\). \\
\textbf{C1.} Quydagi maydon kengaytmasining har birining bazisini toping. Har bir kengaytmaning darajasi qanday?
\(\mathbb{Q}\left( \sqrt{2} \right)\) da \(\mathbb{Q}\left( \sqrt{8} \right)\) \\
\textbf{C2.} Quydagi halqaning barcha ideallarini toping. Bul ideallardan qaysi-biri maksimal bo'ladi?
\(\mathbb{M}_{2}\left( \mathbb{Z} \right)\), elementleri \(\mathbb{Z}\) bol\(g'\)an \(2 \times 2\) matrica \\
\textbf{C3.} Quyidagi akslantirishni gomomorfizm shartlariga tekshiring. \(f(a + ib) = \begin{pmatrix}
a & b \\
 - b & a
\end{pmatrix}\) \\

\end{tabular}
\vspace{1cm}


\begin{tabular}{m{17cm}}
\textbf{100-variant}
\newline

\textbf{T1.} Halqalar. \\
\textbf{T2.} Halqalarning gomomorfizmi va ideallar. \\
\textbf{A1.} Quydagi ko'phadlarning barcha nollarini toping:
\(\mathbb{Z}_{7}\) de \(5x^{4} + 2x^{2} - 3\); \\
\textbf{A2.} Quydagini hisoblang:
\(\mathbb{Z}_{5}\) te \(\left( 3x^{2} + 3x - 4 \right)\left( 4x^{2} + 2 \right)\) \\
\textbf{A3.} \(\mathbb{Z}_{4}\lbrack x\rbrack\) da \(\deg\ p(x) > 1\) bo'ladigan \(p(x)\) birligini toping. Quydagi ko'phadlarning qaysilari \(\mathbb{Q\lbrack}x\rbrack\) da keltirilmaydigan?
\[5x^{5} - 6x^{4} - 3x^{2} + 9x - 15\] \\
\textbf{B1.} Quydagilarning qaysi biri maydon bo'ladi:
\(\mathbb{Q}\left( \sqrt{11} \right) = \left\{ a + b\sqrt{11}:a,b \in \mathbb{Q} \right\}\). \\
\textbf{B2.} Quydagi halqa bo'ladimi:
\(\mathbb{R\lbrack}\sigma\rbrack = \{ x + \sigma \cdot y\ |\ \sigma^{2} = 0\}\). \\
\textbf{B3.} Quydagi matritsalar to'plamining qaysi biri matritsalarni qo'shish va ko'paytirish amallarga qarata halqa bo'ladi .
\(M_{2 \times 2}\mathbb{(R) =}\left\{ \begin{pmatrix}
a & b \\
0 & c
\end{pmatrix}\ :\ a,b,c \in \mathbb{R} \right\}\). \\
\textbf{C1.} Quydagi maydon kengaytmasining har birining bazisini toping. Har bir kengaytmaning darajasi qanday?
\(\mathbb{Q}\) da \(\mathbb{Q}\left( \sqrt{3},\sqrt{5},\sqrt{7} \right)\) \\
\textbf{C2.} Quydagi halqaning barcha ideallarini toping. Bul ideallardan qaysi-biri maksimal bo'ladi?
\[\mathbb{Z}_{18}\] \\
\textbf{C3.} Quyidagi akslantirishni gomomorfizm shartlariga tekshiring. \(f(x) = e^{x}\) \\

\end{tabular}
\vspace{1cm}



\end{document}
