\documentclass{article}
\usepackage[fontsize=12pt]{fontsize}
\usepackage[utf8]{inputenc}
\usepackage[T2A]{fontenc}
% \usepackage{unicode-math}

\usepackage{array}
\usepackage[a4paper,
left=7mm,
right=5mm,
top=7mm,]{geometry}
\usepackage{amsmath}
% \usepackage{amssymbol}
\usepackage{amsfonts}
\usepackage{setspace}
\onehalfspacing  % 1.5 line spacing



\renewcommand{\baselinestretch}{1} 

\everymath{\displaystyle}
\everydisplay{\displaystyle}
% \linespread{1.25}

\DeclareMathOperator{\sign}{sign}


\begin{document}

\pagenumbering{gobble}


\begin{tabular}{m{17cm}}
\textbf{1-variant}
\newline

\textbf{T1.} p-Adikalıq sanlardıń keńisligi (anıqlaması, qásiyetleri, mısallar) \\
\textbf{T2.} Bólshekler maydanı (anıqlaması, qásiyetleri, mısallar) \\
\textbf{A1.} \(\mathbf{Q}\)da \(\sqrt{2} + \sqrt{3}i\)tiń minimal kópaǵzalısın tabıń. \\
\textbf{A2.} Ámellerdi orınlań: \(\mathbb{Z}_{9}\) da \(\left( 7x^{3} + 3x^{2} - x \right) + \left( 6x^{2} - 8x + 4 \right)\) \\
\textbf{A3.} Tómendegi sannıń \(p\)-adikalıq normasın tabıń. \(|\frac{25}{36}|_{2} =\) \\
\textbf{B1.} Tómendegi kóplik kolco dúzedi ma? \(7\mathbb{Z}\) \\
\textbf{B2.} Tómendegi kóplik\(M_{2 \times 2}\left( \mathbb{R} \right)\)kolcosınıń úles kolcosı ekenliginkórsetiń. \(A = \left\{ \left. \ \begin{pmatrix}
a & b \\
0 & a
\end{pmatrix} \right|a,b\mathbb{\in Q} \right\}\) \\
\textbf{B3.} Tómendegi kópaǵzalı\(\mathbf{Z}_{2}\lbrack x\rbrack\)da keltirilmeytuǵınkópaǵzalıboladıma? \(x^{3} + x + 1\) \\
\textbf{C1.} Tómendegi kóplikti maydan shártlerine tekseriń. \(\mathbb{Q}\left\lfloor i \right\rfloor = \left\{ a + bi\ \ :\ \ a,b\mathbb{\in Q} \right\}\) \\
\textbf{C2.} Tómendegi kolconıń úles kóplikleri ideal bolıwın kórsetiń:
\(R = \left\{ \begin{pmatrix}
a & b \\
0 & c
\end{pmatrix}\ |\ a,b,c \in \mathbb{Z} \right\}\), \(I = \left\{ \begin{pmatrix}
0 & a \\
0 & 0
\end{pmatrix}\ |\ a \in \mathbb{Z} \right\}\) \\
\textbf{C3.} Tómendegi sáwlelendiriwdi gomomorfizm shártlerine tekseriń. \(f(x) = \sqrt[3]{x}\) \\

\end{tabular}
\vspace{1cm}


\begin{tabular}{m{17cm}}
\textbf{2-variant}
\newline

\textbf{T1.} Kolcolar (anıqlaması, qásiyetleri, mısallar) \\
\textbf{T2.} Pútinlik oblastı hám maydan (anıqlaması, qásiyetleri, mısallar) \\
\textbf{A1.} \(\mathbf{Q}\)da \(\sqrt{2} + \sqrt{5}\)tiń minimal kópaǵzalısın tabıń. \\
\textbf{A2.} Ámellerdi orınlań: \(\mathbb{Z}_{5}\) te \(\left( x^{2} + 3x - 4 \right)\left( x^{2} - 3 \right)\) \\
\textbf{A3.} Tómendegi sannıń \(p\)-adikalıq normasın tabıń. \(|15|_{3} =\) \\
\textbf{B1.} Tómendegi kóplik kolco dúzedi ma? \(\mathbf{Q} = \left\{ \frac{m}{n},\ \ m \in Z,\ n \in N \right\}\) \\
\textbf{B2.} \(Z_{12}\) kolconıń barlıq úles kolcoların anıqlań. \\
\textbf{B3.} Tómendegi maydanlardıń berilgen kópaǵzalı arqalı ajıralıw maydanın tabıń.
\(\mathbb{Q}\) da \(x^{4} - 5x^{2} + 21\). \\
\textbf{C1.} Tómendegi kóplikti maydan shártlerine tekseriń. \(\mathbb{Z}\left( \sqrt{3} \right) = \left\{ a + b\sqrt{3}:a,b \in \mathbb{Z} \right\}\) \\
\textbf{C2.} Tómendegi kolconıń úles kóplikleri ideal bolıwın kórsetiń:
\(R = \mathbb{Z}_{28}\), \(I = \left\{ \overline{0},\overline{7},\overline{14},\overline{21} \right\}\) \\
\textbf{C3.} Tómendegi sáwlelendiriwdi gomomorfizm shártlerine tekseriń. \(f\left( \begin{pmatrix}
a & 0 \\
0 & a
\end{pmatrix} \right) = a\) \\

\end{tabular}
\vspace{1cm}


\begin{tabular}{m{17cm}}
\textbf{3-variant}
\newline

\textbf{T1.} Kópagzalılar kolcosı (anıqlaması, qásiyetleri, mısallar) \\
\textbf{T2.} Kolco gomomorfizmleri hám ideallar (anıqlaması, qásiyetleri, mısallar) \\
\textbf{A1.} \(\mathbf{Q}\)da \(\sqrt{2 + 2\sqrt{2}}\)tiń minimal kópaǵzalısın tabıń. \\
\textbf{A2.} Tómendegi kópaǵzalınıń barlıq nollerin tabıń: \(\mathbb{Z}_{5}\) de \(3x^{3} - 4x^{2} - x + 4\) \\
\textbf{A3.} Tómendegi sannıń \(p\)-adikalıq normasın tabıń. \(|\frac{3}{4}|_{2} =\) \\
\textbf{B1.} Tómendegi kóplik kolco dúzedi ma? \(\mathbb{Q(}i\sqrt{n}) = \{ x + iy\sqrt{n}\ |\ x,y \in \mathbb{Q}\}\) \\
\textbf{B2.} Tómendegi kóplik\(M_{2 \times 2}\left( \mathbb{R} \right)\) niń úles maydanı ekenliginkórsetiń. \(A = \left\{ \left. \ \begin{pmatrix}
a & b \\
 - b & a
\end{pmatrix} \right|a,b\mathbb{\in Z},a^{2} + b^{2} \neq 0 \right\}\) \\
\textbf{B3.} Tómendegi kópaǵzalı \(\mathbb{Q\lbrack}x\rbrack\) da keltirilmeytuǵın kópaǵzalı boladıma? \(3x^{5} - 4x^{3} - 6x^{2} + 6\) \\
\textbf{C1.} Tómendegi kóplikti maydan shártlerine tekseriń. \(G = \left\{ a^{n},a \neq 0, \pm 1,n \in \mathbb{Z} \right\}\) \\
\textbf{C2.} \(M_{2 \times 2}\left( \mathbf{Z} \right)\)kolcoda\(I = \left\{ \begin{bmatrix}
a & 0 \\
0 & 0
\end{bmatrix}|a\mathbb{\in Z} \right\}\) ideal boladı ma? \\
\textbf{C3.} Tómendegi sáwlelendiriwdi gomomorfizm shártlerine tekseriń.
\[f:x \rightarrow x^{p}\] \\

\end{tabular}
\vspace{1cm}


\begin{tabular}{m{17cm}}
\textbf{4-variant}
\newline

\textbf{T1.} Maydanlar keńeytpesi (anıqlaması, qásiyetleri, mısallar) \\
\textbf{T2.} p-Adikalıq norma, p-Adikalıq norma (anıqlaması, qásiyetleri, mısallar) \\
\textbf{A1.} \(\mathbf{Q}\)da \(\sqrt{2 + \sqrt{2}}\)tiń minimal kópaǵzalısın tabıń. \\
\textbf{A2.} Ámellerdi orınlań: \(\mathbb{Z}_{5}\) te \(\left( 3x^{2} + 3x - 4 \right)\left( x^{2} + 2 \right)\) \\
\textbf{A3.} Tómendegi sannıń \(p\)-adikalıq normasın tabıń. \(|\frac{1}{4}|_{2} =\) \\
\textbf{B1.} Tómendegi kóplik kolco dúzedi ma? \(\mathbb{Q}\left\lfloor i \right\rfloor = \left\{ a + bi\ \ :\ \ a,b\mathbb{\in Q} \right\}\) \\
\textbf{B2.} Tómendegi kóplik\(M_{2 \times 2}\left( \mathbb{R} \right)\)kolcosınıń úles kolcosı ekenliginkórsetiń. \(A = \left\{ \left. \ \begin{pmatrix}
a & 0 \\
0 & 0
\end{pmatrix} \right|a \in Z \right\}\) \\
\textbf{B3.} \(x^{2} - 7\) kópaǵzalı\(\mathbb{Q}(\sqrt{3})\)de keltirilmeytuǵın kópaǵzalı ekenligin kórsetiń. \\
\textbf{C1.} Tómendegi kóplikti maydan shártlerine tekseriń. \(G = \left\{ a^{n},a \neq 0, \pm 1,n \in \mathbb{Z} \right\}\) \\
\textbf{C2.} Tómendegi kolconıń úles kóplikleri ideal bolıwın kórsetiń:
\(R\mathbb{= Z\lbrack}\sqrt{7}\rbrack\), \(I = \{ a + b\sqrt{7}\ |\ \ a,b \in \mathbb{Z,}a - b\ \ jupsan\}\). \\
\textbf{C3.} Tómendegi sáwlelendiriwdi gomomorfizm shártlerine tekseriń. \(f\left( a + \sqrt{2}b \right) = a + bi\) \\

\end{tabular}
\vspace{1cm}


\begin{tabular}{m{17cm}}
\textbf{5-variant}
\newline

\textbf{T1.} Keltirilmeytuǵın kópaǵzalılar (anıqlaması, qásiyetleri, mısallar) \\
\textbf{T2.} Teńlemelerdiń radikallarda sheshiliwi (anıqlaması, qásiyetleri, mısallar) \\
\textbf{A1.} \(\mathbf{Q}\)da \(\sqrt{3 - \sqrt{3}}\)tiń minimal kópaǵzalısın tabıń. \\
\textbf{A2.} Ámellerdi orınlań: \(\mathbb{Z}_{12}\) de \(\left( 5x^{2} + 3x - 2 \right)^{2}\) \\
\textbf{A3.} Tómendegi sannıń \(p\)-adikalıq normasın tabıń. \(|35|_{7} =\) \\
\textbf{B1.} Tómendegi kóplik kolco dúzedi ma? \(R = \left\{ \begin{pmatrix}
a & b \\
2b & a
\end{pmatrix}\ \ :\ \ a,b \in \mathbf{Q} \right\}\) \\
\textbf{B2.} Tómendegi kóplik\(M_{2 \times 2}\left( \mathbb{R} \right)\) niń úles maydanı ekenliginkórsetiń.
\[A = \left\{ \left. \ \begin{pmatrix}
a & b \\
0 & a
\end{pmatrix} \right|a,b\mathbb{\in Z},a \neq 0 \right\}\] \\
\textbf{B3.} Tómendegi maydanlardıń berilgen kópaǵzalı arqalı ajıralıw maydanın tabıń. \(\mathbb{Q}\) da \(x^{4} - 10x^{2} + 21\). \\
\textbf{C1.} Tómendegi kóplikti maydan shártlerine tekseriń. \(7\mathbb{Z}\) \\
\textbf{C2.} Tómendegi kolconıń barlıq ideallarıń tabıń. Bul ideallardan qaysı-biri maksimal boladı? \(\mathbb{M}_{2}\left( \mathbb{R} \right)\), elementleri \(\mathbb{R}\) bolǵan \(2 \times 2\) matrica \\
\textbf{C3.} Tómendegi sáwlelendiriwdi gomomorfizm shártlerine tekseriń.
\[f:\begin{pmatrix}
a & b \\
 - b & a
\end{pmatrix} \rightarrow a + bi\] \\

\end{tabular}
\vspace{1cm}


\begin{tabular}{m{17cm}}
\textbf{6-variant}
\newline

\textbf{T1.} Fundamental teoremalar (anıqlaması, qásiyetleri, mısallar) \\
\textbf{T2.} Maydanlar avtomorfizmleri (anıqlaması, qásiyetleri, mısallar) \\
\textbf{A1.} \(\mathbf{Q}\)da \(\sqrt{3} + \sqrt{2}i\) tiń minimal kópaǵzalısın tabıń. \\
\textbf{A2.} Ámellerdi orınlań: \(\mathbb{Z}_{5}\) te \(\left( x^{2} + 3x - 1 \right)^{2}\) \\
\textbf{A3.} Tómendegi sannıń \(p\)-adikalıq normasın tabıń. \(|\frac{8}{18}|_{5} =\) \\
\textbf{B1.} Tómendegi kóplik kolco dúzedi ma? \(\mathbb{Q(}\sqrt[3]{2}) = \left\{ a + b\sqrt[3]{2}:a,b \in \mathbb{Q} \right\}\) \\
\textbf{B2.} \(Z_{20}\) kolconıń barlıq úles kolcoların anıqlań. \\
\textbf{B3.} Tómendegi maydanlardıń berilgen kópaǵzalı arqalı ajıralıw maydanın tabıń. \(\mathbb{Q}\) da \(x^{4} - 2\) \\
\textbf{C1.} Tómendegi kóplikti maydan shártlerine tekseriń. \(R = \left\{ \begin{pmatrix}
a & b \\
2b & a
\end{pmatrix}\ \ :\ \ a,b \in \mathbf{Q} \right\}\) \\
\textbf{C2.} \(Z_{12}\) niń barlıq idealların tabıń. \\
\textbf{C3.} Tómendegi sáwlelendiriwdi gomomorfizm shártlerine tekseriń. \(f(x) = \sqrt{x}\) \\

\end{tabular}
\vspace{1cm}


\begin{tabular}{m{17cm}}
\textbf{7-variant}
\newline

\textbf{T1.} Maksimal hám ápiwayı ideallar (anıqlaması, qásiyetleri, mısallar) \\
\textbf{T2.} Ajıralatuǵın maydanlar (anıqlaması, qásiyetleri, mısallar) \\
\textbf{A1.} \(\mathbf{Q}\)da \(\sqrt{2} + \sqrt[3]{7}\)tiń minimal kópaǵzalısın tabıń. \\
\textbf{A2.} Ámellerdi orınlań: \(\mathbb{Z}_{12}\) de \(\left( 5x^{2} + 3x - 4 \right) + \left( 4x^{2} - x + 9 \right)\) \\
\textbf{A3.} Tómendegi sannıń \(p\)-adikalıq normasın tabıń. \(|625|_{5} =\) \\
\textbf{B1.} Tómendegi kóplik kolco dúzedi ma? \(G = \left\{ a^{n},a \neq 0, \pm 1,n \in \mathbb{Z} \right\}\) \\
\textbf{B2.} Tómendegi kóplik\(M_{2 \times 2}\left( \mathbb{R} \right)\)kolcosınıń úles kolcosı ekenliginkórsetiń.
\[A = \left\{ \left. \ \begin{pmatrix}
a & b \\
 - b & a
\end{pmatrix} \right|a,b\mathbb{\in R} \right\}\] \\
\textbf{B3.} Tómendegi maydanlardıń berilgen kópaǵzalı arqalı ajıralıw maydanın tabıń. \(\mathbb{Q}\) da \(x^{3} - 3\). \\
\textbf{C1.} \(Z_{18}\) kolconıń barlıq úles kolcoların anıqlań. \\
\textbf{C2.} Tómendegi kolconıń barlıq ideallarıń tabıń. Bul ideallardan qaysı-biri maksimal boladı? \(\mathbb{Z}_{25}\) \\
\textbf{C3.} Tómendegi sáwlelendiriwdi gomomorfizm shártlerine tekseriń. \(f:\begin{pmatrix}
a & b \\
0 & c
\end{pmatrix} \rightarrow a\) \\

\end{tabular}
\vspace{1cm}


\begin{tabular}{m{17cm}}
\textbf{8-variant}
\newline

\textbf{T1.} Shekli maydannıń strukturası (anıqlaması, qásiyetleri, mısallar) \\
\textbf{T2.} Kópagzalılar kolcosı (anıqlaması, qásiyetleri, mısallar) \\
\textbf{A1.} \(\mathbf{Q}\)da \(\sqrt{2} + \sqrt{3}\)tiń minimal kópaǵzalısın tabıń. \\
\textbf{A2.} Tómendegi kópaǵzalınıń barlıq nollerin tabıń: \(\mathbb{Z}_{12}\) de \(5x^{3} + 4x^{2} - x + 9\) \\
\textbf{A3.} Tómendegi sannıń \(p\)-adikalıq normasın tabıń. \(|\frac{9}{12}|_{7} =\) \\
\textbf{B1.} Tómendegi kóplik kolco dúzedi ma? \(R = M_{2 \times 2}\left( \mathbf{Q} \right)\) \\
\textbf{B2.} Tómendegi kóplik\(M_{2 \times 2}\left( \mathbb{R} \right)\) niń úles maydanı ekenligin kórsetiń. \(A = \left\{ \left. \ \begin{pmatrix}
a & 0 \\
2b & a
\end{pmatrix} \right|a,b\mathbb{\in R},a \neq 0 \right\}\) \\
\textbf{B3.} Tómendegi kópaǵzalı \(\mathbb{Q\lbrack}x\rbrack\) da keltirilmeytuǵın kópaǵzalı boladıma? \(x^{4} - 5x^{3} + 3x - 2\) \\
\textbf{C1.} Tómendegi kóplikti maydan shártlerine tekseriń. \(R = M_{2 \times 2}\left( \mathbf{Q} \right)\) \\
\textbf{C2.} Tómendegi kolconıń barlıq ideallarıń tabıń. Bul ideallardan qaysı-biri maksimal boladı? \(\mathbb{M}_{2}\left( \mathbb{Z} \right)\), elementleri \(\mathbb{Z}\) bolǵan \(2 \times 2\) matrica \\
\textbf{C3.} Tómendegi sáwlelendiriwdi gomomorfizm shártlerine tekseriń. \(f(x) = x^{2} + x\) \\

\end{tabular}
\vspace{1cm}


\begin{tabular}{m{17cm}}
\textbf{9-variant}
\newline

\textbf{T1.} Pútinlik oblastı hám maydan (anıqlaması, qásiyetleri, mısallar) \\
\textbf{T2.} Kolcolar (anıqlaması, qásiyetleri, mısallar) \\
\textbf{A1.} \(\mathbf{Q}\)da \(\sqrt{3} + \sqrt[3]{5}\)tiń minimal kópaǵzalısın tabıń. \\
\textbf{A2.} Ámellerdi orınlań: \(\mathbb{Z}_{12}\) de \((3x - 2)^{3}\) \\
\textbf{A3.} Tómendegi sannıń \(p\)-adikalıq normasın tabıń. \(|48|_{3} =\) \\
\textbf{B1.} Tómendegi kóplik kolco dúzedi ma? \(\mathbb{Z}\left( \sqrt{3} \right) = \left\{ a + b\sqrt{3}:a,b \in \mathbb{Z} \right\}\) \\
\textbf{B2.} Tómendegi kóplik\(M_{2 \times 2}\left( \mathbb{R} \right)\)kolcosınıń úles kolcosı ekenliginkórsetiń.
\begin{quote}
\[T = \left\{ \begin{pmatrix}
a + b & b \\
 - b & a
\end{pmatrix}\left| \ \ a,b\mathbb{\in Z} \right.\  \right\}\]
\end{quote} \\
\textbf{B3.} \(x^{2} + 1\) kópaǵzalı \(\mathbb{Z}_{3}\)de keltirilmeytuǵın kópaǵzalı ekenligin kórsetiń. \\
\textbf{C1.} Tómendegi kóplikti maydan shártlerine tekseriń. \(R = M_{2 \times 2}\left( \mathbf{Z} \right)\) \\
\textbf{C2.} Tómendegi kolconıń úles kóplikleri ideal bolıwın kórsetiń:
\(R = \mathbb{Z}_{24}\), \(I = \{\overline{0},\overline{8},\overline{16}\}\). \\
\textbf{C3.} Tómendegi sáwlelendiriwdi gomomorfizm shártlerine tekseriń. \(f(x) = e^{x}\) \\

\end{tabular}
\vspace{1cm}


\begin{tabular}{m{17cm}}
\textbf{10-variant}
\newline

\textbf{T1.} Maksimal hám ápiwayı ideallar (anıqlaması, qásiyetleri, mısallar) \\
\textbf{T2.} Shekli maydannıń strukturası (anıqlaması, qásiyetleri, mısallar) \\
\textbf{A1.} \(\mathbf{Q}\)da \(\sqrt{6 + 3\sqrt{2}}\)tiń minimal kópaǵzalısın tabıń. \\
\textbf{A2.} Ámellerdi orınlań: \(\mathbb{Z}_{5}\) te \(\left( 3x^{2} + 2x - 4 \right) + \left( 4x^{2} + 2 \right)\) \\
\textbf{A3.} Tómendegi sannıń \(p\)-adikalıq normasın tabıń. \(|\frac{7}{15}|_{3} =\) \\
\textbf{B1.} Tómendegi kóplik kolco dúzedi ma? \(\left\{ a + b\sqrt{7}|a,b \in R \right\}\) \\
\textbf{B2.} Tómendegi kóplik\(M_{2 \times 2}\left( \mathbb{R} \right)\)kolcosınıń úles kolcosı ekenliginkórsetiń. \(A = \left\{ \left. \ \begin{pmatrix}
a & b \\
7b & a
\end{pmatrix} \right|a,b \in Z \right\}\) \\
\textbf{B3.} \(x^{4} - 5x^{2} + 6\) kópaǵzalı \(\mathbb{Q}\) de keltirilmeytuǵın kópaǵzalı ekenligin kórsetiń. \\
\textbf{C1.} Tómendegi kóplikti maydan shártlerine tekseriń. \(\mathbf{Q} = \left\{ \frac{m}{n},\ \ m \in Z,\ n \in N \right\}\) \\
\textbf{C2.} Tómendegi kolconıń úles kóplikleri ideal bolıwın kórsetiń:
\(R = \mathbb{Z}_{28}\), \(I = \{\overline{0},\overline{7},\overline{14},\overline{21}\}\). \\
\textbf{C3.} Tómendegi sáwlelendiriwdi gomomorfizm shártlerine tekseriń. \(f(x + iy) = x \cdot y\) \\

\end{tabular}
\vspace{1cm}


\begin{tabular}{m{17cm}}
\textbf{11-variant}
\newline

\textbf{T1.} Kolco gomomorfizmleri hám ideallar (anıqlaması, qásiyetleri, mısallar) \\
\textbf{T2.} Maydanlar avtomorfizmleri (anıqlaması, qásiyetleri, mısallar) \\
\textbf{A1.} \(\mathbf{Q}\)da \(\sqrt{\sqrt[3]{2} - i}\)tiń minimal kópaǵzalısın tabıń. \\
\textbf{A2.} Ámellerdi orınlań: \(\mathbb{Z}_{10}\) da \(\left( 7x^{3} + 3x^{2} - x \right) + \left( 6x^{2} - 8x + 4 \right)\) \\
\textbf{A3.} Tómendegi sannıń \(p\)-adikalıq normasın tabıń. \(|256|_{2} =\) \\
\textbf{B1.} Tómendegi kóplik kolco dúzedi ma? \(\mathbb{Z}\left\lbrack \sqrt{n} \right\rbrack = \left\{ x + y\sqrt{n}\ \ \left| \right.\ x,y \in \mathbb{Z} \right\}\) \\
\textbf{B2.} \(Z_{16}\) kolconıń barlıq úles kolcoların anıqlań. \\
\textbf{B3.} \(x^{2} + x + 1\) kópaǵzalı \(\mathbb{Z}_{5}\) de keltirilmeytuǵın kópaǵzalı ekenligin kórsetiń. \\
\textbf{C1.} Tómendegi kóplikti maydan shártlerine tekseriń. \(\left\{ a + b\sqrt{7}|a,b \in R \right\}\) \\
\textbf{C2.} Tómendegi kolconıń barlıq ideallarıń tabıń. Bul ideallardan qaysı-biri maksimal boladı? \(\mathbb{Z}_{27}\) \\
\textbf{C3.} Tómendegi sáwlelendiriwdi gomomorfizm shártlerine tekseriń. \(f(x) = 5^{x}\) \\

\end{tabular}
\vspace{1cm}


\begin{tabular}{m{17cm}}
\textbf{12-variant}
\newline

\textbf{T1.} Ajıralatuǵın maydanlar (anıqlaması, qásiyetleri, mısallar) \\
\textbf{T2.} Fundamental teoremalar (anıqlaması, qásiyetleri, mısallar) \\
\textbf{A1.} \(\mathbf{Q}\)da \(\sqrt{2 + \sqrt{5}}\)tiń minimal kópaǵzalısın tabıń. \\
\textbf{A2.} Tómendegi kópaǵzalınıń barlıq nollerin tabıń: \(\mathbb{Z}_{2}\) de \(x^{3} + x + 1\) \\
\textbf{A3.} Tómendegi sannıń \(p\)-adikalıq normasın tabıń. \(|729|_{3} =\) \\
\textbf{B1.} Tómendegi kóplik kolco dúzedi ma? \(R = \left\{ a + b\sqrt{2}\ :\ \ \ \ a,b \in \mathbf{Z} \right\}\) \\
\textbf{B2.} Tómendegi kóplik\(M_{2 \times 2}\left( \mathbb{R} \right)\)kolcosınıń úles kolcosı ekenliginkórsetiń. \(A = \left\{ \left. \ \begin{pmatrix}
a & b \\
0 & a
\end{pmatrix} \right|a,b\mathbb{\in R} \right\}\) \\
\textbf{B3.} Tómendegi kópaǵzalı \(\mathbb{Z}_{3}\) da keltirilmeytuǵın kópaǵzalı boladıma? \(x^{3} + 2x + 2\) \\
\textbf{C1.} Tómendegi kóplikti maydan shártlerine tekseriń. \(\mathbb{Q(}i\sqrt{n}) = \{ x + iy\sqrt{n}\ |\ x,y \in \mathbb{Q}\}\) \\
\textbf{C2.} \(\mathbb{Z}_{24}\)tiń barlıq idealların tabıń \\
\textbf{C3.} Tómendegi sáwlelendiriwdi gomomorfizm shártlerine tekseriń.
\[f\left( a - \sqrt{2}b \right) = a + \sqrt{2}b\] \\

\end{tabular}
\vspace{1cm}


\begin{tabular}{m{17cm}}
\textbf{13-variant}
\newline

\textbf{T1.} Maydanlar keńeytpesi (anıqlaması, qásiyetleri, mısallar) \\
\textbf{T2.} Teńlemelerdiń radikallarda sheshiliwi (anıqlaması, qásiyetleri, mısallar) \\
\textbf{A1.} \(\mathbf{Q}\)da \(\sqrt{1/3 + \sqrt{7}}\)tiń minimal kópaǵzalısın tabıń. \\
\textbf{A2.} Ámellerdi orınlań: \(\mathbb{Z}_{12}\) de \(\left( 5x^{2} + 3x - 4 \right)\left( 4x^{2} - x + 9 \right)\) \\
\textbf{A3.} Tómendegi sannıń \(p\)-adikalıq normasın tabıń. \(|6|_{3} =\) \\
\textbf{B1.} Tómendegi kóplik kolco dúzedi ma? \(\mathbb{Q}\left( \sqrt{2} \right) = \left\{ a + b\sqrt{2}:a,b \in \mathbb{Q} \right\}\) \\
\textbf{B2.} Tómendegi kóplik\(M_{2 \times 2}\left( \mathbb{R} \right)\) niń úles maydanı ekenliginkórsetiń. \(A = \left\{ \left. \ \begin{pmatrix}
a & 0 \\
0 & a
\end{pmatrix} \right|a\mathbb{\in R},a \neq 0 \right\}\) \\
\textbf{B3.} \(x^{2} - 3\)kópaǵzalı \(\mathbb{Q}(\sqrt{2})\)de keltirilmeytuǵınkópaǵzalı ekenliginkórsetiń. \\
\textbf{C1.} Tómendegi kóplikti maydan shártlerine tekseriń. \(Z_{p}\) \\
\textbf{C2.} \(M_{2 \times 2}\left( \mathbf{Z} \right)\)kolcoda\(I = \left\{ \begin{bmatrix}
a & 0 \\
b & 0
\end{bmatrix}|a,b\mathbb{\in Z} \right\}\) ideal boladı ma? \\
\textbf{C3.} Tómendegi sáwlelendiriwdi gomomorfizm shártlerine tekseriń. \(f(a + ib) = \begin{pmatrix}
a & b \\
 - b & a
\end{pmatrix}\) \\

\end{tabular}
\vspace{1cm}


\begin{tabular}{m{17cm}}
\textbf{14-variant}
\newline

\textbf{T1.} p-Adikalıq norma, p-Adikalıq norma (anıqlaması, qásiyetleri, mısallar) \\
\textbf{T2.} Keltirilmeytuǵın kópaǵzalılar (anıqlaması, qásiyetleri, mısallar) \\
\textbf{A1.} \(\mathbf{Q}\)da \(\sqrt{\frac{1}{4} + \sqrt{5}}\) tiń minimal kópaǵzalısın tabıń. \\
\textbf{A2.} Ámellerdi orınlań: \(\mathbb{Z}_{5}\) te \(\left( 3x^{2} + 3x - 4 \right)\left( 4x^{2} + 2 \right)\) \\
\textbf{A3.} Tómendegi sannıń \(p\)-adikalıq normasın tabıń. \(|124|_{2} =\) \\
\textbf{B1.} Tómendegi kóplik kolco dúzedi ma? \(R = \left\{ \begin{pmatrix}
a & b \\
2b & a
\end{pmatrix}\ \ :\ \ a,b \in \mathbf{Q} \right\}\) \\
\textbf{B2.} Tómendegi kóplik\(M_{2 \times 2}\left( \mathbb{R} \right)\) niń úles maydanı ekenliginkórsetiń. \(A = \left\{ \left. \ \begin{pmatrix}
a & b\sqrt{2} \\
b\sqrt{2} & a
\end{pmatrix} \right|a,b\mathbb{\in Q},a^{2} - 2b^{2} \neq 0 \right\}\) \\
\textbf{B3.} \(x^{2} + x + 1\) kópaǵzalı \(\mathbb{Z}_{3}\) de keltirilmeytuǵın kópaǵzalı ekenligin kórsetiń. \\
\textbf{C1.} Tómendegi kóplikti maydan shártlerine tekseriń. \(\mathbb{Z}\left\lbrack \sqrt{n} \right\rbrack = \left\{ x + y\sqrt{n}\ \ \left| \right.\ x,y \in \mathbb{Z} \right\}\) \\
\textbf{C2.} Tómendegi kolconıń úles kóplikleri ideal bolıwın kórsetiń:
\(R = \left\{ \begin{pmatrix}
a & b \\
0 & c
\end{pmatrix}\ |\ a,b,c \in \mathbb{Z} \right\}\), \(I = \left\{ \begin{pmatrix}
0 & b \\
0 & c
\end{pmatrix}\ |\ a \in \mathbb{Z} \right\}\). \\
\textbf{C3.} Tómendegi sáwlelendiriwdi gomomorfizm shártlerine tekseriń. \(f(a) = a^{n}\) \\

\end{tabular}
\vspace{1cm}


\begin{tabular}{m{17cm}}
\textbf{15-variant}
\newline

\textbf{T1.} Bólshekler maydanı (anıqlaması, qásiyetleri, mısallar) \\
\textbf{T2.} p-Adikalıq sanlardıń keńisligi (anıqlaması, qásiyetleri, mısallar) \\
\textbf{A1.} \(\mathbf{Q}\)da \(\sqrt{2 + \sqrt{5}}\)tiń minimal kópaǵzalısın tabıń. \\
\textbf{A2.} Ámellerdi orınlań: \(\mathbb{Z}_{10}\) da \(\left( 5x^{2} + 3x - 4 \right)\left( 4x^{2} - x + 9 \right)\) \\
\textbf{A3.} Tómendegi sannıń \(p\)-adikalıq normasın tabıń. \(|\frac{1}{4}|_{2} =\) \\
\textbf{B1.} Tómendegi kóplik kolco dúzedi ma? \(R = M_{2 \times 2}\left( \mathbf{Q} \right)\) \\
\textbf{B2.} Tómendegi kóplik\(M_{2 \times 2}\left( \mathbb{R} \right)\) niń úles maydanı ekenliginkórsetiń. \(A = \left\{ \left. \ \begin{pmatrix}
a & b\sqrt{7} \\
 - b\sqrt{7} & a
\end{pmatrix} \right|a,b\mathbb{\in Q},a^{2} + 7b^{2} \neq 0 \right\}\) \\
\textbf{B3.} Tómendegi maydanlardıń berilgen kópaǵzalı arqalı ajıralıw maydanın tabıń. \(\mathbb{Q}\) da \(x^{4} + 1\). \\
\textbf{C1.} Tómendegi kóplikti maydan shártlerine tekseriń. \(7\mathbb{Z}\) \\
\textbf{C2.} Tómendegi kolconıń úles kóplikleri ideal bolıwın kórsetiń:
\(R = \left\{ \begin{pmatrix}
a & b \\
0 & c
\end{pmatrix}\ |\ a,b,c \in \mathbb{Z} \right\}\), \(I = \left\{ \begin{pmatrix}
0 & b \\
0 & c
\end{pmatrix}\ |\ a \in \mathbb{Z} \right\}\). \\
\textbf{C3.} Tómendegi sáwlelendiriwdi gomomorfizm shártlerine tekseriń.
\[f:\begin{pmatrix}
a & b \\
 - b & a
\end{pmatrix} \rightarrow a + bi\] \\

\end{tabular}
\vspace{1cm}


\begin{tabular}{m{17cm}}
\textbf{16-variant}
\newline

\textbf{T1.} Maydanlar keńeytpesi (anıqlaması, qásiyetleri, mısallar) \\
\textbf{T2.} Keltirilmeytuǵın kópaǵzalılar (anıqlaması, qásiyetleri, mısallar) \\
\textbf{A1.} \(\mathbf{Q}\)da \(\sqrt{2} + \sqrt{3}i\)tiń minimal kópaǵzalısın tabıń. \\
\textbf{A2.} Ámellerdi orınlań: \(\mathbb{Z}_{10}\) da \(\left( 7x^{3} + 3x^{2} - x \right) + \left( 6x^{2} - 8x + 4 \right)\) \\
\textbf{A3.} Tómendegi sannıń \(p\)-adikalıq normasın tabıń. \(|6|_{3} =\) \\
\textbf{B1.} Tómendegi kóplik kolco dúzedi ma? \(\mathbb{Z}\left\lbrack \sqrt{n} \right\rbrack = \left\{ x + y\sqrt{n}\ \ \left| \right.\ x,y \in \mathbb{Z} \right\}\) \\
\textbf{B2.} Tómendegi kóplik\(M_{2 \times 2}\left( \mathbb{R} \right)\)kolcosınıń úles kolcosı ekenliginkórsetiń. \(A = \left\{ \left. \ \begin{pmatrix}
a & b \\
7b & a
\end{pmatrix} \right|a,b \in Z \right\}\) \\
\textbf{B3.} \(x^{2} + x + 1\) kópaǵzalı \(\mathbb{Z}_{5}\) de keltirilmeytuǵın kópaǵzalı ekenligin kórsetiń. \\
\textbf{C1.} Tómendegi kóplikti maydan shártlerine tekseriń. \(\mathbb{Z}\left\lbrack \sqrt{n} \right\rbrack = \left\{ x + y\sqrt{n}\ \ \left| \right.\ x,y \in \mathbb{Z} \right\}\) \\
\textbf{C2.} Tómendegi kolconıń barlıq ideallarıń tabıń. Bul ideallardan qaysı-biri maksimal boladı? \(\mathbb{Z}_{25}\) \\
\textbf{C3.} Tómendegi sáwlelendiriwdi gomomorfizm shártlerine tekseriń.
\[f\left( a - \sqrt{2}b \right) = a + \sqrt{2}b\] \\

\end{tabular}
\vspace{1cm}


\begin{tabular}{m{17cm}}
\textbf{17-variant}
\newline

\textbf{T1.} Maksimal hám ápiwayı ideallar (anıqlaması, qásiyetleri, mısallar) \\
\textbf{T2.} Teńlemelerdiń radikallarda sheshiliwi (anıqlaması, qásiyetleri, mısallar) \\
\textbf{A1.} \(\mathbf{Q}\)da \(\sqrt{\frac{1}{4} + \sqrt{5}}\) tiń minimal kópaǵzalısın tabıń. \\
\textbf{A2.} Ámellerdi orınlań: \(\mathbb{Z}_{5}\) te \(\left( 3x^{2} + 3x - 4 \right)\left( x^{2} + 2 \right)\) \\
\textbf{A3.} Tómendegi sannıń \(p\)-adikalıq normasın tabıń. \(|256|_{2} =\) \\
\textbf{B1.} Tómendegi kóplik kolco dúzedi ma? \(\left\{ a + b\sqrt{7}|a,b \in R \right\}\) \\
\textbf{B2.} Tómendegi kóplik\(M_{2 \times 2}\left( \mathbb{R} \right)\)kolcosınıń úles kolcosı ekenliginkórsetiń. \(A = \left\{ \left. \ \begin{pmatrix}
a & b \\
0 & a
\end{pmatrix} \right|a,b\mathbb{\in R} \right\}\) \\
\textbf{B3.} Tómendegi maydanlardıń berilgen kópaǵzalı arqalı ajıralıw maydanın tabıń. \(\mathbb{Q}\) da \(x^{3} - 3\). \\
\textbf{C1.} Tómendegi kóplikti maydan shártlerine tekseriń. \(R = M_{2 \times 2}\left( \mathbf{Z} \right)\) \\
\textbf{C2.} Tómendegi kolconıń úles kóplikleri ideal bolıwın kórsetiń:
\(R\mathbb{= Z\lbrack}\sqrt{7}\rbrack\), \(I = \{ a + b\sqrt{7}\ |\ \ a,b \in \mathbb{Z,}a - b\ \ jupsan\}\). \\
\textbf{C3.} Tómendegi sáwlelendiriwdi gomomorfizm shártlerine tekseriń. \(f(a) = a^{n}\) \\

\end{tabular}
\vspace{1cm}


\begin{tabular}{m{17cm}}
\textbf{18-variant}
\newline

\textbf{T1.} Kolcolar (anıqlaması, qásiyetleri, mısallar) \\
\textbf{T2.} Fundamental teoremalar (anıqlaması, qásiyetleri, mısallar) \\
\textbf{A1.} \(\mathbf{Q}\)da \(\sqrt{3 - \sqrt{3}}\)tiń minimal kópaǵzalısın tabıń. \\
\textbf{A2.} Ámellerdi orınlań: \(\mathbb{Z}_{12}\) de \(\left( 5x^{2} + 3x - 2 \right)^{2}\) \\
\textbf{A3.} Tómendegi sannıń \(p\)-adikalıq normasın tabıń. \(|124|_{2} =\) \\
\textbf{B1.} Tómendegi kóplik kolco dúzedi ma? \(\mathbb{Q(}i\sqrt{n}) = \{ x + iy\sqrt{n}\ |\ x,y \in \mathbb{Q}\}\) \\
\textbf{B2.} Tómendegi kóplik\(M_{2 \times 2}\left( \mathbb{R} \right)\)kolcosınıń úles kolcosı ekenliginkórsetiń.
\[A = \left\{ \left. \ \begin{pmatrix}
a & b \\
 - b & a
\end{pmatrix} \right|a,b\mathbb{\in R} \right\}\] \\
\textbf{B3.} Tómendegi kópaǵzalı \(\mathbb{Z}_{3}\) da keltirilmeytuǵın kópaǵzalı boladıma? \(x^{3} + 2x + 2\) \\
\textbf{C1.} Tómendegi kóplikti maydan shártlerine tekseriń. \(R = \left\{ \begin{pmatrix}
a & b \\
2b & a
\end{pmatrix}\ \ :\ \ a,b \in \mathbf{Q} \right\}\) \\
\textbf{C2.} \(Z_{12}\) niń barlıq idealların tabıń. \\
\textbf{C3.} Tómendegi sáwlelendiriwdi gomomorfizm shártlerine tekseriń. \(f:\begin{pmatrix}
a & b \\
0 & c
\end{pmatrix} \rightarrow a\) \\

\end{tabular}
\vspace{1cm}


\begin{tabular}{m{17cm}}
\textbf{19-variant}
\newline

\textbf{T1.} Kópagzalılar kolcosı (anıqlaması, qásiyetleri, mısallar) \\
\textbf{T2.} Pútinlik oblastı hám maydan (anıqlaması, qásiyetleri, mısallar) \\
\textbf{A1.} \(\mathbf{Q}\)da \(\sqrt{2 + \sqrt{2}}\)tiń minimal kópaǵzalısın tabıń. \\
\textbf{A2.} Ámellerdi orınlań: \(\mathbb{Z}_{5}\) te \(\left( 3x^{2} + 2x - 4 \right) + \left( 4x^{2} + 2 \right)\) \\
\textbf{A3.} Tómendegi sannıń \(p\)-adikalıq normasın tabıń. \(|\frac{9}{12}|_{7} =\) \\
\textbf{B1.} Tómendegi kóplik kolco dúzedi ma? \(R = \left\{ a + b\sqrt{2}\ :\ \ \ \ a,b \in \mathbf{Z} \right\}\) \\
\textbf{B2.} \(Z_{12}\) kolconıń barlıq úles kolcoların anıqlań. \\
\textbf{B3.} \(x^{4} - 5x^{2} + 6\) kópaǵzalı \(\mathbb{Q}\) de keltirilmeytuǵın kópaǵzalı ekenligin kórsetiń. \\
\textbf{C1.} Tómendegi kóplikti maydan shártlerine tekseriń. \(G = \left\{ a^{n},a \neq 0, \pm 1,n \in \mathbb{Z} \right\}\) \\
\textbf{C2.} Tómendegi kolconıń barlıq ideallarıń tabıń. Bul ideallardan qaysı-biri maksimal boladı? \(\mathbb{Z}_{27}\) \\
\textbf{C3.} Tómendegi sáwlelendiriwdi gomomorfizm shártlerine tekseriń. \(f(x) = x^{2} + x\) \\

\end{tabular}
\vspace{1cm}


\begin{tabular}{m{17cm}}
\textbf{20-variant}
\newline

\textbf{T1.} Bólshekler maydanı (anıqlaması, qásiyetleri, mısallar) \\
\textbf{T2.} Maydanlar avtomorfizmleri (anıqlaması, qásiyetleri, mısallar) \\
\textbf{A1.} \(\mathbf{Q}\)da \(\sqrt{3} + \sqrt{2}i\) tiń minimal kópaǵzalısın tabıń. \\
\textbf{A2.} Ámellerdi orınlań: \(\mathbb{Z}_{5}\) te \(\left( x^{2} + 3x - 4 \right)\left( x^{2} - 3 \right)\) \\
\textbf{A3.} Tómendegi sannıń \(p\)-adikalıq normasın tabıń. \(|35|_{7} =\) \\
\textbf{B1.} Tómendegi kóplik kolco dúzedi ma? \(G = \left\{ a^{n},a \neq 0, \pm 1,n \in \mathbb{Z} \right\}\) \\
\textbf{B2.} Tómendegi kóplik\(M_{2 \times 2}\left( \mathbb{R} \right)\)kolcosınıń úles kolcosı ekenliginkórsetiń.
\begin{quote}
\[T = \left\{ \begin{pmatrix}
a + b & b \\
 - b & a
\end{pmatrix}\left| \ \ a,b\mathbb{\in Z} \right.\  \right\}\]
\end{quote} \\
\textbf{B3.} \(x^{2} - 3\)kópaǵzalı \(\mathbb{Q}(\sqrt{2})\)de keltirilmeytuǵınkópaǵzalı ekenliginkórsetiń. \\
\textbf{C1.} Tómendegi kóplikti maydan shártlerine tekseriń. \(R = M_{2 \times 2}\left( \mathbf{Q} \right)\) \\
\textbf{C2.} Tómendegi kolconıń úles kóplikleri ideal bolıwın kórsetiń:
\(R = \mathbb{Z}_{24}\), \(I = \{\overline{0},\overline{8},\overline{16}\}\). \\
\textbf{C3.} Tómendegi sáwlelendiriwdi gomomorfizm shártlerine tekseriń. \(f\left( a + \sqrt{2}b \right) = a + bi\) \\

\end{tabular}
\vspace{1cm}


\begin{tabular}{m{17cm}}
\textbf{21-variant}
\newline

\textbf{T1.} Kolco gomomorfizmleri hám ideallar (anıqlaması, qásiyetleri, mısallar) \\
\textbf{T2.} Shekli maydannıń strukturası (anıqlaması, qásiyetleri, mısallar) \\
\textbf{A1.} \(\mathbf{Q}\)da \(\sqrt{2} + \sqrt{5}\)tiń minimal kópaǵzalısın tabıń. \\
\textbf{A2.} Ámellerdi orınlań: \(\mathbb{Z}_{9}\) da \(\left( 7x^{3} + 3x^{2} - x \right) + \left( 6x^{2} - 8x + 4 \right)\) \\
\textbf{A3.} Tómendegi sannıń \(p\)-adikalıq normasın tabıń. \(|729|_{3} =\) \\
\textbf{B1.} Tómendegi kóplik kolco dúzedi ma? \(\mathbf{Q} = \left\{ \frac{m}{n},\ \ m \in Z,\ n \in N \right\}\) \\
\textbf{B2.} \(Z_{16}\) kolconıń barlıq úles kolcoların anıqlań. \\
\textbf{B3.} \(x^{2} + x + 1\) kópaǵzalı \(\mathbb{Z}_{3}\) de keltirilmeytuǵın kópaǵzalı ekenligin kórsetiń. \\
\textbf{C1.} Tómendegi kóplikti maydan shártlerine tekseriń. \(G = \left\{ a^{n},a \neq 0, \pm 1,n \in \mathbb{Z} \right\}\) \\
\textbf{C2.} Tómendegi kolconıń barlıq ideallarıń tabıń. Bul ideallardan qaysı-biri maksimal boladı? \(\mathbb{M}_{2}\left( \mathbb{Z} \right)\), elementleri \(\mathbb{Z}\) bolǵan \(2 \times 2\) matrica \\
\textbf{C3.} Tómendegi sáwlelendiriwdi gomomorfizm shártlerine tekseriń.
\[f:x \rightarrow x^{p}\] \\

\end{tabular}
\vspace{1cm}


\begin{tabular}{m{17cm}}
\textbf{22-variant}
\newline

\textbf{T1.} Ajıralatuǵın maydanlar (anıqlaması, qásiyetleri, mısallar) \\
\textbf{T2.} p-Adikalıq norma, p-Adikalıq norma (anıqlaması, qásiyetleri, mısallar) \\
\textbf{A1.} \(\mathbf{Q}\)da \(\sqrt{\sqrt[3]{2} - i}\)tiń minimal kópaǵzalısın tabıń. \\
\textbf{A2.} Tómendegi kópaǵzalınıń barlıq nollerin tabıń: \(\mathbb{Z}_{5}\) de \(3x^{3} - 4x^{2} - x + 4\) \\
\textbf{A3.} Tómendegi sannıń \(p\)-adikalıq normasın tabıń. \(|\frac{3}{4}|_{2} =\) \\
\textbf{B1.} Tómendegi kóplik kolco dúzedi ma? \(\mathbb{Q}\left\lfloor i \right\rfloor = \left\{ a + bi\ \ :\ \ a,b\mathbb{\in Q} \right\}\) \\
\textbf{B2.} Tómendegi kóplik\(M_{2 \times 2}\left( \mathbb{R} \right)\)kolcosınıń úles kolcosı ekenliginkórsetiń. \(A = \left\{ \left. \ \begin{pmatrix}
a & b \\
0 & a
\end{pmatrix} \right|a,b\mathbb{\in Q} \right\}\) \\
\textbf{B3.} Tómendegi maydanlardıń berilgen kópaǵzalı arqalı ajıralıw maydanın tabıń. \(\mathbb{Q}\) da \(x^{4} + 1\). \\
\textbf{C1.} Tómendegi kóplikti maydan shártlerine tekseriń. \(\left\{ a + b\sqrt{7}|a,b \in R \right\}\) \\
\textbf{C2.} Tómendegi kolconıń úles kóplikleri ideal bolıwın kórsetiń:
\(R = \mathbb{Z}_{28}\), \(I = \{\overline{0},\overline{7},\overline{14},\overline{21}\}\). \\
\textbf{C3.} Tómendegi sáwlelendiriwdi gomomorfizm shártlerine tekseriń. \(f(x) = \sqrt{x}\) \\

\end{tabular}
\vspace{1cm}


\begin{tabular}{m{17cm}}
\textbf{23-variant}
\newline

\textbf{T1.} p-Adikalıq sanlardıń keńisligi (anıqlaması, qásiyetleri, mısallar) \\
\textbf{T2.} Kópagzalılar kolcosı (anıqlaması, qásiyetleri, mısallar) \\
\textbf{A1.} \(\mathbf{Q}\)da \(\sqrt{6 + 3\sqrt{2}}\)tiń minimal kópaǵzalısın tabıń. \\
\textbf{A2.} Ámellerdi orınlań: \(\mathbb{Z}_{5}\) te \(\left( 3x^{2} + 3x - 4 \right)\left( 4x^{2} + 2 \right)\) \\
\textbf{A3.} Tómendegi sannıń \(p\)-adikalıq normasın tabıń. \(|\frac{8}{18}|_{5} =\) \\
\textbf{B1.} Tómendegi kóplik kolco dúzedi ma? \(\mathbb{Q}\left( \sqrt{2} \right) = \left\{ a + b\sqrt{2}:a,b \in \mathbb{Q} \right\}\) \\
\textbf{B2.} Tómendegi kóplik\(M_{2 \times 2}\left( \mathbb{R} \right)\) niń úles maydanı ekenliginkórsetiń.
\[A = \left\{ \left. \ \begin{pmatrix}
a & b \\
0 & a
\end{pmatrix} \right|a,b\mathbb{\in Z},a \neq 0 \right\}\] \\
\textbf{B3.} Tómendegi kópaǵzalı \(\mathbb{Q\lbrack}x\rbrack\) da keltirilmeytuǵın kópaǵzalı boladıma? \(3x^{5} - 4x^{3} - 6x^{2} + 6\) \\
\textbf{C1.} \(Z_{18}\) kolconıń barlıq úles kolcoların anıqlań. \\
\textbf{C2.} Tómendegi kolconıń úles kóplikleri ideal bolıwın kórsetiń:
\(R = \left\{ \begin{pmatrix}
a & b \\
0 & c
\end{pmatrix}\ |\ a,b,c \in \mathbb{Z} \right\}\), \(I = \left\{ \begin{pmatrix}
0 & a \\
0 & 0
\end{pmatrix}\ |\ a \in \mathbb{Z} \right\}\) \\
\textbf{C3.} Tómendegi sáwlelendiriwdi gomomorfizm shártlerine tekseriń. \(f\left( \begin{pmatrix}
a & 0 \\
0 & a
\end{pmatrix} \right) = a\) \\

\end{tabular}
\vspace{1cm}


\begin{tabular}{m{17cm}}
\textbf{24-variant}
\newline

\textbf{T1.} Fundamental teoremalar (anıqlaması, qásiyetleri, mısallar) \\
\textbf{T2.} Maksimal hám ápiwayı ideallar (anıqlaması, qásiyetleri, mısallar) \\
\textbf{A1.} \(\mathbf{Q}\)da \(\sqrt{3} + \sqrt[3]{5}\)tiń minimal kópaǵzalısın tabıń. \\
\textbf{A2.} Ámellerdi orınlań: \(\mathbb{Z}_{5}\) te \(\left( x^{2} + 3x - 1 \right)^{2}\) \\
\textbf{A3.} Tómendegi sannıń \(p\)-adikalıq normasın tabıń. \(|625|_{5} =\) \\
\textbf{B1.} Tómendegi kóplik kolco dúzedi ma? \(7\mathbb{Z}\) \\
\textbf{B2.} \(Z_{20}\) kolconıń barlıq úles kolcoların anıqlań. \\
\textbf{B3.} \(x^{2} - 7\) kópaǵzalı\(\mathbb{Q}(\sqrt{3})\)de keltirilmeytuǵın kópaǵzalı ekenligin kórsetiń. \\
\textbf{C1.} Tómendegi kóplikti maydan shártlerine tekseriń. \(\mathbf{Q} = \left\{ \frac{m}{n},\ \ m \in Z,\ n \in N \right\}\) \\
\textbf{C2.} Tómendegi kolconıń barlıq ideallarıń tabıń. Bul ideallardan qaysı-biri maksimal boladı? \(\mathbb{M}_{2}\left( \mathbb{R} \right)\), elementleri \(\mathbb{R}\) bolǵan \(2 \times 2\) matrica \\
\textbf{C3.} Tómendegi sáwlelendiriwdi gomomorfizm shártlerine tekseriń. \(f(x) = e^{x}\) \\

\end{tabular}
\vspace{1cm}


\begin{tabular}{m{17cm}}
\textbf{25-variant}
\newline

\textbf{T1.} Bólshekler maydanı (anıqlaması, qásiyetleri, mısallar) \\
\textbf{T2.} p-Adikalıq sanlardıń keńisligi (anıqlaması, qásiyetleri, mısallar) \\
\textbf{A1.} \(\mathbf{Q}\)da \(\sqrt{1/3 + \sqrt{7}}\)tiń minimal kópaǵzalısın tabıń. \\
\textbf{A2.} Tómendegi kópaǵzalınıń barlıq nollerin tabıń: \(\mathbb{Z}_{2}\) de \(x^{3} + x + 1\) \\
\textbf{A3.} Tómendegi sannıń \(p\)-adikalıq normasın tabıń. \(|48|_{3} =\) \\
\textbf{B1.} Tómendegi kóplik kolco dúzedi ma? \(\mathbb{Z}\left( \sqrt{3} \right) = \left\{ a + b\sqrt{3}:a,b \in \mathbb{Z} \right\}\) \\
\textbf{B2.} Tómendegi kóplik\(M_{2 \times 2}\left( \mathbb{R} \right)\) niń úles maydanı ekenligin kórsetiń. \(A = \left\{ \left. \ \begin{pmatrix}
a & 0 \\
2b & a
\end{pmatrix} \right|a,b\mathbb{\in R},a \neq 0 \right\}\) \\
\textbf{B3.} Tómendegi kópaǵzalı \(\mathbb{Q\lbrack}x\rbrack\) da keltirilmeytuǵın kópaǵzalı boladıma? \(x^{4} - 5x^{3} + 3x - 2\) \\
\textbf{C1.} Tómendegi kóplikti maydan shártlerine tekseriń. \(\mathbb{Q(}i\sqrt{n}) = \{ x + iy\sqrt{n}\ |\ x,y \in \mathbb{Q}\}\) \\
\textbf{C2.} Tómendegi kolconıń úles kóplikleri ideal bolıwın kórsetiń:
\(R = \mathbb{Z}_{28}\), \(I = \left\{ \overline{0},\overline{7},\overline{14},\overline{21} \right\}\) \\
\textbf{C3.} Tómendegi sáwlelendiriwdi gomomorfizm shártlerine tekseriń. \(f(a + ib) = \begin{pmatrix}
a & b \\
 - b & a
\end{pmatrix}\) \\

\end{tabular}
\vspace{1cm}


\begin{tabular}{m{17cm}}
\textbf{26-variant}
\newline

\textbf{T1.} Ajıralatuǵın maydanlar (anıqlaması, qásiyetleri, mısallar) \\
\textbf{T2.} Shekli maydannıń strukturası (anıqlaması, qásiyetleri, mısallar) \\
\textbf{A1.} \(\mathbf{Q}\)da \(\sqrt{2} + \sqrt{3}\)tiń minimal kópaǵzalısın tabıń. \\
\textbf{A2.} Tómendegi kópaǵzalınıń barlıq nollerin tabıń: \(\mathbb{Z}_{12}\) de \(5x^{3} + 4x^{2} - x + 9\) \\
\textbf{A3.} Tómendegi sannıń \(p\)-adikalıq normasın tabıń. \(|\frac{7}{15}|_{3} =\) \\
\textbf{B1.} Tómendegi kóplik kolco dúzedi ma? \(\mathbb{Q(}\sqrt[3]{2}) = \left\{ a + b\sqrt[3]{2}:a,b \in \mathbb{Q} \right\}\) \\
\textbf{B2.} Tómendegi kóplik\(M_{2 \times 2}\left( \mathbb{R} \right)\) niń úles maydanı ekenliginkórsetiń. \(A = \left\{ \left. \ \begin{pmatrix}
a & b\sqrt{2} \\
b\sqrt{2} & a
\end{pmatrix} \right|a,b\mathbb{\in Q},a^{2} - 2b^{2} \neq 0 \right\}\) \\
\textbf{B3.} Tómendegi kópaǵzalı\(\mathbf{Z}_{2}\lbrack x\rbrack\)da keltirilmeytuǵınkópaǵzalıboladıma? \(x^{3} + x + 1\) \\
\textbf{C1.} Tómendegi kóplikti maydan shártlerine tekseriń. \(\mathbb{Z}\left( \sqrt{3} \right) = \left\{ a + b\sqrt{3}:a,b \in \mathbb{Z} \right\}\) \\
\textbf{C2.} \(\mathbb{Z}_{24}\)tiń barlıq idealların tabıń \\
\textbf{C3.} Tómendegi sáwlelendiriwdi gomomorfizm shártlerine tekseriń. \(f(x) = \sqrt[3]{x}\) \\

\end{tabular}
\vspace{1cm}


\begin{tabular}{m{17cm}}
\textbf{27-variant}
\newline

\textbf{T1.} Teńlemelerdiń radikallarda sheshiliwi (anıqlaması, qásiyetleri, mısallar) \\
\textbf{T2.} p-Adikalıq norma, p-Adikalıq norma (anıqlaması, qásiyetleri, mısallar) \\
\textbf{A1.} \(\mathbf{Q}\)da \(\sqrt{2} + \sqrt[3]{7}\)tiń minimal kópaǵzalısın tabıń. \\
\textbf{A2.} Ámellerdi orınlań: \(\mathbb{Z}_{10}\) da \(\left( 5x^{2} + 3x - 4 \right)\left( 4x^{2} - x + 9 \right)\) \\
\textbf{A3.} Tómendegi sannıń \(p\)-adikalıq normasın tabıń. \(|\frac{25}{36}|_{2} =\) \\
\textbf{B1.} Tómendegi kóplik kolco dúzedi ma? \(R = \left\{ \begin{pmatrix}
a & b \\
2b & a
\end{pmatrix}\ \ :\ \ a,b \in \mathbf{Q} \right\}\) \\
\textbf{B2.} Tómendegi kóplik\(M_{2 \times 2}\left( \mathbb{R} \right)\)kolcosınıń úles kolcosı ekenliginkórsetiń. \(A = \left\{ \left. \ \begin{pmatrix}
a & 0 \\
0 & 0
\end{pmatrix} \right|a \in Z \right\}\) \\
\textbf{B3.} Tómendegi maydanlardıń berilgen kópaǵzalı arqalı ajıralıw maydanın tabıń. \(\mathbb{Q}\) da \(x^{4} - 10x^{2} + 21\). \\
\textbf{C1.} Tómendegi kóplikti maydan shártlerine tekseriń. \(Z_{p}\) \\
\textbf{C2.} \(M_{2 \times 2}\left( \mathbf{Z} \right)\)kolcoda\(I = \left\{ \begin{bmatrix}
a & 0 \\
b & 0
\end{bmatrix}|a,b\mathbb{\in Z} \right\}\) ideal boladı ma? \\
\textbf{C3.} Tómendegi sáwlelendiriwdi gomomorfizm shártlerine tekseriń. \(f(x) = 5^{x}\) \\

\end{tabular}
\vspace{1cm}


\begin{tabular}{m{17cm}}
\textbf{28-variant}
\newline

\textbf{T1.} Maydanlar keńeytpesi (anıqlaması, qásiyetleri, mısallar) \\
\textbf{T2.} Kolcolar (anıqlaması, qásiyetleri, mısallar) \\
\textbf{A1.} \(\mathbf{Q}\)da \(\sqrt{2 + 2\sqrt{2}}\)tiń minimal kópaǵzalısın tabıń. \\
\textbf{A2.} Ámellerdi orınlań: \(\mathbb{Z}_{12}\) de \((3x - 2)^{3}\) \\
\textbf{A3.} Tómendegi sannıń \(p\)-adikalıq normasın tabıń. \(|15|_{3} =\) \\
\textbf{B1.} Tómendegi kóplik kolco dúzedi ma? \(\mathbb{Q}\left\lfloor i \right\rfloor = \left\{ a + bi\ \ :\ \ a,b\mathbb{\in Q} \right\}\) \\
\textbf{B2.} Tómendegi kóplik\(M_{2 \times 2}\left( \mathbb{R} \right)\) niń úles maydanı ekenliginkórsetiń. \(A = \left\{ \left. \ \begin{pmatrix}
a & b\sqrt{7} \\
 - b\sqrt{7} & a
\end{pmatrix} \right|a,b\mathbb{\in Q},a^{2} + 7b^{2} \neq 0 \right\}\) \\
\textbf{B3.} Tómendegi maydanlardıń berilgen kópaǵzalı arqalı ajıralıw maydanın tabıń.
\(\mathbb{Q}\) da \(x^{4} - 5x^{2} + 21\). \\
\textbf{C1.} Tómendegi kóplikti maydan shártlerine tekseriń. \(\mathbb{Q}\left\lfloor i \right\rfloor = \left\{ a + bi\ \ :\ \ a,b\mathbb{\in Q} \right\}\) \\
\textbf{C2.} \(M_{2 \times 2}\left( \mathbf{Z} \right)\)kolcoda\(I = \left\{ \begin{bmatrix}
a & 0 \\
0 & 0
\end{bmatrix}|a\mathbb{\in Z} \right\}\) ideal boladı ma? \\
\textbf{C3.} Tómendegi sáwlelendiriwdi gomomorfizm shártlerine tekseriń. \(f(x + iy) = x \cdot y\) \\

\end{tabular}
\vspace{1cm}


\begin{tabular}{m{17cm}}
\textbf{29-variant}
\newline

\textbf{T1.} Kolco gomomorfizmleri hám ideallar (anıqlaması, qásiyetleri, mısallar) \\
\textbf{T2.} Pútinlik oblastı hám maydan (anıqlaması, qásiyetleri, mısallar) \\
\textbf{A1.} \(\mathbf{Q}\)da \(\sqrt{2} + \sqrt[3]{7}\)tiń minimal kópaǵzalısın tabıń. \\
\textbf{A2.} Ámellerdi orınlań: \(\mathbb{Z}_{12}\) de \(\left( 5x^{2} + 3x - 4 \right)\left( 4x^{2} - x + 9 \right)\) \\
\textbf{A3.} Tómendegi sannıń \(p\)-adikalıq normasın tabıń. \(|\frac{1}{4}|_{2} =\) \\
\textbf{B1.} Tómendegi kóplik kolco dúzedi ma? \(\mathbf{Q} = \left\{ \frac{m}{n},\ \ m \in Z,\ n \in N \right\}\) \\
\textbf{B2.} Tómendegi kóplik\(M_{2 \times 2}\left( \mathbb{R} \right)\) niń úles maydanı ekenliginkórsetiń. \(A = \left\{ \left. \ \begin{pmatrix}
a & b \\
 - b & a
\end{pmatrix} \right|a,b\mathbb{\in Z},a^{2} + b^{2} \neq 0 \right\}\) \\
\textbf{B3.} \(x^{2} + 1\) kópaǵzalı \(\mathbb{Z}_{3}\)de keltirilmeytuǵın kópaǵzalı ekenligin kórsetiń. \\
\textbf{C1.} Tómendegi kóplikti maydan shártlerine tekseriń. \(\left\{ a + b\sqrt{7}|a,b \in R \right\}\) \\
\textbf{C2.} \(M_{2 \times 2}\left( \mathbf{Z} \right)\)kolcoda\(I = \left\{ \begin{bmatrix}
a & 0 \\
b & 0
\end{bmatrix}|a,b\mathbb{\in Z} \right\}\) ideal boladı ma? \\
\textbf{C3.} Tómendegi sáwlelendiriwdi gomomorfizm shártlerine tekseriń. \(f:\begin{pmatrix}
a & b \\
0 & c
\end{pmatrix} \rightarrow a\) \\

\end{tabular}
\vspace{1cm}


\begin{tabular}{m{17cm}}
\textbf{30-variant}
\newline

\textbf{T1.} Keltirilmeytuǵın kópaǵzalılar (anıqlaması, qásiyetleri, mısallar) \\
\textbf{T2.} Maydanlar avtomorfizmleri (anıqlaması, qásiyetleri, mısallar) \\
\textbf{A1.} \(\mathbf{Q}\)da \(\sqrt{2} + \sqrt{3}\)tiń minimal kópaǵzalısın tabıń. \\
\textbf{A2.} Ámellerdi orınlań: \(\mathbb{Z}_{12}\) de \(\left( 5x^{2} + 3x - 4 \right) + \left( 4x^{2} - x + 9 \right)\) \\
\textbf{A3.} Tómendegi sannıń \(p\)-adikalıq normasın tabıń. \(|35|_{7} =\) \\
\textbf{B1.} Tómendegi kóplik kolco dúzedi ma? \(\mathbb{Z}\left( \sqrt{3} \right) = \left\{ a + b\sqrt{3}:a,b \in \mathbb{Z} \right\}\) \\
\textbf{B2.} Tómendegi kóplik\(M_{2 \times 2}\left( \mathbb{R} \right)\) niń úles maydanı ekenliginkórsetiń. \(A = \left\{ \left. \ \begin{pmatrix}
a & 0 \\
0 & a
\end{pmatrix} \right|a\mathbb{\in R},a \neq 0 \right\}\) \\
\textbf{B3.} Tómendegi maydanlardıń berilgen kópaǵzalı arqalı ajıralıw maydanın tabıń. \(\mathbb{Q}\) da \(x^{4} - 2\) \\
\textbf{C1.} Tómendegi kóplikti maydan shártlerine tekseriń. \(\mathbb{Q(}i\sqrt{n}) = \{ x + iy\sqrt{n}\ |\ x,y \in \mathbb{Q}\}\) \\
\textbf{C2.} Tómendegi kolconıń úles kóplikleri ideal bolıwın kórsetiń:
\(R = \mathbb{Z}_{28}\), \(I = \{\overline{0},\overline{7},\overline{14},\overline{21}\}\). \\
\textbf{C3.} Tómendegi sáwlelendiriwdi gomomorfizm shártlerine tekseriń. \(f(x + iy) = x \cdot y\) \\

\end{tabular}
\vspace{1cm}


\begin{tabular}{m{17cm}}
\textbf{31-variant}
\newline

\textbf{T1.} p-Adikalıq norma, p-Adikalıq norma (anıqlaması, qásiyetleri, mısallar) \\
\textbf{T2.} Keltirilmeytuǵın kópaǵzalılar (anıqlaması, qásiyetleri, mısallar) \\
\textbf{A1.} \(\mathbf{Q}\)da \(\sqrt{2 + 2\sqrt{2}}\)tiń minimal kópaǵzalısın tabıń. \\
\textbf{A2.} Ámellerdi orınlań: \(\mathbb{Z}_{5}\) te \(\left( x^{2} + 3x - 4 \right)\left( x^{2} - 3 \right)\) \\
\textbf{A3.} Tómendegi sannıń \(p\)-adikalıq normasın tabıń. \(|\frac{8}{18}|_{5} =\) \\
\textbf{B1.} Tómendegi kóplik kolco dúzedi ma? \(G = \left\{ a^{n},a \neq 0, \pm 1,n \in \mathbb{Z} \right\}\) \\
\textbf{B2.} Tómendegi kóplik\(M_{2 \times 2}\left( \mathbb{R} \right)\)kolcosınıń úles kolcosı ekenliginkórsetiń. \(A = \left\{ \left. \ \begin{pmatrix}
a & 0 \\
0 & 0
\end{pmatrix} \right|a \in Z \right\}\) \\
\textbf{B3.} \(x^{2} + 1\) kópaǵzalı \(\mathbb{Z}_{3}\)de keltirilmeytuǵın kópaǵzalı ekenligin kórsetiń. \\
\textbf{C1.} Tómendegi kóplikti maydan shártlerine tekseriń. \(7\mathbb{Z}\) \\
\textbf{C2.} Tómendegi kolconıń barlıq ideallarıń tabıń. Bul ideallardan qaysı-biri maksimal boladı? \(\mathbb{M}_{2}\left( \mathbb{R} \right)\), elementleri \(\mathbb{R}\) bolǵan \(2 \times 2\) matrica \\
\textbf{C3.} Tómendegi sáwlelendiriwdi gomomorfizm shártlerine tekseriń.
\[f:x \rightarrow x^{p}\] \\

\end{tabular}
\vspace{1cm}


\begin{tabular}{m{17cm}}
\textbf{32-variant}
\newline

\textbf{T1.} Teńlemelerdiń radikallarda sheshiliwi (anıqlaması, qásiyetleri, mısallar) \\
\textbf{T2.} Shekli maydannıń strukturası (anıqlaması, qásiyetleri, mısallar) \\
\textbf{A1.} \(\mathbf{Q}\)da \(\sqrt{6 + 3\sqrt{2}}\)tiń minimal kópaǵzalısın tabıń. \\
\textbf{A2.} Ámellerdi orınlań: \(\mathbb{Z}_{5}\) te \(\left( 3x^{2} + 2x - 4 \right) + \left( 4x^{2} + 2 \right)\) \\
\textbf{A3.} Tómendegi sannıń \(p\)-adikalıq normasın tabıń. \(|\frac{7}{15}|_{3} =\) \\
\textbf{B1.} Tómendegi kóplik kolco dúzedi ma? \(\mathbb{Q(}i\sqrt{n}) = \{ x + iy\sqrt{n}\ |\ x,y \in \mathbb{Q}\}\) \\
\textbf{B2.} \(Z_{16}\) kolconıń barlıq úles kolcoların anıqlań. \\
\textbf{B3.} Tómendegi maydanlardıń berilgen kópaǵzalı arqalı ajıralıw maydanın tabıń. \(\mathbb{Q}\) da \(x^{4} + 1\). \\
\textbf{C1.} Tómendegi kóplikti maydan shártlerine tekseriń. \(\mathbb{Z}\left\lbrack \sqrt{n} \right\rbrack = \left\{ x + y\sqrt{n}\ \ \left| \right.\ x,y \in \mathbb{Z} \right\}\) \\
\textbf{C2.} \(M_{2 \times 2}\left( \mathbf{Z} \right)\)kolcoda\(I = \left\{ \begin{bmatrix}
a & 0 \\
0 & 0
\end{bmatrix}|a\mathbb{\in Z} \right\}\) ideal boladı ma? \\
\textbf{C3.} Tómendegi sáwlelendiriwdi gomomorfizm shártlerine tekseriń. \(f(x) = \sqrt[3]{x}\) \\

\end{tabular}
\vspace{1cm}


\begin{tabular}{m{17cm}}
\textbf{33-variant}
\newline

\textbf{T1.} Bólshekler maydanı (anıqlaması, qásiyetleri, mısallar) \\
\textbf{T2.} Kópagzalılar kolcosı (anıqlaması, qásiyetleri, mısallar) \\
\textbf{A1.} \(\mathbf{Q}\)da \(\sqrt{\frac{1}{4} + \sqrt{5}}\) tiń minimal kópaǵzalısın tabıń. \\
\textbf{A2.} Ámellerdi orınlań: \(\mathbb{Z}_{9}\) da \(\left( 7x^{3} + 3x^{2} - x \right) + \left( 6x^{2} - 8x + 4 \right)\) \\
\textbf{A3.} Tómendegi sannıń \(p\)-adikalıq normasın tabıń. \(|\frac{25}{36}|_{2} =\) \\
\textbf{B1.} Tómendegi kóplik kolco dúzedi ma? \(\mathbb{Q}\left( \sqrt{2} \right) = \left\{ a + b\sqrt{2}:a,b \in \mathbb{Q} \right\}\) \\
\textbf{B2.} \(Z_{20}\) kolconıń barlıq úles kolcoların anıqlań. \\
\textbf{B3.} Tómendegi maydanlardıń berilgen kópaǵzalı arqalı ajıralıw maydanın tabıń. \(\mathbb{Q}\) da \(x^{4} - 2\) \\
\textbf{C1.} Tómendegi kóplikti maydan shártlerine tekseriń. \(G = \left\{ a^{n},a \neq 0, \pm 1,n \in \mathbb{Z} \right\}\) \\
\textbf{C2.} Tómendegi kolconıń barlıq ideallarıń tabıń. Bul ideallardan qaysı-biri maksimal boladı? \(\mathbb{Z}_{27}\) \\
\textbf{C3.} Tómendegi sáwlelendiriwdi gomomorfizm shártlerine tekseriń. \(f\left( a + \sqrt{2}b \right) = a + bi\) \\

\end{tabular}
\vspace{1cm}


\begin{tabular}{m{17cm}}
\textbf{34-variant}
\newline

\textbf{T1.} Maksimal hám ápiwayı ideallar (anıqlaması, qásiyetleri, mısallar) \\
\textbf{T2.} Pútinlik oblastı hám maydan (anıqlaması, qásiyetleri, mısallar) \\
\textbf{A1.} \(\mathbf{Q}\)da \(\sqrt{2} + \sqrt{5}\)tiń minimal kópaǵzalısın tabıń. \\
\textbf{A2.} Ámellerdi orınlań: \(\mathbb{Z}_{12}\) de \((3x - 2)^{3}\) \\
\textbf{A3.} Tómendegi sannıń \(p\)-adikalıq normasın tabıń. \(|6|_{3} =\) \\
\textbf{B1.} Tómendegi kóplik kolco dúzedi ma? \(7\mathbb{Z}\) \\
\textbf{B2.} Tómendegi kóplik\(M_{2 \times 2}\left( \mathbb{R} \right)\)kolcosınıń úles kolcosı ekenliginkórsetiń. \(A = \left\{ \left. \ \begin{pmatrix}
a & b \\
7b & a
\end{pmatrix} \right|a,b \in Z \right\}\) \\
\textbf{B3.} \(x^{2} - 7\) kópaǵzalı\(\mathbb{Q}(\sqrt{3})\)de keltirilmeytuǵın kópaǵzalı ekenligin kórsetiń. \\
\textbf{C1.} Tómendegi kóplikti maydan shártlerine tekseriń. \(\mathbb{Q}\left\lfloor i \right\rfloor = \left\{ a + bi\ \ :\ \ a,b\mathbb{\in Q} \right\}\) \\
\textbf{C2.} Tómendegi kolconıń barlıq ideallarıń tabıń. Bul ideallardan qaysı-biri maksimal boladı? \(\mathbb{M}_{2}\left( \mathbb{Z} \right)\), elementleri \(\mathbb{Z}\) bolǵan \(2 \times 2\) matrica \\
\textbf{C3.} Tómendegi sáwlelendiriwdi gomomorfizm shártlerine tekseriń. \(f(x) = x^{2} + x\) \\

\end{tabular}
\vspace{1cm}


\begin{tabular}{m{17cm}}
\textbf{35-variant}
\newline

\textbf{T1.} Kolco gomomorfizmleri hám ideallar (anıqlaması, qásiyetleri, mısallar) \\
\textbf{T2.} Ajıralatuǵın maydanlar (anıqlaması, qásiyetleri, mısallar) \\
\textbf{A1.} \(\mathbf{Q}\)da \(\sqrt{2} + \sqrt{3}i\)tiń minimal kópaǵzalısın tabıń. \\
\textbf{A2.} Ámellerdi orınlań: \(\mathbb{Z}_{12}\) de \(\left( 5x^{2} + 3x - 4 \right) + \left( 4x^{2} - x + 9 \right)\) \\
\textbf{A3.} Tómendegi sannıń \(p\)-adikalıq normasın tabıń. \(|\frac{9}{12}|_{7} =\) \\
\textbf{B1.} Tómendegi kóplik kolco dúzedi ma? \(\mathbb{Q(}\sqrt[3]{2}) = \left\{ a + b\sqrt[3]{2}:a,b \in \mathbb{Q} \right\}\) \\
\textbf{B2.} Tómendegi kóplik\(M_{2 \times 2}\left( \mathbb{R} \right)\) niń úles maydanı ekenliginkórsetiń. \(A = \left\{ \left. \ \begin{pmatrix}
a & b\sqrt{2} \\
b\sqrt{2} & a
\end{pmatrix} \right|a,b\mathbb{\in Q},a^{2} - 2b^{2} \neq 0 \right\}\) \\
\textbf{B3.} Tómendegi kópaǵzalı \(\mathbb{Q\lbrack}x\rbrack\) da keltirilmeytuǵın kópaǵzalı boladıma? \(3x^{5} - 4x^{3} - 6x^{2} + 6\) \\
\textbf{C1.} Tómendegi kóplikti maydan shártlerine tekseriń. \(R = \left\{ \begin{pmatrix}
a & b \\
2b & a
\end{pmatrix}\ \ :\ \ a,b \in \mathbf{Q} \right\}\) \\
\textbf{C2.} Tómendegi kolconıń úles kóplikleri ideal bolıwın kórsetiń:
\(R\mathbb{= Z\lbrack}\sqrt{7}\rbrack\), \(I = \{ a + b\sqrt{7}\ |\ \ a,b \in \mathbb{Z,}a - b\ \ jupsan\}\). \\
\textbf{C3.} Tómendegi sáwlelendiriwdi gomomorfizm shártlerine tekseriń.
\[f\left( a - \sqrt{2}b \right) = a + \sqrt{2}b\] \\

\end{tabular}
\vspace{1cm}


\begin{tabular}{m{17cm}}
\textbf{36-variant}
\newline

\textbf{T1.} Maydanlar avtomorfizmleri (anıqlaması, qásiyetleri, mısallar) \\
\textbf{T2.} Fundamental teoremalar (anıqlaması, qásiyetleri, mısallar) \\
\textbf{A1.} \(\mathbf{Q}\)da \(\sqrt{3} + \sqrt{2}i\) tiń minimal kópaǵzalısın tabıń. \\
\textbf{A2.} Ámellerdi orınlań: \(\mathbb{Z}_{12}\) de \(\left( 5x^{2} + 3x - 4 \right)\left( 4x^{2} - x + 9 \right)\) \\
\textbf{A3.} Tómendegi sannıń \(p\)-adikalıq normasın tabıń. \(|256|_{2} =\) \\
\textbf{B1.} Tómendegi kóplik kolco dúzedi ma? \(R = M_{2 \times 2}\left( \mathbf{Q} \right)\) \\
\textbf{B2.} Tómendegi kóplik\(M_{2 \times 2}\left( \mathbb{R} \right)\)kolcosınıń úles kolcosı ekenliginkórsetiń.
\begin{quote}
\[T = \left\{ \begin{pmatrix}
a + b & b \\
 - b & a
\end{pmatrix}\left| \ \ a,b\mathbb{\in Z} \right.\  \right\}\]
\end{quote} \\
\textbf{B3.} \(x^{2} - 3\)kópaǵzalı \(\mathbb{Q}(\sqrt{2})\)de keltirilmeytuǵınkópaǵzalı ekenliginkórsetiń. \\
\textbf{C1.} \(Z_{18}\) kolconıń barlıq úles kolcoların anıqlań. \\
\textbf{C2.} Tómendegi kolconıń úles kóplikleri ideal bolıwın kórsetiń:
\(R = \left\{ \begin{pmatrix}
a & b \\
0 & c
\end{pmatrix}\ |\ a,b,c \in \mathbb{Z} \right\}\), \(I = \left\{ \begin{pmatrix}
0 & b \\
0 & c
\end{pmatrix}\ |\ a \in \mathbb{Z} \right\}\). \\
\textbf{C3.} Tómendegi sáwlelendiriwdi gomomorfizm shártlerine tekseriń. \(f(a + ib) = \begin{pmatrix}
a & b \\
 - b & a
\end{pmatrix}\) \\

\end{tabular}
\vspace{1cm}


\begin{tabular}{m{17cm}}
\textbf{37-variant}
\newline

\textbf{T1.} Maydanlar keńeytpesi (anıqlaması, qásiyetleri, mısallar) \\
\textbf{T2.} p-Adikalıq sanlardıń keńisligi (anıqlaması, qásiyetleri, mısallar) \\
\textbf{A1.} \(\mathbf{Q}\)da \(\sqrt{1/3 + \sqrt{7}}\)tiń minimal kópaǵzalısın tabıń. \\
\textbf{A2.} Ámellerdi orınlań: \(\mathbb{Z}_{5}\) te \(\left( x^{2} + 3x - 1 \right)^{2}\) \\
\textbf{A3.} Tómendegi sannıń \(p\)-adikalıq normasın tabıń. \(|124|_{2} =\) \\
\textbf{B1.} Tómendegi kóplik kolco dúzedi ma? \(R = \left\{ a + b\sqrt{2}\ :\ \ \ \ a,b \in \mathbf{Z} \right\}\) \\
\textbf{B2.} Tómendegi kóplik\(M_{2 \times 2}\left( \mathbb{R} \right)\)kolcosınıń úles kolcosı ekenliginkórsetiń.
\[A = \left\{ \left. \ \begin{pmatrix}
a & b \\
 - b & a
\end{pmatrix} \right|a,b\mathbb{\in R} \right\}\] \\
\textbf{B3.} Tómendegi kópaǵzalı \(\mathbb{Q\lbrack}x\rbrack\) da keltirilmeytuǵın kópaǵzalı boladıma? \(x^{4} - 5x^{3} + 3x - 2\) \\
\textbf{C1.} Tómendegi kóplikti maydan shártlerine tekseriń. \(R = M_{2 \times 2}\left( \mathbf{Q} \right)\) \\
\textbf{C2.} Tómendegi kolconıń úles kóplikleri ideal bolıwın kórsetiń:
\(R = \left\{ \begin{pmatrix}
a & b \\
0 & c
\end{pmatrix}\ |\ a,b,c \in \mathbb{Z} \right\}\), \(I = \left\{ \begin{pmatrix}
0 & a \\
0 & 0
\end{pmatrix}\ |\ a \in \mathbb{Z} \right\}\) \\
\textbf{C3.} Tómendegi sáwlelendiriwdi gomomorfizm shártlerine tekseriń. \(f(x) = 5^{x}\) \\

\end{tabular}
\vspace{1cm}


\begin{tabular}{m{17cm}}
\textbf{38-variant}
\newline

\textbf{T1.} Kolcolar (anıqlaması, qásiyetleri, mısallar) \\
\textbf{T2.} p-Adikalıq norma, p-Adikalıq norma (anıqlaması, qásiyetleri, mısallar) \\
\textbf{A1.} \(\mathbf{Q}\)da \(\sqrt{\sqrt[3]{2} - i}\)tiń minimal kópaǵzalısın tabıń. \\
\textbf{A2.} Ámellerdi orınlań: \(\mathbb{Z}_{5}\) te \(\left( 3x^{2} + 3x - 4 \right)\left( 4x^{2} + 2 \right)\) \\
\textbf{A3.} Tómendegi sannıń \(p\)-adikalıq normasın tabıń. \(|48|_{3} =\) \\
\textbf{B1.} Tómendegi kóplik kolco dúzedi ma? \(\left\{ a + b\sqrt{7}|a,b \in R \right\}\) \\
\textbf{B2.} Tómendegi kóplik\(M_{2 \times 2}\left( \mathbb{R} \right)\) niń úles maydanı ekenliginkórsetiń. \(A = \left\{ \left. \ \begin{pmatrix}
a & b\sqrt{7} \\
 - b\sqrt{7} & a
\end{pmatrix} \right|a,b\mathbb{\in Q},a^{2} + 7b^{2} \neq 0 \right\}\) \\
\textbf{B3.} Tómendegi maydanlardıń berilgen kópaǵzalı arqalı ajıralıw maydanın tabıń. \(\mathbb{Q}\) da \(x^{4} - 10x^{2} + 21\). \\
\textbf{C1.} Tómendegi kóplikti maydan shártlerine tekseriń. \(Z_{p}\) \\
\textbf{C2.} \(\mathbb{Z}_{24}\)tiń barlıq idealların tabıń \\
\textbf{C3.} Tómendegi sáwlelendiriwdi gomomorfizm shártlerine tekseriń. \(f\left( \begin{pmatrix}
a & 0 \\
0 & a
\end{pmatrix} \right) = a\) \\

\end{tabular}
\vspace{1cm}


\begin{tabular}{m{17cm}}
\textbf{39-variant}
\newline

\textbf{T1.} Teńlemelerdiń radikallarda sheshiliwi (anıqlaması, qásiyetleri, mısallar) \\
\textbf{T2.} Shekli maydannıń strukturası (anıqlaması, qásiyetleri, mısallar) \\
\textbf{A1.} \(\mathbf{Q}\)da \(\sqrt{2 + \sqrt{5}}\)tiń minimal kópaǵzalısın tabıń. \\
\textbf{A2.} Ámellerdi orınlań: \(\mathbb{Z}_{10}\) da \(\left( 7x^{3} + 3x^{2} - x \right) + \left( 6x^{2} - 8x + 4 \right)\) \\
\textbf{A3.} Tómendegi sannıń \(p\)-adikalıq normasın tabıń. \(|729|_{3} =\) \\
\textbf{B1.} Tómendegi kóplik kolco dúzedi ma? \(\mathbb{Z}\left\lbrack \sqrt{n} \right\rbrack = \left\{ x + y\sqrt{n}\ \ \left| \right.\ x,y \in \mathbb{Z} \right\}\) \\
\textbf{B2.} \(Z_{12}\) kolconıń barlıq úles kolcoların anıqlań. \\
\textbf{B3.} Tómendegi kópaǵzalı\(\mathbf{Z}_{2}\lbrack x\rbrack\)da keltirilmeytuǵınkópaǵzalıboladıma? \(x^{3} + x + 1\) \\
\textbf{C1.} Tómendegi kóplikti maydan shártlerine tekseriń. \(G = \left\{ a^{n},a \neq 0, \pm 1,n \in \mathbb{Z} \right\}\) \\
\textbf{C2.} Tómendegi kolconıń úles kóplikleri ideal bolıwın kórsetiń:
\(R = \mathbb{Z}_{28}\), \(I = \left\{ \overline{0},\overline{7},\overline{14},\overline{21} \right\}\) \\
\textbf{C3.} Tómendegi sáwlelendiriwdi gomomorfizm shártlerine tekseriń.
\[f:\begin{pmatrix}
a & b \\
 - b & a
\end{pmatrix} \rightarrow a + bi\] \\

\end{tabular}
\vspace{1cm}


\begin{tabular}{m{17cm}}
\textbf{40-variant}
\newline

\textbf{T1.} Pútinlik oblastı hám maydan (anıqlaması, qásiyetleri, mısallar) \\
\textbf{T2.} Maydanlar keńeytpesi (anıqlaması, qásiyetleri, mısallar) \\
\textbf{A1.} \(\mathbf{Q}\)da \(\sqrt{3} + \sqrt[3]{5}\)tiń minimal kópaǵzalısın tabıń. \\
\textbf{A2.} Ámellerdi orınlań: \(\mathbb{Z}_{10}\) da \(\left( 5x^{2} + 3x - 4 \right)\left( 4x^{2} - x + 9 \right)\) \\
\textbf{A3.} Tómendegi sannıń \(p\)-adikalıq normasın tabıń. \(|625|_{5} =\) \\
\textbf{B1.} Tómendegi kóplik kolco dúzedi ma? \(R = \left\{ \begin{pmatrix}
a & b \\
2b & a
\end{pmatrix}\ \ :\ \ a,b \in \mathbf{Q} \right\}\) \\
\textbf{B2.} Tómendegi kóplik\(M_{2 \times 2}\left( \mathbb{R} \right)\) niń úles maydanı ekenliginkórsetiń. \(A = \left\{ \left. \ \begin{pmatrix}
a & b \\
 - b & a
\end{pmatrix} \right|a,b\mathbb{\in Z},a^{2} + b^{2} \neq 0 \right\}\) \\
\textbf{B3.} \(x^{4} - 5x^{2} + 6\) kópaǵzalı \(\mathbb{Q}\) de keltirilmeytuǵın kópaǵzalı ekenligin kórsetiń. \\
\textbf{C1.} Tómendegi kóplikti maydan shártlerine tekseriń. \(\mathbb{Z}\left( \sqrt{3} \right) = \left\{ a + b\sqrt{3}:a,b \in \mathbb{Z} \right\}\) \\
\textbf{C2.} \(Z_{12}\) niń barlıq idealların tabıń. \\
\textbf{C3.} Tómendegi sáwlelendiriwdi gomomorfizm shártlerine tekseriń. \(f(x) = e^{x}\) \\

\end{tabular}
\vspace{1cm}


\begin{tabular}{m{17cm}}
\textbf{41-variant}
\newline

\textbf{T1.} Fundamental teoremalar (anıqlaması, qásiyetleri, mısallar) \\
\textbf{T2.} Maksimal hám ápiwayı ideallar (anıqlaması, qásiyetleri, mısallar) \\
\textbf{A1.} \(\mathbf{Q}\)da \(\sqrt{2 + \sqrt{2}}\)tiń minimal kópaǵzalısın tabıń. \\
\textbf{A2.} Tómendegi kópaǵzalınıń barlıq nollerin tabıń: \(\mathbb{Z}_{5}\) de \(3x^{3} - 4x^{2} - x + 4\) \\
\textbf{A3.} Tómendegi sannıń \(p\)-adikalıq normasın tabıń. \(|15|_{3} =\) \\
\textbf{B1.} Tómendegi kóplik kolco dúzedi ma? \(\left\{ a + b\sqrt{7}|a,b \in R \right\}\) \\
\textbf{B2.} Tómendegi kóplik\(M_{2 \times 2}\left( \mathbb{R} \right)\) niń úles maydanı ekenliginkórsetiń.
\[A = \left\{ \left. \ \begin{pmatrix}
a & b \\
0 & a
\end{pmatrix} \right|a,b\mathbb{\in Z},a \neq 0 \right\}\] \\
\textbf{B3.} Tómendegi maydanlardıń berilgen kópaǵzalı arqalı ajıralıw maydanın tabıń. \(\mathbb{Q}\) da \(x^{3} - 3\). \\
\textbf{C1.} Tómendegi kóplikti maydan shártlerine tekseriń. \(\mathbf{Q} = \left\{ \frac{m}{n},\ \ m \in Z,\ n \in N \right\}\) \\
\textbf{C2.} Tómendegi kolconıń úles kóplikleri ideal bolıwın kórsetiń:
\(R = \mathbb{Z}_{24}\), \(I = \{\overline{0},\overline{8},\overline{16}\}\). \\
\textbf{C3.} Tómendegi sáwlelendiriwdi gomomorfizm shártlerine tekseriń. \(f(x) = \sqrt{x}\) \\

\end{tabular}
\vspace{1cm}


\begin{tabular}{m{17cm}}
\textbf{42-variant}
\newline

\textbf{T1.} Keltirilmeytuǵın kópaǵzalılar (anıqlaması, qásiyetleri, mısallar) \\
\textbf{T2.} Kolco gomomorfizmleri hám ideallar (anıqlaması, qásiyetleri, mısallar) \\
\textbf{A1.} \(\mathbf{Q}\)da \(\sqrt{3 - \sqrt{3}}\)tiń minimal kópaǵzalısın tabıń. \\
\textbf{A2.} Ámellerdi orınlań: \(\mathbb{Z}_{5}\) te \(\left( 3x^{2} + 3x - 4 \right)\left( x^{2} + 2 \right)\) \\
\textbf{A3.} Tómendegi sannıń \(p\)-adikalıq normasın tabıń. \(|\frac{3}{4}|_{2} =\) \\
\textbf{B1.} Tómendegi kóplik kolco dúzedi ma? \(7\mathbb{Z}\) \\
\textbf{B2.} Tómendegi kóplik\(M_{2 \times 2}\left( \mathbb{R} \right)\)kolcosınıń úles kolcosı ekenliginkórsetiń. \(A = \left\{ \left. \ \begin{pmatrix}
a & b \\
0 & a
\end{pmatrix} \right|a,b\mathbb{\in Q} \right\}\) \\
\textbf{B3.} \(x^{2} + x + 1\) kópaǵzalı \(\mathbb{Z}_{5}\) de keltirilmeytuǵın kópaǵzalı ekenligin kórsetiń. \\
\textbf{C1.} Tómendegi kóplikti maydan shártlerine tekseriń. \(R = M_{2 \times 2}\left( \mathbf{Z} \right)\) \\
\textbf{C2.} Tómendegi kolconıń barlıq ideallarıń tabıń. Bul ideallardan qaysı-biri maksimal boladı? \(\mathbb{Z}_{25}\) \\
\textbf{C3.} Tómendegi sáwlelendiriwdi gomomorfizm shártlerine tekseriń. \(f(a) = a^{n}\) \\

\end{tabular}
\vspace{1cm}


\begin{tabular}{m{17cm}}
\textbf{43-variant}
\newline

\textbf{T1.} Kópagzalılar kolcosı (anıqlaması, qásiyetleri, mısallar) \\
\textbf{T2.} Kolcolar (anıqlaması, qásiyetleri, mısallar) \\
\textbf{A1.} \(\mathbf{Q}\)da \(\sqrt{2} + \sqrt{3}\)tiń minimal kópaǵzalısın tabıń. \\
\textbf{A2.} Tómendegi kópaǵzalınıń barlıq nollerin tabıń: \(\mathbb{Z}_{2}\) de \(x^{3} + x + 1\) \\
\textbf{A3.} Tómendegi sannıń \(p\)-adikalıq normasın tabıń. \(|124|_{2} =\) \\
\textbf{B1.} Tómendegi kóplik kolco dúzedi ma? \(\mathbb{Z}\left( \sqrt{3} \right) = \left\{ a + b\sqrt{3}:a,b \in \mathbb{Z} \right\}\) \\
\textbf{B2.} Tómendegi kóplik\(M_{2 \times 2}\left( \mathbb{R} \right)\)kolcosınıń úles kolcosı ekenliginkórsetiń. \(A = \left\{ \left. \ \begin{pmatrix}
a & b \\
0 & a
\end{pmatrix} \right|a,b\mathbb{\in R} \right\}\) \\
\textbf{B3.} Tómendegi kópaǵzalı \(\mathbb{Z}_{3}\) da keltirilmeytuǵın kópaǵzalı boladıma? \(x^{3} + 2x + 2\) \\
\textbf{C1.} Tómendegi kóplikti maydan shártlerine tekseriń. \(\mathbb{Q}\left\lfloor i \right\rfloor = \left\{ a + bi\ \ :\ \ a,b\mathbb{\in Q} \right\}\) \\
\textbf{C2.} \(Z_{12}\) niń barlıq idealların tabıń. \\
\textbf{C3.} Tómendegi sáwlelendiriwdi gomomorfizm shártlerine tekseriń.
\[f:x \rightarrow x^{p}\] \\

\end{tabular}
\vspace{1cm}


\begin{tabular}{m{17cm}}
\textbf{44-variant}
\newline

\textbf{T1.} Maydanlar avtomorfizmleri (anıqlaması, qásiyetleri, mısallar) \\
\textbf{T2.} p-Adikalıq sanlardıń keńisligi (anıqlaması, qásiyetleri, mısallar) \\
\textbf{A1.} \(\mathbf{Q}\)da \(\sqrt{3} + \sqrt[3]{5}\)tiń minimal kópaǵzalısın tabıń. \\
\textbf{A2.} Ámellerdi orınlań: \(\mathbb{Z}_{12}\) de \(\left( 5x^{2} + 3x - 2 \right)^{2}\) \\
\textbf{A3.} Tómendegi sannıń \(p\)-adikalıq normasın tabıń. \(|\frac{3}{4}|_{2} =\) \\
\textbf{B1.} Tómendegi kóplik kolco dúzedi ma? \(\mathbb{Z}\left\lbrack \sqrt{n} \right\rbrack = \left\{ x + y\sqrt{n}\ \ \left| \right.\ x,y \in \mathbb{Z} \right\}\) \\
\textbf{B2.} Tómendegi kóplik\(M_{2 \times 2}\left( \mathbb{R} \right)\) niń úles maydanı ekenligin kórsetiń. \(A = \left\{ \left. \ \begin{pmatrix}
a & 0 \\
2b & a
\end{pmatrix} \right|a,b\mathbb{\in R},a \neq 0 \right\}\) \\
\textbf{B3.} \(x^{2} + x + 1\) kópaǵzalı \(\mathbb{Z}_{3}\) de keltirilmeytuǵın kópaǵzalı ekenligin kórsetiń. \\
\textbf{C1.} \(Z_{18}\) kolconıń barlıq úles kolcoların anıqlań. \\
\textbf{C2.} Tómendegi kolconıń barlıq ideallarıń tabıń. Bul ideallardan qaysı-biri maksimal boladı? \(\mathbb{M}_{2}\left( \mathbb{Z} \right)\), elementleri \(\mathbb{Z}\) bolǵan \(2 \times 2\) matrica \\
\textbf{C3.} Tómendegi sáwlelendiriwdi gomomorfizm shártlerine tekseriń. \(f\left( a + \sqrt{2}b \right) = a + bi\) \\

\end{tabular}
\vspace{1cm}


\begin{tabular}{m{17cm}}
\textbf{45-variant}
\newline

\textbf{T1.} Ajıralatuǵın maydanlar (anıqlaması, qásiyetleri, mısallar) \\
\textbf{T2.} Bólshekler maydanı (anıqlaması, qásiyetleri, mısallar) \\
\textbf{A1.} \(\mathbf{Q}\)da \(\sqrt{2} + \sqrt{3}i\)tiń minimal kópaǵzalısın tabıń. \\
\textbf{A2.} Tómendegi kópaǵzalınıń barlıq nollerin tabıń: \(\mathbb{Z}_{12}\) de \(5x^{3} + 4x^{2} - x + 9\) \\
\textbf{A3.} Tómendegi sannıń \(p\)-adikalıq normasın tabıń. \(|\frac{9}{12}|_{7} =\) \\
\textbf{B1.} Tómendegi kóplik kolco dúzedi ma? \(\mathbb{Q(}i\sqrt{n}) = \{ x + iy\sqrt{n}\ |\ x,y \in \mathbb{Q}\}\) \\
\textbf{B2.} Tómendegi kóplik\(M_{2 \times 2}\left( \mathbb{R} \right)\) niń úles maydanı ekenliginkórsetiń. \(A = \left\{ \left. \ \begin{pmatrix}
a & 0 \\
0 & a
\end{pmatrix} \right|a\mathbb{\in R},a \neq 0 \right\}\) \\
\textbf{B3.} Tómendegi maydanlardıń berilgen kópaǵzalı arqalı ajıralıw maydanın tabıń.
\(\mathbb{Q}\) da \(x^{4} - 5x^{2} + 21\). \\
\textbf{C1.} Tómendegi kóplikti maydan shártlerine tekseriń. \(R = M_{2 \times 2}\left( \mathbf{Z} \right)\) \\
\textbf{C2.} Tómendegi kolconıń barlıq ideallarıń tabıń. Bul ideallardan qaysı-biri maksimal boladı? \(\mathbb{Z}_{25}\) \\
\textbf{C3.} Tómendegi sáwlelendiriwdi gomomorfizm shártlerine tekseriń. \(f:\begin{pmatrix}
a & b \\
0 & c
\end{pmatrix} \rightarrow a\) \\

\end{tabular}
\vspace{1cm}


\begin{tabular}{m{17cm}}
\textbf{46-variant}
\newline

\textbf{T1.} Kolcolar (anıqlaması, qásiyetleri, mısallar) \\
\textbf{T2.} p-Adikalıq norma, p-Adikalıq norma (anıqlaması, qásiyetleri, mısallar) \\
\textbf{A1.} \(\mathbf{Q}\)da \(\sqrt{\frac{1}{4} + \sqrt{5}}\) tiń minimal kópaǵzalısın tabıń. \\
\textbf{A2.} Ámellerdi orınlań: \(\mathbb{Z}_{5}\) te \(\left( x^{2} + 3x - 4 \right)\left( x^{2} - 3 \right)\) \\
\textbf{A3.} Tómendegi sannıń \(p\)-adikalıq normasın tabıń. \(|\frac{25}{36}|_{2} =\) \\
\textbf{B1.} Tómendegi kóplik kolco dúzedi ma? \(\mathbb{Q}\left( \sqrt{2} \right) = \left\{ a + b\sqrt{2}:a,b \in \mathbb{Q} \right\}\) \\
\textbf{B2.} Tómendegi kóplik\(M_{2 \times 2}\left( \mathbb{R} \right)\)kolcosınıń úles kolcosı ekenliginkórsetiń. \(A = \left\{ \left. \ \begin{pmatrix}
a & b \\
7b & a
\end{pmatrix} \right|a,b \in Z \right\}\) \\
\textbf{B3.} Tómendegi maydanlardıń berilgen kópaǵzalı arqalı ajıralıw maydanın tabıń. \(\mathbb{Q}\) da \(x^{4} - 2\) \\
\textbf{C1.} Tómendegi kóplikti maydan shártlerine tekseriń. \(G = \left\{ a^{n},a \neq 0, \pm 1,n \in \mathbb{Z} \right\}\) \\
\textbf{C2.} \(M_{2 \times 2}\left( \mathbf{Z} \right)\)kolcoda\(I = \left\{ \begin{bmatrix}
a & 0 \\
0 & 0
\end{bmatrix}|a\mathbb{\in Z} \right\}\) ideal boladı ma? \\
\textbf{C3.} Tómendegi sáwlelendiriwdi gomomorfizm shártlerine tekseriń.
\[f\left( a - \sqrt{2}b \right) = a + \sqrt{2}b\] \\

\end{tabular}
\vspace{1cm}


\begin{tabular}{m{17cm}}
\textbf{47-variant}
\newline

\textbf{T1.} Fundamental teoremalar (anıqlaması, qásiyetleri, mısallar) \\
\textbf{T2.} p-Adikalıq sanlardıń keńisligi (anıqlaması, qásiyetleri, mısallar) \\
\textbf{A1.} \(\mathbf{Q}\)da \(\sqrt{3 - \sqrt{3}}\)tiń minimal kópaǵzalısın tabıń. \\
\textbf{A2.} Tómendegi kópaǵzalınıń barlıq nollerin tabıń: \(\mathbb{Z}_{12}\) de \(5x^{3} + 4x^{2} - x + 9\) \\
\textbf{A3.} Tómendegi sannıń \(p\)-adikalıq normasın tabıń. \(|729|_{3} =\) \\
\textbf{B1.} Tómendegi kóplik kolco dúzedi ma? \(\mathbf{Q} = \left\{ \frac{m}{n},\ \ m \in Z,\ n \in N \right\}\) \\
\textbf{B2.} Tómendegi kóplik\(M_{2 \times 2}\left( \mathbb{R} \right)\)kolcosınıń úles kolcosı ekenliginkórsetiń. \(A = \left\{ \left. \ \begin{pmatrix}
a & b \\
0 & a
\end{pmatrix} \right|a,b\mathbb{\in R} \right\}\) \\
\textbf{B3.} Tómendegi kópaǵzalı \(\mathbb{Z}_{3}\) da keltirilmeytuǵın kópaǵzalı boladıma? \(x^{3} + 2x + 2\) \\
\textbf{C1.} Tómendegi kóplikti maydan shártlerine tekseriń. \(7\mathbb{Z}\) \\
\textbf{C2.} Tómendegi kolconıń úles kóplikleri ideal bolıwın kórsetiń:
\(R = \mathbb{Z}_{28}\), \(I = \left\{ \overline{0},\overline{7},\overline{14},\overline{21} \right\}\) \\
\textbf{C3.} Tómendegi sáwlelendiriwdi gomomorfizm shártlerine tekseriń. \(f(x) = \sqrt{x}\) \\

\end{tabular}
\vspace{1cm}


\begin{tabular}{m{17cm}}
\textbf{48-variant}
\newline

\textbf{T1.} Kópagzalılar kolcosı (anıqlaması, qásiyetleri, mısallar) \\
\textbf{T2.} Ajıralatuǵın maydanlar (anıqlaması, qásiyetleri, mısallar) \\
\textbf{A1.} \(\mathbf{Q}\)da \(\sqrt{3} + \sqrt{2}i\) tiń minimal kópaǵzalısın tabıń. \\
\textbf{A2.} Ámellerdi orınlań: \(\mathbb{Z}_{5}\) te \(\left( 3x^{2} + 3x - 4 \right)\left( 4x^{2} + 2 \right)\) \\
\textbf{A3.} Tómendegi sannıń \(p\)-adikalıq normasın tabıń. \(|6|_{3} =\) \\
\textbf{B1.} Tómendegi kóplik kolco dúzedi ma? \(\mathbb{Q(}\sqrt[3]{2}) = \left\{ a + b\sqrt[3]{2}:a,b \in \mathbb{Q} \right\}\) \\
\textbf{B2.} Tómendegi kóplik\(M_{2 \times 2}\left( \mathbb{R} \right)\)kolcosınıń úles kolcosı ekenliginkórsetiń. \(A = \left\{ \left. \ \begin{pmatrix}
a & b \\
0 & a
\end{pmatrix} \right|a,b\mathbb{\in Q} \right\}\) \\
\textbf{B3.} \(x^{2} - 3\)kópaǵzalı \(\mathbb{Q}(\sqrt{2})\)de keltirilmeytuǵınkópaǵzalı ekenliginkórsetiń. \\
\textbf{C1.} Tómendegi kóplikti maydan shártlerine tekseriń. \(Z_{p}\) \\
\textbf{C2.} \(M_{2 \times 2}\left( \mathbf{Z} \right)\)kolcoda\(I = \left\{ \begin{bmatrix}
a & 0 \\
b & 0
\end{bmatrix}|a,b\mathbb{\in Z} \right\}\) ideal boladı ma? \\
\textbf{C3.} Tómendegi sáwlelendiriwdi gomomorfizm shártlerine tekseriń. \(f\left( \begin{pmatrix}
a & 0 \\
0 & a
\end{pmatrix} \right) = a\) \\

\end{tabular}
\vspace{1cm}


\begin{tabular}{m{17cm}}
\textbf{49-variant}
\newline

\textbf{T1.} Keltirilmeytuǵın kópaǵzalılar (anıqlaması, qásiyetleri, mısallar) \\
\textbf{T2.} Pútinlik oblastı hám maydan (anıqlaması, qásiyetleri, mısallar) \\
\textbf{A1.} \(\mathbf{Q}\)da \(\sqrt{2 + \sqrt{5}}\)tiń minimal kópaǵzalısın tabıń. \\
\textbf{A2.} Ámellerdi orınlań: \(\mathbb{Z}_{5}\) te \(\left( 3x^{2} + 3x - 4 \right)\left( x^{2} + 2 \right)\) \\
\textbf{A3.} Tómendegi sannıń \(p\)-adikalıq normasın tabıń. \(|35|_{7} =\) \\
\textbf{B1.} Tómendegi kóplik kolco dúzedi ma? \(R = M_{2 \times 2}\left( \mathbf{Q} \right)\) \\
\textbf{B2.} Tómendegi kóplik\(M_{2 \times 2}\left( \mathbb{R} \right)\) niń úles maydanı ekenliginkórsetiń. \(A = \left\{ \left. \ \begin{pmatrix}
a & b \\
 - b & a
\end{pmatrix} \right|a,b\mathbb{\in Z},a^{2} + b^{2} \neq 0 \right\}\) \\
\textbf{B3.} \(x^{2} + x + 1\) kópaǵzalı \(\mathbb{Z}_{5}\) de keltirilmeytuǵın kópaǵzalı ekenligin kórsetiń. \\
\textbf{C1.} Tómendegi kóplikti maydan shártlerine tekseriń. \(\left\{ a + b\sqrt{7}|a,b \in R \right\}\) \\
\textbf{C2.} Tómendegi kolconıń barlıq ideallarıń tabıń. Bul ideallardan qaysı-biri maksimal boladı? \(\mathbb{M}_{2}\left( \mathbb{R} \right)\), elementleri \(\mathbb{R}\) bolǵan \(2 \times 2\) matrica \\
\textbf{C3.} Tómendegi sáwlelendiriwdi gomomorfizm shártlerine tekseriń. \(f(x + iy) = x \cdot y\) \\

\end{tabular}
\vspace{1cm}


\begin{tabular}{m{17cm}}
\textbf{50-variant}
\newline

\textbf{T1.} Maydanlar avtomorfizmleri (anıqlaması, qásiyetleri, mısallar) \\
\textbf{T2.} Teńlemelerdiń radikallarda sheshiliwi (anıqlaması, qásiyetleri, mısallar) \\
\textbf{A1.} \(\mathbf{Q}\)da \(\sqrt{2} + \sqrt{5}\)tiń minimal kópaǵzalısın tabıń. \\
\textbf{A2.} Ámellerdi orınlań: \(\mathbb{Z}_{12}\) de \(\left( 5x^{2} + 3x - 4 \right)\left( 4x^{2} - x + 9 \right)\) \\
\textbf{A3.} Tómendegi sannıń \(p\)-adikalıq normasın tabıń. \(|625|_{5} =\) \\
\textbf{B1.} Tómendegi kóplik kolco dúzedi ma? \(R = \left\{ a + b\sqrt{2}\ :\ \ \ \ a,b \in \mathbf{Z} \right\}\) \\
\textbf{B2.} Tómendegi kóplik\(M_{2 \times 2}\left( \mathbb{R} \right)\)kolcosınıń úles kolcosı ekenliginkórsetiń.
\begin{quote}
\[T = \left\{ \begin{pmatrix}
a + b & b \\
 - b & a
\end{pmatrix}\left| \ \ a,b\mathbb{\in Z} \right.\  \right\}\]
\end{quote} \\
\textbf{B3.} \(x^{2} + x + 1\) kópaǵzalı \(\mathbb{Z}_{3}\) de keltirilmeytuǵın kópaǵzalı ekenligin kórsetiń. \\
\textbf{C1.} Tómendegi kóplikti maydan shártlerine tekseriń. \(\mathbb{Z}\left\lbrack \sqrt{n} \right\rbrack = \left\{ x + y\sqrt{n}\ \ \left| \right.\ x,y \in \mathbb{Z} \right\}\) \\
\textbf{C2.} Tómendegi kolconıń úles kóplikleri ideal bolıwın kórsetiń:
\(R = \mathbb{Z}_{24}\), \(I = \{\overline{0},\overline{8},\overline{16}\}\). \\
\textbf{C3.} Tómendegi sáwlelendiriwdi gomomorfizm shártlerine tekseriń. \(f(x) = 5^{x}\) \\

\end{tabular}
\vspace{1cm}


\begin{tabular}{m{17cm}}
\textbf{51-variant}
\newline

\textbf{T1.} Kolco gomomorfizmleri hám ideallar (anıqlaması, qásiyetleri, mısallar) \\
\textbf{T2.} Maydanlar keńeytpesi (anıqlaması, qásiyetleri, mısallar) \\
\textbf{A1.} \(\mathbf{Q}\)da \(\sqrt{1/3 + \sqrt{7}}\)tiń minimal kópaǵzalısın tabıń. \\
\textbf{A2.} Ámellerdi orınlań: \(\mathbb{Z}_{12}\) de \(\left( 5x^{2} + 3x - 4 \right) + \left( 4x^{2} - x + 9 \right)\) \\
\textbf{A3.} Tómendegi sannıń \(p\)-adikalıq normasın tabıń. \(|\frac{1}{4}|_{2} =\) \\
\textbf{B1.} Tómendegi kóplik kolco dúzedi ma? \(G = \left\{ a^{n},a \neq 0, \pm 1,n \in \mathbb{Z} \right\}\) \\
\textbf{B2.} Tómendegi kóplik\(M_{2 \times 2}\left( \mathbb{R} \right)\) niń úles maydanı ekenliginkórsetiń.
\[A = \left\{ \left. \ \begin{pmatrix}
a & b \\
0 & a
\end{pmatrix} \right|a,b\mathbb{\in Z},a \neq 0 \right\}\] \\
\textbf{B3.} Tómendegi kópaǵzalı \(\mathbb{Q\lbrack}x\rbrack\) da keltirilmeytuǵın kópaǵzalı boladıma? \(3x^{5} - 4x^{3} - 6x^{2} + 6\) \\
\textbf{C1.} Tómendegi kóplikti maydan shártlerine tekseriń. \(\mathbb{Z}\left( \sqrt{3} \right) = \left\{ a + b\sqrt{3}:a,b \in \mathbb{Z} \right\}\) \\
\textbf{C2.} \(\mathbb{Z}_{24}\)tiń barlıq idealların tabıń \\
\textbf{C3.} Tómendegi sáwlelendiriwdi gomomorfizm shártlerine tekseriń.
\[f:\begin{pmatrix}
a & b \\
 - b & a
\end{pmatrix} \rightarrow a + bi\] \\

\end{tabular}
\vspace{1cm}


\begin{tabular}{m{17cm}}
\textbf{52-variant}
\newline

\textbf{T1.} Shekli maydannıń strukturası (anıqlaması, qásiyetleri, mısallar) \\
\textbf{T2.} Bólshekler maydanı (anıqlaması, qásiyetleri, mısallar) \\
\textbf{A1.} \(\mathbf{Q}\)da \(\sqrt{2 + \sqrt{2}}\)tiń minimal kópaǵzalısın tabıń. \\
\textbf{A2.} Ámellerdi orınlań: \(\mathbb{Z}_{10}\) da \(\left( 5x^{2} + 3x - 4 \right)\left( 4x^{2} - x + 9 \right)\) \\
\textbf{A3.} Tómendegi sannıń \(p\)-adikalıq normasın tabıń. \(|256|_{2} =\) \\
\textbf{B1.} Tómendegi kóplik kolco dúzedi ma? \(\mathbb{Q}\left\lfloor i \right\rfloor = \left\{ a + bi\ \ :\ \ a,b\mathbb{\in Q} \right\}\) \\
\textbf{B2.} Tómendegi kóplik\(M_{2 \times 2}\left( \mathbb{R} \right)\) niń úles maydanı ekenliginkórsetiń. \(A = \left\{ \left. \ \begin{pmatrix}
a & b\sqrt{7} \\
 - b\sqrt{7} & a
\end{pmatrix} \right|a,b\mathbb{\in Q},a^{2} + 7b^{2} \neq 0 \right\}\) \\
\textbf{B3.} Tómendegi maydanlardıń berilgen kópaǵzalı arqalı ajıralıw maydanın tabıń. \(\mathbb{Q}\) da \(x^{4} + 1\). \\
\textbf{C1.} Tómendegi kóplikti maydan shártlerine tekseriń. \(R = \left\{ \begin{pmatrix}
a & b \\
2b & a
\end{pmatrix}\ \ :\ \ a,b \in \mathbf{Q} \right\}\) \\
\textbf{C2.} Tómendegi kolconıń barlıq ideallarıń tabıń. Bul ideallardan qaysı-biri maksimal boladı? \(\mathbb{Z}_{27}\) \\
\textbf{C3.} Tómendegi sáwlelendiriwdi gomomorfizm shártlerine tekseriń. \(f(x) = \sqrt[3]{x}\) \\

\end{tabular}
\vspace{1cm}


\begin{tabular}{m{17cm}}
\textbf{53-variant}
\newline

\textbf{T1.} Maksimal hám ápiwayı ideallar (anıqlaması, qásiyetleri, mısallar) \\
\textbf{T2.} Keltirilmeytuǵın kópaǵzalılar (anıqlaması, qásiyetleri, mısallar) \\
\textbf{A1.} \(\mathbf{Q}\)da \(\sqrt{6 + 3\sqrt{2}}\)tiń minimal kópaǵzalısın tabıń. \\
\textbf{A2.} Ámellerdi orınlań: \(\mathbb{Z}_{12}\) de \(\left( 5x^{2} + 3x - 2 \right)^{2}\) \\
\textbf{A3.} Tómendegi sannıń \(p\)-adikalıq normasın tabıń. \(|15|_{3} =\) \\
\textbf{B1.} Tómendegi kóplik kolco dúzedi ma? \(R = \left\{ a + b\sqrt{2}\ :\ \ \ \ a,b \in \mathbf{Z} \right\}\) \\
\textbf{B2.} \(Z_{12}\) kolconıń barlıq úles kolcoların anıqlań. \\
\textbf{B3.} Tómendegi maydanlardıń berilgen kópaǵzalı arqalı ajıralıw maydanın tabıń. \(\mathbb{Q}\) da \(x^{4} - 10x^{2} + 21\). \\
\textbf{C1.} Tómendegi kóplikti maydan shártlerine tekseriń. \(R = M_{2 \times 2}\left( \mathbf{Q} \right)\) \\
\textbf{C2.} Tómendegi kolconıń úles kóplikleri ideal bolıwın kórsetiń:
\(R = \left\{ \begin{pmatrix}
a & b \\
0 & c
\end{pmatrix}\ |\ a,b,c \in \mathbb{Z} \right\}\), \(I = \left\{ \begin{pmatrix}
0 & b \\
0 & c
\end{pmatrix}\ |\ a \in \mathbb{Z} \right\}\). \\
\textbf{C3.} Tómendegi sáwlelendiriwdi gomomorfizm shártlerine tekseriń. \(f(x) = x^{2} + x\) \\

\end{tabular}
\vspace{1cm}


\begin{tabular}{m{17cm}}
\textbf{54-variant}
\newline

\textbf{T1.} Kolco gomomorfizmleri hám ideallar (anıqlaması, qásiyetleri, mısallar) \\
\textbf{T2.} Bólshekler maydanı (anıqlaması, qásiyetleri, mısallar) \\
\textbf{A1.} \(\mathbf{Q}\)da \(\sqrt{2 + 2\sqrt{2}}\)tiń minimal kópaǵzalısın tabıń. \\
\textbf{A2.} Tómendegi kópaǵzalınıń barlıq nollerin tabıń: \(\mathbb{Z}_{5}\) de \(3x^{3} - 4x^{2} - x + 4\) \\
\textbf{A3.} Tómendegi sannıń \(p\)-adikalıq normasın tabıń. \(|\frac{8}{18}|_{5} =\) \\
\textbf{B1.} Tómendegi kóplik kolco dúzedi ma? \(\mathbb{Q(}i\sqrt{n}) = \{ x + iy\sqrt{n}\ |\ x,y \in \mathbb{Q}\}\) \\
\textbf{B2.} Tómendegi kóplik\(M_{2 \times 2}\left( \mathbb{R} \right)\) niń úles maydanı ekenliginkórsetiń. \(A = \left\{ \left. \ \begin{pmatrix}
a & b\sqrt{2} \\
b\sqrt{2} & a
\end{pmatrix} \right|a,b\mathbb{\in Q},a^{2} - 2b^{2} \neq 0 \right\}\) \\
\textbf{B3.} Tómendegi kópaǵzalı \(\mathbb{Q\lbrack}x\rbrack\) da keltirilmeytuǵın kópaǵzalı boladıma? \(x^{4} - 5x^{3} + 3x - 2\) \\
\textbf{C1.} Tómendegi kóplikti maydan shártlerine tekseriń. \(\mathbb{Q(}i\sqrt{n}) = \{ x + iy\sqrt{n}\ |\ x,y \in \mathbb{Q}\}\) \\
\textbf{C2.} Tómendegi kolconıń úles kóplikleri ideal bolıwın kórsetiń:
\(R = \mathbb{Z}_{28}\), \(I = \{\overline{0},\overline{7},\overline{14},\overline{21}\}\). \\
\textbf{C3.} Tómendegi sáwlelendiriwdi gomomorfizm shártlerine tekseriń. \(f(a + ib) = \begin{pmatrix}
a & b \\
 - b & a
\end{pmatrix}\) \\

\end{tabular}
\vspace{1cm}


\begin{tabular}{m{17cm}}
\textbf{55-variant}
\newline

\textbf{T1.} Ajıralatuǵın maydanlar (anıqlaması, qásiyetleri, mısallar) \\
\textbf{T2.} Pútinlik oblastı hám maydan (anıqlaması, qásiyetleri, mısallar) \\
\textbf{A1.} \(\mathbf{Q}\)da \(\sqrt{2} + \sqrt[3]{7}\)tiń minimal kópaǵzalısın tabıń. \\
\textbf{A2.} Ámellerdi orınlań: \(\mathbb{Z}_{5}\) te \(\left( 3x^{2} + 2x - 4 \right) + \left( 4x^{2} + 2 \right)\) \\
\textbf{A3.} Tómendegi sannıń \(p\)-adikalıq normasın tabıń. \(|\frac{7}{15}|_{3} =\) \\
\textbf{B1.} Tómendegi kóplik kolco dúzedi ma? \(\mathbb{Z}\left( \sqrt{3} \right) = \left\{ a + b\sqrt{3}:a,b \in \mathbb{Z} \right\}\) \\
\textbf{B2.} \(Z_{20}\) kolconıń barlıq úles kolcoların anıqlań. \\
\textbf{B3.} Tómendegi maydanlardıń berilgen kópaǵzalı arqalı ajıralıw maydanın tabıń. \(\mathbb{Q}\) da \(x^{3} - 3\). \\
\textbf{C1.} Tómendegi kóplikti maydan shártlerine tekseriń. \(G = \left\{ a^{n},a \neq 0, \pm 1,n \in \mathbb{Z} \right\}\) \\
\textbf{C2.} Tómendegi kolconıń úles kóplikleri ideal bolıwın kórsetiń:
\(R = \left\{ \begin{pmatrix}
a & b \\
0 & c
\end{pmatrix}\ |\ a,b,c \in \mathbb{Z} \right\}\), \(I = \left\{ \begin{pmatrix}
0 & a \\
0 & 0
\end{pmatrix}\ |\ a \in \mathbb{Z} \right\}\) \\
\textbf{C3.} Tómendegi sáwlelendiriwdi gomomorfizm shártlerine tekseriń. \(f(a) = a^{n}\) \\

\end{tabular}
\vspace{1cm}


\begin{tabular}{m{17cm}}
\textbf{56-variant}
\newline

\textbf{T1.} Kópagzalılar kolcosı (anıqlaması, qásiyetleri, mısallar) \\
\textbf{T2.} p-Adikalıq norma, p-Adikalıq norma (anıqlaması, qásiyetleri, mısallar) \\
\textbf{A1.} \(\mathbf{Q}\)da \(\sqrt{\sqrt[3]{2} - i}\)tiń minimal kópaǵzalısın tabıń. \\
\textbf{A2.} Ámellerdi orınlań: \(\mathbb{Z}_{5}\) te \(\left( x^{2} + 3x - 1 \right)^{2}\) \\
\textbf{A3.} Tómendegi sannıń \(p\)-adikalıq normasın tabıń. \(|48|_{3} =\) \\
\textbf{B1.} Tómendegi kóplik kolco dúzedi ma? \(\left\{ a + b\sqrt{7}|a,b \in R \right\}\) \\
\textbf{B2.} Tómendegi kóplik\(M_{2 \times 2}\left( \mathbb{R} \right)\)kolcosınıń úles kolcosı ekenliginkórsetiń.
\[A = \left\{ \left. \ \begin{pmatrix}
a & b \\
 - b & a
\end{pmatrix} \right|a,b\mathbb{\in R} \right\}\] \\
\textbf{B3.} Tómendegi kópaǵzalı\(\mathbf{Z}_{2}\lbrack x\rbrack\)da keltirilmeytuǵınkópaǵzalıboladıma? \(x^{3} + x + 1\) \\
\textbf{C1.} Tómendegi kóplikti maydan shártlerine tekseriń. \(\mathbf{Q} = \left\{ \frac{m}{n},\ \ m \in Z,\ n \in N \right\}\) \\
\textbf{C2.} Tómendegi kolconıń úles kóplikleri ideal bolıwın kórsetiń:
\(R\mathbb{= Z\lbrack}\sqrt{7}\rbrack\), \(I = \{ a + b\sqrt{7}\ |\ \ a,b \in \mathbb{Z,}a - b\ \ jupsan\}\). \\
\textbf{C3.} Tómendegi sáwlelendiriwdi gomomorfizm shártlerine tekseriń. \(f(x) = e^{x}\) \\

\end{tabular}
\vspace{1cm}


\begin{tabular}{m{17cm}}
\textbf{57-variant}
\newline

\textbf{T1.} Shekli maydannıń strukturası (anıqlaması, qásiyetleri, mısallar) \\
\textbf{T2.} Maydanlar keńeytpesi (anıqlaması, qásiyetleri, mısallar) \\
\textbf{A1.} \(\mathbf{Q}\)da \(\sqrt{2} + \sqrt{5}\)tiń minimal kópaǵzalısın tabıń. \\
\textbf{A2.} Ámellerdi orınlań: \(\mathbb{Z}_{9}\) da \(\left( 7x^{3} + 3x^{2} - x \right) + \left( 6x^{2} - 8x + 4 \right)\) \\
\textbf{A3.} Tómendegi sannıń \(p\)-adikalıq normasın tabıń. \(|48|_{3} =\) \\
\textbf{B1.} Tómendegi kóplik kolco dúzedi ma? \(R = \left\{ \begin{pmatrix}
a & b \\
2b & a
\end{pmatrix}\ \ :\ \ a,b \in \mathbf{Q} \right\}\) \\
\textbf{B2.} Tómendegi kóplik\(M_{2 \times 2}\left( \mathbb{R} \right)\) niń úles maydanı ekenligin kórsetiń. \(A = \left\{ \left. \ \begin{pmatrix}
a & 0 \\
2b & a
\end{pmatrix} \right|a,b\mathbb{\in R},a \neq 0 \right\}\) \\
\textbf{B3.} \(x^{2} - 7\) kópaǵzalı\(\mathbb{Q}(\sqrt{3})\)de keltirilmeytuǵın kópaǵzalı ekenligin kórsetiń. \\
\textbf{C1.} Tómendegi kóplikti maydan shártlerine tekseriń. \(G = \left\{ a^{n},a \neq 0, \pm 1,n \in \mathbb{Z} \right\}\) \\
\textbf{C2.} Tómendegi kolconıń úles kóplikleri ideal bolıwın kórsetiń:
\(R\mathbb{= Z\lbrack}\sqrt{7}\rbrack\), \(I = \{ a + b\sqrt{7}\ |\ \ a,b \in \mathbb{Z,}a - b\ \ jupsan\}\). \\
\textbf{C3.} Tómendegi sáwlelendiriwdi gomomorfizm shártlerine tekseriń. \(f(x) = \sqrt{x}\) \\

\end{tabular}
\vspace{1cm}


\begin{tabular}{m{17cm}}
\textbf{58-variant}
\newline

\textbf{T1.} Kolcolar (anıqlaması, qásiyetleri, mısallar) \\
\textbf{T2.} p-Adikalıq sanlardıń keńisligi (anıqlaması, qásiyetleri, mısallar) \\
\textbf{A1.} \(\mathbf{Q}\)da \(\sqrt{2} + \sqrt{3}\)tiń minimal kópaǵzalısın tabıń. \\
\textbf{A2.} Ámellerdi orınlań: \(\mathbb{Z}_{10}\) da \(\left( 7x^{3} + 3x^{2} - x \right) + \left( 6x^{2} - 8x + 4 \right)\) \\
\textbf{A3.} Tómendegi sannıń \(p\)-adikalıq normasın tabıń. \(|\frac{1}{4}|_{2} =\) \\
\textbf{B1.} Tómendegi kóplik kolco dúzedi ma? \(\mathbb{Q}\left( \sqrt{2} \right) = \left\{ a + b\sqrt{2}:a,b \in \mathbb{Q} \right\}\) \\
\textbf{B2.} Tómendegi kóplik\(M_{2 \times 2}\left( \mathbb{R} \right)\)kolcosınıń úles kolcosı ekenliginkórsetiń. \(A = \left\{ \left. \ \begin{pmatrix}
a & 0 \\
0 & 0
\end{pmatrix} \right|a \in Z \right\}\) \\
\textbf{B3.} Tómendegi maydanlardıń berilgen kópaǵzalı arqalı ajıralıw maydanın tabıń.
\(\mathbb{Q}\) da \(x^{4} - 5x^{2} + 21\). \\
\textbf{C1.} Tómendegi kóplikti maydan shártlerine tekseriń. \(G = \left\{ a^{n},a \neq 0, \pm 1,n \in \mathbb{Z} \right\}\) \\
\textbf{C2.} \(Z_{12}\) niń barlıq idealların tabıń. \\
\textbf{C3.} Tómendegi sáwlelendiriwdi gomomorfizm shártlerine tekseriń. \(f\left( a + \sqrt{2}b \right) = a + bi\) \\

\end{tabular}
\vspace{1cm}


\begin{tabular}{m{17cm}}
\textbf{59-variant}
\newline

\textbf{T1.} Maydanlar avtomorfizmleri (anıqlaması, qásiyetleri, mısallar) \\
\textbf{T2.} Maksimal hám ápiwayı ideallar (anıqlaması, qásiyetleri, mısallar) \\
\textbf{A1.} \(\mathbf{Q}\)da \(\sqrt{6 + 3\sqrt{2}}\)tiń minimal kópaǵzalısın tabıń. \\
\textbf{A2.} Tómendegi kópaǵzalınıń barlıq nollerin tabıń: \(\mathbb{Z}_{2}\) de \(x^{3} + x + 1\) \\
\textbf{A3.} Tómendegi sannıń \(p\)-adikalıq normasın tabıń. \(|625|_{5} =\) \\
\textbf{B1.} Tómendegi kóplik kolco dúzedi ma? \(G = \left\{ a^{n},a \neq 0, \pm 1,n \in \mathbb{Z} \right\}\) \\
\textbf{B2.} \(Z_{16}\) kolconıń barlıq úles kolcoların anıqlań. \\
\textbf{B3.} \(x^{2} + 1\) kópaǵzalı \(\mathbb{Z}_{3}\)de keltirilmeytuǵın kópaǵzalı ekenligin kórsetiń. \\
\textbf{C1.} Tómendegi kóplikti maydan shártlerine tekseriń. \(\left\{ a + b\sqrt{7}|a,b \in R \right\}\) \\
\textbf{C2.} Tómendegi kolconıń úles kóplikleri ideal bolıwın kórsetiń:
\(R = \left\{ \begin{pmatrix}
a & b \\
0 & c
\end{pmatrix}\ |\ a,b,c \in \mathbb{Z} \right\}\), \(I = \left\{ \begin{pmatrix}
0 & b \\
0 & c
\end{pmatrix}\ |\ a \in \mathbb{Z} \right\}\). \\
\textbf{C3.} Tómendegi sáwlelendiriwdi gomomorfizm shártlerine tekseriń. \(f:\begin{pmatrix}
a & b \\
0 & c
\end{pmatrix} \rightarrow a\) \\

\end{tabular}
\vspace{1cm}


\begin{tabular}{m{17cm}}
\textbf{60-variant}
\newline

\textbf{T1.} Teńlemelerdiń radikallarda sheshiliwi (anıqlaması, qásiyetleri, mısallar) \\
\textbf{T2.} Fundamental teoremalar (anıqlaması, qásiyetleri, mısallar) \\
\textbf{A1.} \(\mathbf{Q}\)da \(\sqrt{2} + \sqrt{3}i\)tiń minimal kópaǵzalısın tabıń. \\
\textbf{A2.} Ámellerdi orınlań: \(\mathbb{Z}_{12}\) de \((3x - 2)^{3}\) \\
\textbf{A3.} Tómendegi sannıń \(p\)-adikalıq normasın tabıń. \(|\frac{7}{15}|_{3} =\) \\
\textbf{B1.} Tómendegi kóplik kolco dúzedi ma? \(\mathbb{Z}\left\lbrack \sqrt{n} \right\rbrack = \left\{ x + y\sqrt{n}\ \ \left| \right.\ x,y \in \mathbb{Z} \right\}\) \\
\textbf{B2.} Tómendegi kóplik\(M_{2 \times 2}\left( \mathbb{R} \right)\) niń úles maydanı ekenliginkórsetiń. \(A = \left\{ \left. \ \begin{pmatrix}
a & 0 \\
0 & a
\end{pmatrix} \right|a\mathbb{\in R},a \neq 0 \right\}\) \\
\textbf{B3.} \(x^{4} - 5x^{2} + 6\) kópaǵzalı \(\mathbb{Q}\) de keltirilmeytuǵın kópaǵzalı ekenligin kórsetiń. \\
\textbf{C1.} Tómendegi kóplikti maydan shártlerine tekseriń. \(7\mathbb{Z}\) \\
\textbf{C2.} \(\mathbb{Z}_{24}\)tiń barlıq idealların tabıń \\
\textbf{C3.} Tómendegi sáwlelendiriwdi gomomorfizm shártlerine tekseriń. \(f(a) = a^{n}\) \\

\end{tabular}
\vspace{1cm}


\begin{tabular}{m{17cm}}
\textbf{61-variant}
\newline

\textbf{T1.} Maksimal hám ápiwayı ideallar (anıqlaması, qásiyetleri, mısallar) \\
\textbf{T2.} Bólshekler maydanı (anıqlaması, qásiyetleri, mısallar) \\
\textbf{A1.} \(\mathbf{Q}\)da \(\sqrt{3} + \sqrt[3]{5}\)tiń minimal kópaǵzalısın tabıń. \\
\textbf{A2.} Ámellerdi orınlań: \(\mathbb{Z}_{12}\) de \(\left( 5x^{2} + 3x - 4 \right)\left( 4x^{2} - x + 9 \right)\) \\
\textbf{A3.} Tómendegi sannıń \(p\)-adikalıq normasın tabıń. \(|35|_{7} =\) \\
\textbf{B1.} Tómendegi kóplik kolco dúzedi ma? \(7\mathbb{Z}\) \\
\textbf{B2.} \(Z_{16}\) kolconıń barlıq úles kolcoların anıqlań. \\
\textbf{B3.} Tómendegi maydanlardıń berilgen kópaǵzalı arqalı ajıralıw maydanın tabıń. \(\mathbb{Q}\) da \(x^{3} - 3\). \\
\textbf{C1.} Tómendegi kóplikti maydan shártlerine tekseriń. \(\mathbb{Q(}i\sqrt{n}) = \{ x + iy\sqrt{n}\ |\ x,y \in \mathbb{Q}\}\) \\
\textbf{C2.} Tómendegi kolconıń úles kóplikleri ideal bolıwın kórsetiń:
\(R = \left\{ \begin{pmatrix}
a & b \\
0 & c
\end{pmatrix}\ |\ a,b,c \in \mathbb{Z} \right\}\), \(I = \left\{ \begin{pmatrix}
0 & a \\
0 & 0
\end{pmatrix}\ |\ a \in \mathbb{Z} \right\}\) \\
\textbf{C3.} Tómendegi sáwlelendiriwdi gomomorfizm shártlerine tekseriń.
\[f:\begin{pmatrix}
a & b \\
 - b & a
\end{pmatrix} \rightarrow a + bi\] \\

\end{tabular}
\vspace{1cm}


\begin{tabular}{m{17cm}}
\textbf{62-variant}
\newline

\textbf{T1.} p-Adikalıq sanlardıń keńisligi (anıqlaması, qásiyetleri, mısallar) \\
\textbf{T2.} Ajıralatuǵın maydanlar (anıqlaması, qásiyetleri, mısallar) \\
\textbf{A1.} \(\mathbf{Q}\)da \(\sqrt{2 + 2\sqrt{2}}\)tiń minimal kópaǵzalısın tabıń. \\
\textbf{A2.} Tómendegi kópaǵzalınıń barlıq nollerin tabıń: \(\mathbb{Z}_{5}\) de \(3x^{3} - 4x^{2} - x + 4\) \\
\textbf{A3.} Tómendegi sannıń \(p\)-adikalıq normasın tabıń. \(|\frac{9}{12}|_{7} =\) \\
\textbf{B1.} Tómendegi kóplik kolco dúzedi ma? \(R = M_{2 \times 2}\left( \mathbf{Q} \right)\) \\
\textbf{B2.} Tómendegi kóplik\(M_{2 \times 2}\left( \mathbb{R} \right)\) niń úles maydanı ekenliginkórsetiń. \(A = \left\{ \left. \ \begin{pmatrix}
a & b \\
 - b & a
\end{pmatrix} \right|a,b\mathbb{\in Z},a^{2} + b^{2} \neq 0 \right\}\) \\
\textbf{B3.} Tómendegi maydanlardıń berilgen kópaǵzalı arqalı ajıralıw maydanın tabıń.
\(\mathbb{Q}\) da \(x^{4} - 5x^{2} + 21\). \\
\textbf{C1.} Tómendegi kóplikti maydan shártlerine tekseriń. \(\mathbb{Z}\left( \sqrt{3} \right) = \left\{ a + b\sqrt{3}:a,b \in \mathbb{Z} \right\}\) \\
\textbf{C2.} Tómendegi kolconıń barlıq ideallarıń tabıń. Bul ideallardan qaysı-biri maksimal boladı? \(\mathbb{M}_{2}\left( \mathbb{R} \right)\), elementleri \(\mathbb{R}\) bolǵan \(2 \times 2\) matrica \\
\textbf{C3.} Tómendegi sáwlelendiriwdi gomomorfizm shártlerine tekseriń. \(f(x) = x^{2} + x\) \\

\end{tabular}
\vspace{1cm}


\begin{tabular}{m{17cm}}
\textbf{63-variant}
\newline

\textbf{T1.} Kópagzalılar kolcosı (anıqlaması, qásiyetleri, mısallar) \\
\textbf{T2.} Maydanlar keńeytpesi (anıqlaması, qásiyetleri, mısallar) \\
\textbf{A1.} \(\mathbf{Q}\)da \(\sqrt{1/3 + \sqrt{7}}\)tiń minimal kópaǵzalısın tabıń. \\
\textbf{A2.} Ámellerdi orınlań: \(\mathbb{Z}_{5}\) te \(\left( 3x^{2} + 3x - 4 \right)\left( 4x^{2} + 2 \right)\) \\
\textbf{A3.} Tómendegi sannıń \(p\)-adikalıq normasın tabıń. \(|\frac{25}{36}|_{2} =\) \\
\textbf{B1.} Tómendegi kóplik kolco dúzedi ma? \(\mathbb{Q}\left\lfloor i \right\rfloor = \left\{ a + bi\ \ :\ \ a,b\mathbb{\in Q} \right\}\) \\
\textbf{B2.} Tómendegi kóplik\(M_{2 \times 2}\left( \mathbb{R} \right)\) niń úles maydanı ekenliginkórsetiń. \(A = \left\{ \left. \ \begin{pmatrix}
a & b\sqrt{2} \\
b\sqrt{2} & a
\end{pmatrix} \right|a,b\mathbb{\in Q},a^{2} - 2b^{2} \neq 0 \right\}\) \\
\textbf{B3.} \(x^{2} - 3\)kópaǵzalı \(\mathbb{Q}(\sqrt{2})\)de keltirilmeytuǵınkópaǵzalı ekenliginkórsetiń. \\
\textbf{C1.} Tómendegi kóplikti maydan shártlerine tekseriń. \(R = M_{2 \times 2}\left( \mathbf{Z} \right)\) \\
\textbf{C2.} Tómendegi kolconıń barlıq ideallarıń tabıń. Bul ideallardan qaysı-biri maksimal boladı? \(\mathbb{Z}_{25}\) \\
\textbf{C3.} Tómendegi sáwlelendiriwdi gomomorfizm shártlerine tekseriń. \(f(x + iy) = x \cdot y\) \\

\end{tabular}
\vspace{1cm}


\begin{tabular}{m{17cm}}
\textbf{64-variant}
\newline

\textbf{T1.} Kolco gomomorfizmleri hám ideallar (anıqlaması, qásiyetleri, mısallar) \\
\textbf{T2.} Shekli maydannıń strukturası (anıqlaması, qásiyetleri, mısallar) \\
\textbf{A1.} \(\mathbf{Q}\)da \(\sqrt{2 + \sqrt{5}}\)tiń minimal kópaǵzalısın tabıń. \\
\textbf{A2.} Ámellerdi orınlań: \(\mathbb{Z}_{12}\) de \(\left( 5x^{2} + 3x - 4 \right) + \left( 4x^{2} - x + 9 \right)\) \\
\textbf{A3.} Tómendegi sannıń \(p\)-adikalıq normasın tabıń. \(|15|_{3} =\) \\
\textbf{B1.} Tómendegi kóplik kolco dúzedi ma? \(\mathbb{Q(}\sqrt[3]{2}) = \left\{ a + b\sqrt[3]{2}:a,b \in \mathbb{Q} \right\}\) \\
\textbf{B2.} \(Z_{20}\) kolconıń barlıq úles kolcoların anıqlań. \\
\textbf{B3.} \(x^{4} - 5x^{2} + 6\) kópaǵzalı \(\mathbb{Q}\) de keltirilmeytuǵın kópaǵzalı ekenligin kórsetiń. \\
\textbf{C1.} \(Z_{18}\) kolconıń barlıq úles kolcoların anıqlań. \\
\textbf{C2.} Tómendegi kolconıń barlıq ideallarıń tabıń. Bul ideallardan qaysı-biri maksimal boladı? \(\mathbb{Z}_{27}\) \\
\textbf{C3.} Tómendegi sáwlelendiriwdi gomomorfizm shártlerine tekseriń. \(f(x) = 5^{x}\) \\

\end{tabular}
\vspace{1cm}


\begin{tabular}{m{17cm}}
\textbf{65-variant}
\newline

\textbf{T1.} Kolcolar (anıqlaması, qásiyetleri, mısallar) \\
\textbf{T2.} Keltirilmeytuǵın kópaǵzalılar (anıqlaması, qásiyetleri, mısallar) \\
\textbf{A1.} \(\mathbf{Q}\)da \(\sqrt{\frac{1}{4} + \sqrt{5}}\) tiń minimal kópaǵzalısın tabıń. \\
\textbf{A2.} Ámellerdi orınlań: \(\mathbb{Z}_{5}\) te \(\left( x^{2} + 3x - 1 \right)^{2}\) \\
\textbf{A3.} Tómendegi sannıń \(p\)-adikalıq normasın tabıń. \(|256|_{2} =\) \\
\textbf{B1.} Tómendegi kóplik kolco dúzedi ma? \(\mathbf{Q} = \left\{ \frac{m}{n},\ \ m \in Z,\ n \in N \right\}\) \\
\textbf{B2.} Tómendegi kóplik\(M_{2 \times 2}\left( \mathbb{R} \right)\) niń úles maydanı ekenliginkórsetiń.
\[A = \left\{ \left. \ \begin{pmatrix}
a & b \\
0 & a
\end{pmatrix} \right|a,b\mathbb{\in Z},a \neq 0 \right\}\] \\
\textbf{B3.} \(x^{2} + 1\) kópaǵzalı \(\mathbb{Z}_{3}\)de keltirilmeytuǵın kópaǵzalı ekenligin kórsetiń. \\
\textbf{C1.} Tómendegi kóplikti maydan shártlerine tekseriń. \(R = \left\{ \begin{pmatrix}
a & b \\
2b & a
\end{pmatrix}\ \ :\ \ a,b \in \mathbf{Q} \right\}\) \\
\textbf{C2.} Tómendegi kolconıń úles kóplikleri ideal bolıwın kórsetiń:
\(R = \mathbb{Z}_{28}\), \(I = \{\overline{0},\overline{7},\overline{14},\overline{21}\}\). \\
\textbf{C3.} Tómendegi sáwlelendiriwdi gomomorfizm shártlerine tekseriń.
\[f:x \rightarrow x^{p}\] \\

\end{tabular}
\vspace{1cm}


\begin{tabular}{m{17cm}}
\textbf{66-variant}
\newline

\textbf{T1.} Fundamental teoremalar (anıqlaması, qásiyetleri, mısallar) \\
\textbf{T2.} p-Adikalıq norma, p-Adikalıq norma (anıqlaması, qásiyetleri, mısallar) \\
\textbf{A1.} \(\mathbf{Q}\)da \(\sqrt{3 - \sqrt{3}}\)tiń minimal kópaǵzalısın tabıń. \\
\textbf{A2.} Ámellerdi orınlań: \(\mathbb{Z}_{10}\) da \(\left( 5x^{2} + 3x - 4 \right)\left( 4x^{2} - x + 9 \right)\) \\
\textbf{A3.} Tómendegi sannıń \(p\)-adikalıq normasın tabıń. \(|\frac{8}{18}|_{5} =\) \\
\textbf{B1.} Tómendegi kóplik kolco dúzedi ma? \(G = \left\{ a^{n},a \neq 0, \pm 1,n \in \mathbb{Z} \right\}\) \\
\textbf{B2.} \(Z_{12}\) kolconıń barlıq úles kolcoların anıqlań. \\
\textbf{B3.} \(x^{2} + x + 1\) kópaǵzalı \(\mathbb{Z}_{3}\) de keltirilmeytuǵın kópaǵzalı ekenligin kórsetiń. \\
\textbf{C1.} Tómendegi kóplikti maydan shártlerine tekseriń. \(Z_{p}\) \\
\textbf{C2.} Tómendegi kolconıń barlıq ideallarıń tabıń. Bul ideallardan qaysı-biri maksimal boladı? \(\mathbb{M}_{2}\left( \mathbb{Z} \right)\), elementleri \(\mathbb{Z}\) bolǵan \(2 \times 2\) matrica \\
\textbf{C3.} Tómendegi sáwlelendiriwdi gomomorfizm shártlerine tekseriń. \(f(x) = \sqrt[3]{x}\) \\

\end{tabular}
\vspace{1cm}


\begin{tabular}{m{17cm}}
\textbf{67-variant}
\newline

\textbf{T1.} Teńlemelerdiń radikallarda sheshiliwi (anıqlaması, qásiyetleri, mısallar) \\
\textbf{T2.} Pútinlik oblastı hám maydan (anıqlaması, qásiyetleri, mısallar) \\
\textbf{A1.} \(\mathbf{Q}\)da \(\sqrt{\sqrt[3]{2} - i}\)tiń minimal kópaǵzalısın tabıń. \\
\textbf{A2.} Ámellerdi orınlań: \(\mathbb{Z}_{5}\) te \(\left( 3x^{2} + 2x - 4 \right) + \left( 4x^{2} + 2 \right)\) \\
\textbf{A3.} Tómendegi sannıń \(p\)-adikalıq normasın tabıń. \(|\frac{3}{4}|_{2} =\) \\
\textbf{B1.} Tómendegi kóplik kolco dúzedi ma? \(\mathbb{Z}\left( \sqrt{3} \right) = \left\{ a + b\sqrt{3}:a,b \in \mathbb{Z} \right\}\) \\
\textbf{B2.} Tómendegi kóplik\(M_{2 \times 2}\left( \mathbb{R} \right)\)kolcosınıń úles kolcosı ekenliginkórsetiń. \(A = \left\{ \left. \ \begin{pmatrix}
a & b \\
7b & a
\end{pmatrix} \right|a,b \in Z \right\}\) \\
\textbf{B3.} Tómendegi maydanlardıń berilgen kópaǵzalı arqalı ajıralıw maydanın tabıń. \(\mathbb{Q}\) da \(x^{4} - 10x^{2} + 21\). \\
\textbf{C1.} Tómendegi kóplikti maydan shártlerine tekseriń. \(\mathbb{Q}\left\lfloor i \right\rfloor = \left\{ a + bi\ \ :\ \ a,b\mathbb{\in Q} \right\}\) \\
\textbf{C2.} Tómendegi kolconıń úles kóplikleri ideal bolıwın kórsetiń:
\(R = \mathbb{Z}_{28}\), \(I = \left\{ \overline{0},\overline{7},\overline{14},\overline{21} \right\}\) \\
\textbf{C3.} Tómendegi sáwlelendiriwdi gomomorfizm shártlerine tekseriń.
\[f\left( a - \sqrt{2}b \right) = a + \sqrt{2}b\] \\

\end{tabular}
\vspace{1cm}


\begin{tabular}{m{17cm}}
\textbf{68-variant}
\newline

\textbf{T1.} Maydanlar avtomorfizmleri (anıqlaması, qásiyetleri, mısallar) \\
\textbf{T2.} Shekli maydannıń strukturası (anıqlaması, qásiyetleri, mısallar) \\
\textbf{A1.} \(\mathbf{Q}\)da \(\sqrt{2 + \sqrt{2}}\)tiń minimal kópaǵzalısın tabıń. \\
\textbf{A2.} Ámellerdi orınlań: \(\mathbb{Z}_{9}\) da \(\left( 7x^{3} + 3x^{2} - x \right) + \left( 6x^{2} - 8x + 4 \right)\) \\
\textbf{A3.} Tómendegi sannıń \(p\)-adikalıq normasın tabıń. \(|729|_{3} =\) \\
\textbf{B1.} Tómendegi kóplik kolco dúzedi ma? \(\mathbb{Q(}\sqrt[3]{2}) = \left\{ a + b\sqrt[3]{2}:a,b \in \mathbb{Q} \right\}\) \\
\textbf{B2.} Tómendegi kóplik\(M_{2 \times 2}\left( \mathbb{R} \right)\)kolcosınıń úles kolcosı ekenliginkórsetiń. \(A = \left\{ \left. \ \begin{pmatrix}
a & b \\
0 & a
\end{pmatrix} \right|a,b\mathbb{\in Q} \right\}\) \\
\textbf{B3.} \(x^{2} + x + 1\) kópaǵzalı \(\mathbb{Z}_{5}\) de keltirilmeytuǵın kópaǵzalı ekenligin kórsetiń. \\
\textbf{C1.} Tómendegi kóplikti maydan shártlerine tekseriń. \(R = M_{2 \times 2}\left( \mathbf{Q} \right)\) \\
\textbf{C2.} \(M_{2 \times 2}\left( \mathbf{Z} \right)\)kolcoda\(I = \left\{ \begin{bmatrix}
a & 0 \\
0 & 0
\end{bmatrix}|a\mathbb{\in Z} \right\}\) ideal boladı ma? \\
\textbf{C3.} Tómendegi sáwlelendiriwdi gomomorfizm shártlerine tekseriń. \(f\left( \begin{pmatrix}
a & 0 \\
0 & a
\end{pmatrix} \right) = a\) \\

\end{tabular}
\vspace{1cm}


\begin{tabular}{m{17cm}}
\textbf{69-variant}
\newline

\textbf{T1.} Kópagzalılar kolcosı (anıqlaması, qásiyetleri, mısallar) \\
\textbf{T2.} Ajıralatuǵın maydanlar (anıqlaması, qásiyetleri, mısallar) \\
\textbf{A1.} \(\mathbf{Q}\)da \(\sqrt{3} + \sqrt{2}i\) tiń minimal kópaǵzalısın tabıń. \\
\textbf{A2.} Tómendegi kópaǵzalınıń barlıq nollerin tabıń: \(\mathbb{Z}_{12}\) de \(5x^{3} + 4x^{2} - x + 9\) \\
\textbf{A3.} Tómendegi sannıń \(p\)-adikalıq normasın tabıń. \(|6|_{3} =\) \\
\textbf{B1.} Tómendegi kóplik kolco dúzedi ma? \(7\mathbb{Z}\) \\
\textbf{B2.} Tómendegi kóplik\(M_{2 \times 2}\left( \mathbb{R} \right)\) niń úles maydanı ekenligin kórsetiń. \(A = \left\{ \left. \ \begin{pmatrix}
a & 0 \\
2b & a
\end{pmatrix} \right|a,b\mathbb{\in R},a \neq 0 \right\}\) \\
\textbf{B3.} Tómendegi kópaǵzalı\(\mathbf{Z}_{2}\lbrack x\rbrack\)da keltirilmeytuǵınkópaǵzalıboladıma? \(x^{3} + x + 1\) \\
\textbf{C1.} Tómendegi kóplikti maydan shártlerine tekseriń. \(\mathbb{Z}\left\lbrack \sqrt{n} \right\rbrack = \left\{ x + y\sqrt{n}\ \ \left| \right.\ x,y \in \mathbb{Z} \right\}\) \\
\textbf{C2.} \(M_{2 \times 2}\left( \mathbf{Z} \right)\)kolcoda\(I = \left\{ \begin{bmatrix}
a & 0 \\
b & 0
\end{bmatrix}|a,b\mathbb{\in Z} \right\}\) ideal boladı ma? \\
\textbf{C3.} Tómendegi sáwlelendiriwdi gomomorfizm shártlerine tekseriń. \(f(x) = e^{x}\) \\

\end{tabular}
\vspace{1cm}


\begin{tabular}{m{17cm}}
\textbf{70-variant}
\newline

\textbf{T1.} Kolcolar (anıqlaması, qásiyetleri, mısallar) \\
\textbf{T2.} Maksimal hám ápiwayı ideallar (anıqlaması, qásiyetleri, mısallar) \\
\textbf{A1.} \(\mathbf{Q}\)da \(\sqrt{2} + \sqrt[3]{7}\)tiń minimal kópaǵzalısın tabıń. \\
\textbf{A2.} Ámellerdi orınlań: \(\mathbb{Z}_{5}\) te \(\left( x^{2} + 3x - 4 \right)\left( x^{2} - 3 \right)\) \\
\textbf{A3.} Tómendegi sannıń \(p\)-adikalıq normasın tabıń. \(|124|_{2} =\) \\
\textbf{B1.} Tómendegi kóplik kolco dúzedi ma? \(\mathbb{Z}\left\lbrack \sqrt{n} \right\rbrack = \left\{ x + y\sqrt{n}\ \ \left| \right.\ x,y \in \mathbb{Z} \right\}\) \\
\textbf{B2.} Tómendegi kóplik\(M_{2 \times 2}\left( \mathbb{R} \right)\) niń úles maydanı ekenliginkórsetiń. \(A = \left\{ \left. \ \begin{pmatrix}
a & b\sqrt{7} \\
 - b\sqrt{7} & a
\end{pmatrix} \right|a,b\mathbb{\in Q},a^{2} + 7b^{2} \neq 0 \right\}\) \\
\textbf{B3.} Tómendegi maydanlardıń berilgen kópaǵzalı arqalı ajıralıw maydanın tabıń. \(\mathbb{Q}\) da \(x^{4} - 2\) \\
\textbf{C1.} Tómendegi kóplikti maydan shártlerine tekseriń. \(\mathbf{Q} = \left\{ \frac{m}{n},\ \ m \in Z,\ n \in N \right\}\) \\
\textbf{C2.} Tómendegi kolconıń úles kóplikleri ideal bolıwın kórsetiń:
\(R = \mathbb{Z}_{24}\), \(I = \{\overline{0},\overline{8},\overline{16}\}\). \\
\textbf{C3.} Tómendegi sáwlelendiriwdi gomomorfizm shártlerine tekseriń. \(f(a + ib) = \begin{pmatrix}
a & b \\
 - b & a
\end{pmatrix}\) \\

\end{tabular}
\vspace{1cm}


\begin{tabular}{m{17cm}}
\textbf{71-variant}
\newline

\textbf{T1.} Keltirilmeytuǵın kópaǵzalılar (anıqlaması, qásiyetleri, mısallar) \\
\textbf{T2.} Bólshekler maydanı (anıqlaması, qásiyetleri, mısallar) \\
\textbf{A1.} \(\mathbf{Q}\)da \(\sqrt{6 + 3\sqrt{2}}\)tiń minimal kópaǵzalısın tabıń. \\
\textbf{A2.} Ámellerdi orınlań: \(\mathbb{Z}_{5}\) te \(\left( 3x^{2} + 3x - 4 \right)\left( x^{2} + 2 \right)\) \\
\textbf{A3.} Tómendegi sannıń \(p\)-adikalıq normasın tabıń. \(|15|_{3} =\) \\
\textbf{B1.} Tómendegi kóplik kolco dúzedi ma? \(\mathbb{Q}\left\lfloor i \right\rfloor = \left\{ a + bi\ \ :\ \ a,b\mathbb{\in Q} \right\}\) \\
\textbf{B2.} Tómendegi kóplik\(M_{2 \times 2}\left( \mathbb{R} \right)\)kolcosınıń úles kolcosı ekenliginkórsetiń.
\[A = \left\{ \left. \ \begin{pmatrix}
a & b \\
 - b & a
\end{pmatrix} \right|a,b\mathbb{\in R} \right\}\] \\
\textbf{B3.} Tómendegi kópaǵzalı \(\mathbb{Z}_{3}\) da keltirilmeytuǵın kópaǵzalı boladıma? \(x^{3} + 2x + 2\) \\
\textbf{C1.} Tómendegi kóplikti maydan shártlerine tekseriń. \(R = M_{2 \times 2}\left( \mathbf{Q} \right)\) \\
\textbf{C2.} Tómendegi kolconıń barlıq ideallarıń tabıń. Bul ideallardan qaysı-biri maksimal boladı? \(\mathbb{Z}_{25}\) \\
\textbf{C3.} Tómendegi sáwlelendiriwdi gomomorfizm shártlerine tekseriń. \(f(x) = \sqrt[3]{x}\) \\

\end{tabular}
\vspace{1cm}


\begin{tabular}{m{17cm}}
\textbf{72-variant}
\newline

\textbf{T1.} Teńlemelerdiń radikallarda sheshiliwi (anıqlaması, qásiyetleri, mısallar) \\
\textbf{T2.} Pútinlik oblastı hám maydan (anıqlaması, qásiyetleri, mısallar) \\
\textbf{A1.} \(\mathbf{Q}\)da \(\sqrt{2} + \sqrt{3}i\)tiń minimal kópaǵzalısın tabıń. \\
\textbf{A2.} Ámellerdi orınlań: \(\mathbb{Z}_{10}\) da \(\left( 7x^{3} + 3x^{2} - x \right) + \left( 6x^{2} - 8x + 4 \right)\) \\
\textbf{A3.} Tómendegi sannıń \(p\)-adikalıq normasın tabıń. \(|\frac{25}{36}|_{2} =\) \\
\textbf{B1.} Tómendegi kóplik kolco dúzedi ma? \(\mathbf{Q} = \left\{ \frac{m}{n},\ \ m \in Z,\ n \in N \right\}\) \\
\textbf{B2.} Tómendegi kóplik\(M_{2 \times 2}\left( \mathbb{R} \right)\)kolcosınıń úles kolcosı ekenliginkórsetiń. \(A = \left\{ \left. \ \begin{pmatrix}
a & 0 \\
0 & 0
\end{pmatrix} \right|a \in Z \right\}\) \\
\textbf{B3.} Tómendegi kópaǵzalı \(\mathbb{Q\lbrack}x\rbrack\) da keltirilmeytuǵın kópaǵzalı boladıma? \(x^{4} - 5x^{3} + 3x - 2\) \\
\textbf{C1.} Tómendegi kóplikti maydan shártlerine tekseriń. \(\mathbb{Z}\left\lbrack \sqrt{n} \right\rbrack = \left\{ x + y\sqrt{n}\ \ \left| \right.\ x,y \in \mathbb{Z} \right\}\) \\
\textbf{C2.} \(\mathbb{Z}_{24}\)tiń barlıq idealların tabıń \\
\textbf{C3.} Tómendegi sáwlelendiriwdi gomomorfizm shártlerine tekseriń.
\[f:\begin{pmatrix}
a & b \\
 - b & a
\end{pmatrix} \rightarrow a + bi\] \\

\end{tabular}
\vspace{1cm}


\begin{tabular}{m{17cm}}
\textbf{73-variant}
\newline

\textbf{T1.} p-Adikalıq norma, p-Adikalıq norma (anıqlaması, qásiyetleri, mısallar) \\
\textbf{T2.} Fundamental teoremalar (anıqlaması, qásiyetleri, mısallar) \\
\textbf{A1.} \(\mathbf{Q}\)da \(\sqrt{3} + \sqrt{2}i\) tiń minimal kópaǵzalısın tabıń. \\
\textbf{A2.} Ámellerdi orınlań: \(\mathbb{Z}_{12}\) de \(\left( 5x^{2} + 3x - 2 \right)^{2}\) \\
\textbf{A3.} Tómendegi sannıń \(p\)-adikalıq normasın tabıń. \(|\frac{8}{18}|_{5} =\) \\
\textbf{B1.} Tómendegi kóplik kolco dúzedi ma? \(\mathbb{Q}\left( \sqrt{2} \right) = \left\{ a + b\sqrt{2}:a,b \in \mathbb{Q} \right\}\) \\
\textbf{B2.} Tómendegi kóplik\(M_{2 \times 2}\left( \mathbb{R} \right)\) niń úles maydanı ekenliginkórsetiń. \(A = \left\{ \left. \ \begin{pmatrix}
a & 0 \\
0 & a
\end{pmatrix} \right|a\mathbb{\in R},a \neq 0 \right\}\) \\
\textbf{B3.} Tómendegi kópaǵzalı \(\mathbb{Q\lbrack}x\rbrack\) da keltirilmeytuǵın kópaǵzalı boladıma? \(3x^{5} - 4x^{3} - 6x^{2} + 6\) \\
\textbf{C1.} Tómendegi kóplikti maydan shártlerine tekseriń. \(\mathbb{Z}\left( \sqrt{3} \right) = \left\{ a + b\sqrt{3}:a,b \in \mathbb{Z} \right\}\) \\
\textbf{C2.} \(M_{2 \times 2}\left( \mathbf{Z} \right)\)kolcoda\(I = \left\{ \begin{bmatrix}
a & 0 \\
0 & 0
\end{bmatrix}|a\mathbb{\in Z} \right\}\) ideal boladı ma? \\
\textbf{C3.} Tómendegi sáwlelendiriwdi gomomorfizm shártlerine tekseriń.
\[f:x \rightarrow x^{p}\] \\

\end{tabular}
\vspace{1cm}


\begin{tabular}{m{17cm}}
\textbf{74-variant}
\newline

\textbf{T1.} Maydanlar keńeytpesi (anıqlaması, qásiyetleri, mısallar) \\
\textbf{T2.} Kolco gomomorfizmleri hám ideallar (anıqlaması, qásiyetleri, mısallar) \\
\textbf{A1.} \(\mathbf{Q}\)da \(\sqrt{1/3 + \sqrt{7}}\)tiń minimal kópaǵzalısın tabıń. \\
\textbf{A2.} Ámellerdi orınlań: \(\mathbb{Z}_{12}\) de \((3x - 2)^{3}\) \\
\textbf{A3.} Tómendegi sannıń \(p\)-adikalıq normasın tabıń. \(|\frac{1}{4}|_{2} =\) \\
\textbf{B1.} Tómendegi kóplik kolco dúzedi ma? \(R = M_{2 \times 2}\left( \mathbf{Q} \right)\) \\
\textbf{B2.} Tómendegi kóplik\(M_{2 \times 2}\left( \mathbb{R} \right)\)kolcosınıń úles kolcosı ekenliginkórsetiń. \(A = \left\{ \left. \ \begin{pmatrix}
a & b \\
0 & a
\end{pmatrix} \right|a,b\mathbb{\in R} \right\}\) \\
\textbf{B3.} Tómendegi maydanlardıń berilgen kópaǵzalı arqalı ajıralıw maydanın tabıń. \(\mathbb{Q}\) da \(x^{4} + 1\). \\
\textbf{C1.} Tómendegi kóplikti maydan shártlerine tekseriń. \(\mathbb{Q}\left\lfloor i \right\rfloor = \left\{ a + bi\ \ :\ \ a,b\mathbb{\in Q} \right\}\) \\
\textbf{C2.} Tómendegi kolconıń úles kóplikleri ideal bolıwın kórsetiń:
\(R = \left\{ \begin{pmatrix}
a & b \\
0 & c
\end{pmatrix}\ |\ a,b,c \in \mathbb{Z} \right\}\), \(I = \left\{ \begin{pmatrix}
0 & a \\
0 & 0
\end{pmatrix}\ |\ a \in \mathbb{Z} \right\}\) \\
\textbf{C3.} Tómendegi sáwlelendiriwdi gomomorfizm shártlerine tekseriń. \(f(a + ib) = \begin{pmatrix}
a & b \\
 - b & a
\end{pmatrix}\) \\

\end{tabular}
\vspace{1cm}


\begin{tabular}{m{17cm}}
\textbf{75-variant}
\newline

\textbf{T1.} p-Adikalıq sanlardıń keńisligi (anıqlaması, qásiyetleri, mısallar) \\
\textbf{T2.} Maydanlar avtomorfizmleri (anıqlaması, qásiyetleri, mısallar) \\
\textbf{A1.} \(\mathbf{Q}\)da \(\sqrt{2 + \sqrt{2}}\)tiń minimal kópaǵzalısın tabıń. \\
\textbf{A2.} Tómendegi kópaǵzalınıń barlıq nollerin tabıń: \(\mathbb{Z}_{2}\) de \(x^{3} + x + 1\) \\
\textbf{A3.} Tómendegi sannıń \(p\)-adikalıq normasın tabıń. \(|6|_{3} =\) \\
\textbf{B1.} Tómendegi kóplik kolco dúzedi ma? \(R = \left\{ a + b\sqrt{2}\ :\ \ \ \ a,b \in \mathbf{Z} \right\}\) \\
\textbf{B2.} Tómendegi kóplik\(M_{2 \times 2}\left( \mathbb{R} \right)\)kolcosınıń úles kolcosı ekenliginkórsetiń.
\begin{quote}
\[T = \left\{ \begin{pmatrix}
a + b & b \\
 - b & a
\end{pmatrix}\left| \ \ a,b\mathbb{\in Z} \right.\  \right\}\]
\end{quote} \\
\textbf{B3.} \(x^{2} - 7\) kópaǵzalı\(\mathbb{Q}(\sqrt{3})\)de keltirilmeytuǵın kópaǵzalı ekenligin kórsetiń. \\
\textbf{C1.} \(Z_{18}\) kolconıń barlıq úles kolcoların anıqlań. \\
\textbf{C2.} Tómendegi kolconıń barlıq ideallarıń tabıń. Bul ideallardan qaysı-biri maksimal boladı? \(\mathbb{M}_{2}\left( \mathbb{Z} \right)\), elementleri \(\mathbb{Z}\) bolǵan \(2 \times 2\) matrica \\
\textbf{C3.} Tómendegi sáwlelendiriwdi gomomorfizm shártlerine tekseriń. \(f(x) = 5^{x}\) \\

\end{tabular}
\vspace{1cm}


\begin{tabular}{m{17cm}}
\textbf{76-variant}
\newline

\textbf{T1.} Bólshekler maydanı (anıqlaması, qásiyetleri, mısallar) \\
\textbf{T2.} Shekli maydannıń strukturası (anıqlaması, qásiyetleri, mısallar) \\
\textbf{A1.} \(\mathbf{Q}\)da \(\sqrt{\frac{1}{4} + \sqrt{5}}\) tiń minimal kópaǵzalısın tabıń. \\
\textbf{A2.} Ámellerdi orınlań: \(\mathbb{Z}_{5}\) te \(\left( 3x^{2} + 3x - 4 \right)\left( 4x^{2} + 2 \right)\) \\
\textbf{A3.} Tómendegi sannıń \(p\)-adikalıq normasın tabıń. \(|124|_{2} =\) \\
\textbf{B1.} Tómendegi kóplik kolco dúzedi ma? \(R = \left\{ \begin{pmatrix}
a & b \\
2b & a
\end{pmatrix}\ \ :\ \ a,b \in \mathbf{Q} \right\}\) \\
\textbf{B2.} Tómendegi kóplik\(M_{2 \times 2}\left( \mathbb{R} \right)\)kolcosınıń úles kolcosı ekenliginkórsetiń.
\[A = \left\{ \left. \ \begin{pmatrix}
a & b \\
 - b & a
\end{pmatrix} \right|a,b\mathbb{\in R} \right\}\] \\
\textbf{B3.} Tómendegi maydanlardıń berilgen kópaǵzalı arqalı ajıralıw maydanın tabıń. \(\mathbb{Q}\) da \(x^{4} - 10x^{2} + 21\). \\
\textbf{C1.} Tómendegi kóplikti maydan shártlerine tekseriń. \(7\mathbb{Z}\) \\
\textbf{C2.} \(M_{2 \times 2}\left( \mathbf{Z} \right)\)kolcoda\(I = \left\{ \begin{bmatrix}
a & 0 \\
b & 0
\end{bmatrix}|a,b\mathbb{\in Z} \right\}\) ideal boladı ma? \\
\textbf{C3.} Tómendegi sáwlelendiriwdi gomomorfizm shártlerine tekseriń. \(f\left( a + \sqrt{2}b \right) = a + bi\) \\

\end{tabular}
\vspace{1cm}


\begin{tabular}{m{17cm}}
\textbf{77-variant}
\newline

\textbf{T1.} p-Adikalıq sanlardıń keńisligi (anıqlaması, qásiyetleri, mısallar) \\
\textbf{T2.} Keltirilmeytuǵın kópaǵzalılar (anıqlaması, qásiyetleri, mısallar) \\
\textbf{A1.} \(\mathbf{Q}\)da \(\sqrt{2 + 2\sqrt{2}}\)tiń minimal kópaǵzalısın tabıń. \\
\textbf{A2.} Ámellerdi orınlań: \(\mathbb{Z}_{10}\) da \(\left( 7x^{3} + 3x^{2} - x \right) + \left( 6x^{2} - 8x + 4 \right)\) \\
\textbf{A3.} Tómendegi sannıń \(p\)-adikalıq normasın tabıń. \(|35|_{7} =\) \\
\textbf{B1.} Tómendegi kóplik kolco dúzedi ma? \(\left\{ a + b\sqrt{7}|a,b \in R \right\}\) \\
\textbf{B2.} Tómendegi kóplik\(M_{2 \times 2}\left( \mathbb{R} \right)\) niń úles maydanı ekenligin kórsetiń. \(A = \left\{ \left. \ \begin{pmatrix}
a & 0 \\
2b & a
\end{pmatrix} \right|a,b\mathbb{\in R},a \neq 0 \right\}\) \\
\textbf{B3.} Tómendegi maydanlardıń berilgen kópaǵzalı arqalı ajıralıw maydanın tabıń.
\(\mathbb{Q}\) da \(x^{4} - 5x^{2} + 21\). \\
\textbf{C1.} Tómendegi kóplikti maydan shártlerine tekseriń. \(R = \left\{ \begin{pmatrix}
a & b \\
2b & a
\end{pmatrix}\ \ :\ \ a,b \in \mathbf{Q} \right\}\) \\
\textbf{C2.} \(Z_{12}\) niń barlıq idealların tabıń. \\
\textbf{C3.} Tómendegi sáwlelendiriwdi gomomorfizm shártlerine tekseriń.
\[f\left( a - \sqrt{2}b \right) = a + \sqrt{2}b\] \\

\end{tabular}
\vspace{1cm}


\begin{tabular}{m{17cm}}
\textbf{78-variant}
\newline

\textbf{T1.} Maksimal hám ápiwayı ideallar (anıqlaması, qásiyetleri, mısallar) \\
\textbf{T2.} Maydanlar keńeytpesi (anıqlaması, qásiyetleri, mısallar) \\
\textbf{A1.} \(\mathbf{Q}\)da \(\sqrt{2} + \sqrt{3}\)tiń minimal kópaǵzalısın tabıń. \\
\textbf{A2.} Ámellerdi orınlań: \(\mathbb{Z}_{5}\) te \(\left( x^{2} + 3x - 4 \right)\left( x^{2} - 3 \right)\) \\
\textbf{A3.} Tómendegi sannıń \(p\)-adikalıq normasın tabıń. \(|48|_{3} =\) \\
\textbf{B1.} Tómendegi kóplik kolco dúzedi ma? \(\mathbb{Q(}i\sqrt{n}) = \{ x + iy\sqrt{n}\ |\ x,y \in \mathbb{Q}\}\) \\
\textbf{B2.} Tómendegi kóplik\(M_{2 \times 2}\left( \mathbb{R} \right)\)kolcosınıń úles kolcosı ekenliginkórsetiń. \(A = \left\{ \left. \ \begin{pmatrix}
a & b \\
7b & a
\end{pmatrix} \right|a,b \in Z \right\}\) \\
\textbf{B3.} \(x^{2} - 3\)kópaǵzalı \(\mathbb{Q}(\sqrt{2})\)de keltirilmeytuǵınkópaǵzalı ekenliginkórsetiń. \\
\textbf{C1.} Tómendegi kóplikti maydan shártlerine tekseriń. \(\mathbb{Q(}i\sqrt{n}) = \{ x + iy\sqrt{n}\ |\ x,y \in \mathbb{Q}\}\) \\
\textbf{C2.} Tómendegi kolconıń úles kóplikleri ideal bolıwın kórsetiń:
\(R = \left\{ \begin{pmatrix}
a & b \\
0 & c
\end{pmatrix}\ |\ a,b,c \in \mathbb{Z} \right\}\), \(I = \left\{ \begin{pmatrix}
0 & b \\
0 & c
\end{pmatrix}\ |\ a \in \mathbb{Z} \right\}\). \\
\textbf{C3.} Tómendegi sáwlelendiriwdi gomomorfizm shártlerine tekseriń. \(f(a) = a^{n}\) \\

\end{tabular}
\vspace{1cm}


\begin{tabular}{m{17cm}}
\textbf{79-variant}
\newline

\textbf{T1.} Teńlemelerdiń radikallarda sheshiliwi (anıqlaması, qásiyetleri, mısallar) \\
\textbf{T2.} p-Adikalıq norma, p-Adikalıq norma (anıqlaması, qásiyetleri, mısallar) \\
\textbf{A1.} \(\mathbf{Q}\)da \(\sqrt{3 - \sqrt{3}}\)tiń minimal kópaǵzalısın tabıń. \\
\textbf{A2.} Ámellerdi orınlań: \(\mathbb{Z}_{12}\) de \(\left( 5x^{2} + 3x - 4 \right)\left( 4x^{2} - x + 9 \right)\) \\
\textbf{A3.} Tómendegi sannıń \(p\)-adikalıq normasın tabıń. \(|\frac{3}{4}|_{2} =\) \\
\textbf{B1.} Tómendegi kóplik kolco dúzedi ma? \(G = \left\{ a^{n},a \neq 0, \pm 1,n \in \mathbb{Z} \right\}\) \\
\textbf{B2.} Tómendegi kóplik\(M_{2 \times 2}\left( \mathbb{R} \right)\) niń úles maydanı ekenliginkórsetiń. \(A = \left\{ \left. \ \begin{pmatrix}
a & b\sqrt{7} \\
 - b\sqrt{7} & a
\end{pmatrix} \right|a,b\mathbb{\in Q},a^{2} + 7b^{2} \neq 0 \right\}\) \\
\textbf{B3.} \(x^{2} - 7\) kópaǵzalı\(\mathbb{Q}(\sqrt{3})\)de keltirilmeytuǵın kópaǵzalı ekenligin kórsetiń. \\
\textbf{C1.} Tómendegi kóplikti maydan shártlerine tekseriń. \(Z_{p}\) \\
\textbf{C2.} Tómendegi kolconıń úles kóplikleri ideal bolıwın kórsetiń:
\(R = \mathbb{Z}_{24}\), \(I = \{\overline{0},\overline{8},\overline{16}\}\). \\
\textbf{C3.} Tómendegi sáwlelendiriwdi gomomorfizm shártlerine tekseriń. \(f(x) = e^{x}\) \\

\end{tabular}
\vspace{1cm}


\begin{tabular}{m{17cm}}
\textbf{80-variant}
\newline

\textbf{T1.} Kolco gomomorfizmleri hám ideallar (anıqlaması, qásiyetleri, mısallar) \\
\textbf{T2.} Pútinlik oblastı hám maydan (anıqlaması, qásiyetleri, mısallar) \\
\textbf{A1.} \(\mathbf{Q}\)da \(\sqrt{3} + \sqrt[3]{5}\)tiń minimal kópaǵzalısın tabıń. \\
\textbf{A2.} Ámellerdi orınlań: \(\mathbb{Z}_{5}\) te \(\left( x^{2} + 3x - 1 \right)^{2}\) \\
\textbf{A3.} Tómendegi sannıń \(p\)-adikalıq normasın tabıń. \(|625|_{5} =\) \\
\textbf{B1.} Tómendegi kóplik kolco dúzedi ma? \(\mathbb{Z}\left( \sqrt{3} \right) = \left\{ a + b\sqrt{3}:a,b \in \mathbb{Z} \right\}\) \\
\textbf{B2.} Tómendegi kóplik\(M_{2 \times 2}\left( \mathbb{R} \right)\) niń úles maydanı ekenliginkórsetiń. \(A = \left\{ \left. \ \begin{pmatrix}
a & b\sqrt{2} \\
b\sqrt{2} & a
\end{pmatrix} \right|a,b\mathbb{\in Q},a^{2} - 2b^{2} \neq 0 \right\}\) \\
\textbf{B3.} Tómendegi kópaǵzalı \(\mathbb{Q\lbrack}x\rbrack\) da keltirilmeytuǵın kópaǵzalı boladıma? \(x^{4} - 5x^{3} + 3x - 2\) \\
\textbf{C1.} Tómendegi kóplikti maydan shártlerine tekseriń. \(G = \left\{ a^{n},a \neq 0, \pm 1,n \in \mathbb{Z} \right\}\) \\
\textbf{C2.} Tómendegi kolconıń úles kóplikleri ideal bolıwın kórsetiń:
\(R\mathbb{= Z\lbrack}\sqrt{7}\rbrack\), \(I = \{ a + b\sqrt{7}\ |\ \ a,b \in \mathbb{Z,}a - b\ \ jupsan\}\). \\
\textbf{C3.} Tómendegi sáwlelendiriwdi gomomorfizm shártlerine tekseriń. \(f(x) = x^{2} + x\) \\

\end{tabular}
\vspace{1cm}


\begin{tabular}{m{17cm}}
\textbf{81-variant}
\newline

\textbf{T1.} Kópagzalılar kolcosı (anıqlaması, qásiyetleri, mısallar) \\
\textbf{T2.} Maydanlar avtomorfizmleri (anıqlaması, qásiyetleri, mısallar) \\
\textbf{A1.} \(\mathbf{Q}\)da \(\sqrt{2} + \sqrt[3]{7}\)tiń minimal kópaǵzalısın tabıń. \\
\textbf{A2.} Ámellerdi orınlań: \(\mathbb{Z}_{12}\) de \(\left( 5x^{2} + 3x - 4 \right) + \left( 4x^{2} - x + 9 \right)\) \\
\textbf{A3.} Tómendegi sannıń \(p\)-adikalıq normasın tabıń. \(|\frac{9}{12}|_{7} =\) \\
\textbf{B1.} Tómendegi kóplik kolco dúzedi ma? \(7\mathbb{Z}\) \\
\textbf{B2.} Tómendegi kóplik\(M_{2 \times 2}\left( \mathbb{R} \right)\) niń úles maydanı ekenliginkórsetiń. \(A = \left\{ \left. \ \begin{pmatrix}
a & b \\
 - b & a
\end{pmatrix} \right|a,b\mathbb{\in Z},a^{2} + b^{2} \neq 0 \right\}\) \\
\textbf{B3.} Tómendegi maydanlardıń berilgen kópaǵzalı arqalı ajıralıw maydanın tabıń. \(\mathbb{Q}\) da \(x^{3} - 3\). \\
\textbf{C1.} Tómendegi kóplikti maydan shártlerine tekseriń. \(\mathbf{Q} = \left\{ \frac{m}{n},\ \ m \in Z,\ n \in N \right\}\) \\
\textbf{C2.} Tómendegi kolconıń barlıq ideallarıń tabıń. Bul ideallardan qaysı-biri maksimal boladı? \(\mathbb{Z}_{27}\) \\
\textbf{C3.} Tómendegi sáwlelendiriwdi gomomorfizm shártlerine tekseriń. \(f\left( \begin{pmatrix}
a & 0 \\
0 & a
\end{pmatrix} \right) = a\) \\

\end{tabular}
\vspace{1cm}


\begin{tabular}{m{17cm}}
\textbf{82-variant}
\newline

\textbf{T1.} Fundamental teoremalar (anıqlaması, qásiyetleri, mısallar) \\
\textbf{T2.} Kolcolar (anıqlaması, qásiyetleri, mısallar) \\
\textbf{A1.} \(\mathbf{Q}\)da \(\sqrt{2} + \sqrt{5}\)tiń minimal kópaǵzalısın tabıń. \\
\textbf{A2.} Tómendegi kópaǵzalınıń barlıq nollerin tabıń: \(\mathbb{Z}_{2}\) de \(x^{3} + x + 1\) \\
\textbf{A3.} Tómendegi sannıń \(p\)-adikalıq normasın tabıń. \(|\frac{7}{15}|_{3} =\) \\
\textbf{B1.} Tómendegi kóplik kolco dúzedi ma? \(R = M_{2 \times 2}\left( \mathbf{Q} \right)\) \\
\textbf{B2.} \(Z_{16}\) kolconıń barlıq úles kolcoların anıqlań. \\
\textbf{B3.} \(x^{4} - 5x^{2} + 6\) kópaǵzalı \(\mathbb{Q}\) de keltirilmeytuǵın kópaǵzalı ekenligin kórsetiń. \\
\textbf{C1.} Tómendegi kóplikti maydan shártlerine tekseriń. \(R = M_{2 \times 2}\left( \mathbf{Z} \right)\) \\
\textbf{C2.} Tómendegi kolconıń barlıq ideallarıń tabıń. Bul ideallardan qaysı-biri maksimal boladı? \(\mathbb{M}_{2}\left( \mathbb{R} \right)\), elementleri \(\mathbb{R}\) bolǵan \(2 \times 2\) matrica \\
\textbf{C3.} Tómendegi sáwlelendiriwdi gomomorfizm shártlerine tekseriń. \(f(x + iy) = x \cdot y\) \\

\end{tabular}
\vspace{1cm}


\begin{tabular}{m{17cm}}
\textbf{83-variant}
\newline

\textbf{T1.} Ajıralatuǵın maydanlar (anıqlaması, qásiyetleri, mısallar) \\
\textbf{T2.} Pútinlik oblastı hám maydan (anıqlaması, qásiyetleri, mısallar) \\
\textbf{A1.} \(\mathbf{Q}\)da \(\sqrt{2 + \sqrt{5}}\)tiń minimal kópaǵzalısın tabıń. \\
\textbf{A2.} Ámellerdi orınlań: \(\mathbb{Z}_{10}\) da \(\left( 5x^{2} + 3x - 4 \right)\left( 4x^{2} - x + 9 \right)\) \\
\textbf{A3.} Tómendegi sannıń \(p\)-adikalıq normasın tabıń. \(|729|_{3} =\) \\
\textbf{B1.} Tómendegi kóplik kolco dúzedi ma? \(\mathbf{Q} = \left\{ \frac{m}{n},\ \ m \in Z,\ n \in N \right\}\) \\
\textbf{B2.} Tómendegi kóplik\(M_{2 \times 2}\left( \mathbb{R} \right)\)kolcosınıń úles kolcosı ekenliginkórsetiń. \(A = \left\{ \left. \ \begin{pmatrix}
a & b \\
0 & a
\end{pmatrix} \right|a,b\mathbb{\in R} \right\}\) \\
\textbf{B3.} \(x^{2} + x + 1\) kópaǵzalı \(\mathbb{Z}_{5}\) de keltirilmeytuǵın kópaǵzalı ekenligin kórsetiń. \\
\textbf{C1.} Tómendegi kóplikti maydan shártlerine tekseriń. \(G = \left\{ a^{n},a \neq 0, \pm 1,n \in \mathbb{Z} \right\}\) \\
\textbf{C2.} Tómendegi kolconıń úles kóplikleri ideal bolıwın kórsetiń:
\(R = \mathbb{Z}_{28}\), \(I = \left\{ \overline{0},\overline{7},\overline{14},\overline{21} \right\}\) \\
\textbf{C3.} Tómendegi sáwlelendiriwdi gomomorfizm shártlerine tekseriń. \(f:\begin{pmatrix}
a & b \\
0 & c
\end{pmatrix} \rightarrow a\) \\

\end{tabular}
\vspace{1cm}


\begin{tabular}{m{17cm}}
\textbf{84-variant}
\newline

\textbf{T1.} Shekli maydannıń strukturası (anıqlaması, qásiyetleri, mısallar) \\
\textbf{T2.} Keltirilmeytuǵın kópaǵzalılar (anıqlaması, qásiyetleri, mısallar) \\
\textbf{A1.} \(\mathbf{Q}\)da \(\sqrt{\sqrt[3]{2} - i}\)tiń minimal kópaǵzalısın tabıń. \\
\textbf{A2.} Ámellerdi orınlań: \(\mathbb{Z}_{5}\) te \(\left( 3x^{2} + 2x - 4 \right) + \left( 4x^{2} + 2 \right)\) \\
\textbf{A3.} Tómendegi sannıń \(p\)-adikalıq normasın tabıń. \(|256|_{2} =\) \\
\textbf{B1.} Tómendegi kóplik kolco dúzedi ma? \(\mathbb{Z}\left\lbrack \sqrt{n} \right\rbrack = \left\{ x + y\sqrt{n}\ \ \left| \right.\ x,y \in \mathbb{Z} \right\}\) \\
\textbf{B2.} Tómendegi kóplik\(M_{2 \times 2}\left( \mathbb{R} \right)\)kolcosınıń úles kolcosı ekenliginkórsetiń.
\begin{quote}
\[T = \left\{ \begin{pmatrix}
a + b & b \\
 - b & a
\end{pmatrix}\left| \ \ a,b\mathbb{\in Z} \right.\  \right\}\]
\end{quote} \\
\textbf{B3.} Tómendegi kópaǵzalı \(\mathbb{Q\lbrack}x\rbrack\) da keltirilmeytuǵın kópaǵzalı boladıma? \(3x^{5} - 4x^{3} - 6x^{2} + 6\) \\
\textbf{C1.} Tómendegi kóplikti maydan shártlerine tekseriń. \(\left\{ a + b\sqrt{7}|a,b \in R \right\}\) \\
\textbf{C2.} Tómendegi kolconıń úles kóplikleri ideal bolıwın kórsetiń:
\(R = \mathbb{Z}_{28}\), \(I = \{\overline{0},\overline{7},\overline{14},\overline{21}\}\). \\
\textbf{C3.} Tómendegi sáwlelendiriwdi gomomorfizm shártlerine tekseriń. \(f(x) = \sqrt{x}\) \\

\end{tabular}
\vspace{1cm}


\begin{tabular}{m{17cm}}
\textbf{85-variant}
\newline

\textbf{T1.} Maydanlar avtomorfizmleri (anıqlaması, qásiyetleri, mısallar) \\
\textbf{T2.} Kolco gomomorfizmleri hám ideallar (anıqlaması, qásiyetleri, mısallar) \\
\textbf{A1.} \(\mathbf{Q}\)da \(\sqrt{\sqrt[3]{2} - i}\)tiń minimal kópaǵzalısın tabıń. \\
\textbf{A2.} Ámellerdi orınlań: \(\mathbb{Z}_{9}\) da \(\left( 7x^{3} + 3x^{2} - x \right) + \left( 6x^{2} - 8x + 4 \right)\) \\
\textbf{A3.} Tómendegi sannıń \(p\)-adikalıq normasın tabıń. \(|124|_{2} =\) \\
\textbf{B1.} Tómendegi kóplik kolco dúzedi ma? \(R = \left\{ a + b\sqrt{2}\ :\ \ \ \ a,b \in \mathbf{Z} \right\}\) \\
\textbf{B2.} Tómendegi kóplik\(M_{2 \times 2}\left( \mathbb{R} \right)\) niń úles maydanı ekenliginkórsetiń.
\[A = \left\{ \left. \ \begin{pmatrix}
a & b \\
0 & a
\end{pmatrix} \right|a,b\mathbb{\in Z},a \neq 0 \right\}\] \\
\textbf{B3.} Tómendegi kópaǵzalı\(\mathbf{Z}_{2}\lbrack x\rbrack\)da keltirilmeytuǵınkópaǵzalıboladıma? \(x^{3} + x + 1\) \\
\textbf{C1.} Tómendegi kóplikti maydan shártlerine tekseriń. \(G = \left\{ a^{n},a \neq 0, \pm 1,n \in \mathbb{Z} \right\}\) \\
\textbf{C2.} Tómendegi kolconıń barlıq ideallarıń tabıń. Bul ideallardan qaysı-biri maksimal boladı? \(\mathbb{M}_{2}\left( \mathbb{R} \right)\), elementleri \(\mathbb{R}\) bolǵan \(2 \times 2\) matrica \\
\textbf{C3.} Tómendegi sáwlelendiriwdi gomomorfizm shártlerine tekseriń.
\[f:x \rightarrow x^{p}\] \\

\end{tabular}
\vspace{1cm}


\begin{tabular}{m{17cm}}
\textbf{86-variant}
\newline

\textbf{T1.} Maksimal hám ápiwayı ideallar (anıqlaması, qásiyetleri, mısallar) \\
\textbf{T2.} Kópagzalılar kolcosı (anıqlaması, qásiyetleri, mısallar) \\
\textbf{A1.} \(\mathbf{Q}\)da \(\sqrt{3 - \sqrt{3}}\)tiń minimal kópaǵzalısın tabıń. \\
\textbf{A2.} Ámellerdi orınlań: \(\mathbb{Z}_{12}\) de \((3x - 2)^{3}\) \\
\textbf{A3.} Tómendegi sannıń \(p\)-adikalıq normasın tabıń. \(|\frac{3}{4}|_{2} =\) \\
\textbf{B1.} Tómendegi kóplik kolco dúzedi ma? \(\left\{ a + b\sqrt{7}|a,b \in R \right\}\) \\
\textbf{B2.} Tómendegi kóplik\(M_{2 \times 2}\left( \mathbb{R} \right)\)kolcosınıń úles kolcosı ekenliginkórsetiń. \(A = \left\{ \left. \ \begin{pmatrix}
a & 0 \\
0 & 0
\end{pmatrix} \right|a \in Z \right\}\) \\
\textbf{B3.} \(x^{2} + x + 1\) kópaǵzalı \(\mathbb{Z}_{3}\) de keltirilmeytuǵın kópaǵzalı ekenligin kórsetiń. \\
\textbf{C1.} Tómendegi kóplikti maydan shártlerine tekseriń. \(R = M_{2 \times 2}\left( \mathbf{Z} \right)\) \\
\textbf{C2.} \(M_{2 \times 2}\left( \mathbf{Z} \right)\)kolcoda\(I = \left\{ \begin{bmatrix}
a & 0 \\
0 & 0
\end{bmatrix}|a\mathbb{\in Z} \right\}\) ideal boladı ma? \\
\textbf{C3.} Tómendegi sáwlelendiriwdi gomomorfizm shártlerine tekseriń. \(f(a + ib) = \begin{pmatrix}
a & b \\
 - b & a
\end{pmatrix}\) \\

\end{tabular}
\vspace{1cm}


\begin{tabular}{m{17cm}}
\textbf{87-variant}
\newline

\textbf{T1.} Fundamental teoremalar (anıqlaması, qásiyetleri, mısallar) \\
\textbf{T2.} Ajıralatuǵın maydanlar (anıqlaması, qásiyetleri, mısallar) \\
\textbf{A1.} \(\mathbf{Q}\)da \(\sqrt{2} + \sqrt{3}\)tiń minimal kópaǵzalısın tabıń. \\
\textbf{A2.} Ámellerdi orınlań: \(\mathbb{Z}_{12}\) de \(\left( 5x^{2} + 3x - 2 \right)^{2}\) \\
\textbf{A3.} Tómendegi sannıń \(p\)-adikalıq normasın tabıń. \(|\frac{1}{4}|_{2} =\) \\
\textbf{B1.} Tómendegi kóplik kolco dúzedi ma? \(\mathbb{Q(}i\sqrt{n}) = \{ x + iy\sqrt{n}\ |\ x,y \in \mathbb{Q}\}\) \\
\textbf{B2.} \(Z_{20}\) kolconıń barlıq úles kolcoların anıqlań. \\
\textbf{B3.} Tómendegi maydanlardıń berilgen kópaǵzalı arqalı ajıralıw maydanın tabıń. \(\mathbb{Q}\) da \(x^{4} - 2\) \\
\textbf{C1.} Tómendegi kóplikti maydan shártlerine tekseriń. \(\mathbb{Z}\left\lbrack \sqrt{n} \right\rbrack = \left\{ x + y\sqrt{n}\ \ \left| \right.\ x,y \in \mathbb{Z} \right\}\) \\
\textbf{C2.} \(M_{2 \times 2}\left( \mathbf{Z} \right)\)kolcoda\(I = \left\{ \begin{bmatrix}
a & 0 \\
b & 0
\end{bmatrix}|a,b\mathbb{\in Z} \right\}\) ideal boladı ma? \\
\textbf{C3.} Tómendegi sáwlelendiriwdi gomomorfizm shártlerine tekseriń. \(f\left( \begin{pmatrix}
a & 0 \\
0 & a
\end{pmatrix} \right) = a\) \\

\end{tabular}
\vspace{1cm}


\begin{tabular}{m{17cm}}
\textbf{88-variant}
\newline

\textbf{T1.} Kolcolar (anıqlaması, qásiyetleri, mısallar) \\
\textbf{T2.} Bólshekler maydanı (anıqlaması, qásiyetleri, mısallar) \\
\textbf{A1.} \(\mathbf{Q}\)da \(\sqrt{2 + 2\sqrt{2}}\)tiń minimal kópaǵzalısın tabıń. \\
\textbf{A2.} Tómendegi kópaǵzalınıń barlıq nollerin tabıń: \(\mathbb{Z}_{12}\) de \(5x^{3} + 4x^{2} - x + 9\) \\
\textbf{A3.} Tómendegi sannıń \(p\)-adikalıq normasın tabıń. \(|48|_{3} =\) \\
\textbf{B1.} Tómendegi kóplik kolco dúzedi ma? \(\mathbb{Q}\left( \sqrt{2} \right) = \left\{ a + b\sqrt{2}:a,b \in \mathbb{Q} \right\}\) \\
\textbf{B2.} \(Z_{12}\) kolconıń barlıq úles kolcoların anıqlań. \\
\textbf{B3.} \(x^{2} + 1\) kópaǵzalı \(\mathbb{Z}_{3}\)de keltirilmeytuǵın kópaǵzalı ekenligin kórsetiń. \\
\textbf{C1.} Tómendegi kóplikti maydan shártlerine tekseriń. \(Z_{p}\) \\
\textbf{C2.} Tómendegi kolconıń úles kóplikleri ideal bolıwın kórsetiń:
\(R = \left\{ \begin{pmatrix}
a & b \\
0 & c
\end{pmatrix}\ |\ a,b,c \in \mathbb{Z} \right\}\), \(I = \left\{ \begin{pmatrix}
0 & a \\
0 & 0
\end{pmatrix}\ |\ a \in \mathbb{Z} \right\}\) \\
\textbf{C3.} Tómendegi sáwlelendiriwdi gomomorfizm shártlerine tekseriń. \(f(x + iy) = x \cdot y\) \\

\end{tabular}
\vspace{1cm}


\begin{tabular}{m{17cm}}
\textbf{89-variant}
\newline

\textbf{T1.} p-Adikalıq sanlardıń keńisligi (anıqlaması, qásiyetleri, mısallar) \\
\textbf{T2.} Teńlemelerdiń radikallarda sheshiliwi (anıqlaması, qásiyetleri, mısallar) \\
\textbf{A1.} \(\mathbf{Q}\)da \(\sqrt{1/3 + \sqrt{7}}\)tiń minimal kópaǵzalısın tabıń. \\
\textbf{A2.} Ámellerdi orınlań: \(\mathbb{Z}_{5}\) te \(\left( 3x^{2} + 3x - 4 \right)\left( x^{2} + 2 \right)\) \\
\textbf{A3.} Tómendegi sannıń \(p\)-adikalıq normasın tabıń. \(|35|_{7} =\) \\
\textbf{B1.} Tómendegi kóplik kolco dúzedi ma? \(\mathbb{Q}\left\lfloor i \right\rfloor = \left\{ a + bi\ \ :\ \ a,b\mathbb{\in Q} \right\}\) \\
\textbf{B2.} Tómendegi kóplik\(M_{2 \times 2}\left( \mathbb{R} \right)\)kolcosınıń úles kolcosı ekenliginkórsetiń. \(A = \left\{ \left. \ \begin{pmatrix}
a & b \\
0 & a
\end{pmatrix} \right|a,b\mathbb{\in Q} \right\}\) \\
\textbf{B3.} Tómendegi maydanlardıń berilgen kópaǵzalı arqalı ajıralıw maydanın tabıń. \(\mathbb{Q}\) da \(x^{4} + 1\). \\
\textbf{C1.} Tómendegi kóplikti maydan shártlerine tekseriń. \(R = M_{2 \times 2}\left( \mathbf{Q} \right)\) \\
\textbf{C2.} Tómendegi kolconıń úles kóplikleri ideal bolıwın kórsetiń:
\(R = \mathbb{Z}_{28}\), \(I = \left\{ \overline{0},\overline{7},\overline{14},\overline{21} \right\}\) \\
\textbf{C3.} Tómendegi sáwlelendiriwdi gomomorfizm shártlerine tekseriń. \(f\left( a + \sqrt{2}b \right) = a + bi\) \\

\end{tabular}
\vspace{1cm}


\begin{tabular}{m{17cm}}
\textbf{90-variant}
\newline

\textbf{T1.} Maydanlar keńeytpesi (anıqlaması, qásiyetleri, mısallar) \\
\textbf{T2.} p-Adikalıq norma, p-Adikalıq norma (anıqlaması, qásiyetleri, mısallar) \\
\textbf{A1.} \(\mathbf{Q}\)da \(\sqrt{2} + \sqrt{5}\)tiń minimal kópaǵzalısın tabıń. \\
\textbf{A2.} Tómendegi kópaǵzalınıń barlıq nollerin tabıń: \(\mathbb{Z}_{5}\) de \(3x^{3} - 4x^{2} - x + 4\) \\
\textbf{A3.} Tómendegi sannıń \(p\)-adikalıq normasın tabıń. \(|\frac{8}{18}|_{5} =\) \\
\textbf{B1.} Tómendegi kóplik kolco dúzedi ma? \(R = \left\{ \begin{pmatrix}
a & b \\
2b & a
\end{pmatrix}\ \ :\ \ a,b \in \mathbf{Q} \right\}\) \\
\textbf{B2.} Tómendegi kóplik\(M_{2 \times 2}\left( \mathbb{R} \right)\) niń úles maydanı ekenliginkórsetiń. \(A = \left\{ \left. \ \begin{pmatrix}
a & 0 \\
0 & a
\end{pmatrix} \right|a\mathbb{\in R},a \neq 0 \right\}\) \\
\textbf{B3.} Tómendegi kópaǵzalı \(\mathbb{Z}_{3}\) da keltirilmeytuǵın kópaǵzalı boladıma? \(x^{3} + 2x + 2\) \\
\textbf{C1.} Tómendegi kóplikti maydan shártlerine tekseriń. \(\mathbf{Q} = \left\{ \frac{m}{n},\ \ m \in Z,\ n \in N \right\}\) \\
\textbf{C2.} \(\mathbb{Z}_{24}\)tiń barlıq idealların tabıń \\
\textbf{C3.} Tómendegi sáwlelendiriwdi gomomorfizm shártlerine tekseriń. \(f(x) = x^{2} + x\) \\

\end{tabular}
\vspace{1cm}


\begin{tabular}{m{17cm}}
\textbf{91-variant}
\newline

\textbf{T1.} Pútinlik oblastı hám maydan (anıqlaması, qásiyetleri, mısallar) \\
\textbf{T2.} Kópagzalılar kolcosı (anıqlaması, qásiyetleri, mısallar) \\
\textbf{A1.} \(\mathbf{Q}\)da \(\sqrt{6 + 3\sqrt{2}}\)tiń minimal kópaǵzalısın tabıń. \\
\textbf{A2.} Ámellerdi orınlań: \(\mathbb{Z}_{12}\) de \(\left( 5x^{2} + 3x - 4 \right) + \left( 4x^{2} - x + 9 \right)\) \\
\textbf{A3.} Tómendegi sannıń \(p\)-adikalıq normasın tabıń. \(|6|_{3} =\) \\
\textbf{B1.} Tómendegi kóplik kolco dúzedi ma? \(\mathbb{Q(}\sqrt[3]{2}) = \left\{ a + b\sqrt[3]{2}:a,b \in \mathbb{Q} \right\}\) \\
\textbf{B2.} Tómendegi kóplik\(M_{2 \times 2}\left( \mathbb{R} \right)\)kolcosınıń úles kolcosı ekenliginkórsetiń.
\begin{quote}
\[T = \left\{ \begin{pmatrix}
a + b & b \\
 - b & a
\end{pmatrix}\left| \ \ a,b\mathbb{\in Z} \right.\  \right\}\]
\end{quote} \\
\textbf{B3.} Tómendegi maydanlardıń berilgen kópaǵzalı arqalı ajıralıw maydanın tabıń. \(\mathbb{Q}\) da \(x^{4} + 1\). \\
\textbf{C1.} Tómendegi kóplikti maydan shártlerine tekseriń. \(7\mathbb{Z}\) \\
\textbf{C2.} \(Z_{12}\) niń barlıq idealların tabıń. \\
\textbf{C3.} Tómendegi sáwlelendiriwdi gomomorfizm shártlerine tekseriń. \(f(x) = 5^{x}\) \\

\end{tabular}
\vspace{1cm}


\begin{tabular}{m{17cm}}
\textbf{92-variant}
\newline

\textbf{T1.} Keltirilmeytuǵın kópaǵzalılar (anıqlaması, qásiyetleri, mısallar) \\
\textbf{T2.} Fundamental teoremalar (anıqlaması, qásiyetleri, mısallar) \\
\textbf{A1.} \(\mathbf{Q}\)da \(\sqrt{2} + \sqrt[3]{7}\)tiń minimal kópaǵzalısın tabıń. \\
\textbf{A2.} Ámellerdi orınlań: \(\mathbb{Z}_{9}\) da \(\left( 7x^{3} + 3x^{2} - x \right) + \left( 6x^{2} - 8x + 4 \right)\) \\
\textbf{A3.} Tómendegi sannıń \(p\)-adikalıq normasın tabıń. \(|256|_{2} =\) \\
\textbf{B1.} Tómendegi kóplik kolco dúzedi ma? \(R = \left\{ \begin{pmatrix}
a & b \\
2b & a
\end{pmatrix}\ \ :\ \ a,b \in \mathbf{Q} \right\}\) \\
\textbf{B2.} Tómendegi kóplik\(M_{2 \times 2}\left( \mathbb{R} \right)\) niń úles maydanı ekenliginkórsetiń. \(A = \left\{ \left. \ \begin{pmatrix}
a & b\sqrt{7} \\
 - b\sqrt{7} & a
\end{pmatrix} \right|a,b\mathbb{\in Q},a^{2} + 7b^{2} \neq 0 \right\}\) \\
\textbf{B3.} Tómendegi maydanlardıń berilgen kópaǵzalı arqalı ajıralıw maydanın tabıń. \(\mathbb{Q}\) da \(x^{4} - 2\) \\
\textbf{C1.} Tómendegi kóplikti maydan shártlerine tekseriń. \(G = \left\{ a^{n},a \neq 0, \pm 1,n \in \mathbb{Z} \right\}\) \\
\textbf{C2.} Tómendegi kolconıń barlıq ideallarıń tabıń. Bul ideallardan qaysı-biri maksimal boladı? \(\mathbb{Z}_{27}\) \\
\textbf{C3.} Tómendegi sáwlelendiriwdi gomomorfizm shártlerine tekseriń.
\[f:\begin{pmatrix}
a & b \\
 - b & a
\end{pmatrix} \rightarrow a + bi\] \\

\end{tabular}
\vspace{1cm}


\begin{tabular}{m{17cm}}
\textbf{93-variant}
\newline

\textbf{T1.} p-Adikalıq sanlardıń keńisligi (anıqlaması, qásiyetleri, mısallar) \\
\textbf{T2.} Kolco gomomorfizmleri hám ideallar (anıqlaması, qásiyetleri, mısallar) \\
\textbf{A1.} \(\mathbf{Q}\)da \(\sqrt{2 + \sqrt{5}}\)tiń minimal kópaǵzalısın tabıń. \\
\textbf{A2.} Tómendegi kópaǵzalınıń barlıq nollerin tabıń: \(\mathbb{Z}_{12}\) de \(5x^{3} + 4x^{2} - x + 9\) \\
\textbf{A3.} Tómendegi sannıń \(p\)-adikalıq normasın tabıń. \(|\frac{7}{15}|_{3} =\) \\
\textbf{B1.} Tómendegi kóplik kolco dúzedi ma? \(\mathbf{Q} = \left\{ \frac{m}{n},\ \ m \in Z,\ n \in N \right\}\) \\
\textbf{B2.} \(Z_{20}\) kolconıń barlıq úles kolcoların anıqlań. \\
\textbf{B3.} Tómendegi kópaǵzalı \(\mathbb{Q\lbrack}x\rbrack\) da keltirilmeytuǵın kópaǵzalı boladıma? \(x^{4} - 5x^{3} + 3x - 2\) \\
\textbf{C1.} Tómendegi kóplikti maydan shártlerine tekseriń. \(R = \left\{ \begin{pmatrix}
a & b \\
2b & a
\end{pmatrix}\ \ :\ \ a,b \in \mathbf{Q} \right\}\) \\
\textbf{C2.} Tómendegi kolconıń úles kóplikleri ideal bolıwın kórsetiń:
\(R = \left\{ \begin{pmatrix}
a & b \\
0 & c
\end{pmatrix}\ |\ a,b,c \in \mathbb{Z} \right\}\), \(I = \left\{ \begin{pmatrix}
0 & b \\
0 & c
\end{pmatrix}\ |\ a \in \mathbb{Z} \right\}\). \\
\textbf{C3.} Tómendegi sáwlelendiriwdi gomomorfizm shártlerine tekseriń. \(f(x) = \sqrt[3]{x}\) \\

\end{tabular}
\vspace{1cm}


\begin{tabular}{m{17cm}}
\textbf{94-variant}
\newline

\textbf{T1.} p-Adikalıq norma, p-Adikalıq norma (anıqlaması, qásiyetleri, mısallar) \\
\textbf{T2.} Bólshekler maydanı (anıqlaması, qásiyetleri, mısallar) \\
\textbf{A1.} \(\mathbf{Q}\)da \(\sqrt{3} + \sqrt[3]{5}\)tiń minimal kópaǵzalısın tabıń. \\
\textbf{A2.} Tómendegi kópaǵzalınıń barlıq nollerin tabıń: \(\mathbb{Z}_{5}\) de \(3x^{3} - 4x^{2} - x + 4\) \\
\textbf{A3.} Tómendegi sannıń \(p\)-adikalıq normasın tabıń. \(|625|_{5} =\) \\
\textbf{B1.} Tómendegi kóplik kolco dúzedi ma? \(\mathbb{Q(}\sqrt[3]{2}) = \left\{ a + b\sqrt[3]{2}:a,b \in \mathbb{Q} \right\}\) \\
\textbf{B2.} Tómendegi kóplik\(M_{2 \times 2}\left( \mathbb{R} \right)\)kolcosınıń úles kolcosı ekenliginkórsetiń. \(A = \left\{ \left. \ \begin{pmatrix}
a & b \\
7b & a
\end{pmatrix} \right|a,b \in Z \right\}\) \\
\textbf{B3.} \(x^{2} + 1\) kópaǵzalı \(\mathbb{Z}_{3}\)de keltirilmeytuǵın kópaǵzalı ekenligin kórsetiń. \\
\textbf{C1.} Tómendegi kóplikti maydan shártlerine tekseriń. \(\mathbb{Q}\left\lfloor i \right\rfloor = \left\{ a + bi\ \ :\ \ a,b\mathbb{\in Q} \right\}\) \\
\textbf{C2.} Tómendegi kolconıń úles kóplikleri ideal bolıwın kórsetiń:
\(R\mathbb{= Z\lbrack}\sqrt{7}\rbrack\), \(I = \{ a + b\sqrt{7}\ |\ \ a,b \in \mathbb{Z,}a - b\ \ jupsan\}\). \\
\textbf{C3.} Tómendegi sáwlelendiriwdi gomomorfizm shártlerine tekseriń. \(f(x) = e^{x}\) \\

\end{tabular}
\vspace{1cm}


\begin{tabular}{m{17cm}}
\textbf{95-variant}
\newline

\textbf{T1.} Maydanlar keńeytpesi (anıqlaması, qásiyetleri, mısallar) \\
\textbf{T2.} Teńlemelerdiń radikallarda sheshiliwi (anıqlaması, qásiyetleri, mısallar) \\
\textbf{A1.} \(\mathbf{Q}\)da \(\sqrt{2 + \sqrt{2}}\)tiń minimal kópaǵzalısın tabıń. \\
\textbf{A2.} Ámellerdi orınlań: \(\mathbb{Z}_{5}\) te \(\left( 3x^{2} + 2x - 4 \right) + \left( 4x^{2} + 2 \right)\) \\
\textbf{A3.} Tómendegi sannıń \(p\)-adikalıq normasın tabıń. \(|15|_{3} =\) \\
\textbf{B1.} Tómendegi kóplik kolco dúzedi ma? \(\mathbb{Z}\left( \sqrt{3} \right) = \left\{ a + b\sqrt{3}:a,b \in \mathbb{Z} \right\}\) \\
\textbf{B2.} \(Z_{16}\) kolconıń barlıq úles kolcoların anıqlań. \\
\textbf{B3.} \(x^{2} + x + 1\) kópaǵzalı \(\mathbb{Z}_{3}\) de keltirilmeytuǵın kópaǵzalı ekenligin kórsetiń. \\
\textbf{C1.} Tómendegi kóplikti maydan shártlerine tekseriń. \(\mathbb{Z}\left( \sqrt{3} \right) = \left\{ a + b\sqrt{3}:a,b \in \mathbb{Z} \right\}\) \\
\textbf{C2.} Tómendegi kolconıń úles kóplikleri ideal bolıwın kórsetiń:
\(R = \mathbb{Z}_{28}\), \(I = \{\overline{0},\overline{7},\overline{14},\overline{21}\}\). \\
\textbf{C3.} Tómendegi sáwlelendiriwdi gomomorfizm shártlerine tekseriń. \(f(x) = \sqrt{x}\) \\

\end{tabular}
\vspace{1cm}


\begin{tabular}{m{17cm}}
\textbf{96-variant}
\newline

\textbf{T1.} Maksimal hám ápiwayı ideallar (anıqlaması, qásiyetleri, mısallar) \\
\textbf{T2.} Maydanlar avtomorfizmleri (anıqlaması, qásiyetleri, mısallar) \\
\textbf{A1.} \(\mathbf{Q}\)da \(\sqrt{2} + \sqrt{3}i\)tiń minimal kópaǵzalısın tabıń. \\
\textbf{A2.} Ámellerdi orınlań: \(\mathbb{Z}_{12}\) de \(\left( 5x^{2} + 3x - 2 \right)^{2}\) \\
\textbf{A3.} Tómendegi sannıń \(p\)-adikalıq normasın tabıń. \(|\frac{9}{12}|_{7} =\) \\
\textbf{B1.} Tómendegi kóplik kolco dúzedi ma? \(\mathbb{Z}\left\lbrack \sqrt{n} \right\rbrack = \left\{ x + y\sqrt{n}\ \ \left| \right.\ x,y \in \mathbb{Z} \right\}\) \\
\textbf{B2.} Tómendegi kóplik\(M_{2 \times 2}\left( \mathbb{R} \right)\) niń úles maydanı ekenligin kórsetiń. \(A = \left\{ \left. \ \begin{pmatrix}
a & 0 \\
2b & a
\end{pmatrix} \right|a,b\mathbb{\in R},a \neq 0 \right\}\) \\
\textbf{B3.} Tómendegi kópaǵzalı\(\mathbf{Z}_{2}\lbrack x\rbrack\)da keltirilmeytuǵınkópaǵzalıboladıma? \(x^{3} + x + 1\) \\
\textbf{C1.} Tómendegi kóplikti maydan shártlerine tekseriń. \(\mathbb{Q(}i\sqrt{n}) = \{ x + iy\sqrt{n}\ |\ x,y \in \mathbb{Q}\}\) \\
\textbf{C2.} Tómendegi kolconıń barlıq ideallarıń tabıń. Bul ideallardan qaysı-biri maksimal boladı? \(\mathbb{Z}_{25}\) \\
\textbf{C3.} Tómendegi sáwlelendiriwdi gomomorfizm shártlerine tekseriń. \(f:\begin{pmatrix}
a & b \\
0 & c
\end{pmatrix} \rightarrow a\) \\

\end{tabular}
\vspace{1cm}


\begin{tabular}{m{17cm}}
\textbf{97-variant}
\newline

\textbf{T1.} Shekli maydannıń strukturası (anıqlaması, qásiyetleri, mısallar) \\
\textbf{T2.} Ajıralatuǵın maydanlar (anıqlaması, qásiyetleri, mısallar) \\
\textbf{A1.} \(\mathbf{Q}\)da \(\sqrt{\frac{1}{4} + \sqrt{5}}\) tiń minimal kópaǵzalısın tabıń. \\
\textbf{A2.} Ámellerdi orınlań: \(\mathbb{Z}_{5}\) te \(\left( 3x^{2} + 3x - 4 \right)\left( 4x^{2} + 2 \right)\) \\
\textbf{A3.} Tómendegi sannıń \(p\)-adikalıq normasın tabıń. \(|\frac{25}{36}|_{2} =\) \\
\textbf{B1.} Tómendegi kóplik kolco dúzedi ma? \(R = \left\{ a + b\sqrt{2}\ :\ \ \ \ a,b \in \mathbf{Z} \right\}\) \\
\textbf{B2.} Tómendegi kóplik\(M_{2 \times 2}\left( \mathbb{R} \right)\)kolcosınıń úles kolcosı ekenliginkórsetiń. \(A = \left\{ \left. \ \begin{pmatrix}
a & 0 \\
0 & 0
\end{pmatrix} \right|a \in Z \right\}\) \\
\textbf{B3.} \(x^{2} + x + 1\) kópaǵzalı \(\mathbb{Z}_{5}\) de keltirilmeytuǵın kópaǵzalı ekenligin kórsetiń. \\
\textbf{C1.} \(Z_{18}\) kolconıń barlıq úles kolcoların anıqlań. \\
\textbf{C2.} Tómendegi kolconıń barlıq ideallarıń tabıń. Bul ideallardan qaysı-biri maksimal boladı? \(\mathbb{M}_{2}\left( \mathbb{Z} \right)\), elementleri \(\mathbb{Z}\) bolǵan \(2 \times 2\) matrica \\
\textbf{C3.} Tómendegi sáwlelendiriwdi gomomorfizm shártlerine tekseriń.
\[f\left( a - \sqrt{2}b \right) = a + \sqrt{2}b\] \\

\end{tabular}
\vspace{1cm}


\begin{tabular}{m{17cm}}
\textbf{98-variant}
\newline

\textbf{T1.} Kolcolar (anıqlaması, qásiyetleri, mısallar) \\
\textbf{T2.} Kópagzalılar kolcosı (anıqlaması, qásiyetleri, mısallar) \\
\textbf{A1.} \(\mathbf{Q}\)da \(\sqrt{3} + \sqrt{2}i\) tiń minimal kópaǵzalısın tabıń. \\
\textbf{A2.} Ámellerdi orınlań: \(\mathbb{Z}_{10}\) da \(\left( 7x^{3} + 3x^{2} - x \right) + \left( 6x^{2} - 8x + 4 \right)\) \\
\textbf{A3.} Tómendegi sannıń \(p\)-adikalıq normasın tabıń. \(|729|_{3} =\) \\
\textbf{B1.} Tómendegi kóplik kolco dúzedi ma? \(G = \left\{ a^{n},a \neq 0, \pm 1,n \in \mathbb{Z} \right\}\) \\
\textbf{B2.} Tómendegi kóplik\(M_{2 \times 2}\left( \mathbb{R} \right)\) niń úles maydanı ekenliginkórsetiń. \(A = \left\{ \left. \ \begin{pmatrix}
a & b\sqrt{2} \\
b\sqrt{2} & a
\end{pmatrix} \right|a,b\mathbb{\in Q},a^{2} - 2b^{2} \neq 0 \right\}\) \\
\textbf{B3.} Tómendegi maydanlardıń berilgen kópaǵzalı arqalı ajıralıw maydanın tabıń.
\(\mathbb{Q}\) da \(x^{4} - 5x^{2} + 21\). \\
\textbf{C1.} Tómendegi kóplikti maydan shártlerine tekseriń. \(\left\{ a + b\sqrt{7}|a,b \in R \right\}\) \\
\textbf{C2.} Tómendegi kolconıń úles kóplikleri ideal bolıwın kórsetiń:
\(R = \mathbb{Z}_{24}\), \(I = \{\overline{0},\overline{8},\overline{16}\}\). \\
\textbf{C3.} Tómendegi sáwlelendiriwdi gomomorfizm shártlerine tekseriń. \(f(a) = a^{n}\) \\

\end{tabular}
\vspace{1cm}


\begin{tabular}{m{17cm}}
\textbf{99-variant}
\newline

\textbf{T1.} Keltirilmeytuǵın kópaǵzalılar (anıqlaması, qásiyetleri, mısallar) \\
\textbf{T2.} Ajıralatuǵın maydanlar (anıqlaması, qásiyetleri, mısallar) \\
\textbf{A1.} \(\mathbf{Q}\)da \(\sqrt{2 + \sqrt{2}}\)tiń minimal kópaǵzalısın tabıń. \\
\textbf{A2.} Ámellerdi orınlań: \(\mathbb{Z}_{5}\) te \(\left( x^{2} + 3x - 1 \right)^{2}\) \\
\textbf{A3.} Tómendegi sannıń \(p\)-adikalıq normasın tabıń. \(|729|_{3} =\) \\
\textbf{B1.} Tómendegi kóplik kolco dúzedi ma? \(\mathbb{Q}\left\lfloor i \right\rfloor = \left\{ a + bi\ \ :\ \ a,b\mathbb{\in Q} \right\}\) \\
\textbf{B2.} Tómendegi kóplik\(M_{2 \times 2}\left( \mathbb{R} \right)\) niń úles maydanı ekenliginkórsetiń. \(A = \left\{ \left. \ \begin{pmatrix}
a & b \\
 - b & a
\end{pmatrix} \right|a,b\mathbb{\in Z},a^{2} + b^{2} \neq 0 \right\}\) \\
\textbf{B3.} \(x^{2} - 7\) kópaǵzalı\(\mathbb{Q}(\sqrt{3})\)de keltirilmeytuǵın kópaǵzalı ekenligin kórsetiń. \\
\textbf{C1.} Tómendegi kóplikti maydan shártlerine tekseriń. \(G = \left\{ a^{n},a \neq 0, \pm 1,n \in \mathbb{Z} \right\}\) \\
\textbf{C2.} Tómendegi kolconıń barlıq ideallarıń tabıń. Bul ideallardan qaysı-biri maksimal boladı? \(\mathbb{Z}_{25}\) \\
\textbf{C3.} Tómendegi sáwlelendiriwdi gomomorfizm shártlerine tekseriń.
\[f\left( a - \sqrt{2}b \right) = a + \sqrt{2}b\] \\

\end{tabular}
\vspace{1cm}


\begin{tabular}{m{17cm}}
\textbf{100-variant}
\newline

\textbf{T1.} Maydanlar keńeytpesi (anıqlaması, qásiyetleri, mısallar) \\
\textbf{T2.} Kolco gomomorfizmleri hám ideallar (anıqlaması, qásiyetleri, mısallar) \\
\textbf{A1.} \(\mathbf{Q}\)da \(\sqrt{2 + 2\sqrt{2}}\)tiń minimal kópaǵzalısın tabıń. \\
\textbf{A2.} Tómendegi kópaǵzalınıń barlıq nollerin tabıń: \(\mathbb{Z}_{2}\) de \(x^{3} + x + 1\) \\
\textbf{A3.} Tómendegi sannıń \(p\)-adikalıq normasın tabıń. \(|256|_{2} =\) \\
\textbf{B1.} Tómendegi kóplik kolco dúzedi ma? \(\mathbb{Q(}i\sqrt{n}) = \{ x + iy\sqrt{n}\ |\ x,y \in \mathbb{Q}\}\) \\
\textbf{B2.} \(Z_{12}\) kolconıń barlıq úles kolcoların anıqlań. \\
\textbf{B3.} \(x^{2} - 3\)kópaǵzalı \(\mathbb{Q}(\sqrt{2})\)de keltirilmeytuǵınkópaǵzalı ekenliginkórsetiń. \\
\textbf{C1.} Tómendegi kóplikti maydan shártlerine tekseriń. \(G = \left\{ a^{n},a \neq 0, \pm 1,n \in \mathbb{Z} \right\}\) \\
\textbf{C2.} Tómendegi kolconıń úles kóplikleri ideal bolıwın kórsetiń:
\(R = \mathbb{Z}_{28}\), \(I = \{\overline{0},\overline{7},\overline{14},\overline{21}\}\). \\
\textbf{C3.} Tómendegi sáwlelendiriwdi gomomorfizm shártlerine tekseriń. \(f\left( \begin{pmatrix}
a & 0 \\
0 & a
\end{pmatrix} \right) = a\) \\

\end{tabular}
\vspace{1cm}



\end{document}
