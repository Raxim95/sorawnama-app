\documentclass{article}
\usepackage[fontsize=12pt]{fontsize}
\usepackage[utf8]{inputenc}
% \usepackage[T2A]{fontenc}
% \usepackage{unicode-math}

\usepackage{array}
\usepackage[a4paper,
left=7mm,
right=5mm,
top=7mm,]{geometry}
\usepackage{amsmath}
\usepackage{amsfonts}
\usepackage{amssymb}
\usepackage{setspace}
\usepackage{fontspec}

\setmainfont{Times New Roman}[Weight=900]
% Set the math font and adjust weight to 600 (semi-bold)
\setmainfont{Latin Modern Roman}[Weight=600]



% \renewcommand{\baselinestretch}{1}

\everymath{\displaystyle}
\everydisplay{\displaystyle}
% \linespread{1.25}

\DeclareMathOperator{\sign}{sign}
\DeclareMathOperator{\tg}{tg}
\DeclareMathOperator{\ctg}{ctg}
\DeclareMathOperator{\arctg}{arctg}


\begin{document}

\pagenumbering{gobble}


\begin{tabular}{m{17cm}}
\textbf{1-bilet}

\vspace{0.5cm}

\textbf{T1.} 
Metrik fazolarning uzliksiz akslantirishlari.
 \\
\textbf{T2.} 
Lebeg va Riss teoremalari.
 \\
\textbf{A1.} 
\(A = \{(x,y) \in \mathbb{R}^{2}:\ x = y\},\ B = \{(x,y) \in \mathbb{R}^{2}:\ |x| + |y| \leq 1\}\), \(A,\ B,\ A \cup B,\ A \cap B,\ A \backslash B,\ B \backslash A,\ A \bigtriangleup B\) to'plamlarini aniqlang va tasvirlang.
 \\
\textbf{A2.} 
\((1;\ 7\rbrack\) va \((2;4) \cup \lbrack 9;13\rbrack\) to'plamlari orasida bir qiymatli moslik o'rnating.
 \\
\textbf{A3.} 
\(\lbrack 6,\ 8\rbrack\) kesmada joylashgan sonlarning onlik kasr yozuvida \(8\) raqami qatnashmagan barcha sonlar to'plamining Lebeg o'lchovini toping.
 \\
\textbf{B1.} 
\(\int_{E}^{}f(x)d\mu\) Lebeg integralini hisoblang, \(f(x) = \left\{ \begin{matrix}
\frac{x^{2}}{(x - 5)(x - 6)},\ x \in \mathbb{I} \cap \lbrack 0,\ 4\rbrack \\
3x^{2} - 2,\ x\mathbb{\in Q \cap}\lbrack 0,\ 4\rbrack,\ E = \lbrack 0,\ 4\rbrack
\end{matrix} \right.\ \)
 \\
\textbf{B2.} 
Quyida berilganlar bo'yicha\(\ x,y \in X\) elementlar orasidagi masofani toping: \(X = C\lbrack 0;\ \pi/3\rbrack,\ \rho(x,y) = \max_{0 \leq t \leq \pi/3}|x(t) - y(t)|,x(t) = \sin t,\ y = \cos5t\)
 \\
\textbf{B3.} 
\(A\) va \(B\) to'plamlari orasida o'zaro bir qiymatli moslik o'rnating.\(\ A = \lbrack - 2;4\rbrack\), \(B = ( - 1;9)\).
 \\
\textbf{C1.} 
To'plamning Lebeg o'lchovini toping: \(A = \bigcup_{k = 1}^{\infty}\left( k^{3},k^{3} + 3^{- k} \right)\);
 \\
\textbf{C2.} 
Lebeg integralini (\(\int_{A}^{}{f(x)d\mu}\)) hisoblang: \(f(x) = \frac{( - 1)^{\lbrack x\rbrack}}{\lbrack x\rbrack}\), \(A = \lbrack 1;4)\);
 \\
\textbf{C3.} 
[7;10] to'plamida o'lchovsiz to'plamga misol keltiring.
 \\

\end{tabular}
\vspace{1cm}


\begin{tabular}{m{17cm}}
\textbf{2-bilet}

\vspace{0.5cm}

\textbf{T1.} To'plamlar va ular ustida amallar.
 \\
\textbf{T2.} 
O'lshovli funkciyalar va ularning xossalari.
 \\
\textbf{A1.} 
\(A = \{(x,y) \in \mathbb{R}^{2}:\ max\{|x|,|y|\} \leq 2\},\ B = \{(x,y) \in \mathbb{R}^{2}:\ y \geq x + 1\}\), \(A,\ B,\ A \cup B,\ A \cap B,\ A \backslash B,\ B \backslash A,\ A \bigtriangleup B\) to'plamlarini aniqlang va tasvirlang.
 \\
\textbf{A2.} 
\(\lbrack 0;4)\) va \(\lbrack - 2;0) \cup \lbrack 7;9)\) to'plamlari orasida bir qiymatli moslik o'rnating.
 \\
\textbf{A3.} 
\(\lbrack 3,\ 5\rbrack\) kesmada joylashgan sonlarning onlik kasr yozuvida \(6\) raqami qatnashmagan barcha sonlar to'plamining Lebeg o'lchovini toping.
 \\
\textbf{B1.} 
\(\int_{E}^{}f(x)d\mu\) Lebeg integralini hisoblang, \(f(x) = \left\{ \begin{matrix}
\frac{x^{2}}{(x - 2)(x - 4)},\ x \in \mathbb{I} \cap \lbrack - 1;1\rbrack \\
3x^{2} - 2,\ x\mathbb{\in Q \cap}\lbrack - 1;1\rbrack,\ E = \lbrack - 1;1\rbrack
\end{matrix} \right.\ \)
 \\
\textbf{B2.} 
Quyida berilganlar bo'yicha\(\ x,y \in X\) elementlar orasidagi masofani toping: \(X = C\lbrack 0;\ \pi/4\rbrack,\ \rho(x,y) = \max_{0 \leq t \leq \pi/4}|x(t) - y(t)|,x(t) = \sin t,\ y = \cos3t\)
 \\
\textbf{B3.} 
\(A\) va \(B\) to'plamlari orasida o'zaro bir qiymatli moslik o'rnating.\(\ A = \lbrack - 5;4)\), \(B = \lbrack - 3;11\rbrack\).
 \\
\textbf{C1.} 
To'plamning Lebeg o'lchovini toping: \(A = \bigcup_{k = 1}^{\infty}\left( \frac{1}{2k},\frac{1}{k} \right)\);
 \\
\textbf{C2.} 
Lebeg integralini (\(\int_{A}^{}{f(x)d\mu}\)) hisoblang: \(f(x) = \frac{1}{\lbrack x\rbrack - 1}\), \(A = \lbrack 2;5\rbrack\);
 \\
\textbf{C3.} 
[0;3] to'plamida o'lchovsiz to'plamga misol keltiring.
 \\

\end{tabular}
\vspace{1cm}


\begin{tabular}{m{17cm}}
\textbf{3-bilet}

\vspace{0.5cm}

\textbf{T1.} 
To'plam quvvati va uning xossalari.
 \\
\textbf{T2.} 
Egorov teoremasi.
 \\
\textbf{A1.} 
\(A = \{(x,y) \in \mathbb{R}^{2}:\ x = - y\},\ B = \{(x,y) \in \mathbb{R}^{2}:\ |x| + |y| \leq 2\}\), \(A,\ B,\ A \cup B,\ A \cap B,\ A \backslash B,\ B \backslash A,\ A \bigtriangleup B\) to'plamlarini aniqlang va tasvirlang.
 \\
\textbf{A2.} 
\(\lbrack - 2;\ 2)\) va \(\lbrack 1;3\rbrack \cup (5;7)\) to'plamlari orasida bir qiymatli moslik o'rnating.
 \\
\textbf{A3.} 
\(\lbrack 3,\ 4\rbrack\) kesmada joylashgan sonlarning onlik kasr yozuvida \(1\) raqami qatnashmagan barcha sonlar to'plamining Lebeg o'lchovini toping.
 \\
\textbf{B1.} 
\(\int_{E}^{}f(x)d\mu\) Lebeg integralini hisoblang, \(f(x) = \left\{ \begin{matrix}
\frac{x^{2}}{(x + 2)(x + 4)},\ x \in \mathbb{I} \cap \lbrack 0,\ 4\rbrack \\
3x^{2} - 2,\ x\mathbb{\in Q \cap}\lbrack 0,\ 4\rbrack,\ E = \lbrack 0,\ 4\rbrack
\end{matrix} \right.\ \)
 \\
\textbf{B2.} 
Quyida berilganlar bo'yicha\(\ x,y \in X\) elementlar orasidagi masofani toping: \(X = C\left\lbrack \frac{\pi}{6};\ \frac{\pi}{3} \right\rbrack,\ \rho(x,y) = \max_{\frac{\pi}{6} \leq t \leq \frac{\pi}{3}}|x(t) - y(t)|,x(t) = \ctg (t + \pi/6),\ y = tg\ t\)
 \\
\textbf{B3.} 
\(A\) va \(B\) to'plamlari orasida o'zaro bir qiymatli moslik o'rnating.\(\ A = \lbrack - 7;3)\), \(B = \lbrack - 5;7\rbrack\).
 \\
\textbf{C1.} 
To'plamning Lebeg o'lchovini toping: \(A = \bigcup_{k = 1}^{\infty}\left( \frac{1}{2^{k + 1}},\frac{1}{2^{k}} \right)\);
 \\
\textbf{C2.} 
Lebeg integralini (\(\int_{A}^{}{f(x)d\mu}\)) hisoblang: \(f(x) = 2^{\lbrack x\rbrack}\), \(A = ( - 2;2)\);
 \\
\textbf{C3.} 
[1;4] to'plamida o'lchovsiz to'plamga misol keltiring.
 \\

\end{tabular}
\vspace{1cm}


\begin{tabular}{m{17cm}}
\textbf{4-bilet}

\vspace{0.5cm}

\textbf{T1.} 
Kompakt metrik fazolar.
 \\
\textbf{T2.} 
Tekislikta elementar to'plamlar va ularning o'lshovi.
 \\
\textbf{A1.} 
\(A = \{(x,y) \in \mathbb{R}^{2}:\ max\{|x|,|y|\} \leq 2\},\ B = \{(x,y) \in \mathbb{R}^{2}:\ 4 - x^{2} \geq y\}\), \(A,\ B,\ A \cup B,\ A \cap B,\ A \backslash B,\ B \backslash A,\ A \bigtriangleup B\) to'plamlarini aniqlang va tasvirlang.
 \\
\textbf{A2.} 
\(\lbrack - 3;1\rbrack\) va \(\lbrack - 2;1) \cup \lbrack 4;5\rbrack\) to'plamlari orasida bir qiymatli moslik o'rnating.
 \\
\textbf{A3.} 
\(\lbrack 5,\ 7\rbrack\) kesmada joylashgan sonlarning onlik kasr yozuvida \(7\) raqami qatnashmagan barcha sonlar to'plamining Lebeg o'lchovini toping.
 \\
\textbf{B1.} 
\(\int_{E}^{}f(x)d\mu\) Lebeg integralini hisoblang,\(\ f(x) = \left\{ \begin{matrix}
\frac{x^{2}}{(x + 2)(x + 4)},\ x \in \mathbb{I} \cap \lbrack 2,\ 4\rbrack \\
4x^{3},\ x\mathbb{\in Q \cap}\lbrack 2,\ 4\rbrack,\ E = \lbrack 2,\ 4\rbrack
\end{matrix} \right.\ \)
 \\
\textbf{B2.} 
Quyida berilganlar bo'yicha\(\ x,y \in X\) elementlar orasidagi masofani toping: \(X = C\left\lbrack \frac{\pi}{4};\ \frac{\pi}{2} \right\rbrack,\ \rho(x,y) = \max_{\frac{\pi}{4} \leq t \leq \frac{\pi}{2}}|x(t) - y(t)|,x(t) = \ctg (2t - \pi/6),\ y = tg(\ t - \pi/6)\)
 \\
\textbf{B3.} 
\(A\) va \(B\) to'plamlari orasida o'zaro bir qiymatli moslik o'rnating.\(\ A = \lbrack - 1;7)\), \(B = \lbrack - 3;9\rbrack\).
 \\
\textbf{C1.} 
To'plamning Lebeg o'lchovini toping: \(A = \bigcup_{k = 1}^{\infty}\left\lbrack e^{- 2k},e^{- 2k + 1} \right)\).
 \\
\textbf{C2.} 
Lebeg integralini (\(\int_{A}^{}{f(x)d\mu}\)) hisoblang: \(f(x) = \lbrack x\rbrack - 1\), \(A = \lbrack - 1;3\rbrack\);
 \\
\textbf{C3.} 
[-3;0] to'plamida o'lchovsiz to'plamga misol keltiring.
 \\

\end{tabular}
\vspace{1cm}


\begin{tabular}{m{17cm}}
\textbf{5-bilet}

\vspace{0.5cm}

\textbf{T1.} 
Metrik fazo va unga misollar.
 \\
\textbf{T2.} 
Lebeg va Riss teoremalari.
 \\
\textbf{A1.} 
\(A = \{(x,y) \in \mathbb{R}^{2}:\ xy \geq 0\},\ B = \{(x,y) \in \mathbb{R}^{2}:\ x^{2} + y^{2} \geq 1\}\), \(A,\ B,\ A \cup B,\ A \cap B,\ A \backslash B,\ B \backslash A,\ A \bigtriangleup B\) to'plamlarini aniqlang va tasvirlang.
 \\
\textbf{A2.} 
\(\lbrack - 1;\ 3\rbrack\) va \(\lbrack - 4; - 1) \cup \lbrack 2;3\rbrack\) to'plamlari orasida bir qiymatli moslik o'rnating.
 \\
\textbf{A3.} 
\(\lbrack 1,\ 3\rbrack\) kesmada joylashgan sonlarning onlik kasr yozuvida \(3\) raqami qatnashmagan barcha sonlar to'plamining Lebeg o'lchovini toping.
 \\
\textbf{B1.} 
\(\int_{E}^{}f(x)d\mu\) Lebeg integralini hisoblang, \(E = \lbrack 0,\ 1\rbrack\), \(f(x) = \left\{ \begin{matrix}
\frac{1}{\sqrt{x}},\ x \in \mathbb{I} \cap \lbrack 0,\ 1\rbrack \\
\sin x,\ x\mathbb{\in Q}
\end{matrix} \right.\ \)
 \\
\textbf{B2.} 
Quyida berilganlar bo'yicha\(\ x,y \in X\) elementlar orasidagi masofani toping: \(X = C\lbrack 0;\ \pi/6\rbrack,\ \rho(x,y) = \max_{0 \leq t \leq \pi/6}|x(t) - y(t)|,x(t) = \sin3t,\ y = \cos t\)
 \\
\textbf{B3.} 
\(A\) va \(B\) to'plamlari orasida o'zaro bir qiymatli moslik o'rnating.\(\ A = \lbrack - 6;2\rbrack\), \(B = ( - 7;3)\).
 \\
\textbf{C1.} 
\(P = \{ 0 \leq x \leq 1,\ 0 \leq y \leq 1\}\) va \(Q = \{ 0.3 \leq x \leq 0.8,\ 0 \leq y \leq 1\}\ \)to'g'ri to'rtburchaklar simmetrik ayirmasining o'lchovini toping.
 \\
\textbf{C2.} 
Lebeg integralini (\(\int_{A}^{}{f(x)d\mu}\)) hisoblang: \(f(x) = 2^{\lbrack 2x\rbrack}\), \(A = \lbrack 0;1)\);
 \\
\textbf{C3.} 
[-6;-3] to'plamida o'lchovsiz to'plamga misol keltiring.
 \\

\end{tabular}
\vspace{1cm}


\begin{tabular}{m{17cm}}
\textbf{6-bilet}

\vspace{0.5cm}

\textbf{T1.} 
Metrik fazolarda ochiq va yopiq to'plamlar.
 \\
\textbf{T2.} 
O'lshovli funkciyalar va ularning xossalari.
 \\
\textbf{A1.} 
\(A = \{(x,y) \in \mathbb{R}^{2}:\ x^{2} = y\},\ B = \{(x,y) \in \mathbb{R}^{2}:\ x^{2} + y^{2} \geq 4\}\), \(A,\ B,\ A \cup B,\ A \cap B,\ A \backslash B,\ B \backslash A,\ A \bigtriangleup B\) to'plamlarini aniqlang va tasvirlang.
 \\
\textbf{A2.} 
\(\lbrack 2;6\rbrack\) va \(\lbrack 2;4) \cup \lbrack 11;13\rbrack\) to'plamlari orasida bir qiymatli moslik o'rnating.
 \\
\textbf{A3.} 
\(\lbrack 7,\ 9\rbrack\) kesmada joylashgan sonlarning onlik kasr yozuvida \(9\) raqami qatnashmagan barcha sonlar to'plamining Lebeg o'lchovini toping.
 \\
\textbf{B1.} 
\(\int_{E}^{}f(x)d\mu\) Lebeg integralini hisoblang, \(f(x) = \left\{ \begin{matrix}
\frac{x^{2}}{(x - 2)(x - 4)},\ x \in \mathbb{I} \cap \lbrack - 4; - 1\rbrack \\
3x^{2} - 2,\ x\mathbb{\in Q \cap}\lbrack - 4; - 1\rbrack,E = \lbrack - 4; - 1\rbrack
\end{matrix} \right.\ \)
 \\
\textbf{B2.} 
Quyida berilganlar bo'yicha\(\ x,y \in X\) elementlar orasidagi masofani toping: \(X = C\lbrack 0,\pi\rbrack,\ \rho(x,y) = \max_{0 \leq t \leq \pi}|x(t) - y(t)|,x(t) = \sin2t,\ y = \cos4t\).
 \\
\textbf{B3.} 
\(A\) va \(B\) to'plamlari orasida o'zaro bir qiymatli moslik o'rnating.\(\ A = ( - 4;6\rbrack\), \(B = \lbrack - 2;6\rbrack\).
 \\
\textbf{C1.} 
To'plamning Lebeg o'lchovini toping: \(A = \bigcup_{k = 1}^{\infty}\left( 2k - 2^{- k},2k + \frac{1}{k!} \right)\);
 \\
\textbf{C2.} 
Lebeg integralini (\(\int_{A}^{}{f(x)d\mu}\)) hisoblang: \(f(x) = 2\lbrack x\rbrack\), \(A = ( - 3;3)\);
 \\
\textbf{C3.} 
[6;9] to'plamida o'lchovsiz to'plamga misol keltiring.
 \\

\end{tabular}
\vspace{1cm}


\begin{tabular}{m{17cm}}
\textbf{7-bilet}

\vspace{0.5cm}

\textbf{T1.} To'plamlar va ular ustida amallar.
 \\
\textbf{T2.} 
Tekislikta elementar to'plamlar va ularning o'lshovi.
 \\
\textbf{A1.} 
\(A = \{(x,y) \in \mathbb{R}^{2}:\ xy \leq 0\},\ B = \{(x,y) \in \mathbb{R}^{2}:\ x^{2} + y^{2} \geq 4\}\), \(A,\ B,\ A \cup B,\ A \cap B,\ A \backslash B,\ B \backslash A,\ A \bigtriangleup B\) to'plamlarini aniqlang va tasvirlang.
 \\
\textbf{A2.} 
\(( - 2;6\rbrack\) va \(( - 3; - 1) \cup \lbrack 1;7\rbrack\) to'plamlari orasida bir qiymatli moslik o'rnating.
 \\
\textbf{A3.} 
\(\lbrack 4,\ 6\rbrack\) kesmada joylashgan sonlarning onlik kasr yozuvida \(6\) raqami qatnashmagan barcha sonlar to'plamining Lebeg o'lchovini toping.
 \\
\textbf{B1.} 
\(\int_{E}^{}f(x)d\mu\) Lebeg integralini hisoblang, \(f(x) = \left\{ \begin{matrix}
\frac{x^{2}}{(x - 5)(x - 6)},\ x \in \mathbb{I} \cap \lbrack 0,\ 4\rbrack \\
3x^{2} - 2,\ x\mathbb{\in Q \cap}\lbrack 0,\ 4\rbrack,\ E = \lbrack 0,\ 4\rbrack
\end{matrix} \right.\ \)
 \\
\textbf{B2.} 
Quyida berilganlar bo'yicha\(\ x,y \in X\) elementlar orasidagi masofani toping: \(X = C\left\lbrack \frac{\pi}{6};\ \frac{\pi}{4} \right\rbrack,\ \rho(x,y) = \max_{\frac{\pi}{6} \leq t \leq \frac{\pi}{4}}|x(t) - y(t)|,x(t) = \ctg t,\ y = tg(\ 2t - \frac{\pi}{6})\)
 \\
\textbf{B3.} 
\(A\) va \(B\) to'plamlari orasida o'zaro bir qiymatli moslik o'rnating.\(\ A = ( - 3;3)\), \(B = \lbrack - 1;9\rbrack\).
 \\
\textbf{C1.} 
To'plamning Lebeg o'lchovini toping: \(A = \bigcup_{k = 1}^{\infty}\left( \frac{1}{2k + 1},\frac{1}{2k} \right)\);
 \\
\textbf{C2.} 
Lebeg integralini (\(\int_{A}^{}{f(x)d\mu}\)) hisoblang: \(f(x) = \frac{1}{\lbrack x + 1\rbrack}\), \(A = \lbrack 1;5)\);
 \\
\textbf{C3.} 
[-4;-1] to'plamida o'lchovsiz to'plamga misol keltiring.
 \\

\end{tabular}
\vspace{1cm}


\begin{tabular}{m{17cm}}
\textbf{8-bilet}

\vspace{0.5cm}

\textbf{T1.} 
To'plam quvvati va uning xossalari.
 \\
\textbf{T2.} 
Egorov teoremasi.
 \\
\textbf{A1.} 
\(A = \{(x,y) \in \mathbb{R}^{2}:\ y = - x\},\ B = \{(x,y) \in \mathbb{R}^{2}:\ x^{2} + y^{2} \leq 1\}\), \(A,\ B,\ A \cup B,\ A \cap B,\ A \backslash B,\ B \backslash A,\ A \bigtriangleup B\) to'plamlarini aniqlang va tasvirlang.
 \\
\textbf{A2.} 
\(\lbrack - 4;\ 1)\) va \(\lbrack - 3; - 1) \cup \lbrack 3;6)\) to'plamlari orasida bir qiymatli moslik o'rnating.
 \\
\textbf{A3.} 
\(\lbrack 6,\ 8\rbrack\) kesmada joylashgan sonlarning onlik kasr yozuvida \(9\) raqami qatnashmagan barcha sonlar to'plamining Lebeg o'lchovini toping.
 \\
\textbf{B1.} 
\(\int_{E}^{}f(x)d\mu\) Lebeg integralini hisoblang, \(f(x) = \left\{ \begin{matrix}
\frac{x^{2}}{(x + 3)(x + 2)},\ x \in \mathbb{I} \cap \lbrack 2,\ 4\rbrack \\
3x^{2} - 2,\ x\mathbb{\in Q \cap}\lbrack 2,\ 4\rbrack,\ E = \lbrack 2,\ 4\rbrack
\end{matrix} \right.\ \)
 \\
\textbf{B2.} 
Quyida berilganlar bo'yicha\(\ x,y \in X\) elementlar orasidagi masofani toping: \(X = C\left\lbrack \frac{\pi}{6};\ \frac{\pi}{4} \right\rbrack,\ \rho(x,y) = \max_{\frac{\pi}{6} \leq t \leq \frac{\pi}{4}}|x(t) - y(t)|,x(t) = \ctg (2t - \pi/6),\ y = tg(\ 2t - \pi/6)\)
 \\
\textbf{B3.} 
\(A\) va \(B\) to'plamlari orasida o'zaro bir qiymatli moslik o'rnating.\(\ A = ( - 3;5)\), \(B = \lbrack - 8;6)\).
 \\
\textbf{C1.} 
To'plamning Lebeg o'lchovini toping: \(A = \bigcup_{k = 1}^{\infty}\left( \frac{1}{k + 2},\frac{1}{k} \right)\);
 \\
\textbf{C2.} 
Lebeg integralini (\(\int_{A}^{}{f(x)d\mu}\)) hisoblang: \(f(x) = \frac{1}{\lbrack x\rbrack\lbrack x + 1\rbrack}\), \(A = \lbrack 1;3\rbrack\).
 \\
\textbf{C3.} 
[0;3] to'plamida o'lchovsiz to'plamga misol keltiring.
 \\

\end{tabular}
\vspace{1cm}


\begin{tabular}{m{17cm}}
\textbf{9-bilet}

\vspace{0.5cm}

\textbf{T1.} 
Metrik fazolarning uzliksiz akslantirishlari.
 \\
\textbf{T2.} 
O'lshovli funkciyalar va ularning xossalari.
 \\
\textbf{A1.} 
\(A = \{(x,y) \in \mathbb{R}^{2}:\ |x| + |y| \leq 2\},\ B = \{(x,y) \in \mathbb{R}^{2}:\ 9x^{2} + y^{2} \geq 9\}\),\(A,\ B,\ A \cup B,\ A \cap B,\ A \backslash B,\ B \backslash A,\ A \bigtriangleup B\) to'plamlarini aniqlang va tasvirlang.
 \\
\textbf{A2.} 
\(\lbrack 0;\ 3)\) va \(\lbrack 2;4) \cup \lbrack 5;6)\) to'plamlari orasida bir qiymatli moslik o'rnating.
 \\
\textbf{A3.} 
\(\lbrack 0,\ 2\rbrack\) kesmada joylashgan sonlarning onlik kasr yozuvida \(2\) raqami qatnashmagan barcha sonlar to'plamining Lebeg o'lchovini toping.
 \\
\textbf{B1.} 
\(\int_{E}^{}f(x)d\mu\) Lebeg integralini hisoblang, \(E = \lbrack 0,\ 1\rbrack\), \(f(x) = \left\{ \begin{matrix}
\frac{1}{(x + 1)^{3}}\ x \in \mathbb{I} \cap \lbrack 0,\ 1\rbrack \\
7x,\ x\mathbb{\in Q}
\end{matrix} \right.\ \)
 \\
\textbf{B2.} 
Quyida berilganlar bo'yicha\(\ x,y \in X\) elementlar orasidagi masofani toping: \(X = C\left\lbrack \frac{\pi}{4};\ \frac{\pi}{3} \right\rbrack,\ \rho(x,y) = \max_{\frac{\pi}{4} \leq t \leq \frac{\pi}{3}}|x(t) - y(t)|,x(t) = \ctg (2t + \pi/6),\ y = tg(\ t - \pi/6)\)
 \\
\textbf{B3.} 
\(A\) va \(B\) to'plamlari orasida o'zaro bir qiymatli moslik o'rnating.\(\ A = ( - 1;3)\), \(B = \lbrack 0;9\rbrack\).
 \\
\textbf{C1.} 
To'plamning Lebeg o'lchovini toping: \(A = \bigcup_{k = 1}^{\infty}\left( k,k + \frac{1}{k!} \right)\);
 \\
\textbf{C2.} 
Lebeg integralini (\(\int_{A}^{}{f(x)d\mu}\)) hisoblang: \(f(x) = \lbrack x + 1\rbrack\), \(A = \lbrack - 2;1)\);
 \\
\textbf{C3.} 
[9;12] to'plamida o'lchovsiz to'plamga misol keltiring.
 \\

\end{tabular}
\vspace{1cm}


\begin{tabular}{m{17cm}}
\textbf{10-bilet}

\vspace{0.5cm}

\textbf{T1.} 
Metrik fazolarda ochiq va yopiq to'plamlar.
 \\
\textbf{T2.} 
Lebeg va Riss teoremalari.
 \\
\textbf{A1.} 
\(A = \{(x,y) \in \mathbb{R}^{2}:\ y \geq x^{2}\},\ B = \{(x,y) \in \mathbb{R}^{2}:\ y \leq 4 - x^{2}\}\), \(A,\ B,\ A \cup B,\ A \cap B,\ A \backslash B,\ B \backslash A,\ A \bigtriangleup B\) to'plamlarini aniqlang va tasvirlang.
 \\
\textbf{A2.} 
\(\lbrack 1;\ 5\rbrack\) va \(\lbrack 1;\ 2) \cup \lbrack 7;10\rbrack\) to'plamlari orasida bir qiymatli moslik o'rnating.
 \\
\textbf{A3.} 
\(\lbrack 2,\ 4\rbrack\) kesmada joylashgan sonlarning onlik kasr yozuvida \(5\) raqami qatnashmagan barcha sonlar to'plamining Lebeg o'lchovini toping.
 \\
\textbf{B1.} 
\(\int_{E}^{}f(x)d\mu\) Lebeg integralini hisoblang, \(f(x) = \left\{ \begin{matrix}
\frac{x^{2}}{(x - 5)(x - 7)},\ x \in \mathbb{I} \cap \lbrack 1,\ 4\rbrack \\
3x^{2} - 2,\ x\mathbb{\in Q \cap}\lbrack 1,\ 4\rbrack,\ E = \lbrack 1,\ 4\rbrack
\end{matrix} \right.\ \)
 \\
\textbf{B2.} 
Quyida berilganlar bo'yicha\(\ x,y \in X\) elementlar orasidagi masofani toping: \(X = C\lbrack 0;\ \pi/4\rbrack,\ \rho(x,y) = \max_{0 \leq t \leq \pi/4}|x(t) - y(t)|,x(t) = \sin4t,\ y = \cos2t\)
 \\
\textbf{B3.} 
\(A\) va \(B\) to'plamlari orasida o'zaro bir qiymatli moslik o'rnating.\(\ A = \lbrack - 2;4\rbrack\), \(B = ( - 5;5)\).
 \\
\textbf{C1.} 
To'plamning Lebeg o'lchovini toping: \(A = \bigcup_{k = 1}^{\infty}\left( \frac{1}{3^{k}},\frac{1}{3^{k - 1}} \right)\);
 \\
\textbf{C2.} 
Lebeg integralini (\(\int_{A}^{}{f(x)d\mu}\)) hisoblang: \(f(x) = \frac{1}{\lbrack x - 1\rbrack}\), \(A = (3;6)\);
 \\
\textbf{C3.} 
[-10;-7] to'plamida o'lchovsiz to'plamga misol keltiring.
 \\

\end{tabular}
\vspace{1cm}


\begin{tabular}{m{17cm}}
\textbf{11-bilet}

\vspace{0.5cm}

\textbf{T1.} 
Metrik fazo va unga misollar.
 \\
\textbf{T2.} 
Tekislikta elementar to'plamlar va ularning o'lshovi.
 \\
\textbf{A1.} 
\(A = \{(x,y) \in \mathbb{R}^{2}:\ xy \geq 0\},\ B = \{(x,y) \in \mathbb{R}^{2}:\ |x| + |y - 2| \geq 1\}\), \(A,\ B,\ A \cup B,\ A \cap B,\ A \backslash B,\ B \backslash A,\ A \bigtriangleup B\) to'plamlarini aniqlang va tasvirlang.
 \\
\textbf{A2.} 
\(\lbrack 0;6\rbrack\) va \(\lbrack 0;5) \cup \lbrack 7;8\rbrack\) to'plamlari orasida bir qiymatli moslik o'rnating.
 \\
\textbf{A3.} 
\(\lbrack 7,\ 9\rbrack\) kesmada joylashgan sonlarning onlik kasr yozuvida \(0\) raqami qatnashmagan barcha sonlar to'plamining Lebeg o'lchovini toping.
 \\
\textbf{B1.} 
\(\int_{E}^{}f(x)d\mu\) Lebeg integralini hisoblang, \(E = \lbrack 0,\ 1\rbrack\), \(f(x) = \left\{ \begin{matrix}
\frac{1}{\sqrt{x}},\ x \in \mathbb{I} \cap \lbrack 0,\ 1\rbrack \\
\sin x,\ x\mathbb{\in Q}
\end{matrix} \right.\ \)
 \\
\textbf{B2.} 
Quyida berilganlar bo'yicha\(\ x,y \in X\) elementlar orasidagi masofani toping: \(X = C\left\lbrack \frac{\pi}{4};\ \frac{\pi}{3} \right\rbrack,\ \rho(x,y) = \max_{\frac{\pi}{4} \leq t \leq \frac{\pi}{3}}|x(t) - y(t)|,x(t) = \ctg (2t + \pi/6),\ y = tg(\ t - \pi/6)\)
 \\
\textbf{B3.} 
\(A\) va \(B\) to'plamlari orasida o'zaro bir qiymatli moslik o'rnating.\(\ A = \lbrack - 1;4)\), \(B = \lbrack - 1;7\rbrack\).
 \\
\textbf{C1.} 
To'plamning Lebeg o'lchovini toping: \(A = \bigcup_{k = 1}^{\infty}\left( \frac{1}{k + 1},\frac{1}{k} \right)\);
 \\
\textbf{C2.} 
Lebeg integralini (\(\int_{A}^{}{f(x)d\mu}\)) hisoblang: \(f(x) = sign(x - 1)\), \(A = \lbrack - 1;2)\);
 \\
\textbf{C3.} 
[-12;-9] to'plamida o'lchovsiz to'plamga misol keltiring.
 \\

\end{tabular}
\vspace{1cm}


\begin{tabular}{m{17cm}}
\textbf{12-bilet}

\vspace{0.5cm}

\textbf{T1.} 
Kompakt metrik fazolar.
 \\
\textbf{T2.} 
Egorov teoremasi.
 \\
\textbf{A1.} 
\(A = \{(x,y) \in \mathbb{R}^{2}:\ x \geq y\},\ B = \{(x,y) \in \mathbb{R}^{2}:\ 9x^{2} + y^{2} \leq 36\}\),\(A,\ B,\ A \cup B,\ A \cap B,\ A \backslash B,\ B \backslash A,\ A \bigtriangleup B\) to'plamlarini aniqlang va tasvirlang.
 \\
\textbf{A2.} 
\(\lbrack 2;6)\) va \(\lbrack - 2;1) \cup \lbrack 4;5)\) to'plamlari orasida bir qiymatli moslik o'rnating.
 \\
\textbf{A3.} 
\(\lbrack 8,\ 10\rbrack\) kesmada joylashgan sonlarning onlik kasr yozuvida \(0\) raqami qatnashmagan barcha sonlar to'plamining Lebeg o'lchovini toping.
 \\
\textbf{B1.} 
\(\int_{E}^{}f(x)d\mu\) Lebeg integralini hisoblang, \(f(x) = \left\{ \begin{matrix}
\frac{x^{2}}{(x + 3)(x + 2)},\ x \in \mathbb{I} \cap \lbrack 2,\ 4\rbrack \\
3x^{2} - 2,\ x\mathbb{\in Q \cap}\lbrack 2,\ 4\rbrack,\ E = \lbrack 2,\ 4\rbrack
\end{matrix} \right.\ \)
 \\
\textbf{B2.} 
Quyida berilganlar bo'yicha\(\ x,y \in X\) elementlar orasidagi masofani toping: \(X = C\left\lbrack \frac{\pi}{6};\ \frac{\pi}{4} \right\rbrack,\ \rho(x,y) = \max_{\frac{\pi}{6} \leq t \leq \frac{\pi}{4}}|x(t) - y(t)|,x(t) = \ctg t,\ y = tg(\ 2t - \frac{\pi}{6})\)
 \\
\textbf{B3.} 
\(A\) va \(B\) to'plamlari orasida o'zaro bir qiymatli moslik o'rnating.\(\ A = ( - 2;3\rbrack\), \(B = \lbrack - 2;8\rbrack\).
 \\
\textbf{C1.} 
To'plamning Lebeg o'lchovini toping: \(A = \bigcup_{k = 1}^{\infty}\left( k^{2},k^{2} + 2^{- k} \right)\);
 \\
\textbf{C2.} 
Lebeg integralini (\(\int_{A}^{}{f(x)d\mu}\)) hisoblang: \(f(x) = \frac{1}{\lbrack x\rbrack}\), \(A = (1;4)\);
 \\
\textbf{C3.} 
[-11;-8] to'plamida o'lchovsiz to'plamga misol keltiring.
 \\

\end{tabular}
\vspace{1cm}


\begin{tabular}{m{17cm}}
\textbf{13-bilet}

\vspace{0.5cm}

\textbf{T1.} To'plamlar va ular ustida amallar.
 \\
\textbf{T2.} 
Lebeg va Riss teoremalari.
 \\
\textbf{A1.} 
\(A = \{(x,y) \in \mathbb{R}^{2}:\ max\{|x|,|y|\} \leq 2\},\ B = \{(x,y) \in \mathbb{R}^{2}:\ x^{2} + 1 \leq y\}\), \(A,\ B,\ A \cup B,\ A \cap B,\ A \backslash B,\ B \backslash A,\ A \bigtriangleup B\) to'plamlarini aniqlang va tasvirlang.
 \\
\textbf{A2.} 
\(\lbrack 0;5)\) va \(\lbrack - 2;0) \cup \lbrack 1;4)\) to'plamlari orasida bir qiymatli moslik o'rnating.
 \\
\textbf{A3.} 
\(\lbrack 0,\ 2\rbrack\) kesmada joylashgan sonlarning onlik kasr yozuvida \(3\) raqami qatnashmagan barcha sonlar to'plamining Lebeg o'lchovini toping.
 \\
\textbf{B1.} 
\(\int_{E}^{}f(x)d\mu\) Lebeg integralini hisoblang, \(f(x) = \left\{ \begin{matrix}
\frac{x^{2}}{(x - 5)(x - 6)},\ x \in \mathbb{I} \cap \lbrack 0,\ 4\rbrack \\
3x^{2} - 2,\ x\mathbb{\in Q \cap}\lbrack 0,\ 4\rbrack,\ E = \lbrack 0,\ 4\rbrack
\end{matrix} \right.\ \)
 \\
\textbf{B2.} 
Quyida berilganlar bo'yicha\(\ x,y \in X\) elementlar orasidagi masofani toping: \(X = C\lbrack 0;\ \pi/6\rbrack,\ \rho(x,y) = \max_{0 \leq t \leq \pi/6}|x(t) - y(t)|,x(t) = \sin3t,\ y = \cos t\)
 \\
\textbf{B3.} 
\(A\) va \(B\) to'plamlari orasida o'zaro bir qiymatli moslik o'rnating.\(\ A = ( - 4;3\rbrack\), \(B = \lbrack - 4;10\rbrack\).
 \\
\textbf{C1.} 
To'plamning Lebeg o'lchovini toping: \(A = \bigcup_{k = 1}^{\infty}\left( k - 2^{- k},k + \frac{1}{k!} \right)\);
 \\
\textbf{C2.} 
Lebeg integralini (\(\int_{A}^{}{f(x)d\mu}\)) hisoblang: \(f(x) = 2 - \lbrack x\rbrack\), \(A = \lbrack - 2;3)\);
 \\
\textbf{C3.} 
[3;6] to'plamida o'lchovsiz to'plamga misol keltiring.
 \\

\end{tabular}
\vspace{1cm}


\begin{tabular}{m{17cm}}
\textbf{14-bilet}

\vspace{0.5cm}

\textbf{T1.} 
Metrik fazolarda ochiq va yopiq to'plamlar.
 \\
\textbf{T2.} 
Egorov teoremasi.
 \\
\textbf{A1.} 
\(A = \{(x,y) \in \mathbb{R}^{2}:\ y = x^{2}\},\ B = \{(x,y) \in \mathbb{R}^{2}:\ x^{2} + (y - 1)^{2} \leq 1\}\), \(A,\ B,\ A \cup B,\ A \cap B,\ A \backslash B,\ B \backslash A,\ A \bigtriangleup B\) to'plamlarini aniqlang va tasvirlang.
 \\
\textbf{A2.} 
\(\lbrack 0;5\rbrack\) va \(\lbrack - 2;2) \cup \lbrack 3;4\rbrack\) to'plamlari orasida bir qiymatli moslik o'rnating.
 \\
\textbf{A3.} 
\(\lbrack 4,\ 6\rbrack\) kesmada joylashgan sonlarning onlik kasr yozuvida \(7\) raqami qatnashmagan barcha sonlar to'plamining Lebeg o'lchovini toping.
 \\
\textbf{B1.} 
\(\int_{E}^{}f(x)d\mu\) Lebeg integralini hisoblang, \(E = \lbrack 0,\ 1\rbrack\), \(f(x) = \left\{ \begin{matrix}
\frac{1}{(x + 1)^{3}}\ x \in \mathbb{I} \cap \lbrack 0,\ 1\rbrack \\
7x,\ x\mathbb{\in Q}
\end{matrix} \right.\ \)
 \\
\textbf{B2.} 
Quyida berilganlar bo'yicha\(\ x,y \in X\) elementlar orasidagi masofani toping: \(X = C\lbrack 0,\pi\rbrack,\ \rho(x,y) = \max_{0 \leq t \leq \pi}|x(t) - y(t)|,x(t) = \sin2t,\ y = \cos4t\).
 \\
\textbf{B3.} 
\(A\) va \(B\) to'plamlari orasida o'zaro bir qiymatli moslik o'rnating.\(\ A = ( - 3;4)\), \(B = \lbrack - 2;10)\).
 \\
\textbf{C1.} 
To'plamning Lebeg o'lchovini toping: \(A = \bigcup_{k = 1}^{\infty}\left( k,k + \frac{2}{k(k + 1)} \right)\);
 \\
\textbf{C2.} 
Lebeg integralini (\(\int_{A}^{}{f(x)d\mu}\)) hisoblang: \(f(x) = 2^{( - 1)^{\lbrack x\rbrack}}\), \(A = \lbrack 0;3)\);
 \\
\textbf{C3.} 
[-8;-5] to'plamida o'lchovsiz to'plamga misol keltiring.
 \\

\end{tabular}
\vspace{1cm}


\begin{tabular}{m{17cm}}
\textbf{15-bilet}

\vspace{0.5cm}

\textbf{T1.} 
To'plam quvvati va uning xossalari.
 \\
\textbf{T2.} 
O'lshovli funkciyalar va ularning xossalari.
 \\
\textbf{A1.} 
\(A = \{(x,y) \in \mathbb{R}^{2}:\ |x| + |y| \geq 3\},\ B = \{(x,y) \in \mathbb{R}^{2}:\ max\{|x|,|y|\} \leq 2\}\), \(A,\ B,\ A \cup B,\ A \cap B,\ A \backslash B,\ B \backslash A,\ A \bigtriangleup B\) to'plamlarini aniqlang va tasvirlang.
 \\
\textbf{A2.} 
\(\lbrack - 3;\ 7\rbrack\) va \(\lbrack 2;5) \cup \lbrack 8;15\rbrack\) to'plamlari orasida bir qiymatli moslik o'rnating.
 \\
\textbf{A3.} 
\(\lbrack 1,\ 3\rbrack\) kesmada joylashgan sonlarning onlik kasr yozuvida \(4\) raqami qatnashmagan barcha sonlar to'plamining Lebeg o'lchovini toping.
 \\
\textbf{B1.} 
\(\int_{E}^{}f(x)d\mu\) Lebeg integralini hisoblang, \(f(x) = \left\{ \begin{matrix}
\frac{x^{2}}{(x + 2)(x + 4)},\ x \in \mathbb{I} \cap \lbrack 0,\ 4\rbrack \\
3x^{2} - 2,\ x\mathbb{\in Q \cap}\lbrack 0,\ 4\rbrack,\ E = \lbrack 0,\ 4\rbrack
\end{matrix} \right.\ \)
 \\
\textbf{B2.} 
Quyida berilganlar bo'yicha\(\ x,y \in X\) elementlar orasidagi masofani toping: \(X = C\lbrack 0;\ \pi/4\rbrack,\ \rho(x,y) = \max_{0 \leq t \leq \pi/4}|x(t) - y(t)|,x(t) = \sin t,\ y = \cos3t\)
 \\
\textbf{B3.} 
\(A\) va \(B\) to'plamlari orasida o'zaro bir qiymatli moslik o'rnating.\(\ A = ( - 2;4)\), \(B = \lbrack 2;10)\).
 \\
\textbf{C1.} 
To'plamning Lebeg o'lchovini toping: \(A = \bigcup_{k = 1}^{\infty}\left( k,k + \frac{3}{k(k + 1)} \right)\);
 \\
\textbf{C2.} 
Lebeg integralini (\(\int_{A}^{}{f(x)d\mu}\)) hisoblang: \(f(x) = sign(2x + 1)\), \(A = ( - 1;1\rbrack\).
 \\
\textbf{C3.} 
[-2;1] to'plamida o'lchovsiz to'plamga misol keltiring.
 \\

\end{tabular}
\vspace{1cm}


\begin{tabular}{m{17cm}}
\textbf{16-bilet}

\vspace{0.5cm}

\textbf{T1.} 
Kompakt metrik fazolar.
 \\
\textbf{T2.} 
Tekislikta elementar to'plamlar va ularning o'lshovi.
 \\
\textbf{A1.} 
\(A = \{(x,y) \in \mathbb{R}^{2}:\ x \geq y\},\ B = \{(x,y) \in \mathbb{R}^{2}:\ x^{2} + 4y^{2} \geq 4\}\), \(A,\ B,\ A \cup B,\ A \cap B,\ A \backslash B,\ B \backslash A,\ A \bigtriangleup B\) to'plamlarini aniqlang va tasvirlang.
 \\
\textbf{A2.} 
\(\lbrack - 2;\ 1)\) va \(\lbrack 1;2) \cup \lbrack 3;5)\) to'plamlari orasida bir qiymatli moslik o'rnating.
 \\
\textbf{A3.} 
\(\lbrack 8,\ 10\rbrack\) kesmada joylashgan sonlarning onlik kasr yozuvida \(6\) raqami qatnashmagan barcha sonlar to'plamining Lebeg o'lchovini toping.
 \\
\textbf{B1.} 
\(\int_{E}^{}f(x)d\mu\) Lebeg integralini hisoblang, \(f(x) = \left\{ \begin{matrix}
\frac{x^{2}}{(x - 5)(x - 6)},\ x \in \mathbb{I} \cap \lbrack 0,\ 4\rbrack \\
3x^{2} - 2,\ x\mathbb{\in Q \cap}\lbrack 0,\ 4\rbrack,\ E = \lbrack 0,\ 4\rbrack
\end{matrix} \right.\ \)
 \\
\textbf{B2.} 
Quyida berilganlar bo'yicha\(\ x,y \in X\) elementlar orasidagi masofani toping: \(X = C\left\lbrack \frac{\pi}{6};\ \frac{\pi}{3} \right\rbrack,\ \rho(x,y) = \max_{\frac{\pi}{6} \leq t \leq \frac{\pi}{3}}|x(t) - y(t)|,x(t) = \ctg (t + \pi/6),\ y = tg\ t\)
 \\
\textbf{B3.} 
\(A\) va \(B\) to'plamlari orasida o'zaro bir qiymatli moslik o'rnating.\(\ A = ( - 5;1\rbrack\), \(B = \lbrack - 4;6\rbrack\).
 \\
\textbf{C1.} 
\(P = \{ 0 \leq x \leq 1,\ 0 \leq y \leq 1\}\) va \(Q = \{ 0.3 \leq x \leq 0.8,\ 0 \leq y \leq 1\}\) to'g'ri to'rtburchaklar kesishmasining o'lchovini toping.
 \\
\textbf{C2.} 
Lebeg integralini (\(\int_{A}^{}{f(x)d\mu}\)) hisoblang: \(f(x) = \frac{1}{\lbrack x - 1\rbrack!}\), \(A = (1;3)\);
 \\
\textbf{C3.} 
[4;7] to'plamida o'lchovsiz to'plamga misol keltiring.
 \\

\end{tabular}
\vspace{1cm}


\begin{tabular}{m{17cm}}
\textbf{17-bilet}

\vspace{0.5cm}

\textbf{T1.} 
Metrik fazo va unga misollar.
 \\
\textbf{T2.} 
Egorov teoremasi.
 \\
\textbf{A1.} 
\(\ A = \{(x,y) \in \mathbb{R}^{2}:\ x \leq y\},\ B = \{(x,y) \in \mathbb{R}^{2}:\ 4x^{2} + 9y^{2} \geq 36\}\), \(A,\ B,\ A \cup B,\ A \cap B,\ A \backslash B,\ B \backslash A,\ A \bigtriangleup B\) to'plamlarini aniqlang va tasvirlang.
 \\
\textbf{A2.} 
\(\lbrack - 1;\ 7)\) va \(\lbrack - 2;4) \cup \lbrack 7;9)\) to'plamlari orasida bir qiymatli moslik o'rnating.
 \\
\textbf{A3.} 
\(\lbrack 0,\ 1\rbrack\) kesmada joylashgan sonlarning onlik kasr yozuvida \(1\) raqami qatnashmagan barcha sonlar to'plamining Lebeg o'lchovini toping.
 \\
\textbf{B1.} 
\(\int_{E}^{}f(x)d\mu\) Lebeg integralini hisoblang, \(f(x) = \left\{ \begin{matrix}
\frac{x^{2}}{(x - 2)(x - 4)},\ x \in \mathbb{I} \cap \lbrack - 4; - 1\rbrack \\
3x^{2} - 2,\ x\mathbb{\in Q \cap}\lbrack - 4; - 1\rbrack,E = \lbrack - 4; - 1\rbrack
\end{matrix} \right.\ \)
 \\
\textbf{B2.} 
Quyida berilganlar bo'yicha\(\ x,y \in X\) elementlar orasidagi masofani toping: \(X = C\left\lbrack \frac{\pi}{6};\ \frac{\pi}{4} \right\rbrack,\ \rho(x,y) = \max_{\frac{\pi}{6} \leq t \leq \frac{\pi}{4}}|x(t) - y(t)|,x(t) = \ctg (2t - \pi/6),\ y = tg(\ 2t - \pi/6)\)
 \\
\textbf{B3.} 
\(A\) va \(B\) to'plamlari orasida o'zaro bir qiymatli moslik o'rnating.\(\ A = ( - 5;3)\), \(B = \lbrack - 10;3\rbrack\).
 \\
\textbf{C1.} 
To'plamning Lebeg o'lchovini toping: \(A = \bigcup_{k = 1}^{\infty}\left( \frac{1}{3^{k}},\frac{1}{3^{k - 1}} \right)\);
 \\
\textbf{C2.} 
Lebeg integralini (\(\int_{A}^{}{f(x)d\mu}\)) hisoblang: \(f(x) = sign(x)\), \(A = \lbrack - 2;2)\);
 \\
\textbf{C3.} 
[10;13] to'plamida o'lchovsiz to'plamga misol keltiring.
 \\

\end{tabular}
\vspace{1cm}


\begin{tabular}{m{17cm}}
\textbf{18-bilet}

\vspace{0.5cm}

\textbf{T1.} 
Metrik fazolarning uzliksiz akslantirishlari.
 \\
\textbf{T2.} 
Tekislikta elementar to'plamlar va ularning o'lshovi.
 \\
\textbf{A1.} 
\(A = \{(x,y) \in \mathbb{R}^{2}:\ x = - y\},\ B = \{(x,y) \in \mathbb{R}^{2}:\ (x - 2)^{2} + (y + 3)^{2} \geq 1\}\), \(A,\ B,\ A \cup B,\ A \cap B,\ A \backslash B,\ B \backslash A,\ A \bigtriangleup B\) to'plamlarini aniqlang va tasvirlang.
 \\
\textbf{A2.} 
\(\lbrack - 3;\ 3)\) va \(\lbrack 0;4) \cup \lbrack 7;9)\) to'plamlari orasida bir qiymatli moslik o'rnating.
 \\
\textbf{A3.} 
\(\lbrack 2,\ 4\rbrack\) kesmada joylashgan sonlarning onlik kasr yozuvida \(4\) raqami qatnashmagan barcha sonlar to'plamining Lebeg o'lchovini toping.
 \\
\textbf{B1.} 
\(\int_{E}^{}f(x)d\mu\) Lebeg integralini hisoblang,\(\ f(x) = \left\{ \begin{matrix}
\frac{x^{2}}{(x + 2)(x + 4)},\ x \in \mathbb{I} \cap \lbrack 2,\ 4\rbrack \\
4x^{3},\ x\mathbb{\in Q \cap}\lbrack 2,\ 4\rbrack,\ E = \lbrack 2,\ 4\rbrack
\end{matrix} \right.\ \)
 \\
\textbf{B2.} 
Quyida berilganlar bo'yicha\(\ x,y \in X\) elementlar orasidagi masofani toping: \(X = C\lbrack 0;\ \pi/3\rbrack,\ \rho(x,y) = \max_{0 \leq t \leq \pi/3}|x(t) - y(t)|,x(t) = \sin t,\ y = \cos5t\)
 \\
\textbf{B3.} 
\(A\) va \(B\) to'plamlari orasida o'zaro bir qiymatli moslik o'rnating.\(\ A = ( - 5;3)\), \(B = \lbrack - 2;8\rbrack\).
 \\
\textbf{C1.} 
\(P = \{ 0 \leq x \leq 1,\ 0 \leq y \leq 1\}\) va \(Q = \{ 0.3 \leq x \leq 0.8,\ 0 \leq y \leq 1\}\) to'g'ri to'rtburchaklar kesishmasining o'lchovini toping.
 \\
\textbf{C2.} 
Lebeg integralini (\(\int_{A}^{}{f(x)d\mu}\)) hisoblang: \(f(x) = \frac{1}{\lbrack x\rbrack!}\), \(A = \lbrack 0;4)\);
 \\
\textbf{C3.} 
[-5;-2] to'plamida o'lchovsiz to'plamga misol keltiring.
 \\

\end{tabular}
\vspace{1cm}


\begin{tabular}{m{17cm}}
\textbf{19-bilet}

\vspace{0.5cm}

\textbf{T1.} To'plamlar va ular ustida amallar.
 \\
\textbf{T2.} 
O'lshovli funkciyalar va ularning xossalari.
 \\
\textbf{A1.} 
\(\ A = \{(x,y) \in \mathbb{R}^{2}:\ xy \leq 0\},\ B = \{(x,y) \in \mathbb{R}^{2}:\ |x| + |y| \geq 1\}\), \(A,\ B,\ A \cup B,\ A \cap B,\ A \backslash B,\ B \backslash A,\ A \bigtriangleup B\) to'plamlarini aniqlang va tasvirlang.
 \\
\textbf{A2.} 
\(\lbrack 1;7\rbrack\) va \(\lbrack - 1;4) \cup \lbrack 6;7\rbrack\) to'plamlari orasida bir qiymatli moslik o'rnating.
 \\
\textbf{A3.} 
\(\lbrack 5,\ 7\rbrack\) kesmada joylashgan sonlarning onlik kasr yozuvida \(8\) raqami qatnashmagan barcha sonlar to'plamining Lebeg o'lchovini toping.
 \\
\textbf{B1.} 
\(\int_{E}^{}f(x)d\mu\) Lebeg integralini hisoblang, \(f(x) = \left\{ \begin{matrix}
\frac{x^{2}}{(x - 5)(x - 7)},\ x \in \mathbb{I} \cap \lbrack 1,\ 4\rbrack \\
3x^{2} - 2,\ x\mathbb{\in Q \cap}\lbrack 1,\ 4\rbrack,\ E = \lbrack 1,\ 4\rbrack
\end{matrix} \right.\ \)
 \\
\textbf{B2.} 
Quyida berilganlar bo'yicha\(\ x,y \in X\) elementlar orasidagi masofani toping: \(X = C\lbrack 0;\ \pi/4\rbrack,\ \rho(x,y) = \max_{0 \leq t \leq \pi/4}|x(t) - y(t)|,x(t) = \sin4t,\ y = \cos2t\)
 \\
\textbf{B3.} 
\(A\) va \(B\) to'plamlari orasida o'zaro bir qiymatli moslik o'rnating.\(\ A = \lbrack - 4;4\rbrack\), \(B = ( - 11;3)\).
 \\
\textbf{C1.} 
\(P = \{ 0 \leq x \leq 1,\ 0 \leq y \leq 1\}\) va \(Q = \{ 0.3 \leq x \leq 0.8,\ 0 \leq y \leq 1\}\ \)to'g'ri to'rtburchaklar simmetrik ayirmasining o'lchovini toping.
 \\
\textbf{C2.} 
Lebeg integralini (\(\int_{A}^{}{f(x)d\mu}\)) hisoblang: \(f(x) = sign(x + 1)\), \(A = \lbrack - 2;2\rbrack\);
 \\
\textbf{C3.} 
[-7;-4] to'plamida o'lchovsiz to'plamga misol keltiring.
 \\

\end{tabular}
\vspace{1cm}


\begin{tabular}{m{17cm}}
\textbf{20-bilet}

\vspace{0.5cm}

\textbf{T1.} 
To'plam quvvati va uning xossalari.
 \\
\textbf{T2.} 
Lebeg va Riss teoremalari.
 \\
\textbf{A1.} 
\(A = \{(x,y) \in \mathbb{R}^{2}:\ y = - x^{2}\},\ B = \{(x,y) \in \mathbb{R}^{2}:\ (x + 1)^{2} + (y + 1)^{2} \leq 1\}\), \(A,\ B,\ A \cup B,\ A \cap B,\ A \backslash B,\ B \backslash A,\ A \bigtriangleup B\) to'plamlarini aniqlang va tasvirlang.
 \\
\textbf{A2.} 
\(( - 4;1\rbrack\) va \(( - 1;3) \cup \lbrack 8;9\rbrack\) to'plamlari orasida bir qiymatli moslik o'rnating.
 \\
\textbf{A3.} 
\(\lbrack 3,\ 5\rbrack\) kesmada joylashgan sonlarning onlik kasr yozuvida \(5\) raqami qatnashmagan barcha sonlar to'plamining Lebeg o'lchovini toping.
 \\
\textbf{B1.} 
\(\int_{E}^{}f(x)d\mu\) Lebeg integralini hisoblang, \(f(x) = \left\{ \begin{matrix}
\frac{x^{2}}{(x - 2)(x - 4)},\ x \in \mathbb{I} \cap \lbrack - 1;1\rbrack \\
3x^{2} - 2,\ x\mathbb{\in Q \cap}\lbrack - 1;1\rbrack,\ E = \lbrack - 1;1\rbrack
\end{matrix} \right.\ \)
 \\
\textbf{B2.} 
Quyida berilganlar bo'yicha\(\ x,y \in X\) elementlar orasidagi masofani toping: \(X = C\left\lbrack \frac{\pi}{4};\ \frac{\pi}{2} \right\rbrack,\ \rho(x,y) = \max_{\frac{\pi}{4} \leq t \leq \frac{\pi}{2}}|x(t) - y(t)|,x(t) = \ctg (2t - \pi/6),\ y = tg(\ t - \pi/6)\)
 \\
\textbf{B3.} 
\(A\) va \(B\) to'plamlari orasida o'zaro bir qiymatli moslik o'rnating.\(\ A = ( - 1;4)\), \(B = \lbrack 2;12)\).
 \\
\textbf{C1.} 
To'plamning Lebeg o'lchovini toping: \(A = \bigcup_{k = 1}^{\infty}\left( k,k + \frac{1}{k!} \right)\);
 \\
\textbf{C2.} 
Lebeg integralini (\(\int_{A}^{}{f(x)d\mu}\)) hisoblang: \(f(x) = \frac{1}{\lbrack x\rbrack\lbrack x + 1\rbrack}\), \(A = \lbrack 1;3\rbrack\);
 \\
\textbf{C3.} 
[2;5] to'plamida o'lchovsiz to'plamga misol keltiring.
 \\

\end{tabular}
\vspace{1cm}


\begin{tabular}{m{17cm}}
\textbf{21-bilet}

\vspace{0.5cm}

\textbf{T1.} 
Metrik fazolarning uzliksiz akslantirishlari.
 \\
\textbf{T2.} 
Tekislikta elementar to'plamlar va ularning o'lshovi.
 \\
\textbf{A1.} 
\(\ A = \{(x,y) \in \mathbb{R}^{2}:\ y = x^{2}\},\ B = \{(x,y) \in \mathbb{R}^{2}:\ (x - 1)^{2} + (y - 1)^{2} \leq 4\}\), \(A,\ B,\ A \cup B,\ A \cap B,\ A \backslash B,\ B \backslash A,\ A \bigtriangleup B\) to'plamlarini aniqlang va tasvirlang.
 \\
\textbf{A2.} 
\(\lbrack - 2;\ 4\rbrack\) va \(\lbrack - 2;1) \cup \lbrack 2;5\rbrack\) to'plamlari orasida bir qiymatli moslik o'rnating.
 \\
\textbf{A3.} 
\(\lbrack 2,\ 4\rbrack\) kesmada joylashgan sonlarning onlik kasr yozuvida \(4\) raqami qatnashmagan barcha sonlar to'plamining Lebeg o'lchovini toping.
 \\
\textbf{B1.} 
\(\int_{E}^{}f(x)d\mu\) Lebeg integralini hisoblang, \(E = \lbrack 0,\ 1\rbrack\), \(f(x) = \left\{ \begin{matrix}
\frac{1}{(x + 1)^{3}}\ x \in \mathbb{I} \cap \lbrack 0,\ 1\rbrack \\
7x,\ x\mathbb{\in Q}
\end{matrix} \right.\ \)
 \\
\textbf{B2.} 
Quyida berilganlar bo'yicha\(\ x,y \in X\) elementlar orasidagi masofani toping: \(X = C\lbrack 0;\ \pi/6\rbrack,\ \rho(x,y) = \max_{0 \leq t \leq \pi/6}|x(t) - y(t)|,x(t) = \sin3t,\ y = \cos t\)
 \\
\textbf{B3.} 
\(A\) va \(B\) to'plamlari orasida o'zaro bir qiymatli moslik o'rnating.\(\ A = ( - 5;3)\), \(B = \lbrack - 2;8\rbrack\).
 \\
\textbf{C1.} 
To'plamning Lebeg o'lchovini toping: \(A = \bigcup_{k = 1}^{\infty}\left( \frac{1}{2^{k + 1}},\frac{1}{2^{k}} \right)\);
 \\
\textbf{C2.} 
Lebeg integralini (\(\int_{A}^{}{f(x)d\mu}\)) hisoblang: \(f(x) = \frac{1}{\lbrack x\rbrack - 1}\), \(A = \lbrack 2;5\rbrack\);
 \\
\textbf{C3.} 
[-9;-6] to'plamida o'lchovsiz to'plamga misol keltiring.
 \\

\end{tabular}
\vspace{1cm}


\begin{tabular}{m{17cm}}
\textbf{22-bilet}

\vspace{0.5cm}

\textbf{T1.} 
Kompakt metrik fazolar.
 \\
\textbf{T2.} 
Egorov teoremasi.
 \\
\textbf{A1.} 
\(A = \{(x,y) \in \mathbb{R}^{2}:\ x \leq y\},\ B = \{(x,y) \in \mathbb{R}^{2}:\ 9x^{2} + y^{2} \leq 9\}\), \(A,\ B,\ A \cup B,\ A \cap B,\ A \backslash B,\ B \backslash A,\ A \bigtriangleup B\) to'plamlarini aniqlang va tasvirlang.
 \\
\textbf{A2.} 
\(\lbrack - 2;4)\) va \(\lbrack 0;4) \cup \lbrack 5;7)\) to'plamlari orasida bir qiymatli moslik o'rnating.
 \\
\textbf{A3.} 
\(\lbrack 0,\ 2\rbrack\) kesmada joylashgan sonlarning onlik kasr yozuvida \(2\) raqami qatnashmagan barcha sonlar to'plamining Lebeg o'lchovini toping.
 \\
\textbf{B1.} 
\(\int_{E}^{}f(x)d\mu\) Lebeg integralini hisoblang, \(f(x) = \left\{ \begin{matrix}
\frac{x^{2}}{(x + 3)(x + 2)},\ x \in \mathbb{I} \cap \lbrack 2,\ 4\rbrack \\
3x^{2} - 2,\ x\mathbb{\in Q \cap}\lbrack 2,\ 4\rbrack,\ E = \lbrack 2,\ 4\rbrack
\end{matrix} \right.\ \)
 \\
\textbf{B2.} 
Quyida berilganlar bo'yicha\(\ x,y \in X\) elementlar orasidagi masofani toping: \(X = C\left\lbrack \frac{\pi}{6};\ \frac{\pi}{3} \right\rbrack,\ \rho(x,y) = \max_{\frac{\pi}{6} \leq t \leq \frac{\pi}{3}}|x(t) - y(t)|,x(t) = \ctg (t + \pi/6),\ y = tg\ t\)
 \\
\textbf{B3.} 
\(A\) va \(B\) to'plamlari orasida o'zaro bir qiymatli moslik o'rnating.\(\ A = \lbrack - 4;4\rbrack\), \(B = ( - 11;3)\).
 \\
\textbf{C1.} 
To'plamning Lebeg o'lchovini toping: \(A = \bigcup_{k = 1}^{\infty}\left( k - 2^{- k},k + \frac{1}{k!} \right)\);
 \\
\textbf{C2.} 
Lebeg integralini (\(\int_{A}^{}{f(x)d\mu}\)) hisoblang: \(f(x) = sign(2x + 1)\), \(A = ( - 1;1\rbrack\).
 \\
\textbf{C3.} 
[8;11] to'plamida o'lchovsiz to'plamga misol keltiring.
 \\

\end{tabular}
\vspace{1cm}


\begin{tabular}{m{17cm}}
\textbf{23-bilet}

\vspace{0.5cm}

\textbf{T1.} 
Metrik fazo va unga misollar.
 \\
\textbf{T2.} 
O'lshovli funkciyalar va ularning xossalari.
 \\
\textbf{A1.} 
\(\ A = \{(x,y) \in \mathbb{R}^{2}:\ max\{|x|,|y|\} = 1\},\ B = \{(x,y) \in \mathbb{R}^{2}:\ x^{2} + y^{2} \leq 1\}\), \(A,\ B,\ A \cup B,\ A \cap B,\ A \backslash B,\ B \backslash A,\ A \bigtriangleup B\) to'plamlarini aniqlang va tasvirlang.
 \\
\textbf{A2.} 
\(\lbrack - 3;\ 2\rbrack\) va \(\lbrack 2;4) \cup \lbrack 5;8\rbrack\) to'plamlari orasida bir qiymatli moslik o'rnating.
 \\
\textbf{A3.} 
\(\lbrack 7,\ 9\rbrack\) kesmada joylashgan sonlarning onlik kasr yozuvida \(0\) raqami qatnashmagan barcha sonlar to'plamining Lebeg o'lchovini toping.
 \\
\textbf{B1.} 
\(\int_{E}^{}f(x)d\mu\) Lebeg integralini hisoblang, \(f(x) = \left\{ \begin{matrix}
\frac{x^{2}}{(x - 5)(x - 6)},\ x \in \mathbb{I} \cap \lbrack 0,\ 4\rbrack \\
3x^{2} - 2,\ x\mathbb{\in Q \cap}\lbrack 0,\ 4\rbrack,\ E = \lbrack 0,\ 4\rbrack
\end{matrix} \right.\ \)
 \\
\textbf{B2.} 
Quyida berilganlar bo'yicha\(\ x,y \in X\) elementlar orasidagi masofani toping: \(X = C\lbrack 0;\ \pi/4\rbrack,\ \rho(x,y) = \max_{0 \leq t \leq \pi/4}|x(t) - y(t)|,x(t) = \sin4t,\ y = \cos2t\)
 \\
\textbf{B3.} 
\(A\) va \(B\) to'plamlari orasida o'zaro bir qiymatli moslik o'rnating.\(\ A = \lbrack - 6;2\rbrack\), \(B = ( - 7;3)\).
 \\
\textbf{C1.} 
To'plamning Lebeg o'lchovini toping: \(A = \bigcup_{k = 1}^{\infty}\left\lbrack e^{- 2k},e^{- 2k + 1} \right)\).
 \\
\textbf{C2.} 
Lebeg integralini (\(\int_{A}^{}{f(x)d\mu}\)) hisoblang: \(f(x) = \lbrack x\rbrack - 1\), \(A = \lbrack - 1;3\rbrack\);
 \\
\textbf{C3.} 
[0;3] to'plamida o'lchovsiz to'plamga misol keltiring.
 \\

\end{tabular}
\vspace{1cm}


\begin{tabular}{m{17cm}}
\textbf{24-bilet}

\vspace{0.5cm}

\textbf{T1.} 
Metrik fazolarda ochiq va yopiq to'plamlar.
 \\
\textbf{T2.} 
Lebeg va Riss teoremalari.
 \\
\textbf{A1.} 
\(A = \{(x,y) \in \mathbb{R}^{2}:\ y = - x^{2}\},\ B = \{(x,y) \in \mathbb{R}^{2}:\ (x - 1)^{2} + (y - 1)^{2} \leq 1\}\), \(A,\ B,\ A \cup B,\ A \cap B,\ A \backslash B,\ B \backslash A,\ A \bigtriangleup B\) to'plamlarini aniqlang va tasvirlang.
 \\
\textbf{A2.} 
\(\lbrack 2;\ 5\rbrack\) va \(\lbrack 0;1) \cup \lbrack 3;\ 5\rbrack\) to'plamlari orasida bir qiymatli moslik o'rnating.
 \\
\textbf{A3.} 
\(\lbrack 6,\ 8\rbrack\) kesmada joylashgan sonlarning onlik kasr yozuvida \(9\) raqami qatnashmagan barcha sonlar to'plamining Lebeg o'lchovini toping.
 \\
\textbf{B1.} 
\(\int_{E}^{}f(x)d\mu\) Lebeg integralini hisoblang,\(\ f(x) = \left\{ \begin{matrix}
\frac{x^{2}}{(x + 2)(x + 4)},\ x \in \mathbb{I} \cap \lbrack 2,\ 4\rbrack \\
4x^{3},\ x\mathbb{\in Q \cap}\lbrack 2,\ 4\rbrack,\ E = \lbrack 2,\ 4\rbrack
\end{matrix} \right.\ \)
 \\
\textbf{B2.} 
Quyida berilganlar bo'yicha\(\ x,y \in X\) elementlar orasidagi masofani toping: \(X = C\lbrack 0,\pi\rbrack,\ \rho(x,y) = \max_{0 \leq t \leq \pi}|x(t) - y(t)|,x(t) = \sin2t,\ y = \cos4t\).
 \\
\textbf{B3.} 
\(A\) va \(B\) to'plamlari orasida o'zaro bir qiymatli moslik o'rnating.\(\ A = \lbrack - 1;7)\), \(B = \lbrack - 3;9\rbrack\).
 \\
\textbf{C1.} 
To'plamning Lebeg o'lchovini toping: \(A = \bigcup_{k = 1}^{\infty}\left( \frac{1}{k + 1},\frac{1}{k} \right)\);
 \\
\textbf{C2.} 
Lebeg integralini (\(\int_{A}^{}{f(x)d\mu}\)) hisoblang: \(f(x) = \frac{1}{\lbrack x - 1\rbrack}\), \(A = (3;6)\);
 \\
\textbf{C3.} 
[5;8] to'plamida o'lchovsiz to'plamga misol keltiring.
 \\

\end{tabular}
\vspace{1cm}


\begin{tabular}{m{17cm}}
\textbf{25-bilet}

\vspace{0.5cm}

\textbf{T1.} 
Kompakt metrik fazolar.
 \\
\textbf{T2.} 
Tekislikta elementar to'plamlar va ularning o'lshovi.
 \\
\textbf{A1.} 
\(A = \{(x,y) \in \mathbb{R}^{2}:\ xy \leq 0\},\ B = \{(x,y) \in \mathbb{R}^{2}:\ x^{2} + (y + 1)^{2} \geq 1\}\), \(A,\ B,\ A \cup B,\ A \cap B,\ A \backslash B,\ B \backslash A,\ A \bigtriangleup B\) to'plamlarini aniqlang va tasvirlang.
 \\
\textbf{A2.} 
\(( - 1;5\rbrack\) va \(( - 1;\ 1\rbrack \cup (3;\ 7\rbrack\) to'plamlari orasida bir qiymatli moslik o'rnating.
 \\
\textbf{A3.} 
\(\lbrack 8,\ 10\rbrack\) kesmada joylashgan sonlarning onlik kasr yozuvida \(0\) raqami qatnashmagan barcha sonlar to'plamining Lebeg o'lchovini toping.
 \\
\textbf{B1.} 
\(\int_{E}^{}f(x)d\mu\) Lebeg integralini hisoblang, \(f(x) = \left\{ \begin{matrix}
\frac{x^{2}}{(x - 2)(x - 4)},\ x \in \mathbb{I} \cap \lbrack - 1;1\rbrack \\
3x^{2} - 2,\ x\mathbb{\in Q \cap}\lbrack - 1;1\rbrack,\ E = \lbrack - 1;1\rbrack
\end{matrix} \right.\ \)
 \\
\textbf{B2.} 
Quyida berilganlar bo'yicha\(\ x,y \in X\) elementlar orasidagi masofani toping: \(X = C\left\lbrack \frac{\pi}{6};\ \frac{\pi}{4} \right\rbrack,\ \rho(x,y) = \max_{\frac{\pi}{6} \leq t \leq \frac{\pi}{4}}|x(t) - y(t)|,x(t) = \ctg (2t - \pi/6),\ y = tg(\ 2t - \pi/6)\)
 \\
\textbf{B3.} 
\(A\) va \(B\) to'plamlari orasida o'zaro bir qiymatli moslik o'rnating.\(\ A = \lbrack - 1;4)\), \(B = \lbrack - 1;7\rbrack\).
 \\
\textbf{C1.} 
To'plamning Lebeg o'lchovini toping: \(A = \bigcup_{k = 1}^{\infty}\left( k,k + \frac{3}{k(k + 1)} \right)\);
 \\
\textbf{C2.} 
Lebeg integralini (\(\int_{A}^{}{f(x)d\mu}\)) hisoblang: \(f(x) = sign(x - 1)\), \(A = \lbrack - 1;2)\);
 \\
\textbf{C3.} 
[-1;2] to'plamida o'lchovsiz to'plamga misol keltiring.
 \\

\end{tabular}
\vspace{1cm}


\begin{tabular}{m{17cm}}
\textbf{26-bilet}

\vspace{0.5cm}

\textbf{T1.} 
Metrik fazolarning uzliksiz akslantirishlari.
 \\
\textbf{T2.} 
Lebeg va Riss teoremalari.
 \\
\textbf{A1.} 
\(A = \{(x,y) \in \mathbb{R}^{2}:\ x^{2} = y\},\ B = \{(x,y) \in \mathbb{R}^{2}:\ x^{2} + y^{2} \geq 4\}\), \(A,\ B,\ A \cup B,\ A \cap B,\ A \backslash B,\ B \backslash A,\ A \bigtriangleup B\) to'plamlarini aniqlang va tasvirlang.
 \\
\textbf{A2.} 
\((0;6\rbrack\) va \((2;4) \cup \lbrack 7;11\rbrack\) to'plamlari orasida bir qiymatli moslik o'rnating.
 \\
\textbf{A3.} 
\(\lbrack 4,\ 6\rbrack\) kesmada joylashgan sonlarning onlik kasr yozuvida \(6\) raqami qatnashmagan barcha sonlar to'plamining Lebeg o'lchovini toping.
 \\
\textbf{B1.} 
\(\int_{E}^{}f(x)d\mu\) Lebeg integralini hisoblang, \(f(x) = \left\{ \begin{matrix}
\frac{x^{2}}{(x - 5)(x - 6)},\ x \in \mathbb{I} \cap \lbrack 0,\ 4\rbrack \\
3x^{2} - 2,\ x\mathbb{\in Q \cap}\lbrack 0,\ 4\rbrack,\ E = \lbrack 0,\ 4\rbrack
\end{matrix} \right.\ \)
 \\
\textbf{B2.} 
Quyida berilganlar bo'yicha\(\ x,y \in X\) elementlar orasidagi masofani toping: \(X = C\left\lbrack \frac{\pi}{6};\ \frac{\pi}{4} \right\rbrack,\ \rho(x,y) = \max_{\frac{\pi}{6} \leq t \leq \frac{\pi}{4}}|x(t) - y(t)|,x(t) = \ctg t,\ y = tg(\ 2t - \frac{\pi}{6})\)
 \\
\textbf{B3.} 
\(A\) va \(B\) to'plamlari orasida o'zaro bir qiymatli moslik o'rnating.\(\ A = \lbrack - 2;4\rbrack\), \(B = ( - 1;9)\).
 \\
\textbf{C1.} 
To'plamning Lebeg o'lchovini toping: \(A = \bigcup_{k = 1}^{\infty}\left( \frac{1}{2k + 1},\frac{1}{2k} \right)\);
 \\
\textbf{C2.} 
Lebeg integralini (\(\int_{A}^{}{f(x)d\mu}\)) hisoblang: \(f(x) = 2^{\lbrack 2x\rbrack}\), \(A = \lbrack 0;1)\);
 \\
\textbf{C3.} 
[-2;1] to'plamida o'lchovsiz to'plamga misol keltiring.
 \\

\end{tabular}
\vspace{1cm}


\begin{tabular}{m{17cm}}
\textbf{27-bilet}

\vspace{0.5cm}

\textbf{T1.} 
To'plam quvvati va uning xossalari.
 \\
\textbf{T2.} 
O'lshovli funkciyalar va ularning xossalari.
 \\
\textbf{A1.} 
\(A = \{(x,y) \in \mathbb{R}^{2}:\ y = - x^{2}\},\ B = \{(x,y) \in \mathbb{R}^{2}:\ (x + 1)^{2} + (y + 1)^{2} \leq 1\}\), \(A,\ B,\ A \cup B,\ A \cap B,\ A \backslash B,\ B \backslash A,\ A \bigtriangleup B\) to'plamlarini aniqlang va tasvirlang.
 \\
\textbf{A2.} 
\(\lbrack 1;6\rbrack\) va \(\lbrack 1;4) \cup \lbrack 7;9\rbrack\) to'plamlari orasida bir qiymatli moslik o'rnating.
 \\
\textbf{A3.} 
\(\lbrack 3,\ 5\rbrack\) kesmada joylashgan sonlarning onlik kasr yozuvida \(6\) raqami qatnashmagan barcha sonlar to'plamining Lebeg o'lchovini toping.
 \\
\textbf{B1.} 
\(\int_{E}^{}f(x)d\mu\) Lebeg integralini hisoblang, \(f(x) = \left\{ \begin{matrix}
\frac{x^{2}}{(x - 5)(x - 7)},\ x \in \mathbb{I} \cap \lbrack 1,\ 4\rbrack \\
3x^{2} - 2,\ x\mathbb{\in Q \cap}\lbrack 1,\ 4\rbrack,\ E = \lbrack 1,\ 4\rbrack
\end{matrix} \right.\ \)
 \\
\textbf{B2.} 
Quyida berilganlar bo'yicha\(\ x,y \in X\) elementlar orasidagi masofani toping: \(X = C\lbrack 0;\ \pi/3\rbrack,\ \rho(x,y) = \max_{0 \leq t \leq \pi/3}|x(t) - y(t)|,x(t) = \sin t,\ y = \cos5t\)
 \\
\textbf{B3.} 
\(A\) va \(B\) to'plamlari orasida o'zaro bir qiymatli moslik o'rnating.\(\ A = ( - 5;3)\), \(B = \lbrack - 10;3\rbrack\).
 \\
\textbf{C1.} 
To'plamning Lebeg o'lchovini toping: \(A = \bigcup_{k = 1}^{\infty}\left( k^{3},k^{3} + 3^{- k} \right)\);
 \\
\textbf{C2.} 
Lebeg integralini (\(\int_{A}^{}{f(x)d\mu}\)) hisoblang: \(f(x) = \frac{1}{\lbrack x\rbrack\lbrack x + 1\rbrack}\), \(A = \lbrack 1;3\rbrack\);
 \\
\textbf{C3.} 
[-6;-3] to'plamida o'lchovsiz to'plamga misol keltiring.
 \\

\end{tabular}
\vspace{1cm}


\begin{tabular}{m{17cm}}
\textbf{28-bilet}

\vspace{0.5cm}

\textbf{T1.} 
Metrik fazo va unga misollar.
 \\
\textbf{T2.} 
Egorov teoremasi.
 \\
\textbf{A1.} 
\(A = \{(x,y) \in \mathbb{R}^{2}:\ max\{|x|,|y|\} \leq 2\},\ B = \{(x,y) \in \mathbb{R}^{2}:\ x^{2} + 1 \leq y\}\), \(A,\ B,\ A \cup B,\ A \cap B,\ A \backslash B,\ B \backslash A,\ A \bigtriangleup B\) to'plamlarini aniqlang va tasvirlang.
 \\
\textbf{A2.} 
\((3;6\rbrack\) va \(( - 3; - 1) \cup \lbrack 2;3\rbrack\) to'plamlari orasida bir qiymatli moslik o'rnating.
 \\
\textbf{A3.} 
\(\lbrack 5,\ 7\rbrack\) kesmada joylashgan sonlarning onlik kasr yozuvida \(8\) raqami qatnashmagan barcha sonlar to'plamining Lebeg o'lchovini toping.
 \\
\textbf{B1.} 
\(\int_{E}^{}f(x)d\mu\) Lebeg integralini hisoblang, \(E = \lbrack 0,\ 1\rbrack\), \(f(x) = \left\{ \begin{matrix}
\frac{1}{\sqrt{x}},\ x \in \mathbb{I} \cap \lbrack 0,\ 1\rbrack \\
\sin x,\ x\mathbb{\in Q}
\end{matrix} \right.\ \)
 \\
\textbf{B2.} 
Quyida berilganlar bo'yicha\(\ x,y \in X\) elementlar orasidagi masofani toping: \(X = C\left\lbrack \frac{\pi}{4};\ \frac{\pi}{2} \right\rbrack,\ \rho(x,y) = \max_{\frac{\pi}{4} \leq t \leq \frac{\pi}{2}}|x(t) - y(t)|,x(t) = \ctg (2t - \pi/6),\ y = tg(\ t - \pi/6)\)
 \\
\textbf{B3.} 
\(A\) va \(B\) to'plamlari orasida o'zaro bir qiymatli moslik o'rnating.\(\ A = ( - 2;3\rbrack\), \(B = \lbrack - 2;8\rbrack\).
 \\
\textbf{C1.} 
To'plamning Lebeg o'lchovini toping: \(A = \bigcup_{k = 1}^{\infty}\left( 2k - 2^{- k},2k + \frac{1}{k!} \right)\);
 \\
\textbf{C2.} 
Lebeg integralini (\(\int_{A}^{}{f(x)d\mu}\)) hisoblang: \(f(x) = \frac{1}{\lbrack x\rbrack}\), \(A = (1;4)\);
 \\
\textbf{C3.} 
[-5;-2] to'plamida o'lchovsiz to'plamga misol keltiring.
 \\

\end{tabular}
\vspace{1cm}


\begin{tabular}{m{17cm}}
\textbf{29-bilet}

\vspace{0.5cm}

\textbf{T1.} 
Metrik fazolarda ochiq va yopiq to'plamlar.
 \\
\textbf{T2.} 
Egorov teoremasi.
 \\
\textbf{A1.} 
\(A = \{(x,y) \in \mathbb{R}^{2}:\ xy \leq 0\},\ B = \{(x,y) \in \mathbb{R}^{2}:\ x^{2} + (y + 1)^{2} \geq 1\}\), \(A,\ B,\ A \cup B,\ A \cap B,\ A \backslash B,\ B \backslash A,\ A \bigtriangleup B\) to'plamlarini aniqlang va tasvirlang.
 \\
\textbf{A2.} 
\(\lbrack - 2;5\rbrack\) va \(\lbrack 2;4\rbrack \cup (7;12\rbrack\) to'plamlari orasida bir qiymatli moslik o'rnating.
 \\
\textbf{A3.} 
\(\lbrack 4,\ 6\rbrack\) kesmada joylashgan sonlarning onlik kasr yozuvida \(7\) raqami qatnashmagan barcha sonlar to'plamining Lebeg o'lchovini toping.
 \\
\textbf{B1.} 
\(\int_{E}^{}f(x)d\mu\) Lebeg integralini hisoblang, \(f(x) = \left\{ \begin{matrix}
\frac{x^{2}}{(x - 2)(x - 4)},\ x \in \mathbb{I} \cap \lbrack - 4; - 1\rbrack \\
3x^{2} - 2,\ x\mathbb{\in Q \cap}\lbrack - 4; - 1\rbrack,E = \lbrack - 4; - 1\rbrack
\end{matrix} \right.\ \)
 \\
\textbf{B2.} 
Quyida berilganlar bo'yicha\(\ x,y \in X\) elementlar orasidagi masofani toping: \(X = C\left\lbrack \frac{\pi}{4};\ \frac{\pi}{3} \right\rbrack,\ \rho(x,y) = \max_{\frac{\pi}{4} \leq t \leq \frac{\pi}{3}}|x(t) - y(t)|,x(t) = \ctg (2t + \pi/6),\ y = tg(\ t - \pi/6)\)
 \\
\textbf{B3.} 
\(A\) va \(B\) to'plamlari orasida o'zaro bir qiymatli moslik o'rnating.\(\ A = ( - 2;4)\), \(B = \lbrack 2;10)\).
 \\
\textbf{C1.} 
To'plamning Lebeg o'lchovini toping: \(A = \bigcup_{k = 1}^{\infty}\left( k^{2},k^{2} + 2^{- k} \right)\);
 \\
\textbf{C2.} 
Lebeg integralini (\(\int_{A}^{}{f(x)d\mu}\)) hisoblang: \(f(x) = 2^{\lbrack x\rbrack}\), \(A = ( - 2;2)\);
 \\
\textbf{C3.} 
[-9;-6] to'plamida o'lchovsiz to'plamga misol keltiring.
 \\

\end{tabular}
\vspace{1cm}


\begin{tabular}{m{17cm}}
\textbf{30-bilet}

\vspace{0.5cm}

\textbf{T1.} To'plamlar va ular ustida amallar.
 \\
\textbf{T2.} 
Lebeg va Riss teoremalari.
 \\
\textbf{A1.} 
\(\ A = \{(x,y) \in \mathbb{R}^{2}:\ y = x^{2}\},\ B = \{(x,y) \in \mathbb{R}^{2}:\ (x - 1)^{2} + (y - 1)^{2} \leq 4\}\), \(A,\ B,\ A \cup B,\ A \cap B,\ A \backslash B,\ B \backslash A,\ A \bigtriangleup B\) to'plamlarini aniqlang va tasvirlang.
 \\
\textbf{A2.} 
\(\lbrack 3;\ 7\rbrack\) va \(\lbrack 0;\ 2) \cup \lbrack 6;\ 8\rbrack\) to'plamlari orasida bir qiymatli moslik o'rnating.
 \\
\textbf{A3.} 
\(\lbrack 5,\ 7\rbrack\) kesmada joylashgan sonlarning onlik kasr yozuvida \(7\) raqami qatnashmagan barcha sonlar to'plamining Lebeg o'lchovini toping.
 \\
\textbf{B1.} 
\(\int_{E}^{}f(x)d\mu\) Lebeg integralini hisoblang, \(f(x) = \left\{ \begin{matrix}
\frac{x^{2}}{(x + 2)(x + 4)},\ x \in \mathbb{I} \cap \lbrack 0,\ 4\rbrack \\
3x^{2} - 2,\ x\mathbb{\in Q \cap}\lbrack 0,\ 4\rbrack,\ E = \lbrack 0,\ 4\rbrack
\end{matrix} \right.\ \)
 \\
\textbf{B2.} 
Quyida berilganlar bo'yicha\(\ x,y \in X\) elementlar orasidagi masofani toping: \(X = C\lbrack 0;\ \pi/4\rbrack,\ \rho(x,y) = \max_{0 \leq t \leq \pi/4}|x(t) - y(t)|,x(t) = \sin t,\ y = \cos3t\)
 \\
\textbf{B3.} 
\(A\) va \(B\) to'plamlari orasida o'zaro bir qiymatli moslik o'rnating.\(\ A = \lbrack - 5;4)\), \(B = \lbrack - 3;11\rbrack\).
 \\
\textbf{C1.} 
To'plamning Lebeg o'lchovini toping: \(A = \bigcup_{k = 1}^{\infty}\left( k,k + \frac{2}{k(k + 1)} \right)\);
 \\
\textbf{C2.} 
Lebeg integralini (\(\int_{A}^{}{f(x)d\mu}\)) hisoblang: \(f(x) = 2 - \lbrack x\rbrack\), \(A = \lbrack - 2;3)\);
 \\
\textbf{C3.} 
[0;3] to'plamida o'lchovsiz to'plamga misol keltiring.
 \\

\end{tabular}
\vspace{1cm}


\begin{tabular}{m{17cm}}
\textbf{31-bilet}

\vspace{0.5cm}

\textbf{T1.} 
Metrik fazolarning uzliksiz akslantirishlari.
 \\
\textbf{T2.} 
Tekislikta elementar to'plamlar va ularning o'lshovi.
 \\
\textbf{A1.} 
\(A = \{(x,y) \in \mathbb{R}^{2}:\ x \geq y\},\ B = \{(x,y) \in \mathbb{R}^{2}:\ 9x^{2} + y^{2} \leq 36\}\),\(A,\ B,\ A \cup B,\ A \cap B,\ A \backslash B,\ B \backslash A,\ A \bigtriangleup B\) to'plamlarini aniqlang va tasvirlang.
 \\
\textbf{A2.} 
\(\lbrack - 1;\ 5)\) va \(\lbrack - 1;4) \cup \lbrack 7;8)\) to'plamlari orasida bir qiymatli moslik o'rnating.
 \\
\textbf{A3.} 
\(\lbrack 8,\ 10\rbrack\) kesmada joylashgan sonlarning onlik kasr yozuvida \(6\) raqami qatnashmagan barcha sonlar to'plamining Lebeg o'lchovini toping.
 \\
\textbf{B1.} 
\(\int_{E}^{}f(x)d\mu\) Lebeg integralini hisoblang, \(f(x) = \left\{ \begin{matrix}
\frac{x^{2}}{(x + 2)(x + 4)},\ x \in \mathbb{I} \cap \lbrack 0,\ 4\rbrack \\
3x^{2} - 2,\ x\mathbb{\in Q \cap}\lbrack 0,\ 4\rbrack,\ E = \lbrack 0,\ 4\rbrack
\end{matrix} \right.\ \)
 \\
\textbf{B2.} 
Quyida berilganlar bo'yicha\(\ x,y \in X\) elementlar orasidagi masofani toping: \(X = C\left\lbrack \frac{\pi}{6};\ \frac{\pi}{4} \right\rbrack,\ \rho(x,y) = \max_{\frac{\pi}{6} \leq t \leq \frac{\pi}{4}}|x(t) - y(t)|,x(t) = \ctg t,\ y = tg(\ 2t - \frac{\pi}{6})\)
 \\
\textbf{B3.} 
\(A\) va \(B\) to'plamlari orasida o'zaro bir qiymatli moslik o'rnating.\(\ A = ( - 4;3\rbrack\), \(B = \lbrack - 4;10\rbrack\).
 \\
\textbf{C1.} 
To'plamning Lebeg o'lchovini toping: \(A = \bigcup_{k = 1}^{\infty}\left( \frac{1}{2k},\frac{1}{k} \right)\);
 \\
\textbf{C2.} 
Lebeg integralini (\(\int_{A}^{}{f(x)d\mu}\)) hisoblang: \(f(x) = \frac{( - 1)^{\lbrack x\rbrack}}{\lbrack x\rbrack}\), \(A = \lbrack 1;4)\);
 \\
\textbf{C3.} 
[8;11] to'plamida o'lchovsiz to'plamga misol keltiring.
 \\

\end{tabular}
\vspace{1cm}


\begin{tabular}{m{17cm}}
\textbf{32-bilet}

\vspace{0.5cm}

\textbf{T1.} 
Kompakt metrik fazolar.
 \\
\textbf{T2.} 
O'lshovli funkciyalar va ularning xossalari.
 \\
\textbf{A1.} 
\(A = \{(x,y) \in \mathbb{R}^{2}:\ max\{|x|,|y|\} \leq 2\},\ B = \{(x,y) \in \mathbb{R}^{2}:\ 4 - x^{2} \geq y\}\), \(A,\ B,\ A \cup B,\ A \cap B,\ A \backslash B,\ B \backslash A,\ A \bigtriangleup B\) to'plamlarini aniqlang va tasvirlang.
 \\
\textbf{A2.} 
\(( - 3;\ 4\rbrack\) va \((1;4\rbrack \cup (6;10\rbrack\) to'plamlari orasida bir qiymatli moslik o'rnating.
 \\
\textbf{A3.} 
\(\lbrack 0,\ 2\rbrack\) kesmada joylashgan sonlarning onlik kasr yozuvida \(3\) raqami qatnashmagan barcha sonlar to'plamining Lebeg o'lchovini toping.
 \\
\textbf{B1.} 
\(\int_{E}^{}f(x)d\mu\) Lebeg integralini hisoblang, \(f(x) = \left\{ \begin{matrix}
\frac{x^{2}}{(x - 2)(x - 4)},\ x \in \mathbb{I} \cap \lbrack - 4; - 1\rbrack \\
3x^{2} - 2,\ x\mathbb{\in Q \cap}\lbrack - 4; - 1\rbrack,E = \lbrack - 4; - 1\rbrack
\end{matrix} \right.\ \)
 \\
\textbf{B2.} 
Quyida berilganlar bo'yicha\(\ x,y \in X\) elementlar orasidagi masofani toping: \(X = C\left\lbrack \frac{\pi}{6};\ \frac{\pi}{3} \right\rbrack,\ \rho(x,y) = \max_{\frac{\pi}{6} \leq t \leq \frac{\pi}{3}}|x(t) - y(t)|,x(t) = \ctg (t + \pi/6),\ y = tg\ t\)
 \\
\textbf{B3.} 
\(A\) va \(B\) to'plamlari orasida o'zaro bir qiymatli moslik o'rnating.\(\ A = ( - 3;5)\), \(B = \lbrack - 8;6)\).
 \\
\textbf{C1.} 
To'plamning Lebeg o'lchovini toping: \(A = \bigcup_{k = 1}^{\infty}\left( \frac{1}{k + 2},\frac{1}{k} \right)\);
 \\
\textbf{C2.} 
Lebeg integralini (\(\int_{A}^{}{f(x)d\mu}\)) hisoblang: \(f(x) = \frac{1}{\lbrack x\rbrack!}\), \(A = \lbrack 0;4)\);
 \\
\textbf{C3.} 
[-3;0] to'plamida o'lchovsiz to'plamga misol keltiring.
 \\

\end{tabular}
\vspace{1cm}


\begin{tabular}{m{17cm}}
\textbf{33-bilet}

\vspace{0.5cm}

\textbf{T1.} 
Metrik fazolarda ochiq va yopiq to'plamlar.
 \\
\textbf{T2.} 
Tekislikta elementar to'plamlar va ularning o'lshovi.
 \\
\textbf{A1.} 
\(A = \{(x,y) \in \mathbb{R}^{2}:\ y = x^{2}\},\ B = \{(x,y) \in \mathbb{R}^{2}:\ x^{2} + (y - 1)^{2} \leq 1\}\), \(A,\ B,\ A \cup B,\ A \cap B,\ A \backslash B,\ B \backslash A,\ A \bigtriangleup B\) to'plamlarini aniqlang va tasvirlang.
 \\
\textbf{A2.} 
\(\lbrack 2;\ 7)\) va \(\lbrack - 2; - 1) \cup \lbrack 2;4)\) to'plamlari orasida bir qiymatli moslik o'rnating.
 \\
\textbf{A3.} 
\(\lbrack 1,\ 3\rbrack\) kesmada joylashgan sonlarning onlik kasr yozuvida \(4\) raqami qatnashmagan barcha sonlar to'plamining Lebeg o'lchovini toping.
 \\
\textbf{B1.} 
\(\int_{E}^{}f(x)d\mu\) Lebeg integralini hisoblang, \(f(x) = \left\{ \begin{matrix}
\frac{x^{2}}{(x - 5)(x - 6)},\ x \in \mathbb{I} \cap \lbrack 0,\ 4\rbrack \\
3x^{2} - 2,\ x\mathbb{\in Q \cap}\lbrack 0,\ 4\rbrack,\ E = \lbrack 0,\ 4\rbrack
\end{matrix} \right.\ \)
 \\
\textbf{B2.} 
Quyida berilganlar bo'yicha\(\ x,y \in X\) elementlar orasidagi masofani toping: \(X = C\lbrack 0;\ \pi/3\rbrack,\ \rho(x,y) = \max_{0 \leq t \leq \pi/3}|x(t) - y(t)|,x(t) = \sin t,\ y = \cos5t\)
 \\
\textbf{B3.} 
\(A\) va \(B\) to'plamlari orasida o'zaro bir qiymatli moslik o'rnating.\(\ A = ( - 1;3)\), \(B = \lbrack 0;9\rbrack\).
 \\
\textbf{C1.} 
To'plamning Lebeg o'lchovini toping: \(A = \bigcup_{k = 1}^{\infty}\left( k,k + \frac{2}{k(k + 1)} \right)\);
 \\
\textbf{C2.} 
Lebeg integralini (\(\int_{A}^{}{f(x)d\mu}\)) hisoblang: \(f(x) = \frac{1}{\lbrack x + 1\rbrack}\), \(A = \lbrack 1;5)\);
 \\
\textbf{C3.} 
[0;3] to'plamida o'lchovsiz to'plamga misol keltiring.
 \\

\end{tabular}
\vspace{1cm}


\begin{tabular}{m{17cm}}
\textbf{34-bilet}

\vspace{0.5cm}

\textbf{T1.} To'plamlar va ular ustida amallar.
 \\
\textbf{T2.} 
Egorov teoremasi.
 \\
\textbf{A1.} 
\(\ A = \{(x,y) \in \mathbb{R}^{2}:\ xy \leq 0\},\ B = \{(x,y) \in \mathbb{R}^{2}:\ |x| + |y| \geq 1\}\), \(A,\ B,\ A \cup B,\ A \cap B,\ A \backslash B,\ B \backslash A,\ A \bigtriangleup B\) to'plamlarini aniqlang va tasvirlang.
 \\
\textbf{A2.} 
\(\lbrack - 2;3)\) va \(\lbrack - 3;1) \cup \lbrack 2;3)\) to'plamlari orasida bir qiymatli moslik o'rnating.
 \\
\textbf{A3.} 
\(\lbrack 6,\ 8\rbrack\) kesmada joylashgan sonlarning onlik kasr yozuvida \(8\) raqami qatnashmagan barcha sonlar to'plamining Lebeg o'lchovini toping.
 \\
\textbf{B1.} 
\(\int_{E}^{}f(x)d\mu\) Lebeg integralini hisoblang, \(f(x) = \left\{ \begin{matrix}
\frac{x^{2}}{(x - 5)(x - 7)},\ x \in \mathbb{I} \cap \lbrack 1,\ 4\rbrack \\
3x^{2} - 2,\ x\mathbb{\in Q \cap}\lbrack 1,\ 4\rbrack,\ E = \lbrack 1,\ 4\rbrack
\end{matrix} \right.\ \)
 \\
\textbf{B2.} 
Quyida berilganlar bo'yicha\(\ x,y \in X\) elementlar orasidagi masofani toping: \(X = C\lbrack 0;\ \pi/6\rbrack,\ \rho(x,y) = \max_{0 \leq t \leq \pi/6}|x(t) - y(t)|,x(t) = \sin3t,\ y = \cos t\)
 \\
\textbf{B3.} 
\(A\) va \(B\) to'plamlari orasida o'zaro bir qiymatli moslik o'rnating.\(\ A = \lbrack - 7;3)\), \(B = \lbrack - 5;7\rbrack\).
 \\
\textbf{C1.} 
To'plamning Lebeg o'lchovini toping: \(A = \bigcup_{k = 1}^{\infty}\left( k - 2^{- k},k + \frac{1}{k!} \right)\);
 \\
\textbf{C2.} 
Lebeg integralini (\(\int_{A}^{}{f(x)d\mu}\)) hisoblang: \(f(x) = \lbrack x + 1\rbrack\), \(A = \lbrack - 2;1)\);
 \\
\textbf{C3.} 
[-4;-1] to'plamida o'lchovsiz to'plamga misol keltiring.
 \\

\end{tabular}
\vspace{1cm}


\begin{tabular}{m{17cm}}
\textbf{35-bilet}

\vspace{0.5cm}

\textbf{T1.} 
Metrik fazo va unga misollar.
 \\
\textbf{T2.} 
Lebeg va Riss teoremalari.
 \\
\textbf{A1.} 
\(A = \{(x,y) \in \mathbb{R}^{2}:\ xy \geq 0\},\ B = \{(x,y) \in \mathbb{R}^{2}:\ x^{2} + y^{2} \geq 1\}\), \(A,\ B,\ A \cup B,\ A \cap B,\ A \backslash B,\ B \backslash A,\ A \bigtriangleup B\) to'plamlarini aniqlang va tasvirlang.
 \\
\textbf{A2.} 
\(\lbrack - 4;0)\) va \(\lbrack 0;3) \cup \lbrack 5;6)\) to'plamlari orasida bir qiymatli moslik o'rnating.
 \\
\textbf{A3.} 
\(\lbrack 1,\ 3\rbrack\) kesmada joylashgan sonlarning onlik kasr yozuvida \(3\) raqami qatnashmagan barcha sonlar to'plamining Lebeg o'lchovini toping.
 \\
\textbf{B1.} 
\(\int_{E}^{}f(x)d\mu\) Lebeg integralini hisoblang, \(E = \lbrack 0,\ 1\rbrack\), \(f(x) = \left\{ \begin{matrix}
\frac{1}{\sqrt{x}},\ x \in \mathbb{I} \cap \lbrack 0,\ 1\rbrack \\
\sin x,\ x\mathbb{\in Q}
\end{matrix} \right.\ \)
 \\
\textbf{B2.} 
Quyida berilganlar bo'yicha\(\ x,y \in X\) elementlar orasidagi masofani toping: \(X = C\left\lbrack \frac{\pi}{6};\ \frac{\pi}{4} \right\rbrack,\ \rho(x,y) = \max_{\frac{\pi}{6} \leq t \leq \frac{\pi}{4}}|x(t) - y(t)|,x(t) = \ctg (2t - \pi/6),\ y = tg(\ 2t - \pi/6)\)
 \\
\textbf{B3.} 
\(A\) va \(B\) to'plamlari orasida o'zaro bir qiymatli moslik o'rnating.\(\ A = ( - 3;4)\), \(B = \lbrack - 2;10)\).
 \\
\textbf{C1.} 
To'plamning Lebeg o'lchovini toping: \(A = \bigcup_{k = 1}^{\infty}\left( \frac{1}{3^{k}},\frac{1}{3^{k - 1}} \right)\);
 \\
\textbf{C2.} 
Lebeg integralini (\(\int_{A}^{}{f(x)d\mu}\)) hisoblang: \(f(x) = 2\lbrack x\rbrack\), \(A = ( - 3;3)\);
 \\
\textbf{C3.} 
[2;5] to'plamida o'lchovsiz to'plamga misol keltiring.
 \\

\end{tabular}
\vspace{1cm}


\begin{tabular}{m{17cm}}
\textbf{36-bilet}

\vspace{0.5cm}

\textbf{T1.} 
To'plam quvvati va uning xossalari.
 \\
\textbf{T2.} 
O'lshovli funkciyalar va ularning xossalari.
 \\
\textbf{A1.} 
\(A = \{(x,y) \in \mathbb{R}^{2}:\ xy \geq 0\},\ B = \{(x,y) \in \mathbb{R}^{2}:\ |x| + |y - 2| \geq 1\}\), \(A,\ B,\ A \cup B,\ A \cap B,\ A \backslash B,\ B \backslash A,\ A \bigtriangleup B\) to'plamlarini aniqlang va tasvirlang.
 \\
\textbf{A2.} 
\(\lbrack - 2;\ 4\rbrack\) va \(\lbrack - 2;1) \cup \lbrack 2;5\rbrack\) to'plamlari orasida bir qiymatli moslik o'rnating.
 \\
\textbf{A3.} 
\(\lbrack 3,\ 4\rbrack\) kesmada joylashgan sonlarning onlik kasr yozuvida \(1\) raqami qatnashmagan barcha sonlar to'plamining Lebeg o'lchovini toping.
 \\
\textbf{B1.} 
\(\int_{E}^{}f(x)d\mu\) Lebeg integralini hisoblang, \(f(x) = \left\{ \begin{matrix}
\frac{x^{2}}{(x - 5)(x - 6)},\ x \in \mathbb{I} \cap \lbrack 0,\ 4\rbrack \\
3x^{2} - 2,\ x\mathbb{\in Q \cap}\lbrack 0,\ 4\rbrack,\ E = \lbrack 0,\ 4\rbrack
\end{matrix} \right.\ \)
 \\
\textbf{B2.} 
Quyida berilganlar bo'yicha\(\ x,y \in X\) elementlar orasidagi masofani toping: \(X = C\left\lbrack \frac{\pi}{4};\ \frac{\pi}{3} \right\rbrack,\ \rho(x,y) = \max_{\frac{\pi}{4} \leq t \leq \frac{\pi}{3}}|x(t) - y(t)|,x(t) = \ctg (2t + \pi/6),\ y = tg(\ t - \pi/6)\)
 \\
\textbf{B3.} 
\(A\) va \(B\) to'plamlari orasida o'zaro bir qiymatli moslik o'rnating.\(\ A = \lbrack - 2;4\rbrack\), \(B = ( - 5;5)\).
 \\
\textbf{C1.} 
To'plamning Lebeg o'lchovini toping: \(A = \bigcup_{k = 1}^{\infty}\left( k^{2},k^{2} + 2^{- k} \right)\);
 \\
\textbf{C2.} 
Lebeg integralini (\(\int_{A}^{}{f(x)d\mu}\)) hisoblang: \(f(x) = 2^{( - 1)^{\lbrack x\rbrack}}\), \(A = \lbrack 0;3)\);
 \\
\textbf{C3.} 
[-7;-4] to'plamida o'lchovsiz to'plamga misol keltiring.
 \\

\end{tabular}
\vspace{1cm}


\begin{tabular}{m{17cm}}
\textbf{37-bilet}

\vspace{0.5cm}

\textbf{T1.} 
Kompakt metrik fazolar.
 \\
\textbf{T2.} 
Tekislikta elementar to'plamlar va ularning o'lshovi.
 \\
\textbf{A1.} 
\(\ A = \{(x,y) \in \mathbb{R}^{2}:\ max\{|x|,|y|\} = 1\},\ B = \{(x,y) \in \mathbb{R}^{2}:\ x^{2} + y^{2} \leq 1\}\), \(A,\ B,\ A \cup B,\ A \cap B,\ A \backslash B,\ B \backslash A,\ A \bigtriangleup B\) to'plamlarini aniqlang va tasvirlang.
 \\
\textbf{A2.} 
\(\lbrack - 4;\ 1)\) va \(\lbrack - 3; - 1) \cup \lbrack 3;6)\) to'plamlari orasida bir qiymatli moslik o'rnating.
 \\
\textbf{A3.} 
\(\lbrack 3,\ 5\rbrack\) kesmada joylashgan sonlarning onlik kasr yozuvida \(5\) raqami qatnashmagan barcha sonlar to'plamining Lebeg o'lchovini toping.
 \\
\textbf{B1.} 
\(\int_{E}^{}f(x)d\mu\) Lebeg integralini hisoblang,\(\ f(x) = \left\{ \begin{matrix}
\frac{x^{2}}{(x + 2)(x + 4)},\ x \in \mathbb{I} \cap \lbrack 2,\ 4\rbrack \\
4x^{3},\ x\mathbb{\in Q \cap}\lbrack 2,\ 4\rbrack,\ E = \lbrack 2,\ 4\rbrack
\end{matrix} \right.\ \)
 \\
\textbf{B2.} 
Quyida berilganlar bo'yicha\(\ x,y \in X\) elementlar orasidagi masofani toping: \(X = C\lbrack 0;\ \pi/4\rbrack,\ \rho(x,y) = \max_{0 \leq t \leq \pi/4}|x(t) - y(t)|,x(t) = \sin t,\ y = \cos3t\)
 \\
\textbf{B3.} 
\(A\) va \(B\) to'plamlari orasida o'zaro bir qiymatli moslik o'rnating.\(\ A = ( - 1;4)\), \(B = \lbrack 2;12)\).
 \\
\textbf{C1.} 
To'plamning Lebeg o'lchovini toping: \(A = \bigcup_{k = 1}^{\infty}\left( k,k + \frac{1}{k!} \right)\);
 \\
\textbf{C2.} 
Lebeg integralini (\(\int_{A}^{}{f(x)d\mu}\)) hisoblang: \(f(x) = sign(x)\), \(A = \lbrack - 2;2)\);
 \\
\textbf{C3.} 
[3;6] to'plamida o'lchovsiz to'plamga misol keltiring.
 \\

\end{tabular}
\vspace{1cm}


\begin{tabular}{m{17cm}}
\textbf{38-bilet}

\vspace{0.5cm}

\textbf{T1.} 
Metrik fazolarda ochiq va yopiq to'plamlar.
 \\
\textbf{T2.} 
Lebeg va Riss teoremalari.
 \\
\textbf{A1.} 
\(A = \{(x,y) \in \mathbb{R}^{2}:\ x \leq y\},\ B = \{(x,y) \in \mathbb{R}^{2}:\ 9x^{2} + y^{2} \leq 9\}\), \(A,\ B,\ A \cup B,\ A \cap B,\ A \backslash B,\ B \backslash A,\ A \bigtriangleup B\) to'plamlarini aniqlang va tasvirlang.
 \\
\textbf{A2.} 
\(\lbrack - 3;\ 7\rbrack\) va \(\lbrack 2;5) \cup \lbrack 8;15\rbrack\) to'plamlari orasida bir qiymatli moslik o'rnating.
 \\
\textbf{A3.} 
\(\lbrack 0,\ 1\rbrack\) kesmada joylashgan sonlarning onlik kasr yozuvida \(1\) raqami qatnashmagan barcha sonlar to'plamining Lebeg o'lchovini toping.
 \\
\textbf{B1.} 
\(\int_{E}^{}f(x)d\mu\) Lebeg integralini hisoblang, \(E = \lbrack 0,\ 1\rbrack\), \(f(x) = \left\{ \begin{matrix}
\frac{1}{(x + 1)^{3}}\ x \in \mathbb{I} \cap \lbrack 0,\ 1\rbrack \\
7x,\ x\mathbb{\in Q}
\end{matrix} \right.\ \)
 \\
\textbf{B2.} 
Quyida berilganlar bo'yicha\(\ x,y \in X\) elementlar orasidagi masofani toping: \(X = C\lbrack 0;\ \pi/4\rbrack,\ \rho(x,y) = \max_{0 \leq t \leq \pi/4}|x(t) - y(t)|,x(t) = \sin4t,\ y = \cos2t\)
 \\
\textbf{B3.} 
\(A\) va \(B\) to'plamlari orasida o'zaro bir qiymatli moslik o'rnating.\(\ A = ( - 3;3)\), \(B = \lbrack - 1;9\rbrack\).
 \\
\textbf{C1.} 
To'plamning Lebeg o'lchovini toping: \(A = \bigcup_{k = 1}^{\infty}\left( \frac{1}{2k + 1},\frac{1}{2k} \right)\);
 \\
\textbf{C2.} 
Lebeg integralini (\(\int_{A}^{}{f(x)d\mu}\)) hisoblang: \(f(x) = \frac{1}{\lbrack x - 1\rbrack!}\), \(A = (1;3)\);
 \\
\textbf{C3.} 
[9;12] to'plamida o'lchovsiz to'plamga misol keltiring.
 \\

\end{tabular}
\vspace{1cm}


\begin{tabular}{m{17cm}}
\textbf{39-bilet}

\vspace{0.5cm}

\textbf{T1.} 
To'plam quvvati va uning xossalari.
 \\
\textbf{T2.} 
Egorov teoremasi.
 \\
\textbf{A1.} 
\(A = \{(x,y) \in \mathbb{R}^{2}:\ x = - y\},\ B = \{(x,y) \in \mathbb{R}^{2}:\ |x| + |y| \leq 2\}\), \(A,\ B,\ A \cup B,\ A \cap B,\ A \backslash B,\ B \backslash A,\ A \bigtriangleup B\) to'plamlarini aniqlang va tasvirlang.
 \\
\textbf{A2.} 
\(( - 1;5\rbrack\) va \(( - 1;\ 1\rbrack \cup (3;\ 7\rbrack\) to'plamlari orasida bir qiymatli moslik o'rnating.
 \\
\textbf{A3.} 
\(\lbrack 2,\ 4\rbrack\) kesmada joylashgan sonlarning onlik kasr yozuvida \(5\) raqami qatnashmagan barcha sonlar to'plamining Lebeg o'lchovini toping.
 \\
\textbf{B1.} 
\(\int_{E}^{}f(x)d\mu\) Lebeg integralini hisoblang, \(f(x) = \left\{ \begin{matrix}
\frac{x^{2}}{(x - 2)(x - 4)},\ x \in \mathbb{I} \cap \lbrack - 1;1\rbrack \\
3x^{2} - 2,\ x\mathbb{\in Q \cap}\lbrack - 1;1\rbrack,\ E = \lbrack - 1;1\rbrack
\end{matrix} \right.\ \)
 \\
\textbf{B2.} 
Quyida berilganlar bo'yicha\(\ x,y \in X\) elementlar orasidagi masofani toping: \(X = C\lbrack 0,\pi\rbrack,\ \rho(x,y) = \max_{0 \leq t \leq \pi}|x(t) - y(t)|,x(t) = \sin2t,\ y = \cos4t\).
 \\
\textbf{B3.} 
\(A\) va \(B\) to'plamlari orasida o'zaro bir qiymatli moslik o'rnating.\(\ A = ( - 5;1\rbrack\), \(B = \lbrack - 4;6\rbrack\).
 \\
\textbf{C1.} 
\(P = \{ 0 \leq x \leq 1,\ 0 \leq y \leq 1\}\) va \(Q = \{ 0.3 \leq x \leq 0.8,\ 0 \leq y \leq 1\}\) to'g'ri to'rtburchaklar kesishmasining o'lchovini toping.
 \\
\textbf{C2.} 
Lebeg integralini (\(\int_{A}^{}{f(x)d\mu}\)) hisoblang: \(f(x) = \frac{1}{\lbrack x\rbrack\lbrack x + 1\rbrack}\), \(A = \lbrack 1;3\rbrack\).
 \\
\textbf{C3.} 
[-12;-9] to'plamida o'lchovsiz to'plamga misol keltiring.
 \\

\end{tabular}
\vspace{1cm}


\begin{tabular}{m{17cm}}
\textbf{40-bilet}

\vspace{0.5cm}

\textbf{T1.} 
Metrik fazolarning uzliksiz akslantirishlari.
 \\
\textbf{T2.} 
O'lshovli funkciyalar va ularning xossalari.
 \\
\textbf{A1.} 
\(\ A = \{(x,y) \in \mathbb{R}^{2}:\ x \leq y\},\ B = \{(x,y) \in \mathbb{R}^{2}:\ 4x^{2} + 9y^{2} \geq 36\}\), \(A,\ B,\ A \cup B,\ A \cap B,\ A \backslash B,\ B \backslash A,\ A \bigtriangleup B\) to'plamlarini aniqlang va tasvirlang.
 \\
\textbf{A2.} 
\(\lbrack 1;7\rbrack\) va \(\lbrack - 1;4) \cup \lbrack 6;7\rbrack\) to'plamlari orasida bir qiymatli moslik o'rnating.
 \\
\textbf{A3.} 
\(\lbrack 7,\ 9\rbrack\) kesmada joylashgan sonlarning onlik kasr yozuvida \(9\) raqami qatnashmagan barcha sonlar to'plamining Lebeg o'lchovini toping.
 \\
\textbf{B1.} 
\(\int_{E}^{}f(x)d\mu\) Lebeg integralini hisoblang, \(f(x) = \left\{ \begin{matrix}
\frac{x^{2}}{(x + 3)(x + 2)},\ x \in \mathbb{I} \cap \lbrack 2,\ 4\rbrack \\
3x^{2} - 2,\ x\mathbb{\in Q \cap}\lbrack 2,\ 4\rbrack,\ E = \lbrack 2,\ 4\rbrack
\end{matrix} \right.\ \)
 \\
\textbf{B2.} 
Quyida berilganlar bo'yicha\(\ x,y \in X\) elementlar orasidagi masofani toping: \(X = C\left\lbrack \frac{\pi}{4};\ \frac{\pi}{2} \right\rbrack,\ \rho(x,y) = \max_{\frac{\pi}{4} \leq t \leq \frac{\pi}{2}}|x(t) - y(t)|,x(t) = \ctg (2t - \pi/6),\ y = tg(\ t - \pi/6)\)
 \\
\textbf{B3.} 
\(A\) va \(B\) to'plamlari orasida o'zaro bir qiymatli moslik o'rnating.\(\ A = ( - 4;6\rbrack\), \(B = \lbrack - 2;6\rbrack\).
 \\
\textbf{C1.} 
To'plamning Lebeg o'lchovini toping: \(A = \bigcup_{k = 1}^{\infty}\left( 2k - 2^{- k},2k + \frac{1}{k!} \right)\);
 \\
\textbf{C2.} 
Lebeg integralini (\(\int_{A}^{}{f(x)d\mu}\)) hisoblang: \(f(x) = sign(x + 1)\), \(A = \lbrack - 2;2\rbrack\);
 \\
\textbf{C3.} 
[-1;2] to'plamida o'lchovsiz to'plamga misol keltiring.
 \\

\end{tabular}
\vspace{1cm}


\begin{tabular}{m{17cm}}
\textbf{41-bilet}

\vspace{0.5cm}

\textbf{T1.} 
Metrik fazo va unga misollar.
 \\
\textbf{T2.} 
O'lshovli funkciyalar va ularning xossalari.
 \\
\textbf{A1.} 
\(A = \{(x,y) \in \mathbb{R}^{2}:\ y = - x\},\ B = \{(x,y) \in \mathbb{R}^{2}:\ x^{2} + y^{2} \leq 1\}\), \(A,\ B,\ A \cup B,\ A \cap B,\ A \backslash B,\ B \backslash A,\ A \bigtriangleup B\) to'plamlarini aniqlang va tasvirlang.
 \\
\textbf{A2.} 
\(\lbrack - 1;\ 7)\) va \(\lbrack - 2;4) \cup \lbrack 7;9)\) to'plamlari orasida bir qiymatli moslik o'rnating.
 \\
\textbf{A3.} 
\(\lbrack 3,\ 5\rbrack\) kesmada joylashgan sonlarning onlik kasr yozuvida \(6\) raqami qatnashmagan barcha sonlar to'plamining Lebeg o'lchovini toping.
 \\
\textbf{B1.} 
\(\int_{E}^{}f(x)d\mu\) Lebeg integralini hisoblang, \(f(x) = \left\{ \begin{matrix}
\frac{x^{2}}{(x - 2)(x - 4)},\ x \in \mathbb{I} \cap \lbrack - 1;1\rbrack \\
3x^{2} - 2,\ x\mathbb{\in Q \cap}\lbrack - 1;1\rbrack,\ E = \lbrack - 1;1\rbrack
\end{matrix} \right.\ \)
 \\
\textbf{B2.} 
Quyida berilganlar bo'yicha\(\ x,y \in X\) elementlar orasidagi masofani toping: \(X = C\lbrack 0;\ \pi/3\rbrack,\ \rho(x,y) = \max_{0 \leq t \leq \pi/3}|x(t) - y(t)|,x(t) = \sin t,\ y = \cos5t\)
 \\
\textbf{B3.} 
\(A\) va \(B\) to'plamlari orasida o'zaro bir qiymatli moslik o'rnating.\(\ A = ( - 3;3)\), \(B = \lbrack - 1;9\rbrack\).
 \\
\textbf{C1.} 
To'plamning Lebeg o'lchovini toping: \(A = \bigcup_{k = 1}^{\infty}\left( \frac{1}{k + 1},\frac{1}{k} \right)\);
 \\
\textbf{C2.} 
Lebeg integralini (\(\int_{A}^{}{f(x)d\mu}\)) hisoblang: \(f(x) = \frac{1}{\lbrack x\rbrack}\), \(A = (1;4)\);
 \\
\textbf{C3.} 
[-10;-7] to'plamida o'lchovsiz to'plamga misol keltiring.
 \\

\end{tabular}
\vspace{1cm}


\begin{tabular}{m{17cm}}
\textbf{42-bilet}

\vspace{0.5cm}

\textbf{T1.} To'plamlar va ular ustida amallar.
 \\
\textbf{T2.} 
Lebeg va Riss teoremalari.
 \\
\textbf{A1.} 
\(A = \{(x,y) \in \mathbb{R}^{2}:\ |x| + |y| \leq 2\},\ B = \{(x,y) \in \mathbb{R}^{2}:\ 9x^{2} + y^{2} \geq 9\}\),\(A,\ B,\ A \cup B,\ A \cap B,\ A \backslash B,\ B \backslash A,\ A \bigtriangleup B\) to'plamlarini aniqlang va tasvirlang.
 \\
\textbf{A2.} 
\(\lbrack 2;\ 5\rbrack\) va \(\lbrack 0;1) \cup \lbrack 3;\ 5\rbrack\) to'plamlari orasida bir qiymatli moslik o'rnating.
 \\
\textbf{A3.} 
\(\lbrack 8,\ 10\rbrack\) kesmada joylashgan sonlarning onlik kasr yozuvida \(0\) raqami qatnashmagan barcha sonlar to'plamining Lebeg o'lchovini toping.
 \\
\textbf{B1.} 
\(\int_{E}^{}f(x)d\mu\) Lebeg integralini hisoblang,\(\ f(x) = \left\{ \begin{matrix}
\frac{x^{2}}{(x + 2)(x + 4)},\ x \in \mathbb{I} \cap \lbrack 2,\ 4\rbrack \\
4x^{3},\ x\mathbb{\in Q \cap}\lbrack 2,\ 4\rbrack,\ E = \lbrack 2,\ 4\rbrack
\end{matrix} \right.\ \)
 \\
\textbf{B2.} 
Quyida berilganlar bo'yicha\(\ x,y \in X\) elementlar orasidagi masofani toping: \(X = C\left\lbrack \frac{\pi}{4};\ \frac{\pi}{2} \right\rbrack,\ \rho(x,y) = \max_{\frac{\pi}{4} \leq t \leq \frac{\pi}{2}}|x(t) - y(t)|,x(t) = \ctg (2t - \pi/6),\ y = tg(\ t - \pi/6)\)
 \\
\textbf{B3.} 
\(A\) va \(B\) to'plamlari orasida o'zaro bir qiymatli moslik o'rnating.\(\ A = \lbrack - 2;4\rbrack\), \(B = ( - 5;5)\).
 \\
\textbf{C1.} 
To'plamning Lebeg o'lchovini toping: \(A = \bigcup_{k = 1}^{\infty}\left( k,k + \frac{3}{k(k + 1)} \right)\);
 \\
\textbf{C2.} 
Lebeg integralini (\(\int_{A}^{}{f(x)d\mu}\)) hisoblang: \(f(x) = 2\lbrack x\rbrack\), \(A = ( - 3;3)\);
 \\
\textbf{C3.} 
[5;8] to'plamida o'lchovsiz to'plamga misol keltiring.
 \\

\end{tabular}
\vspace{1cm}


\begin{tabular}{m{17cm}}
\textbf{43-bilet}

\vspace{0.5cm}

\textbf{T1.} 
To'plam quvvati va uning xossalari.
 \\
\textbf{T2.} 
Egorov teoremasi.
 \\
\textbf{A1.} 
\(A = \{(x,y) \in \mathbb{R}^{2}:\ |x| + |y| \geq 3\},\ B = \{(x,y) \in \mathbb{R}^{2}:\ max\{|x|,|y|\} \leq 2\}\), \(A,\ B,\ A \cup B,\ A \cap B,\ A \backslash B,\ B \backslash A,\ A \bigtriangleup B\) to'plamlarini aniqlang va tasvirlang.
 \\
\textbf{A2.} 
\(\lbrack 1;\ 5\rbrack\) va \(\lbrack 1;\ 2) \cup \lbrack 7;10\rbrack\) to'plamlari orasida bir qiymatli moslik o'rnating.
 \\
\textbf{A3.} 
\(\lbrack 3,\ 5\rbrack\) kesmada joylashgan sonlarning onlik kasr yozuvida \(5\) raqami qatnashmagan barcha sonlar to'plamining Lebeg o'lchovini toping.
 \\
\textbf{B1.} 
\(\int_{E}^{}f(x)d\mu\) Lebeg integralini hisoblang, \(f(x) = \left\{ \begin{matrix}
\frac{x^{2}}{(x - 5)(x - 6)},\ x \in \mathbb{I} \cap \lbrack 0,\ 4\rbrack \\
3x^{2} - 2,\ x\mathbb{\in Q \cap}\lbrack 0,\ 4\rbrack,\ E = \lbrack 0,\ 4\rbrack
\end{matrix} \right.\ \)
 \\
\textbf{B2.} 
Quyida berilganlar bo'yicha\(\ x,y \in X\) elementlar orasidagi masofani toping: \(X = C\lbrack 0;\ \pi/6\rbrack,\ \rho(x,y) = \max_{0 \leq t \leq \pi/6}|x(t) - y(t)|,x(t) = \sin3t,\ y = \cos t\)
 \\
\textbf{B3.} 
\(A\) va \(B\) to'plamlari orasida o'zaro bir qiymatli moslik o'rnating.\(\ A = \lbrack - 7;3)\), \(B = \lbrack - 5;7\rbrack\).
 \\
\textbf{C1.} 
To'plamning Lebeg o'lchovini toping: \(A = \bigcup_{k = 1}^{\infty}\left( \frac{1}{2^{k + 1}},\frac{1}{2^{k}} \right)\);
 \\
\textbf{C2.} 
Lebeg integralini (\(\int_{A}^{}{f(x)d\mu}\)) hisoblang: \(f(x) = \frac{1}{\lbrack x - 1\rbrack}\), \(A = (3;6)\);
 \\
\textbf{C3.} 
[0;3] to'plamida o'lchovsiz to'plamga misol keltiring.
 \\

\end{tabular}
\vspace{1cm}


\begin{tabular}{m{17cm}}
\textbf{44-bilet}

\vspace{0.5cm}

\textbf{T1.} 
Metrik fazolarda ochiq va yopiq to'plamlar.
 \\
\textbf{T2.} 
Tekislikta elementar to'plamlar va ularning o'lshovi.
 \\
\textbf{A1.} 
\(A = \{(x,y) \in \mathbb{R}^{2}:\ y = - x^{2}\},\ B = \{(x,y) \in \mathbb{R}^{2}:\ (x - 1)^{2} + (y - 1)^{2} \leq 1\}\), \(A,\ B,\ A \cup B,\ A \cap B,\ A \backslash B,\ B \backslash A,\ A \bigtriangleup B\) to'plamlarini aniqlang va tasvirlang.
 \\
\textbf{A2.} 
\(\lbrack 0;5)\) va \(\lbrack - 2;0) \cup \lbrack 1;4)\) to'plamlari orasida bir qiymatli moslik o'rnating.
 \\
\textbf{A3.} 
\(\lbrack 5,\ 7\rbrack\) kesmada joylashgan sonlarning onlik kasr yozuvida \(7\) raqami qatnashmagan barcha sonlar to'plamining Lebeg o'lchovini toping.
 \\
\textbf{B1.} 
\(\int_{E}^{}f(x)d\mu\) Lebeg integralini hisoblang, \(f(x) = \left\{ \begin{matrix}
\frac{x^{2}}{(x + 2)(x + 4)},\ x \in \mathbb{I} \cap \lbrack 0,\ 4\rbrack \\
3x^{2} - 2,\ x\mathbb{\in Q \cap}\lbrack 0,\ 4\rbrack,\ E = \lbrack 0,\ 4\rbrack
\end{matrix} \right.\ \)
 \\
\textbf{B2.} 
Quyida berilganlar bo'yicha\(\ x,y \in X\) elementlar orasidagi masofani toping: \(X = C\left\lbrack \frac{\pi}{4};\ \frac{\pi}{3} \right\rbrack,\ \rho(x,y) = \max_{\frac{\pi}{4} \leq t \leq \frac{\pi}{3}}|x(t) - y(t)|,x(t) = \ctg (2t + \pi/6),\ y = tg(\ t - \pi/6)\)
 \\
\textbf{B3.} 
\(A\) va \(B\) to'plamlari orasida o'zaro bir qiymatli moslik o'rnating.\(\ A = ( - 3;5)\), \(B = \lbrack - 8;6)\).
 \\
\textbf{C1.} 
To'plamning Lebeg o'lchovini toping: \(A = \bigcup_{k = 1}^{\infty}\left\lbrack e^{- 2k},e^{- 2k + 1} \right)\).
 \\
\textbf{C2.} 
Lebeg integralini (\(\int_{A}^{}{f(x)d\mu}\)) hisoblang: \(f(x) = sign(x + 1)\), \(A = \lbrack - 2;2\rbrack\);
 \\
\textbf{C3.} 
[-8;-5] to'plamida o'lchovsiz to'plamga misol keltiring.
 \\

\end{tabular}
\vspace{1cm}


\begin{tabular}{m{17cm}}
\textbf{45-bilet}

\vspace{0.5cm}

\textbf{T1.} 
Kompakt metrik fazolar.
 \\
\textbf{T2.} 
Egorov teoremasi.
 \\
\textbf{A1.} 
\(A = \{(x,y) \in \mathbb{R}^{2}:\ x \geq y\},\ B = \{(x,y) \in \mathbb{R}^{2}:\ x^{2} + 4y^{2} \geq 4\}\), \(A,\ B,\ A \cup B,\ A \cap B,\ A \backslash B,\ B \backslash A,\ A \bigtriangleup B\) to'plamlarini aniqlang va tasvirlang.
 \\
\textbf{A2.} 
\(\lbrack 0;6\rbrack\) va \(\lbrack 0;5) \cup \lbrack 7;8\rbrack\) to'plamlari orasida bir qiymatli moslik o'rnating.
 \\
\textbf{A3.} 
\(\lbrack 1,\ 3\rbrack\) kesmada joylashgan sonlarning onlik kasr yozuvida \(4\) raqami qatnashmagan barcha sonlar to'plamining Lebeg o'lchovini toping.
 \\
\textbf{B1.} 
\(\int_{E}^{}f(x)d\mu\) Lebeg integralini hisoblang, \(f(x) = \left\{ \begin{matrix}
\frac{x^{2}}{(x + 3)(x + 2)},\ x \in \mathbb{I} \cap \lbrack 2,\ 4\rbrack \\
3x^{2} - 2,\ x\mathbb{\in Q \cap}\lbrack 2,\ 4\rbrack,\ E = \lbrack 2,\ 4\rbrack
\end{matrix} \right.\ \)
 \\
\textbf{B2.} 
Quyida berilganlar bo'yicha\(\ x,y \in X\) elementlar orasidagi masofani toping: \(X = C\left\lbrack \frac{\pi}{6};\ \frac{\pi}{4} \right\rbrack,\ \rho(x,y) = \max_{\frac{\pi}{6} \leq t \leq \frac{\pi}{4}}|x(t) - y(t)|,x(t) = \ctg t,\ y = tg(\ 2t - \frac{\pi}{6})\)
 \\
\textbf{B3.} 
\(A\) va \(B\) to'plamlari orasida o'zaro bir qiymatli moslik o'rnating.\(\ A = \lbrack - 4;4\rbrack\), \(B = ( - 11;3)\).
 \\
\textbf{C1.} 
To'plamning Lebeg o'lchovini toping: \(A = \bigcup_{k = 1}^{\infty}\left( \frac{1}{2k},\frac{1}{k} \right)\);
 \\
\textbf{C2.} 
Lebeg integralini (\(\int_{A}^{}{f(x)d\mu}\)) hisoblang: \(f(x) = sign(x)\), \(A = \lbrack - 2;2)\);
 \\
\textbf{C3.} 
[7;10] to'plamida o'lchovsiz to'plamga misol keltiring.
 \\

\end{tabular}
\vspace{1cm}


\begin{tabular}{m{17cm}}
\textbf{46-bilet}

\vspace{0.5cm}

\textbf{T1.} 
Metrik fazolarning uzliksiz akslantirishlari.
 \\
\textbf{T2.} 
O'lshovli funkciyalar va ularning xossalari.
 \\
\textbf{A1.} 
\(A = \{(x,y) \in \mathbb{R}^{2}:\ y \geq x^{2}\},\ B = \{(x,y) \in \mathbb{R}^{2}:\ y \leq 4 - x^{2}\}\), \(A,\ B,\ A \cup B,\ A \cap B,\ A \backslash B,\ B \backslash A,\ A \bigtriangleup B\) to'plamlarini aniqlang va tasvirlang.
 \\
\textbf{A2.} 
\(\lbrack - 3;\ 3)\) va \(\lbrack 0;4) \cup \lbrack 7;9)\) to'plamlari orasida bir qiymatli moslik o'rnating.
 \\
\textbf{A3.} 
\(\lbrack 2,\ 4\rbrack\) kesmada joylashgan sonlarning onlik kasr yozuvida \(4\) raqami qatnashmagan barcha sonlar to'plamining Lebeg o'lchovini toping.
 \\
\textbf{B1.} 
\(\int_{E}^{}f(x)d\mu\) Lebeg integralini hisoblang, \(f(x) = \left\{ \begin{matrix}
\frac{x^{2}}{(x - 2)(x - 4)},\ x \in \mathbb{I} \cap \lbrack - 4; - 1\rbrack \\
3x^{2} - 2,\ x\mathbb{\in Q \cap}\lbrack - 4; - 1\rbrack,E = \lbrack - 4; - 1\rbrack
\end{matrix} \right.\ \)
 \\
\textbf{B2.} 
Quyida berilganlar bo'yicha\(\ x,y \in X\) elementlar orasidagi masofani toping: \(X = C\left\lbrack \frac{\pi}{6};\ \frac{\pi}{4} \right\rbrack,\ \rho(x,y) = \max_{\frac{\pi}{6} \leq t \leq \frac{\pi}{4}}|x(t) - y(t)|,x(t) = \ctg (2t - \pi/6),\ y = tg(\ 2t - \pi/6)\)
 \\
\textbf{B3.} 
\(A\) va \(B\) to'plamlari orasida o'zaro bir qiymatli moslik o'rnating.\(\ A = \lbrack - 6;2\rbrack\), \(B = ( - 7;3)\).
 \\
\textbf{C1.} 
To'plamning Lebeg o'lchovini toping: \(A = \bigcup_{k = 1}^{\infty}\left( \frac{1}{k + 2},\frac{1}{k} \right)\);
 \\
\textbf{C2.} 
Lebeg integralini (\(\int_{A}^{}{f(x)d\mu}\)) hisoblang: \(f(x) = \frac{1}{\lbrack x + 1\rbrack}\), \(A = \lbrack 1;5)\);
 \\
\textbf{C3.} 
[6;9] to'plamida o'lchovsiz to'plamga misol keltiring.
 \\

\end{tabular}
\vspace{1cm}


\begin{tabular}{m{17cm}}
\textbf{47-bilet}

\vspace{0.5cm}

\textbf{T1.} 
Metrik fazo va unga misollar.
 \\
\textbf{T2.} 
Tekislikta elementar to'plamlar va ularning o'lshovi.
 \\
\textbf{A1.} 
\(A = \{(x,y) \in \mathbb{R}^{2}:\ max\{|x|,|y|\} \leq 2\},\ B = \{(x,y) \in \mathbb{R}^{2}:\ y \geq x + 1\}\), \(A,\ B,\ A \cup B,\ A \cap B,\ A \backslash B,\ B \backslash A,\ A \bigtriangleup B\) to'plamlarini aniqlang va tasvirlang.
 \\
\textbf{A2.} 
\(\lbrack 2;\ 7)\) va \(\lbrack - 2; - 1) \cup \lbrack 2;4)\) to'plamlari orasida bir qiymatli moslik o'rnating.
 \\
\textbf{A3.} 
\(\lbrack 5,\ 7\rbrack\) kesmada joylashgan sonlarning onlik kasr yozuvida \(8\) raqami qatnashmagan barcha sonlar to'plamining Lebeg o'lchovini toping.
 \\
\textbf{B1.} 
\(\int_{E}^{}f(x)d\mu\) Lebeg integralini hisoblang, \(E = \lbrack 0,\ 1\rbrack\), \(f(x) = \left\{ \begin{matrix}
\frac{1}{\sqrt{x}},\ x \in \mathbb{I} \cap \lbrack 0,\ 1\rbrack \\
\sin x,\ x\mathbb{\in Q}
\end{matrix} \right.\ \)
 \\
\textbf{B2.} 
Quyida berilganlar bo'yicha\(\ x,y \in X\) elementlar orasidagi masofani toping: \(X = C\left\lbrack \frac{\pi}{6};\ \frac{\pi}{3} \right\rbrack,\ \rho(x,y) = \max_{\frac{\pi}{6} \leq t \leq \frac{\pi}{3}}|x(t) - y(t)|,x(t) = \ctg (t + \pi/6),\ y = tg\ t\)
 \\
\textbf{B3.} 
\(A\) va \(B\) to'plamlari orasida o'zaro bir qiymatli moslik o'rnating.\(\ A = ( - 2;4)\), \(B = \lbrack 2;10)\).
 \\
\textbf{C1.} 
\(P = \{ 0 \leq x \leq 1,\ 0 \leq y \leq 1\}\) va \(Q = \{ 0.3 \leq x \leq 0.8,\ 0 \leq y \leq 1\}\ \)to'g'ri to'rtburchaklar simmetrik ayirmasining o'lchovini toping.
 \\
\textbf{C2.} 
Lebeg integralini (\(\int_{A}^{}{f(x)d\mu}\)) hisoblang: \(f(x) = 2^{\lbrack x\rbrack}\), \(A = ( - 2;2)\);
 \\
\textbf{C3.} 
[1;4] to'plamida o'lchovsiz to'plamga misol keltiring.
 \\

\end{tabular}
\vspace{1cm}


\begin{tabular}{m{17cm}}
\textbf{48-bilet}

\vspace{0.5cm}

\textbf{T1.} To'plamlar va ular ustida amallar.
 \\
\textbf{T2.} 
Lebeg va Riss teoremalari.
 \\
\textbf{A1.} 
\(A = \{(x,y) \in \mathbb{R}^{2}:\ xy \leq 0\},\ B = \{(x,y) \in \mathbb{R}^{2}:\ x^{2} + y^{2} \geq 4\}\), \(A,\ B,\ A \cup B,\ A \cap B,\ A \backslash B,\ B \backslash A,\ A \bigtriangleup B\) to'plamlarini aniqlang va tasvirlang.
 \\
\textbf{A2.} 
\(\lbrack 3;\ 7\rbrack\) va \(\lbrack 0;\ 2) \cup \lbrack 6;\ 8\rbrack\) to'plamlari orasida bir qiymatli moslik o'rnating.
 \\
\textbf{A3.} 
\(\lbrack 0,\ 2\rbrack\) kesmada joylashgan sonlarning onlik kasr yozuvida \(3\) raqami qatnashmagan barcha sonlar to'plamining Lebeg o'lchovini toping.
 \\
\textbf{B1.} 
\(\int_{E}^{}f(x)d\mu\) Lebeg integralini hisoblang, \(f(x) = \left\{ \begin{matrix}
\frac{x^{2}}{(x - 5)(x - 7)},\ x \in \mathbb{I} \cap \lbrack 1,\ 4\rbrack \\
3x^{2} - 2,\ x\mathbb{\in Q \cap}\lbrack 1,\ 4\rbrack,\ E = \lbrack 1,\ 4\rbrack
\end{matrix} \right.\ \)
 \\
\textbf{B2.} 
Quyida berilganlar bo'yicha\(\ x,y \in X\) elementlar orasidagi masofani toping: \(X = C\lbrack 0;\ \pi/4\rbrack,\ \rho(x,y) = \max_{0 \leq t \leq \pi/4}|x(t) - y(t)|,x(t) = \sin4t,\ y = \cos2t\)
 \\
\textbf{B3.} 
\(A\) va \(B\) to'plamlari orasida o'zaro bir qiymatli moslik o'rnating.\(\ A = \lbrack - 1;4)\), \(B = \lbrack - 1;7\rbrack\).
 \\
\textbf{C1.} 
To'plamning Lebeg o'lchovini toping: \(A = \bigcup_{k = 1}^{\infty}\left( k^{3},k^{3} + 3^{- k} \right)\);
 \\
\textbf{C2.} 
Lebeg integralini (\(\int_{A}^{}{f(x)d\mu}\)) hisoblang: \(f(x) = \frac{( - 1)^{\lbrack x\rbrack}}{\lbrack x\rbrack}\), \(A = \lbrack 1;4)\);
 \\
\textbf{C3.} 
[-11;-8] to'plamida o'lchovsiz to'plamga misol keltiring.
 \\

\end{tabular}
\vspace{1cm}


\begin{tabular}{m{17cm}}
\textbf{49-bilet}

\vspace{0.5cm}

\textbf{T1.} 
To'plam quvvati va uning xossalari.
 \\
\textbf{T2.} 
Tekislikta elementar to'plamlar va ularning o'lshovi.
 \\
\textbf{A1.} 
\(A = \{(x,y) \in \mathbb{R}^{2}:\ x = - y\},\ B = \{(x,y) \in \mathbb{R}^{2}:\ (x - 2)^{2} + (y + 3)^{2} \geq 1\}\), \(A,\ B,\ A \cup B,\ A \cap B,\ A \backslash B,\ B \backslash A,\ A \bigtriangleup B\) to'plamlarini aniqlang va tasvirlang.
 \\
\textbf{A2.} 
\(\lbrack - 2;4)\) va \(\lbrack 0;4) \cup \lbrack 5;7)\) to'plamlari orasida bir qiymatli moslik o'rnating.
 \\
\textbf{A3.} 
\(\lbrack 4,\ 6\rbrack\) kesmada joylashgan sonlarning onlik kasr yozuvida \(7\) raqami qatnashmagan barcha sonlar to'plamining Lebeg o'lchovini toping.
 \\
\textbf{B1.} 
\(\int_{E}^{}f(x)d\mu\) Lebeg integralini hisoblang, \(f(x) = \left\{ \begin{matrix}
\frac{x^{2}}{(x - 5)(x - 6)},\ x \in \mathbb{I} \cap \lbrack 0,\ 4\rbrack \\
3x^{2} - 2,\ x\mathbb{\in Q \cap}\lbrack 0,\ 4\rbrack,\ E = \lbrack 0,\ 4\rbrack
\end{matrix} \right.\ \)
 \\
\textbf{B2.} 
Quyida berilganlar bo'yicha\(\ x,y \in X\) elementlar orasidagi masofani toping: \(X = C\lbrack 0,\pi\rbrack,\ \rho(x,y) = \max_{0 \leq t \leq \pi}|x(t) - y(t)|,x(t) = \sin2t,\ y = \cos4t\).
 \\
\textbf{B3.} 
\(A\) va \(B\) to'plamlari orasida o'zaro bir qiymatli moslik o'rnating.\(\ A = ( - 3;4)\), \(B = \lbrack - 2;10)\).
 \\
\textbf{C1.} 
To'plamning Lebeg o'lchovini toping: \(A = \bigcup_{k = 1}^{\infty}\left( \frac{1}{k + 2},\frac{1}{k} \right)\);
 \\
\textbf{C2.} 
Lebeg integralini (\(\int_{A}^{}{f(x)d\mu}\)) hisoblang: \(f(x) = 2^{( - 1)^{\lbrack x\rbrack}}\), \(A = \lbrack 0;3)\);
 \\
\textbf{C3.} 
[4;7] to'plamida o'lchovsiz to'plamga misol keltiring.
 \\

\end{tabular}
\vspace{1cm}


\begin{tabular}{m{17cm}}
\textbf{50-bilet}

\vspace{0.5cm}

\textbf{T1.} 
Metrik fazolarning uzliksiz akslantirishlari.
 \\
\textbf{T2.} 
O'lshovli funkciyalar va ularning xossalari.
 \\
\textbf{A1.} 
\(A = \{(x,y) \in \mathbb{R}^{2}:\ x = y\},\ B = \{(x,y) \in \mathbb{R}^{2}:\ |x| + |y| \leq 1\}\), \(A,\ B,\ A \cup B,\ A \cap B,\ A \backslash B,\ B \backslash A,\ A \bigtriangleup B\) to'plamlarini aniqlang va tasvirlang.
 \\
\textbf{A2.} 
\(\lbrack - 1;\ 5)\) va \(\lbrack - 1;4) \cup \lbrack 7;8)\) to'plamlari orasida bir qiymatli moslik o'rnating.
 \\
\textbf{A3.} 
\(\lbrack 3,\ 4\rbrack\) kesmada joylashgan sonlarning onlik kasr yozuvida \(1\) raqami qatnashmagan barcha sonlar to'plamining Lebeg o'lchovini toping.
 \\
\textbf{B1.} 
\(\int_{E}^{}f(x)d\mu\) Lebeg integralini hisoblang, \(E = \lbrack 0,\ 1\rbrack\), \(f(x) = \left\{ \begin{matrix}
\frac{1}{(x + 1)^{3}}\ x \in \mathbb{I} \cap \lbrack 0,\ 1\rbrack \\
7x,\ x\mathbb{\in Q}
\end{matrix} \right.\ \)
 \\
\textbf{B2.} 
Quyida berilganlar bo'yicha\(\ x,y \in X\) elementlar orasidagi masofani toping: \(X = C\lbrack 0;\ \pi/4\rbrack,\ \rho(x,y) = \max_{0 \leq t \leq \pi/4}|x(t) - y(t)|,x(t) = \sin t,\ y = \cos3t\)
 \\
\textbf{B3.} 
\(A\) va \(B\) to'plamlari orasida o'zaro bir qiymatli moslik o'rnating.\(\ A = ( - 1;4)\), \(B = \lbrack 2;12)\).
 \\
\textbf{C1.} 
To'plamning Lebeg o'lchovini toping: \(A = \bigcup_{k = 1}^{\infty}\left( k^{3},k^{3} + 3^{- k} \right)\);
 \\
\textbf{C2.} 
Lebeg integralini (\(\int_{A}^{}{f(x)d\mu}\)) hisoblang: \(f(x) = 2^{\lbrack 2x\rbrack}\), \(A = \lbrack 0;1)\);
 \\
\textbf{C3.} 
[10;13] to'plamida o'lchovsiz to'plamga misol keltiring.
 \\

\end{tabular}
\vspace{1cm}



\end{document}
