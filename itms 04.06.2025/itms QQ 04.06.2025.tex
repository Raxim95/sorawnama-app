\documentclass{article}
\usepackage[fontsize=12pt]{fontsize}
\usepackage[utf8]{inputenc}
\usepackage[T2A]{fontenc}
% \usepackage{unicode-math}

\usepackage{array}
\usepackage[a4paper,
left=7mm,
right=5mm,
top=7mm,]{geometry}
\usepackage{amsmath}
% \usepackage{amssymbol}
\usepackage{amsfonts}
\usepackage{graphicx}
\usepackage{setspace}
\onehalfspacing



\renewcommand{\baselinestretch}{1} 

\everymath{\displaystyle}
\everydisplay{\displaystyle}
% \linespread{1.25}

\DeclareMathOperator{\sign}{sign}


\begin{document}

\pagenumbering{gobble}


\begin{tabular}{m{17cm}}
\textbf{1-variant}
\newline

\textbf{T1.} Ǵárezsiz tájiriybelerdiń Bernulli sxeması (binоmiаl bólistiriliw, qásiyetleri).
 \\
\textbf{T2.} Tosınnanlı shamanıń joqarı tártipli momentleri (baslanǵısh hám oraylıq momentleri, qásiyetleri).
 \\
\textbf{A1.} $\left[ 0,1 \right]$ kesindiden tosınnan eki noqat tańlanadı. Birinshi hám ekinshi noqatlar koordinataları arasındaǵı aralıq $0,5$ ten úlken bolıw itimallıǵın tabıń.
 \\
\textbf{A2.} Eki mergen biri-birinen ǵárezsiz bir nıshanǵa bir márteden oq atadı. Birinshi mergenniń nıshanǵa tiygiziw itimallıǵı 0,8 ekinshisiniki bolsa 0,4 ke teń. a) Atıw tamam bolǵannan soń, nıshanda bir oq izi bolıwı itimallıǵın tabıń. b) Sol oq iziniń ekinshi mergenge tiyisli bolıw itimallıǵın tabıń.
 \\
\textbf{A3.} Hárbiriniń júzege asıw itimallıǵı $p$ ǵa teń bolǵan 10 dana Bernulli tájiriybesi ótkerilgende, tómendegi waqıyalardıń itimallıqların tabıń: Tájiriybelerdiń birinshi yarımındaǵı sátliler sanı, tájiriybelerdiń ekinshi yarımındaǵı sátliler sanınan ekige artıq.
 \\
\textbf{B1.} \(f(x) = \frac{C}{e^{x} + e^{- x}}\) bólistiriw tıǵızlıǵı bolıwı ushın \(C\) nege teń bolıwı kerek?
 \\
\textbf{B2.} \includegraphics[width=0.15972in,height=0.23958in]{mediaBpng/image49.png} úzliksiz tosınnanlıq shamanıń tıǵızlıq funkciyaları berilgen. Olarǵa sáykes \includegraphics[width=0.15972in,height=0.19653in]{mediaBpng/image50.png} tosınnanlıq shamanıń \includegraphics[width=0.50278in,height=0.30069in]{mediaBpng/image51.png} tıǵızlıq funkciyasın tabıń. \includegraphics[width=2.30694in,height=0.84028in]{mediaBpng/image64.png} \includegraphics[width=0.85903in,height=0.23958in]{mediaBpng/image65.png}
 \\
\textbf{B3.} Yashikte 10 buyimniń 4 buyımı jaramsiz. Táwekelge alinǵan 7 buyımnıń ishinde 3 buyim jaramsız bolıwı itimallıǵın tabıń.
 \\
\textbf{C1.} Eger \(\left\{ \xi_{n} \right\}\) diskret tosınnanlıq shamalar izbe-izliginiń bólistiriliw nızamları\(P(\xi_{n} = e^{n}) = \frac{1}{n^{2}},\) \(P(\xi_{n} = 0) = 1 - \frac{1}{n^{2}}\) bolsa, onda \(\left\{ \xi_{n} \right\}\) tosınnanlıq shamalar izbe-izliginiń 0 ge bir itimallıq penen jıynaqlılıǵın kórsetiń.
 \\
\textbf{C2.} Eger \(\xi\) tosınnanlı shama \((a,\sigma)\) parametrli normal bólistiriwine iye bolsa, onda onıń xarakteristikalıq funkciyası tabılsın.
 \\
\textbf{C3.} Eger ǵárezsiz \includegraphics[width=0.15208in,height=0.24028in]{mediaCpng/image4.png} hám \includegraphics[width=0.15208in,height=0.19167in]{mediaCpng/image5.png} úzliksiz tosınnanlıq shamalardıń hárbiri \includegraphics[width=0.4in,height=0.24028in]{mediaCpng/image18.png} aralıqta teń ólshemli bólistirilgen bolsa, onda \includegraphics[width=0.84792in,height=0.29583in]{mediaCpng/image19.png} tosınnanlıq shamanıń tıǵızlıq funkciyasın tabıń.
 \\

\end{tabular}
\vspace{1cm}


\begin{tabular}{m{17cm}}
\textbf{2-variant}
\newline

\textbf{T1.} Bayеs formulası (gipotezalar teoreması, dálilleniwi).
 \\
\textbf{T2.} Tıǵızlıq funkciyası (anıqlaması, tiykarǵıqásiyetleri).
 \\
\textbf{A1.} 
Toparda 12 student bar bolıp, olardan 8 student ayrıqsha bahaǵa oqıydı. Dizim boyınsha 9 student ajıratılǵan. Ajıratılǵanlar ishinde ayrıqsha bahaǵa oqıytuǵın 5 student bolıw itimallıǵın tabıń.
 \\
\textbf{A2.} Kitaptıń bir betinde keminde bir baspa qáteligi bolıw itimallıǵı 0,01 ge hám tártip qáteligi bolıw itimallıǵı bolsa 0,3 ke teń. Jámi 500 betli kitapta tómendegi waqıyalardıń júzege asıw itimallıqların tabıń: a) keminde tórt bette baspa qáteligi bolıwı; b) 140 tan 170 ke shekem betlerde tártip qáteligi bolıwı.
 \\
\textbf{A3.} Oyın kubigi taslanǵanda taq ochkonıń túsiw itimallıǵın tabıń.
 \\
\textbf{B1.} $\xi$ tosınnanlı shamanıń \emph{f}(\emph{x}) tıǵızlıq funkciyasi berilgen bolsin. Tómendegilerdi esaplań: a) C; b) \emph{F}(\emph{x}); c) M$\xi$; d) D$\xi$; e) \emph{f}(\emph{x}) hám \emph{F}(\emph{x}) grafiklarin sızıń.\(f(x) = \left\{ \begin{matrix}
\ \ \ \ \ \ 0,\ \ \ \ \ x \notin (0,\ \pi/2)\ \  \\
C\sin x,\ \ \ \ \ x \in (0,\ \pi/2)\ \ 
\end{matrix} \right.\ \)
 \\
\textbf{B2.} Eger \includegraphics[width=0.36181in,height=0.29444in]{mediaBpng/image1.png} ǵárezsiz tosınnanlıq shamalar izbe-izliginiń bólistiriliw nızamları
\includegraphics[width=2.23333in,height=0.50278in]{mediaBpng/image31.png} \includegraphics[width=1.17778in,height=0.50278in]{mediaBpng/image32.png} \includegraphics[width=0.77292in,height=0.25764in]{mediaBpng/image33.png}
bolsa, onda bul izbe-izlik úlken sanlar nızamına boysınama?
 \\
\textbf{B3.} \includegraphics[width=0.15972in,height=0.23958in]{mediaBpng/image49.png} úzliksiz tosınnanlıq shamanıń tıǵızlıq funkciyaları berilgen. Olarǵa sáykes \includegraphics[width=0.15972in,height=0.19653in]{mediaBpng/image50.png} tosınnanlıq shamanıń \includegraphics[width=0.50278in,height=0.30069in]{mediaBpng/image51.png} tıǵızlıq funkciyasın tabıń. \includegraphics[width=2.52778in,height=1.15347in]{mediaBpng/image92.png} \includegraphics[width=0.57639in,height=0.28194in]{mediaBpng/image93.png}
 \\
\textbf{C1.} Eger \(\xi_{1}\) hám \(\xi_{2}\) ǵárezsiz tosınnanlıq shamalardıń hárbiri \(\lbrack 0,1\rbrack\) aralıqta teń ólshemli bólistirilgen bolsa, onda \(\xi_{1} + \xi_{2}\) tosınnanlıq shamanıń tıǵızlıq funkciyasın tabıń.
 \\
\textbf{C2.} Tómende \includegraphics[width=0.46389in,height=0.25625in]{mediaCpng/image42.png} úzliksiz tosınnanlıq vektorlardıń tıǵızlıq funkciyaları berilgen. Olardıń \includegraphics[width=0.48819in,height=0.29583in]{mediaCpng/image43.png} hám \includegraphics[width=0.50417in,height=0.29583in]{mediaCpng/image44.png} marginal tıǵızlıq funkciyaların tabıń; \includegraphics[width=0.15972in,height=0.24028in]{mediaCpng/image45.png} hám \includegraphics[width=0.15972in,height=0.2in]{mediaCpng/image46.png} tosınnanlıq shamalardı ǵárezsizlikke tekseriń: \includegraphics[width=3.07986in,height=0.67986in]{mediaCpng/image59.png}
 \\
\textbf{C3.} Tómende \includegraphics[width=0.46389in,height=0.25625in]{mediaCpng/image42.png} úzliksiz tosınnanlıq vektorlardıń tıǵızlıq funkciyaları berilgen. Olardıń \includegraphics[width=0.48819in,height=0.29583in]{mediaCpng/image43.png} hám \includegraphics[width=0.50417in,height=0.29583in]{mediaCpng/image44.png} marginal tıǵızlıq funkciyaların tabıń; \includegraphics[width=0.15972in,height=0.24028in]{mediaCpng/image45.png} hám \includegraphics[width=0.15972in,height=0.2in]{mediaCpng/image46.png} tosınnanlıq shamalardı ǵárezsizlikke tekseriń: \includegraphics[width=2.65625in,height=0.63194in]{mediaCpng/image57.png}
 \\

\end{tabular}
\vspace{1cm}


\begin{tabular}{m{17cm}}
\textbf{3-variant}
\newline

\textbf{T1.} Bernulli sxemаsı ushın limit teоremаlаr (Muavr-Laplas lokallıq teoreması, qásiyetleri).
 \\
\textbf{T2.} Xarakteristikalıq funkciyalar (anıqlaması, tiykarǵı qásiyetleri).
 \\
\textbf{A1.} Hárbiriniń júzege asıw itimallıǵı $p$ ǵa teń bolǵan 10 dana Bernulli tájiriybesi ótkerilgende, tómendegi waqıyalardıń itimallıqların tabıń: Sátliler sanı 6 dana, sonıń menen birge, sońǵı tájiriybe sátsiz juwmaqlanıwı.
 \\
\textbf{A2.} Eki oyın kubigi taslanǵanda keminde bir oyın kubigindegi túsken ochko jup bolıw itimallıǵın tabıń.
 \\
\textbf{A3.} Jámi $N$ dana lotereya biletleri ishinde $M$ dana lotereya bileti utıslı bolsa, satıp alınǵan $n~\,\left( n\le N \right)$ lotereya biletinen $m\,\,\left( m\le M \right)$ danası utıslı bolıw itimallıǵın tabıń.
 \\
\textbf{B1.} Eger \includegraphics[width=0.36181in,height=0.29444in]{mediaBpng/image1.png} ǵárezsiz tosınnanlıq shamalar izbe-izliginiń bólistiriliw nızamları
\includegraphics[width=2.70556in,height=0.47847in]{mediaBpng/image27.png} \includegraphics[width=0.75486in,height=0.23958in]{mediaBpng/image9.png}
bolsa, onda bul izbe-izlik úlken sanlar nızamına boysınama?
 \\
\textbf{B2.} 
Bólistiriw funkciyasi berilgen: \(\mathbf{F}\mathbf{(}\mathbf{x}\mathbf{)}\mathbf{=}\left\{ \begin{matrix}
\mathbf{0,}\mathbf{\ \ \ \ \ \ \ \ \ \ \ \ \ \ \ \ \ \ \ \ \ \ \ \ \ \ \ \ \ \ \ \ \ \ \ \ \ \ \ \ \ x \leq - a} \\
\frac{\mathbf{1}}{\mathbf{2}}\mathbf{+}\frac{\mathbf{1}}{\mathbf{\pi}}\mathbf{\arcsin}\frac{\mathbf{x}}{\mathbf{a}}\mathbf{,}\mathbf{\ \ \ \ \  - a < x < a}\mathbf{,} \\
\mathbf{1,}\mathbf{\ \ \ \ \ \ \ \ \ \ \ \ \ \ \ \ \ \ \ \ \ \ \ \ \ \ \ \ \ \ \ \ \ \ \ \ \ \ \ \ \ \ \ \ \ x \geq a}
\end{matrix} \right.\ \) a)bólistiriw tiǵizliǵi \(f(x)\  = ?\ \ \ \ \ \ \ \)b) \(\mathbf{P}\left\{ \mathbf{-}\frac{\mathbf{a}}{\mathbf{2}}\mathbf{< \xi <}\frac{\mathbf{a}}{\mathbf{2}} \right\}\mathbf{=}\mathbf{?}\)
 \\
\textbf{B3.} $\xi$ tosınnanlı shamanıń \emph{f}(\emph{x}) tıǵızlıq funkciyasi berilgen bolsin. Tómendegilerdi esaplań: a) C; b) \emph{F}(\emph{x}); c) M$\xi$; d) D$\xi$; e) \emph{f}(\emph{x}) hám \emph{F}(\emph{x}) grafiklarin sızıń.\(f(x) = \left\{ \begin{matrix}
C/\sqrt{1 - x^{2}},\ \ \ \ x \in \lbrack - 1,1\rbrack, \\
\ \ \ \ \ \ \ \ 0,\ \ \ \ \ \ \ \ \ \ \ x \notin \lbrack - 1,1\rbrack.\ \ 
\end{matrix} \right.\ \)
 \\
\textbf{C1.} Eger \(\left( \xi_{1},\xi_{2} \right)\) tosınnanlıq vektordıń bólistiriw funkciyası\(F(x,y) = \sin x \cdot \sin y,\ \ \ 0 \leq x \leq \frac{\pi}{2},\ \ 0 \leq y \leq \frac{\pi}{2}\) bolsa, onda \(\left( \xi_{1},\xi_{2} \right)\) tosınnanlıq noqattıń \(G:x_{1} = \frac{\pi}{6},\ \ x_{2} = \frac{\pi}{2};\ \ y_{1} = \frac{\pi}{4},\ \ y_{2} = \frac{\pi}{3}\) bolǵan tuwrımúyeshlikke túsiw itimallıǵın tabıń.
 \\
\textbf{C2.} Eger \(\xi\sim N\left( a,\sigma^{2} \right)\) bolsa, onda \(\xi\) tosınnanlıq shamanıń joqarı tártipli oraylıq momentlerin tabıń.
 \\
\textbf{C3.} Eger \(\left\{ \xi_{n} \right\}\) ǵárezsiz hám \(\mathbf{\lbrack 0,1\rbrack}\) aralıqta teń ólshemli bólistirilgen tosınnanlıq shamalar izbe-izligi bolsa, onda \(\left\{ \mathbf{\xi}_{\mathbf{(}\mathbf{n}\mathbf{)}}\mathbf{=}\mathbf{\max}\mathbf{\{}\mathbf{\xi}_{\mathbf{1}}\mathbf{,...,}\mathbf{\xi}_{\mathbf{n}}\mathbf{\}} \right\}\) izbe-izlik 1 ge itimallıq boyınsha jıynaqlılıǵın kórsetiń.
 \\

\end{tabular}
\vspace{1cm}


\begin{tabular}{m{17cm}}
\textbf{4-variant}
\newline

\textbf{T1.} Itimallıqlar teoriyası aksiomaları (ólshewli keńislik, itimallıq keńisligi).
 \\
\textbf{T2.} Tiykarǵı diskret bólistiriliwler (Binomial, Puasson hám geometriyalıq bólistiriliwler).
 \\
\textbf{A1.} Radio ustaxanada kúnine 2 radio ońlanadı. Radionıń mexanikalıq bólegi buzılıwı itimallıǵı 0,2 ge hám elektron bólegi buzılıw itimallıǵı 0,005 ke teń. Jıl dawamında ońlanǵan radiolar arasında tómendegi waqıyalardıń júzege asıw itimallıqların tabıń: a) 140 tan 150 ge shekem radiolardıń mexanikalıq bóleginde nasazlıqlar bolǵan; b) besten artıq bolmaǵan radiolardıń elektron bóleginde nasazlıqlar bolǵan.
 \\
\textbf{A2.} Birdey úsh qutı berilgen. Birinshi qutıda $a$ dana aq hám $b$ dana qara sharlar, ekinshi qutıda $c$ dana aq hám $d$ dana qara sharlar; úshinshi qutıda bolsa, tek aq sharlar bar. Qutılardan birewi tosınnan tańlanıp, onnan bir shar alındı. a) Usı alınǵan shardıń aq bolıw itimallıǵın tabıń. b) Alınǵan shar aq bolsa, sol shardıń birinshi qutıǵa tiyisli bolıw itimallıǵın tabıń.
 \\
\textbf{A3.} $\left[ 0,1 \right]$ kesindiden tosınnan eki noqat tańlanadı. Olardıń koordinataları qosındısı, koordinataları kóbeymesi eki esesinen kóp bolıw itimallıǵın tabıń.
 \\
\textbf{B1.} Hár biriniń uzınlıǵı a dan aspaytuǵın eki tosınnanlı alınǵan kesindiler qosındısı a dan úlken bolıwı itimallıǵı qanday?
 \\
\textbf{B2.} Bir reyste pоezdǵа bilet alǵan 860 pаssаjirdiń hár biriniń keshigiw itimаllıǵı 0,04 ge teń. Pоezdǵа bilet alǵan pаssаjirlerdiń keshigiwiniń eń itimallı sаnı tabılsın.
 \\
\textbf{B3.} Bólistiriw tiǵizliǵi \(\mathbf{f}\mathbf{(}\mathbf{x}\mathbf{)}\mathbf{=}\frac{\mathbf{1}}{\mathbf{\sigma}\sqrt{\mathbf{2}\mathbf{\pi}}}\mathbf{e}^{\mathbf{-}\frac{\left( \mathbf{x - m} \right)^{\mathbf{2}}}{\mathbf{2}\mathbf{\sigma}^{\mathbf{2}}}}\)bolsa, usı tosınnanlı shamanıń matematikalıq kútiliwin tabılsın.
 \\
\textbf{C1.} Eger ǵárezsiz \includegraphics[width=0.15208in,height=0.24028in]{mediaCpng/image4.png} hám \includegraphics[width=0.15208in,height=0.19167in]{mediaCpng/image5.png} úzliksiz tosınnanlıq shamalardıń hárbiri \includegraphics[width=0.44028in,height=0.21597in]{mediaCpng/image28.png} parametrli kórsetkishli nızam boyınsha bólistirilgen bolsa, onda \includegraphics[width=0.44028in,height=0.24028in]{mediaCpng/image14.png} tosınnanlıq shamanıń tıǵızlıq funkciyasın tabıń.
 \\
\textbf{C2.} Eger \(\xi_{1}\) hám \(\xi_{2}\) sáykes túrde \(\lambda_{1}\) hám \(\lambda_{2}\) parametrli Puasson bólistiriliwine iye bolǵan ǵárezsiz tosınnanlıq shamalar bolsa, onda \(\xi_{1} + \xi_{2}\) tosınnanlıq shamanıń bólistiriliwin tabıń.
 \\
\textbf{C3.} Tómende \includegraphics[width=0.46389in,height=0.25625in]{mediaCpng/image42.png} úzliksiz tosınnanlıq vektorlardıń tıǵızlıq funkciyaları berilgen. Olardıń \includegraphics[width=0.48819in,height=0.29583in]{mediaCpng/image43.png} hám \includegraphics[width=0.50417in,height=0.29583in]{mediaCpng/image44.png} marginal tıǵızlıq funkciyaların tabıń; \includegraphics[width=0.15972in,height=0.24028in]{mediaCpng/image45.png} hám \includegraphics[width=0.15972in,height=0.2in]{mediaCpng/image46.png} tosınnanlıq shamalardı ǵárezsizlikke tekseriń: \includegraphics[width=3.47986in,height=0.6in]{mediaCpng/image70.png}
 \\

\end{tabular}
\vspace{1cm}


\begin{tabular}{m{17cm}}
\textbf{5-variant}
\newline

\textbf{T1.} Bernulli sxemаsı ushın limit teоremаlаr (Muavr-Laplas integrallıq teoreması, qásiyetleri).
 \\
\textbf{T2.} Bólistiriw funkciyası (anıqlaması, tiykarǵı qásiyetleri).
 \\
\textbf{A1.} Detallar partiyası úsh jumısshı tárepinen tayarlanadı. Birinshi jumısshı barlıq detaldıń 25% in, ekinshi jumısshı 35% in, úshinshisi bolsa 40% in tayarlaydı. Bul úsh jumısshınıń tayarlaǵan detallarınıń sapasız bolıwı waqıyası itimallıqları sáykes túrde 0,05; 0,04 hám 0,02 ge teń. a) Tekseriw ushın partiyadan alınǵan detaldıń sapasız bolıw itimallıǵın tabıń. b) Sol sapasız detaldıń ekinshi jumısshı tárepinen tayarlanǵan bolıw itimallıǵın tabıń.
 \\
\textbf{A2.} Hárbiriniń júzege asıw itimallıǵı $p$ ǵa teń bolǵan 10 dana Bernulli tájiriybesi ótkerilgende, tómendegi waqıyalardıń itimallıqların tabıń: Sátliler sanı, sátsizler sanınan ekige artıq.
 \\
\textbf{A3.} $\left[ 0,2 \right]$ kesindiden tosınnan eki noqat tańlanadı. Olardıń koordinataları kóbeymesi 2 den úlken bolıw itimallıǵın tabıń.
 \\
\textbf{B1.} $\xi$ tosınnanlı shamanıń \emph{f}(\emph{x}) tıǵızlıq funkciyasi berilgen bolsin. Tómendegilerdi esaplań: a) C; b) \emph{F}(\emph{x}); c) M$\xi$; d) D$\xi$; e) \emph{f}(\emph{x}) hám \emph{F}(\emph{x}) grafiklarin sızıń.\(f(x) = \left\{ \begin{matrix}
C\sqrt{1 - x},\ \ \ \ x \in \lbrack 0,1\rbrack, \\
\ \ \ \ \ \ \ \ 0,\ \ \ \ \ \ \ x \notin \lbrack 0,1\rbrack.\ \ 
\end{matrix} \right.\ \)
 \\
\textbf{B2.} \includegraphics[width=0.15972in,height=0.23958in]{mediaBpng/image49.png} úzliksiz tosınnanlıq shamanıń tıǵızlıq funkciyaları berilgen. Olarǵa sáykes \includegraphics[width=0.15972in,height=0.19653in]{mediaBpng/image50.png} tosınnanlıq shamanıń \includegraphics[width=0.50278in,height=0.30069in]{mediaBpng/image51.png} tıǵızlıq funkciyasın tabıń. \includegraphics[width=2.40486in,height=0.84028in]{mediaBpng/image54.png} \includegraphics[width=0.85903in,height=0.23958in]{mediaBpng/image55.png}
 \\
\textbf{B3.} Eger \includegraphics[width=0.36181in,height=0.29444in]{mediaBpng/image1.png} ǵárezsiz tosınnanlıq shamalar izbe-izliginiń bólistiriliw nızamları
\includegraphics[width=2.53958in,height=0.49097in]{mediaBpng/image34.png} \includegraphics[width=1.19028in,height=0.47847in]{mediaBpng/image35.png} \includegraphics[width=0.77292in,height=0.25764in]{mediaBpng/image33.png}
bolsa, onda bul izbe-izlik úlken sanlar nızamına boysınama?
 \\
\textbf{C1.} Eger \(\mathbf{\xi}_{\mathbf{n}}\overset{\mathbf{L}^{\mathbf{2}}}{\rightarrow}\mathbf{\xi}\) bolsa, onda \(n \rightarrow \infty\) de \(\mathbf{M}\mathbf{\xi}_{\mathbf{n}}^{\mathbf{2}}\mathbf{\rightarrow M}\mathbf{\xi}^{\mathbf{2}}\) ekenligin kórsetiń.
 \\
\textbf{C2.} Eger \(\xi\sim E(\lambda)\) bolsa, onda \(\xi\) tosınnanlıq shamanıń joqarı tártipli baslanǵısh momentlerin tabıń.
 \\
\textbf{C3.} 
Eger ǵárezsiz \includegraphics[width=0.15208in,height=0.24028in]{mediaCpng/image1.png} hám \includegraphics[width=0.15208in,height=0.19167in]{mediaCpng/image2.png} úzliksiz tosınnanlıq shamalardıń hárbiri standart normal nızam boyınsha bólistirilgen bolsa, onda \includegraphics[width=0.44028in,height=0.24028in]{mediaCpng/image3.png} tosınnanlıq shamanıń tıǵızlıq funkciyasın tabıń.
 \\

\end{tabular}
\vspace{1cm}


\begin{tabular}{m{17cm}}
\textbf{6-variant}
\newline

\textbf{T1.} Waqıyalar algebrası ($\sigma$-algebra, minimal $\sigma$-algebra).
 \\
\textbf{T2.} Tiykarǵı аbsоlyut úzliksiz bólistiriliwler (nоrmаl bólistiriw, teń ólshewli bólistiriw, kórsetkishli bólistiriw). 
 \\
\textbf{A1.} Elektrotexnika dúkanına hár qaysısı 1000 danadan bolǵan joqarı sapalı hám tómen sapalı muzlatqıshlar alıp kelindi. Joqarı sapalı muzlatqıshtıń defektli bolıwı itimallıǵı 0,001 ge, al tómen sapalı muzlatqıshtıń defektli bolıwı itimallıǵı bolsa 0,03 ke teń. a) Eki dana joqarı sapalı muzlatqıshtıń defektli bolıwı; b) 50 den 70 ke shekem tómen sapalı muzlatqıshlardıń defektli bolıwı itimallıqların tabıń.
 \\
\textbf{A2.} 36 dana kartalar kolodasınan tosınnan alınǵan 3 dana karta ishinde anıq 2 dana valet bolıw itimallıǵın tabıń.
 \\
\textbf{A3.} Eki oyın kubigi taslanǵanda túsken ochkolardıń ayırması 1 den úlken bolıw itimallıǵın tabıń.
 \\
\textbf{B1.} Eki teń ku`shli shaxmatshi shaxmat oynaǵanda 4 partiyadan 3 partiyani utıwı itimallig`i kóppe yamasa 8 partiyadan 5 partiyani utiw itimallıǵı kóp pe?
 \\
\textbf{B2.} {[}-1;1{]} kesindide teń ólshewli bólistirilgen tosınnanlı shamanıń xarakteristikalıq funkciyasın tabıń.
 \\
\textbf{B3.} 
Eger \includegraphics[width=0.36181in,height=0.29444in]{mediaBpng/image1.png} ǵárezsiz tosınnanlıq shamalar izbe-izliginiń bólistiriliw nızamları
\includegraphics[width=2.53958in,height=0.49722in]{mediaBpng/image2.png} \includegraphics[width=0.75486in,height=0.23958in]{mediaBpng/image3.png}
bolsa, onda bul izbe-izlik úlken sanlar nızamına boysınama?
 \\
\textbf{C1.} 
\(\xi\) diskret tosınnanlıq shama \(x_{i} = ( - 1)^{i}i\) mánislerdi \(p_{i} = \frac{1}{i(i + 1)},\) \(\ \ i = 1,\ 2,\ ...\) itimallıqlar menen qabıl etse, onıń matematikalıq kútiliwin tabıń.
 \\
\textbf{C2.} Meyli, \(\left\{ \xi_{n} \right\}\) tosınnanlıq shamalar izbe-izligi óziniń \(\left\{ F_{n}(x) \right\}\) bólistiriw funkciyaları menen berilgen bolsın. Sonda hám tek sonda ǵana, eger \(\lim_{n \rightarrow \infty}\int_{- \infty}^{+ \infty}{\frac{x^{2}}{1 + x^{2}}dF_{n}(x)} = 0\) bolsa, onda \(\mathbf{\xi}_{\mathbf{n}}\overset{\mathbf{P}}{\rightarrow}\mathbf{0}\) ekenligin dálilleń.
 \\
\textbf{C3.} Eger \(\left( \xi_{1},\xi_{2} \right)\) absolyut úziliksiz tosınnanlıq vektordıń \(\xi_{1}\) hám \(\xi_{2}\) komponentaları ǵárezsiz bolıp, olardıń hárbiri standart normal bólistirilgen bolsa, onda \(\left( \xi_{1},\xi_{2} \right)\) tosınnanlıq noqattıń \(D = \left\{ (x,y):\ x^{2} + y^{2} \leq R^{2} \right\}\) oblastqa túsiw itimallıǵın tabıń.
 \\

\end{tabular}
\vspace{1cm}


\begin{tabular}{m{17cm}}
\textbf{7-variant}
\newline

\textbf{T1.} Itimallıq anıqlamaları (klassikalıq, geometriyalıq anıqlamaları).
 \\
\textbf{T2.} Tosınnanlı shamanıń dispersiyası (anıqlaması, qásiyetleri).
 \\
\textbf{A1.} 
Albomda 10 dana jańa hám 12 dana múddeti ótken markalar bar. Albomnan tosınnan 3 marka alınıp, múddeti ótkerildi hám ornına qaytarılıp qoyıldı. Bunnan soń, tosınnan 2 marka alındı. a) Bul 2 marka jańa bolıw itimallıǵın tabıń. b) Sol 2 marka jańa ekenligi belgili bolsa, dáslepki alınǵan 3 markanıń jańa bolıw itimallıǵın tabıń.
 \\
\textbf{A2.} Televizion kapital showda individual oyınshınıń 2000 dollar utıp alıwı itimallıǵı 0,4 ke, al 33000 dollar utıp alıwı itimallıǵı bolsa 0,02 ke teń. Bul oyında 400 oyınshı qatnasqan bolsa, tómendegi waqıyalardıń júzege asıwı itimallıqların tabıń: a) 170 ten 190 ǵa shekem oyınshı 2000 dollar utıwı; b) úshten kóp bolmaǵan oyınshı 33 000 dollar utıwı.
 \\
\textbf{A3.} $\left[ 0,1 \right]$ kesindiden tosınnan eki noqat tańlanadı. Olardıń koordinataları qosındısı 1 den úlken bolmaw hám kóbeymesi 0,09 dan kishi bolmaw itimallıǵın tabıń.
 \\
\textbf{B1.} $\xi$ tosınnanlı shamanıń \emph{f}(\emph{x}) tıǵızlıq funkciyasi berilgen bolsin. Tómendegilerdi esaplań: a) C; b) \emph{F}(\emph{x}); c) M$\xi$; d) D$\xi$; e) \emph{f}(\emph{x}) hám \emph{F}(\emph{x}) grafiklarin sızıń.\(f(x) = \left\{ \begin{matrix}
C\cos x,\ \ \ \ x \in \left\lbrack 0,\frac{\pi}{2} \right\rbrack, \\
\ \ \ \ \ \ \ \ 0,\ \ \ \ \ \ x \notin \left\lbrack 0,\frac{\pi}{2} \right\rbrack.\ \ 
\end{matrix} \right.\ \)
 \\
\textbf{B2.} \includegraphics[width=0.15972in,height=0.23958in]{mediaBpng/image49.png} úzliksiz tosınnanlıq shamanıń tıǵızlıq funkciyaları berilgen. Olarǵa sáykes \includegraphics[width=0.15972in,height=0.19653in]{mediaBpng/image50.png} tosınnanlıq shamanıń \includegraphics[width=0.50278in,height=0.30069in]{mediaBpng/image51.png} tıǵızlıq funkciyasın tabıń. \includegraphics[width=2.20278in,height=0.675in]{mediaBpng/image62.png} \includegraphics[width=0.85903in,height=0.23958in]{mediaBpng/image63.png}
 \\
\textbf{B3.} Tosinnanli $\xi$ shamasiniń bólistiriw tiǵizliǵi: \(\mathbf{f}\mathbf{(}\mathbf{x}\mathbf{)}\mathbf{=}\left\{ \begin{matrix}
\mathbf{0,}\mathbf{x <}\mathbf{0} \\
\mathbf{2}\mathbf{e}^{\mathbf{-}\mathbf{2}\mathbf{x}}\mathbf{,}\mathbf{x \geq}\mathbf{0}
\end{matrix} \right.\ \) bolǵanda, M$\xi$ hám D$\xi$ lerdi tabiń.
 \\
\textbf{C1.} Eger ǵárezsiz \includegraphics[width=0.15208in,height=0.24028in]{mediaCpng/image4.png} hám \includegraphics[width=0.15208in,height=0.19167in]{mediaCpng/image5.png} úzliksiz tosınnanlıq shamalar sáykes túrde, \includegraphics[width=0.43194in,height=0.21597in]{mediaCpng/image12.png} hám \includegraphics[width=0.60833in,height=0.26389in]{mediaCpng/image13.png} parametrli kórsetkishli nızam boyınsha bólistirilgen bolsa, onda \includegraphics[width=0.44028in,height=0.24028in]{mediaCpng/image14.png} tosınnanlıq shamanıń tıǵızlıq funkciyasın tabıń.
 \\
\textbf{C2.} Tómende \includegraphics[width=0.46389in,height=0.25625in]{mediaCpng/image42.png} úzliksiz tosınnanlıq vektorlardıń tıǵızlıq funkciyaları berilgen. Olardıń \includegraphics[width=0.48819in,height=0.29583in]{mediaCpng/image43.png} hám \includegraphics[width=0.50417in,height=0.29583in]{mediaCpng/image44.png} marginal tıǵızlıq funkciyaların tabıń; \includegraphics[width=0.15972in,height=0.24028in]{mediaCpng/image45.png} hám \includegraphics[width=0.15972in,height=0.2in]{mediaCpng/image46.png} tosınnanlıq shamalardı ǵárezsizlikke tekseriń: \includegraphics[width=2.52778in,height=0.59167in]{mediaCpng/image54.png}
 \\
\textbf{C3.} Eger \(\xi\) hám \(\chi^{2}\) ǵárezsiz tosınnanlıq shamalar bolıp, \(\xi\sim N(0,1)\) hám \(\chi^{2}\sim\chi^{2}(n)\) bolsa, onda \(\frac{\xi}{\sqrt{\frac{\chi^{2}}{n}}}\) tosınnanlıq shamanıń tıǵızlıq funkciyasın tabıń.
 \\

\end{tabular}
\vspace{1cm}


\begin{tabular}{m{17cm}}
\textbf{8-variant}
\newline

\textbf{T1.} Tolıq itimallıq formulası (waqıyalardıń tolıq gruppası, dálilleniwi).
 \\
\textbf{T2.} 
Úlken sanlar nızamı (anıqlaması, Chebishev teoreması).
 \\
\textbf{A1.} Hárbiriniń júzege asıw itimallıǵı $p$ ǵa teń bolǵan 10 dana Bernulli tájiriybesi ótkerilgende, tómendegi waqıyalardıń itimallıqların tabıń: Sátliler menen sátsizler izbe-iz keliwi.
 \\
\textbf{A2.} “Sportlotto” oyınında qatnasıwshı kartadaǵı 49 sport túrinen 6 danasın belgileydi. Qatnasıwshınıń qura taslaw nátiyjesinde alınǵan 6 sport túrinen keminde 3 danasın durıs boljaǵan bolıw itimallıǵın tabıń.
 \\
\textbf{A3.} Eki oyın kubigi taslanǵanda túsken eń úlken ochko 4 ten úlken bolıw itimallıǵın tabıń.
 \\
\textbf{B1.} $\xi$ tosınnanlı shamanıń \emph{f}(\emph{x}) tıǵızlıq funkciyasi berilgen bolsin. Tómendegilerdi esaplań: a) C; b) \emph{F}(\emph{x}); c) M$\xi$; d) D$\xi$; e) \emph{f}(\emph{x}) hám \emph{F}(\emph{x}) grafiklarin sızıń.\(f(x) = \left\{ \begin{matrix}
\ \ \ \ \ \ \ \ 0,\ \ \ \ \ \ x \leq 0, \\
Cxe^{- 0.5x},\ \ \ \ \ x > 0.\ \ 
\end{matrix} \right.\ \)
 \\
\textbf{B2.} Eger \includegraphics[width=0.36181in,height=0.29444in]{mediaBpng/image1.png} ǵárezsiz tosınnanlıq shamalar izbe-izliginiń bólistiriliw nızamları
\includegraphics[width=2.53958in,height=0.47847in]{mediaBpng/image25.png} \includegraphics[width=0.75486in,height=0.23958in]{mediaBpng/image9.png}
bolsa, onda bul izbe-izlik úlken sanlar nızamına boysınama?
 \\
\textbf{B3.} Bir nishanaǵa úsh márte atiladi. I, II hám III márte atqandaǵi tiyiw itimalliqlari sáykes: p\textsubscript{1} =0,3; p\textsubscript{2} =0,4; p\textsubscript{3}=0,6. Usi úsh márte atıwdıń nátiyjesinde nishanada eń bolmaǵanda bir oq izi boliwiniń itimallıǵın tabıń.
 \\
\textbf{C1.} Hárqanday \(\varphi_{\xi}(t)\) xarakteristikalıq funkciya ushın \(t \in R\) de \(1 - Re\varphi_{\xi}(2t) \leq 4\left( 1 - Re\varphi_{\xi}(t) \right)\) ekenligin dálilleń.
 \\
\textbf{C2.} Tómende \includegraphics[width=0.46389in,height=0.25625in]{mediaCpng/image42.png} úzliksiz tosınnanlıq vektorlardıń tıǵızlıq funkciyaları berilgen. Olardıń \includegraphics[width=0.48819in,height=0.29583in]{mediaCpng/image43.png} hám \includegraphics[width=0.50417in,height=0.29583in]{mediaCpng/image44.png} marginal tıǵızlıq funkciyaların tabıń; \includegraphics[width=0.15972in,height=0.24028in]{mediaCpng/image45.png} hám \includegraphics[width=0.15972in,height=0.2in]{mediaCpng/image46.png} tosınnanlıq shamalardı ǵárezsizlikke tekseriń: \includegraphics[width=3.54375in,height=0.81597in]{mediaCpng/image71.png} \\
\textbf{C3.} Eger ǵárezsiz \includegraphics[width=0.15208in,height=0.24028in]{mediaCpng/image4.png} hám \includegraphics[width=0.15208in,height=0.19167in]{mediaCpng/image5.png} úzliksiz tosınnanlıq shamalar sáykes túrde, \includegraphics[width=0.43194in,height=0.21597in]{mediaCpng/image12.png} hám \includegraphics[width=0.60833in,height=0.26389in]{mediaCpng/image13.png} parametrli kórsetkishli nızam boyınsha bólistirilgen bolsa, onda \includegraphics[width=0.91181in,height=0.29583in]{mediaCpng/image15.png} tosınnanlıq shamanıń tıǵızlıq funkciyasın tabıń.
 \\

\end{tabular}
\vspace{1cm}


\begin{tabular}{m{17cm}}
\textbf{9-variant}
\newline

\textbf{T1.} Tosınnanlı waqıya (elementar waqıyalar keńisligi, waqıyalar ústinde ámeller).
 \\
\textbf{T2.} Tosınnanlı shamanıń matematikalıq kútiliwi. (anıqlaması, qásiyetleri)
 \\
\textbf{A1.} Bazaǵa 360 dana buyım keltirilgen. Bulardan: 300 danası 1-kárxanada tayarlanǵan bolıp, 250 danası jaramlı; 40 danası 2-kárxanada tayarlanǵan bolıp, 30 danası jaramlı; 20 danası 3-kárxanada tayarlanǵan bolıp, 10 danası jaramlı. a) Bazadan tosınnan alınǵan buyımnıń jaramlı bolıw itimallıǵın tabıń. b) Eger bazadan alınǵan buyım jaramlı bolsa, onda sol buyımnıń 2-kárxanaǵa tiyisli bolıw itimallıǵın tabıń.
 \\
\textbf{A2.} ${{x}^{2}}+2px+q=0$ kvadrat teńlemede $p$ hám $q$ koefficientler $\left[ -1,1 \right]$ kesindiden tosınnan tańlanadı. Kvadrat teńlemeniń oń túbirlerge iye bolıw itimallıǵın tabıń.
 \\
\textbf{A3.} Úsh oyın kubigi taslanǵanda túsken ochkolardıń qosındısı 11 ǵe teń bolıw itimallıǵın tabıń.
 \\
\textbf{B1.} \includegraphics[width=0.15972in,height=0.23958in]{mediaBpng/image49.png} úzliksiz tosınnanlıq shamanıń tıǵızlıq funkciyaları berilgen. Olarǵa sáykes \includegraphics[width=0.15972in,height=0.19653in]{mediaBpng/image50.png} tosınnanlıq shamanıń \includegraphics[width=0.50278in,height=0.30069in]{mediaBpng/image51.png} tıǵızlıq funkciyasın tabıń. \includegraphics[width=2.40486in,height=0.84028in]{mediaBpng/image58.png} \includegraphics[width=0.84028in,height=0.23958in]{mediaBpng/image59.png}
 \\
\textbf{B2.} \includegraphics[width=0.15972in,height=0.23958in]{mediaBpng/image49.png} úzliksiz tosınnanlıq shamanıń tıǵızlıq funkciyaları berilgen. Olarǵa sáykes \includegraphics[width=0.15972in,height=0.19653in]{mediaBpng/image50.png} tosınnanlıq shamanıń \includegraphics[width=0.50278in,height=0.30069in]{mediaBpng/image51.png} tıǵızlıq funkciyasın tabıń. \includegraphics[width=2.62569in,height=0.65in]{mediaBpng/image84.png} \includegraphics[width=0.66875in,height=0.53403in]{mediaBpng/image85.png}
 \\
\textbf{B3.} Oqıtıwshı joqarı matematikadan shegaralıq bahalaw alıw ushın 50 soraw tayarlaǵan. Olardıń ishinde differencial esabınan 20 soraw, integral esabınan 18 soraw, itimallıqlar teoriyasınan 12 soraw bar. Student differencial esabınan 18 sorawǵa, integral esabınan 15 sorawǵa, itimallıqlar teoriyasınan 10 sorawǵa juwap bere alatuǵın bolsa, onıń dus kelgen bir sorawǵa juwap berip, shegaralıq bahalaw tapsırıwınıń itimallıǵın tabıń.
 \\
\textbf{C1.} Eger \(\left\{ \xi_{n} \right\}\) diskret tosınnanlıq shamalar izbe-izliginiń bólistiriliw nızamları\(P\left\{ \xi_{n} = 1 \right\} = P\left\{ \xi_{n} = - 1 \right\} = \frac{1}{2} - \frac{1}{n},\) \(P\left\{ \xi_{n} = 0 \right\} = \frac{2}{n}\) bolsa, onda \(\xi_{n}\overset{d}{\rightarrow}\xi\) bolatuǵın \(\xi\) tosınnanlıq shamanıń bólistiriw funkciyasın tabıń.
 \\
\textbf{C2.} Kóp ólshemli tıǵızlıq funkciyası óziniń marginal tıǵızlıq funkciyaları arqalı bir mánisli anıqlanbaytuǵınlıǵın kórsetiń.
 \\
\textbf{C3.} Eger \(\left\{ \xi_{n} \right\}\) ǵárezsiz tosınnanlıq shamalar izbe-izliginiń bólistiriw funkciyaları \(F_{n}(x) = \left\{ \begin{matrix}
\ 1 - \frac{1}{x + n},\ \ eger\ \ x > 0 \\
 \\
 \\
\ \ \ \ \ \ \ \ \ \ 0,\ \ \ \ \ \ \ \ \ \ \ eger\ \ x \leq 0
\end{matrix} \right.\ \) bolsa, onda bul izbe-izliktiń 0 ge itimallıq boyınsha jıynaqlılıǵın kórsetiń.
 \\

\end{tabular}
\vspace{1cm}


\begin{tabular}{m{17cm}}
\textbf{10-variant}
\newline

\textbf{T1.} Shártli itimallıq (anıqlaması, kóbеytiw tеorеması).
 \\
\textbf{T2.} Kompoziciyalıq formulalar \\
\textbf{A1.} Firmada 7 erkek hám 3 hayal jumısshı isleydi. Tosınnan 3 jumısshı ajıratılıp alındı. Ajıratılıp alınǵan jumısshılardıń barlıǵı erkekler bolıw itimallıǵın tabıń.
 \\
\textbf{A2.} Hárbiriniń júzege asıw itimallıǵı $p$ ǵa teń bolǵan 10 dana Bernulli tájiriybesi ótkerilgende, tómendegi waqıyalardıń itimallıqların tabıń: Sátliler sanı 3 dana, sonıń menen birge, sońǵı tájiriybe sátli juwmaqlanıwı.
 \\
\textbf{A3.} Uzaq aralıqtaǵı nıshanǵa pulemyot hám pistolet penen oq atılmaqta. Pulemyot oǵınıń nıshanǵa tiyiw itimallıǵı 0,02, al pistolet penen bolsa 0,6 ǵa teń. Eger hárbir qural menen nıshanǵa 100 márte oq atılǵan bolsa, tómendegi waqıyalardıń júzege asıw itimallıqların tabıń: a) pulemyot penen nıshanǵa úsh márte tiygiziwi; b) pistolet penen nıshanǵa 17 den 22 mártege shekem tiygiziwi.
 \\
\textbf{B1.} $\xi$ tosınnanlı shamanıń \emph{f}(\emph{x}) tıǵızlıq funkciyasi berilgen bolsin. Tómendegilerdi esaplań: a) C; b) \emph{F}(\emph{x}); c) M$\xi$; d) D$\xi$; e) \emph{f}(\emph{x}) hám \emph{F}(\emph{x}) grafiklarin sızıń.\(f(x) = \left\{ \begin{matrix}
(x + 1)/2,\ \ \ \ x \in \lbrack - 1,0\rbrack, \\
(C - x)/2C,\ x \in (0,C\rbrack, \\
\ \ \ \ \ \ 0,\ \ \ \ \ \ \ \ \ \ \ \ x \notin \lbrack - 1,C\rbrack\ \ 
\end{matrix} \right.\ \)
 \\
\textbf{B2.} Eger \includegraphics[width=0.36181in,height=0.29444in]{mediaBpng/image1.png} ǵárezsiz tosınnanlıq shamalar izbe-izliginiń bólistiriliw funkciyaları
\includegraphics[width=1.79167in,height=0.52153in]{mediaBpng/image48.png}
kórinislerinde bolsa, onda bul izbe-izlik úlken sanlar nızamına boysınama?
 \\
\textbf{B3.} \(f(x) = C \cdot e^{- \frac{x^{2}}{m}}\) tiǵizliq funkciyasi boliwi ushin \emph{C} nege teń boliwi kerek?
 \\
\textbf{C1.} Tómende \includegraphics[width=0.46389in,height=0.25625in]{mediaCpng/image42.png} úzliksiz tosınnanlıq vektorlardıń tıǵızlıq funkciyaları berilgen. Olardıń \includegraphics[width=0.48819in,height=0.29583in]{mediaCpng/image43.png} hám \includegraphics[width=0.50417in,height=0.29583in]{mediaCpng/image44.png} marginal tıǵızlıq funkciyaların tabıń; \includegraphics[width=0.15972in,height=0.24028in]{mediaCpng/image45.png} hám \includegraphics[width=0.15972in,height=0.2in]{mediaCpng/image46.png} tosınnanlıq shamalardı ǵárezsizlikke tekseriń: \includegraphics[width=3.28819in,height=0.84792in]{mediaCpng/image60.png}
 \\
\textbf{C2.} Eger \(\xi\) tosınnanlıq shama standart Koshi bólistiriliwine iye bolsa, onda \(M\min\left( |\xi|,1 \right)\) mánisin tabıń.
 \\
\textbf{C3.} Eger ǵárezsiz \includegraphics[width=0.15208in,height=0.24028in]{mediaCpng/image4.png} hám \includegraphics[width=0.15208in,height=0.19167in]{mediaCpng/image5.png} úzliksiz tosınnanlıq shamalardıń hárbiri \includegraphics[width=0.44028in,height=0.21597in]{mediaCpng/image28.png} parametrli kórsetkishli nızam boyınsha bólistirilgen bolsa, onda \includegraphics[width=0.44028in,height=0.24028in]{mediaCpng/image29.png} tosınnanlıq shamanıń tıǵızlıq funkciyasın tabıń.
 \\

\end{tabular}
\vspace{1cm}


\begin{tabular}{m{17cm}}
\textbf{11-variant}
\newline

\textbf{T1.} Bernulli sxemаsı ushın limit teоremаlаr (Puasson bólistiriliwi, qásiyetleri).
 \\
\textbf{T2.} Oraylıq limit teorema (anıqlaması, ǵárezsiz birdey bólistirilgen tosınnanlı shamalar ushın).
 \\
\textbf{A1.} 150 dana buyımnan ibarat partiyada 5 dana buyım jaramsız. Partiyadan tosınnan 12 dana buyım alınadı. Usı alınǵan 12 dana buyımnıń ishinde 3 dana buyımnıń jaramsız bolıw itimallıǵın tabıń. 
 \\
\textbf{A2.} Dúkan 2000 dana televizor hám 2000 dana radio satıp aldı. Hárbir televizordıń defektli bolıw itimallıǵı 0,004 ke hám hárbir radionıń defektli bolıw itimallıǵı 0,03 ke teń. Usı sawdada tómendegi waqıyalardıń júzege asıwı itimallıqların tabıń: a) keminde úsh televizor defektli bolıwı; b) 33 ten 44 ke shekem radio defektli bolıwı.
 \\
\textbf{A3.} $\left[ -1,2 \right]$ kesindiden tosınnan eki noqat tańlanadı. Olardıń koordinataları qosındısı 1 den úlken bolıw hám kóbeymesi 1 den kishi bolıw itimallıǵın tabıń.
 \\
\textbf{B1.} \includegraphics[width=0.15972in,height=0.23958in]{mediaBpng/image49.png} úzliksiz tosınnanlıq shamanıń tıǵızlıq funkciyaları berilgen. Olarǵa sáykes \includegraphics[width=0.15972in,height=0.19653in]{mediaBpng/image50.png} tosınnanlıq shamanıń \includegraphics[width=0.50278in,height=0.30069in]{mediaBpng/image51.png} tıǵızlıq funkciyasın tabıń. \includegraphics[width=2.325in,height=0.84028in]{mediaBpng/image68.png} \includegraphics[width=0.66875in,height=0.23958in]{mediaBpng/image69.png}
 \\
\textbf{B2.} Qutıda 4 aq hám 5 qara sharlar bar. Qutıdan izbe-iz 2 shar alınadi. Alinǵan 2 shardiń birinshisi aq ekinshisi qara shar boliw itimallıǵın tabıń.
 \\
\textbf{B3.} Eger \includegraphics[width=0.36181in,height=0.29444in]{mediaBpng/image1.png} ǵárezsiz tosınnanlıq shamalar izbe-izliginiń bólistiriliw nızamları
\includegraphics[width=2.53958in,height=0.47847in]{mediaBpng/image26.png} \includegraphics[width=0.75486in,height=0.23958in]{mediaBpng/image9.png}
bolsa, onda bul izbe-izlik úlken sanlar nızamına boysınama?
 \\
\textbf{C1.} Tosınnanlıq vektordıń komponentaları absolyut úziliksizliginen tosınnanlıq vektordıń ózi de absolyut úziliksizligi kelip shıqpaslıǵın kórsetiń.
 \\
\textbf{C2.} Eger \(\xi\) tosınnanlıq shama \(\lbrack 0,\ \pi\rbrack\) aralıqta teń ólshewli bólistirilgen bolsa, onda \(M\sin\xi,\) \(D\sin\xi\) hám \(M\cos\xi,\) \(D\cos\xi\) mánislerin tabıń.
 \\
\textbf{C3.} Eger ǵárezsiz \includegraphics[width=0.15208in,height=0.24028in]{mediaCpng/image4.png} hám \includegraphics[width=0.15208in,height=0.19167in]{mediaCpng/image5.png} úzliksiz tosınnanlıq shamalar sáykes túrde, \includegraphics[width=0.6in,height=0.26389in]{mediaCpng/image16.png} hám \includegraphics[width=0.6in,height=0.26389in]{mediaCpng/image17.png} parametrli kórsetkishli nızam boyınsha bólistirilgen bolsa, onda \includegraphics[width=0.44028in,height=0.24028in]{mediaCpng/image14.png} tosınnanlıq shamanıń tıǵızlıq funkciyasın tabıń.
 \\

\end{tabular}
\vspace{1cm}


\begin{tabular}{m{17cm}}
\textbf{12-variant}
\newline

\textbf{T1.} Bernulli sxemаsı ushın limit teоremаlаr (Puasson bólistiriliwi, qásiyetleri).
 \\
\textbf{T2.} Tosınnanlı shamanıń dispersiyası (anıqlaması, qásiyetleri).
 \\
\textbf{A1.} Eki oyın kubigi taslanǵanda túsken ochkolardıń ayırması 3 ten úlken bolıw itimallıǵın tabıń.
 \\
\textbf{A2.} Oyın kubigi taslandı. Meyli, $m$ - túsken ochkolar sanı bolsın. Soń, nıshanǵa qarata hárbir atıwda $p$ tiyiw itimallıǵı menen $2m$ márte oq atıladı. a) Nıshanǵa eki márte oq tiyiw itimallıǵın tabıń. b) Nıshanǵa eki márte oq tiygen bolsa, $m=3$ bolıw itimallıǵın tabıń.
 \\
\textbf{A3.} Hárbiriniń júzege asıw itimallıǵı $p$ ǵa teń bolǵan 10 dana Bernulli tájiriybesi ótkerilgende, tómendegi waqıyalardıń itimallıqların tabıń: Sátsizler sanı tek 2 dana hám olar arasında 4 dana sátli bolıw.
 \\
\textbf{B1.} Eger \(\xi\) tosınnanlı shama \(\lambda\) parametrli puasson bólistiriwine iye bolsa, onda onıń xarakteristikalıq funkciyası tabılsın.
 \\
\textbf{B2.} $\xi$ tosınnanlı shamanıń \emph{f}(\emph{x}) tıǵızlıq funkciyasi berilgen bolsin. Tómendegilerdi esaplań: a) C; b) \emph{F}(\emph{x}); c) M$\xi$; d) D$\xi$; e) \emph{f}(\emph{x}) hám \emph{F}(\emph{x}) grafiklarin sızıń.\(f(x) = \left\{ \begin{matrix}
C\ln x,\ \ \ \ x \in \lbrack 1,e\rbrack, \\
\ \ \ \ 0,\ \ \ \ \ \ \ x \notin \lbrack 1,e\rbrack.\ \ 
\end{matrix} \right.\ \)
 \\
\textbf{B3.} $\xi$ tosınnanlı shamanıń \emph{f}(\emph{x}) tıǵızlıq funkciyasi berilgen bolsin. Tómendegilerdi esaplań: a) C; b) \emph{F}(\emph{x}); c) M$\xi$; d) D$\xi$; e) \emph{f}(\emph{x}) hám \emph{F}(\emph{x}) grafiklarin sızıń.\(f(x) = \left\{ \begin{matrix}
\ \ \ \ \ \ \ \ 0,\ \ \ \ \ \ x < 1, \\
Ce^{1 - x},\ \ \ \ \ x \geq 1.\ \ 
\end{matrix} \right.\ \)
 \\
\textbf{C1.} Oraylıq limit teorema járdeminde tómendegi teńlikti dálilleń: \(\lim_{n \rightarrow \infty}e^{- n}\sum_{k = 1}^{n}\frac{n^{k}}{k!} = \frac{1}{2}.\)
 \\
\textbf{C2.} Tómende \includegraphics[width=0.46389in,height=0.25625in]{mediaCpng/image42.png} úzliksiz tosınnanlıq vektorlardıń tıǵızlıq funkciyaları berilgen. Olardıń \includegraphics[width=0.48819in,height=0.29583in]{mediaCpng/image43.png} hám \includegraphics[width=0.50417in,height=0.29583in]{mediaCpng/image44.png} marginal tıǵızlıq funkciyaların tabıń; \includegraphics[width=0.15972in,height=0.24028in]{mediaCpng/image45.png} hám \includegraphics[width=0.15972in,height=0.2in]{mediaCpng/image46.png} tosınnanlıq shamalardı ǵárezsizlikke tekseriń: \includegraphics[width=3.13611in,height=0.59167in]{mediaCpng/image49.png}
 \\
\textbf{C3.} Eger \(\xi_{1},\xi_{2}...,\xi_{n}\) ǵárezsiz birdey bólistirilgen tosınnanlıq shamalar standart normal bólistirilgen bolsa, onda \(\xi_{1}^{2} + \xi_{2}^{2} + ...\  + \xi_{n}^{2}\) tosınnanlıq shamanıń tıǵızlıq funkciyasın tabıń.
 \\

\end{tabular}
\vspace{1cm}


\begin{tabular}{m{17cm}}
\textbf{13-variant}
\newline

\textbf{T1.} Bernulli sxemаsı ushın limit teоremаlаr (Muavr-Laplas integrallıq teoreması, qásiyetleri).
 \\
\textbf{T2.} Oraylıq limit teorema (anıqlaması, ǵárezsiz birdey bólistirilgen tosınnanlı shamalar ushın).
 \\
\textbf{A1.} Qutıda 10 dana aq hám 15 dana qara sharlar bar. Tosınnan 5 dana shar alınǵanda, olar ishinde 2 dana aq shar bolıw itimallıǵın tabıń.
 \\
\textbf{A2.} Eki oyın kubigi taslanǵanda túsken ochkolardıń qosındısı 6 dan úlken bolıw itimallıǵın tabıń.
 \\
\textbf{A3.} $\left[ 0,1 \right]$ kesindiden tosınnan eki noqat tańlanadı. Birinshi hám ekinshi noqatlar koordinataları arasındaǵı aralıq $0,7$ den kishi bolıw itimallıǵın tabıń.
 \\
\textbf{B1.} \(f(x) = C \cdot e^{- \frac{(x - m)^{2}}{7}}\) tıǵızlıq funkciyası bolıwı ushin \emph{C} nege teń bolıwı kerek?
 \\
\textbf{B2.} \includegraphics[width=0.15972in,height=0.23958in]{mediaBpng/image49.png} úzliksiz tosınnanlıq shamanıń tıǵızlıq funkciyaları berilgen. Olarǵa sáykes \includegraphics[width=0.15972in,height=0.19653in]{mediaBpng/image50.png} tosınnanlıq shamanıń \includegraphics[width=0.50278in,height=0.30069in]{mediaBpng/image51.png} tıǵızlıq funkciyasın tabıń. \includegraphics[width=1.95069in,height=0.58889in]{mediaBpng/image96.png} \includegraphics[width=0.55208in,height=0.28194in]{mediaBpng/image97.png}
 \\
\textbf{B3.} Eger \includegraphics[width=0.36181in,height=0.29444in]{mediaBpng/image1.png} ǵárezsiz tosınnanlıq shamalar izbe-izliginiń bólistiriliw nızamları
\includegraphics[width=2.50278in,height=0.50278in]{mediaBpng/image4.png} \includegraphics[width=1.16597in,height=0.50278in]{mediaBpng/image5.png} \includegraphics[width=0.75486in,height=0.23958in]{mediaBpng/image6.png}
bolsa, onda bul izbe-izlik úlken sanlar nızamına boysınama?
 \\
\textbf{C1.} Eger ǵárezsiz \includegraphics[width=0.15208in,height=0.24028in]{mediaCpng/image4.png} hám \includegraphics[width=0.15208in,height=0.19167in]{mediaCpng/image5.png} úzliksiz tosınnanlıq shamalardıń hárbiri \includegraphics[width=0.19167in,height=0.16806in]{mediaCpng/image31.png} parametrli
kórsetkishli nızam boyınsha bólistirilgen bolsa, onda \includegraphics[width=0.47986in,height=0.52014in]{mediaCpng/image40.png} tosınnanlıq shamanıń
tıǵızlıq funkciyasın tabıń.
 \\
\textbf{C2.} Meyli, \(\xi_{1},...,\xi_{n}\) tosınnanlıq shamalar ǵárezsiz hám \(\lbrack a,b\rbrack\) aralıqta teń ólshemli bólistirilgen bolıp, \(\eta_{1} = \max\left( \xi_{1},...,\xi_{n} \right)\) hám \(\eta_{2} = \min\left( \xi_{1},...,\xi_{n} \right)\) bolsın. Onda \(\left( \eta_{1},\eta_{2} \right)\) tosınnanlıq vektordıń kovariaciyasın tabıń.
 \\
\textbf{C3.} Tómende \includegraphics[width=0.46389in,height=0.25625in]{mediaCpng/image42.png} úzliksiz tosınnanlıq vektorlardıń tıǵızlıq funkciyaları berilgen. Olardıń \includegraphics[width=0.48819in,height=0.29583in]{mediaCpng/image43.png} hám \includegraphics[width=0.50417in,height=0.29583in]{mediaCpng/image44.png} marginal tıǵızlıq funkciyaların tabıń; \includegraphics[width=0.15972in,height=0.24028in]{mediaCpng/image45.png} hám \includegraphics[width=0.15972in,height=0.2in]{mediaCpng/image46.png} tosınnanlıq shamalardı ǵárezsizlikke tekseriń: \includegraphics[width=3.28819in,height=0.73611in]{mediaCpng/image69.png}
 \\

\end{tabular}
\vspace{1cm}


\begin{tabular}{m{17cm}}
\textbf{14-variant}
\newline

\textbf{T1.} Tosınnanlı waqıya (elementar waqıyalar keńisligi, waqıyalar ústinde ámeller).
 \\
\textbf{T2.} Tosınnanlı shamanıń matematikalıq kútiliwi. (anıqlaması, qásiyetleri)
 \\
\textbf{A1.} Uzaq aralıqtaǵı nıshanǵa pulemyot hám pistolet penen oq atılmaqta. Pulemyot oǵınıń nıshanǵa tiyiw itimallıǵı 0,02, al pistolet penen bolsa 0,6 ǵa teń. Eger hárbir qural menen nıshanǵa 100 márte oq atılǵan bolsa, tómendegi waqıyalardıń júzege asıwı itimallıqların tabıń: a) pulemyot penen nıshanǵa tórt márte tiygiziwi; b) pistolet penen nıshanǵa 70 márte tiygiziwi. \\
\textbf{A2.} Hárbiriniń júzege asıw itimallıǵı $p$ ǵa teń bolǵan 10 dana Bernulli tájiriybesi ótkerilgende, tómendegi waqıyalardıń itimallıqların tabıń: Sátliler sanı 3 dana, sonıń menen birge, olardıń barlıǵı tájiriybelerdiń ekinshi yarımında ámelge asıwı.
 \\
\textbf{A3.} Mikrosxemalardıń 10% i nuqsanlı jaǵdayda bolıp, olar tekseriwden ótkerildi. Ápiwayılastırılǵan tekseriw sınaǵı ótkerildi. Bul tekseriw tómendegishe itimallıqta qátelikke jol qoyadı, yaǵnıy 0,95 itimallıq penen nuqsanlı mikrosxemanı nuqsanlı dep tabadı hám 0,03 itimallıq penen nuqsansız mikrosxemanı nuqsanlı dep tabadı. a) Tekseriwden ótkerilgen mikrosxemanıń nuqsanlı dep tabılıw itimallıǵın tabıń. b) Bul mikrosxemanıń negizinde nuqsansız bolıwı itimallıǵı qanday?
 \\
\textbf{B1.} Radiusı \(r\ (2r < a)\) bolǵan tiyin tosınnanlı túrde tárepi a bolǵan kvadratlarǵa bólingen stolǵa taslandı. Taslanǵan tiyin kvadrattıń bazı bir tárepin kesip ótpewi itimallıǵın tabıń.
 \\
\textbf{B2.} $\xi$ tosınnanlı shamanıń \emph{f}(\emph{x}) tıǵızlıq funkciyasi berilgen bolsin. Tómendegilerdi esaplań: a) C; b) \emph{F}(\emph{x}); c) M$\xi$; d) D$\xi$; e) \emph{f}(\emph{x}) hám \emph{F}(\emph{x}) grafiklarin sızıń.\(f(x) = \left\{ \begin{matrix}
C(1 - |x|),\ \ \ \ x \in \lbrack - 1,1\rbrack, \\
\ \ \ \ \ \ \ \ 0,\ \ \ \ \ \ \ \ \ x \notin \lbrack - 1,1\rbrack.\ \ 
\end{matrix} \right.\ \)
 \\
\textbf{B3.} \includegraphics[width=0.15972in,height=0.23958in]{mediaBpng/image49.png} úzliksiz tosınnanlıq shamanıń tıǵızlıq funkciyaları berilgen. Olarǵa sáykes \includegraphics[width=0.15972in,height=0.19653in]{mediaBpng/image50.png} tosınnanlıq shamanıń \includegraphics[width=0.50278in,height=0.30069in]{mediaBpng/image51.png} tıǵızlıq funkciyasın tabıń. \includegraphics[width=1.95069in,height=0.58264in]{mediaBpng/image70.png} \includegraphics[width=0.85903in,height=0.23958in]{mediaBpng/image71.png}
 \\
\textbf{C1.} Eger \(\mathbf{\xi}_{\mathbf{n}}\overset{\mathbf{L}^{\mathbf{2}}}{\rightarrow}\mathbf{\xi}\) bolsa, onda \(n \rightarrow \infty\) de \(\mathbf{M}\mathbf{\xi}_{\mathbf{n}}\mathbf{\rightarrow M\xi}\) ekenligin kórsetiń.
 \\
\textbf{C2.} 
Eger \(\xi\) tosınnanlıq shama hám \(\left\{ \xi_{n} \right\}\) tosınnanlıq shamalar izbe-izligi ǵárezsiz birdey standart normal bólistirilgen bolsa, onda \(\left\{ \mathbf{\eta}_{\mathbf{n}} \right\}\mathbf{=}\left\{ \frac{\mathbf{\xi}\sqrt{\mathbf{n}}}{\sqrt{\mathbf{\xi}_{\mathbf{1}}^{\mathbf{2}}\mathbf{+}\mathbf{...}\mathbf{+}\mathbf{\xi}_{\mathbf{n}}^{\mathbf{2}}}} \right\}\) tosınnanlıq shamalar izbe-izligining limit bólistiriw funkciyası standart normal bólistiriliw bolıwın kórsetiń.
 \\
\textbf{C3.} Ortasha mánis vektorı \(\left( m_{1},m_{2} \right)\) hám kovariaciyalıq matricası\(K = \begin{pmatrix}
\sigma_{1}^{2} & r\sigma_{1}\sigma_{2} \\
r\sigma_{1}\sigma_{2} & \sigma_{2}^{2}
\end{pmatrix},\ \ \sigma_{1},\ \sigma_{2} > 0,\ \ |r|\  < 1\) bolǵan normal bólistirilgen \(\left( \xi_{1},\xi_{2} \right)\) tosınnanlıq vektordıń tıǵızlıq funkciyasın tabıń.
 \\

\end{tabular}
\vspace{1cm}


\begin{tabular}{m{17cm}}
\textbf{15-variant}
\newline

\textbf{T1.} Bernulli sxemаsı ushın limit teоremаlаr (Muavr-Laplas lokallıq teoreması, qásiyetleri).
 \\
\textbf{T2.} Xarakteristikalıq funkciyalar (anıqlaması, tiykarǵı qásiyetleri).
 \\
\textbf{A1.} Oyın kubigi taslanǵanda 7 ochkonıń túsiw itimallıǵın tabıń.
 \\
\textbf{A2.} Qutıda 90 dana sapalı hám 10 dana sapasız detallar bar. Qutıdan tosınnan alınǵan 10 dana detaldıń ishinde sapasız detaldıń joq bolıw itimallıǵın tabıń.
 \\
\textbf{A3.} Berilgen $1,2,\ldots ,10$ sanlarınıń arasınan tosınnan bir san tańlandı. Meyli, bul san $m$ bolsın. Keyin, $1,2,\ldots ,m$ sanlarınıń arasınan tosınnan bir san tańlandı. a) Bul sannıń 8 ge teń bolıw itimallıǵın tabıń. b) Bul san 8 ge teń bolsa, $m=9$ bolıw itimallıǵın tabıń.
 \\
\textbf{B1.} Eger \includegraphics[width=0.36181in,height=0.29444in]{mediaBpng/image1.png} ǵárezsiz tosınnanlıq shamalar izbe-izliginiń bólistiriliw nızamları
\includegraphics[width=1.51528in,height=0.50278in]{mediaBpng/image19.png} \includegraphics[width=1.62569in,height=0.50278in]{mediaBpng/image20.png} \includegraphics[width=0.75486in,height=0.23958in]{mediaBpng/image9.png}
bolsa, onda bul izbe-izlik úlken sanlar nızamına boysınama?
 \\
\textbf{B2.} Abonent telefon nomerin terip atırıp, aqırǵı úsh cifrdi umıtıp qaldı hám bul nomerlerdiń hár túrli ekenligin eslep olardı táwekeline terdi. Kerekli nomerler terilgen bolıwı itimallıǵın tabıń.
 \\
\textbf{B3.} Eger \(\xi\) tosınnanlı shama \((n,p)\) parametrli binomial bólistiriwine iye bolsa, onda onıń xarakteristikalıq funkciyası tabılsın.
 \\
\textbf{C1.} Eger ǵárezsiz \includegraphics[width=0.15208in,height=0.24028in]{mediaCpng/image4.png} hám \includegraphics[width=0.15208in,height=0.19167in]{mediaCpng/image5.png} úzliksiz tosınnanlıq shamalardıń tıǵızlıq fukciyaları sáykes túrde,
\includegraphics[width=1.6in,height=0.50417in]{mediaCpng/image38.png} hám \includegraphics[width=1.64028in,height=0.52014in]{mediaCpng/image39.png}
bolsa, onda \includegraphics[width=0.35972in,height=0.27222in]{mediaCpng/image27.png} tosınnanlıq shamanıń tıǵızlıq funkciyasın tabıń.
 \\
\textbf{C2.} 
Tómende \includegraphics[width=0.46389in,height=0.25625in]{mediaCpng/image42.png} úzliksiz tosınnanlıq vektorlardıń tıǵızlıq funkciyaları berilgen. Olardıń \includegraphics[width=0.48819in,height=0.29583in]{mediaCpng/image43.png} hám \includegraphics[width=0.50417in,height=0.29583in]{mediaCpng/image44.png} marginal tıǵızlıq funkciyaların tabıń; \includegraphics[width=0.15972in,height=0.24028in]{mediaCpng/image45.png} hám \includegraphics[width=0.15972in,height=0.2in]{mediaCpng/image46.png} tosınnanlıq shamalardı ǵárezsizlikke tekseriń: \includegraphics[width=3.13611in,height=0.53611in]{mediaCpng/image47.png}
 \\
\textbf{C3.} Eger \(\xi_{1}\) hám \(\xi_{2}\) ǵárezsiz tosınnanlıq shamalardıń hárbiri standart normal bólistirilgen bolsa, onda \(\xi_{1} + \xi_{2}\) tosınnanlıq shamanıń tıǵızlıq funkciyasın tabıń.
 \\

\end{tabular}
\vspace{1cm}


\begin{tabular}{m{17cm}}
\textbf{16-variant}
\newline

\textbf{T1.} Itimallıq anıqlamaları (klassikalıq, geometriyalıq anıqlamaları).
 \\
\textbf{T2.} Tosınnanlı shamanıń joqarı tártipli momentleri (baslanǵısh hám oraylıq momentleri, qásiyetleri).
 \\
\textbf{A1.} $x\in \left[ 0,2\pi  \right]$ ushın $2co{{s}^{2}}x-5cosx+1>0$ bolıw itimallıǵın tabıń.
 \\
\textbf{A2.} Jip iyiriw fabrikasında 1500 dana jańa hám 100 dana eski jip iyiriw qurılmaları bar. Bir jumıs kúninde jańa qurılma 0,002 itimallıq penen, al eski qurılma bolsa 0,30 itimallıq penen iyirilip atırǵan jipti úzip aladı. Bir jumıs kúninde tómendegi waqıyalardıń júzege asıwı itimallıqların tabıń: a) jańa qurılma 5 márte jipti úzip alıw; b) eski qurılma 20 ten 25 ǵa shekemgi jipti úzip alıw. 
 \\
\textbf{A3.} Hárbiriniń júzege asıw itimallıǵı $p$ ǵa teń bolǵan 10 dana Bernulli tájiriybesi ótkerilgende, tómendegi waqıyalardıń itimallıqların tabıń: Sátliler sanı 5 den artıq, biraq 8 den kem.
 \\
\textbf{B1.} $\xi$ tosınnanlı shamanıń \emph{f}(\emph{x}) tıǵızlıq funkciyasi berilgen bolsin. Tómendegilerdi esaplań: a) C; b) \emph{F}(\emph{x}); c) M$\xi$; d) D$\xi$; e) \emph{f}(\emph{x}) hám \emph{F}(\emph{x}) grafiklarin sızıń.\(f(x) = \left\{ \begin{matrix}
C\sqrt[3]{1 - x},\ \ \ \ x \in \lbrack 0,1\rbrack, \\
\ \ \ \ \ \ \ \ 0,\ \ \ \ \ \ \ \ \ \ x \notin \lbrack 0,1\rbrack.\ \ 
\end{matrix} \right.\ \)
 \\
\textbf{B2.} Eger \includegraphics[width=0.36181in,height=0.29444in]{mediaBpng/image1.png} ǵárezsiz tosınnanlıq shamalar izbe-izliginiń bólistiriliw nızamları
\includegraphics[width=1.76042in,height=0.50278in]{mediaBpng/image23.png} \includegraphics[width=1.63819in,height=0.50278in]{mediaBpng/image24.png} \includegraphics[width=0.75486in,height=0.23958in]{mediaBpng/image9.png}
bolsa, onda bul izbe-izlik úlken sanlar nızamına boysınama?
 \\
\textbf{B3.} Eger \(\xi\) tosınnanlı shama \(\lbrack a,b\rbrack\) parametrli teń ólshewli bólistiriwine iye bolsa, onda onıń xarakteristikalıq funkciyası tabılsın.
 \\
\textbf{C1.} Eger ǵárezsiz \includegraphics[width=0.15208in,height=0.24028in]{mediaCpng/image4.png} hám \includegraphics[width=0.15208in,height=0.19167in]{mediaCpng/image5.png} úzliksiz tosınnanlıq shamalar sáykes túrde, \includegraphics[width=0.4in,height=0.25625in]{mediaCpng/image9.png} hám \includegraphics[width=0.44028in,height=0.25625in]{mediaCpng/image10.png} parametrli normal nızam boyınsha bólistirilgen bolsa, onda \includegraphics[width=0.44028in,height=0.24028in]{mediaCpng/image11.png} tosınnanlıq shamanıń tıǵızlıq funkciyasın tabıń.
 \\
\textbf{C2.} Tómende \includegraphics[width=0.46389in,height=0.25625in]{mediaCpng/image42.png} úzliksiz tosınnanlıq vektorlardıń tıǵızlıq funkciyaları berilgen. Olardıń \includegraphics[width=0.48819in,height=0.29583in]{mediaCpng/image43.png} hám \includegraphics[width=0.50417in,height=0.29583in]{mediaCpng/image44.png} marginal tıǵızlıq funkciyaların tabıń; \includegraphics[width=0.15972in,height=0.24028in]{mediaCpng/image45.png} hám \includegraphics[width=0.15972in,height=0.2in]{mediaCpng/image46.png} tosınnanlıq shamalardı ǵárezsizlikke tekseriń: \includegraphics[width=2.83194in,height=0.63194in]{mediaCpng/image58.png}
 \\
\textbf{C3.} Meyli, \(\xi_{1},...,\xi_{n}\) tosınnanlıq shamalar ǵárezsiz hám \(\lbrack a,b\rbrack\) aralıqta teń ólshemli bólistirilgen bolıp, \(\eta_{1} = \max\left( \xi_{1},...,\xi_{n} \right)\) hám \(\eta_{2} = \min\left( \xi_{1},...,\xi_{n} \right)\) bolsın. Onda \(\left( \eta_{1},\eta_{2} \right)\) tosınnanlıq vektordıń kovariaciyasın tabıń.
 \\

\end{tabular}
\vspace{1cm}


\begin{tabular}{m{17cm}}
\textbf{17-variant}
\newline

\textbf{T1.} Waqıyalar algebrası ($\sigma$-algebra, minimal $\sigma$-algebra).
 \\
\textbf{T2.} Kompoziciyalıq formulalar \\
\textbf{A1.} Kitaptıń bir betinde keminde bir baspa qáteligi bolıw itimallıǵı 0,02 ge hám tártip qáteligi bolıw itimallıǵı bolsa 0,4 ke teń. Jámi 400 betli kitapta tómendegi waqıyalardıń júzege asıw itimallıqların tabıń: a) keminde bes bette baspa qáteligi bolıwı; b) 170 ten 180 ge shekem betlerde tártip qáteligi bolıwı.
 \\
\textbf{A2.} $\left[ 0,1 \right]$ kesindiden tosınnan eki noqat tańlanadı. Birinshi noqattıń koordinatası ekinshi noqattıń koordinatasınan kishi bolıw itimallıǵın tabıń.
 \\
\textbf{A3.} Shegaralıq bahalaw jumısın tapsırıwǵa kelgen 10 studentten ibarat toparda úshewi ayrıqsha, tórtewi jaqsı, ekewi qanaatlandırarlı hám birewi qanaatlandırarsız tayarlanǵan. Shegaralıq bahalaw jumısınıń variantlarında 20 dana soraw bar. Ayrıqsha tayarlanǵan student barlıq 20 sorawǵa, jaqsı tayarlanǵanı 16 sorawǵa, qanaatlandırarlı tayarlanǵanı 10 sorawǵa, qanaatlandırarsız tayarlanǵanı 5 sorawǵa juwap bere aladı. a) Bul studentlerden qálegen birewi berilgen úsh sorawǵa da durıs juwap beriw itimallıǵın tabıń. b) Sol durıs juwap bergen studenttiń qanaatlandırarsız tayarlanǵan student bolıw itimallıǵın tabıń.
 \\
\textbf{B1.} \includegraphics[width=0.15972in,height=0.23958in]{mediaBpng/image49.png} úzliksiz tosınnanlıq shamanıń tıǵızlıq funkciyaları berilgen. Olarǵa sáykes \includegraphics[width=0.15972in,height=0.19653in]{mediaBpng/image50.png} tosınnanlıq shamanıń \includegraphics[width=0.50278in,height=0.30069in]{mediaBpng/image51.png} tıǵızlıq funkciyasın tabıń. \includegraphics[width=2.44167in,height=0.84028in]{mediaBpng/image60.png} \includegraphics[width=0.88333in,height=0.23958in]{mediaBpng/image61.png}
 \\
\textbf{B2.} Úsh mеrgеn bir-birinеn ǵárеzsiz nıshanaǵa bir márteden oq attı. Birinshi mеrgеnniń nıshanaǵa tiygiziw itimallıǵı 0,6 ǵa, еkinshisiniki 0,8 gе, úshinshisiniki bolsa 0,3 ke tеń. Atıw tamam bolǵannan kеyin nıshanada eki oq izi tabılǵan bolsa, birinshi mеrgеn nıshanaǵa tiygiziwi waqıyası itimallıǵı tabılsın.
 \\
\textbf{B3.} Fakultette 1460 student bar. Keminde 10 studenttiń tuwılǵan kúni 5 sentyabrge tuwra kelip qalıwı waqıyası itimallıǵı tabılsın.
 \\
\textbf{C1.} Eger \(\left( \xi_{1},\xi_{2} \right)\) absolyut úziliksiz tosınnanlıq vektordıń tıǵızlıq funkciyası \(f(x,y) = \left\{ \begin{matrix}
Cxy,\ eger\ (x,y) \in D, \\
 \\
0,\ \ \ \ \ eger\ (x,y) \notin D,
\end{matrix} \right.\ \) bunda \(D = \left\{ (x,y):\ y > - x,\ y < 2,\ x < 0 \right\}\) bolsa, onda \(\xi_{1}\) komponentanıń shártsiz hám shártli tıǵızlıq funkciyaların tabıń. Sonıń menen birge, \(\xi_{1}\) hám \(\xi_{2}\) tosınnanlıq shamalardı ǵárezsizlikke tekseriń.
 \\
\textbf{C2.} Eger \(\left\{ \xi_{n} \right\}\) ǵárezsiz tosınnanlıq shamalar izbe-izligi \(\lbrack 0,1\rbrack\) aralıqta teń ólshemli bólistirilgen bolıp, \(g(x)\) funkciya sol aralıqta úziliksiz bolsa, onda\(\frac{g\left( \xi_{1} \right) + ... + g\left( \xi_{n} \right)}{n}\overset{P}{\rightarrow}\int_{0}^{1}{g(x)}dx\) ekenligin kórsetiń.
 \\
\textbf{C3.} Eger ǵárezsiz \includegraphics[width=0.15208in,height=0.24028in]{mediaCpng/image4.png} hám \includegraphics[width=0.15208in,height=0.19167in]{mediaCpng/image5.png} úzliksiz tosınnanlıq shamalardıń hárbiri \includegraphics[width=0.19167in,height=0.16806in]{mediaCpng/image31.png} parametrli kórsetkishli nızam boyınsha bólistirilgen bolsa, onda \includegraphics[width=0.44028in,height=0.24028in]{mediaCpng/image14.png} tosınnanlıq shamanıń tıǵızlıq funkciyasın tabıń.
 \\

\end{tabular}
\vspace{1cm}


\begin{tabular}{m{17cm}}
\textbf{18-variant}
\newline

\textbf{T1.} Shártli itimallıq (anıqlaması, kóbеytiw tеorеması).
 \\
\textbf{T2.} Bólistiriw funkciyası (anıqlaması, tiykarǵı qásiyetleri).
 \\
\textbf{A1.} 20 komanda eki toparǵa bólinedi. Eki eń kúshli komanda bir toparǵa túspew itimallıǵın tabıń.
 \\
\textbf{A2.} Hárbiriniń júzege asıw itimallıǵı $p$ ǵa teń bolǵan 10 dana Bernulli tájiriybesi ótkerilgende, tómendegi waqıyalardıń itimallıqların tabıń: Sátliler sanı, sátsizler sanınan artıq.
 \\
\textbf{A3.} Oyın kubigi taslanǵanda jup ochkonıń túsiw itimallıǵın tabıń.
 \\
\textbf{B1.} \includegraphics[width=0.15972in,height=0.23958in]{mediaBpng/image49.png} úzliksiz tosınnanlıq shamanıń tıǵızlıq funkciyaları berilgen. Olarǵa sáykes \includegraphics[width=0.15972in,height=0.19653in]{mediaBpng/image50.png} tosınnanlıq shamanıń \includegraphics[width=0.50278in,height=0.30069in]{mediaBpng/image51.png} tıǵızlıq funkciyasın tabıń. \includegraphics[width=2.53958in,height=1.15347in]{mediaBpng/image88.png} \includegraphics[width=0.63194in,height=0.25764in]{mediaBpng/image89.png}
 \\
\textbf{B2.} Parametrleri (0;$\sigma$ ) bolǵan normal nizamniń dispersiyasın tabılsın.
 \\
\textbf{B3.} $\xi$ tosınnanlı shamanıń \emph{f}(\emph{x}) tıǵızlıq funkciyasi berilgen bolsin. Tómendegilerdi esaplań: a) C; b) \emph{F}(\emph{x}); c) M$\xi$; d) D$\xi$; e) \emph{f}(\emph{x}) hám \emph{F}(\emph{x}) grafiklarin sızıń.\(f(x) = \left\{ \begin{matrix}
C/(1 + x^{2}),\ \ \ \ x \in \lbrack 0,\sqrt{3}\rbrack, \\
\ \ \ \ \ \ \ \ 0,\ \ \ \ \ \ \ \ \ \ \ x \notin \lbrack 0,\sqrt{3}\rbrack.\ \ 
\end{matrix} \right.\ \)
 \\
\textbf{C1.} Eger \(\left\{ \xi_{n} \right\}\) ǵárezsiz birdey bólistirilgen tosınnanlıq shamalar izbe-izligi bolıp, onıń bólistiriw funkciyası \(F_{\xi_{1}}(x) = \left\{ \begin{matrix}
\ 1 - e^{\lambda - x},\ \ eger\ \ x \geq \lambda, \\
 \\
\ \ \ \ \ \ 0,\ \ \ \ \ \ \ \ \ \ \ eger\ \ x < \lambda
\end{matrix} \right.\ \) bolsa, onda \(\left\{ \eta_{n} \right\} = \left\{ min(\xi_{1},...,\xi_{n}) \right\}\) izbe-izliktiń \(\mathbf{\lambda}\) ǵa bir itimallıq penen jıynaqlılıǵın kórsetiń.
 \\
\textbf{C2.} Eger \(\xi\) tosınnanlıq shama standart Koshi bólistiriliwine iye bolsa, onda \(M\min\left( |\xi|,1 \right)\) mánisin tabıń.
 \\
\textbf{C3.} Tómende \includegraphics[width=0.46389in,height=0.25625in]{mediaCpng/image42.png} úzliksiz tosınnanlıq vektorlardıń tıǵızlıq funkciyaları berilgen. Olardıń \includegraphics[width=0.48819in,height=0.29583in]{mediaCpng/image43.png} hám \includegraphics[width=0.50417in,height=0.29583in]{mediaCpng/image44.png} marginal tıǵızlıq funkciyaların tabıń; \includegraphics[width=0.15972in,height=0.24028in]{mediaCpng/image45.png} hám \includegraphics[width=0.15972in,height=0.2in]{mediaCpng/image46.png} tosınnanlıq shamalardı ǵárezsizlikke tekseriń: \includegraphics[width=3.20833in,height=0.59167in]{mediaCpng/image50.png}
 \\

\end{tabular}
\vspace{1cm}


\begin{tabular}{m{17cm}}
\textbf{19-variant}
\newline

\textbf{T1.} Ǵárezsiz tájiriybelerdiń Bernulli sxeması (binоmiаl bólistiriliw, qásiyetleri).
 \\
\textbf{T2.} Tıǵızlıq funkciyası (anıqlaması, tiykarǵıqásiyetleri).
 \\
\textbf{A1.} $\left[ 0,1 \right]$ kesindiden tosınnan eki noqat tańlanadı. Olardıń koordinataları kvadratları qosındısı, koordinataları kóbeymesi úsh esesinen kóp bolıw itimallıǵın tabıń.
 \\
\textbf{A2.} Eki oyın kubigi taslanǵanda túsken ochkolardıń qosındısı 2 den úlken, bıraq 5 ten kishi bolıw itimallıǵın tabıń.
 \\
\textbf{A3.} Hárbiriniń júzege asıw itimallıǵı $p$ ǵa teń bolǵan 10 dana Bernulli tájiriybesi ótkerilgende, tómendegi waqıyalardıń itimallıqların tabıń: Sátliler sanı 3 dana, sonıń menen birge, olardıń barlıǵı sońǵı úsh tájiriybede ámelge asıwı.
 \\
\textbf{B1.} Eger \includegraphics[width=0.36181in,height=0.29444in]{mediaBpng/image1.png} ǵárezsiz tosınnanlıq shamalar izbe-izligi \includegraphics[width=0.55208in,height=0.29444in]{mediaBpng/image46.png} aralıqta teń ólshemli bólistirilgen bolsa, onda bul izbe-izlik úlken sanlar nızamına boysınama?
 \\
\textbf{B2.} Eger \includegraphics[width=0.36181in,height=0.29444in]{mediaBpng/image1.png} ǵárezsiz tosınnanlıq shamalar izbe-izliginiń bólistiriliw nızamları
\includegraphics[width=2.38056in,height=0.47847in]{mediaBpng/image36.png} \includegraphics[width=0.75486in,height=0.23958in]{mediaBpng/image9.png}
bolsa, onda bul izbe-izlik úlken sanlar nızamına boysınama?
 \\
\textbf{B3.} Qálegen \(a,b \in \lbrack 0;2\rbrack\) sanları ushın \(D = \left| \begin{matrix}
1 & a \\
a & b
\end{matrix} \right|\) determinanti esaplanadı. \(D > 0\) bolıwı itimallıǵı qanday?
 \\
\textbf{C1.} Eger \(\left( \xi_{1},\xi_{2} \right)\) tosınnanlıq vektordıń bólistiriw funkciyası \(F(x,y) = \left\{ \begin{matrix}
\left( 1 - 2^{- x^{2}} \right)\left( 1 - 2^{- 2y^{2}} \right),\ \ eger\ \ x \geq 0,\ y \geq 0, \\
 \\
 \\
\ \ \ \ \ \ \ \ \ \ \ \ \ \ 0,\ \ \ \ \ \ \ \ \ \ \ \ \ \ \ \ \ \ \ \ \ \ \ basqa\ hallarda
\end{matrix} \right.\ \) bolsa, onda \(F\left( x/\xi_{2} < y \right)\) hám \(F\left( y/\xi_{1} < x \right)\) shártli bólistiriw funkciyaların tabıń. Sonıń menen birge, \(\xi_{1}\) hám \(\xi_{2}\) tosınnanlıq shamalardı ǵárezsizlike tekseriń.
 \\
\textbf{C2.} Ortasha mánis vektorı \(\left( m_{1},m_{2} \right)\) hám kovariaciyalıq matricası\(K = \begin{pmatrix}
\sigma_{1}^{2} & r\sigma_{1}\sigma_{2} \\
r\sigma_{1}\sigma_{2} & \sigma_{2}^{2}
\end{pmatrix},\ \ \sigma_{1},\ \sigma_{2} > 0,\ \ |r|\  < 1\) bolǵan normal bólistirilgen \(\left( \xi_{1},\xi_{2} \right)\) tosınnanlıq vektordıń tıǵızlıq funkciyasın tabıń.
 \\
\textbf{C3.} Tómende \includegraphics[width=0.46389in,height=0.25625in]{mediaCpng/image42.png} úzliksiz tosınnanlıq vektorlardıń tıǵızlıq funkciyaları berilgen. Olardıń \includegraphics[width=0.48819in,height=0.29583in]{mediaCpng/image43.png} hám \includegraphics[width=0.50417in,height=0.29583in]{mediaCpng/image44.png} marginal tıǵızlıq funkciyaların tabıń; \includegraphics[width=0.15972in,height=0.24028in]{mediaCpng/image45.png} hám \includegraphics[width=0.15972in,height=0.2in]{mediaCpng/image46.png} tosınnanlıq shamalardı ǵárezsizlikke tekseriń: \includegraphics[width=3.20833in,height=0.84792in]{mediaCpng/image62.png}
 \\

\end{tabular}
\vspace{1cm}


\begin{tabular}{m{17cm}}
\textbf{20-variant}
\newline

\textbf{T1.} Tolıq itimallıq formulası (waqıyalardıń tolıq gruppası, dálilleniwi).
 \\
\textbf{T2.} Tiykarǵı аbsоlyut úzliksiz bólistiriliwler (nоrmаl bólistiriw, teń ólshewli bólistiriw, kórsetkishli bólistiriw). 
 \\
\textbf{A1.} Toparda 25 student bolıp, olardan 6 student ayrıqsha bahaǵa oqıydı. Dizim boyınsha 10 student ajıratılǵan. Ajıratılǵanlar ishinde ayrıqsha bahaǵa oqıytuǵın 3 student bolıw itimallıǵın tabıń.
 \\
\textbf{A2.} Samolyotqa samolyottan 4 dana ǵárezsiz oq atıldı. Hárbir atılǵan oqtıń tiyiw itimallıǵı 0,3 ke teń. Samolyottı joq etiw ushın (tolıǵı menen isten shıǵarıw ushın) 2 márte tiyiw jetkilikli, 1 márte tiygende 0,6 ǵa teń itimallıq penen isten shıǵadı. a) Tórt márte oq atıw nátiyjesinde samolyottıń tolıǵı menen isten shıǵıw itimallıǵın tabıń. b) Eger samolyot tolıǵı menen isten shıqqan bolsa, ol bir oq tiyip isten shıqqan bolıw itimallıǵın tabıń.
 \\
\textbf{A3.} Radio ustaxanada kúnine 3 radio ońlanadı. Radionıń mexanikalıq bólegi buzılıw itimallıǵı 0,1 ge hám elektron bólegi buzılıw itimallıǵı 0,004 ke teń. Jıl dawamında ońlanǵan radiolar arasında tómendegi waqıyalardıń júzege asıw itimallıqların tabıń: a) 130 dan 140 ǵa shekem radiolardıń mexanikalıq bóleginde nasazlıqlar bolǵan; b) tórtten artıq bolmaǵan radiolardıń elektron bóleginde nasazlıqlar bolǵan.
 \\
\textbf{B1.} \includegraphics[width=0.15972in,height=0.23958in]{mediaBpng/image49.png} úzliksiz tosınnanlıq shamanıń tıǵızlıq funkciyaları berilgen. Olarǵa sáykes \includegraphics[width=0.15972in,height=0.19653in]{mediaBpng/image50.png} tosınnanlıq shamanıń \includegraphics[width=0.50278in,height=0.30069in]{mediaBpng/image51.png} tıǵızlıq funkciyasın tabıń. \includegraphics[width=2.06111in,height=0.58264in]{mediaBpng/image77.png} \includegraphics[width=0.91389in,height=0.49097in]{mediaBpng/image78.png}
 \\
\textbf{B2.} $\xi$ tosınnanlı shamanıń \emph{f}(\emph{x}) tıǵızlıq funkciyasi berilgen bolsin. Tómendegilerdi esaplań: a) C; b) \emph{F}(\emph{x}); c) M$\xi$; d) D$\xi$; e) \emph{f}(\emph{x}) hám \emph{F}(\emph{x}) grafiklarin sızıń.\(f(x) = \left\{ \begin{matrix}
2x/3,\ \ \ \ x \in \lbrack 0,1\rbrack, \\
C(3 - x),\ \ \ x \in (1,3\rbrack, \\
0,\ \ \ keri\ jag'dayda.\ \ 
\end{matrix} \right.\ \)
 \\
\textbf{B3.} İdısta 10 shar bolıp, olardan 3 ewi aq sharlar. İdıstan táwekelge 3 shar alınadı. Tosınnanlı \(\xi\) shaması -- alınǵan aq sharlar sanı. Onıń bólistiriliw nızamın jazıń.
 \\
\textbf{C1.} Eger ǵárezsiz \includegraphics[width=0.15208in,height=0.24028in]{mediaCpng/image4.png} hám \includegraphics[width=0.15208in,height=0.19167in]{mediaCpng/image5.png} úzliksiz tosınnanlıq shamalar sáykes túrde, \includegraphics[width=0.38403in,height=0.24028in]{mediaCpng/image24.png} hám \includegraphics[width=0.41597in,height=0.24028in]{mediaCpng/image25.png} aralıqlarda teń ólshemli bólistirilgen bolsa, onda \includegraphics[width=0.44028in,height=0.24028in]{mediaCpng/image14.png} tosınnanlıq shamanıń tıǵızlıq funkciyasın tabıń.
 \\
\textbf{C2.} Eger \(\left\{ \xi_{n} \right\}\) tosınnanlıq shamalar izbe-izligi \(\mathbf{\xi}_{\mathbf{n}}\overset{\mathbf{P}}{\rightarrow}\mathbf{\xi}\) hám \(\mathbf{\xi}_{\mathbf{n}}\overset{\mathbf{P}}{\rightarrow}\mathbf{\eta}\) bolsa, onda \(\mathbf{P}\left( \mathbf{\xi = \eta} \right)\mathbf{=}\mathbf{1}\) qatnasın dálilleń.
 \\
\textbf{C3.} Eger \(\left( \xi_{1},\xi_{2} \right)\) tosınnanlıq vеktоrdıń tıǵızlıq funkciyası \(f(x,y) = \frac{1}{3\pi}e^{- \ \ \frac{x^{2} + 4y^{2}}{6}}\)bolsa, onda \(\left( \xi_{1},\xi_{2} \right)\) tosınnanlıq noqattıń \(D = \left\{ (x,y):|x| \leq 1,|y| \leq 2 \right\}\) oblastqa túsiw itimallıǵın tabıń.
 \\

\end{tabular}
\vspace{1cm}


\begin{tabular}{m{17cm}}
\textbf{21-variant}
\newline

\textbf{T1.} Bayеs formulası (gipotezalar teoreması, dálilleniwi).
 \\
\textbf{T2.} 
Úlken sanlar nızamı (anıqlaması, Chebishev teoreması).
 \\
\textbf{A1.} Hárbiriniń júzege asıw itimallıǵı $p$ ǵa teń bolǵan 10 dana Bernulli tájiriybesi ótkerilgende, tómendegi waqıyalardıń itimallıqların tabıń: Tájiriybelerdiń birinshi yarımındaǵı sátliler sanı, tájiriybelerdiń ekinshi yarımındaǵı sátliler sanınan kem.
 \\
\textbf{A2.} Uzaq aralıqtaǵı nıshanǵa pulemyot hám pistolet penen oq atılmaqta. Pulemyot oǵınıń nıshanǵa tiyiw itimallıǵı 0,02, al pistolet penen bolsa 0,6 ǵa teń. Eger hárbir qural menen nıshanǵa 100 márte oq atılǵan bolsa, tómendegi waqıyalardıń júzege asıwı itimallıqların tabıń: a) pulemyot penen nıshanǵa úsh mártege shekem tiygiziwi; b) pistolet penen nıshanǵa 60 márte tiygiziwi.
 \\
\textbf{A3.} Qutıda 28 dana birdey sharlar bolıp, olardıń 19 danası qızıl hám 9 danası kók reńdegi sharlar. Tosınnan alınǵan 3 dana shardıń 2 danası kók shar bolıw itimallıǵın tabıń.
 \\
\textbf{B1.} Hár bir sinawda A waqiyasiniń júzege asiw itimalliǵi 0,6 ǵa teń. Ǵárezsiz 5400 sinawdiń 3240 mártesinde A waqiyasiniń júzege asiw itimalliǵin tabiń.
 \\
\textbf{B2.} \includegraphics[width=0.15972in,height=0.23958in]{mediaBpng/image49.png} úzliksiz tosınnanlıq shamanıń tıǵızlıq funkciyaları berilgen. Olarǵa sáykes \includegraphics[width=0.15972in,height=0.19653in]{mediaBpng/image50.png} tosınnanlıq shamanıń \includegraphics[width=0.50278in,height=0.30069in]{mediaBpng/image51.png} tıǵızlıq funkciyasın tabıń. \includegraphics[width=1.95069in,height=0.58264in]{mediaBpng/image70.png} \includegraphics[width=0.57639in,height=0.28194in]{mediaBpng/image79.png}
 \\
\textbf{B3.} $\xi$ tosınnanlı shamanıń \emph{f}(\emph{x}) tıǵızlıq funkciyasi berilgen bolsin. Tómendegilerdi esaplań: a) C; b) \emph{F}(\emph{x}); c) M$\xi$; d) D$\xi$; e) \emph{f}(\emph{x}) hám \emph{F}(\emph{x}) grafiklarin sızıń.\(f(x) = \left\{ \begin{matrix}
C\left( |x| + \frac{1}{4} \right),\ \ \ \ x \in \lbrack - 1,1\rbrack, \\
\ \ \ \ \ \ \ \ 0,\ \ \ \ \ \ \ \ \ \ \ \ \ \ \ x \notin \lbrack - 1,1\rbrack.\ \ 
\end{matrix} \right.\ \)
 \\
\textbf{C1.} Hárqanday \(\varphi_{\xi}(t)\) xarakteristikalıq funkciya ushın \(t \in R\) de \(1 - Re\varphi_{\xi}(2t) \leq 4\left( 1 - Re\varphi_{\xi}(t) \right)\) ekenligin dálilleń.
 \\
\textbf{C2.} Tómende \includegraphics[width=0.46389in,height=0.25625in]{mediaCpng/image42.png} úzliksiz tosınnanlıq vektorlardıń tıǵızlıq funkciyaları berilgen. Olardıń \includegraphics[width=0.48819in,height=0.29583in]{mediaCpng/image43.png} hám \includegraphics[width=0.50417in,height=0.29583in]{mediaCpng/image44.png} marginal tıǵızlıq funkciyaların tabıń; \includegraphics[width=0.15972in,height=0.24028in]{mediaCpng/image45.png} hám \includegraphics[width=0.15972in,height=0.2in]{mediaCpng/image46.png} tosınnanlıq shamalardı ǵárezsizlikke tekseriń: \includegraphics[width=2.56806in,height=0.59167in]{mediaCpng/image53.png}
 \\
\textbf{C3.} Eger \(\chi_{1}^{2}\) hám \(\chi_{2}^{2}\) ǵárezsiz tosınnanlıq shamalar bolıp, \(\chi_{1}^{2}\sim\chi^{2}(n_{1})\) hám \(\chi_{2}^{2}\sim\chi^{2}(n_{2})\) bolsa, onda \(\frac{n_{2}}{n_{1}} \cdot \frac{\chi_{1}^{2}}{\chi_{2}^{2}}\) tosınnanlıq shamanıń tıǵızlıq funkciyasın tabıń.
 \\

\end{tabular}
\vspace{1cm}


\begin{tabular}{m{17cm}}
\textbf{22-variant}
\newline

\textbf{T1.} Itimallıqlar teoriyası aksiomaları (ólshewli keńislik, itimallıq keńisligi).
 \\
\textbf{T2.} Tiykarǵı diskret bólistiriliwler (Binomial, Puasson hám geometriyalıq bólistiriliwler).
 \\
\textbf{A1.} Albomda 8 dana jańa hám 6 dana múddeti ótken markalar bar. Albomnan tosınnan 3 marka alınıp, múddeti jańalandı hám ornına qaytarıp qoyıldı. Bunnan soń, tosınnan 2 marka alındı. a) Bul 2 marka jańa bolıw itimallıǵın tabıń. b) Sol 2 marka jańa ekenligi belgili bolsa, dáslepki alınǵan 3 markanıń múddeti ótken bolıw itimallıǵın tabıń.
 \\
\textbf{A2.} Eki oyın kubigi taslanǵanda túsken ochkolardıń qosındısı 3 ten úlken, bıraq 8 den kishi bolıw itimallıǵın tabıń.
 \\
\textbf{A3.} $\left[ 0,1 \right]$ kesindiden tosınnan eki noqat tańlanadı. Ekinshi noqattıń koordinatası birinshi noqattıń koordinatası eki esesinen úlken bolıw itimallıǵın tabıń.
 \\
\textbf{B1.} Eger \includegraphics[width=0.36181in,height=0.29444in]{mediaBpng/image1.png} ǵárezsiz hám birdey bólistirilgen tosınnanlıq shamalar izbe-izligi \includegraphics[width=0.44792in,height=0.21458in]{mediaBpng/image43.png} parametrli kórsetkishli bólistiriliwine iye bolsa, onda bul izbe-izlik úlken sanlar nızamına boysınama?
 \\
\textbf{B2.} Chebıshev teńsizliginiń járdemi menen normal tosınnanlı shamanıń óziniń matematikalıq kútiliwinen, awısıwınıń úsh orta kvadratlıq awısıwdan úlken bolıwınıń itimallıǵın bahalań.
 \\
\textbf{B3.} Tosinnanli $\xi$ shamasiniń bólistiriw tiǵizliǵi \(\mathbf{f}\mathbf{(}\mathbf{x}\mathbf{)}\mathbf{=}\frac{\mathbf{1}}{\mathbf{2}}\mathbf{e}^{\mathbf{-}\left| \mathbf{x} \right|}\) bolsa, onıń matematikalıq kútiliwin tabıń.
 \\
\textbf{C1.} Eger \(\left\{ \xi_{n} \right\}\) ǵárezsiz tosınnanlıq shamalar izbe-izliginiń bólistiriw funkciyaları \(F_{n}(x) = \left\{ \begin{matrix}
\ 1 - \frac{1}{x + n},\ \ eger\ \ x > 0 \\
 \\
 \\
\ \ \ \ \ \ \ \ \ \ 0,\ \ \ \ \ \ \ \ \ \ \ eger\ \ x \leq 0
\end{matrix} \right.\ \) bolsa, onda bul izbe-izliktiń 0 ge itimallıq boyınsha jıynaqlılıǵın kórsetiń.
 \\
\textbf{C2.} Eger ǵárezsiz \includegraphics[width=0.15208in,height=0.24028in]{mediaCpng/image4.png} hám \includegraphics[width=0.15208in,height=0.19167in]{mediaCpng/image5.png} úzliksiz tosınnanlıq shamalardıń hárbiri \includegraphics[width=0.4in,height=0.24028in]{mediaCpng/image18.png} aralıqta teń ólshemli bólistirilgen bolsa, onda \includegraphics[width=0.50417in,height=0.29583in]{mediaCpng/image21.png} tosınnanlıq shamanıń tıǵızlıq funkciyasın tabıń.
 \\
\textbf{C3.} Eger \(\left\{ \xi_{n} \right\}\) diskret tosınnanlıq shamalar izbe-izliginiń bólistiriliw nızamları\(P(\xi_{n} = e^{n}) = \frac{1}{n^{2}},\) \(P(\xi_{n} = 0) = 1 - \frac{1}{n^{2}}\) bolsa, onda \(\left\{ \xi_{n} \right\}\) tosınnanlıq shamalar izbe-izliginiń 0 ge bir itimallıq penen jıynaqlılıǵın kórsetiń.
 \\

\end{tabular}
\vspace{1cm}


\begin{tabular}{m{17cm}}
\textbf{23-variant}
\newline

\textbf{T1.} Ǵárezsiz tájiriybelerdiń Bernulli sxeması (binоmiаl bólistiriliw, qásiyetleri).
 \\
\textbf{T2.} Kompoziciyalıq formulalar \\
\textbf{A1.} 28 dana dominonıń tolıq komplektinen 7 danası tosınnan tańlanadı. Hárbir domino tasındaǵı ulıwma ochkolar qosındısı 7 den kem bolıw itimallıǵın tabıń.
 \\
\textbf{A2.} Eki oyın kubigi taslanǵanda túsken ochkolardıń qosındısı 4 ten úlken, bıraq 7 den kishi bolıw itimallıǵın tabıń.
 \\
\textbf{A3.} Úsh qutınıń hárbirinde $n$ dana aq ($n\ge 2$) hám $m$ dana qara sharlar bar. Birinshi qutıdan ekinshi qutıǵa tosınnan eki shar, ekinshi qutıdan úshinshi qutıǵa tosınnan bir shar salındı. Keyin, úshinshi qutıdan tosınnan bir shar alındı. а) Bul shardıń aq bolıw itimallıǵın tabıń. b) Sol shar aq bolsa, birinshi qutıdan alınǵan sharlardıń aq bolıw itimallıǵın tabıń.
 \\
\textbf{B1.} $\xi$ tosınnanlı shamanıń \emph{f}(\emph{x}) tıǵızlıq funkciyasi berilgen bolsin. Tómendegilerdi esaplań: a) C; b) \emph{F}(\emph{x}); c) M$\xi$; d) D$\xi$; e) \emph{f}(\emph{x}) hám \emph{F}(\emph{x}) grafiklarin sızıń.\(f(x) = \left\{ \begin{matrix}
\ \ \ \ \ \ \ \ 0,\ \ \ \ \ \ x \leq 0, \\
Cxe^{- x},\ \ \ \ \ x > 0.\ \ 
\end{matrix} \right.\ \)
 \\
\textbf{B2.} Atiwshiniń bir márte nishanaǵa atqanda tiygiziw itimalliǵi 0,75 ke teń. Nishanaǵa 10 márte atqanda 8 mártesinde tiygiziw itimalliǵin tabiń.
 \\
\textbf{B3.} Eger \includegraphics[width=0.36181in,height=0.29444in]{mediaBpng/image1.png} ǵárezsiz tosınnanlıq shamalar izbe-izliginiń bólistiriliw nızamları
\includegraphics[width=1.53958in,height=0.50278in]{mediaBpng/image21.png} \includegraphics[width=1.63819in,height=0.50278in]{mediaBpng/image22.png} \includegraphics[width=0.75486in,height=0.23958in]{mediaBpng/image9.png}
bolsa, onda bul izbe-izlik úlken sanlar nızamına boysınama?
 \\
\textbf{C1.} Eger \(\xi\sim E(\lambda)\) bolsa, onda \(\xi\) tosınnanlıq shamanıń joqarı tártipli baslanǵısh momentlerin tabıń.
 \\
\textbf{C2.} Eger \(\xi_{1},...,\xi_{n}\) ǵárezsiz birdey bólistirilgen tosınnanlıq shamalar \(F(x)\) bólistiriw hám \(f(x)\) tıǵızlıq funkciyalarǵa iye bolsa, onda \(\eta_{1} = \max\left( \xi_{1},...,\xi_{n} \right)\) hám \(\eta_{2} = \min\left( \xi_{1},...,\xi_{n} \right)\) tosınnanlıq shamalardıń bólistiriw hám tıǵızlıq funkciyaların tabıń.
 \\
\textbf{C3.} Eger ǵárezsiz \includegraphics[width=0.15208in,height=0.24028in]{mediaCpng/image4.png} hám úzliksiz \includegraphics[width=0.15208in,height=0.19167in]{mediaCpng/image5.png} tosınnanlıq shamalardıń hárbiri \includegraphics[width=0.4in,height=0.24028in]{mediaCpng/image18.png} aralıqta teń ólshemli bólistirilgen bolsa, onda \includegraphics[width=0.38403in,height=0.24028in]{mediaCpng/image22.png} tosınnanlıq shamanıń tıǵızlıq funkciyasın tabıń.
 \\

\end{tabular}
\vspace{1cm}


\begin{tabular}{m{17cm}}
\textbf{24-variant}
\newline

\textbf{T1.} Bayеs formulası (gipotezalar teoreması, dálilleniwi).
 \\
\textbf{T2.} Tosınnanlı shamanıń matematikalıq kútiliwi. (anıqlaması, qásiyetleri)
 \\
\textbf{A1.} Hárbiriniń júzege asıw itimallıǵı $p$ ǵa teń bolǵan 10 dana Bernulli tájiriybesi ótkerilgende, tómendegi waqıyalardıń itimallıqların tabıń: Sátsizler sanı 2 den artıq, biraq 5 den kem.
 \\
\textbf{A2.} $x\in \left[ -\pi ,\pi  \right]$ ushın $sinx<cosx$ bolıw itimallıǵın tabıń.
 \\
\textbf{A3.} Dúkan 1000 dana televizor hám 1000 dana radio satıp aldı. Hárbir televizordıń defektli bolıwı itimallıǵı 0,005 ke hám hárbir radionıń defektli bolıwı itimallıǵı 0,04 ke teń. Usı sawdada tómendegi waqıyalardıń júzege asıwı itimallıqların tabıń: a) keminde tórt televizor defektli bolıwı; b) 35 ten 45 ke shekem radio defektli bolıwı.
 \\
\textbf{B1.} \includegraphics[width=0.15972in,height=0.23958in]{mediaBpng/image49.png} úzliksiz tosınnanlıq shamanıń tıǵızlıq funkciyaları berilgen. Olarǵa sáykes \includegraphics[width=0.15972in,height=0.19653in]{mediaBpng/image50.png} tosınnanlıq shamanıń \includegraphics[width=0.50278in,height=0.30069in]{mediaBpng/image51.png} tıǵızlıq funkciyasın tabıń. \includegraphics[width=2.15972in,height=0.58889in]{mediaBpng/image90.png} \includegraphics[width=0.71806in,height=0.49097in]{mediaBpng/image91.png}
 \\
\textbf{B2.} Skladqa 30 yashik shisheli buyımlar túsirilgen. Táwekelge alınǵan yashikte buyımlardıń pútin bolıw itimallıǵı 0,9 ǵa teń. Barlıq buyımlar zálelge ushıramaǵan eń kóp itimallı yashikler sanın tabıń.
 \\
\textbf{B3.} $\xi$ tosınnanlı shamanıń \emph{f}(\emph{x}) tıǵızlıq funkciyasi berilgen bolsin. Tómendegilerdi esaplań: a) C; b) \emph{F}(\emph{x}); c) M$\xi$; d) D$\xi$; e) \emph{f}(\emph{x}) hám \emph{F}(\emph{x}) grafiklarin sızıń.\(f(x) = \left\{ \begin{matrix}
\ \ \ \ \ \ \ \ 0,\ \ \ \ \ \ \ \ \ \ \ \ x \leq 0, \\
C/(x + 1)^{4},\ \ \ \ \ x > 0.\ \ 
\end{matrix} \right.\ \)
 \\
\textbf{C1.} Tómende \includegraphics[width=0.46389in,height=0.25625in]{mediaCpng/image42.png} úzliksiz tosınnanlıq vektorlardıń tıǵızlıq funkciyaları berilgen. Olardıń \includegraphics[width=0.48819in,height=0.29583in]{mediaCpng/image43.png} hám \includegraphics[width=0.50417in,height=0.29583in]{mediaCpng/image44.png} marginal tıǵızlıq funkciyaların tabıń; \includegraphics[width=0.15972in,height=0.24028in]{mediaCpng/image45.png} hám \includegraphics[width=0.15972in,height=0.2in]{mediaCpng/image46.png} tosınnanlıq shamalardı ǵárezsizlikke tekseriń: \includegraphics[width=2.64028in,height=0.33611in]{mediaCpng/image51.png}
 \\
\textbf{C2.} Tómende \includegraphics[width=0.46389in,height=0.25625in]{mediaCpng/image42.png} úzliksiz tosınnanlıq vektorlardıń tıǵızlıq funkciyaları berilgen. Olardıń \includegraphics[width=0.48819in,height=0.29583in]{mediaCpng/image43.png} hám \includegraphics[width=0.50417in,height=0.29583in]{mediaCpng/image44.png} marginal tıǵızlıq funkciyaların tabıń; \includegraphics[width=0.15972in,height=0.24028in]{mediaCpng/image45.png} hám \includegraphics[width=0.15972in,height=0.2in]{mediaCpng/image46.png} tosınnanlıq shamalardı ǵárezsizlikke tekseriń: \includegraphics[width=4.04028in,height=0.67986in]{mediaCpng/image65.png}
 \\
\textbf{C3.} Eger \(\left( \xi_{1},\xi_{2} \right)\) absolyut úziliksiz tosınnanlıq vektordıń tıǵızlıq funkciyası \(f(x,y) = \left\{ \begin{matrix}
Ce^{- x - y},\ eger\ \ x \geq 0,y \geq 0, \\
 \\
 \\
\ \ \ \ \ \ \ \ 0,\ \ \ \ \ basqa\ hallarda\ 
\end{matrix} \right.\ \) bolsa, onda \(F(x,y),\) \(F_{\xi_{1}}(x),\) \(F_{\xi_{2}}(y),\) \(f_{\xi_{1}}(x),\) \(f_{\xi_{2}}(y)\) hám \(P\left( \xi_{1} > 0,\xi_{2} < 1 \right)\) itimallıqtı tabıń. Sonıń menen birge, \(\xi_{1}\) hám \(\xi_{2}\) tosınnanlıq shamalardı ǵárezsizlikke tekseriń.
 \\

\end{tabular}
\vspace{1cm}


\begin{tabular}{m{17cm}}
\textbf{25-variant}
\newline

\textbf{T1.} Bernulli sxemаsı ushın limit teоremаlаr (Puasson bólistiriliwi, qásiyetleri).
 \\
\textbf{T2.} Xarakteristikalıq funkciyalar (anıqlaması, tiykarǵı qásiyetleri).
 \\
\textbf{A1.} Birdey úsh qutı berilgen. Birinshi qutıda 2 dana aq hám 1 dana qara sharlar, ekinshi qutıda 3 dana aq hám 1 dana qara sharlar, úshinshi qutıda bolsa 2 dana aq hám 2 dana qara sharlar bar. Tosınnan qutılardan birewi tańlanıp, onıń ishinen bir dana shar alınadı. a) Usı alınǵan shardıń aq shar bolıw itimallıǵın tabıń. b) Alınǵan shar aq bolsa, onıń ekinshi qutıdan alınǵan bolıw itimallıǵın tabıń.
 \\
\textbf{A2.} Eki oyın kubigi taslanǵanda túsken eń kishi ochko 4 ten úlken bolıw itimallıǵın tabıń.
 \\
\textbf{A3.} Balalar baqshasında 300 bala tárbiyalanadı. Tómendegi waqıyalardıń júzege asıwı itimallıqların tabıń: a) anıq eki bala 1-martta tuwılǵan; b) jazda 47 den 52 ge shekem bala tuwılǵan.
 \\
\textbf{B1.} \includegraphics[width=0.15972in,height=0.23958in]{mediaBpng/image49.png} úzliksiz tosınnanlıq shamanıń tıǵızlıq funkciyaları berilgen. Olarǵa sáykes \includegraphics[width=0.15972in,height=0.19653in]{mediaBpng/image50.png} tosınnanlıq shamanıń \includegraphics[width=0.50278in,height=0.30069in]{mediaBpng/image51.png} tıǵızlıq funkciyasın tabıń. \includegraphics[width=2.57083in,height=0.84028in]{mediaBpng/image86.png} \includegraphics[width=0.57639in,height=0.28194in]{mediaBpng/image87.png}
 \\
\textbf{B2.} Puasson nizamına boysiniwshi tosinnanli $\xi$ shamasiniń dispersiyası tabılsın.
 \\
\textbf{B3.} Eger \includegraphics[width=0.36181in,height=0.29444in]{mediaBpng/image1.png} ǵárezsiz tosınnanlıq shamalar izbe-izligi \includegraphics[width=0.52153in,height=0.54583in]{mediaBpng/image45.png} aralıqta teń ólshemli bólistirilgen bolsa, onda bul izbe-izlik úlken sanlar nızamına boysınama?
 \\
\textbf{C1.} 
\(\xi\) diskret tosınnanlıq shama \(x_{i} = ( - 1)^{i}i\) mánislerdi \(p_{i} = \frac{1}{i(i + 1)},\) \(\ \ i = 1,\ 2,\ ...\) itimallıqlar menen qabıl etse, onıń matematikalıq kútiliwin tabıń.
 \\
\textbf{C2.} Meyli, \(\left\{ \xi_{n} \right\}\) tosınnanlıq shamalar izbe-izligi óziniń \(\left\{ F_{n}(x) \right\}\) bólistiriw funkciyaları menen berilgen bolsın. Sonda hám tek sonda ǵana, eger \(\lim_{n \rightarrow \infty}\int_{- \infty}^{+ \infty}{\frac{x^{2}}{1 + x^{2}}dF_{n}(x)} = 0\) bolsa, onda \(\mathbf{\xi}_{\mathbf{n}}\overset{\mathbf{P}}{\rightarrow}\mathbf{0}\) ekenligin dálilleń.
 \\
\textbf{C3.} Eger ǵárezsiz \includegraphics[width=0.15208in,height=0.24028in]{mediaCpng/image4.png} hám \includegraphics[width=0.15208in,height=0.19167in]{mediaCpng/image5.png} úzliksiz tosınnanlıq shamalardıń hárbiri \includegraphics[width=0.4in,height=0.24028in]{mediaCpng/image18.png} aralıqta teń ólshemli bólistirilgen bolsa, onda \includegraphics[width=0.44028in,height=0.24028in]{mediaCpng/image20.png} tosınnanlıq shamanıń tıǵızlıq funkciyasın tabıń.
 \\

\end{tabular}
\vspace{1cm}


\begin{tabular}{m{17cm}}
\textbf{26-variant}
\newline

\textbf{T1.} Tosınnanlı waqıya (elementar waqıyalar keńisligi, waqıyalar ústinde ámeller).
 \\
\textbf{T2.} Tiykarǵı аbsоlyut úzliksiz bólistiriliwler (nоrmаl bólistiriw, teń ólshewli bólistiriw, kórsetkishli bólistiriw). 
 \\
\textbf{A1.} $\left[ 0,1 \right]$ kesindiden tosınnan eki noqat tańlanadı. Ekinshi noqattıń koordinatasınıń birinshi noqattıń koordinatasına qatnası 0,6 dan kishi bolıw itimallıǵın tabıń.
 \\
\textbf{A2.} Qutıda 70 dana joqarı sapalı hám 10 dana tómen sapalı detallar bar. Qutıdan tosınnan alınǵan 6 dana detaldıń ishinde tómen sapalı detaldıń joq bolıw itimallıǵın tabıń.
 \\
\textbf{A3.} Hárbiriniń júzege asıw itimallıǵı $p$ ǵa teń bolǵan 10 dana Bernulli tájiriybesi ótkerilgende, tómendegi waqıyalardıń itimallıqların tabıń: Sátliler sanı keminde 3 dana.
 \\
\textbf{B1.} Eger \includegraphics[width=0.36181in,height=0.29444in]{mediaBpng/image1.png} ǵárezsiz hám birdey bólistirilgen tosınnanlıq shamalar izbe-izligi \includegraphics[width=0.57639in,height=0.27639in]{mediaBpng/image44.png} parametrli kórsetkishli bólistiriliwine bolsa, onda bul izbe-izlik úlken sanlar nızamına boysınama?
 \\
\textbf{B2.} \includegraphics[width=0.15972in,height=0.23958in]{mediaBpng/image49.png} úzliksiz tosınnanlıq shamanıń tıǵızlıq funkciyaları berilgen. Olarǵa sáykes \includegraphics[width=0.15972in,height=0.19653in]{mediaBpng/image50.png} tosınnanlıq shamanıń \includegraphics[width=0.50278in,height=0.30069in]{mediaBpng/image51.png} tıǵızlıq funkciyasın tabıń. \includegraphics[width=2.23333in,height=0.675in]{mediaBpng/image94.png} \includegraphics[width=0.55208in,height=0.28194in]{mediaBpng/image95.png}
 \\
\textbf{B3.} \(M(x,\ y)\) noqat tosınnanlı túrde \(0\  \leq \ x\  \leq \ 1,\ 0\  \leq \ y\  \leq \ 1\) kvadratqa taslandı. \(\min(x,\ y) \leq a\) bolsa, \(a \in (0;1\rbrack\)bolıwı itimallıǵın tabıń.
 \\
\textbf{C1.} Eger ǵárezsiz \includegraphics[width=0.15208in,height=0.24028in]{mediaCpng/image4.png} hám \includegraphics[width=0.15208in,height=0.19167in]{mediaCpng/image5.png} úzliksiz tosınnanlıq shamalardıń hárbiri \includegraphics[width=0.19167in,height=0.16806in]{mediaCpng/image33.png} parametrli kórsetkishli nızam boyınsha bólistirilgen bolsa, onda \includegraphics[width=0.50417in,height=0.29583in]{mediaCpng/image30.png} tosınnanlıq shamanıń tıǵızlıq funkciyasın tabıń.
 \\
\textbf{C2.} Eger \(\xi_{1},\xi_{2}...,\xi_{n}\) ǵárezsiz birdey bólistirilgen tosınnanlıq shamalar standart normal bólistirilgen bolsa, onda \(\xi_{1}^{2} + \xi_{2}^{2} + ...\  + \xi_{n}^{2}\) tosınnanlıq shamanıń tıǵızlıq funkciyasın tabıń.
 \\
\textbf{C3.} Eger \(\left\{ \xi_{n} \right\}\) diskret tosınnanlıq shamalar izbe-izliginiń bólistiriliw nızamları\(P\left\{ \xi_{n} = 1 \right\} = P\left\{ \xi_{n} = - 1 \right\} = \frac{1}{2} - \frac{1}{n},\) \(P\left\{ \xi_{n} = 0 \right\} = \frac{2}{n}\) bolsa, onda \(\xi_{n}\overset{d}{\rightarrow}\xi\) bolatuǵın \(\xi\) tosınnanlıq shamanıń bólistiriw funkciyasın tabıń.
 \\

\end{tabular}
\vspace{1cm}


\begin{tabular}{m{17cm}}
\textbf{27-variant}
\newline

\textbf{T1.} Tolıq itimallıq formulası (waqıyalardıń tolıq gruppası, dálilleniwi).
 \\
\textbf{T2.} 
Úlken sanlar nızamı (anıqlaması, Chebishev teoreması).
 \\
\textbf{A1.} Eki oyın kubigi taslanǵanda túsken ochkolardıń ayırması 1 ge teń bolıw itimallıǵın tabıń.
 \\
\textbf{A2.} Úsh qutınıń hárbirinde $n$ dana aq hám $m$ dana qara sharlar bar. Birinshi hám ekinshi qutıdan tosınnan 1 shardan alınıp, úshinshi qutıǵa salındı. Keyin, úshinshi qutıdan tosınnan bir shar alındı. а) Bul shardıń aq bolıw itimallıǵın tabıń. b) Sol shar aq bolsa, dáslepki eki qutıdan alınǵan sharlardıń aq bolıw itimallıǵın tabıń.
 \\
\textbf{A3.} Jámi 10 bala hám 12 qız bolǵan studentler toparınan 7 student sorawnama ótkeriw ushın tosınnan tańlap alındı. Olar ishinde 3 bala hám 4 qız bolıw itimallıǵın tabıń.
 \\
\textbf{B1.} 
$\xi$ tosınnanlı shamanıń \emph{f}(\emph{x}) tıǵızlıq funkciyasi berilgen bolsin. Tómendegilerdi esaplań: a) C; b) \emph{F}(\emph{x}); c) M$\xi$; d) D$\xi$; e) \emph{f}(\emph{x}) hám \emph{F}(\emph{x}) grafiklarin sızıń.\(f(x) = \left\{ \begin{matrix}
Cx,\ \ \ \ x \in \lbrack 0,1\rbrack, \\
C,\ \ \ \ \ \ \ x \in (1,2\rbrack, \\
0,\ \ \ keri\ jag'dayda.\ \ 
\end{matrix} \right.\ \)
 \\
\textbf{B2.} \(P\left\{ \xi = m \right\} = pq^{m},\ \ \ \ \ \ \ \ m = 0,\ \ 1,\ \ 2,\ \ \ldots\). Usı tosınnanlı $\xi$ shamasınıń matematikalıq kútiliwin tabıń.
 \\
\textbf{B3.} Tosınnanlı $\xi$ shamasiniń bólistiriw tiǵizliǵi berilgen: \(f(x) = A \cdot e^{- 5|x|}\). a) \emph{A}=?
 \\
\textbf{C1.} Tómende \includegraphics[width=0.46389in,height=0.25625in]{mediaCpng/image42.png} úzliksiz tosınnanlıq vektorlardıń tıǵızlıq funkciyaları berilgen. Olardıń \includegraphics[width=0.48819in,height=0.29583in]{mediaCpng/image43.png} hám \includegraphics[width=0.50417in,height=0.29583in]{mediaCpng/image44.png} marginal tıǵızlıq funkciyaların tabıń; \includegraphics[width=0.15972in,height=0.24028in]{mediaCpng/image45.png} hám \includegraphics[width=0.15972in,height=0.2in]{mediaCpng/image46.png} tosınnanlıq shamalardı ǵárezsizlikke tekseriń: \includegraphics[width=2.56806in,height=0.59167in]{mediaCpng/image56.png}
 \\
\textbf{C2.} Eger \(\xi\) tosınnanlı shama \((a,\sigma)\) parametrli normal bólistiriwine iye bolsa, onda onıń xarakteristikalıq funkciyası tabılsın.
 \\
\textbf{C3.} Eger \(\mathbf{\xi}_{\mathbf{n}}\overset{\mathbf{L}^{\mathbf{2}}}{\rightarrow}\mathbf{\xi}\) bolsa, onda \(n \rightarrow \infty\) de \(\mathbf{M}\mathbf{\xi}_{\mathbf{n}}^{\mathbf{2}}\mathbf{\rightarrow M}\mathbf{\xi}^{\mathbf{2}}\) ekenligin kórsetiń.
 \\

\end{tabular}
\vspace{1cm}


\begin{tabular}{m{17cm}}
\textbf{28-variant}
\newline

\textbf{T1.} Shártli itimallıq (anıqlaması, kóbеytiw tеorеması).
 \\
\textbf{T2.} Bólistiriw funkciyası (anıqlaması, tiykarǵı qásiyetleri).
 \\
\textbf{A1.} 
Hárbiriniń júzege asıw itimallıǵı $p$ ǵa teń bolǵan 10 dana Bernulli tájiriybesi ótkerilgende, tómendegi waqıyalardıń itimallıqların tabıń: Sátliler sanı 7 dana.
 \\
\textbf{A2.} Mashina jarısında 500 ekipaj qatnaspaqta. Hárbir ekipaj jarıstan texnikalıq nasazlıqlar sebepli 0,05 itimallıq penen, aydawshınıń keselligi sebepli bolsa 0,01 itimallıq penen shıǵıp ketiw múmkin. a) Aydawshınıń keselligi sebepli 5 ten artıq ekipaj jarıstan shıǵıp ketiwi itimallıǵın tabıń; b) 22 den 28 ge shekem ekipaj texnikalıq nasazlıqlar sebepli jarıstan shıǵıp ketiwi itimallıǵın tabıń.
 \\
\textbf{A3.} $\left[ 0,1 \right]$ kesindiden tosınnan eki noqat tańlanadı. Olardıń koordinataları ayırması modulı $1/6$ den kishi bolıw itimallıǵın tabıń.
 \\
\textbf{B1.} \includegraphics[width=0.15972in,height=0.23958in]{mediaBpng/image49.png} úzliksiz tosınnanlıq shamanıń tıǵızlıq funkciyaları berilgen. Olarǵa sáykes \includegraphics[width=0.15972in,height=0.19653in]{mediaBpng/image50.png} tosınnanlıq shamanıń \includegraphics[width=0.50278in,height=0.30069in]{mediaBpng/image51.png} tıǵızlıq funkciyasın tabıń. \includegraphics[width=2.325in,height=0.84028in]{mediaBpng/image66.png} \includegraphics[width=0.54583in,height=0.53403in]{mediaBpng/image67.png}
 \\
\textbf{B2.} \emph{R} radiuslı dóńgelek ishinen vertikal xordalar júrgiziledi. Tosınnan alınǵan xordanıń radiustan kishi bolıwı itimallıǵın tabıń.
 \\
\textbf{B3.} Eger \includegraphics[width=0.36181in,height=0.29444in]{mediaBpng/image1.png} ǵárezsiz tosınnanlıq shamalar izbe-izliginiń bólistiriliw nızamları
\includegraphics[width=2.41736in,height=0.50278in]{mediaBpng/image10.png} \includegraphics[width=1.50278in,height=0.50278in]{mediaBpng/image11.png} \includegraphics[width=0.75486in,height=0.23958in]{mediaBpng/image12.png}
bolsa, onda bul izbe-izlik úlken sanlar nızamına boysınama?
 \\
\textbf{C1.} Eger \(\left( \xi_{1},\xi_{2} \right)\) absolyut úziliksiz tosınnanlıq vektordıń tıǵızlıq funkciyası \(f(x,y) = \left\{ \begin{matrix}
Cxy,\ eger\ (x,y) \in D, \\
 \\
0,\ \ \ \ \ eger\ (x,y) \notin D,
\end{matrix} \right.\ \) bunda \(D = \left\{ (x,y):\ y > - x,\ y < 2,\ x < 0 \right\}\) bolsa, onda \(\xi_{1}\) komponentanıń shártsiz hám shártli tıǵızlıq funkciyaların tabıń. Sonıń menen birge, \(\xi_{1}\) hám \(\xi_{2}\) tosınnanlıq shamalardı ǵárezsizlikke tekseriń.
 \\
\textbf{C2.} Eger \(\xi\) tosınnanlıq shama \(\lbrack 0,\ \pi\rbrack\) aralıqta teń ólshewli bólistirilgen bolsa, onda \(M\sin\xi,\) \(D\sin\xi\) hám \(M\cos\xi,\) \(D\cos\xi\) mánislerin tabıń.
 \\
\textbf{C3.} Tómende \includegraphics[width=0.46389in,height=0.25625in]{mediaCpng/image42.png} úzliksiz tosınnanlıq vektorlardıń tıǵızlıq funkciyaları berilgen. Olardıń \includegraphics[width=0.48819in,height=0.29583in]{mediaCpng/image43.png} hám \includegraphics[width=0.50417in,height=0.29583in]{mediaCpng/image44.png} marginal tıǵızlıq funkciyaların tabıń; \includegraphics[width=0.15972in,height=0.24028in]{mediaCpng/image45.png} hám \includegraphics[width=0.15972in,height=0.2in]{mediaCpng/image46.png} tosınnanlıq shamalardı ǵárezsizlikke tekseriń: \includegraphics[width=3.24028in,height=0.67986in]{mediaCpng/image61.png}
 \\

\end{tabular}
\vspace{1cm}


\begin{tabular}{m{17cm}}
\textbf{29-variant}
\newline

\textbf{T1.} Waqıyalar algebrası ($\sigma$-algebra, minimal $\sigma$-algebra).
 \\
\textbf{T2.} Tosınnanlı shamanıń joqarı tártipli momentleri (baslanǵısh hám oraylıq momentleri, qásiyetleri).
 \\
\textbf{A1.} Mashina jarısında 600 ekipaj qatnaspaqta. Hárbir ekipaj jarıstan texnikalıq nasazlıqlar sebepli 0,04 itimallıq penen, al aydawshınıń keselligi sebepli bolsa 0,01 itimallıq penen shıǵıp ketiwi múmkin. a) Aydawshınıń keselligi sebepli 4 ten artıq ekipaj jarıstan shıǵıp ketiwi itimallıǵın tabıń; b) 23 ten 27 ge shekem ekipaj texnikalıq nasazlıqlar sebepli jarıstan shıǵıp ketiwi itimallıǵın tabıń.
 \\
\textbf{A2.} Jámi 10 bala hám 12 qız bolǵan studentler toparınan 5 student sorawnama ótkeriw ushın tosınnan tańlap alındı. Olar ishinde keminde bir student qız bolıw itimallıǵın tabıń.
 \\
\textbf{A3.} $\left[ 0,1 \right]$ kesindiden tosınnan eki noqat tańlanadı. Birinshi noqattıń koordinatasınıń ekinshi noqattıń koordinatasına qatnası 0,5 ten úlken bolıw itimallıǵın tabıń.
 \\
\textbf{B1.} $\xi$ tosınnanlı shamanıń \emph{f}(\emph{x}) tıǵızlıq funkciyasi berilgen bolsin. Tómendegilerdi esaplań: a) C; b) \emph{F}(\emph{x}); c) M$\xi$; d) D$\xi$; e) \emph{f}(\emph{x}) hám \emph{F}(\emph{x}) grafiklarin sızıń.\(f(x) = \left\{ \begin{matrix}
C/x,\ \ \ \ x \in \lbrack 1/e,e\rbrack, \\
\ \ \ \ 0,\ \ \ \ \ x \notin \lbrack 1/e,e\rbrack.\ \ 
\end{matrix} \right.\ \)
 \\
\textbf{B2.} Eger \includegraphics[width=0.36181in,height=0.29444in]{mediaBpng/image1.png} ǵárezsiz tosınnanlıq shamalar izbe-izliginiń bólistiriliw nızamları
\includegraphics[width=2.58264in,height=0.49097in]{mediaBpng/image16.png} \includegraphics[width=1.59514in,height=0.47847in]{mediaBpng/image17.png} \includegraphics[width=0.75486in,height=0.23958in]{mediaBpng/image18.png}
bolsa, onda bul izbe-izlik úlken sanlar nızamına boysınama?
 \\
\textbf{B3.} Zavod bazaǵa 5000 sipatli buyim jóneltken. Jolda buyimniń zálelleniw itimalliǵi 0,0002 ge teń. Bazaǵa 3 jaramsiz buyimniń kelip túsiw itimalliǵin tabiń.
 \\
\textbf{C1.} Eger ǵárezsiz \includegraphics[width=0.15208in,height=0.24028in]{mediaCpng/image4.png} hám \includegraphics[width=0.15208in,height=0.19167in]{mediaCpng/image5.png} úzliksiz tosınnanlıq shamalardıń hárbiri \includegraphics[width=0.19167in,height=0.16806in]{mediaCpng/image32.png} parametrli kórsetkishli nızam boyınsha bólistirilgen bolsa, onda \includegraphics[width=0.44028in,height=0.24028in]{mediaCpng/image29.png} tosınnanlıq shamanıń tıǵızlıq funkciyasın tabıń.
 \\
\textbf{C2.} Eger ǵárezsiz \includegraphics[width=0.15208in,height=0.24028in]{mediaCpng/image4.png} hám \includegraphics[width=0.15208in,height=0.19167in]{mediaCpng/image5.png} úzliksiz tosınnanlıq shamalardıń hárbiri \includegraphics[width=0.19167in,height=0.16806in]{mediaCpng/image31.png} parametrli kórsetkishli nızam boyınsha bólistirilgen bolsa, \includegraphics[width=0.47986in,height=0.52014in]{mediaCpng/image41.png} tosınnanlıq shamanıń tıǵızlıq funkciyasın tabıń.
 \\
\textbf{C3.} Tosınnanlıq vektordıń komponentaları absolyut úziliksizliginen tosınnanlıq vektordıń ózi de absolyut úziliksizligi kelip shıqpaslıǵın kórsetiń.
 \\

\end{tabular}
\vspace{1cm}


\begin{tabular}{m{17cm}}
\textbf{30-variant}
\newline

\textbf{T1.} Bernulli sxemаsı ushın limit teоremаlаr (Muavr-Laplas integrallıq teoreması, qásiyetleri).
 \\
\textbf{T2.} Tıǵızlıq funkciyası (anıqlaması, tiykarǵıqásiyetleri).
 \\
\textbf{A1.} Eki oyın kubigi taslanǵanda túsken ochkolardıń qosındısı 7 ge teń bolıw itimallıǵın tabıń.
 \\
\textbf{A2.} Hárbiriniń júzege asıw itimallıǵı $p$ ǵa teń bolǵan 10 dana Bernulli tájiriybesi ótkerilgende, tómendegi waqıyalardıń itimallıqların tabıń: Sátliler sanı tek 2 dana hám olar arasında 3 dana sátsiz bolıw.
 \\
\textbf{A3.} Berilgen $1,2,\ldots ,10$ sanlarınıń arasınan tosınnan bir san tańlandı. Meyli, bul san $m$ bolsın. Soń, $\left[ 0,m \right]$ kesindiden tosınnan $\xi $ noqat tańlandı. a) $\xi >8$ bolıw itimallıǵın tabıń. b) Eger $\xi >8$ bolsa, onda $m=9$ bolıw itimallıǵın tabıń.
 \\
\textbf{B1.} $\xi$ tosınnanlı shamanıń \emph{f}(\emph{x}) tıǵızlıq funkciyasi berilgen bolsin. Tómendegilerdi esaplań: a) C; b) \emph{F}(\emph{x}); c) M$\xi$; d) D$\xi$; e) \emph{f}(\emph{x}) hám \emph{F}(\emph{x}) grafiklarin sızıń.\(f(x) = \left\{ \begin{matrix}
\ \ \ \ \ \ \ \ 0,\ \ \ \ \ \ x < 0, \\
C/(x + 1)^{5},\ \ \ \ \ x \geq 0.\ \ 
\end{matrix} \right.\ \)
 \\
\textbf{B2.} 
\includegraphics[width=0.15972in,height=0.23958in]{mediaBpng/image49.png} úzliksiz tosınnanlıq shamanıń tıǵızlıq funkciyaları berilgen. Olarǵa sáykes \includegraphics[width=0.15972in,height=0.19653in]{mediaBpng/image50.png} tosınnanlıq shamanıń \includegraphics[width=0.50278in,height=0.30069in]{mediaBpng/image51.png} tıǵızlıq funkciyasın tabıń. \includegraphics[width=2.23333in,height=0.84028in]{mediaBpng/image52.png} \includegraphics[width=0.88333in,height=0.23958in]{mediaBpng/image53.png}
 \\
\textbf{B3.} Tosınnanlı $\xi$ shamasınıń tıǵızlıq funkciyasi berilgen: \(f(x) = e^{- 3|x|}\) Usi shamanıń matematikalıq kútiliwin tabıń.
 \\
\textbf{C1.} Tómende \includegraphics[width=0.46389in,height=0.25625in]{mediaCpng/image42.png} úzliksiz tosınnanlıq vektorlardıń tıǵızlıq funkciyaları berilgen. Olardıń \includegraphics[width=0.48819in,height=0.29583in]{mediaCpng/image43.png} hám \includegraphics[width=0.50417in,height=0.29583in]{mediaCpng/image44.png} marginal tıǵızlıq funkciyaların tabıń; \includegraphics[width=0.15972in,height=0.24028in]{mediaCpng/image45.png} hám \includegraphics[width=0.15972in,height=0.2in]{mediaCpng/image46.png} tosınnanlıq shamalardı ǵárezsizlikke tekseriń: \includegraphics[width=3.45625in,height=0.53611in]{mediaCpng/image48.png}
 \\
\textbf{C2.} Eger \(\left\{ \xi_{n} \right\}\) ǵárezsiz hám \(\mathbf{\lbrack 0,1\rbrack}\) aralıqta teń ólshemli bólistirilgen tosınnanlıq shamalar izbe-izligi bolsa, onda \(\left\{ \mathbf{\xi}_{\mathbf{(}\mathbf{n}\mathbf{)}}\mathbf{=}\mathbf{\max}\mathbf{\{}\mathbf{\xi}_{\mathbf{1}}\mathbf{,...,}\mathbf{\xi}_{\mathbf{n}}\mathbf{\}} \right\}\) izbe-izlik 1 ge itimallıq boyınsha jıynaqlılıǵın kórsetiń.
 \\
\textbf{C3.} Eger \(\xi\sim N\left( a,\sigma^{2} \right)\) bolsa, onda \(\xi\) tosınnanlıq shamanıń joqarı tártipli oraylıq momentlerin tabıń.
 \\

\end{tabular}
\vspace{1cm}


\begin{tabular}{m{17cm}}
\textbf{31-variant}
\newline

\textbf{T1.} Bernulli sxemаsı ushın limit teоremаlаr (Muavr-Laplas lokallıq teoreması, qásiyetleri).
 \\
\textbf{T2.} Tiykarǵı diskret bólistiriliwler (Binomial, Puasson hám geometriyalıq bólistiriliwler).
 \\
\textbf{A1.} Firmada 11 erkek hám 6 hayal jumısshı isleydi. Tosınnan 5 jumısshı ajıratılıp alındı. Ajıratılıp alınǵan jumısshılardıń barlıǵı hayal bolıw itimallıǵın tabıń.
 \\
\textbf{A2.} Bazıbir qalada solaqaylar ortasha esapta $1,5$, sol hám oń qollarına teńdey iyelik qılatuǵın adamlar $9$, al qalǵanları ońaqaylar. Jámi 300 adam arasında tómendegi waqıyalardıń júzege asıwı itimallıqların tabıń: a) keminde tórt solaqay boladı; b) 15 ten 20 ǵa shekem sol hám oń qollarına teńdey iyelik qılatuǵın adamlar boladı.
 \\
\textbf{A3.} Hárbiriniń júzege asıw itimallıǵı $p$ ǵa teń bolǵan 10 dana Bernulli tájiriybesi ótkerilgende, tómendegi waqıyalardıń itimallıqların tabıń: Sátsizler sanı kóbi menen 2 dana.
 \\
\textbf{B1.} Barlıq tárepi boyalǵan kub mıń dana birdey ólshemdegi kubiklerge bólingen hám aralastırıp jiberilgen. Tosınnan alınǵan kubiktiń a) bir tárepi; b) eki tárepi; c) úsh tárepi boyalǵan bolıwı itimallıǵın tabıń.
 \\
\textbf{B2.} \includegraphics[width=0.15972in,height=0.23958in]{mediaBpng/image49.png} úzliksiz tosınnanlıq shamanıń tıǵızlıq funkciyaları berilgen. Olarǵa sáykes \includegraphics[width=0.15972in,height=0.19653in]{mediaBpng/image50.png} tosınnanlıq shamanıń \includegraphics[width=0.50278in,height=0.30069in]{mediaBpng/image51.png} tıǵızlıq funkciyasın tabıń. \includegraphics[width=1.95069in,height=0.58264in]{mediaBpng/image70.png} \includegraphics[width=0.84028in,height=0.23958in]{mediaBpng/image72.png}
 \\
\textbf{B3.} Eger \includegraphics[width=0.36181in,height=0.29444in]{mediaBpng/image1.png} ǵárezsiz tosınnanlıq shamalar izbe-izliginiń bólistiriliw nızamları
\includegraphics[width=2.51528in,height=0.52153in]{mediaBpng/image40.png} \includegraphics[width=2.59514in,height=0.47847in]{mediaBpng/image41.png} \includegraphics[width=0.75486in,height=0.23958in]{mediaBpng/image42.png}
bolsa, onda bul izbe-izlik úlken sanlar nızamına boysınama?
 \\
\textbf{C1.} Eger ǵárezsiz \includegraphics[width=0.15208in,height=0.24028in]{mediaCpng/image4.png} hám \includegraphics[width=0.15208in,height=0.19167in]{mediaCpng/image5.png} úzliksiz tosınnanlıq shamalar sáykes túrde, \includegraphics[width=0.4in,height=0.25625in]{mediaCpng/image6.png} hám \includegraphics[width=0.44028in,height=0.25625in]{mediaCpng/image7.png} parametrli normal nızam boyınsha bólistirilgen bolsa, onda \includegraphics[width=0.72778in,height=0.24028in]{mediaCpng/image8.png} tosınnanlıq shamanıń tıǵızlıq funkciyasın tabıń.
 \\
\textbf{C2.} Eger \(\xi\) tosınnanlı shama \((a,\sigma)\) parametrli normal bólistiriwine iye bolsa, onda onıń xarakteristikalıq funkciyası tabılsın.
 \\
\textbf{C3.} Tómende \includegraphics[width=0.46389in,height=0.25625in]{mediaCpng/image42.png} úzliksiz tosınnanlıq vektorlardıń tıǵızlıq funkciyaları berilgen. Olardıń \includegraphics[width=0.48819in,height=0.29583in]{mediaCpng/image43.png} hám \includegraphics[width=0.50417in,height=0.29583in]{mediaCpng/image44.png} marginal tıǵızlıq funkciyaların tabıń; \includegraphics[width=0.15972in,height=0.24028in]{mediaCpng/image45.png} hám \includegraphics[width=0.15972in,height=0.2in]{mediaCpng/image46.png} tosınnanlıq shamalardı ǵárezsizlikke tekseriń: \includegraphics[width=3.31181in,height=0.84792in]{mediaCpng/image67.png}
 \\

\end{tabular}
\vspace{1cm}


\begin{tabular}{m{17cm}}
\textbf{32-variant}
\newline

\textbf{T1.} Itimallıq anıqlamaları (klassikalıq, geometriyalıq anıqlamaları).
 \\
\textbf{T2.} Tosınnanlı shamanıń dispersiyası (anıqlaması, qásiyetleri).
 \\
\textbf{A1.} $\left[ 0,1 \right]$ kesindiden tosınnan eki noqat tańlanadı. Ekinshi noqattıń koordinatası birinshi noqattıń koordinatası úsh esesinen úlken bolıw itimallıǵın tabıń.
 \\
\textbf{A2.} Samolyotqa birimlep úsh márte oq atıladı. Birinshi márte atqanda tiyiw itimallıǵı 0,4 ke, ekinshisinde 0,5 ke, úshinshisinde bolsa 0,7 ge teń. Samolyottı isten shıǵarıw ushın úsh márte oq tiyiwi jetkilikli, bir márte tiygende 0,2 ge teń itimallıq penen qatardan shıǵadı, eki márte tiygende 0,6 ǵa teń itimallıq penen isten shıǵadı. 
a) Úsh márte oq atıw nátiyjesinde samolyottıń tolıǵı menen isten shıǵıw itimallıǵın tabıń. b) Eger samolyot tolıǵı menen isten shıqqan bolsa, ol eki oq tiyip isten shıqqan bolıw itimallıǵın tabıń.
 \\
\textbf{A3.} Eki oyın kubigi taslanǵanda túsken ochkolardıń qosındısı 3 ten úlken bolıw itimallıǵın tabıń.
 \\
\textbf{B1.} $\xi$ tosınnanlı shamanıń \emph{f}(\emph{x}) tıǵızlıq funkciyasi berilgen bolsin. Tómendegilerdi esaplań: a) C; b) \emph{F}(\emph{x}); c) M$\xi$; d) D$\xi$; e) \emph{f}(\emph{x}) hám \emph{F}(\emph{x}) grafiklarin sızıń.\(f(x) = \left\{ \begin{matrix}
C(1 - x/3),\ \ \ \ x \in \lbrack 0,3\rbrack, \\
\ \ \ \ \ \ \ \ 0,\ \ \ \ \ \ \ \ \ \ \ x \notin \lbrack 0,3\rbrack.\ \ 
\end{matrix} \right.\ \)
 \\
\textbf{B2.} Pul lotereyasında 100 bilet shıǵarılǵan. 50 swmlıq 1 utıs, 10 swmlıq 10 utıs bar. Bir lotereya biletiniń iyesi ushın múmkin bolǵan utıstıń bahasınıń bólistiriw nızamın jazıń.
 \\
\textbf{B3.} $\xi$ tosınnanlı shamanıń \emph{f}(\emph{x}) tıǵızlıq funkciyasi berilgen bolsin. Tómendegilerdi esaplań: a) C; b) \emph{F}(\emph{x}); c) M$\xi$; d) D$\xi$; e) \emph{f}(\emph{x}) hám \emph{F}(\emph{x}) grafiklarin sızıń.\(f(x) = \left\{ \begin{matrix}
C(1 - 0.5|x|),\ \ \ \ x \in \lbrack - 2,2\rbrack, \\
\ \ \ \ \ \ \ \ 0,\ \ \ \ \ \ \ \ \ \ \ x \notin \lbrack - 2,2\rbrack.\ \ 
\end{matrix} \right.\ \)
 \\
\textbf{C1.} Eger \(\xi_{1}\) hám \(\xi_{2}\) sáykes túrde \(\lambda_{1}\) hám \(\lambda_{2}\) parametrli Puasson bólistiriliwine iye bolǵan ǵárezsiz tosınnanlıq shamalar bolsa, onda \(\xi_{1} + \xi_{2}\) tosınnanlıq shamanıń bólistiriliwin tabıń.
 \\
\textbf{C2.} Oraylıq limit teorema járdeminde tómendegi teńlikti dálilleń: \(\lim_{n \rightarrow \infty}e^{- n}\sum_{k = 1}^{n}\frac{n^{k}}{k!} = \frac{1}{2}.\)
 \\
\textbf{C3.} Tómende \includegraphics[width=0.46389in,height=0.25625in]{mediaCpng/image42.png} úzliksiz tosınnanlıq vektorlardıń tıǵızlıq funkciyaları berilgen. Olardıń \includegraphics[width=0.48819in,height=0.29583in]{mediaCpng/image43.png} hám \includegraphics[width=0.50417in,height=0.29583in]{mediaCpng/image44.png} marginal tıǵızlıq funkciyaların tabıń; \includegraphics[width=0.15972in,height=0.24028in]{mediaCpng/image45.png} hám \includegraphics[width=0.15972in,height=0.2in]{mediaCpng/image46.png} tosınnanlıq shamalardı ǵárezsizlikke tekseriń: \includegraphics[width=2.79167in,height=0.67986in]{mediaCpng/image63.png}
 \\

\end{tabular}
\vspace{1cm}


\begin{tabular}{m{17cm}}
\textbf{33-variant}
\newline

\textbf{T1.} Itimallıqlar teoriyası aksiomaları (ólshewli keńislik, itimallıq keńisligi).
 \\
\textbf{T2.} Oraylıq limit teorema (anıqlaması, ǵárezsiz birdey bólistirilgen tosınnanlı shamalar ushın).
 \\
\textbf{A1.} Eki oyın kubigi taslanǵanda túsken ochkolardıń kóbeymesi 10 nan asıp ketpew itimallıǵın tabıń.
 \\
\textbf{A2.} Hárbiriniń júzege asıw itimallıǵı $p$ ǵa teń bolǵan 10 dana Bernulli tájiriybesi ótkerilgende, tómendegi waqıyalardıń itimallıqların tabıń: Sátliler sanı 6 dana, sonıń menen birge, olardıń barlıǵı dáslepki altı tájiriybede ámelge asıwı.
 \\
\textbf{A3.} Qutıda 30 dana birdey sharlar bolıp, olardıń 20 danası qızıl hám 10 danası kók reńdegi sharlar. Tosınnan alınǵan 3 dana shardıń 2 danası qızıl shar bolıw itimallıǵın tabıń.
 \\
\textbf{B1.} Birdey kartochkalarǵa A,A,A,E,I,M,M,K,T,T háripleri jazilǵan hám jaqsilap aralastirip tóńkerip jayilǵan. Izbe-iz alinǵan kartochkalardi alinǵan tártibinde jaylastiriw nátiyjesinde «MATEMATIKA» sóziniń kelip shiǵiw itimalliǵin tabiń.
 \\
\textbf{B2.} Eger \includegraphics[width=0.36181in,height=0.29444in]{mediaBpng/image1.png} ǵárezsiz tosınnanlıq shamalar izbe-izliginiń bólistiriliw nızamları
\includegraphics[width=2.62569in,height=0.49097in]{mediaBpng/image37.png} \includegraphics[width=1.55208in,height=0.47847in]{mediaBpng/image38.png} \includegraphics[width=0.77292in,height=0.25764in]{mediaBpng/image39.png}
bolsa, onda bul izbe-izlik úlken sanlar nızamına boysınama?
 \\
\textbf{B3.} Tosınnanlı \(\xi\) shamasınıń bólistiriw tıǵızlıǵı \(f(x) = \frac{1}{2}e^{- |x|}\) bolsa, usı shamanıń xarakteristikalıq funkciyasın tabıń.
 \\
\textbf{C1.} Eger ǵárezsiz \includegraphics[width=0.15208in,height=0.24028in]{mediaCpng/image4.png} hám \includegraphics[width=0.15208in,height=0.19167in]{mediaCpng/image5.png} úzliksiz tosınnanlıq shamalar sáykes túrde, \includegraphics[width=0.4in,height=0.24028in]{mediaCpng/image34.png} aralıqta teń ólshemli hám \includegraphics[width=0.44028in,height=0.21597in]{mediaCpng/image35.png} parametrli kórsetkishli nızam boyınsha bólistirilgen bolsa, onda \includegraphics[width=0.44028in,height=0.24028in]{mediaCpng/image14.png} tosınnanlıq shamanıń tıǵızlıq funkciyasın tabıń.
 \\
\textbf{C2.} 
Eger \(\xi\) tosınnanlıq shama hám \(\left\{ \xi_{n} \right\}\) tosınnanlıq shamalar izbe-izligi ǵárezsiz birdey standart normal bólistirilgen bolsa, onda \(\left\{ \mathbf{\eta}_{\mathbf{n}} \right\}\mathbf{=}\left\{ \frac{\mathbf{\xi}\sqrt{\mathbf{n}}}{\sqrt{\mathbf{\xi}_{\mathbf{1}}^{\mathbf{2}}\mathbf{+}\mathbf{...}\mathbf{+}\mathbf{\xi}_{\mathbf{n}}^{\mathbf{2}}}} \right\}\) tosınnanlıq shamalar izbe-izligining limit bólistiriw funkciyası standart normal bólistiriliw bolıwın kórsetiń.
 \\
\textbf{C3.} Kóp ólshemli tıǵızlıq funkciyası óziniń marginal tıǵızlıq funkciyaları arqalı bir mánisli anıqlanbaytuǵınlıǵın kórsetiń.
 \\

\end{tabular}
\vspace{1cm}


\begin{tabular}{m{17cm}}
\textbf{34-variant}
\newline

\textbf{T1.} Bernulli sxemаsı ushın limit teоremаlаr (Muavr-Laplas lokallıq teoreması, qásiyetleri).
 \\
\textbf{T2.} 
Úlken sanlar nızamı (anıqlaması, Chebishev teoreması).
 \\
\textbf{A1.} Albomda 6 dana jańa hám 10 dana múddeti ótken markalar bar. Albomnan tosınnan 3 marka alıp taslandı. Bunnan soń, tosınnan 2 marka alındı. a) Bul 2 markanıń jańa bolıw itimallıǵın tabıń. b) Sol 2 marka jańa ekenligi belgili bolsa, dáslepki alınǵan 3 markanıń múddeti ótken bolıw itimallıǵın tabıń.
 \\
\textbf{A2.} Studentler eki jıl dawamında matematikalıq kitaplardan hárbirinde 20 máseleni óz ishine alǵan matematikadan 15 tipikalıq esaplardı orınlaydı. Kompyuterde matematikalıq paket járdeminde máseleni nadurıs sheshiwi itimallıǵı 0,01 ge, paket járdemisiz 0,2 ge teń. Úsh jıl ishinde tómendegi waqıyalardıń júzege asıwı itimallıqların tabıń: a) matematikalıq paketten turaqlı túrde paydalanatuǵın student 5 ten kóp bolmaǵan máselelerdi nadurıs sheshken bolsa; b) matematikalıq paketten paydalanbaytuǵın student 50 dan 70 ge shekem máseleni nadurıs sheshedi.
 \\
\textbf{A3.} $\left[ 0,1 \right]$ kesindiden tosınnan eki noqat tańlanadı. Noqatlardıń koordinataları qosındısı 1,5 ten kishi bolıw itimallıǵın tabıń.
 \\
\textbf{B1.} \includegraphics[width=0.15972in,height=0.23958in]{mediaBpng/image49.png} úzliksiz tosınnanlıq shamanıń tıǵızlıq funkciyaları berilgen. Olarǵa sáykes \includegraphics[width=0.15972in,height=0.19653in]{mediaBpng/image50.png} tosınnanlıq shamanıń \includegraphics[width=0.50278in,height=0.30069in]{mediaBpng/image51.png} tıǵızlıq funkciyasın tabıń. \includegraphics[width=2.12292in,height=0.58264in]{mediaBpng/image73.png} \includegraphics[width=0.91389in,height=0.50278in]{mediaBpng/image74.png}
 \\
\textbf{B2.} {[}-1;1{]} kesindide teń ólshewli bólistirilgen tosınnanlı shamanıń xarakteristikalıq funkciyasın tabıń.
 \\
\textbf{B3.} \includegraphics[width=0.15972in,height=0.23958in]{mediaBpng/image49.png} úzliksiz tosınnanlıq shamanıń tıǵızlıq funkciyaları berilgen. Olarǵa sáykes \includegraphics[width=0.15972in,height=0.19653in]{mediaBpng/image50.png} tosınnanlıq shamanıń \includegraphics[width=0.50278in,height=0.30069in]{mediaBpng/image51.png} tıǵızlıq funkciyasın tabıń. \includegraphics[width=2.17153in,height=0.58264in]{mediaBpng/image75.png} \includegraphics[width=0.91389in,height=0.50278in]{mediaBpng/image76.png}
 \\
\textbf{C1.} Eger \(\xi\sim E(\lambda)\) bolsa, onda \(\xi\) tosınnanlıq shamanıń joqarı tártipli baslanǵısh momentlerin tabıń.
 \\
\textbf{C2.} Tómende \includegraphics[width=0.46389in,height=0.25625in]{mediaCpng/image42.png} úzliksiz tosınnanlıq vektorlardıń tıǵızlıq funkciyaları berilgen. Olardıń \includegraphics[width=0.48819in,height=0.29583in]{mediaCpng/image43.png} hám \includegraphics[width=0.50417in,height=0.29583in]{mediaCpng/image44.png} marginal tıǵızlıq funkciyaların tabıń; \includegraphics[width=0.15972in,height=0.24028in]{mediaCpng/image45.png} hám \includegraphics[width=0.15972in,height=0.2in]{mediaCpng/image46.png} tosınnanlıq shamalardı ǵárezsizlikke tekseriń: \includegraphics[width=2.86389in,height=0.73611in]{mediaCpng/image64.png}
 \\
\textbf{C3.} Meyli, \(\xi_{1},...,\xi_{n}\) tosınnanlıq shamalar ǵárezsiz hám \(\lbrack a,b\rbrack\) aralıqta teń ólshemli bólistirilgen bolıp, \(\eta_{1} = \max\left( \xi_{1},...,\xi_{n} \right)\) hám \(\eta_{2} = \min\left( \xi_{1},...,\xi_{n} \right)\) bolsın. Onda \(\left( \eta_{1},\eta_{2} \right)\) tosınnanlıq vektordıń kovariaciyasın tabıń.
 \\

\end{tabular}
\vspace{1cm}


\begin{tabular}{m{17cm}}
\textbf{35-variant}
\newline

\textbf{T1.} Tosınnanlı waqıya (elementar waqıyalar keńisligi, waqıyalar ústinde ámeller).
 \\
\textbf{T2.} Tosınnanlı shamanıń dispersiyası (anıqlaması, qásiyetleri).
 \\
\textbf{A1.} $\left[ 0,2 \right]$ kesindiden tosınnan $x$ hám $y$ noqat tańlanadı. Olar ushın $\left| \begin{matrix}
   1 & x  \\
   x & y  \\
\end{matrix} \right|>0$ bolıw itimallıǵın tabıń.
 \\
\textbf{A2.} Jıldıń qálegen kúninde bala tuwılıw itimallıǵı teń dep esaplap, 200 bala arasında tómendegi waqıyalardıń júzege asıwı itimallıqların tabıń: a) anıq úsh bala 1-yanvarda tuwılǵan; b) báhárde 48 den 53 ke shekem bala tuwılǵan.
 \\
\textbf{A3.} Hárbiriniń júzege asıw itimallıǵı $p$ ǵa teń bolǵan 10 dana Bernulli tájiriybesi ótkerilgende, tómendegi waqıyalardıń itimallıqların tabıń: Sátliler sanı, sátsizler sanınan tórtke kem.
 \\
\textbf{B1.} $\xi$ tosınnanlı shamanıń \emph{f}(\emph{x}) tıǵızlıq funkciyasi berilgen bolsin. Tómendegilerdi esaplań: a) C; b) \emph{F}(\emph{x}); c) M$\xi$; d) D$\xi$; e) \emph{f}(\emph{x}) hám \emph{F}(\emph{x}) grafiklarin sızıń.\(f(x) = \left\{ \begin{matrix}
\ \ \ \ \ \ \ \ 0,\ \ \ \ \ \ x \leq 0, \\
Cxe^{- x},\ \ \ \ \ x > 0.\ \ 
\end{matrix} \right.\ \)
 \\
\textbf{B2.} Eger \includegraphics[width=0.36181in,height=0.29444in]{mediaBpng/image1.png} ǵárezsiz tosınnanlıq shamalar izbe-izligi \includegraphics[width=0.68681in,height=0.54583in]{mediaBpng/image47.png} aralıqta teń ólshemli bólistirilgen bolsa, onda bul izbe-izlik úlken sanlar nızamına boysınama?
 \\
\textbf{B3.} Birdey kartochkalarǵa jazilǵan A,A,A,A,P,R,Q,Q,Q,L háriplerinen tosinnan alinǵan kartochkalardi aliniw tártibinde jaylastiriwdan «QARAQALPAQ» sóziniń kelip shiǵiw itimalliǵin tabiń
 \\
\textbf{C1.} Eger \(\left( \xi_{1},\xi_{2} \right)\) absolyut úziliksiz tosınnanlıq vektordıń \(\xi_{1}\) hám \(\xi_{2}\) komponentaları ǵárezsiz bolıp, olardıń hárbiri standart normal bólistirilgen bolsa, onda \(\left( \xi_{1},\xi_{2} \right)\) tosınnanlıq noqattıń \(D = \left\{ (x,y):\ x^{2} + y^{2} \leq R^{2} \right\}\) oblastqa túsiw itimallıǵın tabıń.
 \\
\textbf{C2.} Eger \(\left\{ \xi_{n} \right\}\) ǵárezsiz tosınnanlıq shamalar izbe-izligi \(\lbrack 0,1\rbrack\) aralıqta teń ólshemli bólistirilgen bolıp, \(g(x)\) funkciya sol aralıqta úziliksiz bolsa, onda\(\frac{g\left( \xi_{1} \right) + ... + g\left( \xi_{n} \right)}{n}\overset{P}{\rightarrow}\int_{0}^{1}{g(x)}dx\) ekenligin kórsetiń.
 \\
\textbf{C3.} Eger ǵárezsiz \includegraphics[width=0.15208in,height=0.24028in]{mediaCpng/image4.png} hám \includegraphics[width=0.15208in,height=0.19167in]{mediaCpng/image5.png} úzliksiz tosınnanlıq shamalardıń hárbiri \includegraphics[width=0.41597in,height=0.24028in]{mediaCpng/image26.png} aralıqta teń ólshemli bólistirilgen bolsa, onda \includegraphics[width=0.35972in,height=0.27222in]{mediaCpng/image27.png} tosınnanlıq shamanıń tıǵızlıq funkciyasın tabıń.
 \\

\end{tabular}
\vspace{1cm}


\begin{tabular}{m{17cm}}
\textbf{36-variant}
\newline

\textbf{T1.} Itimallıqlar teoriyası aksiomaları (ólshewli keńislik, itimallıq keńisligi).
 \\
\textbf{T2.} Tiykarǵı аbsоlyut úzliksiz bólistiriliwler (nоrmаl bólistiriw, teń ólshewli bólistiriw, kórsetkishli bólistiriw). 
 \\
\textbf{A1.} Oyın kubigi taslanǵanda pútin ochkonıń túsiw itimallıǵın tabıń.
 \\
\textbf{A2.} Dushpan nıshanın joq etiw ushın hár túrlı eki samolyot ushıp ketti. Birinshi túrdegi samolyot nıshandı $0,9$ itimallıq penen, ekinshi túrdegi samolyot $0,8$ itimallıq penen joq etiw múmkin. Biraq, dushpannıń hawa hújiminen qorǵanıwı birinshi túrdegi samolyottı $0,95$ itimallıq penen, ekinshi túrdegi samolyottı $0,85$ itimallıq penen urıp túsiriwi múmkin. a) Samolyotlar nıshandı joq etiw itimallıǵın tabıń. b) Nıshan joq etilgen bolsa, onı tek ekinshi samolyot joq etken bolıw itimallıǵın tabıń.
 \\
\textbf{A3.} 36 dana kartalar kolodasınan tosınnan alınǵan 6 dana karta ishinde anıq 5 dana karta birdey reńde hám 1 dana karta basqa reńde bolıw itimallıǵın tabıń.
 \\
\textbf{B1.} İdısta 10 shar bolıp, olardan 3 ewi aq sharlar. İdıstan táwekelge 3 shar alınadı. Tosınnanlı \(\xi\) shaması -- alınǵan aq sharlar sanı. Onıń bólistiriliw nızamın jazıń.
 \\
\textbf{B2.} \includegraphics[width=0.15972in,height=0.23958in]{mediaBpng/image49.png} úzliksiz tosınnanlıq shamanıń tıǵızlıq funkciyaları berilgen. Olarǵa sáykes \includegraphics[width=0.15972in,height=0.19653in]{mediaBpng/image50.png} tosınnanlıq shamanıń \includegraphics[width=0.50278in,height=0.30069in]{mediaBpng/image51.png} tıǵızlıq funkciyasın tabıń. \includegraphics[width=1.50278in,height=0.58889in]{mediaBpng/image82.png} \includegraphics[width=0.88333in,height=0.28194in]{mediaBpng/image83.png}
 \\
\textbf{B3.} $\xi$ tosınnanlı shamanıń \emph{f}(\emph{x}) tıǵızlıq funkciyasi berilgen bolsin. Tómendegilerdi esaplań: a) C; b) \emph{F}(\emph{x}); c) M$\xi$; d) D$\xi$; e) \emph{f}(\emph{x}) hám \emph{F}(\emph{x}) grafiklarin sızıń.\(f(x) = \left\{ \begin{matrix}
C\sqrt[3]{1 - x},\ \ \ \ x \in \lbrack 0,1\rbrack, \\
\ \ \ \ \ \ \ \ 0,\ \ \ \ \ \ \ \ \ \ x \notin \lbrack 0,1\rbrack.\ \ 
\end{matrix} \right.\ \)
 \\
\textbf{C1.} Eger \(\left\{ \xi_{n} \right\}\) ǵárezsiz birdey bólistirilgen tosınnanlıq shamalar izbe-izligi bolıp, onıń bólistiriw funkciyası \(F_{\xi_{1}}(x) = \left\{ \begin{matrix}
\ 1 - e^{\lambda - x},\ \ eger\ \ x \geq \lambda, \\
 \\
\ \ \ \ \ \ 0,\ \ \ \ \ \ \ \ \ \ \ eger\ \ x < \lambda
\end{matrix} \right.\ \) bolsa, onda \(\left\{ \eta_{n} \right\} = \left\{ min(\xi_{1},...,\xi_{n}) \right\}\) izbe-izliktiń \(\mathbf{\lambda}\) ǵa bir itimallıq penen jıynaqlılıǵın kórsetiń.
 \\
\textbf{C2.} Eger \(\xi\) tosınnanlıq shama \(\lbrack 0,\ \pi\rbrack\) aralıqta teń ólshewli bólistirilgen bolsa, onda \(M\sin\xi,\) \(D\sin\xi\) hám \(M\cos\xi,\) \(D\cos\xi\) mánislerin tabıń.
 \\
\textbf{C3.} Tómende \includegraphics[width=0.46389in,height=0.25625in]{mediaCpng/image42.png} úzliksiz tosınnanlıq vektorlardıń tıǵızlıq funkciyaları berilgen. Olardıń \includegraphics[width=0.48819in,height=0.29583in]{mediaCpng/image43.png} hám \includegraphics[width=0.50417in,height=0.29583in]{mediaCpng/image44.png} marginal tıǵızlıq funkciyaların tabıń; \includegraphics[width=0.15972in,height=0.24028in]{mediaCpng/image45.png} hám \includegraphics[width=0.15972in,height=0.2in]{mediaCpng/image46.png} tosınnanlıq shamalardı ǵárezsizlikke tekseriń: \includegraphics[width=2.75208in,height=0.67986in]{mediaCpng/image52.png}
 \\

\end{tabular}
\vspace{1cm}


\begin{tabular}{m{17cm}}
\textbf{37-variant}
\newline

\textbf{T1.} Bernulli sxemаsı ushın limit teоremаlаr (Muavr-Laplas integrallıq teoreması, qásiyetleri).
 \\
\textbf{T2.} Xarakteristikalıq funkciyalar (anıqlaması, tiykarǵı qásiyetleri).
 \\
\textbf{A1.} Zavodta avtomat basqarılatuǵın 14 dana hám qolda basqarılatuǵın 6 dana qurılmalar bar. Avtomat basqarılatuǵın qurılmalar ushın standart bolmaǵan ónimlerdi islep shıǵarıw itimallıǵı $0,001$ ge, al qolda basqarılatuǵın qurılmalar ushın bolsa $0,002$ ge teń. a) Laboratoriya analizine tosınnan alınǵan ónimniń standart bolmaǵan bolıwı itimallıǵın tabıń. b) Eger ónimniń standart emesligi belgili bolsa, onda sol ónimniń qolda basqarılatuǵın qurılmada islep shıǵarılǵanlıǵı itimallıǵın tabıń.
 \\
\textbf{A2.} Qutıda 80 dana joqarı sapalı hám 20 dana tómen sapalı detallar bar. Qutıdan tosınnan alınǵan 14 dana detaldıń ishinde tómen sapalı detaldıń joq bolıw itimallıǵın tabıń.
 \\
\textbf{A3.} Eki oyın kubigi taslanǵanda túsken ochkolardıń ayırması 2 ge teń bolıw itimallıǵın tabıń.
 \\
\textbf{B1.} Hár biriniń uzınlıǵı a dan aspaytuǵın eki tosınnanlı alınǵan kesindiler qosındısı a dan úlken bolıwı itimallıǵı qanday?
 \\
\textbf{B2.} Eger \includegraphics[width=0.36181in,height=0.29444in]{mediaBpng/image1.png} ǵárezsiz tosınnanlıq shamalar izbe-izliginiń bólistiriliw nızamları
\includegraphics[width=2.55833in,height=0.50278in]{mediaBpng/image7.png} \includegraphics[width=1.41736in,height=0.50278in]{mediaBpng/image8.png} \includegraphics[width=0.75486in,height=0.23958in]{mediaBpng/image9.png}
bolsa, onda bul izbe-izlik úlken sanlar nızamına boysınama?
 \\
\textbf{B3.} \includegraphics[width=0.15972in,height=0.23958in]{mediaBpng/image49.png} úzliksiz tosınnanlıq shamanıń tıǵızlıq funkciyaları berilgen. Olarǵa sáykes \includegraphics[width=0.15972in,height=0.19653in]{mediaBpng/image50.png} tosınnanlıq shamanıń \includegraphics[width=0.50278in,height=0.30069in]{mediaBpng/image51.png} tıǵızlıq funkciyasın tabıń. \includegraphics[width=2.62569in,height=0.84028in]{mediaBpng/image56.png} \includegraphics[width=0.85903in,height=0.23958in]{mediaBpng/image57.png}
 \\
\textbf{C1.} Eger ǵárezsiz \includegraphics[width=0.15208in,height=0.24028in]{mediaCpng/image4.png} hám \includegraphics[width=0.15208in,height=0.19167in]{mediaCpng/image5.png} úzliksiz tosınnanlıq shamalar sáykes túrde, \includegraphics[width=0.41597in,height=0.24028in]{mediaCpng/image23.png} hám \includegraphics[width=0.4in,height=0.24028in]{mediaCpng/image18.png} aralıqlarda teń ólshemli bólistirilgen bolsa, onda \includegraphics[width=0.44028in,height=0.24028in]{mediaCpng/image14.png} tosınnanlıq shamanıń tıǵızlıq funkciyasın tabıń.
 \\
\textbf{C2.} Eger \(\left( \xi_{1},\xi_{2} \right)\) tosınnanlıq vektordıń bólistiriw funkciyası \(F(x,y) = \left\{ \begin{matrix}
\left( 1 - 2^{- x^{2}} \right)\left( 1 - 2^{- 2y^{2}} \right),\ \ eger\ \ x \geq 0,\ y \geq 0, \\
 \\
 \\
\ \ \ \ \ \ \ \ \ \ \ \ \ \ 0,\ \ \ \ \ \ \ \ \ \ \ \ \ \ \ \ \ \ \ \ \ \ \ basqa\ hallarda
\end{matrix} \right.\ \) bolsa, onda \(F\left( x/\xi_{2} < y \right)\) hám \(F\left( y/\xi_{1} < x \right)\) shártli bólistiriw funkciyaların tabıń. Sonıń menen birge, \(\xi_{1}\) hám \(\xi_{2}\) tosınnanlıq shamalardı ǵárezsizlike tekseriń.
 \\
\textbf{C3.} Eger ǵárezsiz \includegraphics[width=0.15208in,height=0.24028in]{mediaCpng/image4.png} hám \includegraphics[width=0.15208in,height=0.19167in]{mediaCpng/image5.png} úzliksiz tosınnanlıq shamalardıń hárbiri {[}0,1{]} aralıqta teń ólshemli bólistirilgen bolsa, \includegraphics[width=0.47986in,height=0.52014in]{mediaCpng/image40.png} tosınnanlıq shamanıń tıǵızlıq funkciyasın tabıń.
 \\

\end{tabular}
\vspace{1cm}


\begin{tabular}{m{17cm}}
\textbf{38-variant}
\newline

\textbf{T1.} Bernulli sxemаsı ushın limit teоremаlаr (Puasson bólistiriliwi, qásiyetleri).
 \\
\textbf{T2.} Kompoziciyalıq formulalar \\
\textbf{A1.} $\left( 0,2 \right)$ intervaldan tosınnan $x$ hám $y$ noqat tańlanadı. Olar ushın $xy\le 1$ hám $\frac{y}{x}\le 2$ bolıw itimallıǵın tabıń.
 \\
\textbf{A2.} Studentler úsh jıl dawamında matematikalıq kitaplardan hárbirinde 30 máseleni óz ishine alǵan matematikadan 25 tipikalıq esaplardı orınlaydı. Kompyuterde matematikalıq paket járdeminde máseleni nadurıs sheshiwi itimallıǵı 0,01 ge, paket járdemisiz 0,2 ge teń. Úsh jıl ishinde tómendegi waqıyalardıń júzege asıwı itimallıqların tabıń: a) matematikalıq paketten turaqlı túrde paydalanatuǵın student 4 ten kóp bolmaǵan máselelerdi nadurıs sheshken bolsa; b) matematikalıq paketten paydalanbaytuǵın student 120 dan 180 ge shekem máseleni nadurıs sheshedi.
 \\
\textbf{A3.} Hárbiriniń júzege asıw itimallıǵı $p$ ǵa teń bolǵan 10 dana Bernulli tájiriybesi ótkerilgende, tómendegi waqıyalardıń itimallıqların tabıń: Tájiriybelerdiń birinshi yarımındaǵı sátliler sanı, tájiriybelerdiń ekinshi yarımındaǵı sátliler sanınan artıq.
 \\
\textbf{B1.} \(P\left\{ \xi = m \right\} = pq^{m},\ \ \ \ \ \ \ \ m = 0,\ \ 1,\ \ 2,\ \ \ldots\). Usı tosınnanlı $\xi$ shamasınıń matematikalıq kútiliwin tabıń.
 \\
\textbf{B2.} Barlıq tárepi boyalǵan kub mıń dana birdey ólshemdegi kubiklerge bólingen hám aralastırıp jiberilgen. Tosınnan alınǵan kubiktiń a) bir tárepi; b) eki tárepi; c) úsh tárepi boyalǵan bolıwı itimallıǵın tabıń.
 \\
\textbf{B3.} Eger \includegraphics[width=0.36181in,height=0.29444in]{mediaBpng/image1.png} ǵárezsiz tosınnanlıq shamalar izbe-izliginiń bólistiriliw nızamları
\includegraphics[width=2.66875in,height=0.50278in]{mediaBpng/image14.png} \includegraphics[width=1.53958in,height=0.50278in]{mediaBpng/image15.png} \includegraphics[width=0.75486in,height=0.23958in]{mediaBpng/image12.png}
bolsa, onda bul izbe-izlik úlken sanlar nızamına boysınama?
 \\
\textbf{C1.} Eger \(\xi_{1}\) hám \(\xi_{2}\) ǵárezsiz tosınnanlıq shamalardıń hárbiri \(\lbrack 0,1\rbrack\) aralıqta teń ólshemli bólistirilgen bolsa, onda \(\xi_{1} + \xi_{2}\) tosınnanlıq shamanıń tıǵızlıq funkciyasın tabıń.
 \\
\textbf{C2.} Tómende \includegraphics[width=0.46389in,height=0.25625in]{mediaCpng/image42.png} úzliksiz tosınnanlıq vektorlardıń tıǵızlıq funkciyaları berilgen. Olardıń \includegraphics[width=0.48819in,height=0.29583in]{mediaCpng/image43.png} hám \includegraphics[width=0.50417in,height=0.29583in]{mediaCpng/image44.png} marginal tıǵızlıq funkciyaların tabıń; \includegraphics[width=0.15972in,height=0.24028in]{mediaCpng/image45.png} hám \includegraphics[width=0.15972in,height=0.2in]{mediaCpng/image46.png} tosınnanlıq shamalardı ǵárezsizlikke tekseriń: \includegraphics[width=3.32014in,height=0.84792in]{mediaCpng/image68.png}
 \\
\textbf{C3.} 
\(\xi\) diskret tosınnanlıq shama \(x_{i} = ( - 1)^{i}i\) mánislerdi \(p_{i} = \frac{1}{i(i + 1)},\) \(\ \ i = 1,\ 2,\ ...\) itimallıqlar menen qabıl etse, onıń matematikalıq kútiliwin tabıń.
 \\

\end{tabular}
\vspace{1cm}


\begin{tabular}{m{17cm}}
\textbf{39-variant}
\newline

\textbf{T1.} Tolıq itimallıq formulası (waqıyalardıń tolıq gruppası, dálilleniwi).
 \\
\textbf{T2.} Oraylıq limit teorema (anıqlaması, ǵárezsiz birdey bólistirilgen tosınnanlı shamalar ushın).
 \\
\textbf{A1.} Shegaralıq bahalaw jumısın tapsırıwǵa kelgen 10 studentten ibarat toparda úshewi ayrıqsha, tórtewi jaqsı, ekewi qanaatlandırarlı hám birewi qanaatlandırarsız tayarlanǵan. Shegaralıq bahalaw jumısınıń variantlarında 20 dana soraw bar. Ayrıqsha tayarlanǵan student barlıq 20 sorawǵa, jaqsı tayarlanǵanı 16 sorawǵa, qanaatlandırarlı tayarlanǵanı 10 sorawǵa, qanaatlandırarsız tayarlanǵanı 5 sorawǵa juwap bere aladı. a) Bul studentlerden qálegen birewi berilgen bir sorawǵa durıs juwap beriw itimallıǵın tabıń. b) Sol durıs juwap bergen studenttiń ayrıqsha tayarlanǵan student bolıwı itimallıǵın tabıń.
 \\
\textbf{A2.} 100 dana buyımnan ibarat partiyada 4 dana buyım jaramsız. Partiyadan tosınnan 15 dana buyım alınadı. Usı alınǵan 15 dana buyımnıń ishinde 2 dana buyımnıń jaramsız bolıw itimallıǵın tabıń.
 \\
\textbf{A3.} Hárbiriniń júzege asıw itimallıǵı $p$ ǵa teń bolǵan 10 dana Bernulli tájiriybesi ótkerilgende, tómendegi waqıyalardıń itimallıqların tabıń: Sátliler sanı, sátsizler sanınan kem.
 \\
\textbf{B1.} $\xi$ tosınnanlı shamanıń \emph{f}(\emph{x}) tıǵızlıq funkciyasi berilgen bolsin. Tómendegilerdi esaplań: a) C; b) \emph{F}(\emph{x}); c) M$\xi$; d) D$\xi$; e) \emph{f}(\emph{x}) hám \emph{F}(\emph{x}) grafiklarin sızıń.\(f(x) = \left\{ \begin{matrix}
\ \ \ \ \ \ \ \ 0,\ \ \ \ \ \ x \leq 0, \\
Cxe^{- 0.5x},\ \ \ \ \ x > 0.\ \ 
\end{matrix} \right.\ \)
 \\
\textbf{B2.} \includegraphics[width=0.15972in,height=0.23958in]{mediaBpng/image49.png} úzliksiz tosınnanlıq shamanıń tıǵızlıq funkciyaları berilgen. Olarǵa sáykes \includegraphics[width=0.15972in,height=0.19653in]{mediaBpng/image50.png} tosınnanlıq shamanıń \includegraphics[width=0.50278in,height=0.30069in]{mediaBpng/image51.png} tıǵızlıq funkciyasın tabıń. \includegraphics[width=2.44167in,height=0.63194in]{mediaBpng/image98.png} \includegraphics[width=0.57639in,height=0.28194in]{mediaBpng/image99.png} \\
\textbf{B3.} $\xi$ tosınnanlı shamanıń \emph{f}(\emph{x}) tıǵızlıq funkciyasi berilgen bolsin. Tómendegilerdi esaplań: a) C; b) \emph{F}(\emph{x}); c) M$\xi$; d) D$\xi$; e) \emph{f}(\emph{x}) hám \emph{F}(\emph{x}) grafiklarin sızıń.\(f(x) = \left\{ \begin{matrix}
\ \ \ \ \ \ \ \ 0,\ \ \ \ \ \ \ \ \ \ \ \ x \leq 0, \\
C/(x + 1)^{4},\ \ \ \ \ x > 0.\ \ 
\end{matrix} \right.\ \)
 \\
\textbf{C1.} Eger \(\left\{ \xi_{n} \right\}\) tosınnanlıq shamalar izbe-izligi \(\mathbf{\xi}_{\mathbf{n}}\overset{\mathbf{P}}{\rightarrow}\mathbf{\xi}\) hám \(\mathbf{\xi}_{\mathbf{n}}\overset{\mathbf{P}}{\rightarrow}\mathbf{\eta}\) bolsa, onda \(\mathbf{P}\left( \mathbf{\xi = \eta} \right)\mathbf{=}\mathbf{1}\) qatnasın dálilleń.
 \\
\textbf{C2.} Eger \(\xi\) tosınnanlıq shama standart Koshi bólistiriliwine iye bolsa, onda \(M\min\left( |\xi|,1 \right)\) mánisin tabıń.
 \\
\textbf{C3.} Eger \(\mathbf{\xi}_{\mathbf{n}}\overset{\mathbf{L}^{\mathbf{2}}}{\rightarrow}\mathbf{\xi}\) bolsa, onda \(n \rightarrow \infty\) de \(\mathbf{M}\mathbf{\xi}_{\mathbf{n}}\mathbf{\rightarrow M\xi}\) ekenligin kórsetiń.
 \\

\end{tabular}
\vspace{1cm}


\begin{tabular}{m{17cm}}
\textbf{40-variant}
\newline

\textbf{T1.} Itimallıq anıqlamaları (klassikalıq, geometriyalıq anıqlamaları).
 \\
\textbf{T2.} Tıǵızlıq funkciyası (anıqlaması, tiykarǵıqásiyetleri).
 \\
\textbf{A1.} $\left[ 0,2 \right]$ kesindiden tosınnan $x$ hám $y$ noqat tańlanadı. Olar ushın ${{x}^{2}}\le 4y\le 4x$ bolıw itimallıǵın tabıń.
 \\
\textbf{A2.} Oyın kubigi taslanǵanda 5 ochkonıń túsiw itimallıǵın tabıń.
 \\
\textbf{A3.} Lotereyada úlken hám kishi utıslar oynaladı. Lotereya biletinde úlken utıs shıǵıw itimallıǵı 0,001 ge, al kishisi bolsa 0,01 ge teń. Jámi 1000 dana bilet satıp alınǵanda: a) eki úlken utıslı; b) kishi utıslar 5 ten 15 ke shekem bolıwı waqıyaları itimallıqların tabıń.
 \\
\textbf{B1.} Eger \includegraphics[width=0.36181in,height=0.29444in]{mediaBpng/image1.png} ǵárezsiz tosınnanlıq shamalar izbe-izliginiń bólistiriliw nızamları
\includegraphics[width=2.59514in,height=0.50278in]{mediaBpng/image13.png} \includegraphics[width=1.50278in,height=0.50278in]{mediaBpng/image11.png} \includegraphics[width=0.75486in,height=0.23958in]{mediaBpng/image12.png}
bolsa, onda bul izbe-izlik úlken sanlar nızamına boysınama?
 \\
\textbf{B2.} Tosınnanlı \(\xi\) shamasınıń bólistiriw tıǵızlıǵı \(f(x) = \frac{1}{2}e^{- |x|}\) bolsa, usı shamanıń xarakteristikalıq funkciyasın tabıń.
 \\
\textbf{B3.} Fakultette 1460 student bar. Keminde 10 studenttiń tuwılǵan kúni 5 sentyabrge tuwra kelip qalıwı waqıyası itimallıǵı tabılsın.
 \\
\textbf{C1.} Eger ǵárezsiz \includegraphics[width=0.15208in,height=0.24028in]{mediaCpng/image4.png} hám \includegraphics[width=0.15208in,height=0.19167in]{mediaCpng/image5.png} úzliksiz tosınnanlıq shamalardıń tıǵızlıq fukciyaları sáykes túrde,
\includegraphics[width=1.54375in,height=0.47986in]{mediaCpng/image36.png} hám \includegraphics[width=1.58403in,height=0.47986in]{mediaCpng/image37.png}
bolsa, onda \includegraphics[width=0.44028in,height=0.24028in]{mediaCpng/image14.png} tosınnanlıq shamanıń tıǵızlıq funkciyasın tabıń.
 \\
\textbf{C2.} Tómende \includegraphics[width=0.46389in,height=0.25625in]{mediaCpng/image42.png} úzliksiz tosınnanlıq vektorlardıń tıǵızlıq funkciyaları berilgen. Olardıń \includegraphics[width=0.48819in,height=0.29583in]{mediaCpng/image43.png} hám \includegraphics[width=0.50417in,height=0.29583in]{mediaCpng/image44.png} marginal tıǵızlıq funkciyaların tabıń; \includegraphics[width=0.15972in,height=0.24028in]{mediaCpng/image45.png} hám \includegraphics[width=0.15972in,height=0.2in]{mediaCpng/image46.png} tosınnanlıq shamalardı ǵárezsizlikke tekseriń: \includegraphics[width=2.52778in,height=0.59167in]{mediaCpng/image55.png}
 \\
\textbf{C3.} Eger \(\xi\) hám \(\chi^{2}\) ǵárezsiz tosınnanlıq shamalar bolıp, \(\xi\sim N(0,1)\) hám \(\chi^{2}\sim\chi^{2}(n)\) bolsa, onda \(\frac{\xi}{\sqrt{\frac{\chi^{2}}{n}}}\) tosınnanlıq shamanıń tıǵızlıq funkciyasın tabıń.
 \\

\end{tabular}
\vspace{1cm}


\begin{tabular}{m{17cm}}
\textbf{41-variant}
\newline

\textbf{T1.} Shártli itimallıq (anıqlaması, kóbеytiw tеorеması).
 \\
\textbf{T2.} Tosınnanlı shamanıń joqarı tártipli momentleri (baslanǵısh hám oraylıq momentleri, qásiyetleri).
 \\
\textbf{A1.} 36 dana kartalar kolodasınan tosınnan alınǵan 3 dana kartanıń barlıǵı birdey reńde bolıw itimallıǵın tabıń.
 \\
\textbf{A2.} $\left[ 0,2 \right]$ kesindiden tosınnan eki noqat tańlanadı. Olardıń koordinataları qosındısı 2 den úlken bolıw hám kvadratları qosındısı 4 ten kishi bolıw itimallıǵın tabıń.
 \\
\textbf{A3.} Birinshi qutıda 3 dana aq hám 5 dana qara sharlar bar, ekinshi qutıda 6 dana aq hám 8 dana qara sharlar bar. Birinshi qutıdan tosınnan 2 shar alınıp, ekinshi qutıǵa salındı. Keyin, birinshi qutıdan tosınnan 1 shar alındı. а) Bul shardıń aq bolıw itimallıǵın tabıń. b) Sol shar aq bolsa, birinshi qutıdan alınǵan sharlardıń aq bolıw itimallıǵın tabıń.
 \\
\textbf{B1.} Eger \includegraphics[width=0.36181in,height=0.29444in]{mediaBpng/image1.png} ǵárezsiz tosınnanlıq shamalar izbe-izliginiń bólistiriliw nızamları
\includegraphics[width=2.33125in,height=0.49097in]{mediaBpng/image28.png} \includegraphics[width=1.59514in,height=0.47847in]{mediaBpng/image29.png} \includegraphics[width=0.75486in,height=0.23958in]{mediaBpng/image30.png}
bolsa, onda bul izbe-izlik úlken sanlar nızamına boysınama?
 \\
\textbf{B2.} $\xi$ tosınnanlı shamanıń \emph{f}(\emph{x}) tıǵızlıq funkciyasi berilgen bolsin. Tómendegilerdi esaplań: a) C; b) \emph{F}(\emph{x}); c) M$\xi$; d) D$\xi$; e) \emph{f}(\emph{x}) hám \emph{F}(\emph{x}) grafiklarin sızıń.\(f(x) = \left\{ \begin{matrix}
\ \ \ \ \ \ \ \ 0,\ \ \ \ \ \ x < 1, \\
Ce^{1 - x},\ \ \ \ \ x \geq 1.\ \ 
\end{matrix} \right.\ \)
 \\
\textbf{B3.} Eger \(\xi\) tosınnanlı shama \((n,p)\) parametrli binomial bólistiriwine iye bolsa, onda onıń xarakteristikalıq funkciyası tabılsın.
 \\
\textbf{C1.} Eger \(\chi_{1}^{2}\) hám \(\chi_{2}^{2}\) ǵárezsiz tosınnanlıq shamalar bolıp, \(\chi_{1}^{2}\sim\chi^{2}(n_{1})\) hám \(\chi_{2}^{2}\sim\chi^{2}(n_{2})\) bolsa, onda \(\frac{n_{2}}{n_{1}} \cdot \frac{\chi_{1}^{2}}{\chi_{2}^{2}}\) tosınnanlıq shamanıń tıǵızlıq funkciyasın tabıń.
 \\
\textbf{C2.} Eger \(\xi\sim N\left( a,\sigma^{2} \right)\) bolsa, onda \(\xi\) tosınnanlıq shamanıń joqarı tártipli oraylıq momentlerin tabıń.
 \\
\textbf{C3.} Eger \(\left\{ \xi_{n} \right\}\) diskret tosınnanlıq shamalar izbe-izliginiń bólistiriliw nızamları\(P\left\{ \xi_{n} = 1 \right\} = P\left\{ \xi_{n} = - 1 \right\} = \frac{1}{2} - \frac{1}{n},\) \(P\left\{ \xi_{n} = 0 \right\} = \frac{2}{n}\) bolsa, onda \(\xi_{n}\overset{d}{\rightarrow}\xi\) bolatuǵın \(\xi\) tosınnanlıq shamanıń bólistiriw funkciyasın tabıń.
 \\

\end{tabular}
\vspace{1cm}


\begin{tabular}{m{17cm}}
\textbf{42-variant}
\newline

\textbf{T1.} Waqıyalar algebrası ($\sigma$-algebra, minimal $\sigma$-algebra).
 \\
\textbf{T2.} Tiykarǵı diskret bólistiriliwler (Binomial, Puasson hám geometriyalıq bólistiriliwler).
 \\
\textbf{A1.} 
“Sportlotto” lotereyasında úlken hám kishi utıslar oynaladı. Lotereya biletinde úlken utıs shıǵıw itimallıǵı 0,009 ǵa, al kishisi bolsa 0,02 ge teń. Jámi 500 dana bilet satıp alınǵanda: a) úlken utıslar 5 ten 10 ǵa shekem bolıw; b) 15 ten 20 ǵa shekem kishi utıslar bolıwı waqıyaları itimallıqların tabıń.
 \\
\textbf{A2.} Hárbiriniń júzege asıw itimallıǵı $p$ ǵa teń bolǵan 10 dana Bernulli tájiriybesi ótkerilgende, tómendegi waqıyalardıń itimallıqların tabıń: Sátliler sanı, sátsizler sanınan keminde 2 ese artıq.
 \\
\textbf{A3.} Eki oyın kubigi taslanǵanda túsken ochkolardıń biri ekinshisinen 2 ese kóp bolıw itimallıǵın tabıń.
 \\
\textbf{B1.} \includegraphics[width=0.15972in,height=0.23958in]{mediaBpng/image49.png} úzliksiz tosınnanlıq shamanıń tıǵızlıq funkciyaları berilgen. Olarǵa sáykes \includegraphics[width=0.15972in,height=0.19653in]{mediaBpng/image50.png} tosınnanlıq shamanıń \includegraphics[width=0.50278in,height=0.30069in]{mediaBpng/image51.png} tıǵızlıq funkciyasın tabıń. \includegraphics[width=2.48472in,height=1.15347in]{mediaBpng/image80.png} \includegraphics[width=0.73611in,height=0.23958in]{mediaBpng/image81.png}
 \\
\textbf{B2.} Birdey kartochkalarǵa jazilǵan A,A,A,A,P,R,Q,Q,Q,L háriplerinen tosinnan alinǵan kartochkalardi aliniw tártibinde jaylastiriwdan «QARAQALPAQ» sóziniń kelip shiǵiw itimalliǵin tabiń
 \\
\textbf{B3.} \includegraphics[width=0.15972in,height=0.23958in]{mediaBpng/image49.png} úzliksiz tosınnanlıq shamanıń tıǵızlıq funkciyaları berilgen. Olarǵa sáykes \includegraphics[width=0.15972in,height=0.19653in]{mediaBpng/image50.png} tosınnanlıq shamanıń \includegraphics[width=0.50278in,height=0.30069in]{mediaBpng/image51.png} tıǵızlıq funkciyasın tabıń. \includegraphics[width=2.15972in,height=0.58889in]{mediaBpng/image90.png} \includegraphics[width=0.71806in,height=0.49097in]{mediaBpng/image91.png}
 \\
\textbf{C1.} Eger ǵárezsiz \includegraphics[width=0.15208in,height=0.24028in]{mediaCpng/image4.png} hám \includegraphics[width=0.15208in,height=0.19167in]{mediaCpng/image5.png} úzliksiz tosınnanlıq shamalardıń hárbiri \includegraphics[width=0.44028in,height=0.21597in]{mediaCpng/image28.png} parametrli kórsetkishli nızam boyınsha bólistirilgen bolsa, onda \includegraphics[width=0.50417in,height=0.29583in]{mediaCpng/image30.png} tosınnanlıq shamanıń tıǵızlıq funkciyasın tabıń.
 \\
\textbf{C2.} Tómende \includegraphics[width=0.46389in,height=0.25625in]{mediaCpng/image42.png} úzliksiz tosınnanlıq vektorlardıń tıǵızlıq funkciyaları berilgen. Olardıń \includegraphics[width=0.48819in,height=0.29583in]{mediaCpng/image43.png} hám \includegraphics[width=0.50417in,height=0.29583in]{mediaCpng/image44.png} marginal tıǵızlıq funkciyaların tabıń; \includegraphics[width=0.15972in,height=0.24028in]{mediaCpng/image45.png} hám \includegraphics[width=0.15972in,height=0.2in]{mediaCpng/image46.png} tosınnanlıq shamalardı ǵárezsizlikke tekseriń: \includegraphics[width=3.16806in,height=0.75972in]{mediaCpng/image66.png}
 \\
\textbf{C3.} Eger \(\xi_{1},...,\xi_{n}\) ǵárezsiz birdey bólistirilgen tosınnanlıq shamalar \(F(x)\) bólistiriw hám \(f(x)\) tıǵızlıq funkciyalarǵa iye bolsa, onda \(\eta_{1} = \max\left( \xi_{1},...,\xi_{n} \right)\) hám \(\eta_{2} = \min\left( \xi_{1},...,\xi_{n} \right)\) tosınnanlıq shamalardıń bólistiriw hám tıǵızlıq funkciyaların tabıń.
 \\

\end{tabular}
\vspace{1cm}


\begin{tabular}{m{17cm}}
\textbf{43-variant}
\newline

\textbf{T1.} Ǵárezsiz tájiriybelerdiń Bernulli sxeması (binоmiаl bólistiriliw, qásiyetleri).
 \\
\textbf{T2.} Tosınnanlı shamanıń matematikalıq kútiliwi. (anıqlaması, qásiyetleri)
 \\
\textbf{A1.} Úsh oyın kubigi taslanǵanda túsken ochkolardıń qosındısı 16 dan artıq bolmaw itimallıǵın tabıń.
 \\
\textbf{A2.} Hárbiriniń júzege asıw itimallıǵı $p$ ǵa teń bolǵan 10 dana Bernulli tájiriybesi ótkerilgende, tómendegi waqıyalardıń itimallıqların tabıń: Tájiriybelerdiń birinshi yarımındaǵı sátliler sanı, tájiriybelerdiń ekinshi yarımındaǵı sátliler sanınan eki esege artıq.
 \\
\textbf{A3.} $\left[ 0,1 \right]$ kesindiden tosınnan eki noqat  tańlanadı. Olardıń koordinataları kvadratları qosındısı 1 den úlken bolıw itimallıǵın tabıń.
 \\
\textbf{B1.} Chebıshev teńsizliginiń járdemi menen normal tosınnanlı shamanıń óziniń matematikalıq kútiliwinen, awısıwınıń úsh orta kvadratlıq awısıwdan úlken bolıwınıń itimallıǵın bahalań.
 \\
\textbf{B2.} Úsh mеrgеn bir-birinеn ǵárеzsiz nıshanaǵa bir márteden oq attı. Birinshi mеrgеnniń nıshanaǵa tiygiziw itimallıǵı 0,6 ǵa, еkinshisiniki 0,8 gе, úshinshisiniki bolsa 0,3 ke tеń. Atıw tamam bolǵannan kеyin nıshanada eki oq izi tabılǵan bolsa, birinshi mеrgеn nıshanaǵa tiygiziwi waqıyası itimallıǵı tabılsın.
 \\
\textbf{B3.} $\xi$ tosınnanlı shamanıń \emph{f}(\emph{x}) tıǵızlıq funkciyasi berilgen bolsin. Tómendegilerdi esaplań: a) C; b) \emph{F}(\emph{x}); c) M$\xi$; d) D$\xi$; e) \emph{f}(\emph{x}) hám \emph{F}(\emph{x}) grafiklarin sızıń.\(f(x) = \left\{ \begin{matrix}
C\cos x,\ \ \ \ x \in \left\lbrack 0,\frac{\pi}{2} \right\rbrack, \\
\ \ \ \ \ \ \ \ 0,\ \ \ \ \ \ x \notin \left\lbrack 0,\frac{\pi}{2} \right\rbrack.\ \ 
\end{matrix} \right.\ \)
 \\
\textbf{C1.} Ortasha mánis vektorı \(\left( m_{1},m_{2} \right)\) hám kovariaciyalıq matricası\(K = \begin{pmatrix}
\sigma_{1}^{2} & r\sigma_{1}\sigma_{2} \\
r\sigma_{1}\sigma_{2} & \sigma_{2}^{2}
\end{pmatrix},\ \ \sigma_{1},\ \sigma_{2} > 0,\ \ |r|\  < 1\) bolǵan normal bólistirilgen \(\left( \xi_{1},\xi_{2} \right)\) tosınnanlıq vektordıń tıǵızlıq funkciyasın tabıń.
 \\
\textbf{C2.} Eger \(\mathbf{\xi}_{\mathbf{n}}\overset{\mathbf{L}^{\mathbf{2}}}{\rightarrow}\mathbf{\xi}\) bolsa, onda \(n \rightarrow \infty\) de \(\mathbf{M}\mathbf{\xi}_{\mathbf{n}}\mathbf{\rightarrow M\xi}\) ekenligin kórsetiń.
 \\
\textbf{C3.} Eger ǵárezsiz \includegraphics[width=0.15208in,height=0.24028in]{mediaCpng/image4.png} hám \includegraphics[width=0.15208in,height=0.19167in]{mediaCpng/image5.png} úzliksiz tosınnanlıq shamalardıń hárbiri \includegraphics[width=0.19167in,height=0.16806in]{mediaCpng/image31.png} parametrli kórsetkishli nızam boyınsha bólistirilgen bolsa, onda \includegraphics[width=0.44028in,height=0.24028in]{mediaCpng/image14.png} tosınnanlıq shamanıń tıǵızlıq funkciyasın tabıń.
 \\

\end{tabular}
\vspace{1cm}


\begin{tabular}{m{17cm}}
\textbf{44-variant}
\newline

\textbf{T1.} Bayеs formulası (gipotezalar teoreması, dálilleniwi).
 \\
\textbf{T2.} Bólistiriw funkciyası (anıqlaması, tiykarǵı qásiyetleri).
 \\
\textbf{A1.} Bazıbir qalada solaqaylar ortasha esapta $1$, sol hám oń qollarına teńdey iyelik qılatuǵın adamlar $10$, al qalǵanları ońaqaylar. Jámi 200 adam arasında tómendegi waqıyalardıń júzege asıwı itimallıqların tabıń: a) keminde tórt solaqay boladı; b) 18 den 23 ke shekem sol hám oń qollarına teńdey iyelik qılatuǵın adamlar boladı.
 \\
\textbf{A2.} Berilgen $1,2,\ldots ,10$ sanlarınıń arasınan tosınnan eki san tańlandı. Meyli, bul sanlar ${{m}_{1}}$ hám ${{m}_{2}}$ (${{m}_{1}}<{{m}_{2}}$) bolsın. Soń, ${{m}_{1}},{{m}_{1}}+1,\ldots ,{{m}_{2}}$ sanları arasınan tosınnan bir san tańlandı. a) Bul sannıń 9 ǵa teń bolıw itimallıǵın tabıń. b) Bul san 9 ǵa teń bolsa, ${{m}_{2}}=10$ bolıw itimallıǵın tabıń.
 \\
\textbf{A3.} 28 dana dominonıń tolıq komplektinen 7 danası tosınnan tańlanadı. Olardıń ishinde keminde eki birdey ochko bolıw itimallıǵın tabıń.
 \\
\textbf{B1.} Eger \includegraphics[width=0.36181in,height=0.29444in]{mediaBpng/image1.png} ǵárezsiz tosınnanlıq shamalar izbe-izliginiń bólistiriliw nızamları
\includegraphics[width=2.59514in,height=0.50278in]{mediaBpng/image13.png} \includegraphics[width=1.50278in,height=0.50278in]{mediaBpng/image11.png} \includegraphics[width=0.75486in,height=0.23958in]{mediaBpng/image12.png}
bolsa, onda bul izbe-izlik úlken sanlar nızamına boysınama?
 \\
\textbf{B2.} Birdey kartochkalarǵa A,A,A,E,I,M,M,K,T,T háripleri jazilǵan hám jaqsilap aralastirip tóńkerip jayilǵan. Izbe-iz alinǵan kartochkalardi alinǵan tártibinde jaylastiriw nátiyjesinde «MATEMATIKA» sóziniń kelip shiǵiw itimalliǵin tabiń.
 \\
\textbf{B3.} Eger \includegraphics[width=0.36181in,height=0.29444in]{mediaBpng/image1.png} ǵárezsiz tosınnanlıq shamalar izbe-izliginiń bólistiriliw nızamları
\includegraphics[width=1.51528in,height=0.50278in]{mediaBpng/image19.png} \includegraphics[width=1.62569in,height=0.50278in]{mediaBpng/image20.png} \includegraphics[width=0.75486in,height=0.23958in]{mediaBpng/image9.png}
bolsa, onda bul izbe-izlik úlken sanlar nızamına boysınama?
 \\
\textbf{C1.} Tómende \includegraphics[width=0.46389in,height=0.25625in]{mediaCpng/image42.png} úzliksiz tosınnanlıq vektorlardıń tıǵızlıq funkciyaları berilgen. Olardıń \includegraphics[width=0.48819in,height=0.29583in]{mediaCpng/image43.png} hám \includegraphics[width=0.50417in,height=0.29583in]{mediaCpng/image44.png} marginal tıǵızlıq funkciyaların tabıń; \includegraphics[width=0.15972in,height=0.24028in]{mediaCpng/image45.png} hám \includegraphics[width=0.15972in,height=0.2in]{mediaCpng/image46.png} tosınnanlıq shamalardı ǵárezsizlikke tekseriń: \includegraphics[width=3.24028in,height=0.67986in]{mediaCpng/image61.png}
 \\
\textbf{C2.} Eger ǵárezsiz \includegraphics[width=0.15208in,height=0.24028in]{mediaCpng/image4.png} hám \includegraphics[width=0.15208in,height=0.19167in]{mediaCpng/image5.png} úzliksiz tosınnanlıq shamalardıń hárbiri \includegraphics[width=0.19167in,height=0.16806in]{mediaCpng/image31.png} parametrli kórsetkishli nızam boyınsha bólistirilgen bolsa, \includegraphics[width=0.47986in,height=0.52014in]{mediaCpng/image41.png} tosınnanlıq shamanıń tıǵızlıq funkciyasın tabıń.
 \\
\textbf{C3.} 
Eger \(\xi\) tosınnanlıq shama hám \(\left\{ \xi_{n} \right\}\) tosınnanlıq shamalar izbe-izligi ǵárezsiz birdey standart normal bólistirilgen bolsa, onda \(\left\{ \mathbf{\eta}_{\mathbf{n}} \right\}\mathbf{=}\left\{ \frac{\mathbf{\xi}\sqrt{\mathbf{n}}}{\sqrt{\mathbf{\xi}_{\mathbf{1}}^{\mathbf{2}}\mathbf{+}\mathbf{...}\mathbf{+}\mathbf{\xi}_{\mathbf{n}}^{\mathbf{2}}}} \right\}\) tosınnanlıq shamalar izbe-izligining limit bólistiriw funkciyası standart normal bólistiriliw bolıwın kórsetiń.
 \\

\end{tabular}
\vspace{1cm}


\begin{tabular}{m{17cm}}
\textbf{45-variant}
\newline

\textbf{T1.} Ǵárezsiz tájiriybelerdiń Bernulli sxeması (binоmiаl bólistiriliw, qásiyetleri).
 \\
\textbf{T2.} Tosınnanlı shamanıń joqarı tártipli momentleri (baslanǵısh hám oraylıq momentleri, qásiyetleri).
 \\
\textbf{A1.} Jip iyiriw fabrikasında 1000 dana jańa hám 200 dana eski jip iyiriw qurılmaları bar. Bir jumıs kúninde jańa qurılma 0,003 itimallıq penen, al eski qurılma bolsa 0,20 itimallıq penen iyirilip atırǵan jipti úzip aladı. Bir jumıs kúninde tómendegi waqıyalardıń júzege asıwı itimallıqların tabıń: a) jańa qurılma 3 márte jipti úzip alıwı; b) eski qurılma 10 nan 15 ke shekemgi jipti úzip alıwı. 
 \\
\textbf{A2.} Hárbiriniń júzege asıw itimallıǵı $p$ ǵa teń bolǵan 10 dana Bernulli tájiriybesi ótkerilgende, tómendegi waqıyalardıń itimallıqların tabıń: Sátsizler sanı 4 dana.
 \\
\textbf{A3.} $\left[ 0,1 \right]$ kesindiden tosınnan eki noqat tańlanadı. Birinshi hám ekinshi noqatlardıń koordinataları kvadratlarınıń ayırması $0,25$ ten úlken bolıw itimallıǵın tabıń.
 \\
\textbf{B1.} Eger \(\xi\) tosınnanlı shama \(\lambda\) parametrli puasson bólistiriwine iye bolsa, onda onıń xarakteristikalıq funkciyası tabılsın.
 \\
\textbf{B2.} $\xi$ tosınnanlı shamanıń \emph{f}(\emph{x}) tıǵızlıq funkciyasi berilgen bolsin. Tómendegilerdi esaplań: a) C; b) \emph{F}(\emph{x}); c) M$\xi$; d) D$\xi$; e) \emph{f}(\emph{x}) hám \emph{F}(\emph{x}) grafiklarin sızıń.\(f(x) = \left\{ \begin{matrix}
C/(1 + x^{2}),\ \ \ \ x \in \lbrack 0,\sqrt{3}\rbrack, \\
\ \ \ \ \ \ \ \ 0,\ \ \ \ \ \ \ \ \ \ \ x \notin \lbrack 0,\sqrt{3}\rbrack.\ \ 
\end{matrix} \right.\ \)
 \\
\textbf{B3.} \includegraphics[width=0.15972in,height=0.23958in]{mediaBpng/image49.png} úzliksiz tosınnanlıq shamanıń tıǵızlıq funkciyaları berilgen. Olarǵa sáykes \includegraphics[width=0.15972in,height=0.19653in]{mediaBpng/image50.png} tosınnanlıq shamanıń \includegraphics[width=0.50278in,height=0.30069in]{mediaBpng/image51.png} tıǵızlıq funkciyasın tabıń. \includegraphics[width=2.12292in,height=0.58264in]{mediaBpng/image73.png} \includegraphics[width=0.91389in,height=0.50278in]{mediaBpng/image74.png}
 \\
\textbf{C1.} Eger \(\left( \xi_{1},\xi_{2} \right)\) tosınnanlıq vеktоrdıń tıǵızlıq funkciyası \(f(x,y) = \frac{1}{3\pi}e^{- \ \ \frac{x^{2} + 4y^{2}}{6}}\)bolsa, onda \(\left( \xi_{1},\xi_{2} \right)\) tosınnanlıq noqattıń \(D = \left\{ (x,y):|x| \leq 1,|y| \leq 2 \right\}\) oblastqa túsiw itimallıǵın tabıń.
 \\
\textbf{C2.} Hárqanday \(\varphi_{\xi}(t)\) xarakteristikalıq funkciya ushın \(t \in R\) de \(1 - Re\varphi_{\xi}(2t) \leq 4\left( 1 - Re\varphi_{\xi}(t) \right)\) ekenligin dálilleń.
 \\
\textbf{C3.} Tómende \includegraphics[width=0.46389in,height=0.25625in]{mediaCpng/image42.png} úzliksiz tosınnanlıq vektorlardıń tıǵızlıq funkciyaları berilgen. Olardıń \includegraphics[width=0.48819in,height=0.29583in]{mediaCpng/image43.png} hám \includegraphics[width=0.50417in,height=0.29583in]{mediaCpng/image44.png} marginal tıǵızlıq funkciyaların tabıń; \includegraphics[width=0.15972in,height=0.24028in]{mediaCpng/image45.png} hám \includegraphics[width=0.15972in,height=0.2in]{mediaCpng/image46.png} tosınnanlıq shamalardı ǵárezsizlikke tekseriń: \includegraphics[width=3.20833in,height=0.59167in]{mediaCpng/image50.png}
 \\

\end{tabular}
\vspace{1cm}


\begin{tabular}{m{17cm}}
\textbf{46-variant}
\newline

\textbf{T1.} Waqıyalar algebrası ($\sigma$-algebra, minimal $\sigma$-algebra).
 \\
\textbf{T2.} Xarakteristikalıq funkciyalar (anıqlaması, tiykarǵı qásiyetleri).
 \\
\textbf{A1.} Birdey úsh qutı berilgen. Birinshi qutıda 3 dana aq hám 2 dana qara sharlar, ekinshi qutıda 2 dana aq hám 4 dana qara sharlar, úshinshi qutıda bolsa 4 dana aq hám 3 dana qara sharlar bar. Tosınnan qutılardan birewi tańlanıp, onıń ishinen bir dana shar alınadı. a) Usı alınǵan shardıń qara shar bolıw itimallıǵın tabıń. b) Alınǵan shar qara shar bolsa, onıń úshinshi qutıdan alınǵan bolıw itimallıǵın tabıń.
 \\
\textbf{A2.} 28 dana dominonıń tolıq komplektinen 7 danası tosınnan tańlanadı. Olardıń ishinde keminde 1 dana 6 ochko bolıw itimallıǵın tabıń.
 \\
\textbf{A3.} Eki oyın kubigi taslanǵanda túsken ochkolardıń taq bolıw itimallıǵın tabıń.
 \\
\textbf{B1.} Eger \includegraphics[width=0.36181in,height=0.29444in]{mediaBpng/image1.png} ǵárezsiz tosınnanlıq shamalar izbe-izliginiń bólistiriliw nızamları
\includegraphics[width=2.51528in,height=0.52153in]{mediaBpng/image40.png} \includegraphics[width=2.59514in,height=0.47847in]{mediaBpng/image41.png} \includegraphics[width=0.75486in,height=0.23958in]{mediaBpng/image42.png}
bolsa, onda bul izbe-izlik úlken sanlar nızamına boysınama?
 \\
\textbf{B2.} \includegraphics[width=0.15972in,height=0.23958in]{mediaBpng/image49.png} úzliksiz tosınnanlıq shamanıń tıǵızlıq funkciyaları berilgen. Olarǵa sáykes \includegraphics[width=0.15972in,height=0.19653in]{mediaBpng/image50.png} tosınnanlıq shamanıń \includegraphics[width=0.50278in,height=0.30069in]{mediaBpng/image51.png} tıǵızlıq funkciyasın tabıń. \includegraphics[width=2.23333in,height=0.675in]{mediaBpng/image94.png} \includegraphics[width=0.55208in,height=0.28194in]{mediaBpng/image95.png}
 \\
\textbf{B3.} \(f(x) = C \cdot e^{- \frac{(x - m)^{2}}{7}}\) tıǵızlıq funkciyası bolıwı ushin \emph{C} nege teń bolıwı kerek?
 \\
\textbf{C1.} Eger \(\left( \xi_{1},\xi_{2} \right)\) absolyut úziliksiz tosınnanlıq vektordıń tıǵızlıq funkciyası \(f(x,y) = \left\{ \begin{matrix}
Ce^{- x - y},\ eger\ \ x \geq 0,y \geq 0, \\
 \\
 \\
\ \ \ \ \ \ \ \ 0,\ \ \ \ \ basqa\ hallarda\ 
\end{matrix} \right.\ \) bolsa, onda \(F(x,y),\) \(F_{\xi_{1}}(x),\) \(F_{\xi_{2}}(y),\) \(f_{\xi_{1}}(x),\) \(f_{\xi_{2}}(y)\) hám \(P\left( \xi_{1} > 0,\xi_{2} < 1 \right)\) itimallıqtı tabıń. Sonıń menen birge, \(\xi_{1}\) hám \(\xi_{2}\) tosınnanlıq shamalardı ǵárezsizlikke tekseriń.
 \\
\textbf{C2.} Eger \(\xi\) tosınnanlıq shama standart Koshi bólistiriliwine iye bolsa, onda \(M\min\left( |\xi|,1 \right)\) mánisin tabıń.
 \\
\textbf{C3.} Tómende \includegraphics[width=0.46389in,height=0.25625in]{mediaCpng/image42.png} úzliksiz tosınnanlıq vektorlardıń tıǵızlıq funkciyaları berilgen. Olardıń \includegraphics[width=0.48819in,height=0.29583in]{mediaCpng/image43.png} hám \includegraphics[width=0.50417in,height=0.29583in]{mediaCpng/image44.png} marginal tıǵızlıq funkciyaların tabıń; \includegraphics[width=0.15972in,height=0.24028in]{mediaCpng/image45.png} hám \includegraphics[width=0.15972in,height=0.2in]{mediaCpng/image46.png} tosınnanlıq shamalardı ǵárezsizlikke tekseriń: \includegraphics[width=2.86389in,height=0.73611in]{mediaCpng/image64.png}
 \\

\end{tabular}
\vspace{1cm}


\begin{tabular}{m{17cm}}
\textbf{47-variant}
\newline

\textbf{T1.} Bernulli sxemаsı ushın limit teоremаlаr (Puasson bólistiriliwi, qásiyetleri).
 \\
\textbf{T2.} Tiykarǵı diskret bólistiriliwler (Binomial, Puasson hám geometriyalıq bólistiriliwler).
 \\
\textbf{A1.} $\left[ 0,3 \right]$ kesindiden tosınnan úsh noqat tańlanadı. Olardıń koordinataları qosındısı 3 ten kishi bolıw itimallıǵın tabıń.
 \\
\textbf{A2.} Jámi 14 bala hám 11 qız bolǵan studentler toparınan 6 student sorawnama ótkeriw ushın tosınnan tańlap alındı. Olar ishinde 4 bala bolıw itimallıǵın tabıń.
 \\
\textbf{A3.} Oqıtıwshı matematika páninen shegaralıq bahalaw alıw ushın 50 soraw tayarlaǵan. Olardıń ishinde differencial esabınan 20 soraw, integral esabınan 18 soraw hám qatarlar teoriyasınan 12 soraw bar. Student differencial esabınan 18 sorawǵa, integral esabınan 15 sorawǵa hám qatarlar teoriyasınan 10 sorawǵa juwap bere aladı. a) Studentke berilgen birinshi sorawǵa juwap beriwi itimallıǵın tabıń. b) Eger student sol sorawǵa durıs juwap bergen bolsa, bul sorawdıń integral esabınan bolıwı itimallıǵın tabıń.
 \\
\textbf{B1.} 
$\xi$ tosınnanlı shamanıń \emph{f}(\emph{x}) tıǵızlıq funkciyasi berilgen bolsin. Tómendegilerdi esaplań: a) C; b) \emph{F}(\emph{x}); c) M$\xi$; d) D$\xi$; e) \emph{f}(\emph{x}) hám \emph{F}(\emph{x}) grafiklarin sızıń.\(f(x) = \left\{ \begin{matrix}
Cx,\ \ \ \ x \in \lbrack 0,1\rbrack, \\
C,\ \ \ \ \ \ \ x \in (1,2\rbrack, \\
0,\ \ \ keri\ jag'dayda.\ \ 
\end{matrix} \right.\ \)
 \\
\textbf{B2.} Abonent telefon nomerin terip atırıp, aqırǵı úsh cifrdi umıtıp qaldı hám bul nomerlerdiń hár túrli ekenligin eslep olardı táwekeline terdi. Kerekli nomerler terilgen bolıwı itimallıǵın tabıń.
 \\
\textbf{B3.} \includegraphics[width=0.15972in,height=0.23958in]{mediaBpng/image49.png} úzliksiz tosınnanlıq shamanıń tıǵızlıq funkciyaları berilgen. Olarǵa sáykes \includegraphics[width=0.15972in,height=0.19653in]{mediaBpng/image50.png} tosınnanlıq shamanıń \includegraphics[width=0.50278in,height=0.30069in]{mediaBpng/image51.png} tıǵızlıq funkciyasın tabıń. \includegraphics[width=1.95069in,height=0.58264in]{mediaBpng/image70.png} \includegraphics[width=0.57639in,height=0.28194in]{mediaBpng/image79.png}
 \\
\textbf{C1.} Eger ǵárezsiz \includegraphics[width=0.15208in,height=0.24028in]{mediaCpng/image4.png} hám \includegraphics[width=0.15208in,height=0.19167in]{mediaCpng/image5.png} úzliksiz tosınnanlıq shamalardıń hárbiri \includegraphics[width=0.19167in,height=0.16806in]{mediaCpng/image31.png} parametrli
kórsetkishli nızam boyınsha bólistirilgen bolsa, onda \includegraphics[width=0.47986in,height=0.52014in]{mediaCpng/image40.png} tosınnanlıq shamanıń
tıǵızlıq funkciyasın tabıń.
 \\
\textbf{C2.} Eger \(\left\{ \xi_{n} \right\}\) diskret tosınnanlıq shamalar izbe-izliginiń bólistiriliw nızamları\(P(\xi_{n} = e^{n}) = \frac{1}{n^{2}},\) \(P(\xi_{n} = 0) = 1 - \frac{1}{n^{2}}\) bolsa, onda \(\left\{ \xi_{n} \right\}\) tosınnanlıq shamalar izbe-izliginiń 0 ge bir itimallıq penen jıynaqlılıǵın kórsetiń.
 \\
\textbf{C3.} Tómende \includegraphics[width=0.46389in,height=0.25625in]{mediaCpng/image42.png} úzliksiz tosınnanlıq vektorlardıń tıǵızlıq funkciyaları berilgen. Olardıń \includegraphics[width=0.48819in,height=0.29583in]{mediaCpng/image43.png} hám \includegraphics[width=0.50417in,height=0.29583in]{mediaCpng/image44.png} marginal tıǵızlıq funkciyaların tabıń; \includegraphics[width=0.15972in,height=0.24028in]{mediaCpng/image45.png} hám \includegraphics[width=0.15972in,height=0.2in]{mediaCpng/image46.png} tosınnanlıq shamalardı ǵárezsizlikke tekseriń: \includegraphics[width=2.56806in,height=0.59167in]{mediaCpng/image56.png}
 \\

\end{tabular}
\vspace{1cm}


\begin{tabular}{m{17cm}}
\textbf{48-variant}
\newline

\textbf{T1.} Itimallıq anıqlamaları (klassikalıq, geometriyalıq anıqlamaları).
 \\
\textbf{T2.} Bólistiriw funkciyası (anıqlaması, tiykarǵı qásiyetleri).
 \\
\textbf{A1.} Úsh oyın kubigi taslanǵanda túsken ochkolardıń birdey bolıw itimallıǵın tabıń.
 \\
\textbf{A2.} Hárbiriniń júzege asıw itimallıǵı $p$ ǵa teń bolǵan 10 dana Bernulli tájiriybesi ótkerilgende, tómendegi waqıyalardıń itimallıqların tabıń: Sátsizler sanı joq bolıw.
 \\
\textbf{A3.} “Kim millioner bolıwdı qáleydi” oyınında individual oyınshınıń 1000 dollar utıp alıwı itimallıǵı 0,3 ke, al 32000 dollar utıw itimallıǵı bolsa 0,01 ge teń. Bul oyında 300 oyınshı qatnasqan bolsa, tómendegi waqıyalardıń júzege asıwı itimallıqların tabıń: a) 80 nen 100 ge shekem oyınshı 1000 dollar utıwı; b) tórtten kóp bolmaǵan oyınshı 32000 dollar utıwı.
 \\
\textbf{B1.} $\xi$ tosınnanlı shamanıń \emph{f}(\emph{x}) tıǵızlıq funkciyasi berilgen bolsin. Tómendegilerdi esaplań: a) C; b) \emph{F}(\emph{x}); c) M$\xi$; d) D$\xi$; e) \emph{f}(\emph{x}) hám \emph{F}(\emph{x}) grafiklarin sızıń.\(f(x) = \left\{ \begin{matrix}
C(1 - 0.5|x|),\ \ \ \ x \in \lbrack - 2,2\rbrack, \\
\ \ \ \ \ \ \ \ 0,\ \ \ \ \ \ \ \ \ \ \ x \notin \lbrack - 2,2\rbrack.\ \ 
\end{matrix} \right.\ \)
 \\
\textbf{B2.} Yashikte 10 buyimniń 4 buyımı jaramsiz. Táwekelge alinǵan 7 buyımnıń ishinde 3 buyim jaramsız bolıwı itimallıǵın tabıń.
 \\
\textbf{B3.} Eger \includegraphics[width=0.36181in,height=0.29444in]{mediaBpng/image1.png} ǵárezsiz tosınnanlıq shamalar izbe-izligi \includegraphics[width=0.52153in,height=0.54583in]{mediaBpng/image45.png} aralıqta teń ólshemli bólistirilgen bolsa, onda bul izbe-izlik úlken sanlar nızamına boysınama?
 \\
\textbf{C1.} Eger \(\xi_{1}\) hám \(\xi_{2}\) ǵárezsiz tosınnanlıq shamalardıń hárbiri standart normal bólistirilgen bolsa, onda \(\xi_{1} + \xi_{2}\) tosınnanlıq shamanıń tıǵızlıq funkciyasın tabıń.
 \\
\textbf{C2.} Eger \(\xi\sim N\left( a,\sigma^{2} \right)\) bolsa, onda \(\xi\) tosınnanlıq shamanıń joqarı tártipli oraylıq momentlerin tabıń.
 \\
\textbf{C3.} Eger ǵárezsiz \includegraphics[width=0.15208in,height=0.24028in]{mediaCpng/image4.png} hám \includegraphics[width=0.15208in,height=0.19167in]{mediaCpng/image5.png} úzliksiz tosınnanlıq shamalardıń hárbiri \includegraphics[width=0.44028in,height=0.21597in]{mediaCpng/image28.png} parametrli kórsetkishli nızam boyınsha bólistirilgen bolsa, onda \includegraphics[width=0.50417in,height=0.29583in]{mediaCpng/image30.png} tosınnanlıq shamanıń tıǵızlıq funkciyasın tabıń.
 \\

\end{tabular}
\vspace{1cm}


\begin{tabular}{m{17cm}}
\textbf{49-variant}
\newline

\textbf{T1.} Shártli itimallıq (anıqlaması, kóbеytiw tеorеması).
 \\
\textbf{T2.} Kompoziciyalıq formulalar \\
\textbf{A1.} Elektrotexnika dúkanına hár qaysısı 2000 danadan joqarı sapalı hám tómen sapalı muzlatqıshlar alıp kelindi. Joqarı sapalı muzlatqıshtıń defektli bolıw itimallıǵı 0,002 ge, al tómen sapalı muzlatqıshtıń defektli bolıw itimallıǵı bolsa 0,04 ke teń. a) Úsh dana joqarı sapalı muzlatqıshtıń defektli bolıwı; b) 60 tan 65 ke shekem tómen sapalı muzlatqıshlardıń defektli bolıwı itimallıqların tabıń.
 \\
\textbf{A2.} 36 dana kartalar kolodasınan tosınnan alınǵan 6 dana karta ishinde anıq 2 danası tuz bolıw itimallıǵın tabıń.
 \\
\textbf{A3.} Bazıbir tarawda ónimniń 3% i I fabrika menen, 25% i II fabrika menen, al ónimniń qalǵan bólegi bolsa III fabrika menen óndiriledi. I fabrikada 1% ónim jaramsız, II fabrikada 1,5% ónim jaramsız, III fabrikada 2% ónim jaramsız bolıp shıǵadı. a) Qarıydar satıp alǵan bir buyım jaramsız bolıw itimallıǵın tabıń. b) Sol buyımdı I fabrika islep shıǵarǵan bolıwı itimallıǵın tabıń.
 \\
\textbf{B1.} Tosınnanlı $\xi$ shamasınıń tıǵızlıq funkciyasi berilgen: \(f(x) = e^{- 3|x|}\) Usi shamanıń matematikalıq kútiliwin tabıń.
 \\
\textbf{B2.} \(M(x,\ y)\) noqat tosınnanlı túrde \(0\  \leq \ x\  \leq \ 1,\ 0\  \leq \ y\  \leq \ 1\) kvadratqa taslandı. \(\min(x,\ y) \leq a\) bolsa, \(a \in (0;1\rbrack\)bolıwı itimallıǵın tabıń.
 \\
\textbf{B3.} \includegraphics[width=0.15972in,height=0.23958in]{mediaBpng/image49.png} úzliksiz tosınnanlıq shamanıń tıǵızlıq funkciyaları berilgen. Olarǵa sáykes \includegraphics[width=0.15972in,height=0.19653in]{mediaBpng/image50.png} tosınnanlıq shamanıń \includegraphics[width=0.50278in,height=0.30069in]{mediaBpng/image51.png} tıǵızlıq funkciyasın tabıń. \includegraphics[width=1.95069in,height=0.58264in]{mediaBpng/image70.png} \includegraphics[width=0.84028in,height=0.23958in]{mediaBpng/image72.png}
 \\
\textbf{C1.} Eger \(\left\{ \xi_{n} \right\}\) tosınnanlıq shamalar izbe-izligi \(\mathbf{\xi}_{\mathbf{n}}\overset{\mathbf{P}}{\rightarrow}\mathbf{\xi}\) hám \(\mathbf{\xi}_{\mathbf{n}}\overset{\mathbf{P}}{\rightarrow}\mathbf{\eta}\) bolsa, onda \(\mathbf{P}\left( \mathbf{\xi = \eta} \right)\mathbf{=}\mathbf{1}\) qatnasın dálilleń.
 \\
\textbf{C2.} Eger ǵárezsiz \includegraphics[width=0.15208in,height=0.24028in]{mediaCpng/image4.png} hám \includegraphics[width=0.15208in,height=0.19167in]{mediaCpng/image5.png} úzliksiz tosınnanlıq shamalar sáykes túrde, \includegraphics[width=0.41597in,height=0.24028in]{mediaCpng/image23.png} hám \includegraphics[width=0.4in,height=0.24028in]{mediaCpng/image18.png} aralıqlarda teń ólshemli bólistirilgen bolsa, onda \includegraphics[width=0.44028in,height=0.24028in]{mediaCpng/image14.png} tosınnanlıq shamanıń tıǵızlıq funkciyasın tabıń.
 \\
\textbf{C3.} Tómende \includegraphics[width=0.46389in,height=0.25625in]{mediaCpng/image42.png} úzliksiz tosınnanlıq vektorlardıń tıǵızlıq funkciyaları berilgen. Olardıń \includegraphics[width=0.48819in,height=0.29583in]{mediaCpng/image43.png} hám \includegraphics[width=0.50417in,height=0.29583in]{mediaCpng/image44.png} marginal tıǵızlıq funkciyaların tabıń; \includegraphics[width=0.15972in,height=0.24028in]{mediaCpng/image45.png} hám \includegraphics[width=0.15972in,height=0.2in]{mediaCpng/image46.png} tosınnanlıq shamalardı ǵárezsizlikke tekseriń: \includegraphics[width=3.16806in,height=0.75972in]{mediaCpng/image66.png}
 \\

\end{tabular}
\vspace{1cm}


\begin{tabular}{m{17cm}}
\textbf{50-variant}
\newline

\textbf{T1.} Tolıq itimallıq formulası (waqıyalardıń tolıq gruppası, dálilleniwi).
 \\
\textbf{T2.} Tosınnanlı shamanıń matematikalıq kútiliwi. (anıqlaması, qásiyetleri)
 \\
\textbf{A1.} ${{x}^{2}}+2px+q=0$ kvadrat teńlemede $p$ hám $q$ koefficientler $\left[ -1,1 \right]$ kesindiden tosınnan tańlanadı. Kvadrat teńlemeniń haqıyqıy túbirlerge iye bolıw itimallıǵın tabıń.
 \\
\textbf{A2.} Hárbiriniń júzege asıw itimallıǵı $p$ ǵa teń bolǵan 10 dana Bernulli tájiriybesi ótkerilgende, tómendegi waqıyalardıń itimallıqların tabıń: Sátliler sanı 2 dana, sonıń menen birge, olardıń barlıǵı tájiriybelerdiń birinshi yarımında ámelge asıwı.
 \\
\textbf{A3.} Eki oyın kubigi taslanǵanda túsken ochkolardıń biri ekinshisinen 4 ese kóp bolıw itimallıǵın tabıń.
 \\
\textbf{B1.} $\xi$ tosınnanlı shamanıń \emph{f}(\emph{x}) tıǵızlıq funkciyasi berilgen bolsin. Tómendegilerdi esaplań: a) C; b) \emph{F}(\emph{x}); c) M$\xi$; d) D$\xi$; e) \emph{f}(\emph{x}) hám \emph{F}(\emph{x}) grafiklarin sızıń.\(f(x) = \left\{ \begin{matrix}
C\left( |x| + \frac{1}{4} \right),\ \ \ \ x \in \lbrack - 1,1\rbrack, \\
\ \ \ \ \ \ \ \ 0,\ \ \ \ \ \ \ \ \ \ \ \ \ \ \ x \notin \lbrack - 1,1\rbrack.\ \ 
\end{matrix} \right.\ \)
 \\
\textbf{B2.} Pul lotereyasında 100 bilet shıǵarılǵan. 50 swmlıq 1 utıs, 10 swmlıq 10 utıs bar. Bir lotereya biletiniń iyesi ushın múmkin bolǵan utıstıń bahasınıń bólistiriw nızamın jazıń.
 \\
\textbf{B3.} Eger \includegraphics[width=0.36181in,height=0.29444in]{mediaBpng/image1.png} ǵárezsiz tosınnanlıq shamalar izbe-izliginiń bólistiriliw nızamları
\includegraphics[width=1.76042in,height=0.50278in]{mediaBpng/image23.png} \includegraphics[width=1.63819in,height=0.50278in]{mediaBpng/image24.png} \includegraphics[width=0.75486in,height=0.23958in]{mediaBpng/image9.png}
bolsa, onda bul izbe-izlik úlken sanlar nızamına boysınama?
 \\
\textbf{C1.} Eger \(\left( \xi_{1},\xi_{2} \right)\) tosınnanlıq vektordıń bólistiriw funkciyası\(F(x,y) = \sin x \cdot \sin y,\ \ \ 0 \leq x \leq \frac{\pi}{2},\ \ 0 \leq y \leq \frac{\pi}{2}\) bolsa, onda \(\left( \xi_{1},\xi_{2} \right)\) tosınnanlıq noqattıń \(G:x_{1} = \frac{\pi}{6},\ \ x_{2} = \frac{\pi}{2};\ \ y_{1} = \frac{\pi}{4},\ \ y_{2} = \frac{\pi}{3}\) bolǵan tuwrımúyeshlikke túsiw itimallıǵın tabıń.
 \\
\textbf{C2.} Eger \(\left\{ \xi_{n} \right\}\) ǵárezsiz tosınnanlıq shamalar izbe-izligi \(\lbrack 0,1\rbrack\) aralıqta teń ólshemli bólistirilgen bolıp, \(g(x)\) funkciya sol aralıqta úziliksiz bolsa, onda\(\frac{g\left( \xi_{1} \right) + ... + g\left( \xi_{n} \right)}{n}\overset{P}{\rightarrow}\int_{0}^{1}{g(x)}dx\) ekenligin kórsetiń.
 \\
\textbf{C3.} Eger \(\xi\) tosınnanlı shama \((a,\sigma)\) parametrli normal bólistiriwine iye bolsa, onda onıń xarakteristikalıq funkciyası tabılsın.
 \\

\end{tabular}
\vspace{1cm}


\begin{tabular}{m{17cm}}
\textbf{51-variant}
\newline

\textbf{T1.} Bernulli sxemаsı ushın limit teоremаlаr (Muavr-Laplas integrallıq teoreması, qásiyetleri).
 \\
\textbf{T2.} Tıǵızlıq funkciyası (anıqlaması, tiykarǵıqásiyetleri).
 \\
\textbf{A1.} Hárbiriniń júzege asıw itimallıǵı $p$ ǵa teń bolǵan 10 dana Bernulli tájiriybesi ótkerilgende, tómendegi waqıyalardıń itimallıqların tabıń: Sátliler sanı tek 2 dana hám olar arasında 3 dana sátsiz bolıw.
 \\
\textbf{A2.} Jámi $N$ dana lotereya biletleri ishinde $M$ dana lotereya bileti utıslı bolsa, satıp alınǵan $n~\,\left( n\le N \right)$ lotereya biletinen $m\,\,\left( m\le M \right)$ danası utıslı bolıw itimallıǵın tabıń.
 \\
\textbf{A3.} Oyın kubigi taslanǵanda 7 ochkonıń túsiw itimallıǵın tabıń.
 \\
\textbf{B1.} Tosinnanli $\xi$ shamasiniń bólistiriw tiǵizliǵi: \(\mathbf{f}\mathbf{(}\mathbf{x}\mathbf{)}\mathbf{=}\left\{ \begin{matrix}
\mathbf{0,}\mathbf{x <}\mathbf{0} \\
\mathbf{2}\mathbf{e}^{\mathbf{-}\mathbf{2}\mathbf{x}}\mathbf{,}\mathbf{x \geq}\mathbf{0}
\end{matrix} \right.\ \) bolǵanda, M$\xi$ hám D$\xi$ lerdi tabiń.
 \\
\textbf{B2.} $\xi$ tosınnanlı shamanıń \emph{f}(\emph{x}) tıǵızlıq funkciyasi berilgen bolsin. Tómendegilerdi esaplań: a) C; b) \emph{F}(\emph{x}); c) M$\xi$; d) D$\xi$; e) \emph{f}(\emph{x}) hám \emph{F}(\emph{x}) grafiklarin sızıń.\(f(x) = \left\{ \begin{matrix}
2x/3,\ \ \ \ x \in \lbrack 0,1\rbrack, \\
C(3 - x),\ \ \ x \in (1,3\rbrack, \\
0,\ \ \ keri\ jag'dayda.\ \ 
\end{matrix} \right.\ \)
 \\
\textbf{B3.} Hár bir sinawda A waqiyasiniń júzege asiw itimalliǵi 0,6 ǵa teń. Ǵárezsiz 5400 sinawdiń 3240 mártesinde A waqiyasiniń júzege asiw itimalliǵin tabiń.
 \\
\textbf{C1.} Kóp ólshemli tıǵızlıq funkciyası óziniń marginal tıǵızlıq funkciyaları arqalı bir mánisli anıqlanbaytuǵınlıǵın kórsetiń.
 \\
\textbf{C2.} Ortasha mánis vektorı \(\left( m_{1},m_{2} \right)\) hám kovariaciyalıq matricası\(K = \begin{pmatrix}
\sigma_{1}^{2} & r\sigma_{1}\sigma_{2} \\
r\sigma_{1}\sigma_{2} & \sigma_{2}^{2}
\end{pmatrix},\ \ \sigma_{1},\ \sigma_{2} > 0,\ \ |r|\  < 1\) bolǵan normal bólistirilgen \(\left( \xi_{1},\xi_{2} \right)\) tosınnanlıq vektordıń tıǵızlıq funkciyasın tabıń.
 \\
\textbf{C3.} Eger ǵárezsiz \includegraphics[width=0.15208in,height=0.24028in]{mediaCpng/image4.png} hám \includegraphics[width=0.15208in,height=0.19167in]{mediaCpng/image5.png} úzliksiz tosınnanlıq shamalardıń tıǵızlıq fukciyaları sáykes túrde,
\includegraphics[width=1.6in,height=0.50417in]{mediaCpng/image38.png} hám \includegraphics[width=1.64028in,height=0.52014in]{mediaCpng/image39.png}
bolsa, onda \includegraphics[width=0.35972in,height=0.27222in]{mediaCpng/image27.png} tosınnanlıq shamanıń tıǵızlıq funkciyasın tabıń.
 \\

\end{tabular}
\vspace{1cm}


\begin{tabular}{m{17cm}}
\textbf{52-variant}
\newline

\textbf{T1.} Tosınnanlı waqıya (elementar waqıyalar keńisligi, waqıyalar ústinde ámeller).
 \\
\textbf{T2.} Tosınnanlı shamanıń dispersiyası (anıqlaması, qásiyetleri).
 \\
\textbf{A1.} $\left[ 0,1 \right]$ kesindiden tosınnan eki noqat tańlanadı. Olardıń koordinataları kvadratları qosındısı, koordinataları kóbeymesi úsh esesinen kóp bolıw itimallıǵın tabıń.
 \\
\textbf{A2.} Birdey úsh qutı berilgen. Birinshi qutıda 3 dana aq hám 2 dana qara sharlar, ekinshi qutıda 2 dana aq hám 4 dana qara sharlar, úshinshi qutıda bolsa 4 dana aq hám 3 dana qara sharlar bar. Tosınnan qutılardan birewi tańlanıp, onıń ishinen bir dana shar alınadı. a) Usı alınǵan shardıń qara shar bolıw itimallıǵın tabıń. b) Alınǵan shar qara shar bolsa, onıń úshinshi qutıdan alınǵan bolıw itimallıǵın tabıń.
 \\
\textbf{A3.} Dúkan 2000 dana televizor hám 2000 dana radio satıp aldı. Hárbir televizordıń defektli bolıw itimallıǵı 0,004 ke hám hárbir radionıń defektli bolıw itimallıǵı 0,03 ke teń. Usı sawdada tómendegi waqıyalardıń júzege asıwı itimallıqların tabıń: a) keminde úsh televizor defektli bolıwı; b) 33 ten 44 ke shekem radio defektli bolıwı.
 \\
\textbf{B1.} Eger \includegraphics[width=0.36181in,height=0.29444in]{mediaBpng/image1.png} ǵárezsiz tosınnanlıq shamalar izbe-izliginiń bólistiriliw nızamları
\includegraphics[width=2.38056in,height=0.47847in]{mediaBpng/image36.png} \includegraphics[width=0.75486in,height=0.23958in]{mediaBpng/image9.png}
bolsa, onda bul izbe-izlik úlken sanlar nızamına boysınama?
 \\
\textbf{B2.} \includegraphics[width=0.15972in,height=0.23958in]{mediaBpng/image49.png} úzliksiz tosınnanlıq shamanıń tıǵızlıq funkciyaları berilgen. Olarǵa sáykes \includegraphics[width=0.15972in,height=0.19653in]{mediaBpng/image50.png} tosınnanlıq shamanıń \includegraphics[width=0.50278in,height=0.30069in]{mediaBpng/image51.png} tıǵızlıq funkciyasın tabıń. \includegraphics[width=2.17153in,height=0.58264in]{mediaBpng/image75.png} \includegraphics[width=0.91389in,height=0.50278in]{mediaBpng/image76.png}
 \\
\textbf{B3.} 
Bólistiriw funkciyasi berilgen: \(\mathbf{F}\mathbf{(}\mathbf{x}\mathbf{)}\mathbf{=}\left\{ \begin{matrix}
\mathbf{0,}\mathbf{\ \ \ \ \ \ \ \ \ \ \ \ \ \ \ \ \ \ \ \ \ \ \ \ \ \ \ \ \ \ \ \ \ \ \ \ \ \ \ \ \ x \leq - a} \\
\frac{\mathbf{1}}{\mathbf{2}}\mathbf{+}\frac{\mathbf{1}}{\mathbf{\pi}}\mathbf{\arcsin}\frac{\mathbf{x}}{\mathbf{a}}\mathbf{,}\mathbf{\ \ \ \ \  - a < x < a}\mathbf{,} \\
\mathbf{1,}\mathbf{\ \ \ \ \ \ \ \ \ \ \ \ \ \ \ \ \ \ \ \ \ \ \ \ \ \ \ \ \ \ \ \ \ \ \ \ \ \ \ \ \ \ \ \ \ x \geq a}
\end{matrix} \right.\ \) a)bólistiriw tiǵizliǵi \(f(x)\  = ?\ \ \ \ \ \ \ \)b) \(\mathbf{P}\left\{ \mathbf{-}\frac{\mathbf{a}}{\mathbf{2}}\mathbf{< \xi <}\frac{\mathbf{a}}{\mathbf{2}} \right\}\mathbf{=}\mathbf{?}\)
 \\
\textbf{C1.} Eger \(\mathbf{\xi}_{\mathbf{n}}\overset{\mathbf{L}^{\mathbf{2}}}{\rightarrow}\mathbf{\xi}\) bolsa, onda \(n \rightarrow \infty\) de \(\mathbf{M}\mathbf{\xi}_{\mathbf{n}}^{\mathbf{2}}\mathbf{\rightarrow M}\mathbf{\xi}^{\mathbf{2}}\) ekenligin kórsetiń.
 \\
\textbf{C2.} Tómende \includegraphics[width=0.46389in,height=0.25625in]{mediaCpng/image42.png} úzliksiz tosınnanlıq vektorlardıń tıǵızlıq funkciyaları berilgen. Olardıń \includegraphics[width=0.48819in,height=0.29583in]{mediaCpng/image43.png} hám \includegraphics[width=0.50417in,height=0.29583in]{mediaCpng/image44.png} marginal tıǵızlıq funkciyaların tabıń; \includegraphics[width=0.15972in,height=0.24028in]{mediaCpng/image45.png} hám \includegraphics[width=0.15972in,height=0.2in]{mediaCpng/image46.png} tosınnanlıq shamalardı ǵárezsizlikke tekseriń: \includegraphics[width=3.54375in,height=0.81597in]{mediaCpng/image71.png} \\
\textbf{C3.} Eger \(\xi\) hám \(\chi^{2}\) ǵárezsiz tosınnanlıq shamalar bolıp, \(\xi\sim N(0,1)\) hám \(\chi^{2}\sim\chi^{2}(n)\) bolsa, onda \(\frac{\xi}{\sqrt{\frac{\chi^{2}}{n}}}\) tosınnanlıq shamanıń tıǵızlıq funkciyasın tabıń.
 \\

\end{tabular}
\vspace{1cm}


\begin{tabular}{m{17cm}}
\textbf{53-variant}
\newline

\textbf{T1.} Bayеs formulası (gipotezalar teoreması, dálilleniwi).
 \\
\textbf{T2.} Oraylıq limit teorema (anıqlaması, ǵárezsiz birdey bólistirilgen tosınnanlı shamalar ushın).
 \\
\textbf{A1.} Birdey úsh qutı berilgen. Birinshi qutıda 2 dana aq hám 1 dana qara sharlar, ekinshi qutıda 3 dana aq hám 1 dana qara sharlar, úshinshi qutıda bolsa 2 dana aq hám 2 dana qara sharlar bar. Tosınnan qutılardan birewi tańlanıp, onıń ishinen bir dana shar alınadı. a) Usı alınǵan shardıń aq shar bolıw itimallıǵın tabıń. b) Alınǵan shar aq bolsa, onıń ekinshi qutıdan alınǵan bolıw itimallıǵın tabıń.
 \\
\textbf{A2.} Jıldıń qálegen kúninde bala tuwılıw itimallıǵı teń dep esaplap, 200 bala arasında tómendegi waqıyalardıń júzege asıwı itimallıqların tabıń: a) anıq úsh bala 1-yanvarda tuwılǵan; b) báhárde 48 den 53 ke shekem bala tuwılǵan.
 \\
\textbf{A3.} $\left[ 0,2 \right]$ kesindiden tosınnan eki noqat tańlanadı. Olardıń koordinataları qosındısı 2 den úlken bolıw hám kvadratları qosındısı 4 ten kishi bolıw itimallıǵın tabıń.
 \\
\textbf{B1.} Radiusı \(r\ (2r < a)\) bolǵan tiyin tosınnanlı túrde tárepi a bolǵan kvadratlarǵa bólingen stolǵa taslandı. Taslanǵan tiyin kvadrattıń bazı bir tárepin kesip ótpewi itimallıǵın tabıń.
 \\
\textbf{B2.} \includegraphics[width=0.15972in,height=0.23958in]{mediaBpng/image49.png} úzliksiz tosınnanlıq shamanıń tıǵızlıq funkciyaları berilgen. Olarǵa sáykes \includegraphics[width=0.15972in,height=0.19653in]{mediaBpng/image50.png} tosınnanlıq shamanıń \includegraphics[width=0.50278in,height=0.30069in]{mediaBpng/image51.png} tıǵızlıq funkciyasın tabıń. \includegraphics[width=2.20278in,height=0.675in]{mediaBpng/image62.png} \includegraphics[width=0.85903in,height=0.23958in]{mediaBpng/image63.png}
 \\
\textbf{B3.} $\xi$ tosınnanlı shamanıń \emph{f}(\emph{x}) tıǵızlıq funkciyasi berilgen bolsin. Tómendegilerdi esaplań: a) C; b) \emph{F}(\emph{x}); c) M$\xi$; d) D$\xi$; e) \emph{f}(\emph{x}) hám \emph{F}(\emph{x}) grafiklarin sızıń.\(f(x) = \left\{ \begin{matrix}
C(1 - x/3),\ \ \ \ x \in \lbrack 0,3\rbrack, \\
\ \ \ \ \ \ \ \ 0,\ \ \ \ \ \ \ \ \ \ \ x \notin \lbrack 0,3\rbrack.\ \ 
\end{matrix} \right.\ \)
 \\
\textbf{C1.} Tómende \includegraphics[width=0.46389in,height=0.25625in]{mediaCpng/image42.png} úzliksiz tosınnanlıq vektorlardıń tıǵızlıq funkciyaları berilgen. Olardıń \includegraphics[width=0.48819in,height=0.29583in]{mediaCpng/image43.png} hám \includegraphics[width=0.50417in,height=0.29583in]{mediaCpng/image44.png} marginal tıǵızlıq funkciyaların tabıń; \includegraphics[width=0.15972in,height=0.24028in]{mediaCpng/image45.png} hám \includegraphics[width=0.15972in,height=0.2in]{mediaCpng/image46.png} tosınnanlıq shamalardı ǵárezsizlikke tekseriń: \includegraphics[width=3.31181in,height=0.84792in]{mediaCpng/image67.png}
 \\
\textbf{C2.} Eger \(\left\{ \xi_{n} \right\}\) ǵárezsiz birdey bólistirilgen tosınnanlıq shamalar izbe-izligi bolıp, onıń bólistiriw funkciyası \(F_{\xi_{1}}(x) = \left\{ \begin{matrix}
\ 1 - e^{\lambda - x},\ \ eger\ \ x \geq \lambda, \\
 \\
\ \ \ \ \ \ 0,\ \ \ \ \ \ \ \ \ \ \ eger\ \ x < \lambda
\end{matrix} \right.\ \) bolsa, onda \(\left\{ \eta_{n} \right\} = \left\{ min(\xi_{1},...,\xi_{n}) \right\}\) izbe-izliktiń \(\mathbf{\lambda}\) ǵa bir itimallıq penen jıynaqlılıǵın kórsetiń.
 \\
\textbf{C3.} Meyli, \(\xi_{1},...,\xi_{n}\) tosınnanlıq shamalar ǵárezsiz hám \(\lbrack a,b\rbrack\) aralıqta teń ólshemli bólistirilgen bolıp, \(\eta_{1} = \max\left( \xi_{1},...,\xi_{n} \right)\) hám \(\eta_{2} = \min\left( \xi_{1},...,\xi_{n} \right)\) bolsın. Onda \(\left( \eta_{1},\eta_{2} \right)\) tosınnanlıq vektordıń kovariaciyasın tabıń.
 \\

\end{tabular}
\vspace{1cm}


\begin{tabular}{m{17cm}}
\textbf{54-variant}
\newline

\textbf{T1.} Itimallıqlar teoriyası aksiomaları (ólshewli keńislik, itimallıq keńisligi).
 \\
\textbf{T2.} 
Úlken sanlar nızamı (anıqlaması, Chebishev teoreması).
 \\
\textbf{A1.} Hárbiriniń júzege asıw itimallıǵı $p$ ǵa teń bolǵan 10 dana Bernulli tájiriybesi ótkerilgende, tómendegi waqıyalardıń itimallıqların tabıń: Sátliler sanı keminde 3 dana.
 \\
\textbf{A2.} 100 dana buyımnan ibarat partiyada 4 dana buyım jaramsız. Partiyadan tosınnan 15 dana buyım alınadı. Usı alınǵan 15 dana buyımnıń ishinde 2 dana buyımnıń jaramsız bolıw itimallıǵın tabıń.
 \\
\textbf{A3.} Eki oyın kubigi taslanǵanda túsken ochkolardıń biri ekinshisinen 2 ese kóp bolıw itimallıǵın tabıń.
 \\
\textbf{B1.} Eger \includegraphics[width=0.36181in,height=0.29444in]{mediaBpng/image1.png} ǵárezsiz tosınnanlıq shamalar izbe-izliginiń bólistiriliw nızamları
\includegraphics[width=2.70556in,height=0.47847in]{mediaBpng/image27.png} \includegraphics[width=0.75486in,height=0.23958in]{mediaBpng/image9.png}
bolsa, onda bul izbe-izlik úlken sanlar nızamına boysınama?
 \\
\textbf{B2.} \includegraphics[width=0.15972in,height=0.23958in]{mediaBpng/image49.png} úzliksiz tosınnanlıq shamanıń tıǵızlıq funkciyaları berilgen. Olarǵa sáykes \includegraphics[width=0.15972in,height=0.19653in]{mediaBpng/image50.png} tosınnanlıq shamanıń \includegraphics[width=0.50278in,height=0.30069in]{mediaBpng/image51.png} tıǵızlıq funkciyasın tabıń. \includegraphics[width=1.95069in,height=0.58264in]{mediaBpng/image70.png} \includegraphics[width=0.85903in,height=0.23958in]{mediaBpng/image71.png}
 \\
\textbf{B3.} Eger \includegraphics[width=0.36181in,height=0.29444in]{mediaBpng/image1.png} ǵárezsiz tosınnanlıq shamalar izbe-izligi \includegraphics[width=0.55208in,height=0.29444in]{mediaBpng/image46.png} aralıqta teń ólshemli bólistirilgen bolsa, onda bul izbe-izlik úlken sanlar nızamına boysınama?
 \\
\textbf{C1.} Eger ǵárezsiz \includegraphics[width=0.15208in,height=0.24028in]{mediaCpng/image4.png} hám \includegraphics[width=0.15208in,height=0.19167in]{mediaCpng/image5.png} úzliksiz tosınnanlıq shamalar sáykes túrde, \includegraphics[width=0.43194in,height=0.21597in]{mediaCpng/image12.png} hám \includegraphics[width=0.60833in,height=0.26389in]{mediaCpng/image13.png} parametrli kórsetkishli nızam boyınsha bólistirilgen bolsa, onda \includegraphics[width=0.91181in,height=0.29583in]{mediaCpng/image15.png} tosınnanlıq shamanıń tıǵızlıq funkciyasın tabıń.
 \\
\textbf{C2.} 
Eger ǵárezsiz \includegraphics[width=0.15208in,height=0.24028in]{mediaCpng/image1.png} hám \includegraphics[width=0.15208in,height=0.19167in]{mediaCpng/image2.png} úzliksiz tosınnanlıq shamalardıń hárbiri standart normal nızam boyınsha bólistirilgen bolsa, onda \includegraphics[width=0.44028in,height=0.24028in]{mediaCpng/image3.png} tosınnanlıq shamanıń tıǵızlıq funkciyasın tabıń.
 \\
\textbf{C3.} Eger \(\left( \xi_{1},\xi_{2} \right)\) tosınnanlıq vеktоrdıń tıǵızlıq funkciyası \(f(x,y) = \frac{1}{3\pi}e^{- \ \ \frac{x^{2} + 4y^{2}}{6}}\)bolsa, onda \(\left( \xi_{1},\xi_{2} \right)\) tosınnanlıq noqattıń \(D = \left\{ (x,y):|x| \leq 1,|y| \leq 2 \right\}\) oblastqa túsiw itimallıǵın tabıń.
 \\

\end{tabular}
\vspace{1cm}


\begin{tabular}{m{17cm}}
\textbf{55-variant}
\newline

\textbf{T1.} Bernulli sxemаsı ushın limit teоremаlаr (Muavr-Laplas lokallıq teoreması, qásiyetleri).
 \\
\textbf{T2.} Tiykarǵı аbsоlyut úzliksiz bólistiriliwler (nоrmаl bólistiriw, teń ólshewli bólistiriw, kórsetkishli bólistiriw). 
 \\
\textbf{A1.} Kitaptıń bir betinde keminde bir baspa qáteligi bolıw itimallıǵı 0,01 ge hám tártip qáteligi bolıw itimallıǵı bolsa 0,3 ke teń. Jámi 500 betli kitapta tómendegi waqıyalardıń júzege asıw itimallıqların tabıń: a) keminde tórt bette baspa qáteligi bolıwı; b) 140 tan 170 ke shekem betlerde tártip qáteligi bolıwı.
 \\
\textbf{A2.} 
Albomda 10 dana jańa hám 12 dana múddeti ótken markalar bar. Albomnan tosınnan 3 marka alınıp, múddeti ótkerildi hám ornına qaytarılıp qoyıldı. Bunnan soń, tosınnan 2 marka alındı. a) Bul 2 marka jańa bolıw itimallıǵın tabıń. b) Sol 2 marka jańa ekenligi belgili bolsa, dáslepki alınǵan 3 markanıń jańa bolıw itimallıǵın tabıń.
 \\
\textbf{A3.} $\left[ 0,1 \right]$ kesindiden tosınnan eki noqat tańlanadı. Noqatlardıń koordinataları qosındısı 1,5 ten kishi bolıw itimallıǵın tabıń.
 \\
\textbf{B1.} Parametrleri (0;$\sigma$ ) bolǵan normal nizamniń dispersiyasın tabılsın.
 \\
\textbf{B2.} $\xi$ tosınnanlı shamanıń \emph{f}(\emph{x}) tıǵızlıq funkciyasi berilgen bolsin. Tómendegilerdi esaplań: a) C; b) \emph{F}(\emph{x}); c) M$\xi$; d) D$\xi$; e) \emph{f}(\emph{x}) hám \emph{F}(\emph{x}) grafiklarin sızıń.\(f(x) = \left\{ \begin{matrix}
C\sqrt{1 - x},\ \ \ \ x \in \lbrack 0,1\rbrack, \\
\ \ \ \ \ \ \ \ 0,\ \ \ \ \ \ \ x \notin \lbrack 0,1\rbrack.\ \ 
\end{matrix} \right.\ \)
 \\
\textbf{B3.} Qálegen \(a,b \in \lbrack 0;2\rbrack\) sanları ushın \(D = \left| \begin{matrix}
1 & a \\
a & b
\end{matrix} \right|\) determinanti esaplanadı. \(D > 0\) bolıwı itimallıǵı qanday?
 \\
\textbf{C1.} Tómende \includegraphics[width=0.46389in,height=0.25625in]{mediaCpng/image42.png} úzliksiz tosınnanlıq vektorlardıń tıǵızlıq funkciyaları berilgen. Olardıń \includegraphics[width=0.48819in,height=0.29583in]{mediaCpng/image43.png} hám \includegraphics[width=0.50417in,height=0.29583in]{mediaCpng/image44.png} marginal tıǵızlıq funkciyaların tabıń; \includegraphics[width=0.15972in,height=0.24028in]{mediaCpng/image45.png} hám \includegraphics[width=0.15972in,height=0.2in]{mediaCpng/image46.png} tosınnanlıq shamalardı ǵárezsizlikke tekseriń: \includegraphics[width=4.04028in,height=0.67986in]{mediaCpng/image65.png}
 \\
\textbf{C2.} Eger \(\left\{ \xi_{n} \right\}\) ǵárezsiz hám \(\mathbf{\lbrack 0,1\rbrack}\) aralıqta teń ólshemli bólistirilgen tosınnanlıq shamalar izbe-izligi bolsa, onda \(\left\{ \mathbf{\xi}_{\mathbf{(}\mathbf{n}\mathbf{)}}\mathbf{=}\mathbf{\max}\mathbf{\{}\mathbf{\xi}_{\mathbf{1}}\mathbf{,...,}\mathbf{\xi}_{\mathbf{n}}\mathbf{\}} \right\}\) izbe-izlik 1 ge itimallıq boyınsha jıynaqlılıǵın kórsetiń.
 \\
\textbf{C3.} Eger \(\xi\sim E(\lambda)\) bolsa, onda \(\xi\) tosınnanlıq shamanıń joqarı tártipli baslanǵısh momentlerin tabıń.
 \\

\end{tabular}
\vspace{1cm}


\begin{tabular}{m{17cm}}
\textbf{56-variant}
\newline

\textbf{T1.} Shártli itimallıq (anıqlaması, kóbеytiw tеorеması).
 \\
\textbf{T2.} Kompoziciyalıq formulalar \\
\textbf{A1.} Hárbiriniń júzege asıw itimallıǵı $p$ ǵa teń bolǵan 10 dana Bernulli tájiriybesi ótkerilgende, tómendegi waqıyalardıń itimallıqların tabıń: Sátliler sanı, sátsizler sanınan keminde 2 ese artıq.
 \\
\textbf{A2.} 20 komanda eki toparǵa bólinedi. Eki eń kúshli komanda bir toparǵa túspew itimallıǵın tabıń.
 \\
\textbf{A3.} Eki oyın kubigi taslanǵanda túsken ochkolardıń qosındısı 3 ten úlken bolıw itimallıǵın tabıń.
 \\
\textbf{B1.} Eger \includegraphics[width=0.36181in,height=0.29444in]{mediaBpng/image1.png} ǵárezsiz hám birdey bólistirilgen tosınnanlıq shamalar izbe-izligi \includegraphics[width=0.44792in,height=0.21458in]{mediaBpng/image43.png} parametrli kórsetkishli bólistiriliwine iye bolsa, onda bul izbe-izlik úlken sanlar nızamına boysınama?
 \\
\textbf{B2.} $\xi$ tosınnanlı shamanıń \emph{f}(\emph{x}) tıǵızlıq funkciyasi berilgen bolsin. Tómendegilerdi esaplań: a) C; b) \emph{F}(\emph{x}); c) M$\xi$; d) D$\xi$; e) \emph{f}(\emph{x}) hám \emph{F}(\emph{x}) grafiklarin sızıń.\(f(x) = \left\{ \begin{matrix}
C\ln x,\ \ \ \ x \in \lbrack 1,e\rbrack, \\
\ \ \ \ 0,\ \ \ \ \ \ \ x \notin \lbrack 1,e\rbrack.\ \ 
\end{matrix} \right.\ \)
 \\
\textbf{B3.} \includegraphics[width=0.15972in,height=0.23958in]{mediaBpng/image49.png} úzliksiz tosınnanlıq shamanıń tıǵızlıq funkciyaları berilgen. Olarǵa sáykes \includegraphics[width=0.15972in,height=0.19653in]{mediaBpng/image50.png} tosınnanlıq shamanıń \includegraphics[width=0.50278in,height=0.30069in]{mediaBpng/image51.png} tıǵızlıq funkciyasın tabıń. \includegraphics[width=2.48472in,height=1.15347in]{mediaBpng/image80.png} \includegraphics[width=0.73611in,height=0.23958in]{mediaBpng/image81.png}
 \\
\textbf{C1.} Eger ǵárezsiz \includegraphics[width=0.15208in,height=0.24028in]{mediaCpng/image4.png} hám \includegraphics[width=0.15208in,height=0.19167in]{mediaCpng/image5.png} úzliksiz tosınnanlıq shamalardıń hárbiri \includegraphics[width=0.4in,height=0.24028in]{mediaCpng/image18.png} aralıqta teń ólshemli bólistirilgen bolsa, onda \includegraphics[width=0.50417in,height=0.29583in]{mediaCpng/image21.png} tosınnanlıq shamanıń tıǵızlıq funkciyasın tabıń.
 \\
\textbf{C2.} 
\(\xi\) diskret tosınnanlıq shama \(x_{i} = ( - 1)^{i}i\) mánislerdi \(p_{i} = \frac{1}{i(i + 1)},\) \(\ \ i = 1,\ 2,\ ...\) itimallıqlar menen qabıl etse, onıń matematikalıq kútiliwin tabıń.
 \\
\textbf{C3.} Meyli, \(\left\{ \xi_{n} \right\}\) tosınnanlıq shamalar izbe-izligi óziniń \(\left\{ F_{n}(x) \right\}\) bólistiriw funkciyaları menen berilgen bolsın. Sonda hám tek sonda ǵana, eger \(\lim_{n \rightarrow \infty}\int_{- \infty}^{+ \infty}{\frac{x^{2}}{1 + x^{2}}dF_{n}(x)} = 0\) bolsa, onda \(\mathbf{\xi}_{\mathbf{n}}\overset{\mathbf{P}}{\rightarrow}\mathbf{0}\) ekenligin dálilleń.
 \\

\end{tabular}
\vspace{1cm}


\begin{tabular}{m{17cm}}
\textbf{57-variant}
\newline

\textbf{T1.} Waqıyalar algebrası ($\sigma$-algebra, minimal $\sigma$-algebra).
 \\
\textbf{T2.} Oraylıq limit teorema (anıqlaması, ǵárezsiz birdey bólistirilgen tosınnanlı shamalar ushın).
 \\
\textbf{A1.} Eki oyın kubigi taslanǵanda túsken ochkolardıń qosındısı 2 den úlken, bıraq 5 ten kishi bolıw itimallıǵın tabıń.
 \\
\textbf{A2.} Bazıbir tarawda ónimniń 3% i I fabrika menen, 25% i II fabrika menen, al ónimniń qalǵan bólegi bolsa III fabrika menen óndiriledi. I fabrikada 1% ónim jaramsız, II fabrikada 1,5% ónim jaramsız, III fabrikada 2% ónim jaramsız bolıp shıǵadı. a) Qarıydar satıp alǵan bir buyım jaramsız bolıw itimallıǵın tabıń. b) Sol buyımdı I fabrika islep shıǵarǵan bolıwı itimallıǵın tabıń.
 \\
\textbf{A3.} Qutıda 80 dana joqarı sapalı hám 20 dana tómen sapalı detallar bar. Qutıdan tosınnan alınǵan 14 dana detaldıń ishinde tómen sapalı detaldıń joq bolıw itimallıǵın tabıń.
 \\
\textbf{B1.} \(f(x) = C \cdot e^{- \frac{x^{2}}{m}}\) tiǵizliq funkciyasi boliwi ushin \emph{C} nege teń boliwi kerek?
 \\
\textbf{B2.} Eki teń ku`shli shaxmatshi shaxmat oynaǵanda 4 partiyadan 3 partiyani utıwı itimallig`i kóppe yamasa 8 partiyadan 5 partiyani utiw itimallıǵı kóp pe?
 \\
\textbf{B3.} Tosınnanlı $\xi$ shamasiniń bólistiriw tiǵizliǵi berilgen: \(f(x) = A \cdot e^{- 5|x|}\). a) \emph{A}=?
 \\
\textbf{C1.} Tómende \includegraphics[width=0.46389in,height=0.25625in]{mediaCpng/image42.png} úzliksiz tosınnanlıq vektorlardıń tıǵızlıq funkciyaları berilgen. Olardıń \includegraphics[width=0.48819in,height=0.29583in]{mediaCpng/image43.png} hám \includegraphics[width=0.50417in,height=0.29583in]{mediaCpng/image44.png} marginal tıǵızlıq funkciyaların tabıń; \includegraphics[width=0.15972in,height=0.24028in]{mediaCpng/image45.png} hám \includegraphics[width=0.15972in,height=0.2in]{mediaCpng/image46.png} tosınnanlıq shamalardı ǵárezsizlikke tekseriń: \includegraphics[width=2.83194in,height=0.63194in]{mediaCpng/image58.png}
 \\
\textbf{C2.} Eger \(\left( \xi_{1},\xi_{2} \right)\) absolyut úziliksiz tosınnanlıq vektordıń tıǵızlıq funkciyası \(f(x,y) = \left\{ \begin{matrix}
Cxy,\ eger\ (x,y) \in D, \\
 \\
0,\ \ \ \ \ eger\ (x,y) \notin D,
\end{matrix} \right.\ \) bunda \(D = \left\{ (x,y):\ y > - x,\ y < 2,\ x < 0 \right\}\) bolsa, onda \(\xi_{1}\) komponentanıń shártsiz hám shártli tıǵızlıq funkciyaların tabıń. Sonıń menen birge, \(\xi_{1}\) hám \(\xi_{2}\) tosınnanlıq shamalardı ǵárezsizlikke tekseriń.
 \\
\textbf{C3.} Tómende \includegraphics[width=0.46389in,height=0.25625in]{mediaCpng/image42.png} úzliksiz tosınnanlıq vektorlardıń tıǵızlıq funkciyaları berilgen. Olardıń \includegraphics[width=0.48819in,height=0.29583in]{mediaCpng/image43.png} hám \includegraphics[width=0.50417in,height=0.29583in]{mediaCpng/image44.png} marginal tıǵızlıq funkciyaların tabıń; \includegraphics[width=0.15972in,height=0.24028in]{mediaCpng/image45.png} hám \includegraphics[width=0.15972in,height=0.2in]{mediaCpng/image46.png} tosınnanlıq shamalardı ǵárezsizlikke tekseriń: \includegraphics[width=3.28819in,height=0.73611in]{mediaCpng/image69.png}
 \\

\end{tabular}
\vspace{1cm}


\begin{tabular}{m{17cm}}
\textbf{58-variant}
\newline

\textbf{T1.} Tosınnanlı waqıya (elementar waqıyalar keńisligi, waqıyalar ústinde ámeller).
 \\
\textbf{T2.} Tosınnanlı shamanıń dispersiyası (anıqlaması, qásiyetleri).
 \\
\textbf{A1.} Elektrotexnika dúkanına hár qaysısı 1000 danadan bolǵan joqarı sapalı hám tómen sapalı muzlatqıshlar alıp kelindi. Joqarı sapalı muzlatqıshtıń defektli bolıwı itimallıǵı 0,001 ge, al tómen sapalı muzlatqıshtıń defektli bolıwı itimallıǵı bolsa 0,03 ke teń. a) Eki dana joqarı sapalı muzlatqıshtıń defektli bolıwı; b) 50 den 70 ke shekem tómen sapalı muzlatqıshlardıń defektli bolıwı itimallıqların tabıń.
 \\
\textbf{A2.} Hárbiriniń júzege asıw itimallıǵı $p$ ǵa teń bolǵan 10 dana Bernulli tájiriybesi ótkerilgende, tómendegi waqıyalardıń itimallıqların tabıń: Tájiriybelerdiń birinshi yarımındaǵı sátliler sanı, tájiriybelerdiń ekinshi yarımındaǵı sátliler sanınan eki esege artıq.
 \\
\textbf{A3.} $\left[ 0,1 \right]$ kesindiden tosınnan eki noqat tańlanadı. Ekinshi noqattıń koordinatası birinshi noqattıń koordinatası eki esesinen úlken bolıw itimallıǵın tabıń.
 \\
\textbf{B1.} Oqıtıwshı joqarı matematikadan shegaralıq bahalaw alıw ushın 50 soraw tayarlaǵan. Olardıń ishinde differencial esabınan 20 soraw, integral esabınan 18 soraw, itimallıqlar teoriyasınan 12 soraw bar. Student differencial esabınan 18 sorawǵa, integral esabınan 15 sorawǵa, itimallıqlar teoriyasınan 10 sorawǵa juwap bere alatuǵın bolsa, onıń dus kelgen bir sorawǵa juwap berip, shegaralıq bahalaw tapsırıwınıń itimallıǵın tabıń.
 \\
\textbf{B2.} $\xi$ tosınnanlı shamanıń \emph{f}(\emph{x}) tıǵızlıq funkciyasi berilgen bolsin. Tómendegilerdi esaplań: a) C; b) \emph{F}(\emph{x}); c) M$\xi$; d) D$\xi$; e) \emph{f}(\emph{x}) hám \emph{F}(\emph{x}) grafiklarin sızıń.\(f(x) = \left\{ \begin{matrix}
\ \ \ \ \ \ \ \ 0,\ \ \ \ \ \ x < 0, \\
C/(x + 1)^{5},\ \ \ \ \ x \geq 0.\ \ 
\end{matrix} \right.\ \)
 \\
\textbf{B3.} \includegraphics[width=0.15972in,height=0.23958in]{mediaBpng/image49.png} úzliksiz tosınnanlıq shamanıń tıǵızlıq funkciyaları berilgen. Olarǵa sáykes \includegraphics[width=0.15972in,height=0.19653in]{mediaBpng/image50.png} tosınnanlıq shamanıń \includegraphics[width=0.50278in,height=0.30069in]{mediaBpng/image51.png} tıǵızlıq funkciyasın tabıń. \includegraphics[width=2.325in,height=0.84028in]{mediaBpng/image66.png} \includegraphics[width=0.54583in,height=0.53403in]{mediaBpng/image67.png}
 \\
\textbf{C1.} Tosınnanlıq vektordıń komponentaları absolyut úziliksizliginen tosınnanlıq vektordıń ózi de absolyut úziliksizligi kelip shıqpaslıǵın kórsetiń.
 \\
\textbf{C2.} Oraylıq limit teorema járdeminde tómendegi teńlikti dálilleń: \(\lim_{n \rightarrow \infty}e^{- n}\sum_{k = 1}^{n}\frac{n^{k}}{k!} = \frac{1}{2}.\)
 \\
\textbf{C3.} Hárqanday \(\varphi_{\xi}(t)\) xarakteristikalıq funkciya ushın \(t \in R\) de \(1 - Re\varphi_{\xi}(2t) \leq 4\left( 1 - Re\varphi_{\xi}(t) \right)\) ekenligin dálilleń.
 \\

\end{tabular}
\vspace{1cm}


\begin{tabular}{m{17cm}}
\textbf{59-variant}
\newline

\textbf{T1.} Bayеs formulası (gipotezalar teoreması, dálilleniwi).
 \\
\textbf{T2.} Bólistiriw funkciyası (anıqlaması, tiykarǵı qásiyetleri).
 \\
\textbf{A1.} $\left[ 0,1 \right]$ kesindiden tosınnan eki noqat tańlanadı. Olardıń koordinataları ayırması modulı $1/6$ den kishi bolıw itimallıǵın tabıń.
 \\
\textbf{A2.} Dushpan nıshanın joq etiw ushın hár túrlı eki samolyot ushıp ketti. Birinshi túrdegi samolyot nıshandı $0,9$ itimallıq penen, ekinshi túrdegi samolyot $0,8$ itimallıq penen joq etiw múmkin. Biraq, dushpannıń hawa hújiminen qorǵanıwı birinshi túrdegi samolyottı $0,95$ itimallıq penen, ekinshi túrdegi samolyottı $0,85$ itimallıq penen urıp túsiriwi múmkin. a) Samolyotlar nıshandı joq etiw itimallıǵın tabıń. b) Nıshan joq etilgen bolsa, onı tek ekinshi samolyot joq etken bolıw itimallıǵın tabıń.
 \\
\textbf{A3.} Elektrotexnika dúkanına hár qaysısı 2000 danadan joqarı sapalı hám tómen sapalı muzlatqıshlar alıp kelindi. Joqarı sapalı muzlatqıshtıń defektli bolıw itimallıǵı 0,002 ge, al tómen sapalı muzlatqıshtıń defektli bolıw itimallıǵı bolsa 0,04 ke teń. a) Úsh dana joqarı sapalı muzlatqıshtıń defektli bolıwı; b) 60 tan 65 ke shekem tómen sapalı muzlatqıshlardıń defektli bolıwı itimallıqların tabıń.
 \\
\textbf{B1.} Eger \includegraphics[width=0.36181in,height=0.29444in]{mediaBpng/image1.png} ǵárezsiz tosınnanlıq shamalar izbe-izliginiń bólistiriliw nızamları
\includegraphics[width=2.50278in,height=0.50278in]{mediaBpng/image4.png} \includegraphics[width=1.16597in,height=0.50278in]{mediaBpng/image5.png} \includegraphics[width=0.75486in,height=0.23958in]{mediaBpng/image6.png}
bolsa, onda bul izbe-izlik úlken sanlar nızamına boysınama?
 \\
\textbf{B2.} Bir nishanaǵa úsh márte atiladi. I, II hám III márte atqandaǵi tiyiw itimalliqlari sáykes: p\textsubscript{1} =0,3; p\textsubscript{2} =0,4; p\textsubscript{3}=0,6. Usi úsh márte atıwdıń nátiyjesinde nishanada eń bolmaǵanda bir oq izi boliwiniń itimallıǵın tabıń.
 \\
\textbf{B3.} $\xi$ tosınnanlı shamanıń \emph{f}(\emph{x}) tıǵızlıq funkciyasi berilgen bolsin. Tómendegilerdi esaplań: a) C; b) \emph{F}(\emph{x}); c) M$\xi$; d) D$\xi$; e) \emph{f}(\emph{x}) hám \emph{F}(\emph{x}) grafiklarin sızıń.\(f(x) = \left\{ \begin{matrix}
\ \ \ \ \ \ 0,\ \ \ \ \ x \notin (0,\ \pi/2)\ \  \\
C\sin x,\ \ \ \ \ x \in (0,\ \pi/2)\ \ 
\end{matrix} \right.\ \)
 \\
\textbf{C1.} Eger ǵárezsiz \includegraphics[width=0.15208in,height=0.24028in]{mediaCpng/image4.png} hám \includegraphics[width=0.15208in,height=0.19167in]{mediaCpng/image5.png} úzliksiz tosınnanlıq shamalardıń hárbiri \includegraphics[width=0.44028in,height=0.21597in]{mediaCpng/image28.png} parametrli kórsetkishli nızam boyınsha bólistirilgen bolsa, onda \includegraphics[width=0.44028in,height=0.24028in]{mediaCpng/image29.png} tosınnanlıq shamanıń tıǵızlıq funkciyasın tabıń.
 \\
\textbf{C2.} Eger \(\left\{ \xi_{n} \right\}\) ǵárezsiz tosınnanlıq shamalar izbe-izliginiń bólistiriw funkciyaları \(F_{n}(x) = \left\{ \begin{matrix}
\ 1 - \frac{1}{x + n},\ \ eger\ \ x > 0 \\
 \\
 \\
\ \ \ \ \ \ \ \ \ \ 0,\ \ \ \ \ \ \ \ \ \ \ eger\ \ x \leq 0
\end{matrix} \right.\ \) bolsa, onda bul izbe-izliktiń 0 ge itimallıq boyınsha jıynaqlılıǵın kórsetiń.
 \\
\textbf{C3.} Eger \(\xi_{1},...,\xi_{n}\) ǵárezsiz birdey bólistirilgen tosınnanlıq shamalar \(F(x)\) bólistiriw hám \(f(x)\) tıǵızlıq funkciyalarǵa iye bolsa, onda \(\eta_{1} = \max\left( \xi_{1},...,\xi_{n} \right)\) hám \(\eta_{2} = \min\left( \xi_{1},...,\xi_{n} \right)\) tosınnanlıq shamalardıń bólistiriw hám tıǵızlıq funkciyaların tabıń.
 \\

\end{tabular}
\vspace{1cm}


\begin{tabular}{m{17cm}}
\textbf{60-variant}
\newline

\textbf{T1.} Bernulli sxemаsı ushın limit teоremаlаr (Puasson bólistiriliwi, qásiyetleri).
 \\
\textbf{T2.} Tiykarǵı аbsоlyut úzliksiz bólistiriliwler (nоrmаl bólistiriw, teń ólshewli bólistiriw, kórsetkishli bólistiriw). 
 \\
\textbf{A1.} Jámi 14 bala hám 11 qız bolǵan studentler toparınan 6 student sorawnama ótkeriw ushın tosınnan tańlap alındı. Olar ishinde 4 bala bolıw itimallıǵın tabıń.
 \\
\textbf{A2.} Hárbiriniń júzege asıw itimallıǵı $p$ ǵa teń bolǵan 10 dana Bernulli tájiriybesi ótkerilgende, tómendegi waqıyalardıń itimallıqların tabıń: Sátliler sanı, sátsizler sanınan kem.
 \\
\textbf{A3.} Eki oyın kubigi taslanǵanda túsken ochkolardıń kóbeymesi 10 nan asıp ketpew itimallıǵın tabıń.
 \\
\textbf{B1.} \includegraphics[width=0.15972in,height=0.23958in]{mediaBpng/image49.png} úzliksiz tosınnanlıq shamanıń tıǵızlıq funkciyaları berilgen. Olarǵa sáykes \includegraphics[width=0.15972in,height=0.19653in]{mediaBpng/image50.png} tosınnanlıq shamanıń \includegraphics[width=0.50278in,height=0.30069in]{mediaBpng/image51.png} tıǵızlıq funkciyasın tabıń. \includegraphics[width=2.62569in,height=0.84028in]{mediaBpng/image56.png} \includegraphics[width=0.85903in,height=0.23958in]{mediaBpng/image57.png}
 \\
\textbf{B2.} Eger \includegraphics[width=0.36181in,height=0.29444in]{mediaBpng/image1.png} ǵárezsiz tosınnanlıq shamalar izbe-izliginiń bólistiriliw nızamları
\includegraphics[width=2.66875in,height=0.50278in]{mediaBpng/image14.png} \includegraphics[width=1.53958in,height=0.50278in]{mediaBpng/image15.png} \includegraphics[width=0.75486in,height=0.23958in]{mediaBpng/image12.png}
bolsa, onda bul izbe-izlik úlken sanlar nızamına boysınama?
 \\
\textbf{B3.} \(f(x) = \frac{C}{e^{x} + e^{- x}}\) bólistiriw tıǵızlıǵı bolıwı ushın \(C\) nege teń bolıwı kerek?
 \\
\textbf{C1.} Eger ǵárezsiz \includegraphics[width=0.15208in,height=0.24028in]{mediaCpng/image4.png} hám úzliksiz \includegraphics[width=0.15208in,height=0.19167in]{mediaCpng/image5.png} tosınnanlıq shamalardıń hárbiri \includegraphics[width=0.4in,height=0.24028in]{mediaCpng/image18.png} aralıqta teń ólshemli bólistirilgen bolsa, onda \includegraphics[width=0.38403in,height=0.24028in]{mediaCpng/image22.png} tosınnanlıq shamanıń tıǵızlıq funkciyasın tabıń.
 \\
\textbf{C2.} Tómende \includegraphics[width=0.46389in,height=0.25625in]{mediaCpng/image42.png} úzliksiz tosınnanlıq vektorlardıń tıǵızlıq funkciyaları berilgen. Olardıń \includegraphics[width=0.48819in,height=0.29583in]{mediaCpng/image43.png} hám \includegraphics[width=0.50417in,height=0.29583in]{mediaCpng/image44.png} marginal tıǵızlıq funkciyaların tabıń; \includegraphics[width=0.15972in,height=0.24028in]{mediaCpng/image45.png} hám \includegraphics[width=0.15972in,height=0.2in]{mediaCpng/image46.png} tosınnanlıq shamalardı ǵárezsizlikke tekseriń: \includegraphics[width=2.56806in,height=0.59167in]{mediaCpng/image53.png}
 \\
\textbf{C3.} Eger \(\xi\) tosınnanlıq shama \(\lbrack 0,\ \pi\rbrack\) aralıqta teń ólshewli bólistirilgen bolsa, onda \(M\sin\xi,\) \(D\sin\xi\) hám \(M\cos\xi,\) \(D\cos\xi\) mánislerin tabıń.
 \\

\end{tabular}
\vspace{1cm}


\begin{tabular}{m{17cm}}
\textbf{61-variant}
\newline

\textbf{T1.} Bernulli sxemаsı ushın limit teоremаlаr (Muavr-Laplas integrallıq teoreması, qásiyetleri).
 \\
\textbf{T2.} Tıǵızlıq funkciyası (anıqlaması, tiykarǵıqásiyetleri).
 \\
\textbf{A1.} Detallar partiyası úsh jumısshı tárepinen tayarlanadı. Birinshi jumısshı barlıq detaldıń 25% in, ekinshi jumısshı 35% in, úshinshisi bolsa 40% in tayarlaydı. Bul úsh jumısshınıń tayarlaǵan detallarınıń sapasız bolıwı waqıyası itimallıqları sáykes túrde 0,05; 0,04 hám 0,02 ge teń. a) Tekseriw ushın partiyadan alınǵan detaldıń sapasız bolıw itimallıǵın tabıń. b) Sol sapasız detaldıń ekinshi jumısshı tárepinen tayarlanǵan bolıw itimallıǵın tabıń.
 \\
\textbf{A2.} Hárbiriniń júzege asıw itimallıǵı $p$ ǵa teń bolǵan 10 dana Bernulli tájiriybesi ótkerilgende, tómendegi waqıyalardıń itimallıqların tabıń: Sátliler sanı 6 dana, sonıń menen birge, sońǵı tájiriybe sátsiz juwmaqlanıwı.
 \\
\textbf{A3.} Eki oyın kubigi taslanǵanda túsken eń kishi ochko 4 ten úlken bolıw itimallıǵın tabıń.
 \\
\textbf{B1.} Eger \includegraphics[width=0.36181in,height=0.29444in]{mediaBpng/image1.png} ǵárezsiz tosınnanlıq shamalar izbe-izliginiń bólistiriliw nızamları
\includegraphics[width=2.55833in,height=0.50278in]{mediaBpng/image7.png} \includegraphics[width=1.41736in,height=0.50278in]{mediaBpng/image8.png} \includegraphics[width=0.75486in,height=0.23958in]{mediaBpng/image9.png}
bolsa, onda bul izbe-izlik úlken sanlar nızamına boysınama?
 \\
\textbf{B2.} Bólistiriw tiǵizliǵi \(\mathbf{f}\mathbf{(}\mathbf{x}\mathbf{)}\mathbf{=}\frac{\mathbf{1}}{\mathbf{\sigma}\sqrt{\mathbf{2}\mathbf{\pi}}}\mathbf{e}^{\mathbf{-}\frac{\left( \mathbf{x - m} \right)^{\mathbf{2}}}{\mathbf{2}\mathbf{\sigma}^{\mathbf{2}}}}\)bolsa, usı tosınnanlı shamanıń matematikalıq kútiliwin tabılsın.
 \\
\textbf{B3.} $\xi$ tosınnanlı shamanıń \emph{f}(\emph{x}) tıǵızlıq funkciyasi berilgen bolsin. Tómendegilerdi esaplań: a) C; b) \emph{F}(\emph{x}); c) M$\xi$; d) D$\xi$; e) \emph{f}(\emph{x}) hám \emph{F}(\emph{x}) grafiklarin sızıń.\(f(x) = \left\{ \begin{matrix}
C/x,\ \ \ \ x \in \lbrack 1/e,e\rbrack, \\
\ \ \ \ 0,\ \ \ \ \ x \notin \lbrack 1/e,e\rbrack.\ \ 
\end{matrix} \right.\ \)
 \\
\textbf{C1.} Tómende \includegraphics[width=0.46389in,height=0.25625in]{mediaCpng/image42.png} úzliksiz tosınnanlıq vektorlardıń tıǵızlıq funkciyaları berilgen. Olardıń \includegraphics[width=0.48819in,height=0.29583in]{mediaCpng/image43.png} hám \includegraphics[width=0.50417in,height=0.29583in]{mediaCpng/image44.png} marginal tıǵızlıq funkciyaların tabıń; \includegraphics[width=0.15972in,height=0.24028in]{mediaCpng/image45.png} hám \includegraphics[width=0.15972in,height=0.2in]{mediaCpng/image46.png} tosınnanlıq shamalardı ǵárezsizlikke tekseriń: \includegraphics[width=3.32014in,height=0.84792in]{mediaCpng/image68.png}
 \\
\textbf{C2.} 
\(\xi\) diskret tosınnanlıq shama \(x_{i} = ( - 1)^{i}i\) mánislerdi \(p_{i} = \frac{1}{i(i + 1)},\) \(\ \ i = 1,\ 2,\ ...\) itimallıqlar menen qabıl etse, onıń matematikalıq kútiliwin tabıń.
 \\
\textbf{C3.} Eger \(\mathbf{\xi}_{\mathbf{n}}\overset{\mathbf{L}^{\mathbf{2}}}{\rightarrow}\mathbf{\xi}\) bolsa, onda \(n \rightarrow \infty\) de \(\mathbf{M}\mathbf{\xi}_{\mathbf{n}}^{\mathbf{2}}\mathbf{\rightarrow M}\mathbf{\xi}^{\mathbf{2}}\) ekenligin kórsetiń.
 \\

\end{tabular}
\vspace{1cm}


\begin{tabular}{m{17cm}}
\textbf{62-variant}
\newline

\textbf{T1.} Tolıq itimallıq formulası (waqıyalardıń tolıq gruppası, dálilleniwi).
 \\
\textbf{T2.} Xarakteristikalıq funkciyalar (anıqlaması, tiykarǵı qásiyetleri).
 \\
\textbf{A1.} Uzaq aralıqtaǵı nıshanǵa pulemyot hám pistolet penen oq atılmaqta. Pulemyot oǵınıń nıshanǵa tiyiw itimallıǵı 0,02, al pistolet penen bolsa 0,6 ǵa teń. Eger hárbir qural menen nıshanǵa 100 márte oq atılǵan bolsa, tómendegi waqıyalardıń júzege asıwı itimallıqların tabıń: a) pulemyot penen nıshanǵa úsh mártege shekem tiygiziwi; b) pistolet penen nıshanǵa 60 márte tiygiziwi.
 \\
\textbf{A2.} $\left[ 0,1 \right]$ kesindiden tosınnan eki noqat  tańlanadı. Olardıń koordinataları kvadratları qosındısı 1 den úlken bolıw itimallıǵın tabıń.
 \\
\textbf{A3.} 36 dana kartalar kolodasınan tosınnan alınǵan 3 dana kartanıń barlıǵı birdey reńde bolıw itimallıǵın tabıń.
 \\
\textbf{B1.} Bir reyste pоezdǵа bilet alǵan 860 pаssаjirdiń hár biriniń keshigiw itimаllıǵı 0,04 ge teń. Pоezdǵа bilet alǵan pаssаjirlerdiń keshigiwiniń eń itimallı sаnı tabılsın.
 \\
\textbf{B2.} \includegraphics[width=0.15972in,height=0.23958in]{mediaBpng/image49.png} úzliksiz tosınnanlıq shamanıń tıǵızlıq funkciyaları berilgen. Olarǵa sáykes \includegraphics[width=0.15972in,height=0.19653in]{mediaBpng/image50.png} tosınnanlıq shamanıń \includegraphics[width=0.50278in,height=0.30069in]{mediaBpng/image51.png} tıǵızlıq funkciyasın tabıń. \includegraphics[width=1.95069in,height=0.58889in]{mediaBpng/image96.png} \includegraphics[width=0.55208in,height=0.28194in]{mediaBpng/image97.png}
 \\
\textbf{B3.} Eger \(\xi\) tosınnanlı shama \(\lbrack a,b\rbrack\) parametrli teń ólshewli bólistiriwine iye bolsa, onda onıń xarakteristikalıq funkciyası tabılsın.
 \\
\textbf{C1.} Eger ǵárezsiz \includegraphics[width=0.15208in,height=0.24028in]{mediaCpng/image4.png} hám \includegraphics[width=0.15208in,height=0.19167in]{mediaCpng/image5.png} úzliksiz tosınnanlıq shamalardıń hárbiri \includegraphics[width=0.44028in,height=0.21597in]{mediaCpng/image28.png} parametrli kórsetkishli nızam boyınsha bólistirilgen bolsa, onda \includegraphics[width=0.44028in,height=0.24028in]{mediaCpng/image14.png} tosınnanlıq shamanıń tıǵızlıq funkciyasın tabıń.
 \\
\textbf{C2.} Eger \(\xi_{1}\) hám \(\xi_{2}\) ǵárezsiz tosınnanlıq shamalardıń hárbiri standart normal bólistirilgen bolsa, onda \(\xi_{1} + \xi_{2}\) tosınnanlıq shamanıń tıǵızlıq funkciyasın tabıń.
 \\
\textbf{C3.} Eger \(\xi_{1}\) hám \(\xi_{2}\) ǵárezsiz tosınnanlıq shamalardıń hárbiri \(\lbrack 0,1\rbrack\) aralıqta teń ólshemli bólistirilgen bolsa, onda \(\xi_{1} + \xi_{2}\) tosınnanlıq shamanıń tıǵızlıq funkciyasın tabıń.
 \\

\end{tabular}
\vspace{1cm}


\begin{tabular}{m{17cm}}
\textbf{63-variant}
\newline

\textbf{T1.} Itimallıq anıqlamaları (klassikalıq, geometriyalıq anıqlamaları).
 \\
\textbf{T2.} Tosınnanlı shamanıń matematikalıq kútiliwi. (anıqlaması, qásiyetleri)
 \\
\textbf{A1.} Hárbiriniń júzege asıw itimallıǵı $p$ ǵa teń bolǵan 10 dana Bernulli tájiriybesi ótkerilgende, tómendegi waqıyalardıń itimallıqların tabıń: Sátliler sanı 6 dana, sonıń menen birge, olardıń barlıǵı dáslepki altı tájiriybede ámelge asıwı.
 \\
\textbf{A2.} Oyın kubigi taslandı. Meyli, $m$ - túsken ochkolar sanı bolsın. Soń, nıshanǵa qarata hárbir atıwda $p$ tiyiw itimallıǵı menen $2m$ márte oq atıladı. a) Nıshanǵa eki márte oq tiyiw itimallıǵın tabıń. b) Nıshanǵa eki márte oq tiygen bolsa, $m=3$ bolıw itimallıǵın tabıń.
 \\
\textbf{A3.} $\left[ 0,3 \right]$ kesindiden tosınnan úsh noqat tańlanadı. Olardıń koordinataları qosındısı 3 ten kishi bolıw itimallıǵın tabıń.
 \\
\textbf{B1.} $\xi$ tosınnanlı shamanıń \emph{f}(\emph{x}) tıǵızlıq funkciyasi berilgen bolsin. Tómendegilerdi esaplań: a) C; b) \emph{F}(\emph{x}); c) M$\xi$; d) D$\xi$; e) \emph{f}(\emph{x}) hám \emph{F}(\emph{x}) grafiklarin sızıń.\(f(x) = \left\{ \begin{matrix}
(x + 1)/2,\ \ \ \ x \in \lbrack - 1,0\rbrack, \\
(C - x)/2C,\ x \in (0,C\rbrack, \\
\ \ \ \ \ \ 0,\ \ \ \ \ \ \ \ \ \ \ \ x \notin \lbrack - 1,C\rbrack\ \ 
\end{matrix} \right.\ \)
 \\
\textbf{B2.} Atiwshiniń bir márte nishanaǵa atqanda tiygiziw itimalliǵi 0,75 ke teń. Nishanaǵa 10 márte atqanda 8 mártesinde tiygiziw itimalliǵin tabiń.
 \\
\textbf{B3.} \includegraphics[width=0.15972in,height=0.23958in]{mediaBpng/image49.png} úzliksiz tosınnanlıq shamanıń tıǵızlıq funkciyaları berilgen. Olarǵa sáykes \includegraphics[width=0.15972in,height=0.19653in]{mediaBpng/image50.png} tosınnanlıq shamanıń \includegraphics[width=0.50278in,height=0.30069in]{mediaBpng/image51.png} tıǵızlıq funkciyasın tabıń. \includegraphics[width=2.44167in,height=0.63194in]{mediaBpng/image98.png} \includegraphics[width=0.57639in,height=0.28194in]{mediaBpng/image99.png} \\
\textbf{C1.} Tómende \includegraphics[width=0.46389in,height=0.25625in]{mediaCpng/image42.png} úzliksiz tosınnanlıq vektorlardıń tıǵızlıq funkciyaları berilgen. Olardıń \includegraphics[width=0.48819in,height=0.29583in]{mediaCpng/image43.png} hám \includegraphics[width=0.50417in,height=0.29583in]{mediaCpng/image44.png} marginal tıǵızlıq funkciyaların tabıń; \includegraphics[width=0.15972in,height=0.24028in]{mediaCpng/image45.png} hám \includegraphics[width=0.15972in,height=0.2in]{mediaCpng/image46.png} tosınnanlıq shamalardı ǵárezsizlikke tekseriń: \includegraphics[width=2.79167in,height=0.67986in]{mediaCpng/image63.png}
 \\
\textbf{C2.} Eger \(\left\{ \xi_{n} \right\}\) diskret tosınnanlıq shamalar izbe-izliginiń bólistiriliw nızamları\(P(\xi_{n} = e^{n}) = \frac{1}{n^{2}},\) \(P(\xi_{n} = 0) = 1 - \frac{1}{n^{2}}\) bolsa, onda \(\left\{ \xi_{n} \right\}\) tosınnanlıq shamalar izbe-izliginiń 0 ge bir itimallıq penen jıynaqlılıǵın kórsetiń.
 \\
\textbf{C3.} Eger ǵárezsiz \includegraphics[width=0.15208in,height=0.24028in]{mediaCpng/image4.png} hám \includegraphics[width=0.15208in,height=0.19167in]{mediaCpng/image5.png} úzliksiz tosınnanlıq shamalar sáykes túrde, \includegraphics[width=0.4in,height=0.25625in]{mediaCpng/image9.png} hám \includegraphics[width=0.44028in,height=0.25625in]{mediaCpng/image10.png} parametrli normal nızam boyınsha bólistirilgen bolsa, onda \includegraphics[width=0.44028in,height=0.24028in]{mediaCpng/image11.png} tosınnanlıq shamanıń tıǵızlıq funkciyasın tabıń.
 \\

\end{tabular}
\vspace{1cm}


\begin{tabular}{m{17cm}}
\textbf{64-variant}
\newline

\textbf{T1.} Bernulli sxemаsı ushın limit teоremаlаr (Muavr-Laplas lokallıq teoreması, qásiyetleri).
 \\
\textbf{T2.} 
Úlken sanlar nızamı (anıqlaması, Chebishev teoreması).
 \\
\textbf{A1.} Studentler úsh jıl dawamında matematikalıq kitaplardan hárbirinde 30 máseleni óz ishine alǵan matematikadan 25 tipikalıq esaplardı orınlaydı. Kompyuterde matematikalıq paket járdeminde máseleni nadurıs sheshiwi itimallıǵı 0,01 ge, paket járdemisiz 0,2 ge teń. Úsh jıl ishinde tómendegi waqıyalardıń júzege asıwı itimallıqların tabıń: a) matematikalıq paketten turaqlı túrde paydalanatuǵın student 4 ten kóp bolmaǵan máselelerdi nadurıs sheshken bolsa; b) matematikalıq paketten paydalanbaytuǵın student 120 dan 180 ge shekem máseleni nadurıs sheshedi.
 \\
\textbf{A2.} 28 dana dominonıń tolıq komplektinen 7 danası tosınnan tańlanadı. Olardıń ishinde keminde 1 dana 6 ochko bolıw itimallıǵın tabıń.
 \\
\textbf{A3.} Oyın kubigi taslanǵanda jup ochkonıń túsiw itimallıǵın tabıń.
 \\
\textbf{B1.} Eger \includegraphics[width=0.36181in,height=0.29444in]{mediaBpng/image1.png} ǵárezsiz tosınnanlıq shamalar izbe-izliginiń bólistiriliw nızamları
\includegraphics[width=2.41736in,height=0.50278in]{mediaBpng/image10.png} \includegraphics[width=1.50278in,height=0.50278in]{mediaBpng/image11.png} \includegraphics[width=0.75486in,height=0.23958in]{mediaBpng/image12.png}
bolsa, onda bul izbe-izlik úlken sanlar nızamına boysınama?
 \\
\textbf{B2.} $\xi$ tosınnanlı shamanıń \emph{f}(\emph{x}) tıǵızlıq funkciyasi berilgen bolsin. Tómendegilerdi esaplań: a) C; b) \emph{F}(\emph{x}); c) M$\xi$; d) D$\xi$; e) \emph{f}(\emph{x}) hám \emph{F}(\emph{x}) grafiklarin sızıń.\(f(x) = \left\{ \begin{matrix}
C/\sqrt{1 - x^{2}},\ \ \ \ x \in \lbrack - 1,1\rbrack, \\
\ \ \ \ \ \ \ \ 0,\ \ \ \ \ \ \ \ \ \ \ x \notin \lbrack - 1,1\rbrack.\ \ 
\end{matrix} \right.\ \)
 \\
\textbf{B3.} Tosinnanli $\xi$ shamasiniń bólistiriw tiǵizliǵi \(\mathbf{f}\mathbf{(}\mathbf{x}\mathbf{)}\mathbf{=}\frac{\mathbf{1}}{\mathbf{2}}\mathbf{e}^{\mathbf{-}\left| \mathbf{x} \right|}\) bolsa, onıń matematikalıq kútiliwin tabıń.
 \\
\textbf{C1.} Eger \(\xi\sim N\left( a,\sigma^{2} \right)\) bolsa, onda \(\xi\) tosınnanlıq shamanıń joqarı tártipli oraylıq momentlerin tabıń.
 \\
\textbf{C2.} Eger ǵárezsiz \includegraphics[width=0.15208in,height=0.24028in]{mediaCpng/image4.png} hám \includegraphics[width=0.15208in,height=0.19167in]{mediaCpng/image5.png} úzliksiz tosınnanlıq shamalar sáykes túrde, \includegraphics[width=0.43194in,height=0.21597in]{mediaCpng/image12.png} hám \includegraphics[width=0.60833in,height=0.26389in]{mediaCpng/image13.png} parametrli kórsetkishli nızam boyınsha bólistirilgen bolsa, onda \includegraphics[width=0.44028in,height=0.24028in]{mediaCpng/image14.png} tosınnanlıq shamanıń tıǵızlıq funkciyasın tabıń.
 \\
\textbf{C3.} 
Eger \(\xi\) tosınnanlıq shama hám \(\left\{ \xi_{n} \right\}\) tosınnanlıq shamalar izbe-izligi ǵárezsiz birdey standart normal bólistirilgen bolsa, onda \(\left\{ \mathbf{\eta}_{\mathbf{n}} \right\}\mathbf{=}\left\{ \frac{\mathbf{\xi}\sqrt{\mathbf{n}}}{\sqrt{\mathbf{\xi}_{\mathbf{1}}^{\mathbf{2}}\mathbf{+}\mathbf{...}\mathbf{+}\mathbf{\xi}_{\mathbf{n}}^{\mathbf{2}}}} \right\}\) tosınnanlıq shamalar izbe-izligining limit bólistiriw funkciyası standart normal bólistiriliw bolıwın kórsetiń.
 \\

\end{tabular}
\vspace{1cm}


\begin{tabular}{m{17cm}}
\textbf{65-variant}
\newline

\textbf{T1.} Ǵárezsiz tájiriybelerdiń Bernulli sxeması (binоmiаl bólistiriliw, qásiyetleri).
 \\
\textbf{T2.} Tiykarǵı diskret bólistiriliwler (Binomial, Puasson hám geometriyalıq bólistiriliwler).
 \\
\textbf{A1.} $\left[ 0,1 \right]$ kesindiden tosınnan eki noqat tańlanadı. Birinshi hám ekinshi noqatlar koordinataları arasındaǵı aralıq $0,7$ den kishi bolıw itimallıǵın tabıń.
 \\
\textbf{A2.} Qutıda 90 dana sapalı hám 10 dana sapasız detallar bar. Qutıdan tosınnan alınǵan 10 dana detaldıń ishinde sapasız detaldıń joq bolıw itimallıǵın tabıń.
 \\
\textbf{A3.} Radio ustaxanada kúnine 2 radio ońlanadı. Radionıń mexanikalıq bólegi buzılıwı itimallıǵı 0,2 ge hám elektron bólegi buzılıw itimallıǵı 0,005 ke teń. Jıl dawamında ońlanǵan radiolar arasında tómendegi waqıyalardıń júzege asıw itimallıqların tabıń: a) 140 tan 150 ge shekem radiolardıń mexanikalıq bóleginde nasazlıqlar bolǵan; b) besten artıq bolmaǵan radiolardıń elektron bóleginde nasazlıqlar bolǵan.
 \\
\textbf{B1.} \includegraphics[width=0.15972in,height=0.23958in]{mediaBpng/image49.png} úzliksiz tosınnanlıq shamanıń tıǵızlıq funkciyaları berilgen. Olarǵa sáykes \includegraphics[width=0.15972in,height=0.19653in]{mediaBpng/image50.png} tosınnanlıq shamanıń \includegraphics[width=0.50278in,height=0.30069in]{mediaBpng/image51.png} tıǵızlıq funkciyasın tabıń. \includegraphics[width=2.53958in,height=1.15347in]{mediaBpng/image88.png} \includegraphics[width=0.63194in,height=0.25764in]{mediaBpng/image89.png}
 \\
\textbf{B2.} Eger \includegraphics[width=0.36181in,height=0.29444in]{mediaBpng/image1.png} ǵárezsiz tosınnanlıq shamalar izbe-izliginiń bólistiriliw nızamları
\includegraphics[width=2.53958in,height=0.47847in]{mediaBpng/image26.png} \includegraphics[width=0.75486in,height=0.23958in]{mediaBpng/image9.png}
bolsa, onda bul izbe-izlik úlken sanlar nızamına boysınama?
 \\
\textbf{B3.} Skladqa 30 yashik shisheli buyımlar túsirilgen. Táwekelge alınǵan yashikte buyımlardıń pútin bolıw itimallıǵı 0,9 ǵa teń. Barlıq buyımlar zálelge ushıramaǵan eń kóp itimallı yashikler sanın tabıń.
 \\
\textbf{C1.} Eger \(\xi\) tosınnanlı shama \((a,\sigma)\) parametrli normal bólistiriwine iye bolsa, onda onıń xarakteristikalıq funkciyası tabılsın.
 \\
\textbf{C2.} Tómende \includegraphics[width=0.46389in,height=0.25625in]{mediaCpng/image42.png} úzliksiz tosınnanlıq vektorlardıń tıǵızlıq funkciyaları berilgen. Olardıń \includegraphics[width=0.48819in,height=0.29583in]{mediaCpng/image43.png} hám \includegraphics[width=0.50417in,height=0.29583in]{mediaCpng/image44.png} marginal tıǵızlıq funkciyaların tabıń; \includegraphics[width=0.15972in,height=0.24028in]{mediaCpng/image45.png} hám \includegraphics[width=0.15972in,height=0.2in]{mediaCpng/image46.png} tosınnanlıq shamalardı ǵárezsizlikke tekseriń: \includegraphics[width=3.28819in,height=0.84792in]{mediaCpng/image60.png}
 \\
\textbf{C3.} Eger \(\left( \xi_{1},\xi_{2} \right)\) absolyut úziliksiz tosınnanlıq vektordıń \(\xi_{1}\) hám \(\xi_{2}\) komponentaları ǵárezsiz bolıp, olardıń hárbiri standart normal bólistirilgen bolsa, onda \(\left( \xi_{1},\xi_{2} \right)\) tosınnanlıq noqattıń \(D = \left\{ (x,y):\ x^{2} + y^{2} \leq R^{2} \right\}\) oblastqa túsiw itimallıǵın tabıń.
 \\

\end{tabular}
\vspace{1cm}


\begin{tabular}{m{17cm}}
\textbf{66-variant}
\newline

\textbf{T1.} Itimallıqlar teoriyası aksiomaları (ólshewli keńislik, itimallıq keńisligi).
 \\
\textbf{T2.} Tosınnanlı shamanıń joqarı tártipli momentleri (baslanǵısh hám oraylıq momentleri, qásiyetleri).
 \\
\textbf{A1.} Shegaralıq bahalaw jumısın tapsırıwǵa kelgen 10 studentten ibarat toparda úshewi ayrıqsha, tórtewi jaqsı, ekewi qanaatlandırarlı hám birewi qanaatlandırarsız tayarlanǵan. Shegaralıq bahalaw jumısınıń variantlarında 20 dana soraw bar. Ayrıqsha tayarlanǵan student barlıq 20 sorawǵa, jaqsı tayarlanǵanı 16 sorawǵa, qanaatlandırarlı tayarlanǵanı 10 sorawǵa, qanaatlandırarsız tayarlanǵanı 5 sorawǵa juwap bere aladı. a) Bul studentlerden qálegen birewi berilgen bir sorawǵa durıs juwap beriw itimallıǵın tabıń. b) Sol durıs juwap bergen studenttiń ayrıqsha tayarlanǵan student bolıwı itimallıǵın tabıń.
 \\
\textbf{A2.} Hárbiriniń júzege asıw itimallıǵı $p$ ǵa teń bolǵan 10 dana Bernulli tájiriybesi ótkerilgende, tómendegi waqıyalardıń itimallıqların tabıń: Sátliler sanı 3 dana, sonıń menen birge, sońǵı tájiriybe sátli juwmaqlanıwı.
 \\
\textbf{A3.} Eki oyın kubigi taslanǵanda túsken ochkolardıń ayırması 1 ge teń bolıw itimallıǵın tabıń.
 \\
\textbf{B1.} $\xi$ tosınnanlı shamanıń \emph{f}(\emph{x}) tıǵızlıq funkciyasi berilgen bolsin. Tómendegilerdi esaplań: a) C; b) \emph{F}(\emph{x}); c) M$\xi$; d) D$\xi$; e) \emph{f}(\emph{x}) hám \emph{F}(\emph{x}) grafiklarin sızıń.\(f(x) = \left\{ \begin{matrix}
C(1 - |x|),\ \ \ \ x \in \lbrack - 1,1\rbrack, \\
\ \ \ \ \ \ \ \ 0,\ \ \ \ \ \ \ \ \ x \notin \lbrack - 1,1\rbrack.\ \ 
\end{matrix} \right.\ \)
 \\
\textbf{B2.} Puasson nizamına boysiniwshi tosinnanli $\xi$ shamasiniń dispersiyası tabılsın.
 \\
\textbf{B3.} Zavod bazaǵa 5000 sipatli buyim jóneltken. Jolda buyimniń zálelleniw itimalliǵi 0,0002 ge teń. Bazaǵa 3 jaramsiz buyimniń kelip túsiw itimalliǵin tabiń.
 \\
\textbf{C1.} Meyli, \(\left\{ \xi_{n} \right\}\) tosınnanlıq shamalar izbe-izligi óziniń \(\left\{ F_{n}(x) \right\}\) bólistiriw funkciyaları menen berilgen bolsın. Sonda hám tek sonda ǵana, eger \(\lim_{n \rightarrow \infty}\int_{- \infty}^{+ \infty}{\frac{x^{2}}{1 + x^{2}}dF_{n}(x)} = 0\) bolsa, onda \(\mathbf{\xi}_{\mathbf{n}}\overset{\mathbf{P}}{\rightarrow}\mathbf{0}\) ekenligin dálilleń.
 \\
\textbf{C2.} Eger \(\left( \xi_{1},\xi_{2} \right)\) absolyut úziliksiz tosınnanlıq vektordıń tıǵızlıq funkciyası \(f(x,y) = \left\{ \begin{matrix}
Ce^{- x - y},\ eger\ \ x \geq 0,y \geq 0, \\
 \\
 \\
\ \ \ \ \ \ \ \ 0,\ \ \ \ \ basqa\ hallarda\ 
\end{matrix} \right.\ \) bolsa, onda \(F(x,y),\) \(F_{\xi_{1}}(x),\) \(F_{\xi_{2}}(y),\) \(f_{\xi_{1}}(x),\) \(f_{\xi_{2}}(y)\) hám \(P\left( \xi_{1} > 0,\xi_{2} < 1 \right)\) itimallıqtı tabıń. Sonıń menen birge, \(\xi_{1}\) hám \(\xi_{2}\) tosınnanlıq shamalardı ǵárezsizlikke tekseriń.
 \\
\textbf{C3.} 
Tómende \includegraphics[width=0.46389in,height=0.25625in]{mediaCpng/image42.png} úzliksiz tosınnanlıq vektorlardıń tıǵızlıq funkciyaları berilgen. Olardıń \includegraphics[width=0.48819in,height=0.29583in]{mediaCpng/image43.png} hám \includegraphics[width=0.50417in,height=0.29583in]{mediaCpng/image44.png} marginal tıǵızlıq funkciyaların tabıń; \includegraphics[width=0.15972in,height=0.24028in]{mediaCpng/image45.png} hám \includegraphics[width=0.15972in,height=0.2in]{mediaCpng/image46.png} tosınnanlıq shamalardı ǵárezsizlikke tekseriń: \includegraphics[width=3.13611in,height=0.53611in]{mediaCpng/image47.png}
 \\

\end{tabular}
\vspace{1cm}


\begin{tabular}{m{17cm}}
\textbf{67-variant}
\newline

\textbf{T1.} Tolıq itimallıq formulası (waqıyalardıń tolıq gruppası, dálilleniwi).
 \\
\textbf{T2.} Tiykarǵı аbsоlyut úzliksiz bólistiriliwler (nоrmаl bólistiriw, teń ólshewli bólistiriw, kórsetkishli bólistiriw). 
 \\
\textbf{A1.} Eki oyın kubigi taslanǵanda túsken ochkolardıń qosındısı 7 ge teń bolıw itimallıǵın tabıń.
 \\
\textbf{A2.} Firmada 7 erkek hám 3 hayal jumısshı isleydi. Tosınnan 3 jumısshı ajıratılıp alındı. Ajıratılıp alınǵan jumısshılardıń barlıǵı erkekler bolıw itimallıǵın tabıń.
 \\
\textbf{A3.} $\left[ 0,1 \right]$ kesindiden tosınnan eki noqat tańlanadı. Olardıń koordinataları qosındısı, koordinataları kóbeymesi eki esesinen kóp bolıw itimallıǵın tabıń.
 \\
\textbf{B1.} \includegraphics[width=0.15972in,height=0.23958in]{mediaBpng/image49.png} úzliksiz tosınnanlıq shamanıń tıǵızlıq funkciyaları berilgen. Olarǵa sáykes \includegraphics[width=0.15972in,height=0.19653in]{mediaBpng/image50.png} tosınnanlıq shamanıń \includegraphics[width=0.50278in,height=0.30069in]{mediaBpng/image51.png} tıǵızlıq funkciyasın tabıń. \includegraphics[width=2.57083in,height=0.84028in]{mediaBpng/image86.png} \includegraphics[width=0.57639in,height=0.28194in]{mediaBpng/image87.png}
 \\
\textbf{B2.} Eger \includegraphics[width=0.36181in,height=0.29444in]{mediaBpng/image1.png} ǵárezsiz tosınnanlıq shamalar izbe-izliginiń bólistiriliw nızamları
\includegraphics[width=2.62569in,height=0.49097in]{mediaBpng/image37.png} \includegraphics[width=1.55208in,height=0.47847in]{mediaBpng/image38.png} \includegraphics[width=0.77292in,height=0.25764in]{mediaBpng/image39.png}
bolsa, onda bul izbe-izlik úlken sanlar nızamına boysınama?
 \\
\textbf{B3.} Eger \includegraphics[width=0.36181in,height=0.29444in]{mediaBpng/image1.png} ǵárezsiz tosınnanlıq shamalar izbe-izliginiń bólistiriliw nızamları
\includegraphics[width=2.53958in,height=0.49097in]{mediaBpng/image34.png} \includegraphics[width=1.19028in,height=0.47847in]{mediaBpng/image35.png} \includegraphics[width=0.77292in,height=0.25764in]{mediaBpng/image33.png}
bolsa, onda bul izbe-izlik úlken sanlar nızamına boysınama?
 \\
\textbf{C1.} Eger ǵárezsiz \includegraphics[width=0.15208in,height=0.24028in]{mediaCpng/image4.png} hám \includegraphics[width=0.15208in,height=0.19167in]{mediaCpng/image5.png} úzliksiz tosınnanlıq shamalar sáykes túrde, \includegraphics[width=0.38403in,height=0.24028in]{mediaCpng/image24.png} hám \includegraphics[width=0.41597in,height=0.24028in]{mediaCpng/image25.png} aralıqlarda teń ólshemli bólistirilgen bolsa, onda \includegraphics[width=0.44028in,height=0.24028in]{mediaCpng/image14.png} tosınnanlıq shamanıń tıǵızlıq funkciyasın tabıń.
 \\
\textbf{C2.} Eger \(\xi\) tosınnanlıq shama standart Koshi bólistiriliwine iye bolsa, onda \(M\min\left( |\xi|,1 \right)\) mánisin tabıń.
 \\
\textbf{C3.} Eger ǵárezsiz \includegraphics[width=0.15208in,height=0.24028in]{mediaCpng/image4.png} hám \includegraphics[width=0.15208in,height=0.19167in]{mediaCpng/image5.png} úzliksiz tosınnanlıq shamalar sáykes túrde, \includegraphics[width=0.4in,height=0.25625in]{mediaCpng/image6.png} hám \includegraphics[width=0.44028in,height=0.25625in]{mediaCpng/image7.png} parametrli normal nızam boyınsha bólistirilgen bolsa, onda \includegraphics[width=0.72778in,height=0.24028in]{mediaCpng/image8.png} tosınnanlıq shamanıń tıǵızlıq funkciyasın tabıń.
 \\

\end{tabular}
\vspace{1cm}


\begin{tabular}{m{17cm}}
\textbf{68-variant}
\newline

\textbf{T1.} Tosınnanlı waqıya (elementar waqıyalar keńisligi, waqıyalar ústinde ámeller).
 \\
\textbf{T2.} Tosınnanlı shamanıń matematikalıq kútiliwi. (anıqlaması, qásiyetleri)
 \\
\textbf{A1.} Úsh qutınıń hárbirinde $n$ dana aq ($n\ge 2$) hám $m$ dana qara sharlar bar. Birinshi qutıdan ekinshi qutıǵa tosınnan eki shar, ekinshi qutıdan úshinshi qutıǵa tosınnan bir shar salındı. Keyin, úshinshi qutıdan tosınnan bir shar alındı. а) Bul shardıń aq bolıw itimallıǵın tabıń. b) Sol shar aq bolsa, birinshi qutıdan alınǵan sharlardıń aq bolıw itimallıǵın tabıń.
 \\
\textbf{A2.} Hárbiriniń júzege asıw itimallıǵı $p$ ǵa teń bolǵan 10 dana Bernulli tájiriybesi ótkerilgende, tómendegi waqıyalardıń itimallıqların tabıń: Sátliler sanı 2 dana, sonıń menen birge, olardıń barlıǵı tájiriybelerdiń birinshi yarımında ámelge asıwı.
 \\
\textbf{A3.} “Kim millioner bolıwdı qáleydi” oyınında individual oyınshınıń 1000 dollar utıp alıwı itimallıǵı 0,3 ke, al 32000 dollar utıw itimallıǵı bolsa 0,01 ge teń. Bul oyında 300 oyınshı qatnasqan bolsa, tómendegi waqıyalardıń júzege asıwı itimallıqların tabıń: a) 80 nen 100 ge shekem oyınshı 1000 dollar utıwı; b) tórtten kóp bolmaǵan oyınshı 32000 dollar utıwı.
 \\
\textbf{B1.} $\xi$ tosınnanlı shamanıń \emph{f}(\emph{x}) tıǵızlıq funkciyasi berilgen bolsin. Tómendegilerdi esaplań: a) C; b) \emph{F}(\emph{x}); c) M$\xi$; d) D$\xi$; e) \emph{f}(\emph{x}) hám \emph{F}(\emph{x}) grafiklarin sızıń.\(f(x) = \left\{ \begin{matrix}
\ \ \ \ \ \ 0,\ \ \ \ \ x \notin (0,\ \pi/2)\ \  \\
C\sin x,\ \ \ \ \ x \in (0,\ \pi/2)\ \ 
\end{matrix} \right.\ \)
 \\
\textbf{B2.} \includegraphics[width=0.15972in,height=0.23958in]{mediaBpng/image49.png} úzliksiz tosınnanlıq shamanıń tıǵızlıq funkciyaları berilgen. Olarǵa sáykes \includegraphics[width=0.15972in,height=0.19653in]{mediaBpng/image50.png} tosınnanlıq shamanıń \includegraphics[width=0.50278in,height=0.30069in]{mediaBpng/image51.png} tıǵızlıq funkciyasın tabıń. \includegraphics[width=2.62569in,height=0.65in]{mediaBpng/image84.png} \includegraphics[width=0.66875in,height=0.53403in]{mediaBpng/image85.png}
 \\
\textbf{B3.} Tosınnanlı \(\xi\) shamasınıń bólistiriw tıǵızlıǵı \(f(x) = \frac{1}{2}e^{- |x|}\) bolsa, usı shamanıń xarakteristikalıq funkciyasın tabıń.
 \\
\textbf{C1.} Eger \(\xi_{1},\xi_{2}...,\xi_{n}\) ǵárezsiz birdey bólistirilgen tosınnanlıq shamalar standart normal bólistirilgen bolsa, onda \(\xi_{1}^{2} + \xi_{2}^{2} + ...\  + \xi_{n}^{2}\) tosınnanlıq shamanıń tıǵızlıq funkciyasın tabıń.
 \\
\textbf{C2.} Eger \(\left\{ \xi_{n} \right\}\) diskret tosınnanlıq shamalar izbe-izliginiń bólistiriliw nızamları\(P\left\{ \xi_{n} = 1 \right\} = P\left\{ \xi_{n} = - 1 \right\} = \frac{1}{2} - \frac{1}{n},\) \(P\left\{ \xi_{n} = 0 \right\} = \frac{2}{n}\) bolsa, onda \(\xi_{n}\overset{d}{\rightarrow}\xi\) bolatuǵın \(\xi\) tosınnanlıq shamanıń bólistiriw funkciyasın tabıń.
 \\
\textbf{C3.} Ortasha mánis vektorı \(\left( m_{1},m_{2} \right)\) hám kovariaciyalıq matricası\(K = \begin{pmatrix}
\sigma_{1}^{2} & r\sigma_{1}\sigma_{2} \\
r\sigma_{1}\sigma_{2} & \sigma_{2}^{2}
\end{pmatrix},\ \ \sigma_{1},\ \sigma_{2} > 0,\ \ |r|\  < 1\) bolǵan normal bólistirilgen \(\left( \xi_{1},\xi_{2} \right)\) tosınnanlıq vektordıń tıǵızlıq funkciyasın tabıń.
 \\

\end{tabular}
\vspace{1cm}


\begin{tabular}{m{17cm}}
\textbf{69-variant}
\newline

\textbf{T1.} Shártli itimallıq (anıqlaması, kóbеytiw tеorеması).
 \\
\textbf{T2.} Tiykarǵı diskret bólistiriliwler (Binomial, Puasson hám geometriyalıq bólistiriliwler).
 \\
\textbf{A1.} Birdey úsh qutı berilgen. Birinshi qutıda $a$ dana aq hám $b$ dana qara sharlar, ekinshi qutıda $c$ dana aq hám $d$ dana qara sharlar; úshinshi qutıda bolsa, tek aq sharlar bar. Qutılardan birewi tosınnan tańlanıp, onnan bir shar alındı. a) Usı alınǵan shardıń aq bolıw itimallıǵın tabıń. b) Alınǵan shar aq bolsa, sol shardıń birinshi qutıǵa tiyisli bolıw itimallıǵın tabıń.
 \\
\textbf{A2.} Hárbiriniń júzege asıw itimallıǵı $p$ ǵa teń bolǵan 10 dana Bernulli tájiriybesi ótkerilgende, tómendegi waqıyalardıń itimallıqların tabıń: Sátsizler sanı 2 den artıq, biraq 5 den kem.
 \\
\textbf{A3.} Eki oyın kubigi taslanǵanda túsken ochkolardıń ayırması 2 ge teń bolıw itimallıǵın tabıń.
 \\
\textbf{B1.} \emph{R} radiuslı dóńgelek ishinen vertikal xordalar júrgiziledi. Tosınnan alınǵan xordanıń radiustan kishi bolıwı itimallıǵın tabıń.
 \\
\textbf{B2.} \(P\left\{ \xi = m \right\} = pq^{m},\ \ \ \ \ \ \ \ m = 0,\ \ 1,\ \ 2,\ \ \ldots\). Usı tosınnanlı $\xi$ shamasınıń matematikalıq kútiliwin tabıń.
 \\
\textbf{B3.} Eger \includegraphics[width=0.36181in,height=0.29444in]{mediaBpng/image1.png} ǵárezsiz tosınnanlıq shamalar izbe-izliginiń bólistiriliw funkciyaları
\includegraphics[width=1.79167in,height=0.52153in]{mediaBpng/image48.png}
kórinislerinde bolsa, onda bul izbe-izlik úlken sanlar nızamına boysınama?
 \\
\textbf{C1.} Tómende \includegraphics[width=0.46389in,height=0.25625in]{mediaCpng/image42.png} úzliksiz tosınnanlıq vektorlardıń tıǵızlıq funkciyaları berilgen. Olardıń \includegraphics[width=0.48819in,height=0.29583in]{mediaCpng/image43.png} hám \includegraphics[width=0.50417in,height=0.29583in]{mediaCpng/image44.png} marginal tıǵızlıq funkciyaların tabıń; \includegraphics[width=0.15972in,height=0.24028in]{mediaCpng/image45.png} hám \includegraphics[width=0.15972in,height=0.2in]{mediaCpng/image46.png} tosınnanlıq shamalardı ǵárezsizlikke tekseriń: \includegraphics[width=2.52778in,height=0.59167in]{mediaCpng/image55.png}
 \\
\textbf{C2.} Eger \(\xi_{1}\) hám \(\xi_{2}\) sáykes túrde \(\lambda_{1}\) hám \(\lambda_{2}\) parametrli Puasson bólistiriliwine iye bolǵan ǵárezsiz tosınnanlıq shamalar bolsa, onda \(\xi_{1} + \xi_{2}\) tosınnanlıq shamanıń bólistiriliwin tabıń.
 \\
\textbf{C3.} Tómende \includegraphics[width=0.46389in,height=0.25625in]{mediaCpng/image42.png} úzliksiz tosınnanlıq vektorlardıń tıǵızlıq funkciyaları berilgen. Olardıń \includegraphics[width=0.48819in,height=0.29583in]{mediaCpng/image43.png} hám \includegraphics[width=0.50417in,height=0.29583in]{mediaCpng/image44.png} marginal tıǵızlıq funkciyaların tabıń; \includegraphics[width=0.15972in,height=0.24028in]{mediaCpng/image45.png} hám \includegraphics[width=0.15972in,height=0.2in]{mediaCpng/image46.png} tosınnanlıq shamalardı ǵárezsizlikke tekseriń: \includegraphics[width=2.75208in,height=0.67986in]{mediaCpng/image52.png}
 \\

\end{tabular}
\vspace{1cm}


\begin{tabular}{m{17cm}}
\textbf{70-variant}
\newline

\textbf{T1.} Itimallıqlar teoriyası aksiomaları (ólshewli keńislik, itimallıq keńisligi).
 \\
\textbf{T2.} Tosınnanlı shamanıń dispersiyası (anıqlaması, qásiyetleri).
 \\
\textbf{A1.} Qutıda 70 dana joqarı sapalı hám 10 dana tómen sapalı detallar bar. Qutıdan tosınnan alınǵan 6 dana detaldıń ishinde tómen sapalı detaldıń joq bolıw itimallıǵın tabıń.
 \\
\textbf{A2.} $\left( 0,2 \right)$ intervaldan tosınnan $x$ hám $y$ noqat tańlanadı. Olar ushın $xy\le 1$ hám $\frac{y}{x}\le 2$ bolıw itimallıǵın tabıń.
 \\
\textbf{A3.} Televizion kapital showda individual oyınshınıń 2000 dollar utıp alıwı itimallıǵı 0,4 ke, al 33000 dollar utıp alıwı itimallıǵı bolsa 0,02 ke teń. Bul oyında 400 oyınshı qatnasqan bolsa, tómendegi waqıyalardıń júzege asıwı itimallıqların tabıń: a) 170 ten 190 ǵa shekem oyınshı 2000 dollar utıwı; b) úshten kóp bolmaǵan oyınshı 33 000 dollar utıwı.
 \\
\textbf{B1.} Qutıda 4 aq hám 5 qara sharlar bar. Qutıdan izbe-iz 2 shar alınadi. Alinǵan 2 shardiń birinshisi aq ekinshisi qara shar boliw itimallıǵın tabıń.
 \\
\textbf{B2.} \includegraphics[width=0.15972in,height=0.23958in]{mediaBpng/image49.png} úzliksiz tosınnanlıq shamanıń tıǵızlıq funkciyaları berilgen. Olarǵa sáykes \includegraphics[width=0.15972in,height=0.19653in]{mediaBpng/image50.png} tosınnanlıq shamanıń \includegraphics[width=0.50278in,height=0.30069in]{mediaBpng/image51.png} tıǵızlıq funkciyasın tabıń. \includegraphics[width=2.52778in,height=1.15347in]{mediaBpng/image92.png} \includegraphics[width=0.57639in,height=0.28194in]{mediaBpng/image93.png}
 \\
\textbf{B3.} $\xi$ tosınnanlı shamanıń \emph{f}(\emph{x}) tıǵızlıq funkciyasi berilgen bolsin. Tómendegilerdi esaplań: a) C; b) \emph{F}(\emph{x}); c) M$\xi$; d) D$\xi$; e) \emph{f}(\emph{x}) hám \emph{F}(\emph{x}) grafiklarin sızıń.\(f(x) = \left\{ \begin{matrix}
C\sqrt[3]{1 - x},\ \ \ \ x \in \lbrack 0,1\rbrack, \\
\ \ \ \ \ \ \ \ 0,\ \ \ \ \ \ \ \ \ \ x \notin \lbrack 0,1\rbrack.\ \ 
\end{matrix} \right.\ \)
 \\
\textbf{C1.} Meyli, \(\xi_{1},...,\xi_{n}\) tosınnanlıq shamalar ǵárezsiz hám \(\lbrack a,b\rbrack\) aralıqta teń ólshemli bólistirilgen bolıp, \(\eta_{1} = \max\left( \xi_{1},...,\xi_{n} \right)\) hám \(\eta_{2} = \min\left( \xi_{1},...,\xi_{n} \right)\) bolsın. Onda \(\left( \eta_{1},\eta_{2} \right)\) tosınnanlıq vektordıń kovariaciyasın tabıń.
 \\
\textbf{C2.} Eger ǵárezsiz \includegraphics[width=0.15208in,height=0.24028in]{mediaCpng/image4.png} hám \includegraphics[width=0.15208in,height=0.19167in]{mediaCpng/image5.png} úzliksiz tosınnanlıq shamalardıń hárbiri \includegraphics[width=0.41597in,height=0.24028in]{mediaCpng/image26.png} aralıqta teń ólshemli bólistirilgen bolsa, onda \includegraphics[width=0.35972in,height=0.27222in]{mediaCpng/image27.png} tosınnanlıq shamanıń tıǵızlıq funkciyasın tabıń.
 \\
\textbf{C3.} Eger \(\left\{ \xi_{n} \right\}\) ǵárezsiz hám \(\mathbf{\lbrack 0,1\rbrack}\) aralıqta teń ólshemli bólistirilgen tosınnanlıq shamalar izbe-izligi bolsa, onda \(\left\{ \mathbf{\xi}_{\mathbf{(}\mathbf{n}\mathbf{)}}\mathbf{=}\mathbf{\max}\mathbf{\{}\mathbf{\xi}_{\mathbf{1}}\mathbf{,...,}\mathbf{\xi}_{\mathbf{n}}\mathbf{\}} \right\}\) izbe-izlik 1 ge itimallıq boyınsha jıynaqlılıǵın kórsetiń.
 \\

\end{tabular}
\vspace{1cm}


\begin{tabular}{m{17cm}}
\textbf{71-variant}
\newline

\textbf{T1.} Bernulli sxemаsı ushın limit teоremаlаr (Muavr-Laplas integrallıq teoreması, qásiyetleri).
 \\
\textbf{T2.} Bólistiriw funkciyası (anıqlaması, tiykarǵı qásiyetleri).
 \\
\textbf{A1.} Hárbiriniń júzege asıw itimallıǵı $p$ ǵa teń bolǵan 10 dana Bernulli tájiriybesi ótkerilgende, tómendegi waqıyalardıń itimallıqların tabıń: Sátliler sanı, sátsizler sanınan tórtke kem.
 \\
\textbf{A2.} Qutıda 30 dana birdey sharlar bolıp, olardıń 20 danası qızıl hám 10 danası kók reńdegi sharlar. Tosınnan alınǵan 3 dana shardıń 2 danası qızıl shar bolıw itimallıǵın tabıń.
 \\
\textbf{A3.} Oyın kubigi taslanǵanda 5 ochkonıń túsiw itimallıǵın tabıń.
 \\
\textbf{B1.} Radiusı \(r\ (2r < a)\) bolǵan tiyin tosınnanlı túrde tárepi a bolǵan kvadratlarǵa bólingen stolǵa taslandı. Taslanǵan tiyin kvadrattıń bazı bir tárepin kesip ótpewi itimallıǵın tabıń.
 \\
\textbf{B2.} 
Bólistiriw funkciyasi berilgen: \(\mathbf{F}\mathbf{(}\mathbf{x}\mathbf{)}\mathbf{=}\left\{ \begin{matrix}
\mathbf{0,}\mathbf{\ \ \ \ \ \ \ \ \ \ \ \ \ \ \ \ \ \ \ \ \ \ \ \ \ \ \ \ \ \ \ \ \ \ \ \ \ \ \ \ \ x \leq - a} \\
\frac{\mathbf{1}}{\mathbf{2}}\mathbf{+}\frac{\mathbf{1}}{\mathbf{\pi}}\mathbf{\arcsin}\frac{\mathbf{x}}{\mathbf{a}}\mathbf{,}\mathbf{\ \ \ \ \  - a < x < a}\mathbf{,} \\
\mathbf{1,}\mathbf{\ \ \ \ \ \ \ \ \ \ \ \ \ \ \ \ \ \ \ \ \ \ \ \ \ \ \ \ \ \ \ \ \ \ \ \ \ \ \ \ \ \ \ \ \ x \geq a}
\end{matrix} \right.\ \) a)bólistiriw tiǵizliǵi \(f(x)\  = ?\ \ \ \ \ \ \ \)b) \(\mathbf{P}\left\{ \mathbf{-}\frac{\mathbf{a}}{\mathbf{2}}\mathbf{< \xi <}\frac{\mathbf{a}}{\mathbf{2}} \right\}\mathbf{=}\mathbf{?}\)
 \\
\textbf{B3.} $\xi$ tosınnanlı shamanıń \emph{f}(\emph{x}) tıǵızlıq funkciyasi berilgen bolsin. Tómendegilerdi esaplań: a) C; b) \emph{F}(\emph{x}); c) M$\xi$; d) D$\xi$; e) \emph{f}(\emph{x}) hám \emph{F}(\emph{x}) grafiklarin sızıń.\(f(x) = \left\{ \begin{matrix}
C/x,\ \ \ \ x \in \lbrack 1/e,e\rbrack, \\
\ \ \ \ 0,\ \ \ \ \ x \notin \lbrack 1/e,e\rbrack.\ \ 
\end{matrix} \right.\ \)
 \\
\textbf{C1.} Tómende \includegraphics[width=0.46389in,height=0.25625in]{mediaCpng/image42.png} úzliksiz tosınnanlıq vektorlardıń tıǵızlıq funkciyaları berilgen. Olardıń \includegraphics[width=0.48819in,height=0.29583in]{mediaCpng/image43.png} hám \includegraphics[width=0.50417in,height=0.29583in]{mediaCpng/image44.png} marginal tıǵızlıq funkciyaların tabıń; \includegraphics[width=0.15972in,height=0.24028in]{mediaCpng/image45.png} hám \includegraphics[width=0.15972in,height=0.2in]{mediaCpng/image46.png} tosınnanlıq shamalardı ǵárezsizlikke tekseriń: \includegraphics[width=3.07986in,height=0.67986in]{mediaCpng/image59.png}
 \\
\textbf{C2.} Eger ǵárezsiz \includegraphics[width=0.15208in,height=0.24028in]{mediaCpng/image4.png} hám \includegraphics[width=0.15208in,height=0.19167in]{mediaCpng/image5.png} úzliksiz tosınnanlıq shamalardıń hárbiri \includegraphics[width=0.19167in,height=0.16806in]{mediaCpng/image32.png} parametrli kórsetkishli nızam boyınsha bólistirilgen bolsa, onda \includegraphics[width=0.44028in,height=0.24028in]{mediaCpng/image29.png} tosınnanlıq shamanıń tıǵızlıq funkciyasın tabıń.
 \\
\textbf{C3.} Eger \(\chi_{1}^{2}\) hám \(\chi_{2}^{2}\) ǵárezsiz tosınnanlıq shamalar bolıp, \(\chi_{1}^{2}\sim\chi^{2}(n_{1})\) hám \(\chi_{2}^{2}\sim\chi^{2}(n_{2})\) bolsa, onda \(\frac{n_{2}}{n_{1}} \cdot \frac{\chi_{1}^{2}}{\chi_{2}^{2}}\) tosınnanlıq shamanıń tıǵızlıq funkciyasın tabıń.
 \\

\end{tabular}
\vspace{1cm}


\begin{tabular}{m{17cm}}
\textbf{72-variant}
\newline

\textbf{T1.} Ǵárezsiz tájiriybelerdiń Bernulli sxeması (binоmiаl bólistiriliw, qásiyetleri).
 \\
\textbf{T2.} Kompoziciyalıq formulalar \\
\textbf{A1.} Lotereyada úlken hám kishi utıslar oynaladı. Lotereya biletinde úlken utıs shıǵıw itimallıǵı 0,001 ge, al kishisi bolsa 0,01 ge teń. Jámi 1000 dana bilet satıp alınǵanda: a) eki úlken utıslı; b) kishi utıslar 5 ten 15 ke shekem bolıwı waqıyaları itimallıqların tabıń.
 \\
\textbf{A2.} Mikrosxemalardıń 10% i nuqsanlı jaǵdayda bolıp, olar tekseriwden ótkerildi. Ápiwayılastırılǵan tekseriw sınaǵı ótkerildi. Bul tekseriw tómendegishe itimallıqta qátelikke jol qoyadı, yaǵnıy 0,95 itimallıq penen nuqsanlı mikrosxemanı nuqsanlı dep tabadı hám 0,03 itimallıq penen nuqsansız mikrosxemanı nuqsanlı dep tabadı. a) Tekseriwden ótkerilgen mikrosxemanıń nuqsanlı dep tabılıw itimallıǵın tabıń. b) Bul mikrosxemanıń negizinde nuqsansız bolıwı itimallıǵı qanday?
 \\
\textbf{A3.} ${{x}^{2}}+2px+q=0$ kvadrat teńlemede $p$ hám $q$ koefficientler $\left[ -1,1 \right]$ kesindiden tosınnan tańlanadı. Kvadrat teńlemeniń haqıyqıy túbirlerge iye bolıw itimallıǵın tabıń.
 \\
\textbf{B1.} \includegraphics[width=0.15972in,height=0.23958in]{mediaBpng/image49.png} úzliksiz tosınnanlıq shamanıń tıǵızlıq funkciyaları berilgen. Olarǵa sáykes \includegraphics[width=0.15972in,height=0.19653in]{mediaBpng/image50.png} tosınnanlıq shamanıń \includegraphics[width=0.50278in,height=0.30069in]{mediaBpng/image51.png} tıǵızlıq funkciyasın tabıń. \includegraphics[width=2.325in,height=0.84028in]{mediaBpng/image68.png} \includegraphics[width=0.66875in,height=0.23958in]{mediaBpng/image69.png}
 \\
\textbf{B2.} Eger \includegraphics[width=0.36181in,height=0.29444in]{mediaBpng/image1.png} ǵárezsiz tosınnanlıq shamalar izbe-izliginiń bólistiriliw nızamları
\includegraphics[width=2.53958in,height=0.47847in]{mediaBpng/image25.png} \includegraphics[width=0.75486in,height=0.23958in]{mediaBpng/image9.png}
bolsa, onda bul izbe-izlik úlken sanlar nızamına boysınama?
 \\
\textbf{B3.} Eger \(\xi\) tosınnanlı shama \(\lbrack a,b\rbrack\) parametrli teń ólshewli bólistiriwine iye bolsa, onda onıń xarakteristikalıq funkciyası tabılsın.
 \\
\textbf{C1.} Eger \(\left\{ \xi_{n} \right\}\) ǵárezsiz birdey bólistirilgen tosınnanlıq shamalar izbe-izligi bolıp, onıń bólistiriw funkciyası \(F_{\xi_{1}}(x) = \left\{ \begin{matrix}
\ 1 - e^{\lambda - x},\ \ eger\ \ x \geq \lambda, \\
 \\
\ \ \ \ \ \ 0,\ \ \ \ \ \ \ \ \ \ \ eger\ \ x < \lambda
\end{matrix} \right.\ \) bolsa, onda \(\left\{ \eta_{n} \right\} = \left\{ min(\xi_{1},...,\xi_{n}) \right\}\) izbe-izliktiń \(\mathbf{\lambda}\) ǵa bir itimallıq penen jıynaqlılıǵın kórsetiń.
 \\
\textbf{C2.} Eger \(\xi\) tosınnanlıq shama \(\lbrack 0,\ \pi\rbrack\) aralıqta teń ólshewli bólistirilgen bolsa, onda \(M\sin\xi,\) \(D\sin\xi\) hám \(M\cos\xi,\) \(D\cos\xi\) mánislerin tabıń.
 \\
\textbf{C3.} Hárqanday \(\varphi_{\xi}(t)\) xarakteristikalıq funkciya ushın \(t \in R\) de \(1 - Re\varphi_{\xi}(2t) \leq 4\left( 1 - Re\varphi_{\xi}(t) \right)\) ekenligin dálilleń.
 \\

\end{tabular}
\vspace{1cm}


\begin{tabular}{m{17cm}}
\textbf{73-variant}
\newline

\textbf{T1.} Waqıyalar algebrası ($\sigma$-algebra, minimal $\sigma$-algebra).
 \\
\textbf{T2.} Tıǵızlıq funkciyası (anıqlaması, tiykarǵıqásiyetleri).
 \\
\textbf{A1.} 150 dana buyımnan ibarat partiyada 5 dana buyım jaramsız. Partiyadan tosınnan 12 dana buyım alınadı. Usı alınǵan 12 dana buyımnıń ishinde 3 dana buyımnıń jaramsız bolıw itimallıǵın tabıń. 
 \\
\textbf{A2.} $\left[ 0,2 \right]$ kesindiden tosınnan $x$ hám $y$ noqat tańlanadı. Olar ushın ${{x}^{2}}\le 4y\le 4x$ bolıw itimallıǵın tabıń.
 \\
\textbf{A3.} Úsh oyın kubigi taslanǵanda túsken ochkolardıń birdey bolıw itimallıǵın tabıń.
 \\
\textbf{B1.} \includegraphics[width=0.15972in,height=0.23958in]{mediaBpng/image49.png} úzliksiz tosınnanlıq shamanıń tıǵızlıq funkciyaları berilgen. Olarǵa sáykes \includegraphics[width=0.15972in,height=0.19653in]{mediaBpng/image50.png} tosınnanlıq shamanıń \includegraphics[width=0.50278in,height=0.30069in]{mediaBpng/image51.png} tıǵızlıq funkciyasın tabıń. \includegraphics[width=2.40486in,height=0.84028in]{mediaBpng/image54.png} \includegraphics[width=0.85903in,height=0.23958in]{mediaBpng/image55.png}
 \\
\textbf{B2.} Eger \includegraphics[width=0.36181in,height=0.29444in]{mediaBpng/image1.png} ǵárezsiz hám birdey bólistirilgen tosınnanlıq shamalar izbe-izligi \includegraphics[width=0.57639in,height=0.27639in]{mediaBpng/image44.png} parametrli kórsetkishli bólistiriliwine bolsa, onda bul izbe-izlik úlken sanlar nızamına boysınama?
 \\
\textbf{B3.} Abonent telefon nomerin terip atırıp, aqırǵı úsh cifrdi umıtıp qaldı hám bul nomerlerdiń hár túrli ekenligin eslep olardı táwekeline terdi. Kerekli nomerler terilgen bolıwı itimallıǵın tabıń.
 \\
\textbf{C1.} Eger \(\left\{ \xi_{n} \right\}\) tosınnanlıq shamalar izbe-izligi \(\mathbf{\xi}_{\mathbf{n}}\overset{\mathbf{P}}{\rightarrow}\mathbf{\xi}\) hám \(\mathbf{\xi}_{\mathbf{n}}\overset{\mathbf{P}}{\rightarrow}\mathbf{\eta}\) bolsa, onda \(\mathbf{P}\left( \mathbf{\xi = \eta} \right)\mathbf{=}\mathbf{1}\) qatnasın dálilleń.
 \\
\textbf{C2.} Eger ǵárezsiz \includegraphics[width=0.15208in,height=0.24028in]{mediaCpng/image4.png} hám \includegraphics[width=0.15208in,height=0.19167in]{mediaCpng/image5.png} úzliksiz tosınnanlıq shamalardıń hárbiri {[}0,1{]} aralıqta teń ólshemli bólistirilgen bolsa, \includegraphics[width=0.47986in,height=0.52014in]{mediaCpng/image40.png} tosınnanlıq shamanıń tıǵızlıq funkciyasın tabıń.
 \\
\textbf{C3.} Eger \(\left( \xi_{1},\xi_{2} \right)\) tosınnanlıq vektordıń bólistiriw funkciyası \(F(x,y) = \left\{ \begin{matrix}
\left( 1 - 2^{- x^{2}} \right)\left( 1 - 2^{- 2y^{2}} \right),\ \ eger\ \ x \geq 0,\ y \geq 0, \\
 \\
 \\
\ \ \ \ \ \ \ \ \ \ \ \ \ \ 0,\ \ \ \ \ \ \ \ \ \ \ \ \ \ \ \ \ \ \ \ \ \ \ basqa\ hallarda
\end{matrix} \right.\ \) bolsa, onda \(F\left( x/\xi_{2} < y \right)\) hám \(F\left( y/\xi_{1} < x \right)\) shártli bólistiriw funkciyaların tabıń. Sonıń menen birge, \(\xi_{1}\) hám \(\xi_{2}\) tosınnanlıq shamalardı ǵárezsizlike tekseriń.
 \\

\end{tabular}
\vspace{1cm}


\begin{tabular}{m{17cm}}
\textbf{74-variant}
\newline

\textbf{T1.} Bayеs formulası (gipotezalar teoreması, dálilleniwi).
 \\
\textbf{T2.} 
Úlken sanlar nızamı (anıqlaması, Chebishev teoreması).
 \\
\textbf{A1.} Dúkan 1000 dana televizor hám 1000 dana radio satıp aldı. Hárbir televizordıń defektli bolıwı itimallıǵı 0,005 ke hám hárbir radionıń defektli bolıwı itimallıǵı 0,04 ke teń. Usı sawdada tómendegi waqıyalardıń júzege asıwı itimallıqların tabıń: a) keminde tórt televizor defektli bolıwı; b) 35 ten 45 ke shekem radio defektli bolıwı.
 \\
\textbf{A2.} Hárbiriniń júzege asıw itimallıǵı $p$ ǵa teń bolǵan 10 dana Bernulli tájiriybesi ótkerilgende, tómendegi waqıyalardıń itimallıqların tabıń: Sátliler sanı, sátsizler sanınan ekige artıq.
 \\
\textbf{A3.} Oqıtıwshı matematika páninen shegaralıq bahalaw alıw ushın 50 soraw tayarlaǵan. Olardıń ishinde differencial esabınan 20 soraw, integral esabınan 18 soraw hám qatarlar teoriyasınan 12 soraw bar. Student differencial esabınan 18 sorawǵa, integral esabınan 15 sorawǵa hám qatarlar teoriyasınan 10 sorawǵa juwap bere aladı. a) Studentke berilgen birinshi sorawǵa juwap beriwi itimallıǵın tabıń. b) Eger student sol sorawǵa durıs juwap bergen bolsa, bul sorawdıń integral esabınan bolıwı itimallıǵın tabıń.
 \\
\textbf{B1.} $\xi$ tosınnanlı shamanıń \emph{f}(\emph{x}) tıǵızlıq funkciyasi berilgen bolsin. Tómendegilerdi esaplań: a) C; b) \emph{F}(\emph{x}); c) M$\xi$; d) D$\xi$; e) \emph{f}(\emph{x}) hám \emph{F}(\emph{x}) grafiklarin sızıń.\(f(x) = \left\{ \begin{matrix}
C\ln x,\ \ \ \ x \in \lbrack 1,e\rbrack, \\
\ \ \ \ 0,\ \ \ \ \ \ \ x \notin \lbrack 1,e\rbrack.\ \ 
\end{matrix} \right.\ \)
 \\
\textbf{B2.} $\xi$ tosınnanlı shamanıń \emph{f}(\emph{x}) tıǵızlıq funkciyasi berilgen bolsin. Tómendegilerdi esaplań: a) C; b) \emph{F}(\emph{x}); c) M$\xi$; d) D$\xi$; e) \emph{f}(\emph{x}) hám \emph{F}(\emph{x}) grafiklarin sızıń.\(f(x) = \left\{ \begin{matrix}
C\left( |x| + \frac{1}{4} \right),\ \ \ \ x \in \lbrack - 1,1\rbrack, \\
\ \ \ \ \ \ \ \ 0,\ \ \ \ \ \ \ \ \ \ \ \ \ \ \ x \notin \lbrack - 1,1\rbrack.\ \ 
\end{matrix} \right.\ \)
 \\
\textbf{B3.} \includegraphics[width=0.15972in,height=0.23958in]{mediaBpng/image49.png} úzliksiz tosınnanlıq shamanıń tıǵızlıq funkciyaları berilgen. Olarǵa sáykes \includegraphics[width=0.15972in,height=0.19653in]{mediaBpng/image50.png} tosınnanlıq shamanıń \includegraphics[width=0.50278in,height=0.30069in]{mediaBpng/image51.png} tıǵızlıq funkciyasın tabıń. \includegraphics[width=2.40486in,height=0.84028in]{mediaBpng/image58.png} \includegraphics[width=0.84028in,height=0.23958in]{mediaBpng/image59.png}
 \\
\textbf{C1.} Tómende \includegraphics[width=0.46389in,height=0.25625in]{mediaCpng/image42.png} úzliksiz tosınnanlıq vektorlardıń tıǵızlıq funkciyaları berilgen. Olardıń \includegraphics[width=0.48819in,height=0.29583in]{mediaCpng/image43.png} hám \includegraphics[width=0.50417in,height=0.29583in]{mediaCpng/image44.png} marginal tıǵızlıq funkciyaların tabıń; \includegraphics[width=0.15972in,height=0.24028in]{mediaCpng/image45.png} hám \includegraphics[width=0.15972in,height=0.2in]{mediaCpng/image46.png} tosınnanlıq shamalardı ǵárezsizlikke tekseriń: \includegraphics[width=2.64028in,height=0.33611in]{mediaCpng/image51.png}
 \\
\textbf{C2.} Eger \(\left\{ \xi_{n} \right\}\) ǵárezsiz tosınnanlıq shamalar izbe-izliginiń bólistiriw funkciyaları \(F_{n}(x) = \left\{ \begin{matrix}
\ 1 - \frac{1}{x + n},\ \ eger\ \ x > 0 \\
 \\
 \\
\ \ \ \ \ \ \ \ \ \ 0,\ \ \ \ \ \ \ \ \ \ \ eger\ \ x \leq 0
\end{matrix} \right.\ \) bolsa, onda bul izbe-izliktiń 0 ge itimallıq boyınsha jıynaqlılıǵın kórsetiń.
 \\
\textbf{C3.} Eger ǵárezsiz \includegraphics[width=0.15208in,height=0.24028in]{mediaCpng/image4.png} hám \includegraphics[width=0.15208in,height=0.19167in]{mediaCpng/image5.png} úzliksiz tosınnanlıq shamalardıń hárbiri \includegraphics[width=0.4in,height=0.24028in]{mediaCpng/image18.png} aralıqta teń ólshemli bólistirilgen bolsa, onda \includegraphics[width=0.44028in,height=0.24028in]{mediaCpng/image20.png} tosınnanlıq shamanıń tıǵızlıq funkciyasın tabıń.
 \\

\end{tabular}
\vspace{1cm}


\begin{tabular}{m{17cm}}
\textbf{75-variant}
\newline

\textbf{T1.} Bernulli sxemаsı ushın limit teоremаlаr (Puasson bólistiriliwi, qásiyetleri).
 \\
\textbf{T2.} Tosınnanlı shamanıń joqarı tártipli momentleri (baslanǵısh hám oraylıq momentleri, qásiyetleri).
 \\
\textbf{A1.} Berilgen $1,2,\ldots ,10$ sanlarınıń arasınan tosınnan bir san tańlandı. Meyli, bul san $m$ bolsın. Keyin, $1,2,\ldots ,m$ sanlarınıń arasınan tosınnan bir san tańlandı. a) Bul sannıń 8 ge teń bolıw itimallıǵın tabıń. b) Bul san 8 ge teń bolsa, $m=9$ bolıw itimallıǵın tabıń.
 \\
\textbf{A2.} Mashina jarısında 500 ekipaj qatnaspaqta. Hárbir ekipaj jarıstan texnikalıq nasazlıqlar sebepli 0,05 itimallıq penen, aydawshınıń keselligi sebepli bolsa 0,01 itimallıq penen shıǵıp ketiw múmkin. a) Aydawshınıń keselligi sebepli 5 ten artıq ekipaj jarıstan shıǵıp ketiwi itimallıǵın tabıń; b) 22 den 28 ge shekem ekipaj texnikalıq nasazlıqlar sebepli jarıstan shıǵıp ketiwi itimallıǵın tabıń.
 \\
\textbf{A3.} Hárbiriniń júzege asıw itimallıǵı $p$ ǵa teń bolǵan 10 dana Bernulli tájiriybesi ótkerilgende, tómendegi waqıyalardıń itimallıqların tabıń: Sátliler sanı 3 dana, sonıń menen birge, olardıń barlıǵı tájiriybelerdiń ekinshi yarımında ámelge asıwı.
 \\
\textbf{B1.} \(f(x) = C \cdot e^{- \frac{(x - m)^{2}}{7}}\) tıǵızlıq funkciyası bolıwı ushin \emph{C} nege teń bolıwı kerek?
 \\
\textbf{B2.} Hár biriniń uzınlıǵı a dan aspaytuǵın eki tosınnanlı alınǵan kesindiler qosındısı a dan úlken bolıwı itimallıǵı qanday?
 \\
\textbf{B3.} Eger \includegraphics[width=0.36181in,height=0.29444in]{mediaBpng/image1.png} ǵárezsiz tosınnanlıq shamalar izbe-izliginiń bólistiriliw nızamları
\includegraphics[width=2.58264in,height=0.49097in]{mediaBpng/image16.png} \includegraphics[width=1.59514in,height=0.47847in]{mediaBpng/image17.png} \includegraphics[width=0.75486in,height=0.23958in]{mediaBpng/image18.png}
bolsa, onda bul izbe-izlik úlken sanlar nızamına boysınama?
 \\
\textbf{C1.} Eger \(\xi\sim E(\lambda)\) bolsa, onda \(\xi\) tosınnanlıq shamanıń joqarı tártipli baslanǵısh momentlerin tabıń.
 \\
\textbf{C2.} Eger \(\left( \xi_{1},\xi_{2} \right)\) tosınnanlıq vektordıń bólistiriw funkciyası\(F(x,y) = \sin x \cdot \sin y,\ \ \ 0 \leq x \leq \frac{\pi}{2},\ \ 0 \leq y \leq \frac{\pi}{2}\) bolsa, onda \(\left( \xi_{1},\xi_{2} \right)\) tosınnanlıq noqattıń \(G:x_{1} = \frac{\pi}{6},\ \ x_{2} = \frac{\pi}{2};\ \ y_{1} = \frac{\pi}{4},\ \ y_{2} = \frac{\pi}{3}\) bolǵan tuwrımúyeshlikke túsiw itimallıǵın tabıń.
 \\
\textbf{C3.} Tómende \includegraphics[width=0.46389in,height=0.25625in]{mediaCpng/image42.png} úzliksiz tosınnanlıq vektorlardıń tıǵızlıq funkciyaları berilgen. Olardıń \includegraphics[width=0.48819in,height=0.29583in]{mediaCpng/image43.png} hám \includegraphics[width=0.50417in,height=0.29583in]{mediaCpng/image44.png} marginal tıǵızlıq funkciyaların tabıń; \includegraphics[width=0.15972in,height=0.24028in]{mediaCpng/image45.png} hám \includegraphics[width=0.15972in,height=0.2in]{mediaCpng/image46.png} tosınnanlıq shamalardı ǵárezsizlikke tekseriń: \includegraphics[width=3.47986in,height=0.6in]{mediaCpng/image70.png}
 \\

\end{tabular}
\vspace{1cm}


\begin{tabular}{m{17cm}}
\textbf{76-variant}
\newline

\textbf{T1.} Bernulli sxemаsı ushın limit teоremаlаr (Muavr-Laplas lokallıq teoreması, qásiyetleri).
 \\
\textbf{T2.} Oraylıq limit teorema (anıqlaması, ǵárezsiz birdey bólistirilgen tosınnanlı shamalar ushın).
 \\
\textbf{A1.} Jámi 10 bala hám 12 qız bolǵan studentler toparınan 5 student sorawnama ótkeriw ushın tosınnan tańlap alındı. Olar ishinde keminde bir student qız bolıw itimallıǵın tabıń.
 \\
\textbf{A2.} Úsh oyın kubigi taslanǵanda túsken ochkolardıń qosındısı 11 ǵe teń bolıw itimallıǵın tabıń.
 \\
\textbf{A3.} $\left[ 0,1 \right]$ kesindiden tosınnan eki noqat tańlanadı. Ekinshi noqattıń koordinatasınıń birinshi noqattıń koordinatasına qatnası 0,6 dan kishi bolıw itimallıǵın tabıń.
 \\
\textbf{B1.} Tosınnanlı $\xi$ shamasiniń bólistiriw tiǵizliǵi berilgen: \(f(x) = A \cdot e^{- 5|x|}\). a) \emph{A}=?
 \\
\textbf{B2.} Eger \includegraphics[width=0.36181in,height=0.29444in]{mediaBpng/image1.png} ǵárezsiz tosınnanlıq shamalar izbe-izliginiń bólistiriliw nızamları
\includegraphics[width=2.33125in,height=0.49097in]{mediaBpng/image28.png} \includegraphics[width=1.59514in,height=0.47847in]{mediaBpng/image29.png} \includegraphics[width=0.75486in,height=0.23958in]{mediaBpng/image30.png}
bolsa, onda bul izbe-izlik úlken sanlar nızamına boysınama?
 \\
\textbf{B3.} Eki teń ku`shli shaxmatshi shaxmat oynaǵanda 4 partiyadan 3 partiyani utıwı itimallig`i kóppe yamasa 8 partiyadan 5 partiyani utiw itimallıǵı kóp pe?
 \\
\textbf{C1.} Ortasha mánis vektorı \(\left( m_{1},m_{2} \right)\) hám kovariaciyalıq matricası\(K = \begin{pmatrix}
\sigma_{1}^{2} & r\sigma_{1}\sigma_{2} \\
r\sigma_{1}\sigma_{2} & \sigma_{2}^{2}
\end{pmatrix},\ \ \sigma_{1},\ \sigma_{2} > 0,\ \ |r|\  < 1\) bolǵan normal bólistirilgen \(\left( \xi_{1},\xi_{2} \right)\) tosınnanlıq vektordıń tıǵızlıq funkciyasın tabıń.
 \\
\textbf{C2.} Eger \(\left( \xi_{1},\xi_{2} \right)\) absolyut úziliksiz tosınnanlıq vektordıń tıǵızlıq funkciyası \(f(x,y) = \left\{ \begin{matrix}
Ce^{- x - y},\ eger\ \ x \geq 0,y \geq 0, \\
 \\
 \\
\ \ \ \ \ \ \ \ 0,\ \ \ \ \ basqa\ hallarda\ 
\end{matrix} \right.\ \) bolsa, onda \(F(x,y),\) \(F_{\xi_{1}}(x),\) \(F_{\xi_{2}}(y),\) \(f_{\xi_{1}}(x),\) \(f_{\xi_{2}}(y)\) hám \(P\left( \xi_{1} > 0,\xi_{2} < 1 \right)\) itimallıqtı tabıń. Sonıń menen birge, \(\xi_{1}\) hám \(\xi_{2}\) tosınnanlıq shamalardı ǵárezsizlikke tekseriń.
 \\
\textbf{C3.} Tómende \includegraphics[width=0.46389in,height=0.25625in]{mediaCpng/image42.png} úzliksiz tosınnanlıq vektorlardıń tıǵızlıq funkciyaları berilgen. Olardıń \includegraphics[width=0.48819in,height=0.29583in]{mediaCpng/image43.png} hám \includegraphics[width=0.50417in,height=0.29583in]{mediaCpng/image44.png} marginal tıǵızlıq funkciyaların tabıń; \includegraphics[width=0.15972in,height=0.24028in]{mediaCpng/image45.png} hám \includegraphics[width=0.15972in,height=0.2in]{mediaCpng/image46.png} tosınnanlıq shamalardı ǵárezsizlikke tekseriń: \includegraphics[width=3.20833in,height=0.84792in]{mediaCpng/image62.png}
 \\

\end{tabular}
\vspace{1cm}


\begin{tabular}{m{17cm}}
\textbf{77-variant}
\newline

\textbf{T1.} Itimallıq anıqlamaları (klassikalıq, geometriyalıq anıqlamaları).
 \\
\textbf{T2.} Xarakteristikalıq funkciyalar (anıqlaması, tiykarǵı qásiyetleri).
 \\
\textbf{A1.} Hárbiriniń júzege asıw itimallıǵı $p$ ǵa teń bolǵan 10 dana Bernulli tájiriybesi ótkerilgende, tómendegi waqıyalardıń itimallıqların tabıń: Sátliler sanı, sátsizler sanınan artıq.
 \\
\textbf{A2.} Berilgen $1,2,\ldots ,10$ sanlarınıń arasınan tosınnan bir san tańlandı. Meyli, bul san $m$ bolsın. Soń, $\left[ 0,m \right]$ kesindiden tosınnan $\xi $ noqat tańlandı. a) $\xi >8$ bolıw itimallıǵın tabıń. b) Eger $\xi >8$ bolsa, onda $m=9$ bolıw itimallıǵın tabıń.
 \\
\textbf{A3.} Oyın kubigi taslanǵanda taq ochkonıń túsiw itimallıǵın tabıń.
 \\
\textbf{B1.} \includegraphics[width=0.15972in,height=0.23958in]{mediaBpng/image49.png} úzliksiz tosınnanlıq shamanıń tıǵızlıq funkciyaları berilgen. Olarǵa sáykes \includegraphics[width=0.15972in,height=0.19653in]{mediaBpng/image50.png} tosınnanlıq shamanıń \includegraphics[width=0.50278in,height=0.30069in]{mediaBpng/image51.png} tıǵızlıq funkciyasın tabıń. \includegraphics[width=2.06111in,height=0.58264in]{mediaBpng/image77.png} \includegraphics[width=0.91389in,height=0.49097in]{mediaBpng/image78.png}
 \\
\textbf{B2.} $\xi$ tosınnanlı shamanıń \emph{f}(\emph{x}) tıǵızlıq funkciyasi berilgen bolsin. Tómendegilerdi esaplań: a) C; b) \emph{F}(\emph{x}); c) M$\xi$; d) D$\xi$; e) \emph{f}(\emph{x}) hám \emph{F}(\emph{x}) grafiklarin sızıń.\(f(x) = \left\{ \begin{matrix}
C\sqrt{1 - x},\ \ \ \ x \in \lbrack 0,1\rbrack, \\
\ \ \ \ \ \ \ \ 0,\ \ \ \ \ \ \ x \notin \lbrack 0,1\rbrack.\ \ 
\end{matrix} \right.\ \)
 \\
\textbf{B3.} Atiwshiniń bir márte nishanaǵa atqanda tiygiziw itimalliǵi 0,75 ke teń. Nishanaǵa 10 márte atqanda 8 mártesinde tiygiziw itimalliǵin tabiń.
 \\
\textbf{C1.} Eger ǵárezsiz \includegraphics[width=0.15208in,height=0.24028in]{mediaCpng/image4.png} hám \includegraphics[width=0.15208in,height=0.19167in]{mediaCpng/image5.png} úzliksiz tosınnanlıq shamalardıń tıǵızlıq fukciyaları sáykes túrde,
\includegraphics[width=1.54375in,height=0.47986in]{mediaCpng/image36.png} hám \includegraphics[width=1.58403in,height=0.47986in]{mediaCpng/image37.png}
bolsa, onda \includegraphics[width=0.44028in,height=0.24028in]{mediaCpng/image14.png} tosınnanlıq shamanıń tıǵızlıq funkciyasın tabıń.
 \\
\textbf{C2.} Eger \(\mathbf{\xi}_{\mathbf{n}}\overset{\mathbf{L}^{\mathbf{2}}}{\rightarrow}\mathbf{\xi}\) bolsa, onda \(n \rightarrow \infty\) de \(\mathbf{M}\mathbf{\xi}_{\mathbf{n}}\mathbf{\rightarrow M\xi}\) ekenligin kórsetiń.
 \\
\textbf{C3.} Tómende \includegraphics[width=0.46389in,height=0.25625in]{mediaCpng/image42.png} úzliksiz tosınnanlıq vektorlardıń tıǵızlıq funkciyaları berilgen. Olardıń \includegraphics[width=0.48819in,height=0.29583in]{mediaCpng/image43.png} hám \includegraphics[width=0.50417in,height=0.29583in]{mediaCpng/image44.png} marginal tıǵızlıq funkciyaların tabıń; \includegraphics[width=0.15972in,height=0.24028in]{mediaCpng/image45.png} hám \includegraphics[width=0.15972in,height=0.2in]{mediaCpng/image46.png} tosınnanlıq shamalardı ǵárezsizlikke tekseriń: \includegraphics[width=2.65625in,height=0.63194in]{mediaCpng/image57.png}
 \\

\end{tabular}
\vspace{1cm}


\begin{tabular}{m{17cm}}
\textbf{78-variant}
\newline

\textbf{T1.} Shártli itimallıq (anıqlaması, kóbеytiw tеorеması).
 \\
\textbf{T2.} Kompoziciyalıq formulalar \\
\textbf{A1.} $\left[ 0,1 \right]$ kesindiden tosınnan eki noqat tańlanadı. Ekinshi noqattıń koordinatası birinshi noqattıń koordinatası úsh esesinen úlken bolıw itimallıǵın tabıń.
 \\
\textbf{A2.} Uzaq aralıqtaǵı nıshanǵa pulemyot hám pistolet penen oq atılmaqta. Pulemyot oǵınıń nıshanǵa tiyiw itimallıǵı 0,02, al pistolet penen bolsa 0,6 ǵa teń. Eger hárbir qural menen nıshanǵa 100 márte oq atılǵan bolsa, tómendegi waqıyalardıń júzege asıw itimallıqların tabıń: a) pulemyot penen nıshanǵa úsh márte tiygiziwi; b) pistolet penen nıshanǵa 17 den 22 mártege shekem tiygiziwi.
 \\
\textbf{A3.} “Sportlotto” oyınında qatnasıwshı kartadaǵı 49 sport túrinen 6 danasın belgileydi. Qatnasıwshınıń qura taslaw nátiyjesinde alınǵan 6 sport túrinen keminde 3 danasın durıs boljaǵan bolıw itimallıǵın tabıń.
 \\
\textbf{B1.} 
Eger \includegraphics[width=0.36181in,height=0.29444in]{mediaBpng/image1.png} ǵárezsiz tosınnanlıq shamalar izbe-izliginiń bólistiriliw nızamları
\includegraphics[width=2.53958in,height=0.49722in]{mediaBpng/image2.png} \includegraphics[width=0.75486in,height=0.23958in]{mediaBpng/image3.png}
bolsa, onda bul izbe-izlik úlken sanlar nızamına boysınama?
 \\
\textbf{B2.} $\xi$ tosınnanlı shamanıń \emph{f}(\emph{x}) tıǵızlıq funkciyasi berilgen bolsin. Tómendegilerdi esaplań: a) C; b) \emph{F}(\emph{x}); c) M$\xi$; d) D$\xi$; e) \emph{f}(\emph{x}) hám \emph{F}(\emph{x}) grafiklarin sızıń.\(f(x) = \left\{ \begin{matrix}
\ \ \ \ \ \ \ \ 0,\ \ \ \ \ \ x \leq 0, \\
Cxe^{- x},\ \ \ \ \ x > 0.\ \ 
\end{matrix} \right.\ \)
 \\
\textbf{B3.} Eger \(\xi\) tosınnanlı shama \(\lambda\) parametrli puasson bólistiriwine iye bolsa, onda onıń xarakteristikalıq funkciyası tabılsın.
 \\
\textbf{C1.} Eger \(\left\{ \xi_{n} \right\}\) ǵárezsiz tosınnanlıq shamalar izbe-izligi \(\lbrack 0,1\rbrack\) aralıqta teń ólshemli bólistirilgen bolıp, \(g(x)\) funkciya sol aralıqta úziliksiz bolsa, onda\(\frac{g\left( \xi_{1} \right) + ... + g\left( \xi_{n} \right)}{n}\overset{P}{\rightarrow}\int_{0}^{1}{g(x)}dx\) ekenligin kórsetiń.
 \\
\textbf{C2.} Eger ǵárezsiz \includegraphics[width=0.15208in,height=0.24028in]{mediaCpng/image4.png} hám \includegraphics[width=0.15208in,height=0.19167in]{mediaCpng/image5.png} úzliksiz tosınnanlıq shamalar sáykes túrde, \includegraphics[width=0.4in,height=0.24028in]{mediaCpng/image34.png} aralıqta teń ólshemli hám \includegraphics[width=0.44028in,height=0.21597in]{mediaCpng/image35.png} parametrli kórsetkishli nızam boyınsha bólistirilgen bolsa, onda \includegraphics[width=0.44028in,height=0.24028in]{mediaCpng/image14.png} tosınnanlıq shamanıń tıǵızlıq funkciyasın tabıń.
 \\
\textbf{C3.} Eger \(\xi\) tosınnanlı shama \((a,\sigma)\) parametrli normal bólistiriwine iye bolsa, onda onıń xarakteristikalıq funkciyası tabılsın.
 \\

\end{tabular}
\vspace{1cm}


\begin{tabular}{m{17cm}}
\textbf{79-variant}
\newline

\textbf{T1.} Bayеs formulası (gipotezalar teoreması, dálilleniwi).
 \\
\textbf{T2.} Tiykarǵı аbsоlyut úzliksiz bólistiriliwler (nоrmаl bólistiriw, teń ólshewli bólistiriw, kórsetkishli bólistiriw). 
 \\
\textbf{A1.} Albomda 6 dana jańa hám 10 dana múddeti ótken markalar bar. Albomnan tosınnan 3 marka alıp taslandı. Bunnan soń, tosınnan 2 marka alındı. a) Bul 2 markanıń jańa bolıw itimallıǵın tabıń. b) Sol 2 marka jańa ekenligi belgili bolsa, dáslepki alınǵan 3 markanıń múddeti ótken bolıw itimallıǵın tabıń.
 \\
\textbf{A2.} $\left[ 0,1 \right]$ kesindiden tosınnan eki noqat tańlanadı. Birinshi hám ekinshi noqatlar koordinataları arasındaǵı aralıq $0,5$ ten úlken bolıw itimallıǵın tabıń.
 \\
\textbf{A3.} Eki oyın kubigi taslanǵanda túsken eń úlken ochko 4 ten úlken bolıw itimallıǵın tabıń.
 \\
\textbf{B1.} 
\includegraphics[width=0.15972in,height=0.23958in]{mediaBpng/image49.png} úzliksiz tosınnanlıq shamanıń tıǵızlıq funkciyaları berilgen. Olarǵa sáykes \includegraphics[width=0.15972in,height=0.19653in]{mediaBpng/image50.png} tosınnanlıq shamanıń \includegraphics[width=0.50278in,height=0.30069in]{mediaBpng/image51.png} tıǵızlıq funkciyasın tabıń. \includegraphics[width=2.23333in,height=0.84028in]{mediaBpng/image52.png} \includegraphics[width=0.88333in,height=0.23958in]{mediaBpng/image53.png}
 \\
\textbf{B2.} Eger \includegraphics[width=0.36181in,height=0.29444in]{mediaBpng/image1.png} ǵárezsiz tosınnanlıq shamalar izbe-izliginiń bólistiriliw nızamları
\includegraphics[width=2.23333in,height=0.50278in]{mediaBpng/image31.png} \includegraphics[width=1.17778in,height=0.50278in]{mediaBpng/image32.png} \includegraphics[width=0.77292in,height=0.25764in]{mediaBpng/image33.png}
bolsa, onda bul izbe-izlik úlken sanlar nızamına boysınama?
 \\
\textbf{B3.} \includegraphics[width=0.15972in,height=0.23958in]{mediaBpng/image49.png} úzliksiz tosınnanlıq shamanıń tıǵızlıq funkciyaları berilgen. Olarǵa sáykes \includegraphics[width=0.15972in,height=0.19653in]{mediaBpng/image50.png} tosınnanlıq shamanıń \includegraphics[width=0.50278in,height=0.30069in]{mediaBpng/image51.png} tıǵızlıq funkciyasın tabıń. \includegraphics[width=2.44167in,height=0.84028in]{mediaBpng/image60.png} \includegraphics[width=0.88333in,height=0.23958in]{mediaBpng/image61.png}
 \\
\textbf{C1.} Eger \(\chi_{1}^{2}\) hám \(\chi_{2}^{2}\) ǵárezsiz tosınnanlıq shamalar bolıp, \(\chi_{1}^{2}\sim\chi^{2}(n_{1})\) hám \(\chi_{2}^{2}\sim\chi^{2}(n_{2})\) bolsa, onda \(\frac{n_{2}}{n_{1}} \cdot \frac{\chi_{1}^{2}}{\chi_{2}^{2}}\) tosınnanlıq shamanıń tıǵızlıq funkciyasın tabıń.
 \\
\textbf{C2.} Tosınnanlıq vektordıń komponentaları absolyut úziliksizliginen tosınnanlıq vektordıń ózi de absolyut úziliksizligi kelip shıqpaslıǵın kórsetiń.
 \\
\textbf{C3.} Eger ǵárezsiz \includegraphics[width=0.15208in,height=0.24028in]{mediaCpng/image4.png} hám \includegraphics[width=0.15208in,height=0.19167in]{mediaCpng/image5.png} úzliksiz tosınnanlıq shamalardıń hárbiri \includegraphics[width=0.19167in,height=0.16806in]{mediaCpng/image33.png} parametrli kórsetkishli nızam boyınsha bólistirilgen bolsa, onda \includegraphics[width=0.50417in,height=0.29583in]{mediaCpng/image30.png} tosınnanlıq shamanıń tıǵızlıq funkciyasın tabıń.
 \\

\end{tabular}
\vspace{1cm}


\begin{tabular}{m{17cm}}
\textbf{80-variant}
\newline

\textbf{T1.} Itimallıq anıqlamaları (klassikalıq, geometriyalıq anıqlamaları).
 \\
\textbf{T2.} Tosınnanlı shamanıń matematikalıq kútiliwi. (anıqlaması, qásiyetleri)
 \\
\textbf{A1.} Uzaq aralıqtaǵı nıshanǵa pulemyot hám pistolet penen oq atılmaqta. Pulemyot oǵınıń nıshanǵa tiyiw itimallıǵı 0,02, al pistolet penen bolsa 0,6 ǵa teń. Eger hárbir qural menen nıshanǵa 100 márte oq atılǵan bolsa, tómendegi waqıyalardıń júzege asıwı itimallıqların tabıń: a) pulemyot penen nıshanǵa tórt márte tiygiziwi; b) pistolet penen nıshanǵa 70 márte tiygiziwi. \\
\textbf{A2.} Hárbiriniń júzege asıw itimallıǵı $p$ ǵa teń bolǵan 10 dana Bernulli tájiriybesi ótkerilgende, tómendegi waqıyalardıń itimallıqların tabıń: Sátsizler sanı kóbi menen 2 dana.
 \\
\textbf{A3.} 36 dana kartalar kolodasınan tosınnan alınǵan 6 dana karta ishinde anıq 5 dana karta birdey reńde hám 1 dana karta basqa reńde bolıw itimallıǵın tabıń.
 \\
\textbf{B1.} Tosınnanlı $\xi$ shamasınıń tıǵızlıq funkciyasi berilgen: \(f(x) = e^{- 3|x|}\) Usi shamanıń matematikalıq kútiliwin tabıń.
 \\
\textbf{B2.} $\xi$ tosınnanlı shamanıń \emph{f}(\emph{x}) tıǵızlıq funkciyasi berilgen bolsin. Tómendegilerdi esaplań: a) C; b) \emph{F}(\emph{x}); c) M$\xi$; d) D$\xi$; e) \emph{f}(\emph{x}) hám \emph{F}(\emph{x}) grafiklarin sızıń.\(f(x) = \left\{ \begin{matrix}
(x + 1)/2,\ \ \ \ x \in \lbrack - 1,0\rbrack, \\
(C - x)/2C,\ x \in (0,C\rbrack, \\
\ \ \ \ \ \ 0,\ \ \ \ \ \ \ \ \ \ \ \ x \notin \lbrack - 1,C\rbrack\ \ 
\end{matrix} \right.\ \)
 \\
\textbf{B3.} Birdey kartochkalarǵa A,A,A,E,I,M,M,K,T,T háripleri jazilǵan hám jaqsilap aralastirip tóńkerip jayilǵan. Izbe-iz alinǵan kartochkalardi alinǵan tártibinde jaylastiriw nátiyjesinde «MATEMATIKA» sóziniń kelip shiǵiw itimalliǵin tabiń.
 \\
\textbf{C1.} Tómende \includegraphics[width=0.46389in,height=0.25625in]{mediaCpng/image42.png} úzliksiz tosınnanlıq vektorlardıń tıǵızlıq funkciyaları berilgen. Olardıń \includegraphics[width=0.48819in,height=0.29583in]{mediaCpng/image43.png} hám \includegraphics[width=0.50417in,height=0.29583in]{mediaCpng/image44.png} marginal tıǵızlıq funkciyaların tabıń; \includegraphics[width=0.15972in,height=0.24028in]{mediaCpng/image45.png} hám \includegraphics[width=0.15972in,height=0.2in]{mediaCpng/image46.png} tosınnanlıq shamalardı ǵárezsizlikke tekseriń: \includegraphics[width=3.45625in,height=0.53611in]{mediaCpng/image48.png}
 \\
\textbf{C2.} Hárqanday \(\varphi_{\xi}(t)\) xarakteristikalıq funkciya ushın \(t \in R\) de \(1 - Re\varphi_{\xi}(2t) \leq 4\left( 1 - Re\varphi_{\xi}(t) \right)\) ekenligin dálilleń.
 \\
\textbf{C3.} Oraylıq limit teorema járdeminde tómendegi teńlikti dálilleń: \(\lim_{n \rightarrow \infty}e^{- n}\sum_{k = 1}^{n}\frac{n^{k}}{k!} = \frac{1}{2}.\)
 \\

\end{tabular}
\vspace{1cm}


\begin{tabular}{m{17cm}}
\textbf{81-variant}
\newline

\textbf{T1.} Bernulli sxemаsı ushın limit teоremаlаr (Muavr-Laplas integrallıq teoreması, qásiyetleri).
 \\
\textbf{T2.} Oraylıq limit teorema (anıqlaması, ǵárezsiz birdey bólistirilgen tosınnanlı shamalar ushın).
 \\
\textbf{A1.} Jip iyiriw fabrikasında 1000 dana jańa hám 200 dana eski jip iyiriw qurılmaları bar. Bir jumıs kúninde jańa qurılma 0,003 itimallıq penen, al eski qurılma bolsa 0,20 itimallıq penen iyirilip atırǵan jipti úzip aladı. Bir jumıs kúninde tómendegi waqıyalardıń júzege asıwı itimallıqların tabıń: a) jańa qurılma 3 márte jipti úzip alıwı; b) eski qurılma 10 nan 15 ke shekemgi jipti úzip alıwı. 
 \\
\textbf{A2.} Eki oyın kubigi taslanǵanda túsken ochkolardıń ayırması 1 den úlken bolıw itimallıǵın tabıń.
 \\
\textbf{A3.} 
Toparda 12 student bar bolıp, olardan 8 student ayrıqsha bahaǵa oqıydı. Dizim boyınsha 9 student ajıratılǵan. Ajıratılǵanlar ishinde ayrıqsha bahaǵa oqıytuǵın 5 student bolıw itimallıǵın tabıń.
 \\
\textbf{B1.} Hár bir sinawda A waqiyasiniń júzege asiw itimalliǵi 0,6 ǵa teń. Ǵárezsiz 5400 sinawdiń 3240 mártesinde A waqiyasiniń júzege asiw itimalliǵin tabiń.
 \\
\textbf{B2.} 
$\xi$ tosınnanlı shamanıń \emph{f}(\emph{x}) tıǵızlıq funkciyasi berilgen bolsin. Tómendegilerdi esaplań: a) C; b) \emph{F}(\emph{x}); c) M$\xi$; d) D$\xi$; e) \emph{f}(\emph{x}) hám \emph{F}(\emph{x}) grafiklarin sızıń.\(f(x) = \left\{ \begin{matrix}
Cx,\ \ \ \ x \in \lbrack 0,1\rbrack, \\
C,\ \ \ \ \ \ \ x \in (1,2\rbrack, \\
0,\ \ \ keri\ jag'dayda.\ \ 
\end{matrix} \right.\ \)
 \\
\textbf{B3.} Eger \includegraphics[width=0.36181in,height=0.29444in]{mediaBpng/image1.png} ǵárezsiz tosınnanlıq shamalar izbe-izliginiń bólistiriliw nızamları
\includegraphics[width=1.53958in,height=0.50278in]{mediaBpng/image21.png} \includegraphics[width=1.63819in,height=0.50278in]{mediaBpng/image22.png} \includegraphics[width=0.75486in,height=0.23958in]{mediaBpng/image9.png}
bolsa, onda bul izbe-izlik úlken sanlar nızamına boysınama?
 \\
\textbf{C1.} Meyli, \(\xi_{1},...,\xi_{n}\) tosınnanlıq shamalar ǵárezsiz hám \(\lbrack a,b\rbrack\) aralıqta teń ólshemli bólistirilgen bolıp, \(\eta_{1} = \max\left( \xi_{1},...,\xi_{n} \right)\) hám \(\eta_{2} = \min\left( \xi_{1},...,\xi_{n} \right)\) bolsın. Onda \(\left( \eta_{1},\eta_{2} \right)\) tosınnanlıq vektordıń kovariaciyasın tabıń.
 \\
\textbf{C2.} Eger ǵárezsiz \includegraphics[width=0.15208in,height=0.24028in]{mediaCpng/image4.png} hám \includegraphics[width=0.15208in,height=0.19167in]{mediaCpng/image5.png} úzliksiz tosınnanlıq shamalar sáykes túrde, \includegraphics[width=0.6in,height=0.26389in]{mediaCpng/image16.png} hám \includegraphics[width=0.6in,height=0.26389in]{mediaCpng/image17.png} parametrli kórsetkishli nızam boyınsha bólistirilgen bolsa, onda \includegraphics[width=0.44028in,height=0.24028in]{mediaCpng/image14.png} tosınnanlıq shamanıń tıǵızlıq funkciyasın tabıń.
 \\
\textbf{C3.} Eger \(\left( \xi_{1},\xi_{2} \right)\) tosınnanlıq vеktоrdıń tıǵızlıq funkciyası \(f(x,y) = \frac{1}{3\pi}e^{- \ \ \frac{x^{2} + 4y^{2}}{6}}\)bolsa, onda \(\left( \xi_{1},\xi_{2} \right)\) tosınnanlıq noqattıń \(D = \left\{ (x,y):|x| \leq 1,|y| \leq 2 \right\}\) oblastqa túsiw itimallıǵın tabıń.
 \\

\end{tabular}
\vspace{1cm}


\begin{tabular}{m{17cm}}
\textbf{82-variant}
\newline

\textbf{T1.} Waqıyalar algebrası ($\sigma$-algebra, minimal $\sigma$-algebra).
 \\
\textbf{T2.} Tiykarǵı diskret bólistiriliwler (Binomial, Puasson hám geometriyalıq bólistiriliwler).
 \\
\textbf{A1.} Albomda 8 dana jańa hám 6 dana múddeti ótken markalar bar. Albomnan tosınnan 3 marka alınıp, múddeti jańalandı hám ornına qaytarıp qoyıldı. Bunnan soń, tosınnan 2 marka alındı. a) Bul 2 marka jańa bolıw itimallıǵın tabıń. b) Sol 2 marka jańa ekenligi belgili bolsa, dáslepki alınǵan 3 markanıń múddeti ótken bolıw itimallıǵın tabıń.
 \\
\textbf{A2.} $\left[ 0,1 \right]$ kesindiden tosınnan eki noqat tańlanadı. Olardıń koordinataları qosındısı 1 den úlken bolmaw hám kóbeymesi 0,09 dan kishi bolmaw itimallıǵın tabıń.
 \\
\textbf{A3.} Hárbiriniń júzege asıw itimallıǵı $p$ ǵa teń bolǵan 10 dana Bernulli tájiriybesi ótkerilgende, tómendegi waqıyalardıń itimallıqların tabıń: Tájiriybelerdiń birinshi yarımındaǵı sátliler sanı, tájiriybelerdiń ekinshi yarımındaǵı sátliler sanınan ekige artıq.
 \\
\textbf{B1.} Eger \(\xi\) tosınnanlı shama \((n,p)\) parametrli binomial bólistiriwine iye bolsa, onda onıń xarakteristikalıq funkciyası tabılsın.
 \\
\textbf{B2.} \includegraphics[width=0.15972in,height=0.23958in]{mediaBpng/image49.png} úzliksiz tosınnanlıq shamanıń tıǵızlıq funkciyaları berilgen. Olarǵa sáykes \includegraphics[width=0.15972in,height=0.19653in]{mediaBpng/image50.png} tosınnanlıq shamanıń \includegraphics[width=0.50278in,height=0.30069in]{mediaBpng/image51.png} tıǵızlıq funkciyasın tabıń. \includegraphics[width=1.50278in,height=0.58889in]{mediaBpng/image82.png} \includegraphics[width=0.88333in,height=0.28194in]{mediaBpng/image83.png}
 \\
\textbf{B3.} $\xi$ tosınnanlı shamanıń \emph{f}(\emph{x}) tıǵızlıq funkciyasi berilgen bolsin. Tómendegilerdi esaplań: a) C; b) \emph{F}(\emph{x}); c) M$\xi$; d) D$\xi$; e) \emph{f}(\emph{x}) hám \emph{F}(\emph{x}) grafiklarin sızıń.\(f(x) = \left\{ \begin{matrix}
\ \ \ \ \ \ \ \ 0,\ \ \ \ \ \ \ \ \ \ \ \ x \leq 0, \\
C/(x + 1)^{4},\ \ \ \ \ x > 0.\ \ 
\end{matrix} \right.\ \)
 \\
\textbf{C1.} Eger \(\left\{ \xi_{n} \right\}\) ǵárezsiz tosınnanlıq shamalar izbe-izligi \(\lbrack 0,1\rbrack\) aralıqta teń ólshemli bólistirilgen bolıp, \(g(x)\) funkciya sol aralıqta úziliksiz bolsa, onda\(\frac{g\left( \xi_{1} \right) + ... + g\left( \xi_{n} \right)}{n}\overset{P}{\rightarrow}\int_{0}^{1}{g(x)}dx\) ekenligin kórsetiń.
 \\
\textbf{C2.} Tómende \includegraphics[width=0.46389in,height=0.25625in]{mediaCpng/image42.png} úzliksiz tosınnanlıq vektorlardıń tıǵızlıq funkciyaları berilgen. Olardıń \includegraphics[width=0.48819in,height=0.29583in]{mediaCpng/image43.png} hám \includegraphics[width=0.50417in,height=0.29583in]{mediaCpng/image44.png} marginal tıǵızlıq funkciyaların tabıń; \includegraphics[width=0.15972in,height=0.24028in]{mediaCpng/image45.png} hám \includegraphics[width=0.15972in,height=0.2in]{mediaCpng/image46.png} tosınnanlıq shamalardı ǵárezsizlikke tekseriń: \includegraphics[width=2.52778in,height=0.59167in]{mediaCpng/image54.png}
 \\
\textbf{C3.} Eger \(\xi\sim E(\lambda)\) bolsa, onda \(\xi\) tosınnanlıq shamanıń joqarı tártipli baslanǵısh momentlerin tabıń.
 \\

\end{tabular}
\vspace{1cm}


\begin{tabular}{m{17cm}}
\textbf{83-variant}
\newline

\textbf{T1.} Itimallıqlar teoriyası aksiomaları (ólshewli keńislik, itimallıq keńisligi).
 \\
\textbf{T2.} Tosınnanlı shamanıń dispersiyası (anıqlaması, qásiyetleri).
 \\
\textbf{A1.} 
“Sportlotto” lotereyasında úlken hám kishi utıslar oynaladı. Lotereya biletinde úlken utıs shıǵıw itimallıǵı 0,009 ǵa, al kishisi bolsa 0,02 ge teń. Jámi 500 dana bilet satıp alınǵanda: a) úlken utıslar 5 ten 10 ǵa shekem bolıw; b) 15 ten 20 ǵa shekem kishi utıslar bolıwı waqıyaları itimallıqların tabıń.
 \\
\textbf{A2.} Samolyotqa birimlep úsh márte oq atıladı. Birinshi márte atqanda tiyiw itimallıǵı 0,4 ke, ekinshisinde 0,5 ke, úshinshisinde bolsa 0,7 ge teń. Samolyottı isten shıǵarıw ushın úsh márte oq tiyiwi jetkilikli, bir márte tiygende 0,2 ge teń itimallıq penen qatardan shıǵadı, eki márte tiygende 0,6 ǵa teń itimallıq penen isten shıǵadı. 
a) Úsh márte oq atıw nátiyjesinde samolyottıń tolıǵı menen isten shıǵıw itimallıǵın tabıń. b) Eger samolyot tolıǵı menen isten shıqqan bolsa, ol eki oq tiyip isten shıqqan bolıw itimallıǵın tabıń.
 \\
\textbf{A3.} Hárbiriniń júzege asıw itimallıǵı $p$ ǵa teń bolǵan 10 dana Bernulli tájiriybesi ótkerilgende, tómendegi waqıyalardıń itimallıqların tabıń: Tájiriybelerdiń birinshi yarımındaǵı sátliler sanı, tájiriybelerdiń ekinshi yarımındaǵı sátliler sanınan artıq.
 \\
\textbf{B1.} \includegraphics[width=0.15972in,height=0.23958in]{mediaBpng/image49.png} úzliksiz tosınnanlıq shamanıń tıǵızlıq funkciyaları berilgen. Olarǵa sáykes \includegraphics[width=0.15972in,height=0.19653in]{mediaBpng/image50.png} tosınnanlıq shamanıń \includegraphics[width=0.50278in,height=0.30069in]{mediaBpng/image51.png} tıǵızlıq funkciyasın tabıń. \includegraphics[width=2.30694in,height=0.84028in]{mediaBpng/image64.png} \includegraphics[width=0.85903in,height=0.23958in]{mediaBpng/image65.png}
 \\
\textbf{B2.} \emph{R} radiuslı dóńgelek ishinen vertikal xordalar júrgiziledi. Tosınnan alınǵan xordanıń radiustan kishi bolıwı itimallıǵın tabıń.
 \\
\textbf{B3.} Eger \includegraphics[width=0.36181in,height=0.29444in]{mediaBpng/image1.png} ǵárezsiz tosınnanlıq shamalar izbe-izligi \includegraphics[width=0.68681in,height=0.54583in]{mediaBpng/image47.png} aralıqta teń ólshemli bólistirilgen bolsa, onda bul izbe-izlik úlken sanlar nızamına boysınama?
 \\
\textbf{C1.} Tómende \includegraphics[width=0.46389in,height=0.25625in]{mediaCpng/image42.png} úzliksiz tosınnanlıq vektorlardıń tıǵızlıq funkciyaları berilgen. Olardıń \includegraphics[width=0.48819in,height=0.29583in]{mediaCpng/image43.png} hám \includegraphics[width=0.50417in,height=0.29583in]{mediaCpng/image44.png} marginal tıǵızlıq funkciyaların tabıń; \includegraphics[width=0.15972in,height=0.24028in]{mediaCpng/image45.png} hám \includegraphics[width=0.15972in,height=0.2in]{mediaCpng/image46.png} tosınnanlıq shamalardı ǵárezsizlikke tekseriń: \includegraphics[width=3.13611in,height=0.59167in]{mediaCpng/image49.png}
 \\
\textbf{C2.} Eger \(\mathbf{\xi}_{\mathbf{n}}\overset{\mathbf{L}^{\mathbf{2}}}{\rightarrow}\mathbf{\xi}\) bolsa, onda \(n \rightarrow \infty\) de \(\mathbf{M}\mathbf{\xi}_{\mathbf{n}}\mathbf{\rightarrow M\xi}\) ekenligin kórsetiń.
 \\
\textbf{C3.} Eger ǵárezsiz \includegraphics[width=0.15208in,height=0.24028in]{mediaCpng/image4.png} hám \includegraphics[width=0.15208in,height=0.19167in]{mediaCpng/image5.png} úzliksiz tosınnanlıq shamalardıń hárbiri \includegraphics[width=0.4in,height=0.24028in]{mediaCpng/image18.png} aralıqta teń ólshemli bólistirilgen bolsa, onda \includegraphics[width=0.84792in,height=0.29583in]{mediaCpng/image19.png} tosınnanlıq shamanıń tıǵızlıq funkciyasın tabıń.
 \\

\end{tabular}
\vspace{1cm}


\begin{tabular}{m{17cm}}
\textbf{84-variant}
\newline

\textbf{T1.} Tolıq itimallıq formulası (waqıyalardıń tolıq gruppası, dálilleniwi).
 \\
\textbf{T2.} Bólistiriw funkciyası (anıqlaması, tiykarǵı qásiyetleri).
 \\
\textbf{A1.} Eki oyın kubigi taslanǵanda keminde bir oyın kubigindegi túsken ochko jup bolıw itimallıǵın tabıń.
 \\
\textbf{A2.} $\left[ 0,1 \right]$ kesindiden tosınnan eki noqat tańlanadı. Birinshi noqattıń koordinatasınıń ekinshi noqattıń koordinatasına qatnası 0,5 ten úlken bolıw itimallıǵın tabıń.
 \\
\textbf{A3.} 36 dana kartalar kolodasınan tosınnan alınǵan 6 dana karta ishinde anıq 2 danası tuz bolıw itimallıǵın tabıń.
 \\
\textbf{B1.} Bólistiriw tiǵizliǵi \(\mathbf{f}\mathbf{(}\mathbf{x}\mathbf{)}\mathbf{=}\frac{\mathbf{1}}{\mathbf{\sigma}\sqrt{\mathbf{2}\mathbf{\pi}}}\mathbf{e}^{\mathbf{-}\frac{\left( \mathbf{x - m} \right)^{\mathbf{2}}}{\mathbf{2}\mathbf{\sigma}^{\mathbf{2}}}}\)bolsa, usı tosınnanlı shamanıń matematikalıq kútiliwin tabılsın.
 \\
\textbf{B2.} İdısta 10 shar bolıp, olardan 3 ewi aq sharlar. İdıstan táwekelge 3 shar alınadı. Tosınnanlı \(\xi\) shaması -- alınǵan aq sharlar sanı. Onıń bólistiriliw nızamın jazıń.
 \\
\textbf{B3.} Qutıda 4 aq hám 5 qara sharlar bar. Qutıdan izbe-iz 2 shar alınadi. Alinǵan 2 shardiń birinshisi aq ekinshisi qara shar boliw itimallıǵın tabıń.
 \\
\textbf{C1.} Eger \(\left( \xi_{1},\xi_{2} \right)\) tosınnanlıq vektordıń bólistiriw funkciyası\(F(x,y) = \sin x \cdot \sin y,\ \ \ 0 \leq x \leq \frac{\pi}{2},\ \ 0 \leq y \leq \frac{\pi}{2}\) bolsa, onda \(\left( \xi_{1},\xi_{2} \right)\) tosınnanlıq noqattıń \(G:x_{1} = \frac{\pi}{6},\ \ x_{2} = \frac{\pi}{2};\ \ y_{1} = \frac{\pi}{4},\ \ y_{2} = \frac{\pi}{3}\) bolǵan tuwrımúyeshlikke túsiw itimallıǵın tabıń.
 \\
\textbf{C2.} Eger ǵárezsiz \includegraphics[width=0.15208in,height=0.24028in]{mediaCpng/image4.png} hám \includegraphics[width=0.15208in,height=0.19167in]{mediaCpng/image5.png} úzliksiz tosınnanlıq shamalar sáykes túrde, \includegraphics[width=0.43194in,height=0.21597in]{mediaCpng/image12.png} hám \includegraphics[width=0.60833in,height=0.26389in]{mediaCpng/image13.png} parametrli kórsetkishli nızam boyınsha bólistirilgen bolsa, onda \includegraphics[width=0.44028in,height=0.24028in]{mediaCpng/image14.png} tosınnanlıq shamanıń tıǵızlıq funkciyasın tabıń.
 \\
\textbf{C3.} Meyli, \(\left\{ \xi_{n} \right\}\) tosınnanlıq shamalar izbe-izligi óziniń \(\left\{ F_{n}(x) \right\}\) bólistiriw funkciyaları menen berilgen bolsın. Sonda hám tek sonda ǵana, eger \(\lim_{n \rightarrow \infty}\int_{- \infty}^{+ \infty}{\frac{x^{2}}{1 + x^{2}}dF_{n}(x)} = 0\) bolsa, onda \(\mathbf{\xi}_{\mathbf{n}}\overset{\mathbf{P}}{\rightarrow}\mathbf{0}\) ekenligin dálilleń.
 \\

\end{tabular}
\vspace{1cm}


\begin{tabular}{m{17cm}}
\textbf{85-variant}
\newline

\textbf{T1.} Ǵárezsiz tájiriybelerdiń Bernulli sxeması (binоmiаl bólistiriliw, qásiyetleri).
 \\
\textbf{T2.} Tıǵızlıq funkciyası (anıqlaması, tiykarǵıqásiyetleri).
 \\
\textbf{A1.} Hárbiriniń júzege asıw itimallıǵı $p$ ǵa teń bolǵan 10 dana Bernulli tájiriybesi ótkerilgende, tómendegi waqıyalardıń itimallıqların tabıń: Sátsizler sanı tek 2 dana hám olar arasında 4 dana sátli bolıw.
 \\
\textbf{A2.} Radio ustaxanada kúnine 3 radio ońlanadı. Radionıń mexanikalıq bólegi buzılıw itimallıǵı 0,1 ge hám elektron bólegi buzılıw itimallıǵı 0,004 ke teń. Jıl dawamında ońlanǵan radiolar arasında tómendegi waqıyalardıń júzege asıw itimallıqların tabıń: a) 130 dan 140 ǵa shekem radiolardıń mexanikalıq bóleginde nasazlıqlar bolǵan; b) tórtten artıq bolmaǵan radiolardıń elektron bóleginde nasazlıqlar bolǵan.
 \\
\textbf{A3.} Qutıda 10 dana aq hám 15 dana qara sharlar bar. Tosınnan 5 dana shar alınǵanda, olar ishinde 2 dana aq shar bolıw itimallıǵın tabıń.
 \\
\textbf{B1.} \includegraphics[width=0.15972in,height=0.23958in]{mediaBpng/image49.png} úzliksiz tosınnanlıq shamanıń tıǵızlıq funkciyaları berilgen. Olarǵa sáykes \includegraphics[width=0.15972in,height=0.19653in]{mediaBpng/image50.png} tosınnanlıq shamanıń \includegraphics[width=0.50278in,height=0.30069in]{mediaBpng/image51.png} tıǵızlıq funkciyasın tabıń. \includegraphics[width=2.53958in,height=1.15347in]{mediaBpng/image88.png} \includegraphics[width=0.63194in,height=0.25764in]{mediaBpng/image89.png}
 \\
\textbf{B2.} Eger \includegraphics[width=0.36181in,height=0.29444in]{mediaBpng/image1.png} ǵárezsiz tosınnanlıq shamalar izbe-izliginiń bólistiriliw nızamları
\includegraphics[width=2.55833in,height=0.50278in]{mediaBpng/image7.png} \includegraphics[width=1.41736in,height=0.50278in]{mediaBpng/image8.png} \includegraphics[width=0.75486in,height=0.23958in]{mediaBpng/image9.png}
bolsa, onda bul izbe-izlik úlken sanlar nızamına boysınama?
 \\
\textbf{B3.} $\xi$ tosınnanlı shamanıń \emph{f}(\emph{x}) tıǵızlıq funkciyasi berilgen bolsin. Tómendegilerdi esaplań: a) C; b) \emph{F}(\emph{x}); c) M$\xi$; d) D$\xi$; e) \emph{f}(\emph{x}) hám \emph{F}(\emph{x}) grafiklarin sızıń.\(f(x) = \left\{ \begin{matrix}
C/\sqrt{1 - x^{2}},\ \ \ \ x \in \lbrack - 1,1\rbrack, \\
\ \ \ \ \ \ \ \ 0,\ \ \ \ \ \ \ \ \ \ \ x \notin \lbrack - 1,1\rbrack.\ \ 
\end{matrix} \right.\ \)
 \\
\textbf{C1.} Eger \(\xi\) hám \(\chi^{2}\) ǵárezsiz tosınnanlıq shamalar bolıp, \(\xi\sim N(0,1)\) hám \(\chi^{2}\sim\chi^{2}(n)\) bolsa, onda \(\frac{\xi}{\sqrt{\frac{\chi^{2}}{n}}}\) tosınnanlıq shamanıń tıǵızlıq funkciyasın tabıń.
 \\
\textbf{C2.} Eger \(\xi\sim N\left( a,\sigma^{2} \right)\) bolsa, onda \(\xi\) tosınnanlıq shamanıń joqarı tártipli oraylıq momentlerin tabıń.
 \\
\textbf{C3.} Tómende \includegraphics[width=0.46389in,height=0.25625in]{mediaCpng/image42.png} úzliksiz tosınnanlıq vektorlardıń tıǵızlıq funkciyaları berilgen. Olardıń \includegraphics[width=0.48819in,height=0.29583in]{mediaCpng/image43.png} hám \includegraphics[width=0.50417in,height=0.29583in]{mediaCpng/image44.png} marginal tıǵızlıq funkciyaların tabıń; \includegraphics[width=0.15972in,height=0.24028in]{mediaCpng/image45.png} hám \includegraphics[width=0.15972in,height=0.2in]{mediaCpng/image46.png} tosınnanlıq shamalardı ǵárezsizlikke tekseriń: \includegraphics[width=3.54375in,height=0.81597in]{mediaCpng/image71.png} \\

\end{tabular}
\vspace{1cm}


\begin{tabular}{m{17cm}}
\textbf{86-variant}
\newline

\textbf{T1.} Bernulli sxemаsı ushın limit teоremаlаr (Muavr-Laplas lokallıq teoreması, qásiyetleri).
 \\
\textbf{T2.} Xarakteristikalıq funkciyalar (anıqlaması, tiykarǵı qásiyetleri).
 \\
\textbf{A1.} $\left[ -1,2 \right]$ kesindiden tosınnan eki noqat tańlanadı. Olardıń koordinataları qosındısı 1 den úlken bolıw hám kóbeymesi 1 den kishi bolıw itimallıǵın tabıń.
 \\
\textbf{A2.} Berilgen $1,2,\ldots ,10$ sanlarınıń arasınan tosınnan eki san tańlandı. Meyli, bul sanlar ${{m}_{1}}$ hám ${{m}_{2}}$ (${{m}_{1}}<{{m}_{2}}$) bolsın. Soń, ${{m}_{1}},{{m}_{1}}+1,\ldots ,{{m}_{2}}$ sanları arasınan tosınnan bir san tańlandı. a) Bul sannıń 9 ǵa teń bolıw itimallıǵın tabıń. b) Bul san 9 ǵa teń bolsa, ${{m}_{2}}=10$ bolıw itimallıǵın tabıń.
 \\
\textbf{A3.} Eki oyın kubigi taslanǵanda túsken ochkolardıń taq bolıw itimallıǵın tabıń.
 \\
\textbf{B1.} \includegraphics[width=0.15972in,height=0.23958in]{mediaBpng/image49.png} úzliksiz tosınnanlıq shamanıń tıǵızlıq funkciyaları berilgen. Olarǵa sáykes \includegraphics[width=0.15972in,height=0.19653in]{mediaBpng/image50.png} tosınnanlıq shamanıń \includegraphics[width=0.50278in,height=0.30069in]{mediaBpng/image51.png} tıǵızlıq funkciyasın tabıń. \includegraphics[width=2.44167in,height=0.63194in]{mediaBpng/image98.png} \includegraphics[width=0.57639in,height=0.28194in]{mediaBpng/image99.png} \\
\textbf{B2.} Pul lotereyasında 100 bilet shıǵarılǵan. 50 swmlıq 1 utıs, 10 swmlıq 10 utıs bar. Bir lotereya biletiniń iyesi ushın múmkin bolǵan utıstıń bahasınıń bólistiriw nızamın jazıń.
 \\
\textbf{B3.} $\xi$ tosınnanlı shamanıń \emph{f}(\emph{x}) tıǵızlıq funkciyasi berilgen bolsin. Tómendegilerdi esaplań: a) C; b) \emph{F}(\emph{x}); c) M$\xi$; d) D$\xi$; e) \emph{f}(\emph{x}) hám \emph{F}(\emph{x}) grafiklarin sızıń.\(f(x) = \left\{ \begin{matrix}
2x/3,\ \ \ \ x \in \lbrack 0,1\rbrack, \\
C(3 - x),\ \ \ x \in (1,3\rbrack, \\
0,\ \ \ keri\ jag'dayda.\ \ 
\end{matrix} \right.\ \)
 \\
\textbf{C1.} 
\(\xi\) diskret tosınnanlıq shama \(x_{i} = ( - 1)^{i}i\) mánislerdi \(p_{i} = \frac{1}{i(i + 1)},\) \(\ \ i = 1,\ 2,\ ...\) itimallıqlar menen qabıl etse, onıń matematikalıq kútiliwin tabıń.
 \\
\textbf{C2.} Eger ǵárezsiz \includegraphics[width=0.15208in,height=0.24028in]{mediaCpng/image4.png} hám \includegraphics[width=0.15208in,height=0.19167in]{mediaCpng/image5.png} úzliksiz tosınnanlıq shamalardıń hárbiri \includegraphics[width=0.4in,height=0.24028in]{mediaCpng/image18.png} aralıqta teń ólshemli bólistirilgen bolsa, onda \includegraphics[width=0.50417in,height=0.29583in]{mediaCpng/image21.png} tosınnanlıq shamanıń tıǵızlıq funkciyasın tabıń.
 \\
\textbf{C3.} Tómende \includegraphics[width=0.46389in,height=0.25625in]{mediaCpng/image42.png} úzliksiz tosınnanlıq vektorlardıń tıǵızlıq funkciyaları berilgen. Olardıń \includegraphics[width=0.48819in,height=0.29583in]{mediaCpng/image43.png} hám \includegraphics[width=0.50417in,height=0.29583in]{mediaCpng/image44.png} marginal tıǵızlıq funkciyaların tabıń; \includegraphics[width=0.15972in,height=0.24028in]{mediaCpng/image45.png} hám \includegraphics[width=0.15972in,height=0.2in]{mediaCpng/image46.png} tosınnanlıq shamalardı ǵárezsizlikke tekseriń: \includegraphics[width=3.31181in,height=0.84792in]{mediaCpng/image67.png}
 \\

\end{tabular}
\vspace{1cm}


\begin{tabular}{m{17cm}}
\textbf{87-variant}
\newline

\textbf{T1.} Bernulli sxemаsı ushın limit teоremаlаr (Puasson bólistiriliwi, qásiyetleri).
 \\
\textbf{T2.} Tosınnanlı shamanıń joqarı tártipli momentleri (baslanǵısh hám oraylıq momentleri, qásiyetleri).
 \\
\textbf{A1.} Bazaǵa 360 dana buyım keltirilgen. Bulardan: 300 danası 1-kárxanada tayarlanǵan bolıp, 250 danası jaramlı; 40 danası 2-kárxanada tayarlanǵan bolıp, 30 danası jaramlı; 20 danası 3-kárxanada tayarlanǵan bolıp, 10 danası jaramlı. a) Bazadan tosınnan alınǵan buyımnıń jaramlı bolıw itimallıǵın tabıń. b) Eger bazadan alınǵan buyım jaramlı bolsa, onda sol buyımnıń 2-kárxanaǵa tiyisli bolıw itimallıǵın tabıń.
 \\
\textbf{A2.} Hárbiriniń júzege asıw itimallıǵı $p$ ǵa teń bolǵan 10 dana Bernulli tájiriybesi ótkerilgende, tómendegi waqıyalardıń itimallıqların tabıń: Sátsizler sanı joq bolıw.
 \\
\textbf{A3.} $x\in \left[ -\pi ,\pi  \right]$ ushın $sinx<cosx$ bolıw itimallıǵın tabıń.
 \\
\textbf{B1.} 
Eger \includegraphics[width=0.36181in,height=0.29444in]{mediaBpng/image1.png} ǵárezsiz tosınnanlıq shamalar izbe-izliginiń bólistiriliw nızamları
\includegraphics[width=2.53958in,height=0.49722in]{mediaBpng/image2.png} \includegraphics[width=0.75486in,height=0.23958in]{mediaBpng/image3.png}
bolsa, onda bul izbe-izlik úlken sanlar nızamına boysınama?
 \\
\textbf{B2.} Qálegen \(a,b \in \lbrack 0;2\rbrack\) sanları ushın \(D = \left| \begin{matrix}
1 & a \\
a & b
\end{matrix} \right|\) determinanti esaplanadı. \(D > 0\) bolıwı itimallıǵı qanday?
 \\
\textbf{B3.} \includegraphics[width=0.15972in,height=0.23958in]{mediaBpng/image49.png} úzliksiz tosınnanlıq shamanıń tıǵızlıq funkciyaları berilgen. Olarǵa sáykes \includegraphics[width=0.15972in,height=0.19653in]{mediaBpng/image50.png} tosınnanlıq shamanıń \includegraphics[width=0.50278in,height=0.30069in]{mediaBpng/image51.png} tıǵızlıq funkciyasın tabıń. \includegraphics[width=1.95069in,height=0.58264in]{mediaBpng/image70.png} \includegraphics[width=0.85903in,height=0.23958in]{mediaBpng/image71.png}
 \\
\textbf{C1.} Eger \(\left( \xi_{1},\xi_{2} \right)\) absolyut úziliksiz tosınnanlıq vektordıń \(\xi_{1}\) hám \(\xi_{2}\) komponentaları ǵárezsiz bolıp, olardıń hárbiri standart normal bólistirilgen bolsa, onda \(\left( \xi_{1},\xi_{2} \right)\) tosınnanlıq noqattıń \(D = \left\{ (x,y):\ x^{2} + y^{2} \leq R^{2} \right\}\) oblastqa túsiw itimallıǵın tabıń.
 \\
\textbf{C2.} Eger \(\left\{ \xi_{n} \right\}\) diskret tosınnanlıq shamalar izbe-izliginiń bólistiriliw nızamları\(P\left\{ \xi_{n} = 1 \right\} = P\left\{ \xi_{n} = - 1 \right\} = \frac{1}{2} - \frac{1}{n},\) \(P\left\{ \xi_{n} = 0 \right\} = \frac{2}{n}\) bolsa, onda \(\xi_{n}\overset{d}{\rightarrow}\xi\) bolatuǵın \(\xi\) tosınnanlıq shamanıń bólistiriw funkciyasın tabıń.
 \\
\textbf{C3.} Eger \(\mathbf{\xi}_{\mathbf{n}}\overset{\mathbf{L}^{\mathbf{2}}}{\rightarrow}\mathbf{\xi}\) bolsa, onda \(n \rightarrow \infty\) de \(\mathbf{M}\mathbf{\xi}_{\mathbf{n}}^{\mathbf{2}}\mathbf{\rightarrow M}\mathbf{\xi}^{\mathbf{2}}\) ekenligin kórsetiń.
 \\

\end{tabular}
\vspace{1cm}


\begin{tabular}{m{17cm}}
\textbf{88-variant}
\newline

\textbf{T1.} Tosınnanlı waqıya (elementar waqıyalar keńisligi, waqıyalar ústinde ámeller).
 \\
\textbf{T2.} 
Úlken sanlar nızamı (anıqlaması, Chebishev teoreması).
 \\
\textbf{A1.} 28 dana dominonıń tolıq komplektinen 7 danası tosınnan tańlanadı. Hárbir domino tasındaǵı ulıwma ochkolar qosındısı 7 den kem bolıw itimallıǵın tabıń.
 \\
\textbf{A2.} Eki oyın kubigi taslanǵanda túsken ochkolardıń biri ekinshisinen 4 ese kóp bolıw itimallıǵın tabıń.
 \\
\textbf{A3.} Jip iyiriw fabrikasında 1500 dana jańa hám 100 dana eski jip iyiriw qurılmaları bar. Bir jumıs kúninde jańa qurılma 0,002 itimallıq penen, al eski qurılma bolsa 0,30 itimallıq penen iyirilip atırǵan jipti úzip aladı. Bir jumıs kúninde tómendegi waqıyalardıń júzege asıwı itimallıqların tabıń: a) jańa qurılma 5 márte jipti úzip alıw; b) eski qurılma 20 ten 25 ǵa shekemgi jipti úzip alıw. 
 \\
\textbf{B1.} Tosinnanli $\xi$ shamasiniń bólistiriw tiǵizliǵi \(\mathbf{f}\mathbf{(}\mathbf{x}\mathbf{)}\mathbf{=}\frac{\mathbf{1}}{\mathbf{2}}\mathbf{e}^{\mathbf{-}\left| \mathbf{x} \right|}\) bolsa, onıń matematikalıq kútiliwin tabıń.
 \\
\textbf{B2.} Fakultette 1460 student bar. Keminde 10 studenttiń tuwılǵan kúni 5 sentyabrge tuwra kelip qalıwı waqıyası itimallıǵı tabılsın.
 \\
\textbf{B3.} Eger \includegraphics[width=0.36181in,height=0.29444in]{mediaBpng/image1.png} ǵárezsiz tosınnanlıq shamalar izbe-izliginiń bólistiriliw nızamları
\includegraphics[width=1.51528in,height=0.50278in]{mediaBpng/image19.png} \includegraphics[width=1.62569in,height=0.50278in]{mediaBpng/image20.png} \includegraphics[width=0.75486in,height=0.23958in]{mediaBpng/image9.png}
bolsa, onda bul izbe-izlik úlken sanlar nızamına boysınama?
 \\
\textbf{C1.} Eger \(\xi_{1}\) hám \(\xi_{2}\) ǵárezsiz tosınnanlıq shamalardıń hárbiri \(\lbrack 0,1\rbrack\) aralıqta teń ólshemli bólistirilgen bolsa, onda \(\xi_{1} + \xi_{2}\) tosınnanlıq shamanıń tıǵızlıq funkciyasın tabıń.
 \\
\textbf{C2.} Tómende \includegraphics[width=0.46389in,height=0.25625in]{mediaCpng/image42.png} úzliksiz tosınnanlıq vektorlardıń tıǵızlıq funkciyaları berilgen. Olardıń \includegraphics[width=0.48819in,height=0.29583in]{mediaCpng/image43.png} hám \includegraphics[width=0.50417in,height=0.29583in]{mediaCpng/image44.png} marginal tıǵızlıq funkciyaların tabıń; \includegraphics[width=0.15972in,height=0.24028in]{mediaCpng/image45.png} hám \includegraphics[width=0.15972in,height=0.2in]{mediaCpng/image46.png} tosınnanlıq shamalardı ǵárezsizlikke tekseriń: \includegraphics[width=3.28819in,height=0.84792in]{mediaCpng/image60.png}
 \\
\textbf{C3.} Eger \(\xi\) tosınnanlıq shama \(\lbrack 0,\ \pi\rbrack\) aralıqta teń ólshewli bólistirilgen bolsa, onda \(M\sin\xi,\) \(D\sin\xi\) hám \(M\cos\xi,\) \(D\cos\xi\) mánislerin tabıń.
 \\

\end{tabular}
\vspace{1cm}


\begin{tabular}{m{17cm}}
\textbf{89-variant}
\newline

\textbf{T1.} Ǵárezsiz tájiriybelerdiń Bernulli sxeması (binоmiаl bólistiriliw, qásiyetleri).
 \\
\textbf{T2.} Xarakteristikalıq funkciyalar (anıqlaması, tiykarǵı qásiyetleri).
 \\
\textbf{A1.} Hárbiriniń júzege asıw itimallıǵı $p$ ǵa teń bolǵan 10 dana Bernulli tájiriybesi ótkerilgende, tómendegi waqıyalardıń itimallıqların tabıń: Sátliler menen sátsizler izbe-iz keliwi.
 \\
\textbf{A2.} 36 dana kartalar kolodasınan tosınnan alınǵan 3 dana karta ishinde anıq 2 dana valet bolıw itimallıǵın tabıń.
 \\
\textbf{A3.} $\left[ 0,2 \right]$ kesindiden tosınnan $x$ hám $y$ noqat tańlanadı. Olar ushın $\left| \begin{matrix}
   1 & x  \\
   x & y  \\
\end{matrix} \right|>0$ bolıw itimallıǵın tabıń.
 \\
\textbf{B1.} $\xi$ tosınnanlı shamanıń \emph{f}(\emph{x}) tıǵızlıq funkciyasi berilgen bolsin. Tómendegilerdi esaplań: a) C; b) \emph{F}(\emph{x}); c) M$\xi$; d) D$\xi$; e) \emph{f}(\emph{x}) hám \emph{F}(\emph{x}) grafiklarin sızıń.\(f(x) = \left\{ \begin{matrix}
\ \ \ \ \ \ \ \ 0,\ \ \ \ \ \ x < 1, \\
Ce^{1 - x},\ \ \ \ \ x \geq 1.\ \ 
\end{matrix} \right.\ \)
 \\
\textbf{B2.} \includegraphics[width=0.15972in,height=0.23958in]{mediaBpng/image49.png} úzliksiz tosınnanlıq shamanıń tıǵızlıq funkciyaları berilgen. Olarǵa sáykes \includegraphics[width=0.15972in,height=0.19653in]{mediaBpng/image50.png} tosınnanlıq shamanıń \includegraphics[width=0.50278in,height=0.30069in]{mediaBpng/image51.png} tıǵızlıq funkciyasın tabıń. \includegraphics[width=1.50278in,height=0.58889in]{mediaBpng/image82.png} \includegraphics[width=0.88333in,height=0.28194in]{mediaBpng/image83.png}
 \\
\textbf{B3.} Parametrleri (0;$\sigma$ ) bolǵan normal nizamniń dispersiyasın tabılsın.
 \\
\textbf{C1.} Eger ǵárezsiz \includegraphics[width=0.15208in,height=0.24028in]{mediaCpng/image4.png} hám \includegraphics[width=0.15208in,height=0.19167in]{mediaCpng/image5.png} úzliksiz tosınnanlıq shamalardıń hárbiri \includegraphics[width=0.44028in,height=0.21597in]{mediaCpng/image28.png} parametrli kórsetkishli nızam boyınsha bólistirilgen bolsa, onda \includegraphics[width=0.50417in,height=0.29583in]{mediaCpng/image30.png} tosınnanlıq shamanıń tıǵızlıq funkciyasın tabıń.
 \\
\textbf{C2.} Eger ǵárezsiz \includegraphics[width=0.15208in,height=0.24028in]{mediaCpng/image4.png} hám \includegraphics[width=0.15208in,height=0.19167in]{mediaCpng/image5.png} úzliksiz tosınnanlıq shamalar sáykes túrde, \includegraphics[width=0.43194in,height=0.21597in]{mediaCpng/image12.png} hám \includegraphics[width=0.60833in,height=0.26389in]{mediaCpng/image13.png} parametrli kórsetkishli nızam boyınsha bólistirilgen bolsa, onda \includegraphics[width=0.91181in,height=0.29583in]{mediaCpng/image15.png} tosınnanlıq shamanıń tıǵızlıq funkciyasın tabıń.
 \\
\textbf{C3.} Eger \(\left\{ \xi_{n} \right\}\) diskret tosınnanlıq shamalar izbe-izliginiń bólistiriliw nızamları\(P(\xi_{n} = e^{n}) = \frac{1}{n^{2}},\) \(P(\xi_{n} = 0) = 1 - \frac{1}{n^{2}}\) bolsa, onda \(\left\{ \xi_{n} \right\}\) tosınnanlıq shamalar izbe-izliginiń 0 ge bir itimallıq penen jıynaqlılıǵın kórsetiń.
 \\

\end{tabular}
\vspace{1cm}


\begin{tabular}{m{17cm}}
\textbf{90-variant}
\newline

\textbf{T1.} Bayеs formulası (gipotezalar teoreması, dálilleniwi).
 \\
\textbf{T2.} Kompoziciyalıq formulalar \\
\textbf{A1.} Shegaralıq bahalaw jumısın tapsırıwǵa kelgen 10 studentten ibarat toparda úshewi ayrıqsha, tórtewi jaqsı, ekewi qanaatlandırarlı hám birewi qanaatlandırarsız tayarlanǵan. Shegaralıq bahalaw jumısınıń variantlarında 20 dana soraw bar. Ayrıqsha tayarlanǵan student barlıq 20 sorawǵa, jaqsı tayarlanǵanı 16 sorawǵa, qanaatlandırarlı tayarlanǵanı 10 sorawǵa, qanaatlandırarsız tayarlanǵanı 5 sorawǵa juwap bere aladı. a) Bul studentlerden qálegen birewi berilgen úsh sorawǵa da durıs juwap beriw itimallıǵın tabıń. b) Sol durıs juwap bergen studenttiń qanaatlandırarsız tayarlanǵan student bolıw itimallıǵın tabıń.
 \\
\textbf{A2.} Balalar baqshasında 300 bala tárbiyalanadı. Tómendegi waqıyalardıń júzege asıwı itimallıqların tabıń: a) anıq eki bala 1-martta tuwılǵan; b) jazda 47 den 52 ge shekem bala tuwılǵan.
 \\
\textbf{A3.} Úsh oyın kubigi taslanǵanda túsken ochkolardıń qosındısı 16 dan artıq bolmaw itimallıǵın tabıń.
 \\
\textbf{B1.} Eger \includegraphics[width=0.36181in,height=0.29444in]{mediaBpng/image1.png} ǵárezsiz tosınnanlıq shamalar izbe-izligi \includegraphics[width=0.52153in,height=0.54583in]{mediaBpng/image45.png} aralıqta teń ólshemli bólistirilgen bolsa, onda bul izbe-izlik úlken sanlar nızamına boysınama?
 \\
\textbf{B2.} $\xi$ tosınnanlı shamanıń \emph{f}(\emph{x}) tıǵızlıq funkciyasi berilgen bolsin. Tómendegilerdi esaplań: a) C; b) \emph{F}(\emph{x}); c) M$\xi$; d) D$\xi$; e) \emph{f}(\emph{x}) hám \emph{F}(\emph{x}) grafiklarin sızıń.\(f(x) = \left\{ \begin{matrix}
C/(1 + x^{2}),\ \ \ \ x \in \lbrack 0,\sqrt{3}\rbrack, \\
\ \ \ \ \ \ \ \ 0,\ \ \ \ \ \ \ \ \ \ \ x \notin \lbrack 0,\sqrt{3}\rbrack.\ \ 
\end{matrix} \right.\ \)
 \\
\textbf{B3.} \(M(x,\ y)\) noqat tosınnanlı túrde \(0\  \leq \ x\  \leq \ 1,\ 0\  \leq \ y\  \leq \ 1\) kvadratqa taslandı. \(\min(x,\ y) \leq a\) bolsa, \(a \in (0;1\rbrack\)bolıwı itimallıǵın tabıń.
 \\
\textbf{C1.} Eger \(\xi_{1}\) hám \(\xi_{2}\) sáykes túrde \(\lambda_{1}\) hám \(\lambda_{2}\) parametrli Puasson bólistiriliwine iye bolǵan ǵárezsiz tosınnanlıq shamalar bolsa, onda \(\xi_{1} + \xi_{2}\) tosınnanlıq shamanıń bólistiriliwin tabıń.
 \\
\textbf{C2.} Tómende \includegraphics[width=0.46389in,height=0.25625in]{mediaCpng/image42.png} úzliksiz tosınnanlıq vektorlardıń tıǵızlıq funkciyaları berilgen. Olardıń \includegraphics[width=0.48819in,height=0.29583in]{mediaCpng/image43.png} hám \includegraphics[width=0.50417in,height=0.29583in]{mediaCpng/image44.png} marginal tıǵızlıq funkciyaların tabıń; \includegraphics[width=0.15972in,height=0.24028in]{mediaCpng/image45.png} hám \includegraphics[width=0.15972in,height=0.2in]{mediaCpng/image46.png} tosınnanlıq shamalardı ǵárezsizlikke tekseriń: \includegraphics[width=4.04028in,height=0.67986in]{mediaCpng/image65.png}
 \\
\textbf{C3.} Eger \(\xi\) tosınnanlıq shama standart Koshi bólistiriliwine iye bolsa, onda \(M\min\left( |\xi|,1 \right)\) mánisin tabıń.
 \\

\end{tabular}
\vspace{1cm}


\begin{tabular}{m{17cm}}
\textbf{91-variant}
\newline

\textbf{T1.} Bernulli sxemаsı ushın limit teоremаlаr (Muavr-Laplas integrallıq teoreması, qásiyetleri).
 \\
\textbf{T2.} Tıǵızlıq funkciyası (anıqlaması, tiykarǵıqásiyetleri).
 \\
\textbf{A1.} Studentler eki jıl dawamında matematikalıq kitaplardan hárbirinde 20 máseleni óz ishine alǵan matematikadan 15 tipikalıq esaplardı orınlaydı. Kompyuterde matematikalıq paket járdeminde máseleni nadurıs sheshiwi itimallıǵı 0,01 ge, paket járdemisiz 0,2 ge teń. Úsh jıl ishinde tómendegi waqıyalardıń júzege asıwı itimallıqların tabıń: a) matematikalıq paketten turaqlı túrde paydalanatuǵın student 5 ten kóp bolmaǵan máselelerdi nadurıs sheshken bolsa; b) matematikalıq paketten paydalanbaytuǵın student 50 dan 70 ge shekem máseleni nadurıs sheshedi.
 \\
\textbf{A2.} $\left[ 0,1 \right]$ kesindiden tosınnan eki noqat tańlanadı. Birinshi hám ekinshi noqatlardıń koordinataları kvadratlarınıń ayırması $0,25$ ten úlken bolıw itimallıǵın tabıń.
 \\
\textbf{A3.} Jámi 10 bala hám 12 qız bolǵan studentler toparınan 7 student sorawnama ótkeriw ushın tosınnan tańlap alındı. Olar ishinde 3 bala hám 4 qız bolıw itimallıǵın tabıń.
 \\
\textbf{B1.} \(f(x) = C \cdot e^{- \frac{x^{2}}{m}}\) tiǵizliq funkciyasi boliwi ushin \emph{C} nege teń boliwi kerek?
 \\
\textbf{B2.} $\xi$ tosınnanlı shamanıń \emph{f}(\emph{x}) tıǵızlıq funkciyasi berilgen bolsin. Tómendegilerdi esaplań: a) C; b) \emph{F}(\emph{x}); c) M$\xi$; d) D$\xi$; e) \emph{f}(\emph{x}) hám \emph{F}(\emph{x}) grafiklarin sızıń.\(f(x) = \left\{ \begin{matrix}
C(1 - x/3),\ \ \ \ x \in \lbrack 0,3\rbrack, \\
\ \ \ \ \ \ \ \ 0,\ \ \ \ \ \ \ \ \ \ \ x \notin \lbrack 0,3\rbrack.\ \ 
\end{matrix} \right.\ \)
 \\
\textbf{B3.} \includegraphics[width=0.15972in,height=0.23958in]{mediaBpng/image49.png} úzliksiz tosınnanlıq shamanıń tıǵızlıq funkciyaları berilgen. Olarǵa sáykes \includegraphics[width=0.15972in,height=0.19653in]{mediaBpng/image50.png} tosınnanlıq shamanıń \includegraphics[width=0.50278in,height=0.30069in]{mediaBpng/image51.png} tıǵızlıq funkciyasın tabıń. \includegraphics[width=2.17153in,height=0.58264in]{mediaBpng/image75.png} \includegraphics[width=0.91389in,height=0.50278in]{mediaBpng/image76.png}
 \\
\textbf{C1.} Eger \(\xi\) tosınnanlıq shama standart Koshi bólistiriliwine iye bolsa, onda \(M\min\left( |\xi|,1 \right)\) mánisin tabıń.
 \\
\textbf{C2.} Eger \(\xi_{1},...,\xi_{n}\) ǵárezsiz birdey bólistirilgen tosınnanlıq shamalar \(F(x)\) bólistiriw hám \(f(x)\) tıǵızlıq funkciyalarǵa iye bolsa, onda \(\eta_{1} = \max\left( \xi_{1},...,\xi_{n} \right)\) hám \(\eta_{2} = \min\left( \xi_{1},...,\xi_{n} \right)\) tosınnanlıq shamalardıń bólistiriw hám tıǵızlıq funkciyaların tabıń.
 \\
\textbf{C3.} Eger \(\left\{ \xi_{n} \right\}\) ǵárezsiz hám \(\mathbf{\lbrack 0,1\rbrack}\) aralıqta teń ólshemli bólistirilgen tosınnanlıq shamalar izbe-izligi bolsa, onda \(\left\{ \mathbf{\xi}_{\mathbf{(}\mathbf{n}\mathbf{)}}\mathbf{=}\mathbf{\max}\mathbf{\{}\mathbf{\xi}_{\mathbf{1}}\mathbf{,...,}\mathbf{\xi}_{\mathbf{n}}\mathbf{\}} \right\}\) izbe-izlik 1 ge itimallıq boyınsha jıynaqlılıǵın kórsetiń.
 \\

\end{tabular}
\vspace{1cm}


\begin{tabular}{m{17cm}}
\textbf{92-variant}
\newline

\textbf{T1.} Bernulli sxemаsı ushın limit teоremаlаr (Puasson bólistiriliwi, qásiyetleri).
 \\
\textbf{T2.} Tosınnanlı shamanıń dispersiyası (anıqlaması, qásiyetleri).
 \\
\textbf{A1.} Eki oyın kubigi taslanǵanda túsken ochkolardıń qosındısı 4 ten úlken, bıraq 7 den kishi bolıw itimallıǵın tabıń.
 \\
\textbf{A2.} Hárbiriniń júzege asıw itimallıǵı $p$ ǵa teń bolǵan 10 dana Bernulli tájiriybesi ótkerilgende, tómendegi waqıyalardıń itimallıqların tabıń: Sátliler sanı 3 dana, sonıń menen birge, olardıń barlıǵı sońǵı úsh tájiriybede ámelge asıwı.
 \\
\textbf{A3.} Úsh qutınıń hárbirinde $n$ dana aq hám $m$ dana qara sharlar bar. Birinshi hám ekinshi qutıdan tosınnan 1 shardan alınıp, úshinshi qutıǵa salındı. Keyin, úshinshi qutıdan tosınnan bir shar alındı. а) Bul shardıń aq bolıw itimallıǵın tabıń. b) Sol shar aq bolsa, dáslepki eki qutıdan alınǵan sharlardıń aq bolıw itimallıǵın tabıń.
 \\
\textbf{B1.} Eger \includegraphics[width=0.36181in,height=0.29444in]{mediaBpng/image1.png} ǵárezsiz tosınnanlıq shamalar izbe-izliginiń bólistiriliw nızamları
\includegraphics[width=2.51528in,height=0.52153in]{mediaBpng/image40.png} \includegraphics[width=2.59514in,height=0.47847in]{mediaBpng/image41.png} \includegraphics[width=0.75486in,height=0.23958in]{mediaBpng/image42.png}
bolsa, onda bul izbe-izlik úlken sanlar nızamına boysınama?
 \\
\textbf{B2.} Bir nishanaǵa úsh márte atiladi. I, II hám III márte atqandaǵi tiyiw itimalliqlari sáykes: p\textsubscript{1} =0,3; p\textsubscript{2} =0,4; p\textsubscript{3}=0,6. Usi úsh márte atıwdıń nátiyjesinde nishanada eń bolmaǵanda bir oq izi boliwiniń itimallıǵın tabıń.
 \\
\textbf{B3.} $\xi$ tosınnanlı shamanıń \emph{f}(\emph{x}) tıǵızlıq funkciyasi berilgen bolsin. Tómendegilerdi esaplań: a) C; b) \emph{F}(\emph{x}); c) M$\xi$; d) D$\xi$; e) \emph{f}(\emph{x}) hám \emph{F}(\emph{x}) grafiklarin sızıń.\(f(x) = \left\{ \begin{matrix}
C\cos x,\ \ \ \ x \in \left\lbrack 0,\frac{\pi}{2} \right\rbrack, \\
\ \ \ \ \ \ \ \ 0,\ \ \ \ \ \ x \notin \left\lbrack 0,\frac{\pi}{2} \right\rbrack.\ \ 
\end{matrix} \right.\ \)
 \\
\textbf{C1.} Eger ǵárezsiz \includegraphics[width=0.15208in,height=0.24028in]{mediaCpng/image4.png} hám \includegraphics[width=0.15208in,height=0.19167in]{mediaCpng/image5.png} úzliksiz tosınnanlıq shamalardıń hárbiri \includegraphics[width=0.44028in,height=0.21597in]{mediaCpng/image28.png} parametrli kórsetkishli nızam boyınsha bólistirilgen bolsa, onda \includegraphics[width=0.44028in,height=0.24028in]{mediaCpng/image14.png} tosınnanlıq shamanıń tıǵızlıq funkciyasın tabıń.
 \\
\textbf{C2.} Tómende \includegraphics[width=0.46389in,height=0.25625in]{mediaCpng/image42.png} úzliksiz tosınnanlıq vektorlardıń tıǵızlıq funkciyaları berilgen. Olardıń \includegraphics[width=0.48819in,height=0.29583in]{mediaCpng/image43.png} hám \includegraphics[width=0.50417in,height=0.29583in]{mediaCpng/image44.png} marginal tıǵızlıq funkciyaların tabıń; \includegraphics[width=0.15972in,height=0.24028in]{mediaCpng/image45.png} hám \includegraphics[width=0.15972in,height=0.2in]{mediaCpng/image46.png} tosınnanlıq shamalardı ǵárezsizlikke tekseriń: \includegraphics[width=3.20833in,height=0.59167in]{mediaCpng/image50.png}
 \\
\textbf{C3.} Eger \(\left( \xi_{1},\xi_{2} \right)\) absolyut úziliksiz tosınnanlıq vektordıń tıǵızlıq funkciyası \(f(x,y) = \left\{ \begin{matrix}
Cxy,\ eger\ (x,y) \in D, \\
 \\
0,\ \ \ \ \ eger\ (x,y) \notin D,
\end{matrix} \right.\ \) bunda \(D = \left\{ (x,y):\ y > - x,\ y < 2,\ x < 0 \right\}\) bolsa, onda \(\xi_{1}\) komponentanıń shártsiz hám shártli tıǵızlıq funkciyaların tabıń. Sonıń menen birge, \(\xi_{1}\) hám \(\xi_{2}\) tosınnanlıq shamalardı ǵárezsizlikke tekseriń.
 \\

\end{tabular}
\vspace{1cm}


\begin{tabular}{m{17cm}}
\textbf{93-variant}
\newline

\textbf{T1.} Itimallıq anıqlamaları (klassikalıq, geometriyalıq anıqlamaları).
 \\
\textbf{T2.} Bólistiriw funkciyası (anıqlaması, tiykarǵı qásiyetleri).
 \\
\textbf{A1.} $\left[ 0,1 \right]$ kesindiden tosınnan eki noqat tańlanadı. Birinshi noqattıń koordinatası ekinshi noqattıń koordinatasınan kishi bolıw itimallıǵın tabıń.
 \\
\textbf{A2.} Hárbiriniń júzege asıw itimallıǵı $p$ ǵa teń bolǵan 10 dana Bernulli tájiriybesi ótkerilgende, tómendegi waqıyalardıń itimallıqların tabıń: Sátsizler sanı 4 dana.
 \\
\textbf{A3.} Eki mergen biri-birinen ǵárezsiz bir nıshanǵa bir márteden oq atadı. Birinshi mergenniń nıshanǵa tiygiziw itimallıǵı 0,8 ekinshisiniki bolsa 0,4 ke teń. a) Atıw tamam bolǵannan soń, nıshanda bir oq izi bolıwı itimallıǵın tabıń. b) Sol oq iziniń ekinshi mergenge tiyisli bolıw itimallıǵın tabıń.
 \\
\textbf{B1.} Zavod bazaǵa 5000 sipatli buyim jóneltken. Jolda buyimniń zálelleniw itimalliǵi 0,0002 ge teń. Bazaǵa 3 jaramsiz buyimniń kelip túsiw itimalliǵin tabiń.
 \\
\textbf{B2.} Eger \includegraphics[width=0.36181in,height=0.29444in]{mediaBpng/image1.png} ǵárezsiz tosınnanlıq shamalar izbe-izliginiń bólistiriliw nızamları
\includegraphics[width=1.53958in,height=0.50278in]{mediaBpng/image21.png} \includegraphics[width=1.63819in,height=0.50278in]{mediaBpng/image22.png} \includegraphics[width=0.75486in,height=0.23958in]{mediaBpng/image9.png}
bolsa, onda bul izbe-izlik úlken sanlar nızamına boysınama?
 \\
\textbf{B3.} \includegraphics[width=0.15972in,height=0.23958in]{mediaBpng/image49.png} úzliksiz tosınnanlıq shamanıń tıǵızlıq funkciyaları berilgen. Olarǵa sáykes \includegraphics[width=0.15972in,height=0.19653in]{mediaBpng/image50.png} tosınnanlıq shamanıń \includegraphics[width=0.50278in,height=0.30069in]{mediaBpng/image51.png} tıǵızlıq funkciyasın tabıń. \includegraphics[width=2.325in,height=0.84028in]{mediaBpng/image66.png} \includegraphics[width=0.54583in,height=0.53403in]{mediaBpng/image67.png}
 \\
\textbf{C1.} Eger ǵárezsiz \includegraphics[width=0.15208in,height=0.24028in]{mediaCpng/image4.png} hám \includegraphics[width=0.15208in,height=0.19167in]{mediaCpng/image5.png} úzliksiz tosınnanlıq shamalar sáykes túrde, \includegraphics[width=0.4in,height=0.25625in]{mediaCpng/image6.png} hám \includegraphics[width=0.44028in,height=0.25625in]{mediaCpng/image7.png} parametrli normal nızam boyınsha bólistirilgen bolsa, onda \includegraphics[width=0.72778in,height=0.24028in]{mediaCpng/image8.png} tosınnanlıq shamanıń tıǵızlıq funkciyasın tabıń.
 \\
\textbf{C2.} Eger \(\xi\sim N\left( a,\sigma^{2} \right)\) bolsa, onda \(\xi\) tosınnanlıq shamanıń joqarı tártipli oraylıq momentlerin tabıń.
 \\
\textbf{C3.} Eger \(\left\{ \xi_{n} \right\}\) ǵárezsiz birdey bólistirilgen tosınnanlıq shamalar izbe-izligi bolıp, onıń bólistiriw funkciyası \(F_{\xi_{1}}(x) = \left\{ \begin{matrix}
\ 1 - e^{\lambda - x},\ \ eger\ \ x \geq \lambda, \\
 \\
\ \ \ \ \ \ 0,\ \ \ \ \ \ \ \ \ \ \ eger\ \ x < \lambda
\end{matrix} \right.\ \) bolsa, onda \(\left\{ \eta_{n} \right\} = \left\{ min(\xi_{1},...,\xi_{n}) \right\}\) izbe-izliktiń \(\mathbf{\lambda}\) ǵa bir itimallıq penen jıynaqlılıǵın kórsetiń.
 \\

\end{tabular}
\vspace{1cm}


\begin{tabular}{m{17cm}}
\textbf{94-variant}
\newline

\textbf{T1.} Bernulli sxemаsı ushın limit teоremаlаr (Muavr-Laplas lokallıq teoreması, qásiyetleri).
 \\
\textbf{T2.} 
Úlken sanlar nızamı (anıqlaması, Chebishev teoreması).
 \\
\textbf{A1.} Eki oyın kubigi taslanǵanda túsken ochkolardıń qosındısı 6 dan úlken bolıw itimallıǵın tabıń.
 \\
\textbf{A2.} Bazıbir qalada solaqaylar ortasha esapta $1$, sol hám oń qollarına teńdey iyelik qılatuǵın adamlar $10$, al qalǵanları ońaqaylar. Jámi 200 adam arasında tómendegi waqıyalardıń júzege asıwı itimallıqların tabıń: a) keminde tórt solaqay boladı; b) 18 den 23 ke shekem sol hám oń qollarına teńdey iyelik qılatuǵın adamlar boladı.
 \\
\textbf{A3.} Toparda 25 student bolıp, olardan 6 student ayrıqsha bahaǵa oqıydı. Dizim boyınsha 10 student ajıratılǵan. Ajıratılǵanlar ishinde ayrıqsha bahaǵa oqıytuǵın 3 student bolıw itimallıǵın tabıń.
 \\
\textbf{B1.} {[}-1;1{]} kesindide teń ólshewli bólistirilgen tosınnanlı shamanıń xarakteristikalıq funkciyasın tabıń.
 \\
\textbf{B2.} Eger \includegraphics[width=0.36181in,height=0.29444in]{mediaBpng/image1.png} ǵárezsiz hám birdey bólistirilgen tosınnanlıq shamalar izbe-izligi \includegraphics[width=0.44792in,height=0.21458in]{mediaBpng/image43.png} parametrli kórsetkishli bólistiriliwine iye bolsa, onda bul izbe-izlik úlken sanlar nızamına boysınama?
 \\
\textbf{B3.} \(f(x) = \frac{C}{e^{x} + e^{- x}}\) bólistiriw tıǵızlıǵı bolıwı ushın \(C\) nege teń bolıwı kerek?
 \\
\textbf{C1.} 
Tómende \includegraphics[width=0.46389in,height=0.25625in]{mediaCpng/image42.png} úzliksiz tosınnanlıq vektorlardıń tıǵızlıq funkciyaları berilgen. Olardıń \includegraphics[width=0.48819in,height=0.29583in]{mediaCpng/image43.png} hám \includegraphics[width=0.50417in,height=0.29583in]{mediaCpng/image44.png} marginal tıǵızlıq funkciyaların tabıń; \includegraphics[width=0.15972in,height=0.24028in]{mediaCpng/image45.png} hám \includegraphics[width=0.15972in,height=0.2in]{mediaCpng/image46.png} tosınnanlıq shamalardı ǵárezsizlikke tekseriń: \includegraphics[width=3.13611in,height=0.53611in]{mediaCpng/image47.png}
 \\
\textbf{C2.} Tómende \includegraphics[width=0.46389in,height=0.25625in]{mediaCpng/image42.png} úzliksiz tosınnanlıq vektorlardıń tıǵızlıq funkciyaları berilgen. Olardıń \includegraphics[width=0.48819in,height=0.29583in]{mediaCpng/image43.png} hám \includegraphics[width=0.50417in,height=0.29583in]{mediaCpng/image44.png} marginal tıǵızlıq funkciyaların tabıń; \includegraphics[width=0.15972in,height=0.24028in]{mediaCpng/image45.png} hám \includegraphics[width=0.15972in,height=0.2in]{mediaCpng/image46.png} tosınnanlıq shamalardı ǵárezsizlikke tekseriń: \includegraphics[width=3.20833in,height=0.84792in]{mediaCpng/image62.png}
 \\
\textbf{C3.} Eger ǵárezsiz \includegraphics[width=0.15208in,height=0.24028in]{mediaCpng/image4.png} hám \includegraphics[width=0.15208in,height=0.19167in]{mediaCpng/image5.png} úzliksiz tosınnanlıq shamalar sáykes túrde, \includegraphics[width=0.4in,height=0.24028in]{mediaCpng/image34.png} aralıqta teń ólshemli hám \includegraphics[width=0.44028in,height=0.21597in]{mediaCpng/image35.png} parametrli kórsetkishli nızam boyınsha bólistirilgen bolsa, onda \includegraphics[width=0.44028in,height=0.24028in]{mediaCpng/image14.png} tosınnanlıq shamanıń tıǵızlıq funkciyasın tabıń.
 \\

\end{tabular}
\vspace{1cm}


\begin{tabular}{m{17cm}}
\textbf{95-variant}
\newline

\textbf{T1.} Itimallıqlar teoriyası aksiomaları (ólshewli keńislik, itimallıq keńisligi).
 \\
\textbf{T2.} Tosınnanlı shamanıń joqarı tártipli momentleri (baslanǵısh hám oraylıq momentleri, qásiyetleri).
 \\
\textbf{A1.} Qutıda 28 dana birdey sharlar bolıp, olardıń 19 danası qızıl hám 9 danası kók reńdegi sharlar. Tosınnan alınǵan 3 dana shardıń 2 danası kók shar bolıw itimallıǵın tabıń.
 \\
\textbf{A2.} $\left[ 0,2 \right]$ kesindiden tosınnan eki noqat tańlanadı. Olardıń koordinataları kóbeymesi 2 den úlken bolıw itimallıǵın tabıń.
 \\
\textbf{A3.} Kitaptıń bir betinde keminde bir baspa qáteligi bolıw itimallıǵı 0,02 ge hám tártip qáteligi bolıw itimallıǵı bolsa 0,4 ke teń. Jámi 400 betli kitapta tómendegi waqıyalardıń júzege asıw itimallıqların tabıń: a) keminde bes bette baspa qáteligi bolıwı; b) 170 ten 180 ge shekem betlerde tártip qáteligi bolıwı.
 \\
\textbf{B1.} $\xi$ tosınnanlı shamanıń \emph{f}(\emph{x}) tıǵızlıq funkciyasi berilgen bolsin. Tómendegilerdi esaplań: a) C; b) \emph{F}(\emph{x}); c) M$\xi$; d) D$\xi$; e) \emph{f}(\emph{x}) hám \emph{F}(\emph{x}) grafiklarin sızıń.\(f(x) = \left\{ \begin{matrix}
C(1 - |x|),\ \ \ \ x \in \lbrack - 1,1\rbrack, \\
\ \ \ \ \ \ \ \ 0,\ \ \ \ \ \ \ \ \ x \notin \lbrack - 1,1\rbrack.\ \ 
\end{matrix} \right.\ \)
 \\
\textbf{B2.} Oqıtıwshı joqarı matematikadan shegaralıq bahalaw alıw ushın 50 soraw tayarlaǵan. Olardıń ishinde differencial esabınan 20 soraw, integral esabınan 18 soraw, itimallıqlar teoriyasınan 12 soraw bar. Student differencial esabınan 18 sorawǵa, integral esabınan 15 sorawǵa, itimallıqlar teoriyasınan 10 sorawǵa juwap bere alatuǵın bolsa, onıń dus kelgen bir sorawǵa juwap berip, shegaralıq bahalaw tapsırıwınıń itimallıǵın tabıń.
 \\
\textbf{B3.} \includegraphics[width=0.15972in,height=0.23958in]{mediaBpng/image49.png} úzliksiz tosınnanlıq shamanıń tıǵızlıq funkciyaları berilgen. Olarǵa sáykes \includegraphics[width=0.15972in,height=0.19653in]{mediaBpng/image50.png} tosınnanlıq shamanıń \includegraphics[width=0.50278in,height=0.30069in]{mediaBpng/image51.png} tıǵızlıq funkciyasın tabıń. \includegraphics[width=2.44167in,height=0.84028in]{mediaBpng/image60.png} \includegraphics[width=0.88333in,height=0.23958in]{mediaBpng/image61.png}
 \\
\textbf{C1.} Kóp ólshemli tıǵızlıq funkciyası óziniń marginal tıǵızlıq funkciyaları arqalı bir mánisli anıqlanbaytuǵınlıǵın kórsetiń.
 \\
\textbf{C2.} 
Eger \(\xi\) tosınnanlıq shama hám \(\left\{ \xi_{n} \right\}\) tosınnanlıq shamalar izbe-izligi ǵárezsiz birdey standart normal bólistirilgen bolsa, onda \(\left\{ \mathbf{\eta}_{\mathbf{n}} \right\}\mathbf{=}\left\{ \frac{\mathbf{\xi}\sqrt{\mathbf{n}}}{\sqrt{\mathbf{\xi}_{\mathbf{1}}^{\mathbf{2}}\mathbf{+}\mathbf{...}\mathbf{+}\mathbf{\xi}_{\mathbf{n}}^{\mathbf{2}}}} \right\}\) tosınnanlıq shamalar izbe-izligining limit bólistiriw funkciyası standart normal bólistiriliw bolıwın kórsetiń.
 \\
\textbf{C3.} 
\(\xi\) diskret tosınnanlıq shama \(x_{i} = ( - 1)^{i}i\) mánislerdi \(p_{i} = \frac{1}{i(i + 1)},\) \(\ \ i = 1,\ 2,\ ...\) itimallıqlar menen qabıl etse, onıń matematikalıq kútiliwin tabıń.
 \\

\end{tabular}
\vspace{1cm}


\begin{tabular}{m{17cm}}
\textbf{96-variant}
\newline

\textbf{T1.} Shártli itimallıq (anıqlaması, kóbеytiw tеorеması).
 \\
\textbf{T2.} Tosınnanlı shamanıń matematikalıq kútiliwi. (anıqlaması, qásiyetleri)
 \\
\textbf{A1.} Oyın kubigi taslanǵanda pútin ochkonıń túsiw itimallıǵın tabıń.
 \\
\textbf{A2.} Samolyotqa samolyottan 4 dana ǵárezsiz oq atıldı. Hárbir atılǵan oqtıń tiyiw itimallıǵı 0,3 ke teń. Samolyottı joq etiw ushın (tolıǵı menen isten shıǵarıw ushın) 2 márte tiyiw jetkilikli, 1 márte tiygende 0,6 ǵa teń itimallıq penen isten shıǵadı. a) Tórt márte oq atıw nátiyjesinde samolyottıń tolıǵı menen isten shıǵıw itimallıǵın tabıń. b) Eger samolyot tolıǵı menen isten shıqqan bolsa, ol bir oq tiyip isten shıqqan bolıw itimallıǵın tabıń.
 \\
\textbf{A3.} Hárbiriniń júzege asıw itimallıǵı $p$ ǵa teń bolǵan 10 dana Bernulli tájiriybesi ótkerilgende, tómendegi waqıyalardıń itimallıqların tabıń: Tájiriybelerdiń birinshi yarımındaǵı sátliler sanı, tájiriybelerdiń ekinshi yarımındaǵı sátliler sanınan kem.
 \\
\textbf{B1.} Tosinnanli $\xi$ shamasiniń bólistiriw tiǵizliǵi: \(\mathbf{f}\mathbf{(}\mathbf{x}\mathbf{)}\mathbf{=}\left\{ \begin{matrix}
\mathbf{0,}\mathbf{x <}\mathbf{0} \\
\mathbf{2}\mathbf{e}^{\mathbf{-}\mathbf{2}\mathbf{x}}\mathbf{,}\mathbf{x \geq}\mathbf{0}
\end{matrix} \right.\ \) bolǵanda, M$\xi$ hám D$\xi$ lerdi tabiń.
 \\
\textbf{B2.} Barlıq tárepi boyalǵan kub mıń dana birdey ólshemdegi kubiklerge bólingen hám aralastırıp jiberilgen. Tosınnan alınǵan kubiktiń a) bir tárepi; b) eki tárepi; c) úsh tárepi boyalǵan bolıwı itimallıǵın tabıń.
 \\
\textbf{B3.} $\xi$ tosınnanlı shamanıń \emph{f}(\emph{x}) tıǵızlıq funkciyasi berilgen bolsin. Tómendegilerdi esaplań: a) C; b) \emph{F}(\emph{x}); c) M$\xi$; d) D$\xi$; e) \emph{f}(\emph{x}) hám \emph{F}(\emph{x}) grafiklarin sızıń.\(f(x) = \left\{ \begin{matrix}
\ \ \ \ \ \ \ \ 0,\ \ \ \ \ \ x \leq 0, \\
Cxe^{- 0.5x},\ \ \ \ \ x > 0.\ \ 
\end{matrix} \right.\ \)
 \\
\textbf{C1.} Eger \(\xi_{1}\) hám \(\xi_{2}\) ǵárezsiz tosınnanlıq shamalardıń hárbiri standart normal bólistirilgen bolsa, onda \(\xi_{1} + \xi_{2}\) tosınnanlıq shamanıń tıǵızlıq funkciyasın tabıń.
 \\
\textbf{C2.} Eger ǵárezsiz \includegraphics[width=0.15208in,height=0.24028in]{mediaCpng/image4.png} hám \includegraphics[width=0.15208in,height=0.19167in]{mediaCpng/image5.png} úzliksiz tosınnanlıq shamalardıń hárbiri \includegraphics[width=0.44028in,height=0.21597in]{mediaCpng/image28.png} parametrli kórsetkishli nızam boyınsha bólistirilgen bolsa, onda \includegraphics[width=0.44028in,height=0.24028in]{mediaCpng/image29.png} tosınnanlıq shamanıń tıǵızlıq funkciyasın tabıń.
 \\
\textbf{C3.} Tómende \includegraphics[width=0.46389in,height=0.25625in]{mediaCpng/image42.png} úzliksiz tosınnanlıq vektorlardıń tıǵızlıq funkciyaları berilgen. Olardıń \includegraphics[width=0.48819in,height=0.29583in]{mediaCpng/image43.png} hám \includegraphics[width=0.50417in,height=0.29583in]{mediaCpng/image44.png} marginal tıǵızlıq funkciyaların tabıń; \includegraphics[width=0.15972in,height=0.24028in]{mediaCpng/image45.png} hám \includegraphics[width=0.15972in,height=0.2in]{mediaCpng/image46.png} tosınnanlıq shamalardı ǵárezsizlikke tekseriń: \includegraphics[width=2.83194in,height=0.63194in]{mediaCpng/image58.png}
 \\

\end{tabular}
\vspace{1cm}


\begin{tabular}{m{17cm}}
\textbf{97-variant}
\newline

\textbf{T1.} Tosınnanlı waqıya (elementar waqıyalar keńisligi, waqıyalar ústinde ámeller).
 \\
\textbf{T2.} Tiykarǵı diskret bólistiriliwler (Binomial, Puasson hám geometriyalıq bólistiriliwler).
 \\
\textbf{A1.} Hárbiriniń júzege asıw itimallıǵı $p$ ǵa teń bolǵan 10 dana Bernulli tájiriybesi ótkerilgende, tómendegi waqıyalardıń itimallıqların tabıń: Sátliler sanı 5 den artıq, biraq 8 den kem.
 \\
\textbf{A2.} Eki oyın kubigi taslanǵanda túsken ochkolardıń ayırması 3 ten úlken bolıw itimallıǵın tabıń.
 \\
\textbf{A3.} Bazıbir qalada solaqaylar ortasha esapta $1,5$, sol hám oń qollarına teńdey iyelik qılatuǵın adamlar $9$, al qalǵanları ońaqaylar. Jámi 300 adam arasında tómendegi waqıyalardıń júzege asıwı itimallıqların tabıń: a) keminde tórt solaqay boladı; b) 15 ten 20 ǵa shekem sol hám oń qollarına teńdey iyelik qılatuǵın adamlar boladı.
 \\
\textbf{B1.} Eger \includegraphics[width=0.36181in,height=0.29444in]{mediaBpng/image1.png} ǵárezsiz tosınnanlıq shamalar izbe-izligi \includegraphics[width=0.55208in,height=0.29444in]{mediaBpng/image46.png} aralıqta teń ólshemli bólistirilgen bolsa, onda bul izbe-izlik úlken sanlar nızamına boysınama?
 \\
\textbf{B2.} 
\includegraphics[width=0.15972in,height=0.23958in]{mediaBpng/image49.png} úzliksiz tosınnanlıq shamanıń tıǵızlıq funkciyaları berilgen. Olarǵa sáykes \includegraphics[width=0.15972in,height=0.19653in]{mediaBpng/image50.png} tosınnanlıq shamanıń \includegraphics[width=0.50278in,height=0.30069in]{mediaBpng/image51.png} tıǵızlıq funkciyasın tabıń. \includegraphics[width=2.23333in,height=0.84028in]{mediaBpng/image52.png} \includegraphics[width=0.88333in,height=0.23958in]{mediaBpng/image53.png}
 \\
\textbf{B3.} Birdey kartochkalarǵa jazilǵan A,A,A,A,P,R,Q,Q,Q,L háriplerinen tosinnan alinǵan kartochkalardi aliniw tártibinde jaylastiriwdan «QARAQALPAQ» sóziniń kelip shiǵiw itimalliǵin tabiń
 \\
\textbf{C1.} Eger \(\left\{ \xi_{n} \right\}\) ǵárezsiz tosınnanlıq shamalar izbe-izliginiń bólistiriw funkciyaları \(F_{n}(x) = \left\{ \begin{matrix}
\ 1 - \frac{1}{x + n},\ \ eger\ \ x > 0 \\
 \\
 \\
\ \ \ \ \ \ \ \ \ \ 0,\ \ \ \ \ \ \ \ \ \ \ eger\ \ x \leq 0
\end{matrix} \right.\ \) bolsa, onda bul izbe-izliktiń 0 ge itimallıq boyınsha jıynaqlılıǵın kórsetiń.
 \\
\textbf{C2.} Eger \(\xi\sim E(\lambda)\) bolsa, onda \(\xi\) tosınnanlıq shamanıń joqarı tártipli baslanǵısh momentlerin tabıń.
 \\
\textbf{C3.} Eger \(\xi\) tosınnanlıq shama \(\lbrack 0,\ \pi\rbrack\) aralıqta teń ólshewli bólistirilgen bolsa, onda \(M\sin\xi,\) \(D\sin\xi\) hám \(M\cos\xi,\) \(D\cos\xi\) mánislerin tabıń.
 \\

\end{tabular}
\vspace{1cm}


\begin{tabular}{m{17cm}}
\textbf{98-variant}
\newline

\textbf{T1.} Waqıyalar algebrası ($\sigma$-algebra, minimal $\sigma$-algebra).
 \\
\textbf{T2.} Oraylıq limit teorema (anıqlaması, ǵárezsiz birdey bólistirilgen tosınnanlı shamalar ushın).
 \\
\textbf{A1.} $x\in \left[ 0,2\pi  \right]$ ushın $2co{{s}^{2}}x-5cosx+1>0$ bolıw itimallıǵın tabıń.
 \\
\textbf{A2.} Birinshi qutıda 3 dana aq hám 5 dana qara sharlar bar, ekinshi qutıda 6 dana aq hám 8 dana qara sharlar bar. Birinshi qutıdan tosınnan 2 shar alınıp, ekinshi qutıǵa salındı. Keyin, birinshi qutıdan tosınnan 1 shar alındı. а) Bul shardıń aq bolıw itimallıǵın tabıń. b) Sol shar aq bolsa, birinshi qutıdan alınǵan sharlardıń aq bolıw itimallıǵın tabıń.
 \\
\textbf{A3.} Firmada 11 erkek hám 6 hayal jumısshı isleydi. Tosınnan 5 jumısshı ajıratılıp alındı. Ajıratılıp alınǵan jumısshılardıń barlıǵı hayal bolıw itimallıǵın tabıń.
 \\
\textbf{B1.} $\xi$ tosınnanlı shamanıń \emph{f}(\emph{x}) tıǵızlıq funkciyasi berilgen bolsin. Tómendegilerdi esaplań: a) C; b) \emph{F}(\emph{x}); c) M$\xi$; d) D$\xi$; e) \emph{f}(\emph{x}) hám \emph{F}(\emph{x}) grafiklarin sızıń.\(f(x) = \left\{ \begin{matrix}
\ \ \ \ \ \ \ \ 0,\ \ \ \ \ \ x < 0, \\
C/(x + 1)^{5},\ \ \ \ \ x \geq 0.\ \ 
\end{matrix} \right.\ \)
 \\
\textbf{B2.} Eger \includegraphics[width=0.36181in,height=0.29444in]{mediaBpng/image1.png} ǵárezsiz tosınnanlıq shamalar izbe-izliginiń bólistiriliw nızamları
\includegraphics[width=1.76042in,height=0.50278in]{mediaBpng/image23.png} \includegraphics[width=1.63819in,height=0.50278in]{mediaBpng/image24.png} \includegraphics[width=0.75486in,height=0.23958in]{mediaBpng/image9.png}
bolsa, onda bul izbe-izlik úlken sanlar nızamına boysınama?
 \\
\textbf{B3.} \includegraphics[width=0.15972in,height=0.23958in]{mediaBpng/image49.png} úzliksiz tosınnanlıq shamanıń tıǵızlıq funkciyaları berilgen. Olarǵa sáykes \includegraphics[width=0.15972in,height=0.19653in]{mediaBpng/image50.png} tosınnanlıq shamanıń \includegraphics[width=0.50278in,height=0.30069in]{mediaBpng/image51.png} tıǵızlıq funkciyasın tabıń. \includegraphics[width=2.62569in,height=0.65in]{mediaBpng/image84.png} \includegraphics[width=0.66875in,height=0.53403in]{mediaBpng/image85.png}
 \\
\textbf{C1.} Eger \(\left( \xi_{1},\xi_{2} \right)\) tosınnanlıq vektordıń bólistiriw funkciyası \(F(x,y) = \left\{ \begin{matrix}
\left( 1 - 2^{- x^{2}} \right)\left( 1 - 2^{- 2y^{2}} \right),\ \ eger\ \ x \geq 0,\ y \geq 0, \\
 \\
 \\
\ \ \ \ \ \ \ \ \ \ \ \ \ \ 0,\ \ \ \ \ \ \ \ \ \ \ \ \ \ \ \ \ \ \ \ \ \ \ basqa\ hallarda
\end{matrix} \right.\ \) bolsa, onda \(F\left( x/\xi_{2} < y \right)\) hám \(F\left( y/\xi_{1} < x \right)\) shártli bólistiriw funkciyaların tabıń. Sonıń menen birge, \(\xi_{1}\) hám \(\xi_{2}\) tosınnanlıq shamalardı ǵárezsizlike tekseriń.
 \\
\textbf{C2.} Tómende \includegraphics[width=0.46389in,height=0.25625in]{mediaCpng/image42.png} úzliksiz tosınnanlıq vektorlardıń tıǵızlıq funkciyaları berilgen. Olardıń \includegraphics[width=0.48819in,height=0.29583in]{mediaCpng/image43.png} hám \includegraphics[width=0.50417in,height=0.29583in]{mediaCpng/image44.png} marginal tıǵızlıq funkciyaların tabıń; \includegraphics[width=0.15972in,height=0.24028in]{mediaCpng/image45.png} hám \includegraphics[width=0.15972in,height=0.2in]{mediaCpng/image46.png} tosınnanlıq shamalardı ǵárezsizlikke tekseriń: \includegraphics[width=2.52778in,height=0.59167in]{mediaCpng/image55.png}
 \\
\textbf{C3.} Oraylıq limit teorema járdeminde tómendegi teńlikti dálilleń: \(\lim_{n \rightarrow \infty}e^{- n}\sum_{k = 1}^{n}\frac{n^{k}}{k!} = \frac{1}{2}.\)
 \\

\end{tabular}
\vspace{1cm}


\begin{tabular}{m{17cm}}
\textbf{99-variant}
\newline

\textbf{T1.} Tolıq itimallıq formulası (waqıyalardıń tolıq gruppası, dálilleniwi).
 \\
\textbf{T2.} Tiykarǵı аbsоlyut úzliksiz bólistiriliwler (nоrmаl bólistiriw, teń ólshewli bólistiriw, kórsetkishli bólistiriw). 
 \\
\textbf{A1.} 
Hárbiriniń júzege asıw itimallıǵı $p$ ǵa teń bolǵan 10 dana Bernulli tájiriybesi ótkerilgende, tómendegi waqıyalardıń itimallıqların tabıń: Sátliler sanı 7 dana.
 \\
\textbf{A2.} Zavodta avtomat basqarılatuǵın 14 dana hám qolda basqarılatuǵın 6 dana qurılmalar bar. Avtomat basqarılatuǵın qurılmalar ushın standart bolmaǵan ónimlerdi islep shıǵarıw itimallıǵı $0,001$ ge, al qolda basqarılatuǵın qurılmalar ushın bolsa $0,002$ ge teń. a) Laboratoriya analizine tosınnan alınǵan ónimniń standart bolmaǵan bolıwı itimallıǵın tabıń. b) Eger ónimniń standart emesligi belgili bolsa, onda sol ónimniń qolda basqarılatuǵın qurılmada islep shıǵarılǵanlıǵı itimallıǵın tabıń.
 \\
\textbf{A3.} 28 dana dominonıń tolıq komplektinen 7 danası tosınnan tańlanadı. Olardıń ishinde keminde eki birdey ochko bolıw itimallıǵın tabıń.
 \\
\textbf{B1.} Puasson nizamına boysiniwshi tosinnanli $\xi$ shamasiniń dispersiyası tabılsın.
 \\
\textbf{B2.} Úsh mеrgеn bir-birinеn ǵárеzsiz nıshanaǵa bir márteden oq attı. Birinshi mеrgеnniń nıshanaǵa tiygiziw itimallıǵı 0,6 ǵa, еkinshisiniki 0,8 gе, úshinshisiniki bolsa 0,3 ke tеń. Atıw tamam bolǵannan kеyin nıshanada eki oq izi tabılǵan bolsa, birinshi mеrgеn nıshanaǵa tiygiziwi waqıyası itimallıǵı tabılsın.
 \\
\textbf{B3.} \includegraphics[width=0.15972in,height=0.23958in]{mediaBpng/image49.png} úzliksiz tosınnanlıq shamanıń tıǵızlıq funkciyaları berilgen. Olarǵa sáykes \includegraphics[width=0.15972in,height=0.19653in]{mediaBpng/image50.png} tosınnanlıq shamanıń \includegraphics[width=0.50278in,height=0.30069in]{mediaBpng/image51.png} tıǵızlıq funkciyasın tabıń. \includegraphics[width=2.15972in,height=0.58889in]{mediaBpng/image90.png} \includegraphics[width=0.71806in,height=0.49097in]{mediaBpng/image91.png}
 \\
\textbf{C1.} Eger ǵárezsiz \includegraphics[width=0.15208in,height=0.24028in]{mediaCpng/image4.png} hám \includegraphics[width=0.15208in,height=0.19167in]{mediaCpng/image5.png} úzliksiz tosınnanlıq shamalardıń hárbiri \includegraphics[width=0.19167in,height=0.16806in]{mediaCpng/image31.png} parametrli kórsetkishli nızam boyınsha bólistirilgen bolsa, onda \includegraphics[width=0.44028in,height=0.24028in]{mediaCpng/image14.png} tosınnanlıq shamanıń tıǵızlıq funkciyasın tabıń.
 \\
\textbf{C2.} Eger \(\xi_{1},\xi_{2}...,\xi_{n}\) ǵárezsiz birdey bólistirilgen tosınnanlıq shamalar standart normal bólistirilgen bolsa, onda \(\xi_{1}^{2} + \xi_{2}^{2} + ...\  + \xi_{n}^{2}\) tosınnanlıq shamanıń tıǵızlıq funkciyasın tabıń.
 \\
\textbf{C3.} Ortasha mánis vektorı \(\left( m_{1},m_{2} \right)\) hám kovariaciyalıq matricası\(K = \begin{pmatrix}
\sigma_{1}^{2} & r\sigma_{1}\sigma_{2} \\
r\sigma_{1}\sigma_{2} & \sigma_{2}^{2}
\end{pmatrix},\ \ \sigma_{1},\ \sigma_{2} > 0,\ \ |r|\  < 1\) bolǵan normal bólistirilgen \(\left( \xi_{1},\xi_{2} \right)\) tosınnanlıq vektordıń tıǵızlıq funkciyasın tabıń.
 \\

\end{tabular}
\vspace{1cm}


\begin{tabular}{m{17cm}}
\textbf{100-variant}
\newline

\textbf{T1.} Itimallıqlar teoriyası aksiomaları (ólshewli keńislik, itimallıq keńisligi).
 \\
\textbf{T2.} Kompoziciyalıq formulalar \\
\textbf{A1.} ${{x}^{2}}+2px+q=0$ kvadrat teńlemede $p$ hám $q$ koefficientler $\left[ -1,1 \right]$ kesindiden tosınnan tańlanadı. Kvadrat teńlemeniń oń túbirlerge iye bolıw itimallıǵın tabıń.
 \\
\textbf{A2.} Eki oyın kubigi taslanǵanda túsken ochkolardıń qosındısı 3 ten úlken, bıraq 8 den kishi bolıw itimallıǵın tabıń.
 \\
\textbf{A3.} Mashina jarısında 600 ekipaj qatnaspaqta. Hárbir ekipaj jarıstan texnikalıq nasazlıqlar sebepli 0,04 itimallıq penen, al aydawshınıń keselligi sebepli bolsa 0,01 itimallıq penen shıǵıp ketiwi múmkin. a) Aydawshınıń keselligi sebepli 4 ten artıq ekipaj jarıstan shıǵıp ketiwi itimallıǵın tabıń; b) 23 ten 27 ge shekem ekipaj texnikalıq nasazlıqlar sebepli jarıstan shıǵıp ketiwi itimallıǵın tabıń.
 \\
\textbf{B1.} $\xi$ tosınnanlı shamanıń \emph{f}(\emph{x}) tıǵızlıq funkciyasi berilgen bolsin. Tómendegilerdi esaplań: a) C; b) \emph{F}(\emph{x}); c) M$\xi$; d) D$\xi$; e) \emph{f}(\emph{x}) hám \emph{F}(\emph{x}) grafiklarin sızıń.\(f(x) = \left\{ \begin{matrix}
C(1 - 0.5|x|),\ \ \ \ x \in \lbrack - 2,2\rbrack, \\
\ \ \ \ \ \ \ \ 0,\ \ \ \ \ \ \ \ \ \ \ x \notin \lbrack - 2,2\rbrack.\ \ 
\end{matrix} \right.\ \)
 \\
\textbf{B2.} Chebıshev teńsizliginiń járdemi menen normal tosınnanlı shamanıń óziniń matematikalıq kútiliwinen, awısıwınıń úsh orta kvadratlıq awısıwdan úlken bolıwınıń itimallıǵın bahalań.
 \\
\textbf{B3.} Eger \includegraphics[width=0.36181in,height=0.29444in]{mediaBpng/image1.png} ǵárezsiz tosınnanlıq shamalar izbe-izliginiń bólistiriliw nızamları
\includegraphics[width=2.58264in,height=0.49097in]{mediaBpng/image16.png} \includegraphics[width=1.59514in,height=0.47847in]{mediaBpng/image17.png} \includegraphics[width=0.75486in,height=0.23958in]{mediaBpng/image18.png}
bolsa, onda bul izbe-izlik úlken sanlar nızamına boysınama?
 \\
\textbf{C1.} Tómende \includegraphics[width=0.46389in,height=0.25625in]{mediaCpng/image42.png} úzliksiz tosınnanlıq vektorlardıń tıǵızlıq funkciyaları berilgen. Olardıń \includegraphics[width=0.48819in,height=0.29583in]{mediaCpng/image43.png} hám \includegraphics[width=0.50417in,height=0.29583in]{mediaCpng/image44.png} marginal tıǵızlıq funkciyaların tabıń; \includegraphics[width=0.15972in,height=0.24028in]{mediaCpng/image45.png} hám \includegraphics[width=0.15972in,height=0.2in]{mediaCpng/image46.png} tosınnanlıq shamalardı ǵárezsizlikke tekseriń: \includegraphics[width=3.28819in,height=0.73611in]{mediaCpng/image69.png}
 \\
\textbf{C2.} Eger \(\left\{ \xi_{n} \right\}\) tosınnanlıq shamalar izbe-izligi \(\mathbf{\xi}_{\mathbf{n}}\overset{\mathbf{P}}{\rightarrow}\mathbf{\xi}\) hám \(\mathbf{\xi}_{\mathbf{n}}\overset{\mathbf{P}}{\rightarrow}\mathbf{\eta}\) bolsa, onda \(\mathbf{P}\left( \mathbf{\xi = \eta} \right)\mathbf{=}\mathbf{1}\) qatnasın dálilleń.
 \\
\textbf{C3.} Eger ǵárezsiz \includegraphics[width=0.15208in,height=0.24028in]{mediaCpng/image4.png} hám \includegraphics[width=0.15208in,height=0.19167in]{mediaCpng/image5.png} úzliksiz tosınnanlıq shamalardıń hárbiri \includegraphics[width=0.19167in,height=0.16806in]{mediaCpng/image32.png} parametrli kórsetkishli nızam boyınsha bólistirilgen bolsa, onda \includegraphics[width=0.44028in,height=0.24028in]{mediaCpng/image29.png} tosınnanlıq shamanıń tıǵızlıq funkciyasın tabıń.
 \\

\end{tabular}
\vspace{1cm}



\end{document}
