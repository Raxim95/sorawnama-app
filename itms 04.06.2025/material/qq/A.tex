Oyın kubigi taslanǵanda 5 ochkonıń túsiw itimallıǵın tabıń.
====
Oyın kubigi taslanǵanda taq ochkonıń túsiw itimallıǵın tabıń.
====
Oyın kubigi taslanǵanda jup ochkonıń túsiw itimallıǵın tabıń.
====
Oyın kubigi taslanǵanda pútin ochkonıń túsiw itimallıǵın tabıń.
====
Oyın kubigi taslanǵanda 7 ochkonıń túsiw itimallıǵın tabıń.
====
Eki oyın kubigi taslanǵanda túsken ochkolardıń qosındısı 7 ge teń bolıw itimallıǵın tabıń.
====
Eki oyın kubigi taslanǵanda túsken ochkolardıń qosındısı 3 ten úlken bolıw itimallıǵın tabıń.
====
Eki oyın kubigi taslanǵanda túsken ochkolardıń qosındısı 6 dan úlken bolıw itimallıǵın tabıń.
====
Eki oyın kubigi taslanǵanda túsken ochkolardıń qosındısı 4 ten úlken, bıraq 7 den kishi bolıw itimallıǵın tabıń.
====
Eki oyın kubigi taslanǵanda túsken ochkolardıń qosındısı 3 ten úlken, bıraq 8 den kishi bolıw itimallıǵın tabıń.
====
Eki oyın kubigi taslanǵanda túsken ochkolardıń qosındısı 2 den úlken, bıraq 5 ten kishi bolıw itimallıǵın tabıń.
====
Eki oyın kubigi taslanǵanda túsken ochkolardıń ayırması 2 ge teń bolıw itimallıǵın tabıń.
====
Eki oyın kubigi taslanǵanda túsken ochkolardıń ayırması 1 ge teń bolıw itimallıǵın tabıń.
====
Eki oyın kubigi taslanǵanda túsken ochkolardıń biri ekinshisinen 2 ese kóp bolıw itimallıǵın tabıń.
====
Eki oyın kubigi taslanǵanda túsken ochkolardıń biri ekinshisinen 4 ese kóp bolıw itimallıǵın tabıń.
====
Eki oyın kubigi taslanǵanda túsken ochkolardıń ayırması 1 den úlken bolıw itimallıǵın tabıń.
====
Eki oyın kubigi taslanǵanda túsken ochkolardıń ayırması 3 ten úlken bolıw itimallıǵın tabıń.
====
Eki oyın kubigi taslanǵanda túsken ochkolardıń kóbeymesi 10 nan asıp ketpew itimallıǵın tabıń.
====
Eki oyın kubigi taslanǵanda túsken eń úlken ochko 4 ten úlken bolıw itimallıǵın tabıń.
====
Eki oyın kubigi taslanǵanda túsken eń kishi ochko 4 ten úlken bolıw itimallıǵın tabıń.
====
Eki oyın kubigi taslanǵanda keminde bir oyın kubigindegi túsken ochko jup bolıw itimallıǵın tabıń.
====
Eki oyın kubigi taslanǵanda túsken ochkolardıń taq bolıw itimallıǵın tabıń.
====
Úsh oyın kubigi taslanǵanda túsken ochkolardıń qosındısı 11 ǵe teń bolıw itimallıǵın tabıń.
====
Úsh oyın kubigi taslanǵanda túsken ochkolardıń qosındısı 16 dan artıq bolmaw itimallıǵın tabıń.
====
Úsh oyın kubigi taslanǵanda túsken ochkolardıń birdey bolıw itimallıǵın tabıń.
++++
Toparda 12 student bar bolıp, olardan 8 student ayrıqsha bahaǵa oqıydı. Dizim boyınsha 9 student ajıratılǵan. Ajıratılǵanlar ishinde ayrıqsha bahaǵa oqıytuǵın 5 student bolıw itimallıǵın tabıń.
====
Firmada 7 erkek hám 3 hayal jumısshı isleydi. Tosınnan 3 jumısshı ajıratılıp alındı. Ajıratılıp alınǵan jumısshılardıń barlıǵı erkekler bolıw itimallıǵın tabıń.
====
100 dana buyımnan ibarat partiyada 4 dana buyım jaramsız. Partiyadan tosınnan 15 dana buyım alınadı. Usı alınǵan 15 dana buyımnıń ishinde 2 dana buyımnıń jaramsız bolıw itimallıǵın tabıń.
====
Qutıda 90 dana sapalı hám 10 dana sapasız detallar bar. Qutıdan tosınnan alınǵan 10 dana detaldıń ishinde sapasız detaldıń joq bolıw itimallıǵın tabıń.
====
Qutıda 30 dana birdey sharlar bolıp, olardıń 20 danası qızıl hám 10 danası kók reńdegi sharlar. Tosınnan alınǵan 3 dana shardıń 2 danası qızıl shar bolıw itimallıǵın tabıń.
====
Jámi $N$ dana lotereya biletleri ishinde $M$ dana lotereya bileti utıslı bolsa, satıp alınǵan $n~\,\left( n\le N \right)$ lotereya biletinen $m\,\,\left( m\le M \right)$ danası utıslı bolıw itimallıǵın tabıń.
====
“Sportlotto” oyınında qatnasıwshı kartadaǵı 49 sport túrinen 6 danasın belgileydi. Qatnasıwshınıń qura taslaw nátiyjesinde alınǵan 6 sport túrinen keminde 3 danasın durıs boljaǵan bolıw itimallıǵın tabıń.
====
36 dana kartalar kolodasınan tosınnan alınǵan 6 dana karta ishinde anıq 2 danası tuz bolıw itimallıǵın tabıń.
====
36 dana kartalar kolodasınan tosınnan alınǵan 3 dana karta ishinde anıq 2 dana valet bolıw itimallıǵın tabıń.
====
36 dana kartalar kolodasınan tosınnan alınǵan 6 dana karta ishinde anıq 5 dana karta birdey reńde hám 1 dana karta basqa reńde bolıw itimallıǵın tabıń.
====
36 dana kartalar kolodasınan tosınnan alınǵan 3 dana kartanıń barlıǵı birdey reńde bolıw itimallıǵın tabıń.
====
Qutıda 10 dana aq hám 15 dana qara sharlar bar. Tosınnan 5 dana shar alınǵanda, olar ishinde 2 dana aq shar bolıw itimallıǵın tabıń.
====
28 dana dominonıń tolıq komplektinen 7 danası tosınnan tańlanadı. Olardıń ishinde keminde 1 dana 6 ochko bolıw itimallıǵın tabıń.
====
28 dana dominonıń tolıq komplektinen 7 danası tosınnan tańlanadı. Olardıń ishinde keminde eki birdey ochko bolıw itimallıǵın tabıń.
====
28 dana dominonıń tolıq komplektinen 7 danası tosınnan tańlanadı. Hárbir domino tasındaǵı ulıwma ochkolar qosındısı 7 den kem bolıw itimallıǵın tabıń.
====
Qutıda 70 dana joqarı sapalı hám 10 dana tómen sapalı detallar bar. Qutıdan tosınnan alınǵan 6 dana detaldıń ishinde tómen sapalı detaldıń joq bolıw itimallıǵın tabıń.
====
20 komanda eki toparǵa bólinedi. Eki eń kúshli komanda bir toparǵa túspew itimallıǵın tabıń.
====
Jámi 10 bala hám 12 qız bolǵan studentler toparınan 5 student sorawnama ótkeriw ushın tosınnan tańlap alındı. Olar ishinde keminde bir student qız bolıw itimallıǵın tabıń.
====
Jámi 14 bala hám 11 qız bolǵan studentler toparınan 6 student sorawnama ótkeriw ushın tosınnan tańlap alındı. Olar ishinde 4 bala bolıw itimallıǵın tabıń.
====
Jámi 10 bala hám 12 qız bolǵan studentler toparınan 7 student sorawnama ótkeriw ushın tosınnan tańlap alındı. Olar ishinde 3 bala hám 4 qız bolıw itimallıǵın tabıń.
====
Toparda 25 student bolıp, olardan 6 student ayrıqsha bahaǵa oqıydı. Dizim boyınsha 10 student ajıratılǵan. Ajıratılǵanlar ishinde ayrıqsha bahaǵa oqıytuǵın 3 student bolıw itimallıǵın tabıń.
====
Firmada 11 erkek hám 6 hayal jumısshı isleydi. Tosınnan 5 jumısshı ajıratılıp alındı. Ajıratılıp alınǵan jumısshılardıń barlıǵı hayal bolıw itimallıǵın tabıń.
====
150 dana buyımnan ibarat partiyada 5 dana buyım jaramsız. Partiyadan tosınnan 12 dana buyım alınadı. Usı alınǵan 12 dana buyımnıń ishinde 3 dana buyımnıń jaramsız bolıw itimallıǵın tabıń. 
====
Qutıda 80 dana joqarı sapalı hám 20 dana tómen sapalı detallar bar. Qutıdan tosınnan alınǵan 14 dana detaldıń ishinde tómen sapalı detaldıń joq bolıw itimallıǵın tabıń.
====
Qutıda 28 dana birdey sharlar bolıp, olardıń 19 danası qızıl hám 9 danası kók reńdegi sharlar. Tosınnan alınǵan 3 dana shardıń 2 danası kók shar bolıw itimallıǵın tabıń.
++++
====
$\left[ 0,1 \right]$ kesindiden tosınnan eki noqat tańlanadı. Birinshi noqattıń koordinatası ekinshi noqattıń koordinatasınan kishi bolıw itimallıǵın tabıń.
====
$\left[ 0,1 \right]$ kesindiden tosınnan eki noqat tańlanadı. Ekinshi noqattıń koordinatası birinshi noqattıń koordinatası eki esesinen úlken bolıw itimallıǵın tabıń.
====
$\left[ 0,1 \right]$ kesindiden tosınnan eki noqat tańlanadı. Ekinshi noqattıń koordinatası birinshi noqattıń koordinatası úsh esesinen úlken bolıw itimallıǵın tabıń.
====
$\left[ 0,1 \right]$ kesindiden tosınnan eki noqat tańlanadı. Birinshi hám ekinshi noqatlar koordinataları arasındaǵı aralıq $0,5$ ten úlken bolıw itimallıǵın tabıń.
====
$\left[ 0,1 \right]$ kesindiden tosınnan eki noqat tańlanadı. Birinshi hám ekinshi noqatlar koordinataları arasındaǵı aralıq $0,7$ den kishi bolıw itimallıǵın tabıń.
====
$\left[ 0,1 \right]$ kesindiden tosınnan eki noqat tańlanadı. Noqatlardıń koordinataları qosındısı 1,5 ten kishi bolıw itimallıǵın tabıń.
====
$\left[ 0,1 \right]$ kesindiden tosınnan eki noqat tańlanadı. Birinshi noqattıń koordinatasınıń ekinshi noqattıń koordinatasına qatnası 0,5 ten úlken bolıw itimallıǵın tabıń.
====
$\left[ 0,1 \right]$ kesindiden tosınnan eki noqat tańlanadı. Ekinshi noqattıń koordinatasınıń birinshi noqattıń koordinatasına qatnası 0,6 dan kishi bolıw itimallıǵın tabıń.
====
$\left[ 0,1 \right]$ kesindiden tosınnan eki noqat  tańlanadı. Olardıń koordinataları kvadratları qosındısı 1 den úlken bolıw itimallıǵın tabıń.
====
$\left[ 0,1 \right]$ kesindiden tosınnan eki noqat tańlanadı. Birinshi hám ekinshi noqatlardıń koordinataları kvadratlarınıń ayırması $0,25$ ten úlken bolıw itimallıǵın tabıń.
====
$\left[ 0,1 \right]$ kesindiden tosınnan eki noqat tańlanadı. Olardıń koordinataları qosındısı, koordinataları kóbeymesi eki esesinen kóp bolıw itimallıǵın tabıń.
====
$\left[ 0,1 \right]$ kesindiden tosınnan eki noqat tańlanadı. Olardıń koordinataları kvadratları qosındısı, koordinataları kóbeymesi úsh esesinen kóp bolıw itimallıǵın tabıń.
====
$\left[ 0,1 \right]$ kesindiden tosınnan eki noqat tańlanadı. Olardıń koordinataları ayırması modulı $1/6$ den kishi bolıw itimallıǵın tabıń.
====
$\left[ 0,1 \right]$ kesindiden tosınnan eki noqat tańlanadı. Olardıń koordinataları qosındısı 1 den úlken bolmaw hám kóbeymesi 0,09 dan kishi bolmaw itimallıǵın tabıń.
====
$\left[ 0,2 \right]$ kesindiden tosınnan eki noqat tańlanadı. Olardıń koordinataları qosındısı 2 den úlken bolıw hám kvadratları qosındısı 4 ten kishi bolıw itimallıǵın tabıń.
====
$\left[ -1,2 \right]$ kesindiden tosınnan eki noqat tańlanadı. Olardıń koordinataları qosındısı 1 den úlken bolıw hám kóbeymesi 1 den kishi bolıw itimallıǵın tabıń.
====
$\left[ 0,2 \right]$ kesindiden tosınnan eki noqat tańlanadı. Olardıń koordinataları kóbeymesi 2 den úlken bolıw itimallıǵın tabıń.
====
$\left[ 0,2 \right]$ kesindiden tosınnan $x$ hám $y$ noqat tańlanadı. Olar ushın $\left| \begin{matrix}
   1 & x  \\
   x & y  \\
\end{matrix} \right|>0$ bolıw itimallıǵın tabıń.
====
$x\in \left[ -\pi ,\pi  \right]$ ushın $sinx<cosx$ bolıw itimallıǵın tabıń.
====
$x\in \left[ 0,2\pi  \right]$ ushın $2co{{s}^{2}}x-5cosx+1>0$ bolıw itimallıǵın tabıń.
====
${{x}^{2}}+2px+q=0$ kvadrat teńlemede $p$ hám $q$ koefficientler $\left[ -1,1 \right]$ kesindiden tosınnan tańlanadı. Kvadrat teńlemeniń haqıyqıy túbirlerge iye bolıw itimallıǵın tabıń.
====
${{x}^{2}}+2px+q=0$ kvadrat teńlemede $p$ hám $q$ koefficientler $\left[ -1,1 \right]$ kesindiden tosınnan tańlanadı. Kvadrat teńlemeniń oń túbirlerge iye bolıw itimallıǵın tabıń.
====
$\left[ 0,2 \right]$ kesindiden tosınnan $x$ hám $y$ noqat tańlanadı. Olar ushın ${{x}^{2}}\le 4y\le 4x$ bolıw itimallıǵın tabıń.
====
$\left( 0,2 \right)$ intervaldan tosınnan $x$ hám $y$ noqat tańlanadı. Olar ushın $xy\le 1$ hám $\frac{y}{x}\le 2$ bolıw itimallıǵın tabıń.
====
$\left[ 0,3 \right]$ kesindiden tosınnan úsh noqat tańlanadı. Olardıń koordinataları qosındısı 3 ten kishi bolıw itimallıǵın tabıń.
++++
Albomda 10 dana jańa hám 12 dana múddeti ótken markalar bar. Albomnan tosınnan 3 marka alınıp, múddeti ótkerildi hám ornına qaytarılıp qoyıldı. Bunnan soń, tosınnan 2 marka alındı. a) Bul 2 marka jańa bolıw itimallıǵın tabıń. b) Sol 2 marka jańa ekenligi belgili bolsa, dáslepki alınǵan 3 markanıń jańa bolıw itimallıǵın tabıń.
====
Albomda 6 dana jańa hám 10 dana múddeti ótken markalar bar. Albomnan tosınnan 3 marka alıp taslandı. Bunnan soń, tosınnan 2 marka alındı. a) Bul 2 markanıń jańa bolıw itimallıǵın tabıń. b) Sol 2 marka jańa ekenligi belgili bolsa, dáslepki alınǵan 3 markanıń múddeti ótken bolıw itimallıǵın tabıń.
====
Albomda 8 dana jańa hám 6 dana múddeti ótken markalar bar. Albomnan tosınnan 3 marka alınıp, múddeti jańalandı hám ornına qaytarıp qoyıldı. Bunnan soń, tosınnan 2 marka alındı. a) Bul 2 marka jańa bolıw itimallıǵın tabıń. b) Sol 2 marka jańa ekenligi belgili bolsa, dáslepki alınǵan 3 markanıń múddeti ótken bolıw itimallıǵın tabıń.
====
Birinshi qutıda 3 dana aq hám 5 dana qara sharlar bar, ekinshi qutıda 6 dana aq hám 8 dana qara sharlar bar. Birinshi qutıdan tosınnan 2 shar alınıp, ekinshi qutıǵa salındı. Keyin, birinshi qutıdan tosınnan 1 shar alındı. а) Bul shardıń aq bolıw itimallıǵın tabıń. b) Sol shar aq bolsa, birinshi qutıdan alınǵan sharlardıń aq bolıw itimallıǵın tabıń.
====
Úsh qutınıń hárbirinde $n$ dana aq hám $m$ dana qara sharlar bar. Birinshi hám ekinshi qutıdan tosınnan 1 shardan alınıp, úshinshi qutıǵa salındı. Keyin, úshinshi qutıdan tosınnan bir shar alındı. а) Bul shardıń aq bolıw itimallıǵın tabıń. b) Sol shar aq bolsa, dáslepki eki qutıdan alınǵan sharlardıń aq bolıw itimallıǵın tabıń.
====
Úsh qutınıń hárbirinde $n$ dana aq ($n\ge 2$) hám $m$ dana qara sharlar bar. Birinshi qutıdan ekinshi qutıǵa tosınnan eki shar, ekinshi qutıdan úshinshi qutıǵa tosınnan bir shar salındı. Keyin, úshinshi qutıdan tosınnan bir shar alındı. а) Bul shardıń aq bolıw itimallıǵın tabıń. b) Sol shar aq bolsa, birinshi qutıdan alınǵan sharlardıń aq bolıw itimallıǵın tabıń.
====
Berilgen $1,2,\ldots ,10$ sanlarınıń arasınan tosınnan bir san tańlandı. Meyli, bul san $m$ bolsın. Keyin, $1,2,\ldots ,m$ sanlarınıń arasınan tosınnan bir san tańlandı. a) Bul sannıń 8 ge teń bolıw itimallıǵın tabıń. b) Bul san 8 ge teń bolsa, $m=9$ bolıw itimallıǵın tabıń.
====
Berilgen $1,2,\ldots ,10$ sanlarınıń arasınan tosınnan bir san tańlandı. Meyli, bul san $m$ bolsın. Soń, $\left[ 0,m \right]$ kesindiden tosınnan $\xi $ noqat tańlandı. a) $\xi >8$ bolıw itimallıǵın tabıń. b) Eger $\xi >8$ bolsa, onda $m=9$ bolıw itimallıǵın tabıń.
====
Berilgen $1,2,\ldots ,10$ sanlarınıń arasınan tosınnan eki san tańlandı. Meyli, bul sanlar ${{m}_{1}}$ hám ${{m}_{2}}$ (${{m}_{1}}<{{m}_{2}}$) bolsın. Soń, ${{m}_{1}},{{m}_{1}}+1,\ldots ,{{m}_{2}}$ sanları arasınan tosınnan bir san tańlandı. a) Bul sannıń 9 ǵa teń bolıw itimallıǵın tabıń. b) Bul san 9 ǵa teń bolsa, ${{m}_{2}}=10$ bolıw itimallıǵın tabıń.
====
Oyın kubigi taslandı. Meyli, $m$ - túsken ochkolar sanı bolsın. Soń, nıshanǵa qarata hárbir atıwda $p$ tiyiw itimallıǵı menen $2m$ márte oq atıladı. a) Nıshanǵa eki márte oq tiyiw itimallıǵın tabıń. b) Nıshanǵa eki márte oq tiygen bolsa, $m=3$ bolıw itimallıǵın tabıń.
====
Birdey úsh qutı berilgen. Birinshi qutıda 2 dana aq hám 1 dana qara sharlar, ekinshi qutıda 3 dana aq hám 1 dana qara sharlar, úshinshi qutıda bolsa 2 dana aq hám 2 dana qara sharlar bar. Tosınnan qutılardan birewi tańlanıp, onıń ishinen bir dana shar alınadı. a) Usı alınǵan shardıń aq shar bolıw itimallıǵın tabıń. b) Alınǵan shar aq bolsa, onıń ekinshi qutıdan alınǵan bolıw itimallıǵın tabıń.
====
Oqıtıwshı matematika páninen shegaralıq bahalaw alıw ushın 50 soraw tayarlaǵan. Olardıń ishinde differencial esabınan 20 soraw, integral esabınan 18 soraw hám qatarlar teoriyasınan 12 soraw bar. Student differencial esabınan 18 sorawǵa, integral esabınan 15 sorawǵa hám qatarlar teoriyasınan 10 sorawǵa juwap bere aladı. a) Studentke berilgen birinshi sorawǵa juwap beriwi itimallıǵın tabıń. b) Eger student sol sorawǵa durıs juwap bergen bolsa, bul sorawdıń integral esabınan bolıwı itimallıǵın tabıń.
====
Birdey úsh qutı berilgen. Birinshi qutıda 3 dana aq hám 2 dana qara sharlar, ekinshi qutıda 2 dana aq hám 4 dana qara sharlar, úshinshi qutıda bolsa 4 dana aq hám 3 dana qara sharlar bar. Tosınnan qutılardan birewi tańlanıp, onıń ishinen bir dana shar alınadı. a) Usı alınǵan shardıń qara shar bolıw itimallıǵın tabıń. b) Alınǵan shar qara shar bolsa, onıń úshinshi qutıdan alınǵan bolıw itimallıǵın tabıń.
====
Samolyotqa birimlep úsh márte oq atıladı. Birinshi márte atqanda tiyiw itimallıǵı 0,4 ke, ekinshisinde 0,5 ke, úshinshisinde bolsa 0,7 ge teń. Samolyottı isten shıǵarıw ushın úsh márte oq tiyiwi jetkilikli, bir márte tiygende 0,2 ge teń itimallıq penen qatardan shıǵadı, eki márte tiygende 0,6 ǵa teń itimallıq penen isten shıǵadı. 
a) Úsh márte oq atıw nátiyjesinde samolyottıń tolıǵı menen isten shıǵıw itimallıǵın tabıń. b) Eger samolyot tolıǵı menen isten shıqqan bolsa, ol eki oq tiyip isten shıqqan bolıw itimallıǵın tabıń.
====
Samolyotqa samolyottan 4 dana ǵárezsiz oq atıldı. Hárbir atılǵan oqtıń tiyiw itimallıǵı 0,3 ke teń. Samolyottı joq etiw ushın (tolıǵı menen isten shıǵarıw ushın) 2 márte tiyiw jetkilikli, 1 márte tiygende 0,6 ǵa teń itimallıq penen isten shıǵadı. a) Tórt márte oq atıw nátiyjesinde samolyottıń tolıǵı menen isten shıǵıw itimallıǵın tabıń. b) Eger samolyot tolıǵı menen isten shıqqan bolsa, ol bir oq tiyip isten shıqqan bolıw itimallıǵın tabıń.
====
Bazaǵa 360 dana buyım keltirilgen. Bulardan: 300 danası 1-kárxanada tayarlanǵan bolıp, 250 danası jaramlı; 40 danası 2-kárxanada tayarlanǵan bolıp, 30 danası jaramlı; 20 danası 3-kárxanada tayarlanǵan bolıp, 10 danası jaramlı. a) Bazadan tosınnan alınǵan buyımnıń jaramlı bolıw itimallıǵın tabıń. b) Eger bazadan alınǵan buyım jaramlı bolsa, onda sol buyımnıń 2-kárxanaǵa tiyisli bolıw itimallıǵın tabıń.
====
Birdey úsh qutı berilgen. Birinshi qutıda $a$ dana aq hám $b$ dana qara sharlar, ekinshi qutıda $c$ dana aq hám $d$ dana qara sharlar; úshinshi qutıda bolsa, tek aq sharlar bar. Qutılardan birewi tosınnan tańlanıp, onnan bir shar alındı. a) Usı alınǵan shardıń aq bolıw itimallıǵın tabıń. b) Alınǵan shar aq bolsa, sol shardıń birinshi qutıǵa tiyisli bolıw itimallıǵın tabıń.
====
Mikrosxemalardıń 10% i nuqsanlı jaǵdayda bolıp, olar tekseriwden ótkerildi. Ápiwayılastırılǵan tekseriw sınaǵı ótkerildi. Bul tekseriw tómendegishe itimallıqta qátelikke jol qoyadı, yaǵnıy 0,95 itimallıq penen nuqsanlı mikrosxemanı nuqsanlı dep tabadı hám 0,03 itimallıq penen nuqsansız mikrosxemanı nuqsanlı dep tabadı. a) Tekseriwden ótkerilgen mikrosxemanıń nuqsanlı dep tabılıw itimallıǵın tabıń. b) Bul mikrosxemanıń negizinde nuqsansız bolıwı itimallıǵı qanday?
====
Detallar partiyası úsh jumısshı tárepinen tayarlanadı. Birinshi jumısshı barlıq detaldıń 25% in, ekinshi jumısshı 35% in, úshinshisi bolsa 40% in tayarlaydı. Bul úsh jumısshınıń tayarlaǵan detallarınıń sapasız bolıwı waqıyası itimallıqları sáykes túrde 0,05; 0,04 hám 0,02 ge teń. a) Tekseriw ushın partiyadan alınǵan detaldıń sapasız bolıw itimallıǵın tabıń. b) Sol sapasız detaldıń ekinshi jumısshı tárepinen tayarlanǵan bolıw itimallıǵın tabıń.
====
Eki mergen biri-birinen ǵárezsiz bir nıshanǵa bir márteden oq atadı. Birinshi mergenniń nıshanǵa tiygiziw itimallıǵı 0,8 ekinshisiniki bolsa 0,4 ke teń. a) Atıw tamam bolǵannan soń, nıshanda bir oq izi bolıwı itimallıǵın tabıń. b) Sol oq iziniń ekinshi mergenge tiyisli bolıw itimallıǵın tabıń.
====
Zavodta avtomat basqarılatuǵın 14 dana hám qolda basqarılatuǵın 6 dana qurılmalar bar. Avtomat basqarılatuǵın qurılmalar ushın standart bolmaǵan ónimlerdi islep shıǵarıw itimallıǵı $0,001$ ge, al qolda basqarılatuǵın qurılmalar ushın bolsa $0,002$ ge teń. a) Laboratoriya analizine tosınnan alınǵan ónimniń standart bolmaǵan bolıwı itimallıǵın tabıń. b) Eger ónimniń standart emesligi belgili bolsa, onda sol ónimniń qolda basqarılatuǵın qurılmada islep shıǵarılǵanlıǵı itimallıǵın tabıń.
====
Bazıbir tarawda ónimniń 3% i I fabrika menen, 25% i II fabrika menen, al ónimniń qalǵan bólegi bolsa III fabrika menen óndiriledi. I fabrikada 1% ónim jaramsız, II fabrikada 1,5% ónim jaramsız, III fabrikada 2% ónim jaramsız bolıp shıǵadı. a) Qarıydar satıp alǵan bir buyım jaramsız bolıw itimallıǵın tabıń. b) Sol buyımdı I fabrika islep shıǵarǵan bolıwı itimallıǵın tabıń.
====
Shegaralıq bahalaw jumısın tapsırıwǵa kelgen 10 studentten ibarat toparda úshewi ayrıqsha, tórtewi jaqsı, ekewi qanaatlandırarlı hám birewi qanaatlandırarsız tayarlanǵan. Shegaralıq bahalaw jumısınıń variantlarında 20 dana soraw bar. Ayrıqsha tayarlanǵan student barlıq 20 sorawǵa, jaqsı tayarlanǵanı 16 sorawǵa, qanaatlandırarlı tayarlanǵanı 10 sorawǵa, qanaatlandırarsız tayarlanǵanı 5 sorawǵa juwap bere aladı. a) Bul studentlerden qálegen birewi berilgen bir sorawǵa durıs juwap beriw itimallıǵın tabıń. b) Sol durıs juwap bergen studenttiń ayrıqsha tayarlanǵan student bolıwı itimallıǵın tabıń.
====
Shegaralıq bahalaw jumısın tapsırıwǵa kelgen 10 studentten ibarat toparda úshewi ayrıqsha, tórtewi jaqsı, ekewi qanaatlandırarlı hám birewi qanaatlandırarsız tayarlanǵan. Shegaralıq bahalaw jumısınıń variantlarında 20 dana soraw bar. Ayrıqsha tayarlanǵan student barlıq 20 sorawǵa, jaqsı tayarlanǵanı 16 sorawǵa, qanaatlandırarlı tayarlanǵanı 10 sorawǵa, qanaatlandırarsız tayarlanǵanı 5 sorawǵa juwap bere aladı. a) Bul studentlerden qálegen birewi berilgen úsh sorawǵa da durıs juwap beriw itimallıǵın tabıń. b) Sol durıs juwap bergen studenttiń qanaatlandırarsız tayarlanǵan student bolıw itimallıǵın tabıń.
====
Dushpan nıshanın joq etiw ushın hár túrlı eki samolyot ushıp ketti. Birinshi túrdegi samolyot nıshandı $0,9$ itimallıq penen, ekinshi túrdegi samolyot $0,8$ itimallıq penen joq etiw múmkin. Biraq, dushpannıń hawa hújiminen qorǵanıwı birinshi túrdegi samolyottı $0,95$ itimallıq penen, ekinshi túrdegi samolyottı $0,85$ itimallıq penen urıp túsiriwi múmkin. a) Samolyotlar nıshandı joq etiw itimallıǵın tabıń. b) Nıshan joq etilgen bolsa, onı tek ekinshi samolyot joq etken bolıw itimallıǵın tabıń.
++++
Hárbiriniń júzege asıw itimallıǵı $p$ ǵa teń bolǵan 10 dana Bernulli tájiriybesi ótkerilgende, tómendegi waqıyalardıń itimallıqların tabıń: Sátliler sanı 7 dana.
====
Hárbiriniń júzege asıw itimallıǵı $p$ ǵa teń bolǵan 10 dana Bernulli tájiriybesi ótkerilgende, tómendegi waqıyalardıń itimallıqların tabıń: Sátsizler sanı 4 dana.
====
Hárbiriniń júzege asıw itimallıǵı $p$ ǵa teń bolǵan 10 dana Bernulli tájiriybesi ótkerilgende, tómendegi waqıyalardıń itimallıqların tabıń: Sátsizler sanı joq bolıw.
====
Hárbiriniń júzege asıw itimallıǵı $p$ ǵa teń bolǵan 10 dana Bernulli tájiriybesi ótkerilgende, tómendegi waqıyalardıń itimallıqların tabıń: Sátliler sanı keminde 3 dana.
====
Hárbiriniń júzege asıw itimallıǵı $p$ ǵa teń bolǵan 10 dana Bernulli tájiriybesi ótkerilgende, tómendegi waqıyalardıń itimallıqların tabıń: Sátsizler sanı kóbi menen 2 dana.
====
Hárbiriniń júzege asıw itimallıǵı $p$ ǵa teń bolǵan 10 dana Bernulli tájiriybesi ótkerilgende, tómendegi waqıyalardıń itimallıqların tabıń: Sátliler sanı 5 den artıq, biraq 8 den kem.
====
Hárbiriniń júzege asıw itimallıǵı $p$ ǵa teń bolǵan 10 dana Bernulli tájiriybesi ótkerilgende, tómendegi waqıyalardıń itimallıqların tabıń: Sátsizler sanı 2 den artıq, biraq 5 den kem.
====
Hárbiriniń júzege asıw itimallıǵı $p$ ǵa teń bolǵan 10 dana Bernulli tájiriybesi ótkerilgende, tómendegi waqıyalardıń itimallıqların tabıń: Tájiriybelerdiń birinshi yarımındaǵı sátliler sanı, tájiriybelerdiń ekinshi yarımındaǵı sátliler sanınan ekige artıq.
====
Hárbiriniń júzege asıw itimallıǵı $p$ ǵa teń bolǵan 10 dana Bernulli tájiriybesi ótkerilgende, tómendegi waqıyalardıń itimallıqların tabıń: Tájiriybelerdiń birinshi yarımındaǵı sátliler sanı, tájiriybelerdiń ekinshi yarımındaǵı sátliler sanınan eki esege artıq.
====
Hárbiriniń júzege asıw itimallıǵı $p$ ǵa teń bolǵan 10 dana Bernulli tájiriybesi ótkerilgende, tómendegi waqıyalardıń itimallıqların tabıń: Tájiriybelerdiń birinshi yarımındaǵı sátliler sanı, tájiriybelerdiń ekinshi yarımındaǵı sátliler sanınan kem.
====
Hárbiriniń júzege asıw itimallıǵı $p$ ǵa teń bolǵan 10 dana Bernulli tájiriybesi ótkerilgende, tómendegi waqıyalardıń itimallıqların tabıń: Tájiriybelerdiń birinshi yarımındaǵı sátliler sanı, tájiriybelerdiń ekinshi yarımındaǵı sátliler sanınan artıq.
====
Hárbiriniń júzege asıw itimallıǵı $p$ ǵa teń bolǵan 10 dana Bernulli tájiriybesi ótkerilgende, tómendegi waqıyalardıń itimallıqların tabıń: Sátliler sanı 3 dana, sonıń menen birge, sońǵı tájiriybe sátli juwmaqlanıwı.
====
Hárbiriniń júzege asıw itimallıǵı $p$ ǵa teń bolǵan 10 dana Bernulli tájiriybesi ótkerilgende, tómendegi waqıyalardıń itimallıqların tabıń: Sátliler sanı 6 dana, sonıń menen birge, sońǵı tájiriybe sátsiz juwmaqlanıwı.
====
Hárbiriniń júzege asıw itimallıǵı $p$ ǵa teń bolǵan 10 dana Bernulli tájiriybesi ótkerilgende, tómendegi waqıyalardıń itimallıqların tabıń: Sátliler menen sátsizler izbe-iz keliwi.
====
Hárbiriniń júzege asıw itimallıǵı $p$ ǵa teń bolǵan 10 dana Bernulli tájiriybesi ótkerilgende, tómendegi waqıyalardıń itimallıqların tabıń: Sátliler sanı 3 dana, sonıń menen birge, olardıń barlıǵı sońǵı úsh tájiriybede ámelge asıwı.
====
Hárbiriniń júzege asıw itimallıǵı $p$ ǵa teń bolǵan 10 dana Bernulli tájiriybesi ótkerilgende, tómendegi waqıyalardıń itimallıqların tabıń: Sátliler sanı 6 dana, sonıń menen birge, olardıń barlıǵı dáslepki altı tájiriybede ámelge asıwı.
====
Hárbiriniń júzege asıw itimallıǵı $p$ ǵa teń bolǵan 10 dana Bernulli tájiriybesi ótkerilgende, tómendegi waqıyalardıń itimallıqların tabıń: Sátliler sanı 3 dana, sonıń menen birge, olardıń barlıǵı tájiriybelerdiń ekinshi yarımında ámelge asıwı.
====
Hárbiriniń júzege asıw itimallıǵı $p$ ǵa teń bolǵan 10 dana Bernulli tájiriybesi ótkerilgende, tómendegi waqıyalardıń itimallıqların tabıń: Sátliler sanı 2 dana, sonıń menen birge, olardıń barlıǵı tájiriybelerdiń birinshi yarımında ámelge asıwı.
====
Hárbiriniń júzege asıw itimallıǵı $p$ ǵa teń bolǵan 10 dana Bernulli tájiriybesi ótkerilgende, tómendegi waqıyalardıń itimallıqların tabıń: Sátliler sanı, sátsizler sanınan artıq.
====
Hárbiriniń júzege asıw itimallıǵı $p$ ǵa teń bolǵan 10 dana Bernulli tájiriybesi ótkerilgende, tómendegi waqıyalardıń itimallıqların tabıń: Sátliler sanı, sátsizler sanınan ekige artıq.
====
Hárbiriniń júzege asıw itimallıǵı $p$ ǵa teń bolǵan 10 dana Bernulli tájiriybesi ótkerilgende, tómendegi waqıyalardıń itimallıqların tabıń: Sátliler sanı, sátsizler sanınan kem.
====
Hárbiriniń júzege asıw itimallıǵı $p$ ǵa teń bolǵan 10 dana Bernulli tájiriybesi ótkerilgende, tómendegi waqıyalardıń itimallıqların tabıń: Sátliler sanı, sátsizler sanınan tórtke kem.
====
Hárbiriniń júzege asıw itimallıǵı $p$ ǵa teń bolǵan 10 dana Bernulli tájiriybesi ótkerilgende, tómendegi waqıyalardıń itimallıqların tabıń: Sátliler sanı, sátsizler sanınan keminde 2 ese artıq.
====
Hárbiriniń júzege asıw itimallıǵı $p$ ǵa teń bolǵan 10 dana Bernulli tájiriybesi ótkerilgende, tómendegi waqıyalardıń itimallıqların tabıń: Sátsizler sanı tek 2 dana hám olar arasında 4 dana sátli bolıw.
====
Hárbiriniń júzege asıw itimallıǵı $p$ ǵa teń bolǵan 10 dana Bernulli tájiriybesi ótkerilgende, tómendegi waqıyalardıń itimallıqların tabıń: Sátliler sanı tek 2 dana hám olar arasında 3 dana sátsiz bolıw.
++++
“Sportlotto” lotereyasında úlken hám kishi utıslar oynaladı. Lotereya biletinde úlken utıs shıǵıw itimallıǵı 0,009 ǵa, al kishisi bolsa 0,02 ge teń. Jámi 500 dana bilet satıp alınǵanda: a) úlken utıslar 5 ten 10 ǵa shekem bolıw; b) 15 ten 20 ǵa shekem kishi utıslar bolıwı waqıyaları itimallıqların tabıń.
====
Lotereyada úlken hám kishi utıslar oynaladı. Lotereya biletinde úlken utıs shıǵıw itimallıǵı 0,001 ge, al kishisi bolsa 0,01 ge teń. Jámi 1000 dana bilet satıp alınǵanda: a) eki úlken utıslı; b) kishi utıslar 5 ten 15 ke shekem bolıwı waqıyaları itimallıqların tabıń.
====
Studentler úsh jıl dawamında matematikalıq kitaplardan hárbirinde 30 máseleni óz ishine alǵan matematikadan 25 tipikalıq esaplardı orınlaydı. Kompyuterde matematikalıq paket járdeminde máseleni nadurıs sheshiwi itimallıǵı 0,01 ge, paket járdemisiz 0,2 ge teń. Úsh jıl ishinde tómendegi waqıyalardıń júzege asıwı itimallıqların tabıń: a) matematikalıq paketten turaqlı túrde paydalanatuǵın student 4 ten kóp bolmaǵan máselelerdi nadurıs sheshken bolsa; b) matematikalıq paketten paydalanbaytuǵın student 120 dan 180 ge shekem máseleni nadurıs sheshedi.
====
Studentler eki jıl dawamında matematikalıq kitaplardan hárbirinde 20 máseleni óz ishine alǵan matematikadan 15 tipikalıq esaplardı orınlaydı. Kompyuterde matematikalıq paket járdeminde máseleni nadurıs sheshiwi itimallıǵı 0,01 ge, paket járdemisiz 0,2 ge teń. Úsh jıl ishinde tómendegi waqıyalardıń júzege asıwı itimallıqların tabıń: a) matematikalıq paketten turaqlı túrde paydalanatuǵın student 5 ten kóp bolmaǵan máselelerdi nadurıs sheshken bolsa; b) matematikalıq paketten paydalanbaytuǵın student 50 dan 70 ge shekem máseleni nadurıs sheshedi.
====
Bazıbir qalada solaqaylar ortasha esapta $1,5$, sol hám oń qollarına teńdey iyelik qılatuǵın adamlar $9$, al qalǵanları ońaqaylar. Jámi 300 adam arasında tómendegi waqıyalardıń júzege asıwı itimallıqların tabıń: a) keminde tórt solaqay boladı; b) 15 ten 20 ǵa shekem sol hám oń qollarına teńdey iyelik qılatuǵın adamlar boladı.
====
Bazıbir qalada solaqaylar ortasha esapta $1$, sol hám oń qollarına teńdey iyelik qılatuǵın adamlar $10$, al qalǵanları ońaqaylar. Jámi 200 adam arasında tómendegi waqıyalardıń júzege asıwı itimallıqların tabıń: a) keminde tórt solaqay boladı; b) 18 den 23 ke shekem sol hám oń qollarına teńdey iyelik qılatuǵın adamlar boladı.
====
Balalar baqshasında 300 bala tárbiyalanadı. Tómendegi waqıyalardıń júzege asıwı itimallıqların tabıń: a) anıq eki bala 1-martta tuwılǵan; b) jazda 47 den 52 ge shekem bala tuwılǵan.
====
Jıldıń qálegen kúninde bala tuwılıw itimallıǵı teń dep esaplap, 200 bala arasında tómendegi waqıyalardıń júzege asıwı itimallıqların tabıń: a) anıq úsh bala 1-yanvarda tuwılǵan; b) báhárde 48 den 53 ke shekem bala tuwılǵan.
====
Mashina jarısında 600 ekipaj qatnaspaqta. Hárbir ekipaj jarıstan texnikalıq nasazlıqlar sebepli 0,04 itimallıq penen, al aydawshınıń keselligi sebepli bolsa 0,01 itimallıq penen shıǵıp ketiwi múmkin. a) Aydawshınıń keselligi sebepli 4 ten artıq ekipaj jarıstan shıǵıp ketiwi itimallıǵın tabıń; b) 23 ten 27 ge shekem ekipaj texnikalıq nasazlıqlar sebepli jarıstan shıǵıp ketiwi itimallıǵın tabıń.
====
Mashina jarısında 500 ekipaj qatnaspaqta. Hárbir ekipaj jarıstan texnikalıq nasazlıqlar sebepli 0,05 itimallıq penen, aydawshınıń keselligi sebepli bolsa 0,01 itimallıq penen shıǵıp ketiw múmkin. a) Aydawshınıń keselligi sebepli 5 ten artıq ekipaj jarıstan shıǵıp ketiwi itimallıǵın tabıń; b) 22 den 28 ge shekem ekipaj texnikalıq nasazlıqlar sebepli jarıstan shıǵıp ketiwi itimallıǵın tabıń.
====
Dúkan 2000 dana televizor hám 2000 dana radio satıp aldı. Hárbir televizordıń defektli bolıw itimallıǵı 0,004 ke hám hárbir radionıń defektli bolıw itimallıǵı 0,03 ke teń. Usı sawdada tómendegi waqıyalardıń júzege asıwı itimallıqların tabıń: a) keminde úsh televizor defektli bolıwı; b) 33 ten 44 ke shekem radio defektli bolıwı.
====
Dúkan 1000 dana televizor hám 1000 dana radio satıp aldı. Hárbir televizordıń defektli bolıwı itimallıǵı 0,005 ke hám hárbir radionıń defektli bolıwı itimallıǵı 0,04 ke teń. Usı sawdada tómendegi waqıyalardıń júzege asıwı itimallıqların tabıń: a) keminde tórt televizor defektli bolıwı; b) 35 ten 45 ke shekem radio defektli bolıwı.
====
Radio ustaxanada kúnine 3 radio ońlanadı. Radionıń mexanikalıq bólegi buzılıw itimallıǵı 0,1 ge hám elektron bólegi buzılıw itimallıǵı 0,004 ke teń. Jıl dawamında ońlanǵan radiolar arasında tómendegi waqıyalardıń júzege asıw itimallıqların tabıń: a) 130 dan 140 ǵa shekem radiolardıń mexanikalıq bóleginde nasazlıqlar bolǵan; b) tórtten artıq bolmaǵan radiolardıń elektron bóleginde nasazlıqlar bolǵan.
====
Radio ustaxanada kúnine 2 radio ońlanadı. Radionıń mexanikalıq bólegi buzılıwı itimallıǵı 0,2 ge hám elektron bólegi buzılıw itimallıǵı 0,005 ke teń. Jıl dawamında ońlanǵan radiolar arasında tómendegi waqıyalardıń júzege asıw itimallıqların tabıń: a) 140 tan 150 ge shekem radiolardıń mexanikalıq bóleginde nasazlıqlar bolǵan; b) besten artıq bolmaǵan radiolardıń elektron bóleginde nasazlıqlar bolǵan.
====
Kitaptıń bir betinde keminde bir baspa qáteligi bolıw itimallıǵı 0,02 ge hám tártip qáteligi bolıw itimallıǵı bolsa 0,4 ke teń. Jámi 400 betli kitapta tómendegi waqıyalardıń júzege asıw itimallıqların tabıń: a) keminde bes bette baspa qáteligi bolıwı; b) 170 ten 180 ge shekem betlerde tártip qáteligi bolıwı.
====
Kitaptıń bir betinde keminde bir baspa qáteligi bolıw itimallıǵı 0,01 ge hám tártip qáteligi bolıw itimallıǵı bolsa 0,3 ke teń. Jámi 500 betli kitapta tómendegi waqıyalardıń júzege asıw itimallıqların tabıń: a) keminde tórt bette baspa qáteligi bolıwı; b) 140 tan 170 ke shekem betlerde tártip qáteligi bolıwı.
====
Uzaq aralıqtaǵı nıshanǵa pulemyot hám pistolet penen oq atılmaqta. Pulemyot oǵınıń nıshanǵa tiyiw itimallıǵı 0,02, al pistolet penen bolsa 0,6 ǵa teń. Eger hárbir qural menen nıshanǵa 100 márte oq atılǵan bolsa, tómendegi waqıyalardıń júzege asıw itimallıqların tabıń: a) pulemyot penen nıshanǵa úsh márte tiygiziwi; b) pistolet penen nıshanǵa 17 den 22 mártege shekem tiygiziwi.
====
Uzaq aralıqtaǵı nıshanǵa pulemyot hám pistolet penen oq atılmaqta. Pulemyot oǵınıń nıshanǵa tiyiw itimallıǵı 0,02, al pistolet penen bolsa 0,6 ǵa teń. Eger hárbir qural menen nıshanǵa 100 márte oq atılǵan bolsa, tómendegi waqıyalardıń júzege asıwı itimallıqların tabıń: a) pulemyot penen nıshanǵa úsh mártege shekem tiygiziwi; b) pistolet penen nıshanǵa 60 márte tiygiziwi.
====
Televizion kapital showda individual oyınshınıń 2000 dollar utıp alıwı itimallıǵı 0,4 ke, al 33000 dollar utıp alıwı itimallıǵı bolsa 0,02 ke teń. Bul oyında 400 oyınshı qatnasqan bolsa, tómendegi waqıyalardıń júzege asıwı itimallıqların tabıń: a) 170 ten 190 ǵa shekem oyınshı 2000 dollar utıwı; b) úshten kóp bolmaǵan oyınshı 33 000 dollar utıwı.
====
“Kim millioner bolıwdı qáleydi” oyınında individual oyınshınıń 1000 dollar utıp alıwı itimallıǵı 0,3 ke, al 32000 dollar utıw itimallıǵı bolsa 0,01 ge teń. Bul oyında 300 oyınshı qatnasqan bolsa, tómendegi waqıyalardıń júzege asıwı itimallıqların tabıń: a) 80 nen 100 ge shekem oyınshı 1000 dollar utıwı; b) tórtten kóp bolmaǵan oyınshı 32000 dollar utıwı.
====
Elektrotexnika dúkanına hár qaysısı 1000 danadan bolǵan joqarı sapalı hám tómen sapalı muzlatqıshlar alıp kelindi. Joqarı sapalı muzlatqıshtıń defektli bolıwı itimallıǵı 0,001 ge, al tómen sapalı muzlatqıshtıń defektli bolıwı itimallıǵı bolsa 0,03 ke teń. a) Eki dana joqarı sapalı muzlatqıshtıń defektli bolıwı; b) 50 den 70 ke shekem tómen sapalı muzlatqıshlardıń defektli bolıwı itimallıqların tabıń.
====
Elektrotexnika dúkanına hár qaysısı 2000 danadan joqarı sapalı hám tómen sapalı muzlatqıshlar alıp kelindi. Joqarı sapalı muzlatqıshtıń defektli bolıw itimallıǵı 0,002 ge, al tómen sapalı muzlatqıshtıń defektli bolıw itimallıǵı bolsa 0,04 ke teń. a) Úsh dana joqarı sapalı muzlatqıshtıń defektli bolıwı; b) 60 tan 65 ke shekem tómen sapalı muzlatqıshlardıń defektli bolıwı itimallıqların tabıń.
====
Jip iyiriw fabrikasında 1000 dana jańa hám 200 dana eski jip iyiriw qurılmaları bar. Bir jumıs kúninde jańa qurılma 0,003 itimallıq penen, al eski qurılma bolsa 0,20 itimallıq penen iyirilip atırǵan jipti úzip aladı. Bir jumıs kúninde tómendegi waqıyalardıń júzege asıwı itimallıqların tabıń: a) jańa qurılma 3 márte jipti úzip alıwı; b) eski qurılma 10 nan 15 ke shekemgi jipti úzip alıwı. 
====
Jip iyiriw fabrikasında 1500 dana jańa hám 100 dana eski jip iyiriw qurılmaları bar. Bir jumıs kúninde jańa qurılma 0,002 itimallıq penen, al eski qurılma bolsa 0,30 itimallıq penen iyirilip atırǵan jipti úzip aladı. Bir jumıs kúninde tómendegi waqıyalardıń júzege asıwı itimallıqların tabıń: a) jańa qurılma 5 márte jipti úzip alıw; b) eski qurılma 20 ten 25 ǵa shekemgi jipti úzip alıw. 
====
Uzaq aralıqtaǵı nıshanǵa pulemyot hám pistolet penen oq atılmaqta. Pulemyot oǵınıń nıshanǵa tiyiw itimallıǵı 0,02, al pistolet penen bolsa 0,6 ǵa teń. Eger hárbir qural menen nıshanǵa 100 márte oq atılǵan bolsa, tómendegi waqıyalardıń júzege asıwı itimallıqların tabıń: a) pulemyot penen nıshanǵa tórt márte tiygiziwi; b) pistolet penen nıshanǵa 70 márte tiygiziwi.