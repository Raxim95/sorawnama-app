Analitikalıq geometriya pániniń predmeti hám usılları.
====
Vektor túsinigi. Vektorlar ústindegi sızıqlı ámeller.
====
Sızıqlı baylanıslı hám sızıqlı baylanıslı bolmaǵan vektorlar.
====
Vektordıń koordinataları.
====
Vektorlardıń skalyar kóbeymesi.
====
Vektorlardıń vektor kóbeymesi hám aralas kóbeymesi.
====
Koordinataları menen berilgen vektorlardıń skalyar, vektor hám aralas kóbeymeleri.
++++
Tegislikte hám keńislikte dekart koordinatalar sistemasın almastırıw.
====
Tegislikte tuwrı sızıqtıń tenlemeleri.
====
Tegisliktegi tuwrı sızıqlardıń óz ara jaylasıwı.
====
Noqattan tuwrı sızıqqa shekem bolǵan aralıq. Tuwrılar dástesi.
====
Tegisliktiń tenlemeleri. Tegisliklerdiń óz ara jaylasıwı.
====
Keńisliktegi tuwrı sızıqtıń tenlemeleri. Tuwrı sızıqlardıń óz ara jaylasıwı.
====
Tegislik hám tuwrı sızıqlardıń óz ara jaylasıwı.
====
Noqattan tegislikke shekem, keńislikte noqattan tuwrı sızıqqa shekem hám ayqash tuwrı sızıqlar arasındaǵı aralıq.
++++
$M_1 (1; -2) $, $M_2 (2; 1) $ noqatlar berilgen.
Tómendegi kesindilerdiń koordinata kósherlerine proekciyaların tabıń: $\overline{M_1M_2}$
====
Kvadrattıń eki qońsılas tóbeleri $A (3; -7) $ hám
$B (-1;4) $ berilgen. Onıń maydanın esaplań.
====
Kvadrattıń eki qarama-qarsı tóbeleri $P (3; 5) $ hám
$Q (1; -3) $ berilgen. Onıń maydanın esaplań.
====
Eki tóbesi $A (-3; 2) $ hám $B (1; 6) $ noqatlarda
jaylasqan durıs úshmúyeshliktiń maydanın esaplań.
====
$ABCD$ parallelogrammnıń úsh tóbesi $A (3; -7) $,
$B (5; -7) $, $C (-2; 5) $ berilgen, tórtinshi ushı $D$,
$B$ tóbesine qarama-qarsı. Sol parallelogrammnıń diagonalları
uzınlıqların anıqlań.
====
Berilgen $A (3; -5) $, $B (-2; -7) $ hám
$C (18; 1) $ noqatlar bir tuwrı sızıqta jatıwın dálilleń.
====
$A (2;2) $, $B (-1;6) $, $C (-5;3) $ hám $D (-2;-1) $
noqatları kvadrat tóbeleri ekenligin dálilleń.
====
Bir tekli elementten islengen qatardıń tóbeleri
$A (3;-5) $ hám $B (-1;1) $ noqatlarda jaylasqan. Onıń awırlıǵı
orayınıń koordinatasın anıqlań.
====
Bir tekli elementten islengen qatardıń awırlıq orayı
$M (1;4) $ noqatında, bir tóbesi $P (-2;2) $ noqatında jaylasqan. Bul
qatardıń ekinshi ushı $Q$ nın koordinataların anıqlań.
====
Úshmúyeshlik tóbelerinin koordinataları berilgen
$A (1;-3) $, $B (3;-5) $ hám $C (-5;7) $. Tárepleriniń ortaların
anıqlań.
====
$M (2;-1) $, $N (-1;4) $ hám $P (-2;2) $ noqatlar
úshmúyeshlik táreplerinin ortaları. Tóbeleriniń koordinataların
anıqlań.
====
Parallelogramnıń tóbeleri
$A (3;-5) $, $B (5;-3) $, $C (-1;3) $ berilgen. $B$ tóbesine
qarama-qarsı jaylasqan $D$ ushın anıqlań.
====
Parallelogrammnıń eki qońsılas tóbeleri $A (-3;5) $, $B (1;7) $
hám dioganallarınıń kesilisiw noqatı $M (1;1) $ berilgen. Qalǵan eki
tóbesin anıqlań.
====
$ABCD$-parallelogramnıń úsh tóbesi
$A (2;3) $, $B (4;-1) $ hám $C (0;5) $ berilgen. Tórtinshi $D$
tóbesin tabıń.
====
Úshmúyeshliktiń tóbeleri $A (1;4) $, $B (3;-9) $, $C (-5;2) $
berilgen. $B$ tóbesinen ótkerilgen mediananıń uzınlıǵın anıqlań.
====
$A (1;-3) $ hám $B (4;3) $ noqatlardı tutastırıwshı
kesindi teń úsh bólekke bólindi. Bóliwshiler noqatlarınıń koordinataların
anıqlań.
====
$A (4;2) $, $B (7;-2) $ hám $C (1;6) $ noqatlar bir tekli
sımnan islengen úshmúyeshlik tóbeleri. Sol úshmúyeshliktiń awırlıq orayın tabıń.
====
Tóbeleri $A (2;-3) $, $B (3;2) $ hám $C (-2;5) $
noqatlarında jaylasqan úshmúyeshliktiń maydanın esaplań.
====
Tóbeleri $M_1 (-3;2) $, $M_2 (5;-2) $ hám $M_3 (1;3) $
noqatlarında jaylasqan úshmúyeshliktiń maydanın esaplań.
====
Tóbeleri $M (3;-4) $, $N (-2;3) $ hám $P (4;5) $
noqatlarında jaylasqan úshmúyeshliktiń maydanın esaplań.
====
Uch uchi $A (-2;3), \ B (4;-5) $ va
$C (-3;1)$ noqatlarda jaylasqan parallelogrammnıń maydanın anıqlań.
====
Bir tekli tórtmúyeshli plastinkanıń tóbeleri berilgen:
$A (2;1), \ B (5;3), \ C (-1;7) $ hám $D (-7;5) $. Onıń awırlıq orayı
koordinataların anıqlań.
====
Bir tekli bes múyeshli plastinkanıń tóbeleri berilgen:
$A (2;3), B (0;6), C (-1;5), D (0;1) $ hám $E (1;1) $. Onıń awırlıǵı
orayınıń koordinataların anıqlań.
====
Eki tóbesi $A (3;1) $ hám $B (1;-3) $ noqatlarda, a
úshinshi $C$ tóbesi $Oy$ kósherine tiyisli úshmúyeshliktiń
maydanı $S=3$ qa teń. $C$ tóbesiniń koordinataların anıqlań.
====
Eki tóbesi $A (2;1) $ hám $B (3;-2) $ noqatlarda, hám
úshinshi $C$ tóbesi $Ox$ kósherine tiyisli bolǵan úshmúyeshliktiń
maydanı $S=4$ qa teń. $C$ tóbesiniń koordinataların anıqlań.
++++
Berilgen $M_1 (3; 1) $, $M_2 (2; 3) $, M_3 (6; 3) $,
$M_4 (-3;-3) $. $M_5 (3;-1) $, $M_6 (-2; 1) $ noqatlardıń qaysıları
$2x-3y-3=0 tuwrı sızıqqa tiyisli hám qaysıları tiyisli
emes.
====
$P1$, $P2$, $P3$, $P4$, $P5$ noqatlar
3x-2y-6=0 tuwrı sızıqqa tiyisli hám abscissaları sáykes túrde
4, 0, 2, -2, -6 ģa teń. Olardıń ordinataların tabıń.
====
$Q_1$, $Q_2$, $Q_3$, $Q_4$, $Q_5$ noqatlar
$x-3y+2=0$ tuwrı sızıqqa tiyisli hám ordinataları sáykes túrde
1, 0, 2, -1, 3 ke teń. Olardıń abscissaların tabıń.
====
$5x-y+3=0$ tuwrı sızıqtıń $k$ múyeshi
koefficientin hám $Oy$ kósherinen kesip algan kesindiniń algebralıq
mánisin anıqlań $b$.
====
$2x+3y-6=0$ tuwrı sızıqtıń $k$ múyeshi
koefficientin hám $Oy$ kósherinen kesip algan kesindiniń algebralıq
mánisin anıqlań $b$.
====
$5x+3y+2=0$ tuwrı sızıqtıń $k$ múyeshi
koefficientin hám $Oy$ kósherinen kesip algan kesindiniń algebralıq
mánisin anıqlań $b$.
====
$3x+2y=0$ tuwrı sızıqtıń $k$ múyeshi
koefficientin hám $Oy$ kósherinen kesip algan kesindiniń algebralıq
mánisin anıqlań $b$.
====
$y-3=0$ tuwrı sızıqtıń $k$ múyeshi
koefficientin hám $Oy$ kósherinen kesip algan kesindiniń algebralıq
mánisin anıqlań $b$.
====
Ulıwma teńleme menen berilgen tuwrı sızıqlardıń
óz ara jaylasıwın anıqlań, eger kesilisiwshi bolsa kesilisiw noqatın
tabıń: $12x+15y-39=0, 16x-9y-23=0$.
====
Ulıwma teńleme menen berilgen tuwrı sızıqlardıń
óz ara jaylasıwın anıqlań, eger kesilisiwshi bolsa kesilisiw noqatın
tabıń: $3x+2y-27=0, x+5y-35=0$.
====
Ulıwma teńleme menen berilgen tuwrı sızıqlardıń
óz ara jaylasıwın anıqlań, eger kesilisiwshi bolsa kesilisiw noqatın
tabıń: $12x+59y-19=0, 8x+33y-19=0$.
====
Ulıwma teńleme menen berilgen tuwrı sızıqlardıń
óz ara jaylasıwın anıqlań, eger kesilisiwshi bolsa kesilisiw noqatın
tabıń: $6x+10y+9=0, 3x+5y-6=0$.
====
Ulıwma teńleme menen berilgen tuwrı sızıqlardıń
óz ara jaylasıwın anıqlań, eger kesilisiwshi bolsa kesilisiw noqatın
tabıń: $14x-9y-24=0, 7x-2y-17=0$.
====
Ulıwma teńleme menen berilgen tuwrı sızıqlardıń
óz ara jaylasıwın anıqlań, eger kesilisiwshi bolsa kesilisiw noqatın
tabıń: $2x-3y+12=0, 4x-6y-21=0$.
====
Ulıwma teńleme menen berilgen tuwrı sızıqlardıń
óz ara jaylasıwın anıqlań, eger kesilisiwshi bolsa kesilisiw noqatın
tabıń: $2y+9=0, y-5=0$.
====
Ulıwma teńleme menen berilgen tuwrı sızıqlardıń
óz ara jaylasıwın anıqlań, eger kesilisiwshi bolsa kesilisiw noqatın
tabıń: $4x-7=0, 3x+8=0$.
====
Ulıwma teńleme menen berilgen tuwrı sızıqlardıń
óz ara jaylasıwın anıqlań, eger kesilisiwshi bolsa kesilisiw noqatın
tabıń: $2x-5y+1=0, 6x-15y+3=0$.
====
Ulıwma teńleme menen berilgen tuwrı sızıqlardıń
óz ara jaylasıwın anıqlań, eger kesilisiwshi bolsa kesilisiw noqatın
tabıń: $x-5=0, y+12=0$.
====
Ulıwma teńleme menen berilgen tuwrı sızıqlardıń
óz ara jaylasıwın anıqlań, eger kesilisiwshi bolsa kesilisiw noqatın
tabıń: $x\sqrt{2}+12=0, 4x+24\sqrt{2}=0$.
====
Ulıwma teńleme menen berilgen tuwrı sızıqlardıń
óz ara jaylasıwın anıqlań, eger kesilisiwshi bolsa kesilisiw noqatın
tabıń: $3x+y\sqrt{3}=0, x\sqrt{3}+3y-6=0$.
====
$a$ hám $b$ parametrlerinin qanday mánislerinde
$ax-2y-1=0$, $6x-4y-b=0$ tuwrı sızıqlar uliwma noqatqa iye boladı?
====
$a$ hám $b$ parametrlerinin qanday mánislerinde
$ax-2y-1=0$, $6x-4y-b=0$ tuwrı sızıqlar parallel boladı?
====
$a$ hám $b$ parametrlerinin qanday mánislerinde
$ax-2y-1=0$, $6x-4y-b=0$ tuwrı sızıqlar kesilisedi?
====
$m$ hám $n$ parametrlerinin qanday mánislerinde
$mx+8y+n=0$, $2x+my-1=0$ tuwrı sızıqlar parallel boladı?
====
$m$ parametriniń qanday mánislerinde
$ (m-1) x+my-5=0$, $mx+ (2m-1) y+7=0$ tuwrı sızıqlar abscissa
kósherinde jatıwshı noqatta kesilisedi.
====
$m$ parametriniń qanday mánislerinde
$mx+ (2m+3) y+m+6=0$, $ (2m+1) x+ (m-1) y+m-2=0$ tuwrı sızıqlar ordinata
kósherinde jatıwshı noqatta kesilisedi.
====
$3x-y+2=0$, $4x-5y+5=0$, $2x+3y-1=0$
tuwrı sızıqlar bir noqatta kesilisedi me?
====
$5x+3y-7=0$, $x-2y-4=0$, $3x-y+3=0$
tuwrı sızıqlar bir noqatta kesilisedi me?
====
$x+2y-17=0$, $2x-y+1=0$, $x+2y-3=0$
tuwrı sızıqlar bir noqatta kesilisedi me?
====
$2x-y+2=0$, $4x-2y+4=0$, $6x-3y+6=0$
tuwrı sızıqlar bir noqatta kesilisedi me?
====
$5x-3y+15=0$ tuwrı sızıqtıń koordinata múyeshinen
kesip algan úshmúyeshliktiń maydanın esaplań.
====
$M (-3;8) $ noqattan ótip, koordinata kósherlerinen ótedi
teń kesindilerdi kesip ótetuģın tuwrı sızıqlardıń teńlemesin dúziń.
====
$M (3;3) $ noqattan ótip, koordinata kósherlerinen teń
kesindilerdi kesip ótetuģın tuwrı sızıqlardıń teńlemesin dúziń.
====
$P (2;2) $ noqattan ótip, koordinata múyeshinen ótedi
maydanı 1 ge teń úshmúyeshlik kesip ótetuģın tuwrı sızıqlardıń
teńlemesin dúziń.
====
$B (-5;5) $ noqattan ótip, koordinata múyeshinen ótedi
maydanı 50 ge teń úshmúyeshlik kesip ótetuģın tuwrı sızıqlardıń teńlemesin
dúziń.
====
$P (8;6) $ noqattan ótip, koordinata múyeshinen ótedi
maydanı 12 ge teń úshmúyeshlik kesip ótetuģın tuwrı sızıqlardıń teńlemesin dúziw
dúziń.
====
$P (12;6) $ noqattan ótip, koordinata múyeshinen ótedi
maydanı 150 ge teń úshmúyeshlik kesip ótetuģın tuwrı sızıqlardıń
teńlemesin dúziń.
====
$M (4;3) $ noqattan, koordinata múyeshinen
maydanı 3 ke teń úshmúyeshlikti kesip ótetuģın tuwrı sızıq júrgizildi.
Usi tuwrı sızıqtıń koordinata kósherleri menen kesilisiw noqatları
koordinataların anıqlań.
====
$A (3;-2) $ noqattan $3x+4y-15=0$ tuwrı sızıqqa
ge shekemgi jiljıwdı hám aralıqtı esaplań.
++++
Vektor koordinata kósherleri menen tómendegi múyeshlerdi payda ete aladı ma:
$\alpha = 45^{{^\circ}},\beta = 60^{{^\circ}},\gamma = 120^{{^\circ}}$.
====
Vektor koordinata kósherleri menen tómendegi múyeshlerdi payda ete aladı ma:
$\alpha = 45^{{^\circ}},\ \beta = 135^{{^\circ}},\ \gamma = 60^{{^\circ}}$.
====
Vektor koordinata kósherleri menen tómendegi múyeshlerdi payda etiwi
múmkin be: $\alpha = 90^{{^\circ}},\ \beta = 150^{{^\circ}}$,
$\gamma = 60^{{^\circ}}?$
====
Tegislikte eki vektor
$\overrightarrow{p} = \{ 2; - 3\}$, $\overrightarrow{q} = \{ 1;2\}$.
$\overrightarrow{a} = \{9;4\}$ vektorınıń
$\overrightarrow{p},\ \overrightarrow{q}$ bazis boyinsha jayılması tabılsın.
====
Tórtmúyeshliktiń tóbeleri berilgen:
$A (1; - 2;2) $, $B (1;4;0),C (- 4;1;1) $ hám $D (- 5; -5;3) $. Onıń diagonalları $AC$ hám $BD$ óz ara
perpendikulyarlıǵın dálilleń.
====
$\alpha$
qanday mánislerde
$\overrightarrow{a} = \alpha\overrightarrow{i} - 3\overrightarrow{j} + 2\overrightarrow{k}$
hám
$\overrightarrow{b} = \overrightarrow{i} + 2\overrightarrow{j} - \alpha\overrightarrow{k}$
vektorlar óz ara perpendikulyar bolatuģının anıqlań.
====
$\overrightarrow{a} = \{ 2; - 4;4\}$ hám $\overrightarrow{b} = \{ - 3;2;6\}$
vektorlar payda etken múyesh kosinusın esaplań.
====
Úshmúyeshliktiń tóbeleri
$A (- 1; - 2;4) $, $B (- 4; - 2;0) $ hám $C (3; -2;1) $. Onıń $B$ tóbesindegi
ishki múyeshti anıqlań.
====
Úshmúyeshliktiń tóbeleri
$A (3;2; 3) $, $B (5;1; - 1) $ hám $C (1; -2;1) $. Onıń $A$ tóbesindegi sırtqı múyeshi aniqlan.
====
Tóbeleri $A (1;2;1), B (3;-1;7) $ hám $C (7;4;-2) $ bolǵan úshmúyeshliktiń
ishki múyeshlerin esaplap tabıń. Bul úshmúyeshliktiń teń qaptallı ekenligin dálilleń.
====
$\overrightarrow{a}$ hám $\overrightarrow{b}$ vektorlar
$\varphi = \pi/6$ múyesh payda etedi.
$|\overrightarrow{a}| = 6,|\overrightarrow{b}| = 5$ ekenligin bilip,
$\left| \left\lbrack \overrightarrow{a},\overrightarrow{b} \right\rbrack \right|$ shamalardı esaplań.
====
Berilgen: $\overrightarrow{a}| = 10,|\overrightarrow{b}| = 2$ hám
$\left(\overrightarrow{a},\overrightarrow{b} \right) = 12$. Esaplań
$\left| \left\lbrack \overrightarrow{a},\overrightarrow{b} \right\rbrack \right|$.
====
Berilgen: $\overrightarrow{a}| = 3,|\overrightarrow{b}| = 26$ hám
$\lbrack\overrightarrow{a},\overrightarrow{b}\rbrack| = 72$. Esaplań
$\left(\overrightarrow{a},\overrightarrow{b} \right) $.
====
$\overrightarrow{a}
= \{ 1; - 1;3\}, \ \overrightarrow{b} = \{ - 2;1\}$, $\overrightarrow{c} = \{3; -2;5\}$ vektorlar berilgen. Esaplań:
$ (\lbrack\overrightarrow{a},\overrightarrow{b}\rbrack,\overrightarrow{c}) $.
====
Eger \(a = \{ 2;3; - 1\}, \ b = \{ 1; - 1;3\}, \ c = \{ 1;9; - 11\}\) bolsa, $\overrightarrow{a}, \overrightarrow{b}, \overrightarrow{c}$ vektorlar komplanar boliwin tekseriń.
====
Eger \(a = \{ 3; - 2;1\}, \ b = \{ 2;1;2\}, \ c = \{ 3; - 1; - 2\}\) bolsa, $\overrightarrow{a}, \overrightarrow{b}, \overrightarrow{c}$ vektorlar komplanar boliwin tekseriń.
====
Eger \(a = \{ 2; - 1;2\}, \ b = \{ 1;2; - 3\}, \ c = \{ 3; - 4;7\}\) bolsa, $\overrightarrow{a}, \overrightarrow{b}, \overrightarrow{c}$ vektorlar komplanar boliwin tekseriń.
++++
Eki qarama-qarsı tóbeleri \(P (4;9) \) hám \(Q (-2; 1) \) noqatlarında jaylasqan rombtıń tárepi uzınlıǵı \(5\sqrt{10}\). Bul
romb maydanın esaplań.
====
Eki qarama-qarsı tóbeleri $P (3; -4) $ hám $Q (l;2) $ noqatlarda jaylasqan rombtıń tárepi uzınlıǵı \(5\sqrt{2}\). Sol romb biyikliginiń uzınlıǵın esaplań.
====
Tóbeleri $A_1 (1; 1), A_2 (2; 3) $ hám $A (5;-1) $
noqatlarında jaylasqan úshmúyeshliktiń tuwrı múyeshli ekenligin dálilleń.
====
Tóbeleri \(M_{1} (1;1), M_{2} (0,2) \) hám
\(M_{3} (2;-1) \) noqatlarda jaylasqan úshmúyeshliktiń ishki
múyeshleri arasında ótpeytuģın múyesh bar yaki joq ekenligin anıqlań.
====
Tóbeleri \(M (-1;3),\ N (1,2) \ \) hám \(P (0;4) \)
noqatlarında jaylasqan úshmúyeshliktiń ishki múyeshleri ótkir múyesh
ekenligin dálilleń.
====
Úshmúyeshliktiń tóbeleri \(A (5;0),\ B (0;1) \) hám \(C (3;3) \)
noqatlarında. Oniń ishki múyeshlerin tabıń.
====
Úshmúyeshliktiń tóbeleri
\(A\left(-\sqrt{3};1 \right),\ B (0;2) \) hám
\(C\left(-2\sqrt{3};2 \right) \) noqatlarda. Onıń $A$
tóbesindegi sırtqı múyeshti tabıń.
====
Abscissa kósherinde sonday $M$ noqattı tabıń,
\(N (2;-3) \) noqattan uzaqlıǵı 5 ke teń bolǵan.
====
Ordinata kósherinde sonday $M$ noqattı tabıń.
\(N (-8;13) \) noqattan uzaqlıǵı 17 ge teń bolǵan.
====
Eki noqat berilgen \(M (2;2) \) hám \(N (5;-2) \); abscissa kósherinde sonday $P$ noqattı tabıń, $MPN$ múyeshi tuwrı múyesh bolsin.
====
\(M_{1} (1;2) \) noqatqa, \(A (1;0) \) hám \(B (-1;-2) \)
noqatlarınan ótiwshi tuwrı sızıqqa salıstırganda simmetriyalı bolǵan \(M_{2}\) noqattıń koordinataların tabıń.
====
Úshmúyeshliktiń tóbeleri \(A (2;-5),\ B (1;-2),\ C (4;7) \)
berilgen. $AC$ tárepi menen $B$ tóbesiniń ishki múyeshi
bissektrisasınıń kesilisiw noqatın tabıń.
====
Úshmúyeshliktiń tóbeleri
\(A (3;-5),\ B (-3;3),\ C (-1;-2) \) berilgen. $A$ tóbesiniń ishki
múyeshi bessektrisanıń uzınlıǵın anıqlań.
++++
Bir tuwrı sızıqqa tiyisli \(A (1;-1),\ B (3;3) \) hám
\(C (4;5) \) noqatlar berilgen. Hárbir noqattıń, qalǵan eki noqat arqalı anıqlanatuģın kesindiniń bóliw qatnasın anıqlań $\lambda$.
====
\(P (2;2) \) hám \(Q (1;5) \) noqatlar menen teń úsh
bólingen kesindiniń tóbeleri $A$ hám $B$ noqatlarınıń
koordinataların anıqlań.
====
Tuwrı \(M_{1} (-12;-13) \) hám \(M_{2} (-2;-5) \)
noqatlarınan ótedi. Sol tuwrı sızıqta abscissası 3 ke teń noqattı tabıń.
====
Tuwrı sızıq \(M (2;-3) \) hám \(N (-6;5) \) noqatlardan ótedi.
Usi tuwrı sızıqta ordinatasi $-5$ qa teń noqatti tabıń.
====
Tuwrı sızıq \(A (7;-3) \) hám \(B (23;-6) \) noqatlardan ótedi.
Sol tuwrı sızıqtıń abscissa kósheri menen kesilisiw noqatın tabıń.
====
Tuwrı \(A (5;2) \) hám \(B (-4; -7) \) noqatlarınan ótedi.
Sol tuwrı sızıqtıń ordinata kósheri menen kesilisiw noqatın tabıń.
====
Tórtmúyeshliktiń tóbeleri
\(A (-3;12),\ B (3;-4),\ C (5;-4) \) hám \(D (5;8) \) berilgen. Bul
tórtmúyeshliktiń $AC$ diagonalı $BD$ diagonalı qanday
qatnasında bolıwın anıqlań.
====
Tórtmúyeshliktiń tóbeleri
\(A (-2;14),\ B (4;-2),\ C (6;-2) \) hám \(D (6;10) \) berilgen. Bul
tórtmúyeshliktiń $AC$ hám $BD$ diagonallarınıń kesilisiwi
noqatın tabıń.
====
Úshmúyeshliktiń tóbeleri \(A (3;6),\ B (-1;3) \) hám
\(C (2: 1) \) noqatlarda jaylasqan. $C$ tóbesinen túsirilgen biyiklik uzınlıǵın esaplań.
====
Parallelogramnıń úsh tóbesi \(A (3;7),\ B (2;-3) \) hám
\(C (-1;4) \) noqatlarda jaylasqan. $B$ tóbesinen $AC$
tárepinen túsirilgen biyiklik uzınlıǵın esaplań.
====
Eki tóbesi \(A (3;1) \) hám \(B (1;-3) \) noqatlarda, hám
awırlıq orayı $Ox$ kósherine tiyisli úshmúyeshliktiń maydanı
\(S=3\) ge teń. Úshinshi $C$ tóbesiniń koordinataların anıqlań.
++++
Berilgen tuwrı sızıqlardıń kesilisiw noqatın tabıń:
$(3x-4y-29=0, 2x+5y+19=0) $.
====
$ABC$ úshmúyeshliginiń tárepleri:
\(AB:4x+3y-5=0,\ BC:x-3y+10=0,\ AC:x-2=0\)
teńlemeleri menen berilgen. Tóbeleriniń koordinataların anıqlań.
====
Parallelogramnıń eki tárepi teńlemeleri
\(8x+3y+1=0,\ 2x+y-1=0\) hám bir diagonalı teńlemesi
\(3x+2y+3=0\) berilgen. Parallelogramm tóbeleri koordinataların
anıqlań.
====
Úshmúyeshliktiń tárepleri \(x+5y-7=0\),
\(3x-2y-4=0\), \(7x+y+19=0\) tuwrı sızıqlarda jatadi. Onıń
maydanın esaplań.
====
Ulıwma teńlemesi \(2x-5y+4=0\) bolǵan durıs
berilgen. \(M (-3,5) \) noqattan ótip, berilgen tuwrı sızıqqa: a) parallel;
b) perpendikulyar bolǵan tuwrı sızıqlar teńlemesin dúziń.
====
Tuwrı tórtmúyeshliktiń bir tóbesi \(A (2;-3) \), hám eki tárepiniń
niń teńlemeleri \(2x+3y+9=0,\ 3x-2y-7=0\)
berilgen. Qalǵan eki táreptiń teńlemelerin dúziń.
====
\(N (5;8) \) noqattıń, \(5x-11y-43=0\) tuwrı sızıqtaǵı
proekciyasın tabıń.
====
Tómendegi hárbir tuwrı sızıqlar juplıǵı ushın, olarǵa parallel
bólip, anıq ortasınan ótiwshi tuwrı teńlemeni dúziń: $3x-2y-3=0$, $3x-2y-17=0$.
====
Berilgen eki noqattan ótiwshi tuwrı sızıqtıń múyesh
koefficienti $k$ nı esaplań: $A (-4;3) $, $B (1;8) $.
====
Úshmúyeshlik tóbeleri \(A (1;0),\ B (5;-2),\ C (3;2) \)
koordinataları menen berilgen. Úshmúyeshlikler tárepleriniń hám
medianalarınıń teńlemelerin dúziń.
====
\(P (3;8) \) hám \(Q (-1;-6) \) noqatlardan ótken
tuwrı sızıqtıń koordinata kósherleri menen kesilisiw noqatların tabıń.
====
Dóńes tórtmúyeshliktiń tóbeleri
\(A (-2;-6),\ B (7;6),\ C (3;9) \) hám \(D (-3;1) \) noqatlarda
jaylasqan. Diagonallarınıń kesilisiw noqatı tabılsın.
====
$ABCD$ parallelogramnıń eki qońsılas tóbeleri
\(A (3,3),\ B (-1;7) \) hám diagonallarının kesilisiw noqatı
\(E (2;-4) \) berilgen. Sol parallelogramm tárepleriniń teńlemelerin
dúziń.
++++
Tuwrı tórtmúyeshliktiń eki tárepi
\(5x+2y-7=0,\ 5x+2y-36=0\) hám diagonalı
\(3x+7y-10=0\) teńlemeler menen berilgen. Qalǵan eki tárepi
teńlemelerin dúziń.
====
Berilgen tuwrı sızıqlar arasındaǵı múyeshti anıqlań: $3x+2y+4=0, 5x-y+1=0$.
====
Qırları
\(7x+y+31=0,\ 3x+4y-1=0,\ x-7y-17=0\) teńlemeler
menen berilgen úshmúyeshliktiń teń qaptallı ekenligin dálilleń.
Máseleni úshmúyeshliktiń
múyeshlerin tabıw arqalı sheshiń.
====
\(N (4;-5) \) noqattan ótip, $2x+5y-7=0$
tuwrı sızıqlarına parallel tuwrı sızıqlardıń teńlemesin dúziń. Máseleniń múyesh
koefficientti esaplamastan sheshiń.
====
Tómende berilgen tuwrı sızıqlar juplıǵınan qaysıları
perpendikulyar ekenligin anıqlań: $4x+y+6=0, 2x-8y-13=0$.
====
Eki tuwrı sızıqtıń arasındaģı múyeshti tabıń: $2x+y-9=0, 3x-y+11=0$.
====
Parallel tuwrı sızıqlar arasındaģı aralıqtı esaplań: $5x-12y+13=0, 5x-12y-26=0$.
====
Kvadrattıń eki tárepi
\(5x-12y+65=0,\ 5x-12y-26=0\) tuwrı sızıqlarda
jatıwın bilgen halda, maydanın esaplań.
====
\(P (2;7) \) noqattan ótip, \(Q (1;2) \) noqatqa shekem
aralıǵı 5 ke teń bolǵan tuwrı sızıqlardıń teńlemesin dúziń.
====
\(M (7;-2) \) noqattan ótip, \(N (4;-6) \) noqatqa
ǵa shekem bolǵan aralıǵı 5 ke teń bolǵan tuwrı sızıqlardıń teńlemesin dúziń.
====
\(A (4;-5) \) noqattan ótip, \(B (-2;3) \) noqatqa
aralıǵı 12 ge teń bolǵan tuwrı sızıqlardıń teńlemesin dúziń.
====
Berilgen \(8x-15y-25=0\) tuwrı sızıqtan awısı -2 ge teń
teń bolǵan noqatlardıń geometriyalıq ornı teńlemesin dúziń.
====
Berilgen \(3x-4y-10=0\) tuwrı sızıqqa parallel hám onnan
$d=3$ aralıqta jatıwshı tuwrı sızıqlardıń tenlemesin dúziń.
====
Berilgen parallel tuwrı sızıqlardan teń aralıqta jatıwshı
noqatlardıń geometriyalıq ornı teńlemesin dúziń: $2x+y+7=0, 2x+y-3=0$.
====
\(P (1;-2) \) noqat hám koordinatalar bası, berilgen eki
tuwrınıń: $12x-5y-7=0, 3x+4y-8=0$.
kesilisiwinen payda bolǵan birdey múyeshte me yáki vertikal
múyeshlerde jata ma?
====
\(P (2;3) \) hám \(Q (5;-1) \) noqatlar, berilgen eki
tuwrınıń: $12x-y-7=0, 13x+4y-5=0$.
kesilisiwinen payda bolǵan birdey múyeshte me yáki vertikal 
múyeshlerde jata ma?
====
Koordinata bası, tárepleriniń tenlemeleri
\(8x+3y+31=0,\ x+8y-19=0,\ 7x-5y-11=0\) menen
berilgen úshmúyeshliktiń sırtında yamasa ishinde jatıwın anıqlań.
====
\(P (−3;2) \) noqat, táreplerinin tenlemeleri
\(x+y-4=0,\ 3x-7y+8=0,\ 4x-y-31=0\) menen
berilgen úshmúyeshliktiń sırtında yamasa ishinde jatıwın anıqlań.
====
Koordinata bası, berilgen tuwrı sızıqlardıń:
\(3x+y-4=0\) hám \(3x-2y+6=0\) kesilispesinde payda 
bolǵan súyir yamasa doǵal múyeshke tiyisli ekenligin anıqlań.
====
\(M (2;-5) \) noqat, berilgen tuwrı sızıqlardıń:
\(3x+5y-4=0\) hám \(x-2y+3=0\) kesilisiwinde payda 
bolǵan súyir yamasa doǵal múyeshke tiyisli ekenligin anıqlań.
====
\(4x+3y-1=0\) hám \(3x-2y+5=0\)
tuwrı sızıqlardıń kesilisiw noqatınan ótip (bul noqattı anıqlamay), ordinata
kósherinen \(b=4\) kesindini kesip ótetuģın tuwrı sızıq teńlemesin dúziń.
====
\(2x+y-2=0\) hám \(x-5y-3=0\)
tuwrı sızıqlardıń kesilisiw noqatınan ótip (bul noqattı anıqlamay), tóbeleri
\(A (-1;-4) \) hám \(B (5;-6) \) noqatlarda jaylasqan kesindiniń
tuwrı sızıqtıń ortasınan ótiwshi tuwrı sızıqtıń teńlemesin dúziń.
====
Tóbeleri \(A (4;-4),\ B (6;-1) \) hám \(C (-1;2) \)
noqatlarında jaylasqan bir tekli plastinkadan jasalǵan úshmúyeshliktiń
awırlıq orayınan ótip, tómende berilgen 
\(\alpha (2x+3y-1) +\beta (3x-4y-3) =0\) tuwrı sızıqlar dástesine 
tiyisli tuwrınıń teńlemesin dúziń.
++++
Tegislikte úsh vektor $\vec{a} = \{ 3; - 2\}$, $\vec{b} = \{ - 2;1\}$ hám $\vec{c} = \{ 7; - 4\}$ berilgen. Bul úsh vektorlardıń hár biriniń qalǵan ekewin bazis sıpatında qabıl etip, jayılmasın tabıń.
====
$\vec{a}$ hám $\vec{b}$ vektorlar $\varphi = 2\pi/3$ múyesh payda etedi. $|\vec{a}| = 3,|\vec{b}| = 4$ ekenligi belgili. Esaplań:
$\left(\vec{a},\vec{b} \right) $.
====
$\vec{a}$ hám $\vec{b}$ vektorlar $\varphi = 2\pi/3$ múyesh payda etedi. $|\vec{a}| = 3,|\vec{b}| = 4$ ekenligi belgili. Esaplań:
${\vec{a}}^{2}$.
====
$\vec{a}$ hám $\vec{b}$ vektorlar $\varphi = 2\pi/3$ múyesh payda etedi. $|\vec{a}| = 3,|\vec{b}| = 4$ ekenligi belgili. Esaplań:
${\vec{b}}^{2}$.
====
$\vec{a}$ hám $\vec{b}$ vektorlar $\varphi = 2\pi/3$ múyesh payda etedi. $|\vec{a}| = 3,|\vec{b}| = 4$ ekenligi belgili. Esaplań:
$ (\vec{a} + \vec{b}) ^{2}$.
====
$\vec{a}$ hám $\vec{b}$ vektorlar $\varphi = 2\pi/3$ múyesh payda etedi. $|\vec{a}| = 3,|\vec{b}| = 4$ ekenligi belgili. Esaplań:
$\left(3\vec{a} - 2\vec{b},\vec{a} + 2\vec{b} \right) $.
====
$\vec{a}$ hám $\vec{b}$ vektorlar $\varphi = 2\pi/3$ múyesh payda etedi. $|\vec{a}| = 3,|\vec{b}| = 4$ ekenligi belgili. Esaplań: $ (3\vec{a} + 2\vec{b}) ^{2}$.
====
$\vec{a}$ hám $\vec{b}$ vektorlar óz ara perpendikulyar; $\vec{c}$ vektor olar menen $\pi/3$ qa teń bolǵan múyeshler payda etedi; $|\vec{a}| = 3$, $|\vec{b}| = 5,\ |\vec{c}| = 8$ ekenligi belgili, tómendegilerdi esaplań:
$\left(3\vec{a} - 2\vec{b},\vec{b} + 3\vec{c} \right) $.
====
$\vec{a}$ hám $\vec{b}$ vektorlar óz ara perpendikulyar; $\vec{c}$ vektor olar menen $\pi/3$ qa teń bolǵan múyeshler payda etedi; $|\vec{a}| = 3$, $|\vec{b}| = 5,\ |\vec{c}| = 8$ ekenligi belgili, tómendegilerdi esaplań:
$ (\vec{a} + \vec{b} + \vec{c}) ^{2}$.
====
$\vec{a}$ hám $\vec{b}$ vektorlar óz ara perpendikulyar; $\vec{c}$ vektor olar menen $\pi/3$ qa teń bolǵan múyeshler payda etedi; $|\vec{a}| = 3$, $|\vec{b}| = 5,\ |\vec{c}| = 8$ ekenligi belgili, tómendegilerdi esaplań:
$ (\vec{a} + 2\vec{b} - 3\vec{c}) ^{2}$.
====
$\vec{a} + \vec{b} + \vec{c} = 0$ shártti qanaatlandırıwshı $\vec{a},\ \vec{b}$ hám $\vec{c}$ vektorlar berilgen. $|\vec{a}| = 3,\ |\vec{b}| = 1$ hám $|\vec{c}| = 4$ ekenligi belgili, $\left(\vec{a},\vec{b} \right) + \left(\vec{b},\vec{c} \right) + (\vec{c}) $ ańlatpanı esaplań.
====
$|\vec{a}| = 3,|\vec{b}| = 5$ berilgen. $\alpha$ nıń qanday mánisinde $\vec{a} + \alpha\vec{b}$, $\vec{a} - \alpha\vec{b}$ vektorlar óz ara perpendikulyar bolatuģının anıqlań.
====
$a$ hám $b$ vektorlar $\varphi = \pi/6$ múyesh payda etedi; $|a| = \sqrt{3},|b| = 1$ ekenligi belgili. $p = a + b$ hám $q = a - b$ vektorlar arasındaǵı $\alpha$ múyeshti esaplań.
====
$\vec{a} = \{ 6; - 8; - 7,5\}$ vektorģa kollinear bolǵan $\vec{x}$ vektor $Oz$ kósheri menen súyir múyesh payda etedi. $|\vec{x}| = 50$ ekenligin bilgen halda onıń koordinataların tabıń.
++++
$\vec{a} = \{ 2;1; - 1\}$ vektorģa kollinear bolǵan hám $\left(\vec{x},\vec{a} \right) = 3$ shártti qanaatlandırıwshı $\vec{x}$ vektordi tabıń.
====
$\vec{a}$ hám $\vec{b}$ vektorlar óz ara perpendikulyar. $|\vec{a}| = 3,|\vec{b}| = 4$ ekenligi belgili, tómendegilerdi esaplań:
$|\lbrack\vec{a} + \vec{b},\vec{a} - \vec{b}\rbrack|$.
====
$\vec{a}$ hám $\vec{b}$ vektorlar óz ara perpendikulyar. $|\vec{a}| = 3,|\vec{b}| = 4$ ekenligi belgili, tómendegilerdi esaplań:
$|\lbrack 3\vec{a} - \vec{b},\vec{a}-2\vec{b}\rbrack|$.
====
$\vec{a}$ hám $\vec{b}$ vektorlar $\varphi = 2\pi/3$ múyesh payda etedi. $|\vec{a}| = 1,|\vec{b}| = 2$ ekenligin bilip, tómendegilerdi esaplań:
$\lbrack\vec{a},\vec{b}\rbrack^{2}$.
====
$\vec{a}$ hám $\vec{b}$ vektorlar $\varphi = 2\pi/3$ múyesh payda etedi. $|\vec{a}| = 1,|\vec{b}| = 2$ ekenligin bilip, tómendegilerdi esaplań:
$\lbrack 2\overrightarrow{a} + \overrightarrow{b},\overrightarrow{a} + 2\overrightarrow{b}\rbrack^{2}$.
====
$\vec{a}$ hám $\vec{b}$ vektorlar $\varphi = 2\pi/3$ múyesh payda etedi. $|\vec{a}| = 1,|\vec{b}| = 2$ ekenligin bilip, tómendegilerdi esaplań:
$\lbrack\overrightarrow{a} + 3\overrightarrow{b},3\overrightarrow{a} - \overrightarrow{b}\rbrack^{2}$
====
$\vec{a} = \{ 3; - 1; - 2\}$ hám $\vec{b} = \{ 1;2; - 1\}$ vektorlar berilgen. Tómendegi vektor kóbeymelerdiń koordinataların tabıń:
$\left\lbrack \vec{a},\vec{b} \right\rbrack$.
====
$\vec{a} = \{ 3; - 1; - 2\}$ hám $\vec{b} = \{ 1;2; - 1\}$ vektorlar berilgen. Tómendegi vektor kóbeymelerdiń koordinataların tabıń:
$\left\lbrack 2\vec{a} + \vec{b},\vec{b} \right\rbrack$.
====
$\vec{a} = \{ 3; - 1; - 2\}$ hám $\vec{b} = \{ 1;2; - 1\}$ vektorlar berilgen. Tómendegi vektor kóbeymelerdiń koordinataların tabıń:
$\left\lbrack 2\vec{a} - \vec{b},2\vec{a} + \vec{b} \right\rbrack$.
====
$A (2; -1;2), B (1;2; 1) $ hám $C (3;2;1) $ noqatlar berilgen. Tómendegi vektor kóbeymelerdiń koordinataların tabıń:
$\lbrack\overline{AB},\overline{BC}\rbrack$.
====
$A (2; -1;2), B (1;2; 1) $ hám $C (3;2;1) $ noqatlar berilgen. Tómendegi vektor kóbeymelerdiń koordinataların tabıń:
$\lbrack\overline{BC} - 2\overline{CA},\overline{CB}\rbrack$.