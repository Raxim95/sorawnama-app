1. Линейные пространства. Линейные подпространства. Сумма и пересечение подпространств.
3. Ортогональное дополнение и ортогональная проекция.
5. Методы приведения квадратичной формы к каноническому форму.
7. Комплексные евклидовы пространства.
9. Линейные преобразования и их матрица.
11. Обратное преобразование.
13. Инвариантные подпространства. Собственные векторы и собственные значения.
15. Самосопряженные преобразования и их канонический вид.
17. Взаимозаменяемые преобразования.
19. Полиномиальные матрицы и диагональные нормальные формы.
++++
2. Евклидово пространство. Неравенство Коши-Буняковского. Процесс ортогонализации.
4. Линейные, билинейные, и квадратичные формы. Преобразование матрицы линейного вида при изменении базиса.
6. Положительно определенные квадратичные формы.
8. Квадратичные формы в комплексном пространстве и их канонические виды.
10. Ядро, образ линйеного преобразования.
12. Связь между матрицами линейных преобразовании в разных базисах.
14. Сопряженное преобразование для данного преобразования.
16. Унитарные преобразования и их собственные значения и канонический вид.
18. Нормальные преобразования и их канонический вид.
20. Подобные матрицы.