Доказать, что векторы \(\overrightarrow{a} = (1;\ \ 2;\ \ 1),\) \(\overrightarrow{b} = (1;\ \ 1;\ \  - 3)\) и \(\overrightarrow{c} = ( - 1;\ \ 2;\ \ 1)\) образуют базис пространства \(\mathbf{R}^{3},\) и найти координаты вектора \(\overrightarrow{d} = (0;\ \ 10;\ \  - 2)\) в этом базисе.
====
Доказать, что векторы \(\overrightarrow{a} = (1;\ \ 2;\ \ 1),\) \(\overrightarrow{b} = (1;\ \ 1;\ \ 3)\) и \(\overrightarrow{c} = ( - 1;\ \ 2;\ \ 1)\) образуют базис пространства \(\mathbf{R}^{3},\) и найти координаты вектора \(\overrightarrow{d} = (0;\ \ 10;\ \  - 2)\) в этом базисе.
====
Доказать, что векторы \(\overrightarrow{a} = (2;\ \ 1;\ \ 1),\) \(\overrightarrow{b} = (1;\ \ 2;\ \ 1)\) и \(\overrightarrow{c} = (2;\ \ 3;\ \  - 1)\) образуют базис пространства \(\mathbf{R}^{3},\) и найти координаты вектора \(\overrightarrow{d} = (2;\ \ 3;\ \  - 1)\) в этом базисе.
====
Доказать, что векторы \(\overrightarrow{a} = (2;\ \ 1;\ \  - 3),\) \(\overrightarrow{b} = ( - 1;\ \ 2;\ \ 4)\) и \(\overrightarrow{c} = (3;\ \  - 4;\ \ 2)\) образуют базис пространства \(\mathbf{R}^{3},\) и найти координаты вектора \(\overrightarrow{d} = ( - 4;\ \ 19;\ \ 3)\) в этом базисе.
====
Доказать, что векторы \(\overrightarrow{a} = (3;\ \ 4;\ \ 3),\) \(\overrightarrow{b} = ( - 2;\ \ 3;\ \ 1)\) и \(\overrightarrow{c} = (4;\ \  - 2;\ \ 3)\) образуют базис пространства \(\mathbf{R}^{3},\) и найти координаты вектора \(\overrightarrow{d} = ( - 17;\ \ 18;\ \  - 7)\) в этом базисе.
====
Доказать, что векторы \(\overrightarrow{a} = (3;\ \ 5;\ \ 4),\) \(\overrightarrow{b} = (4;\ \ 3;\ \ 2)\) и \(\overrightarrow{c} = ( - 1;\ \  - 4;\ \ 3)\) образуют базис пространства \(\mathbf{R}^{3},\) и найти координаты вектора \(\overrightarrow{d} = ( - 2;\ \  - 2;\ \ 5)\) в этом базисе.
====
Доказать, что векторы \(\overrightarrow{a} = (1;\ \ 0;\ \ 1),\) \(\overrightarrow{b} = (1;\ \ 1;\ \ 1)\) и \(\overrightarrow{c} = ( - 1;\ \ 2;\ \ 1)\) образуют базис пространства \(\mathbf{R}^{3},\) и найти координаты вектора \(\overrightarrow{d} = (0;\ \ 10;\ \ 3)\) в этом базисе.
====
Доказать, что векторы \(\overrightarrow{a} = (2;\ \ 1;1),\) \(\overrightarrow{b} = ( - 1;\ \ 2;\ \ 4)\) и \(\overrightarrow{c} = (3;\ \ 3;\ \ 2)\) образуют базис пространства \(\mathbf{R}^{3},\) и найти координаты вектора \(\overrightarrow{d} = ( - 4;\ \ 2;\ \ 4)\) в этом базисе.
====
Доказать, что векторы \(\overrightarrow{a} = (1;\ \ 2;\ \ 1),\) \(\overrightarrow{b} = (1;\ \ 1;\ \ 3)\) и \(\overrightarrow{c} = ( - 1;\ \ 2;\ \ 1)\) образуют базис пространства \(\mathbf{R}^{3},\) и найти координаты вектора \(\overrightarrow{d} = (0;\ \ 10;\ \  - 2)\) в этом базисе.
====
Доказать, что векторы \(\overrightarrow{a} = (2;\ \ 1;\ \ 1),\) \(\overrightarrow{b} = (1;\ \ 2;\ \ 1)\) и \(\overrightarrow{c} = (2;\ \ 1;\ \ 1)\) образуют базис пространства \(\mathbf{R}^{3},\) и найти координаты вектора \(\overrightarrow{d} = (1;\ \ 3;\ \ 1)\) в этом базисе.
====
Доказать, что векторы \(\overrightarrow{a} = (3;\ \ 1;\ \ 2),\) \(\overrightarrow{b} = (2;\ \  - 3;\ \ 1)\) и \(\overrightarrow{c} = (4;\ \  - 2;\ \ 3)\) образуют базис пространства \(\mathbf{R}^{3},\) и найти координаты вектора \(\overrightarrow{d} = ( - 7;\ \ 8;\ \ 7)\) в этом базисе.
====
Доказать, что векторы \(\overrightarrow{a} = (3;\ \ 1;\ \ 0),\) \(\overrightarrow{b} = (4;\ \ 3;\ \ 2)\) и \(\overrightarrow{c} = ( - 1;\ \  - 4;\ \ 3)\) образуют базис пространства \(\mathbf{R}^{3},\) и найти координаты вектора \(\overrightarrow{d} = ( - 1;\ \ 2;\ \ 5)\) в этом базисе.
++++
Известно, что оператор \emph{A} переводит базисные векторы \(\overrightarrow{i} = (1;\ \ 0;\ \ 0),\) \(\overrightarrow{j} = (0;\ \ 1;\ \ 0),\) \(\overrightarrow{k} = (0;\ \ 0;\ \ 1)\) линейного пространства \(\mathbf{R}^{3}\) в векторы \({\overline{a}}_{1} = (1;1;0),\) \({\overline{a}}_{2} = (3;\ \ 2;\ \ 1),\) \({\overline{a}}_{3} = (0;\ \ 1;\ \ 1).\) В базисе \(\overrightarrow{i},\overrightarrow{j},\overrightarrow{k}\) найти: 1)матрицу оператора \(A\) ; 2)образ вектора \(\overline{b} = (1;\ \  - 2;\ \  - 3).\)
====
Известно, что оператор \emph{A} переводит базисные векторы \(\overrightarrow{i} = (1;\ \ 0;\ \ 0),\) \(\overrightarrow{j} = (0;\ \ 1;\ \ 0),\) \(\overrightarrow{k} = (0;\ \ 0;\ \ 1)\) линейного пространства \(\mathbf{R}^{3}\) в векторы \({\overline{a}}_{1} = (0;\ \ 1;\ \ 1),\) \({\overline{a}}_{2} = (3;\ \ 1;\ \ 1),\) \({\overline{a}}_{3} = (3;0;1).\) В базисе \(\overrightarrow{i},\overrightarrow{j},\overrightarrow{k}\) найти: 1)матрицу оператора \(A\) ; 2)образ вектора \(\overline{b} = (1;\ \  - 1;\ \  - 1).\)
====
Известно, что оператор \emph{A} переводит базисные векторы \(\overrightarrow{i} = (1;\ \ 0;\ \ 0),\) \(\overrightarrow{j} = (0;\ \ 1;\ \ 0),\) \(\overrightarrow{k} = (0;\ \ 0;\ \ 1)\) линейного пространства \(\mathbf{R}^{3}\) в векторы \({\overline{a}}_{1} = (2;\ \ 1;1),\) \({\overline{a}}_{2} = (3;\ \ 2;\ \ 1),\) \({\overline{a}}_{3} = (3;\ \ 1;\ \ 1).\) В базисе \(\overrightarrow{i},\overrightarrow{j},\overrightarrow{k}\) найти: 1)матрицу оператора \(A\) ; 2)образ вектора \(\overline{b} = (1;\ \  - 2;\ \ 3).\)
====
Известно, что оператор \emph{A} переводит базисные векторы \(\overrightarrow{i} = (1;\ \ 0;\ \ 0),\) \(\overrightarrow{j} = (0;\ \ 1;\ \ 0),\) \(\overrightarrow{k} = (0;\ \ 0;\ \ 1)\) линейного пространства \(\mathbf{R}^{3}\) в векторы \({\overline{a}}_{1} = (1;\ \ 1;\ \ 1),{\overline{a}}_{2} = (3;\ \ 0;\ \ 1),\) \({\overline{a}}_{3} = (0;\ \ 2;\ \ 1).\)В базисе \(\overrightarrow{i},\overrightarrow{j},\overrightarrow{k}\) найти:1)матрицу оператора \(A\) ;2)образ вектора \(\overline{b} = (1;\ \ 2;\ \  - 2).\)
====
Известно, что оператор \emph{A} переводит базисные векторы \(\overrightarrow{i} = (1;\ \ 0;\ \ 0),\) \(\overrightarrow{j} = (0;\ \ 1;\ \ 0),\) \(\overrightarrow{k} = (0;\ \ 0;\ \ 1)\) линейного пространства \(\mathbf{R}^{3}\) в векторы \({\overline{a}}_{1} = (1;\ \ 0;\ \ 1),\) \({\overline{a}}_{2} = (0;\ \ 1;\ \ 1),\) \({\overline{a}}_{3} = (3;\ \ 1;\ \ 1).\) В базисе \(\overrightarrow{i},\overrightarrow{j},\overrightarrow{k}\) найти: 1)матрицу оператора \(A\) ; 2)образ вектора \(\overline{b} = (1;\ \  - 2;\ \  - 3).\)
====
Известно, что оператор \emph{A} переводит базисные векторы \(\overrightarrow{i} = (1;\ \ 0;\ \ 0),\) \(\overrightarrow{j} = (0;\ \ 1;\ \ 0),\) \(\overrightarrow{k} = (0;\ \ 0;\ \ 1)\) линейного пространства \(\mathbf{R}^{3}\) в векторы \({\overline{a}}_{1} = (1;\ \ 1;\ \ 0),\) \({\overline{a}}_{2} = (3;\ \ 2;\ \ 1),\) \({\overline{a}}_{3} = (1;2;\ \ 1).\) В базисе \(\overrightarrow{i},\overrightarrow{j},\overrightarrow{k}\) найти: 1)матрицу оператора \(A\) ; 2)образ вектора \(\overline{b} = (1;\ \ 1;\ \  - 2).\)
====
Известно, что оператор \emph{A} переводит базисные векторы \(\overrightarrow{i} = (1;\ \ 0;\ \ 0),\) \(\overrightarrow{j} = (0;\ \ 1;\ \ 0),\) \(\overrightarrow{k} = (0;\ \ 0;\ \ 1)\) линейного пространства \(\mathbf{R}^{3}\) в векторы \({\overline{a}}_{1} = (1;\ \ 1;\ \ 0),\) \({\overline{a}}_{2} = (3;\ \ 2;\ \ 1),\) \({\overline{a}}_{3} = (3;\ \ 1;\ \ 1).\) В базисе \(\overrightarrow{i},\overrightarrow{j},\overrightarrow{k}\) найти: 1)матрицу оператора \(A\) ; 2)образ вектора \(\overline{b} = (1;\ \ 2;\ \ 3).\)
====
Известно, что оператор \emph{A} переводит базисные векторы \(\overrightarrow{i} = (1;\ \ 0;\ \ 0),\) \(\overrightarrow{j} = (0;\ \ 1;\ \ 0),\) \(\overrightarrow{k} = (0;\ \ 0;\ \ 1)\) линейного пространства \(\mathbf{R}^{3}\) в векторы \({\overline{a}}_{1} = (1;\ \ 1;\ \ 1),{\overline{a}}_{2} = (3;\ \ 0;\ \ 1),\) \({\overline{a}}_{3} = (3;\ \ 1;\ \  - 1).\)В базисе \(\overrightarrow{i},\overrightarrow{j},\overrightarrow{k}\) найти:1)матрицу оператора \(A\) ;2)образ вектора \(\overline{b} = (1;\ \ 2;\ \ 1).\)
====
Известно, что оператор \emph{A} переводит базисные векторы \(\overrightarrow{i} = (1;\ \ 0;\ \ 0),\) \(\overrightarrow{j} = (0;\ \ 1;\ \ 0),\) \(\overrightarrow{k} = (0;\ \ 0;\ \ 1)\) линейного пространства \(\mathbf{R}^{3}\) в векторы \({\overline{a}}_{1} = (1;\ \ 0;\ \ 1),\) \({\overline{a}}_{2} = (3;\ \ 2;\ \ 1),\) \({\overline{a}}_{3} = (3;\ \ 1;\ \ 1).\) В базисе \(\overrightarrow{i},\overrightarrow{j},\overrightarrow{k}\) найти: 1)матрицу оператора \(A\) ; 2)образ вектора \(\overline{b} = (1;\ \  - 2;\ \ 3).\)
====
Известно, что оператор \emph{A} переводит базисные векторы \(\overrightarrow{i} = (1;\ \ 0;\ \ 0),\) \(\overrightarrow{j} = (0;\ \ 1;\ \ 0),\) \(\overrightarrow{k} = (0;\ \ 0;\ \ 1)\) линейного пространства \(\mathbf{R}^{3}\) в векторы \({\overline{a}}_{1} = (1;\ \ 0;\ \ 1),\) \({\overline{a}}_{2} = (0;\ \ 2;\ \ 1),\) \({\overline{a}}_{3} = (3;\ \ 1;\ \ 1).\) В базисе \(\overrightarrow{i},\overrightarrow{j},\overrightarrow{k}\) найти: 1)матрицу оператора \(A\) ; 2)образ вектора \(\overline{b} = (1;\ \ 2;\ \ 3).\)
====
Известно, что оператор \emph{A} переводит базисные векторы \(\overrightarrow{i} = (1;\ \ 0;\ \ 0),\) \(\overrightarrow{j} = (0;\ \ 1;\ \ 0),\) \(\overrightarrow{k} = (0;\ \ 0;\ \ 1)\) линейного пространства \(\mathbf{R}^{3}\) в векторы \({\overline{a}}_{1} = (0;\ \ 1;\ \ 1),\) \({\overline{a}}_{2} = (3;\ \ 1;\ \ 1),\) \({\overline{a}}_{3} = (3;\ \ 1;\ \ 1).\) В базисе \(\overrightarrow{i},\overrightarrow{j},\overrightarrow{k}\) найти: 1)матрицу оператора \(A\) ; 2)образ вектора \(\overline{b} = (1;\ \ 1;\ \ 1).\)
====
Известно, что оператор \emph{A} переводит базисные векторы \(\overrightarrow{i} = (1;\ \ 0;\ \ 0),\) \(\overrightarrow{j} = (0;\ \ 1;\ \ 0),\) \(\overrightarrow{k} = (0;\ \ 0;\ \ 1)\) линейного пространства \(\mathbf{R}^{3}\) в векторы \({\overline{a}}_{1} = (1;\ \ 1;\ \ 1),\) \({\overline{a}}_{2} = (3;\ \ 2;\ \ 1),\) \({\overline{a}}_{3} = (0;\ \ 1;\ \ 1).\) В базисе \(\overrightarrow{i},\overrightarrow{j},\overrightarrow{k}\) найти: 1)матрицу оператора \(A\) ; 2)образ вектора \(\overline{b} = (1;\ \  - 2;\ \ 3).\)
++++
В пространстве \(\mathbb{C}^{3}\) со скалярным произведением \(\left\langle x,y \right\rangle = \sum_{k = 1}^{3}{x_{k}\overline{y_{k}}}\), найдите сопряженный оператор \(A^{*}\) для заданного оператора \(A\). Является ли \(A\)самосопряженным? \(Ax = \left( ix_{1} + x_{3},x_{2} + ix_{1},x_{1} + ix_{3} \right)\);
====
В пространстве \(\mathbb{C}^{3}\) со скалярным произведением \(\left\langle x,y \right\rangle = \sum_{k = 1}^{3}{x_{k}\overline{y_{k}}}\), найдите сопряженный оператор \(A^{*}\) для заданного оператора \(A\). Является ли \(A\)самосопряженным? \(Ax = \left( ix_{1} + x_{2},x_{1} + ix_{2},x_{2} + ix_{3} \right)\);
====
В пространстве \(\mathbb{C}^{3}\) со скалярным произведением \(\left\langle x,y \right\rangle = \sum_{k = 1}^{3}{x_{k}\overline{y_{k}}}\), найдите сопряженный оператор \(A^{*}\) для заданного оператора \(A\). Является ли \(A\)самосопряженным? \(Ax = \left( ix_{1} + 2ix_{3},x_{3},x_{1} - 2ix_{3} \right)\);
====
В пространстве \(\mathbb{C}^{3}\) со скалярным произведением \(\left\langle x,y \right\rangle = \sum_{k = 1}^{3}{x_{k}\overline{y_{k}}}\), найдите сопряженный оператор \(A^{*}\) для заданного оператора \(A\). Является ли \(A\)самосопряженным? \(Ax = \left( x_{1} + ix_{3},x_{3} + 2ix_{2},ix_{2} - 2ix_{3} \right)\);
====
В пространстве \(\mathbb{C}^{3}\) со скалярным произведением \(\left\langle x,y \right\rangle = \sum_{k = 1}^{3}{x_{k}\overline{y_{k}}}\), найдите сопряженный оператор \(A^{*}\) для заданного оператора \(A\). Является ли \(A\)самосопряженным?\(Ax = \left( x_{1} + 2ix_{3},2ix_{1} + ix_{2},x_{1} + ix_{3} \right)\);
====
В пространстве \(\mathbb{C}^{3}\) со скалярным произведением \(\left\langle x,y \right\rangle = \sum_{k = 1}^{3}{x_{k}\overline{y_{k}}}\), найдите сопряженный оператор \(A^{*}\) для заданного оператора \(A\). Является ли \(A\)самосопряженным? \(Ax = \left( ix_{1} + x_{3},ix_{3} - ix_{2},x_{1} - ix_{3} \right)\);
====
В пространстве \(\mathbb{C}^{3}\) со скалярным произведением \(\left\langle x,y \right\rangle = \sum_{k = 1}^{3}{x_{k}\overline{y_{k}}}\), найдите сопряженный оператор \(A^{*}\) для заданного оператора \(A\). Является ли \(A\)самосопряженным? \(Ax = \left( x_{1} + 2ix_{3},ix_{2} - x_{3},x_{2} - ix_{3} \right)\);
====
В пространстве \(\mathbb{C}^{3}\) со скалярным произведением \(\left\langle x,y \right\rangle = \sum_{k = 1}^{3}{x_{k}\overline{y_{k}}}\), найдите сопряженный оператор \(A^{*}\) для заданного оператора \(A\). Является ли \(A\)самосопряженным? \(Ax = \left( 3ix_{1} + x_{2},x_{1} + 2ix_{2},ix_{2} - x_{3} \right)\);
====
В пространстве \(\mathbb{C}^{3}\) со скалярным произведением \(\left\langle x,y \right\rangle = \sum_{k = 1}^{3}{x_{k}\overline{y_{k}}}\), найдите сопряженный оператор \(A^{*}\) для заданного оператора \(A\). Является ли \(A\)самосопряженным? \(Ax = \left( x_{2} + ix_{3},x_{1} - ix_{2},x_{1} + ix_{2} + x_{3} \right)\)
====
В пространстве \(\mathbb{C}^{3}\) со скалярным произведением \(\left\langle x,y \right\rangle = \sum_{k = 1}^{3}{x_{k}\overline{y_{k}}}\), найдите сопряженный оператор \(A^{*}\) для заданного оператора \(A\). Является ли \(A\)самосопряженным? \(Ax = \left( 2ix_{1} + ix_{3},x_{1} + x_{2} + ix_{3},ix_{3} \right)\);
====
В пространстве \(\mathbb{C}^{3}\) со скалярным произведением \(\left\langle x,y \right\rangle = \sum_{k = 1}^{3}{x_{k}\overline{y_{k}}}\), найдите сопряженный оператор \(A^{*}\) для заданного оператора \(A\). Является ли \(A\)самосопряженным?\(Ax = \left( x_{1} + 2ix_{2},x_{3} - ix_{2},x_{1} - ix_{2} - 2ix_{3} \right)\);
====
В пространстве \(\mathbb{C}^{3}\) со скалярным произведением \(\left\langle x,y \right\rangle = \sum_{k = 1}^{3}{x_{k}\overline{y_{k}}}\), найдите сопряженный оператор \(A^{*}\) для заданного оператора \(A\). Является ли \(A\)самосопряженным?\(Ax = \left( x_{1} - 2ix_{2},x_{3} + 2ix_{2},ix_{2} + 2ix_{3} \right)\);
++++
С помощью процесса ортогонализации Грамма- Шмидта ортонормировать следующие системы векторов, используя стандартное скалярное произведение:
====
\(a_{1} = (1;1;1)\), \(a_{2} = (1;2;3)\), \(a_{3} = (1;1;2)\);
====
С помощью процесса ортогонализации Грамма- Шмидта ортонормировать следующие системы векторов, используя стандартное скалярное произведение:
====
\(a_{1} = (2; - 1;3)\), \(a_{2} = (3;2; - 5)\), \(a_{3} = (1; - 1;1)\);
====
С помощью процесса ортогонализации Грамма- Шмидта ортонормировать следующие системы векторов, используя стандартное скалярное произведение:
====
\(a_{1} = (1;2;1)\), \(a_{2} = (2;3;3)\), \(a_{3} = (3;8;2)\);
====
С помощью процесса ортогонализации Грамма- Шмидта ортонормировать следующие системы векторов, используя стандартное скалярное произведение:
====
\(a_{1} = ( - 1;3;7)\),\(a_{2} = (0;2; - 1)\), \(a_{3} = (1; - 2; - 8)\);
====
С помощью процесса ортогонализации Грамма- Шмидта ортонормировать следующие системы векторов, используя стандартное скалярное произведение:
====
\(a_{1} = (2;0;1)\), \(a_{2} = ( - 1;2;3)\), \(a_{3} = ( - 1;1;1)\);
====
С помощью процесса ортогонализации Грамма- Шмидта ортонормировать следующие системы векторов, используя стандартное скалярное произведение:
====
\(a_{1} = (0;1; - 2)\),\(a_{2} = (1; - 1;1)\), \(a_{3} = ( - 2;0;3)\);
====
С помощью процесса ортогонализации Грамма- Шмидта ортонормировать следующие системы векторов, используя стандартное скалярное произведение:
====
\(a_{1} = (1;0;2)\), \(a_{2} = (3; - 1;4)\), \(a_{3} = (2; - 2;1)\);
====
С помощью процесса ортогонализации Грамма- Шмидта ортонормировать следующие системы векторов, используя стандартное скалярное произведение:
====
\(a_{1} = ( - 2;3;1)\),\(a_{2} = (0;2;1)\), \(a_{3} = (1;2;1)\);
====
С помощью процесса ортогонализации Грамма- Шмидта ортонормировать следующие системы векторов, используя стандартное скалярное произведение:
====
\(a_{1} = ( - 3;0;1)\), \(a_{2} = (0;2;3)\), \(a_{3} = ( - 1; - 1; - 1)\);
====
С помощью процесса ортогонализации Грамма- Шмидта ортонормировать следующие системы векторов, используя стандартное скалярное произведение:
====
\(a_{1} = (3;1; - 1)\), \(a_{2} = ( - 2;0;1)\), \(a_{3} = (2;7;3)\);
====
С помощью процесса ортогонализации Грамма- Шмидта ортонормировать следующие системы векторов, используя стандартное скалярное произведение:
====
\(a_{1} = (2;4;3)\), \(a_{2} = (3; - 1;4)\), \(a_{3} = (1;5; - 1)\);
====
С помощью процесса ортогонализации Грамма- Шмидта ортонормировать следующие системы векторов, используя стандартное скалярное произведение:
====
\(a_{1} = (4;1;3)\), \(a_{2} = (0;7; - 2)\), \(a_{3} = (4;8;0)\);
++++
Найти собственные значения и собственные векторы оператора \emph{А}, заданного в некотором базисе пространства \(V^{3}\) матрицей \(A = \begin{bmatrix}
0 & - 1 & 1 \\
 - 1 & 0 & 1 \\
1 & 1 & 0
\end{bmatrix}.\)
====
Найти собственные значения и собственные векторы оператора \emph{А}, заданного в некотором базисе пространства \(V^{3}\) матрицей \(A = \begin{bmatrix}
2 & 1 & 0 \\
1 & 2 & 0 \\
0 & 0 & - 5
\end{bmatrix};\)
====
Найти собственные значения и собственные векторы оператора \emph{А}, заданного в некотором базисе пространства \(V^{3}\) матрицей \(A = \begin{bmatrix}
0 & - 2 & 0 \\
 - 2 & 6 & - 2 \\
0 & - 2 & 5
\end{bmatrix};\)
====
Найти собственные значения и собственные векторы оператора \emph{А}, заданного в некотором базисе пространства \(V^{3}\) матрицей \(A = \begin{bmatrix}
0 & - 1 & 1 \\
 - 1 & 0 & 1 \\
1 & 1 & 0
\end{bmatrix};\)
====
Найти собственные значения и собственные векторы оператора \emph{А}, заданного в некотором базисе пространства \(V^{3}\) матрицей \(A = \begin{bmatrix}
 - 1 & 1 & 0 \\
 - 4 & 3 & 0 \\
 - 2 & 1 & 1
\end{bmatrix};\)
====
Найти собственные значения и собственные векторы оператора \emph{А}, заданного в некотором базисе пространства \(V^{3}\) матрицей \(A = \begin{bmatrix}
 - 1 & - 2 & 0 \\
0 & - 2 & 0 \\
2 & 2 & 1
\end{bmatrix}.\)
====
Найти собственные значения и собственные векторы оператора \emph{А}, заданного в некотором базисе пространства \(V^{3}\) матрицей \(A = \begin{pmatrix}
2 & - 1 & 2 \\
1 & 0 & 2 \\
 - 2 & 1 & - 1
\end{pmatrix}\)
====
Найти собственные значения и собственные векторы оператора \emph{А}, заданного в некотором базисе пространства \(V^{3}\) матрицей \(A = \begin{pmatrix}
1 & - 2 & - 1 \\
 - 1 & 1 & 1 \\
1 & 0 & - 1
\end{pmatrix}\)
====
Найти собственные значения и собственные векторы оператора \emph{А}, заданного в некотором базисе пространства \(V^{3}\) матрицей \(A = \begin{pmatrix}
2 & 1 & 0 \\
1 & 3 & - 1 \\
 - 1 & 2 & 3
\end{pmatrix}\).
====
Найти собственные значения и собственные векторы оператора \emph{А}, заданного в некотором базисе пространства \(V^{3}\) матрицей \(A = \begin{pmatrix}
2 & - 1 & 2 \\
5 & - 3 & 3 \\
 - 1 & 0 & - 2
\end{pmatrix}\).
====
Найти собственные значения и собственные векторы оператора \emph{А}, заданного в некотором базисе пространства \(V^{3}\) матрицей \(A = \begin{pmatrix}
1 & - 1 & 1 \\
1 & 1 & - 1 \\
2 & - 1 & 0
\end{pmatrix}\).
====
Найти собственные значения и собственные векторы оператора \emph{А}, заданного в некотором базисе пространства \(V^{3}\) матрицей \(A = \begin{pmatrix}
0 & 1 & 2 \\
 - 1 & 0 & - 2 \\
 - 2 & 2 & 0
\end{pmatrix}\).
++++
Приведите квадратичные формы \(G_{1}\) и \(G_{2}\) к каноническому виду. \(G_{1} = x_{1}^{2} + x_{2}^{2} + 3x_{3}^{2} + 4x_{1}x_{2} + 2x_{1}x_{3} + 2x_{2}x_{3}\), \(G_{2} = x_{1}x_{2} + x_{1}x_{3} + x_{2}x_{3}\)
====
Приведите квадратичные формы \(G_{1}\) и \(G_{2}\) к каноническому виду. \(G_{1} = x_{1}^{2} - 2x_{2}^{2} + x_{3}^{2} + 2x_{1}x_{2} + 4x_{1}x_{3} + 2x_{2}x_{3}\), \(G_{2} = 2x_{1}x_{3} - 4x_{2}x_{3}\)
====
Приведите квадратичные формы \(G_{1}\) и \(G_{2}\) к каноническому виду. \(G_{1} = x_{1}^{2} - 3x_{3}^{2} - 2x_{1}x_{2} - 2x_{1}x_{3} - 6x_{2}x_{3}\), \(G_{2} = 2x_{1}x_{2} - x_{1}x_{3} + 2x_{2}x_{3}\)
====
Приведите квадратичные формы \(G_{1}\) и \(G_{2}\) к каноническому виду. \(G_{1} = x_{1}^{2} + 5x_{2}^{2} - 4x_{3}^{2} + 2x_{1}x_{3} - 4x_{1}x_{2}\), \(G_{2} = - 4x_{1}x_{2} + 2x_{1}x_{3}\)
====
Приведите квадратичные формы \(G_{1}\) и \(G_{2}\) к каноническому виду. \(G_{1} = 4x_{1}^{2} + x_{2}^{2} + x_{3}^{2} - 4x_{1}x_{2} + 4x_{1}x_{3} - 3x_{2}x_{3}\), \(G_{2} = x_{1}x_{2} + 6x_{1}x_{3} - 4x_{2}x_{3}\)
====
Приведите квадратичные формы \(G_{1}\) и \(G_{2}\) к каноническому виду. \(G_{1} = 2x_{1}^{2} + x_{2}^{2} + x_{3}^{2} + 4x_{1}x_{2} - 2x_{1}x_{3}\), \(G_{2} = x_{1}x_{2} + x_{1}x_{3} + 4x_{2}x_{3}\)
====
Приведите квадратичные формы \(G_{1}\) и \(G_{2}\) к каноническому виду. \(G_{1} = 2x_{1}^{2} + 6x_{2}^{2} - 4x_{3}^{2} - 2x_{1}x_{3} + 4x_{1}x_{2} - 8x_{2}x_{3}\), \(G_{2} = x_{2}x_{3} - 2x_{1}x_{3}\)
====
Приведите квадратичные формы \(G_{1}\) и \(G_{2}\) к каноническому виду. \(G_{1} = 2x_{1}^{2} + 3x_{2}^{2} + 4x_{3}^{2} - 2x_{1}x_{2} + 4x_{1}x_{3} - 3x_{2}x_{3}\), \(G_{2} = x_{1}x_{3} - 2x_{2}x_{3}\)
====
Приведите квадратичные формы \(G_{1}\) и \(G_{2}\) к каноническому виду. \(G_{1} = 3x_{1}^{2} - 2x_{2}^{2} + 2x_{3}^{2} + 4x_{1}x_{2} - 3x_{1}x_{3} - x_{2}x_{3}\), \(G_{2} = 2x_{1}x_{3} + 4x_{1}x_{2} - 2x_{2}x_{3}\)
====
Приведите квадратичные формы \(G_{1}\) и \(G_{2}\) к каноническому виду. \(G_{1} = 5x_{1}^{2} + 6x_{2}^{2} - 3x_{3}^{2} + 4x_{1}x_{2} - 2x_{2}x_{3}\), \(G_{2} = 6x_{2}x_{3} - x_{1}x_{2}\)
====
Приведите квадратичные формы \(G_{1}\) и \(G_{2}\) к каноническому виду. \(G_{1} = 3x_{1}^{2} - 2x_{2}^{2} + 2x_{1}x_{3} - 4x_{2}x_{3}\), \(G_{2} = x_{1}x_{2} + x_{2}x_{3}\)
====
Приведите квадратичные формы \(G_{1}\) и \(G_{2}\) к каноническому виду. \(G_{1} = x_{1}^{2} - 2x_{3}^{2} + 2x_{1}x_{3} - 6x_{1}x_{2}\), \(G_{2} = 6x_{2}x_{3} - 4x_{1}x_{2} + x_{1}x_{3}\)
++++
В базисе \((e_{1},\ \ e_{2},\ \ e_{3})\) пространства \(V^{3}\) оператор \emph{A} имеет матрицу \(A = \begin{bmatrix}
1 & 2 & 3 \\
0 & 1 & 2 \\
3 & 1 & 2
\end{bmatrix}\ \ .\) Найти матрицу \emph{B} этого же оператора в базисе \(({e'}_{1},\ \ {e'}_{2},\ \ {e'}_{3}),\) где \({e'}_{1} = e_{1} + 2e_{2},\) \({e'}_{2} = e_{1} - e_{3},\) \({e'}_{3} = e_{1} + e_{2} + e_{3}.\)
====
В базисе \((e_{1},\ \ e_{2},\ \ e_{3})\) пространства \(V^{3}\) оператор \emph{A} имеет матрицу \(A = \begin{bmatrix}
2 & 0 & - 1 \\
3 & 2 & 0 \\
 - 1 & 4 & 3
\end{bmatrix}\ \ .\) Найти матрицу \emph{B} этого же оператора в базисе \(({e'}_{1},\ \ {e'}_{2},\ \ {e'}_{3}),\) где \({e'}_{1} = e_{1} - e_{3},\) \({e'}_{2} = e_{2} + e_{3},\) \({e'}_{3} = e_{3}.\)
====
В базисе \((e_{1},\ \ e_{2},\ \ e_{3})\) пространства \(V^{3}\) оператор \emph{A} имеет матрицу \(A = \begin{bmatrix}
5 & - 2 & 1 \\
 - 1 & 0 & 4 \\
3 & 1 & 2
\end{bmatrix}\ \ .\) Найти матрицу \emph{B} этого же оператора в базисе \(({e'}_{1},\ \ {e'}_{2},\ \ {e'}_{3}),\) где \({e'}_{1} = 2e_{1} + 3e_{3},\) \({e'}_{2} = - e_{2},\) \({e'}_{3} = e_{1} + e_{2} + e_{3}.\)
====
В базисе \((e_{1},\ \ e_{2},\ \ e_{3})\) пространства \(V^{3}\) оператор \emph{A} имеет матрицу \(A = \begin{bmatrix}
1 & 3 & - 1 \\
2 & 0 & 4 \\
1 & 1 & 1
\end{bmatrix}\ \ .\) Найти матрицу \emph{B} этого же оператора в базисе \(({e'}_{1},\ \ {e'}_{2},\ \ {e'}_{3}),\) где \({e'}_{1} = 2e_{1} + e_{2}\), \({e'}_{2} = - e_{1} + 2e_{2} + 3e_{3}\),\({e'}_{3} = - e_{1} + e_{2} + e_{3}\)
====
В базисе \((e_{1},\ \ e_{2},\ \ e_{3})\) пространства \(V^{3}\) оператор \emph{A} имеет матрицу \(A = \begin{bmatrix}
0 & 1 & - 3 \\
2 & 4 & 1 \\
0 & 3 & - 3
\end{bmatrix}\ \ .\) Найти матрицу \emph{B} этого же оператора в базисе \(({e'}_{1},\ \ {e'}_{2},\ \ {e'}_{3}),\) где \({e'}_{1} = 2e_{1} + e_{2}\), \({e'}_{2} = - e_{1} + 2e_{2} + 3e_{3}\),\({e'}_{3} = - e_{1} + e_{2} + e_{3}\)
====
В базисе \((e_{1},\ \ e_{2},\ \ e_{3})\) пространства \(V^{3}\) оператор \emph{A} имеет матрицу \(A = \begin{bmatrix}
1 & - 1 & 2 \\
0 & 3 & - 1 \\
4 & 2 & 2
\end{bmatrix}\ \ .\) Найти матрицу \emph{B} этого же оператора в базисе \(({e'}_{1},\ \ {e'}_{2},\ \ {e'}_{3}),\) где \({e'}_{1} = e_{1} + 2e_{2},\) \({e'}_{2} = e_{1} - e_{3},\) \({e'}_{3} = e_{1} + e_{2} + e_{3}.\)
====
В базисе \((e_{1},\ \ e_{2},\ \ e_{3})\) пространства \(V^{3}\) оператор \emph{A} имеет матрицу \(A = \begin{bmatrix}
 - 1 & 2 & 1 \\
0 & 1 & - 4 \\
5 & - 1 & 2
\end{bmatrix}\ \ .\) Найти матрицу \emph{B} этого же оператора в базисе \(({e'}_{1},\ \ {e'}_{2},\ \ {e'}_{3}),\) где \({e'}_{1} = e_{1} - e_{3},\) \({e'}_{2} = e_{2} + e_{3},\) \({e'}_{3} = e_{3}.\)
====
В базисе \((e_{1},\ \ e_{2},\ \ e_{3})\) пространства \(V^{3}\) оператор \emph{A} имеет матрицу \(A = \begin{bmatrix}
0 & 1 & - 2 \\
3 & 5 & 1 \\
 - 1 & 2 & 0
\end{bmatrix}\ \ .\) Найти матрицу \emph{B} этого же оператора в базисе \(({e'}_{1},\ \ {e'}_{2},\ \ {e'}_{3}),\) где \({e'}_{1} = e_{1} + 2e_{2},\) \({e'}_{2} = e_{1} - e_{3},\) \({e'}_{3} = e_{1} + e_{2} + e_{3}.\)
====
В базисе \((e_{1},\ \ e_{2},\ \ e_{3})\) пространства \(V^{3}\) оператор \emph{A} имеет матрицу \(A = \begin{bmatrix}
3 & 2 & - 1 \\
4 & 0 & 2 \\
 - 1 & 2 & - 1
\end{bmatrix}\ \ .\) Найти матрицу \emph{B} этого же оператора в базисе \(({e'}_{1},\ \ {e'}_{2},\ \ {e'}_{3}),\) где \({e'}_{1} = 2e_{1} + e_{2}\), \({e'}_{2} = - e_{1} + 2e_{2} + 3e_{3}\),\({e'}_{3} = - e_{1} + e_{2} + e_{3}\)
====
В базисе \((e_{1},\ \ e_{2},\ \ e_{3})\) пространства \(V^{3}\) оператор \emph{A} имеет матрицу \(A = \begin{bmatrix}
 - 3 & 1 & 4 \\
0 & 3 & 2 \\
 - 5 & - 1 & 2
\end{bmatrix}\ \ .\) Найти матрицу \emph{B} этого же оператора в базисе \(({e'}_{1},\ \ {e'}_{2},\ \ {e'}_{3}),\) где \({e'}_{1} = 2e_{1} + e_{2}\), \({e'}_{2} = - e_{1} + 2e_{2} + 3e_{3}\),\({e'}_{3} = - e_{1} + e_{2} + e_{3}\)
====
В базисе \((e_{1},\ \ e_{2},\ \ e_{3})\) пространства \(V^{3}\) оператор \emph{A} имеет матрицу \(A = \begin{bmatrix}
 - 1 & 2 & 4 \\
 - 4 & 2 & 0 \\
3 & 3 & - 3
\end{bmatrix}\ \ .\) Найти матрицу \emph{B} этого же оператора в базисе \(({e'}_{1},\ \ {e'}_{2},\ \ {e'}_{3}),\) где \({e'}_{1} = 2e_{1} + e_{2}\), \({e'}_{2} = - e_{1} + 2e_{2} + 3e_{3}\),\({e'}_{3} = - e_{1} + e_{2} + e_{3}\)
====
В базисе \((e_{1},\ \ e_{2},\ \ e_{3})\) пространства \(V^{3}\) оператор \emph{A} имеет матрицу \(A = \begin{bmatrix}
4 & 0 & 1 \\
 - 2 & - 2 & 3 \\
0 & 2 & - 1
\end{bmatrix}\ \ .\) Найти матрицу \emph{B} этого же оператора в базисе \(({e'}_{1},\ \ {e'}_{2},\ \ {e'}_{3}),\) где \({e'}_{1} = e_{1} + 2e_{2},\) \({e'}_{2} = e_{1} - e_{3},\) \({e'}_{3} = e_{1} + e_{2} + e_{3}.\)
++++
Даны векторы \(e_{1},e_{2},e_{3}\), \(a_{1},a_{2},a_{3}\) линейного пространства \(R^{3}\). Найдите матрицу перехода от базиса \(e_{1},e_{2},e_{3}\) к базису \(a_{1},a_{2},a_{3}\).
\(e_{1} = (2,1, - 3)\),\(e_{2} = (3,2, - 5)\),\(e_{3} = (1, - 1,1)\) и \(a_{1} = (0,1, - 2)\),\(a_{2} = ( - 2,0,3)\),\(a_{3} = (1, - 1,1)\)
====
Даны векторы \(e_{1},e_{2},e_{3}\), \(a_{1},a_{2},a_{3}\) линейного пространства \(R^{3}\). Найдите матрицу перехода от базиса \(e_{1},e_{2},e_{3}\) к базису \(a_{1},a_{2},a_{3}\).
\(e_{1} = (2,0,1)\),\(e_{2} = ( - 1,2,3)\),\(e_{3} = ( - 1,1,1)\) и \(a_{1} = (1,0,2)\),\(a_{2} = (3, - 1,4)\),\(a_{3} = (2, - 2,1)\)
====
Даны векторы \(e_{1},e_{2},e_{3}\), \(a_{1},a_{2},a_{3}\) линейного пространства \(R^{3}\). Найдите матрицу перехода от базиса \(e_{1},e_{2},e_{3}\) к базису \(a_{1},a_{2},a_{3}\).
\(e_{1} = (3,5,8)\),\(e_{2} = (5,14,13)\),\(e_{3} = (1,9,2)\) и \(a_{1} = ( - 2,3,1)\),\(a_{2} = (0,2,1)\),\(a_{3} = (1,2,1)\)
====
Даны векторы \(e_{1},e_{2},e_{3}\), \(a_{1},a_{2},a_{3}\) линейного пространства \(R^{3}\). Найдите матрицу перехода от базиса \(e_{1},e_{2},e_{3}\) к базису \(a_{1},a_{2},a_{3}\).
\(e_{1} = (2,0,1)\),\(e_{2} = ( - 1,2,3)\),\(e_{3} = ( - 1,1,1)\) и \(a_{1} = ( - 3,0,1)\),\(a_{2} = (0,2,3)\),\(a_{3} = ( - 1, - 1, - 1)\)
====
Даны векторы \(e_{1},e_{2},e_{3}\), \(a_{1},a_{2},a_{3}\) линейного пространства \(R^{3}\). Найдите матрицу перехода от базиса \(e_{1},e_{2},e_{3}\) к базису \(a_{1},a_{2},a_{3}\).
\(e_{1} = (0,1, - 2)\),\(e_{2} = ( - 2,0,3)\),\(e_{3} = (1, - 1,1)\) и \(a_{1} = (3,1, - 1)\),\(a_{2} = ( - 2,0,1)\),\(a_{3} = (2,7,3)\)
====
Даны векторы \(e_{1},e_{2},e_{3}\), \(a_{1},a_{2},a_{3}\) линейного пространства \(R^{3}\). Найдите матрицу перехода от базиса \(e_{1},e_{2},e_{3}\) к базису \(a_{1},a_{2},a_{3}\).
\(e_{1} = (1,0,2)\),\(e_{2} = (3, - 1,4)\),\(e_{3} = (2, - 2,1)\) и \(a_{1} = (4,0,5)\),\(a_{2} = ( - 2,1,3)\),\(a_{3} = ( - 5,1, - 1)\)
====
Даны векторы \(e_{1},e_{2},e_{3}\), \(a_{1},a_{2},a_{3}\) линейного пространства \(R^{3}\). Найдите матрицу перехода от базиса \(e_{1},e_{2},e_{3}\) к базису \(a_{1},a_{2},a_{3}\).
\(e_{1} = ( - 2,3,1)\),\(e_{2} = (0,2,1)\),\(e_{3} = (1,2,1)\) и \(a_{1} = ( - 1,3,7)\),\(a_{2} = (0,2, - 1)\),\(a_{3} = (1, - 2, - 8)\)
====
Даны векторы \(e_{1},e_{2},e_{3}\), \(a_{1},a_{2},a_{3}\) линейного пространства \(R^{3}\). Найдите матрицу перехода от базиса \(e_{1},e_{2},e_{3}\) к базису \(a_{1},a_{2},a_{3}\).
\(e_{1} = ( - 3,0,1)\),\(e_{2} = (0,2,3)\),\(e_{3} = ( - 1, - 1, - 1)\) и \(a_{1} = (1,1,1)\),\(a_{2} = (1,1,2)\),\(a_{3} = (1,2,3)\)
====
Даны векторы \(e_{1},e_{2},e_{3}\), \(a_{1},a_{2},a_{3}\) линейного пространства \(R^{3}\). Найдите матрицу перехода от базиса \(e_{1},e_{2},e_{3}\) к базису \(a_{1},a_{2},a_{3}\).
\(e_{1} = (3,1, - 1)\),\(e_{2} = ( - 2,0,1)\),\(e_{3} = (2,7,3)\) и \(a_{1} = (2,1, - 3)\),\(a_{2} = (3,2, - 5)\),\(a_{3} = (1, - 1,1)\)
====
10.Даны векторы \(e_{1},e_{2},e_{3}\), \(a_{1},a_{2},a_{3}\) линейного пространства \(R^{3}\). Найдите матрицу перехода от базиса \(e_{1},e_{2},e_{3}\) к базису \(a_{1},a_{2},a_{3}\).
\(e_{1} = (4,0,5)\),\(e_{2} = ( - 2,1,3)\),\(e_{3} = ( - 5,1, - 1)\) и \(a_{1} = (1,2,1)\),\(a_{2} = (2,3,3)\),\(a_{3} = (3,8,2)\)
====
11.Даны векторы \(e_{1},e_{2},e_{3}\), \(a_{1},a_{2},a_{3}\) линейного пространства \(R^{3}\). Найдите матрицу перехода от базиса \(e_{1},e_{2},e_{3}\) к базису \(a_{1},a_{2},a_{3}\).
\(e_{1} = ( - 1,3,7)\),\(e_{2} = (0,2, - 1)\),\(e_{3} = (1, - 2, - 8)\) и \(a_{1} = (0,3, - 2)\),\(a_{2} = (1, - 1, - 8)\),\(a_{3} = ( - 1,2,7)\)
====
12.Даны векторы \(e_{1},e_{2},e_{3}\), \(a_{1},a_{2},a_{3}\) линейного пространства \(R^{3}\). Найдите матрицу перехода от базиса \(e_{1},e_{2},e_{3}\) к базису \(a_{1},a_{2},a_{3}\).
\(e_{1} = (1,1,1)\),\(e_{2} = (1,1,2)\),\(e_{3} = (1,2,3)\) и \(a_{1} = (2,0,1)\),\(a_{2} = ( - 1,2,3)\),\(a_{3} = ( - 1,1,1)\)
++++
Найти жорданову нормальную форму матрицы \(A = \begin{pmatrix}
 - 1 & 1 & - 2 \\
3 & - 3 & 6 \\
2 & - 2 & 4
\end{pmatrix}\).
====
Найти жорданову нормальную форму матрицы \(A = \begin{pmatrix}
2 & - 1 & - 1 \\
2 & - 1 & - 2 \\
 - 1 & 1 & 2
\end{pmatrix}\).
====
Найти жорданову нормальную форму матрицы \(A = \begin{pmatrix}
2 & 1 & 1 \\
1 & 2 & 1 \\
1 & 1 & 2
\end{pmatrix}\).
====
Найти жорданову нормальную форму матрицы \(A = \begin{pmatrix}
0 & 3 & 1 \\
3 & 0 & 1 \\
 - 2 & 2 & 1
\end{pmatrix}\)
====
Найти жорданову нормальную форму матрицы \(A = \begin{pmatrix}
0 & 1 & 0 \\
 - 4 & 4 & 0 \\
0 & 0 & 2
\end{pmatrix}\)
====
Найти жорданову нормальную форму матрицы \(A = \begin{pmatrix}
2 & - 1 & - 1 \\
2 & - 1 & - 2 \\
 - 1 & 1 & 2
\end{pmatrix}\)
====
Найти жорданову нормальную форму матрицы \(A = \begin{pmatrix}
2 & - 1 & 2 \\
5 & - 3 & 3 \\
 - 1 & 0 & - 2
\end{pmatrix}\).
====
Найти жорданову нормальную форму матрицы \(A = \begin{pmatrix}
 - 1 & 4 & 3 \\
 - 2 & 5 & 3 \\
2 & - 4 & - 2
\end{pmatrix}\).
====
Найти жорданову нормальную форму матрицы \(A = \begin{pmatrix}
 - 1 & 3 & - 1 \\
 - 3 & 5 & - 1 \\
 - 3 & 3 & 1
\end{pmatrix}\).
====
Найти жорданову нормальную форму матрицы \(A = \begin{pmatrix}
3 & - 2 & 6 \\
 - 2 & 6 & 3 \\
6 & 3 & - 2
\end{pmatrix}\).
====
Найти жорданову нормальную форму матрицы \(A = \begin{pmatrix}
 - 1 & 4 & 3 \\
 - 2 & 5 & 3 \\
2 & - 4 & - 2
\end{pmatrix}\).
====
Найти жорданову нормальную форму матрицы \(A = \begin{pmatrix}
1 & 2 & 1 \\
1 & 2 & 4 \\
 - 1 & - 2 & - 3
\end{pmatrix}\).