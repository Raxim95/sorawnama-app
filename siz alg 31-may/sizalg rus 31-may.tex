\documentclass{article}
\usepackage[fontsize=11pt]{fontsize}
\usepackage[utf8]{inputenc}
\usepackage[T2A]{fontenc}
% \usepackage{unicode-math}

\usepackage{array}
\usepackage[a4paper,
left=7mm,
right=5mm,
top=7mm,]{geometry}
\usepackage{amsmath}
% \usepackage{amssymbol}
\usepackage{amsfonts}
\usepackage{setspace}



\renewcommand{\baselinestretch}{1} 

\everymath{\displaystyle}
\everydisplay{\displaystyle}
% \linespread{1.25}

\DeclareMathOperator{\sign}{sign}


\begin{document}

\pagenumbering{gobble}


\begin{tabular}{m{17cm}}
\textbf{1-вариант}
\newline

T1. 7. Комплексные евклидовы пространства. \\
T2. 6. Положительно определенные квадратичные формы. \\
A1. Доказать, что векторы \(\overrightarrow{a} = (1;\ \ 2;\ \ 1),\) \(\overrightarrow{b} = (1;\ \ 1;\ \  - 3)\) и \(\overrightarrow{c} = ( - 1;\ \ 2;\ \ 1)\) образуют базис пространства \(\mathbf{R}^{3},\) и найти координаты вектора \(\overrightarrow{d} = (0;\ \ 10;\ \  - 2)\) в этом базисе. \\
A2. Известно, что оператор \emph{A} переводит базисные векторы \(\overrightarrow{i} = (1;\ \ 0;\ \ 0),\) \(\overrightarrow{j} = (0;\ \ 1;\ \ 0),\) \(\overrightarrow{k} = (0;\ \ 0;\ \ 1)\) линейного пространства \(\mathbf{R}^{3}\) в векторы \({\overline{a}}_{1} = (1;\ \ 1;\ \ 0),\) \({\overline{a}}_{2} = (3;\ \ 2;\ \ 1),\) \({\overline{a}}_{3} = (1;2;\ \ 1).\) В базисе \(\overrightarrow{i},\overrightarrow{j},\overrightarrow{k}\) найти: 1)матрицу оператора \(A\) ; 2)образ вектора \(\overline{b} = (1;\ \ 1;\ \  - 2).\) \\
A3. В пространстве \(\mathbb{C}^{3}\) со скалярным произведением \(\left\langle x,y \right\rangle = \sum_{k = 1}^{3}{x_{k}\overline{y_{k}}}\), найдите сопряженный оператор \(A^{*}\) для заданного оператора \(A\). Является ли \(A\)самосопряженным?\(Ax = \left( x_{1} - 2ix_{2},x_{3} + 2ix_{2},ix_{2} + 2ix_{3} \right)\);
 \\
B1. С помощью процесса ортогонализации Грамма- Шмидта ортонормировать следующие системы векторов, используя стандартное скалярное произведение: \\
B2. Найти собственные значения и собственные векторы оператора \emph{А}, заданного в некотором базисе пространства \(V^{3}\) матрицей \(A = \begin{bmatrix}
0 & - 2 & 0 \\
 - 2 & 6 & - 2 \\
0 & - 2 & 5
\end{bmatrix};\) \\
B3. Приведите квадратичные формы \(G_{1}\) и \(G_{2}\) к каноническому виду. \(G_{1} = 3x_{1}^{2} - 2x_{2}^{2} + 2x_{3}^{2} + 4x_{1}x_{2} - 3x_{1}x_{3} - x_{2}x_{3}\), \(G_{2} = 2x_{1}x_{3} + 4x_{1}x_{2} - 2x_{2}x_{3}\) \\
C1. В базисе \((e_{1},\ \ e_{2},\ \ e_{3})\) пространства \(V^{3}\) оператор \emph{A} имеет матрицу \(A = \begin{bmatrix}
1 & - 1 & 2 \\
0 & 3 & - 1 \\
4 & 2 & 2
\end{bmatrix}\ \ .\) Найти матрицу \emph{B} этого же оператора в базисе \(({e'}_{1},\ \ {e'}_{2},\ \ {e'}_{3}),\) где \({e'}_{1} = e_{1} + 2e_{2},\) \({e'}_{2} = e_{1} - e_{3},\) \({e'}_{3} = e_{1} + e_{2} + e_{3}.\) \\
C2. Даны векторы \(e_{1},e_{2},e_{3}\), \(a_{1},a_{2},a_{3}\) линейного пространства \(R^{3}\). Найдите матрицу перехода от базиса \(e_{1},e_{2},e_{3}\) к базису \(a_{1},a_{2},a_{3}\).
\(e_{1} = (2,0,1)\),\(e_{2} = ( - 1,2,3)\),\(e_{3} = ( - 1,1,1)\) и \(a_{1} = ( - 3,0,1)\),\(a_{2} = (0,2,3)\),\(a_{3} = ( - 1, - 1, - 1)\) \\
C3. Найти жорданову нормальную форму матрицы \(A = \begin{pmatrix}
0 & 3 & 1 \\
3 & 0 & 1 \\
 - 2 & 2 & 1
\end{pmatrix}\) \\

\end{tabular}
\vspace{1cm}


\begin{tabular}{m{17cm}}
\textbf{2-вариант}
\newline

T1. 17. Взаимозаменяемые преобразования. \\
T2. 14. Сопряженное преобразование для данного преобразования. \\
A1. Доказать, что векторы \(\overrightarrow{a} = (3;\ \ 1;\ \ 0),\) \(\overrightarrow{b} = (4;\ \ 3;\ \ 2)\) и \(\overrightarrow{c} = ( - 1;\ \  - 4;\ \ 3)\) образуют базис пространства \(\mathbf{R}^{3},\) и найти координаты вектора \(\overrightarrow{d} = ( - 1;\ \ 2;\ \ 5)\) в этом базисе.
 \\
A2. 
Известно, что оператор \emph{A} переводит базисные векторы \(\overrightarrow{i} = (1;\ \ 0;\ \ 0),\) \(\overrightarrow{j} = (0;\ \ 1;\ \ 0),\) \(\overrightarrow{k} = (0;\ \ 0;\ \ 1)\) линейного пространства \(\mathbf{R}^{3}\) в векторы \({\overline{a}}_{1} = (1;1;0),\) \({\overline{a}}_{2} = (3;\ \ 2;\ \ 1),\) \({\overline{a}}_{3} = (0;\ \ 1;\ \ 1).\) В базисе \(\overrightarrow{i},\overrightarrow{j},\overrightarrow{k}\) найти: 1)матрицу оператора \(A\) ; 2)образ вектора \(\overline{b} = (1;\ \  - 2;\ \  - 3).\) \\
A3. В пространстве \(\mathbb{C}^{3}\) со скалярным произведением \(\left\langle x,y \right\rangle = \sum_{k = 1}^{3}{x_{k}\overline{y_{k}}}\), найдите сопряженный оператор \(A^{*}\) для заданного оператора \(A\). Является ли \(A\)самосопряженным? \(Ax = \left( x_{2} + ix_{3},x_{1} - ix_{2},x_{1} + ix_{2} + x_{3} \right)\) \\
B1. \(a_{1} = (1;2;1)\), \(a_{2} = (2;3;3)\), \(a_{3} = (3;8;2)\); \\
B2. Найти собственные значения и собственные векторы оператора \emph{А}, заданного в некотором базисе пространства \(V^{3}\) матрицей \(A = \begin{bmatrix}
 - 1 & 1 & 0 \\
 - 4 & 3 & 0 \\
 - 2 & 1 & 1
\end{bmatrix};\) \\
B3. Приведите квадратичные формы \(G_{1}\) и \(G_{2}\) к каноническому виду. \(G_{1} = 2x_{1}^{2} + 6x_{2}^{2} - 4x_{3}^{2} - 2x_{1}x_{3} + 4x_{1}x_{2} - 8x_{2}x_{3}\), \(G_{2} = x_{2}x_{3} - 2x_{1}x_{3}\) \\
C1. В базисе \((e_{1},\ \ e_{2},\ \ e_{3})\) пространства \(V^{3}\) оператор \emph{A} имеет матрицу \(A = \begin{bmatrix}
4 & 0 & 1 \\
 - 2 & - 2 & 3 \\
0 & 2 & - 1
\end{bmatrix}\ \ .\) Найти матрицу \emph{B} этого же оператора в базисе \(({e'}_{1},\ \ {e'}_{2},\ \ {e'}_{3}),\) где \({e'}_{1} = e_{1} + 2e_{2},\) \({e'}_{2} = e_{1} - e_{3},\) \({e'}_{3} = e_{1} + e_{2} + e_{3}.\)
 \\
C2. 12.Даны векторы \(e_{1},e_{2},e_{3}\), \(a_{1},a_{2},a_{3}\) линейного пространства \(R^{3}\). Найдите матрицу перехода от базиса \(e_{1},e_{2},e_{3}\) к базису \(a_{1},a_{2},a_{3}\).
\(e_{1} = (1,1,1)\),\(e_{2} = (1,1,2)\),\(e_{3} = (1,2,3)\) и \(a_{1} = (2,0,1)\),\(a_{2} = ( - 1,2,3)\),\(a_{3} = ( - 1,1,1)\)
 \\
C3. Найти жорданову нормальную форму матрицы \(A = \begin{pmatrix}
3 & - 2 & 6 \\
 - 2 & 6 & 3 \\
6 & 3 & - 2
\end{pmatrix}\). \\

\end{tabular}
\vspace{1cm}


\begin{tabular}{m{17cm}}
\textbf{3-вариант}
\newline

T1. 1. Линейные пространства. Линейные подпространства. Сумма и пересечение подпространств. \\
T2. 16. Унитарные преобразования и их собственные значения и канонический вид. \\
A1. Доказать, что векторы \(\overrightarrow{a} = (2;\ \ 1;\ \  - 3),\) \(\overrightarrow{b} = ( - 1;\ \ 2;\ \ 4)\) и \(\overrightarrow{c} = (3;\ \  - 4;\ \ 2)\) образуют базис пространства \(\mathbf{R}^{3},\) и найти координаты вектора \(\overrightarrow{d} = ( - 4;\ \ 19;\ \ 3)\) в этом базисе. \\
A2. Известно, что оператор \emph{A} переводит базисные векторы \(\overrightarrow{i} = (1;\ \ 0;\ \ 0),\) \(\overrightarrow{j} = (0;\ \ 1;\ \ 0),\) \(\overrightarrow{k} = (0;\ \ 0;\ \ 1)\) линейного пространства \(\mathbf{R}^{3}\) в векторы \({\overline{a}}_{1} = (1;\ \ 0;\ \ 1),\) \({\overline{a}}_{2} = (0;\ \ 2;\ \ 1),\) \({\overline{a}}_{3} = (3;\ \ 1;\ \ 1).\) В базисе \(\overrightarrow{i},\overrightarrow{j},\overrightarrow{k}\) найти: 1)матрицу оператора \(A\) ; 2)образ вектора \(\overline{b} = (1;\ \ 2;\ \ 3).\) \\
A3. В пространстве \(\mathbb{C}^{3}\) со скалярным произведением \(\left\langle x,y \right\rangle = \sum_{k = 1}^{3}{x_{k}\overline{y_{k}}}\), найдите сопряженный оператор \(A^{*}\) для заданного оператора \(A\). Является ли \(A\)самосопряженным? \(Ax = \left( x_{1} + 2ix_{3},ix_{2} - x_{3},x_{2} - ix_{3} \right)\); \\
B1. \(a_{1} = (1;1;1)\), \(a_{2} = (1;2;3)\), \(a_{3} = (1;1;2)\); \\
B2. Найти собственные значения и собственные векторы оператора \emph{А}, заданного в некотором базисе пространства \(V^{3}\) матрицей \(A = \begin{pmatrix}
0 & 1 & 2 \\
 - 1 & 0 & - 2 \\
 - 2 & 2 & 0
\end{pmatrix}\).
 \\
B3. Приведите квадратичные формы \(G_{1}\) и \(G_{2}\) к каноническому виду. \(G_{1} = x_{1}^{2} - 2x_{3}^{2} + 2x_{1}x_{3} - 6x_{1}x_{2}\), \(G_{2} = 6x_{2}x_{3} - 4x_{1}x_{2} + x_{1}x_{3}\)
 \\
C1. В базисе \((e_{1},\ \ e_{2},\ \ e_{3})\) пространства \(V^{3}\) оператор \emph{A} имеет матрицу \(A = \begin{bmatrix}
 - 3 & 1 & 4 \\
0 & 3 & 2 \\
 - 5 & - 1 & 2
\end{bmatrix}\ \ .\) Найти матрицу \emph{B} этого же оператора в базисе \(({e'}_{1},\ \ {e'}_{2},\ \ {e'}_{3}),\) где \({e'}_{1} = 2e_{1} + e_{2}\), \({e'}_{2} = - e_{1} + 2e_{2} + 3e_{3}\),\({e'}_{3} = - e_{1} + e_{2} + e_{3}\) \\
C2. Даны векторы \(e_{1},e_{2},e_{3}\), \(a_{1},a_{2},a_{3}\) линейного пространства \(R^{3}\). Найдите матрицу перехода от базиса \(e_{1},e_{2},e_{3}\) к базису \(a_{1},a_{2},a_{3}\).
\(e_{1} = (1,0,2)\),\(e_{2} = (3, - 1,4)\),\(e_{3} = (2, - 2,1)\) и \(a_{1} = (4,0,5)\),\(a_{2} = ( - 2,1,3)\),\(a_{3} = ( - 5,1, - 1)\) \\
C3. Найти жорданову нормальную форму матрицы \(A = \begin{pmatrix}
2 & - 1 & - 1 \\
2 & - 1 & - 2 \\
 - 1 & 1 & 2
\end{pmatrix}\) \\

\end{tabular}
\vspace{1cm}


\begin{tabular}{m{17cm}}
\textbf{4-вариант}
\newline

T1. 15. Самосопряженные преобразования и их канонический вид. \\
T2. 12. Связь между матрицами линейных преобразовании в разных базисах. \\
A1. Доказать, что векторы \(\overrightarrow{a} = (3;\ \ 4;\ \ 3),\) \(\overrightarrow{b} = ( - 2;\ \ 3;\ \ 1)\) и \(\overrightarrow{c} = (4;\ \  - 2;\ \ 3)\) образуют базис пространства \(\mathbf{R}^{3},\) и найти координаты вектора \(\overrightarrow{d} = ( - 17;\ \ 18;\ \  - 7)\) в этом базисе. \\
A2. Известно, что оператор \emph{A} переводит базисные векторы \(\overrightarrow{i} = (1;\ \ 0;\ \ 0),\) \(\overrightarrow{j} = (0;\ \ 1;\ \ 0),\) \(\overrightarrow{k} = (0;\ \ 0;\ \ 1)\) линейного пространства \(\mathbf{R}^{3}\) в векторы \({\overline{a}}_{1} = (1;\ \ 0;\ \ 1),\) \({\overline{a}}_{2} = (3;\ \ 2;\ \ 1),\) \({\overline{a}}_{3} = (3;\ \ 1;\ \ 1).\) В базисе \(\overrightarrow{i},\overrightarrow{j},\overrightarrow{k}\) найти: 1)матрицу оператора \(A\) ; 2)образ вектора \(\overline{b} = (1;\ \  - 2;\ \ 3).\) \\
A3. В пространстве \(\mathbb{C}^{3}\) со скалярным произведением \(\left\langle x,y \right\rangle = \sum_{k = 1}^{3}{x_{k}\overline{y_{k}}}\), найдите сопряженный оператор \(A^{*}\) для заданного оператора \(A\). Является ли \(A\)самосопряженным? \(Ax = \left( ix_{1} + x_{3},ix_{3} - ix_{2},x_{1} - ix_{3} \right)\); \\
B1. \(a_{1} = ( - 3;0;1)\), \(a_{2} = (0;2;3)\), \(a_{3} = ( - 1; - 1; - 1)\); \\
B2. 
Найти собственные значения и собственные векторы оператора \emph{А}, заданного в некотором базисе пространства \(V^{3}\) матрицей \(A = \begin{bmatrix}
0 & - 1 & 1 \\
 - 1 & 0 & 1 \\
1 & 1 & 0
\end{bmatrix}.\) \\
B3. Приведите квадратичные формы \(G_{1}\) и \(G_{2}\) к каноническому виду. \(G_{1} = 4x_{1}^{2} + x_{2}^{2} + x_{3}^{2} - 4x_{1}x_{2} + 4x_{1}x_{3} - 3x_{2}x_{3}\), \(G_{2} = x_{1}x_{2} + 6x_{1}x_{3} - 4x_{2}x_{3}\) \\
C1. В базисе \((e_{1},\ \ e_{2},\ \ e_{3})\) пространства \(V^{3}\) оператор \emph{A} имеет матрицу \(A = \begin{bmatrix}
5 & - 2 & 1 \\
 - 1 & 0 & 4 \\
3 & 1 & 2
\end{bmatrix}\ \ .\) Найти матрицу \emph{B} этого же оператора в базисе \(({e'}_{1},\ \ {e'}_{2},\ \ {e'}_{3}),\) где \({e'}_{1} = 2e_{1} + 3e_{3},\) \({e'}_{2} = - e_{2},\) \({e'}_{3} = e_{1} + e_{2} + e_{3}.\) \\
C2. 
Даны векторы \(e_{1},e_{2},e_{3}\), \(a_{1},a_{2},a_{3}\) линейного пространства \(R^{3}\). Найдите матрицу перехода от базиса \(e_{1},e_{2},e_{3}\) к базису \(a_{1},a_{2},a_{3}\).
\(e_{1} = (2,1, - 3)\),\(e_{2} = (3,2, - 5)\),\(e_{3} = (1, - 1,1)\) и \(a_{1} = (0,1, - 2)\),\(a_{2} = ( - 2,0,3)\),\(a_{3} = (1, - 1,1)\) \\
C3. Найти жорданову нормальную форму матрицы \(A = \begin{pmatrix}
 - 1 & 3 & - 1 \\
 - 3 & 5 & - 1 \\
 - 3 & 3 & 1
\end{pmatrix}\). \\

\end{tabular}
\vspace{1cm}


\begin{tabular}{m{17cm}}
\textbf{5-вариант}
\newline

T1. 19. Полиномиальные матрицы и диагональные нормальные формы. \\
T2. 10. Ядро, образ линйеного преобразования. \\
A1. Доказать, что векторы \(\overrightarrow{a} = (1;\ \ 0;\ \ 1),\) \(\overrightarrow{b} = (1;\ \ 1;\ \ 1)\) и \(\overrightarrow{c} = ( - 1;\ \ 2;\ \ 1)\) образуют базис пространства \(\mathbf{R}^{3},\) и найти координаты вектора \(\overrightarrow{d} = (0;\ \ 10;\ \ 3)\) в этом базисе. \\
A2. Известно, что оператор \emph{A} переводит базисные векторы \(\overrightarrow{i} = (1;\ \ 0;\ \ 0),\) \(\overrightarrow{j} = (0;\ \ 1;\ \ 0),\) \(\overrightarrow{k} = (0;\ \ 0;\ \ 1)\) линейного пространства \(\mathbf{R}^{3}\) в векторы \({\overline{a}}_{1} = (1;\ \ 1;\ \ 0),\) \({\overline{a}}_{2} = (3;\ \ 2;\ \ 1),\) \({\overline{a}}_{3} = (3;\ \ 1;\ \ 1).\) В базисе \(\overrightarrow{i},\overrightarrow{j},\overrightarrow{k}\) найти: 1)матрицу оператора \(A\) ; 2)образ вектора \(\overline{b} = (1;\ \ 2;\ \ 3).\) \\
A3. В пространстве \(\mathbb{C}^{3}\) со скалярным произведением \(\left\langle x,y \right\rangle = \sum_{k = 1}^{3}{x_{k}\overline{y_{k}}}\), найдите сопряженный оператор \(A^{*}\) для заданного оператора \(A\). Является ли \(A\)самосопряженным? \(Ax = \left( ix_{1} + x_{2},x_{1} + ix_{2},x_{2} + ix_{3} \right)\); \\
B1. \(a_{1} = (0;1; - 2)\),\(a_{2} = (1; - 1;1)\), \(a_{3} = ( - 2;0;3)\); \\
B2. Найти собственные значения и собственные векторы оператора \emph{А}, заданного в некотором базисе пространства \(V^{3}\) матрицей \(A = \begin{pmatrix}
1 & - 2 & - 1 \\
 - 1 & 1 & 1 \\
1 & 0 & - 1
\end{pmatrix}\) \\
B3. Приведите квадратичные формы \(G_{1}\) и \(G_{2}\) к каноническому виду. \(G_{1} = x_{1}^{2} + 5x_{2}^{2} - 4x_{3}^{2} + 2x_{1}x_{3} - 4x_{1}x_{2}\), \(G_{2} = - 4x_{1}x_{2} + 2x_{1}x_{3}\) \\
C1. 
В базисе \((e_{1},\ \ e_{2},\ \ e_{3})\) пространства \(V^{3}\) оператор \emph{A} имеет матрицу \(A = \begin{bmatrix}
1 & 2 & 3 \\
0 & 1 & 2 \\
3 & 1 & 2
\end{bmatrix}\ \ .\) Найти матрицу \emph{B} этого же оператора в базисе \(({e'}_{1},\ \ {e'}_{2},\ \ {e'}_{3}),\) где \({e'}_{1} = e_{1} + 2e_{2},\) \({e'}_{2} = e_{1} - e_{3},\) \({e'}_{3} = e_{1} + e_{2} + e_{3}.\) \\
C2. Даны векторы \(e_{1},e_{2},e_{3}\), \(a_{1},a_{2},a_{3}\) линейного пространства \(R^{3}\). Найдите матрицу перехода от базиса \(e_{1},e_{2},e_{3}\) к базису \(a_{1},a_{2},a_{3}\).
\(e_{1} = (0,1, - 2)\),\(e_{2} = ( - 2,0,3)\),\(e_{3} = (1, - 1,1)\) и \(a_{1} = (3,1, - 1)\),\(a_{2} = ( - 2,0,1)\),\(a_{3} = (2,7,3)\) \\
C3. Найти жорданову нормальную форму матрицы \(A = \begin{pmatrix}
 - 1 & 4 & 3 \\
 - 2 & 5 & 3 \\
2 & - 4 & - 2
\end{pmatrix}\). \\

\end{tabular}
\vspace{1cm}


\begin{tabular}{m{17cm}}
\textbf{6-вариант}
\newline

T1. 11. Обратное преобразование. \\
T2. 4. Линейные, билинейные, и квадратичные формы. Преобразование матрицы линейного вида при изменении базиса. \\
A1. Доказать, что векторы \(\overrightarrow{a} = (2;\ \ 1;1),\) \(\overrightarrow{b} = ( - 1;\ \ 2;\ \ 4)\) и \(\overrightarrow{c} = (3;\ \ 3;\ \ 2)\) образуют базис пространства \(\mathbf{R}^{3},\) и найти координаты вектора \(\overrightarrow{d} = ( - 4;\ \ 2;\ \ 4)\) в этом базисе. \\
A2. Известно, что оператор \emph{A} переводит базисные векторы \(\overrightarrow{i} = (1;\ \ 0;\ \ 0),\) \(\overrightarrow{j} = (0;\ \ 1;\ \ 0),\) \(\overrightarrow{k} = (0;\ \ 0;\ \ 1)\) линейного пространства \(\mathbf{R}^{3}\) в векторы \({\overline{a}}_{1} = (0;\ \ 1;\ \ 1),\) \({\overline{a}}_{2} = (3;\ \ 1;\ \ 1),\) \({\overline{a}}_{3} = (3;0;1).\) В базисе \(\overrightarrow{i},\overrightarrow{j},\overrightarrow{k}\) найти: 1)матрицу оператора \(A\) ; 2)образ вектора \(\overline{b} = (1;\ \  - 1;\ \  - 1).\) \\
A3. В пространстве \(\mathbb{C}^{3}\) со скалярным произведением \(\left\langle x,y \right\rangle = \sum_{k = 1}^{3}{x_{k}\overline{y_{k}}}\), найдите сопряженный оператор \(A^{*}\) для заданного оператора \(A\). Является ли \(A\)самосопряженным? \(Ax = \left( x_{1} + ix_{3},x_{3} + 2ix_{2},ix_{2} - 2ix_{3} \right)\); \\
B1. \(a_{1} = (2; - 1;3)\), \(a_{2} = (3;2; - 5)\), \(a_{3} = (1; - 1;1)\); \\
B2. Найти собственные значения и собственные векторы оператора \emph{А}, заданного в некотором базисе пространства \(V^{3}\) матрицей \(A = \begin{pmatrix}
2 & - 1 & 2 \\
1 & 0 & 2 \\
 - 2 & 1 & - 1
\end{pmatrix}\) \\
B3. Приведите квадратичные формы \(G_{1}\) и \(G_{2}\) к каноническому виду. \(G_{1} = 2x_{1}^{2} + 3x_{2}^{2} + 4x_{3}^{2} - 2x_{1}x_{2} + 4x_{1}x_{3} - 3x_{2}x_{3}\), \(G_{2} = x_{1}x_{3} - 2x_{2}x_{3}\) \\
C1. В базисе \((e_{1},\ \ e_{2},\ \ e_{3})\) пространства \(V^{3}\) оператор \emph{A} имеет матрицу \(A = \begin{bmatrix}
0 & 1 & - 2 \\
3 & 5 & 1 \\
 - 1 & 2 & 0
\end{bmatrix}\ \ .\) Найти матрицу \emph{B} этого же оператора в базисе \(({e'}_{1},\ \ {e'}_{2},\ \ {e'}_{3}),\) где \({e'}_{1} = e_{1} + 2e_{2},\) \({e'}_{2} = e_{1} - e_{3},\) \({e'}_{3} = e_{1} + e_{2} + e_{3}.\) \\
C2. 10.Даны векторы \(e_{1},e_{2},e_{3}\), \(a_{1},a_{2},a_{3}\) линейного пространства \(R^{3}\). Найдите матрицу перехода от базиса \(e_{1},e_{2},e_{3}\) к базису \(a_{1},a_{2},a_{3}\).
\(e_{1} = (4,0,5)\),\(e_{2} = ( - 2,1,3)\),\(e_{3} = ( - 5,1, - 1)\) и \(a_{1} = (1,2,1)\),\(a_{2} = (2,3,3)\),\(a_{3} = (3,8,2)\) \\
C3. Найти жорданову нормальную форму матрицы \(A = \begin{pmatrix}
 - 1 & 4 & 3 \\
 - 2 & 5 & 3 \\
2 & - 4 & - 2
\end{pmatrix}\). \\

\end{tabular}
\vspace{1cm}


\begin{tabular}{m{17cm}}
\textbf{7-вариант}
\newline

T1. 9. Линейные преобразования и их матрица. \\
T2. 2. Евклидово пространство. Неравенство Коши-Буняковского. Процесс ортогонализации. \\
A1. Доказать, что векторы \(\overrightarrow{a} = (1;\ \ 2;\ \ 1),\) \(\overrightarrow{b} = (1;\ \ 1;\ \ 3)\) и \(\overrightarrow{c} = ( - 1;\ \ 2;\ \ 1)\) образуют базис пространства \(\mathbf{R}^{3},\) и найти координаты вектора \(\overrightarrow{d} = (0;\ \ 10;\ \  - 2)\) в этом базисе. \\
A2. Известно, что оператор \emph{A} переводит базисные векторы \(\overrightarrow{i} = (1;\ \ 0;\ \ 0),\) \(\overrightarrow{j} = (0;\ \ 1;\ \ 0),\) \(\overrightarrow{k} = (0;\ \ 0;\ \ 1)\) линейного пространства \(\mathbf{R}^{3}\) в векторы \({\overline{a}}_{1} = (1;\ \ 1;\ \ 1),{\overline{a}}_{2} = (3;\ \ 0;\ \ 1),\) \({\overline{a}}_{3} = (3;\ \ 1;\ \  - 1).\)В базисе \(\overrightarrow{i},\overrightarrow{j},\overrightarrow{k}\) найти:1)матрицу оператора \(A\) ;2)образ вектора \(\overline{b} = (1;\ \ 2;\ \ 1).\) \\
A3. В пространстве \(\mathbb{C}^{3}\) со скалярным произведением \(\left\langle x,y \right\rangle = \sum_{k = 1}^{3}{x_{k}\overline{y_{k}}}\), найдите сопряженный оператор \(A^{*}\) для заданного оператора \(A\). Является ли \(A\)самосопряженным? \(Ax = \left( 2ix_{1} + ix_{3},x_{1} + x_{2} + ix_{3},ix_{3} \right)\); \\
B1. С помощью процесса ортогонализации Грамма- Шмидта ортонормировать следующие системы векторов, используя стандартное скалярное произведение: \\
B2. Найти собственные значения и собственные векторы оператора \emph{А}, заданного в некотором базисе пространства \(V^{3}\) матрицей \(A = \begin{pmatrix}
1 & - 1 & 1 \\
1 & 1 & - 1 \\
2 & - 1 & 0
\end{pmatrix}\). \\
B3. 
Приведите квадратичные формы \(G_{1}\) и \(G_{2}\) к каноническому виду. \(G_{1} = x_{1}^{2} + x_{2}^{2} + 3x_{3}^{2} + 4x_{1}x_{2} + 2x_{1}x_{3} + 2x_{2}x_{3}\), \(G_{2} = x_{1}x_{2} + x_{1}x_{3} + x_{2}x_{3}\) \\
C1. В базисе \((e_{1},\ \ e_{2},\ \ e_{3})\) пространства \(V^{3}\) оператор \emph{A} имеет матрицу \(A = \begin{bmatrix}
2 & 0 & - 1 \\
3 & 2 & 0 \\
 - 1 & 4 & 3
\end{bmatrix}\ \ .\) Найти матрицу \emph{B} этого же оператора в базисе \(({e'}_{1},\ \ {e'}_{2},\ \ {e'}_{3}),\) где \({e'}_{1} = e_{1} - e_{3},\) \({e'}_{2} = e_{2} + e_{3},\) \({e'}_{3} = e_{3}.\) \\
C2. Даны векторы \(e_{1},e_{2},e_{3}\), \(a_{1},a_{2},a_{3}\) линейного пространства \(R^{3}\). Найдите матрицу перехода от базиса \(e_{1},e_{2},e_{3}\) к базису \(a_{1},a_{2},a_{3}\).
\(e_{1} = (2,0,1)\),\(e_{2} = ( - 1,2,3)\),\(e_{3} = ( - 1,1,1)\) и \(a_{1} = (1,0,2)\),\(a_{2} = (3, - 1,4)\),\(a_{3} = (2, - 2,1)\) \\
C3. Найти жорданову нормальную форму матрицы \(A = \begin{pmatrix}
2 & 1 & 1 \\
1 & 2 & 1 \\
1 & 1 & 2
\end{pmatrix}\). \\

\end{tabular}
\vspace{1cm}


\begin{tabular}{m{17cm}}
\textbf{8-вариант}
\newline

T1. 5. Методы приведения квадратичной формы к каноническому форму. \\
T2. 8. Квадратичные формы в комплексном пространстве и их канонические виды. \\
A1. Доказать, что векторы \(\overrightarrow{a} = (2;\ \ 1;\ \ 1),\) \(\overrightarrow{b} = (1;\ \ 2;\ \ 1)\) и \(\overrightarrow{c} = (2;\ \ 3;\ \  - 1)\) образуют базис пространства \(\mathbf{R}^{3},\) и найти координаты вектора \(\overrightarrow{d} = (2;\ \ 3;\ \  - 1)\) в этом базисе. \\
A2. Известно, что оператор \emph{A} переводит базисные векторы \(\overrightarrow{i} = (1;\ \ 0;\ \ 0),\) \(\overrightarrow{j} = (0;\ \ 1;\ \ 0),\) \(\overrightarrow{k} = (0;\ \ 0;\ \ 1)\) линейного пространства \(\mathbf{R}^{3}\) в векторы \({\overline{a}}_{1} = (1;\ \ 1;\ \ 1),\) \({\overline{a}}_{2} = (3;\ \ 2;\ \ 1),\) \({\overline{a}}_{3} = (0;\ \ 1;\ \ 1).\) В базисе \(\overrightarrow{i},\overrightarrow{j},\overrightarrow{k}\) найти: 1)матрицу оператора \(A\) ; 2)образ вектора \(\overline{b} = (1;\ \  - 2;\ \ 3).\)
 \\
A3. В пространстве \(\mathbb{C}^{3}\) со скалярным произведением \(\left\langle x,y \right\rangle = \sum_{k = 1}^{3}{x_{k}\overline{y_{k}}}\), найдите сопряженный оператор \(A^{*}\) для заданного оператора \(A\). Является ли \(A\)самосопряженным?\(Ax = \left( x_{1} + 2ix_{2},x_{3} - ix_{2},x_{1} - ix_{2} - 2ix_{3} \right)\); \\
B1. \(a_{1} = (3;1; - 1)\), \(a_{2} = ( - 2;0;1)\), \(a_{3} = (2;7;3)\); \\
B2. Найти собственные значения и собственные векторы оператора \emph{А}, заданного в некотором базисе пространства \(V^{3}\) матрицей \(A = \begin{bmatrix}
 - 1 & - 2 & 0 \\
0 & - 2 & 0 \\
2 & 2 & 1
\end{bmatrix}.\) \\
B3. Приведите квадратичные формы \(G_{1}\) и \(G_{2}\) к каноническому виду. \(G_{1} = 2x_{1}^{2} + x_{2}^{2} + x_{3}^{2} + 4x_{1}x_{2} - 2x_{1}x_{3}\), \(G_{2} = x_{1}x_{2} + x_{1}x_{3} + 4x_{2}x_{3}\) \\
C1. В базисе \((e_{1},\ \ e_{2},\ \ e_{3})\) пространства \(V^{3}\) оператор \emph{A} имеет матрицу \(A = \begin{bmatrix}
0 & 1 & - 3 \\
2 & 4 & 1 \\
0 & 3 & - 3
\end{bmatrix}\ \ .\) Найти матрицу \emph{B} этого же оператора в базисе \(({e'}_{1},\ \ {e'}_{2},\ \ {e'}_{3}),\) где \({e'}_{1} = 2e_{1} + e_{2}\), \({e'}_{2} = - e_{1} + 2e_{2} + 3e_{3}\),\({e'}_{3} = - e_{1} + e_{2} + e_{3}\) \\
C2. Даны векторы \(e_{1},e_{2},e_{3}\), \(a_{1},a_{2},a_{3}\) линейного пространства \(R^{3}\). Найдите матрицу перехода от базиса \(e_{1},e_{2},e_{3}\) к базису \(a_{1},a_{2},a_{3}\).
\(e_{1} = (3,5,8)\),\(e_{2} = (5,14,13)\),\(e_{3} = (1,9,2)\) и \(a_{1} = ( - 2,3,1)\),\(a_{2} = (0,2,1)\),\(a_{3} = (1,2,1)\) \\
C3. Найти жорданову нормальную форму матрицы \(A = \begin{pmatrix}
0 & 1 & 0 \\
 - 4 & 4 & 0 \\
0 & 0 & 2
\end{pmatrix}\) \\

\end{tabular}
\vspace{1cm}


\begin{tabular}{m{17cm}}
\textbf{9-вариант}
\newline

T1. 13. Инвариантные подпространства. Собственные векторы и собственные значения. \\
T2. 20. Подобные матрицы. \\
A1. Доказать, что векторы \(\overrightarrow{a} = (2;\ \ 1;\ \ 1),\) \(\overrightarrow{b} = (1;\ \ 2;\ \ 1)\) и \(\overrightarrow{c} = (2;\ \ 1;\ \ 1)\) образуют базис пространства \(\mathbf{R}^{3},\) и найти координаты вектора \(\overrightarrow{d} = (1;\ \ 3;\ \ 1)\) в этом базисе. \\
A2. Известно, что оператор \emph{A} переводит базисные векторы \(\overrightarrow{i} = (1;\ \ 0;\ \ 0),\) \(\overrightarrow{j} = (0;\ \ 1;\ \ 0),\) \(\overrightarrow{k} = (0;\ \ 0;\ \ 1)\) линейного пространства \(\mathbf{R}^{3}\) в векторы \({\overline{a}}_{1} = (1;\ \ 0;\ \ 1),\) \({\overline{a}}_{2} = (0;\ \ 1;\ \ 1),\) \({\overline{a}}_{3} = (3;\ \ 1;\ \ 1).\) В базисе \(\overrightarrow{i},\overrightarrow{j},\overrightarrow{k}\) найти: 1)матрицу оператора \(A\) ; 2)образ вектора \(\overline{b} = (1;\ \  - 2;\ \  - 3).\) \\
A3. 
В пространстве \(\mathbb{C}^{3}\) со скалярным произведением \(\left\langle x,y \right\rangle = \sum_{k = 1}^{3}{x_{k}\overline{y_{k}}}\), найдите сопряженный оператор \(A^{*}\) для заданного оператора \(A\). Является ли \(A\)самосопряженным? \(Ax = \left( ix_{1} + x_{3},x_{2} + ix_{1},x_{1} + ix_{3} \right)\); \\
B1. 
С помощью процесса ортогонализации Грамма- Шмидта ортонормировать следующие системы векторов, используя стандартное скалярное произведение: \\
B2. Найти собственные значения и собственные векторы оператора \emph{А}, заданного в некотором базисе пространства \(V^{3}\) матрицей \(A = \begin{bmatrix}
2 & 1 & 0 \\
1 & 2 & 0 \\
0 & 0 & - 5
\end{bmatrix};\) \\
B3. Приведите квадратичные формы \(G_{1}\) и \(G_{2}\) к каноническому виду. \(G_{1} = x_{1}^{2} - 2x_{2}^{2} + x_{3}^{2} + 2x_{1}x_{2} + 4x_{1}x_{3} + 2x_{2}x_{3}\), \(G_{2} = 2x_{1}x_{3} - 4x_{2}x_{3}\) \\
C1. В базисе \((e_{1},\ \ e_{2},\ \ e_{3})\) пространства \(V^{3}\) оператор \emph{A} имеет матрицу \(A = \begin{bmatrix}
 - 1 & 2 & 1 \\
0 & 1 & - 4 \\
5 & - 1 & 2
\end{bmatrix}\ \ .\) Найти матрицу \emph{B} этого же оператора в базисе \(({e'}_{1},\ \ {e'}_{2},\ \ {e'}_{3}),\) где \({e'}_{1} = e_{1} - e_{3},\) \({e'}_{2} = e_{2} + e_{3},\) \({e'}_{3} = e_{3}.\) \\
C2. Даны векторы \(e_{1},e_{2},e_{3}\), \(a_{1},a_{2},a_{3}\) линейного пространства \(R^{3}\). Найдите матрицу перехода от базиса \(e_{1},e_{2},e_{3}\) к базису \(a_{1},a_{2},a_{3}\).
\(e_{1} = (3,1, - 1)\),\(e_{2} = ( - 2,0,1)\),\(e_{3} = (2,7,3)\) и \(a_{1} = (2,1, - 3)\),\(a_{2} = (3,2, - 5)\),\(a_{3} = (1, - 1,1)\) \\
C3. Найти жорданову нормальную форму матрицы \(A = \begin{pmatrix}
2 & - 1 & - 1 \\
2 & - 1 & - 2 \\
 - 1 & 1 & 2
\end{pmatrix}\). \\

\end{tabular}
\vspace{1cm}


\begin{tabular}{m{17cm}}
\textbf{10-вариант}
\newline

T1. 3. Ортогональное дополнение и ортогональная проекция. \\
T2. 18. Нормальные преобразования и их канонический вид. \\
A1. Доказать, что векторы \(\overrightarrow{a} = (3;\ \ 1;\ \ 2),\) \(\overrightarrow{b} = (2;\ \  - 3;\ \ 1)\) и \(\overrightarrow{c} = (4;\ \  - 2;\ \ 3)\) образуют базис пространства \(\mathbf{R}^{3},\) и найти координаты вектора \(\overrightarrow{d} = ( - 7;\ \ 8;\ \ 7)\) в этом базисе. \\
A2. Известно, что оператор \emph{A} переводит базисные векторы \(\overrightarrow{i} = (1;\ \ 0;\ \ 0),\) \(\overrightarrow{j} = (0;\ \ 1;\ \ 0),\) \(\overrightarrow{k} = (0;\ \ 0;\ \ 1)\) линейного пространства \(\mathbf{R}^{3}\) в векторы \({\overline{a}}_{1} = (2;\ \ 1;1),\) \({\overline{a}}_{2} = (3;\ \ 2;\ \ 1),\) \({\overline{a}}_{3} = (3;\ \ 1;\ \ 1).\) В базисе \(\overrightarrow{i},\overrightarrow{j},\overrightarrow{k}\) найти: 1)матрицу оператора \(A\) ; 2)образ вектора \(\overline{b} = (1;\ \  - 2;\ \ 3).\) \\
A3. В пространстве \(\mathbb{C}^{3}\) со скалярным произведением \(\left\langle x,y \right\rangle = \sum_{k = 1}^{3}{x_{k}\overline{y_{k}}}\), найдите сопряженный оператор \(A^{*}\) для заданного оператора \(A\). Является ли \(A\)самосопряженным? \(Ax = \left( 3ix_{1} + x_{2},x_{1} + 2ix_{2},ix_{2} - x_{3} \right)\); \\
B1. С помощью процесса ортогонализации Грамма- Шмидта ортонормировать следующие системы векторов, используя стандартное скалярное произведение: \\
B2. Найти собственные значения и собственные векторы оператора \emph{А}, заданного в некотором базисе пространства \(V^{3}\) матрицей \(A = \begin{bmatrix}
0 & - 1 & 1 \\
 - 1 & 0 & 1 \\
1 & 1 & 0
\end{bmatrix};\) \\
B3. Приведите квадратичные формы \(G_{1}\) и \(G_{2}\) к каноническому виду. \(G_{1} = x_{1}^{2} - 3x_{3}^{2} - 2x_{1}x_{2} - 2x_{1}x_{3} - 6x_{2}x_{3}\), \(G_{2} = 2x_{1}x_{2} - x_{1}x_{3} + 2x_{2}x_{3}\) \\
C1. В базисе \((e_{1},\ \ e_{2},\ \ e_{3})\) пространства \(V^{3}\) оператор \emph{A} имеет матрицу \(A = \begin{bmatrix}
 - 1 & 2 & 4 \\
 - 4 & 2 & 0 \\
3 & 3 & - 3
\end{bmatrix}\ \ .\) Найти матрицу \emph{B} этого же оператора в базисе \(({e'}_{1},\ \ {e'}_{2},\ \ {e'}_{3}),\) где \({e'}_{1} = 2e_{1} + e_{2}\), \({e'}_{2} = - e_{1} + 2e_{2} + 3e_{3}\),\({e'}_{3} = - e_{1} + e_{2} + e_{3}\) \\
C2. Даны векторы \(e_{1},e_{2},e_{3}\), \(a_{1},a_{2},a_{3}\) линейного пространства \(R^{3}\). Найдите матрицу перехода от базиса \(e_{1},e_{2},e_{3}\) к базису \(a_{1},a_{2},a_{3}\).
\(e_{1} = ( - 2,3,1)\),\(e_{2} = (0,2,1)\),\(e_{3} = (1,2,1)\) и \(a_{1} = ( - 1,3,7)\),\(a_{2} = (0,2, - 1)\),\(a_{3} = (1, - 2, - 8)\) \\
C3. Найти жорданову нормальную форму матрицы \(A = \begin{pmatrix}
1 & 2 & 1 \\
1 & 2 & 4 \\
 - 1 & - 2 & - 3
\end{pmatrix}\). \\

\end{tabular}
\vspace{1cm}


\begin{tabular}{m{17cm}}
\textbf{11-вариант}
\newline

T1. 5. Методы приведения квадратичной формы к каноническому форму. \\
T2. 20. Подобные матрицы. \\
A1. Доказать, что векторы \(\overrightarrow{a} = (1;\ \ 2;\ \ 1),\) \(\overrightarrow{b} = (1;\ \ 1;\ \ 3)\) и \(\overrightarrow{c} = ( - 1;\ \ 2;\ \ 1)\) образуют базис пространства \(\mathbf{R}^{3},\) и найти координаты вектора \(\overrightarrow{d} = (0;\ \ 10;\ \  - 2)\) в этом базисе. \\
A2. Известно, что оператор \emph{A} переводит базисные векторы \(\overrightarrow{i} = (1;\ \ 0;\ \ 0),\) \(\overrightarrow{j} = (0;\ \ 1;\ \ 0),\) \(\overrightarrow{k} = (0;\ \ 0;\ \ 1)\) линейного пространства \(\mathbf{R}^{3}\) в векторы \({\overline{a}}_{1} = (0;\ \ 1;\ \ 1),\) \({\overline{a}}_{2} = (3;\ \ 1;\ \ 1),\) \({\overline{a}}_{3} = (3;\ \ 1;\ \ 1).\) В базисе \(\overrightarrow{i},\overrightarrow{j},\overrightarrow{k}\) найти: 1)матрицу оператора \(A\) ; 2)образ вектора \(\overline{b} = (1;\ \ 1;\ \ 1).\) \\
A3. В пространстве \(\mathbb{C}^{3}\) со скалярным произведением \(\left\langle x,y \right\rangle = \sum_{k = 1}^{3}{x_{k}\overline{y_{k}}}\), найдите сопряженный оператор \(A^{*}\) для заданного оператора \(A\). Является ли \(A\)самосопряженным?\(Ax = \left( x_{1} + 2ix_{3},2ix_{1} + ix_{2},x_{1} + ix_{3} \right)\); \\
B1. С помощью процесса ортогонализации Грамма- Шмидта ортонормировать следующие системы векторов, используя стандартное скалярное произведение: \\
B2. Найти собственные значения и собственные векторы оператора \emph{А}, заданного в некотором базисе пространства \(V^{3}\) матрицей \(A = \begin{pmatrix}
2 & 1 & 0 \\
1 & 3 & - 1 \\
 - 1 & 2 & 3
\end{pmatrix}\). \\
B3. Приведите квадратичные формы \(G_{1}\) и \(G_{2}\) к каноническому виду. \(G_{1} = 3x_{1}^{2} - 2x_{2}^{2} + 2x_{1}x_{3} - 4x_{2}x_{3}\), \(G_{2} = x_{1}x_{2} + x_{2}x_{3}\) \\
C1. В базисе \((e_{1},\ \ e_{2},\ \ e_{3})\) пространства \(V^{3}\) оператор \emph{A} имеет матрицу \(A = \begin{bmatrix}
1 & 3 & - 1 \\
2 & 0 & 4 \\
1 & 1 & 1
\end{bmatrix}\ \ .\) Найти матрицу \emph{B} этого же оператора в базисе \(({e'}_{1},\ \ {e'}_{2},\ \ {e'}_{3}),\) где \({e'}_{1} = 2e_{1} + e_{2}\), \({e'}_{2} = - e_{1} + 2e_{2} + 3e_{3}\),\({e'}_{3} = - e_{1} + e_{2} + e_{3}\) \\
C2. Даны векторы \(e_{1},e_{2},e_{3}\), \(a_{1},a_{2},a_{3}\) линейного пространства \(R^{3}\). Найдите матрицу перехода от базиса \(e_{1},e_{2},e_{3}\) к базису \(a_{1},a_{2},a_{3}\).
\(e_{1} = ( - 3,0,1)\),\(e_{2} = (0,2,3)\),\(e_{3} = ( - 1, - 1, - 1)\) и \(a_{1} = (1,1,1)\),\(a_{2} = (1,1,2)\),\(a_{3} = (1,2,3)\) \\
C3. Найти жорданову нормальную форму матрицы \(A = \begin{pmatrix}
2 & - 1 & 2 \\
5 & - 3 & 3 \\
 - 1 & 0 & - 2
\end{pmatrix}\). \\

\end{tabular}
\vspace{1cm}


\begin{tabular}{m{17cm}}
\textbf{12-вариант}
\newline

T1. 17. Взаимозаменяемые преобразования. \\
T2. 12. Связь между матрицами линейных преобразовании в разных базисах. \\
A1. Доказать, что векторы \(\overrightarrow{a} = (3;\ \ 5;\ \ 4),\) \(\overrightarrow{b} = (4;\ \ 3;\ \ 2)\) и \(\overrightarrow{c} = ( - 1;\ \  - 4;\ \ 3)\) образуют базис пространства \(\mathbf{R}^{3},\) и найти координаты вектора \(\overrightarrow{d} = ( - 2;\ \  - 2;\ \ 5)\) в этом базисе. \\
A2. Известно, что оператор \emph{A} переводит базисные векторы \(\overrightarrow{i} = (1;\ \ 0;\ \ 0),\) \(\overrightarrow{j} = (0;\ \ 1;\ \ 0),\) \(\overrightarrow{k} = (0;\ \ 0;\ \ 1)\) линейного пространства \(\mathbf{R}^{3}\) в векторы \({\overline{a}}_{1} = (1;\ \ 1;\ \ 1),{\overline{a}}_{2} = (3;\ \ 0;\ \ 1),\) \({\overline{a}}_{3} = (0;\ \ 2;\ \ 1).\)В базисе \(\overrightarrow{i},\overrightarrow{j},\overrightarrow{k}\) найти:1)матрицу оператора \(A\) ;2)образ вектора \(\overline{b} = (1;\ \ 2;\ \  - 2).\) \\
A3. В пространстве \(\mathbb{C}^{3}\) со скалярным произведением \(\left\langle x,y \right\rangle = \sum_{k = 1}^{3}{x_{k}\overline{y_{k}}}\), найдите сопряженный оператор \(A^{*}\) для заданного оператора \(A\). Является ли \(A\)самосопряженным? \(Ax = \left( ix_{1} + 2ix_{3},x_{3},x_{1} - 2ix_{3} \right)\); \\
B1. \(a_{1} = ( - 2;3;1)\),\(a_{2} = (0;2;1)\), \(a_{3} = (1;2;1)\); \\
B2. Найти собственные значения и собственные векторы оператора \emph{А}, заданного в некотором базисе пространства \(V^{3}\) матрицей \(A = \begin{pmatrix}
2 & - 1 & 2 \\
5 & - 3 & 3 \\
 - 1 & 0 & - 2
\end{pmatrix}\). \\
B3. Приведите квадратичные формы \(G_{1}\) и \(G_{2}\) к каноническому виду. \(G_{1} = 5x_{1}^{2} + 6x_{2}^{2} - 3x_{3}^{2} + 4x_{1}x_{2} - 2x_{2}x_{3}\), \(G_{2} = 6x_{2}x_{3} - x_{1}x_{2}\) \\
C1. В базисе \((e_{1},\ \ e_{2},\ \ e_{3})\) пространства \(V^{3}\) оператор \emph{A} имеет матрицу \(A = \begin{bmatrix}
3 & 2 & - 1 \\
4 & 0 & 2 \\
 - 1 & 2 & - 1
\end{bmatrix}\ \ .\) Найти матрицу \emph{B} этого же оператора в базисе \(({e'}_{1},\ \ {e'}_{2},\ \ {e'}_{3}),\) где \({e'}_{1} = 2e_{1} + e_{2}\), \({e'}_{2} = - e_{1} + 2e_{2} + 3e_{3}\),\({e'}_{3} = - e_{1} + e_{2} + e_{3}\) \\
C2. 11.Даны векторы \(e_{1},e_{2},e_{3}\), \(a_{1},a_{2},a_{3}\) линейного пространства \(R^{3}\). Найдите матрицу перехода от базиса \(e_{1},e_{2},e_{3}\) к базису \(a_{1},a_{2},a_{3}\).
\(e_{1} = ( - 1,3,7)\),\(e_{2} = (0,2, - 1)\),\(e_{3} = (1, - 2, - 8)\) и \(a_{1} = (0,3, - 2)\),\(a_{2} = (1, - 1, - 8)\),\(a_{3} = ( - 1,2,7)\) \\
C3. 
Найти жорданову нормальную форму матрицы \(A = \begin{pmatrix}
 - 1 & 1 & - 2 \\
3 & - 3 & 6 \\
2 & - 2 & 4
\end{pmatrix}\). \\

\end{tabular}
\vspace{1cm}


\begin{tabular}{m{17cm}}
\textbf{13-вариант}
\newline

T1. 19. Полиномиальные матрицы и диагональные нормальные формы. \\
T2. 4. Линейные, билинейные, и квадратичные формы. Преобразование матрицы линейного вида при изменении базиса. \\
A1. Доказать, что векторы \(\overrightarrow{a} = (3;\ \ 1;\ \ 2),\) \(\overrightarrow{b} = (2;\ \  - 3;\ \ 1)\) и \(\overrightarrow{c} = (4;\ \  - 2;\ \ 3)\) образуют базис пространства \(\mathbf{R}^{3},\) и найти координаты вектора \(\overrightarrow{d} = ( - 7;\ \ 8;\ \ 7)\) в этом базисе. \\
A2. Известно, что оператор \emph{A} переводит базисные векторы \(\overrightarrow{i} = (1;\ \ 0;\ \ 0),\) \(\overrightarrow{j} = (0;\ \ 1;\ \ 0),\) \(\overrightarrow{k} = (0;\ \ 0;\ \ 1)\) линейного пространства \(\mathbf{R}^{3}\) в векторы \({\overline{a}}_{1} = (1;\ \ 0;\ \ 1),\) \({\overline{a}}_{2} = (0;\ \ 2;\ \ 1),\) \({\overline{a}}_{3} = (3;\ \ 1;\ \ 1).\) В базисе \(\overrightarrow{i},\overrightarrow{j},\overrightarrow{k}\) найти: 1)матрицу оператора \(A\) ; 2)образ вектора \(\overline{b} = (1;\ \ 2;\ \ 3).\) \\
A3. В пространстве \(\mathbb{C}^{3}\) со скалярным произведением \(\left\langle x,y \right\rangle = \sum_{k = 1}^{3}{x_{k}\overline{y_{k}}}\), найдите сопряженный оператор \(A^{*}\) для заданного оператора \(A\). Является ли \(A\)самосопряженным? \(Ax = \left( ix_{1} + 2ix_{3},x_{3},x_{1} - 2ix_{3} \right)\); \\
B1. С помощью процесса ортогонализации Грамма- Шмидта ортонормировать следующие системы векторов, используя стандартное скалярное произведение: \\
B2. Найти собственные значения и собственные векторы оператора \emph{А}, заданного в некотором базисе пространства \(V^{3}\) матрицей \(A = \begin{pmatrix}
0 & 1 & 2 \\
 - 1 & 0 & - 2 \\
 - 2 & 2 & 0
\end{pmatrix}\).
 \\
B3. Приведите квадратичные формы \(G_{1}\) и \(G_{2}\) к каноническому виду. \(G_{1} = x_{1}^{2} - 2x_{2}^{2} + x_{3}^{2} + 2x_{1}x_{2} + 4x_{1}x_{3} + 2x_{2}x_{3}\), \(G_{2} = 2x_{1}x_{3} - 4x_{2}x_{3}\) \\
C1. В базисе \((e_{1},\ \ e_{2},\ \ e_{3})\) пространства \(V^{3}\) оператор \emph{A} имеет матрицу \(A = \begin{bmatrix}
5 & - 2 & 1 \\
 - 1 & 0 & 4 \\
3 & 1 & 2
\end{bmatrix}\ \ .\) Найти матрицу \emph{B} этого же оператора в базисе \(({e'}_{1},\ \ {e'}_{2},\ \ {e'}_{3}),\) где \({e'}_{1} = 2e_{1} + 3e_{3},\) \({e'}_{2} = - e_{2},\) \({e'}_{3} = e_{1} + e_{2} + e_{3}.\) \\
C2. Даны векторы \(e_{1},e_{2},e_{3}\), \(a_{1},a_{2},a_{3}\) линейного пространства \(R^{3}\). Найдите матрицу перехода от базиса \(e_{1},e_{2},e_{3}\) к базису \(a_{1},a_{2},a_{3}\).
\(e_{1} = (0,1, - 2)\),\(e_{2} = ( - 2,0,3)\),\(e_{3} = (1, - 1,1)\) и \(a_{1} = (3,1, - 1)\),\(a_{2} = ( - 2,0,1)\),\(a_{3} = (2,7,3)\) \\
C3. Найти жорданову нормальную форму матрицы \(A = \begin{pmatrix}
3 & - 2 & 6 \\
 - 2 & 6 & 3 \\
6 & 3 & - 2
\end{pmatrix}\). \\

\end{tabular}
\vspace{1cm}


\begin{tabular}{m{17cm}}
\textbf{14-вариант}
\newline

T1. 9. Линейные преобразования и их матрица. \\
T2. 2. Евклидово пространство. Неравенство Коши-Буняковского. Процесс ортогонализации. \\
A1. Доказать, что векторы \(\overrightarrow{a} = (2;\ \ 1;1),\) \(\overrightarrow{b} = ( - 1;\ \ 2;\ \ 4)\) и \(\overrightarrow{c} = (3;\ \ 3;\ \ 2)\) образуют базис пространства \(\mathbf{R}^{3},\) и найти координаты вектора \(\overrightarrow{d} = ( - 4;\ \ 2;\ \ 4)\) в этом базисе. \\
A2. 
Известно, что оператор \emph{A} переводит базисные векторы \(\overrightarrow{i} = (1;\ \ 0;\ \ 0),\) \(\overrightarrow{j} = (0;\ \ 1;\ \ 0),\) \(\overrightarrow{k} = (0;\ \ 0;\ \ 1)\) линейного пространства \(\mathbf{R}^{3}\) в векторы \({\overline{a}}_{1} = (1;1;0),\) \({\overline{a}}_{2} = (3;\ \ 2;\ \ 1),\) \({\overline{a}}_{3} = (0;\ \ 1;\ \ 1).\) В базисе \(\overrightarrow{i},\overrightarrow{j},\overrightarrow{k}\) найти: 1)матрицу оператора \(A\) ; 2)образ вектора \(\overline{b} = (1;\ \  - 2;\ \  - 3).\) \\
A3. В пространстве \(\mathbb{C}^{3}\) со скалярным произведением \(\left\langle x,y \right\rangle = \sum_{k = 1}^{3}{x_{k}\overline{y_{k}}}\), найдите сопряженный оператор \(A^{*}\) для заданного оператора \(A\). Является ли \(A\)самосопряженным? \(Ax = \left( 3ix_{1} + x_{2},x_{1} + 2ix_{2},ix_{2} - x_{3} \right)\); \\
B1. С помощью процесса ортогонализации Грамма- Шмидта ортонормировать следующие системы векторов, используя стандартное скалярное произведение: \\
B2. Найти собственные значения и собственные векторы оператора \emph{А}, заданного в некотором базисе пространства \(V^{3}\) матрицей \(A = \begin{pmatrix}
1 & - 1 & 1 \\
1 & 1 & - 1 \\
2 & - 1 & 0
\end{pmatrix}\). \\
B3. Приведите квадратичные формы \(G_{1}\) и \(G_{2}\) к каноническому виду. \(G_{1} = x_{1}^{2} - 3x_{3}^{2} - 2x_{1}x_{2} - 2x_{1}x_{3} - 6x_{2}x_{3}\), \(G_{2} = 2x_{1}x_{2} - x_{1}x_{3} + 2x_{2}x_{3}\) \\
C1. В базисе \((e_{1},\ \ e_{2},\ \ e_{3})\) пространства \(V^{3}\) оператор \emph{A} имеет матрицу \(A = \begin{bmatrix}
1 & 3 & - 1 \\
2 & 0 & 4 \\
1 & 1 & 1
\end{bmatrix}\ \ .\) Найти матрицу \emph{B} этого же оператора в базисе \(({e'}_{1},\ \ {e'}_{2},\ \ {e'}_{3}),\) где \({e'}_{1} = 2e_{1} + e_{2}\), \({e'}_{2} = - e_{1} + 2e_{2} + 3e_{3}\),\({e'}_{3} = - e_{1} + e_{2} + e_{3}\) \\
C2. Даны векторы \(e_{1},e_{2},e_{3}\), \(a_{1},a_{2},a_{3}\) линейного пространства \(R^{3}\). Найдите матрицу перехода от базиса \(e_{1},e_{2},e_{3}\) к базису \(a_{1},a_{2},a_{3}\).
\(e_{1} = (2,0,1)\),\(e_{2} = ( - 1,2,3)\),\(e_{3} = ( - 1,1,1)\) и \(a_{1} = (1,0,2)\),\(a_{2} = (3, - 1,4)\),\(a_{3} = (2, - 2,1)\) \\
C3. Найти жорданову нормальную форму матрицы \(A = \begin{pmatrix}
2 & - 1 & 2 \\
5 & - 3 & 3 \\
 - 1 & 0 & - 2
\end{pmatrix}\). \\

\end{tabular}
\vspace{1cm}


\begin{tabular}{m{17cm}}
\textbf{15-вариант}
\newline

T1. 3. Ортогональное дополнение и ортогональная проекция. \\
T2. 6. Положительно определенные квадратичные формы. \\
A1. Доказать, что векторы \(\overrightarrow{a} = (3;\ \ 1;\ \ 0),\) \(\overrightarrow{b} = (4;\ \ 3;\ \ 2)\) и \(\overrightarrow{c} = ( - 1;\ \  - 4;\ \ 3)\) образуют базис пространства \(\mathbf{R}^{3},\) и найти координаты вектора \(\overrightarrow{d} = ( - 1;\ \ 2;\ \ 5)\) в этом базисе.
 \\
A2. Известно, что оператор \emph{A} переводит базисные векторы \(\overrightarrow{i} = (1;\ \ 0;\ \ 0),\) \(\overrightarrow{j} = (0;\ \ 1;\ \ 0),\) \(\overrightarrow{k} = (0;\ \ 0;\ \ 1)\) линейного пространства \(\mathbf{R}^{3}\) в векторы \({\overline{a}}_{1} = (0;\ \ 1;\ \ 1),\) \({\overline{a}}_{2} = (3;\ \ 1;\ \ 1),\) \({\overline{a}}_{3} = (3;\ \ 1;\ \ 1).\) В базисе \(\overrightarrow{i},\overrightarrow{j},\overrightarrow{k}\) найти: 1)матрицу оператора \(A\) ; 2)образ вектора \(\overline{b} = (1;\ \ 1;\ \ 1).\) \\
A3. В пространстве \(\mathbb{C}^{3}\) со скалярным произведением \(\left\langle x,y \right\rangle = \sum_{k = 1}^{3}{x_{k}\overline{y_{k}}}\), найдите сопряженный оператор \(A^{*}\) для заданного оператора \(A\). Является ли \(A\)самосопряженным? \(Ax = \left( x_{1} + ix_{3},x_{3} + 2ix_{2},ix_{2} - 2ix_{3} \right)\); \\
B1. С помощью процесса ортогонализации Грамма- Шмидта ортонормировать следующие системы векторов, используя стандартное скалярное произведение: \\
B2. Найти собственные значения и собственные векторы оператора \emph{А}, заданного в некотором базисе пространства \(V^{3}\) матрицей \(A = \begin{pmatrix}
1 & - 2 & - 1 \\
 - 1 & 1 & 1 \\
1 & 0 & - 1
\end{pmatrix}\) \\
B3. Приведите квадратичные формы \(G_{1}\) и \(G_{2}\) к каноническому виду. \(G_{1} = x_{1}^{2} + 5x_{2}^{2} - 4x_{3}^{2} + 2x_{1}x_{3} - 4x_{1}x_{2}\), \(G_{2} = - 4x_{1}x_{2} + 2x_{1}x_{3}\) \\
C1. В базисе \((e_{1},\ \ e_{2},\ \ e_{3})\) пространства \(V^{3}\) оператор \emph{A} имеет матрицу \(A = \begin{bmatrix}
 - 3 & 1 & 4 \\
0 & 3 & 2 \\
 - 5 & - 1 & 2
\end{bmatrix}\ \ .\) Найти матрицу \emph{B} этого же оператора в базисе \(({e'}_{1},\ \ {e'}_{2},\ \ {e'}_{3}),\) где \({e'}_{1} = 2e_{1} + e_{2}\), \({e'}_{2} = - e_{1} + 2e_{2} + 3e_{3}\),\({e'}_{3} = - e_{1} + e_{2} + e_{3}\) \\
C2. Даны векторы \(e_{1},e_{2},e_{3}\), \(a_{1},a_{2},a_{3}\) линейного пространства \(R^{3}\). Найдите матрицу перехода от базиса \(e_{1},e_{2},e_{3}\) к базису \(a_{1},a_{2},a_{3}\).
\(e_{1} = ( - 2,3,1)\),\(e_{2} = (0,2,1)\),\(e_{3} = (1,2,1)\) и \(a_{1} = ( - 1,3,7)\),\(a_{2} = (0,2, - 1)\),\(a_{3} = (1, - 2, - 8)\) \\
C3. Найти жорданову нормальную форму матрицы \(A = \begin{pmatrix}
 - 1 & 3 & - 1 \\
 - 3 & 5 & - 1 \\
 - 3 & 3 & 1
\end{pmatrix}\). \\

\end{tabular}
\vspace{1cm}


\begin{tabular}{m{17cm}}
\textbf{16-вариант}
\newline

T1. 11. Обратное преобразование. \\
T2. 8. Квадратичные формы в комплексном пространстве и их канонические виды. \\
A1. Доказать, что векторы \(\overrightarrow{a} = (3;\ \ 4;\ \ 3),\) \(\overrightarrow{b} = ( - 2;\ \ 3;\ \ 1)\) и \(\overrightarrow{c} = (4;\ \  - 2;\ \ 3)\) образуют базис пространства \(\mathbf{R}^{3},\) и найти координаты вектора \(\overrightarrow{d} = ( - 17;\ \ 18;\ \  - 7)\) в этом базисе. \\
A2. Известно, что оператор \emph{A} переводит базисные векторы \(\overrightarrow{i} = (1;\ \ 0;\ \ 0),\) \(\overrightarrow{j} = (0;\ \ 1;\ \ 0),\) \(\overrightarrow{k} = (0;\ \ 0;\ \ 1)\) линейного пространства \(\mathbf{R}^{3}\) в векторы \({\overline{a}}_{1} = (2;\ \ 1;1),\) \({\overline{a}}_{2} = (3;\ \ 2;\ \ 1),\) \({\overline{a}}_{3} = (3;\ \ 1;\ \ 1).\) В базисе \(\overrightarrow{i},\overrightarrow{j},\overrightarrow{k}\) найти: 1)матрицу оператора \(A\) ; 2)образ вектора \(\overline{b} = (1;\ \  - 2;\ \ 3).\) \\
A3. В пространстве \(\mathbb{C}^{3}\) со скалярным произведением \(\left\langle x,y \right\rangle = \sum_{k = 1}^{3}{x_{k}\overline{y_{k}}}\), найдите сопряженный оператор \(A^{*}\) для заданного оператора \(A\). Является ли \(A\)самосопряженным? \(Ax = \left( x_{1} + 2ix_{3},ix_{2} - x_{3},x_{2} - ix_{3} \right)\); \\
B1. С помощью процесса ортогонализации Грамма- Шмидта ортонормировать следующие системы векторов, используя стандартное скалярное произведение: \\
B2. Найти собственные значения и собственные векторы оператора \emph{А}, заданного в некотором базисе пространства \(V^{3}\) матрицей \(A = \begin{bmatrix}
 - 1 & - 2 & 0 \\
0 & - 2 & 0 \\
2 & 2 & 1
\end{bmatrix}.\) \\
B3. Приведите квадратичные формы \(G_{1}\) и \(G_{2}\) к каноническому виду. \(G_{1} = 2x_{1}^{2} + x_{2}^{2} + x_{3}^{2} + 4x_{1}x_{2} - 2x_{1}x_{3}\), \(G_{2} = x_{1}x_{2} + x_{1}x_{3} + 4x_{2}x_{3}\) \\
C1. 
В базисе \((e_{1},\ \ e_{2},\ \ e_{3})\) пространства \(V^{3}\) оператор \emph{A} имеет матрицу \(A = \begin{bmatrix}
1 & 2 & 3 \\
0 & 1 & 2 \\
3 & 1 & 2
\end{bmatrix}\ \ .\) Найти матрицу \emph{B} этого же оператора в базисе \(({e'}_{1},\ \ {e'}_{2},\ \ {e'}_{3}),\) где \({e'}_{1} = e_{1} + 2e_{2},\) \({e'}_{2} = e_{1} - e_{3},\) \({e'}_{3} = e_{1} + e_{2} + e_{3}.\) \\
C2. 10.Даны векторы \(e_{1},e_{2},e_{3}\), \(a_{1},a_{2},a_{3}\) линейного пространства \(R^{3}\). Найдите матрицу перехода от базиса \(e_{1},e_{2},e_{3}\) к базису \(a_{1},a_{2},a_{3}\).
\(e_{1} = (4,0,5)\),\(e_{2} = ( - 2,1,3)\),\(e_{3} = ( - 5,1, - 1)\) и \(a_{1} = (1,2,1)\),\(a_{2} = (2,3,3)\),\(a_{3} = (3,8,2)\) \\
C3. Найти жорданову нормальную форму матрицы \(A = \begin{pmatrix}
1 & 2 & 1 \\
1 & 2 & 4 \\
 - 1 & - 2 & - 3
\end{pmatrix}\). \\

\end{tabular}
\vspace{1cm}


\begin{tabular}{m{17cm}}
\textbf{17-вариант}
\newline

T1. 13. Инвариантные подпространства. Собственные векторы и собственные значения. \\
T2. 14. Сопряженное преобразование для данного преобразования. \\
A1. Доказать, что векторы \(\overrightarrow{a} = (2;\ \ 1;\ \  - 3),\) \(\overrightarrow{b} = ( - 1;\ \ 2;\ \ 4)\) и \(\overrightarrow{c} = (3;\ \  - 4;\ \ 2)\) образуют базис пространства \(\mathbf{R}^{3},\) и найти координаты вектора \(\overrightarrow{d} = ( - 4;\ \ 19;\ \ 3)\) в этом базисе. \\
A2. Известно, что оператор \emph{A} переводит базисные векторы \(\overrightarrow{i} = (1;\ \ 0;\ \ 0),\) \(\overrightarrow{j} = (0;\ \ 1;\ \ 0),\) \(\overrightarrow{k} = (0;\ \ 0;\ \ 1)\) линейного пространства \(\mathbf{R}^{3}\) в векторы \({\overline{a}}_{1} = (1;\ \ 1;\ \ 1),\) \({\overline{a}}_{2} = (3;\ \ 2;\ \ 1),\) \({\overline{a}}_{3} = (0;\ \ 1;\ \ 1).\) В базисе \(\overrightarrow{i},\overrightarrow{j},\overrightarrow{k}\) найти: 1)матрицу оператора \(A\) ; 2)образ вектора \(\overline{b} = (1;\ \  - 2;\ \ 3).\)
 \\
A3. 
В пространстве \(\mathbb{C}^{3}\) со скалярным произведением \(\left\langle x,y \right\rangle = \sum_{k = 1}^{3}{x_{k}\overline{y_{k}}}\), найдите сопряженный оператор \(A^{*}\) для заданного оператора \(A\). Является ли \(A\)самосопряженным? \(Ax = \left( ix_{1} + x_{3},x_{2} + ix_{1},x_{1} + ix_{3} \right)\); \\
B1. \(a_{1} = (4;1;3)\), \(a_{2} = (0;7; - 2)\), \(a_{3} = (4;8;0)\);
 \\
B2. Найти собственные значения и собственные векторы оператора \emph{А}, заданного в некотором базисе пространства \(V^{3}\) матрицей \(A = \begin{pmatrix}
2 & - 1 & 2 \\
1 & 0 & 2 \\
 - 2 & 1 & - 1
\end{pmatrix}\) \\
B3. 
Приведите квадратичные формы \(G_{1}\) и \(G_{2}\) к каноническому виду. \(G_{1} = x_{1}^{2} + x_{2}^{2} + 3x_{3}^{2} + 4x_{1}x_{2} + 2x_{1}x_{3} + 2x_{2}x_{3}\), \(G_{2} = x_{1}x_{2} + x_{1}x_{3} + x_{2}x_{3}\) \\
C1. В базисе \((e_{1},\ \ e_{2},\ \ e_{3})\) пространства \(V^{3}\) оператор \emph{A} имеет матрицу \(A = \begin{bmatrix}
 - 1 & 2 & 4 \\
 - 4 & 2 & 0 \\
3 & 3 & - 3
\end{bmatrix}\ \ .\) Найти матрицу \emph{B} этого же оператора в базисе \(({e'}_{1},\ \ {e'}_{2},\ \ {e'}_{3}),\) где \({e'}_{1} = 2e_{1} + e_{2}\), \({e'}_{2} = - e_{1} + 2e_{2} + 3e_{3}\),\({e'}_{3} = - e_{1} + e_{2} + e_{3}\) \\
C2. 11.Даны векторы \(e_{1},e_{2},e_{3}\), \(a_{1},a_{2},a_{3}\) линейного пространства \(R^{3}\). Найдите матрицу перехода от базиса \(e_{1},e_{2},e_{3}\) к базису \(a_{1},a_{2},a_{3}\).
\(e_{1} = ( - 1,3,7)\),\(e_{2} = (0,2, - 1)\),\(e_{3} = (1, - 2, - 8)\) и \(a_{1} = (0,3, - 2)\),\(a_{2} = (1, - 1, - 8)\),\(a_{3} = ( - 1,2,7)\) \\
C3. Найти жорданову нормальную форму матрицы \(A = \begin{pmatrix}
2 & - 1 & - 1 \\
2 & - 1 & - 2 \\
 - 1 & 1 & 2
\end{pmatrix}\). \\

\end{tabular}
\vspace{1cm}


\begin{tabular}{m{17cm}}
\textbf{18-вариант}
\newline

T1. 1. Линейные пространства. Линейные подпространства. Сумма и пересечение подпространств. \\
T2. 18. Нормальные преобразования и их канонический вид. \\
A1. Доказать, что векторы \(\overrightarrow{a} = (1;\ \ 2;\ \ 1),\) \(\overrightarrow{b} = (1;\ \ 1;\ \  - 3)\) и \(\overrightarrow{c} = ( - 1;\ \ 2;\ \ 1)\) образуют базис пространства \(\mathbf{R}^{3},\) и найти координаты вектора \(\overrightarrow{d} = (0;\ \ 10;\ \  - 2)\) в этом базисе. \\
A2. Известно, что оператор \emph{A} переводит базисные векторы \(\overrightarrow{i} = (1;\ \ 0;\ \ 0),\) \(\overrightarrow{j} = (0;\ \ 1;\ \ 0),\) \(\overrightarrow{k} = (0;\ \ 0;\ \ 1)\) линейного пространства \(\mathbf{R}^{3}\) в векторы \({\overline{a}}_{1} = (1;\ \ 1;\ \ 1),{\overline{a}}_{2} = (3;\ \ 0;\ \ 1),\) \({\overline{a}}_{3} = (0;\ \ 2;\ \ 1).\)В базисе \(\overrightarrow{i},\overrightarrow{j},\overrightarrow{k}\) найти:1)матрицу оператора \(A\) ;2)образ вектора \(\overline{b} = (1;\ \ 2;\ \  - 2).\) \\
A3. В пространстве \(\mathbb{C}^{3}\) со скалярным произведением \(\left\langle x,y \right\rangle = \sum_{k = 1}^{3}{x_{k}\overline{y_{k}}}\), найдите сопряженный оператор \(A^{*}\) для заданного оператора \(A\). Является ли \(A\)самосопряженным? \(Ax = \left( ix_{1} + x_{2},x_{1} + ix_{2},x_{2} + ix_{3} \right)\); \\
B1. \(a_{1} = (2;0;1)\), \(a_{2} = ( - 1;2;3)\), \(a_{3} = ( - 1;1;1)\); \\
B2. Найти собственные значения и собственные векторы оператора \emph{А}, заданного в некотором базисе пространства \(V^{3}\) матрицей \(A = \begin{bmatrix}
 - 1 & 1 & 0 \\
 - 4 & 3 & 0 \\
 - 2 & 1 & 1
\end{bmatrix};\) \\
B3. Приведите квадратичные формы \(G_{1}\) и \(G_{2}\) к каноническому виду. \(G_{1} = 4x_{1}^{2} + x_{2}^{2} + x_{3}^{2} - 4x_{1}x_{2} + 4x_{1}x_{3} - 3x_{2}x_{3}\), \(G_{2} = x_{1}x_{2} + 6x_{1}x_{3} - 4x_{2}x_{3}\) \\
C1. В базисе \((e_{1},\ \ e_{2},\ \ e_{3})\) пространства \(V^{3}\) оператор \emph{A} имеет матрицу \(A = \begin{bmatrix}
1 & - 1 & 2 \\
0 & 3 & - 1 \\
4 & 2 & 2
\end{bmatrix}\ \ .\) Найти матрицу \emph{B} этого же оператора в базисе \(({e'}_{1},\ \ {e'}_{2},\ \ {e'}_{3}),\) где \({e'}_{1} = e_{1} + 2e_{2},\) \({e'}_{2} = e_{1} - e_{3},\) \({e'}_{3} = e_{1} + e_{2} + e_{3}.\) \\
C2. Даны векторы \(e_{1},e_{2},e_{3}\), \(a_{1},a_{2},a_{3}\) линейного пространства \(R^{3}\). Найдите матрицу перехода от базиса \(e_{1},e_{2},e_{3}\) к базису \(a_{1},a_{2},a_{3}\).
\(e_{1} = (1,0,2)\),\(e_{2} = (3, - 1,4)\),\(e_{3} = (2, - 2,1)\) и \(a_{1} = (4,0,5)\),\(a_{2} = ( - 2,1,3)\),\(a_{3} = ( - 5,1, - 1)\) \\
C3. Найти жорданову нормальную форму матрицы \(A = \begin{pmatrix}
0 & 1 & 0 \\
 - 4 & 4 & 0 \\
0 & 0 & 2
\end{pmatrix}\) \\

\end{tabular}
\vspace{1cm}


\begin{tabular}{m{17cm}}
\textbf{19-вариант}
\newline

T1. 7. Комплексные евклидовы пространства. \\
T2. 10. Ядро, образ линйеного преобразования. \\
A1. Доказать, что векторы \(\overrightarrow{a} = (1;\ \ 2;\ \ 1),\) \(\overrightarrow{b} = (1;\ \ 1;\ \ 3)\) и \(\overrightarrow{c} = ( - 1;\ \ 2;\ \ 1)\) образуют базис пространства \(\mathbf{R}^{3},\) и найти координаты вектора \(\overrightarrow{d} = (0;\ \ 10;\ \  - 2)\) в этом базисе. \\
A2. Известно, что оператор \emph{A} переводит базисные векторы \(\overrightarrow{i} = (1;\ \ 0;\ \ 0),\) \(\overrightarrow{j} = (0;\ \ 1;\ \ 0),\) \(\overrightarrow{k} = (0;\ \ 0;\ \ 1)\) линейного пространства \(\mathbf{R}^{3}\) в векторы \({\overline{a}}_{1} = (0;\ \ 1;\ \ 1),\) \({\overline{a}}_{2} = (3;\ \ 1;\ \ 1),\) \({\overline{a}}_{3} = (3;0;1).\) В базисе \(\overrightarrow{i},\overrightarrow{j},\overrightarrow{k}\) найти: 1)матрицу оператора \(A\) ; 2)образ вектора \(\overline{b} = (1;\ \  - 1;\ \  - 1).\) \\
A3. В пространстве \(\mathbb{C}^{3}\) со скалярным произведением \(\left\langle x,y \right\rangle = \sum_{k = 1}^{3}{x_{k}\overline{y_{k}}}\), найдите сопряженный оператор \(A^{*}\) для заданного оператора \(A\). Является ли \(A\)самосопряженным? \(Ax = \left( 2ix_{1} + ix_{3},x_{1} + x_{2} + ix_{3},ix_{3} \right)\); \\
B1. С помощью процесса ортогонализации Грамма- Шмидта ортонормировать следующие системы векторов, используя стандартное скалярное произведение: \\
B2. Найти собственные значения и собственные векторы оператора \emph{А}, заданного в некотором базисе пространства \(V^{3}\) матрицей \(A = \begin{pmatrix}
2 & 1 & 0 \\
1 & 3 & - 1 \\
 - 1 & 2 & 3
\end{pmatrix}\). \\
B3. Приведите квадратичные формы \(G_{1}\) и \(G_{2}\) к каноническому виду. \(G_{1} = 3x_{1}^{2} - 2x_{2}^{2} + 2x_{3}^{2} + 4x_{1}x_{2} - 3x_{1}x_{3} - x_{2}x_{3}\), \(G_{2} = 2x_{1}x_{3} + 4x_{1}x_{2} - 2x_{2}x_{3}\) \\
C1. В базисе \((e_{1},\ \ e_{2},\ \ e_{3})\) пространства \(V^{3}\) оператор \emph{A} имеет матрицу \(A = \begin{bmatrix}
4 & 0 & 1 \\
 - 2 & - 2 & 3 \\
0 & 2 & - 1
\end{bmatrix}\ \ .\) Найти матрицу \emph{B} этого же оператора в базисе \(({e'}_{1},\ \ {e'}_{2},\ \ {e'}_{3}),\) где \({e'}_{1} = e_{1} + 2e_{2},\) \({e'}_{2} = e_{1} - e_{3},\) \({e'}_{3} = e_{1} + e_{2} + e_{3}.\)
 \\
C2. Даны векторы \(e_{1},e_{2},e_{3}\), \(a_{1},a_{2},a_{3}\) линейного пространства \(R^{3}\). Найдите матрицу перехода от базиса \(e_{1},e_{2},e_{3}\) к базису \(a_{1},a_{2},a_{3}\).
\(e_{1} = (3,5,8)\),\(e_{2} = (5,14,13)\),\(e_{3} = (1,9,2)\) и \(a_{1} = ( - 2,3,1)\),\(a_{2} = (0,2,1)\),\(a_{3} = (1,2,1)\) \\
C3. Найти жорданову нормальную форму матрицы \(A = \begin{pmatrix}
 - 1 & 4 & 3 \\
 - 2 & 5 & 3 \\
2 & - 4 & - 2
\end{pmatrix}\). \\

\end{tabular}
\vspace{1cm}


\begin{tabular}{m{17cm}}
\textbf{20-вариант}
\newline

T1. 15. Самосопряженные преобразования и их канонический вид. \\
T2. 16. Унитарные преобразования и их собственные значения и канонический вид. \\
A1. Доказать, что векторы \(\overrightarrow{a} = (2;\ \ 1;\ \ 1),\) \(\overrightarrow{b} = (1;\ \ 2;\ \ 1)\) и \(\overrightarrow{c} = (2;\ \ 1;\ \ 1)\) образуют базис пространства \(\mathbf{R}^{3},\) и найти координаты вектора \(\overrightarrow{d} = (1;\ \ 3;\ \ 1)\) в этом базисе. \\
A2. Известно, что оператор \emph{A} переводит базисные векторы \(\overrightarrow{i} = (1;\ \ 0;\ \ 0),\) \(\overrightarrow{j} = (0;\ \ 1;\ \ 0),\) \(\overrightarrow{k} = (0;\ \ 0;\ \ 1)\) линейного пространства \(\mathbf{R}^{3}\) в векторы \({\overline{a}}_{1} = (1;\ \ 1;\ \ 0),\) \({\overline{a}}_{2} = (3;\ \ 2;\ \ 1),\) \({\overline{a}}_{3} = (3;\ \ 1;\ \ 1).\) В базисе \(\overrightarrow{i},\overrightarrow{j},\overrightarrow{k}\) найти: 1)матрицу оператора \(A\) ; 2)образ вектора \(\overline{b} = (1;\ \ 2;\ \ 3).\) \\
A3. В пространстве \(\mathbb{C}^{3}\) со скалярным произведением \(\left\langle x,y \right\rangle = \sum_{k = 1}^{3}{x_{k}\overline{y_{k}}}\), найдите сопряженный оператор \(A^{*}\) для заданного оператора \(A\). Является ли \(A\)самосопряженным?\(Ax = \left( x_{1} - 2ix_{2},x_{3} + 2ix_{2},ix_{2} + 2ix_{3} \right)\);
 \\
B1. \(a_{1} = (1;0;2)\), \(a_{2} = (3; - 1;4)\), \(a_{3} = (2; - 2;1)\); \\
B2. 
Найти собственные значения и собственные векторы оператора \emph{А}, заданного в некотором базисе пространства \(V^{3}\) матрицей \(A = \begin{bmatrix}
0 & - 1 & 1 \\
 - 1 & 0 & 1 \\
1 & 1 & 0
\end{bmatrix}.\) \\
B3. Приведите квадратичные формы \(G_{1}\) и \(G_{2}\) к каноническому виду. \(G_{1} = 3x_{1}^{2} - 2x_{2}^{2} + 2x_{1}x_{3} - 4x_{2}x_{3}\), \(G_{2} = x_{1}x_{2} + x_{2}x_{3}\) \\
C1. В базисе \((e_{1},\ \ e_{2},\ \ e_{3})\) пространства \(V^{3}\) оператор \emph{A} имеет матрицу \(A = \begin{bmatrix}
 - 1 & 2 & 1 \\
0 & 1 & - 4 \\
5 & - 1 & 2
\end{bmatrix}\ \ .\) Найти матрицу \emph{B} этого же оператора в базисе \(({e'}_{1},\ \ {e'}_{2},\ \ {e'}_{3}),\) где \({e'}_{1} = e_{1} - e_{3},\) \({e'}_{2} = e_{2} + e_{3},\) \({e'}_{3} = e_{3}.\) \\
C2. 12.Даны векторы \(e_{1},e_{2},e_{3}\), \(a_{1},a_{2},a_{3}\) линейного пространства \(R^{3}\). Найдите матрицу перехода от базиса \(e_{1},e_{2},e_{3}\) к базису \(a_{1},a_{2},a_{3}\).
\(e_{1} = (1,1,1)\),\(e_{2} = (1,1,2)\),\(e_{3} = (1,2,3)\) и \(a_{1} = (2,0,1)\),\(a_{2} = ( - 1,2,3)\),\(a_{3} = ( - 1,1,1)\)
 \\
C3. Найти жорданову нормальную форму матрицы \(A = \begin{pmatrix}
2 & - 1 & - 1 \\
2 & - 1 & - 2 \\
 - 1 & 1 & 2
\end{pmatrix}\) \\

\end{tabular}
\vspace{1cm}


\begin{tabular}{m{17cm}}
\textbf{21-вариант}
\newline

T1. 5. Методы приведения квадратичной формы к каноническому форму. \\
T2. 16. Унитарные преобразования и их собственные значения и канонический вид. \\
A1. Доказать, что векторы \(\overrightarrow{a} = (1;\ \ 0;\ \ 1),\) \(\overrightarrow{b} = (1;\ \ 1;\ \ 1)\) и \(\overrightarrow{c} = ( - 1;\ \ 2;\ \ 1)\) образуют базис пространства \(\mathbf{R}^{3},\) и найти координаты вектора \(\overrightarrow{d} = (0;\ \ 10;\ \ 3)\) в этом базисе. \\
A2. Известно, что оператор \emph{A} переводит базисные векторы \(\overrightarrow{i} = (1;\ \ 0;\ \ 0),\) \(\overrightarrow{j} = (0;\ \ 1;\ \ 0),\) \(\overrightarrow{k} = (0;\ \ 0;\ \ 1)\) линейного пространства \(\mathbf{R}^{3}\) в векторы \({\overline{a}}_{1} = (1;\ \ 1;\ \ 0),\) \({\overline{a}}_{2} = (3;\ \ 2;\ \ 1),\) \({\overline{a}}_{3} = (1;2;\ \ 1).\) В базисе \(\overrightarrow{i},\overrightarrow{j},\overrightarrow{k}\) найти: 1)матрицу оператора \(A\) ; 2)образ вектора \(\overline{b} = (1;\ \ 1;\ \  - 2).\) \\
A3. В пространстве \(\mathbb{C}^{3}\) со скалярным произведением \(\left\langle x,y \right\rangle = \sum_{k = 1}^{3}{x_{k}\overline{y_{k}}}\), найдите сопряженный оператор \(A^{*}\) для заданного оператора \(A\). Является ли \(A\)самосопряженным?\(Ax = \left( x_{1} + 2ix_{3},2ix_{1} + ix_{2},x_{1} + ix_{3} \right)\); \\
B1. \(a_{1} = (2;4;3)\), \(a_{2} = (3; - 1;4)\), \(a_{3} = (1;5; - 1)\); \\
B2. Найти собственные значения и собственные векторы оператора \emph{А}, заданного в некотором базисе пространства \(V^{3}\) матрицей \(A = \begin{bmatrix}
2 & 1 & 0 \\
1 & 2 & 0 \\
0 & 0 & - 5
\end{bmatrix};\) \\
B3. Приведите квадратичные формы \(G_{1}\) и \(G_{2}\) к каноническому виду. \(G_{1} = 5x_{1}^{2} + 6x_{2}^{2} - 3x_{3}^{2} + 4x_{1}x_{2} - 2x_{2}x_{3}\), \(G_{2} = 6x_{2}x_{3} - x_{1}x_{2}\) \\
C1. В базисе \((e_{1},\ \ e_{2},\ \ e_{3})\) пространства \(V^{3}\) оператор \emph{A} имеет матрицу \(A = \begin{bmatrix}
2 & 0 & - 1 \\
3 & 2 & 0 \\
 - 1 & 4 & 3
\end{bmatrix}\ \ .\) Найти матрицу \emph{B} этого же оператора в базисе \(({e'}_{1},\ \ {e'}_{2},\ \ {e'}_{3}),\) где \({e'}_{1} = e_{1} - e_{3},\) \({e'}_{2} = e_{2} + e_{3},\) \({e'}_{3} = e_{3}.\) \\
C2. 
Даны векторы \(e_{1},e_{2},e_{3}\), \(a_{1},a_{2},a_{3}\) линейного пространства \(R^{3}\). Найдите матрицу перехода от базиса \(e_{1},e_{2},e_{3}\) к базису \(a_{1},a_{2},a_{3}\).
\(e_{1} = (2,1, - 3)\),\(e_{2} = (3,2, - 5)\),\(e_{3} = (1, - 1,1)\) и \(a_{1} = (0,1, - 2)\),\(a_{2} = ( - 2,0,3)\),\(a_{3} = (1, - 1,1)\) \\
C3. Найти жорданову нормальную форму матрицы \(A = \begin{pmatrix}
0 & 3 & 1 \\
3 & 0 & 1 \\
 - 2 & 2 & 1
\end{pmatrix}\) \\

\end{tabular}
\vspace{1cm}


\begin{tabular}{m{17cm}}
\textbf{22-вариант}
\newline

T1. 9. Линейные преобразования и их матрица. \\
T2. 6. Положительно определенные квадратичные формы. \\
A1. Доказать, что векторы \(\overrightarrow{a} = (2;\ \ 1;\ \ 1),\) \(\overrightarrow{b} = (1;\ \ 2;\ \ 1)\) и \(\overrightarrow{c} = (2;\ \ 3;\ \  - 1)\) образуют базис пространства \(\mathbf{R}^{3},\) и найти координаты вектора \(\overrightarrow{d} = (2;\ \ 3;\ \  - 1)\) в этом базисе. \\
A2. Известно, что оператор \emph{A} переводит базисные векторы \(\overrightarrow{i} = (1;\ \ 0;\ \ 0),\) \(\overrightarrow{j} = (0;\ \ 1;\ \ 0),\) \(\overrightarrow{k} = (0;\ \ 0;\ \ 1)\) линейного пространства \(\mathbf{R}^{3}\) в векторы \({\overline{a}}_{1} = (1;\ \ 0;\ \ 1),\) \({\overline{a}}_{2} = (0;\ \ 1;\ \ 1),\) \({\overline{a}}_{3} = (3;\ \ 1;\ \ 1).\) В базисе \(\overrightarrow{i},\overrightarrow{j},\overrightarrow{k}\) найти: 1)матрицу оператора \(A\) ; 2)образ вектора \(\overline{b} = (1;\ \  - 2;\ \  - 3).\) \\
A3. В пространстве \(\mathbb{C}^{3}\) со скалярным произведением \(\left\langle x,y \right\rangle = \sum_{k = 1}^{3}{x_{k}\overline{y_{k}}}\), найдите сопряженный оператор \(A^{*}\) для заданного оператора \(A\). Является ли \(A\)самосопряженным? \(Ax = \left( ix_{1} + x_{3},ix_{3} - ix_{2},x_{1} - ix_{3} \right)\); \\
B1. С помощью процесса ортогонализации Грамма- Шмидта ортонормировать следующие системы векторов, используя стандартное скалярное произведение: \\
B2. Найти собственные значения и собственные векторы оператора \emph{А}, заданного в некотором базисе пространства \(V^{3}\) матрицей \(A = \begin{bmatrix}
0 & - 1 & 1 \\
 - 1 & 0 & 1 \\
1 & 1 & 0
\end{bmatrix};\) \\
B3. Приведите квадратичные формы \(G_{1}\) и \(G_{2}\) к каноническому виду. \(G_{1} = x_{1}^{2} - 2x_{3}^{2} + 2x_{1}x_{3} - 6x_{1}x_{2}\), \(G_{2} = 6x_{2}x_{3} - 4x_{1}x_{2} + x_{1}x_{3}\)
 \\
C1. В базисе \((e_{1},\ \ e_{2},\ \ e_{3})\) пространства \(V^{3}\) оператор \emph{A} имеет матрицу \(A = \begin{bmatrix}
0 & 1 & - 2 \\
3 & 5 & 1 \\
 - 1 & 2 & 0
\end{bmatrix}\ \ .\) Найти матрицу \emph{B} этого же оператора в базисе \(({e'}_{1},\ \ {e'}_{2},\ \ {e'}_{3}),\) где \({e'}_{1} = e_{1} + 2e_{2},\) \({e'}_{2} = e_{1} - e_{3},\) \({e'}_{3} = e_{1} + e_{2} + e_{3}.\) \\
C2. Даны векторы \(e_{1},e_{2},e_{3}\), \(a_{1},a_{2},a_{3}\) линейного пространства \(R^{3}\). Найдите матрицу перехода от базиса \(e_{1},e_{2},e_{3}\) к базису \(a_{1},a_{2},a_{3}\).
\(e_{1} = (2,0,1)\),\(e_{2} = ( - 1,2,3)\),\(e_{3} = ( - 1,1,1)\) и \(a_{1} = ( - 3,0,1)\),\(a_{2} = (0,2,3)\),\(a_{3} = ( - 1, - 1, - 1)\) \\
C3. 
Найти жорданову нормальную форму матрицы \(A = \begin{pmatrix}
 - 1 & 1 & - 2 \\
3 & - 3 & 6 \\
2 & - 2 & 4
\end{pmatrix}\). \\

\end{tabular}
\vspace{1cm}


\begin{tabular}{m{17cm}}
\textbf{23-вариант}
\newline

T1. 11. Обратное преобразование. \\
T2. 14. Сопряженное преобразование для данного преобразования. \\
A1. Доказать, что векторы \(\overrightarrow{a} = (3;\ \ 5;\ \ 4),\) \(\overrightarrow{b} = (4;\ \ 3;\ \ 2)\) и \(\overrightarrow{c} = ( - 1;\ \  - 4;\ \ 3)\) образуют базис пространства \(\mathbf{R}^{3},\) и найти координаты вектора \(\overrightarrow{d} = ( - 2;\ \  - 2;\ \ 5)\) в этом базисе. \\
A2. Известно, что оператор \emph{A} переводит базисные векторы \(\overrightarrow{i} = (1;\ \ 0;\ \ 0),\) \(\overrightarrow{j} = (0;\ \ 1;\ \ 0),\) \(\overrightarrow{k} = (0;\ \ 0;\ \ 1)\) линейного пространства \(\mathbf{R}^{3}\) в векторы \({\overline{a}}_{1} = (1;\ \ 1;\ \ 1),{\overline{a}}_{2} = (3;\ \ 0;\ \ 1),\) \({\overline{a}}_{3} = (3;\ \ 1;\ \  - 1).\)В базисе \(\overrightarrow{i},\overrightarrow{j},\overrightarrow{k}\) найти:1)матрицу оператора \(A\) ;2)образ вектора \(\overline{b} = (1;\ \ 2;\ \ 1).\) \\
A3. В пространстве \(\mathbb{C}^{3}\) со скалярным произведением \(\left\langle x,y \right\rangle = \sum_{k = 1}^{3}{x_{k}\overline{y_{k}}}\), найдите сопряженный оператор \(A^{*}\) для заданного оператора \(A\). Является ли \(A\)самосопряженным?\(Ax = \left( x_{1} + 2ix_{2},x_{3} - ix_{2},x_{1} - ix_{2} - 2ix_{3} \right)\); \\
B1. С помощью процесса ортогонализации Грамма- Шмидта ортонормировать следующие системы векторов, используя стандартное скалярное произведение: \\
B2. Найти собственные значения и собственные векторы оператора \emph{А}, заданного в некотором базисе пространства \(V^{3}\) матрицей \(A = \begin{bmatrix}
0 & - 2 & 0 \\
 - 2 & 6 & - 2 \\
0 & - 2 & 5
\end{bmatrix};\) \\
B3. Приведите квадратичные формы \(G_{1}\) и \(G_{2}\) к каноническому виду. \(G_{1} = 2x_{1}^{2} + 6x_{2}^{2} - 4x_{3}^{2} - 2x_{1}x_{3} + 4x_{1}x_{2} - 8x_{2}x_{3}\), \(G_{2} = x_{2}x_{3} - 2x_{1}x_{3}\) \\
C1. В базисе \((e_{1},\ \ e_{2},\ \ e_{3})\) пространства \(V^{3}\) оператор \emph{A} имеет матрицу \(A = \begin{bmatrix}
3 & 2 & - 1 \\
4 & 0 & 2 \\
 - 1 & 2 & - 1
\end{bmatrix}\ \ .\) Найти матрицу \emph{B} этого же оператора в базисе \(({e'}_{1},\ \ {e'}_{2},\ \ {e'}_{3}),\) где \({e'}_{1} = 2e_{1} + e_{2}\), \({e'}_{2} = - e_{1} + 2e_{2} + 3e_{3}\),\({e'}_{3} = - e_{1} + e_{2} + e_{3}\) \\
C2. Даны векторы \(e_{1},e_{2},e_{3}\), \(a_{1},a_{2},a_{3}\) линейного пространства \(R^{3}\). Найдите матрицу перехода от базиса \(e_{1},e_{2},e_{3}\) к базису \(a_{1},a_{2},a_{3}\).
\(e_{1} = (3,1, - 1)\),\(e_{2} = ( - 2,0,1)\),\(e_{3} = (2,7,3)\) и \(a_{1} = (2,1, - 3)\),\(a_{2} = (3,2, - 5)\),\(a_{3} = (1, - 1,1)\) \\
C3. Найти жорданову нормальную форму матрицы \(A = \begin{pmatrix}
2 & 1 & 1 \\
1 & 2 & 1 \\
1 & 1 & 2
\end{pmatrix}\). \\

\end{tabular}
\vspace{1cm}


\begin{tabular}{m{17cm}}
\textbf{24-вариант}
\newline

T1. 15. Самосопряженные преобразования и их канонический вид. \\
T2. 8. Квадратичные формы в комплексном пространстве и их канонические виды. \\
A1. Доказать, что векторы \(\overrightarrow{a} = (1;\ \ 2;\ \ 1),\) \(\overrightarrow{b} = (1;\ \ 1;\ \ 3)\) и \(\overrightarrow{c} = ( - 1;\ \ 2;\ \ 1)\) образуют базис пространства \(\mathbf{R}^{3},\) и найти координаты вектора \(\overrightarrow{d} = (0;\ \ 10;\ \  - 2)\) в этом базисе. \\
A2. Известно, что оператор \emph{A} переводит базисные векторы \(\overrightarrow{i} = (1;\ \ 0;\ \ 0),\) \(\overrightarrow{j} = (0;\ \ 1;\ \ 0),\) \(\overrightarrow{k} = (0;\ \ 0;\ \ 1)\) линейного пространства \(\mathbf{R}^{3}\) в векторы \({\overline{a}}_{1} = (1;\ \ 0;\ \ 1),\) \({\overline{a}}_{2} = (3;\ \ 2;\ \ 1),\) \({\overline{a}}_{3} = (3;\ \ 1;\ \ 1).\) В базисе \(\overrightarrow{i},\overrightarrow{j},\overrightarrow{k}\) найти: 1)матрицу оператора \(A\) ; 2)образ вектора \(\overline{b} = (1;\ \  - 2;\ \ 3).\) \\
A3. В пространстве \(\mathbb{C}^{3}\) со скалярным произведением \(\left\langle x,y \right\rangle = \sum_{k = 1}^{3}{x_{k}\overline{y_{k}}}\), найдите сопряженный оператор \(A^{*}\) для заданного оператора \(A\). Является ли \(A\)самосопряженным? \(Ax = \left( x_{2} + ix_{3},x_{1} - ix_{2},x_{1} + ix_{2} + x_{3} \right)\) \\
B1. \(a_{1} = ( - 1;3;7)\),\(a_{2} = (0;2; - 1)\), \(a_{3} = (1; - 2; - 8)\); \\
B2. Найти собственные значения и собственные векторы оператора \emph{А}, заданного в некотором базисе пространства \(V^{3}\) матрицей \(A = \begin{pmatrix}
2 & - 1 & 2 \\
5 & - 3 & 3 \\
 - 1 & 0 & - 2
\end{pmatrix}\). \\
B3. Приведите квадратичные формы \(G_{1}\) и \(G_{2}\) к каноническому виду. \(G_{1} = 2x_{1}^{2} + 3x_{2}^{2} + 4x_{3}^{2} - 2x_{1}x_{2} + 4x_{1}x_{3} - 3x_{2}x_{3}\), \(G_{2} = x_{1}x_{3} - 2x_{2}x_{3}\) \\
C1. В базисе \((e_{1},\ \ e_{2},\ \ e_{3})\) пространства \(V^{3}\) оператор \emph{A} имеет матрицу \(A = \begin{bmatrix}
0 & 1 & - 3 \\
2 & 4 & 1 \\
0 & 3 & - 3
\end{bmatrix}\ \ .\) Найти матрицу \emph{B} этого же оператора в базисе \(({e'}_{1},\ \ {e'}_{2},\ \ {e'}_{3}),\) где \({e'}_{1} = 2e_{1} + e_{2}\), \({e'}_{2} = - e_{1} + 2e_{2} + 3e_{3}\),\({e'}_{3} = - e_{1} + e_{2} + e_{3}\) \\
C2. Даны векторы \(e_{1},e_{2},e_{3}\), \(a_{1},a_{2},a_{3}\) линейного пространства \(R^{3}\). Найдите матрицу перехода от базиса \(e_{1},e_{2},e_{3}\) к базису \(a_{1},a_{2},a_{3}\).
\(e_{1} = ( - 3,0,1)\),\(e_{2} = (0,2,3)\),\(e_{3} = ( - 1, - 1, - 1)\) и \(a_{1} = (1,1,1)\),\(a_{2} = (1,1,2)\),\(a_{3} = (1,2,3)\) \\
C3. Найти жорданову нормальную форму матрицы \(A = \begin{pmatrix}
 - 1 & 4 & 3 \\
 - 2 & 5 & 3 \\
2 & - 4 & - 2
\end{pmatrix}\). \\

\end{tabular}
\vspace{1cm}


\begin{tabular}{m{17cm}}
\textbf{25-вариант}
\newline

T1. 1. Линейные пространства. Линейные подпространства. Сумма и пересечение подпространств. \\
T2. 4. Линейные, билинейные, и квадратичные формы. Преобразование матрицы линейного вида при изменении базиса. \\
A1. Доказать, что векторы \(\overrightarrow{a} = (1;\ \ 2;\ \ 1),\) \(\overrightarrow{b} = (1;\ \ 1;\ \ 3)\) и \(\overrightarrow{c} = ( - 1;\ \ 2;\ \ 1)\) образуют базис пространства \(\mathbf{R}^{3},\) и найти координаты вектора \(\overrightarrow{d} = (0;\ \ 10;\ \  - 2)\) в этом базисе. \\
A2. Известно, что оператор \emph{A} переводит базисные векторы \(\overrightarrow{i} = (1;\ \ 0;\ \ 0),\) \(\overrightarrow{j} = (0;\ \ 1;\ \ 0),\) \(\overrightarrow{k} = (0;\ \ 0;\ \ 1)\) линейного пространства \(\mathbf{R}^{3}\) в векторы \({\overline{a}}_{1} = (1;\ \ 0;\ \ 1),\) \({\overline{a}}_{2} = (0;\ \ 1;\ \ 1),\) \({\overline{a}}_{3} = (3;\ \ 1;\ \ 1).\) В базисе \(\overrightarrow{i},\overrightarrow{j},\overrightarrow{k}\) найти: 1)матрицу оператора \(A\) ; 2)образ вектора \(\overline{b} = (1;\ \  - 2;\ \  - 3).\) \\
A3. В пространстве \(\mathbb{C}^{3}\) со скалярным произведением \(\left\langle x,y \right\rangle = \sum_{k = 1}^{3}{x_{k}\overline{y_{k}}}\), найдите сопряженный оператор \(A^{*}\) для заданного оператора \(A\). Является ли \(A\)самосопряженным?\(Ax = \left( x_{1} - 2ix_{2},x_{3} + 2ix_{2},ix_{2} + 2ix_{3} \right)\);
 \\
B1. \(a_{1} = (0;1; - 2)\),\(a_{2} = (1; - 1;1)\), \(a_{3} = ( - 2;0;3)\); \\
B2. Найти собственные значения и собственные векторы оператора \emph{А}, заданного в некотором базисе пространства \(V^{3}\) матрицей \(A = \begin{pmatrix}
1 & - 1 & 1 \\
1 & 1 & - 1 \\
2 & - 1 & 0
\end{pmatrix}\). \\
B3. Приведите квадратичные формы \(G_{1}\) и \(G_{2}\) к каноническому виду. \(G_{1} = 2x_{1}^{2} + 6x_{2}^{2} - 4x_{3}^{2} - 2x_{1}x_{3} + 4x_{1}x_{2} - 8x_{2}x_{3}\), \(G_{2} = x_{2}x_{3} - 2x_{1}x_{3}\) \\
C1. В базисе \((e_{1},\ \ e_{2},\ \ e_{3})\) пространства \(V^{3}\) оператор \emph{A} имеет матрицу \(A = \begin{bmatrix}
5 & - 2 & 1 \\
 - 1 & 0 & 4 \\
3 & 1 & 2
\end{bmatrix}\ \ .\) Найти матрицу \emph{B} этого же оператора в базисе \(({e'}_{1},\ \ {e'}_{2},\ \ {e'}_{3}),\) где \({e'}_{1} = 2e_{1} + 3e_{3},\) \({e'}_{2} = - e_{2},\) \({e'}_{3} = e_{1} + e_{2} + e_{3}.\) \\
C2. 
Даны векторы \(e_{1},e_{2},e_{3}\), \(a_{1},a_{2},a_{3}\) линейного пространства \(R^{3}\). Найдите матрицу перехода от базиса \(e_{1},e_{2},e_{3}\) к базису \(a_{1},a_{2},a_{3}\).
\(e_{1} = (2,1, - 3)\),\(e_{2} = (3,2, - 5)\),\(e_{3} = (1, - 1,1)\) и \(a_{1} = (0,1, - 2)\),\(a_{2} = ( - 2,0,3)\),\(a_{3} = (1, - 1,1)\) \\
C3. Найти жорданову нормальную форму матрицы \(A = \begin{pmatrix}
2 & 1 & 1 \\
1 & 2 & 1 \\
1 & 1 & 2
\end{pmatrix}\). \\

\end{tabular}
\vspace{1cm}


\begin{tabular}{m{17cm}}
\textbf{26-вариант}
\newline

T1. 19. Полиномиальные матрицы и диагональные нормальные формы. \\
T2. 10. Ядро, образ линйеного преобразования. \\
A1. Доказать, что векторы \(\overrightarrow{a} = (3;\ \ 4;\ \ 3),\) \(\overrightarrow{b} = ( - 2;\ \ 3;\ \ 1)\) и \(\overrightarrow{c} = (4;\ \  - 2;\ \ 3)\) образуют базис пространства \(\mathbf{R}^{3},\) и найти координаты вектора \(\overrightarrow{d} = ( - 17;\ \ 18;\ \  - 7)\) в этом базисе. \\
A2. Известно, что оператор \emph{A} переводит базисные векторы \(\overrightarrow{i} = (1;\ \ 0;\ \ 0),\) \(\overrightarrow{j} = (0;\ \ 1;\ \ 0),\) \(\overrightarrow{k} = (0;\ \ 0;\ \ 1)\) линейного пространства \(\mathbf{R}^{3}\) в векторы \({\overline{a}}_{1} = (1;\ \ 0;\ \ 1),\) \({\overline{a}}_{2} = (3;\ \ 2;\ \ 1),\) \({\overline{a}}_{3} = (3;\ \ 1;\ \ 1).\) В базисе \(\overrightarrow{i},\overrightarrow{j},\overrightarrow{k}\) найти: 1)матрицу оператора \(A\) ; 2)образ вектора \(\overline{b} = (1;\ \  - 2;\ \ 3).\) \\
A3. В пространстве \(\mathbb{C}^{3}\) со скалярным произведением \(\left\langle x,y \right\rangle = \sum_{k = 1}^{3}{x_{k}\overline{y_{k}}}\), найдите сопряженный оператор \(A^{*}\) для заданного оператора \(A\). Является ли \(A\)самосопряженным?\(Ax = \left( x_{1} + 2ix_{3},2ix_{1} + ix_{2},x_{1} + ix_{3} \right)\); \\
B1. С помощью процесса ортогонализации Грамма- Шмидта ортонормировать следующие системы векторов, используя стандартное скалярное произведение: \\
B2. Найти собственные значения и собственные векторы оператора \emph{А}, заданного в некотором базисе пространства \(V^{3}\) матрицей \(A = \begin{pmatrix}
0 & 1 & 2 \\
 - 1 & 0 & - 2 \\
 - 2 & 2 & 0
\end{pmatrix}\).
 \\
B3. Приведите квадратичные формы \(G_{1}\) и \(G_{2}\) к каноническому виду. \(G_{1} = 2x_{1}^{2} + 3x_{2}^{2} + 4x_{3}^{2} - 2x_{1}x_{2} + 4x_{1}x_{3} - 3x_{2}x_{3}\), \(G_{2} = x_{1}x_{3} - 2x_{2}x_{3}\) \\
C1. 
В базисе \((e_{1},\ \ e_{2},\ \ e_{3})\) пространства \(V^{3}\) оператор \emph{A} имеет матрицу \(A = \begin{bmatrix}
1 & 2 & 3 \\
0 & 1 & 2 \\
3 & 1 & 2
\end{bmatrix}\ \ .\) Найти матрицу \emph{B} этого же оператора в базисе \(({e'}_{1},\ \ {e'}_{2},\ \ {e'}_{3}),\) где \({e'}_{1} = e_{1} + 2e_{2},\) \({e'}_{2} = e_{1} - e_{3},\) \({e'}_{3} = e_{1} + e_{2} + e_{3}.\) \\
C2. 10.Даны векторы \(e_{1},e_{2},e_{3}\), \(a_{1},a_{2},a_{3}\) линейного пространства \(R^{3}\). Найдите матрицу перехода от базиса \(e_{1},e_{2},e_{3}\) к базису \(a_{1},a_{2},a_{3}\).
\(e_{1} = (4,0,5)\),\(e_{2} = ( - 2,1,3)\),\(e_{3} = ( - 5,1, - 1)\) и \(a_{1} = (1,2,1)\),\(a_{2} = (2,3,3)\),\(a_{3} = (3,8,2)\) \\
C3. Найти жорданову нормальную форму матрицы \(A = \begin{pmatrix}
 - 1 & 4 & 3 \\
 - 2 & 5 & 3 \\
2 & - 4 & - 2
\end{pmatrix}\). \\

\end{tabular}
\vspace{1cm}


\begin{tabular}{m{17cm}}
\textbf{27-вариант}
\newline

T1. 3. Ортогональное дополнение и ортогональная проекция. \\
T2. 12. Связь между матрицами линейных преобразовании в разных базисах. \\
A1. Доказать, что векторы \(\overrightarrow{a} = (1;\ \ 2;\ \ 1),\) \(\overrightarrow{b} = (1;\ \ 1;\ \  - 3)\) и \(\overrightarrow{c} = ( - 1;\ \ 2;\ \ 1)\) образуют базис пространства \(\mathbf{R}^{3},\) и найти координаты вектора \(\overrightarrow{d} = (0;\ \ 10;\ \  - 2)\) в этом базисе. \\
A2. Известно, что оператор \emph{A} переводит базисные векторы \(\overrightarrow{i} = (1;\ \ 0;\ \ 0),\) \(\overrightarrow{j} = (0;\ \ 1;\ \ 0),\) \(\overrightarrow{k} = (0;\ \ 0;\ \ 1)\) линейного пространства \(\mathbf{R}^{3}\) в векторы \({\overline{a}}_{1} = (1;\ \ 1;\ \ 1),\) \({\overline{a}}_{2} = (3;\ \ 2;\ \ 1),\) \({\overline{a}}_{3} = (0;\ \ 1;\ \ 1).\) В базисе \(\overrightarrow{i},\overrightarrow{j},\overrightarrow{k}\) найти: 1)матрицу оператора \(A\) ; 2)образ вектора \(\overline{b} = (1;\ \  - 2;\ \ 3).\)
 \\
A3. В пространстве \(\mathbb{C}^{3}\) со скалярным произведением \(\left\langle x,y \right\rangle = \sum_{k = 1}^{3}{x_{k}\overline{y_{k}}}\), найдите сопряженный оператор \(A^{*}\) для заданного оператора \(A\). Является ли \(A\)самосопряженным? \(Ax = \left( x_{1} + ix_{3},x_{3} + 2ix_{2},ix_{2} - 2ix_{3} \right)\); \\
B1. \(a_{1} = (2;0;1)\), \(a_{2} = ( - 1;2;3)\), \(a_{3} = ( - 1;1;1)\); \\
B2. Найти собственные значения и собственные векторы оператора \emph{А}, заданного в некотором базисе пространства \(V^{3}\) матрицей \(A = \begin{pmatrix}
1 & - 2 & - 1 \\
 - 1 & 1 & 1 \\
1 & 0 & - 1
\end{pmatrix}\) \\
B3. Приведите квадратичные формы \(G_{1}\) и \(G_{2}\) к каноническому виду. \(G_{1} = 3x_{1}^{2} - 2x_{2}^{2} + 2x_{1}x_{3} - 4x_{2}x_{3}\), \(G_{2} = x_{1}x_{2} + x_{2}x_{3}\) \\
C1. В базисе \((e_{1},\ \ e_{2},\ \ e_{3})\) пространства \(V^{3}\) оператор \emph{A} имеет матрицу \(A = \begin{bmatrix}
1 & - 1 & 2 \\
0 & 3 & - 1 \\
4 & 2 & 2
\end{bmatrix}\ \ .\) Найти матрицу \emph{B} этого же оператора в базисе \(({e'}_{1},\ \ {e'}_{2},\ \ {e'}_{3}),\) где \({e'}_{1} = e_{1} + 2e_{2},\) \({e'}_{2} = e_{1} - e_{3},\) \({e'}_{3} = e_{1} + e_{2} + e_{3}.\) \\
C2. 11.Даны векторы \(e_{1},e_{2},e_{3}\), \(a_{1},a_{2},a_{3}\) линейного пространства \(R^{3}\). Найдите матрицу перехода от базиса \(e_{1},e_{2},e_{3}\) к базису \(a_{1},a_{2},a_{3}\).
\(e_{1} = ( - 1,3,7)\),\(e_{2} = (0,2, - 1)\),\(e_{3} = (1, - 2, - 8)\) и \(a_{1} = (0,3, - 2)\),\(a_{2} = (1, - 1, - 8)\),\(a_{3} = ( - 1,2,7)\) \\
C3. Найти жорданову нормальную форму матрицы \(A = \begin{pmatrix}
2 & - 1 & - 1 \\
2 & - 1 & - 2 \\
 - 1 & 1 & 2
\end{pmatrix}\) \\

\end{tabular}
\vspace{1cm}


\begin{tabular}{m{17cm}}
\textbf{28-вариант}
\newline

T1. 17. Взаимозаменяемые преобразования. \\
T2. 18. Нормальные преобразования и их канонический вид. \\
A1. Доказать, что векторы \(\overrightarrow{a} = (2;\ \ 1;\ \  - 3),\) \(\overrightarrow{b} = ( - 1;\ \ 2;\ \ 4)\) и \(\overrightarrow{c} = (3;\ \  - 4;\ \ 2)\) образуют базис пространства \(\mathbf{R}^{3},\) и найти координаты вектора \(\overrightarrow{d} = ( - 4;\ \ 19;\ \ 3)\) в этом базисе. \\
A2. 
Известно, что оператор \emph{A} переводит базисные векторы \(\overrightarrow{i} = (1;\ \ 0;\ \ 0),\) \(\overrightarrow{j} = (0;\ \ 1;\ \ 0),\) \(\overrightarrow{k} = (0;\ \ 0;\ \ 1)\) линейного пространства \(\mathbf{R}^{3}\) в векторы \({\overline{a}}_{1} = (1;1;0),\) \({\overline{a}}_{2} = (3;\ \ 2;\ \ 1),\) \({\overline{a}}_{3} = (0;\ \ 1;\ \ 1).\) В базисе \(\overrightarrow{i},\overrightarrow{j},\overrightarrow{k}\) найти: 1)матрицу оператора \(A\) ; 2)образ вектора \(\overline{b} = (1;\ \  - 2;\ \  - 3).\) \\
A3. В пространстве \(\mathbb{C}^{3}\) со скалярным произведением \(\left\langle x,y \right\rangle = \sum_{k = 1}^{3}{x_{k}\overline{y_{k}}}\), найдите сопряженный оператор \(A^{*}\) для заданного оператора \(A\). Является ли \(A\)самосопряженным? \(Ax = \left( 3ix_{1} + x_{2},x_{1} + 2ix_{2},ix_{2} - x_{3} \right)\); \\
B1. С помощью процесса ортогонализации Грамма- Шмидта ортонормировать следующие системы векторов, используя стандартное скалярное произведение: \\
B2. Найти собственные значения и собственные векторы оператора \emph{А}, заданного в некотором базисе пространства \(V^{3}\) матрицей \(A = \begin{pmatrix}
2 & - 1 & 2 \\
1 & 0 & 2 \\
 - 2 & 1 & - 1
\end{pmatrix}\) \\
B3. Приведите квадратичные формы \(G_{1}\) и \(G_{2}\) к каноническому виду. \(G_{1} = x_{1}^{2} - 3x_{3}^{2} - 2x_{1}x_{2} - 2x_{1}x_{3} - 6x_{2}x_{3}\), \(G_{2} = 2x_{1}x_{2} - x_{1}x_{3} + 2x_{2}x_{3}\) \\
C1. В базисе \((e_{1},\ \ e_{2},\ \ e_{3})\) пространства \(V^{3}\) оператор \emph{A} имеет матрицу \(A = \begin{bmatrix}
 - 3 & 1 & 4 \\
0 & 3 & 2 \\
 - 5 & - 1 & 2
\end{bmatrix}\ \ .\) Найти матрицу \emph{B} этого же оператора в базисе \(({e'}_{1},\ \ {e'}_{2},\ \ {e'}_{3}),\) где \({e'}_{1} = 2e_{1} + e_{2}\), \({e'}_{2} = - e_{1} + 2e_{2} + 3e_{3}\),\({e'}_{3} = - e_{1} + e_{2} + e_{3}\) \\
C2. Даны векторы \(e_{1},e_{2},e_{3}\), \(a_{1},a_{2},a_{3}\) линейного пространства \(R^{3}\). Найдите матрицу перехода от базиса \(e_{1},e_{2},e_{3}\) к базису \(a_{1},a_{2},a_{3}\).
\(e_{1} = (1,0,2)\),\(e_{2} = (3, - 1,4)\),\(e_{3} = (2, - 2,1)\) и \(a_{1} = (4,0,5)\),\(a_{2} = ( - 2,1,3)\),\(a_{3} = ( - 5,1, - 1)\) \\
C3. 
Найти жорданову нормальную форму матрицы \(A = \begin{pmatrix}
 - 1 & 1 & - 2 \\
3 & - 3 & 6 \\
2 & - 2 & 4
\end{pmatrix}\). \\

\end{tabular}
\vspace{1cm}


\begin{tabular}{m{17cm}}
\textbf{29-вариант}
\newline

T1. 13. Инвариантные подпространства. Собственные векторы и собственные значения. \\
T2. 20. Подобные матрицы. \\
A1. Доказать, что векторы \(\overrightarrow{a} = (1;\ \ 2;\ \ 1),\) \(\overrightarrow{b} = (1;\ \ 1;\ \ 3)\) и \(\overrightarrow{c} = ( - 1;\ \ 2;\ \ 1)\) образуют базис пространства \(\mathbf{R}^{3},\) и найти координаты вектора \(\overrightarrow{d} = (0;\ \ 10;\ \  - 2)\) в этом базисе. \\
A2. Известно, что оператор \emph{A} переводит базисные векторы \(\overrightarrow{i} = (1;\ \ 0;\ \ 0),\) \(\overrightarrow{j} = (0;\ \ 1;\ \ 0),\) \(\overrightarrow{k} = (0;\ \ 0;\ \ 1)\) линейного пространства \(\mathbf{R}^{3}\) в векторы \({\overline{a}}_{1} = (1;\ \ 1;\ \ 1),{\overline{a}}_{2} = (3;\ \ 0;\ \ 1),\) \({\overline{a}}_{3} = (3;\ \ 1;\ \  - 1).\)В базисе \(\overrightarrow{i},\overrightarrow{j},\overrightarrow{k}\) найти:1)матрицу оператора \(A\) ;2)образ вектора \(\overline{b} = (1;\ \ 2;\ \ 1).\) \\
A3. В пространстве \(\mathbb{C}^{3}\) со скалярным произведением \(\left\langle x,y \right\rangle = \sum_{k = 1}^{3}{x_{k}\overline{y_{k}}}\), найдите сопряженный оператор \(A^{*}\) для заданного оператора \(A\). Является ли \(A\)самосопряженным? \(Ax = \left( ix_{1} + x_{3},ix_{3} - ix_{2},x_{1} - ix_{3} \right)\); \\
B1. \(a_{1} = ( - 2;3;1)\),\(a_{2} = (0;2;1)\), \(a_{3} = (1;2;1)\); \\
B2. Найти собственные значения и собственные векторы оператора \emph{А}, заданного в некотором базисе пространства \(V^{3}\) матрицей \(A = \begin{pmatrix}
2 & 1 & 0 \\
1 & 3 & - 1 \\
 - 1 & 2 & 3
\end{pmatrix}\). \\
B3. 
Приведите квадратичные формы \(G_{1}\) и \(G_{2}\) к каноническому виду. \(G_{1} = x_{1}^{2} + x_{2}^{2} + 3x_{3}^{2} + 4x_{1}x_{2} + 2x_{1}x_{3} + 2x_{2}x_{3}\), \(G_{2} = x_{1}x_{2} + x_{1}x_{3} + x_{2}x_{3}\) \\
C1. В базисе \((e_{1},\ \ e_{2},\ \ e_{3})\) пространства \(V^{3}\) оператор \emph{A} имеет матрицу \(A = \begin{bmatrix}
 - 1 & 2 & 4 \\
 - 4 & 2 & 0 \\
3 & 3 & - 3
\end{bmatrix}\ \ .\) Найти матрицу \emph{B} этого же оператора в базисе \(({e'}_{1},\ \ {e'}_{2},\ \ {e'}_{3}),\) где \({e'}_{1} = 2e_{1} + e_{2}\), \({e'}_{2} = - e_{1} + 2e_{2} + 3e_{3}\),\({e'}_{3} = - e_{1} + e_{2} + e_{3}\) \\
C2. Даны векторы \(e_{1},e_{2},e_{3}\), \(a_{1},a_{2},a_{3}\) линейного пространства \(R^{3}\). Найдите матрицу перехода от базиса \(e_{1},e_{2},e_{3}\) к базису \(a_{1},a_{2},a_{3}\).
\(e_{1} = (3,1, - 1)\),\(e_{2} = ( - 2,0,1)\),\(e_{3} = (2,7,3)\) и \(a_{1} = (2,1, - 3)\),\(a_{2} = (3,2, - 5)\),\(a_{3} = (1, - 1,1)\) \\
C3. Найти жорданову нормальную форму матрицы \(A = \begin{pmatrix}
3 & - 2 & 6 \\
 - 2 & 6 & 3 \\
6 & 3 & - 2
\end{pmatrix}\). \\

\end{tabular}
\vspace{1cm}


\begin{tabular}{m{17cm}}
\textbf{30-вариант}
\newline

T1. 7. Комплексные евклидовы пространства. \\
T2. 2. Евклидово пространство. Неравенство Коши-Буняковского. Процесс ортогонализации. \\
A1. Доказать, что векторы \(\overrightarrow{a} = (2;\ \ 1;1),\) \(\overrightarrow{b} = ( - 1;\ \ 2;\ \ 4)\) и \(\overrightarrow{c} = (3;\ \ 3;\ \ 2)\) образуют базис пространства \(\mathbf{R}^{3},\) и найти координаты вектора \(\overrightarrow{d} = ( - 4;\ \ 2;\ \ 4)\) в этом базисе. \\
A2. Известно, что оператор \emph{A} переводит базисные векторы \(\overrightarrow{i} = (1;\ \ 0;\ \ 0),\) \(\overrightarrow{j} = (0;\ \ 1;\ \ 0),\) \(\overrightarrow{k} = (0;\ \ 0;\ \ 1)\) линейного пространства \(\mathbf{R}^{3}\) в векторы \({\overline{a}}_{1} = (0;\ \ 1;\ \ 1),\) \({\overline{a}}_{2} = (3;\ \ 1;\ \ 1),\) \({\overline{a}}_{3} = (3;0;1).\) В базисе \(\overrightarrow{i},\overrightarrow{j},\overrightarrow{k}\) найти: 1)матрицу оператора \(A\) ; 2)образ вектора \(\overline{b} = (1;\ \  - 1;\ \  - 1).\) \\
A3. В пространстве \(\mathbb{C}^{3}\) со скалярным произведением \(\left\langle x,y \right\rangle = \sum_{k = 1}^{3}{x_{k}\overline{y_{k}}}\), найдите сопряженный оператор \(A^{*}\) для заданного оператора \(A\). Является ли \(A\)самосопряженным? \(Ax = \left( x_{1} + 2ix_{3},ix_{2} - x_{3},x_{2} - ix_{3} \right)\); \\
B1. \(a_{1} = (1;0;2)\), \(a_{2} = (3; - 1;4)\), \(a_{3} = (2; - 2;1)\); \\
B2. Найти собственные значения и собственные векторы оператора \emph{А}, заданного в некотором базисе пространства \(V^{3}\) матрицей \(A = \begin{pmatrix}
2 & - 1 & 2 \\
5 & - 3 & 3 \\
 - 1 & 0 & - 2
\end{pmatrix}\). \\
B3. Приведите квадратичные формы \(G_{1}\) и \(G_{2}\) к каноническому виду. \(G_{1} = 3x_{1}^{2} - 2x_{2}^{2} + 2x_{3}^{2} + 4x_{1}x_{2} - 3x_{1}x_{3} - x_{2}x_{3}\), \(G_{2} = 2x_{1}x_{3} + 4x_{1}x_{2} - 2x_{2}x_{3}\) \\
C1. В базисе \((e_{1},\ \ e_{2},\ \ e_{3})\) пространства \(V^{3}\) оператор \emph{A} имеет матрицу \(A = \begin{bmatrix}
0 & 1 & - 2 \\
3 & 5 & 1 \\
 - 1 & 2 & 0
\end{bmatrix}\ \ .\) Найти матрицу \emph{B} этого же оператора в базисе \(({e'}_{1},\ \ {e'}_{2},\ \ {e'}_{3}),\) где \({e'}_{1} = e_{1} + 2e_{2},\) \({e'}_{2} = e_{1} - e_{3},\) \({e'}_{3} = e_{1} + e_{2} + e_{3}.\) \\
C2. Даны векторы \(e_{1},e_{2},e_{3}\), \(a_{1},a_{2},a_{3}\) линейного пространства \(R^{3}\). Найдите матрицу перехода от базиса \(e_{1},e_{2},e_{3}\) к базису \(a_{1},a_{2},a_{3}\).
\(e_{1} = (0,1, - 2)\),\(e_{2} = ( - 2,0,3)\),\(e_{3} = (1, - 1,1)\) и \(a_{1} = (3,1, - 1)\),\(a_{2} = ( - 2,0,1)\),\(a_{3} = (2,7,3)\) \\
C3. Найти жорданову нормальную форму матрицы \(A = \begin{pmatrix}
0 & 3 & 1 \\
3 & 0 & 1 \\
 - 2 & 2 & 1
\end{pmatrix}\) \\

\end{tabular}
\vspace{1cm}


\begin{tabular}{m{17cm}}
\textbf{31-вариант}
\newline

T1. 15. Самосопряженные преобразования и их канонический вид. \\
T2. 2. Евклидово пространство. Неравенство Коши-Буняковского. Процесс ортогонализации. \\
A1. Доказать, что векторы \(\overrightarrow{a} = (3;\ \ 1;\ \ 2),\) \(\overrightarrow{b} = (2;\ \  - 3;\ \ 1)\) и \(\overrightarrow{c} = (4;\ \  - 2;\ \ 3)\) образуют базис пространства \(\mathbf{R}^{3},\) и найти координаты вектора \(\overrightarrow{d} = ( - 7;\ \ 8;\ \ 7)\) в этом базисе. \\
A2. Известно, что оператор \emph{A} переводит базисные векторы \(\overrightarrow{i} = (1;\ \ 0;\ \ 0),\) \(\overrightarrow{j} = (0;\ \ 1;\ \ 0),\) \(\overrightarrow{k} = (0;\ \ 0;\ \ 1)\) линейного пространства \(\mathbf{R}^{3}\) в векторы \({\overline{a}}_{1} = (1;\ \ 1;\ \ 0),\) \({\overline{a}}_{2} = (3;\ \ 2;\ \ 1),\) \({\overline{a}}_{3} = (1;2;\ \ 1).\) В базисе \(\overrightarrow{i},\overrightarrow{j},\overrightarrow{k}\) найти: 1)матрицу оператора \(A\) ; 2)образ вектора \(\overline{b} = (1;\ \ 1;\ \  - 2).\) \\
A3. В пространстве \(\mathbb{C}^{3}\) со скалярным произведением \(\left\langle x,y \right\rangle = \sum_{k = 1}^{3}{x_{k}\overline{y_{k}}}\), найдите сопряженный оператор \(A^{*}\) для заданного оператора \(A\). Является ли \(A\)самосопряженным?\(Ax = \left( x_{1} + 2ix_{2},x_{3} - ix_{2},x_{1} - ix_{2} - 2ix_{3} \right)\); \\
B1. С помощью процесса ортогонализации Грамма- Шмидта ортонормировать следующие системы векторов, используя стандартное скалярное произведение: \\
B2. Найти собственные значения и собственные векторы оператора \emph{А}, заданного в некотором базисе пространства \(V^{3}\) матрицей \(A = \begin{bmatrix}
0 & - 2 & 0 \\
 - 2 & 6 & - 2 \\
0 & - 2 & 5
\end{bmatrix};\) \\
B3. Приведите квадратичные формы \(G_{1}\) и \(G_{2}\) к каноническому виду. \(G_{1} = 5x_{1}^{2} + 6x_{2}^{2} - 3x_{3}^{2} + 4x_{1}x_{2} - 2x_{2}x_{3}\), \(G_{2} = 6x_{2}x_{3} - x_{1}x_{2}\) \\
C1. В базисе \((e_{1},\ \ e_{2},\ \ e_{3})\) пространства \(V^{3}\) оператор \emph{A} имеет матрицу \(A = \begin{bmatrix}
 - 1 & 2 & 1 \\
0 & 1 & - 4 \\
5 & - 1 & 2
\end{bmatrix}\ \ .\) Найти матрицу \emph{B} этого же оператора в базисе \(({e'}_{1},\ \ {e'}_{2},\ \ {e'}_{3}),\) где \({e'}_{1} = e_{1} - e_{3},\) \({e'}_{2} = e_{2} + e_{3},\) \({e'}_{3} = e_{3}.\) \\
C2. Даны векторы \(e_{1},e_{2},e_{3}\), \(a_{1},a_{2},a_{3}\) линейного пространства \(R^{3}\). Найдите матрицу перехода от базиса \(e_{1},e_{2},e_{3}\) к базису \(a_{1},a_{2},a_{3}\).
\(e_{1} = ( - 3,0,1)\),\(e_{2} = (0,2,3)\),\(e_{3} = ( - 1, - 1, - 1)\) и \(a_{1} = (1,1,1)\),\(a_{2} = (1,1,2)\),\(a_{3} = (1,2,3)\) \\
C3. Найти жорданову нормальную форму матрицы \(A = \begin{pmatrix}
1 & 2 & 1 \\
1 & 2 & 4 \\
 - 1 & - 2 & - 3
\end{pmatrix}\). \\

\end{tabular}
\vspace{1cm}


\begin{tabular}{m{17cm}}
\textbf{32-вариант}
\newline

T1. 1. Линейные пространства. Линейные подпространства. Сумма и пересечение подпространств. \\
T2. 8. Квадратичные формы в комплексном пространстве и их канонические виды. \\
A1. Доказать, что векторы \(\overrightarrow{a} = (3;\ \ 1;\ \ 0),\) \(\overrightarrow{b} = (4;\ \ 3;\ \ 2)\) и \(\overrightarrow{c} = ( - 1;\ \  - 4;\ \ 3)\) образуют базис пространства \(\mathbf{R}^{3},\) и найти координаты вектора \(\overrightarrow{d} = ( - 1;\ \ 2;\ \ 5)\) в этом базисе.
 \\
A2. Известно, что оператор \emph{A} переводит базисные векторы \(\overrightarrow{i} = (1;\ \ 0;\ \ 0),\) \(\overrightarrow{j} = (0;\ \ 1;\ \ 0),\) \(\overrightarrow{k} = (0;\ \ 0;\ \ 1)\) линейного пространства \(\mathbf{R}^{3}\) в векторы \({\overline{a}}_{1} = (1;\ \ 1;\ \ 1),{\overline{a}}_{2} = (3;\ \ 0;\ \ 1),\) \({\overline{a}}_{3} = (0;\ \ 2;\ \ 1).\)В базисе \(\overrightarrow{i},\overrightarrow{j},\overrightarrow{k}\) найти:1)матрицу оператора \(A\) ;2)образ вектора \(\overline{b} = (1;\ \ 2;\ \  - 2).\) \\
A3. В пространстве \(\mathbb{C}^{3}\) со скалярным произведением \(\left\langle x,y \right\rangle = \sum_{k = 1}^{3}{x_{k}\overline{y_{k}}}\), найдите сопряженный оператор \(A^{*}\) для заданного оператора \(A\). Является ли \(A\)самосопряженным? \(Ax = \left( ix_{1} + x_{2},x_{1} + ix_{2},x_{2} + ix_{3} \right)\); \\
B1. С помощью процесса ортогонализации Грамма- Шмидта ортонормировать следующие системы векторов, используя стандартное скалярное произведение: \\
B2. Найти собственные значения и собственные векторы оператора \emph{А}, заданного в некотором базисе пространства \(V^{3}\) матрицей \(A = \begin{bmatrix}
2 & 1 & 0 \\
1 & 2 & 0 \\
0 & 0 & - 5
\end{bmatrix};\) \\
B3. Приведите квадратичные формы \(G_{1}\) и \(G_{2}\) к каноническому виду. \(G_{1} = 2x_{1}^{2} + x_{2}^{2} + x_{3}^{2} + 4x_{1}x_{2} - 2x_{1}x_{3}\), \(G_{2} = x_{1}x_{2} + x_{1}x_{3} + 4x_{2}x_{3}\) \\
C1. В базисе \((e_{1},\ \ e_{2},\ \ e_{3})\) пространства \(V^{3}\) оператор \emph{A} имеет матрицу \(A = \begin{bmatrix}
3 & 2 & - 1 \\
4 & 0 & 2 \\
 - 1 & 2 & - 1
\end{bmatrix}\ \ .\) Найти матрицу \emph{B} этого же оператора в базисе \(({e'}_{1},\ \ {e'}_{2},\ \ {e'}_{3}),\) где \({e'}_{1} = 2e_{1} + e_{2}\), \({e'}_{2} = - e_{1} + 2e_{2} + 3e_{3}\),\({e'}_{3} = - e_{1} + e_{2} + e_{3}\) \\
C2. Даны векторы \(e_{1},e_{2},e_{3}\), \(a_{1},a_{2},a_{3}\) линейного пространства \(R^{3}\). Найдите матрицу перехода от базиса \(e_{1},e_{2},e_{3}\) к базису \(a_{1},a_{2},a_{3}\).
\(e_{1} = ( - 2,3,1)\),\(e_{2} = (0,2,1)\),\(e_{3} = (1,2,1)\) и \(a_{1} = ( - 1,3,7)\),\(a_{2} = (0,2, - 1)\),\(a_{3} = (1, - 2, - 8)\) \\
C3. Найти жорданову нормальную форму матрицы \(A = \begin{pmatrix}
2 & - 1 & - 1 \\
2 & - 1 & - 2 \\
 - 1 & 1 & 2
\end{pmatrix}\). \\

\end{tabular}
\vspace{1cm}


\begin{tabular}{m{17cm}}
\textbf{33-вариант}
\newline

T1. 11. Обратное преобразование. \\
T2. 10. Ядро, образ линйеного преобразования. \\
A1. Доказать, что векторы \(\overrightarrow{a} = (3;\ \ 5;\ \ 4),\) \(\overrightarrow{b} = (4;\ \ 3;\ \ 2)\) и \(\overrightarrow{c} = ( - 1;\ \  - 4;\ \ 3)\) образуют базис пространства \(\mathbf{R}^{3},\) и найти координаты вектора \(\overrightarrow{d} = ( - 2;\ \  - 2;\ \ 5)\) в этом базисе. \\
A2. Известно, что оператор \emph{A} переводит базисные векторы \(\overrightarrow{i} = (1;\ \ 0;\ \ 0),\) \(\overrightarrow{j} = (0;\ \ 1;\ \ 0),\) \(\overrightarrow{k} = (0;\ \ 0;\ \ 1)\) линейного пространства \(\mathbf{R}^{3}\) в векторы \({\overline{a}}_{1} = (0;\ \ 1;\ \ 1),\) \({\overline{a}}_{2} = (3;\ \ 1;\ \ 1),\) \({\overline{a}}_{3} = (3;\ \ 1;\ \ 1).\) В базисе \(\overrightarrow{i},\overrightarrow{j},\overrightarrow{k}\) найти: 1)матрицу оператора \(A\) ; 2)образ вектора \(\overline{b} = (1;\ \ 1;\ \ 1).\) \\
A3. В пространстве \(\mathbb{C}^{3}\) со скалярным произведением \(\left\langle x,y \right\rangle = \sum_{k = 1}^{3}{x_{k}\overline{y_{k}}}\), найдите сопряженный оператор \(A^{*}\) для заданного оператора \(A\). Является ли \(A\)самосопряженным? \(Ax = \left( ix_{1} + 2ix_{3},x_{3},x_{1} - 2ix_{3} \right)\); \\
B1. \(a_{1} = ( - 3;0;1)\), \(a_{2} = (0;2;3)\), \(a_{3} = ( - 1; - 1; - 1)\); \\
B2. 
Найти собственные значения и собственные векторы оператора \emph{А}, заданного в некотором базисе пространства \(V^{3}\) матрицей \(A = \begin{bmatrix}
0 & - 1 & 1 \\
 - 1 & 0 & 1 \\
1 & 1 & 0
\end{bmatrix}.\) \\
B3. Приведите квадратичные формы \(G_{1}\) и \(G_{2}\) к каноническому виду. \(G_{1} = 4x_{1}^{2} + x_{2}^{2} + x_{3}^{2} - 4x_{1}x_{2} + 4x_{1}x_{3} - 3x_{2}x_{3}\), \(G_{2} = x_{1}x_{2} + 6x_{1}x_{3} - 4x_{2}x_{3}\) \\
C1. В базисе \((e_{1},\ \ e_{2},\ \ e_{3})\) пространства \(V^{3}\) оператор \emph{A} имеет матрицу \(A = \begin{bmatrix}
0 & 1 & - 3 \\
2 & 4 & 1 \\
0 & 3 & - 3
\end{bmatrix}\ \ .\) Найти матрицу \emph{B} этого же оператора в базисе \(({e'}_{1},\ \ {e'}_{2},\ \ {e'}_{3}),\) где \({e'}_{1} = 2e_{1} + e_{2}\), \({e'}_{2} = - e_{1} + 2e_{2} + 3e_{3}\),\({e'}_{3} = - e_{1} + e_{2} + e_{3}\) \\
C2. 12.Даны векторы \(e_{1},e_{2},e_{3}\), \(a_{1},a_{2},a_{3}\) линейного пространства \(R^{3}\). Найдите матрицу перехода от базиса \(e_{1},e_{2},e_{3}\) к базису \(a_{1},a_{2},a_{3}\).
\(e_{1} = (1,1,1)\),\(e_{2} = (1,1,2)\),\(e_{3} = (1,2,3)\) и \(a_{1} = (2,0,1)\),\(a_{2} = ( - 1,2,3)\),\(a_{3} = ( - 1,1,1)\)
 \\
C3. Найти жорданову нормальную форму матрицы \(A = \begin{pmatrix}
 - 1 & 4 & 3 \\
 - 2 & 5 & 3 \\
2 & - 4 & - 2
\end{pmatrix}\). \\

\end{tabular}
\vspace{1cm}


\begin{tabular}{m{17cm}}
\textbf{34-вариант}
\newline

T1. 9. Линейные преобразования и их матрица. \\
T2. 6. Положительно определенные квадратичные формы. \\
A1. Доказать, что векторы \(\overrightarrow{a} = (2;\ \ 1;\ \ 1),\) \(\overrightarrow{b} = (1;\ \ 2;\ \ 1)\) и \(\overrightarrow{c} = (2;\ \ 3;\ \  - 1)\) образуют базис пространства \(\mathbf{R}^{3},\) и найти координаты вектора \(\overrightarrow{d} = (2;\ \ 3;\ \  - 1)\) в этом базисе. \\
A2. Известно, что оператор \emph{A} переводит базисные векторы \(\overrightarrow{i} = (1;\ \ 0;\ \ 0),\) \(\overrightarrow{j} = (0;\ \ 1;\ \ 0),\) \(\overrightarrow{k} = (0;\ \ 0;\ \ 1)\) линейного пространства \(\mathbf{R}^{3}\) в векторы \({\overline{a}}_{1} = (2;\ \ 1;1),\) \({\overline{a}}_{2} = (3;\ \ 2;\ \ 1),\) \({\overline{a}}_{3} = (3;\ \ 1;\ \ 1).\) В базисе \(\overrightarrow{i},\overrightarrow{j},\overrightarrow{k}\) найти: 1)матрицу оператора \(A\) ; 2)образ вектора \(\overline{b} = (1;\ \  - 2;\ \ 3).\) \\
A3. В пространстве \(\mathbb{C}^{3}\) со скалярным произведением \(\left\langle x,y \right\rangle = \sum_{k = 1}^{3}{x_{k}\overline{y_{k}}}\), найдите сопряженный оператор \(A^{*}\) для заданного оператора \(A\). Является ли \(A\)самосопряженным? \(Ax = \left( x_{2} + ix_{3},x_{1} - ix_{2},x_{1} + ix_{2} + x_{3} \right)\) \\
B1. С помощью процесса ортогонализации Грамма- Шмидта ортонормировать следующие системы векторов, используя стандартное скалярное произведение: \\
B2. Найти собственные значения и собственные векторы оператора \emph{А}, заданного в некотором базисе пространства \(V^{3}\) матрицей \(A = \begin{bmatrix}
0 & - 1 & 1 \\
 - 1 & 0 & 1 \\
1 & 1 & 0
\end{bmatrix};\) \\
B3. Приведите квадратичные формы \(G_{1}\) и \(G_{2}\) к каноническому виду. \(G_{1} = x_{1}^{2} + 5x_{2}^{2} - 4x_{3}^{2} + 2x_{1}x_{3} - 4x_{1}x_{2}\), \(G_{2} = - 4x_{1}x_{2} + 2x_{1}x_{3}\) \\
C1. В базисе \((e_{1},\ \ e_{2},\ \ e_{3})\) пространства \(V^{3}\) оператор \emph{A} имеет матрицу \(A = \begin{bmatrix}
1 & 3 & - 1 \\
2 & 0 & 4 \\
1 & 1 & 1
\end{bmatrix}\ \ .\) Найти матрицу \emph{B} этого же оператора в базисе \(({e'}_{1},\ \ {e'}_{2},\ \ {e'}_{3}),\) где \({e'}_{1} = 2e_{1} + e_{2}\), \({e'}_{2} = - e_{1} + 2e_{2} + 3e_{3}\),\({e'}_{3} = - e_{1} + e_{2} + e_{3}\) \\
C2. Даны векторы \(e_{1},e_{2},e_{3}\), \(a_{1},a_{2},a_{3}\) линейного пространства \(R^{3}\). Найдите матрицу перехода от базиса \(e_{1},e_{2},e_{3}\) к базису \(a_{1},a_{2},a_{3}\).
\(e_{1} = (3,5,8)\),\(e_{2} = (5,14,13)\),\(e_{3} = (1,9,2)\) и \(a_{1} = ( - 2,3,1)\),\(a_{2} = (0,2,1)\),\(a_{3} = (1,2,1)\) \\
C3. Найти жорданову нормальную форму матрицы \(A = \begin{pmatrix}
2 & - 1 & 2 \\
5 & - 3 & 3 \\
 - 1 & 0 & - 2
\end{pmatrix}\). \\

\end{tabular}
\vspace{1cm}


\begin{tabular}{m{17cm}}
\textbf{35-вариант}
\newline

T1. 17. Взаимозаменяемые преобразования. \\
T2. 16. Унитарные преобразования и их собственные значения и канонический вид. \\
A1. Доказать, что векторы \(\overrightarrow{a} = (1;\ \ 0;\ \ 1),\) \(\overrightarrow{b} = (1;\ \ 1;\ \ 1)\) и \(\overrightarrow{c} = ( - 1;\ \ 2;\ \ 1)\) образуют базис пространства \(\mathbf{R}^{3},\) и найти координаты вектора \(\overrightarrow{d} = (0;\ \ 10;\ \ 3)\) в этом базисе. \\
A2. Известно, что оператор \emph{A} переводит базисные векторы \(\overrightarrow{i} = (1;\ \ 0;\ \ 0),\) \(\overrightarrow{j} = (0;\ \ 1;\ \ 0),\) \(\overrightarrow{k} = (0;\ \ 0;\ \ 1)\) линейного пространства \(\mathbf{R}^{3}\) в векторы \({\overline{a}}_{1} = (1;\ \ 1;\ \ 0),\) \({\overline{a}}_{2} = (3;\ \ 2;\ \ 1),\) \({\overline{a}}_{3} = (3;\ \ 1;\ \ 1).\) В базисе \(\overrightarrow{i},\overrightarrow{j},\overrightarrow{k}\) найти: 1)матрицу оператора \(A\) ; 2)образ вектора \(\overline{b} = (1;\ \ 2;\ \ 3).\) \\
A3. В пространстве \(\mathbb{C}^{3}\) со скалярным произведением \(\left\langle x,y \right\rangle = \sum_{k = 1}^{3}{x_{k}\overline{y_{k}}}\), найдите сопряженный оператор \(A^{*}\) для заданного оператора \(A\). Является ли \(A\)самосопряженным? \(Ax = \left( 2ix_{1} + ix_{3},x_{1} + x_{2} + ix_{3},ix_{3} \right)\); \\
B1. \(a_{1} = (4;1;3)\), \(a_{2} = (0;7; - 2)\), \(a_{3} = (4;8;0)\);
 \\
B2. Найти собственные значения и собственные векторы оператора \emph{А}, заданного в некотором базисе пространства \(V^{3}\) матрицей \(A = \begin{bmatrix}
 - 1 & 1 & 0 \\
 - 4 & 3 & 0 \\
 - 2 & 1 & 1
\end{bmatrix};\) \\
B3. Приведите квадратичные формы \(G_{1}\) и \(G_{2}\) к каноническому виду. \(G_{1} = x_{1}^{2} - 2x_{3}^{2} + 2x_{1}x_{3} - 6x_{1}x_{2}\), \(G_{2} = 6x_{2}x_{3} - 4x_{1}x_{2} + x_{1}x_{3}\)
 \\
C1. В базисе \((e_{1},\ \ e_{2},\ \ e_{3})\) пространства \(V^{3}\) оператор \emph{A} имеет матрицу \(A = \begin{bmatrix}
2 & 0 & - 1 \\
3 & 2 & 0 \\
 - 1 & 4 & 3
\end{bmatrix}\ \ .\) Найти матрицу \emph{B} этого же оператора в базисе \(({e'}_{1},\ \ {e'}_{2},\ \ {e'}_{3}),\) где \({e'}_{1} = e_{1} - e_{3},\) \({e'}_{2} = e_{2} + e_{3},\) \({e'}_{3} = e_{3}.\) \\
C2. Даны векторы \(e_{1},e_{2},e_{3}\), \(a_{1},a_{2},a_{3}\) линейного пространства \(R^{3}\). Найдите матрицу перехода от базиса \(e_{1},e_{2},e_{3}\) к базису \(a_{1},a_{2},a_{3}\).
\(e_{1} = (2,0,1)\),\(e_{2} = ( - 1,2,3)\),\(e_{3} = ( - 1,1,1)\) и \(a_{1} = ( - 3,0,1)\),\(a_{2} = (0,2,3)\),\(a_{3} = ( - 1, - 1, - 1)\) \\
C3. Найти жорданову нормальную форму матрицы \(A = \begin{pmatrix}
0 & 1 & 0 \\
 - 4 & 4 & 0 \\
0 & 0 & 2
\end{pmatrix}\) \\

\end{tabular}
\vspace{1cm}


\begin{tabular}{m{17cm}}
\textbf{36-вариант}
\newline

T1. 13. Инвариантные подпространства. Собственные векторы и собственные значения. \\
T2. 12. Связь между матрицами линейных преобразовании в разных базисах. \\
A1. Доказать, что векторы \(\overrightarrow{a} = (2;\ \ 1;\ \ 1),\) \(\overrightarrow{b} = (1;\ \ 2;\ \ 1)\) и \(\overrightarrow{c} = (2;\ \ 1;\ \ 1)\) образуют базис пространства \(\mathbf{R}^{3},\) и найти координаты вектора \(\overrightarrow{d} = (1;\ \ 3;\ \ 1)\) в этом базисе. \\
A2. Известно, что оператор \emph{A} переводит базисные векторы \(\overrightarrow{i} = (1;\ \ 0;\ \ 0),\) \(\overrightarrow{j} = (0;\ \ 1;\ \ 0),\) \(\overrightarrow{k} = (0;\ \ 0;\ \ 1)\) линейного пространства \(\mathbf{R}^{3}\) в векторы \({\overline{a}}_{1} = (1;\ \ 0;\ \ 1),\) \({\overline{a}}_{2} = (0;\ \ 2;\ \ 1),\) \({\overline{a}}_{3} = (3;\ \ 1;\ \ 1).\) В базисе \(\overrightarrow{i},\overrightarrow{j},\overrightarrow{k}\) найти: 1)матрицу оператора \(A\) ; 2)образ вектора \(\overline{b} = (1;\ \ 2;\ \ 3).\) \\
A3. 
В пространстве \(\mathbb{C}^{3}\) со скалярным произведением \(\left\langle x,y \right\rangle = \sum_{k = 1}^{3}{x_{k}\overline{y_{k}}}\), найдите сопряженный оператор \(A^{*}\) для заданного оператора \(A\). Является ли \(A\)самосопряженным? \(Ax = \left( ix_{1} + x_{3},x_{2} + ix_{1},x_{1} + ix_{3} \right)\); \\
B1. \(a_{1} = (1;2;1)\), \(a_{2} = (2;3;3)\), \(a_{3} = (3;8;2)\); \\
B2. Найти собственные значения и собственные векторы оператора \emph{А}, заданного в некотором базисе пространства \(V^{3}\) матрицей \(A = \begin{bmatrix}
 - 1 & - 2 & 0 \\
0 & - 2 & 0 \\
2 & 2 & 1
\end{bmatrix}.\) \\
B3. Приведите квадратичные формы \(G_{1}\) и \(G_{2}\) к каноническому виду. \(G_{1} = x_{1}^{2} - 2x_{2}^{2} + x_{3}^{2} + 2x_{1}x_{2} + 4x_{1}x_{3} + 2x_{2}x_{3}\), \(G_{2} = 2x_{1}x_{3} - 4x_{2}x_{3}\) \\
C1. В базисе \((e_{1},\ \ e_{2},\ \ e_{3})\) пространства \(V^{3}\) оператор \emph{A} имеет матрицу \(A = \begin{bmatrix}
4 & 0 & 1 \\
 - 2 & - 2 & 3 \\
0 & 2 & - 1
\end{bmatrix}\ \ .\) Найти матрицу \emph{B} этого же оператора в базисе \(({e'}_{1},\ \ {e'}_{2},\ \ {e'}_{3}),\) где \({e'}_{1} = e_{1} + 2e_{2},\) \({e'}_{2} = e_{1} - e_{3},\) \({e'}_{3} = e_{1} + e_{2} + e_{3}.\)
 \\
C2. Даны векторы \(e_{1},e_{2},e_{3}\), \(a_{1},a_{2},a_{3}\) линейного пространства \(R^{3}\). Найдите матрицу перехода от базиса \(e_{1},e_{2},e_{3}\) к базису \(a_{1},a_{2},a_{3}\).
\(e_{1} = (2,0,1)\),\(e_{2} = ( - 1,2,3)\),\(e_{3} = ( - 1,1,1)\) и \(a_{1} = (1,0,2)\),\(a_{2} = (3, - 1,4)\),\(a_{3} = (2, - 2,1)\) \\
C3. Найти жорданову нормальную форму матрицы \(A = \begin{pmatrix}
 - 1 & 3 & - 1 \\
 - 3 & 5 & - 1 \\
 - 3 & 3 & 1
\end{pmatrix}\). \\

\end{tabular}
\vspace{1cm}


\begin{tabular}{m{17cm}}
\textbf{37-вариант}
\newline

T1. 7. Комплексные евклидовы пространства. \\
T2. 18. Нормальные преобразования и их канонический вид. \\
A1. Доказать, что векторы \(\overrightarrow{a} = (1;\ \ 2;\ \ 1),\) \(\overrightarrow{b} = (1;\ \ 1;\ \ 3)\) и \(\overrightarrow{c} = ( - 1;\ \ 2;\ \ 1)\) образуют базис пространства \(\mathbf{R}^{3},\) и найти координаты вектора \(\overrightarrow{d} = (0;\ \ 10;\ \  - 2)\) в этом базисе. \\
A2. Известно, что оператор \emph{A} переводит базисные векторы \(\overrightarrow{i} = (1;\ \ 0;\ \ 0),\) \(\overrightarrow{j} = (0;\ \ 1;\ \ 0),\) \(\overrightarrow{k} = (0;\ \ 0;\ \ 1)\) линейного пространства \(\mathbf{R}^{3}\) в векторы \({\overline{a}}_{1} = (2;\ \ 1;1),\) \({\overline{a}}_{2} = (3;\ \ 2;\ \ 1),\) \({\overline{a}}_{3} = (3;\ \ 1;\ \ 1).\) В базисе \(\overrightarrow{i},\overrightarrow{j},\overrightarrow{k}\) найти: 1)матрицу оператора \(A\) ; 2)образ вектора \(\overline{b} = (1;\ \  - 2;\ \ 3).\) \\
A3. В пространстве \(\mathbb{C}^{3}\) со скалярным произведением \(\left\langle x,y \right\rangle = \sum_{k = 1}^{3}{x_{k}\overline{y_{k}}}\), найдите сопряженный оператор \(A^{*}\) для заданного оператора \(A\). Является ли \(A\)самосопряженным? \(Ax = \left( ix_{1} + 2ix_{3},x_{3},x_{1} - 2ix_{3} \right)\); \\
B1. 
С помощью процесса ортогонализации Грамма- Шмидта ортонормировать следующие системы векторов, используя стандартное скалярное произведение: \\
B2. Найти собственные значения и собственные векторы оператора \emph{А}, заданного в некотором базисе пространства \(V^{3}\) матрицей \(A = \begin{bmatrix}
 - 1 & 1 & 0 \\
 - 4 & 3 & 0 \\
 - 2 & 1 & 1
\end{bmatrix};\) \\
B3. Приведите квадратичные формы \(G_{1}\) и \(G_{2}\) к каноническому виду. \(G_{1} = 3x_{1}^{2} - 2x_{2}^{2} + 2x_{3}^{2} + 4x_{1}x_{2} - 3x_{1}x_{3} - x_{2}x_{3}\), \(G_{2} = 2x_{1}x_{3} + 4x_{1}x_{2} - 2x_{2}x_{3}\) \\
C1. В базисе \((e_{1},\ \ e_{2},\ \ e_{3})\) пространства \(V^{3}\) оператор \emph{A} имеет матрицу \(A = \begin{bmatrix}
2 & 0 & - 1 \\
3 & 2 & 0 \\
 - 1 & 4 & 3
\end{bmatrix}\ \ .\) Найти матрицу \emph{B} этого же оператора в базисе \(({e'}_{1},\ \ {e'}_{2},\ \ {e'}_{3}),\) где \({e'}_{1} = e_{1} - e_{3},\) \({e'}_{2} = e_{2} + e_{3},\) \({e'}_{3} = e_{3}.\) \\
C2. 10.Даны векторы \(e_{1},e_{2},e_{3}\), \(a_{1},a_{2},a_{3}\) линейного пространства \(R^{3}\). Найдите матрицу перехода от базиса \(e_{1},e_{2},e_{3}\) к базису \(a_{1},a_{2},a_{3}\).
\(e_{1} = (4,0,5)\),\(e_{2} = ( - 2,1,3)\),\(e_{3} = ( - 5,1, - 1)\) и \(a_{1} = (1,2,1)\),\(a_{2} = (2,3,3)\),\(a_{3} = (3,8,2)\) \\
C3. Найти жорданову нормальную форму матрицы \(A = \begin{pmatrix}
2 & - 1 & - 1 \\
2 & - 1 & - 2 \\
 - 1 & 1 & 2
\end{pmatrix}\). \\

\end{tabular}
\vspace{1cm}


\begin{tabular}{m{17cm}}
\textbf{38-вариант}
\newline

T1. 3. Ортогональное дополнение и ортогональная проекция. \\
T2. 20. Подобные матрицы. \\
A1. Доказать, что векторы \(\overrightarrow{a} = (3;\ \ 1;\ \ 2),\) \(\overrightarrow{b} = (2;\ \  - 3;\ \ 1)\) и \(\overrightarrow{c} = (4;\ \  - 2;\ \ 3)\) образуют базис пространства \(\mathbf{R}^{3},\) и найти координаты вектора \(\overrightarrow{d} = ( - 7;\ \ 8;\ \ 7)\) в этом базисе. \\
A2. Известно, что оператор \emph{A} переводит базисные векторы \(\overrightarrow{i} = (1;\ \ 0;\ \ 0),\) \(\overrightarrow{j} = (0;\ \ 1;\ \ 0),\) \(\overrightarrow{k} = (0;\ \ 0;\ \ 1)\) линейного пространства \(\mathbf{R}^{3}\) в векторы \({\overline{a}}_{1} = (1;\ \ 1;\ \ 1),\) \({\overline{a}}_{2} = (3;\ \ 2;\ \ 1),\) \({\overline{a}}_{3} = (0;\ \ 1;\ \ 1).\) В базисе \(\overrightarrow{i},\overrightarrow{j},\overrightarrow{k}\) найти: 1)матрицу оператора \(A\) ; 2)образ вектора \(\overline{b} = (1;\ \  - 2;\ \ 3).\)
 \\
A3. В пространстве \(\mathbb{C}^{3}\) со скалярным произведением \(\left\langle x,y \right\rangle = \sum_{k = 1}^{3}{x_{k}\overline{y_{k}}}\), найдите сопряженный оператор \(A^{*}\) для заданного оператора \(A\). Является ли \(A\)самосопряженным? \(Ax = \left( 3ix_{1} + x_{2},x_{1} + 2ix_{2},ix_{2} - x_{3} \right)\); \\
B1. С помощью процесса ортогонализации Грамма- Шмидта ортонормировать следующие системы векторов, используя стандартное скалярное произведение: \\
B2. Найти собственные значения и собственные векторы оператора \emph{А}, заданного в некотором базисе пространства \(V^{3}\) матрицей \(A = \begin{pmatrix}
1 & - 1 & 1 \\
1 & 1 & - 1 \\
2 & - 1 & 0
\end{pmatrix}\). \\
B3. Приведите квадратичные формы \(G_{1}\) и \(G_{2}\) к каноническому виду. \(G_{1} = x_{1}^{2} - 2x_{2}^{2} + x_{3}^{2} + 2x_{1}x_{2} + 4x_{1}x_{3} + 2x_{2}x_{3}\), \(G_{2} = 2x_{1}x_{3} - 4x_{2}x_{3}\) \\
C1. В базисе \((e_{1},\ \ e_{2},\ \ e_{3})\) пространства \(V^{3}\) оператор \emph{A} имеет матрицу \(A = \begin{bmatrix}
5 & - 2 & 1 \\
 - 1 & 0 & 4 \\
3 & 1 & 2
\end{bmatrix}\ \ .\) Найти матрицу \emph{B} этого же оператора в базисе \(({e'}_{1},\ \ {e'}_{2},\ \ {e'}_{3}),\) где \({e'}_{1} = 2e_{1} + 3e_{3},\) \({e'}_{2} = - e_{2},\) \({e'}_{3} = e_{1} + e_{2} + e_{3}.\) \\
C2. Даны векторы \(e_{1},e_{2},e_{3}\), \(a_{1},a_{2},a_{3}\) линейного пространства \(R^{3}\). Найдите матрицу перехода от базиса \(e_{1},e_{2},e_{3}\) к базису \(a_{1},a_{2},a_{3}\).
\(e_{1} = (2,0,1)\),\(e_{2} = ( - 1,2,3)\),\(e_{3} = ( - 1,1,1)\) и \(a_{1} = (1,0,2)\),\(a_{2} = (3, - 1,4)\),\(a_{3} = (2, - 2,1)\) \\
C3. Найти жорданову нормальную форму матрицы \(A = \begin{pmatrix}
2 & 1 & 1 \\
1 & 2 & 1 \\
1 & 1 & 2
\end{pmatrix}\). \\

\end{tabular}
\vspace{1cm}


\begin{tabular}{m{17cm}}
\textbf{39-вариант}
\newline

T1. 5. Методы приведения квадратичной формы к каноническому форму. \\
T2. 4. Линейные, билинейные, и квадратичные формы. Преобразование матрицы линейного вида при изменении базиса. \\
A1. Доказать, что векторы \(\overrightarrow{a} = (2;\ \ 1;1),\) \(\overrightarrow{b} = ( - 1;\ \ 2;\ \ 4)\) и \(\overrightarrow{c} = (3;\ \ 3;\ \ 2)\) образуют базис пространства \(\mathbf{R}^{3},\) и найти координаты вектора \(\overrightarrow{d} = ( - 4;\ \ 2;\ \ 4)\) в этом базисе. \\
A2. Известно, что оператор \emph{A} переводит базисные векторы \(\overrightarrow{i} = (1;\ \ 0;\ \ 0),\) \(\overrightarrow{j} = (0;\ \ 1;\ \ 0),\) \(\overrightarrow{k} = (0;\ \ 0;\ \ 1)\) линейного пространства \(\mathbf{R}^{3}\) в векторы \({\overline{a}}_{1} = (1;\ \ 0;\ \ 1),\) \({\overline{a}}_{2} = (0;\ \ 1;\ \ 1),\) \({\overline{a}}_{3} = (3;\ \ 1;\ \ 1).\) В базисе \(\overrightarrow{i},\overrightarrow{j},\overrightarrow{k}\) найти: 1)матрицу оператора \(A\) ; 2)образ вектора \(\overline{b} = (1;\ \  - 2;\ \  - 3).\) \\
A3. В пространстве \(\mathbb{C}^{3}\) со скалярным произведением \(\left\langle x,y \right\rangle = \sum_{k = 1}^{3}{x_{k}\overline{y_{k}}}\), найдите сопряженный оператор \(A^{*}\) для заданного оператора \(A\). Является ли \(A\)самосопряженным?\(Ax = \left( x_{1} + 2ix_{2},x_{3} - ix_{2},x_{1} - ix_{2} - 2ix_{3} \right)\); \\
B1. С помощью процесса ортогонализации Грамма- Шмидта ортонормировать следующие системы векторов, используя стандартное скалярное произведение: \\
B2. Найти собственные значения и собственные векторы оператора \emph{А}, заданного в некотором базисе пространства \(V^{3}\) матрицей \(A = \begin{bmatrix}
0 & - 1 & 1 \\
 - 1 & 0 & 1 \\
1 & 1 & 0
\end{bmatrix};\) \\
B3. Приведите квадратичные формы \(G_{1}\) и \(G_{2}\) к каноническому виду. \(G_{1} = 3x_{1}^{2} - 2x_{2}^{2} + 2x_{1}x_{3} - 4x_{2}x_{3}\), \(G_{2} = x_{1}x_{2} + x_{2}x_{3}\) \\
C1. В базисе \((e_{1},\ \ e_{2},\ \ e_{3})\) пространства \(V^{3}\) оператор \emph{A} имеет матрицу \(A = \begin{bmatrix}
 - 3 & 1 & 4 \\
0 & 3 & 2 \\
 - 5 & - 1 & 2
\end{bmatrix}\ \ .\) Найти матрицу \emph{B} этого же оператора в базисе \(({e'}_{1},\ \ {e'}_{2},\ \ {e'}_{3}),\) где \({e'}_{1} = 2e_{1} + e_{2}\), \({e'}_{2} = - e_{1} + 2e_{2} + 3e_{3}\),\({e'}_{3} = - e_{1} + e_{2} + e_{3}\) \\
C2. Даны векторы \(e_{1},e_{2},e_{3}\), \(a_{1},a_{2},a_{3}\) линейного пространства \(R^{3}\). Найдите матрицу перехода от базиса \(e_{1},e_{2},e_{3}\) к базису \(a_{1},a_{2},a_{3}\).
\(e_{1} = (2,0,1)\),\(e_{2} = ( - 1,2,3)\),\(e_{3} = ( - 1,1,1)\) и \(a_{1} = ( - 3,0,1)\),\(a_{2} = (0,2,3)\),\(a_{3} = ( - 1, - 1, - 1)\) \\
C3. Найти жорданову нормальную форму матрицы \(A = \begin{pmatrix}
1 & 2 & 1 \\
1 & 2 & 4 \\
 - 1 & - 2 & - 3
\end{pmatrix}\). \\

\end{tabular}
\vspace{1cm}


\begin{tabular}{m{17cm}}
\textbf{40-вариант}
\newline

T1. 19. Полиномиальные матрицы и диагональные нормальные формы. \\
T2. 14. Сопряженное преобразование для данного преобразования. \\
A1. Доказать, что векторы \(\overrightarrow{a} = (3;\ \ 5;\ \ 4),\) \(\overrightarrow{b} = (4;\ \ 3;\ \ 2)\) и \(\overrightarrow{c} = ( - 1;\ \  - 4;\ \ 3)\) образуют базис пространства \(\mathbf{R}^{3},\) и найти координаты вектора \(\overrightarrow{d} = ( - 2;\ \  - 2;\ \ 5)\) в этом базисе. \\
A2. Известно, что оператор \emph{A} переводит базисные векторы \(\overrightarrow{i} = (1;\ \ 0;\ \ 0),\) \(\overrightarrow{j} = (0;\ \ 1;\ \ 0),\) \(\overrightarrow{k} = (0;\ \ 0;\ \ 1)\) линейного пространства \(\mathbf{R}^{3}\) в векторы \({\overline{a}}_{1} = (1;\ \ 1;\ \ 0),\) \({\overline{a}}_{2} = (3;\ \ 2;\ \ 1),\) \({\overline{a}}_{3} = (3;\ \ 1;\ \ 1).\) В базисе \(\overrightarrow{i},\overrightarrow{j},\overrightarrow{k}\) найти: 1)матрицу оператора \(A\) ; 2)образ вектора \(\overline{b} = (1;\ \ 2;\ \ 3).\) \\
A3. В пространстве \(\mathbb{C}^{3}\) со скалярным произведением \(\left\langle x,y \right\rangle = \sum_{k = 1}^{3}{x_{k}\overline{y_{k}}}\), найдите сопряженный оператор \(A^{*}\) для заданного оператора \(A\). Является ли \(A\)самосопряженным? \(Ax = \left( ix_{1} + x_{3},ix_{3} - ix_{2},x_{1} - ix_{3} \right)\); \\
B1. \(a_{1} = (2;4;3)\), \(a_{2} = (3; - 1;4)\), \(a_{3} = (1;5; - 1)\); \\
B2. Найти собственные значения и собственные векторы оператора \emph{А}, заданного в некотором базисе пространства \(V^{3}\) матрицей \(A = \begin{pmatrix}
2 & - 1 & 2 \\
5 & - 3 & 3 \\
 - 1 & 0 & - 2
\end{pmatrix}\). \\
B3. Приведите квадратичные формы \(G_{1}\) и \(G_{2}\) к каноническому виду. \(G_{1} = 4x_{1}^{2} + x_{2}^{2} + x_{3}^{2} - 4x_{1}x_{2} + 4x_{1}x_{3} - 3x_{2}x_{3}\), \(G_{2} = x_{1}x_{2} + 6x_{1}x_{3} - 4x_{2}x_{3}\) \\
C1. В базисе \((e_{1},\ \ e_{2},\ \ e_{3})\) пространства \(V^{3}\) оператор \emph{A} имеет матрицу \(A = \begin{bmatrix}
 - 1 & 2 & 4 \\
 - 4 & 2 & 0 \\
3 & 3 & - 3
\end{bmatrix}\ \ .\) Найти матрицу \emph{B} этого же оператора в базисе \(({e'}_{1},\ \ {e'}_{2},\ \ {e'}_{3}),\) где \({e'}_{1} = 2e_{1} + e_{2}\), \({e'}_{2} = - e_{1} + 2e_{2} + 3e_{3}\),\({e'}_{3} = - e_{1} + e_{2} + e_{3}\) \\
C2. Даны векторы \(e_{1},e_{2},e_{3}\), \(a_{1},a_{2},a_{3}\) линейного пространства \(R^{3}\). Найдите матрицу перехода от базиса \(e_{1},e_{2},e_{3}\) к базису \(a_{1},a_{2},a_{3}\).
\(e_{1} = (3,1, - 1)\),\(e_{2} = ( - 2,0,1)\),\(e_{3} = (2,7,3)\) и \(a_{1} = (2,1, - 3)\),\(a_{2} = (3,2, - 5)\),\(a_{3} = (1, - 1,1)\) \\
C3. Найти жорданову нормальную форму матрицы \(A = \begin{pmatrix}
2 & - 1 & - 1 \\
2 & - 1 & - 2 \\
 - 1 & 1 & 2
\end{pmatrix}\) \\

\end{tabular}
\vspace{1cm}


\begin{tabular}{m{17cm}}
\textbf{41-вариант}
\newline

T1. 5. Методы приведения квадратичной формы к каноническому форму. \\
T2. 2. Евклидово пространство. Неравенство Коши-Буняковского. Процесс ортогонализации. \\
A1. Доказать, что векторы \(\overrightarrow{a} = (1;\ \ 2;\ \ 1),\) \(\overrightarrow{b} = (1;\ \ 1;\ \  - 3)\) и \(\overrightarrow{c} = ( - 1;\ \ 2;\ \ 1)\) образуют базис пространства \(\mathbf{R}^{3},\) и найти координаты вектора \(\overrightarrow{d} = (0;\ \ 10;\ \  - 2)\) в этом базисе. \\
A2. Известно, что оператор \emph{A} переводит базисные векторы \(\overrightarrow{i} = (1;\ \ 0;\ \ 0),\) \(\overrightarrow{j} = (0;\ \ 1;\ \ 0),\) \(\overrightarrow{k} = (0;\ \ 0;\ \ 1)\) линейного пространства \(\mathbf{R}^{3}\) в векторы \({\overline{a}}_{1} = (1;\ \ 1;\ \ 1),{\overline{a}}_{2} = (3;\ \ 0;\ \ 1),\) \({\overline{a}}_{3} = (0;\ \ 2;\ \ 1).\)В базисе \(\overrightarrow{i},\overrightarrow{j},\overrightarrow{k}\) найти:1)матрицу оператора \(A\) ;2)образ вектора \(\overline{b} = (1;\ \ 2;\ \  - 2).\) \\
A3. В пространстве \(\mathbb{C}^{3}\) со скалярным произведением \(\left\langle x,y \right\rangle = \sum_{k = 1}^{3}{x_{k}\overline{y_{k}}}\), найдите сопряженный оператор \(A^{*}\) для заданного оператора \(A\). Является ли \(A\)самосопряженным? \(Ax = \left( 2ix_{1} + ix_{3},x_{1} + x_{2} + ix_{3},ix_{3} \right)\); \\
B1. С помощью процесса ортогонализации Грамма- Шмидта ортонормировать следующие системы векторов, используя стандартное скалярное произведение: \\
B2. Найти собственные значения и собственные векторы оператора \emph{А}, заданного в некотором базисе пространства \(V^{3}\) матрицей \(A = \begin{pmatrix}
1 & - 2 & - 1 \\
 - 1 & 1 & 1 \\
1 & 0 & - 1
\end{pmatrix}\) \\
B3. Приведите квадратичные формы \(G_{1}\) и \(G_{2}\) к каноническому виду. \(G_{1} = 2x_{1}^{2} + 3x_{2}^{2} + 4x_{3}^{2} - 2x_{1}x_{2} + 4x_{1}x_{3} - 3x_{2}x_{3}\), \(G_{2} = x_{1}x_{3} - 2x_{2}x_{3}\) \\
C1. В базисе \((e_{1},\ \ e_{2},\ \ e_{3})\) пространства \(V^{3}\) оператор \emph{A} имеет матрицу \(A = \begin{bmatrix}
0 & 1 & - 2 \\
3 & 5 & 1 \\
 - 1 & 2 & 0
\end{bmatrix}\ \ .\) Найти матрицу \emph{B} этого же оператора в базисе \(({e'}_{1},\ \ {e'}_{2},\ \ {e'}_{3}),\) где \({e'}_{1} = e_{1} + 2e_{2},\) \({e'}_{2} = e_{1} - e_{3},\) \({e'}_{3} = e_{1} + e_{2} + e_{3}.\) \\
C2. Даны векторы \(e_{1},e_{2},e_{3}\), \(a_{1},a_{2},a_{3}\) линейного пространства \(R^{3}\). Найдите матрицу перехода от базиса \(e_{1},e_{2},e_{3}\) к базису \(a_{1},a_{2},a_{3}\).
\(e_{1} = ( - 3,0,1)\),\(e_{2} = (0,2,3)\),\(e_{3} = ( - 1, - 1, - 1)\) и \(a_{1} = (1,1,1)\),\(a_{2} = (1,1,2)\),\(a_{3} = (1,2,3)\) \\
C3. Найти жорданову нормальную форму матрицы \(A = \begin{pmatrix}
2 & - 1 & 2 \\
5 & - 3 & 3 \\
 - 1 & 0 & - 2
\end{pmatrix}\). \\

\end{tabular}
\vspace{1cm}


\begin{tabular}{m{17cm}}
\textbf{42-вариант}
\newline

T1. 1. Линейные пространства. Линейные подпространства. Сумма и пересечение подпространств. \\
T2. 10. Ядро, образ линйеного преобразования. \\
A1. Доказать, что векторы \(\overrightarrow{a} = (1;\ \ 2;\ \ 1),\) \(\overrightarrow{b} = (1;\ \ 1;\ \ 3)\) и \(\overrightarrow{c} = ( - 1;\ \ 2;\ \ 1)\) образуют базис пространства \(\mathbf{R}^{3},\) и найти координаты вектора \(\overrightarrow{d} = (0;\ \ 10;\ \  - 2)\) в этом базисе. \\
A2. 
Известно, что оператор \emph{A} переводит базисные векторы \(\overrightarrow{i} = (1;\ \ 0;\ \ 0),\) \(\overrightarrow{j} = (0;\ \ 1;\ \ 0),\) \(\overrightarrow{k} = (0;\ \ 0;\ \ 1)\) линейного пространства \(\mathbf{R}^{3}\) в векторы \({\overline{a}}_{1} = (1;1;0),\) \({\overline{a}}_{2} = (3;\ \ 2;\ \ 1),\) \({\overline{a}}_{3} = (0;\ \ 1;\ \ 1).\) В базисе \(\overrightarrow{i},\overrightarrow{j},\overrightarrow{k}\) найти: 1)матрицу оператора \(A\) ; 2)образ вектора \(\overline{b} = (1;\ \  - 2;\ \  - 3).\) \\
A3. В пространстве \(\mathbb{C}^{3}\) со скалярным произведением \(\left\langle x,y \right\rangle = \sum_{k = 1}^{3}{x_{k}\overline{y_{k}}}\), найдите сопряженный оператор \(A^{*}\) для заданного оператора \(A\). Является ли \(A\)самосопряженным? \(Ax = \left( x_{1} + ix_{3},x_{3} + 2ix_{2},ix_{2} - 2ix_{3} \right)\); \\
B1. С помощью процесса ортогонализации Грамма- Шмидта ортонормировать следующие системы векторов, используя стандартное скалярное произведение: \\
B2. 
Найти собственные значения и собственные векторы оператора \emph{А}, заданного в некотором базисе пространства \(V^{3}\) матрицей \(A = \begin{bmatrix}
0 & - 1 & 1 \\
 - 1 & 0 & 1 \\
1 & 1 & 0
\end{bmatrix}.\) \\
B3. Приведите квадратичные формы \(G_{1}\) и \(G_{2}\) к каноническому виду. \(G_{1} = x_{1}^{2} - 3x_{3}^{2} - 2x_{1}x_{2} - 2x_{1}x_{3} - 6x_{2}x_{3}\), \(G_{2} = 2x_{1}x_{2} - x_{1}x_{3} + 2x_{2}x_{3}\) \\
C1. В базисе \((e_{1},\ \ e_{2},\ \ e_{3})\) пространства \(V^{3}\) оператор \emph{A} имеет матрицу \(A = \begin{bmatrix}
3 & 2 & - 1 \\
4 & 0 & 2 \\
 - 1 & 2 & - 1
\end{bmatrix}\ \ .\) Найти матрицу \emph{B} этого же оператора в базисе \(({e'}_{1},\ \ {e'}_{2},\ \ {e'}_{3}),\) где \({e'}_{1} = 2e_{1} + e_{2}\), \({e'}_{2} = - e_{1} + 2e_{2} + 3e_{3}\),\({e'}_{3} = - e_{1} + e_{2} + e_{3}\) \\
C2. Даны векторы \(e_{1},e_{2},e_{3}\), \(a_{1},a_{2},a_{3}\) линейного пространства \(R^{3}\). Найдите матрицу перехода от базиса \(e_{1},e_{2},e_{3}\) к базису \(a_{1},a_{2},a_{3}\).
\(e_{1} = (0,1, - 2)\),\(e_{2} = ( - 2,0,3)\),\(e_{3} = (1, - 1,1)\) и \(a_{1} = (3,1, - 1)\),\(a_{2} = ( - 2,0,1)\),\(a_{3} = (2,7,3)\) \\
C3. 
Найти жорданову нормальную форму матрицы \(A = \begin{pmatrix}
 - 1 & 1 & - 2 \\
3 & - 3 & 6 \\
2 & - 2 & 4
\end{pmatrix}\). \\

\end{tabular}
\vspace{1cm}


\begin{tabular}{m{17cm}}
\textbf{43-вариант}
\newline

T1. 15. Самосопряженные преобразования и их канонический вид. \\
T2. 16. Унитарные преобразования и их собственные значения и канонический вид. \\
A1. Доказать, что векторы \(\overrightarrow{a} = (2;\ \ 1;\ \  - 3),\) \(\overrightarrow{b} = ( - 1;\ \ 2;\ \ 4)\) и \(\overrightarrow{c} = (3;\ \  - 4;\ \ 2)\) образуют базис пространства \(\mathbf{R}^{3},\) и найти координаты вектора \(\overrightarrow{d} = ( - 4;\ \ 19;\ \ 3)\) в этом базисе. \\
A2. Известно, что оператор \emph{A} переводит базисные векторы \(\overrightarrow{i} = (1;\ \ 0;\ \ 0),\) \(\overrightarrow{j} = (0;\ \ 1;\ \ 0),\) \(\overrightarrow{k} = (0;\ \ 0;\ \ 1)\) линейного пространства \(\mathbf{R}^{3}\) в векторы \({\overline{a}}_{1} = (1;\ \ 0;\ \ 1),\) \({\overline{a}}_{2} = (3;\ \ 2;\ \ 1),\) \({\overline{a}}_{3} = (3;\ \ 1;\ \ 1).\) В базисе \(\overrightarrow{i},\overrightarrow{j},\overrightarrow{k}\) найти: 1)матрицу оператора \(A\) ; 2)образ вектора \(\overline{b} = (1;\ \  - 2;\ \ 3).\) \\
A3. В пространстве \(\mathbb{C}^{3}\) со скалярным произведением \(\left\langle x,y \right\rangle = \sum_{k = 1}^{3}{x_{k}\overline{y_{k}}}\), найдите сопряженный оператор \(A^{*}\) для заданного оператора \(A\). Является ли \(A\)самосопряженным? \(Ax = \left( x_{1} + 2ix_{3},ix_{2} - x_{3},x_{2} - ix_{3} \right)\); \\
B1. \(a_{1} = (3;1; - 1)\), \(a_{2} = ( - 2;0;1)\), \(a_{3} = (2;7;3)\); \\
B2. Найти собственные значения и собственные векторы оператора \emph{А}, заданного в некотором базисе пространства \(V^{3}\) матрицей \(A = \begin{pmatrix}
2 & 1 & 0 \\
1 & 3 & - 1 \\
 - 1 & 2 & 3
\end{pmatrix}\). \\
B3. Приведите квадратичные формы \(G_{1}\) и \(G_{2}\) к каноническому виду. \(G_{1} = 2x_{1}^{2} + 6x_{2}^{2} - 4x_{3}^{2} - 2x_{1}x_{3} + 4x_{1}x_{2} - 8x_{2}x_{3}\), \(G_{2} = x_{2}x_{3} - 2x_{1}x_{3}\) \\
C1. В базисе \((e_{1},\ \ e_{2},\ \ e_{3})\) пространства \(V^{3}\) оператор \emph{A} имеет матрицу \(A = \begin{bmatrix}
0 & 1 & - 3 \\
2 & 4 & 1 \\
0 & 3 & - 3
\end{bmatrix}\ \ .\) Найти матрицу \emph{B} этого же оператора в базисе \(({e'}_{1},\ \ {e'}_{2},\ \ {e'}_{3}),\) где \({e'}_{1} = 2e_{1} + e_{2}\), \({e'}_{2} = - e_{1} + 2e_{2} + 3e_{3}\),\({e'}_{3} = - e_{1} + e_{2} + e_{3}\) \\
C2. 11.Даны векторы \(e_{1},e_{2},e_{3}\), \(a_{1},a_{2},a_{3}\) линейного пространства \(R^{3}\). Найдите матрицу перехода от базиса \(e_{1},e_{2},e_{3}\) к базису \(a_{1},a_{2},a_{3}\).
\(e_{1} = ( - 1,3,7)\),\(e_{2} = (0,2, - 1)\),\(e_{3} = (1, - 2, - 8)\) и \(a_{1} = (0,3, - 2)\),\(a_{2} = (1, - 1, - 8)\),\(a_{3} = ( - 1,2,7)\) \\
C3. Найти жорданову нормальную форму матрицы \(A = \begin{pmatrix}
3 & - 2 & 6 \\
 - 2 & 6 & 3 \\
6 & 3 & - 2
\end{pmatrix}\). \\

\end{tabular}
\vspace{1cm}


\begin{tabular}{m{17cm}}
\textbf{44-вариант}
\newline

T1. 17. Взаимозаменяемые преобразования. \\
T2. 4. Линейные, билинейные, и квадратичные формы. Преобразование матрицы линейного вида при изменении базиса. \\
A1. Доказать, что векторы \(\overrightarrow{a} = (1;\ \ 0;\ \ 1),\) \(\overrightarrow{b} = (1;\ \ 1;\ \ 1)\) и \(\overrightarrow{c} = ( - 1;\ \ 2;\ \ 1)\) образуют базис пространства \(\mathbf{R}^{3},\) и найти координаты вектора \(\overrightarrow{d} = (0;\ \ 10;\ \ 3)\) в этом базисе. \\
A2. Известно, что оператор \emph{A} переводит базисные векторы \(\overrightarrow{i} = (1;\ \ 0;\ \ 0),\) \(\overrightarrow{j} = (0;\ \ 1;\ \ 0),\) \(\overrightarrow{k} = (0;\ \ 0;\ \ 1)\) линейного пространства \(\mathbf{R}^{3}\) в векторы \({\overline{a}}_{1} = (1;\ \ 1;\ \ 1),{\overline{a}}_{2} = (3;\ \ 0;\ \ 1),\) \({\overline{a}}_{3} = (3;\ \ 1;\ \  - 1).\)В базисе \(\overrightarrow{i},\overrightarrow{j},\overrightarrow{k}\) найти:1)матрицу оператора \(A\) ;2)образ вектора \(\overline{b} = (1;\ \ 2;\ \ 1).\) \\
A3. В пространстве \(\mathbb{C}^{3}\) со скалярным произведением \(\left\langle x,y \right\rangle = \sum_{k = 1}^{3}{x_{k}\overline{y_{k}}}\), найдите сопряженный оператор \(A^{*}\) для заданного оператора \(A\). Является ли \(A\)самосопряженным?\(Ax = \left( x_{1} - 2ix_{2},x_{3} + 2ix_{2},ix_{2} + 2ix_{3} \right)\);
 \\
B1. \(a_{1} = (1;1;1)\), \(a_{2} = (1;2;3)\), \(a_{3} = (1;1;2)\); \\
B2. Найти собственные значения и собственные векторы оператора \emph{А}, заданного в некотором базисе пространства \(V^{3}\) матрицей \(A = \begin{pmatrix}
2 & - 1 & 2 \\
1 & 0 & 2 \\
 - 2 & 1 & - 1
\end{pmatrix}\) \\
B3. Приведите квадратичные формы \(G_{1}\) и \(G_{2}\) к каноническому виду. \(G_{1} = 5x_{1}^{2} + 6x_{2}^{2} - 3x_{3}^{2} + 4x_{1}x_{2} - 2x_{2}x_{3}\), \(G_{2} = 6x_{2}x_{3} - x_{1}x_{2}\) \\
C1. 
В базисе \((e_{1},\ \ e_{2},\ \ e_{3})\) пространства \(V^{3}\) оператор \emph{A} имеет матрицу \(A = \begin{bmatrix}
1 & 2 & 3 \\
0 & 1 & 2 \\
3 & 1 & 2
\end{bmatrix}\ \ .\) Найти матрицу \emph{B} этого же оператора в базисе \(({e'}_{1},\ \ {e'}_{2},\ \ {e'}_{3}),\) где \({e'}_{1} = e_{1} + 2e_{2},\) \({e'}_{2} = e_{1} - e_{3},\) \({e'}_{3} = e_{1} + e_{2} + e_{3}.\) \\
C2. Даны векторы \(e_{1},e_{2},e_{3}\), \(a_{1},a_{2},a_{3}\) линейного пространства \(R^{3}\). Найдите матрицу перехода от базиса \(e_{1},e_{2},e_{3}\) к базису \(a_{1},a_{2},a_{3}\).
\(e_{1} = ( - 2,3,1)\),\(e_{2} = (0,2,1)\),\(e_{3} = (1,2,1)\) и \(a_{1} = ( - 1,3,7)\),\(a_{2} = (0,2, - 1)\),\(a_{3} = (1, - 2, - 8)\) \\
C3. Найти жорданову нормальную форму матрицы \(A = \begin{pmatrix}
 - 1 & 3 & - 1 \\
 - 3 & 5 & - 1 \\
 - 3 & 3 & 1
\end{pmatrix}\). \\

\end{tabular}
\vspace{1cm}


\begin{tabular}{m{17cm}}
\textbf{45-вариант}
\newline

T1. 13. Инвариантные подпространства. Собственные векторы и собственные значения. \\
T2. 18. Нормальные преобразования и их канонический вид. \\
A1. Доказать, что векторы \(\overrightarrow{a} = (2;\ \ 1;\ \ 1),\) \(\overrightarrow{b} = (1;\ \ 2;\ \ 1)\) и \(\overrightarrow{c} = (2;\ \ 1;\ \ 1)\) образуют базис пространства \(\mathbf{R}^{3},\) и найти координаты вектора \(\overrightarrow{d} = (1;\ \ 3;\ \ 1)\) в этом базисе. \\
A2. Известно, что оператор \emph{A} переводит базисные векторы \(\overrightarrow{i} = (1;\ \ 0;\ \ 0),\) \(\overrightarrow{j} = (0;\ \ 1;\ \ 0),\) \(\overrightarrow{k} = (0;\ \ 0;\ \ 1)\) линейного пространства \(\mathbf{R}^{3}\) в векторы \({\overline{a}}_{1} = (0;\ \ 1;\ \ 1),\) \({\overline{a}}_{2} = (3;\ \ 1;\ \ 1),\) \({\overline{a}}_{3} = (3;\ \ 1;\ \ 1).\) В базисе \(\overrightarrow{i},\overrightarrow{j},\overrightarrow{k}\) найти: 1)матрицу оператора \(A\) ; 2)образ вектора \(\overline{b} = (1;\ \ 1;\ \ 1).\) \\
A3. 
В пространстве \(\mathbb{C}^{3}\) со скалярным произведением \(\left\langle x,y \right\rangle = \sum_{k = 1}^{3}{x_{k}\overline{y_{k}}}\), найдите сопряженный оператор \(A^{*}\) для заданного оператора \(A\). Является ли \(A\)самосопряженным? \(Ax = \left( ix_{1} + x_{3},x_{2} + ix_{1},x_{1} + ix_{3} \right)\); \\
B1. \(a_{1} = ( - 1;3;7)\),\(a_{2} = (0;2; - 1)\), \(a_{3} = (1; - 2; - 8)\); \\
B2. Найти собственные значения и собственные векторы оператора \emph{А}, заданного в некотором базисе пространства \(V^{3}\) матрицей \(A = \begin{bmatrix}
0 & - 2 & 0 \\
 - 2 & 6 & - 2 \\
0 & - 2 & 5
\end{bmatrix};\) \\
B3. Приведите квадратичные формы \(G_{1}\) и \(G_{2}\) к каноническому виду. \(G_{1} = x_{1}^{2} + 5x_{2}^{2} - 4x_{3}^{2} + 2x_{1}x_{3} - 4x_{1}x_{2}\), \(G_{2} = - 4x_{1}x_{2} + 2x_{1}x_{3}\) \\
C1. В базисе \((e_{1},\ \ e_{2},\ \ e_{3})\) пространства \(V^{3}\) оператор \emph{A} имеет матрицу \(A = \begin{bmatrix}
4 & 0 & 1 \\
 - 2 & - 2 & 3 \\
0 & 2 & - 1
\end{bmatrix}\ \ .\) Найти матрицу \emph{B} этого же оператора в базисе \(({e'}_{1},\ \ {e'}_{2},\ \ {e'}_{3}),\) где \({e'}_{1} = e_{1} + 2e_{2},\) \({e'}_{2} = e_{1} - e_{3},\) \({e'}_{3} = e_{1} + e_{2} + e_{3}.\)
 \\
C2. Даны векторы \(e_{1},e_{2},e_{3}\), \(a_{1},a_{2},a_{3}\) линейного пространства \(R^{3}\). Найдите матрицу перехода от базиса \(e_{1},e_{2},e_{3}\) к базису \(a_{1},a_{2},a_{3}\).
\(e_{1} = (3,5,8)\),\(e_{2} = (5,14,13)\),\(e_{3} = (1,9,2)\) и \(a_{1} = ( - 2,3,1)\),\(a_{2} = (0,2,1)\),\(a_{3} = (1,2,1)\) \\
C3. Найти жорданову нормальную форму матрицы \(A = \begin{pmatrix}
 - 1 & 4 & 3 \\
 - 2 & 5 & 3 \\
2 & - 4 & - 2
\end{pmatrix}\). \\

\end{tabular}
\vspace{1cm}


\begin{tabular}{m{17cm}}
\textbf{46-вариант}
\newline

T1. 11. Обратное преобразование. \\
T2. 12. Связь между матрицами линейных преобразовании в разных базисах. \\
A1. Доказать, что векторы \(\overrightarrow{a} = (3;\ \ 1;\ \ 0),\) \(\overrightarrow{b} = (4;\ \ 3;\ \ 2)\) и \(\overrightarrow{c} = ( - 1;\ \  - 4;\ \ 3)\) образуют базис пространства \(\mathbf{R}^{3},\) и найти координаты вектора \(\overrightarrow{d} = ( - 1;\ \ 2;\ \ 5)\) в этом базисе.
 \\
A2. Известно, что оператор \emph{A} переводит базисные векторы \(\overrightarrow{i} = (1;\ \ 0;\ \ 0),\) \(\overrightarrow{j} = (0;\ \ 1;\ \ 0),\) \(\overrightarrow{k} = (0;\ \ 0;\ \ 1)\) линейного пространства \(\mathbf{R}^{3}\) в векторы \({\overline{a}}_{1} = (1;\ \ 1;\ \ 0),\) \({\overline{a}}_{2} = (3;\ \ 2;\ \ 1),\) \({\overline{a}}_{3} = (1;2;\ \ 1).\) В базисе \(\overrightarrow{i},\overrightarrow{j},\overrightarrow{k}\) найти: 1)матрицу оператора \(A\) ; 2)образ вектора \(\overline{b} = (1;\ \ 1;\ \  - 2).\) \\
A3. В пространстве \(\mathbb{C}^{3}\) со скалярным произведением \(\left\langle x,y \right\rangle = \sum_{k = 1}^{3}{x_{k}\overline{y_{k}}}\), найдите сопряженный оператор \(A^{*}\) для заданного оператора \(A\). Является ли \(A\)самосопряженным? \(Ax = \left( ix_{1} + x_{2},x_{1} + ix_{2},x_{2} + ix_{3} \right)\); \\
B1. С помощью процесса ортогонализации Грамма- Шмидта ортонормировать следующие системы векторов, используя стандартное скалярное произведение: \\
B2. Найти собственные значения и собственные векторы оператора \emph{А}, заданного в некотором базисе пространства \(V^{3}\) матрицей \(A = \begin{bmatrix}
2 & 1 & 0 \\
1 & 2 & 0 \\
0 & 0 & - 5
\end{bmatrix};\) \\
B3. 
Приведите квадратичные формы \(G_{1}\) и \(G_{2}\) к каноническому виду. \(G_{1} = x_{1}^{2} + x_{2}^{2} + 3x_{3}^{2} + 4x_{1}x_{2} + 2x_{1}x_{3} + 2x_{2}x_{3}\), \(G_{2} = x_{1}x_{2} + x_{1}x_{3} + x_{2}x_{3}\) \\
C1. В базисе \((e_{1},\ \ e_{2},\ \ e_{3})\) пространства \(V^{3}\) оператор \emph{A} имеет матрицу \(A = \begin{bmatrix}
1 & - 1 & 2 \\
0 & 3 & - 1 \\
4 & 2 & 2
\end{bmatrix}\ \ .\) Найти матрицу \emph{B} этого же оператора в базисе \(({e'}_{1},\ \ {e'}_{2},\ \ {e'}_{3}),\) где \({e'}_{1} = e_{1} + 2e_{2},\) \({e'}_{2} = e_{1} - e_{3},\) \({e'}_{3} = e_{1} + e_{2} + e_{3}.\) \\
C2. 
Даны векторы \(e_{1},e_{2},e_{3}\), \(a_{1},a_{2},a_{3}\) линейного пространства \(R^{3}\). Найдите матрицу перехода от базиса \(e_{1},e_{2},e_{3}\) к базису \(a_{1},a_{2},a_{3}\).
\(e_{1} = (2,1, - 3)\),\(e_{2} = (3,2, - 5)\),\(e_{3} = (1, - 1,1)\) и \(a_{1} = (0,1, - 2)\),\(a_{2} = ( - 2,0,3)\),\(a_{3} = (1, - 1,1)\) \\
C3. Найти жорданову нормальную форму матрицы \(A = \begin{pmatrix}
0 & 1 & 0 \\
 - 4 & 4 & 0 \\
0 & 0 & 2
\end{pmatrix}\) \\

\end{tabular}
\vspace{1cm}


\begin{tabular}{m{17cm}}
\textbf{47-вариант}
\newline

T1. 9. Линейные преобразования и их матрица. \\
T2. 20. Подобные матрицы. \\
A1. Доказать, что векторы \(\overrightarrow{a} = (3;\ \ 4;\ \ 3),\) \(\overrightarrow{b} = ( - 2;\ \ 3;\ \ 1)\) и \(\overrightarrow{c} = (4;\ \  - 2;\ \ 3)\) образуют базис пространства \(\mathbf{R}^{3},\) и найти координаты вектора \(\overrightarrow{d} = ( - 17;\ \ 18;\ \  - 7)\) в этом базисе. \\
A2. Известно, что оператор \emph{A} переводит базисные векторы \(\overrightarrow{i} = (1;\ \ 0;\ \ 0),\) \(\overrightarrow{j} = (0;\ \ 1;\ \ 0),\) \(\overrightarrow{k} = (0;\ \ 0;\ \ 1)\) линейного пространства \(\mathbf{R}^{3}\) в векторы \({\overline{a}}_{1} = (1;\ \ 0;\ \ 1),\) \({\overline{a}}_{2} = (0;\ \ 2;\ \ 1),\) \({\overline{a}}_{3} = (3;\ \ 1;\ \ 1).\) В базисе \(\overrightarrow{i},\overrightarrow{j},\overrightarrow{k}\) найти: 1)матрицу оператора \(A\) ; 2)образ вектора \(\overline{b} = (1;\ \ 2;\ \ 3).\) \\
A3. В пространстве \(\mathbb{C}^{3}\) со скалярным произведением \(\left\langle x,y \right\rangle = \sum_{k = 1}^{3}{x_{k}\overline{y_{k}}}\), найдите сопряженный оператор \(A^{*}\) для заданного оператора \(A\). Является ли \(A\)самосопряженным?\(Ax = \left( x_{1} + 2ix_{3},2ix_{1} + ix_{2},x_{1} + ix_{3} \right)\); \\
B1. \(a_{1} = (2; - 1;3)\), \(a_{2} = (3;2; - 5)\), \(a_{3} = (1; - 1;1)\); \\
B2. Найти собственные значения и собственные векторы оператора \emph{А}, заданного в некотором базисе пространства \(V^{3}\) матрицей \(A = \begin{pmatrix}
0 & 1 & 2 \\
 - 1 & 0 & - 2 \\
 - 2 & 2 & 0
\end{pmatrix}\).
 \\
B3. Приведите квадратичные формы \(G_{1}\) и \(G_{2}\) к каноническому виду. \(G_{1} = 2x_{1}^{2} + x_{2}^{2} + x_{3}^{2} + 4x_{1}x_{2} - 2x_{1}x_{3}\), \(G_{2} = x_{1}x_{2} + x_{1}x_{3} + 4x_{2}x_{3}\) \\
C1. В базисе \((e_{1},\ \ e_{2},\ \ e_{3})\) пространства \(V^{3}\) оператор \emph{A} имеет матрицу \(A = \begin{bmatrix}
 - 1 & 2 & 1 \\
0 & 1 & - 4 \\
5 & - 1 & 2
\end{bmatrix}\ \ .\) Найти матрицу \emph{B} этого же оператора в базисе \(({e'}_{1},\ \ {e'}_{2},\ \ {e'}_{3}),\) где \({e'}_{1} = e_{1} - e_{3},\) \({e'}_{2} = e_{2} + e_{3},\) \({e'}_{3} = e_{3}.\) \\
C2. 12.Даны векторы \(e_{1},e_{2},e_{3}\), \(a_{1},a_{2},a_{3}\) линейного пространства \(R^{3}\). Найдите матрицу перехода от базиса \(e_{1},e_{2},e_{3}\) к базису \(a_{1},a_{2},a_{3}\).
\(e_{1} = (1,1,1)\),\(e_{2} = (1,1,2)\),\(e_{3} = (1,2,3)\) и \(a_{1} = (2,0,1)\),\(a_{2} = ( - 1,2,3)\),\(a_{3} = ( - 1,1,1)\)
 \\
C3. Найти жорданову нормальную форму матрицы \(A = \begin{pmatrix}
0 & 3 & 1 \\
3 & 0 & 1 \\
 - 2 & 2 & 1
\end{pmatrix}\) \\

\end{tabular}
\vspace{1cm}


\begin{tabular}{m{17cm}}
\textbf{48-вариант}
\newline

T1. 7. Комплексные евклидовы пространства. \\
T2. 8. Квадратичные формы в комплексном пространстве и их канонические виды. \\
A1. Доказать, что векторы \(\overrightarrow{a} = (2;\ \ 1;\ \ 1),\) \(\overrightarrow{b} = (1;\ \ 2;\ \ 1)\) и \(\overrightarrow{c} = (2;\ \ 3;\ \  - 1)\) образуют базис пространства \(\mathbf{R}^{3},\) и найти координаты вектора \(\overrightarrow{d} = (2;\ \ 3;\ \  - 1)\) в этом базисе. \\
A2. Известно, что оператор \emph{A} переводит базисные векторы \(\overrightarrow{i} = (1;\ \ 0;\ \ 0),\) \(\overrightarrow{j} = (0;\ \ 1;\ \ 0),\) \(\overrightarrow{k} = (0;\ \ 0;\ \ 1)\) линейного пространства \(\mathbf{R}^{3}\) в векторы \({\overline{a}}_{1} = (0;\ \ 1;\ \ 1),\) \({\overline{a}}_{2} = (3;\ \ 1;\ \ 1),\) \({\overline{a}}_{3} = (3;0;1).\) В базисе \(\overrightarrow{i},\overrightarrow{j},\overrightarrow{k}\) найти: 1)матрицу оператора \(A\) ; 2)образ вектора \(\overline{b} = (1;\ \  - 1;\ \  - 1).\) \\
A3. В пространстве \(\mathbb{C}^{3}\) со скалярным произведением \(\left\langle x,y \right\rangle = \sum_{k = 1}^{3}{x_{k}\overline{y_{k}}}\), найдите сопряженный оператор \(A^{*}\) для заданного оператора \(A\). Является ли \(A\)самосопряженным? \(Ax = \left( x_{2} + ix_{3},x_{1} - ix_{2},x_{1} + ix_{2} + x_{3} \right)\) \\
B1. С помощью процесса ортогонализации Грамма- Шмидта ортонормировать следующие системы векторов, используя стандартное скалярное произведение: \\
B2. Найти собственные значения и собственные векторы оператора \emph{А}, заданного в некотором базисе пространства \(V^{3}\) матрицей \(A = \begin{bmatrix}
 - 1 & - 2 & 0 \\
0 & - 2 & 0 \\
2 & 2 & 1
\end{bmatrix}.\) \\
B3. Приведите квадратичные формы \(G_{1}\) и \(G_{2}\) к каноническому виду. \(G_{1} = x_{1}^{2} - 2x_{3}^{2} + 2x_{1}x_{3} - 6x_{1}x_{2}\), \(G_{2} = 6x_{2}x_{3} - 4x_{1}x_{2} + x_{1}x_{3}\)
 \\
C1. В базисе \((e_{1},\ \ e_{2},\ \ e_{3})\) пространства \(V^{3}\) оператор \emph{A} имеет матрицу \(A = \begin{bmatrix}
1 & 3 & - 1 \\
2 & 0 & 4 \\
1 & 1 & 1
\end{bmatrix}\ \ .\) Найти матрицу \emph{B} этого же оператора в базисе \(({e'}_{1},\ \ {e'}_{2},\ \ {e'}_{3}),\) где \({e'}_{1} = 2e_{1} + e_{2}\), \({e'}_{2} = - e_{1} + 2e_{2} + 3e_{3}\),\({e'}_{3} = - e_{1} + e_{2} + e_{3}\) \\
C2. Даны векторы \(e_{1},e_{2},e_{3}\), \(a_{1},a_{2},a_{3}\) линейного пространства \(R^{3}\). Найдите матрицу перехода от базиса \(e_{1},e_{2},e_{3}\) к базису \(a_{1},a_{2},a_{3}\).
\(e_{1} = (1,0,2)\),\(e_{2} = (3, - 1,4)\),\(e_{3} = (2, - 2,1)\) и \(a_{1} = (4,0,5)\),\(a_{2} = ( - 2,1,3)\),\(a_{3} = ( - 5,1, - 1)\) \\
C3. Найти жорданову нормальную форму матрицы \(A = \begin{pmatrix}
 - 1 & 4 & 3 \\
 - 2 & 5 & 3 \\
2 & - 4 & - 2
\end{pmatrix}\). \\

\end{tabular}
\vspace{1cm}


\begin{tabular}{m{17cm}}
\textbf{49-вариант}
\newline

T1. 19. Полиномиальные матрицы и диагональные нормальные формы. \\
T2. 6. Положительно определенные квадратичные формы. \\
A1. Доказать, что векторы \(\overrightarrow{a} = (2;\ \ 1;\ \ 1),\) \(\overrightarrow{b} = (1;\ \ 2;\ \ 1)\) и \(\overrightarrow{c} = (2;\ \ 1;\ \ 1)\) образуют базис пространства \(\mathbf{R}^{3},\) и найти координаты вектора \(\overrightarrow{d} = (1;\ \ 3;\ \ 1)\) в этом базисе. \\
A2. Известно, что оператор \emph{A} переводит базисные векторы \(\overrightarrow{i} = (1;\ \ 0;\ \ 0),\) \(\overrightarrow{j} = (0;\ \ 1;\ \ 0),\) \(\overrightarrow{k} = (0;\ \ 0;\ \ 1)\) линейного пространства \(\mathbf{R}^{3}\) в векторы \({\overline{a}}_{1} = (1;\ \ 1;\ \ 1),\) \({\overline{a}}_{2} = (3;\ \ 2;\ \ 1),\) \({\overline{a}}_{3} = (0;\ \ 1;\ \ 1).\) В базисе \(\overrightarrow{i},\overrightarrow{j},\overrightarrow{k}\) найти: 1)матрицу оператора \(A\) ; 2)образ вектора \(\overline{b} = (1;\ \  - 2;\ \ 3).\)
 \\
A3. В пространстве \(\mathbb{C}^{3}\) со скалярным произведением \(\left\langle x,y \right\rangle = \sum_{k = 1}^{3}{x_{k}\overline{y_{k}}}\), найдите сопряженный оператор \(A^{*}\) для заданного оператора \(A\). Является ли \(A\)самосопряженным?\(Ax = \left( x_{1} - 2ix_{2},x_{3} + 2ix_{2},ix_{2} + 2ix_{3} \right)\);
 \\
B1. \(a_{1} = (1;1;1)\), \(a_{2} = (1;2;3)\), \(a_{3} = (1;1;2)\); \\
B2. Найти собственные значения и собственные векторы оператора \emph{А}, заданного в некотором базисе пространства \(V^{3}\) матрицей \(A = \begin{pmatrix}
2 & 1 & 0 \\
1 & 3 & - 1 \\
 - 1 & 2 & 3
\end{pmatrix}\). \\
B3. Приведите квадратичные формы \(G_{1}\) и \(G_{2}\) к каноническому виду. \(G_{1} = 2x_{1}^{2} + 3x_{2}^{2} + 4x_{3}^{2} - 2x_{1}x_{2} + 4x_{1}x_{3} - 3x_{2}x_{3}\), \(G_{2} = x_{1}x_{3} - 2x_{2}x_{3}\) \\
C1. В базисе \((e_{1},\ \ e_{2},\ \ e_{3})\) пространства \(V^{3}\) оператор \emph{A} имеет матрицу \(A = \begin{bmatrix}
1 & 3 & - 1 \\
2 & 0 & 4 \\
1 & 1 & 1
\end{bmatrix}\ \ .\) Найти матрицу \emph{B} этого же оператора в базисе \(({e'}_{1},\ \ {e'}_{2},\ \ {e'}_{3}),\) где \({e'}_{1} = 2e_{1} + e_{2}\), \({e'}_{2} = - e_{1} + 2e_{2} + 3e_{3}\),\({e'}_{3} = - e_{1} + e_{2} + e_{3}\) \\
C2. Даны векторы \(e_{1},e_{2},e_{3}\), \(a_{1},a_{2},a_{3}\) линейного пространства \(R^{3}\). Найдите матрицу перехода от базиса \(e_{1},e_{2},e_{3}\) к базису \(a_{1},a_{2},a_{3}\).
\(e_{1} = (1,0,2)\),\(e_{2} = (3, - 1,4)\),\(e_{3} = (2, - 2,1)\) и \(a_{1} = (4,0,5)\),\(a_{2} = ( - 2,1,3)\),\(a_{3} = ( - 5,1, - 1)\) \\
C3. Найти жорданову нормальную форму матрицы \(A = \begin{pmatrix}
2 & - 1 & 2 \\
5 & - 3 & 3 \\
 - 1 & 0 & - 2
\end{pmatrix}\). \\

\end{tabular}
\vspace{1cm}


\begin{tabular}{m{17cm}}
\textbf{50-вариант}
\newline

T1. 3. Ортогональное дополнение и ортогональная проекция. \\
T2. 14. Сопряженное преобразование для данного преобразования. \\
A1. Доказать, что векторы \(\overrightarrow{a} = (1;\ \ 0;\ \ 1),\) \(\overrightarrow{b} = (1;\ \ 1;\ \ 1)\) и \(\overrightarrow{c} = ( - 1;\ \ 2;\ \ 1)\) образуют базис пространства \(\mathbf{R}^{3},\) и найти координаты вектора \(\overrightarrow{d} = (0;\ \ 10;\ \ 3)\) в этом базисе. \\
A2. Известно, что оператор \emph{A} переводит базисные векторы \(\overrightarrow{i} = (1;\ \ 0;\ \ 0),\) \(\overrightarrow{j} = (0;\ \ 1;\ \ 0),\) \(\overrightarrow{k} = (0;\ \ 0;\ \ 1)\) линейного пространства \(\mathbf{R}^{3}\) в векторы \({\overline{a}}_{1} = (1;\ \ 1;\ \ 1),{\overline{a}}_{2} = (3;\ \ 0;\ \ 1),\) \({\overline{a}}_{3} = (3;\ \ 1;\ \  - 1).\)В базисе \(\overrightarrow{i},\overrightarrow{j},\overrightarrow{k}\) найти:1)матрицу оператора \(A\) ;2)образ вектора \(\overline{b} = (1;\ \ 2;\ \ 1).\) \\
A3. В пространстве \(\mathbb{C}^{3}\) со скалярным произведением \(\left\langle x,y \right\rangle = \sum_{k = 1}^{3}{x_{k}\overline{y_{k}}}\), найдите сопряженный оператор \(A^{*}\) для заданного оператора \(A\). Является ли \(A\)самосопряженным? \(Ax = \left( ix_{1} + x_{3},ix_{3} - ix_{2},x_{1} - ix_{3} \right)\); \\
B1. С помощью процесса ортогонализации Грамма- Шмидта ортонормировать следующие системы векторов, используя стандартное скалярное произведение: \\
B2. Найти собственные значения и собственные векторы оператора \emph{А}, заданного в некотором базисе пространства \(V^{3}\) матрицей \(A = \begin{bmatrix}
 - 1 & 1 & 0 \\
 - 4 & 3 & 0 \\
 - 2 & 1 & 1
\end{bmatrix};\) \\
B3. Приведите квадратичные формы \(G_{1}\) и \(G_{2}\) к каноническому виду. \(G_{1} = 3x_{1}^{2} - 2x_{2}^{2} + 2x_{3}^{2} + 4x_{1}x_{2} - 3x_{1}x_{3} - x_{2}x_{3}\), \(G_{2} = 2x_{1}x_{3} + 4x_{1}x_{2} - 2x_{2}x_{3}\) \\
C1. В базисе \((e_{1},\ \ e_{2},\ \ e_{3})\) пространства \(V^{3}\) оператор \emph{A} имеет матрицу \(A = \begin{bmatrix}
0 & 1 & - 3 \\
2 & 4 & 1 \\
0 & 3 & - 3
\end{bmatrix}\ \ .\) Найти матрицу \emph{B} этого же оператора в базисе \(({e'}_{1},\ \ {e'}_{2},\ \ {e'}_{3}),\) где \({e'}_{1} = 2e_{1} + e_{2}\), \({e'}_{2} = - e_{1} + 2e_{2} + 3e_{3}\),\({e'}_{3} = - e_{1} + e_{2} + e_{3}\) \\
C2. 10.Даны векторы \(e_{1},e_{2},e_{3}\), \(a_{1},a_{2},a_{3}\) линейного пространства \(R^{3}\). Найдите матрицу перехода от базиса \(e_{1},e_{2},e_{3}\) к базису \(a_{1},a_{2},a_{3}\).
\(e_{1} = (4,0,5)\),\(e_{2} = ( - 2,1,3)\),\(e_{3} = ( - 5,1, - 1)\) и \(a_{1} = (1,2,1)\),\(a_{2} = (2,3,3)\),\(a_{3} = (3,8,2)\) \\
C3. Найти жорданову нормальную форму матрицы \(A = \begin{pmatrix}
2 & 1 & 1 \\
1 & 2 & 1 \\
1 & 1 & 2
\end{pmatrix}\). \\

\end{tabular}
\vspace{1cm}


\begin{tabular}{m{17cm}}
\textbf{51-вариант}
\newline

T1. 15. Самосопряженные преобразования и их канонический вид. \\
T2. 10. Ядро, образ линйеного преобразования. \\
A1. Доказать, что векторы \(\overrightarrow{a} = (1;\ \ 2;\ \ 1),\) \(\overrightarrow{b} = (1;\ \ 1;\ \  - 3)\) и \(\overrightarrow{c} = ( - 1;\ \ 2;\ \ 1)\) образуют базис пространства \(\mathbf{R}^{3},\) и найти координаты вектора \(\overrightarrow{d} = (0;\ \ 10;\ \  - 2)\) в этом базисе. \\
A2. Известно, что оператор \emph{A} переводит базисные векторы \(\overrightarrow{i} = (1;\ \ 0;\ \ 0),\) \(\overrightarrow{j} = (0;\ \ 1;\ \ 0),\) \(\overrightarrow{k} = (0;\ \ 0;\ \ 1)\) линейного пространства \(\mathbf{R}^{3}\) в векторы \({\overline{a}}_{1} = (1;\ \ 0;\ \ 1),\) \({\overline{a}}_{2} = (3;\ \ 2;\ \ 1),\) \({\overline{a}}_{3} = (3;\ \ 1;\ \ 1).\) В базисе \(\overrightarrow{i},\overrightarrow{j},\overrightarrow{k}\) найти: 1)матрицу оператора \(A\) ; 2)образ вектора \(\overline{b} = (1;\ \  - 2;\ \ 3).\) \\
A3. В пространстве \(\mathbb{C}^{3}\) со скалярным произведением \(\left\langle x,y \right\rangle = \sum_{k = 1}^{3}{x_{k}\overline{y_{k}}}\), найдите сопряженный оператор \(A^{*}\) для заданного оператора \(A\). Является ли \(A\)самосопряженным? \(Ax = \left( x_{1} + 2ix_{3},ix_{2} - x_{3},x_{2} - ix_{3} \right)\); \\
B1. С помощью процесса ортогонализации Грамма- Шмидта ортонормировать следующие системы векторов, используя стандартное скалярное произведение: \\
B2. Найти собственные значения и собственные векторы оператора \emph{А}, заданного в некотором базисе пространства \(V^{3}\) матрицей \(A = \begin{bmatrix}
0 & - 1 & 1 \\
 - 1 & 0 & 1 \\
1 & 1 & 0
\end{bmatrix};\) \\
B3. Приведите квадратичные формы \(G_{1}\) и \(G_{2}\) к каноническому виду. \(G_{1} = 4x_{1}^{2} + x_{2}^{2} + x_{3}^{2} - 4x_{1}x_{2} + 4x_{1}x_{3} - 3x_{2}x_{3}\), \(G_{2} = x_{1}x_{2} + 6x_{1}x_{3} - 4x_{2}x_{3}\) \\
C1. В базисе \((e_{1},\ \ e_{2},\ \ e_{3})\) пространства \(V^{3}\) оператор \emph{A} имеет матрицу \(A = \begin{bmatrix}
3 & 2 & - 1 \\
4 & 0 & 2 \\
 - 1 & 2 & - 1
\end{bmatrix}\ \ .\) Найти матрицу \emph{B} этого же оператора в базисе \(({e'}_{1},\ \ {e'}_{2},\ \ {e'}_{3}),\) где \({e'}_{1} = 2e_{1} + e_{2}\), \({e'}_{2} = - e_{1} + 2e_{2} + 3e_{3}\),\({e'}_{3} = - e_{1} + e_{2} + e_{3}\) \\
C2. Даны векторы \(e_{1},e_{2},e_{3}\), \(a_{1},a_{2},a_{3}\) линейного пространства \(R^{3}\). Найдите матрицу перехода от базиса \(e_{1},e_{2},e_{3}\) к базису \(a_{1},a_{2},a_{3}\).
\(e_{1} = ( - 3,0,1)\),\(e_{2} = (0,2,3)\),\(e_{3} = ( - 1, - 1, - 1)\) и \(a_{1} = (1,1,1)\),\(a_{2} = (1,1,2)\),\(a_{3} = (1,2,3)\) \\
C3. Найти жорданову нормальную форму матрицы \(A = \begin{pmatrix}
 - 1 & 4 & 3 \\
 - 2 & 5 & 3 \\
2 & - 4 & - 2
\end{pmatrix}\). \\

\end{tabular}
\vspace{1cm}


\begin{tabular}{m{17cm}}
\textbf{52-вариант}
\newline

T1. 9. Линейные преобразования и их матрица. \\
T2. 2. Евклидово пространство. Неравенство Коши-Буняковского. Процесс ортогонализации. \\
A1. Доказать, что векторы \(\overrightarrow{a} = (2;\ \ 1;1),\) \(\overrightarrow{b} = ( - 1;\ \ 2;\ \ 4)\) и \(\overrightarrow{c} = (3;\ \ 3;\ \ 2)\) образуют базис пространства \(\mathbf{R}^{3},\) и найти координаты вектора \(\overrightarrow{d} = ( - 4;\ \ 2;\ \ 4)\) в этом базисе. \\
A2. Известно, что оператор \emph{A} переводит базисные векторы \(\overrightarrow{i} = (1;\ \ 0;\ \ 0),\) \(\overrightarrow{j} = (0;\ \ 1;\ \ 0),\) \(\overrightarrow{k} = (0;\ \ 0;\ \ 1)\) линейного пространства \(\mathbf{R}^{3}\) в векторы \({\overline{a}}_{1} = (1;\ \ 0;\ \ 1),\) \({\overline{a}}_{2} = (0;\ \ 1;\ \ 1),\) \({\overline{a}}_{3} = (3;\ \ 1;\ \ 1).\) В базисе \(\overrightarrow{i},\overrightarrow{j},\overrightarrow{k}\) найти: 1)матрицу оператора \(A\) ; 2)образ вектора \(\overline{b} = (1;\ \  - 2;\ \  - 3).\) \\
A3. В пространстве \(\mathbb{C}^{3}\) со скалярным произведением \(\left\langle x,y \right\rangle = \sum_{k = 1}^{3}{x_{k}\overline{y_{k}}}\), найдите сопряженный оператор \(A^{*}\) для заданного оператора \(A\). Является ли \(A\)самосопряженным?\(Ax = \left( x_{1} + 2ix_{2},x_{3} - ix_{2},x_{1} - ix_{2} - 2ix_{3} \right)\); \\
B1. \(a_{1} = (2;4;3)\), \(a_{2} = (3; - 1;4)\), \(a_{3} = (1;5; - 1)\); \\
B2. Найти собственные значения и собственные векторы оператора \emph{А}, заданного в некотором базисе пространства \(V^{3}\) матрицей \(A = \begin{pmatrix}
1 & - 1 & 1 \\
1 & 1 & - 1 \\
2 & - 1 & 0
\end{pmatrix}\). \\
B3. Приведите квадратичные формы \(G_{1}\) и \(G_{2}\) к каноническому виду. \(G_{1} = 2x_{1}^{2} + x_{2}^{2} + x_{3}^{2} + 4x_{1}x_{2} - 2x_{1}x_{3}\), \(G_{2} = x_{1}x_{2} + x_{1}x_{3} + 4x_{2}x_{3}\) \\
C1. В базисе \((e_{1},\ \ e_{2},\ \ e_{3})\) пространства \(V^{3}\) оператор \emph{A} имеет матрицу \(A = \begin{bmatrix}
 - 1 & 2 & 4 \\
 - 4 & 2 & 0 \\
3 & 3 & - 3
\end{bmatrix}\ \ .\) Найти матрицу \emph{B} этого же оператора в базисе \(({e'}_{1},\ \ {e'}_{2},\ \ {e'}_{3}),\) где \({e'}_{1} = 2e_{1} + e_{2}\), \({e'}_{2} = - e_{1} + 2e_{2} + 3e_{3}\),\({e'}_{3} = - e_{1} + e_{2} + e_{3}\) \\
C2. Даны векторы \(e_{1},e_{2},e_{3}\), \(a_{1},a_{2},a_{3}\) линейного пространства \(R^{3}\). Найдите матрицу перехода от базиса \(e_{1},e_{2},e_{3}\) к базису \(a_{1},a_{2},a_{3}\).
\(e_{1} = ( - 2,3,1)\),\(e_{2} = (0,2,1)\),\(e_{3} = (1,2,1)\) и \(a_{1} = ( - 1,3,7)\),\(a_{2} = (0,2, - 1)\),\(a_{3} = (1, - 2, - 8)\) \\
C3. Найти жорданову нормальную форму матрицы \(A = \begin{pmatrix}
3 & - 2 & 6 \\
 - 2 & 6 & 3 \\
6 & 3 & - 2
\end{pmatrix}\). \\

\end{tabular}
\vspace{1cm}


\begin{tabular}{m{17cm}}
\textbf{53-вариант}
\newline

T1. 19. Полиномиальные матрицы и диагональные нормальные формы. \\
T2. 16. Унитарные преобразования и их собственные значения и канонический вид. \\
A1. Доказать, что векторы \(\overrightarrow{a} = (2;\ \ 1;\ \ 1),\) \(\overrightarrow{b} = (1;\ \ 2;\ \ 1)\) и \(\overrightarrow{c} = (2;\ \ 3;\ \  - 1)\) образуют базис пространства \(\mathbf{R}^{3},\) и найти координаты вектора \(\overrightarrow{d} = (2;\ \ 3;\ \  - 1)\) в этом базисе. \\
A2. Известно, что оператор \emph{A} переводит базисные векторы \(\overrightarrow{i} = (1;\ \ 0;\ \ 0),\) \(\overrightarrow{j} = (0;\ \ 1;\ \ 0),\) \(\overrightarrow{k} = (0;\ \ 0;\ \ 1)\) линейного пространства \(\mathbf{R}^{3}\) в векторы \({\overline{a}}_{1} = (1;\ \ 1;\ \ 0),\) \({\overline{a}}_{2} = (3;\ \ 2;\ \ 1),\) \({\overline{a}}_{3} = (3;\ \ 1;\ \ 1).\) В базисе \(\overrightarrow{i},\overrightarrow{j},\overrightarrow{k}\) найти: 1)матрицу оператора \(A\) ; 2)образ вектора \(\overline{b} = (1;\ \ 2;\ \ 3).\) \\
A3. В пространстве \(\mathbb{C}^{3}\) со скалярным произведением \(\left\langle x,y \right\rangle = \sum_{k = 1}^{3}{x_{k}\overline{y_{k}}}\), найдите сопряженный оператор \(A^{*}\) для заданного оператора \(A\). Является ли \(A\)самосопряженным? \(Ax = \left( x_{1} + ix_{3},x_{3} + 2ix_{2},ix_{2} - 2ix_{3} \right)\); \\
B1. \(a_{1} = (2;0;1)\), \(a_{2} = ( - 1;2;3)\), \(a_{3} = ( - 1;1;1)\); \\
B2. Найти собственные значения и собственные векторы оператора \emph{А}, заданного в некотором базисе пространства \(V^{3}\) матрицей \(A = \begin{pmatrix}
1 & - 2 & - 1 \\
 - 1 & 1 & 1 \\
1 & 0 & - 1
\end{pmatrix}\) \\
B3. Приведите квадратичные формы \(G_{1}\) и \(G_{2}\) к каноническому виду. \(G_{1} = x_{1}^{2} - 2x_{2}^{2} + x_{3}^{2} + 2x_{1}x_{2} + 4x_{1}x_{3} + 2x_{2}x_{3}\), \(G_{2} = 2x_{1}x_{3} - 4x_{2}x_{3}\) \\
C1. В базисе \((e_{1},\ \ e_{2},\ \ e_{3})\) пространства \(V^{3}\) оператор \emph{A} имеет матрицу \(A = \begin{bmatrix}
0 & 1 & - 2 \\
3 & 5 & 1 \\
 - 1 & 2 & 0
\end{bmatrix}\ \ .\) Найти матрицу \emph{B} этого же оператора в базисе \(({e'}_{1},\ \ {e'}_{2},\ \ {e'}_{3}),\) где \({e'}_{1} = e_{1} + 2e_{2},\) \({e'}_{2} = e_{1} - e_{3},\) \({e'}_{3} = e_{1} + e_{2} + e_{3}.\) \\
C2. 11.Даны векторы \(e_{1},e_{2},e_{3}\), \(a_{1},a_{2},a_{3}\) линейного пространства \(R^{3}\). Найдите матрицу перехода от базиса \(e_{1},e_{2},e_{3}\) к базису \(a_{1},a_{2},a_{3}\).
\(e_{1} = ( - 1,3,7)\),\(e_{2} = (0,2, - 1)\),\(e_{3} = (1, - 2, - 8)\) и \(a_{1} = (0,3, - 2)\),\(a_{2} = (1, - 1, - 8)\),\(a_{3} = ( - 1,2,7)\) \\
C3. 
Найти жорданову нормальную форму матрицы \(A = \begin{pmatrix}
 - 1 & 1 & - 2 \\
3 & - 3 & 6 \\
2 & - 2 & 4
\end{pmatrix}\). \\

\end{tabular}
\vspace{1cm}


\begin{tabular}{m{17cm}}
\textbf{54-вариант}
\newline

T1. 7. Комплексные евклидовы пространства. \\
T2. 8. Квадратичные формы в комплексном пространстве и их канонические виды. \\
A1. Доказать, что векторы \(\overrightarrow{a} = (1;\ \ 2;\ \ 1),\) \(\overrightarrow{b} = (1;\ \ 1;\ \ 3)\) и \(\overrightarrow{c} = ( - 1;\ \ 2;\ \ 1)\) образуют базис пространства \(\mathbf{R}^{3},\) и найти координаты вектора \(\overrightarrow{d} = (0;\ \ 10;\ \  - 2)\) в этом базисе. \\
A2. Известно, что оператор \emph{A} переводит базисные векторы \(\overrightarrow{i} = (1;\ \ 0;\ \ 0),\) \(\overrightarrow{j} = (0;\ \ 1;\ \ 0),\) \(\overrightarrow{k} = (0;\ \ 0;\ \ 1)\) линейного пространства \(\mathbf{R}^{3}\) в векторы \({\overline{a}}_{1} = (1;\ \ 1;\ \ 1),{\overline{a}}_{2} = (3;\ \ 0;\ \ 1),\) \({\overline{a}}_{3} = (0;\ \ 2;\ \ 1).\)В базисе \(\overrightarrow{i},\overrightarrow{j},\overrightarrow{k}\) найти:1)матрицу оператора \(A\) ;2)образ вектора \(\overline{b} = (1;\ \ 2;\ \  - 2).\) \\
A3. В пространстве \(\mathbb{C}^{3}\) со скалярным произведением \(\left\langle x,y \right\rangle = \sum_{k = 1}^{3}{x_{k}\overline{y_{k}}}\), найдите сопряженный оператор \(A^{*}\) для заданного оператора \(A\). Является ли \(A\)самосопряженным? \(Ax = \left( ix_{1} + 2ix_{3},x_{3},x_{1} - 2ix_{3} \right)\); \\
B1. С помощью процесса ортогонализации Грамма- Шмидта ортонормировать следующие системы векторов, используя стандартное скалярное произведение: \\
B2. Найти собственные значения и собственные векторы оператора \emph{А}, заданного в некотором базисе пространства \(V^{3}\) матрицей \(A = \begin{pmatrix}
2 & - 1 & 2 \\
1 & 0 & 2 \\
 - 2 & 1 & - 1
\end{pmatrix}\) \\
B3. Приведите квадратичные формы \(G_{1}\) и \(G_{2}\) к каноническому виду. \(G_{1} = x_{1}^{2} - 2x_{3}^{2} + 2x_{1}x_{3} - 6x_{1}x_{2}\), \(G_{2} = 6x_{2}x_{3} - 4x_{1}x_{2} + x_{1}x_{3}\)
 \\
C1. В базисе \((e_{1},\ \ e_{2},\ \ e_{3})\) пространства \(V^{3}\) оператор \emph{A} имеет матрицу \(A = \begin{bmatrix}
 - 1 & 2 & 1 \\
0 & 1 & - 4 \\
5 & - 1 & 2
\end{bmatrix}\ \ .\) Найти матрицу \emph{B} этого же оператора в базисе \(({e'}_{1},\ \ {e'}_{2},\ \ {e'}_{3}),\) где \({e'}_{1} = e_{1} - e_{3},\) \({e'}_{2} = e_{2} + e_{3},\) \({e'}_{3} = e_{3}.\) \\
C2. Даны векторы \(e_{1},e_{2},e_{3}\), \(a_{1},a_{2},a_{3}\) линейного пространства \(R^{3}\). Найдите матрицу перехода от базиса \(e_{1},e_{2},e_{3}\) к базису \(a_{1},a_{2},a_{3}\).
\(e_{1} = (2,0,1)\),\(e_{2} = ( - 1,2,3)\),\(e_{3} = ( - 1,1,1)\) и \(a_{1} = ( - 3,0,1)\),\(a_{2} = (0,2,3)\),\(a_{3} = ( - 1, - 1, - 1)\) \\
C3. Найти жорданову нормальную форму матрицы \(A = \begin{pmatrix}
 - 1 & 4 & 3 \\
 - 2 & 5 & 3 \\
2 & - 4 & - 2
\end{pmatrix}\). \\

\end{tabular}
\vspace{1cm}


\begin{tabular}{m{17cm}}
\textbf{55-вариант}
\newline

T1. 1. Линейные пространства. Линейные подпространства. Сумма и пересечение подпространств. \\
T2. 20. Подобные матрицы. \\
A1. Доказать, что векторы \(\overrightarrow{a} = (2;\ \ 1;\ \  - 3),\) \(\overrightarrow{b} = ( - 1;\ \ 2;\ \ 4)\) и \(\overrightarrow{c} = (3;\ \  - 4;\ \ 2)\) образуют базис пространства \(\mathbf{R}^{3},\) и найти координаты вектора \(\overrightarrow{d} = ( - 4;\ \ 19;\ \ 3)\) в этом базисе. \\
A2. Известно, что оператор \emph{A} переводит базисные векторы \(\overrightarrow{i} = (1;\ \ 0;\ \ 0),\) \(\overrightarrow{j} = (0;\ \ 1;\ \ 0),\) \(\overrightarrow{k} = (0;\ \ 0;\ \ 1)\) линейного пространства \(\mathbf{R}^{3}\) в векторы \({\overline{a}}_{1} = (1;\ \ 0;\ \ 1),\) \({\overline{a}}_{2} = (0;\ \ 2;\ \ 1),\) \({\overline{a}}_{3} = (3;\ \ 1;\ \ 1).\) В базисе \(\overrightarrow{i},\overrightarrow{j},\overrightarrow{k}\) найти: 1)матрицу оператора \(A\) ; 2)образ вектора \(\overline{b} = (1;\ \ 2;\ \ 3).\) \\
A3. В пространстве \(\mathbb{C}^{3}\) со скалярным произведением \(\left\langle x,y \right\rangle = \sum_{k = 1}^{3}{x_{k}\overline{y_{k}}}\), найдите сопряженный оператор \(A^{*}\) для заданного оператора \(A\). Является ли \(A\)самосопряженным? \(Ax = \left( ix_{1} + x_{2},x_{1} + ix_{2},x_{2} + ix_{3} \right)\); \\
B1. \(a_{1} = ( - 1;3;7)\),\(a_{2} = (0;2; - 1)\), \(a_{3} = (1; - 2; - 8)\); \\
B2. 
Найти собственные значения и собственные векторы оператора \emph{А}, заданного в некотором базисе пространства \(V^{3}\) матрицей \(A = \begin{bmatrix}
0 & - 1 & 1 \\
 - 1 & 0 & 1 \\
1 & 1 & 0
\end{bmatrix}.\) \\
B3. 
Приведите квадратичные формы \(G_{1}\) и \(G_{2}\) к каноническому виду. \(G_{1} = x_{1}^{2} + x_{2}^{2} + 3x_{3}^{2} + 4x_{1}x_{2} + 2x_{1}x_{3} + 2x_{2}x_{3}\), \(G_{2} = x_{1}x_{2} + x_{1}x_{3} + x_{2}x_{3}\) \\
C1. В базисе \((e_{1},\ \ e_{2},\ \ e_{3})\) пространства \(V^{3}\) оператор \emph{A} имеет матрицу \(A = \begin{bmatrix}
4 & 0 & 1 \\
 - 2 & - 2 & 3 \\
0 & 2 & - 1
\end{bmatrix}\ \ .\) Найти матрицу \emph{B} этого же оператора в базисе \(({e'}_{1},\ \ {e'}_{2},\ \ {e'}_{3}),\) где \({e'}_{1} = e_{1} + 2e_{2},\) \({e'}_{2} = e_{1} - e_{3},\) \({e'}_{3} = e_{1} + e_{2} + e_{3}.\)
 \\
C2. Даны векторы \(e_{1},e_{2},e_{3}\), \(a_{1},a_{2},a_{3}\) линейного пространства \(R^{3}\). Найдите матрицу перехода от базиса \(e_{1},e_{2},e_{3}\) к базису \(a_{1},a_{2},a_{3}\).
\(e_{1} = (3,1, - 1)\),\(e_{2} = ( - 2,0,1)\),\(e_{3} = (2,7,3)\) и \(a_{1} = (2,1, - 3)\),\(a_{2} = (3,2, - 5)\),\(a_{3} = (1, - 1,1)\) \\
C3. Найти жорданову нормальную форму матрицы \(A = \begin{pmatrix}
 - 1 & 3 & - 1 \\
 - 3 & 5 & - 1 \\
 - 3 & 3 & 1
\end{pmatrix}\). \\

\end{tabular}
\vspace{1cm}


\begin{tabular}{m{17cm}}
\textbf{56-вариант}
\newline

T1. 17. Взаимозаменяемые преобразования. \\
T2. 12. Связь между матрицами линейных преобразовании в разных базисах. \\
A1. Доказать, что векторы \(\overrightarrow{a} = (3;\ \ 5;\ \ 4),\) \(\overrightarrow{b} = (4;\ \ 3;\ \ 2)\) и \(\overrightarrow{c} = ( - 1;\ \  - 4;\ \ 3)\) образуют базис пространства \(\mathbf{R}^{3},\) и найти координаты вектора \(\overrightarrow{d} = ( - 2;\ \  - 2;\ \ 5)\) в этом базисе. \\
A2. Известно, что оператор \emph{A} переводит базисные векторы \(\overrightarrow{i} = (1;\ \ 0;\ \ 0),\) \(\overrightarrow{j} = (0;\ \ 1;\ \ 0),\) \(\overrightarrow{k} = (0;\ \ 0;\ \ 1)\) линейного пространства \(\mathbf{R}^{3}\) в векторы \({\overline{a}}_{1} = (2;\ \ 1;1),\) \({\overline{a}}_{2} = (3;\ \ 2;\ \ 1),\) \({\overline{a}}_{3} = (3;\ \ 1;\ \ 1).\) В базисе \(\overrightarrow{i},\overrightarrow{j},\overrightarrow{k}\) найти: 1)матрицу оператора \(A\) ; 2)образ вектора \(\overline{b} = (1;\ \  - 2;\ \ 3).\) \\
A3. В пространстве \(\mathbb{C}^{3}\) со скалярным произведением \(\left\langle x,y \right\rangle = \sum_{k = 1}^{3}{x_{k}\overline{y_{k}}}\), найдите сопряженный оператор \(A^{*}\) для заданного оператора \(A\). Является ли \(A\)самосопряженным? \(Ax = \left( 3ix_{1} + x_{2},x_{1} + 2ix_{2},ix_{2} - x_{3} \right)\); \\
B1. С помощью процесса ортогонализации Грамма- Шмидта ортонормировать следующие системы векторов, используя стандартное скалярное произведение: \\
B2. Найти собственные значения и собственные векторы оператора \emph{А}, заданного в некотором базисе пространства \(V^{3}\) матрицей \(A = \begin{bmatrix}
0 & - 2 & 0 \\
 - 2 & 6 & - 2 \\
0 & - 2 & 5
\end{bmatrix};\) \\
B3. Приведите квадратичные формы \(G_{1}\) и \(G_{2}\) к каноническому виду. \(G_{1} = 5x_{1}^{2} + 6x_{2}^{2} - 3x_{3}^{2} + 4x_{1}x_{2} - 2x_{2}x_{3}\), \(G_{2} = 6x_{2}x_{3} - x_{1}x_{2}\) \\
C1. В базисе \((e_{1},\ \ e_{2},\ \ e_{3})\) пространства \(V^{3}\) оператор \emph{A} имеет матрицу \(A = \begin{bmatrix}
1 & - 1 & 2 \\
0 & 3 & - 1 \\
4 & 2 & 2
\end{bmatrix}\ \ .\) Найти матрицу \emph{B} этого же оператора в базисе \(({e'}_{1},\ \ {e'}_{2},\ \ {e'}_{3}),\) где \({e'}_{1} = e_{1} + 2e_{2},\) \({e'}_{2} = e_{1} - e_{3},\) \({e'}_{3} = e_{1} + e_{2} + e_{3}.\) \\
C2. 
Даны векторы \(e_{1},e_{2},e_{3}\), \(a_{1},a_{2},a_{3}\) линейного пространства \(R^{3}\). Найдите матрицу перехода от базиса \(e_{1},e_{2},e_{3}\) к базису \(a_{1},a_{2},a_{3}\).
\(e_{1} = (2,1, - 3)\),\(e_{2} = (3,2, - 5)\),\(e_{3} = (1, - 1,1)\) и \(a_{1} = (0,1, - 2)\),\(a_{2} = ( - 2,0,3)\),\(a_{3} = (1, - 1,1)\) \\
C3. Найти жорданову нормальную форму матрицы \(A = \begin{pmatrix}
0 & 3 & 1 \\
3 & 0 & 1 \\
 - 2 & 2 & 1
\end{pmatrix}\) \\

\end{tabular}
\vspace{1cm}


\begin{tabular}{m{17cm}}
\textbf{57-вариант}
\newline

T1. 11. Обратное преобразование. \\
T2. 14. Сопряженное преобразование для данного преобразования. \\
A1. Доказать, что векторы \(\overrightarrow{a} = (3;\ \ 1;\ \ 2),\) \(\overrightarrow{b} = (2;\ \  - 3;\ \ 1)\) и \(\overrightarrow{c} = (4;\ \  - 2;\ \ 3)\) образуют базис пространства \(\mathbf{R}^{3},\) и найти координаты вектора \(\overrightarrow{d} = ( - 7;\ \ 8;\ \ 7)\) в этом базисе. \\
A2. Известно, что оператор \emph{A} переводит базисные векторы \(\overrightarrow{i} = (1;\ \ 0;\ \ 0),\) \(\overrightarrow{j} = (0;\ \ 1;\ \ 0),\) \(\overrightarrow{k} = (0;\ \ 0;\ \ 1)\) линейного пространства \(\mathbf{R}^{3}\) в векторы \({\overline{a}}_{1} = (1;\ \ 1;\ \ 0),\) \({\overline{a}}_{2} = (3;\ \ 2;\ \ 1),\) \({\overline{a}}_{3} = (1;2;\ \ 1).\) В базисе \(\overrightarrow{i},\overrightarrow{j},\overrightarrow{k}\) найти: 1)матрицу оператора \(A\) ; 2)образ вектора \(\overline{b} = (1;\ \ 1;\ \  - 2).\) \\
A3. В пространстве \(\mathbb{C}^{3}\) со скалярным произведением \(\left\langle x,y \right\rangle = \sum_{k = 1}^{3}{x_{k}\overline{y_{k}}}\), найдите сопряженный оператор \(A^{*}\) для заданного оператора \(A\). Является ли \(A\)самосопряженным? \(Ax = \left( x_{2} + ix_{3},x_{1} - ix_{2},x_{1} + ix_{2} + x_{3} \right)\) \\
B1. С помощью процесса ортогонализации Грамма- Шмидта ортонормировать следующие системы векторов, используя стандартное скалярное произведение: \\
B2. Найти собственные значения и собственные векторы оператора \emph{А}, заданного в некотором базисе пространства \(V^{3}\) матрицей \(A = \begin{pmatrix}
2 & - 1 & 2 \\
5 & - 3 & 3 \\
 - 1 & 0 & - 2
\end{pmatrix}\). \\
B3. Приведите квадратичные формы \(G_{1}\) и \(G_{2}\) к каноническому виду. \(G_{1} = x_{1}^{2} + 5x_{2}^{2} - 4x_{3}^{2} + 2x_{1}x_{3} - 4x_{1}x_{2}\), \(G_{2} = - 4x_{1}x_{2} + 2x_{1}x_{3}\) \\
C1. В базисе \((e_{1},\ \ e_{2},\ \ e_{3})\) пространства \(V^{3}\) оператор \emph{A} имеет матрицу \(A = \begin{bmatrix}
2 & 0 & - 1 \\
3 & 2 & 0 \\
 - 1 & 4 & 3
\end{bmatrix}\ \ .\) Найти матрицу \emph{B} этого же оператора в базисе \(({e'}_{1},\ \ {e'}_{2},\ \ {e'}_{3}),\) где \({e'}_{1} = e_{1} - e_{3},\) \({e'}_{2} = e_{2} + e_{3},\) \({e'}_{3} = e_{3}.\) \\
C2. Даны векторы \(e_{1},e_{2},e_{3}\), \(a_{1},a_{2},a_{3}\) линейного пространства \(R^{3}\). Найдите матрицу перехода от базиса \(e_{1},e_{2},e_{3}\) к базису \(a_{1},a_{2},a_{3}\).
\(e_{1} = (3,5,8)\),\(e_{2} = (5,14,13)\),\(e_{3} = (1,9,2)\) и \(a_{1} = ( - 2,3,1)\),\(a_{2} = (0,2,1)\),\(a_{3} = (1,2,1)\) \\
C3. Найти жорданову нормальную форму матрицы \(A = \begin{pmatrix}
0 & 1 & 0 \\
 - 4 & 4 & 0 \\
0 & 0 & 2
\end{pmatrix}\) \\

\end{tabular}
\vspace{1cm}


\begin{tabular}{m{17cm}}
\textbf{58-вариант}
\newline

T1. 13. Инвариантные подпространства. Собственные векторы и собственные значения. \\
T2. 4. Линейные, билинейные, и квадратичные формы. Преобразование матрицы линейного вида при изменении базиса. \\
A1. Доказать, что векторы \(\overrightarrow{a} = (1;\ \ 2;\ \ 1),\) \(\overrightarrow{b} = (1;\ \ 1;\ \ 3)\) и \(\overrightarrow{c} = ( - 1;\ \ 2;\ \ 1)\) образуют базис пространства \(\mathbf{R}^{3},\) и найти координаты вектора \(\overrightarrow{d} = (0;\ \ 10;\ \  - 2)\) в этом базисе. \\
A2. Известно, что оператор \emph{A} переводит базисные векторы \(\overrightarrow{i} = (1;\ \ 0;\ \ 0),\) \(\overrightarrow{j} = (0;\ \ 1;\ \ 0),\) \(\overrightarrow{k} = (0;\ \ 0;\ \ 1)\) линейного пространства \(\mathbf{R}^{3}\) в векторы \({\overline{a}}_{1} = (0;\ \ 1;\ \ 1),\) \({\overline{a}}_{2} = (3;\ \ 1;\ \ 1),\) \({\overline{a}}_{3} = (3;\ \ 1;\ \ 1).\) В базисе \(\overrightarrow{i},\overrightarrow{j},\overrightarrow{k}\) найти: 1)матрицу оператора \(A\) ; 2)образ вектора \(\overline{b} = (1;\ \ 1;\ \ 1).\) \\
A3. В пространстве \(\mathbb{C}^{3}\) со скалярным произведением \(\left\langle x,y \right\rangle = \sum_{k = 1}^{3}{x_{k}\overline{y_{k}}}\), найдите сопряженный оператор \(A^{*}\) для заданного оператора \(A\). Является ли \(A\)самосопряженным?\(Ax = \left( x_{1} + 2ix_{3},2ix_{1} + ix_{2},x_{1} + ix_{3} \right)\); \\
B1. \(a_{1} = (0;1; - 2)\),\(a_{2} = (1; - 1;1)\), \(a_{3} = ( - 2;0;3)\); \\
B2. Найти собственные значения и собственные векторы оператора \emph{А}, заданного в некотором базисе пространства \(V^{3}\) матрицей \(A = \begin{pmatrix}
0 & 1 & 2 \\
 - 1 & 0 & - 2 \\
 - 2 & 2 & 0
\end{pmatrix}\).
 \\
B3. Приведите квадратичные формы \(G_{1}\) и \(G_{2}\) к каноническому виду. \(G_{1} = 2x_{1}^{2} + 6x_{2}^{2} - 4x_{3}^{2} - 2x_{1}x_{3} + 4x_{1}x_{2} - 8x_{2}x_{3}\), \(G_{2} = x_{2}x_{3} - 2x_{1}x_{3}\) \\
C1. В базисе \((e_{1},\ \ e_{2},\ \ e_{3})\) пространства \(V^{3}\) оператор \emph{A} имеет матрицу \(A = \begin{bmatrix}
5 & - 2 & 1 \\
 - 1 & 0 & 4 \\
3 & 1 & 2
\end{bmatrix}\ \ .\) Найти матрицу \emph{B} этого же оператора в базисе \(({e'}_{1},\ \ {e'}_{2},\ \ {e'}_{3}),\) где \({e'}_{1} = 2e_{1} + 3e_{3},\) \({e'}_{2} = - e_{2},\) \({e'}_{3} = e_{1} + e_{2} + e_{3}.\) \\
C2. 12.Даны векторы \(e_{1},e_{2},e_{3}\), \(a_{1},a_{2},a_{3}\) линейного пространства \(R^{3}\). Найдите матрицу перехода от базиса \(e_{1},e_{2},e_{3}\) к базису \(a_{1},a_{2},a_{3}\).
\(e_{1} = (1,1,1)\),\(e_{2} = (1,1,2)\),\(e_{3} = (1,2,3)\) и \(a_{1} = (2,0,1)\),\(a_{2} = ( - 1,2,3)\),\(a_{3} = ( - 1,1,1)\)
 \\
C3. Найти жорданову нормальную форму матрицы \(A = \begin{pmatrix}
2 & - 1 & - 1 \\
2 & - 1 & - 2 \\
 - 1 & 1 & 2
\end{pmatrix}\). \\

\end{tabular}
\vspace{1cm}


\begin{tabular}{m{17cm}}
\textbf{59-вариант}
\newline

T1. 5. Методы приведения квадратичной формы к каноническому форму. \\
T2. 6. Положительно определенные квадратичные формы. \\
A1. Доказать, что векторы \(\overrightarrow{a} = (3;\ \ 4;\ \ 3),\) \(\overrightarrow{b} = ( - 2;\ \ 3;\ \ 1)\) и \(\overrightarrow{c} = (4;\ \  - 2;\ \ 3)\) образуют базис пространства \(\mathbf{R}^{3},\) и найти координаты вектора \(\overrightarrow{d} = ( - 17;\ \ 18;\ \  - 7)\) в этом базисе. \\
A2. 
Известно, что оператор \emph{A} переводит базисные векторы \(\overrightarrow{i} = (1;\ \ 0;\ \ 0),\) \(\overrightarrow{j} = (0;\ \ 1;\ \ 0),\) \(\overrightarrow{k} = (0;\ \ 0;\ \ 1)\) линейного пространства \(\mathbf{R}^{3}\) в векторы \({\overline{a}}_{1} = (1;1;0),\) \({\overline{a}}_{2} = (3;\ \ 2;\ \ 1),\) \({\overline{a}}_{3} = (0;\ \ 1;\ \ 1).\) В базисе \(\overrightarrow{i},\overrightarrow{j},\overrightarrow{k}\) найти: 1)матрицу оператора \(A\) ; 2)образ вектора \(\overline{b} = (1;\ \  - 2;\ \  - 3).\) \\
A3. В пространстве \(\mathbb{C}^{3}\) со скалярным произведением \(\left\langle x,y \right\rangle = \sum_{k = 1}^{3}{x_{k}\overline{y_{k}}}\), найдите сопряженный оператор \(A^{*}\) для заданного оператора \(A\). Является ли \(A\)самосопряженным? \(Ax = \left( 2ix_{1} + ix_{3},x_{1} + x_{2} + ix_{3},ix_{3} \right)\); \\
B1. С помощью процесса ортогонализации Грамма- Шмидта ортонормировать следующие системы векторов, используя стандартное скалярное произведение: \\
B2. Найти собственные значения и собственные векторы оператора \emph{А}, заданного в некотором базисе пространства \(V^{3}\) матрицей \(A = \begin{bmatrix}
2 & 1 & 0 \\
1 & 2 & 0 \\
0 & 0 & - 5
\end{bmatrix};\) \\
B3. Приведите квадратичные формы \(G_{1}\) и \(G_{2}\) к каноническому виду. \(G_{1} = x_{1}^{2} - 3x_{3}^{2} - 2x_{1}x_{2} - 2x_{1}x_{3} - 6x_{2}x_{3}\), \(G_{2} = 2x_{1}x_{2} - x_{1}x_{3} + 2x_{2}x_{3}\) \\
C1. В базисе \((e_{1},\ \ e_{2},\ \ e_{3})\) пространства \(V^{3}\) оператор \emph{A} имеет матрицу \(A = \begin{bmatrix}
 - 3 & 1 & 4 \\
0 & 3 & 2 \\
 - 5 & - 1 & 2
\end{bmatrix}\ \ .\) Найти матрицу \emph{B} этого же оператора в базисе \(({e'}_{1},\ \ {e'}_{2},\ \ {e'}_{3}),\) где \({e'}_{1} = 2e_{1} + e_{2}\), \({e'}_{2} = - e_{1} + 2e_{2} + 3e_{3}\),\({e'}_{3} = - e_{1} + e_{2} + e_{3}\) \\
C2. Даны векторы \(e_{1},e_{2},e_{3}\), \(a_{1},a_{2},a_{3}\) линейного пространства \(R^{3}\). Найдите матрицу перехода от базиса \(e_{1},e_{2},e_{3}\) к базису \(a_{1},a_{2},a_{3}\).
\(e_{1} = (2,0,1)\),\(e_{2} = ( - 1,2,3)\),\(e_{3} = ( - 1,1,1)\) и \(a_{1} = (1,0,2)\),\(a_{2} = (3, - 1,4)\),\(a_{3} = (2, - 2,1)\) \\
C3. Найти жорданову нормальную форму матрицы \(A = \begin{pmatrix}
1 & 2 & 1 \\
1 & 2 & 4 \\
 - 1 & - 2 & - 3
\end{pmatrix}\). \\

\end{tabular}
\vspace{1cm}


\begin{tabular}{m{17cm}}
\textbf{60-вариант}
\newline

T1. 3. Ортогональное дополнение и ортогональная проекция. \\
T2. 18. Нормальные преобразования и их канонический вид. \\
A1. Доказать, что векторы \(\overrightarrow{a} = (3;\ \ 1;\ \ 0),\) \(\overrightarrow{b} = (4;\ \ 3;\ \ 2)\) и \(\overrightarrow{c} = ( - 1;\ \  - 4;\ \ 3)\) образуют базис пространства \(\mathbf{R}^{3},\) и найти координаты вектора \(\overrightarrow{d} = ( - 1;\ \ 2;\ \ 5)\) в этом базисе.
 \\
A2. Известно, что оператор \emph{A} переводит базисные векторы \(\overrightarrow{i} = (1;\ \ 0;\ \ 0),\) \(\overrightarrow{j} = (0;\ \ 1;\ \ 0),\) \(\overrightarrow{k} = (0;\ \ 0;\ \ 1)\) линейного пространства \(\mathbf{R}^{3}\) в векторы \({\overline{a}}_{1} = (0;\ \ 1;\ \ 1),\) \({\overline{a}}_{2} = (3;\ \ 1;\ \ 1),\) \({\overline{a}}_{3} = (3;0;1).\) В базисе \(\overrightarrow{i},\overrightarrow{j},\overrightarrow{k}\) найти: 1)матрицу оператора \(A\) ; 2)образ вектора \(\overline{b} = (1;\ \  - 1;\ \  - 1).\) \\
A3. 
В пространстве \(\mathbb{C}^{3}\) со скалярным произведением \(\left\langle x,y \right\rangle = \sum_{k = 1}^{3}{x_{k}\overline{y_{k}}}\), найдите сопряженный оператор \(A^{*}\) для заданного оператора \(A\). Является ли \(A\)самосопряженным? \(Ax = \left( ix_{1} + x_{3},x_{2} + ix_{1},x_{1} + ix_{3} \right)\); \\
B1. \(a_{1} = (3;1; - 1)\), \(a_{2} = ( - 2;0;1)\), \(a_{3} = (2;7;3)\); \\
B2. Найти собственные значения и собственные векторы оператора \emph{А}, заданного в некотором базисе пространства \(V^{3}\) матрицей \(A = \begin{bmatrix}
 - 1 & - 2 & 0 \\
0 & - 2 & 0 \\
2 & 2 & 1
\end{bmatrix}.\) \\
B3. Приведите квадратичные формы \(G_{1}\) и \(G_{2}\) к каноническому виду. \(G_{1} = 3x_{1}^{2} - 2x_{2}^{2} + 2x_{1}x_{3} - 4x_{2}x_{3}\), \(G_{2} = x_{1}x_{2} + x_{2}x_{3}\) \\
C1. 
В базисе \((e_{1},\ \ e_{2},\ \ e_{3})\) пространства \(V^{3}\) оператор \emph{A} имеет матрицу \(A = \begin{bmatrix}
1 & 2 & 3 \\
0 & 1 & 2 \\
3 & 1 & 2
\end{bmatrix}\ \ .\) Найти матрицу \emph{B} этого же оператора в базисе \(({e'}_{1},\ \ {e'}_{2},\ \ {e'}_{3}),\) где \({e'}_{1} = e_{1} + 2e_{2},\) \({e'}_{2} = e_{1} - e_{3},\) \({e'}_{3} = e_{1} + e_{2} + e_{3}.\) \\
C2. Даны векторы \(e_{1},e_{2},e_{3}\), \(a_{1},a_{2},a_{3}\) линейного пространства \(R^{3}\). Найдите матрицу перехода от базиса \(e_{1},e_{2},e_{3}\) к базису \(a_{1},a_{2},a_{3}\).
\(e_{1} = (0,1, - 2)\),\(e_{2} = ( - 2,0,3)\),\(e_{3} = (1, - 1,1)\) и \(a_{1} = (3,1, - 1)\),\(a_{2} = ( - 2,0,1)\),\(a_{3} = (2,7,3)\) \\
C3. Найти жорданову нормальную форму матрицы \(A = \begin{pmatrix}
2 & - 1 & - 1 \\
2 & - 1 & - 2 \\
 - 1 & 1 & 2
\end{pmatrix}\) \\

\end{tabular}
\vspace{1cm}


\begin{tabular}{m{17cm}}
\textbf{61-вариант}
\newline

T1. 17. Взаимозаменяемые преобразования. \\
T2. 6. Положительно определенные квадратичные формы. \\
A1. Доказать, что векторы \(\overrightarrow{a} = (2;\ \ 1;\ \  - 3),\) \(\overrightarrow{b} = ( - 1;\ \ 2;\ \ 4)\) и \(\overrightarrow{c} = (3;\ \  - 4;\ \ 2)\) образуют базис пространства \(\mathbf{R}^{3},\) и найти координаты вектора \(\overrightarrow{d} = ( - 4;\ \ 19;\ \ 3)\) в этом базисе. \\
A2. Известно, что оператор \emph{A} переводит базисные векторы \(\overrightarrow{i} = (1;\ \ 0;\ \ 0),\) \(\overrightarrow{j} = (0;\ \ 1;\ \ 0),\) \(\overrightarrow{k} = (0;\ \ 0;\ \ 1)\) линейного пространства \(\mathbf{R}^{3}\) в векторы \({\overline{a}}_{1} = (1;\ \ 1;\ \ 1),\) \({\overline{a}}_{2} = (3;\ \ 2;\ \ 1),\) \({\overline{a}}_{3} = (0;\ \ 1;\ \ 1).\) В базисе \(\overrightarrow{i},\overrightarrow{j},\overrightarrow{k}\) найти: 1)матрицу оператора \(A\) ; 2)образ вектора \(\overline{b} = (1;\ \  - 2;\ \ 3).\)
 \\
A3. В пространстве \(\mathbb{C}^{3}\) со скалярным произведением \(\left\langle x,y \right\rangle = \sum_{k = 1}^{3}{x_{k}\overline{y_{k}}}\), найдите сопряженный оператор \(A^{*}\) для заданного оператора \(A\). Является ли \(A\)самосопряженным? \(Ax = \left( ix_{1} + x_{2},x_{1} + ix_{2},x_{2} + ix_{3} \right)\); \\
B1. С помощью процесса ортогонализации Грамма- Шмидта ортонормировать следующие системы векторов, используя стандартное скалярное произведение: \\
B2. Найти собственные значения и собственные векторы оператора \emph{А}, заданного в некотором базисе пространства \(V^{3}\) матрицей \(A = \begin{bmatrix}
2 & 1 & 0 \\
1 & 2 & 0 \\
0 & 0 & - 5
\end{bmatrix};\) \\
B3. Приведите квадратичные формы \(G_{1}\) и \(G_{2}\) к каноническому виду. \(G_{1} = 3x_{1}^{2} - 2x_{2}^{2} + 2x_{1}x_{3} - 4x_{2}x_{3}\), \(G_{2} = x_{1}x_{2} + x_{2}x_{3}\) \\
C1. В базисе \((e_{1},\ \ e_{2},\ \ e_{3})\) пространства \(V^{3}\) оператор \emph{A} имеет матрицу \(A = \begin{bmatrix}
 - 3 & 1 & 4 \\
0 & 3 & 2 \\
 - 5 & - 1 & 2
\end{bmatrix}\ \ .\) Найти матрицу \emph{B} этого же оператора в базисе \(({e'}_{1},\ \ {e'}_{2},\ \ {e'}_{3}),\) где \({e'}_{1} = 2e_{1} + e_{2}\), \({e'}_{2} = - e_{1} + 2e_{2} + 3e_{3}\),\({e'}_{3} = - e_{1} + e_{2} + e_{3}\) \\
C2. Даны векторы \(e_{1},e_{2},e_{3}\), \(a_{1},a_{2},a_{3}\) линейного пространства \(R^{3}\). Найдите матрицу перехода от базиса \(e_{1},e_{2},e_{3}\) к базису \(a_{1},a_{2},a_{3}\).
\(e_{1} = (1,0,2)\),\(e_{2} = (3, - 1,4)\),\(e_{3} = (2, - 2,1)\) и \(a_{1} = (4,0,5)\),\(a_{2} = ( - 2,1,3)\),\(a_{3} = ( - 5,1, - 1)\) \\
C3. Найти жорданову нормальную форму матрицы \(A = \begin{pmatrix}
0 & 3 & 1 \\
3 & 0 & 1 \\
 - 2 & 2 & 1
\end{pmatrix}\) \\

\end{tabular}
\vspace{1cm}


\begin{tabular}{m{17cm}}
\textbf{62-вариант}
\newline

T1. 9. Линейные преобразования и их матрица. \\
T2. 18. Нормальные преобразования и их канонический вид. \\
A1. Доказать, что векторы \(\overrightarrow{a} = (3;\ \ 1;\ \ 0),\) \(\overrightarrow{b} = (4;\ \ 3;\ \ 2)\) и \(\overrightarrow{c} = ( - 1;\ \  - 4;\ \ 3)\) образуют базис пространства \(\mathbf{R}^{3},\) и найти координаты вектора \(\overrightarrow{d} = ( - 1;\ \ 2;\ \ 5)\) в этом базисе.
 \\
A2. Известно, что оператор \emph{A} переводит базисные векторы \(\overrightarrow{i} = (1;\ \ 0;\ \ 0),\) \(\overrightarrow{j} = (0;\ \ 1;\ \ 0),\) \(\overrightarrow{k} = (0;\ \ 0;\ \ 1)\) линейного пространства \(\mathbf{R}^{3}\) в векторы \({\overline{a}}_{1} = (1;\ \ 0;\ \ 1),\) \({\overline{a}}_{2} = (0;\ \ 1;\ \ 1),\) \({\overline{a}}_{3} = (3;\ \ 1;\ \ 1).\) В базисе \(\overrightarrow{i},\overrightarrow{j},\overrightarrow{k}\) найти: 1)матрицу оператора \(A\) ; 2)образ вектора \(\overline{b} = (1;\ \  - 2;\ \  - 3).\) \\
A3. В пространстве \(\mathbb{C}^{3}\) со скалярным произведением \(\left\langle x,y \right\rangle = \sum_{k = 1}^{3}{x_{k}\overline{y_{k}}}\), найдите сопряженный оператор \(A^{*}\) для заданного оператора \(A\). Является ли \(A\)самосопряженным?\(Ax = \left( x_{1} - 2ix_{2},x_{3} + 2ix_{2},ix_{2} + 2ix_{3} \right)\);
 \\
B1. \(a_{1} = (2; - 1;3)\), \(a_{2} = (3;2; - 5)\), \(a_{3} = (1; - 1;1)\); \\
B2. Найти собственные значения и собственные векторы оператора \emph{А}, заданного в некотором базисе пространства \(V^{3}\) матрицей \(A = \begin{bmatrix}
0 & - 2 & 0 \\
 - 2 & 6 & - 2 \\
0 & - 2 & 5
\end{bmatrix};\) \\
B3. Приведите квадратичные формы \(G_{1}\) и \(G_{2}\) к каноническому виду. \(G_{1} = 3x_{1}^{2} - 2x_{2}^{2} + 2x_{3}^{2} + 4x_{1}x_{2} - 3x_{1}x_{3} - x_{2}x_{3}\), \(G_{2} = 2x_{1}x_{3} + 4x_{1}x_{2} - 2x_{2}x_{3}\) \\
C1. В базисе \((e_{1},\ \ e_{2},\ \ e_{3})\) пространства \(V^{3}\) оператор \emph{A} имеет матрицу \(A = \begin{bmatrix}
0 & 1 & - 3 \\
2 & 4 & 1 \\
0 & 3 & - 3
\end{bmatrix}\ \ .\) Найти матрицу \emph{B} этого же оператора в базисе \(({e'}_{1},\ \ {e'}_{2},\ \ {e'}_{3}),\) где \({e'}_{1} = 2e_{1} + e_{2}\), \({e'}_{2} = - e_{1} + 2e_{2} + 3e_{3}\),\({e'}_{3} = - e_{1} + e_{2} + e_{3}\) \\
C2. Даны векторы \(e_{1},e_{2},e_{3}\), \(a_{1},a_{2},a_{3}\) линейного пространства \(R^{3}\). Найдите матрицу перехода от базиса \(e_{1},e_{2},e_{3}\) к базису \(a_{1},a_{2},a_{3}\).
\(e_{1} = (3,5,8)\),\(e_{2} = (5,14,13)\),\(e_{3} = (1,9,2)\) и \(a_{1} = ( - 2,3,1)\),\(a_{2} = (0,2,1)\),\(a_{3} = (1,2,1)\) \\
C3. Найти жорданову нормальную форму матрицы \(A = \begin{pmatrix}
 - 1 & 4 & 3 \\
 - 2 & 5 & 3 \\
2 & - 4 & - 2
\end{pmatrix}\). \\

\end{tabular}
\vspace{1cm}


\begin{tabular}{m{17cm}}
\textbf{63-вариант}
\newline

T1. 3. Ортогональное дополнение и ортогональная проекция. \\
T2. 10. Ядро, образ линйеного преобразования. \\
A1. Доказать, что векторы \(\overrightarrow{a} = (3;\ \ 1;\ \ 2),\) \(\overrightarrow{b} = (2;\ \  - 3;\ \ 1)\) и \(\overrightarrow{c} = (4;\ \  - 2;\ \ 3)\) образуют базис пространства \(\mathbf{R}^{3},\) и найти координаты вектора \(\overrightarrow{d} = ( - 7;\ \ 8;\ \ 7)\) в этом базисе. \\
A2. 
Известно, что оператор \emph{A} переводит базисные векторы \(\overrightarrow{i} = (1;\ \ 0;\ \ 0),\) \(\overrightarrow{j} = (0;\ \ 1;\ \ 0),\) \(\overrightarrow{k} = (0;\ \ 0;\ \ 1)\) линейного пространства \(\mathbf{R}^{3}\) в векторы \({\overline{a}}_{1} = (1;1;0),\) \({\overline{a}}_{2} = (3;\ \ 2;\ \ 1),\) \({\overline{a}}_{3} = (0;\ \ 1;\ \ 1).\) В базисе \(\overrightarrow{i},\overrightarrow{j},\overrightarrow{k}\) найти: 1)матрицу оператора \(A\) ; 2)образ вектора \(\overline{b} = (1;\ \  - 2;\ \  - 3).\) \\
A3. В пространстве \(\mathbb{C}^{3}\) со скалярным произведением \(\left\langle x,y \right\rangle = \sum_{k = 1}^{3}{x_{k}\overline{y_{k}}}\), найдите сопряженный оператор \(A^{*}\) для заданного оператора \(A\). Является ли \(A\)самосопряженным? \(Ax = \left( 3ix_{1} + x_{2},x_{1} + 2ix_{2},ix_{2} - x_{3} \right)\); \\
B1. С помощью процесса ортогонализации Грамма- Шмидта ортонормировать следующие системы векторов, используя стандартное скалярное произведение: \\
B2. Найти собственные значения и собственные векторы оператора \emph{А}, заданного в некотором базисе пространства \(V^{3}\) матрицей \(A = \begin{pmatrix}
2 & - 1 & 2 \\
5 & - 3 & 3 \\
 - 1 & 0 & - 2
\end{pmatrix}\). \\
B3. Приведите квадратичные формы \(G_{1}\) и \(G_{2}\) к каноническому виду. \(G_{1} = 2x_{1}^{2} + 3x_{2}^{2} + 4x_{3}^{2} - 2x_{1}x_{2} + 4x_{1}x_{3} - 3x_{2}x_{3}\), \(G_{2} = x_{1}x_{3} - 2x_{2}x_{3}\) \\
C1. В базисе \((e_{1},\ \ e_{2},\ \ e_{3})\) пространства \(V^{3}\) оператор \emph{A} имеет матрицу \(A = \begin{bmatrix}
4 & 0 & 1 \\
 - 2 & - 2 & 3 \\
0 & 2 & - 1
\end{bmatrix}\ \ .\) Найти матрицу \emph{B} этого же оператора в базисе \(({e'}_{1},\ \ {e'}_{2},\ \ {e'}_{3}),\) где \({e'}_{1} = e_{1} + 2e_{2},\) \({e'}_{2} = e_{1} - e_{3},\) \({e'}_{3} = e_{1} + e_{2} + e_{3}.\)
 \\
C2. Даны векторы \(e_{1},e_{2},e_{3}\), \(a_{1},a_{2},a_{3}\) линейного пространства \(R^{3}\). Найдите матрицу перехода от базиса \(e_{1},e_{2},e_{3}\) к базису \(a_{1},a_{2},a_{3}\).
\(e_{1} = (0,1, - 2)\),\(e_{2} = ( - 2,0,3)\),\(e_{3} = (1, - 1,1)\) и \(a_{1} = (3,1, - 1)\),\(a_{2} = ( - 2,0,1)\),\(a_{3} = (2,7,3)\) \\
C3. Найти жорданову нормальную форму матрицы \(A = \begin{pmatrix}
 - 1 & 3 & - 1 \\
 - 3 & 5 & - 1 \\
 - 3 & 3 & 1
\end{pmatrix}\). \\

\end{tabular}
\vspace{1cm}


\begin{tabular}{m{17cm}}
\textbf{64-вариант}
\newline

T1. 7. Комплексные евклидовы пространства. \\
T2. 20. Подобные матрицы. \\
A1. Доказать, что векторы \(\overrightarrow{a} = (2;\ \ 1;1),\) \(\overrightarrow{b} = ( - 1;\ \ 2;\ \ 4)\) и \(\overrightarrow{c} = (3;\ \ 3;\ \ 2)\) образуют базис пространства \(\mathbf{R}^{3},\) и найти координаты вектора \(\overrightarrow{d} = ( - 4;\ \ 2;\ \ 4)\) в этом базисе. \\
A2. Известно, что оператор \emph{A} переводит базисные векторы \(\overrightarrow{i} = (1;\ \ 0;\ \ 0),\) \(\overrightarrow{j} = (0;\ \ 1;\ \ 0),\) \(\overrightarrow{k} = (0;\ \ 0;\ \ 1)\) линейного пространства \(\mathbf{R}^{3}\) в векторы \({\overline{a}}_{1} = (0;\ \ 1;\ \ 1),\) \({\overline{a}}_{2} = (3;\ \ 1;\ \ 1),\) \({\overline{a}}_{3} = (3;0;1).\) В базисе \(\overrightarrow{i},\overrightarrow{j},\overrightarrow{k}\) найти: 1)матрицу оператора \(A\) ; 2)образ вектора \(\overline{b} = (1;\ \  - 1;\ \  - 1).\) \\
A3. В пространстве \(\mathbb{C}^{3}\) со скалярным произведением \(\left\langle x,y \right\rangle = \sum_{k = 1}^{3}{x_{k}\overline{y_{k}}}\), найдите сопряженный оператор \(A^{*}\) для заданного оператора \(A\). Является ли \(A\)самосопряженным?\(Ax = \left( x_{1} + 2ix_{3},2ix_{1} + ix_{2},x_{1} + ix_{3} \right)\); \\
B1. \(a_{1} = (1;2;1)\), \(a_{2} = (2;3;3)\), \(a_{3} = (3;8;2)\); \\
B2. Найти собственные значения и собственные векторы оператора \emph{А}, заданного в некотором базисе пространства \(V^{3}\) матрицей \(A = \begin{bmatrix}
 - 1 & 1 & 0 \\
 - 4 & 3 & 0 \\
 - 2 & 1 & 1
\end{bmatrix};\) \\
B3. Приведите квадратичные формы \(G_{1}\) и \(G_{2}\) к каноническому виду. \(G_{1} = x_{1}^{2} - 2x_{3}^{2} + 2x_{1}x_{3} - 6x_{1}x_{2}\), \(G_{2} = 6x_{2}x_{3} - 4x_{1}x_{2} + x_{1}x_{3}\)
 \\
C1. В базисе \((e_{1},\ \ e_{2},\ \ e_{3})\) пространства \(V^{3}\) оператор \emph{A} имеет матрицу \(A = \begin{bmatrix}
 - 1 & 2 & 4 \\
 - 4 & 2 & 0 \\
3 & 3 & - 3
\end{bmatrix}\ \ .\) Найти матрицу \emph{B} этого же оператора в базисе \(({e'}_{1},\ \ {e'}_{2},\ \ {e'}_{3}),\) где \({e'}_{1} = 2e_{1} + e_{2}\), \({e'}_{2} = - e_{1} + 2e_{2} + 3e_{3}\),\({e'}_{3} = - e_{1} + e_{2} + e_{3}\) \\
C2. 
Даны векторы \(e_{1},e_{2},e_{3}\), \(a_{1},a_{2},a_{3}\) линейного пространства \(R^{3}\). Найдите матрицу перехода от базиса \(e_{1},e_{2},e_{3}\) к базису \(a_{1},a_{2},a_{3}\).
\(e_{1} = (2,1, - 3)\),\(e_{2} = (3,2, - 5)\),\(e_{3} = (1, - 1,1)\) и \(a_{1} = (0,1, - 2)\),\(a_{2} = ( - 2,0,3)\),\(a_{3} = (1, - 1,1)\) \\
C3. Найти жорданову нормальную форму матрицы \(A = \begin{pmatrix}
2 & - 1 & - 1 \\
2 & - 1 & - 2 \\
 - 1 & 1 & 2
\end{pmatrix}\). \\

\end{tabular}
\vspace{1cm}


\begin{tabular}{m{17cm}}
\textbf{65-вариант}
\newline

T1. 1. Линейные пространства. Линейные подпространства. Сумма и пересечение подпространств. \\
T2. 8. Квадратичные формы в комплексном пространстве и их канонические виды. \\
A1. Доказать, что векторы \(\overrightarrow{a} = (3;\ \ 5;\ \ 4),\) \(\overrightarrow{b} = (4;\ \ 3;\ \ 2)\) и \(\overrightarrow{c} = ( - 1;\ \  - 4;\ \ 3)\) образуют базис пространства \(\mathbf{R}^{3},\) и найти координаты вектора \(\overrightarrow{d} = ( - 2;\ \  - 2;\ \ 5)\) в этом базисе. \\
A2. Известно, что оператор \emph{A} переводит базисные векторы \(\overrightarrow{i} = (1;\ \ 0;\ \ 0),\) \(\overrightarrow{j} = (0;\ \ 1;\ \ 0),\) \(\overrightarrow{k} = (0;\ \ 0;\ \ 1)\) линейного пространства \(\mathbf{R}^{3}\) в векторы \({\overline{a}}_{1} = (1;\ \ 0;\ \ 1),\) \({\overline{a}}_{2} = (0;\ \ 2;\ \ 1),\) \({\overline{a}}_{3} = (3;\ \ 1;\ \ 1).\) В базисе \(\overrightarrow{i},\overrightarrow{j},\overrightarrow{k}\) найти: 1)матрицу оператора \(A\) ; 2)образ вектора \(\overline{b} = (1;\ \ 2;\ \ 3).\) \\
A3. В пространстве \(\mathbb{C}^{3}\) со скалярным произведением \(\left\langle x,y \right\rangle = \sum_{k = 1}^{3}{x_{k}\overline{y_{k}}}\), найдите сопряженный оператор \(A^{*}\) для заданного оператора \(A\). Является ли \(A\)самосопряженным? \(Ax = \left( 2ix_{1} + ix_{3},x_{1} + x_{2} + ix_{3},ix_{3} \right)\); \\
B1. 
С помощью процесса ортогонализации Грамма- Шмидта ортонормировать следующие системы векторов, используя стандартное скалярное произведение: \\
B2. Найти собственные значения и собственные векторы оператора \emph{А}, заданного в некотором базисе пространства \(V^{3}\) матрицей \(A = \begin{pmatrix}
2 & - 1 & 2 \\
1 & 0 & 2 \\
 - 2 & 1 & - 1
\end{pmatrix}\) \\
B3. Приведите квадратичные формы \(G_{1}\) и \(G_{2}\) к каноническому виду. \(G_{1} = x_{1}^{2} - 2x_{2}^{2} + x_{3}^{2} + 2x_{1}x_{2} + 4x_{1}x_{3} + 2x_{2}x_{3}\), \(G_{2} = 2x_{1}x_{3} - 4x_{2}x_{3}\) \\
C1. 
В базисе \((e_{1},\ \ e_{2},\ \ e_{3})\) пространства \(V^{3}\) оператор \emph{A} имеет матрицу \(A = \begin{bmatrix}
1 & 2 & 3 \\
0 & 1 & 2 \\
3 & 1 & 2
\end{bmatrix}\ \ .\) Найти матрицу \emph{B} этого же оператора в базисе \(({e'}_{1},\ \ {e'}_{2},\ \ {e'}_{3}),\) где \({e'}_{1} = e_{1} + 2e_{2},\) \({e'}_{2} = e_{1} - e_{3},\) \({e'}_{3} = e_{1} + e_{2} + e_{3}.\) \\
C2. Даны векторы \(e_{1},e_{2},e_{3}\), \(a_{1},a_{2},a_{3}\) линейного пространства \(R^{3}\). Найдите матрицу перехода от базиса \(e_{1},e_{2},e_{3}\) к базису \(a_{1},a_{2},a_{3}\).
\(e_{1} = (3,1, - 1)\),\(e_{2} = ( - 2,0,1)\),\(e_{3} = (2,7,3)\) и \(a_{1} = (2,1, - 3)\),\(a_{2} = (3,2, - 5)\),\(a_{3} = (1, - 1,1)\) \\
C3. 
Найти жорданову нормальную форму матрицы \(A = \begin{pmatrix}
 - 1 & 1 & - 2 \\
3 & - 3 & 6 \\
2 & - 2 & 4
\end{pmatrix}\). \\

\end{tabular}
\vspace{1cm}


\begin{tabular}{m{17cm}}
\textbf{66-вариант}
\newline

T1. 15. Самосопряженные преобразования и их канонический вид. \\
T2. 14. Сопряженное преобразование для данного преобразования. \\
A1. Доказать, что векторы \(\overrightarrow{a} = (3;\ \ 4;\ \ 3),\) \(\overrightarrow{b} = ( - 2;\ \ 3;\ \ 1)\) и \(\overrightarrow{c} = (4;\ \  - 2;\ \ 3)\) образуют базис пространства \(\mathbf{R}^{3},\) и найти координаты вектора \(\overrightarrow{d} = ( - 17;\ \ 18;\ \  - 7)\) в этом базисе. \\
A2. Известно, что оператор \emph{A} переводит базисные векторы \(\overrightarrow{i} = (1;\ \ 0;\ \ 0),\) \(\overrightarrow{j} = (0;\ \ 1;\ \ 0),\) \(\overrightarrow{k} = (0;\ \ 0;\ \ 1)\) линейного пространства \(\mathbf{R}^{3}\) в векторы \({\overline{a}}_{1} = (2;\ \ 1;1),\) \({\overline{a}}_{2} = (3;\ \ 2;\ \ 1),\) \({\overline{a}}_{3} = (3;\ \ 1;\ \ 1).\) В базисе \(\overrightarrow{i},\overrightarrow{j},\overrightarrow{k}\) найти: 1)матрицу оператора \(A\) ; 2)образ вектора \(\overline{b} = (1;\ \  - 2;\ \ 3).\) \\
A3. 
В пространстве \(\mathbb{C}^{3}\) со скалярным произведением \(\left\langle x,y \right\rangle = \sum_{k = 1}^{3}{x_{k}\overline{y_{k}}}\), найдите сопряженный оператор \(A^{*}\) для заданного оператора \(A\). Является ли \(A\)самосопряженным? \(Ax = \left( ix_{1} + x_{3},x_{2} + ix_{1},x_{1} + ix_{3} \right)\); \\
B1. С помощью процесса ортогонализации Грамма- Шмидта ортонормировать следующие системы векторов, используя стандартное скалярное произведение: \\
B2. Найти собственные значения и собственные векторы оператора \emph{А}, заданного в некотором базисе пространства \(V^{3}\) матрицей \(A = \begin{pmatrix}
1 & - 1 & 1 \\
1 & 1 & - 1 \\
2 & - 1 & 0
\end{pmatrix}\). \\
B3. Приведите квадратичные формы \(G_{1}\) и \(G_{2}\) к каноническому виду. \(G_{1} = 2x_{1}^{2} + x_{2}^{2} + x_{3}^{2} + 4x_{1}x_{2} - 2x_{1}x_{3}\), \(G_{2} = x_{1}x_{2} + x_{1}x_{3} + 4x_{2}x_{3}\) \\
C1. В базисе \((e_{1},\ \ e_{2},\ \ e_{3})\) пространства \(V^{3}\) оператор \emph{A} имеет матрицу \(A = \begin{bmatrix}
1 & - 1 & 2 \\
0 & 3 & - 1 \\
4 & 2 & 2
\end{bmatrix}\ \ .\) Найти матрицу \emph{B} этого же оператора в базисе \(({e'}_{1},\ \ {e'}_{2},\ \ {e'}_{3}),\) где \({e'}_{1} = e_{1} + 2e_{2},\) \({e'}_{2} = e_{1} - e_{3},\) \({e'}_{3} = e_{1} + e_{2} + e_{3}.\) \\
C2. Даны векторы \(e_{1},e_{2},e_{3}\), \(a_{1},a_{2},a_{3}\) линейного пространства \(R^{3}\). Найдите матрицу перехода от базиса \(e_{1},e_{2},e_{3}\) к базису \(a_{1},a_{2},a_{3}\).
\(e_{1} = (2,0,1)\),\(e_{2} = ( - 1,2,3)\),\(e_{3} = ( - 1,1,1)\) и \(a_{1} = ( - 3,0,1)\),\(a_{2} = (0,2,3)\),\(a_{3} = ( - 1, - 1, - 1)\) \\
C3. Найти жорданову нормальную форму матрицы \(A = \begin{pmatrix}
0 & 1 & 0 \\
 - 4 & 4 & 0 \\
0 & 0 & 2
\end{pmatrix}\) \\

\end{tabular}
\vspace{1cm}


\begin{tabular}{m{17cm}}
\textbf{67-вариант}
\newline

T1. 19. Полиномиальные матрицы и диагональные нормальные формы. \\
T2. 16. Унитарные преобразования и их собственные значения и канонический вид. \\
A1. Доказать, что векторы \(\overrightarrow{a} = (2;\ \ 1;\ \ 1),\) \(\overrightarrow{b} = (1;\ \ 2;\ \ 1)\) и \(\overrightarrow{c} = (2;\ \ 3;\ \  - 1)\) образуют базис пространства \(\mathbf{R}^{3},\) и найти координаты вектора \(\overrightarrow{d} = (2;\ \ 3;\ \  - 1)\) в этом базисе. \\
A2. Известно, что оператор \emph{A} переводит базисные векторы \(\overrightarrow{i} = (1;\ \ 0;\ \ 0),\) \(\overrightarrow{j} = (0;\ \ 1;\ \ 0),\) \(\overrightarrow{k} = (0;\ \ 0;\ \ 1)\) линейного пространства \(\mathbf{R}^{3}\) в векторы \({\overline{a}}_{1} = (0;\ \ 1;\ \ 1),\) \({\overline{a}}_{2} = (3;\ \ 1;\ \ 1),\) \({\overline{a}}_{3} = (3;\ \ 1;\ \ 1).\) В базисе \(\overrightarrow{i},\overrightarrow{j},\overrightarrow{k}\) найти: 1)матрицу оператора \(A\) ; 2)образ вектора \(\overline{b} = (1;\ \ 1;\ \ 1).\) \\
A3. В пространстве \(\mathbb{C}^{3}\) со скалярным произведением \(\left\langle x,y \right\rangle = \sum_{k = 1}^{3}{x_{k}\overline{y_{k}}}\), найдите сопряженный оператор \(A^{*}\) для заданного оператора \(A\). Является ли \(A\)самосопряженным? \(Ax = \left( x_{2} + ix_{3},x_{1} - ix_{2},x_{1} + ix_{2} + x_{3} \right)\) \\
B1. С помощью процесса ортогонализации Грамма- Шмидта ортонормировать следующие системы векторов, используя стандартное скалярное произведение: \\
B2. Найти собственные значения и собственные векторы оператора \emph{А}, заданного в некотором базисе пространства \(V^{3}\) матрицей \(A = \begin{pmatrix}
2 & 1 & 0 \\
1 & 3 & - 1 \\
 - 1 & 2 & 3
\end{pmatrix}\). \\
B3. Приведите квадратичные формы \(G_{1}\) и \(G_{2}\) к каноническому виду. \(G_{1} = x_{1}^{2} + 5x_{2}^{2} - 4x_{3}^{2} + 2x_{1}x_{3} - 4x_{1}x_{2}\), \(G_{2} = - 4x_{1}x_{2} + 2x_{1}x_{3}\) \\
C1. В базисе \((e_{1},\ \ e_{2},\ \ e_{3})\) пространства \(V^{3}\) оператор \emph{A} имеет матрицу \(A = \begin{bmatrix}
0 & 1 & - 2 \\
3 & 5 & 1 \\
 - 1 & 2 & 0
\end{bmatrix}\ \ .\) Найти матрицу \emph{B} этого же оператора в базисе \(({e'}_{1},\ \ {e'}_{2},\ \ {e'}_{3}),\) где \({e'}_{1} = e_{1} + 2e_{2},\) \({e'}_{2} = e_{1} - e_{3},\) \({e'}_{3} = e_{1} + e_{2} + e_{3}.\) \\
C2. Даны векторы \(e_{1},e_{2},e_{3}\), \(a_{1},a_{2},a_{3}\) линейного пространства \(R^{3}\). Найдите матрицу перехода от базиса \(e_{1},e_{2},e_{3}\) к базису \(a_{1},a_{2},a_{3}\).
\(e_{1} = (2,0,1)\),\(e_{2} = ( - 1,2,3)\),\(e_{3} = ( - 1,1,1)\) и \(a_{1} = (1,0,2)\),\(a_{2} = (3, - 1,4)\),\(a_{3} = (2, - 2,1)\) \\
C3. Найти жорданову нормальную форму матрицы \(A = \begin{pmatrix}
 - 1 & 4 & 3 \\
 - 2 & 5 & 3 \\
2 & - 4 & - 2
\end{pmatrix}\). \\

\end{tabular}
\vspace{1cm}


\begin{tabular}{m{17cm}}
\textbf{68-вариант}
\newline

T1. 5. Методы приведения квадратичной формы к каноническому форму. \\
T2. 2. Евклидово пространство. Неравенство Коши-Буняковского. Процесс ортогонализации. \\
A1. Доказать, что векторы \(\overrightarrow{a} = (1;\ \ 2;\ \ 1),\) \(\overrightarrow{b} = (1;\ \ 1;\ \ 3)\) и \(\overrightarrow{c} = ( - 1;\ \ 2;\ \ 1)\) образуют базис пространства \(\mathbf{R}^{3},\) и найти координаты вектора \(\overrightarrow{d} = (0;\ \ 10;\ \  - 2)\) в этом базисе. \\
A2. Известно, что оператор \emph{A} переводит базисные векторы \(\overrightarrow{i} = (1;\ \ 0;\ \ 0),\) \(\overrightarrow{j} = (0;\ \ 1;\ \ 0),\) \(\overrightarrow{k} = (0;\ \ 0;\ \ 1)\) линейного пространства \(\mathbf{R}^{3}\) в векторы \({\overline{a}}_{1} = (1;\ \ 1;\ \ 0),\) \({\overline{a}}_{2} = (3;\ \ 2;\ \ 1),\) \({\overline{a}}_{3} = (1;2;\ \ 1).\) В базисе \(\overrightarrow{i},\overrightarrow{j},\overrightarrow{k}\) найти: 1)матрицу оператора \(A\) ; 2)образ вектора \(\overline{b} = (1;\ \ 1;\ \  - 2).\) \\
A3. В пространстве \(\mathbb{C}^{3}\) со скалярным произведением \(\left\langle x,y \right\rangle = \sum_{k = 1}^{3}{x_{k}\overline{y_{k}}}\), найдите сопряженный оператор \(A^{*}\) для заданного оператора \(A\). Является ли \(A\)самосопряженным? \(Ax = \left( ix_{1} + 2ix_{3},x_{3},x_{1} - 2ix_{3} \right)\); \\
B1. \(a_{1} = (4;1;3)\), \(a_{2} = (0;7; - 2)\), \(a_{3} = (4;8;0)\);
 \\
B2. Найти собственные значения и собственные векторы оператора \emph{А}, заданного в некотором базисе пространства \(V^{3}\) матрицей \(A = \begin{pmatrix}
1 & - 2 & - 1 \\
 - 1 & 1 & 1 \\
1 & 0 & - 1
\end{pmatrix}\) \\
B3. 
Приведите квадратичные формы \(G_{1}\) и \(G_{2}\) к каноническому виду. \(G_{1} = x_{1}^{2} + x_{2}^{2} + 3x_{3}^{2} + 4x_{1}x_{2} + 2x_{1}x_{3} + 2x_{2}x_{3}\), \(G_{2} = x_{1}x_{2} + x_{1}x_{3} + x_{2}x_{3}\) \\
C1. В базисе \((e_{1},\ \ e_{2},\ \ e_{3})\) пространства \(V^{3}\) оператор \emph{A} имеет матрицу \(A = \begin{bmatrix}
1 & 3 & - 1 \\
2 & 0 & 4 \\
1 & 1 & 1
\end{bmatrix}\ \ .\) Найти матрицу \emph{B} этого же оператора в базисе \(({e'}_{1},\ \ {e'}_{2},\ \ {e'}_{3}),\) где \({e'}_{1} = 2e_{1} + e_{2}\), \({e'}_{2} = - e_{1} + 2e_{2} + 3e_{3}\),\({e'}_{3} = - e_{1} + e_{2} + e_{3}\) \\
C2. 12.Даны векторы \(e_{1},e_{2},e_{3}\), \(a_{1},a_{2},a_{3}\) линейного пространства \(R^{3}\). Найдите матрицу перехода от базиса \(e_{1},e_{2},e_{3}\) к базису \(a_{1},a_{2},a_{3}\).
\(e_{1} = (1,1,1)\),\(e_{2} = (1,1,2)\),\(e_{3} = (1,2,3)\) и \(a_{1} = (2,0,1)\),\(a_{2} = ( - 1,2,3)\),\(a_{3} = ( - 1,1,1)\)
 \\
C3. Найти жорданову нормальную форму матрицы \(A = \begin{pmatrix}
2 & 1 & 1 \\
1 & 2 & 1 \\
1 & 1 & 2
\end{pmatrix}\). \\

\end{tabular}
\vspace{1cm}


\begin{tabular}{m{17cm}}
\textbf{69-вариант}
\newline

T1. 11. Обратное преобразование. \\
T2. 12. Связь между матрицами линейных преобразовании в разных базисах. \\
A1. Доказать, что векторы \(\overrightarrow{a} = (2;\ \ 1;\ \ 1),\) \(\overrightarrow{b} = (1;\ \ 2;\ \ 1)\) и \(\overrightarrow{c} = (2;\ \ 1;\ \ 1)\) образуют базис пространства \(\mathbf{R}^{3},\) и найти координаты вектора \(\overrightarrow{d} = (1;\ \ 3;\ \ 1)\) в этом базисе. \\
A2. Известно, что оператор \emph{A} переводит базисные векторы \(\overrightarrow{i} = (1;\ \ 0;\ \ 0),\) \(\overrightarrow{j} = (0;\ \ 1;\ \ 0),\) \(\overrightarrow{k} = (0;\ \ 0;\ \ 1)\) линейного пространства \(\mathbf{R}^{3}\) в векторы \({\overline{a}}_{1} = (1;\ \ 1;\ \ 0),\) \({\overline{a}}_{2} = (3;\ \ 2;\ \ 1),\) \({\overline{a}}_{3} = (3;\ \ 1;\ \ 1).\) В базисе \(\overrightarrow{i},\overrightarrow{j},\overrightarrow{k}\) найти: 1)матрицу оператора \(A\) ; 2)образ вектора \(\overline{b} = (1;\ \ 2;\ \ 3).\) \\
A3. В пространстве \(\mathbb{C}^{3}\) со скалярным произведением \(\left\langle x,y \right\rangle = \sum_{k = 1}^{3}{x_{k}\overline{y_{k}}}\), найдите сопряженный оператор \(A^{*}\) для заданного оператора \(A\). Является ли \(A\)самосопряженным? \(Ax = \left( ix_{1} + x_{3},ix_{3} - ix_{2},x_{1} - ix_{3} \right)\); \\
B1. С помощью процесса ортогонализации Грамма- Шмидта ортонормировать следующие системы векторов, используя стандартное скалярное произведение: \\
B2. Найти собственные значения и собственные векторы оператора \emph{А}, заданного в некотором базисе пространства \(V^{3}\) матрицей \(A = \begin{bmatrix}
 - 1 & - 2 & 0 \\
0 & - 2 & 0 \\
2 & 2 & 1
\end{bmatrix}.\) \\
B3. Приведите квадратичные формы \(G_{1}\) и \(G_{2}\) к каноническому виду. \(G_{1} = 5x_{1}^{2} + 6x_{2}^{2} - 3x_{3}^{2} + 4x_{1}x_{2} - 2x_{2}x_{3}\), \(G_{2} = 6x_{2}x_{3} - x_{1}x_{2}\) \\
C1. В базисе \((e_{1},\ \ e_{2},\ \ e_{3})\) пространства \(V^{3}\) оператор \emph{A} имеет матрицу \(A = \begin{bmatrix}
 - 1 & 2 & 1 \\
0 & 1 & - 4 \\
5 & - 1 & 2
\end{bmatrix}\ \ .\) Найти матрицу \emph{B} этого же оператора в базисе \(({e'}_{1},\ \ {e'}_{2},\ \ {e'}_{3}),\) где \({e'}_{1} = e_{1} - e_{3},\) \({e'}_{2} = e_{2} + e_{3},\) \({e'}_{3} = e_{3}.\) \\
C2. 10.Даны векторы \(e_{1},e_{2},e_{3}\), \(a_{1},a_{2},a_{3}\) линейного пространства \(R^{3}\). Найдите матрицу перехода от базиса \(e_{1},e_{2},e_{3}\) к базису \(a_{1},a_{2},a_{3}\).
\(e_{1} = (4,0,5)\),\(e_{2} = ( - 2,1,3)\),\(e_{3} = ( - 5,1, - 1)\) и \(a_{1} = (1,2,1)\),\(a_{2} = (2,3,3)\),\(a_{3} = (3,8,2)\) \\
C3. Найти жорданову нормальную форму матрицы \(A = \begin{pmatrix}
1 & 2 & 1 \\
1 & 2 & 4 \\
 - 1 & - 2 & - 3
\end{pmatrix}\). \\

\end{tabular}
\vspace{1cm}


\begin{tabular}{m{17cm}}
\textbf{70-вариант}
\newline

T1. 13. Инвариантные подпространства. Собственные векторы и собственные значения. \\
T2. 4. Линейные, билинейные, и квадратичные формы. Преобразование матрицы линейного вида при изменении базиса. \\
A1. Доказать, что векторы \(\overrightarrow{a} = (1;\ \ 2;\ \ 1),\) \(\overrightarrow{b} = (1;\ \ 1;\ \ 3)\) и \(\overrightarrow{c} = ( - 1;\ \ 2;\ \ 1)\) образуют базис пространства \(\mathbf{R}^{3},\) и найти координаты вектора \(\overrightarrow{d} = (0;\ \ 10;\ \  - 2)\) в этом базисе. \\
A2. Известно, что оператор \emph{A} переводит базисные векторы \(\overrightarrow{i} = (1;\ \ 0;\ \ 0),\) \(\overrightarrow{j} = (0;\ \ 1;\ \ 0),\) \(\overrightarrow{k} = (0;\ \ 0;\ \ 1)\) линейного пространства \(\mathbf{R}^{3}\) в векторы \({\overline{a}}_{1} = (1;\ \ 1;\ \ 1),{\overline{a}}_{2} = (3;\ \ 0;\ \ 1),\) \({\overline{a}}_{3} = (0;\ \ 2;\ \ 1).\)В базисе \(\overrightarrow{i},\overrightarrow{j},\overrightarrow{k}\) найти:1)матрицу оператора \(A\) ;2)образ вектора \(\overline{b} = (1;\ \ 2;\ \  - 2).\) \\
A3. В пространстве \(\mathbb{C}^{3}\) со скалярным произведением \(\left\langle x,y \right\rangle = \sum_{k = 1}^{3}{x_{k}\overline{y_{k}}}\), найдите сопряженный оператор \(A^{*}\) для заданного оператора \(A\). Является ли \(A\)самосопряженным?\(Ax = \left( x_{1} + 2ix_{2},x_{3} - ix_{2},x_{1} - ix_{2} - 2ix_{3} \right)\); \\
B1. \(a_{1} = ( - 3;0;1)\), \(a_{2} = (0;2;3)\), \(a_{3} = ( - 1; - 1; - 1)\); \\
B2. Найти собственные значения и собственные векторы оператора \emph{А}, заданного в некотором базисе пространства \(V^{3}\) матрицей \(A = \begin{bmatrix}
0 & - 1 & 1 \\
 - 1 & 0 & 1 \\
1 & 1 & 0
\end{bmatrix};\) \\
B3. Приведите квадратичные формы \(G_{1}\) и \(G_{2}\) к каноническому виду. \(G_{1} = 2x_{1}^{2} + 6x_{2}^{2} - 4x_{3}^{2} - 2x_{1}x_{3} + 4x_{1}x_{2} - 8x_{2}x_{3}\), \(G_{2} = x_{2}x_{3} - 2x_{1}x_{3}\) \\
C1. В базисе \((e_{1},\ \ e_{2},\ \ e_{3})\) пространства \(V^{3}\) оператор \emph{A} имеет матрицу \(A = \begin{bmatrix}
3 & 2 & - 1 \\
4 & 0 & 2 \\
 - 1 & 2 & - 1
\end{bmatrix}\ \ .\) Найти матрицу \emph{B} этого же оператора в базисе \(({e'}_{1},\ \ {e'}_{2},\ \ {e'}_{3}),\) где \({e'}_{1} = 2e_{1} + e_{2}\), \({e'}_{2} = - e_{1} + 2e_{2} + 3e_{3}\),\({e'}_{3} = - e_{1} + e_{2} + e_{3}\) \\
C2. Даны векторы \(e_{1},e_{2},e_{3}\), \(a_{1},a_{2},a_{3}\) линейного пространства \(R^{3}\). Найдите матрицу перехода от базиса \(e_{1},e_{2},e_{3}\) к базису \(a_{1},a_{2},a_{3}\).
\(e_{1} = ( - 2,3,1)\),\(e_{2} = (0,2,1)\),\(e_{3} = (1,2,1)\) и \(a_{1} = ( - 1,3,7)\),\(a_{2} = (0,2, - 1)\),\(a_{3} = (1, - 2, - 8)\) \\
C3. Найти жорданову нормальную форму матрицы \(A = \begin{pmatrix}
2 & - 1 & - 1 \\
2 & - 1 & - 2 \\
 - 1 & 1 & 2
\end{pmatrix}\) \\

\end{tabular}
\vspace{1cm}


\begin{tabular}{m{17cm}}
\textbf{71-вариант}
\newline

T1. 5. Методы приведения квадратичной формы к каноническому форму. \\
T2. 14. Сопряженное преобразование для данного преобразования. \\
A1. Доказать, что векторы \(\overrightarrow{a} = (1;\ \ 2;\ \ 1),\) \(\overrightarrow{b} = (1;\ \ 1;\ \  - 3)\) и \(\overrightarrow{c} = ( - 1;\ \ 2;\ \ 1)\) образуют базис пространства \(\mathbf{R}^{3},\) и найти координаты вектора \(\overrightarrow{d} = (0;\ \ 10;\ \  - 2)\) в этом базисе. \\
A2. Известно, что оператор \emph{A} переводит базисные векторы \(\overrightarrow{i} = (1;\ \ 0;\ \ 0),\) \(\overrightarrow{j} = (0;\ \ 1;\ \ 0),\) \(\overrightarrow{k} = (0;\ \ 0;\ \ 1)\) линейного пространства \(\mathbf{R}^{3}\) в векторы \({\overline{a}}_{1} = (1;\ \ 1;\ \ 1),{\overline{a}}_{2} = (3;\ \ 0;\ \ 1),\) \({\overline{a}}_{3} = (3;\ \ 1;\ \  - 1).\)В базисе \(\overrightarrow{i},\overrightarrow{j},\overrightarrow{k}\) найти:1)матрицу оператора \(A\) ;2)образ вектора \(\overline{b} = (1;\ \ 2;\ \ 1).\) \\
A3. В пространстве \(\mathbb{C}^{3}\) со скалярным произведением \(\left\langle x,y \right\rangle = \sum_{k = 1}^{3}{x_{k}\overline{y_{k}}}\), найдите сопряженный оператор \(A^{*}\) для заданного оператора \(A\). Является ли \(A\)самосопряженным? \(Ax = \left( x_{1} + 2ix_{3},ix_{2} - x_{3},x_{2} - ix_{3} \right)\); \\
B1. \(a_{1} = ( - 2;3;1)\),\(a_{2} = (0;2;1)\), \(a_{3} = (1;2;1)\); \\
B2. Найти собственные значения и собственные векторы оператора \emph{А}, заданного в некотором базисе пространства \(V^{3}\) матрицей \(A = \begin{pmatrix}
0 & 1 & 2 \\
 - 1 & 0 & - 2 \\
 - 2 & 2 & 0
\end{pmatrix}\).
 \\
B3. Приведите квадратичные формы \(G_{1}\) и \(G_{2}\) к каноническому виду. \(G_{1} = 4x_{1}^{2} + x_{2}^{2} + x_{3}^{2} - 4x_{1}x_{2} + 4x_{1}x_{3} - 3x_{2}x_{3}\), \(G_{2} = x_{1}x_{2} + 6x_{1}x_{3} - 4x_{2}x_{3}\) \\
C1. В базисе \((e_{1},\ \ e_{2},\ \ e_{3})\) пространства \(V^{3}\) оператор \emph{A} имеет матрицу \(A = \begin{bmatrix}
2 & 0 & - 1 \\
3 & 2 & 0 \\
 - 1 & 4 & 3
\end{bmatrix}\ \ .\) Найти матрицу \emph{B} этого же оператора в базисе \(({e'}_{1},\ \ {e'}_{2},\ \ {e'}_{3}),\) где \({e'}_{1} = e_{1} - e_{3},\) \({e'}_{2} = e_{2} + e_{3},\) \({e'}_{3} = e_{3}.\) \\
C2. 11.Даны векторы \(e_{1},e_{2},e_{3}\), \(a_{1},a_{2},a_{3}\) линейного пространства \(R^{3}\). Найдите матрицу перехода от базиса \(e_{1},e_{2},e_{3}\) к базису \(a_{1},a_{2},a_{3}\).
\(e_{1} = ( - 1,3,7)\),\(e_{2} = (0,2, - 1)\),\(e_{3} = (1, - 2, - 8)\) и \(a_{1} = (0,3, - 2)\),\(a_{2} = (1, - 1, - 8)\),\(a_{3} = ( - 1,2,7)\) \\
C3. Найти жорданову нормальную форму матрицы \(A = \begin{pmatrix}
3 & - 2 & 6 \\
 - 2 & 6 & 3 \\
6 & 3 & - 2
\end{pmatrix}\). \\

\end{tabular}
\vspace{1cm}


\begin{tabular}{m{17cm}}
\textbf{72-вариант}
\newline

T1. 1. Линейные пространства. Линейные подпространства. Сумма и пересечение подпространств. \\
T2. 10. Ядро, образ линйеного преобразования. \\
A1. Доказать, что векторы \(\overrightarrow{a} = (1;\ \ 0;\ \ 1),\) \(\overrightarrow{b} = (1;\ \ 1;\ \ 1)\) и \(\overrightarrow{c} = ( - 1;\ \ 2;\ \ 1)\) образуют базис пространства \(\mathbf{R}^{3},\) и найти координаты вектора \(\overrightarrow{d} = (0;\ \ 10;\ \ 3)\) в этом базисе. \\
A2. Известно, что оператор \emph{A} переводит базисные векторы \(\overrightarrow{i} = (1;\ \ 0;\ \ 0),\) \(\overrightarrow{j} = (0;\ \ 1;\ \ 0),\) \(\overrightarrow{k} = (0;\ \ 0;\ \ 1)\) линейного пространства \(\mathbf{R}^{3}\) в векторы \({\overline{a}}_{1} = (1;\ \ 0;\ \ 1),\) \({\overline{a}}_{2} = (3;\ \ 2;\ \ 1),\) \({\overline{a}}_{3} = (3;\ \ 1;\ \ 1).\) В базисе \(\overrightarrow{i},\overrightarrow{j},\overrightarrow{k}\) найти: 1)матрицу оператора \(A\) ; 2)образ вектора \(\overline{b} = (1;\ \  - 2;\ \ 3).\) \\
A3. В пространстве \(\mathbb{C}^{3}\) со скалярным произведением \(\left\langle x,y \right\rangle = \sum_{k = 1}^{3}{x_{k}\overline{y_{k}}}\), найдите сопряженный оператор \(A^{*}\) для заданного оператора \(A\). Является ли \(A\)самосопряженным? \(Ax = \left( x_{1} + ix_{3},x_{3} + 2ix_{2},ix_{2} - 2ix_{3} \right)\); \\
B1. \(a_{1} = (1;0;2)\), \(a_{2} = (3; - 1;4)\), \(a_{3} = (2; - 2;1)\); \\
B2. 
Найти собственные значения и собственные векторы оператора \emph{А}, заданного в некотором базисе пространства \(V^{3}\) матрицей \(A = \begin{bmatrix}
0 & - 1 & 1 \\
 - 1 & 0 & 1 \\
1 & 1 & 0
\end{bmatrix}.\) \\
B3. Приведите квадратичные формы \(G_{1}\) и \(G_{2}\) к каноническому виду. \(G_{1} = x_{1}^{2} - 3x_{3}^{2} - 2x_{1}x_{2} - 2x_{1}x_{3} - 6x_{2}x_{3}\), \(G_{2} = 2x_{1}x_{2} - x_{1}x_{3} + 2x_{2}x_{3}\) \\
C1. В базисе \((e_{1},\ \ e_{2},\ \ e_{3})\) пространства \(V^{3}\) оператор \emph{A} имеет матрицу \(A = \begin{bmatrix}
5 & - 2 & 1 \\
 - 1 & 0 & 4 \\
3 & 1 & 2
\end{bmatrix}\ \ .\) Найти матрицу \emph{B} этого же оператора в базисе \(({e'}_{1},\ \ {e'}_{2},\ \ {e'}_{3}),\) где \({e'}_{1} = 2e_{1} + 3e_{3},\) \({e'}_{2} = - e_{2},\) \({e'}_{3} = e_{1} + e_{2} + e_{3}.\) \\
C2. Даны векторы \(e_{1},e_{2},e_{3}\), \(a_{1},a_{2},a_{3}\) линейного пространства \(R^{3}\). Найдите матрицу перехода от базиса \(e_{1},e_{2},e_{3}\) к базису \(a_{1},a_{2},a_{3}\).
\(e_{1} = ( - 3,0,1)\),\(e_{2} = (0,2,3)\),\(e_{3} = ( - 1, - 1, - 1)\) и \(a_{1} = (1,1,1)\),\(a_{2} = (1,1,2)\),\(a_{3} = (1,2,3)\) \\
C3. Найти жорданову нормальную форму матрицы \(A = \begin{pmatrix}
2 & - 1 & 2 \\
5 & - 3 & 3 \\
 - 1 & 0 & - 2
\end{pmatrix}\). \\

\end{tabular}
\vspace{1cm}


\begin{tabular}{m{17cm}}
\textbf{73-вариант}
\newline

T1. 9. Линейные преобразования и их матрица. \\
T2. 2. Евклидово пространство. Неравенство Коши-Буняковского. Процесс ортогонализации. \\
A1. Доказать, что векторы \(\overrightarrow{a} = (3;\ \ 5;\ \ 4),\) \(\overrightarrow{b} = (4;\ \ 3;\ \ 2)\) и \(\overrightarrow{c} = ( - 1;\ \  - 4;\ \ 3)\) образуют базис пространства \(\mathbf{R}^{3},\) и найти координаты вектора \(\overrightarrow{d} = ( - 2;\ \  - 2;\ \ 5)\) в этом базисе. \\
A2. Известно, что оператор \emph{A} переводит базисные векторы \(\overrightarrow{i} = (1;\ \ 0;\ \ 0),\) \(\overrightarrow{j} = (0;\ \ 1;\ \ 0),\) \(\overrightarrow{k} = (0;\ \ 0;\ \ 1)\) линейного пространства \(\mathbf{R}^{3}\) в векторы \({\overline{a}}_{1} = (1;\ \ 1;\ \ 0),\) \({\overline{a}}_{2} = (3;\ \ 2;\ \ 1),\) \({\overline{a}}_{3} = (1;2;\ \ 1).\) В базисе \(\overrightarrow{i},\overrightarrow{j},\overrightarrow{k}\) найти: 1)матрицу оператора \(A\) ; 2)образ вектора \(\overline{b} = (1;\ \ 1;\ \  - 2).\) \\
A3. В пространстве \(\mathbb{C}^{3}\) со скалярным произведением \(\left\langle x,y \right\rangle = \sum_{k = 1}^{3}{x_{k}\overline{y_{k}}}\), найдите сопряженный оператор \(A^{*}\) для заданного оператора \(A\). Является ли \(A\)самосопряженным? \(Ax = \left( ix_{1} + x_{3},ix_{3} - ix_{2},x_{1} - ix_{3} \right)\); \\
B1. С помощью процесса ортогонализации Грамма- Шмидта ортонормировать следующие системы векторов, используя стандартное скалярное произведение: \\
B2. Найти собственные значения и собственные векторы оператора \emph{А}, заданного в некотором базисе пространства \(V^{3}\) матрицей \(A = \begin{bmatrix}
0 & - 2 & 0 \\
 - 2 & 6 & - 2 \\
0 & - 2 & 5
\end{bmatrix};\) \\
B3. Приведите квадратичные формы \(G_{1}\) и \(G_{2}\) к каноническому виду. \(G_{1} = x_{1}^{2} + 5x_{2}^{2} - 4x_{3}^{2} + 2x_{1}x_{3} - 4x_{1}x_{2}\), \(G_{2} = - 4x_{1}x_{2} + 2x_{1}x_{3}\) \\
C1. 
В базисе \((e_{1},\ \ e_{2},\ \ e_{3})\) пространства \(V^{3}\) оператор \emph{A} имеет матрицу \(A = \begin{bmatrix}
1 & 2 & 3 \\
0 & 1 & 2 \\
3 & 1 & 2
\end{bmatrix}\ \ .\) Найти матрицу \emph{B} этого же оператора в базисе \(({e'}_{1},\ \ {e'}_{2},\ \ {e'}_{3}),\) где \({e'}_{1} = e_{1} + 2e_{2},\) \({e'}_{2} = e_{1} - e_{3},\) \({e'}_{3} = e_{1} + e_{2} + e_{3}.\) \\
C2. Даны векторы \(e_{1},e_{2},e_{3}\), \(a_{1},a_{2},a_{3}\) линейного пространства \(R^{3}\). Найдите матрицу перехода от базиса \(e_{1},e_{2},e_{3}\) к базису \(a_{1},a_{2},a_{3}\).
\(e_{1} = (0,1, - 2)\),\(e_{2} = ( - 2,0,3)\),\(e_{3} = (1, - 1,1)\) и \(a_{1} = (3,1, - 1)\),\(a_{2} = ( - 2,0,1)\),\(a_{3} = (2,7,3)\) \\
C3. 
Найти жорданову нормальную форму матрицы \(A = \begin{pmatrix}
 - 1 & 1 & - 2 \\
3 & - 3 & 6 \\
2 & - 2 & 4
\end{pmatrix}\). \\

\end{tabular}
\vspace{1cm}


\begin{tabular}{m{17cm}}
\textbf{74-вариант}
\newline

T1. 11. Обратное преобразование. \\
T2. 18. Нормальные преобразования и их канонический вид. \\
A1. Доказать, что векторы \(\overrightarrow{a} = (1;\ \ 2;\ \ 1),\) \(\overrightarrow{b} = (1;\ \ 1;\ \ 3)\) и \(\overrightarrow{c} = ( - 1;\ \ 2;\ \ 1)\) образуют базис пространства \(\mathbf{R}^{3},\) и найти координаты вектора \(\overrightarrow{d} = (0;\ \ 10;\ \  - 2)\) в этом базисе. \\
A2. Известно, что оператор \emph{A} переводит базисные векторы \(\overrightarrow{i} = (1;\ \ 0;\ \ 0),\) \(\overrightarrow{j} = (0;\ \ 1;\ \ 0),\) \(\overrightarrow{k} = (0;\ \ 0;\ \ 1)\) линейного пространства \(\mathbf{R}^{3}\) в векторы \({\overline{a}}_{1} = (1;\ \ 0;\ \ 1),\) \({\overline{a}}_{2} = (0;\ \ 1;\ \ 1),\) \({\overline{a}}_{3} = (3;\ \ 1;\ \ 1).\) В базисе \(\overrightarrow{i},\overrightarrow{j},\overrightarrow{k}\) найти: 1)матрицу оператора \(A\) ; 2)образ вектора \(\overline{b} = (1;\ \  - 2;\ \  - 3).\) \\
A3. В пространстве \(\mathbb{C}^{3}\) со скалярным произведением \(\left\langle x,y \right\rangle = \sum_{k = 1}^{3}{x_{k}\overline{y_{k}}}\), найдите сопряженный оператор \(A^{*}\) для заданного оператора \(A\). Является ли \(A\)самосопряженным?\(Ax = \left( x_{1} + 2ix_{2},x_{3} - ix_{2},x_{1} - ix_{2} - 2ix_{3} \right)\); \\
B1. С помощью процесса ортогонализации Грамма- Шмидта ортонормировать следующие системы векторов, используя стандартное скалярное произведение: \\
B2. 
Найти собственные значения и собственные векторы оператора \emph{А}, заданного в некотором базисе пространства \(V^{3}\) матрицей \(A = \begin{bmatrix}
0 & - 1 & 1 \\
 - 1 & 0 & 1 \\
1 & 1 & 0
\end{bmatrix}.\) \\
B3. Приведите квадратичные формы \(G_{1}\) и \(G_{2}\) к каноническому виду. \(G_{1} = x_{1}^{2} - 2x_{2}^{2} + x_{3}^{2} + 2x_{1}x_{2} + 4x_{1}x_{3} + 2x_{2}x_{3}\), \(G_{2} = 2x_{1}x_{3} - 4x_{2}x_{3}\) \\
C1. В базисе \((e_{1},\ \ e_{2},\ \ e_{3})\) пространства \(V^{3}\) оператор \emph{A} имеет матрицу \(A = \begin{bmatrix}
2 & 0 & - 1 \\
3 & 2 & 0 \\
 - 1 & 4 & 3
\end{bmatrix}\ \ .\) Найти матрицу \emph{B} этого же оператора в базисе \(({e'}_{1},\ \ {e'}_{2},\ \ {e'}_{3}),\) где \({e'}_{1} = e_{1} - e_{3},\) \({e'}_{2} = e_{2} + e_{3},\) \({e'}_{3} = e_{3}.\) \\
C2. Даны векторы \(e_{1},e_{2},e_{3}\), \(a_{1},a_{2},a_{3}\) линейного пространства \(R^{3}\). Найдите матрицу перехода от базиса \(e_{1},e_{2},e_{3}\) к базису \(a_{1},a_{2},a_{3}\).
\(e_{1} = (2,0,1)\),\(e_{2} = ( - 1,2,3)\),\(e_{3} = ( - 1,1,1)\) и \(a_{1} = ( - 3,0,1)\),\(a_{2} = (0,2,3)\),\(a_{3} = ( - 1, - 1, - 1)\) \\
C3. Найти жорданову нормальную форму матрицы \(A = \begin{pmatrix}
3 & - 2 & 6 \\
 - 2 & 6 & 3 \\
6 & 3 & - 2
\end{pmatrix}\). \\

\end{tabular}
\vspace{1cm}


\begin{tabular}{m{17cm}}
\textbf{75-вариант}
\newline

T1. 3. Ортогональное дополнение и ортогональная проекция. \\
T2. 4. Линейные, билинейные, и квадратичные формы. Преобразование матрицы линейного вида при изменении базиса. \\
A1. Доказать, что векторы \(\overrightarrow{a} = (1;\ \ 2;\ \ 1),\) \(\overrightarrow{b} = (1;\ \ 1;\ \ 3)\) и \(\overrightarrow{c} = ( - 1;\ \ 2;\ \ 1)\) образуют базис пространства \(\mathbf{R}^{3},\) и найти координаты вектора \(\overrightarrow{d} = (0;\ \ 10;\ \  - 2)\) в этом базисе. \\
A2. Известно, что оператор \emph{A} переводит базисные векторы \(\overrightarrow{i} = (1;\ \ 0;\ \ 0),\) \(\overrightarrow{j} = (0;\ \ 1;\ \ 0),\) \(\overrightarrow{k} = (0;\ \ 0;\ \ 1)\) линейного пространства \(\mathbf{R}^{3}\) в векторы \({\overline{a}}_{1} = (1;\ \ 1;\ \ 1),{\overline{a}}_{2} = (3;\ \ 0;\ \ 1),\) \({\overline{a}}_{3} = (0;\ \ 2;\ \ 1).\)В базисе \(\overrightarrow{i},\overrightarrow{j},\overrightarrow{k}\) найти:1)матрицу оператора \(A\) ;2)образ вектора \(\overline{b} = (1;\ \ 2;\ \  - 2).\) \\
A3. В пространстве \(\mathbb{C}^{3}\) со скалярным произведением \(\left\langle x,y \right\rangle = \sum_{k = 1}^{3}{x_{k}\overline{y_{k}}}\), найдите сопряженный оператор \(A^{*}\) для заданного оператора \(A\). Является ли \(A\)самосопряженным? \(Ax = \left( x_{2} + ix_{3},x_{1} - ix_{2},x_{1} + ix_{2} + x_{3} \right)\) \\
B1. \(a_{1} = (1;0;2)\), \(a_{2} = (3; - 1;4)\), \(a_{3} = (2; - 2;1)\); \\
B2. Найти собственные значения и собственные векторы оператора \emph{А}, заданного в некотором базисе пространства \(V^{3}\) матрицей \(A = \begin{bmatrix}
2 & 1 & 0 \\
1 & 2 & 0 \\
0 & 0 & - 5
\end{bmatrix};\) \\
B3. Приведите квадратичные формы \(G_{1}\) и \(G_{2}\) к каноническому виду. \(G_{1} = x_{1}^{2} - 2x_{3}^{2} + 2x_{1}x_{3} - 6x_{1}x_{2}\), \(G_{2} = 6x_{2}x_{3} - 4x_{1}x_{2} + x_{1}x_{3}\)
 \\
C1. В базисе \((e_{1},\ \ e_{2},\ \ e_{3})\) пространства \(V^{3}\) оператор \emph{A} имеет матрицу \(A = \begin{bmatrix}
4 & 0 & 1 \\
 - 2 & - 2 & 3 \\
0 & 2 & - 1
\end{bmatrix}\ \ .\) Найти матрицу \emph{B} этого же оператора в базисе \(({e'}_{1},\ \ {e'}_{2},\ \ {e'}_{3}),\) где \({e'}_{1} = e_{1} + 2e_{2},\) \({e'}_{2} = e_{1} - e_{3},\) \({e'}_{3} = e_{1} + e_{2} + e_{3}.\)
 \\
C2. Даны векторы \(e_{1},e_{2},e_{3}\), \(a_{1},a_{2},a_{3}\) линейного пространства \(R^{3}\). Найдите матрицу перехода от базиса \(e_{1},e_{2},e_{3}\) к базису \(a_{1},a_{2},a_{3}\).
\(e_{1} = (1,0,2)\),\(e_{2} = (3, - 1,4)\),\(e_{3} = (2, - 2,1)\) и \(a_{1} = (4,0,5)\),\(a_{2} = ( - 2,1,3)\),\(a_{3} = ( - 5,1, - 1)\) \\
C3. Найти жорданову нормальную форму матрицы \(A = \begin{pmatrix}
0 & 1 & 0 \\
 - 4 & 4 & 0 \\
0 & 0 & 2
\end{pmatrix}\) \\

\end{tabular}
\vspace{1cm}


\begin{tabular}{m{17cm}}
\textbf{76-вариант}
\newline

T1. 7. Комплексные евклидовы пространства. \\
T2. 12. Связь между матрицами линейных преобразовании в разных базисах. \\
A1. Доказать, что векторы \(\overrightarrow{a} = (2;\ \ 1;1),\) \(\overrightarrow{b} = ( - 1;\ \ 2;\ \ 4)\) и \(\overrightarrow{c} = (3;\ \ 3;\ \ 2)\) образуют базис пространства \(\mathbf{R}^{3},\) и найти координаты вектора \(\overrightarrow{d} = ( - 4;\ \ 2;\ \ 4)\) в этом базисе. \\
A2. Известно, что оператор \emph{A} переводит базисные векторы \(\overrightarrow{i} = (1;\ \ 0;\ \ 0),\) \(\overrightarrow{j} = (0;\ \ 1;\ \ 0),\) \(\overrightarrow{k} = (0;\ \ 0;\ \ 1)\) линейного пространства \(\mathbf{R}^{3}\) в векторы \({\overline{a}}_{1} = (1;\ \ 1;\ \ 0),\) \({\overline{a}}_{2} = (3;\ \ 2;\ \ 1),\) \({\overline{a}}_{3} = (3;\ \ 1;\ \ 1).\) В базисе \(\overrightarrow{i},\overrightarrow{j},\overrightarrow{k}\) найти: 1)матрицу оператора \(A\) ; 2)образ вектора \(\overline{b} = (1;\ \ 2;\ \ 3).\) \\
A3. В пространстве \(\mathbb{C}^{3}\) со скалярным произведением \(\left\langle x,y \right\rangle = \sum_{k = 1}^{3}{x_{k}\overline{y_{k}}}\), найдите сопряженный оператор \(A^{*}\) для заданного оператора \(A\). Является ли \(A\)самосопряженным? \(Ax = \left( 2ix_{1} + ix_{3},x_{1} + x_{2} + ix_{3},ix_{3} \right)\); \\
B1. С помощью процесса ортогонализации Грамма- Шмидта ортонормировать следующие системы векторов, используя стандартное скалярное произведение: \\
B2. Найти собственные значения и собственные векторы оператора \emph{А}, заданного в некотором базисе пространства \(V^{3}\) матрицей \(A = \begin{pmatrix}
0 & 1 & 2 \\
 - 1 & 0 & - 2 \\
 - 2 & 2 & 0
\end{pmatrix}\).
 \\
B3. Приведите квадратичные формы \(G_{1}\) и \(G_{2}\) к каноническому виду. \(G_{1} = 4x_{1}^{2} + x_{2}^{2} + x_{3}^{2} - 4x_{1}x_{2} + 4x_{1}x_{3} - 3x_{2}x_{3}\), \(G_{2} = x_{1}x_{2} + 6x_{1}x_{3} - 4x_{2}x_{3}\) \\
C1. В базисе \((e_{1},\ \ e_{2},\ \ e_{3})\) пространства \(V^{3}\) оператор \emph{A} имеет матрицу \(A = \begin{bmatrix}
 - 1 & 2 & 1 \\
0 & 1 & - 4 \\
5 & - 1 & 2
\end{bmatrix}\ \ .\) Найти матрицу \emph{B} этого же оператора в базисе \(({e'}_{1},\ \ {e'}_{2},\ \ {e'}_{3}),\) где \({e'}_{1} = e_{1} - e_{3},\) \({e'}_{2} = e_{2} + e_{3},\) \({e'}_{3} = e_{3}.\) \\
C2. 11.Даны векторы \(e_{1},e_{2},e_{3}\), \(a_{1},a_{2},a_{3}\) линейного пространства \(R^{3}\). Найдите матрицу перехода от базиса \(e_{1},e_{2},e_{3}\) к базису \(a_{1},a_{2},a_{3}\).
\(e_{1} = ( - 1,3,7)\),\(e_{2} = (0,2, - 1)\),\(e_{3} = (1, - 2, - 8)\) и \(a_{1} = (0,3, - 2)\),\(a_{2} = (1, - 1, - 8)\),\(a_{3} = ( - 1,2,7)\) \\
C3. Найти жорданову нормальную форму матрицы \(A = \begin{pmatrix}
2 & - 1 & - 1 \\
2 & - 1 & - 2 \\
 - 1 & 1 & 2
\end{pmatrix}\). \\

\end{tabular}
\vspace{1cm}


\begin{tabular}{m{17cm}}
\textbf{77-вариант}
\newline

T1. 17. Взаимозаменяемые преобразования. \\
T2. 8. Квадратичные формы в комплексном пространстве и их канонические виды. \\
A1. Доказать, что векторы \(\overrightarrow{a} = (3;\ \ 1;\ \ 2),\) \(\overrightarrow{b} = (2;\ \  - 3;\ \ 1)\) и \(\overrightarrow{c} = (4;\ \  - 2;\ \ 3)\) образуют базис пространства \(\mathbf{R}^{3},\) и найти координаты вектора \(\overrightarrow{d} = ( - 7;\ \ 8;\ \ 7)\) в этом базисе. \\
A2. Известно, что оператор \emph{A} переводит базисные векторы \(\overrightarrow{i} = (1;\ \ 0;\ \ 0),\) \(\overrightarrow{j} = (0;\ \ 1;\ \ 0),\) \(\overrightarrow{k} = (0;\ \ 0;\ \ 1)\) линейного пространства \(\mathbf{R}^{3}\) в векторы \({\overline{a}}_{1} = (1;\ \ 0;\ \ 1),\) \({\overline{a}}_{2} = (0;\ \ 2;\ \ 1),\) \({\overline{a}}_{3} = (3;\ \ 1;\ \ 1).\) В базисе \(\overrightarrow{i},\overrightarrow{j},\overrightarrow{k}\) найти: 1)матрицу оператора \(A\) ; 2)образ вектора \(\overline{b} = (1;\ \ 2;\ \ 3).\) \\
A3. В пространстве \(\mathbb{C}^{3}\) со скалярным произведением \(\left\langle x,y \right\rangle = \sum_{k = 1}^{3}{x_{k}\overline{y_{k}}}\), найдите сопряженный оператор \(A^{*}\) для заданного оператора \(A\). Является ли \(A\)самосопряженным?\(Ax = \left( x_{1} - 2ix_{2},x_{3} + 2ix_{2},ix_{2} + 2ix_{3} \right)\);
 \\
B1. \(a_{1} = (2;0;1)\), \(a_{2} = ( - 1;2;3)\), \(a_{3} = ( - 1;1;1)\); \\
B2. Найти собственные значения и собственные векторы оператора \emph{А}, заданного в некотором базисе пространства \(V^{3}\) матрицей \(A = \begin{pmatrix}
2 & - 1 & 2 \\
1 & 0 & 2 \\
 - 2 & 1 & - 1
\end{pmatrix}\) \\
B3. 
Приведите квадратичные формы \(G_{1}\) и \(G_{2}\) к каноническому виду. \(G_{1} = x_{1}^{2} + x_{2}^{2} + 3x_{3}^{2} + 4x_{1}x_{2} + 2x_{1}x_{3} + 2x_{2}x_{3}\), \(G_{2} = x_{1}x_{2} + x_{1}x_{3} + x_{2}x_{3}\) \\
C1. В базисе \((e_{1},\ \ e_{2},\ \ e_{3})\) пространства \(V^{3}\) оператор \emph{A} имеет матрицу \(A = \begin{bmatrix}
 - 1 & 2 & 4 \\
 - 4 & 2 & 0 \\
3 & 3 & - 3
\end{bmatrix}\ \ .\) Найти матрицу \emph{B} этого же оператора в базисе \(({e'}_{1},\ \ {e'}_{2},\ \ {e'}_{3}),\) где \({e'}_{1} = 2e_{1} + e_{2}\), \({e'}_{2} = - e_{1} + 2e_{2} + 3e_{3}\),\({e'}_{3} = - e_{1} + e_{2} + e_{3}\) \\
C2. 
Даны векторы \(e_{1},e_{2},e_{3}\), \(a_{1},a_{2},a_{3}\) линейного пространства \(R^{3}\). Найдите матрицу перехода от базиса \(e_{1},e_{2},e_{3}\) к базису \(a_{1},a_{2},a_{3}\).
\(e_{1} = (2,1, - 3)\),\(e_{2} = (3,2, - 5)\),\(e_{3} = (1, - 1,1)\) и \(a_{1} = (0,1, - 2)\),\(a_{2} = ( - 2,0,3)\),\(a_{3} = (1, - 1,1)\) \\
C3. Найти жорданову нормальную форму матрицы \(A = \begin{pmatrix}
 - 1 & 3 & - 1 \\
 - 3 & 5 & - 1 \\
 - 3 & 3 & 1
\end{pmatrix}\). \\

\end{tabular}
\vspace{1cm}


\begin{tabular}{m{17cm}}
\textbf{78-вариант}
\newline

T1. 13. Инвариантные подпространства. Собственные векторы и собственные значения. \\
T2. 16. Унитарные преобразования и их собственные значения и канонический вид. \\
A1. Доказать, что векторы \(\overrightarrow{a} = (3;\ \ 4;\ \ 3),\) \(\overrightarrow{b} = ( - 2;\ \ 3;\ \ 1)\) и \(\overrightarrow{c} = (4;\ \  - 2;\ \ 3)\) образуют базис пространства \(\mathbf{R}^{3},\) и найти координаты вектора \(\overrightarrow{d} = ( - 17;\ \ 18;\ \  - 7)\) в этом базисе. \\
A2. Известно, что оператор \emph{A} переводит базисные векторы \(\overrightarrow{i} = (1;\ \ 0;\ \ 0),\) \(\overrightarrow{j} = (0;\ \ 1;\ \ 0),\) \(\overrightarrow{k} = (0;\ \ 0;\ \ 1)\) линейного пространства \(\mathbf{R}^{3}\) в векторы \({\overline{a}}_{1} = (2;\ \ 1;1),\) \({\overline{a}}_{2} = (3;\ \ 2;\ \ 1),\) \({\overline{a}}_{3} = (3;\ \ 1;\ \ 1).\) В базисе \(\overrightarrow{i},\overrightarrow{j},\overrightarrow{k}\) найти: 1)матрицу оператора \(A\) ; 2)образ вектора \(\overline{b} = (1;\ \  - 2;\ \ 3).\) \\
A3. 
В пространстве \(\mathbb{C}^{3}\) со скалярным произведением \(\left\langle x,y \right\rangle = \sum_{k = 1}^{3}{x_{k}\overline{y_{k}}}\), найдите сопряженный оператор \(A^{*}\) для заданного оператора \(A\). Является ли \(A\)самосопряженным? \(Ax = \left( ix_{1} + x_{3},x_{2} + ix_{1},x_{1} + ix_{3} \right)\); \\
B1. \(a_{1} = (4;1;3)\), \(a_{2} = (0;7; - 2)\), \(a_{3} = (4;8;0)\);
 \\
B2. Найти собственные значения и собственные векторы оператора \emph{А}, заданного в некотором базисе пространства \(V^{3}\) матрицей \(A = \begin{pmatrix}
2 & - 1 & 2 \\
5 & - 3 & 3 \\
 - 1 & 0 & - 2
\end{pmatrix}\). \\
B3. Приведите квадратичные формы \(G_{1}\) и \(G_{2}\) к каноническому виду. \(G_{1} = 2x_{1}^{2} + x_{2}^{2} + x_{3}^{2} + 4x_{1}x_{2} - 2x_{1}x_{3}\), \(G_{2} = x_{1}x_{2} + x_{1}x_{3} + 4x_{2}x_{3}\) \\
C1. В базисе \((e_{1},\ \ e_{2},\ \ e_{3})\) пространства \(V^{3}\) оператор \emph{A} имеет матрицу \(A = \begin{bmatrix}
1 & - 1 & 2 \\
0 & 3 & - 1 \\
4 & 2 & 2
\end{bmatrix}\ \ .\) Найти матрицу \emph{B} этого же оператора в базисе \(({e'}_{1},\ \ {e'}_{2},\ \ {e'}_{3}),\) где \({e'}_{1} = e_{1} + 2e_{2},\) \({e'}_{2} = e_{1} - e_{3},\) \({e'}_{3} = e_{1} + e_{2} + e_{3}.\) \\
C2. 10.Даны векторы \(e_{1},e_{2},e_{3}\), \(a_{1},a_{2},a_{3}\) линейного пространства \(R^{3}\). Найдите матрицу перехода от базиса \(e_{1},e_{2},e_{3}\) к базису \(a_{1},a_{2},a_{3}\).
\(e_{1} = (4,0,5)\),\(e_{2} = ( - 2,1,3)\),\(e_{3} = ( - 5,1, - 1)\) и \(a_{1} = (1,2,1)\),\(a_{2} = (2,3,3)\),\(a_{3} = (3,8,2)\) \\
C3. Найти жорданову нормальную форму матрицы \(A = \begin{pmatrix}
2 & - 1 & 2 \\
5 & - 3 & 3 \\
 - 1 & 0 & - 2
\end{pmatrix}\). \\

\end{tabular}
\vspace{1cm}


\begin{tabular}{m{17cm}}
\textbf{79-вариант}
\newline

T1. 15. Самосопряженные преобразования и их канонический вид. \\
T2. 6. Положительно определенные квадратичные формы. \\
A1. Доказать, что векторы \(\overrightarrow{a} = (1;\ \ 0;\ \ 1),\) \(\overrightarrow{b} = (1;\ \ 1;\ \ 1)\) и \(\overrightarrow{c} = ( - 1;\ \ 2;\ \ 1)\) образуют базис пространства \(\mathbf{R}^{3},\) и найти координаты вектора \(\overrightarrow{d} = (0;\ \ 10;\ \ 3)\) в этом базисе. \\
A2. Известно, что оператор \emph{A} переводит базисные векторы \(\overrightarrow{i} = (1;\ \ 0;\ \ 0),\) \(\overrightarrow{j} = (0;\ \ 1;\ \ 0),\) \(\overrightarrow{k} = (0;\ \ 0;\ \ 1)\) линейного пространства \(\mathbf{R}^{3}\) в векторы \({\overline{a}}_{1} = (1;\ \ 1;\ \ 1),\) \({\overline{a}}_{2} = (3;\ \ 2;\ \ 1),\) \({\overline{a}}_{3} = (0;\ \ 1;\ \ 1).\) В базисе \(\overrightarrow{i},\overrightarrow{j},\overrightarrow{k}\) найти: 1)матрицу оператора \(A\) ; 2)образ вектора \(\overline{b} = (1;\ \  - 2;\ \ 3).\)
 \\
A3. В пространстве \(\mathbb{C}^{3}\) со скалярным произведением \(\left\langle x,y \right\rangle = \sum_{k = 1}^{3}{x_{k}\overline{y_{k}}}\), найдите сопряженный оператор \(A^{*}\) для заданного оператора \(A\). Является ли \(A\)самосопряженным? \(Ax = \left( 3ix_{1} + x_{2},x_{1} + 2ix_{2},ix_{2} - x_{3} \right)\); \\
B1. \(a_{1} = (3;1; - 1)\), \(a_{2} = ( - 2;0;1)\), \(a_{3} = (2;7;3)\); \\
B2. Найти собственные значения и собственные векторы оператора \emph{А}, заданного в некотором базисе пространства \(V^{3}\) матрицей \(A = \begin{bmatrix}
 - 1 & 1 & 0 \\
 - 4 & 3 & 0 \\
 - 2 & 1 & 1
\end{bmatrix};\) \\
B3. Приведите квадратичные формы \(G_{1}\) и \(G_{2}\) к каноническому виду. \(G_{1} = 2x_{1}^{2} + 6x_{2}^{2} - 4x_{3}^{2} - 2x_{1}x_{3} + 4x_{1}x_{2} - 8x_{2}x_{3}\), \(G_{2} = x_{2}x_{3} - 2x_{1}x_{3}\) \\
C1. В базисе \((e_{1},\ \ e_{2},\ \ e_{3})\) пространства \(V^{3}\) оператор \emph{A} имеет матрицу \(A = \begin{bmatrix}
5 & - 2 & 1 \\
 - 1 & 0 & 4 \\
3 & 1 & 2
\end{bmatrix}\ \ .\) Найти матрицу \emph{B} этого же оператора в базисе \(({e'}_{1},\ \ {e'}_{2},\ \ {e'}_{3}),\) где \({e'}_{1} = 2e_{1} + 3e_{3},\) \({e'}_{2} = - e_{2},\) \({e'}_{3} = e_{1} + e_{2} + e_{3}.\) \\
C2. Даны векторы \(e_{1},e_{2},e_{3}\), \(a_{1},a_{2},a_{3}\) линейного пространства \(R^{3}\). Найдите матрицу перехода от базиса \(e_{1},e_{2},e_{3}\) к базису \(a_{1},a_{2},a_{3}\).
\(e_{1} = ( - 2,3,1)\),\(e_{2} = (0,2,1)\),\(e_{3} = (1,2,1)\) и \(a_{1} = ( - 1,3,7)\),\(a_{2} = (0,2, - 1)\),\(a_{3} = (1, - 2, - 8)\) \\
C3. Найти жорданову нормальную форму матрицы \(A = \begin{pmatrix}
 - 1 & 4 & 3 \\
 - 2 & 5 & 3 \\
2 & - 4 & - 2
\end{pmatrix}\). \\

\end{tabular}
\vspace{1cm}


\begin{tabular}{m{17cm}}
\textbf{80-вариант}
\newline

T1. 19. Полиномиальные матрицы и диагональные нормальные формы. \\
T2. 20. Подобные матрицы. \\
A1. Доказать, что векторы \(\overrightarrow{a} = (2;\ \ 1;\ \  - 3),\) \(\overrightarrow{b} = ( - 1;\ \ 2;\ \ 4)\) и \(\overrightarrow{c} = (3;\ \  - 4;\ \ 2)\) образуют базис пространства \(\mathbf{R}^{3},\) и найти координаты вектора \(\overrightarrow{d} = ( - 4;\ \ 19;\ \ 3)\) в этом базисе. \\
A2. 
Известно, что оператор \emph{A} переводит базисные векторы \(\overrightarrow{i} = (1;\ \ 0;\ \ 0),\) \(\overrightarrow{j} = (0;\ \ 1;\ \ 0),\) \(\overrightarrow{k} = (0;\ \ 0;\ \ 1)\) линейного пространства \(\mathbf{R}^{3}\) в векторы \({\overline{a}}_{1} = (1;1;0),\) \({\overline{a}}_{2} = (3;\ \ 2;\ \ 1),\) \({\overline{a}}_{3} = (0;\ \ 1;\ \ 1).\) В базисе \(\overrightarrow{i},\overrightarrow{j},\overrightarrow{k}\) найти: 1)матрицу оператора \(A\) ; 2)образ вектора \(\overline{b} = (1;\ \  - 2;\ \  - 3).\) \\
A3. В пространстве \(\mathbb{C}^{3}\) со скалярным произведением \(\left\langle x,y \right\rangle = \sum_{k = 1}^{3}{x_{k}\overline{y_{k}}}\), найдите сопряженный оператор \(A^{*}\) для заданного оператора \(A\). Является ли \(A\)самосопряженным? \(Ax = \left( ix_{1} + 2ix_{3},x_{3},x_{1} - 2ix_{3} \right)\); \\
B1. С помощью процесса ортогонализации Грамма- Шмидта ортонормировать следующие системы векторов, используя стандартное скалярное произведение: \\
B2. Найти собственные значения и собственные векторы оператора \emph{А}, заданного в некотором базисе пространства \(V^{3}\) матрицей \(A = \begin{bmatrix}
0 & - 1 & 1 \\
 - 1 & 0 & 1 \\
1 & 1 & 0
\end{bmatrix};\) \\
B3. Приведите квадратичные формы \(G_{1}\) и \(G_{2}\) к каноническому виду. \(G_{1} = 3x_{1}^{2} - 2x_{2}^{2} + 2x_{1}x_{3} - 4x_{2}x_{3}\), \(G_{2} = x_{1}x_{2} + x_{2}x_{3}\) \\
C1. В базисе \((e_{1},\ \ e_{2},\ \ e_{3})\) пространства \(V^{3}\) оператор \emph{A} имеет матрицу \(A = \begin{bmatrix}
0 & 1 & - 2 \\
3 & 5 & 1 \\
 - 1 & 2 & 0
\end{bmatrix}\ \ .\) Найти матрицу \emph{B} этого же оператора в базисе \(({e'}_{1},\ \ {e'}_{2},\ \ {e'}_{3}),\) где \({e'}_{1} = e_{1} + 2e_{2},\) \({e'}_{2} = e_{1} - e_{3},\) \({e'}_{3} = e_{1} + e_{2} + e_{3}.\) \\
C2. Даны векторы \(e_{1},e_{2},e_{3}\), \(a_{1},a_{2},a_{3}\) линейного пространства \(R^{3}\). Найдите матрицу перехода от базиса \(e_{1},e_{2},e_{3}\) к базису \(a_{1},a_{2},a_{3}\).
\(e_{1} = (3,5,8)\),\(e_{2} = (5,14,13)\),\(e_{3} = (1,9,2)\) и \(a_{1} = ( - 2,3,1)\),\(a_{2} = (0,2,1)\),\(a_{3} = (1,2,1)\) \\
C3. Найти жорданову нормальную форму матрицы \(A = \begin{pmatrix}
1 & 2 & 1 \\
1 & 2 & 4 \\
 - 1 & - 2 & - 3
\end{pmatrix}\). \\

\end{tabular}
\vspace{1cm}


\begin{tabular}{m{17cm}}
\textbf{81-вариант}
\newline

T1. 13. Инвариантные подпространства. Собственные векторы и собственные значения. \\
T2. 10. Ядро, образ линйеного преобразования. \\
A1. Доказать, что векторы \(\overrightarrow{a} = (1;\ \ 2;\ \ 1),\) \(\overrightarrow{b} = (1;\ \ 1;\ \  - 3)\) и \(\overrightarrow{c} = ( - 1;\ \ 2;\ \ 1)\) образуют базис пространства \(\mathbf{R}^{3},\) и найти координаты вектора \(\overrightarrow{d} = (0;\ \ 10;\ \  - 2)\) в этом базисе. \\
A2. Известно, что оператор \emph{A} переводит базисные векторы \(\overrightarrow{i} = (1;\ \ 0;\ \ 0),\) \(\overrightarrow{j} = (0;\ \ 1;\ \ 0),\) \(\overrightarrow{k} = (0;\ \ 0;\ \ 1)\) линейного пространства \(\mathbf{R}^{3}\) в векторы \({\overline{a}}_{1} = (0;\ \ 1;\ \ 1),\) \({\overline{a}}_{2} = (3;\ \ 1;\ \ 1),\) \({\overline{a}}_{3} = (3;\ \ 1;\ \ 1).\) В базисе \(\overrightarrow{i},\overrightarrow{j},\overrightarrow{k}\) найти: 1)матрицу оператора \(A\) ; 2)образ вектора \(\overline{b} = (1;\ \ 1;\ \ 1).\) \\
A3. В пространстве \(\mathbb{C}^{3}\) со скалярным произведением \(\left\langle x,y \right\rangle = \sum_{k = 1}^{3}{x_{k}\overline{y_{k}}}\), найдите сопряженный оператор \(A^{*}\) для заданного оператора \(A\). Является ли \(A\)самосопряженным?\(Ax = \left( x_{1} + 2ix_{3},2ix_{1} + ix_{2},x_{1} + ix_{3} \right)\); \\
B1. \(a_{1} = (2; - 1;3)\), \(a_{2} = (3;2; - 5)\), \(a_{3} = (1; - 1;1)\); \\
B2. Найти собственные значения и собственные векторы оператора \emph{А}, заданного в некотором базисе пространства \(V^{3}\) матрицей \(A = \begin{pmatrix}
1 & - 1 & 1 \\
1 & 1 & - 1 \\
2 & - 1 & 0
\end{pmatrix}\). \\
B3. Приведите квадратичные формы \(G_{1}\) и \(G_{2}\) к каноническому виду. \(G_{1} = x_{1}^{2} - 3x_{3}^{2} - 2x_{1}x_{2} - 2x_{1}x_{3} - 6x_{2}x_{3}\), \(G_{2} = 2x_{1}x_{2} - x_{1}x_{3} + 2x_{2}x_{3}\) \\
C1. В базисе \((e_{1},\ \ e_{2},\ \ e_{3})\) пространства \(V^{3}\) оператор \emph{A} имеет матрицу \(A = \begin{bmatrix}
 - 3 & 1 & 4 \\
0 & 3 & 2 \\
 - 5 & - 1 & 2
\end{bmatrix}\ \ .\) Найти матрицу \emph{B} этого же оператора в базисе \(({e'}_{1},\ \ {e'}_{2},\ \ {e'}_{3}),\) где \({e'}_{1} = 2e_{1} + e_{2}\), \({e'}_{2} = - e_{1} + 2e_{2} + 3e_{3}\),\({e'}_{3} = - e_{1} + e_{2} + e_{3}\) \\
C2. Даны векторы \(e_{1},e_{2},e_{3}\), \(a_{1},a_{2},a_{3}\) линейного пространства \(R^{3}\). Найдите матрицу перехода от базиса \(e_{1},e_{2},e_{3}\) к базису \(a_{1},a_{2},a_{3}\).
\(e_{1} = (2,0,1)\),\(e_{2} = ( - 1,2,3)\),\(e_{3} = ( - 1,1,1)\) и \(a_{1} = (1,0,2)\),\(a_{2} = (3, - 1,4)\),\(a_{3} = (2, - 2,1)\) \\
C3. Найти жорданову нормальную форму матрицы \(A = \begin{pmatrix}
0 & 3 & 1 \\
3 & 0 & 1 \\
 - 2 & 2 & 1
\end{pmatrix}\) \\

\end{tabular}
\vspace{1cm}


\begin{tabular}{m{17cm}}
\textbf{82-вариант}
\newline

T1. 11. Обратное преобразование. \\
T2. 2. Евклидово пространство. Неравенство Коши-Буняковского. Процесс ортогонализации. \\
A1. Доказать, что векторы \(\overrightarrow{a} = (2;\ \ 1;\ \ 1),\) \(\overrightarrow{b} = (1;\ \ 2;\ \ 1)\) и \(\overrightarrow{c} = (2;\ \ 1;\ \ 1)\) образуют базис пространства \(\mathbf{R}^{3},\) и найти координаты вектора \(\overrightarrow{d} = (1;\ \ 3;\ \ 1)\) в этом базисе. \\
A2. Известно, что оператор \emph{A} переводит базисные векторы \(\overrightarrow{i} = (1;\ \ 0;\ \ 0),\) \(\overrightarrow{j} = (0;\ \ 1;\ \ 0),\) \(\overrightarrow{k} = (0;\ \ 0;\ \ 1)\) линейного пространства \(\mathbf{R}^{3}\) в векторы \({\overline{a}}_{1} = (0;\ \ 1;\ \ 1),\) \({\overline{a}}_{2} = (3;\ \ 1;\ \ 1),\) \({\overline{a}}_{3} = (3;0;1).\) В базисе \(\overrightarrow{i},\overrightarrow{j},\overrightarrow{k}\) найти: 1)матрицу оператора \(A\) ; 2)образ вектора \(\overline{b} = (1;\ \  - 1;\ \  - 1).\) \\
A3. В пространстве \(\mathbb{C}^{3}\) со скалярным произведением \(\left\langle x,y \right\rangle = \sum_{k = 1}^{3}{x_{k}\overline{y_{k}}}\), найдите сопряженный оператор \(A^{*}\) для заданного оператора \(A\). Является ли \(A\)самосопряженным? \(Ax = \left( x_{1} + ix_{3},x_{3} + 2ix_{2},ix_{2} - 2ix_{3} \right)\); \\
B1. С помощью процесса ортогонализации Грамма- Шмидта ортонормировать следующие системы векторов, используя стандартное скалярное произведение: \\
B2. Найти собственные значения и собственные векторы оператора \emph{А}, заданного в некотором базисе пространства \(V^{3}\) матрицей \(A = \begin{pmatrix}
2 & 1 & 0 \\
1 & 3 & - 1 \\
 - 1 & 2 & 3
\end{pmatrix}\). \\
B3. Приведите квадратичные формы \(G_{1}\) и \(G_{2}\) к каноническому виду. \(G_{1} = 3x_{1}^{2} - 2x_{2}^{2} + 2x_{3}^{2} + 4x_{1}x_{2} - 3x_{1}x_{3} - x_{2}x_{3}\), \(G_{2} = 2x_{1}x_{3} + 4x_{1}x_{2} - 2x_{2}x_{3}\) \\
C1. В базисе \((e_{1},\ \ e_{2},\ \ e_{3})\) пространства \(V^{3}\) оператор \emph{A} имеет матрицу \(A = \begin{bmatrix}
3 & 2 & - 1 \\
4 & 0 & 2 \\
 - 1 & 2 & - 1
\end{bmatrix}\ \ .\) Найти матрицу \emph{B} этого же оператора в базисе \(({e'}_{1},\ \ {e'}_{2},\ \ {e'}_{3}),\) где \({e'}_{1} = 2e_{1} + e_{2}\), \({e'}_{2} = - e_{1} + 2e_{2} + 3e_{3}\),\({e'}_{3} = - e_{1} + e_{2} + e_{3}\) \\
C2. 12.Даны векторы \(e_{1},e_{2},e_{3}\), \(a_{1},a_{2},a_{3}\) линейного пространства \(R^{3}\). Найдите матрицу перехода от базиса \(e_{1},e_{2},e_{3}\) к базису \(a_{1},a_{2},a_{3}\).
\(e_{1} = (1,1,1)\),\(e_{2} = (1,1,2)\),\(e_{3} = (1,2,3)\) и \(a_{1} = (2,0,1)\),\(a_{2} = ( - 1,2,3)\),\(a_{3} = ( - 1,1,1)\)
 \\
C3. Найти жорданову нормальную форму матрицы \(A = \begin{pmatrix}
2 & - 1 & - 1 \\
2 & - 1 & - 2 \\
 - 1 & 1 & 2
\end{pmatrix}\) \\

\end{tabular}
\vspace{1cm}


\begin{tabular}{m{17cm}}
\textbf{83-вариант}
\newline

T1. 9. Линейные преобразования и их матрица. \\
T2. 4. Линейные, билинейные, и квадратичные формы. Преобразование матрицы линейного вида при изменении базиса. \\
A1. Доказать, что векторы \(\overrightarrow{a} = (3;\ \ 1;\ \ 0),\) \(\overrightarrow{b} = (4;\ \ 3;\ \ 2)\) и \(\overrightarrow{c} = ( - 1;\ \  - 4;\ \ 3)\) образуют базис пространства \(\mathbf{R}^{3},\) и найти координаты вектора \(\overrightarrow{d} = ( - 1;\ \ 2;\ \ 5)\) в этом базисе.
 \\
A2. Известно, что оператор \emph{A} переводит базисные векторы \(\overrightarrow{i} = (1;\ \ 0;\ \ 0),\) \(\overrightarrow{j} = (0;\ \ 1;\ \ 0),\) \(\overrightarrow{k} = (0;\ \ 0;\ \ 1)\) линейного пространства \(\mathbf{R}^{3}\) в векторы \({\overline{a}}_{1} = (1;\ \ 1;\ \ 1),{\overline{a}}_{2} = (3;\ \ 0;\ \ 1),\) \({\overline{a}}_{3} = (3;\ \ 1;\ \  - 1).\)В базисе \(\overrightarrow{i},\overrightarrow{j},\overrightarrow{k}\) найти:1)матрицу оператора \(A\) ;2)образ вектора \(\overline{b} = (1;\ \ 2;\ \ 1).\) \\
A3. В пространстве \(\mathbb{C}^{3}\) со скалярным произведением \(\left\langle x,y \right\rangle = \sum_{k = 1}^{3}{x_{k}\overline{y_{k}}}\), найдите сопряженный оператор \(A^{*}\) для заданного оператора \(A\). Является ли \(A\)самосопряженным? \(Ax = \left( ix_{1} + x_{2},x_{1} + ix_{2},x_{2} + ix_{3} \right)\); \\
B1. С помощью процесса ортогонализации Грамма- Шмидта ортонормировать следующие системы векторов, используя стандартное скалярное произведение: \\
B2. Найти собственные значения и собственные векторы оператора \emph{А}, заданного в некотором базисе пространства \(V^{3}\) матрицей \(A = \begin{pmatrix}
1 & - 2 & - 1 \\
 - 1 & 1 & 1 \\
1 & 0 & - 1
\end{pmatrix}\) \\
B3. Приведите квадратичные формы \(G_{1}\) и \(G_{2}\) к каноническому виду. \(G_{1} = 2x_{1}^{2} + 3x_{2}^{2} + 4x_{3}^{2} - 2x_{1}x_{2} + 4x_{1}x_{3} - 3x_{2}x_{3}\), \(G_{2} = x_{1}x_{3} - 2x_{2}x_{3}\) \\
C1. В базисе \((e_{1},\ \ e_{2},\ \ e_{3})\) пространства \(V^{3}\) оператор \emph{A} имеет матрицу \(A = \begin{bmatrix}
0 & 1 & - 3 \\
2 & 4 & 1 \\
0 & 3 & - 3
\end{bmatrix}\ \ .\) Найти матрицу \emph{B} этого же оператора в базисе \(({e'}_{1},\ \ {e'}_{2},\ \ {e'}_{3}),\) где \({e'}_{1} = 2e_{1} + e_{2}\), \({e'}_{2} = - e_{1} + 2e_{2} + 3e_{3}\),\({e'}_{3} = - e_{1} + e_{2} + e_{3}\) \\
C2. Даны векторы \(e_{1},e_{2},e_{3}\), \(a_{1},a_{2},a_{3}\) линейного пространства \(R^{3}\). Найдите матрицу перехода от базиса \(e_{1},e_{2},e_{3}\) к базису \(a_{1},a_{2},a_{3}\).
\(e_{1} = ( - 3,0,1)\),\(e_{2} = (0,2,3)\),\(e_{3} = ( - 1, - 1, - 1)\) и \(a_{1} = (1,1,1)\),\(a_{2} = (1,1,2)\),\(a_{3} = (1,2,3)\) \\
C3. Найти жорданову нормальную форму матрицы \(A = \begin{pmatrix}
 - 1 & 4 & 3 \\
 - 2 & 5 & 3 \\
2 & - 4 & - 2
\end{pmatrix}\). \\

\end{tabular}
\vspace{1cm}


\begin{tabular}{m{17cm}}
\textbf{84-вариант}
\newline

T1. 3. Ортогональное дополнение и ортогональная проекция. \\
T2. 16. Унитарные преобразования и их собственные значения и канонический вид. \\
A1. Доказать, что векторы \(\overrightarrow{a} = (2;\ \ 1;\ \ 1),\) \(\overrightarrow{b} = (1;\ \ 2;\ \ 1)\) и \(\overrightarrow{c} = (2;\ \ 3;\ \  - 1)\) образуют базис пространства \(\mathbf{R}^{3},\) и найти координаты вектора \(\overrightarrow{d} = (2;\ \ 3;\ \  - 1)\) в этом базисе. \\
A2. Известно, что оператор \emph{A} переводит базисные векторы \(\overrightarrow{i} = (1;\ \ 0;\ \ 0),\) \(\overrightarrow{j} = (0;\ \ 1;\ \ 0),\) \(\overrightarrow{k} = (0;\ \ 0;\ \ 1)\) линейного пространства \(\mathbf{R}^{3}\) в векторы \({\overline{a}}_{1} = (1;\ \ 0;\ \ 1),\) \({\overline{a}}_{2} = (3;\ \ 2;\ \ 1),\) \({\overline{a}}_{3} = (3;\ \ 1;\ \ 1).\) В базисе \(\overrightarrow{i},\overrightarrow{j},\overrightarrow{k}\) найти: 1)матрицу оператора \(A\) ; 2)образ вектора \(\overline{b} = (1;\ \  - 2;\ \ 3).\) \\
A3. В пространстве \(\mathbb{C}^{3}\) со скалярным произведением \(\left\langle x,y \right\rangle = \sum_{k = 1}^{3}{x_{k}\overline{y_{k}}}\), найдите сопряженный оператор \(A^{*}\) для заданного оператора \(A\). Является ли \(A\)самосопряженным? \(Ax = \left( x_{1} + 2ix_{3},ix_{2} - x_{3},x_{2} - ix_{3} \right)\); \\
B1. С помощью процесса ортогонализации Грамма- Шмидта ортонормировать следующие системы векторов, используя стандартное скалярное произведение: \\
B2. Найти собственные значения и собственные векторы оператора \emph{А}, заданного в некотором базисе пространства \(V^{3}\) матрицей \(A = \begin{bmatrix}
 - 1 & - 2 & 0 \\
0 & - 2 & 0 \\
2 & 2 & 1
\end{bmatrix}.\) \\
B3. Приведите квадратичные формы \(G_{1}\) и \(G_{2}\) к каноническому виду. \(G_{1} = 5x_{1}^{2} + 6x_{2}^{2} - 3x_{3}^{2} + 4x_{1}x_{2} - 2x_{2}x_{3}\), \(G_{2} = 6x_{2}x_{3} - x_{1}x_{2}\) \\
C1. В базисе \((e_{1},\ \ e_{2},\ \ e_{3})\) пространства \(V^{3}\) оператор \emph{A} имеет матрицу \(A = \begin{bmatrix}
1 & 3 & - 1 \\
2 & 0 & 4 \\
1 & 1 & 1
\end{bmatrix}\ \ .\) Найти матрицу \emph{B} этого же оператора в базисе \(({e'}_{1},\ \ {e'}_{2},\ \ {e'}_{3}),\) где \({e'}_{1} = 2e_{1} + e_{2}\), \({e'}_{2} = - e_{1} + 2e_{2} + 3e_{3}\),\({e'}_{3} = - e_{1} + e_{2} + e_{3}\) \\
C2. Даны векторы \(e_{1},e_{2},e_{3}\), \(a_{1},a_{2},a_{3}\) линейного пространства \(R^{3}\). Найдите матрицу перехода от базиса \(e_{1},e_{2},e_{3}\) к базису \(a_{1},a_{2},a_{3}\).
\(e_{1} = (3,1, - 1)\),\(e_{2} = ( - 2,0,1)\),\(e_{3} = (2,7,3)\) и \(a_{1} = (2,1, - 3)\),\(a_{2} = (3,2, - 5)\),\(a_{3} = (1, - 1,1)\) \\
C3. Найти жорданову нормальную форму матрицы \(A = \begin{pmatrix}
2 & 1 & 1 \\
1 & 2 & 1 \\
1 & 1 & 2
\end{pmatrix}\). \\

\end{tabular}
\vspace{1cm}


\begin{tabular}{m{17cm}}
\textbf{85-вариант}
\newline

T1. 17. Взаимозаменяемые преобразования. \\
T2. 14. Сопряженное преобразование для данного преобразования. \\
A1. Доказать, что векторы \(\overrightarrow{a} = (2;\ \ 1;\ \  - 3),\) \(\overrightarrow{b} = ( - 1;\ \ 2;\ \ 4)\) и \(\overrightarrow{c} = (3;\ \  - 4;\ \ 2)\) образуют базис пространства \(\mathbf{R}^{3},\) и найти координаты вектора \(\overrightarrow{d} = ( - 4;\ \ 19;\ \ 3)\) в этом базисе. \\
A2. Известно, что оператор \emph{A} переводит базисные векторы \(\overrightarrow{i} = (1;\ \ 0;\ \ 0),\) \(\overrightarrow{j} = (0;\ \ 1;\ \ 0),\) \(\overrightarrow{k} = (0;\ \ 0;\ \ 1)\) линейного пространства \(\mathbf{R}^{3}\) в векторы \({\overline{a}}_{1} = (1;\ \ 1;\ \ 1),{\overline{a}}_{2} = (3;\ \ 0;\ \ 1),\) \({\overline{a}}_{3} = (3;\ \ 1;\ \  - 1).\)В базисе \(\overrightarrow{i},\overrightarrow{j},\overrightarrow{k}\) найти:1)матрицу оператора \(A\) ;2)образ вектора \(\overline{b} = (1;\ \ 2;\ \ 1).\) \\
A3. В пространстве \(\mathbb{C}^{3}\) со скалярным произведением \(\left\langle x,y \right\rangle = \sum_{k = 1}^{3}{x_{k}\overline{y_{k}}}\), найдите сопряженный оператор \(A^{*}\) для заданного оператора \(A\). Является ли \(A\)самосопряженным? \(Ax = \left( x_{1} + 2ix_{3},ix_{2} - x_{3},x_{2} - ix_{3} \right)\); \\
B1. \(a_{1} = (1;2;1)\), \(a_{2} = (2;3;3)\), \(a_{3} = (3;8;2)\); \\
B2. Найти собственные значения и собственные векторы оператора \emph{А}, заданного в некотором базисе пространства \(V^{3}\) матрицей \(A = \begin{pmatrix}
1 & - 2 & - 1 \\
 - 1 & 1 & 1 \\
1 & 0 & - 1
\end{pmatrix}\) \\
B3. Приведите квадратичные формы \(G_{1}\) и \(G_{2}\) к каноническому виду. \(G_{1} = x_{1}^{2} - 2x_{3}^{2} + 2x_{1}x_{3} - 6x_{1}x_{2}\), \(G_{2} = 6x_{2}x_{3} - 4x_{1}x_{2} + x_{1}x_{3}\)
 \\
C1. В базисе \((e_{1},\ \ e_{2},\ \ e_{3})\) пространства \(V^{3}\) оператор \emph{A} имеет матрицу \(A = \begin{bmatrix}
3 & 2 & - 1 \\
4 & 0 & 2 \\
 - 1 & 2 & - 1
\end{bmatrix}\ \ .\) Найти матрицу \emph{B} этого же оператора в базисе \(({e'}_{1},\ \ {e'}_{2},\ \ {e'}_{3}),\) где \({e'}_{1} = 2e_{1} + e_{2}\), \({e'}_{2} = - e_{1} + 2e_{2} + 3e_{3}\),\({e'}_{3} = - e_{1} + e_{2} + e_{3}\) \\
C2. Даны векторы \(e_{1},e_{2},e_{3}\), \(a_{1},a_{2},a_{3}\) линейного пространства \(R^{3}\). Найдите матрицу перехода от базиса \(e_{1},e_{2},e_{3}\) к базису \(a_{1},a_{2},a_{3}\).
\(e_{1} = (2,0,1)\),\(e_{2} = ( - 1,2,3)\),\(e_{3} = ( - 1,1,1)\) и \(a_{1} = ( - 3,0,1)\),\(a_{2} = (0,2,3)\),\(a_{3} = ( - 1, - 1, - 1)\) \\
C3. Найти жорданову нормальную форму матрицы \(A = \begin{pmatrix}
2 & - 1 & 2 \\
5 & - 3 & 3 \\
 - 1 & 0 & - 2
\end{pmatrix}\). \\

\end{tabular}
\vspace{1cm}


\begin{tabular}{m{17cm}}
\textbf{86-вариант}
\newline

T1. 15. Самосопряженные преобразования и их канонический вид. \\
T2. 8. Квадратичные формы в комплексном пространстве и их канонические виды. \\
A1. Доказать, что векторы \(\overrightarrow{a} = (1;\ \ 0;\ \ 1),\) \(\overrightarrow{b} = (1;\ \ 1;\ \ 1)\) и \(\overrightarrow{c} = ( - 1;\ \ 2;\ \ 1)\) образуют базис пространства \(\mathbf{R}^{3},\) и найти координаты вектора \(\overrightarrow{d} = (0;\ \ 10;\ \ 3)\) в этом базисе. \\
A2. Известно, что оператор \emph{A} переводит базисные векторы \(\overrightarrow{i} = (1;\ \ 0;\ \ 0),\) \(\overrightarrow{j} = (0;\ \ 1;\ \ 0),\) \(\overrightarrow{k} = (0;\ \ 0;\ \ 1)\) линейного пространства \(\mathbf{R}^{3}\) в векторы \({\overline{a}}_{1} = (1;\ \ 1;\ \ 0),\) \({\overline{a}}_{2} = (3;\ \ 2;\ \ 1),\) \({\overline{a}}_{3} = (1;2;\ \ 1).\) В базисе \(\overrightarrow{i},\overrightarrow{j},\overrightarrow{k}\) найти: 1)матрицу оператора \(A\) ; 2)образ вектора \(\overline{b} = (1;\ \ 1;\ \  - 2).\) \\
A3. В пространстве \(\mathbb{C}^{3}\) со скалярным произведением \(\left\langle x,y \right\rangle = \sum_{k = 1}^{3}{x_{k}\overline{y_{k}}}\), найдите сопряженный оператор \(A^{*}\) для заданного оператора \(A\). Является ли \(A\)самосопряженным? \(Ax = \left( 3ix_{1} + x_{2},x_{1} + 2ix_{2},ix_{2} - x_{3} \right)\); \\
B1. \(a_{1} = ( - 1;3;7)\),\(a_{2} = (0;2; - 1)\), \(a_{3} = (1; - 2; - 8)\); \\
B2. Найти собственные значения и собственные векторы оператора \emph{А}, заданного в некотором базисе пространства \(V^{3}\) матрицей \(A = \begin{pmatrix}
2 & - 1 & 2 \\
5 & - 3 & 3 \\
 - 1 & 0 & - 2
\end{pmatrix}\). \\
B3. Приведите квадратичные формы \(G_{1}\) и \(G_{2}\) к каноническому виду. \(G_{1} = 2x_{1}^{2} + x_{2}^{2} + x_{3}^{2} + 4x_{1}x_{2} - 2x_{1}x_{3}\), \(G_{2} = x_{1}x_{2} + x_{1}x_{3} + 4x_{2}x_{3}\) \\
C1. В базисе \((e_{1},\ \ e_{2},\ \ e_{3})\) пространства \(V^{3}\) оператор \emph{A} имеет матрицу \(A = \begin{bmatrix}
0 & 1 & - 3 \\
2 & 4 & 1 \\
0 & 3 & - 3
\end{bmatrix}\ \ .\) Найти матрицу \emph{B} этого же оператора в базисе \(({e'}_{1},\ \ {e'}_{2},\ \ {e'}_{3}),\) где \({e'}_{1} = 2e_{1} + e_{2}\), \({e'}_{2} = - e_{1} + 2e_{2} + 3e_{3}\),\({e'}_{3} = - e_{1} + e_{2} + e_{3}\) \\
C2. 12.Даны векторы \(e_{1},e_{2},e_{3}\), \(a_{1},a_{2},a_{3}\) линейного пространства \(R^{3}\). Найдите матрицу перехода от базиса \(e_{1},e_{2},e_{3}\) к базису \(a_{1},a_{2},a_{3}\).
\(e_{1} = (1,1,1)\),\(e_{2} = (1,1,2)\),\(e_{3} = (1,2,3)\) и \(a_{1} = (2,0,1)\),\(a_{2} = ( - 1,2,3)\),\(a_{3} = ( - 1,1,1)\)
 \\
C3. Найти жорданову нормальную форму матрицы \(A = \begin{pmatrix}
 - 1 & 4 & 3 \\
 - 2 & 5 & 3 \\
2 & - 4 & - 2
\end{pmatrix}\). \\

\end{tabular}
\vspace{1cm}


\begin{tabular}{m{17cm}}
\textbf{87-вариант}
\newline

T1. 1. Линейные пространства. Линейные подпространства. Сумма и пересечение подпространств. \\
T2. 6. Положительно определенные квадратичные формы. \\
A1. Доказать, что векторы \(\overrightarrow{a} = (1;\ \ 2;\ \ 1),\) \(\overrightarrow{b} = (1;\ \ 1;\ \  - 3)\) и \(\overrightarrow{c} = ( - 1;\ \ 2;\ \ 1)\) образуют базис пространства \(\mathbf{R}^{3},\) и найти координаты вектора \(\overrightarrow{d} = (0;\ \ 10;\ \  - 2)\) в этом базисе. \\
A2. Известно, что оператор \emph{A} переводит базисные векторы \(\overrightarrow{i} = (1;\ \ 0;\ \ 0),\) \(\overrightarrow{j} = (0;\ \ 1;\ \ 0),\) \(\overrightarrow{k} = (0;\ \ 0;\ \ 1)\) линейного пространства \(\mathbf{R}^{3}\) в векторы \({\overline{a}}_{1} = (1;\ \ 0;\ \ 1),\) \({\overline{a}}_{2} = (0;\ \ 2;\ \ 1),\) \({\overline{a}}_{3} = (3;\ \ 1;\ \ 1).\) В базисе \(\overrightarrow{i},\overrightarrow{j},\overrightarrow{k}\) найти: 1)матрицу оператора \(A\) ; 2)образ вектора \(\overline{b} = (1;\ \ 2;\ \ 3).\) \\
A3. В пространстве \(\mathbb{C}^{3}\) со скалярным произведением \(\left\langle x,y \right\rangle = \sum_{k = 1}^{3}{x_{k}\overline{y_{k}}}\), найдите сопряженный оператор \(A^{*}\) для заданного оператора \(A\). Является ли \(A\)самосопряженным? \(Ax = \left( 2ix_{1} + ix_{3},x_{1} + x_{2} + ix_{3},ix_{3} \right)\); \\
B1. \(a_{1} = (0;1; - 2)\),\(a_{2} = (1; - 1;1)\), \(a_{3} = ( - 2;0;3)\); \\
B2. Найти собственные значения и собственные векторы оператора \emph{А}, заданного в некотором базисе пространства \(V^{3}\) матрицей \(A = \begin{pmatrix}
1 & - 1 & 1 \\
1 & 1 & - 1 \\
2 & - 1 & 0
\end{pmatrix}\). \\
B3. Приведите квадратичные формы \(G_{1}\) и \(G_{2}\) к каноническому виду. \(G_{1} = x_{1}^{2} - 3x_{3}^{2} - 2x_{1}x_{2} - 2x_{1}x_{3} - 6x_{2}x_{3}\), \(G_{2} = 2x_{1}x_{2} - x_{1}x_{3} + 2x_{2}x_{3}\) \\
C1. В базисе \((e_{1},\ \ e_{2},\ \ e_{3})\) пространства \(V^{3}\) оператор \emph{A} имеет матрицу \(A = \begin{bmatrix}
 - 1 & 2 & 1 \\
0 & 1 & - 4 \\
5 & - 1 & 2
\end{bmatrix}\ \ .\) Найти матрицу \emph{B} этого же оператора в базисе \(({e'}_{1},\ \ {e'}_{2},\ \ {e'}_{3}),\) где \({e'}_{1} = e_{1} - e_{3},\) \({e'}_{2} = e_{2} + e_{3},\) \({e'}_{3} = e_{3}.\) \\
C2. 10.Даны векторы \(e_{1},e_{2},e_{3}\), \(a_{1},a_{2},a_{3}\) линейного пространства \(R^{3}\). Найдите матрицу перехода от базиса \(e_{1},e_{2},e_{3}\) к базису \(a_{1},a_{2},a_{3}\).
\(e_{1} = (4,0,5)\),\(e_{2} = ( - 2,1,3)\),\(e_{3} = ( - 5,1, - 1)\) и \(a_{1} = (1,2,1)\),\(a_{2} = (2,3,3)\),\(a_{3} = (3,8,2)\) \\
C3. Найти жорданову нормальную форму матрицы \(A = \begin{pmatrix}
3 & - 2 & 6 \\
 - 2 & 6 & 3 \\
6 & 3 & - 2
\end{pmatrix}\). \\

\end{tabular}
\vspace{1cm}


\begin{tabular}{m{17cm}}
\textbf{88-вариант}
\newline

T1. 5. Методы приведения квадратичной формы к каноническому форму. \\
T2. 18. Нормальные преобразования и их канонический вид. \\
A1. Доказать, что векторы \(\overrightarrow{a} = (2;\ \ 1;\ \ 1),\) \(\overrightarrow{b} = (1;\ \ 2;\ \ 1)\) и \(\overrightarrow{c} = (2;\ \ 1;\ \ 1)\) образуют базис пространства \(\mathbf{R}^{3},\) и найти координаты вектора \(\overrightarrow{d} = (1;\ \ 3;\ \ 1)\) в этом базисе. \\
A2. Известно, что оператор \emph{A} переводит базисные векторы \(\overrightarrow{i} = (1;\ \ 0;\ \ 0),\) \(\overrightarrow{j} = (0;\ \ 1;\ \ 0),\) \(\overrightarrow{k} = (0;\ \ 0;\ \ 1)\) линейного пространства \(\mathbf{R}^{3}\) в векторы \({\overline{a}}_{1} = (0;\ \ 1;\ \ 1),\) \({\overline{a}}_{2} = (3;\ \ 1;\ \ 1),\) \({\overline{a}}_{3} = (3;0;1).\) В базисе \(\overrightarrow{i},\overrightarrow{j},\overrightarrow{k}\) найти: 1)матрицу оператора \(A\) ; 2)образ вектора \(\overline{b} = (1;\ \  - 1;\ \  - 1).\) \\
A3. В пространстве \(\mathbb{C}^{3}\) со скалярным произведением \(\left\langle x,y \right\rangle = \sum_{k = 1}^{3}{x_{k}\overline{y_{k}}}\), найдите сопряженный оператор \(A^{*}\) для заданного оператора \(A\). Является ли \(A\)самосопряженным? \(Ax = \left( ix_{1} + x_{2},x_{1} + ix_{2},x_{2} + ix_{3} \right)\); \\
B1. С помощью процесса ортогонализации Грамма- Шмидта ортонормировать следующие системы векторов, используя стандартное скалярное произведение: \\
B2. Найти собственные значения и собственные векторы оператора \emph{А}, заданного в некотором базисе пространства \(V^{3}\) матрицей \(A = \begin{bmatrix}
 - 1 & - 2 & 0 \\
0 & - 2 & 0 \\
2 & 2 & 1
\end{bmatrix}.\) \\
B3. Приведите квадратичные формы \(G_{1}\) и \(G_{2}\) к каноническому виду. \(G_{1} = 5x_{1}^{2} + 6x_{2}^{2} - 3x_{3}^{2} + 4x_{1}x_{2} - 2x_{2}x_{3}\), \(G_{2} = 6x_{2}x_{3} - x_{1}x_{2}\) \\
C1. В базисе \((e_{1},\ \ e_{2},\ \ e_{3})\) пространства \(V^{3}\) оператор \emph{A} имеет матрицу \(A = \begin{bmatrix}
 - 3 & 1 & 4 \\
0 & 3 & 2 \\
 - 5 & - 1 & 2
\end{bmatrix}\ \ .\) Найти матрицу \emph{B} этого же оператора в базисе \(({e'}_{1},\ \ {e'}_{2},\ \ {e'}_{3}),\) где \({e'}_{1} = 2e_{1} + e_{2}\), \({e'}_{2} = - e_{1} + 2e_{2} + 3e_{3}\),\({e'}_{3} = - e_{1} + e_{2} + e_{3}\) \\
C2. Даны векторы \(e_{1},e_{2},e_{3}\), \(a_{1},a_{2},a_{3}\) линейного пространства \(R^{3}\). Найдите матрицу перехода от базиса \(e_{1},e_{2},e_{3}\) к базису \(a_{1},a_{2},a_{3}\).
\(e_{1} = ( - 2,3,1)\),\(e_{2} = (0,2,1)\),\(e_{3} = (1,2,1)\) и \(a_{1} = ( - 1,3,7)\),\(a_{2} = (0,2, - 1)\),\(a_{3} = (1, - 2, - 8)\) \\
C3. Найти жорданову нормальную форму матрицы \(A = \begin{pmatrix}
 - 1 & 4 & 3 \\
 - 2 & 5 & 3 \\
2 & - 4 & - 2
\end{pmatrix}\). \\

\end{tabular}
\vspace{1cm}


\begin{tabular}{m{17cm}}
\textbf{89-вариант}
\newline

T1. 19. Полиномиальные матрицы и диагональные нормальные формы. \\
T2. 12. Связь между матрицами линейных преобразовании в разных базисах. \\
A1. Доказать, что векторы \(\overrightarrow{a} = (3;\ \ 1;\ \ 0),\) \(\overrightarrow{b} = (4;\ \ 3;\ \ 2)\) и \(\overrightarrow{c} = ( - 1;\ \  - 4;\ \ 3)\) образуют базис пространства \(\mathbf{R}^{3},\) и найти координаты вектора \(\overrightarrow{d} = ( - 1;\ \ 2;\ \ 5)\) в этом базисе.
 \\
A2. Известно, что оператор \emph{A} переводит базисные векторы \(\overrightarrow{i} = (1;\ \ 0;\ \ 0),\) \(\overrightarrow{j} = (0;\ \ 1;\ \ 0),\) \(\overrightarrow{k} = (0;\ \ 0;\ \ 1)\) линейного пространства \(\mathbf{R}^{3}\) в векторы \({\overline{a}}_{1} = (0;\ \ 1;\ \ 1),\) \({\overline{a}}_{2} = (3;\ \ 1;\ \ 1),\) \({\overline{a}}_{3} = (3;\ \ 1;\ \ 1).\) В базисе \(\overrightarrow{i},\overrightarrow{j},\overrightarrow{k}\) найти: 1)матрицу оператора \(A\) ; 2)образ вектора \(\overline{b} = (1;\ \ 1;\ \ 1).\) \\
A3. В пространстве \(\mathbb{C}^{3}\) со скалярным произведением \(\left\langle x,y \right\rangle = \sum_{k = 1}^{3}{x_{k}\overline{y_{k}}}\), найдите сопряженный оператор \(A^{*}\) для заданного оператора \(A\). Является ли \(A\)самосопряженным?\(Ax = \left( x_{1} - 2ix_{2},x_{3} + 2ix_{2},ix_{2} + 2ix_{3} \right)\);
 \\
B1. \(a_{1} = ( - 2;3;1)\),\(a_{2} = (0;2;1)\), \(a_{3} = (1;2;1)\); \\
B2. Найти собственные значения и собственные векторы оператора \emph{А}, заданного в некотором базисе пространства \(V^{3}\) матрицей \(A = \begin{pmatrix}
0 & 1 & 2 \\
 - 1 & 0 & - 2 \\
 - 2 & 2 & 0
\end{pmatrix}\).
 \\
B3. Приведите квадратичные формы \(G_{1}\) и \(G_{2}\) к каноническому виду. \(G_{1} = 3x_{1}^{2} - 2x_{2}^{2} + 2x_{3}^{2} + 4x_{1}x_{2} - 3x_{1}x_{3} - x_{2}x_{3}\), \(G_{2} = 2x_{1}x_{3} + 4x_{1}x_{2} - 2x_{2}x_{3}\) \\
C1. В базисе \((e_{1},\ \ e_{2},\ \ e_{3})\) пространства \(V^{3}\) оператор \emph{A} имеет матрицу \(A = \begin{bmatrix}
1 & 3 & - 1 \\
2 & 0 & 4 \\
1 & 1 & 1
\end{bmatrix}\ \ .\) Найти матрицу \emph{B} этого же оператора в базисе \(({e'}_{1},\ \ {e'}_{2},\ \ {e'}_{3}),\) где \({e'}_{1} = 2e_{1} + e_{2}\), \({e'}_{2} = - e_{1} + 2e_{2} + 3e_{3}\),\({e'}_{3} = - e_{1} + e_{2} + e_{3}\) \\
C2. Даны векторы \(e_{1},e_{2},e_{3}\), \(a_{1},a_{2},a_{3}\) линейного пространства \(R^{3}\). Найдите матрицу перехода от базиса \(e_{1},e_{2},e_{3}\) к базису \(a_{1},a_{2},a_{3}\).
\(e_{1} = (2,0,1)\),\(e_{2} = ( - 1,2,3)\),\(e_{3} = ( - 1,1,1)\) и \(a_{1} = (1,0,2)\),\(a_{2} = (3, - 1,4)\),\(a_{3} = (2, - 2,1)\) \\
C3. Найти жорданову нормальную форму матрицы \(A = \begin{pmatrix}
1 & 2 & 1 \\
1 & 2 & 4 \\
 - 1 & - 2 & - 3
\end{pmatrix}\). \\

\end{tabular}
\vspace{1cm}


\begin{tabular}{m{17cm}}
\textbf{90-вариант}
\newline

T1. 7. Комплексные евклидовы пространства. \\
T2. 20. Подобные матрицы. \\
A1. Доказать, что векторы \(\overrightarrow{a} = (2;\ \ 1;1),\) \(\overrightarrow{b} = ( - 1;\ \ 2;\ \ 4)\) и \(\overrightarrow{c} = (3;\ \ 3;\ \ 2)\) образуют базис пространства \(\mathbf{R}^{3},\) и найти координаты вектора \(\overrightarrow{d} = ( - 4;\ \ 2;\ \ 4)\) в этом базисе. \\
A2. Известно, что оператор \emph{A} переводит базисные векторы \(\overrightarrow{i} = (1;\ \ 0;\ \ 0),\) \(\overrightarrow{j} = (0;\ \ 1;\ \ 0),\) \(\overrightarrow{k} = (0;\ \ 0;\ \ 1)\) линейного пространства \(\mathbf{R}^{3}\) в векторы \({\overline{a}}_{1} = (1;\ \ 1;\ \ 1),{\overline{a}}_{2} = (3;\ \ 0;\ \ 1),\) \({\overline{a}}_{3} = (0;\ \ 2;\ \ 1).\)В базисе \(\overrightarrow{i},\overrightarrow{j},\overrightarrow{k}\) найти:1)матрицу оператора \(A\) ;2)образ вектора \(\overline{b} = (1;\ \ 2;\ \  - 2).\) \\
A3. В пространстве \(\mathbb{C}^{3}\) со скалярным произведением \(\left\langle x,y \right\rangle = \sum_{k = 1}^{3}{x_{k}\overline{y_{k}}}\), найдите сопряженный оператор \(A^{*}\) для заданного оператора \(A\). Является ли \(A\)самосопряженным? \(Ax = \left( ix_{1} + 2ix_{3},x_{3},x_{1} - 2ix_{3} \right)\); \\
B1. \(a_{1} = (1;1;1)\), \(a_{2} = (1;2;3)\), \(a_{3} = (1;1;2)\); \\
B2. Найти собственные значения и собственные векторы оператора \emph{А}, заданного в некотором базисе пространства \(V^{3}\) матрицей \(A = \begin{pmatrix}
2 & - 1 & 2 \\
1 & 0 & 2 \\
 - 2 & 1 & - 1
\end{pmatrix}\) \\
B3. Приведите квадратичные формы \(G_{1}\) и \(G_{2}\) к каноническому виду. \(G_{1} = x_{1}^{2} - 2x_{2}^{2} + x_{3}^{2} + 2x_{1}x_{2} + 4x_{1}x_{3} + 2x_{2}x_{3}\), \(G_{2} = 2x_{1}x_{3} - 4x_{2}x_{3}\) \\
C1. В базисе \((e_{1},\ \ e_{2},\ \ e_{3})\) пространства \(V^{3}\) оператор \emph{A} имеет матрицу \(A = \begin{bmatrix}
1 & - 1 & 2 \\
0 & 3 & - 1 \\
4 & 2 & 2
\end{bmatrix}\ \ .\) Найти матрицу \emph{B} этого же оператора в базисе \(({e'}_{1},\ \ {e'}_{2},\ \ {e'}_{3}),\) где \({e'}_{1} = e_{1} + 2e_{2},\) \({e'}_{2} = e_{1} - e_{3},\) \({e'}_{3} = e_{1} + e_{2} + e_{3}.\) \\
C2. Даны векторы \(e_{1},e_{2},e_{3}\), \(a_{1},a_{2},a_{3}\) линейного пространства \(R^{3}\). Найдите матрицу перехода от базиса \(e_{1},e_{2},e_{3}\) к базису \(a_{1},a_{2},a_{3}\).
\(e_{1} = ( - 3,0,1)\),\(e_{2} = (0,2,3)\),\(e_{3} = ( - 1, - 1, - 1)\) и \(a_{1} = (1,1,1)\),\(a_{2} = (1,1,2)\),\(a_{3} = (1,2,3)\) \\
C3. Найти жорданову нормальную форму матрицы \(A = \begin{pmatrix}
 - 1 & 3 & - 1 \\
 - 3 & 5 & - 1 \\
 - 3 & 3 & 1
\end{pmatrix}\). \\

\end{tabular}
\vspace{1cm}


\begin{tabular}{m{17cm}}
\textbf{91-вариант}
\newline

T1. 11. Обратное преобразование. \\
T2. 20. Подобные матрицы. \\
A1. Доказать, что векторы \(\overrightarrow{a} = (2;\ \ 1;\ \ 1),\) \(\overrightarrow{b} = (1;\ \ 2;\ \ 1)\) и \(\overrightarrow{c} = (2;\ \ 3;\ \  - 1)\) образуют базис пространства \(\mathbf{R}^{3},\) и найти координаты вектора \(\overrightarrow{d} = (2;\ \ 3;\ \  - 1)\) в этом базисе. \\
A2. Известно, что оператор \emph{A} переводит базисные векторы \(\overrightarrow{i} = (1;\ \ 0;\ \ 0),\) \(\overrightarrow{j} = (0;\ \ 1;\ \ 0),\) \(\overrightarrow{k} = (0;\ \ 0;\ \ 1)\) линейного пространства \(\mathbf{R}^{3}\) в векторы \({\overline{a}}_{1} = (1;\ \ 0;\ \ 1),\) \({\overline{a}}_{2} = (0;\ \ 1;\ \ 1),\) \({\overline{a}}_{3} = (3;\ \ 1;\ \ 1).\) В базисе \(\overrightarrow{i},\overrightarrow{j},\overrightarrow{k}\) найти: 1)матрицу оператора \(A\) ; 2)образ вектора \(\overline{b} = (1;\ \  - 2;\ \  - 3).\) \\
A3. В пространстве \(\mathbb{C}^{3}\) со скалярным произведением \(\left\langle x,y \right\rangle = \sum_{k = 1}^{3}{x_{k}\overline{y_{k}}}\), найдите сопряженный оператор \(A^{*}\) для заданного оператора \(A\). Является ли \(A\)самосопряженным?\(Ax = \left( x_{1} + 2ix_{3},2ix_{1} + ix_{2},x_{1} + ix_{3} \right)\); \\
B1. С помощью процесса ортогонализации Грамма- Шмидта ортонормировать следующие системы векторов, используя стандартное скалярное произведение: \\
B2. Найти собственные значения и собственные векторы оператора \emph{А}, заданного в некотором базисе пространства \(V^{3}\) матрицей \(A = \begin{bmatrix}
0 & - 2 & 0 \\
 - 2 & 6 & - 2 \\
0 & - 2 & 5
\end{bmatrix};\) \\
B3. Приведите квадратичные формы \(G_{1}\) и \(G_{2}\) к каноническому виду. \(G_{1} = x_{1}^{2} + 5x_{2}^{2} - 4x_{3}^{2} + 2x_{1}x_{3} - 4x_{1}x_{2}\), \(G_{2} = - 4x_{1}x_{2} + 2x_{1}x_{3}\) \\
C1. В базисе \((e_{1},\ \ e_{2},\ \ e_{3})\) пространства \(V^{3}\) оператор \emph{A} имеет матрицу \(A = \begin{bmatrix}
5 & - 2 & 1 \\
 - 1 & 0 & 4 \\
3 & 1 & 2
\end{bmatrix}\ \ .\) Найти матрицу \emph{B} этого же оператора в базисе \(({e'}_{1},\ \ {e'}_{2},\ \ {e'}_{3}),\) где \({e'}_{1} = 2e_{1} + 3e_{3},\) \({e'}_{2} = - e_{2},\) \({e'}_{3} = e_{1} + e_{2} + e_{3}.\) \\
C2. Даны векторы \(e_{1},e_{2},e_{3}\), \(a_{1},a_{2},a_{3}\) линейного пространства \(R^{3}\). Найдите матрицу перехода от базиса \(e_{1},e_{2},e_{3}\) к базису \(a_{1},a_{2},a_{3}\).
\(e_{1} = (3,5,8)\),\(e_{2} = (5,14,13)\),\(e_{3} = (1,9,2)\) и \(a_{1} = ( - 2,3,1)\),\(a_{2} = (0,2,1)\),\(a_{3} = (1,2,1)\) \\
C3. Найти жорданову нормальную форму матрицы \(A = \begin{pmatrix}
2 & - 1 & - 1 \\
2 & - 1 & - 2 \\
 - 1 & 1 & 2
\end{pmatrix}\). \\

\end{tabular}
\vspace{1cm}


\begin{tabular}{m{17cm}}
\textbf{92-вариант}
\newline

T1. 13. Инвариантные подпространства. Собственные векторы и собственные значения. \\
T2. 6. Положительно определенные квадратичные формы. \\
A1. Доказать, что векторы \(\overrightarrow{a} = (3;\ \ 5;\ \ 4),\) \(\overrightarrow{b} = (4;\ \ 3;\ \ 2)\) и \(\overrightarrow{c} = ( - 1;\ \  - 4;\ \ 3)\) образуют базис пространства \(\mathbf{R}^{3},\) и найти координаты вектора \(\overrightarrow{d} = ( - 2;\ \  - 2;\ \ 5)\) в этом базисе. \\
A2. Известно, что оператор \emph{A} переводит базисные векторы \(\overrightarrow{i} = (1;\ \ 0;\ \ 0),\) \(\overrightarrow{j} = (0;\ \ 1;\ \ 0),\) \(\overrightarrow{k} = (0;\ \ 0;\ \ 1)\) линейного пространства \(\mathbf{R}^{3}\) в векторы \({\overline{a}}_{1} = (1;\ \ 1;\ \ 0),\) \({\overline{a}}_{2} = (3;\ \ 2;\ \ 1),\) \({\overline{a}}_{3} = (3;\ \ 1;\ \ 1).\) В базисе \(\overrightarrow{i},\overrightarrow{j},\overrightarrow{k}\) найти: 1)матрицу оператора \(A\) ; 2)образ вектора \(\overline{b} = (1;\ \ 2;\ \ 3).\) \\
A3. В пространстве \(\mathbb{C}^{3}\) со скалярным произведением \(\left\langle x,y \right\rangle = \sum_{k = 1}^{3}{x_{k}\overline{y_{k}}}\), найдите сопряженный оператор \(A^{*}\) для заданного оператора \(A\). Является ли \(A\)самосопряженным? \(Ax = \left( ix_{1} + x_{3},ix_{3} - ix_{2},x_{1} - ix_{3} \right)\); \\
B1. С помощью процесса ортогонализации Грамма- Шмидта ортонормировать следующие системы векторов, используя стандартное скалярное произведение: \\
B2. 
Найти собственные значения и собственные векторы оператора \emph{А}, заданного в некотором базисе пространства \(V^{3}\) матрицей \(A = \begin{bmatrix}
0 & - 1 & 1 \\
 - 1 & 0 & 1 \\
1 & 1 & 0
\end{bmatrix}.\) \\
B3. Приведите квадратичные формы \(G_{1}\) и \(G_{2}\) к каноническому виду. \(G_{1} = 2x_{1}^{2} + 3x_{2}^{2} + 4x_{3}^{2} - 2x_{1}x_{2} + 4x_{1}x_{3} - 3x_{2}x_{3}\), \(G_{2} = x_{1}x_{3} - 2x_{2}x_{3}\) \\
C1. В базисе \((e_{1},\ \ e_{2},\ \ e_{3})\) пространства \(V^{3}\) оператор \emph{A} имеет матрицу \(A = \begin{bmatrix}
 - 1 & 2 & 4 \\
 - 4 & 2 & 0 \\
3 & 3 & - 3
\end{bmatrix}\ \ .\) Найти матрицу \emph{B} этого же оператора в базисе \(({e'}_{1},\ \ {e'}_{2},\ \ {e'}_{3}),\) где \({e'}_{1} = 2e_{1} + e_{2}\), \({e'}_{2} = - e_{1} + 2e_{2} + 3e_{3}\),\({e'}_{3} = - e_{1} + e_{2} + e_{3}\) \\
C2. 
Даны векторы \(e_{1},e_{2},e_{3}\), \(a_{1},a_{2},a_{3}\) линейного пространства \(R^{3}\). Найдите матрицу перехода от базиса \(e_{1},e_{2},e_{3}\) к базису \(a_{1},a_{2},a_{3}\).
\(e_{1} = (2,1, - 3)\),\(e_{2} = (3,2, - 5)\),\(e_{3} = (1, - 1,1)\) и \(a_{1} = (0,1, - 2)\),\(a_{2} = ( - 2,0,3)\),\(a_{3} = (1, - 1,1)\) \\
C3. Найти жорданову нормальную форму матрицы \(A = \begin{pmatrix}
2 & - 1 & - 1 \\
2 & - 1 & - 2 \\
 - 1 & 1 & 2
\end{pmatrix}\) \\

\end{tabular}
\vspace{1cm}


\begin{tabular}{m{17cm}}
\textbf{93-вариант}
\newline

T1. 3. Ортогональное дополнение и ортогональная проекция. \\
T2. 12. Связь между матрицами линейных преобразовании в разных базисах. \\
A1. Доказать, что векторы \(\overrightarrow{a} = (1;\ \ 2;\ \ 1),\) \(\overrightarrow{b} = (1;\ \ 1;\ \ 3)\) и \(\overrightarrow{c} = ( - 1;\ \ 2;\ \ 1)\) образуют базис пространства \(\mathbf{R}^{3},\) и найти координаты вектора \(\overrightarrow{d} = (0;\ \ 10;\ \  - 2)\) в этом базисе. \\
A2. Известно, что оператор \emph{A} переводит базисные векторы \(\overrightarrow{i} = (1;\ \ 0;\ \ 0),\) \(\overrightarrow{j} = (0;\ \ 1;\ \ 0),\) \(\overrightarrow{k} = (0;\ \ 0;\ \ 1)\) линейного пространства \(\mathbf{R}^{3}\) в векторы \({\overline{a}}_{1} = (2;\ \ 1;1),\) \({\overline{a}}_{2} = (3;\ \ 2;\ \ 1),\) \({\overline{a}}_{3} = (3;\ \ 1;\ \ 1).\) В базисе \(\overrightarrow{i},\overrightarrow{j},\overrightarrow{k}\) найти: 1)матрицу оператора \(A\) ; 2)образ вектора \(\overline{b} = (1;\ \  - 2;\ \ 3).\) \\
A3. В пространстве \(\mathbb{C}^{3}\) со скалярным произведением \(\left\langle x,y \right\rangle = \sum_{k = 1}^{3}{x_{k}\overline{y_{k}}}\), найдите сопряженный оператор \(A^{*}\) для заданного оператора \(A\). Является ли \(A\)самосопряженным? \(Ax = \left( x_{1} + ix_{3},x_{3} + 2ix_{2},ix_{2} - 2ix_{3} \right)\); \\
B1. \(a_{1} = (2;4;3)\), \(a_{2} = (3; - 1;4)\), \(a_{3} = (1;5; - 1)\); \\
B2. Найти собственные значения и собственные векторы оператора \emph{А}, заданного в некотором базисе пространства \(V^{3}\) матрицей \(A = \begin{bmatrix}
2 & 1 & 0 \\
1 & 2 & 0 \\
0 & 0 & - 5
\end{bmatrix};\) \\
B3. Приведите квадратичные формы \(G_{1}\) и \(G_{2}\) к каноническому виду. \(G_{1} = 3x_{1}^{2} - 2x_{2}^{2} + 2x_{1}x_{3} - 4x_{2}x_{3}\), \(G_{2} = x_{1}x_{2} + x_{2}x_{3}\) \\
C1. В базисе \((e_{1},\ \ e_{2},\ \ e_{3})\) пространства \(V^{3}\) оператор \emph{A} имеет матрицу \(A = \begin{bmatrix}
4 & 0 & 1 \\
 - 2 & - 2 & 3 \\
0 & 2 & - 1
\end{bmatrix}\ \ .\) Найти матрицу \emph{B} этого же оператора в базисе \(({e'}_{1},\ \ {e'}_{2},\ \ {e'}_{3}),\) где \({e'}_{1} = e_{1} + 2e_{2},\) \({e'}_{2} = e_{1} - e_{3},\) \({e'}_{3} = e_{1} + e_{2} + e_{3}.\)
 \\
C2. Даны векторы \(e_{1},e_{2},e_{3}\), \(a_{1},a_{2},a_{3}\) линейного пространства \(R^{3}\). Найдите матрицу перехода от базиса \(e_{1},e_{2},e_{3}\) к базису \(a_{1},a_{2},a_{3}\).
\(e_{1} = (0,1, - 2)\),\(e_{2} = ( - 2,0,3)\),\(e_{3} = (1, - 1,1)\) и \(a_{1} = (3,1, - 1)\),\(a_{2} = ( - 2,0,1)\),\(a_{3} = (2,7,3)\) \\
C3. Найти жорданову нормальную форму матрицы \(A = \begin{pmatrix}
2 & 1 & 1 \\
1 & 2 & 1 \\
1 & 1 & 2
\end{pmatrix}\). \\

\end{tabular}
\vspace{1cm}


\begin{tabular}{m{17cm}}
\textbf{94-вариант}
\newline

T1. 15. Самосопряженные преобразования и их канонический вид. \\
T2. 14. Сопряженное преобразование для данного преобразования. \\
A1. Доказать, что векторы \(\overrightarrow{a} = (3;\ \ 1;\ \ 2),\) \(\overrightarrow{b} = (2;\ \  - 3;\ \ 1)\) и \(\overrightarrow{c} = (4;\ \  - 2;\ \ 3)\) образуют базис пространства \(\mathbf{R}^{3},\) и найти координаты вектора \(\overrightarrow{d} = ( - 7;\ \ 8;\ \ 7)\) в этом базисе. \\
A2. 
Известно, что оператор \emph{A} переводит базисные векторы \(\overrightarrow{i} = (1;\ \ 0;\ \ 0),\) \(\overrightarrow{j} = (0;\ \ 1;\ \ 0),\) \(\overrightarrow{k} = (0;\ \ 0;\ \ 1)\) линейного пространства \(\mathbf{R}^{3}\) в векторы \({\overline{a}}_{1} = (1;1;0),\) \({\overline{a}}_{2} = (3;\ \ 2;\ \ 1),\) \({\overline{a}}_{3} = (0;\ \ 1;\ \ 1).\) В базисе \(\overrightarrow{i},\overrightarrow{j},\overrightarrow{k}\) найти: 1)матрицу оператора \(A\) ; 2)образ вектора \(\overline{b} = (1;\ \  - 2;\ \  - 3).\) \\
A3. 
В пространстве \(\mathbb{C}^{3}\) со скалярным произведением \(\left\langle x,y \right\rangle = \sum_{k = 1}^{3}{x_{k}\overline{y_{k}}}\), найдите сопряженный оператор \(A^{*}\) для заданного оператора \(A\). Является ли \(A\)самосопряженным? \(Ax = \left( ix_{1} + x_{3},x_{2} + ix_{1},x_{1} + ix_{3} \right)\); \\
B1. 
С помощью процесса ортогонализации Грамма- Шмидта ортонормировать следующие системы векторов, используя стандартное скалярное произведение: \\
B2. Найти собственные значения и собственные векторы оператора \emph{А}, заданного в некотором базисе пространства \(V^{3}\) матрицей \(A = \begin{bmatrix}
 - 1 & 1 & 0 \\
 - 4 & 3 & 0 \\
 - 2 & 1 & 1
\end{bmatrix};\) \\
B3. Приведите квадратичные формы \(G_{1}\) и \(G_{2}\) к каноническому виду. \(G_{1} = 4x_{1}^{2} + x_{2}^{2} + x_{3}^{2} - 4x_{1}x_{2} + 4x_{1}x_{3} - 3x_{2}x_{3}\), \(G_{2} = x_{1}x_{2} + 6x_{1}x_{3} - 4x_{2}x_{3}\) \\
C1. В базисе \((e_{1},\ \ e_{2},\ \ e_{3})\) пространства \(V^{3}\) оператор \emph{A} имеет матрицу \(A = \begin{bmatrix}
0 & 1 & - 2 \\
3 & 5 & 1 \\
 - 1 & 2 & 0
\end{bmatrix}\ \ .\) Найти матрицу \emph{B} этого же оператора в базисе \(({e'}_{1},\ \ {e'}_{2},\ \ {e'}_{3}),\) где \({e'}_{1} = e_{1} + 2e_{2},\) \({e'}_{2} = e_{1} - e_{3},\) \({e'}_{3} = e_{1} + e_{2} + e_{3}.\) \\
C2. Даны векторы \(e_{1},e_{2},e_{3}\), \(a_{1},a_{2},a_{3}\) линейного пространства \(R^{3}\). Найдите матрицу перехода от базиса \(e_{1},e_{2},e_{3}\) к базису \(a_{1},a_{2},a_{3}\).
\(e_{1} = (1,0,2)\),\(e_{2} = (3, - 1,4)\),\(e_{3} = (2, - 2,1)\) и \(a_{1} = (4,0,5)\),\(a_{2} = ( - 2,1,3)\),\(a_{3} = ( - 5,1, - 1)\) \\
C3. Найти жорданову нормальную форму матрицы \(A = \begin{pmatrix}
0 & 1 & 0 \\
 - 4 & 4 & 0 \\
0 & 0 & 2
\end{pmatrix}\) \\

\end{tabular}
\vspace{1cm}


\begin{tabular}{m{17cm}}
\textbf{95-вариант}
\newline

T1. 7. Комплексные евклидовы пространства. \\
T2. 10. Ядро, образ линйеного преобразования. \\
A1. Доказать, что векторы \(\overrightarrow{a} = (3;\ \ 4;\ \ 3),\) \(\overrightarrow{b} = ( - 2;\ \ 3;\ \ 1)\) и \(\overrightarrow{c} = (4;\ \  - 2;\ \ 3)\) образуют базис пространства \(\mathbf{R}^{3},\) и найти координаты вектора \(\overrightarrow{d} = ( - 17;\ \ 18;\ \  - 7)\) в этом базисе. \\
A2. Известно, что оператор \emph{A} переводит базисные векторы \(\overrightarrow{i} = (1;\ \ 0;\ \ 0),\) \(\overrightarrow{j} = (0;\ \ 1;\ \ 0),\) \(\overrightarrow{k} = (0;\ \ 0;\ \ 1)\) линейного пространства \(\mathbf{R}^{3}\) в векторы \({\overline{a}}_{1} = (1;\ \ 0;\ \ 1),\) \({\overline{a}}_{2} = (3;\ \ 2;\ \ 1),\) \({\overline{a}}_{3} = (3;\ \ 1;\ \ 1).\) В базисе \(\overrightarrow{i},\overrightarrow{j},\overrightarrow{k}\) найти: 1)матрицу оператора \(A\) ; 2)образ вектора \(\overline{b} = (1;\ \  - 2;\ \ 3).\) \\
A3. В пространстве \(\mathbb{C}^{3}\) со скалярным произведением \(\left\langle x,y \right\rangle = \sum_{k = 1}^{3}{x_{k}\overline{y_{k}}}\), найдите сопряженный оператор \(A^{*}\) для заданного оператора \(A\). Является ли \(A\)самосопряженным? \(Ax = \left( x_{2} + ix_{3},x_{1} - ix_{2},x_{1} + ix_{2} + x_{3} \right)\) \\
B1. С помощью процесса ортогонализации Грамма- Шмидта ортонормировать следующие системы векторов, используя стандартное скалярное произведение: \\
B2. Найти собственные значения и собственные векторы оператора \emph{А}, заданного в некотором базисе пространства \(V^{3}\) матрицей \(A = \begin{bmatrix}
0 & - 1 & 1 \\
 - 1 & 0 & 1 \\
1 & 1 & 0
\end{bmatrix};\) \\
B3. 
Приведите квадратичные формы \(G_{1}\) и \(G_{2}\) к каноническому виду. \(G_{1} = x_{1}^{2} + x_{2}^{2} + 3x_{3}^{2} + 4x_{1}x_{2} + 2x_{1}x_{3} + 2x_{2}x_{3}\), \(G_{2} = x_{1}x_{2} + x_{1}x_{3} + x_{2}x_{3}\) \\
C1. 
В базисе \((e_{1},\ \ e_{2},\ \ e_{3})\) пространства \(V^{3}\) оператор \emph{A} имеет матрицу \(A = \begin{bmatrix}
1 & 2 & 3 \\
0 & 1 & 2 \\
3 & 1 & 2
\end{bmatrix}\ \ .\) Найти матрицу \emph{B} этого же оператора в базисе \(({e'}_{1},\ \ {e'}_{2},\ \ {e'}_{3}),\) где \({e'}_{1} = e_{1} + 2e_{2},\) \({e'}_{2} = e_{1} - e_{3},\) \({e'}_{3} = e_{1} + e_{2} + e_{3}.\) \\
C2. Даны векторы \(e_{1},e_{2},e_{3}\), \(a_{1},a_{2},a_{3}\) линейного пространства \(R^{3}\). Найдите матрицу перехода от базиса \(e_{1},e_{2},e_{3}\) к базису \(a_{1},a_{2},a_{3}\).
\(e_{1} = (3,1, - 1)\),\(e_{2} = ( - 2,0,1)\),\(e_{3} = (2,7,3)\) и \(a_{1} = (2,1, - 3)\),\(a_{2} = (3,2, - 5)\),\(a_{3} = (1, - 1,1)\) \\
C3. Найти жорданову нормальную форму матрицы \(A = \begin{pmatrix}
0 & 3 & 1 \\
3 & 0 & 1 \\
 - 2 & 2 & 1
\end{pmatrix}\) \\

\end{tabular}
\vspace{1cm}


\begin{tabular}{m{17cm}}
\textbf{96-вариант}
\newline

T1. 17. Взаимозаменяемые преобразования. \\
T2. 2. Евклидово пространство. Неравенство Коши-Буняковского. Процесс ортогонализации. \\
A1. Доказать, что векторы \(\overrightarrow{a} = (1;\ \ 2;\ \ 1),\) \(\overrightarrow{b} = (1;\ \ 1;\ \ 3)\) и \(\overrightarrow{c} = ( - 1;\ \ 2;\ \ 1)\) образуют базис пространства \(\mathbf{R}^{3},\) и найти координаты вектора \(\overrightarrow{d} = (0;\ \ 10;\ \  - 2)\) в этом базисе. \\
A2. Известно, что оператор \emph{A} переводит базисные векторы \(\overrightarrow{i} = (1;\ \ 0;\ \ 0),\) \(\overrightarrow{j} = (0;\ \ 1;\ \ 0),\) \(\overrightarrow{k} = (0;\ \ 0;\ \ 1)\) линейного пространства \(\mathbf{R}^{3}\) в векторы \({\overline{a}}_{1} = (1;\ \ 1;\ \ 1),\) \({\overline{a}}_{2} = (3;\ \ 2;\ \ 1),\) \({\overline{a}}_{3} = (0;\ \ 1;\ \ 1).\) В базисе \(\overrightarrow{i},\overrightarrow{j},\overrightarrow{k}\) найти: 1)матрицу оператора \(A\) ; 2)образ вектора \(\overline{b} = (1;\ \  - 2;\ \ 3).\)
 \\
A3. В пространстве \(\mathbb{C}^{3}\) со скалярным произведением \(\left\langle x,y \right\rangle = \sum_{k = 1}^{3}{x_{k}\overline{y_{k}}}\), найдите сопряженный оператор \(A^{*}\) для заданного оператора \(A\). Является ли \(A\)самосопряженным?\(Ax = \left( x_{1} + 2ix_{2},x_{3} - ix_{2},x_{1} - ix_{2} - 2ix_{3} \right)\); \\
B1. \(a_{1} = ( - 3;0;1)\), \(a_{2} = (0;2;3)\), \(a_{3} = ( - 1; - 1; - 1)\); \\
B2. Найти собственные значения и собственные векторы оператора \emph{А}, заданного в некотором базисе пространства \(V^{3}\) матрицей \(A = \begin{pmatrix}
2 & 1 & 0 \\
1 & 3 & - 1 \\
 - 1 & 2 & 3
\end{pmatrix}\). \\
B3. Приведите квадратичные формы \(G_{1}\) и \(G_{2}\) к каноническому виду. \(G_{1} = 2x_{1}^{2} + 6x_{2}^{2} - 4x_{3}^{2} - 2x_{1}x_{3} + 4x_{1}x_{2} - 8x_{2}x_{3}\), \(G_{2} = x_{2}x_{3} - 2x_{1}x_{3}\) \\
C1. В базисе \((e_{1},\ \ e_{2},\ \ e_{3})\) пространства \(V^{3}\) оператор \emph{A} имеет матрицу \(A = \begin{bmatrix}
2 & 0 & - 1 \\
3 & 2 & 0 \\
 - 1 & 4 & 3
\end{bmatrix}\ \ .\) Найти матрицу \emph{B} этого же оператора в базисе \(({e'}_{1},\ \ {e'}_{2},\ \ {e'}_{3}),\) где \({e'}_{1} = e_{1} - e_{3},\) \({e'}_{2} = e_{2} + e_{3},\) \({e'}_{3} = e_{3}.\) \\
C2. 11.Даны векторы \(e_{1},e_{2},e_{3}\), \(a_{1},a_{2},a_{3}\) линейного пространства \(R^{3}\). Найдите матрицу перехода от базиса \(e_{1},e_{2},e_{3}\) к базису \(a_{1},a_{2},a_{3}\).
\(e_{1} = ( - 1,3,7)\),\(e_{2} = (0,2, - 1)\),\(e_{3} = (1, - 2, - 8)\) и \(a_{1} = (0,3, - 2)\),\(a_{2} = (1, - 1, - 8)\),\(a_{3} = ( - 1,2,7)\) \\
C3. 
Найти жорданову нормальную форму матрицы \(A = \begin{pmatrix}
 - 1 & 1 & - 2 \\
3 & - 3 & 6 \\
2 & - 2 & 4
\end{pmatrix}\). \\

\end{tabular}
\vspace{1cm}


\begin{tabular}{m{17cm}}
\textbf{97-вариант}
\newline

T1. 9. Линейные преобразования и их матрица. \\
T2. 4. Линейные, билинейные, и квадратичные формы. Преобразование матрицы линейного вида при изменении базиса. \\
A1. Доказать, что векторы \(\overrightarrow{a} = (3;\ \ 1;\ \ 0),\) \(\overrightarrow{b} = (4;\ \ 3;\ \ 2)\) и \(\overrightarrow{c} = ( - 1;\ \  - 4;\ \ 3)\) образуют базис пространства \(\mathbf{R}^{3},\) и найти координаты вектора \(\overrightarrow{d} = ( - 1;\ \ 2;\ \ 5)\) в этом базисе.
 \\
A2. 
Известно, что оператор \emph{A} переводит базисные векторы \(\overrightarrow{i} = (1;\ \ 0;\ \ 0),\) \(\overrightarrow{j} = (0;\ \ 1;\ \ 0),\) \(\overrightarrow{k} = (0;\ \ 0;\ \ 1)\) линейного пространства \(\mathbf{R}^{3}\) в векторы \({\overline{a}}_{1} = (1;1;0),\) \({\overline{a}}_{2} = (3;\ \ 2;\ \ 1),\) \({\overline{a}}_{3} = (0;\ \ 1;\ \ 1).\) В базисе \(\overrightarrow{i},\overrightarrow{j},\overrightarrow{k}\) найти: 1)матрицу оператора \(A\) ; 2)образ вектора \(\overline{b} = (1;\ \  - 2;\ \  - 3).\) \\
A3. В пространстве \(\mathbb{C}^{3}\) со скалярным произведением \(\left\langle x,y \right\rangle = \sum_{k = 1}^{3}{x_{k}\overline{y_{k}}}\), найдите сопряженный оператор \(A^{*}\) для заданного оператора \(A\). Является ли \(A\)самосопряженным? \(Ax = \left( ix_{1} + x_{3},ix_{3} - ix_{2},x_{1} - ix_{3} \right)\); \\
B1. С помощью процесса ортогонализации Грамма- Шмидта ортонормировать следующие системы векторов, используя стандартное скалярное произведение: \\
B2. Найти собственные значения и собственные векторы оператора \emph{А}, заданного в некотором базисе пространства \(V^{3}\) матрицей \(A = \begin{pmatrix}
0 & 1 & 2 \\
 - 1 & 0 & - 2 \\
 - 2 & 2 & 0
\end{pmatrix}\).
 \\
B3. Приведите квадратичные формы \(G_{1}\) и \(G_{2}\) к каноническому виду. \(G_{1} = 2x_{1}^{2} + 6x_{2}^{2} - 4x_{3}^{2} - 2x_{1}x_{3} + 4x_{1}x_{2} - 8x_{2}x_{3}\), \(G_{2} = x_{2}x_{3} - 2x_{1}x_{3}\) \\
C1. В базисе \((e_{1},\ \ e_{2},\ \ e_{3})\) пространства \(V^{3}\) оператор \emph{A} имеет матрицу \(A = \begin{bmatrix}
 - 3 & 1 & 4 \\
0 & 3 & 2 \\
 - 5 & - 1 & 2
\end{bmatrix}\ \ .\) Найти матрицу \emph{B} этого же оператора в базисе \(({e'}_{1},\ \ {e'}_{2},\ \ {e'}_{3}),\) где \({e'}_{1} = 2e_{1} + e_{2}\), \({e'}_{2} = - e_{1} + 2e_{2} + 3e_{3}\),\({e'}_{3} = - e_{1} + e_{2} + e_{3}\) \\
C2. 
Даны векторы \(e_{1},e_{2},e_{3}\), \(a_{1},a_{2},a_{3}\) линейного пространства \(R^{3}\). Найдите матрицу перехода от базиса \(e_{1},e_{2},e_{3}\) к базису \(a_{1},a_{2},a_{3}\).
\(e_{1} = (2,1, - 3)\),\(e_{2} = (3,2, - 5)\),\(e_{3} = (1, - 1,1)\) и \(a_{1} = (0,1, - 2)\),\(a_{2} = ( - 2,0,3)\),\(a_{3} = (1, - 1,1)\) \\
C3. Найти жорданову нормальную форму матрицы \(A = \begin{pmatrix}
 - 1 & 4 & 3 \\
 - 2 & 5 & 3 \\
2 & - 4 & - 2
\end{pmatrix}\). \\

\end{tabular}
\vspace{1cm}


\begin{tabular}{m{17cm}}
\textbf{98-вариант}
\newline

T1. 19. Полиномиальные матрицы и диагональные нормальные формы. \\
T2. 8. Квадратичные формы в комплексном пространстве и их канонические виды. \\
A1. Доказать, что векторы \(\overrightarrow{a} = (1;\ \ 0;\ \ 1),\) \(\overrightarrow{b} = (1;\ \ 1;\ \ 1)\) и \(\overrightarrow{c} = ( - 1;\ \ 2;\ \ 1)\) образуют базис пространства \(\mathbf{R}^{3},\) и найти координаты вектора \(\overrightarrow{d} = (0;\ \ 10;\ \ 3)\) в этом базисе. \\
A2. Известно, что оператор \emph{A} переводит базисные векторы \(\overrightarrow{i} = (1;\ \ 0;\ \ 0),\) \(\overrightarrow{j} = (0;\ \ 1;\ \ 0),\) \(\overrightarrow{k} = (0;\ \ 0;\ \ 1)\) линейного пространства \(\mathbf{R}^{3}\) в векторы \({\overline{a}}_{1} = (1;\ \ 0;\ \ 1),\) \({\overline{a}}_{2} = (3;\ \ 2;\ \ 1),\) \({\overline{a}}_{3} = (3;\ \ 1;\ \ 1).\) В базисе \(\overrightarrow{i},\overrightarrow{j},\overrightarrow{k}\) найти: 1)матрицу оператора \(A\) ; 2)образ вектора \(\overline{b} = (1;\ \  - 2;\ \ 3).\) \\
A3. 
В пространстве \(\mathbb{C}^{3}\) со скалярным произведением \(\left\langle x,y \right\rangle = \sum_{k = 1}^{3}{x_{k}\overline{y_{k}}}\), найдите сопряженный оператор \(A^{*}\) для заданного оператора \(A\). Является ли \(A\)самосопряженным? \(Ax = \left( ix_{1} + x_{3},x_{2} + ix_{1},x_{1} + ix_{3} \right)\); \\
B1. \(a_{1} = (2;0;1)\), \(a_{2} = ( - 1;2;3)\), \(a_{3} = ( - 1;1;1)\); \\
B2. Найти собственные значения и собственные векторы оператора \emph{А}, заданного в некотором базисе пространства \(V^{3}\) матрицей \(A = \begin{bmatrix}
 - 1 & 1 & 0 \\
 - 4 & 3 & 0 \\
 - 2 & 1 & 1
\end{bmatrix};\) \\
B3. Приведите квадратичные формы \(G_{1}\) и \(G_{2}\) к каноническому виду. \(G_{1} = 3x_{1}^{2} - 2x_{2}^{2} + 2x_{1}x_{3} - 4x_{2}x_{3}\), \(G_{2} = x_{1}x_{2} + x_{2}x_{3}\) \\
C1. 
В базисе \((e_{1},\ \ e_{2},\ \ e_{3})\) пространства \(V^{3}\) оператор \emph{A} имеет матрицу \(A = \begin{bmatrix}
1 & 2 & 3 \\
0 & 1 & 2 \\
3 & 1 & 2
\end{bmatrix}\ \ .\) Найти матрицу \emph{B} этого же оператора в базисе \(({e'}_{1},\ \ {e'}_{2},\ \ {e'}_{3}),\) где \({e'}_{1} = e_{1} + 2e_{2},\) \({e'}_{2} = e_{1} - e_{3},\) \({e'}_{3} = e_{1} + e_{2} + e_{3}.\) \\
C2. Даны векторы \(e_{1},e_{2},e_{3}\), \(a_{1},a_{2},a_{3}\) линейного пространства \(R^{3}\). Найдите матрицу перехода от базиса \(e_{1},e_{2},e_{3}\) к базису \(a_{1},a_{2},a_{3}\).
\(e_{1} = (2,0,1)\),\(e_{2} = ( - 1,2,3)\),\(e_{3} = ( - 1,1,1)\) и \(a_{1} = (1,0,2)\),\(a_{2} = (3, - 1,4)\),\(a_{3} = (2, - 2,1)\) \\
C3. Найти жорданову нормальную форму матрицы \(A = \begin{pmatrix}
 - 1 & 4 & 3 \\
 - 2 & 5 & 3 \\
2 & - 4 & - 2
\end{pmatrix}\). \\

\end{tabular}
\vspace{1cm}


\begin{tabular}{m{17cm}}
\textbf{99-вариант}
\newline

T1. 5. Методы приведения квадратичной формы к каноническому форму. \\
T2. 18. Нормальные преобразования и их канонический вид. \\
A1. Доказать, что векторы \(\overrightarrow{a} = (2;\ \ 1;1),\) \(\overrightarrow{b} = ( - 1;\ \ 2;\ \ 4)\) и \(\overrightarrow{c} = (3;\ \ 3;\ \ 2)\) образуют базис пространства \(\mathbf{R}^{3},\) и найти координаты вектора \(\overrightarrow{d} = ( - 4;\ \ 2;\ \ 4)\) в этом базисе. \\
A2. Известно, что оператор \emph{A} переводит базисные векторы \(\overrightarrow{i} = (1;\ \ 0;\ \ 0),\) \(\overrightarrow{j} = (0;\ \ 1;\ \ 0),\) \(\overrightarrow{k} = (0;\ \ 0;\ \ 1)\) линейного пространства \(\mathbf{R}^{3}\) в векторы \({\overline{a}}_{1} = (1;\ \ 0;\ \ 1),\) \({\overline{a}}_{2} = (0;\ \ 1;\ \ 1),\) \({\overline{a}}_{3} = (3;\ \ 1;\ \ 1).\) В базисе \(\overrightarrow{i},\overrightarrow{j},\overrightarrow{k}\) найти: 1)матрицу оператора \(A\) ; 2)образ вектора \(\overline{b} = (1;\ \  - 2;\ \  - 3).\) \\
A3. В пространстве \(\mathbb{C}^{3}\) со скалярным произведением \(\left\langle x,y \right\rangle = \sum_{k = 1}^{3}{x_{k}\overline{y_{k}}}\), найдите сопряженный оператор \(A^{*}\) для заданного оператора \(A\). Является ли \(A\)самосопряженным?\(Ax = \left( x_{1} - 2ix_{2},x_{3} + 2ix_{2},ix_{2} + 2ix_{3} \right)\);
 \\
B1. \(a_{1} = (1;0;2)\), \(a_{2} = (3; - 1;4)\), \(a_{3} = (2; - 2;1)\); \\
B2. Найти собственные значения и собственные векторы оператора \emph{А}, заданного в некотором базисе пространства \(V^{3}\) матрицей \(A = \begin{pmatrix}
1 & - 2 & - 1 \\
 - 1 & 1 & 1 \\
1 & 0 & - 1
\end{pmatrix}\) \\
B3. Приведите квадратичные формы \(G_{1}\) и \(G_{2}\) к каноническому виду. \(G_{1} = 2x_{1}^{2} + x_{2}^{2} + x_{3}^{2} + 4x_{1}x_{2} - 2x_{1}x_{3}\), \(G_{2} = x_{1}x_{2} + x_{1}x_{3} + 4x_{2}x_{3}\) \\
C1. В базисе \((e_{1},\ \ e_{2},\ \ e_{3})\) пространства \(V^{3}\) оператор \emph{A} имеет матрицу \(A = \begin{bmatrix}
1 & - 1 & 2 \\
0 & 3 & - 1 \\
4 & 2 & 2
\end{bmatrix}\ \ .\) Найти матрицу \emph{B} этого же оператора в базисе \(({e'}_{1},\ \ {e'}_{2},\ \ {e'}_{3}),\) где \({e'}_{1} = e_{1} + 2e_{2},\) \({e'}_{2} = e_{1} - e_{3},\) \({e'}_{3} = e_{1} + e_{2} + e_{3}.\) \\
C2. Даны векторы \(e_{1},e_{2},e_{3}\), \(a_{1},a_{2},a_{3}\) линейного пространства \(R^{3}\). Найдите матрицу перехода от базиса \(e_{1},e_{2},e_{3}\) к базису \(a_{1},a_{2},a_{3}\).
\(e_{1} = (0,1, - 2)\),\(e_{2} = ( - 2,0,3)\),\(e_{3} = (1, - 1,1)\) и \(a_{1} = (3,1, - 1)\),\(a_{2} = ( - 2,0,1)\),\(a_{3} = (2,7,3)\) \\
C3. Найти жорданову нормальную форму матрицы \(A = \begin{pmatrix}
3 & - 2 & 6 \\
 - 2 & 6 & 3 \\
6 & 3 & - 2
\end{pmatrix}\). \\

\end{tabular}
\vspace{1cm}


\begin{tabular}{m{17cm}}
\textbf{100-вариант}
\newline

T1. 1. Линейные пространства. Линейные подпространства. Сумма и пересечение подпространств. \\
T2. 16. Унитарные преобразования и их собственные значения и канонический вид. \\
A1. Доказать, что векторы \(\overrightarrow{a} = (3;\ \ 1;\ \ 2),\) \(\overrightarrow{b} = (2;\ \  - 3;\ \ 1)\) и \(\overrightarrow{c} = (4;\ \  - 2;\ \ 3)\) образуют базис пространства \(\mathbf{R}^{3},\) и найти координаты вектора \(\overrightarrow{d} = ( - 7;\ \ 8;\ \ 7)\) в этом базисе. \\
A2. Известно, что оператор \emph{A} переводит базисные векторы \(\overrightarrow{i} = (1;\ \ 0;\ \ 0),\) \(\overrightarrow{j} = (0;\ \ 1;\ \ 0),\) \(\overrightarrow{k} = (0;\ \ 0;\ \ 1)\) линейного пространства \(\mathbf{R}^{3}\) в векторы \({\overline{a}}_{1} = (1;\ \ 0;\ \ 1),\) \({\overline{a}}_{2} = (0;\ \ 2;\ \ 1),\) \({\overline{a}}_{3} = (3;\ \ 1;\ \ 1).\) В базисе \(\overrightarrow{i},\overrightarrow{j},\overrightarrow{k}\) найти: 1)матрицу оператора \(A\) ; 2)образ вектора \(\overline{b} = (1;\ \ 2;\ \ 3).\) \\
A3. В пространстве \(\mathbb{C}^{3}\) со скалярным произведением \(\left\langle x,y \right\rangle = \sum_{k = 1}^{3}{x_{k}\overline{y_{k}}}\), найдите сопряженный оператор \(A^{*}\) для заданного оператора \(A\). Является ли \(A\)самосопряженным? \(Ax = \left( x_{1} + 2ix_{3},ix_{2} - x_{3},x_{2} - ix_{3} \right)\); \\
B1. \(a_{1} = (0;1; - 2)\),\(a_{2} = (1; - 1;1)\), \(a_{3} = ( - 2;0;3)\); \\
B2. Найти собственные значения и собственные векторы оператора \emph{А}, заданного в некотором базисе пространства \(V^{3}\) матрицей \(A = \begin{pmatrix}
2 & - 1 & 2 \\
1 & 0 & 2 \\
 - 2 & 1 & - 1
\end{pmatrix}\) \\
B3. Приведите квадратичные формы \(G_{1}\) и \(G_{2}\) к каноническому виду. \(G_{1} = x_{1}^{2} - 2x_{3}^{2} + 2x_{1}x_{3} - 6x_{1}x_{2}\), \(G_{2} = 6x_{2}x_{3} - 4x_{1}x_{2} + x_{1}x_{3}\)
 \\
C1. В базисе \((e_{1},\ \ e_{2},\ \ e_{3})\) пространства \(V^{3}\) оператор \emph{A} имеет матрицу \(A = \begin{bmatrix}
 - 1 & 2 & 1 \\
0 & 1 & - 4 \\
5 & - 1 & 2
\end{bmatrix}\ \ .\) Найти матрицу \emph{B} этого же оператора в базисе \(({e'}_{1},\ \ {e'}_{2},\ \ {e'}_{3}),\) где \({e'}_{1} = e_{1} - e_{3},\) \({e'}_{2} = e_{2} + e_{3},\) \({e'}_{3} = e_{3}.\) \\
C2. 12.Даны векторы \(e_{1},e_{2},e_{3}\), \(a_{1},a_{2},a_{3}\) линейного пространства \(R^{3}\). Найдите матрицу перехода от базиса \(e_{1},e_{2},e_{3}\) к базису \(a_{1},a_{2},a_{3}\).
\(e_{1} = (1,1,1)\),\(e_{2} = (1,1,2)\),\(e_{3} = (1,2,3)\) и \(a_{1} = (2,0,1)\),\(a_{2} = ( - 1,2,3)\),\(a_{3} = ( - 1,1,1)\)
 \\
C3. Найти жорданову нормальную форму матрицы \(A = \begin{pmatrix}
0 & 3 & 1 \\
3 & 0 & 1 \\
 - 2 & 2 & 1
\end{pmatrix}\) \\

\end{tabular}
\vspace{1cm}



\end{document}
