\documentclass{article}
\usepackage[fontsize=11pt]{fontsize}
\usepackage[utf8]{inputenc}
\usepackage[T2A]{fontenc}
% \usepackage{unicode-math}

\usepackage{array}
\usepackage[a4paper,
left=7mm,
right=5mm,
top=7mm,]{geometry}
\usepackage{amsmath}
% \usepackage{amssymbol}
\usepackage{amsfonts}
\usepackage{setspace}



\renewcommand{\baselinestretch}{1} 

\everymath{\displaystyle}
\everydisplay{\displaystyle}
% \linespread{1.25}

\DeclareMathOperator{\sign}{sign}


\begin{document}

\pagenumbering{gobble}


\begin{tabular}{m{17cm}}
\textbf{1-variant}
\newline

T1. Qálegen \(a\) parametri hám \(x\) ushın \(x(a - x) \leq a^{2}/4\) teńsizligi orınlı bolıwın dálilleń. \\
T2. Bezu teoreması hám onıń qollanılıwı. \\
A1. Teńsizlikti sheshiń: \(\frac{x^{3} + 3x^{2} - x - 3}{x^{2} + 3x - 10} < 0\). \\
A2. Teńlemeni sheshiń. \(\sqrt{x^{2} + x + 4} + \sqrt{x^{2} + x + 1} = \sqrt{2x^{2} + 2x + 9}\). \\
A3. Teńlemeni sheshiń. \((x - 4)^{3} + (x - 4)^{2} + (x - 4)(x - 3) + (x - 3)^{2} + (x - 3)^{3} = 6\). \\
B1. \(P(x + 3)\) kópaǵzalını \(x + 1\) ga bólgende qaldıq -3, \(Q(2x - 1)\) kópaǵzalını \(x - 1\)ga bólgende qaldıq 2 bolsa, \(P(x + 4) + x^{2}Q(x + 3)\) kópaǵzalını \(x + 2\) ga bólgendegi qaldıqni tabıń. \\
B2. Tómendegi aytımdı qálegen natural san ushın matematikalıq induksiya metodi járdeminde dálilleń: \(1^{3} + 2^{3} + 3^{3} + ... + n^{3} = \left( \frac{n(n + 1)}{2} \right)^{2}\); \\
B3. Tómendegi aytımdı qálegen natural san ushın matematikalıq induksiya metodi járdeminde dálilleń: \(5^{n} - 4n + 15\) sanı 16 ga eseli ; \\
C1. Teńsizlikti sheshiń: \(C_{10}^{x - 1} > 2C_{10}^{x}\) \\
C2. Teń qaptallı úshmúyeshlik ultanidagi múyesh \(\alpha\) ga teń. Shu múyesh uchidan ultanga \(\beta(\beta < \alpha)\) múyesh ostida tuwrı sızıq túsirilgen, u úshmúyeshlikni eki bólekga ajıratadı. Payda bolǵan úshmúyeshliklar maydanlarınıń qatnasini tabıń. \\
C3. Eger \(S\) úshmúyeshliktiń maydanı, \(b\) hám \(c\) onıń táreplari bolsa, \(S \leq \frac{b^{2} + c^{2}}{4}\) bolıwın dálilleń. \\

\end{tabular}
\vspace{1cm}


\begin{tabular}{m{17cm}}
\textbf{2-variant}
\newline

T1. Haqıyqıy \(a_{1},\ a_{2},\ .\ .\ .\ ,\ a_{n},\ b_{1},\ b_{2},\ .\ .\ .\ ,\ b_{n}\) sanları ushın \(\left( a_{1}b_{1} + a_{2}b_{2} + \ .\ .\ .\  + a_{n}b_{n} \right)^{2} \leq \left( a_{1}^{2} + a_{2}^{2} + \ .\ .\ .\  + a_{n}^{2} \right)\left( b_{1}^{2} + b_{2}^{2} + \ .\ .\ .\  + b_{n}^{2} \right)\) Koshi teńsizligin dálilleń. \\
T2. \(a\) parametriniń qanday mánisinde \(P(x) = x^{2017} + ax - 5\) kóp aǵzalısı \((x + 1)\) kóp aǵzalısına qaldıqsız bólinedi? \\
A1. Teńsizlikti sheshiń: \(\sqrt{x^{2} - 4x} > x - 3\). \\
A2. Teńlemeni sheshiń. \(\sqrt{x + 8 + 2\sqrt{x + 7}} + \sqrt{x + 1 - \sqrt{x + 7}} = 4\). \\
A3. Teńsizlikti sheshiń: \(\sqrt{x + 3} + \sqrt{x - 2} - \sqrt{2x + 4} > 0\). \\
B1. \(P(x) = x^{33} - 2ax^{21} + x^{8} + 8\) kópaǵzalıi berilgan. \(a\) nıń qaysi qiymati ushın \(P(x)\) kópaǵzalıi \(x + 1\) ga qaldıqsiz bóliadi? \\
B2. Tómendegi aytımdı qálegen natural san ushın matematikalıq induksiya metodi járdeminde dálilleń: \(\frac{1}{4 \cdot 5} + \frac{1}{5 \cdot 6} + \frac{1}{6 \cdot 7} + \ldots + \frac{1}{(n + 3) \cdot (n + 4)} = \frac{n}{4 \cdot (n + 4)}\). \\
B3. Tómendegi aytımdı qálegen natural san ushın matematikalıq induksiya metodi járdeminde dálilleń: \(2n^{3} + 3n^{2} + 7n\) sanı 6 ga eseli ; \\
C1. \((a + b)^{n}\) ańlatpa jayılmasinıń barcha koeffitsiyentlari yig`indisi 4096 ga teń bolsa, Onıń eń úlken koeffitsiyentin tabıń. \\
C2. Parallelogrammnıń táreplari \(a\) hám \(b\), ular arasındaǵı múyesh \(\alpha\). bolsa, paralllelogramm ichki múyeshlari bissektrisalari kesilisiwinen payda bolǵan tórtmúyeshlik maydanın tabıń. \\
C3. \(R\) radiusli dóńgelekga bitta umumiy uchga ega bolǵan durıs úshmúyeshlik hám kvadrat ishley sızılǵan. Olardıń kesilisken bóleginıń maydanıni tabıń. \\

\end{tabular}
\vspace{1cm}


\begin{tabular}{m{17cm}}
\textbf{3-variant}
\newline

T1. Qálegen \(a,b,c \in (0;1)\) sanları ushın \(a(1 - b) > 1/4,\ b(1 - c) > 1/4,\ c(1 - a) > 1/4\) teńsizlikleri bir waqıtta orınlı bola almaytuǵınlıǵin dálilleń. \\
T2. Fales teoreması hám onıń qollanılıwı. \\
A1. Teńlemeni sheshiń. \(\sqrt{x} + \frac{2x + 1}{x + 2} = 2\). \\
A2. Teńlemeni sheshiń. \(\sqrt[3]{x - 1} + \sqrt[3]{x - 2} - \sqrt{2x - 3} = 0\). \\
A3. Teńlemeni sheshiń \((x + 1)^{5} + (x - 1)^{5} = 32x\). \\
B1. \(P(x)\) kópaǵzalını \(3x^{2} - 4x + 1\) ga bólgenimizdeqaldıq \(6x - 11\) bolsa, \(P(x)\) kópaǵzalını \(3x - 1\)ga bólgende qaldıqni tabıń. \\
B2. Tómendegi aytımdı qálegen natural san ushın matematikalıq induksiya metodi járdeminde dálilleń: \(\frac{1}{1 \cdot 4} + \frac{1}{4 \cdot 7} + \frac{1}{7 \cdot 10} + \ldots + \frac{1}{(3n - 2) \cdot (3n + 1)} = \frac{n}{(3n + 1)}\). \\
B3. Tómendegi aytımdı qálegen natural san ushın matematikalıq induksiya metodi járdeminde dálilleń: \(n^{3} + (n + 1)^{3} + (n + 2)^{3}\) sanı 9 ga eseli ; \\
C1. \((x + 1)^{3} + (x + 1)^{4} + (x + 1)^{5} + ... + (x + 1)^{10}\) ańlatpada \(x^{3}\) aldıńda ǵı koeffitsiyentti tabıń \\
C2. \(\bigtriangleup ABC\) da \(\angle A\) múyesh \(\angle B\) dan eki marta úlken bólib, \(AC = b,AB = c\). \emph{BC} tárepnıń uzunligi tabılsın. \\
C3. Ya. Bernulli teńsizligi. Eger\(x \geq - 1\) bolsa, onda qálegen natural \(n\) sanı ushın \((1 + x)^{n} \geq 1 + nx\) teńsizlik orinli bolıwın dálilleń. \\

\end{tabular}
\vspace{1cm}


\begin{tabular}{m{17cm}}
\textbf{4-variant}
\newline

T1. Mına \(P(x) = x^{5} + 11x^{4} + 37x^{3} + 35x^{2} - 44x - 40\) kóp aǵzalısı \(Q(x) = x^{2} + 3x + 2\) kóp aǵzalısına qaldıqsız bólinedime? \\
T2. Matematikalıq induksiya metodı hám onıń qollanılıwına mısallar. \\
A1. Teńlemeni sheshiń \(\left( x^{2} - 4x + 6 \right)^{2} - 4\left( x^{2} - 4x + 6 \right) + 6 = x\). \\
A2. Teńlemeni sheshiń. \(\sqrt{3x^{2} - 2x + 15} + \sqrt{3x^{2} - 2x + 8} = 7\). \\
A3. Teńlemeni sheshiń. \(\sqrt[3]{x} + \sqrt[3]{x - 16} = \sqrt[3]{x - 8}\). \\
B1. \(P(x + 1) + P(x - 3) = 2x^{2} - 10x + 16\) bolsa, \(P(x)\) ni tabıń. \\
B2. Tómendegi aytımdı qálegen natural san ushın matematikalıq induksiya metodi járdeminde dálilleń: \(1^{2} + 3^{2} + 5^{2} + ... + (2n - 1)^{2} = \frac{n\left( 4n^{2} - 1 \right)}{3}\); \\
B3. Tómendegi aytımdı qálegen natural san ushın matematikalıq induksiya metodi járdeminde dálilleń: \(5 \cdot 2^{3n - 2} + 3^{3n - 1}\) sanı 19 ga eseli \\
C1. \(\left( 2x^{\ ^{2}} - \frac{b}{2x^{3}} \right)^{10}\) binom jayılmasinıń \(x\) qatnashmagan aǵzasın tabıń. \\
C2. Tuwrı múyeshli úshmúyeshlikda katetlarnıń qatnasi 3:2 kabi, biyiklik bolsa gipotenuzani shunday eki kesindiga ajıratadı, olardan birinıń uzunligi ekinshisinen 2 ga úlken. Gipotenuzanıń uzunligi tabılsın. \\
C3. Eger a,b - oń sanlar bolsa, tómendegi teńsizlikti dálilleń: \(\sqrt[3]{\frac{a}{b}} + \sqrt[3]{\frac{b}{a}} \leq \sqrt[3]{2(a + b)\left( \frac{1}{a} + \frac{1}{b} \right)}\) \\

\end{tabular}
\vspace{1cm}


\begin{tabular}{m{17cm}}
\textbf{5-variant}
\newline

T1. \(P(x) = (x - 1)^{20}\left( x^{2} + 25 \right)\) kóp aǵzalisınıń koefficentleri qosındısın tabıń. \\
T2. Simmetriyalıq kóp aǵzalılar. \\
A1. Teńlemeni sheshiń \(\left( x^{2} - 6x \right)^{2} - 2(x - 3)^{2} = 81\). \\
A2. Teńlemeni sheshiń. \(\frac{4x}{x^{2} + x + 3} + \frac{5x}{x^{2} - 5x + 3} = - \frac{3}{2}\). \\
A3. Teńsizlikti sheshiń:\(x^{2}\left( x^{4} + 36 \right) - 6\sqrt{3}\left( x^{4} + 4 \right) < 0\). \\
B1. \(P(x + 3) = x^{2} - x + n\) bolsa. \(P(x - 2)\) kópaǵzalını \(x - 3\) ga bólgende qaldıq \(10\) bolsa, \(n\) ni tabıń. \\
B2. Tómendegi aytımdı qálegen natural san ushın matematikalıq induksiya metodi járdeminde dálilleń: \(\left( 1 - \frac{1}{4} \right)\left( 1 - \frac{1}{9} \right)...\left( 1 - \frac{1}{n^{2}} \right) = \frac{n + 1}{2n}\), \(n \geq 2\) \\
B3. Tómendegi aytımdı qálegen natural san ushın matematikalıq induksiya metodi járdeminde dálilleń: \(5^{n + 2} + 26 \cdot 5^{n} + 8^{2n + 1}\) sanı 59 ga eseli; \\
C1. \(\left( x\sqrt{x} - \frac{1}{x^{4}} \right)^{n}\) binom jayılmasında 3-aǵza koeffitsiyenti 2-aǵza koeffitsiyentidan 44 ga úlken.Ozod hadini tabıń. \\
C2. Úshmúyeshliktiń táreplari \(a\) hám \(b\), bissektrisasi \(l_{c} = l\). \(l\) ni bilgan holda Onıń maydanıni tabıń. \\
C3. Durıs úshmúyeshliktiń tárepi a ga teń. Tárepini diametr deb esaplap dóńgelek jasalǵan. Úshmúyeshliktiń usı dóńgelekten sirtindaǵi bólegi maydanın tabıń. \\

\end{tabular}
\vspace{1cm}


\begin{tabular}{m{17cm}}
\textbf{6-variant}
\newline

T1. \(2^{81} + 1\) sanı 9 sanına qaldıqsız bóliniwin dálilleń. \\
T2. \(b\) parametriniń qanday mánisinde \(x^{3} + 17x^{2} + bx - 17 = 0\) teńlemesiniń korenleri pútin sanlardan turadı? \\
A1. Teńlemeni sheshiń \((x + 4)(x + 1) - 3\sqrt{x^{2} + 5x + 2} = 6\). \\
A2. Teńlemeni sheshiń. \(\frac{z}{z + 1} - 2\sqrt{\frac{z + 1}{2}} = 3\). \\
A3. Teńlemeni sheshiń. \((\sqrt{x + 1} + \sqrt{x})^{3} + (\sqrt{x + 1} + \sqrt{x})^{2} = 2\). \\
B1. \(P(x + 2) + P(x - 1) = - 2x^{2} - 2x + 7\) bolsa, \(P(x)\) ni \(x + 4\) ga bólgendegi qaldıqni tabıń. \\
B2. Tómendegi aytımdı qálegen natural san ushın matematikalıq induksiya metodi járdeminde dálilleń: \(1 \cdot 2 + 2 \cdot 3 + 3 \cdot 4 + \ldots + n \cdot (n + 1) = \frac{n \cdot (n + 1) \cdot (n + 2)}{3}\). \\
B3. Tómendegi aytımdı qálegen natural san ushın matematikalıq induksiya metodi járdeminde dálilleń: \(5^{2n + 1} + 3^{n + 2} \cdot 2^{n - 1}\) sanı 19 ga eseli ; \\
C1. Birdeylikti dálilleń:\(\sum_{j = 0}^{n}C_{n}^{j}( - 1)^{j} = 0\); \\
C2. \(\bigtriangleup ABC\) da \(AB = 2sm,BD\) mediana, \(BD = 1sm\), \(\angle BDA = 30^{{^\circ}}\). Úshmúyeshliktiń maydanı esaplansın. \\
C3. Eger a,b,c - oń sanlar bolsa, tómendegi teńsizlikti dálilleń: \(\sqrt{\mathbf{a}^{\mathbf{2}}\mathbf{+ ab +}\mathbf{b}^{\mathbf{2}}}\mathbf{+}\sqrt{\mathbf{b}^{\mathbf{2}}\mathbf{+ bc +}\mathbf{c}^{\mathbf{2}}}\mathbf{>}\sqrt{\mathbf{a}^{\mathbf{2}}\mathbf{+ ac +}\mathbf{c}^{\mathbf{2}}}\) \\

\end{tabular}
\vspace{1cm}


\begin{tabular}{m{17cm}}
\textbf{7-variant}
\newline

T1. Mına \(P(0) = 20\) hám \(P(1) = 100\) shártlerin qanaǵatlandıratuǵın \(P(x)\) kóp aǵzalısı bar bolama? \\
T2. Kombinatorika elementleri hám Nyuton binomı. \\
A1. Teńlemeni sheshiń \(\sqrt{\frac{18 - 7x - x^{2}}{8 - 6x + x^{2}}} + \sqrt{\frac{8 - 6x + x^{2}}{18 - 7x - x^{2}}} = \frac{13}{6}\). \\
A2. Teńlemeni sheshiń \(\left( x^{2} + 10x + 10 \right)\left( x^{2} + x + 10 \right) = 10x^{2}\) . \\
A3. Teńlemeni sheshiń. \(\sqrt{x} + \frac{2x + 1}{x + 2} = 2\). \\
B1. \(P(x) = (x - 5)^{2n + 1} + (x - 1)^{2n + 3}\) kópaǵzalını \(x - 3\) ga bólgende qaldıq \(3 \cdot 2^{3n - 4}\) bolsa, \(n\) ni tabıń. \\
B2. Tómendegi aytımdı qálegen natural san ushın matematikalıq induksiya metodi járdeminde dálilleń: \(1 \cdot 2 + 2 \cdot 3 + 3 \cdot 4 + ... + n(n + 1) = \frac{n(n + 1)(n + 2)}{3}\); \\
B3. Tómendegi aytımdı qálegen natural san ushın matematikalıq induksiya metodi járdeminde dálilleń: \(6^{2n - 2} + 3^{n + 1} + 3^{n - 1}\) sanı 11 eseli ; \\
C1. \(x(1 - x)^{4} + x^{2}(1 + 2x)^{8} + x^{3}(1 + 3x)^{12}\) ańlatpada \(x^{4}\) aldıńdaǵı koeffitsiyentti tabıń. \\
C2. \(ABCD(AD\| BC)\) trapetsiya diagonallari \(O\) noqatda kesilisedi. Eger\emph{AOD} úshmúyeshliktiń maydanı \(a^{2}\) ga, \emph{BOC} úshmúyeshliktiń maydanı \(b^{2}\) ga teńligi ma'lum bolsa, trapetsiya maydanın tabıń. \\
C3. Eger \(S\) úshmúyeshliktiń maydanı, \(b\) hám \(c\) onıń táreplari bolsa, \(S \leq \frac{b^{2} + c^{2}}{4}\) bolıwın dálilleń. \\

\end{tabular}
\vspace{1cm}


\begin{tabular}{m{17cm}}
\textbf{8-variant}
\newline

T1. Qosındısı berge teń bolǵan \(x,y,z\) oń sanları ushın \(\frac{1}{x} + \frac{1}{y} + \frac{1}{z} \geq 9\) teńsizligi orınlı bolıwın dálilleń. \\
T2. \(x\) ózgeriwshiniń qálegen pútin mánisinde \(ax^{2} + bx + c\) ush aǵzalısınıń mánisi pútin bolıwı ushın \(2a,\ a + b\) hám \(c\) sanlarınıń pútin bolıwı zárurli hám jetkilikli ekenligin dálilleń. \\
A1. Teńlemeni sheshiń \(\left( x^{2} + 10x + 10 \right)\left( x^{2} + x + 10 \right) = 10x^{2}\) . \\
A2. Teńlemeni sheshiń \(\left( x^{2} - 6x \right)^{2} - 2(x - 3)^{2} = 81\). \\
A3. Teńlemeni sheshiń. \(\frac{4x}{x^{2} + x + 3} + \frac{5x}{x^{2} - 5x + 3} = - \frac{3}{2}\). \\
B1. \(P(2x - 1) + P(x - 1) = 10x^{2} - 12x + 2\) bolsa, \(P(x)\) ni tabıń. \\
B2. Tómendegi aytımdı qálegen natural san ushın matematikalıq induksiya metodi járdeminde dálilleń: \(1 \cdot 1! + 2 \cdot 2! + 3 \cdot 3! + \ldots + n \cdot n! = (n + 1)! - 1\). \\
B3. Tómendegi aytımdı qálegen natural san ushın matematikalıq induksiya metodi járdeminde dálilleń:\(7^{n} - 1\) sanı 6 ga eseli; \\
C1. Birdeylikti dálilleń: \(\sum_{j = 0}^{n}C_{n}^{j} = 2^{n}\); \\
C2. Tuwrı múyeshli úshmúyeshlikda katetlar 7 sm hám 24 sm ga teń. Tuwrı múyeshnıń bissektrisasi túsirilgen. Bu bissektrisa gipotenuzani qanday uzunlikdagi kesindilerga ajıratadı? \\
C3. Eger a,b,c - oń sanlar bolsa, tómendegi teńsizlikti dálilleń: \(\sqrt{\mathbf{a}^{\mathbf{2}}\mathbf{+ ab +}\mathbf{b}^{\mathbf{2}}}\mathbf{+}\sqrt{\mathbf{b}^{\mathbf{2}}\mathbf{+ bc +}\mathbf{c}^{\mathbf{2}}}\mathbf{>}\sqrt{\mathbf{a}^{\mathbf{2}}\mathbf{+ ac +}\mathbf{c}^{\mathbf{2}}}\) \\

\end{tabular}
\vspace{1cm}


\begin{tabular}{m{17cm}}
\textbf{9-variant}
\newline

T1. \(P(x) = x^{6} - 3x^{5} + x^{4} - 6x^{2} + 2x - 6\) kóp aǵzalısınıń pútin korenlerin tabıń. \\
T2. Pifagor teoreması hám onıń dálilleniwleri. \\
A1. Teńlemeni sheshiń \(\left( x^{2} - 4x + 6 \right)^{2} - 4\left( x^{2} - 4x + 6 \right) + 6 = x\). \\
A2. Teńlemeni sheshiń \((x + 1)^{5} + (x - 1)^{5} = 32x\). \\
A3. Teńlemeni sheshiń. \(\sqrt{x + 8 + 2\sqrt{x + 7}} + \sqrt{x + 1 - \sqrt{x + 7}} = 4\). \\
B1. \(P(x) = x^{4} - 2x + 2^{n + 1}\) kópaǵzalını \(x - 2^{n}\) ga bólgende qaldıq \(2^{n - 2}\) bolsa, \(n\) ni tabıń. \\
B2. Tómendegi aytımdı qálegen natural san ushın matematikalıq induksiya metodi járdeminde dálilleń: \(2^{2} + 6^{2} + \ldots + (4n - 2)^{2} = \frac{4n(2n - 1)(2n + 1)}{3}\). \\
B3. Tómendegi aytımdı qálegen natural san ushın matematikalıq induksiya metodi járdeminde dálilleń: \(n\left( 2n^{2} - 3n + 1 \right)\) sanı 6 ga eseli ; \\
C1. \(\left( x^{3} - \frac{3}{x^{2}} \right)^{10}\) binom jayılmasinıń \(x\) qatnashmagan aǵzasın tabıń. \\
C2. Orayları \(O_{1}\) hám \(O_{2}\) noqatlarda hám radiusi \(R\) bolǵan eki teń sheńberlar sirtlay urinadi. \(l\) tuwrı sızıq bu sheńberlarni A, B, C hám \(D\) noqatlarda shunday kesib ótadiki, \(AB = BC = CD\) bóledi. \(O_{1}ADO_{2}\) tórtmúyeshlik maydanıni tabıń. \\
C3. Ya. Bernulli teńsizligi. Eger\(x \geq - 1\) bolsa, onda qálegen natural \(n\) sanı ushın \((1 + x)^{n} \geq 1 + nx\) teńsizlik orinli bolıwın dálilleń. \\

\end{tabular}
\vspace{1cm}


\begin{tabular}{m{17cm}}
\textbf{10-variant}
\newline

T1. \(n\) dárejeniń qanday mánislerinde \((x + 1)^{n} + (x - 1)^{n}\) ańlatpası \(x\) ańlatpaǵa qaldıqsız bólinedi? \\
T2. Mına \(P(0) = 20\) hám \(P(1) = 100\) shártlerin qanaǵatlandıratuǵın \(P(x)\) kóp aǵzalısı bar bolama? \\
A1. Teńlemeni sheshiń. \(\sqrt[3]{x} + \sqrt[3]{x - 16} = \sqrt[3]{x - 8}\). \\
A2. Teńlemeni sheshiń. \(\sqrt[3]{x - 1} + \sqrt[3]{x - 2} - \sqrt{2x - 3} = 0\). \\
A3. Teńsizlikti sheshiń: \(\sqrt{x^{2} - 4x} > x - 3\). \\
B1. \(P(x + n) = (x + n)^{3} + (x - n)^{2} + x + n + 6\) kópaǵzalıi berilgan. \(P(x)\) kópaǵzalıi \(x - n\) ga qaldıqsiz bólinse, \(n\) ni tabıń. \\
B2. Tómendegi aytımdı qálegen natural san ushın matematikalıq induksiya metodi járdeminde dálilleń: \(1^{2} + 2^{2} + 3^{2} + ... + n^{2} = \frac{n(n + 1)(2n + 1)}{6}\); \\
B3. Tómendegi aytımdı qálegen natural san ushın matematikalıq induksiya metodi járdeminde dálilleń: \(5^{2n + 1} + 3^{n + 2} \cdot 2^{n - 1}\) sanı 19 ga eseli ; \\
C1. Birdeylikti dálilleń:\(C_{n}^{j} = C_{n}^{n - j}\); \\
C2. \emph{ABC} úshmúyeshliktiń \emph{AB} tárepinda jaylasqan \(N\) noqatdan \(NQ\| AC\) hám \(NP\| BC\) tuwrı sızıqlar túsirilgen. Eger \emph{BNQ} úshmúyeshliktiń maydanı \(S_{1}\) ga, \emph{ANP} úshmúyeshliktiń maydanı \(S_{2}\) ga teńligi ma'lum bolsa, \emph{ABC} úshmúyeshliktiń maydanıni tabıń. \\
C3. Durıs úshmúyeshliktiń tárepi a ga teń. Tárepini diametr deb esaplap dóńgelek jasalǵan. Úshmúyeshliktiń usı dóńgelekten sirtindaǵi bólegi maydanın tabıń. \\

\end{tabular}
\vspace{1cm}


\begin{tabular}{m{17cm}}
\textbf{11-variant}
\newline

T1. Matematikalıq induksiya metodı hám onıń qollanılıwına mısallar. \\
T2. Simmetriyalıq kóp aǵzalılar. \\
A1. Teńlemeni sheshiń \(\sqrt{\frac{18 - 7x - x^{2}}{8 - 6x + x^{2}}} + \sqrt{\frac{8 - 6x + x^{2}}{18 - 7x - x^{2}}} = \frac{13}{6}\). \\
A2. Teńsizlikti sheshiń: \(\frac{x^{3} + 3x^{2} - x - 3}{x^{2} + 3x - 10} < 0\). \\
A3. Teńlemeni sheshiń \((x + 4)(x + 1) - 3\sqrt{x^{2} + 5x + 2} = 6\). \\
B1. \(P(x) = (x - 5)^{2n + 1} + (x - 1)^{2n + 3}\) kópaǵzalını \(x - 3\) ga bólgende qaldıq \(3 \cdot 2^{3n - 4}\) bolsa, \(n\) ni tabıń. \\
B2. Tómendegi aytımdı qálegen natural san ushın matematikalıq induksiya metodi járdeminde dálilleń: \(\frac{1}{1 \cdot 5} + \frac{1}{5 \cdot 9} + ... + \frac{1}{(4n - 3)(4n + 1)} = \frac{n}{4n + 1}\); \\
B3. Tómendegi aytımdı qálegen natural san ushın matematikalıq induksiya metodi járdeminde dálilleń: \(2n^{3} + 3n^{2} + 7n\) sanı 6 ga eseli ; \\
C1. Teńsizlikti sheshiń \(C_{13}^{x} < C_{13}^{x + 2}\), \(x \in N\) \\
C2. Teń qaptallı \(ABC(AB = BC)\) úshmúyeshlikda \emph{AD} bissektrisa túsirilgen. Eger\(S_{ABD} = S_{1},S_{\bigtriangleup ADC} = S_{2}\) bolsa, \emph{AC} ni tabıń. \\
C3. Eger a,b - oń sanlar bolsa, tómendegi teńsizlikti dálilleń: \(\sqrt[3]{\frac{a}{b}} + \sqrt[3]{\frac{b}{a}} \leq \sqrt[3]{2(a + b)\left( \frac{1}{a} + \frac{1}{b} \right)}\) \\

\end{tabular}
\vspace{1cm}


\begin{tabular}{m{17cm}}
\textbf{12-variant}
\newline

T1. \(P(x) = x^{6} - 3x^{5} + x^{4} - 6x^{2} + 2x - 6\) kóp aǵzalısınıń pútin korenlerin tabıń. \\
T2. Qálegen \(a\) parametri hám \(x\) ushın \(x(a - x) \leq a^{2}/4\) teńsizligi orınlı bolıwın dálilleń. \\
A1. Teńlemeni sheshiń. \(\sqrt{3x^{2} - 2x + 15} + \sqrt{3x^{2} - 2x + 8} = 7\). \\
A2. Teńlemeni sheshiń. \((x - 4)^{3} + (x - 4)^{2} + (x - 4)(x - 3) + (x - 3)^{2} + (x - 3)^{3} = 6\). \\
A3. Teńsizlikti sheshiń:\(x^{2}\left( x^{4} + 36 \right) - 6\sqrt{3}\left( x^{4} + 4 \right) < 0\). \\
B1. \(P(x + 2) + P(x - 1) = - 2x^{2} - 2x + 7\) bolsa, \(P(x)\) ni \(x + 4\) ga bólgendegi qaldıqni tabıń. \\
B2. Tómendegi aytımdı qálegen natural san ushın matematikalıq induksiya metodi járdeminde dálilleń: \(\frac{1}{4 \cdot 5} + \frac{1}{5 \cdot 6} + \frac{1}{6 \cdot 7} + \ldots + \frac{1}{(n + 3) \cdot (n + 4)} = \frac{n}{4 \cdot (n + 4)}\). \\
B3. Tómendegi aytımdı qálegen natural san ushın matematikalıq induksiya metodi járdeminde dálilleń: \(n^{3} + (n + 1)^{3} + (n + 2)^{3}\) sanı 9 ga eseli ; \\
C1. Birdeylikti dálilleń: \(C_{n + 1}^{j + 1} = C_{n}^{j} + C_{n}^{j + 1}\); \\
C2. Eki birdey radiusli dóńgeleklar sonday jaylasqan, olardıń orayları arasındaǵı aralıq radiusqa teń. Dóńgeleklar kesilisken bólegi maydanınıń, kesilisken bólegine ishley sızılǵan kvadrat maydanına qatnasin tabıń. \\
C3. \(R\) radiusli dóńgelekga bitta umumiy uchga ega bolǵan durıs úshmúyeshlik hám kvadrat ishley sızılǵan. Olardıń kesilisken bóleginıń maydanıni tabıń. \\

\end{tabular}
\vspace{1cm}


\begin{tabular}{m{17cm}}
\textbf{13-variant}
\newline

T1. Bezu teoreması hám onıń qollanılıwı. \\
T2. \(b\) parametriniń qanday mánisinde \(x^{3} + 17x^{2} + bx - 17 = 0\) teńlemesiniń korenleri pútin sanlardan turadı? \\
A1. Teńlemeni sheshiń. \((\sqrt{x + 1} + \sqrt{x})^{3} + (\sqrt{x + 1} + \sqrt{x})^{2} = 2\). \\
A2. Teńlemeni sheshiń. \(\sqrt{x^{2} + x + 4} + \sqrt{x^{2} + x + 1} = \sqrt{2x^{2} + 2x + 9}\). \\
A3. Teńsizlikti sheshiń: \(\sqrt{x + 3} + \sqrt{x - 2} - \sqrt{2x + 4} > 0\). \\
B1. \(P(x + 1) + P(x - 3) = 2x^{2} - 10x + 16\) bolsa, \(P(x)\) ni tabıń. \\
B2. Tómendegi aytımdı qálegen natural san ushın matematikalıq induksiya metodi járdeminde dálilleń: \(1^{3} + 2^{3} + 3^{3} + ... + n^{3} = \left( \frac{n(n + 1)}{2} \right)^{2}\); \\
B3. Tómendegi aytımdı qálegen natural san ushın matematikalıq induksiya metodi járdeminde dálilleń: \(5^{n} - 4n + 15\) sanı 16 ga eseli ; \\
C1. \(5C_{n}^{3} = C_{n + 2}^{4}\) bolsa, \(n\) ni tabıń. \\
C2. Tuwrı múyeshli úshmúyeshliktiń katetlari \(b\) hám \(c\) ga teń. Tuwrı múyesh bissektrisasinıń uzunligi tabılsın. \\
C3. Eger \(S\) úshmúyeshliktiń maydanı, \(b\) hám \(c\) onıń táreplari bolsa, \(S \leq \frac{b^{2} + c^{2}}{4}\) bolıwın dálilleń. \\

\end{tabular}
\vspace{1cm}


\begin{tabular}{m{17cm}}
\textbf{14-variant}
\newline

T1. Pifagor teoreması hám onıń dálilleniwleri. \\
T2. Fales teoreması hám onıń qollanılıwı. \\
A1. Teńlemeni sheshiń. \(\frac{z}{z + 1} - 2\sqrt{\frac{z + 1}{2}} = 3\). \\
A2. Teńlemeni sheshiń \(\left( x^{2} - 6x \right)^{2} - 2(x - 3)^{2} = 81\). \\
A3. Teńlemeni sheshiń \((x + 4)(x + 1) - 3\sqrt{x^{2} + 5x + 2} = 6\). \\
B1. \(P(x + 3)\) kópaǵzalını \(x + 1\) ga bólgende qaldıq -3, \(Q(2x - 1)\) kópaǵzalını \(x - 1\)ga bólgende qaldıq 2 bolsa, \(P(x + 4) + x^{2}Q(x + 3)\) kópaǵzalını \(x + 2\) ga bólgendegi qaldıqni tabıń. \\
B2. Tómendegi aytımdı qálegen natural san ushın matematikalıq induksiya metodi járdeminde dálilleń: \(1 \cdot 2 + 2 \cdot 3 + 3 \cdot 4 + ... + n(n + 1) = \frac{n(n + 1)(n + 2)}{3}\); \\
B3. Tómendegi aytımdı qálegen natural san ushın matematikalıq induksiya metodi járdeminde dálilleń: \(5 \cdot 2^{3n - 2} + 3^{3n - 1}\) sanı 19 ga eseli \\
C1. \(C_{n + 4}^{n + 1} - C_{n + 3}^{n} = 15(n + 2)\) bolsa, \(n\) ni tabıń. \\
C2. Teń qaptallı úshmúyeshliktiń maydanı \(S\) ga teń. Qaptal táreplariga túsirilgen medianalari arasındaǵı múyesh \(\alpha\) ga teń. Úshmúyeshlik ultanini tabıń. \\
C3. Ya. Bernulli teńsizligi. Eger\(x \geq - 1\) bolsa, onda qálegen natural \(n\) sanı ushın \((1 + x)^{n} \geq 1 + nx\) teńsizlik orinli bolıwın dálilleń. \\

\end{tabular}
\vspace{1cm}


\begin{tabular}{m{17cm}}
\textbf{15-variant}
\newline

T1. \(n\) dárejeniń qanday mánislerinde \((x + 1)^{n} + (x - 1)^{n}\) ańlatpası \(x\) ańlatpaǵa qaldıqsız bólinedi? \\
T2. \(x\) ózgeriwshiniń qálegen pútin mánisinde \(ax^{2} + bx + c\) ush aǵzalısınıń mánisi pútin bolıwı ushın \(2a,\ a + b\) hám \(c\) sanlarınıń pútin bolıwı zárurli hám jetkilikli ekenligin dálilleń. \\
A1. Teńlemeni sheshiń. \(\sqrt[3]{x - 1} + \sqrt[3]{x - 2} - \sqrt{2x - 3} = 0\). \\
A2. Teńlemeni sheshiń. \(\sqrt{x^{2} + x + 4} + \sqrt{x^{2} + x + 1} = \sqrt{2x^{2} + 2x + 9}\). \\
A3. Teńlemeni sheshiń. \(\sqrt{x} + \frac{2x + 1}{x + 2} = 2\). \\
B1. \(P(x) = x^{4} - 2x + 2^{n + 1}\) kópaǵzalını \(x - 2^{n}\) ga bólgende qaldıq \(2^{n - 2}\) bolsa, \(n\) ni tabıń. \\
B2. Tómendegi aytımdı qálegen natural san ushın matematikalıq induksiya metodi járdeminde dálilleń: \(1 \cdot 2 + 2 \cdot 3 + 3 \cdot 4 + \ldots + n \cdot (n + 1) = \frac{n \cdot (n + 1) \cdot (n + 2)}{3}\). \\
B3. Tómendegi aytımdı qálegen natural san ushın matematikalıq induksiya metodi járdeminde dálilleń: \(n\left( 2n^{2} - 3n + 1 \right)\) sanı 6 ga eseli ; \\
C1. Teńlemeni sheshiń \(\frac{C_{2x}^{x + 1}}{C_{2x + 1}^{x - 1}} = \frac{2}{3}\), \(x \in N\) \\
C2. Bir burchagi \(60^{{^\circ}}\) bolǵan úshmúyeshlikka ishley sızılǵan sheńberdiń uriniw noqati shu múyeshke qarama- qarama-qarsı tárepini \(a\) hám \(b\) kesindilerga ajıratadı. Úshmúyeshlik maydanıni tabıń. \\
C3. Eger a,b - oń sanlar bolsa, tómendegi teńsizlikti dálilleń: \(\sqrt[3]{\frac{a}{b}} + \sqrt[3]{\frac{b}{a}} \leq \sqrt[3]{2(a + b)\left( \frac{1}{a} + \frac{1}{b} \right)}\) \\

\end{tabular}
\vspace{1cm}


\begin{tabular}{m{17cm}}
\textbf{16-variant}
\newline

T1. Haqıyqıy \(a_{1},\ a_{2},\ .\ .\ .\ ,\ a_{n},\ b_{1},\ b_{2},\ .\ .\ .\ ,\ b_{n}\) sanları ushın \(\left( a_{1}b_{1} + a_{2}b_{2} + \ .\ .\ .\  + a_{n}b_{n} \right)^{2} \leq \left( a_{1}^{2} + a_{2}^{2} + \ .\ .\ .\  + a_{n}^{2} \right)\left( b_{1}^{2} + b_{2}^{2} + \ .\ .\ .\  + b_{n}^{2} \right)\) Koshi teńsizligin dálilleń. \\
T2. \(a\) parametriniń qanday mánisinde \(P(x) = x^{2017} + ax - 5\) kóp aǵzalısı \((x + 1)\) kóp aǵzalısına qaldıqsız bólinedi? \\
A1. Teńlemeni sheshiń. \(\sqrt[3]{x} + \sqrt[3]{x - 16} = \sqrt[3]{x - 8}\). \\
A2. Teńsizlikti sheshiń: \(\sqrt{x^{2} - 4x} > x - 3\). \\
A3. Teńlemeni sheshiń \(\left( x^{2} + 10x + 10 \right)\left( x^{2} + x + 10 \right) = 10x^{2}\) . \\
B1. \(P(2x - 1) + P(x - 1) = 10x^{2} - 12x + 2\) bolsa, \(P(x)\) ni tabıń. \\
B2. Tómendegi aytımdı qálegen natural san ushın matematikalıq induksiya metodi járdeminde dálilleń: \(\frac{1}{1 \cdot 4} + \frac{1}{4 \cdot 7} + \frac{1}{7 \cdot 10} + \ldots + \frac{1}{(3n - 2) \cdot (3n + 1)} = \frac{n}{(3n + 1)}\). \\
B3. Tómendegi aytımdı qálegen natural san ushın matematikalıq induksiya metodi járdeminde dálilleń: \(6^{2n - 2} + 3^{n + 1} + 3^{n - 1}\) sanı 11 eseli ; \\
C1. \(\left( \sqrt{x} + \frac{1}{\sqrt[3]{x^{2}}} \right)^{n}\) binom jayılmasında 5-aǵza koeffitsiyentinıń 3-aǵza koeffitsiyentine qatnasi 7:2 ga teń. \(x\) nıń darajasi 1 ga teń bolǵan aǵzasın tabıń. \\
C2. \emph{ABC} úshmúyeshliktiń \(B\) uchidan \emph{AC} tárepiga \emph{BD} kesindi ótkazildi. \emph{BD} kesindi bu úshmúyeshliktiń maydanıni teń ekige bóledi. Eger\(AC = a\) bolsa, \emph{AD} hám \emph{DC} kesindilarnıń uzınlıqlarıni tabıń. \\
C3. Durıs úshmúyeshliktiń tárepi a ga teń. Tárepini diametr deb esaplap dóńgelek jasalǵan. Úshmúyeshliktiń usı dóńgelekten sirtindaǵi bólegi maydanın tabıń. \\

\end{tabular}
\vspace{1cm}


\begin{tabular}{m{17cm}}
\textbf{17-variant}
\newline

T1. Qosındısı berge teń bolǵan \(x,y,z\) oń sanları ushın \(\frac{1}{x} + \frac{1}{y} + \frac{1}{z} \geq 9\) teńsizligi orınlı bolıwın dálilleń. \\
T2. Kombinatorika elementleri hám Nyuton binomı. \\
A1. Teńlemeni sheshiń \(\sqrt{\frac{18 - 7x - x^{2}}{8 - 6x + x^{2}}} + \sqrt{\frac{8 - 6x + x^{2}}{18 - 7x - x^{2}}} = \frac{13}{6}\). \\
A2. Teńlemeni sheshiń \((x + 1)^{5} + (x - 1)^{5} = 32x\). \\
A3. Teńlemeni sheshiń. \(\sqrt{3x^{2} - 2x + 15} + \sqrt{3x^{2} - 2x + 8} = 7\). \\
B1. \(P(x) = x^{33} - 2ax^{21} + x^{8} + 8\) kópaǵzalıi berilgan. \(a\) nıń qaysi qiymati ushın \(P(x)\) kópaǵzalıi \(x + 1\) ga qaldıqsiz bóliadi? \\
B2. Tómendegi aytımdı qálegen natural san ushın matematikalıq induksiya metodi járdeminde dálilleń: \(\frac{1}{1 \cdot 5} + \frac{1}{5 \cdot 9} + ... + \frac{1}{(4n - 3)(4n + 1)} = \frac{n}{4n + 1}\); \\
B3. Tómendegi aytımdı qálegen natural san ushın matematikalıq induksiya metodi járdeminde dálilleń:\(7^{n} - 1\) sanı 6 ga eseli; \\
C1. \(\frac{1}{C_{4}^{n}} = \frac{1}{C_{5}^{n}} + \frac{1}{C_{6}^{n}}\) bolsa, \(n\) ni tabıń \\
C2. Durıs úshmúyeshliktiń uchlari uchta parallel tuwrı sızıqlarda yotadi. Eger ortadagi tuwrı sızıqdan chekkalardagi tuwrı sızıqlargacha bolǵan aralıq \(a\) hám \(b\) ga teń bolsa, úshmúyeshliktiń tárepini tabıń. \\
C3. Eger a,b,c - oń sanlar bolsa, tómendegi teńsizlikti dálilleń: \(\sqrt{\mathbf{a}^{\mathbf{2}}\mathbf{+ ab +}\mathbf{b}^{\mathbf{2}}}\mathbf{+}\sqrt{\mathbf{b}^{\mathbf{2}}\mathbf{+ bc +}\mathbf{c}^{\mathbf{2}}}\mathbf{>}\sqrt{\mathbf{a}^{\mathbf{2}}\mathbf{+ ac +}\mathbf{c}^{\mathbf{2}}}\) \\

\end{tabular}
\vspace{1cm}


\begin{tabular}{m{17cm}}
\textbf{18-variant}
\newline

T1. \(2^{81} + 1\) sanı 9 sanına qaldıqsız bóliniwin dálilleń. \\
T2. Mına \(P(x) = x^{5} + 11x^{4} + 37x^{3} + 35x^{2} - 44x - 40\) kóp aǵzalısı \(Q(x) = x^{2} + 3x + 2\) kóp aǵzalısına qaldıqsız bólinedime? \\
A1. Teńsizlikti sheshiń: \(\sqrt{x + 3} + \sqrt{x - 2} - \sqrt{2x + 4} > 0\). \\
A2. Teńlemeni sheshiń \(\left( x^{2} - 4x + 6 \right)^{2} - 4\left( x^{2} - 4x + 6 \right) + 6 = x\). \\
A3. Teńlemeni sheshiń. \((x - 4)^{3} + (x - 4)^{2} + (x - 4)(x - 3) + (x - 3)^{2} + (x - 3)^{3} = 6\). \\
B1. \(P(x + 3) = x^{2} - x + n\) bolsa. \(P(x - 2)\) kópaǵzalını \(x - 3\) ga bólgende qaldıq \(10\) bolsa, \(n\) ni tabıń. \\
B2. Tómendegi aytımdı qálegen natural san ushın matematikalıq induksiya metodi járdeminde dálilleń: \(1^{2} + 3^{2} + 5^{2} + ... + (2n - 1)^{2} = \frac{n\left( 4n^{2} - 1 \right)}{3}\); \\
B3. Tómendegi aytımdı qálegen natural san ushın matematikalıq induksiya metodi járdeminde dálilleń: \(5^{n + 2} + 26 \cdot 5^{n} + 8^{2n + 1}\) sanı 59 ga eseli; \\
C1. Birdeylikti dálilleń:\(C_{n + 2}^{j + 2} = C_{n}^{j} + 2C_{n}^{j + 1} + C_{n}^{j + 2}\); \\
C2. \(\bigtriangleup ABC\) da \(AB = 3sm,AC = 5sm,\angle BAC = 120^{{^\circ}}.BD\) bissektrisanıń uzunligi tabılsın. \\
C3. \(R\) radiusli dóńgelekga bitta umumiy uchga ega bolǵan durıs úshmúyeshlik hám kvadrat ishley sızılǵan. Olardıń kesilisken bóleginıń maydanıni tabıń. \\

\end{tabular}
\vspace{1cm}


\begin{tabular}{m{17cm}}
\textbf{19-variant}
\newline

T1. \(P(x) = (x - 1)^{20}\left( x^{2} + 25 \right)\) kóp aǵzalisınıń koefficentleri qosındısın tabıń. \\
T2. Qálegen \(a,b,c \in (0;1)\) sanları ushın \(a(1 - b) > 1/4,\ b(1 - c) > 1/4,\ c(1 - a) > 1/4\) teńsizlikleri bir waqıtta orınlı bola almaytuǵınlıǵin dálilleń. \\
A1. Teńlemeni sheshiń. \(\frac{z}{z + 1} - 2\sqrt{\frac{z + 1}{2}} = 3\). \\
A2. Teńsizlikti sheshiń:\(x^{2}\left( x^{4} + 36 \right) - 6\sqrt{3}\left( x^{4} + 4 \right) < 0\). \\
A3. Teńlemeni sheshiń. \((\sqrt{x + 1} + \sqrt{x})^{3} + (\sqrt{x + 1} + \sqrt{x})^{2} = 2\). \\
B1. \(P(x)\) kópaǵzalını \(3x^{2} - 4x + 1\) ga bólgenimizdeqaldıq \(6x - 11\) bolsa, \(P(x)\) kópaǵzalını \(3x - 1\)ga bólgende qaldıqni tabıń. \\
B2. Tómendegi aytımdı qálegen natural san ushın matematikalıq induksiya metodi járdeminde dálilleń: \(1^{2} + 2^{2} + 3^{2} + ... + n^{2} = \frac{n(n + 1)(2n + 1)}{6}\); \\
B3. Tómendegi aytımdı qálegen natural san ushın matematikalıq induksiya metodi járdeminde dálilleń: \(6^{2n - 2} + 3^{n + 1} + 3^{n - 1}\) sanı 11 eseli ; \\
C1. Teńsizlikti sheshiń \(5C_{x}^{3} < C_{x + 2}^{4}\), \(x \in N\) \\
C2. Egerteń qaptallı úshmúyeshliktiń perimetri 32 dm , orta sızıǵı 6 dm ga teń bolsa, Onıń táreplari uzınlıqları tabılsın. \\
C3. Eger a,b,c - oń sanlar bolsa, tómendegi teńsizlikti dálilleń: \(\sqrt{\mathbf{a}^{\mathbf{2}}\mathbf{+ ab +}\mathbf{b}^{\mathbf{2}}}\mathbf{+}\sqrt{\mathbf{b}^{\mathbf{2}}\mathbf{+ bc +}\mathbf{c}^{\mathbf{2}}}\mathbf{>}\sqrt{\mathbf{a}^{\mathbf{2}}\mathbf{+ ac +}\mathbf{c}^{\mathbf{2}}}\) \\

\end{tabular}
\vspace{1cm}


\begin{tabular}{m{17cm}}
\textbf{20-variant}
\newline

T1. Qosındısı berge teń bolǵan \(x,y,z\) oń sanları ushın \(\frac{1}{x} + \frac{1}{y} + \frac{1}{z} \geq 9\) teńsizligi orınlı bolıwın dálilleń. \\
T2. Pifagor teoreması hám onıń dálilleniwleri. \\
A1. Teńlemeni sheshiń. \(\sqrt{x + 8 + 2\sqrt{x + 7}} + \sqrt{x + 1 - \sqrt{x + 7}} = 4\). \\
A2. Teńlemeni sheshiń. \(\frac{4x}{x^{2} + x + 3} + \frac{5x}{x^{2} - 5x + 3} = - \frac{3}{2}\). \\
A3. Teńsizlikti sheshiń: \(\frac{x^{3} + 3x^{2} - x - 3}{x^{2} + 3x - 10} < 0\). \\
B1. \(P(x + n) = (x + n)^{3} + (x - n)^{2} + x + n + 6\) kópaǵzalıi berilgan. \(P(x)\) kópaǵzalıi \(x - n\) ga qaldıqsiz bólinse, \(n\) ni tabıń. \\
B2. Tómendegi aytımdı qálegen natural san ushın matematikalıq induksiya metodi járdeminde dálilleń: \(\left( 1 - \frac{1}{4} \right)\left( 1 - \frac{1}{9} \right)...\left( 1 - \frac{1}{n^{2}} \right) = \frac{n + 1}{2n}\), \(n \geq 2\) \\
B3. Tómendegi aytımdı qálegen natural san ushın matematikalıq induksiya metodi járdeminde dálilleń: \(5 \cdot 2^{3n - 2} + 3^{3n - 1}\) sanı 19 ga eseli \\
C1. Birdeylikti dálilleń: \(C_{n + k}^{j + k} = \sum_{s = 0}^{k}C_{n}^{j + s}C_{k}^{s}\); \\
C2. Úshmúyeshliktiń perimetri \(4,5dm\) ga teń, bissektrisa bolsa qarama-qarsı tárepni uzınlıqları 6 hám 9 sm ga teń bolǵan kesindilerga ajıratadı. Úshmúyeshliktiń táreplari tabılsın. \\
C3. \(R\) radiusli dóńgelekga bitta umumiy uchga ega bolǵan durıs úshmúyeshlik hám kvadrat ishley sızılǵan. Olardıń kesilisken bóleginıń maydanıni tabıń. \\

\end{tabular}
\vspace{1cm}


\begin{tabular}{m{17cm}}
\textbf{21-variant}
\newline

T1. Mına \(P(x) = x^{5} + 11x^{4} + 37x^{3} + 35x^{2} - 44x - 40\) kóp aǵzalısı \(Q(x) = x^{2} + 3x + 2\) kóp aǵzalısına qaldıqsız bólinedime? \\
T2. \(b\) parametriniń qanday mánisinde \(x^{3} + 17x^{2} + bx - 17 = 0\) teńlemesiniń korenleri pútin sanlardan turadı? \\
A1. Teńlemeni sheshiń \(\left( x^{2} + 10x + 10 \right)\left( x^{2} + x + 10 \right) = 10x^{2}\) . \\
A2. Teńlemeni sheshiń. \(\sqrt{x + 8 + 2\sqrt{x + 7}} + \sqrt{x + 1 - \sqrt{x + 7}} = 4\). \\
A3. Teńlemeni sheshiń. \(\sqrt{x^{2} + x + 4} + \sqrt{x^{2} + x + 1} = \sqrt{2x^{2} + 2x + 9}\). \\
B1. \(P(x + n) = (x + n)^{3} + (x - n)^{2} + x + n + 6\) kópaǵzalıi berilgan. \(P(x)\) kópaǵzalıi \(x - n\) ga qaldıqsiz bólinse, \(n\) ni tabıń. \\
B2. Tómendegi aytımdı qálegen natural san ushın matematikalıq induksiya metodi járdeminde dálilleń: \(2^{2} + 6^{2} + \ldots + (4n - 2)^{2} = \frac{4n(2n - 1)(2n + 1)}{3}\). \\
B3. Tómendegi aytımdı qálegen natural san ushın matematikalıq induksiya metodi járdeminde dálilleń: \(5^{n + 2} + 26 \cdot 5^{n} + 8^{2n + 1}\) sanı 59 ga eseli; \\
C1. \((a + b)^{n}\) ańlatpa jayılmasinıń barcha koeffitsiyentlari yig`indisi 4096 ga teń bolsa, Onıń eń úlken koeffitsiyentin tabıń. \\
C2. Tuwrı múyeshli úshmúyeshliktiń tuwrı burchagi bissektrisasi shu uchdan túsirilgen mediana hám biyiklik arasındaǵı múyeshni ham teń ekige bóliniwini dálilleń. \\
C3. Ya. Bernulli teńsizligi. Eger\(x \geq - 1\) bolsa, onda qálegen natural \(n\) sanı ushın \((1 + x)^{n} \geq 1 + nx\) teńsizlik orinli bolıwın dálilleń. \\

\end{tabular}
\vspace{1cm}


\begin{tabular}{m{17cm}}
\textbf{22-variant}
\newline

T1. \(a\) parametriniń qanday mánisinde \(P(x) = x^{2017} + ax - 5\) kóp aǵzalısı \((x + 1)\) kóp aǵzalısına qaldıqsız bólinedi? \\
T2. \(x\) ózgeriwshiniń qálegen pútin mánisinde \(ax^{2} + bx + c\) ush aǵzalısınıń mánisi pútin bolıwı ushın \(2a,\ a + b\) hám \(c\) sanlarınıń pútin bolıwı zárurli hám jetkilikli ekenligin dálilleń. \\
A1. Teńsizlikti sheshiń:\(x^{2}\left( x^{4} + 36 \right) - 6\sqrt{3}\left( x^{4} + 4 \right) < 0\). \\
A2. Teńlemeni sheshiń \((x + 1)^{5} + (x - 1)^{5} = 32x\). \\
A3. Teńlemeni sheshiń. \(\sqrt{3x^{2} - 2x + 15} + \sqrt{3x^{2} - 2x + 8} = 7\). \\
B1. \(P(x + 3)\) kópaǵzalını \(x + 1\) ga bólgende qaldıq -3, \(Q(2x - 1)\) kópaǵzalını \(x - 1\)ga bólgende qaldıq 2 bolsa, \(P(x + 4) + x^{2}Q(x + 3)\) kópaǵzalını \(x + 2\) ga bólgendegi qaldıqni tabıń. \\
B2. Tómendegi aytımdı qálegen natural san ushın matematikalıq induksiya metodi járdeminde dálilleń: \(1 \cdot 1! + 2 \cdot 2! + 3 \cdot 3! + \ldots + n \cdot n! = (n + 1)! - 1\). \\
B3. Tómendegi aytımdı qálegen natural san ushın matematikalıq induksiya metodi járdeminde dálilleń: \(n^{3} + (n + 1)^{3} + (n + 2)^{3}\) sanı 9 ga eseli ; \\
C1. Teńsizlikti sheshiń \(C_{13}^{x} < C_{13}^{x + 2}\), \(x \in N\) \\
C2. Tuwrı múyeshli úshmúyeshliktiń biyikligi gipotenuzani uzınlıqları \emph{x} hám \emph{y} ga teń bolǵan kesindilerga ajıratadı. Úshmúyeshliktiń maydanı esaplansın. \\
C3. Durıs úshmúyeshliktiń tárepi a ga teń. Tárepini diametr deb esaplap dóńgelek jasalǵan. Úshmúyeshliktiń usı dóńgelekten sirtindaǵi bólegi maydanın tabıń. \\

\end{tabular}
\vspace{1cm}


\begin{tabular}{m{17cm}}
\textbf{23-variant}
\newline

T1. \(P(x) = (x - 1)^{20}\left( x^{2} + 25 \right)\) kóp aǵzalisınıń koefficentleri qosındısın tabıń. \\
T2. \(P(x) = x^{6} - 3x^{5} + x^{4} - 6x^{2} + 2x - 6\) kóp aǵzalısınıń pútin korenlerin tabıń. \\
A1. Teńlemeni sheshiń \(\sqrt{\frac{18 - 7x - x^{2}}{8 - 6x + x^{2}}} + \sqrt{\frac{8 - 6x + x^{2}}{18 - 7x - x^{2}}} = \frac{13}{6}\). \\
A2. Teńlemeni sheshiń. \(\sqrt[3]{x - 1} + \sqrt[3]{x - 2} - \sqrt{2x - 3} = 0\). \\
A3. Teńsizlikti sheshiń: \(\frac{x^{3} + 3x^{2} - x - 3}{x^{2} + 3x - 10} < 0\). \\
B1. \(P(x) = x^{33} - 2ax^{21} + x^{8} + 8\) kópaǵzalıi berilgan. \(a\) nıń qaysi qiymati ushın \(P(x)\) kópaǵzalıi \(x + 1\) ga qaldıqsiz bóliadi? \\
B2. Tómendegi aytımdı qálegen natural san ushın matematikalıq induksiya metodi járdeminde dálilleń: \(1^{2} + 3^{2} + 5^{2} + ... + (2n - 1)^{2} = \frac{n\left( 4n^{2} - 1 \right)}{3}\); \\
B3. Tómendegi aytımdı qálegen natural san ushın matematikalıq induksiya metodi járdeminde dálilleń: \(5^{n} - 4n + 15\) sanı 16 ga eseli ; \\
C1. Teńlemeni sheshiń \(\frac{C_{2x}^{x + 1}}{C_{2x + 1}^{x - 1}} = \frac{2}{3}\), \(x \in N\) \\
C2. \emph{ABC} úshmúyeshliktiń \emph{AC}, \emph{BC} hám \emph{AB} táreplarida \emph{CMPA}, \emph{BEFC} hám \emph{ADKB} kvadratlar jasalǵan. Eger\(AB = 13\), \(AC = 14,BC = 15\) ekanligi ma'lum bolsa, \emph{DKEFMP} altimúyeshliktıń maydanın tabıń. \\
C3. Eger \(S\) úshmúyeshliktiń maydanı, \(b\) hám \(c\) onıń táreplari bolsa, \(S \leq \frac{b^{2} + c^{2}}{4}\) bolıwın dálilleń. \\

\end{tabular}
\vspace{1cm}


\begin{tabular}{m{17cm}}
\textbf{24-variant}
\newline

T1. Fales teoreması hám onıń qollanılıwı. \\
T2. Matematikalıq induksiya metodı hám onıń qollanılıwına mısallar. \\
A1. Teńlemeni sheshiń \((x + 4)(x + 1) - 3\sqrt{x^{2} + 5x + 2} = 6\). \\
A2. Teńlemeni sheshiń. \(\frac{4x}{x^{2} + x + 3} + \frac{5x}{x^{2} - 5x + 3} = - \frac{3}{2}\). \\
A3. Teńlemeni sheshiń. \(\sqrt[3]{x} + \sqrt[3]{x - 16} = \sqrt[3]{x - 8}\). \\
B1. \(P(x + 3) = x^{2} - x + n\) bolsa. \(P(x - 2)\) kópaǵzalını \(x - 3\) ga bólgende qaldıq \(10\) bolsa, \(n\) ni tabıń. \\
B2. Tómendegi aytımdı qálegen natural san ushın matematikalıq induksiya metodi járdeminde dálilleń: \(\left( 1 - \frac{1}{4} \right)\left( 1 - \frac{1}{9} \right)...\left( 1 - \frac{1}{n^{2}} \right) = \frac{n + 1}{2n}\), \(n \geq 2\) \\
B3. Tómendegi aytımdı qálegen natural san ushın matematikalıq induksiya metodi járdeminde dálilleń:\(7^{n} - 1\) sanı 6 ga eseli; \\
C1. \(5C_{n}^{3} = C_{n + 2}^{4}\) bolsa, \(n\) ni tabıń. \\
C2. \emph{ABC} úshmúyeshlik berilgan. Onıń medianalaridan \(\bigtriangleup A_{1}B_{1}C_{1}\) jasalǵan. \(\bigtriangleup ABC\) hám \(\bigtriangleup A_{1}B_{1}C_{1}\) maydanlarınıń qatnasi tabılsın. \\
C3. Eger a,b - oń sanlar bolsa, tómendegi teńsizlikti dálilleń: \(\sqrt[3]{\frac{a}{b}} + \sqrt[3]{\frac{b}{a}} \leq \sqrt[3]{2(a + b)\left( \frac{1}{a} + \frac{1}{b} \right)}\) \\

\end{tabular}
\vspace{1cm}


\begin{tabular}{m{17cm}}
\textbf{25-variant}
\newline

T1. Haqıyqıy \(a_{1},\ a_{2},\ .\ .\ .\ ,\ a_{n},\ b_{1},\ b_{2},\ .\ .\ .\ ,\ b_{n}\) sanları ushın \(\left( a_{1}b_{1} + a_{2}b_{2} + \ .\ .\ .\  + a_{n}b_{n} \right)^{2} \leq \left( a_{1}^{2} + a_{2}^{2} + \ .\ .\ .\  + a_{n}^{2} \right)\left( b_{1}^{2} + b_{2}^{2} + \ .\ .\ .\  + b_{n}^{2} \right)\) Koshi teńsizligin dálilleń. \\
T2. Bezu teoreması hám onıń qollanılıwı. \\
A1. Teńlemeni sheshiń \(\left( x^{2} - 6x \right)^{2} - 2(x - 3)^{2} = 81\). \\
A2. Teńlemeni sheshiń. \(\frac{z}{z + 1} - 2\sqrt{\frac{z + 1}{2}} = 3\). \\
A3. Teńsizlikti sheshiń: \(\sqrt{x^{2} - 4x} > x - 3\). \\
B1. \(P(x)\) kópaǵzalını \(3x^{2} - 4x + 1\) ga bólgenimizdeqaldıq \(6x - 11\) bolsa, \(P(x)\) kópaǵzalını \(3x - 1\)ga bólgende qaldıqni tabıń. \\
B2. Tómendegi aytımdı qálegen natural san ushın matematikalıq induksiya metodi járdeminde dálilleń: \(\frac{1}{1 \cdot 5} + \frac{1}{5 \cdot 9} + ... + \frac{1}{(4n - 3)(4n + 1)} = \frac{n}{4n + 1}\); \\
B3. Tómendegi aytımdı qálegen natural san ushın matematikalıq induksiya metodi járdeminde dálilleń: \(n\left( 2n^{2} - 3n + 1 \right)\) sanı 6 ga eseli ; \\
C1. \(x(1 - x)^{4} + x^{2}(1 + 2x)^{8} + x^{3}(1 + 3x)^{12}\) ańlatpada \(x^{4}\) aldıńdaǵı koeffitsiyentti tabıń. \\
C2. Tuwrı múyeshli úshmúyeshliktiń biyikligi gipotenuzani uzınlıqları 18 hám 32 sm ga teń bolǵan kesindilerga ajıratadı. Úshmúyeshliktiń maydanı esaplansın. \\
C3. Eger \(S\) úshmúyeshliktiń maydanı, \(b\) hám \(c\) onıń táreplari bolsa, \(S \leq \frac{b^{2} + c^{2}}{4}\) bolıwın dálilleń. \\

\end{tabular}
\vspace{1cm}


\begin{tabular}{m{17cm}}
\textbf{26-variant}
\newline

T1. \(2^{81} + 1\) sanı 9 sanına qaldıqsız bóliniwin dálilleń. \\
T2. Qálegen \(a\) parametri hám \(x\) ushın \(x(a - x) \leq a^{2}/4\) teńsizligi orınlı bolıwın dálilleń. \\
A1. Teńlemeni sheshiń \(\left( x^{2} - 4x + 6 \right)^{2} - 4\left( x^{2} - 4x + 6 \right) + 6 = x\). \\
A2. Teńsizlikti sheshiń: \(\sqrt{x + 3} + \sqrt{x - 2} - \sqrt{2x + 4} > 0\). \\
A3. Teńlemeni sheshiń. \((x - 4)^{3} + (x - 4)^{2} + (x - 4)(x - 3) + (x - 3)^{2} + (x - 3)^{3} = 6\). \\
B1. \(P(2x - 1) + P(x - 1) = 10x^{2} - 12x + 2\) bolsa, \(P(x)\) ni tabıń. \\
B2. Tómendegi aytımdı qálegen natural san ushın matematikalıq induksiya metodi járdeminde dálilleń: \(2^{2} + 6^{2} + \ldots + (4n - 2)^{2} = \frac{4n(2n - 1)(2n + 1)}{3}\). \\
B3. Tómendegi aytımdı qálegen natural san ushın matematikalıq induksiya metodi járdeminde dálilleń: \(5^{2n + 1} + 3^{n + 2} \cdot 2^{n - 1}\) sanı 19 ga eseli ; \\
C1. Birdeylikti dálilleń: \(\sum_{j = 0}^{n}C_{n}^{j} = 2^{n}\); \\
C2. Úshmúyeshliktiń a, b hám \(c\) táreplari arifmetik progressiya quraydı. \(ac = 6Rr\) bolıwın dálilleń. Bu yerda \(R\) hám \(r\) sirtlay hám ishki sızılǵan sheńberlernıń radiuslari. \\
C3. Ya. Bernulli teńsizligi. Eger\(x \geq - 1\) bolsa, onda qálegen natural \(n\) sanı ushın \((1 + x)^{n} \geq 1 + nx\) teńsizlik orinli bolıwın dálilleń. \\

\end{tabular}
\vspace{1cm}


\begin{tabular}{m{17cm}}
\textbf{27-variant}
\newline

T1. \(n\) dárejeniń qanday mánislerinde \((x + 1)^{n} + (x - 1)^{n}\) ańlatpası \(x\) ańlatpaǵa qaldıqsız bólinedi? \\
T2. Simmetriyalıq kóp aǵzalılar. \\
A1. Teńlemeni sheshiń. \((\sqrt{x + 1} + \sqrt{x})^{3} + (\sqrt{x + 1} + \sqrt{x})^{2} = 2\). \\
A2. Teńlemeni sheshiń. \(\sqrt{x} + \frac{2x + 1}{x + 2} = 2\). \\
A3. Teńsizlikti sheshiń:\(x^{2}\left( x^{4} + 36 \right) - 6\sqrt{3}\left( x^{4} + 4 \right) < 0\). \\
B1. \(P(x) = x^{4} - 2x + 2^{n + 1}\) kópaǵzalını \(x - 2^{n}\) ga bólgende qaldıq \(2^{n - 2}\) bolsa, \(n\) ni tabıń. \\
B2. Tómendegi aytımdı qálegen natural san ushın matematikalıq induksiya metodi járdeminde dálilleń: \(1 \cdot 2 + 2 \cdot 3 + 3 \cdot 4 + ... + n(n + 1) = \frac{n(n + 1)(n + 2)}{3}\); \\
B3. Tómendegi aytımdı qálegen natural san ushın matematikalıq induksiya metodi járdeminde dálilleń: \(2n^{3} + 3n^{2} + 7n\) sanı 6 ga eseli ; \\
C1. \(\left( 2x^{\ ^{2}} - \frac{b}{2x^{3}} \right)^{10}\) binom jayılmasinıń \(x\) qatnashmagan aǵzasın tabıń. \\
C2. \emph{ABCD} parallelogrammnıń \emph{AD} tárepi \(n\) ta teń bólekke bólingen. Birinchi bólinish noqatsi \(P\) hám \(B\) uch bilan birlashtirilgan. \emph{BP} tuwrı sızıq \emph{AC} dioganaldan Onıń \(\frac{1}{n + 1}\) bólegiga teń \emph{AQ} kesindi ajratishini dálilleń. \\
C3. \(R\) radiusli dóńgelekga bitta umumiy uchga ega bolǵan durıs úshmúyeshlik hám kvadrat ishley sızılǵan. Olardıń kesilisken bóleginıń maydanıni tabıń. \\

\end{tabular}
\vspace{1cm}


\begin{tabular}{m{17cm}}
\textbf{28-variant}
\newline

T1. Mına \(P(0) = 20\) hám \(P(1) = 100\) shártlerin qanaǵatlandıratuǵın \(P(x)\) kóp aǵzalısı bar bolama? \\
T2. Qálegen \(a,b,c \in (0;1)\) sanları ushın \(a(1 - b) > 1/4,\ b(1 - c) > 1/4,\ c(1 - a) > 1/4\) teńsizlikleri bir waqıtta orınlı bola almaytuǵınlıǵin dálilleń. \\
A1. Teńlemeni sheshiń \(\left( x^{2} + 10x + 10 \right)\left( x^{2} + x + 10 \right) = 10x^{2}\) . \\
A2. Teńlemeni sheshiń. \(\frac{4x}{x^{2} + x + 3} + \frac{5x}{x^{2} - 5x + 3} = - \frac{3}{2}\). \\
A3. Teńlemeni sheshiń \((x + 4)(x + 1) - 3\sqrt{x^{2} + 5x + 2} = 6\). \\
B1. \(P(x + 1) + P(x - 3) = 2x^{2} - 10x + 16\) bolsa, \(P(x)\) ni tabıń. \\
B2. Tómendegi aytımdı qálegen natural san ushın matematikalıq induksiya metodi járdeminde dálilleń: \(\frac{1}{1 \cdot 4} + \frac{1}{4 \cdot 7} + \frac{1}{7 \cdot 10} + \ldots + \frac{1}{(3n - 2) \cdot (3n + 1)} = \frac{n}{(3n + 1)}\). \\
B3. Tómendegi aytımdı qálegen natural san ushın matematikalıq induksiya metodi járdeminde dálilleń: \(5 \cdot 2^{3n - 2} + 3^{3n - 1}\) sanı 19 ga eseli \\
C1. \((x + 1)^{3} + (x + 1)^{4} + (x + 1)^{5} + ... + (x + 1)^{10}\) ańlatpada \(x^{3}\) aldıńda ǵı koeffitsiyentti tabıń \\
C2. Tuwrı múyeshli úshmúyeshlik ótkir múyeshlarinıń bisєektrisalari AD hám BK \(AB^{2} = AD \cdot BK\) bolsa, úshmúyeshliktiń múyeshlarini tabıń. \\
C3. Durıs úshmúyeshliktiń tárepi a ga teń. Tárepini diametr deb esaplap dóńgelek jasalǵan. Úshmúyeshliktiń usı dóńgelekten sirtindaǵi bólegi maydanın tabıń. \\

\end{tabular}
\vspace{1cm}


\begin{tabular}{m{17cm}}
\textbf{29-variant}
\newline

T1. Kombinatorika elementleri hám Nyuton binomı. \\
T2. Haqıyqıy \(a_{1},\ a_{2},\ .\ .\ .\ ,\ a_{n},\ b_{1},\ b_{2},\ .\ .\ .\ ,\ b_{n}\) sanları ushın \(\left( a_{1}b_{1} + a_{2}b_{2} + \ .\ .\ .\  + a_{n}b_{n} \right)^{2} \leq \left( a_{1}^{2} + a_{2}^{2} + \ .\ .\ .\  + a_{n}^{2} \right)\left( b_{1}^{2} + b_{2}^{2} + \ .\ .\ .\  + b_{n}^{2} \right)\) Koshi teńsizligin dálilleń. \\
A1. Teńlemeni sheshiń. \((x - 4)^{3} + (x - 4)^{2} + (x - 4)(x - 3) + (x - 3)^{2} + (x - 3)^{3} = 6\). \\
A2. Teńlemeni sheshiń \(\left( x^{2} - 4x + 6 \right)^{2} - 4\left( x^{2} - 4x + 6 \right) + 6 = x\). \\
A3. Teńlemeni sheshiń. \((\sqrt{x + 1} + \sqrt{x})^{3} + (\sqrt{x + 1} + \sqrt{x})^{2} = 2\). \\
B1. \(P(x) = (x - 5)^{2n + 1} + (x - 1)^{2n + 3}\) kópaǵzalını \(x - 3\) ga bólgende qaldıq \(3 \cdot 2^{3n - 4}\) bolsa, \(n\) ni tabıń. \\
B2. Tómendegi aytımdı qálegen natural san ushın matematikalıq induksiya metodi járdeminde dálilleń: \(1^{2} + 2^{2} + 3^{2} + ... + n^{2} = \frac{n(n + 1)(2n + 1)}{6}\); \\
B3. Tómendegi aytımdı qálegen natural san ushın matematikalıq induksiya metodi járdeminde dálilleń: \(n^{3} + (n + 1)^{3} + (n + 2)^{3}\) sanı 9 ga eseli ; \\
C1. \(\left( x\sqrt{x} - \frac{1}{x^{4}} \right)^{n}\) binom jayılmasında 3-aǵza koeffitsiyenti 2-aǵza koeffitsiyentidan 44 ga úlken.Ozod hadini tabıń. \\
C2. Ultanlari \(x\) hám 3 bolǵan trapetsiyada diagonallar órtalari arasındaǵı aralıqni \(x\) nıń funksiyasi sifatida ańlatpalań. \(x\) nіnń qanday qiymatida bu aralıq 1 ga teń bóledi? \\
C3. Eger a,b,c - oń sanlar bolsa, tómendegi teńsizlikti dálilleń: \(\sqrt{\mathbf{a}^{\mathbf{2}}\mathbf{+ ab +}\mathbf{b}^{\mathbf{2}}}\mathbf{+}\sqrt{\mathbf{b}^{\mathbf{2}}\mathbf{+ bc +}\mathbf{c}^{\mathbf{2}}}\mathbf{>}\sqrt{\mathbf{a}^{\mathbf{2}}\mathbf{+ ac +}\mathbf{c}^{\mathbf{2}}}\) \\

\end{tabular}
\vspace{1cm}


\begin{tabular}{m{17cm}}
\textbf{30-variant}
\newline

T1. Qálegen \(a,b,c \in (0;1)\) sanları ushın \(a(1 - b) > 1/4,\ b(1 - c) > 1/4,\ c(1 - a) > 1/4\) teńsizlikleri bir waqıtta orınlı bola almaytuǵınlıǵin dálilleń. \\
T2. Pifagor teoreması hám onıń dálilleniwleri. \\
A1. Teńlemeni sheshiń. \(\frac{z}{z + 1} - 2\sqrt{\frac{z + 1}{2}} = 3\). \\
A2. Teńlemeni sheshiń. \(\sqrt{x + 8 + 2\sqrt{x + 7}} + \sqrt{x + 1 - \sqrt{x + 7}} = 4\). \\
A3. Teńlemeni sheshiń \(\sqrt{\frac{18 - 7x - x^{2}}{8 - 6x + x^{2}}} + \sqrt{\frac{8 - 6x + x^{2}}{18 - 7x - x^{2}}} = \frac{13}{6}\). \\
B1. \(P(x + 2) + P(x - 1) = - 2x^{2} - 2x + 7\) bolsa, \(P(x)\) ni \(x + 4\) ga bólgendegi qaldıqni tabıń. \\
B2. Tómendegi aytımdı qálegen natural san ushın matematikalıq induksiya metodi járdeminde dálilleń: \(1^{3} + 2^{3} + 3^{3} + ... + n^{3} = \left( \frac{n(n + 1)}{2} \right)^{2}\); \\
B3. Tómendegi aytımdı qálegen natural san ushın matematikalıq induksiya metodi járdeminde dálilleń: \(5^{n} - 4n + 15\) sanı 16 ga eseli ; \\
C1. Birdeylikti dálilleń: \(C_{n + k}^{j + k} = \sum_{s = 0}^{k}C_{n}^{j + s}C_{k}^{s}\); \\
C2. Úshmúyeshliktiń ishida olińan noqatdan Onıń táreplariga parallel tuwrı sızıqlar túsirilgen. Ular úshmúyeshlikni 6 bólekga bóledi. Eger payda bolǵan úshmúyeshliklarnıń maydanları \(S_{1},S_{2}\) hám \(S_{3}\) bolsa, berilgan úshmúyeshlik maydanın tabıń. \\
C3. Eger a,b - oń sanlar bolsa, tómendegi teńsizlikti dálilleń: \(\sqrt[3]{\frac{a}{b}} + \sqrt[3]{\frac{b}{a}} \leq \sqrt[3]{2(a + b)\left( \frac{1}{a} + \frac{1}{b} \right)}\) \\

\end{tabular}
\vspace{1cm}


\begin{tabular}{m{17cm}}
\textbf{31-variant}
\newline

T1. Fales teoreması hám onıń qollanılıwı. \\
T2. Mına \(P(x) = x^{5} + 11x^{4} + 37x^{3} + 35x^{2} - 44x - 40\) kóp aǵzalısı \(Q(x) = x^{2} + 3x + 2\) kóp aǵzalısına qaldıqsız bólinedime? \\
A1. Teńsizlikti sheshiń: \(\sqrt{x^{2} - 4x} > x - 3\). \\
A2. Teńlemeni sheshiń. \(\sqrt{x^{2} + x + 4} + \sqrt{x^{2} + x + 1} = \sqrt{2x^{2} + 2x + 9}\). \\
A3. Teńlemeni sheshiń \((x + 1)^{5} + (x - 1)^{5} = 32x\). \\
B1. \(P(x + n) = (x + n)^{3} + (x - n)^{2} + x + n + 6\) kópaǵzalıi berilgan. \(P(x)\) kópaǵzalıi \(x - n\) ga qaldıqsiz bólinse, \(n\) ni tabıń. \\
B2. Tómendegi aytımdı qálegen natural san ushın matematikalıq induksiya metodi járdeminde dálilleń: \(1 \cdot 1! + 2 \cdot 2! + 3 \cdot 3! + \ldots + n \cdot n! = (n + 1)! - 1\). \\
B3. Tómendegi aytımdı qálegen natural san ushın matematikalıq induksiya metodi járdeminde dálilleń:\(7^{n} - 1\) sanı 6 ga eseli; \\
C1. \(\frac{1}{C_{4}^{n}} = \frac{1}{C_{5}^{n}} + \frac{1}{C_{6}^{n}}\) bolsa, \(n\) ni tabıń \\
C2. Teń qaptallı úshmúyeshliktiń qaptal tárepi 13 sm , qaptal tárepine túsirilgen biyiklik 5 sm ga teń. Úshmúyeshlik ultaninıń uzunligi tabılsın. \\
C3. Ya. Bernulli teńsizligi. Eger\(x \geq - 1\) bolsa, onda qálegen natural \(n\) sanı ushın \((1 + x)^{n} \geq 1 + nx\) teńsizlik orinli bolıwın dálilleń. \\

\end{tabular}
\vspace{1cm}


\begin{tabular}{m{17cm}}
\textbf{32-variant}
\newline

T1. Simmetriyalıq kóp aǵzalılar. \\
T2. Bezu teoreması hám onıń qollanılıwı. \\
A1. Teńlemeni sheshiń. \(\sqrt[3]{x} + \sqrt[3]{x - 16} = \sqrt[3]{x - 8}\). \\
A2. Teńlemeni sheshiń. \(\sqrt[3]{x - 1} + \sqrt[3]{x - 2} - \sqrt{2x - 3} = 0\). \\
A3. Teńsizlikti sheshiń: \(\frac{x^{3} + 3x^{2} - x - 3}{x^{2} + 3x - 10} < 0\). \\
B1. \(P(2x - 1) + P(x - 1) = 10x^{2} - 12x + 2\) bolsa, \(P(x)\) ni tabıń. \\
B2. Tómendegi aytımdı qálegen natural san ushın matematikalıq induksiya metodi járdeminde dálilleń: \(\frac{1}{4 \cdot 5} + \frac{1}{5 \cdot 6} + \frac{1}{6 \cdot 7} + \ldots + \frac{1}{(n + 3) \cdot (n + 4)} = \frac{n}{4 \cdot (n + 4)}\). \\
B3. Tómendegi aytımdı qálegen natural san ushın matematikalıq induksiya metodi járdeminde dálilleń: \(5^{2n + 1} + 3^{n + 2} \cdot 2^{n - 1}\) sanı 19 ga eseli ; \\
C1. Birdeylikti dálilleń: \(C_{n + 1}^{j + 1} = C_{n}^{j} + C_{n}^{j + 1}\); \\
C2. Úshmúyeshliktiń ultaniga túsirilgen biyikligi \(h\) ga teń. Úshmúyeshliktiń ultaniga parallel kesindi úshmúyeshliktiń maydanıni teń ekiga bóledi. Úshmúyeshliktiń ushınan usı kesindige shekem bolǵan aralıq tabılsın. \\
C3. Durıs úshmúyeshliktiń tárepi a ga teń. Tárepini diametr deb esaplap dóńgelek jasalǵan. Úshmúyeshliktiń usı dóńgelekten sirtindaǵi bólegi maydanın tabıń. \\

\end{tabular}
\vspace{1cm}


\begin{tabular}{m{17cm}}
\textbf{33-variant}
\newline

T1. \(n\) dárejeniń qanday mánislerinde \((x + 1)^{n} + (x - 1)^{n}\) ańlatpası \(x\) ańlatpaǵa qaldıqsız bólinedi? \\
T2. \(x\) ózgeriwshiniń qálegen pútin mánisinde \(ax^{2} + bx + c\) ush aǵzalısınıń mánisi pútin bolıwı ushın \(2a,\ a + b\) hám \(c\) sanlarınıń pútin bolıwı zárurli hám jetkilikli ekenligin dálilleń. \\
A1. Teńlemeni sheshiń. \(\sqrt{3x^{2} - 2x + 15} + \sqrt{3x^{2} - 2x + 8} = 7\). \\
A2. Teńlemeni sheshiń \(\left( x^{2} - 6x \right)^{2} - 2(x - 3)^{2} = 81\). \\
A3. Teńlemeni sheshiń. \(\sqrt{x} + \frac{2x + 1}{x + 2} = 2\). \\
B1. \(P(x + 3) = x^{2} - x + n\) bolsa. \(P(x - 2)\) kópaǵzalını \(x - 3\) ga bólgende qaldıq \(10\) bolsa, \(n\) ni tabıń. \\
B2. Tómendegi aytımdı qálegen natural san ushın matematikalıq induksiya metodi járdeminde dálilleń: \(1 \cdot 2 + 2 \cdot 3 + 3 \cdot 4 + \ldots + n \cdot (n + 1) = \frac{n \cdot (n + 1) \cdot (n + 2)}{3}\). \\
B3. Tómendegi aytımdı qálegen natural san ushın matematikalıq induksiya metodi járdeminde dálilleń: \(6^{2n - 2} + 3^{n + 1} + 3^{n - 1}\) sanı 11 eseli ; \\
C1. Birdeylikti dálilleń:\(\sum_{j = 0}^{n}C_{n}^{j}( - 1)^{j} = 0\); \\
C2. Úshmúyeshliktiń ultaniga túsirilgen biyikligi \(h\) ga teń. Úshmúyeshliktiń ultaniga parallel kesindi úshmúyeshliktiń maydanıni teń ekiga bóledi. Úshmúyeshliktiń ushınan usı kesindige shekem bolǵan aralıq tabılsın. \\
C3. Eger a,b,c - oń sanlar bolsa, tómendegi teńsizlikti dálilleń: \(\sqrt{\mathbf{a}^{\mathbf{2}}\mathbf{+ ab +}\mathbf{b}^{\mathbf{2}}}\mathbf{+}\sqrt{\mathbf{b}^{\mathbf{2}}\mathbf{+ bc +}\mathbf{c}^{\mathbf{2}}}\mathbf{>}\sqrt{\mathbf{a}^{\mathbf{2}}\mathbf{+ ac +}\mathbf{c}^{\mathbf{2}}}\) \\

\end{tabular}
\vspace{1cm}


\begin{tabular}{m{17cm}}
\textbf{34-variant}
\newline

T1. Qosındısı berge teń bolǵan \(x,y,z\) oń sanları ushın \(\frac{1}{x} + \frac{1}{y} + \frac{1}{z} \geq 9\) teńsizligi orınlı bolıwın dálilleń. \\
T2. Matematikalıq induksiya metodı hám onıń qollanılıwına mısallar. \\
A1. Teńsizlikti sheshiń: \(\sqrt{x + 3} + \sqrt{x - 2} - \sqrt{2x + 4} > 0\). \\
A2. Teńlemeni sheshiń \(\left( x^{2} + 10x + 10 \right)\left( x^{2} + x + 10 \right) = 10x^{2}\) . \\
A3. Teńlemeni sheshiń. \(\frac{z}{z + 1} - 2\sqrt{\frac{z + 1}{2}} = 3\). \\
B1. \(P(x) = x^{33} - 2ax^{21} + x^{8} + 8\) kópaǵzalıi berilgan. \(a\) nıń qaysi qiymati ushın \(P(x)\) kópaǵzalıi \(x + 1\) ga qaldıqsiz bóliadi? \\
B2. Tómendegi aytımdı qálegen natural san ushın matematikalıq induksiya metodi járdeminde dálilleń: \(\frac{1}{1 \cdot 5} + \frac{1}{5 \cdot 9} + ... + \frac{1}{(4n - 3)(4n + 1)} = \frac{n}{4n + 1}\); \\
B3. Tómendegi aytımdı qálegen natural san ushın matematikalıq induksiya metodi járdeminde dálilleń: \(5^{n + 2} + 26 \cdot 5^{n} + 8^{2n + 1}\) sanı 59 ga eseli; \\
C1. \(\left( \sqrt{x} + \frac{1}{\sqrt[3]{x^{2}}} \right)^{n}\) binom jayılmasında 5-aǵza koeffitsiyentinıń 3-aǵza koeffitsiyentine qatnasi 7:2 ga teń. \(x\) nıń darajasi 1 ga teń bolǵan aǵzasın tabıń. \\
C2. Tuwrı múyeshli úshmúyeshlikda katetlar 7 sm hám 24 sm ga teń. Tuwrı múyeshnıń bissektrisasi túsirilgen. Bu bissektrisa gipotenuzani qanday uzunlikdagi kesindilerga ajıratadı? \\
C3. Eger a,b - oń sanlar bolsa, tómendegi teńsizlikti dálilleń: \(\sqrt[3]{\frac{a}{b}} + \sqrt[3]{\frac{b}{a}} \leq \sqrt[3]{2(a + b)\left( \frac{1}{a} + \frac{1}{b} \right)}\) \\

\end{tabular}
\vspace{1cm}


\begin{tabular}{m{17cm}}
\textbf{35-variant}
\newline

T1. \(2^{81} + 1\) sanı 9 sanına qaldıqsız bóliniwin dálilleń. \\
T2. \(P(x) = (x - 1)^{20}\left( x^{2} + 25 \right)\) kóp aǵzalisınıń koefficentleri qosındısın tabıń. \\
A1. Teńlemeni sheshiń. \(\sqrt[3]{x - 1} + \sqrt[3]{x - 2} - \sqrt{2x - 3} = 0\). \\
A2. Teńlemeni sheshiń. \((x - 4)^{3} + (x - 4)^{2} + (x - 4)(x - 3) + (x - 3)^{2} + (x - 3)^{3} = 6\). \\
A3. Teńlemeni sheshiń. \(\frac{4x}{x^{2} + x + 3} + \frac{5x}{x^{2} - 5x + 3} = - \frac{3}{2}\). \\
B1. \(P(x + 3)\) kópaǵzalını \(x + 1\) ga bólgende qaldıq -3, \(Q(2x - 1)\) kópaǵzalını \(x - 1\)ga bólgende qaldıq 2 bolsa, \(P(x + 4) + x^{2}Q(x + 3)\) kópaǵzalını \(x + 2\) ga bólgendegi qaldıqni tabıń. \\
B2. Tómendegi aytımdı qálegen natural san ushın matematikalıq induksiya metodi járdeminde dálilleń: \(1^{3} + 2^{3} + 3^{3} + ... + n^{3} = \left( \frac{n(n + 1)}{2} \right)^{2}\); \\
B3. Tómendegi aytımdı qálegen natural san ushın matematikalıq induksiya metodi járdeminde dálilleń: \(n\left( 2n^{2} - 3n + 1 \right)\) sanı 6 ga eseli ; \\
C1. \(\left( x^{3} - \frac{3}{x^{2}} \right)^{10}\) binom jayılmasinıń \(x\) qatnashmagan aǵzasın tabıń. \\
C2. Eki birdey radiusli dóńgeleklar sonday jaylasqan, olardıń orayları arasındaǵı aralıq radiusqa teń. Dóńgeleklar kesilisken bólegi maydanınıń, kesilisken bólegine ishley sızılǵan kvadrat maydanına qatnasin tabıń. \\
C3. \(R\) radiusli dóńgelekga bitta umumiy uchga ega bolǵan durıs úshmúyeshlik hám kvadrat ishley sızılǵan. Olardıń kesilisken bóleginıń maydanıni tabıń. \\

\end{tabular}
\vspace{1cm}


\begin{tabular}{m{17cm}}
\textbf{36-variant}
\newline

T1. \(b\) parametriniń qanday mánisinde \(x^{3} + 17x^{2} + bx - 17 = 0\) teńlemesiniń korenleri pútin sanlardan turadı? \\
T2. Qálegen \(a\) parametri hám \(x\) ushın \(x(a - x) \leq a^{2}/4\) teńsizligi orınlı bolıwın dálilleń. \\
A1. Teńlemeni sheshiń. \(\sqrt{x} + \frac{2x + 1}{x + 2} = 2\). \\
A2. Teńsizlikti sheshiń: \(\sqrt{x^{2} - 4x} > x - 3\). \\
A3. Teńsizlikti sheshiń:\(x^{2}\left( x^{4} + 36 \right) - 6\sqrt{3}\left( x^{4} + 4 \right) < 0\). \\
B1. \(P(x + 1) + P(x - 3) = 2x^{2} - 10x + 16\) bolsa, \(P(x)\) ni tabıń. \\
B2. Tómendegi aytımdı qálegen natural san ushın matematikalıq induksiya metodi járdeminde dálilleń: \(1 \cdot 2 + 2 \cdot 3 + 3 \cdot 4 + ... + n(n + 1) = \frac{n(n + 1)(n + 2)}{3}\); \\
B3. Tómendegi aytımdı qálegen natural san ushın matematikalıq induksiya metodi járdeminde dálilleń: \(2n^{3} + 3n^{2} + 7n\) sanı 6 ga eseli ; \\
C1. Teńsizlikti sheshiń \(5C_{x}^{3} < C_{x + 2}^{4}\), \(x \in N\) \\
C2. \emph{ABC} úshmúyeshliktiń \emph{AB} tárepinda jaylasqan \(N\) noqatdan \(NQ\| AC\) hám \(NP\| BC\) tuwrı sızıqlar túsirilgen. Eger \emph{BNQ} úshmúyeshliktiń maydanı \(S_{1}\) ga, \emph{ANP} úshmúyeshliktiń maydanı \(S_{2}\) ga teńligi ma'lum bolsa, \emph{ABC} úshmúyeshliktiń maydanıni tabıń. \\
C3. Eger \(S\) úshmúyeshliktiń maydanı, \(b\) hám \(c\) onıń táreplari bolsa, \(S \leq \frac{b^{2} + c^{2}}{4}\) bolıwın dálilleń. \\

\end{tabular}
\vspace{1cm}


\begin{tabular}{m{17cm}}
\textbf{37-variant}
\newline

T1. \(a\) parametriniń qanday mánisinde \(P(x) = x^{2017} + ax - 5\) kóp aǵzalısı \((x + 1)\) kóp aǵzalısına qaldıqsız bólinedi? \\
T2. Kombinatorika elementleri hám Nyuton binomı. \\
A1. Teńlemeni sheshiń. \(\sqrt[3]{x} + \sqrt[3]{x - 16} = \sqrt[3]{x - 8}\). \\
A2. Teńlemeni sheshiń. \((\sqrt{x + 1} + \sqrt{x})^{3} + (\sqrt{x + 1} + \sqrt{x})^{2} = 2\). \\
A3. Teńlemeni sheshiń \((x + 1)^{5} + (x - 1)^{5} = 32x\). \\
B1. \(P(x) = (x - 5)^{2n + 1} + (x - 1)^{2n + 3}\) kópaǵzalını \(x - 3\) ga bólgende qaldıq \(3 \cdot 2^{3n - 4}\) bolsa, \(n\) ni tabıń. \\
B2. Tómendegi aytımdı qálegen natural san ushın matematikalıq induksiya metodi járdeminde dálilleń: \(2^{2} + 6^{2} + \ldots + (4n - 2)^{2} = \frac{4n(2n - 1)(2n + 1)}{3}\). \\
B3. Tómendegi aytımdı qálegen natural san ushın matematikalıq induksiya metodi járdeminde dálilleń: \(5 \cdot 2^{3n - 2} + 3^{3n - 1}\) sanı 19 ga eseli \\
C1. Birdeylikti dálilleń:\(C_{n}^{j} = C_{n}^{n - j}\); \\
C2. Ultanlari \(x\) hám 3 bolǵan trapetsiyada diagonallar órtalari arasındaǵı aralıqni \(x\) nıń funksiyasi sifatida ańlatpalań. \(x\) nіnń qanday qiymatida bu aralıq 1 ga teń bóledi? \\
C3. Eger a,b - oń sanlar bolsa, tómendegi teńsizlikti dálilleń: \(\sqrt[3]{\frac{a}{b}} + \sqrt[3]{\frac{b}{a}} \leq \sqrt[3]{2(a + b)\left( \frac{1}{a} + \frac{1}{b} \right)}\) \\

\end{tabular}
\vspace{1cm}


\begin{tabular}{m{17cm}}
\textbf{38-variant}
\newline

T1. \(P(x) = x^{6} - 3x^{5} + x^{4} - 6x^{2} + 2x - 6\) kóp aǵzalısınıń pútin korenlerin tabıń. \\
T2. Mına \(P(0) = 20\) hám \(P(1) = 100\) shártlerin qanaǵatlandıratuǵın \(P(x)\) kóp aǵzalısı bar bolama? \\
A1. Teńlemeni sheshiń. \(\sqrt{x^{2} + x + 4} + \sqrt{x^{2} + x + 1} = \sqrt{2x^{2} + 2x + 9}\). \\
A2. Teńsizlikti sheshiń: \(\frac{x^{3} + 3x^{2} - x - 3}{x^{2} + 3x - 10} < 0\). \\
A3. Teńlemeni sheshiń. \(\sqrt{3x^{2} - 2x + 15} + \sqrt{3x^{2} - 2x + 8} = 7\). \\
B1. \(P(x + 2) + P(x - 1) = - 2x^{2} - 2x + 7\) bolsa, \(P(x)\) ni \(x + 4\) ga bólgendegi qaldıqni tabıń. \\
B2. Tómendegi aytımdı qálegen natural san ushın matematikalıq induksiya metodi járdeminde dálilleń: \(\frac{1}{1 \cdot 4} + \frac{1}{4 \cdot 7} + \frac{1}{7 \cdot 10} + \ldots + \frac{1}{(3n - 2) \cdot (3n + 1)} = \frac{n}{(3n + 1)}\). \\
B3. Tómendegi aytımdı qálegen natural san ushın matematikalıq induksiya metodi járdeminde dálilleń: \(5^{n + 2} + 26 \cdot 5^{n} + 8^{2n + 1}\) sanı 59 ga eseli; \\
C1. \(C_{n + 4}^{n + 1} - C_{n + 3}^{n} = 15(n + 2)\) bolsa, \(n\) ni tabıń. \\
C2. \emph{ABC} úshmúyeshliktiń \(B\) uchidan \emph{AC} tárepiga \emph{BD} kesindi ótkazildi. \emph{BD} kesindi bu úshmúyeshliktiń maydanıni teń ekige bóledi. Eger\(AC = a\) bolsa, \emph{AD} hám \emph{DC} kesindilarnıń uzınlıqlarıni tabıń. \\
C3. Eger \(S\) úshmúyeshliktiń maydanı, \(b\) hám \(c\) onıń táreplari bolsa, \(S \leq \frac{b^{2} + c^{2}}{4}\) bolıwın dálilleń. \\

\end{tabular}
\vspace{1cm}


\begin{tabular}{m{17cm}}
\textbf{39-variant}
\newline

T1. \(P(x) = (x - 1)^{20}\left( x^{2} + 25 \right)\) kóp aǵzalisınıń koefficentleri qosındısın tabıń. \\
T2. \(2^{81} + 1\) sanı 9 sanına qaldıqsız bóliniwin dálilleń. \\
A1. Teńlemeni sheshiń \(\left( x^{2} - 6x \right)^{2} - 2(x - 3)^{2} = 81\). \\
A2. Teńlemeni sheshiń. \(\sqrt{x + 8 + 2\sqrt{x + 7}} + \sqrt{x + 1 - \sqrt{x + 7}} = 4\). \\
A3. Teńlemeni sheshiń \((x + 4)(x + 1) - 3\sqrt{x^{2} + 5x + 2} = 6\). \\
B1. \(P(x)\) kópaǵzalını \(3x^{2} - 4x + 1\) ga bólgenimizdeqaldıq \(6x - 11\) bolsa, \(P(x)\) kópaǵzalını \(3x - 1\)ga bólgende qaldıqni tabıń. \\
B2. Tómendegi aytımdı qálegen natural san ushın matematikalıq induksiya metodi járdeminde dálilleń: \(\frac{1}{4 \cdot 5} + \frac{1}{5 \cdot 6} + \frac{1}{6 \cdot 7} + \ldots + \frac{1}{(n + 3) \cdot (n + 4)} = \frac{n}{4 \cdot (n + 4)}\). \\
B3. Tómendegi aytımdı qálegen natural san ushın matematikalıq induksiya metodi járdeminde dálilleń: \(6^{2n - 2} + 3^{n + 1} + 3^{n - 1}\) sanı 11 eseli ; \\
C1. Birdeylikti dálilleń:\(C_{n + 2}^{j + 2} = C_{n}^{j} + 2C_{n}^{j + 1} + C_{n}^{j + 2}\); \\
C2. Tuwrı múyeshli úshmúyeshliktiń katetlari \(b\) hám \(c\) ga teń. Tuwrı múyesh bissektrisasinıń uzunligi tabılsın. \\
C3. Ya. Bernulli teńsizligi. Eger\(x \geq - 1\) bolsa, onda qálegen natural \(n\) sanı ushın \((1 + x)^{n} \geq 1 + nx\) teńsizlik orinli bolıwın dálilleń. \\

\end{tabular}
\vspace{1cm}


\begin{tabular}{m{17cm}}
\textbf{40-variant}
\newline

T1. Qálegen \(a\) parametri hám \(x\) ushın \(x(a - x) \leq a^{2}/4\) teńsizligi orınlı bolıwın dálilleń. \\
T2. \(a\) parametriniń qanday mánisinde \(P(x) = x^{2017} + ax - 5\) kóp aǵzalısı \((x + 1)\) kóp aǵzalısına qaldıqsız bólinedi? \\
A1. Teńsizlikti sheshiń: \(\sqrt{x + 3} + \sqrt{x - 2} - \sqrt{2x + 4} > 0\). \\
A2. Teńlemeni sheshiń \(\sqrt{\frac{18 - 7x - x^{2}}{8 - 6x + x^{2}}} + \sqrt{\frac{8 - 6x + x^{2}}{18 - 7x - x^{2}}} = \frac{13}{6}\). \\
A3. Teńlemeni sheshiń \(\left( x^{2} - 4x + 6 \right)^{2} - 4\left( x^{2} - 4x + 6 \right) + 6 = x\). \\
B1. \(P(x) = x^{4} - 2x + 2^{n + 1}\) kópaǵzalını \(x - 2^{n}\) ga bólgende qaldıq \(2^{n - 2}\) bolsa, \(n\) ni tabıń. \\
B2. Tómendegi aytımdı qálegen natural san ushın matematikalıq induksiya metodi járdeminde dálilleń: \(1 \cdot 1! + 2 \cdot 2! + 3 \cdot 3! + \ldots + n \cdot n! = (n + 1)! - 1\). \\
B3. Tómendegi aytımdı qálegen natural san ushın matematikalıq induksiya metodi járdeminde dálilleń: \(2n^{3} + 3n^{2} + 7n\) sanı 6 ga eseli ; \\
C1. Teńsizlikti sheshiń: \(C_{10}^{x - 1} > 2C_{10}^{x}\) \\
C2. Teń qaptallı \(ABC(AB = BC)\) úshmúyeshlikda \emph{AD} bissektrisa túsirilgen. Eger\(S_{ABD} = S_{1},S_{\bigtriangleup ADC} = S_{2}\) bolsa, \emph{AC} ni tabıń. \\
C3. \(R\) radiusli dóńgelekga bitta umumiy uchga ega bolǵan durıs úshmúyeshlik hám kvadrat ishley sızılǵan. Olardıń kesilisken bóleginıń maydanıni tabıń. \\

\end{tabular}
\vspace{1cm}


\begin{tabular}{m{17cm}}
\textbf{41-variant}
\newline

T1. Haqıyqıy \(a_{1},\ a_{2},\ .\ .\ .\ ,\ a_{n},\ b_{1},\ b_{2},\ .\ .\ .\ ,\ b_{n}\) sanları ushın \(\left( a_{1}b_{1} + a_{2}b_{2} + \ .\ .\ .\  + a_{n}b_{n} \right)^{2} \leq \left( a_{1}^{2} + a_{2}^{2} + \ .\ .\ .\  + a_{n}^{2} \right)\left( b_{1}^{2} + b_{2}^{2} + \ .\ .\ .\  + b_{n}^{2} \right)\) Koshi teńsizligin dálilleń. \\
T2. \(n\) dárejeniń qanday mánislerinde \((x + 1)^{n} + (x - 1)^{n}\) ańlatpası \(x\) ańlatpaǵa qaldıqsız bólinedi? \\
A1. Teńsizlikti sheshiń: \(\sqrt{x + 3} + \sqrt{x - 2} - \sqrt{2x + 4} > 0\). \\
A2. Teńlemeni sheshiń. \((x - 4)^{3} + (x - 4)^{2} + (x - 4)(x - 3) + (x - 3)^{2} + (x - 3)^{3} = 6\). \\
A3. Teńlemeni sheshiń \(\left( x^{2} - 6x \right)^{2} - 2(x - 3)^{2} = 81\). \\
B1. \(P(x + 2) + P(x - 1) = - 2x^{2} - 2x + 7\) bolsa, \(P(x)\) ni \(x + 4\) ga bólgendegi qaldıqni tabıń. \\
B2. Tómendegi aytımdı qálegen natural san ushın matematikalıq induksiya metodi járdeminde dálilleń: \(1^{2} + 3^{2} + 5^{2} + ... + (2n - 1)^{2} = \frac{n\left( 4n^{2} - 1 \right)}{3}\); \\
B3. Tómendegi aytımdı qálegen natural san ushın matematikalıq induksiya metodi járdeminde dálilleń: \(5^{2n + 1} + 3^{n + 2} \cdot 2^{n - 1}\) sanı 19 ga eseli ; \\
C1. \(\left( \sqrt{x} + \frac{1}{\sqrt[3]{x^{2}}} \right)^{n}\) binom jayılmasında 5-aǵza koeffitsiyentinıń 3-aǵza koeffitsiyentine qatnasi 7:2 ga teń. \(x\) nıń darajasi 1 ga teń bolǵan aǵzasın tabıń. \\
C2. Teń qaptallı úshmúyeshliktiń qaptal tárepi 13 sm , qaptal tárepine túsirilgen biyiklik 5 sm ga teń. Úshmúyeshlik ultaninıń uzunligi tabılsın. \\
C3. Durıs úshmúyeshliktiń tárepi a ga teń. Tárepini diametr deb esaplap dóńgelek jasalǵan. Úshmúyeshliktiń usı dóńgelekten sirtindaǵi bólegi maydanın tabıń. \\

\end{tabular}
\vspace{1cm}


\begin{tabular}{m{17cm}}
\textbf{42-variant}
\newline

T1. Mına \(P(0) = 20\) hám \(P(1) = 100\) shártlerin qanaǵatlandıratuǵın \(P(x)\) kóp aǵzalısı bar bolama? \\
T2. Mına \(P(x) = x^{5} + 11x^{4} + 37x^{3} + 35x^{2} - 44x - 40\) kóp aǵzalısı \(Q(x) = x^{2} + 3x + 2\) kóp aǵzalısına qaldıqsız bólinedime? \\
A1. Teńlemeni sheshiń. \(\sqrt[3]{x - 1} + \sqrt[3]{x - 2} - \sqrt{2x - 3} = 0\). \\
A2. Teńlemeni sheshiń. \(\frac{z}{z + 1} - 2\sqrt{\frac{z + 1}{2}} = 3\). \\
A3. Teńlemeni sheshiń \((x + 4)(x + 1) - 3\sqrt{x^{2} + 5x + 2} = 6\). \\
B1. \(P(x + 3) = x^{2} - x + n\) bolsa. \(P(x - 2)\) kópaǵzalını \(x - 3\) ga bólgende qaldıq \(10\) bolsa, \(n\) ni tabıń. \\
B2. Tómendegi aytımdı qálegen natural san ushın matematikalıq induksiya metodi járdeminde dálilleń: \(1 \cdot 2 + 2 \cdot 3 + 3 \cdot 4 + \ldots + n \cdot (n + 1) = \frac{n \cdot (n + 1) \cdot (n + 2)}{3}\). \\
B3. Tómendegi aytımdı qálegen natural san ushın matematikalıq induksiya metodi járdeminde dálilleń: \(n\left( 2n^{2} - 3n + 1 \right)\) sanı 6 ga eseli ; \\
C1. Birdeylikti dálilleń:\(C_{n + 2}^{j + 2} = C_{n}^{j} + 2C_{n}^{j + 1} + C_{n}^{j + 2}\); \\
C2. \(\bigtriangleup ABC\) da \(\angle A\) múyesh \(\angle B\) dan eki marta úlken bólib, \(AC = b,AB = c\). \emph{BC} tárepnıń uzunligi tabılsın. \\
C3. Eger a,b,c - oń sanlar bolsa, tómendegi teńsizlikti dálilleń: \(\sqrt{\mathbf{a}^{\mathbf{2}}\mathbf{+ ab +}\mathbf{b}^{\mathbf{2}}}\mathbf{+}\sqrt{\mathbf{b}^{\mathbf{2}}\mathbf{+ bc +}\mathbf{c}^{\mathbf{2}}}\mathbf{>}\sqrt{\mathbf{a}^{\mathbf{2}}\mathbf{+ ac +}\mathbf{c}^{\mathbf{2}}}\) \\

\end{tabular}
\vspace{1cm}


\begin{tabular}{m{17cm}}
\textbf{43-variant}
\newline

T1. \(x\) ózgeriwshiniń qálegen pútin mánisinde \(ax^{2} + bx + c\) ush aǵzalısınıń mánisi pútin bolıwı ushın \(2a,\ a + b\) hám \(c\) sanlarınıń pútin bolıwı zárurli hám jetkilikli ekenligin dálilleń. \\
T2. Qálegen \(a,b,c \in (0;1)\) sanları ushın \(a(1 - b) > 1/4,\ b(1 - c) > 1/4,\ c(1 - a) > 1/4\) teńsizlikleri bir waqıtta orınlı bola almaytuǵınlıǵin dálilleń. \\
A1. Teńsizlikti sheshiń:\(x^{2}\left( x^{4} + 36 \right) - 6\sqrt{3}\left( x^{4} + 4 \right) < 0\). \\
A2. Teńsizlikti sheshiń: \(\frac{x^{3} + 3x^{2} - x - 3}{x^{2} + 3x - 10} < 0\). \\
A3. Teńlemeni sheshiń. \(\sqrt{x} + \frac{2x + 1}{x + 2} = 2\). \\
B1. \(P(x)\) kópaǵzalını \(3x^{2} - 4x + 1\) ga bólgenimizdeqaldıq \(6x - 11\) bolsa, \(P(x)\) kópaǵzalını \(3x - 1\)ga bólgende qaldıqni tabıń. \\
B2. Tómendegi aytımdı qálegen natural san ushın matematikalıq induksiya metodi járdeminde dálilleń: \(1^{2} + 2^{2} + 3^{2} + ... + n^{2} = \frac{n(n + 1)(2n + 1)}{6}\); \\
B3. Tómendegi aytımdı qálegen natural san ushın matematikalıq induksiya metodi járdeminde dálilleń: \(5^{n} - 4n + 15\) sanı 16 ga eseli ; \\
C1. Birdeylikti dálilleń:\(\sum_{j = 0}^{n}C_{n}^{j}( - 1)^{j} = 0\); \\
C2. Egerteń qaptallı úshmúyeshliktiń perimetri 32 dm , orta sızıǵı 6 dm ga teń bolsa, Onıń táreplari uzınlıqları tabılsın. \\
C3. Durıs úshmúyeshliktiń tárepi a ga teń. Tárepini diametr deb esaplap dóńgelek jasalǵan. Úshmúyeshliktiń usı dóńgelekten sirtindaǵi bólegi maydanın tabıń. \\

\end{tabular}
\vspace{1cm}


\begin{tabular}{m{17cm}}
\textbf{44-variant}
\newline

T1. Bezu teoreması hám onıń qollanılıwı. \\
T2. Fales teoreması hám onıń qollanılıwı. \\
A1. Teńlemeni sheshiń \(\left( x^{2} - 4x + 6 \right)^{2} - 4\left( x^{2} - 4x + 6 \right) + 6 = x\). \\
A2. Teńlemeni sheshiń \(\left( x^{2} + 10x + 10 \right)\left( x^{2} + x + 10 \right) = 10x^{2}\) . \\
A3. Teńlemeni sheshiń. \((\sqrt{x + 1} + \sqrt{x})^{3} + (\sqrt{x + 1} + \sqrt{x})^{2} = 2\). \\
B1. \(P(x) = x^{33} - 2ax^{21} + x^{8} + 8\) kópaǵzalıi berilgan. \(a\) nıń qaysi qiymati ushın \(P(x)\) kópaǵzalıi \(x + 1\) ga qaldıqsiz bóliadi? \\
B2. Tómendegi aytımdı qálegen natural san ushın matematikalıq induksiya metodi járdeminde dálilleń: \(\left( 1 - \frac{1}{4} \right)\left( 1 - \frac{1}{9} \right)...\left( 1 - \frac{1}{n^{2}} \right) = \frac{n + 1}{2n}\), \(n \geq 2\) \\
B3. Tómendegi aytımdı qálegen natural san ushın matematikalıq induksiya metodi járdeminde dálilleń:\(7^{n} - 1\) sanı 6 ga eseli; \\
C1. \(5C_{n}^{3} = C_{n + 2}^{4}\) bolsa, \(n\) ni tabıń. \\
C2. \(\bigtriangleup ABC\) da \(AB = 3sm,AC = 5sm,\angle BAC = 120^{{^\circ}}.BD\) bissektrisanıń uzunligi tabılsın. \\
C3. Eger \(S\) úshmúyeshliktiń maydanı, \(b\) hám \(c\) onıń táreplari bolsa, \(S \leq \frac{b^{2} + c^{2}}{4}\) bolıwın dálilleń. \\

\end{tabular}
\vspace{1cm}


\begin{tabular}{m{17cm}}
\textbf{45-variant}
\newline

T1. \(P(x) = x^{6} - 3x^{5} + x^{4} - 6x^{2} + 2x - 6\) kóp aǵzalısınıń pútin korenlerin tabıń. \\
T2. Qosındısı berge teń bolǵan \(x,y,z\) oń sanları ushın \(\frac{1}{x} + \frac{1}{y} + \frac{1}{z} \geq 9\) teńsizligi orınlı bolıwın dálilleń. \\
A1. Teńlemeni sheshiń \(\sqrt{\frac{18 - 7x - x^{2}}{8 - 6x + x^{2}}} + \sqrt{\frac{8 - 6x + x^{2}}{18 - 7x - x^{2}}} = \frac{13}{6}\). \\
A2. Teńlemeni sheshiń. \(\frac{4x}{x^{2} + x + 3} + \frac{5x}{x^{2} - 5x + 3} = - \frac{3}{2}\). \\
A3. Teńlemeni sheshiń. \(\sqrt{3x^{2} - 2x + 15} + \sqrt{3x^{2} - 2x + 8} = 7\). \\
B1. \(P(x) = (x - 5)^{2n + 1} + (x - 1)^{2n + 3}\) kópaǵzalını \(x - 3\) ga bólgende qaldıq \(3 \cdot 2^{3n - 4}\) bolsa, \(n\) ni tabıń. \\
B2. Tómendegi aytımdı qálegen natural san ushın matematikalıq induksiya metodi járdeminde dálilleń: \(1 \cdot 1! + 2 \cdot 2! + 3 \cdot 3! + \ldots + n \cdot n! = (n + 1)! - 1\). \\
B3. Tómendegi aytımdı qálegen natural san ushın matematikalıq induksiya metodi járdeminde dálilleń: \(n^{3} + (n + 1)^{3} + (n + 2)^{3}\) sanı 9 ga eseli ; \\
C1. \((x + 1)^{3} + (x + 1)^{4} + (x + 1)^{5} + ... + (x + 1)^{10}\) ańlatpada \(x^{3}\) aldıńda ǵı koeffitsiyentti tabıń \\
C2. Bir burchagi \(60^{{^\circ}}\) bolǵan úshmúyeshlikka ishley sızılǵan sheńberdiń uriniw noqati shu múyeshke qarama- qarama-qarsı tárepini \(a\) hám \(b\) kesindilerga ajıratadı. Úshmúyeshlik maydanıni tabıń. \\
C3. \(R\) radiusli dóńgelekga bitta umumiy uchga ega bolǵan durıs úshmúyeshlik hám kvadrat ishley sızılǵan. Olardıń kesilisken bóleginıń maydanıni tabıń. \\

\end{tabular}
\vspace{1cm}


\begin{tabular}{m{17cm}}
\textbf{46-variant}
\newline

T1. \(b\) parametriniń qanday mánisinde \(x^{3} + 17x^{2} + bx - 17 = 0\) teńlemesiniń korenleri pútin sanlardan turadı? \\
T2. Matematikalıq induksiya metodı hám onıń qollanılıwına mısallar. \\
A1. Teńlemeni sheshiń. \(\sqrt{x^{2} + x + 4} + \sqrt{x^{2} + x + 1} = \sqrt{2x^{2} + 2x + 9}\). \\
A2. Teńsizlikti sheshiń: \(\sqrt{x^{2} - 4x} > x - 3\). \\
A3. Teńlemeni sheshiń. \(\sqrt[3]{x} + \sqrt[3]{x - 16} = \sqrt[3]{x - 8}\). \\
B1. \(P(x + 1) + P(x - 3) = 2x^{2} - 10x + 16\) bolsa, \(P(x)\) ni tabıń. \\
B2. Tómendegi aytımdı qálegen natural san ushın matematikalıq induksiya metodi járdeminde dálilleń: \(\frac{1}{1 \cdot 4} + \frac{1}{4 \cdot 7} + \frac{1}{7 \cdot 10} + \ldots + \frac{1}{(3n - 2) \cdot (3n + 1)} = \frac{n}{(3n + 1)}\). \\
B3. Tómendegi aytımdı qálegen natural san ushın matematikalıq induksiya metodi járdeminde dálilleń: \(6^{2n - 2} + 3^{n + 1} + 3^{n - 1}\) sanı 11 eseli ; \\
C1. \(\left( 2x^{\ ^{2}} - \frac{b}{2x^{3}} \right)^{10}\) binom jayılmasinıń \(x\) qatnashmagan aǵzasın tabıń. \\
C2. Tuwrı múyeshli úshmúyeshlikda katetlarnıń qatnasi 3:2 kabi, biyiklik bolsa gipotenuzani shunday eki kesindiga ajıratadı, olardan birinıń uzunligi ekinshisinen 2 ga úlken. Gipotenuzanıń uzunligi tabılsın. \\
C3. Eger a,b,c - oń sanlar bolsa, tómendegi teńsizlikti dálilleń: \(\sqrt{\mathbf{a}^{\mathbf{2}}\mathbf{+ ab +}\mathbf{b}^{\mathbf{2}}}\mathbf{+}\sqrt{\mathbf{b}^{\mathbf{2}}\mathbf{+ bc +}\mathbf{c}^{\mathbf{2}}}\mathbf{>}\sqrt{\mathbf{a}^{\mathbf{2}}\mathbf{+ ac +}\mathbf{c}^{\mathbf{2}}}\) \\

\end{tabular}
\vspace{1cm}


\begin{tabular}{m{17cm}}
\textbf{47-variant}
\newline

T1. Kombinatorika elementleri hám Nyuton binomı. \\
T2. Pifagor teoreması hám onıń dálilleniwleri. \\
A1. Teńlemeni sheshiń \((x + 1)^{5} + (x - 1)^{5} = 32x\). \\
A2. Teńlemeni sheshiń. \(\sqrt{x + 8 + 2\sqrt{x + 7}} + \sqrt{x + 1 - \sqrt{x + 7}} = 4\). \\
A3. Teńlemeni sheshiń. \(\sqrt{x^{2} + x + 4} + \sqrt{x^{2} + x + 1} = \sqrt{2x^{2} + 2x + 9}\). \\
B1. \(P(x) = x^{4} - 2x + 2^{n + 1}\) kópaǵzalını \(x - 2^{n}\) ga bólgende qaldıq \(2^{n - 2}\) bolsa, \(n\) ni tabıń. \\
B2. Tómendegi aytımdı qálegen natural san ushın matematikalıq induksiya metodi járdeminde dálilleń: \(1 \cdot 2 + 2 \cdot 3 + 3 \cdot 4 + \ldots + n \cdot (n + 1) = \frac{n \cdot (n + 1) \cdot (n + 2)}{3}\). \\
B3. Tómendegi aytımdı qálegen natural san ushın matematikalıq induksiya metodi járdeminde dálilleń: \(n\left( 2n^{2} - 3n + 1 \right)\) sanı 6 ga eseli ; \\
C1. Teńlemeni sheshiń \(\frac{C_{2x}^{x + 1}}{C_{2x + 1}^{x - 1}} = \frac{2}{3}\), \(x \in N\) \\
C2. \emph{ABC} úshmúyeshliktiń \emph{AC}, \emph{BC} hám \emph{AB} táreplarida \emph{CMPA}, \emph{BEFC} hám \emph{ADKB} kvadratlar jasalǵan. Eger\(AB = 13\), \(AC = 14,BC = 15\) ekanligi ma'lum bolsa, \emph{DKEFMP} altimúyeshliktıń maydanın tabıń. \\
C3. Eger a,b - oń sanlar bolsa, tómendegi teńsizlikti dálilleń: \(\sqrt[3]{\frac{a}{b}} + \sqrt[3]{\frac{b}{a}} \leq \sqrt[3]{2(a + b)\left( \frac{1}{a} + \frac{1}{b} \right)}\) \\

\end{tabular}
\vspace{1cm}


\begin{tabular}{m{17cm}}
\textbf{48-variant}
\newline

T1. Simmetriyalıq kóp aǵzalılar. \\
T2. Haqıyqıy \(a_{1},\ a_{2},\ .\ .\ .\ ,\ a_{n},\ b_{1},\ b_{2},\ .\ .\ .\ ,\ b_{n}\) sanları ushın \(\left( a_{1}b_{1} + a_{2}b_{2} + \ .\ .\ .\  + a_{n}b_{n} \right)^{2} \leq \left( a_{1}^{2} + a_{2}^{2} + \ .\ .\ .\  + a_{n}^{2} \right)\left( b_{1}^{2} + b_{2}^{2} + \ .\ .\ .\  + b_{n}^{2} \right)\) Koshi teńsizligin dálilleń. \\
A1. Teńsizlikti sheshiń: \(\sqrt{x^{2} - 4x} > x - 3\). \\
A2. Teńlemeni sheshiń. \(\sqrt{3x^{2} - 2x + 15} + \sqrt{3x^{2} - 2x + 8} = 7\). \\
A3. Teńlemeni sheshiń \(\sqrt{\frac{18 - 7x - x^{2}}{8 - 6x + x^{2}}} + \sqrt{\frac{8 - 6x + x^{2}}{18 - 7x - x^{2}}} = \frac{13}{6}\). \\
B1. \(P(x + n) = (x + n)^{3} + (x - n)^{2} + x + n + 6\) kópaǵzalıi berilgan. \(P(x)\) kópaǵzalıi \(x - n\) ga qaldıqsiz bólinse, \(n\) ni tabıń. \\
B2. Tómendegi aytımdı qálegen natural san ushın matematikalıq induksiya metodi járdeminde dálilleń: \(\left( 1 - \frac{1}{4} \right)\left( 1 - \frac{1}{9} \right)...\left( 1 - \frac{1}{n^{2}} \right) = \frac{n + 1}{2n}\), \(n \geq 2\) \\
B3. Tómendegi aytımdı qálegen natural san ushın matematikalıq induksiya metodi járdeminde dálilleń: \(5^{n + 2} + 26 \cdot 5^{n} + 8^{2n + 1}\) sanı 59 ga eseli; \\
C1. \(C_{n + 4}^{n + 1} - C_{n + 3}^{n} = 15(n + 2)\) bolsa, \(n\) ni tabıń. \\
C2. Teń qaptallı úshmúyeshlik ultanidagi múyesh \(\alpha\) ga teń. Shu múyesh uchidan ultanga \(\beta(\beta < \alpha)\) múyesh ostida tuwrı sızıq túsirilgen, u úshmúyeshlikni eki bólekga ajıratadı. Payda bolǵan úshmúyeshliklar maydanlarınıń qatnasini tabıń. \\
C3. Ya. Bernulli teńsizligi. Eger\(x \geq - 1\) bolsa, onda qálegen natural \(n\) sanı ushın \((1 + x)^{n} \geq 1 + nx\) teńsizlik orinli bolıwın dálilleń. \\

\end{tabular}
\vspace{1cm}


\begin{tabular}{m{17cm}}
\textbf{49-variant}
\newline

T1. Fales teoreması hám onıń qollanılıwı. \\
T2. Matematikalıq induksiya metodı hám onıń qollanılıwına mısallar. \\
A1. Teńlemeni sheshiń. \(\sqrt[3]{x - 1} + \sqrt[3]{x - 2} - \sqrt{2x - 3} = 0\). \\
A2. Teńsizlikti sheshiń: \(\sqrt{x + 3} + \sqrt{x - 2} - \sqrt{2x + 4} > 0\). \\
A3. Teńlemeni sheshiń \(\left( x^{2} + 10x + 10 \right)\left( x^{2} + x + 10 \right) = 10x^{2}\) . \\
B1. \(P(2x - 1) + P(x - 1) = 10x^{2} - 12x + 2\) bolsa, \(P(x)\) ni tabıń. \\
B2. Tómendegi aytımdı qálegen natural san ushın matematikalıq induksiya metodi járdeminde dálilleń: \(1^{3} + 2^{3} + 3^{3} + ... + n^{3} = \left( \frac{n(n + 1)}{2} \right)^{2}\); \\
B3. Tómendegi aytımdı qálegen natural san ushın matematikalıq induksiya metodi járdeminde dálilleń: \(5 \cdot 2^{3n - 2} + 3^{3n - 1}\) sanı 19 ga eseli \\
C1. \(\frac{1}{C_{4}^{n}} = \frac{1}{C_{5}^{n}} + \frac{1}{C_{6}^{n}}\) bolsa, \(n\) ni tabıń \\
C2. Tuwrı múyeshli úshmúyeshlik ótkir múyeshlarinıń bisєektrisalari AD hám BK \(AB^{2} = AD \cdot BK\) bolsa, úshmúyeshliktiń múyeshlarini tabıń. \\
C3. Eger a,b,c - oń sanlar bolsa, tómendegi teńsizlikti dálilleń: \(\sqrt{\mathbf{a}^{\mathbf{2}}\mathbf{+ ab +}\mathbf{b}^{\mathbf{2}}}\mathbf{+}\sqrt{\mathbf{b}^{\mathbf{2}}\mathbf{+ bc +}\mathbf{c}^{\mathbf{2}}}\mathbf{>}\sqrt{\mathbf{a}^{\mathbf{2}}\mathbf{+ ac +}\mathbf{c}^{\mathbf{2}}}\) \\

\end{tabular}
\vspace{1cm}


\begin{tabular}{m{17cm}}
\textbf{50-variant}
\newline

T1. \(n\) dárejeniń qanday mánislerinde \((x + 1)^{n} + (x - 1)^{n}\) ańlatpası \(x\) ańlatpaǵa qaldıqsız bólinedi? \\
T2. \(b\) parametriniń qanday mánisinde \(x^{3} + 17x^{2} + bx - 17 = 0\) teńlemesiniń korenleri pútin sanlardan turadı? \\
A1. Teńlemeni sheshiń. \((\sqrt{x + 1} + \sqrt{x})^{3} + (\sqrt{x + 1} + \sqrt{x})^{2} = 2\). \\
A2. Teńlemeni sheshiń. \(\sqrt{x + 8 + 2\sqrt{x + 7}} + \sqrt{x + 1 - \sqrt{x + 7}} = 4\). \\
A3. Teńlemeni sheshiń. \(\frac{z}{z + 1} - 2\sqrt{\frac{z + 1}{2}} = 3\). \\
B1. \(P(x + 3)\) kópaǵzalını \(x + 1\) ga bólgende qaldıq -3, \(Q(2x - 1)\) kópaǵzalını \(x - 1\)ga bólgende qaldıq 2 bolsa, \(P(x + 4) + x^{2}Q(x + 3)\) kópaǵzalını \(x + 2\) ga bólgendegi qaldıqni tabıń. \\
B2. Tómendegi aytımdı qálegen natural san ushın matematikalıq induksiya metodi járdeminde dálilleń: \(\frac{1}{4 \cdot 5} + \frac{1}{5 \cdot 6} + \frac{1}{6 \cdot 7} + \ldots + \frac{1}{(n + 3) \cdot (n + 4)} = \frac{n}{4 \cdot (n + 4)}\). \\
B3. Tómendegi aytımdı qálegen natural san ushın matematikalıq induksiya metodi járdeminde dálilleń: \(5^{n} - 4n + 15\) sanı 16 ga eseli ; \\
C1. Birdeylikti dálilleń: \(C_{n + k}^{j + k} = \sum_{s = 0}^{k}C_{n}^{j + s}C_{k}^{s}\); \\
C2. Úshmúyeshliktiń perimetri \(4,5dm\) ga teń, bissektrisa bolsa qarama-qarsı tárepni uzınlıqları 6 hám 9 sm ga teń bolǵan kesindilerga ajıratadı. Úshmúyeshliktiń táreplari tabılsın. \\
C3. Eger \(S\) úshmúyeshliktiń maydanı, \(b\) hám \(c\) onıń táreplari bolsa, \(S \leq \frac{b^{2} + c^{2}}{4}\) bolıwın dálilleń. \\

\end{tabular}
\vspace{1cm}


\begin{tabular}{m{17cm}}
\textbf{51-variant}
\newline

T1. \(P(x) = (x - 1)^{20}\left( x^{2} + 25 \right)\) kóp aǵzalisınıń koefficentleri qosındısın tabıń. \\
T2. Bezu teoreması hám onıń qollanılıwı. \\
A1. Teńlemeni sheshiń \((x + 4)(x + 1) - 3\sqrt{x^{2} + 5x + 2} = 6\). \\
A2. Teńlemeni sheshiń. \(\frac{4x}{x^{2} + x + 3} + \frac{5x}{x^{2} - 5x + 3} = - \frac{3}{2}\). \\
A3. Teńlemeni sheshiń. \((x - 4)^{3} + (x - 4)^{2} + (x - 4)(x - 3) + (x - 3)^{2} + (x - 3)^{3} = 6\). \\
B1. \(P(x + 3)\) kópaǵzalını \(x + 1\) ga bólgende qaldıq -3, \(Q(2x - 1)\) kópaǵzalını \(x - 1\)ga bólgende qaldıq 2 bolsa, \(P(x + 4) + x^{2}Q(x + 3)\) kópaǵzalını \(x + 2\) ga bólgendegi qaldıqni tabıń. \\
B2. Tómendegi aytımdı qálegen natural san ushın matematikalıq induksiya metodi járdeminde dálilleń: \(\frac{1}{1 \cdot 5} + \frac{1}{5 \cdot 9} + ... + \frac{1}{(4n - 3)(4n + 1)} = \frac{n}{4n + 1}\); \\
B3. Tómendegi aytımdı qálegen natural san ushın matematikalıq induksiya metodi járdeminde dálilleń: \(2n^{3} + 3n^{2} + 7n\) sanı 6 ga eseli ; \\
C1. Teńsizlikti sheshiń \(5C_{x}^{3} < C_{x + 2}^{4}\), \(x \in N\) \\
C2. Tuwrı múyeshli úshmúyeshliktiń tuwrı burchagi bissektrisasi shu uchdan túsirilgen mediana hám biyiklik arasındaǵı múyeshni ham teń ekige bóliniwini dálilleń. \\
C3. \(R\) radiusli dóńgelekga bitta umumiy uchga ega bolǵan durıs úshmúyeshlik hám kvadrat ishley sızılǵan. Olardıń kesilisken bóleginıń maydanıni tabıń. \\

\end{tabular}
\vspace{1cm}


\begin{tabular}{m{17cm}}
\textbf{52-variant}
\newline

T1. Pifagor teoreması hám onıń dálilleniwleri. \\
T2. Mına \(P(x) = x^{5} + 11x^{4} + 37x^{3} + 35x^{2} - 44x - 40\) kóp aǵzalısı \(Q(x) = x^{2} + 3x + 2\) kóp aǵzalısına qaldıqsız bólinedime? \\
A1. Teńsizlikti sheshiń: \(\frac{x^{3} + 3x^{2} - x - 3}{x^{2} + 3x - 10} < 0\). \\
A2. Teńlemeni sheshiń. \(\sqrt{x} + \frac{2x + 1}{x + 2} = 2\). \\
A3. Teńlemeni sheshiń \(\left( x^{2} - 6x \right)^{2} - 2(x - 3)^{2} = 81\). \\
B1. \(P(x) = x^{4} - 2x + 2^{n + 1}\) kópaǵzalını \(x - 2^{n}\) ga bólgende qaldıq \(2^{n - 2}\) bolsa, \(n\) ni tabıń. \\
B2. Tómendegi aytımdı qálegen natural san ushın matematikalıq induksiya metodi járdeminde dálilleń: \(1 \cdot 2 + 2 \cdot 3 + 3 \cdot 4 + ... + n(n + 1) = \frac{n(n + 1)(n + 2)}{3}\); \\
B3. Tómendegi aytımdı qálegen natural san ushın matematikalıq induksiya metodi járdeminde dálilleń: \(n^{3} + (n + 1)^{3} + (n + 2)^{3}\) sanı 9 ga eseli ; \\
C1. \(\left( x^{3} - \frac{3}{x^{2}} \right)^{10}\) binom jayılmasinıń \(x\) qatnashmagan aǵzasın tabıń. \\
C2. Úshmúyeshliktiń ishida olińan noqatdan Onıń táreplariga parallel tuwrı sızıqlar túsirilgen. Ular úshmúyeshlikni 6 bólekga bóledi. Eger payda bolǵan úshmúyeshliklarnıń maydanları \(S_{1},S_{2}\) hám \(S_{3}\) bolsa, berilgan úshmúyeshlik maydanın tabıń. \\
C3. Eger a,b - oń sanlar bolsa, tómendegi teńsizlikti dálilleń: \(\sqrt[3]{\frac{a}{b}} + \sqrt[3]{\frac{b}{a}} \leq \sqrt[3]{2(a + b)\left( \frac{1}{a} + \frac{1}{b} \right)}\) \\

\end{tabular}
\vspace{1cm}


\begin{tabular}{m{17cm}}
\textbf{53-variant}
\newline

T1. \(P(x) = x^{6} - 3x^{5} + x^{4} - 6x^{2} + 2x - 6\) kóp aǵzalısınıń pútin korenlerin tabıń. \\
T2. Qálegen \(a,b,c \in (0;1)\) sanları ushın \(a(1 - b) > 1/4,\ b(1 - c) > 1/4,\ c(1 - a) > 1/4\) teńsizlikleri bir waqıtta orınlı bola almaytuǵınlıǵin dálilleń. \\
A1. Teńlemeni sheshiń \((x + 1)^{5} + (x - 1)^{5} = 32x\). \\
A2. Teńsizlikti sheshiń:\(x^{2}\left( x^{4} + 36 \right) - 6\sqrt{3}\left( x^{4} + 4 \right) < 0\). \\
A3. Teńlemeni sheshiń. \(\sqrt[3]{x} + \sqrt[3]{x - 16} = \sqrt[3]{x - 8}\). \\
B1. \(P(x + 2) + P(x - 1) = - 2x^{2} - 2x + 7\) bolsa, \(P(x)\) ni \(x + 4\) ga bólgendegi qaldıqni tabıń. \\
B2. Tómendegi aytımdı qálegen natural san ushın matematikalıq induksiya metodi járdeminde dálilleń: \(1^{2} + 3^{2} + 5^{2} + ... + (2n - 1)^{2} = \frac{n\left( 4n^{2} - 1 \right)}{3}\); \\
B3. Tómendegi aytımdı qálegen natural san ushın matematikalıq induksiya metodi járdeminde dálilleń:\(7^{n} - 1\) sanı 6 ga eseli; \\
C1. \((a + b)^{n}\) ańlatpa jayılmasinıń barcha koeffitsiyentlari yig`indisi 4096 ga teń bolsa, Onıń eń úlken koeffitsiyentin tabıń. \\
C2. \(ABCD(AD\| BC)\) trapetsiya diagonallari \(O\) noqatda kesilisedi. Eger\emph{AOD} úshmúyeshliktiń maydanı \(a^{2}\) ga, \emph{BOC} úshmúyeshliktiń maydanı \(b^{2}\) ga teńligi ma'lum bolsa, trapetsiya maydanın tabıń. \\
C3. Ya. Bernulli teńsizligi. Eger\(x \geq - 1\) bolsa, onda qálegen natural \(n\) sanı ushın \((1 + x)^{n} \geq 1 + nx\) teńsizlik orinli bolıwın dálilleń. \\

\end{tabular}
\vspace{1cm}


\begin{tabular}{m{17cm}}
\textbf{54-variant}
\newline

T1. Qálegen \(a\) parametri hám \(x\) ushın \(x(a - x) \leq a^{2}/4\) teńsizligi orınlı bolıwın dálilleń. \\
T2. \(2^{81} + 1\) sanı 9 sanına qaldıqsız bóliniwin dálilleń. \\
A1. Teńlemeni sheshiń \(\left( x^{2} - 4x + 6 \right)^{2} - 4\left( x^{2} - 4x + 6 \right) + 6 = x\). \\
A2. Teńlemeni sheshiń \(\sqrt{\frac{18 - 7x - x^{2}}{8 - 6x + x^{2}}} + \sqrt{\frac{8 - 6x + x^{2}}{18 - 7x - x^{2}}} = \frac{13}{6}\). \\
A3. Teńlemeni sheshiń. \((x - 4)^{3} + (x - 4)^{2} + (x - 4)(x - 3) + (x - 3)^{2} + (x - 3)^{3} = 6\). \\
B1. \(P(2x - 1) + P(x - 1) = 10x^{2} - 12x + 2\) bolsa, \(P(x)\) ni tabıń. \\
B2. Tómendegi aytımdı qálegen natural san ushın matematikalıq induksiya metodi járdeminde dálilleń: \(1^{2} + 2^{2} + 3^{2} + ... + n^{2} = \frac{n(n + 1)(2n + 1)}{6}\); \\
B3. Tómendegi aytımdı qálegen natural san ushın matematikalıq induksiya metodi járdeminde dálilleń: \(5^{2n + 1} + 3^{n + 2} \cdot 2^{n - 1}\) sanı 19 ga eseli ; \\
C1. Birdeylikti dálilleń: \(\sum_{j = 0}^{n}C_{n}^{j} = 2^{n}\); \\
C2. Úshmúyeshliktiń táreplari \(a\) hám \(b\), bissektrisasi \(l_{c} = l\). \(l\) ni bilgan holda Onıń maydanıni tabıń. \\
C3. Durıs úshmúyeshliktiń tárepi a ga teń. Tárepini diametr deb esaplap dóńgelek jasalǵan. Úshmúyeshliktiń usı dóńgelekten sirtindaǵi bólegi maydanın tabıń. \\

\end{tabular}
\vspace{1cm}


\begin{tabular}{m{17cm}}
\textbf{55-variant}
\newline

T1. \(x\) ózgeriwshiniń qálegen pútin mánisinde \(ax^{2} + bx + c\) ush aǵzalısınıń mánisi pútin bolıwı ushın \(2a,\ a + b\) hám \(c\) sanlarınıń pútin bolıwı zárurli hám jetkilikli ekenligin dálilleń. \\
T2. Kombinatorika elementleri hám Nyuton binomı. \\
A1. Teńsizlikti sheshiń: \(\sqrt{x + 3} + \sqrt{x - 2} - \sqrt{2x + 4} > 0\). \\
A2. Teńsizlikti sheshiń:\(x^{2}\left( x^{4} + 36 \right) - 6\sqrt{3}\left( x^{4} + 4 \right) < 0\). \\
A3. Teńlemeni sheshiń. \(\frac{4x}{x^{2} + x + 3} + \frac{5x}{x^{2} - 5x + 3} = - \frac{3}{2}\). \\
B1. \(P(x) = x^{33} - 2ax^{21} + x^{8} + 8\) kópaǵzalıi berilgan. \(a\) nıń qaysi qiymati ushın \(P(x)\) kópaǵzalıi \(x + 1\) ga qaldıqsiz bóliadi? \\
B2. Tómendegi aytımdı qálegen natural san ushın matematikalıq induksiya metodi járdeminde dálilleń: \(2^{2} + 6^{2} + \ldots + (4n - 2)^{2} = \frac{4n(2n - 1)(2n + 1)}{3}\). \\
B3. Tómendegi aytımdı qálegen natural san ushın matematikalıq induksiya metodi járdeminde dálilleń: \(5 \cdot 2^{3n - 2} + 3^{3n - 1}\) sanı 19 ga eseli \\
C1. \(x(1 - x)^{4} + x^{2}(1 + 2x)^{8} + x^{3}(1 + 3x)^{12}\) ańlatpada \(x^{4}\) aldıńdaǵı koeffitsiyentti tabıń. \\
C2. \emph{ABC} úshmúyeshlik berilgan. Onıń medianalaridan \(\bigtriangleup A_{1}B_{1}C_{1}\) jasalǵan. \(\bigtriangleup ABC\) hám \(\bigtriangleup A_{1}B_{1}C_{1}\) maydanlarınıń qatnasi tabılsın. \\
C3. Ya. Bernulli teńsizligi. Eger\(x \geq - 1\) bolsa, onda qálegen natural \(n\) sanı ushın \((1 + x)^{n} \geq 1 + nx\) teńsizlik orinli bolıwın dálilleń. \\

\end{tabular}
\vspace{1cm}


\begin{tabular}{m{17cm}}
\textbf{56-variant}
\newline

T1. Simmetriyalıq kóp aǵzalılar. \\
T2. Qosındısı berge teń bolǵan \(x,y,z\) oń sanları ushın \(\frac{1}{x} + \frac{1}{y} + \frac{1}{z} \geq 9\) teńsizligi orınlı bolıwın dálilleń. \\
A1. Teńlemeni sheshiń. \(\sqrt[3]{x - 1} + \sqrt[3]{x - 2} - \sqrt{2x - 3} = 0\). \\
A2. Teńlemeni sheshiń. \((\sqrt{x + 1} + \sqrt{x})^{3} + (\sqrt{x + 1} + \sqrt{x})^{2} = 2\). \\
A3. Teńlemeni sheshiń. \(\sqrt{x^{2} + x + 4} + \sqrt{x^{2} + x + 1} = \sqrt{2x^{2} + 2x + 9}\). \\
B1. \(P(x)\) kópaǵzalını \(3x^{2} - 4x + 1\) ga bólgenimizdeqaldıq \(6x - 11\) bolsa, \(P(x)\) kópaǵzalını \(3x - 1\)ga bólgende qaldıqni tabıń. \\
B2. Tómendegi aytımdı qálegen natural san ushın matematikalıq induksiya metodi járdeminde dálilleń: \(1 \cdot 2 + 2 \cdot 3 + 3 \cdot 4 + ... + n(n + 1) = \frac{n(n + 1)(n + 2)}{3}\); \\
B3. Tómendegi aytımdı qálegen natural san ushın matematikalıq induksiya metodi járdeminde dálilleń: \(2n^{3} + 3n^{2} + 7n\) sanı 6 ga eseli ; \\
C1. \(\left( x\sqrt{x} - \frac{1}{x^{4}} \right)^{n}\) binom jayılmasında 3-aǵza koeffitsiyenti 2-aǵza koeffitsiyentidan 44 ga úlken.Ozod hadini tabıń. \\
C2. Parallelogrammnıń táreplari \(a\) hám \(b\), ular arasındaǵı múyesh \(\alpha\). bolsa, paralllelogramm ichki múyeshlari bissektrisalari kesilisiwinen payda bolǵan tórtmúyeshlik maydanın tabıń. \\
C3. Eger a,b,c - oń sanlar bolsa, tómendegi teńsizlikti dálilleń: \(\sqrt{\mathbf{a}^{\mathbf{2}}\mathbf{+ ab +}\mathbf{b}^{\mathbf{2}}}\mathbf{+}\sqrt{\mathbf{b}^{\mathbf{2}}\mathbf{+ bc +}\mathbf{c}^{\mathbf{2}}}\mathbf{>}\sqrt{\mathbf{a}^{\mathbf{2}}\mathbf{+ ac +}\mathbf{c}^{\mathbf{2}}}\) \\

\end{tabular}
\vspace{1cm}


\begin{tabular}{m{17cm}}
\textbf{57-variant}
\newline

T1. Mına \(P(0) = 20\) hám \(P(1) = 100\) shártlerin qanaǵatlandıratuǵın \(P(x)\) kóp aǵzalısı bar bolama? \\
T2. \(a\) parametriniń qanday mánisinde \(P(x) = x^{2017} + ax - 5\) kóp aǵzalısı \((x + 1)\) kóp aǵzalısına qaldıqsız bólinedi? \\
A1. Teńsizlikti sheshiń: \(\sqrt{x^{2} - 4x} > x - 3\). \\
A2. Teńlemeni sheshiń. \(\sqrt[3]{x} + \sqrt[3]{x - 16} = \sqrt[3]{x - 8}\). \\
A3. Teńsizlikti sheshiń: \(\frac{x^{3} + 3x^{2} - x - 3}{x^{2} + 3x - 10} < 0\). \\
B1. \(P(x + 3) = x^{2} - x + n\) bolsa. \(P(x - 2)\) kópaǵzalını \(x - 3\) ga bólgende qaldıq \(10\) bolsa, \(n\) ni tabıń. \\
B2. Tómendegi aytımdı qálegen natural san ushın matematikalıq induksiya metodi járdeminde dálilleń: \(2^{2} + 6^{2} + \ldots + (4n - 2)^{2} = \frac{4n(2n - 1)(2n + 1)}{3}\). \\
B3. Tómendegi aytımdı qálegen natural san ushın matematikalıq induksiya metodi járdeminde dálilleń: \(5^{n} - 4n + 15\) sanı 16 ga eseli ; \\
C1. Birdeylikti dálilleń: \(C_{n + 1}^{j + 1} = C_{n}^{j} + C_{n}^{j + 1}\); \\
C2. Tuwrı múyeshli úshmúyeshliktiń biyikligi gipotenuzani uzınlıqları \emph{x} hám \emph{y} ga teń bolǵan kesindilerga ajıratadı. Úshmúyeshliktiń maydanı esaplansın. \\
C3. \(R\) radiusli dóńgelekga bitta umumiy uchga ega bolǵan durıs úshmúyeshlik hám kvadrat ishley sızılǵan. Olardıń kesilisken bóleginıń maydanıni tabıń. \\

\end{tabular}
\vspace{1cm}


\begin{tabular}{m{17cm}}
\textbf{58-variant}
\newline

T1. \(P(x) = x^{6} - 3x^{5} + x^{4} - 6x^{2} + 2x - 6\) kóp aǵzalısınıń pútin korenlerin tabıń. \\
T2. Matematikalıq induksiya metodı hám onıń qollanılıwına mısallar. \\
A1. Teńlemeni sheshiń \((x + 1)^{5} + (x - 1)^{5} = 32x\). \\
A2. Teńlemeni sheshiń. \(\sqrt{3x^{2} - 2x + 15} + \sqrt{3x^{2} - 2x + 8} = 7\). \\
A3. Teńlemeni sheshiń \(\left( x^{2} + 10x + 10 \right)\left( x^{2} + x + 10 \right) = 10x^{2}\) . \\
B1. \(P(x + 1) + P(x - 3) = 2x^{2} - 10x + 16\) bolsa, \(P(x)\) ni tabıń. \\
B2. Tómendegi aytımdı qálegen natural san ushın matematikalıq induksiya metodi járdeminde dálilleń: \(1^{2} + 2^{2} + 3^{2} + ... + n^{2} = \frac{n(n + 1)(2n + 1)}{6}\); \\
B3. Tómendegi aytımdı qálegen natural san ushın matematikalıq induksiya metodi járdeminde dálilleń: \(5^{2n + 1} + 3^{n + 2} \cdot 2^{n - 1}\) sanı 19 ga eseli ; \\
C1. Birdeylikti dálilleń:\(C_{n}^{j} = C_{n}^{n - j}\); \\
C2. Tuwrı múyeshli úshmúyeshliktiń biyikligi gipotenuzani uzınlıqları 18 hám 32 sm ga teń bolǵan kesindilerga ajıratadı. Úshmúyeshliktiń maydanı esaplansın. \\
C3. Durıs úshmúyeshliktiń tárepi a ga teń. Tárepini diametr deb esaplap dóńgelek jasalǵan. Úshmúyeshliktiń usı dóńgelekten sirtindaǵi bólegi maydanın tabıń. \\

\end{tabular}
\vspace{1cm}


\begin{tabular}{m{17cm}}
\textbf{59-variant}
\newline

T1. Qálegen \(a,b,c \in (0;1)\) sanları ushın \(a(1 - b) > 1/4,\ b(1 - c) > 1/4,\ c(1 - a) > 1/4\) teńsizlikleri bir waqıtta orınlı bola almaytuǵınlıǵin dálilleń. \\
T2. \(x\) ózgeriwshiniń qálegen pútin mánisinde \(ax^{2} + bx + c\) ush aǵzalısınıń mánisi pútin bolıwı ushın \(2a,\ a + b\) hám \(c\) sanlarınıń pútin bolıwı zárurli hám jetkilikli ekenligin dálilleń. \\
A1. Teńlemeni sheshiń. \(\sqrt{x + 8 + 2\sqrt{x + 7}} + \sqrt{x + 1 - \sqrt{x + 7}} = 4\). \\
A2. Teńlemeni sheshiń \(\left( x^{2} - 6x \right)^{2} - 2(x - 3)^{2} = 81\). \\
A3. Teńlemeni sheshiń. \(\frac{z}{z + 1} - 2\sqrt{\frac{z + 1}{2}} = 3\). \\
B1. \(P(x + n) = (x + n)^{3} + (x - n)^{2} + x + n + 6\) kópaǵzalıi berilgan. \(P(x)\) kópaǵzalıi \(x - n\) ga qaldıqsiz bólinse, \(n\) ni tabıń. \\
B2. Tómendegi aytımdı qálegen natural san ushın matematikalıq induksiya metodi járdeminde dálilleń: \(\frac{1}{1 \cdot 5} + \frac{1}{5 \cdot 9} + ... + \frac{1}{(4n - 3)(4n + 1)} = \frac{n}{4n + 1}\); \\
B3. Tómendegi aytımdı qálegen natural san ushın matematikalıq induksiya metodi járdeminde dálilleń: \(n\left( 2n^{2} - 3n + 1 \right)\) sanı 6 ga eseli ; \\
C1. Teńsizlikti sheshiń: \(C_{10}^{x - 1} > 2C_{10}^{x}\) \\
C2. Durıs úshmúyeshliktiń uchlari uchta parallel tuwrı sızıqlarda yotadi. Eger ortadagi tuwrı sızıqdan chekkalardagi tuwrı sızıqlargacha bolǵan aralıq \(a\) hám \(b\) ga teń bolsa, úshmúyeshliktiń tárepini tabıń. \\
C3. Eger a,b - oń sanlar bolsa, tómendegi teńsizlikti dálilleń: \(\sqrt[3]{\frac{a}{b}} + \sqrt[3]{\frac{b}{a}} \leq \sqrt[3]{2(a + b)\left( \frac{1}{a} + \frac{1}{b} \right)}\) \\

\end{tabular}
\vspace{1cm}


\begin{tabular}{m{17cm}}
\textbf{60-variant}
\newline

T1. Simmetriyalıq kóp aǵzalılar. \\
T2. Fales teoreması hám onıń qollanılıwı. \\
A1. Teńlemeni sheshiń. \(\sqrt{x} + \frac{2x + 1}{x + 2} = 2\). \\
A2. Teńlemeni sheshiń \((x + 4)(x + 1) - 3\sqrt{x^{2} + 5x + 2} = 6\). \\
A3. Teńlemeni sheshiń \(\left( x^{2} - 4x + 6 \right)^{2} - 4\left( x^{2} - 4x + 6 \right) + 6 = x\). \\
B1. \(P(x) = (x - 5)^{2n + 1} + (x - 1)^{2n + 3}\) kópaǵzalını \(x - 3\) ga bólgende qaldıq \(3 \cdot 2^{3n - 4}\) bolsa, \(n\) ni tabıń. \\
B2. Tómendegi aytımdı qálegen natural san ushın matematikalıq induksiya metodi járdeminde dálilleń: \(\frac{1}{4 \cdot 5} + \frac{1}{5 \cdot 6} + \frac{1}{6 \cdot 7} + \ldots + \frac{1}{(n + 3) \cdot (n + 4)} = \frac{n}{4 \cdot (n + 4)}\). \\
B3. Tómendegi aytımdı qálegen natural san ushın matematikalıq induksiya metodi járdeminde dálilleń:\(7^{n} - 1\) sanı 6 ga eseli; \\
C1. Teńsizlikti sheshiń \(C_{13}^{x} < C_{13}^{x + 2}\), \(x \in N\) \\
C2. \(\bigtriangleup ABC\) da \(AB = 2sm,BD\) mediana, \(BD = 1sm\), \(\angle BDA = 30^{{^\circ}}\). Úshmúyeshliktiń maydanı esaplansın. \\
C3. Eger \(S\) úshmúyeshliktiń maydanı, \(b\) hám \(c\) onıń táreplari bolsa, \(S \leq \frac{b^{2} + c^{2}}{4}\) bolıwın dálilleń. \\

\end{tabular}
\vspace{1cm}


\begin{tabular}{m{17cm}}
\textbf{61-variant}
\newline

T1. \(2^{81} + 1\) sanı 9 sanına qaldıqsız bóliniwin dálilleń. \\
T2. \(n\) dárejeniń qanday mánislerinde \((x + 1)^{n} + (x - 1)^{n}\) ańlatpası \(x\) ańlatpaǵa qaldıqsız bólinedi? \\
A1. Teńlemeni sheshiń. \(\sqrt{x} + \frac{2x + 1}{x + 2} = 2\). \\
A2. Teńlemeni sheshiń. \(\sqrt[3]{x - 1} + \sqrt[3]{x - 2} - \sqrt{2x - 3} = 0\). \\
A3. Teńsizlikti sheshiń: \(\sqrt{x + 3} + \sqrt{x - 2} - \sqrt{2x + 4} > 0\). \\
B1. \(P(x + 2) + P(x - 1) = - 2x^{2} - 2x + 7\) bolsa, \(P(x)\) ni \(x + 4\) ga bólgendegi qaldıqni tabıń. \\
B2. Tómendegi aytımdı qálegen natural san ushın matematikalıq induksiya metodi járdeminde dálilleń: \(1^{3} + 2^{3} + 3^{3} + ... + n^{3} = \left( \frac{n(n + 1)}{2} \right)^{2}\); \\
B3. Tómendegi aytımdı qálegen natural san ushın matematikalıq induksiya metodi járdeminde dálilleń: \(5^{n + 2} + 26 \cdot 5^{n} + 8^{2n + 1}\) sanı 59 ga eseli; \\
C1. \((x + 1)^{3} + (x + 1)^{4} + (x + 1)^{5} + ... + (x + 1)^{10}\) ańlatpada \(x^{3}\) aldıńda ǵı koeffitsiyentti tabıń \\
C2. Teń qaptallı úshmúyeshliktiń maydanı \(S\) ga teń. Qaptal táreplariga túsirilgen medianalari arasındaǵı múyesh \(\alpha\) ga teń. Úshmúyeshlik ultanini tabıń. \\
C3. Eger a,b,c - oń sanlar bolsa, tómendegi teńsizlikti dálilleń: \(\sqrt{\mathbf{a}^{\mathbf{2}}\mathbf{+ ab +}\mathbf{b}^{\mathbf{2}}}\mathbf{+}\sqrt{\mathbf{b}^{\mathbf{2}}\mathbf{+ bc +}\mathbf{c}^{\mathbf{2}}}\mathbf{>}\sqrt{\mathbf{a}^{\mathbf{2}}\mathbf{+ ac +}\mathbf{c}^{\mathbf{2}}}\) \\

\end{tabular}
\vspace{1cm}


\begin{tabular}{m{17cm}}
\textbf{62-variant}
\newline

T1. \(a\) parametriniń qanday mánisinde \(P(x) = x^{2017} + ax - 5\) kóp aǵzalısı \((x + 1)\) kóp aǵzalısına qaldıqsız bólinedi? \\
T2. \(P(x) = (x - 1)^{20}\left( x^{2} + 25 \right)\) kóp aǵzalisınıń koefficentleri qosındısın tabıń. \\
A1. Teńlemeni sheshiń. \(\frac{z}{z + 1} - 2\sqrt{\frac{z + 1}{2}} = 3\). \\
A2. Teńlemeni sheshiń \(\sqrt{\frac{18 - 7x - x^{2}}{8 - 6x + x^{2}}} + \sqrt{\frac{8 - 6x + x^{2}}{18 - 7x - x^{2}}} = \frac{13}{6}\). \\
A3. Teńlemeni sheshiń. \(\sqrt{3x^{2} - 2x + 15} + \sqrt{3x^{2} - 2x + 8} = 7\). \\
B1. \(P(x) = x^{4} - 2x + 2^{n + 1}\) kópaǵzalını \(x - 2^{n}\) ga bólgende qaldıq \(2^{n - 2}\) bolsa, \(n\) ni tabıń. \\
B2. Tómendegi aytımdı qálegen natural san ushın matematikalıq induksiya metodi járdeminde dálilleń: \(1^{2} + 3^{2} + 5^{2} + ... + (2n - 1)^{2} = \frac{n\left( 4n^{2} - 1 \right)}{3}\); \\
B3. Tómendegi aytımdı qálegen natural san ushın matematikalıq induksiya metodi járdeminde dálilleń: \(6^{2n - 2} + 3^{n + 1} + 3^{n - 1}\) sanı 11 eseli ; \\
C1. Birdeylikti dálilleń: \(\sum_{j = 0}^{n}C_{n}^{j} = 2^{n}\); \\
C2. Úshmúyeshliktiń a, b hám \(c\) táreplari arifmetik progressiya quraydı. \(ac = 6Rr\) bolıwın dálilleń. Bu yerda \(R\) hám \(r\) sirtlay hám ishki sızılǵan sheńberlernıń radiuslari. \\
C3. Ya. Bernulli teńsizligi. Eger\(x \geq - 1\) bolsa, onda qálegen natural \(n\) sanı ushın \((1 + x)^{n} \geq 1 + nx\) teńsizlik orinli bolıwın dálilleń. \\

\end{tabular}
\vspace{1cm}


\begin{tabular}{m{17cm}}
\textbf{63-variant}
\newline

T1. Mına \(P(x) = x^{5} + 11x^{4} + 37x^{3} + 35x^{2} - 44x - 40\) kóp aǵzalısı \(Q(x) = x^{2} + 3x + 2\) kóp aǵzalısına qaldıqsız bólinedime? \\
T2. \(b\) parametriniń qanday mánisinde \(x^{3} + 17x^{2} + bx - 17 = 0\) teńlemesiniń korenleri pútin sanlardan turadı? \\
A1. Teńlemeni sheshiń. \((x - 4)^{3} + (x - 4)^{2} + (x - 4)(x - 3) + (x - 3)^{2} + (x - 3)^{3} = 6\). \\
A2. Teńlemeni sheshiń \(\left( x^{2} - 4x + 6 \right)^{2} - 4\left( x^{2} - 4x + 6 \right) + 6 = x\). \\
A3. Teńlemeni sheshiń. \(\sqrt{x^{2} + x + 4} + \sqrt{x^{2} + x + 1} = \sqrt{2x^{2} + 2x + 9}\). \\
B1. \(P(x + 1) + P(x - 3) = 2x^{2} - 10x + 16\) bolsa, \(P(x)\) ni tabıń. \\
B2. Tómendegi aytımdı qálegen natural san ushın matematikalıq induksiya metodi járdeminde dálilleń: \(1 \cdot 1! + 2 \cdot 2! + 3 \cdot 3! + \ldots + n \cdot n! = (n + 1)! - 1\). \\
B3. Tómendegi aytımdı qálegen natural san ushın matematikalıq induksiya metodi járdeminde dálilleń: \(n^{3} + (n + 1)^{3} + (n + 2)^{3}\) sanı 9 ga eseli ; \\
C1. Teńsizlikti sheshiń \(5C_{x}^{3} < C_{x + 2}^{4}\), \(x \in N\) \\
C2. \emph{ABCD} parallelogrammnıń \emph{AD} tárepi \(n\) ta teń bólekke bólingen. Birinchi bólinish noqatsi \(P\) hám \(B\) uch bilan birlashtirilgan. \emph{BP} tuwrı sızıq \emph{AC} dioganaldan Onıń \(\frac{1}{n + 1}\) bólegiga teń \emph{AQ} kesindi ajratishini dálilleń. \\
C3. Eger \(S\) úshmúyeshliktiń maydanı, \(b\) hám \(c\) onıń táreplari bolsa, \(S \leq \frac{b^{2} + c^{2}}{4}\) bolıwın dálilleń. \\

\end{tabular}
\vspace{1cm}


\begin{tabular}{m{17cm}}
\textbf{64-variant}
\newline

T1. Qálegen \(a\) parametri hám \(x\) ushın \(x(a - x) \leq a^{2}/4\) teńsizligi orınlı bolıwın dálilleń. \\
T2. Kombinatorika elementleri hám Nyuton binomı. \\
A1. Teńlemeni sheshiń. \(\sqrt{x + 8 + 2\sqrt{x + 7}} + \sqrt{x + 1 - \sqrt{x + 7}} = 4\). \\
A2. Teńlemeni sheshiń. \((\sqrt{x + 1} + \sqrt{x})^{3} + (\sqrt{x + 1} + \sqrt{x})^{2} = 2\). \\
A3. Teńlemeni sheshiń. \(\sqrt[3]{x} + \sqrt[3]{x - 16} = \sqrt[3]{x - 8}\). \\
B1. \(P(x)\) kópaǵzalını \(3x^{2} - 4x + 1\) ga bólgenimizdeqaldıq \(6x - 11\) bolsa, \(P(x)\) kópaǵzalını \(3x - 1\)ga bólgende qaldıqni tabıń. \\
B2. Tómendegi aytımdı qálegen natural san ushın matematikalıq induksiya metodi járdeminde dálilleń: \(\frac{1}{1 \cdot 4} + \frac{1}{4 \cdot 7} + \frac{1}{7 \cdot 10} + \ldots + \frac{1}{(3n - 2) \cdot (3n + 1)} = \frac{n}{(3n + 1)}\). \\
B3. Tómendegi aytımdı qálegen natural san ushın matematikalıq induksiya metodi járdeminde dálilleń: \(5^{2n + 1} + 3^{n + 2} \cdot 2^{n - 1}\) sanı 19 ga eseli ; \\
C1. Birdeylikti dálilleń:\(C_{n}^{j} = C_{n}^{n - j}\); \\
C2. Orayları \(O_{1}\) hám \(O_{2}\) noqatlarda hám radiusi \(R\) bolǵan eki teń sheńberlar sirtlay urinadi. \(l\) tuwrı sızıq bu sheńberlarni A, B, C hám \(D\) noqatlarda shunday kesib ótadiki, \(AB = BC = CD\) bóledi. \(O_{1}ADO_{2}\) tórtmúyeshlik maydanıni tabıń. \\
C3. \(R\) radiusli dóńgelekga bitta umumiy uchga ega bolǵan durıs úshmúyeshlik hám kvadrat ishley sızılǵan. Olardıń kesilisken bóleginıń maydanıni tabıń. \\

\end{tabular}
\vspace{1cm}


\begin{tabular}{m{17cm}}
\textbf{65-variant}
\newline

T1. Pifagor teoreması hám onıń dálilleniwleri. \\
T2. Mına \(P(0) = 20\) hám \(P(1) = 100\) shártlerin qanaǵatlandıratuǵın \(P(x)\) kóp aǵzalısı bar bolama? \\
A1. Teńsizlikti sheshiń: \(\sqrt{x^{2} - 4x} > x - 3\). \\
A2. Teńlemeni sheshiń \((x + 1)^{5} + (x - 1)^{5} = 32x\). \\
A3. Teńlemeni sheshiń \((x + 4)(x + 1) - 3\sqrt{x^{2} + 5x + 2} = 6\). \\
B1. \(P(x) = x^{33} - 2ax^{21} + x^{8} + 8\) kópaǵzalıi berilgan. \(a\) nıń qaysi qiymati ushın \(P(x)\) kópaǵzalıi \(x + 1\) ga qaldıqsiz bóliadi? \\
B2. Tómendegi aytımdı qálegen natural san ushın matematikalıq induksiya metodi járdeminde dálilleń: \(1 \cdot 2 + 2 \cdot 3 + 3 \cdot 4 + \ldots + n \cdot (n + 1) = \frac{n \cdot (n + 1) \cdot (n + 2)}{3}\). \\
B3. Tómendegi aytımdı qálegen natural san ushın matematikalıq induksiya metodi járdeminde dálilleń: \(5 \cdot 2^{3n - 2} + 3^{3n - 1}\) sanı 19 ga eseli \\
C1. \(5C_{n}^{3} = C_{n + 2}^{4}\) bolsa, \(n\) ni tabıń. \\
C2. Durıs úshmúyeshliktiń uchlari uchta parallel tuwrı sızıqlarda yotadi. Eger ortadagi tuwrı sızıqdan chekkalardagi tuwrı sızıqlargacha bolǵan aralıq \(a\) hám \(b\) ga teń bolsa, úshmúyeshliktiń tárepini tabıń. \\
C3. Durıs úshmúyeshliktiń tárepi a ga teń. Tárepini diametr deb esaplap dóńgelek jasalǵan. Úshmúyeshliktiń usı dóńgelekten sirtindaǵi bólegi maydanın tabıń. \\

\end{tabular}
\vspace{1cm}


\begin{tabular}{m{17cm}}
\textbf{66-variant}
\newline

T1. Qosındısı berge teń bolǵan \(x,y,z\) oń sanları ushın \(\frac{1}{x} + \frac{1}{y} + \frac{1}{z} \geq 9\) teńsizligi orınlı bolıwın dálilleń. \\
T2. Bezu teoreması hám onıń qollanılıwı. \\
A1. Teńlemeni sheshiń \(\left( x^{2} - 6x \right)^{2} - 2(x - 3)^{2} = 81\). \\
A2. Teńlemeni sheshiń. \(\frac{4x}{x^{2} + x + 3} + \frac{5x}{x^{2} - 5x + 3} = - \frac{3}{2}\). \\
A3. Teńsizlikti sheshiń: \(\frac{x^{3} + 3x^{2} - x - 3}{x^{2} + 3x - 10} < 0\). \\
B1. \(P(x + 3) = x^{2} - x + n\) bolsa. \(P(x - 2)\) kópaǵzalını \(x - 3\) ga bólgende qaldıq \(10\) bolsa, \(n\) ni tabıń. \\
B2. Tómendegi aytımdı qálegen natural san ushın matematikalıq induksiya metodi járdeminde dálilleń: \(\left( 1 - \frac{1}{4} \right)\left( 1 - \frac{1}{9} \right)...\left( 1 - \frac{1}{n^{2}} \right) = \frac{n + 1}{2n}\), \(n \geq 2\) \\
B3. Tómendegi aytımdı qálegen natural san ushın matematikalıq induksiya metodi járdeminde dálilleń: \(6^{2n - 2} + 3^{n + 1} + 3^{n - 1}\) sanı 11 eseli ; \\
C1. \(\left( x\sqrt{x} - \frac{1}{x^{4}} \right)^{n}\) binom jayılmasında 3-aǵza koeffitsiyenti 2-aǵza koeffitsiyentidan 44 ga úlken.Ozod hadini tabıń. \\
C2. \emph{ABC} úshmúyeshliktiń \emph{AC}, \emph{BC} hám \emph{AB} táreplarida \emph{CMPA}, \emph{BEFC} hám \emph{ADKB} kvadratlar jasalǵan. Eger\(AB = 13\), \(AC = 14,BC = 15\) ekanligi ma'lum bolsa, \emph{DKEFMP} altimúyeshliktıń maydanın tabıń. \\
C3. Eger a,b - oń sanlar bolsa, tómendegi teńsizlikti dálilleń: \(\sqrt[3]{\frac{a}{b}} + \sqrt[3]{\frac{b}{a}} \leq \sqrt[3]{2(a + b)\left( \frac{1}{a} + \frac{1}{b} \right)}\) \\

\end{tabular}
\vspace{1cm}


\begin{tabular}{m{17cm}}
\textbf{67-variant}
\newline

T1. Haqıyqıy \(a_{1},\ a_{2},\ .\ .\ .\ ,\ a_{n},\ b_{1},\ b_{2},\ .\ .\ .\ ,\ b_{n}\) sanları ushın \(\left( a_{1}b_{1} + a_{2}b_{2} + \ .\ .\ .\  + a_{n}b_{n} \right)^{2} \leq \left( a_{1}^{2} + a_{2}^{2} + \ .\ .\ .\  + a_{n}^{2} \right)\left( b_{1}^{2} + b_{2}^{2} + \ .\ .\ .\  + b_{n}^{2} \right)\) Koshi teńsizligin dálilleń. \\
T2. Qálegen \(a\) parametri hám \(x\) ushın \(x(a - x) \leq a^{2}/4\) teńsizligi orınlı bolıwın dálilleń. \\
A1. Teńsizlikti sheshiń:\(x^{2}\left( x^{4} + 36 \right) - 6\sqrt{3}\left( x^{4} + 4 \right) < 0\). \\
A2. Teńlemeni sheshiń \(\left( x^{2} + 10x + 10 \right)\left( x^{2} + x + 10 \right) = 10x^{2}\) . \\
A3. Teńlemeni sheshiń \((x + 4)(x + 1) - 3\sqrt{x^{2} + 5x + 2} = 6\). \\
B1. \(P(2x - 1) + P(x - 1) = 10x^{2} - 12x + 2\) bolsa, \(P(x)\) ni tabıń. \\
B2. Tómendegi aytımdı qálegen natural san ushın matematikalıq induksiya metodi járdeminde dálilleń: \(1^{2} + 2^{2} + 3^{2} + ... + n^{2} = \frac{n(n + 1)(2n + 1)}{6}\); \\
B3. Tómendegi aytımdı qálegen natural san ushın matematikalıq induksiya metodi járdeminde dálilleń: \(2n^{3} + 3n^{2} + 7n\) sanı 6 ga eseli ; \\
C1. \(\left( x^{3} - \frac{3}{x^{2}} \right)^{10}\) binom jayılmasinıń \(x\) qatnashmagan aǵzasın tabıń. \\
C2. \emph{ABC} úshmúyeshliktiń \emph{AB} tárepinda jaylasqan \(N\) noqatdan \(NQ\| AC\) hám \(NP\| BC\) tuwrı sızıqlar túsirilgen. Eger \emph{BNQ} úshmúyeshliktiń maydanı \(S_{1}\) ga, \emph{ANP} úshmúyeshliktiń maydanı \(S_{2}\) ga teńligi ma'lum bolsa, \emph{ABC} úshmúyeshliktiń maydanıni tabıń. \\
C3. Durıs úshmúyeshliktiń tárepi a ga teń. Tárepini diametr deb esaplap dóńgelek jasalǵan. Úshmúyeshliktiń usı dóńgelekten sirtindaǵi bólegi maydanın tabıń. \\

\end{tabular}
\vspace{1cm}


\begin{tabular}{m{17cm}}
\textbf{68-variant}
\newline

T1. Kombinatorika elementleri hám Nyuton binomı. \\
T2. Simmetriyalıq kóp aǵzalılar. \\
A1. Teńlemeni sheshiń. \((\sqrt{x + 1} + \sqrt{x})^{3} + (\sqrt{x + 1} + \sqrt{x})^{2} = 2\). \\
A2. Teńlemeni sheshiń. \(\sqrt{3x^{2} - 2x + 15} + \sqrt{3x^{2} - 2x + 8} = 7\). \\
A3. Teńsizlikti sheshiń:\(x^{2}\left( x^{4} + 36 \right) - 6\sqrt{3}\left( x^{4} + 4 \right) < 0\). \\
B1. \(P(x) = (x - 5)^{2n + 1} + (x - 1)^{2n + 3}\) kópaǵzalını \(x - 3\) ga bólgende qaldıq \(3 \cdot 2^{3n - 4}\) bolsa, \(n\) ni tabıń. \\
B2. Tómendegi aytımdı qálegen natural san ushın matematikalıq induksiya metodi járdeminde dálilleń: \(\left( 1 - \frac{1}{4} \right)\left( 1 - \frac{1}{9} \right)...\left( 1 - \frac{1}{n^{2}} \right) = \frac{n + 1}{2n}\), \(n \geq 2\) \\
B3. Tómendegi aytımdı qálegen natural san ushın matematikalıq induksiya metodi járdeminde dálilleń: \(n^{3} + (n + 1)^{3} + (n + 2)^{3}\) sanı 9 ga eseli ; \\
C1. Birdeylikti dálilleń: \(C_{n + k}^{j + k} = \sum_{s = 0}^{k}C_{n}^{j + s}C_{k}^{s}\); \\
C2. \(\bigtriangleup ABC\) da \(AB = 2sm,BD\) mediana, \(BD = 1sm\), \(\angle BDA = 30^{{^\circ}}\). Úshmúyeshliktiń maydanı esaplansın. \\
C3. Eger a,b - oń sanlar bolsa, tómendegi teńsizlikti dálilleń: \(\sqrt[3]{\frac{a}{b}} + \sqrt[3]{\frac{b}{a}} \leq \sqrt[3]{2(a + b)\left( \frac{1}{a} + \frac{1}{b} \right)}\) \\

\end{tabular}
\vspace{1cm}


\begin{tabular}{m{17cm}}
\textbf{69-variant}
\newline

T1. Fales teoreması hám onıń qollanılıwı. \\
T2. \(P(x) = (x - 1)^{20}\left( x^{2} + 25 \right)\) kóp aǵzalisınıń koefficentleri qosındısın tabıń. \\
A1. Teńsizlikti sheshiń: \(\sqrt{x^{2} - 4x} > x - 3\). \\
A2. Teńsizlikti sheshiń: \(\sqrt{x + 3} + \sqrt{x - 2} - \sqrt{2x + 4} > 0\). \\
A3. Teńlemeni sheshiń \(\left( x^{2} + 10x + 10 \right)\left( x^{2} + x + 10 \right) = 10x^{2}\) . \\
B1. \(P(x + 3)\) kópaǵzalını \(x + 1\) ga bólgende qaldıq -3, \(Q(2x - 1)\) kópaǵzalını \(x - 1\)ga bólgende qaldıq 2 bolsa, \(P(x + 4) + x^{2}Q(x + 3)\) kópaǵzalını \(x + 2\) ga bólgendegi qaldıqni tabıń. \\
B2. Tómendegi aytımdı qálegen natural san ushın matematikalıq induksiya metodi járdeminde dálilleń: \(\frac{1}{4 \cdot 5} + \frac{1}{5 \cdot 6} + \frac{1}{6 \cdot 7} + \ldots + \frac{1}{(n + 3) \cdot (n + 4)} = \frac{n}{4 \cdot (n + 4)}\). \\
B3. Tómendegi aytımdı qálegen natural san ushın matematikalıq induksiya metodi járdeminde dálilleń: \(5^{n} - 4n + 15\) sanı 16 ga eseli ; \\
C1. Teńsizlikti sheshiń: \(C_{10}^{x - 1} > 2C_{10}^{x}\) \\
C2. Úshmúyeshliktiń ultaniga túsirilgen biyikligi \(h\) ga teń. Úshmúyeshliktiń ultaniga parallel kesindi úshmúyeshliktiń maydanıni teń ekiga bóledi. Úshmúyeshliktiń ushınan usı kesindige shekem bolǵan aralıq tabılsın. \\
C3. \(R\) radiusli dóńgelekga bitta umumiy uchga ega bolǵan durıs úshmúyeshlik hám kvadrat ishley sızılǵan. Olardıń kesilisken bóleginıń maydanıni tabıń. \\

\end{tabular}
\vspace{1cm}


\begin{tabular}{m{17cm}}
\textbf{70-variant}
\newline

T1. Haqıyqıy \(a_{1},\ a_{2},\ .\ .\ .\ ,\ a_{n},\ b_{1},\ b_{2},\ .\ .\ .\ ,\ b_{n}\) sanları ushın \(\left( a_{1}b_{1} + a_{2}b_{2} + \ .\ .\ .\  + a_{n}b_{n} \right)^{2} \leq \left( a_{1}^{2} + a_{2}^{2} + \ .\ .\ .\  + a_{n}^{2} \right)\left( b_{1}^{2} + b_{2}^{2} + \ .\ .\ .\  + b_{n}^{2} \right)\) Koshi teńsizligin dálilleń. \\
T2. Bezu teoreması hám onıń qollanılıwı. \\
A1. Teńlemeni sheshiń. \(\sqrt{x} + \frac{2x + 1}{x + 2} = 2\). \\
A2. Teńsizlikti sheshiń: \(\frac{x^{3} + 3x^{2} - x - 3}{x^{2} + 3x - 10} < 0\). \\
A3. Teńlemeni sheshiń. \(\frac{z}{z + 1} - 2\sqrt{\frac{z + 1}{2}} = 3\). \\
B1. \(P(x + n) = (x + n)^{3} + (x - n)^{2} + x + n + 6\) kópaǵzalıi berilgan. \(P(x)\) kópaǵzalıi \(x - n\) ga qaldıqsiz bólinse, \(n\) ni tabıń. \\
B2. Tómendegi aytımdı qálegen natural san ushın matematikalıq induksiya metodi járdeminde dálilleń: \(1 \cdot 2 + 2 \cdot 3 + 3 \cdot 4 + \ldots + n \cdot (n + 1) = \frac{n \cdot (n + 1) \cdot (n + 2)}{3}\). \\
B3. Tómendegi aytımdı qálegen natural san ushın matematikalıq induksiya metodi járdeminde dálilleń:\(7^{n} - 1\) sanı 6 ga eseli; \\
C1. \(\left( \sqrt{x} + \frac{1}{\sqrt[3]{x^{2}}} \right)^{n}\) binom jayılmasında 5-aǵza koeffitsiyentinıń 3-aǵza koeffitsiyentine qatnasi 7:2 ga teń. \(x\) nıń darajasi 1 ga teń bolǵan aǵzasın tabıń. \\
C2. Tuwrı múyeshli úshmúyeshliktiń tuwrı burchagi bissektrisasi shu uchdan túsirilgen mediana hám biyiklik arasındaǵı múyeshni ham teń ekige bóliniwini dálilleń. \\
C3. Ya. Bernulli teńsizligi. Eger\(x \geq - 1\) bolsa, onda qálegen natural \(n\) sanı ushın \((1 + x)^{n} \geq 1 + nx\) teńsizlik orinli bolıwın dálilleń. \\

\end{tabular}
\vspace{1cm}


\begin{tabular}{m{17cm}}
\textbf{71-variant}
\newline

T1. \(x\) ózgeriwshiniń qálegen pútin mánisinde \(ax^{2} + bx + c\) ush aǵzalısınıń mánisi pútin bolıwı ushın \(2a,\ a + b\) hám \(c\) sanlarınıń pútin bolıwı zárurli hám jetkilikli ekenligin dálilleń. \\
T2. Mına \(P(0) = 20\) hám \(P(1) = 100\) shártlerin qanaǵatlandıratuǵın \(P(x)\) kóp aǵzalısı bar bolama? \\
A1. Teńlemeni sheshiń. \(\sqrt[3]{x - 1} + \sqrt[3]{x - 2} - \sqrt{2x - 3} = 0\). \\
A2. Teńlemeni sheshiń. \((x - 4)^{3} + (x - 4)^{2} + (x - 4)(x - 3) + (x - 3)^{2} + (x - 3)^{3} = 6\). \\
A3. Teńlemeni sheshiń \(\left( x^{2} - 6x \right)^{2} - 2(x - 3)^{2} = 81\). \\
B1. \(P(2x - 1) + P(x - 1) = 10x^{2} - 12x + 2\) bolsa, \(P(x)\) ni tabıń. \\
B2. Tómendegi aytımdı qálegen natural san ushın matematikalıq induksiya metodi járdeminde dálilleń: \(1^{3} + 2^{3} + 3^{3} + ... + n^{3} = \left( \frac{n(n + 1)}{2} \right)^{2}\); \\
B3. Tómendegi aytımdı qálegen natural san ushın matematikalıq induksiya metodi járdeminde dálilleń: \(5^{n + 2} + 26 \cdot 5^{n} + 8^{2n + 1}\) sanı 59 ga eseli; \\
C1. Birdeylikti dálilleń:\(C_{n + 2}^{j + 2} = C_{n}^{j} + 2C_{n}^{j + 1} + C_{n}^{j + 2}\); \\
C2. Tuwrı múyeshli úshmúyeshlik ótkir múyeshlarinıń bisєektrisalari AD hám BK \(AB^{2} = AD \cdot BK\) bolsa, úshmúyeshliktiń múyeshlarini tabıń. \\
C3. Eger \(S\) úshmúyeshliktiń maydanı, \(b\) hám \(c\) onıń táreplari bolsa, \(S \leq \frac{b^{2} + c^{2}}{4}\) bolıwın dálilleń. \\

\end{tabular}
\vspace{1cm}


\begin{tabular}{m{17cm}}
\textbf{72-variant}
\newline

T1. \(n\) dárejeniń qanday mánislerinde \((x + 1)^{n} + (x - 1)^{n}\) ańlatpası \(x\) ańlatpaǵa qaldıqsız bólinedi? \\
T2. Matematikalıq induksiya metodı hám onıń qollanılıwına mısallar. \\
A1. Teńlemeni sheshiń. \(\frac{4x}{x^{2} + x + 3} + \frac{5x}{x^{2} - 5x + 3} = - \frac{3}{2}\). \\
A2. Teńlemeni sheshiń \((x + 1)^{5} + (x - 1)^{5} = 32x\). \\
A3. Teńlemeni sheshiń. \(\sqrt[3]{x} + \sqrt[3]{x - 16} = \sqrt[3]{x - 8}\). \\
B1. \(P(x + 3)\) kópaǵzalını \(x + 1\) ga bólgende qaldıq -3, \(Q(2x - 1)\) kópaǵzalını \(x - 1\)ga bólgende qaldıq 2 bolsa, \(P(x + 4) + x^{2}Q(x + 3)\) kópaǵzalını \(x + 2\) ga bólgendegi qaldıqni tabıń. \\
B2. Tómendegi aytımdı qálegen natural san ushın matematikalıq induksiya metodi járdeminde dálilleń: \(1 \cdot 2 + 2 \cdot 3 + 3 \cdot 4 + ... + n(n + 1) = \frac{n(n + 1)(n + 2)}{3}\); \\
B3. Tómendegi aytımdı qálegen natural san ushın matematikalıq induksiya metodi járdeminde dálilleń: \(n\left( 2n^{2} - 3n + 1 \right)\) sanı 6 ga eseli ; \\
C1. Teńsizlikti sheshiń \(C_{13}^{x} < C_{13}^{x + 2}\), \(x \in N\) \\
C2. \(ABCD(AD\| BC)\) trapetsiya diagonallari \(O\) noqatda kesilisedi. Eger\emph{AOD} úshmúyeshliktiń maydanı \(a^{2}\) ga, \emph{BOC} úshmúyeshliktiń maydanı \(b^{2}\) ga teńligi ma'lum bolsa, trapetsiya maydanın tabıń. \\
C3. Eger a,b,c - oń sanlar bolsa, tómendegi teńsizlikti dálilleń: \(\sqrt{\mathbf{a}^{\mathbf{2}}\mathbf{+ ab +}\mathbf{b}^{\mathbf{2}}}\mathbf{+}\sqrt{\mathbf{b}^{\mathbf{2}}\mathbf{+ bc +}\mathbf{c}^{\mathbf{2}}}\mathbf{>}\sqrt{\mathbf{a}^{\mathbf{2}}\mathbf{+ ac +}\mathbf{c}^{\mathbf{2}}}\) \\

\end{tabular}
\vspace{1cm}


\begin{tabular}{m{17cm}}
\textbf{73-variant}
\newline

T1. \(2^{81} + 1\) sanı 9 sanına qaldıqsız bóliniwin dálilleń. \\
T2. \(a\) parametriniń qanday mánisinde \(P(x) = x^{2017} + ax - 5\) kóp aǵzalısı \((x + 1)\) kóp aǵzalısına qaldıqsız bólinedi? \\
A1. Teńlemeni sheshiń \(\sqrt{\frac{18 - 7x - x^{2}}{8 - 6x + x^{2}}} + \sqrt{\frac{8 - 6x + x^{2}}{18 - 7x - x^{2}}} = \frac{13}{6}\). \\
A2. Teńlemeni sheshiń. \(\sqrt{x^{2} + x + 4} + \sqrt{x^{2} + x + 1} = \sqrt{2x^{2} + 2x + 9}\). \\
A3. Teńlemeni sheshiń \(\left( x^{2} - 4x + 6 \right)^{2} - 4\left( x^{2} - 4x + 6 \right) + 6 = x\). \\
B1. \(P(x)\) kópaǵzalını \(3x^{2} - 4x + 1\) ga bólgenimizdeqaldıq \(6x - 11\) bolsa, \(P(x)\) kópaǵzalını \(3x - 1\)ga bólgende qaldıqni tabıń. \\
B2. Tómendegi aytımdı qálegen natural san ushın matematikalıq induksiya metodi járdeminde dálilleń: \(2^{2} + 6^{2} + \ldots + (4n - 2)^{2} = \frac{4n(2n - 1)(2n + 1)}{3}\). \\
B3. Tómendegi aytımdı qálegen natural san ushın matematikalıq induksiya metodi járdeminde dálilleń:\(7^{n} - 1\) sanı 6 ga eseli; \\
C1. Birdeylikti dálilleń:\(\sum_{j = 0}^{n}C_{n}^{j}( - 1)^{j} = 0\); \\
C2. \(\bigtriangleup ABC\) da \(\angle A\) múyesh \(\angle B\) dan eki marta úlken bólib, \(AC = b,AB = c\). \emph{BC} tárepnıń uzunligi tabılsın. \\
C3. Durıs úshmúyeshliktiń tárepi a ga teń. Tárepini diametr deb esaplap dóńgelek jasalǵan. Úshmúyeshliktiń usı dóńgelekten sirtindaǵi bólegi maydanın tabıń. \\

\end{tabular}
\vspace{1cm}


\begin{tabular}{m{17cm}}
\textbf{74-variant}
\newline

T1. \(P(x) = x^{6} - 3x^{5} + x^{4} - 6x^{2} + 2x - 6\) kóp aǵzalısınıń pútin korenlerin tabıń. \\
T2. \(b\) parametriniń qanday mánisinde \(x^{3} + 17x^{2} + bx - 17 = 0\) teńlemesiniń korenleri pútin sanlardan turadı? \\
A1. Teńlemeni sheshiń. \(\sqrt{x + 8 + 2\sqrt{x + 7}} + \sqrt{x + 1 - \sqrt{x + 7}} = 4\). \\
A2. Teńlemeni sheshiń. \(\sqrt[3]{x - 1} + \sqrt[3]{x - 2} - \sqrt{2x - 3} = 0\). \\
A3. Teńlemeni sheshiń \(\left( x^{2} - 4x + 6 \right)^{2} - 4\left( x^{2} - 4x + 6 \right) + 6 = x\). \\
B1. \(P(x) = x^{33} - 2ax^{21} + x^{8} + 8\) kópaǵzalıi berilgan. \(a\) nıń qaysi qiymati ushın \(P(x)\) kópaǵzalıi \(x + 1\) ga qaldıqsiz bóliadi? \\
B2. Tómendegi aytımdı qálegen natural san ushın matematikalıq induksiya metodi járdeminde dálilleń: \(1 \cdot 1! + 2 \cdot 2! + 3 \cdot 3! + \ldots + n \cdot n! = (n + 1)! - 1\). \\
B3. Tómendegi aytımdı qálegen natural san ushın matematikalıq induksiya metodi járdeminde dálilleń: \(5^{n} - 4n + 15\) sanı 16 ga eseli ; \\
C1. \(C_{n + 4}^{n + 1} - C_{n + 3}^{n} = 15(n + 2)\) bolsa, \(n\) ni tabıń. \\
C2. Ultanlari \(x\) hám 3 bolǵan trapetsiyada diagonallar órtalari arasındaǵı aralıqni \(x\) nıń funksiyasi sifatida ańlatpalań. \(x\) nіnń qanday qiymatida bu aralıq 1 ga teń bóledi? \\
C3. Eger \(S\) úshmúyeshliktiń maydanı, \(b\) hám \(c\) onıń táreplari bolsa, \(S \leq \frac{b^{2} + c^{2}}{4}\) bolıwın dálilleń. \\

\end{tabular}
\vspace{1cm}


\begin{tabular}{m{17cm}}
\textbf{75-variant}
\newline

T1. Mına \(P(x) = x^{5} + 11x^{4} + 37x^{3} + 35x^{2} - 44x - 40\) kóp aǵzalısı \(Q(x) = x^{2} + 3x + 2\) kóp aǵzalısına qaldıqsız bólinedime? \\
T2. Qálegen \(a,b,c \in (0;1)\) sanları ushın \(a(1 - b) > 1/4,\ b(1 - c) > 1/4,\ c(1 - a) > 1/4\) teńsizlikleri bir waqıtta orınlı bola almaytuǵınlıǵin dálilleń. \\
A1. Teńlemeni sheshiń \(\sqrt{\frac{18 - 7x - x^{2}}{8 - 6x + x^{2}}} + \sqrt{\frac{8 - 6x + x^{2}}{18 - 7x - x^{2}}} = \frac{13}{6}\). \\
A2. Teńlemeni sheshiń \((x + 1)^{5} + (x - 1)^{5} = 32x\). \\
A3. Teńsizlikti sheshiń: \(\frac{x^{3} + 3x^{2} - x - 3}{x^{2} + 3x - 10} < 0\). \\
B1. \(P(x) = x^{4} - 2x + 2^{n + 1}\) kópaǵzalını \(x - 2^{n}\) ga bólgende qaldıq \(2^{n - 2}\) bolsa, \(n\) ni tabıń. \\
B2. Tómendegi aytımdı qálegen natural san ushın matematikalıq induksiya metodi járdeminde dálilleń: \(1^{2} + 3^{2} + 5^{2} + ... + (2n - 1)^{2} = \frac{n\left( 4n^{2} - 1 \right)}{3}\); \\
B3. Tómendegi aytımdı qálegen natural san ushın matematikalıq induksiya metodi járdeminde dálilleń: \(6^{2n - 2} + 3^{n + 1} + 3^{n - 1}\) sanı 11 eseli ; \\
C1. \(x(1 - x)^{4} + x^{2}(1 + 2x)^{8} + x^{3}(1 + 3x)^{12}\) ańlatpada \(x^{4}\) aldıńdaǵı koeffitsiyentti tabıń. \\
C2. Tuwrı múyeshli úshmúyeshliktiń biyikligi gipotenuzani uzınlıqları 18 hám 32 sm ga teń bolǵan kesindilerga ajıratadı. Úshmúyeshliktiń maydanı esaplansın. \\
C3. Eger a,b - oń sanlar bolsa, tómendegi teńsizlikti dálilleń: \(\sqrt[3]{\frac{a}{b}} + \sqrt[3]{\frac{b}{a}} \leq \sqrt[3]{2(a + b)\left( \frac{1}{a} + \frac{1}{b} \right)}\) \\

\end{tabular}
\vspace{1cm}


\begin{tabular}{m{17cm}}
\textbf{76-variant}
\newline

T1. Pifagor teoreması hám onıń dálilleniwleri. \\
T2. Qosındısı berge teń bolǵan \(x,y,z\) oń sanları ushın \(\frac{1}{x} + \frac{1}{y} + \frac{1}{z} \geq 9\) teńsizligi orınlı bolıwın dálilleń. \\
A1. Teńsizlikti sheshiń: \(\sqrt{x + 3} + \sqrt{x - 2} - \sqrt{2x + 4} > 0\). \\
A2. Teńsizlikti sheshiń: \(\sqrt{x^{2} - 4x} > x - 3\). \\
A3. Teńlemeni sheshiń. \((x - 4)^{3} + (x - 4)^{2} + (x - 4)(x - 3) + (x - 3)^{2} + (x - 3)^{3} = 6\). \\
B1. \(P(x) = (x - 5)^{2n + 1} + (x - 1)^{2n + 3}\) kópaǵzalını \(x - 3\) ga bólgende qaldıq \(3 \cdot 2^{3n - 4}\) bolsa, \(n\) ni tabıń. \\
B2. Tómendegi aytımdı qálegen natural san ushın matematikalıq induksiya metodi járdeminde dálilleń: \(\frac{1}{1 \cdot 5} + \frac{1}{5 \cdot 9} + ... + \frac{1}{(4n - 3)(4n + 1)} = \frac{n}{4n + 1}\); \\
B3. Tómendegi aytımdı qálegen natural san ushın matematikalıq induksiya metodi járdeminde dálilleń: \(5^{n + 2} + 26 \cdot 5^{n} + 8^{2n + 1}\) sanı 59 ga eseli; \\
C1. \((a + b)^{n}\) ańlatpa jayılmasinıń barcha koeffitsiyentlari yig`indisi 4096 ga teń bolsa, Onıń eń úlken koeffitsiyentin tabıń. \\
C2. \emph{ABC} úshmúyeshlik berilgan. Onıń medianalaridan \(\bigtriangleup A_{1}B_{1}C_{1}\) jasalǵan. \(\bigtriangleup ABC\) hám \(\bigtriangleup A_{1}B_{1}C_{1}\) maydanlarınıń qatnasi tabılsın. \\
C3. Ya. Bernulli teńsizligi. Eger\(x \geq - 1\) bolsa, onda qálegen natural \(n\) sanı ushın \((1 + x)^{n} \geq 1 + nx\) teńsizlik orinli bolıwın dálilleń. \\

\end{tabular}
\vspace{1cm}


\begin{tabular}{m{17cm}}
\textbf{77-variant}
\newline

T1. \(x\) ózgeriwshiniń qálegen pútin mánisinde \(ax^{2} + bx + c\) ush aǵzalısınıń mánisi pútin bolıwı ushın \(2a,\ a + b\) hám \(c\) sanlarınıń pútin bolıwı zárurli hám jetkilikli ekenligin dálilleń. \\
T2. \(2^{81} + 1\) sanı 9 sanına qaldıqsız bóliniwin dálilleń. \\
A1. Teńlemeni sheshiń. \(\sqrt{x + 8 + 2\sqrt{x + 7}} + \sqrt{x + 1 - \sqrt{x + 7}} = 4\). \\
A2. Teńlemeni sheshiń \((x + 4)(x + 1) - 3\sqrt{x^{2} + 5x + 2} = 6\). \\
A3. Teńlemeni sheshiń \(\left( x^{2} - 6x \right)^{2} - 2(x - 3)^{2} = 81\). \\
B1. \(P(x + n) = (x + n)^{3} + (x - n)^{2} + x + n + 6\) kópaǵzalıi berilgan. \(P(x)\) kópaǵzalıi \(x - n\) ga qaldıqsiz bólinse, \(n\) ni tabıń. \\
B2. Tómendegi aytımdı qálegen natural san ushın matematikalıq induksiya metodi járdeminde dálilleń: \(\frac{1}{1 \cdot 4} + \frac{1}{4 \cdot 7} + \frac{1}{7 \cdot 10} + \ldots + \frac{1}{(3n - 2) \cdot (3n + 1)} = \frac{n}{(3n + 1)}\). \\
B3. Tómendegi aytımdı qálegen natural san ushın matematikalıq induksiya metodi járdeminde dálilleń: \(2n^{3} + 3n^{2} + 7n\) sanı 6 ga eseli ; \\
C1. Birdeylikti dálilleń: \(C_{n + 1}^{j + 1} = C_{n}^{j} + C_{n}^{j + 1}\); \\
C2. Úshmúyeshliktiń perimetri \(4,5dm\) ga teń, bissektrisa bolsa qarama-qarsı tárepni uzınlıqları 6 hám 9 sm ga teń bolǵan kesindilerga ajıratadı. Úshmúyeshliktiń táreplari tabılsın. \\
C3. Eger a,b,c - oń sanlar bolsa, tómendegi teńsizlikti dálilleń: \(\sqrt{\mathbf{a}^{\mathbf{2}}\mathbf{+ ab +}\mathbf{b}^{\mathbf{2}}}\mathbf{+}\sqrt{\mathbf{b}^{\mathbf{2}}\mathbf{+ bc +}\mathbf{c}^{\mathbf{2}}}\mathbf{>}\sqrt{\mathbf{a}^{\mathbf{2}}\mathbf{+ ac +}\mathbf{c}^{\mathbf{2}}}\) \\

\end{tabular}
\vspace{1cm}


\begin{tabular}{m{17cm}}
\textbf{78-variant}
\newline

T1. \(a\) parametriniń qanday mánisinde \(P(x) = x^{2017} + ax - 5\) kóp aǵzalısı \((x + 1)\) kóp aǵzalısına qaldıqsız bólinedi? \\
T2. \(P(x) = (x - 1)^{20}\left( x^{2} + 25 \right)\) kóp aǵzalisınıń koefficentleri qosındısın tabıń. \\
A1. Teńlemeni sheshiń. \((\sqrt{x + 1} + \sqrt{x})^{3} + (\sqrt{x + 1} + \sqrt{x})^{2} = 2\). \\
A2. Teńlemeni sheshiń. \(\sqrt[3]{x} + \sqrt[3]{x - 16} = \sqrt[3]{x - 8}\). \\
A3. Teńlemeni sheshiń. \(\sqrt{3x^{2} - 2x + 15} + \sqrt{3x^{2} - 2x + 8} = 7\). \\
B1. \(P(x + 1) + P(x - 3) = 2x^{2} - 10x + 16\) bolsa, \(P(x)\) ni tabıń. \\
B2. Tómendegi aytımdı qálegen natural san ushın matematikalıq induksiya metodi járdeminde dálilleń: \(\frac{1}{4 \cdot 5} + \frac{1}{5 \cdot 6} + \frac{1}{6 \cdot 7} + \ldots + \frac{1}{(n + 3) \cdot (n + 4)} = \frac{n}{4 \cdot (n + 4)}\). \\
B3. Tómendegi aytımdı qálegen natural san ushın matematikalıq induksiya metodi járdeminde dálilleń: \(5 \cdot 2^{3n - 2} + 3^{3n - 1}\) sanı 19 ga eseli \\
C1. \(\frac{1}{C_{4}^{n}} = \frac{1}{C_{5}^{n}} + \frac{1}{C_{6}^{n}}\) bolsa, \(n\) ni tabıń \\
C2. Tuwrı múyeshli úshmúyeshlikda katetlarnıń qatnasi 3:2 kabi, biyiklik bolsa gipotenuzani shunday eki kesindiga ajıratadı, olardan birinıń uzunligi ekinshisinen 2 ga úlken. Gipotenuzanıń uzunligi tabılsın. \\
C3. \(R\) radiusli dóńgelekga bitta umumiy uchga ega bolǵan durıs úshmúyeshlik hám kvadrat ishley sızılǵan. Olardıń kesilisken bóleginıń maydanıni tabıń. \\

\end{tabular}
\vspace{1cm}


\begin{tabular}{m{17cm}}
\textbf{79-variant}
\newline

T1. Bezu teoreması hám onıń qollanılıwı. \\
T2. Mına \(P(x) = x^{5} + 11x^{4} + 37x^{3} + 35x^{2} - 44x - 40\) kóp aǵzalısı \(Q(x) = x^{2} + 3x + 2\) kóp aǵzalısına qaldıqsız bólinedime? \\
A1. Teńlemeni sheshiń. \(\sqrt{x} + \frac{2x + 1}{x + 2} = 2\). \\
A2. Teńlemeni sheshiń. \(\frac{z}{z + 1} - 2\sqrt{\frac{z + 1}{2}} = 3\). \\
A3. Teńlemeni sheshiń. \(\frac{4x}{x^{2} + x + 3} + \frac{5x}{x^{2} - 5x + 3} = - \frac{3}{2}\). \\
B1. \(P(x + 2) + P(x - 1) = - 2x^{2} - 2x + 7\) bolsa, \(P(x)\) ni \(x + 4\) ga bólgendegi qaldıqni tabıń. \\
B2. Tómendegi aytımdı qálegen natural san ushın matematikalıq induksiya metodi járdeminde dálilleń: \(\frac{1}{1 \cdot 5} + \frac{1}{5 \cdot 9} + ... + \frac{1}{(4n - 3)(4n + 1)} = \frac{n}{4n + 1}\); \\
B3. Tómendegi aytımdı qálegen natural san ushın matematikalıq induksiya metodi járdeminde dálilleń: \(n\left( 2n^{2} - 3n + 1 \right)\) sanı 6 ga eseli ; \\
C1. Teńlemeni sheshiń \(\frac{C_{2x}^{x + 1}}{C_{2x + 1}^{x - 1}} = \frac{2}{3}\), \(x \in N\) \\
C2. Teń qaptallı úshmúyeshliktiń maydanı \(S\) ga teń. Qaptal táreplariga túsirilgen medianalari arasındaǵı múyesh \(\alpha\) ga teń. Úshmúyeshlik ultanini tabıń. \\
C3. \(R\) radiusli dóńgelekga bitta umumiy uchga ega bolǵan durıs úshmúyeshlik hám kvadrat ishley sızılǵan. Olardıń kesilisken bóleginıń maydanıni tabıń. \\

\end{tabular}
\vspace{1cm}


\begin{tabular}{m{17cm}}
\textbf{80-variant}
\newline

T1. Fales teoreması hám onıń qollanılıwı. \\
T2. Mına \(P(0) = 20\) hám \(P(1) = 100\) shártlerin qanaǵatlandıratuǵın \(P(x)\) kóp aǵzalısı bar bolama? \\
A1. Teńlemeni sheshiń \(\left( x^{2} + 10x + 10 \right)\left( x^{2} + x + 10 \right) = 10x^{2}\) . \\
A2. Teńlemeni sheshiń. \(\sqrt{x^{2} + x + 4} + \sqrt{x^{2} + x + 1} = \sqrt{2x^{2} + 2x + 9}\). \\
A3. Teńsizlikti sheshiń:\(x^{2}\left( x^{4} + 36 \right) - 6\sqrt{3}\left( x^{4} + 4 \right) < 0\). \\
B1. \(P(x + 3) = x^{2} - x + n\) bolsa. \(P(x - 2)\) kópaǵzalını \(x - 3\) ga bólgende qaldıq \(10\) bolsa, \(n\) ni tabıń. \\
B2. Tómendegi aytımdı qálegen natural san ushın matematikalıq induksiya metodi járdeminde dálilleń: \(\frac{1}{1 \cdot 4} + \frac{1}{4 \cdot 7} + \frac{1}{7 \cdot 10} + \ldots + \frac{1}{(3n - 2) \cdot (3n + 1)} = \frac{n}{(3n + 1)}\). \\
B3. Tómendegi aytımdı qálegen natural san ushın matematikalıq induksiya metodi járdeminde dálilleń: \(5^{2n + 1} + 3^{n + 2} \cdot 2^{n - 1}\) sanı 19 ga eseli ; \\
C1. \(\left( 2x^{\ ^{2}} - \frac{b}{2x^{3}} \right)^{10}\) binom jayılmasinıń \(x\) qatnashmagan aǵzasın tabıń. \\
C2. Tuwrı múyeshli úshmúyeshliktiń katetlari \(b\) hám \(c\) ga teń. Tuwrı múyesh bissektrisasinıń uzunligi tabılsın. \\
C3. Eger a,b - oń sanlar bolsa, tómendegi teńsizlikti dálilleń: \(\sqrt[3]{\frac{a}{b}} + \sqrt[3]{\frac{b}{a}} \leq \sqrt[3]{2(a + b)\left( \frac{1}{a} + \frac{1}{b} \right)}\) \\

\end{tabular}
\vspace{1cm}


\begin{tabular}{m{17cm}}
\textbf{81-variant}
\newline

T1. \(b\) parametriniń qanday mánisinde \(x^{3} + 17x^{2} + bx - 17 = 0\) teńlemesiniń korenleri pútin sanlardan turadı? \\
T2. Qálegen \(a,b,c \in (0;1)\) sanları ushın \(a(1 - b) > 1/4,\ b(1 - c) > 1/4,\ c(1 - a) > 1/4\) teńsizlikleri bir waqıtta orınlı bola almaytuǵınlıǵin dálilleń. \\
A1. Teńlemeni sheshiń. \(\sqrt{x + 8 + 2\sqrt{x + 7}} + \sqrt{x + 1 - \sqrt{x + 7}} = 4\). \\
A2. Teńlemeni sheshiń \(\sqrt{\frac{18 - 7x - x^{2}}{8 - 6x + x^{2}}} + \sqrt{\frac{8 - 6x + x^{2}}{18 - 7x - x^{2}}} = \frac{13}{6}\). \\
A3. Teńlemeni sheshiń. \(\sqrt[3]{x - 1} + \sqrt[3]{x - 2} - \sqrt{2x - 3} = 0\). \\
B1. \(P(x + 3)\) kópaǵzalını \(x + 1\) ga bólgende qaldıq -3, \(Q(2x - 1)\) kópaǵzalını \(x - 1\)ga bólgende qaldıq 2 bolsa, \(P(x + 4) + x^{2}Q(x + 3)\) kópaǵzalını \(x + 2\) ga bólgendegi qaldıqni tabıń. \\
B2. Tómendegi aytımdı qálegen natural san ushın matematikalıq induksiya metodi járdeminde dálilleń: \(1^{2} + 3^{2} + 5^{2} + ... + (2n - 1)^{2} = \frac{n\left( 4n^{2} - 1 \right)}{3}\); \\
B3. Tómendegi aytımdı qálegen natural san ushın matematikalıq induksiya metodi járdeminde dálilleń: \(n^{3} + (n + 1)^{3} + (n + 2)^{3}\) sanı 9 ga eseli ; \\
C1. \((x + 1)^{3} + (x + 1)^{4} + (x + 1)^{5} + ... + (x + 1)^{10}\) ańlatpada \(x^{3}\) aldıńda ǵı koeffitsiyentti tabıń \\
C2. Úshmúyeshliktiń a, b hám \(c\) táreplari arifmetik progressiya quraydı. \(ac = 6Rr\) bolıwın dálilleń. Bu yerda \(R\) hám \(r\) sirtlay hám ishki sızılǵan sheńberlernıń radiuslari. \\
C3. Ya. Bernulli teńsizligi. Eger\(x \geq - 1\) bolsa, onda qálegen natural \(n\) sanı ushın \((1 + x)^{n} \geq 1 + nx\) teńsizlik orinli bolıwın dálilleń. \\

\end{tabular}
\vspace{1cm}


\begin{tabular}{m{17cm}}
\textbf{82-variant}
\newline

T1. Kombinatorika elementleri hám Nyuton binomı. \\
T2. Qálegen \(a\) parametri hám \(x\) ushın \(x(a - x) \leq a^{2}/4\) teńsizligi orınlı bolıwın dálilleń. \\
A1. Teńsizlikti sheshiń: \(\sqrt{x^{2} - 4x} > x - 3\). \\
A2. Teńlemeni sheshiń. \(\frac{4x}{x^{2} + x + 3} + \frac{5x}{x^{2} - 5x + 3} = - \frac{3}{2}\). \\
A3. Teńlemeni sheshiń. \(\sqrt{x} + \frac{2x + 1}{x + 2} = 2\). \\
B1. \(P(2x - 1) + P(x - 1) = 10x^{2} - 12x + 2\) bolsa, \(P(x)\) ni tabıń. \\
B2. Tómendegi aytımdı qálegen natural san ushın matematikalıq induksiya metodi járdeminde dálilleń: \(1 \cdot 2 + 2 \cdot 3 + 3 \cdot 4 + \ldots + n \cdot (n + 1) = \frac{n \cdot (n + 1) \cdot (n + 2)}{3}\). \\
B3. Tómendegi aytımdı qálegen natural san ushın matematikalıq induksiya metodi járdeminde dálilleń: \(n\left( 2n^{2} - 3n + 1 \right)\) sanı 6 ga eseli ; \\
C1. \(x(1 - x)^{4} + x^{2}(1 + 2x)^{8} + x^{3}(1 + 3x)^{12}\) ańlatpada \(x^{4}\) aldıńdaǵı koeffitsiyentti tabıń. \\
C2. Teń qaptallı úshmúyeshlik ultanidagi múyesh \(\alpha\) ga teń. Shu múyesh uchidan ultanga \(\beta(\beta < \alpha)\) múyesh ostida tuwrı sızıq túsirilgen, u úshmúyeshlikni eki bólekga ajıratadı. Payda bolǵan úshmúyeshliklar maydanlarınıń qatnasini tabıń. \\
C3. Durıs úshmúyeshliktiń tárepi a ga teń. Tárepini diametr deb esaplap dóńgelek jasalǵan. Úshmúyeshliktiń usı dóńgelekten sirtindaǵi bólegi maydanın tabıń. \\

\end{tabular}
\vspace{1cm}


\begin{tabular}{m{17cm}}
\textbf{83-variant}
\newline

T1. Qosındısı berge teń bolǵan \(x,y,z\) oń sanları ushın \(\frac{1}{x} + \frac{1}{y} + \frac{1}{z} \geq 9\) teńsizligi orınlı bolıwın dálilleń. \\
T2. Simmetriyalıq kóp aǵzalılar. \\
A1. Teńlemeni sheshiń. \((x - 4)^{3} + (x - 4)^{2} + (x - 4)(x - 3) + (x - 3)^{2} + (x - 3)^{3} = 6\). \\
A2. Teńlemeni sheshiń \(\left( x^{2} - 6x \right)^{2} - 2(x - 3)^{2} = 81\). \\
A3. Teńlemeni sheshiń. \(\sqrt[3]{x} + \sqrt[3]{x - 16} = \sqrt[3]{x - 8}\). \\
B1. \(P(x)\) kópaǵzalını \(3x^{2} - 4x + 1\) ga bólgenimizdeqaldıq \(6x - 11\) bolsa, \(P(x)\) kópaǵzalını \(3x - 1\)ga bólgende qaldıqni tabıń. \\
B2. Tómendegi aytımdı qálegen natural san ushın matematikalıq induksiya metodi járdeminde dálilleń: \(1^{3} + 2^{3} + 3^{3} + ... + n^{3} = \left( \frac{n(n + 1)}{2} \right)^{2}\); \\
B3. Tómendegi aytımdı qálegen natural san ushın matematikalıq induksiya metodi járdeminde dálilleń: \(5^{2n + 1} + 3^{n + 2} \cdot 2^{n - 1}\) sanı 19 ga eseli ; \\
C1. Teńsizlikti sheshiń: \(C_{10}^{x - 1} > 2C_{10}^{x}\) \\
C2. Eki birdey radiusli dóńgeleklar sonday jaylasqan, olardıń orayları arasındaǵı aralıq radiusqa teń. Dóńgeleklar kesilisken bólegi maydanınıń, kesilisken bólegine ishley sızılǵan kvadrat maydanına qatnasin tabıń. \\
C3. Eger a,b,c - oń sanlar bolsa, tómendegi teńsizlikti dálilleń: \(\sqrt{\mathbf{a}^{\mathbf{2}}\mathbf{+ ab +}\mathbf{b}^{\mathbf{2}}}\mathbf{+}\sqrt{\mathbf{b}^{\mathbf{2}}\mathbf{+ bc +}\mathbf{c}^{\mathbf{2}}}\mathbf{>}\sqrt{\mathbf{a}^{\mathbf{2}}\mathbf{+ ac +}\mathbf{c}^{\mathbf{2}}}\) \\

\end{tabular}
\vspace{1cm}


\begin{tabular}{m{17cm}}
\textbf{84-variant}
\newline

T1. \(P(x) = x^{6} - 3x^{5} + x^{4} - 6x^{2} + 2x - 6\) kóp aǵzalısınıń pútin korenlerin tabıń. \\
T2. Matematikalıq induksiya metodı hám onıń qollanılıwına mısallar. \\
A1. Teńlemeni sheshiń \(\left( x^{2} + 10x + 10 \right)\left( x^{2} + x + 10 \right) = 10x^{2}\) . \\
A2. Teńlemeni sheshiń. \(\sqrt{x^{2} + x + 4} + \sqrt{x^{2} + x + 1} = \sqrt{2x^{2} + 2x + 9}\). \\
A3. Teńlemeni sheshiń \((x + 1)^{5} + (x - 1)^{5} = 32x\). \\
B1. \(P(x + 3) = x^{2} - x + n\) bolsa. \(P(x - 2)\) kópaǵzalını \(x - 3\) ga bólgende qaldıq \(10\) bolsa, \(n\) ni tabıń. \\
B2. Tómendegi aytımdı qálegen natural san ushın matematikalıq induksiya metodi járdeminde dálilleń: \(1^{2} + 2^{2} + 3^{2} + ... + n^{2} = \frac{n(n + 1)(2n + 1)}{6}\); \\
B3. Tómendegi aytımdı qálegen natural san ushın matematikalıq induksiya metodi járdeminde dálilleń: \(5^{n} - 4n + 15\) sanı 16 ga eseli ; \\
C1. Teńsizlikti sheshiń \(C_{13}^{x} < C_{13}^{x + 2}\), \(x \in N\) \\
C2. Tuwrı múyeshli úshmúyeshliktiń biyikligi gipotenuzani uzınlıqları \emph{x} hám \emph{y} ga teń bolǵan kesindilerga ajıratadı. Úshmúyeshliktiń maydanı esaplansın. \\
C3. Eger \(S\) úshmúyeshliktiń maydanı, \(b\) hám \(c\) onıń táreplari bolsa, \(S \leq \frac{b^{2} + c^{2}}{4}\) bolıwın dálilleń. \\

\end{tabular}
\vspace{1cm}


\begin{tabular}{m{17cm}}
\textbf{85-variant}
\newline

T1. Haqıyqıy \(a_{1},\ a_{2},\ .\ .\ .\ ,\ a_{n},\ b_{1},\ b_{2},\ .\ .\ .\ ,\ b_{n}\) sanları ushın \(\left( a_{1}b_{1} + a_{2}b_{2} + \ .\ .\ .\  + a_{n}b_{n} \right)^{2} \leq \left( a_{1}^{2} + a_{2}^{2} + \ .\ .\ .\  + a_{n}^{2} \right)\left( b_{1}^{2} + b_{2}^{2} + \ .\ .\ .\  + b_{n}^{2} \right)\) Koshi teńsizligin dálilleń. \\
T2. Pifagor teoreması hám onıń dálilleniwleri. \\
A1. Teńsizlikti sheshiń: \(\sqrt{x + 3} + \sqrt{x - 2} - \sqrt{2x + 4} > 0\). \\
A2. Teńlemeni sheshiń. \((\sqrt{x + 1} + \sqrt{x})^{3} + (\sqrt{x + 1} + \sqrt{x})^{2} = 2\). \\
A3. Teńlemeni sheshiń \((x + 4)(x + 1) - 3\sqrt{x^{2} + 5x + 2} = 6\). \\
B1. \(P(x + n) = (x + n)^{3} + (x - n)^{2} + x + n + 6\) kópaǵzalıi berilgan. \(P(x)\) kópaǵzalıi \(x - n\) ga qaldıqsiz bólinse, \(n\) ni tabıń. \\
B2. Tómendegi aytımdı qálegen natural san ushın matematikalıq induksiya metodi járdeminde dálilleń: \(1 \cdot 1! + 2 \cdot 2! + 3 \cdot 3! + \ldots + n \cdot n! = (n + 1)! - 1\). \\
B3. Tómendegi aytımdı qálegen natural san ushın matematikalıq induksiya metodi járdeminde dálilleń: \(6^{2n - 2} + 3^{n + 1} + 3^{n - 1}\) sanı 11 eseli ; \\
C1. \(\left( x\sqrt{x} - \frac{1}{x^{4}} \right)^{n}\) binom jayılmasında 3-aǵza koeffitsiyenti 2-aǵza koeffitsiyentidan 44 ga úlken.Ozod hadini tabıń. \\
C2. \emph{ABC} úshmúyeshliktiń \(B\) uchidan \emph{AC} tárepiga \emph{BD} kesindi ótkazildi. \emph{BD} kesindi bu úshmúyeshliktiń maydanıni teń ekige bóledi. Eger\(AC = a\) bolsa, \emph{AD} hám \emph{DC} kesindilarnıń uzınlıqlarıni tabıń. \\
C3. Durıs úshmúyeshliktiń tárepi a ga teń. Tárepini diametr deb esaplap dóńgelek jasalǵan. Úshmúyeshliktiń usı dóńgelekten sirtindaǵi bólegi maydanın tabıń. \\

\end{tabular}
\vspace{1cm}


\begin{tabular}{m{17cm}}
\textbf{86-variant}
\newline

T1. \(n\) dárejeniń qanday mánislerinde \((x + 1)^{n} + (x - 1)^{n}\) ańlatpası \(x\) ańlatpaǵa qaldıqsız bólinedi? \\
T2. \(P(x) = x^{6} - 3x^{5} + x^{4} - 6x^{2} + 2x - 6\) kóp aǵzalısınıń pútin korenlerin tabıń. \\
A1. Teńlemeni sheshiń \(\left( x^{2} - 4x + 6 \right)^{2} - 4\left( x^{2} - 4x + 6 \right) + 6 = x\). \\
A2. Teńsizlikti sheshiń:\(x^{2}\left( x^{4} + 36 \right) - 6\sqrt{3}\left( x^{4} + 4 \right) < 0\). \\
A3. Teńlemeni sheshiń. \(\frac{z}{z + 1} - 2\sqrt{\frac{z + 1}{2}} = 3\). \\
B1. \(P(x + 1) + P(x - 3) = 2x^{2} - 10x + 16\) bolsa, \(P(x)\) ni tabıń. \\
B2. Tómendegi aytımdı qálegen natural san ushın matematikalıq induksiya metodi járdeminde dálilleń: \(2^{2} + 6^{2} + \ldots + (4n - 2)^{2} = \frac{4n(2n - 1)(2n + 1)}{3}\). \\
B3. Tómendegi aytımdı qálegen natural san ushın matematikalıq induksiya metodi járdeminde dálilleń: \(2n^{3} + 3n^{2} + 7n\) sanı 6 ga eseli ; \\
C1. \(\left( \sqrt{x} + \frac{1}{\sqrt[3]{x^{2}}} \right)^{n}\) binom jayılmasında 5-aǵza koeffitsiyentinıń 3-aǵza koeffitsiyentine qatnasi 7:2 ga teń. \(x\) nıń darajasi 1 ga teń bolǵan aǵzasın tabıń. \\
C2. \(\bigtriangleup ABC\) da \(AB = 3sm,AC = 5sm,\angle BAC = 120^{{^\circ}}.BD\) bissektrisanıń uzunligi tabılsın. \\
C3. Ya. Bernulli teńsizligi. Eger\(x \geq - 1\) bolsa, onda qálegen natural \(n\) sanı ushın \((1 + x)^{n} \geq 1 + nx\) teńsizlik orinli bolıwın dálilleń. \\

\end{tabular}
\vspace{1cm}


\begin{tabular}{m{17cm}}
\textbf{87-variant}
\newline

T1. \(x\) ózgeriwshiniń qálegen pútin mánisinde \(ax^{2} + bx + c\) ush aǵzalısınıń mánisi pútin bolıwı ushın \(2a,\ a + b\) hám \(c\) sanlarınıń pútin bolıwı zárurli hám jetkilikli ekenligin dálilleń. \\
T2. Mına \(P(x) = x^{5} + 11x^{4} + 37x^{3} + 35x^{2} - 44x - 40\) kóp aǵzalısı \(Q(x) = x^{2} + 3x + 2\) kóp aǵzalısına qaldıqsız bólinedime? \\
A1. Teńlemeni sheshiń. \(\sqrt{3x^{2} - 2x + 15} + \sqrt{3x^{2} - 2x + 8} = 7\). \\
A2. Teńsizlikti sheshiń: \(\frac{x^{3} + 3x^{2} - x - 3}{x^{2} + 3x - 10} < 0\). \\
A3. Teńlemeni sheshiń \((x + 4)(x + 1) - 3\sqrt{x^{2} + 5x + 2} = 6\). \\
B1. \(P(x) = x^{4} - 2x + 2^{n + 1}\) kópaǵzalını \(x - 2^{n}\) ga bólgende qaldıq \(2^{n - 2}\) bolsa, \(n\) ni tabıń. \\
B2. Tómendegi aytımdı qálegen natural san ushın matematikalıq induksiya metodi járdeminde dálilleń: \(1 \cdot 2 + 2 \cdot 3 + 3 \cdot 4 + ... + n(n + 1) = \frac{n(n + 1)(n + 2)}{3}\); \\
B3. Tómendegi aytımdı qálegen natural san ushın matematikalıq induksiya metodi járdeminde dálilleń: \(5^{n + 2} + 26 \cdot 5^{n} + 8^{2n + 1}\) sanı 59 ga eseli; \\
C1. \(\left( 2x^{\ ^{2}} - \frac{b}{2x^{3}} \right)^{10}\) binom jayılmasinıń \(x\) qatnashmagan aǵzasın tabıń. \\
C2. Úshmúyeshliktiń ishida olińan noqatdan Onıń táreplariga parallel tuwrı sızıqlar túsirilgen. Ular úshmúyeshlikni 6 bólekga bóledi. Eger payda bolǵan úshmúyeshliklarnıń maydanları \(S_{1},S_{2}\) hám \(S_{3}\) bolsa, berilgan úshmúyeshlik maydanın tabıń. \\
C3. Eger a,b,c - oń sanlar bolsa, tómendegi teńsizlikti dálilleń: \(\sqrt{\mathbf{a}^{\mathbf{2}}\mathbf{+ ab +}\mathbf{b}^{\mathbf{2}}}\mathbf{+}\sqrt{\mathbf{b}^{\mathbf{2}}\mathbf{+ bc +}\mathbf{c}^{\mathbf{2}}}\mathbf{>}\sqrt{\mathbf{a}^{\mathbf{2}}\mathbf{+ ac +}\mathbf{c}^{\mathbf{2}}}\) \\

\end{tabular}
\vspace{1cm}


\begin{tabular}{m{17cm}}
\textbf{88-variant}
\newline

T1. Kombinatorika elementleri hám Nyuton binomı. \\
T2. Simmetriyalıq kóp aǵzalılar. \\
A1. Teńlemeni sheshiń. \(\sqrt[3]{x} + \sqrt[3]{x - 16} = \sqrt[3]{x - 8}\). \\
A2. Teńlemeni sheshiń. \((x - 4)^{3} + (x - 4)^{2} + (x - 4)(x - 3) + (x - 3)^{2} + (x - 3)^{3} = 6\). \\
A3. Teńlemeni sheshiń \(\left( x^{2} - 4x + 6 \right)^{2} - 4\left( x^{2} - 4x + 6 \right) + 6 = x\). \\
B1. \(P(x) = x^{33} - 2ax^{21} + x^{8} + 8\) kópaǵzalıi berilgan. \(a\) nıń qaysi qiymati ushın \(P(x)\) kópaǵzalıi \(x + 1\) ga qaldıqsiz bóliadi? \\
B2. Tómendegi aytımdı qálegen natural san ushın matematikalıq induksiya metodi járdeminde dálilleń: \(\left( 1 - \frac{1}{4} \right)\left( 1 - \frac{1}{9} \right)...\left( 1 - \frac{1}{n^{2}} \right) = \frac{n + 1}{2n}\), \(n \geq 2\) \\
B3. Tómendegi aytımdı qálegen natural san ushın matematikalıq induksiya metodi járdeminde dálilleń: \(5 \cdot 2^{3n - 2} + 3^{3n - 1}\) sanı 19 ga eseli \\
C1. Birdeylikti dálilleń: \(\sum_{j = 0}^{n}C_{n}^{j} = 2^{n}\); \\
C2. Parallelogrammnıń táreplari \(a\) hám \(b\), ular arasındaǵı múyesh \(\alpha\). bolsa, paralllelogramm ichki múyeshlari bissektrisalari kesilisiwinen payda bolǵan tórtmúyeshlik maydanın tabıń. \\
C3. Eger a,b - oń sanlar bolsa, tómendegi teńsizlikti dálilleń: \(\sqrt[3]{\frac{a}{b}} + \sqrt[3]{\frac{b}{a}} \leq \sqrt[3]{2(a + b)\left( \frac{1}{a} + \frac{1}{b} \right)}\) \\

\end{tabular}
\vspace{1cm}


\begin{tabular}{m{17cm}}
\textbf{89-variant}
\newline

T1. \(b\) parametriniń qanday mánisinde \(x^{3} + 17x^{2} + bx - 17 = 0\) teńlemesiniń korenleri pútin sanlardan turadı? \\
T2. Bezu teoreması hám onıń qollanılıwı. \\
A1. Teńlemeni sheshiń. \(\frac{4x}{x^{2} + x + 3} + \frac{5x}{x^{2} - 5x + 3} = - \frac{3}{2}\). \\
A2. Teńlemeni sheshiń. \((\sqrt{x + 1} + \sqrt{x})^{3} + (\sqrt{x + 1} + \sqrt{x})^{2} = 2\). \\
A3. Teńsizlikti sheshiń: \(\sqrt{x^{2} - 4x} > x - 3\). \\
B1. \(P(x) = (x - 5)^{2n + 1} + (x - 1)^{2n + 3}\) kópaǵzalını \(x - 3\) ga bólgende qaldıq \(3 \cdot 2^{3n - 4}\) bolsa, \(n\) ni tabıń. \\
B2. Tómendegi aytımdı qálegen natural san ushın matematikalıq induksiya metodi járdeminde dálilleń: \(1^{2} + 3^{2} + 5^{2} + ... + (2n - 1)^{2} = \frac{n\left( 4n^{2} - 1 \right)}{3}\); \\
B3. Tómendegi aytımdı qálegen natural san ushın matematikalıq induksiya metodi járdeminde dálilleń: \(n^{3} + (n + 1)^{3} + (n + 2)^{3}\) sanı 9 ga eseli ; \\
C1. Birdeylikti dálilleń:\(\sum_{j = 0}^{n}C_{n}^{j}( - 1)^{j} = 0\); \\
C2. Egerteń qaptallı úshmúyeshliktiń perimetri 32 dm , orta sızıǵı 6 dm ga teń bolsa, Onıń táreplari uzınlıqları tabılsın. \\
C3. Eger \(S\) úshmúyeshliktiń maydanı, \(b\) hám \(c\) onıń táreplari bolsa, \(S \leq \frac{b^{2} + c^{2}}{4}\) bolıwın dálilleń. \\

\end{tabular}
\vspace{1cm}


\begin{tabular}{m{17cm}}
\textbf{90-variant}
\newline

T1. \(2^{81} + 1\) sanı 9 sanına qaldıqsız bóliniwin dálilleń. \\
T2. Mına \(P(0) = 20\) hám \(P(1) = 100\) shártlerin qanaǵatlandıratuǵın \(P(x)\) kóp aǵzalısı bar bolama? \\
A1. Teńsizlikti sheshiń:\(x^{2}\left( x^{4} + 36 \right) - 6\sqrt{3}\left( x^{4} + 4 \right) < 0\). \\
A2. Teńlemeni sheshiń. \(\sqrt{x + 8 + 2\sqrt{x + 7}} + \sqrt{x + 1 - \sqrt{x + 7}} = 4\). \\
A3. Teńlemeni sheshiń \(\left( x^{2} + 10x + 10 \right)\left( x^{2} + x + 10 \right) = 10x^{2}\) . \\
B1. \(P(x + 2) + P(x - 1) = - 2x^{2} - 2x + 7\) bolsa, \(P(x)\) ni \(x + 4\) ga bólgendegi qaldıqni tabıń. \\
B2. Tómendegi aytımdı qálegen natural san ushın matematikalıq induksiya metodi járdeminde dálilleń: \(1^{3} + 2^{3} + 3^{3} + ... + n^{3} = \left( \frac{n(n + 1)}{2} \right)^{2}\); \\
B3. Tómendegi aytımdı qálegen natural san ushın matematikalıq induksiya metodi járdeminde dálilleń:\(7^{n} - 1\) sanı 6 ga eseli; \\
C1. \(\frac{1}{C_{4}^{n}} = \frac{1}{C_{5}^{n}} + \frac{1}{C_{6}^{n}}\) bolsa, \(n\) ni tabıń \\
C2. Teń qaptallı \(ABC(AB = BC)\) úshmúyeshlikda \emph{AD} bissektrisa túsirilgen. Eger\(S_{ABD} = S_{1},S_{\bigtriangleup ADC} = S_{2}\) bolsa, \emph{AC} ni tabıń. \\
C3. \(R\) radiusli dóńgelekga bitta umumiy uchga ega bolǵan durıs úshmúyeshlik hám kvadrat ishley sızılǵan. Olardıń kesilisken bóleginıń maydanıni tabıń. \\

\end{tabular}
\vspace{1cm}


\begin{tabular}{m{17cm}}
\textbf{91-variant}
\newline

T1. Qosındısı berge teń bolǵan \(x,y,z\) oń sanları ushın \(\frac{1}{x} + \frac{1}{y} + \frac{1}{z} \geq 9\) teńsizligi orınlı bolıwın dálilleń. \\
T2. Fales teoreması hám onıń qollanılıwı. \\
A1. Teńlemeni sheshiń. \(\sqrt{3x^{2} - 2x + 15} + \sqrt{3x^{2} - 2x + 8} = 7\). \\
A2. Teńsizlikti sheshiń: \(\frac{x^{3} + 3x^{2} - x - 3}{x^{2} + 3x - 10} < 0\). \\
A3. Teńlemeni sheshiń \((x + 1)^{5} + (x - 1)^{5} = 32x\). \\
B1. \(P(x + 1) + P(x - 3) = 2x^{2} - 10x + 16\) bolsa, \(P(x)\) ni tabıń. \\
B2. Tómendegi aytımdı qálegen natural san ushın matematikalıq induksiya metodi járdeminde dálilleń: \(\frac{1}{4 \cdot 5} + \frac{1}{5 \cdot 6} + \frac{1}{6 \cdot 7} + \ldots + \frac{1}{(n + 3) \cdot (n + 4)} = \frac{n}{4 \cdot (n + 4)}\). \\
B3. Tómendegi aytımdı qálegen natural san ushın matematikalıq induksiya metodi járdeminde dálilleń: \(5^{n + 2} + 26 \cdot 5^{n} + 8^{2n + 1}\) sanı 59 ga eseli; \\
C1. Teńsizlikti sheshiń \(5C_{x}^{3} < C_{x + 2}^{4}\), \(x \in N\) \\
C2. \emph{ABCD} parallelogrammnıń \emph{AD} tárepi \(n\) ta teń bólekke bólingen. Birinchi bólinish noqatsi \(P\) hám \(B\) uch bilan birlashtirilgan. \emph{BP} tuwrı sızıq \emph{AC} dioganaldan Onıń \(\frac{1}{n + 1}\) bólegiga teń \emph{AQ} kesindi ajratishini dálilleń. \\
C3. Eger \(S\) úshmúyeshliktiń maydanı, \(b\) hám \(c\) onıń táreplari bolsa, \(S \leq \frac{b^{2} + c^{2}}{4}\) bolıwın dálilleń. \\

\end{tabular}
\vspace{1cm}


\begin{tabular}{m{17cm}}
\textbf{92-variant}
\newline

T1. Haqıyqıy \(a_{1},\ a_{2},\ .\ .\ .\ ,\ a_{n},\ b_{1},\ b_{2},\ .\ .\ .\ ,\ b_{n}\) sanları ushın \(\left( a_{1}b_{1} + a_{2}b_{2} + \ .\ .\ .\  + a_{n}b_{n} \right)^{2} \leq \left( a_{1}^{2} + a_{2}^{2} + \ .\ .\ .\  + a_{n}^{2} \right)\left( b_{1}^{2} + b_{2}^{2} + \ .\ .\ .\  + b_{n}^{2} \right)\) Koshi teńsizligin dálilleń. \\
T2. Qálegen \(a,b,c \in (0;1)\) sanları ushın \(a(1 - b) > 1/4,\ b(1 - c) > 1/4,\ c(1 - a) > 1/4\) teńsizlikleri bir waqıtta orınlı bola almaytuǵınlıǵin dálilleń. \\
A1. Teńlemeni sheshiń \(\left( x^{2} - 6x \right)^{2} - 2(x - 3)^{2} = 81\). \\
A2. Teńlemeni sheshiń \(\sqrt{\frac{18 - 7x - x^{2}}{8 - 6x + x^{2}}} + \sqrt{\frac{8 - 6x + x^{2}}{18 - 7x - x^{2}}} = \frac{13}{6}\). \\
A3. Teńlemeni sheshiń. \(\sqrt{x} + \frac{2x + 1}{x + 2} = 2\). \\
B1. \(P(x + 3) = x^{2} - x + n\) bolsa. \(P(x - 2)\) kópaǵzalını \(x - 3\) ga bólgende qaldıq \(10\) bolsa, \(n\) ni tabıń. \\
B2. Tómendegi aytımdı qálegen natural san ushın matematikalıq induksiya metodi járdeminde dálilleń: \(\frac{1}{1 \cdot 4} + \frac{1}{4 \cdot 7} + \frac{1}{7 \cdot 10} + \ldots + \frac{1}{(3n - 2) \cdot (3n + 1)} = \frac{n}{(3n + 1)}\). \\
B3. Tómendegi aytımdı qálegen natural san ushın matematikalıq induksiya metodi járdeminde dálilleń: \(5 \cdot 2^{3n - 2} + 3^{3n - 1}\) sanı 19 ga eseli \\
C1. Birdeylikti dálilleń:\(C_{n + 2}^{j + 2} = C_{n}^{j} + 2C_{n}^{j + 1} + C_{n}^{j + 2}\); \\
C2. Orayları \(O_{1}\) hám \(O_{2}\) noqatlarda hám radiusi \(R\) bolǵan eki teń sheńberlar sirtlay urinadi. \(l\) tuwrı sızıq bu sheńberlarni A, B, C hám \(D\) noqatlarda shunday kesib ótadiki, \(AB = BC = CD\) bóledi. \(O_{1}ADO_{2}\) tórtmúyeshlik maydanıni tabıń. \\
C3. Ya. Bernulli teńsizligi. Eger\(x \geq - 1\) bolsa, onda qálegen natural \(n\) sanı ushın \((1 + x)^{n} \geq 1 + nx\) teńsizlik orinli bolıwın dálilleń. \\

\end{tabular}
\vspace{1cm}


\begin{tabular}{m{17cm}}
\textbf{93-variant}
\newline

T1. Pifagor teoreması hám onıń dálilleniwleri. \\
T2. \(P(x) = (x - 1)^{20}\left( x^{2} + 25 \right)\) kóp aǵzalisınıń koefficentleri qosındısın tabıń. \\
A1. Teńlemeni sheshiń. \(\frac{z}{z + 1} - 2\sqrt{\frac{z + 1}{2}} = 3\). \\
A2. Teńsizlikti sheshiń: \(\sqrt{x + 3} + \sqrt{x - 2} - \sqrt{2x + 4} > 0\). \\
A3. Teńlemeni sheshiń. \(\sqrt{x^{2} + x + 4} + \sqrt{x^{2} + x + 1} = \sqrt{2x^{2} + 2x + 9}\). \\
B1. \(P(x) = x^{4} - 2x + 2^{n + 1}\) kópaǵzalını \(x - 2^{n}\) ga bólgende qaldıq \(2^{n - 2}\) bolsa, \(n\) ni tabıń. \\
B2. Tómendegi aytımdı qálegen natural san ushın matematikalıq induksiya metodi járdeminde dálilleń: \(1 \cdot 1! + 2 \cdot 2! + 3 \cdot 3! + \ldots + n \cdot n! = (n + 1)! - 1\). \\
B3. Tómendegi aytımdı qálegen natural san ushın matematikalıq induksiya metodi járdeminde dálilleń: \(n\left( 2n^{2} - 3n + 1 \right)\) sanı 6 ga eseli ; \\
C1. Birdeylikti dálilleń: \(C_{n + 1}^{j + 1} = C_{n}^{j} + C_{n}^{j + 1}\); \\
C2. Bir burchagi \(60^{{^\circ}}\) bolǵan úshmúyeshlikka ishley sızılǵan sheńberdiń uriniw noqati shu múyeshke qarama- qarama-qarsı tárepini \(a\) hám \(b\) kesindilerga ajıratadı. Úshmúyeshlik maydanıni tabıń. \\
C3. Eger a,b,c - oń sanlar bolsa, tómendegi teńsizlikti dálilleń: \(\sqrt{\mathbf{a}^{\mathbf{2}}\mathbf{+ ab +}\mathbf{b}^{\mathbf{2}}}\mathbf{+}\sqrt{\mathbf{b}^{\mathbf{2}}\mathbf{+ bc +}\mathbf{c}^{\mathbf{2}}}\mathbf{>}\sqrt{\mathbf{a}^{\mathbf{2}}\mathbf{+ ac +}\mathbf{c}^{\mathbf{2}}}\) \\

\end{tabular}
\vspace{1cm}


\begin{tabular}{m{17cm}}
\textbf{94-variant}
\newline

T1. \(a\) parametriniń qanday mánisinde \(P(x) = x^{2017} + ax - 5\) kóp aǵzalısı \((x + 1)\) kóp aǵzalısına qaldıqsız bólinedi? \\
T2. Matematikalıq induksiya metodı hám onıń qollanılıwına mısallar. \\
A1. Teńlemeni sheshiń. \(\sqrt[3]{x - 1} + \sqrt[3]{x - 2} - \sqrt{2x - 3} = 0\). \\
A2. Teńlemeni sheshiń. \(\sqrt[3]{x} + \sqrt[3]{x - 16} = \sqrt[3]{x - 8}\). \\
A3. Teńlemeni sheshiń. \(\frac{4x}{x^{2} + x + 3} + \frac{5x}{x^{2} - 5x + 3} = - \frac{3}{2}\). \\
B1. \(P(x + 3)\) kópaǵzalını \(x + 1\) ga bólgende qaldıq -3, \(Q(2x - 1)\) kópaǵzalını \(x - 1\)ga bólgende qaldıq 2 bolsa, \(P(x + 4) + x^{2}Q(x + 3)\) kópaǵzalını \(x + 2\) ga bólgendegi qaldıqni tabıń. \\
B2. Tómendegi aytımdı qálegen natural san ushın matematikalıq induksiya metodi járdeminde dálilleń: \(\left( 1 - \frac{1}{4} \right)\left( 1 - \frac{1}{9} \right)...\left( 1 - \frac{1}{n^{2}} \right) = \frac{n + 1}{2n}\), \(n \geq 2\) \\
B3. Tómendegi aytımdı qálegen natural san ushın matematikalıq induksiya metodi járdeminde dálilleń:\(7^{n} - 1\) sanı 6 ga eseli; \\
C1. Teńlemeni sheshiń \(\frac{C_{2x}^{x + 1}}{C_{2x + 1}^{x - 1}} = \frac{2}{3}\), \(x \in N\) \\
C2. Úshmúyeshliktiń táreplari \(a\) hám \(b\), bissektrisasi \(l_{c} = l\). \(l\) ni bilgan holda Onıń maydanıni tabıń. \\
C3. Eger a,b - oń sanlar bolsa, tómendegi teńsizlikti dálilleń: \(\sqrt[3]{\frac{a}{b}} + \sqrt[3]{\frac{b}{a}} \leq \sqrt[3]{2(a + b)\left( \frac{1}{a} + \frac{1}{b} \right)}\) \\

\end{tabular}
\vspace{1cm}


\begin{tabular}{m{17cm}}
\textbf{95-variant}
\newline

T1. Qálegen \(a\) parametri hám \(x\) ushın \(x(a - x) \leq a^{2}/4\) teńsizligi orınlı bolıwın dálilleń. \\
T2. \(n\) dárejeniń qanday mánislerinde \((x + 1)^{n} + (x - 1)^{n}\) ańlatpası \(x\) ańlatpaǵa qaldıqsız bólinedi? \\
A1. Teńlemeni sheshiń \(\sqrt{\frac{18 - 7x - x^{2}}{8 - 6x + x^{2}}} + \sqrt{\frac{8 - 6x + x^{2}}{18 - 7x - x^{2}}} = \frac{13}{6}\). \\
A2. Teńlemeni sheshiń \(\left( x^{2} - 4x + 6 \right)^{2} - 4\left( x^{2} - 4x + 6 \right) + 6 = x\). \\
A3. Teńsizlikti sheshiń: \(\sqrt{x + 3} + \sqrt{x - 2} - \sqrt{2x + 4} > 0\). \\
B1. \(P(x)\) kópaǵzalını \(3x^{2} - 4x + 1\) ga bólgenimizdeqaldıq \(6x - 11\) bolsa, \(P(x)\) kópaǵzalını \(3x - 1\)ga bólgende qaldıqni tabıń. \\
B2. Tómendegi aytımdı qálegen natural san ushın matematikalıq induksiya metodi járdeminde dálilleń: \(\frac{1}{1 \cdot 5} + \frac{1}{5 \cdot 9} + ... + \frac{1}{(4n - 3)(4n + 1)} = \frac{n}{4n + 1}\); \\
B3. Tómendegi aytımdı qálegen natural san ushın matematikalıq induksiya metodi járdeminde dálilleń: \(n^{3} + (n + 1)^{3} + (n + 2)^{3}\) sanı 9 ga eseli ; \\
C1. \(5C_{n}^{3} = C_{n + 2}^{4}\) bolsa, \(n\) ni tabıń. \\
C2. Teń qaptallı úshmúyeshliktiń qaptal tárepi 13 sm , qaptal tárepine túsirilgen biyiklik 5 sm ga teń. Úshmúyeshlik ultaninıń uzunligi tabılsın. \\
C3. Durıs úshmúyeshliktiń tárepi a ga teń. Tárepini diametr deb esaplap dóńgelek jasalǵan. Úshmúyeshliktiń usı dóńgelekten sirtindaǵi bólegi maydanın tabıń. \\

\end{tabular}
\vspace{1cm}


\begin{tabular}{m{17cm}}
\textbf{96-variant}
\newline

T1. Kombinatorika elementleri hám Nyuton binomı. \\
T2. Mına \(P(0) = 20\) hám \(P(1) = 100\) shártlerin qanaǵatlandıratuǵın \(P(x)\) kóp aǵzalısı bar bolama? \\
A1. Teńsizlikti sheshiń:\(x^{2}\left( x^{4} + 36 \right) - 6\sqrt{3}\left( x^{4} + 4 \right) < 0\). \\
A2. Teńlemeni sheshiń \(\left( x^{2} - 6x \right)^{2} - 2(x - 3)^{2} = 81\). \\
A3. Teńlemeni sheshiń \(\left( x^{2} + 10x + 10 \right)\left( x^{2} + x + 10 \right) = 10x^{2}\) . \\
B1. \(P(2x - 1) + P(x - 1) = 10x^{2} - 12x + 2\) bolsa, \(P(x)\) ni tabıń. \\
B2. Tómendegi aytımdı qálegen natural san ushın matematikalıq induksiya metodi járdeminde dálilleń: \(2^{2} + 6^{2} + \ldots + (4n - 2)^{2} = \frac{4n(2n - 1)(2n + 1)}{3}\). \\
B3. Tómendegi aytımdı qálegen natural san ushın matematikalıq induksiya metodi járdeminde dálilleń: \(5^{2n + 1} + 3^{n + 2} \cdot 2^{n - 1}\) sanı 19 ga eseli ; \\
C1. Birdeylikti dálilleń:\(C_{n}^{j} = C_{n}^{n - j}\); \\
C2. Tuwrı múyeshli úshmúyeshlikda katetlar 7 sm hám 24 sm ga teń. Tuwrı múyeshnıń bissektrisasi túsirilgen. Bu bissektrisa gipotenuzani qanday uzunlikdagi kesindilerga ajıratadı? \\
C3. \(R\) radiusli dóńgelekga bitta umumiy uchga ega bolǵan durıs úshmúyeshlik hám kvadrat ishley sızılǵan. Olardıń kesilisken bóleginıń maydanıni tabıń. \\

\end{tabular}
\vspace{1cm}


\begin{tabular}{m{17cm}}
\textbf{97-variant}
\newline

T1. Simmetriyalıq kóp aǵzalılar. \\
T2. \(b\) parametriniń qanday mánisinde \(x^{3} + 17x^{2} + bx - 17 = 0\) teńlemesiniń korenleri pútin sanlardan turadı? \\
A1. Teńsizlikti sheshiń: \(\frac{x^{3} + 3x^{2} - x - 3}{x^{2} + 3x - 10} < 0\). \\
A2. Teńlemeni sheshiń \((x + 4)(x + 1) - 3\sqrt{x^{2} + 5x + 2} = 6\). \\
A3. Teńlemeni sheshiń. \(\frac{z}{z + 1} - 2\sqrt{\frac{z + 1}{2}} = 3\). \\
B1. \(P(x + n) = (x + n)^{3} + (x - n)^{2} + x + n + 6\) kópaǵzalıi berilgan. \(P(x)\) kópaǵzalıi \(x - n\) ga qaldıqsiz bólinse, \(n\) ni tabıń. \\
B2. Tómendegi aytımdı qálegen natural san ushın matematikalıq induksiya metodi járdeminde dálilleń: \(1^{2} + 2^{2} + 3^{2} + ... + n^{2} = \frac{n(n + 1)(2n + 1)}{6}\); \\
B3. Tómendegi aytımdı qálegen natural san ushın matematikalıq induksiya metodi járdeminde dálilleń: \(6^{2n - 2} + 3^{n + 1} + 3^{n - 1}\) sanı 11 eseli ; \\
C1. \(\left( x^{3} - \frac{3}{x^{2}} \right)^{10}\) binom jayılmasinıń \(x\) qatnashmagan aǵzasın tabıń. \\
C2. Úshmúyeshliktiń perimetri \(4,5dm\) ga teń, bissektrisa bolsa qarama-qarsı tárepni uzınlıqları 6 hám 9 sm ga teń bolǵan kesindilerga ajıratadı. Úshmúyeshliktiń táreplari tabılsın. \\
C3. \(R\) radiusli dóńgelekga bitta umumiy uchga ega bolǵan durıs úshmúyeshlik hám kvadrat ishley sızılǵan. Olardıń kesilisken bóleginıń maydanıni tabıń. \\

\end{tabular}
\vspace{1cm}


\begin{tabular}{m{17cm}}
\textbf{98-variant}
\newline

T1. \(P(x) = (x - 1)^{20}\left( x^{2} + 25 \right)\) kóp aǵzalisınıń koefficentleri qosındısın tabıń. \\
T2. \(a\) parametriniń qanday mánisinde \(P(x) = x^{2017} + ax - 5\) kóp aǵzalısı \((x + 1)\) kóp aǵzalısına qaldıqsız bólinedi? \\
A1. Teńsizlikti sheshiń: \(\sqrt{x^{2} - 4x} > x - 3\). \\
A2. Teńlemeni sheshiń. \(\sqrt{3x^{2} - 2x + 15} + \sqrt{3x^{2} - 2x + 8} = 7\). \\
A3. Teńlemeni sheshiń. \((\sqrt{x + 1} + \sqrt{x})^{3} + (\sqrt{x + 1} + \sqrt{x})^{2} = 2\). \\
B1. \(P(x) = x^{33} - 2ax^{21} + x^{8} + 8\) kópaǵzalıi berilgan. \(a\) nıń qaysi qiymati ushın \(P(x)\) kópaǵzalıi \(x + 1\) ga qaldıqsiz bóliadi? \\
B2. Tómendegi aytımdı qálegen natural san ushın matematikalıq induksiya metodi járdeminde dálilleń: \(1 \cdot 2 + 2 \cdot 3 + 3 \cdot 4 + \ldots + n \cdot (n + 1) = \frac{n \cdot (n + 1) \cdot (n + 2)}{3}\). \\
B3. Tómendegi aytımdı qálegen natural san ushın matematikalıq induksiya metodi járdeminde dálilleń: \(2n^{3} + 3n^{2} + 7n\) sanı 6 ga eseli ; \\
C1. \(C_{n + 4}^{n + 1} - C_{n + 3}^{n} = 15(n + 2)\) bolsa, \(n\) ni tabıń. \\
C2. Ultanlari \(x\) hám 3 bolǵan trapetsiyada diagonallar órtalari arasındaǵı aralıqni \(x\) nıń funksiyasi sifatida ańlatpalań. \(x\) nіnń qanday qiymatida bu aralıq 1 ga teń bóledi? \\
C3. Durıs úshmúyeshliktiń tárepi a ga teń. Tárepini diametr deb esaplap dóńgelek jasalǵan. Úshmúyeshliktiń usı dóńgelekten sirtindaǵi bólegi maydanın tabıń. \\

\end{tabular}
\vspace{1cm}


\begin{tabular}{m{17cm}}
\textbf{99-variant}
\newline

T1. Pifagor teoreması hám onıń dálilleniwleri. \\
T2. Qálegen \(a,b,c \in (0;1)\) sanları ushın \(a(1 - b) > 1/4,\ b(1 - c) > 1/4,\ c(1 - a) > 1/4\) teńsizlikleri bir waqıtta orınlı bola almaytuǵınlıǵin dálilleń. \\
A1. Teńlemeni sheshiń. \(\sqrt[3]{x - 1} + \sqrt[3]{x - 2} - \sqrt{2x - 3} = 0\). \\
A2. Teńlemeni sheshiń. \(\sqrt{x + 8 + 2\sqrt{x + 7}} + \sqrt{x + 1 - \sqrt{x + 7}} = 4\). \\
A3. Teńlemeni sheshiń. \(\sqrt{x} + \frac{2x + 1}{x + 2} = 2\). \\
B1. \(P(x + 2) + P(x - 1) = - 2x^{2} - 2x + 7\) bolsa, \(P(x)\) ni \(x + 4\) ga bólgendegi qaldıqni tabıń. \\
B2. Tómendegi aytımdı qálegen natural san ushın matematikalıq induksiya metodi járdeminde dálilleń: \(1 \cdot 2 + 2 \cdot 3 + 3 \cdot 4 + ... + n(n + 1) = \frac{n(n + 1)(n + 2)}{3}\); \\
B3. Tómendegi aytımdı qálegen natural san ushın matematikalıq induksiya metodi járdeminde dálilleń: \(5^{n} - 4n + 15\) sanı 16 ga eseli ; \\
C1. \((a + b)^{n}\) ańlatpa jayılmasinıń barcha koeffitsiyentlari yig`indisi 4096 ga teń bolsa, Onıń eń úlken koeffitsiyentin tabıń. \\
C2. Úshmúyeshliktiń ultaniga túsirilgen biyikligi \(h\) ga teń. Úshmúyeshliktiń ultaniga parallel kesindi úshmúyeshliktiń maydanıni teń ekiga bóledi. Úshmúyeshliktiń ushınan usı kesindige shekem bolǵan aralıq tabılsın. \\
C3. Ya. Bernulli teńsizligi. Eger\(x \geq - 1\) bolsa, onda qálegen natural \(n\) sanı ushın \((1 + x)^{n} \geq 1 + nx\) teńsizlik orinli bolıwın dálilleń. \\

\end{tabular}
\vspace{1cm}


\begin{tabular}{m{17cm}}
\textbf{100-variant}
\newline

T1. \(x\) ózgeriwshiniń qálegen pútin mánisinde \(ax^{2} + bx + c\) ush aǵzalısınıń mánisi pútin bolıwı ushın \(2a,\ a + b\) hám \(c\) sanlarınıń pútin bolıwı zárurli hám jetkilikli ekenligin dálilleń. \\
T2. Mına \(P(x) = x^{5} + 11x^{4} + 37x^{3} + 35x^{2} - 44x - 40\) kóp aǵzalısı \(Q(x) = x^{2} + 3x + 2\) kóp aǵzalısına qaldıqsız bólinedime? \\
A1. Teńlemeni sheshiń. \(\sqrt{x^{2} + x + 4} + \sqrt{x^{2} + x + 1} = \sqrt{2x^{2} + 2x + 9}\). \\
A2. Teńlemeni sheshiń \((x + 1)^{5} + (x - 1)^{5} = 32x\). \\
A3. Teńlemeni sheshiń. \((x - 4)^{3} + (x - 4)^{2} + (x - 4)(x - 3) + (x - 3)^{2} + (x - 3)^{3} = 6\). \\
B1. \(P(x) = (x - 5)^{2n + 1} + (x - 1)^{2n + 3}\) kópaǵzalını \(x - 3\) ga bólgende qaldıq \(3 \cdot 2^{3n - 4}\) bolsa, \(n\) ni tabıń. \\
B2. Tómendegi aytımdı qálegen natural san ushın matematikalıq induksiya metodi járdeminde dálilleń: \(1^{2} + 3^{2} + 5^{2} + ... + (2n - 1)^{2} = \frac{n\left( 4n^{2} - 1 \right)}{3}\); \\
B3. Tómendegi aytımdı qálegen natural san ushın matematikalıq induksiya metodi járdeminde dálilleń: \(6^{2n - 2} + 3^{n + 1} + 3^{n - 1}\) sanı 11 eseli ; \\
C1. Birdeylikti dálilleń: \(C_{n + k}^{j + k} = \sum_{s = 0}^{k}C_{n}^{j + s}C_{k}^{s}\); \\
C2. Egerteń qaptallı úshmúyeshliktiń perimetri 32 dm , orta sızıǵı 6 dm ga teń bolsa, Onıń táreplari uzınlıqları tabılsın. \\
C3. Eger \(S\) úshmúyeshliktiń maydanı, \(b\) hám \(c\) onıń táreplari bolsa, \(S \leq \frac{b^{2} + c^{2}}{4}\) bolıwın dálilleń. \\

\end{tabular}
\vspace{1cm}



\end{document}
