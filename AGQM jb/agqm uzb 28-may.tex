\documentclass{article}
\usepackage[fontsize=11pt]{fontsize}
\usepackage[utf8]{inputenc}
\usepackage[T2A]{fontenc}
% \usepackage{unicode-math}

\usepackage{array}
\usepackage[a4paper,
left=7mm,
right=5mm,
top=7mm,]{geometry}
\usepackage{amsmath}
% \usepackage{amssymbol}
\usepackage{amsfonts}
\usepackage{setspace}



\renewcommand{\baselinestretch}{1} 

\everymath{\displaystyle}
\everydisplay{\displaystyle}
% \linespread{1.25}

\DeclareMathOperator{\sign}{sign}


\begin{document}

\pagenumbering{gobble}


\begin{tabular}{m{17cm}}
\textbf{1-variant}
\newline

T1. Ushbu \(P(x) = x^{5} + 11x^{4} + 37x^{3} + 35x^{2} - 44x - 40\) ko'phadi \(Q(x) = x^{2} + 3x + 2\) ko'phadiga qoldiqsiz bo'linadimi? \\
T2. \(a\) parametrining qanday qiymatlarida \(P(x) = x^{2017} + ax - 5\) ko'phadi \((x + 1)\) ko'phadiga qoldiqsiz bo'linadi? \\
A1. Tengsizlikni yeching:\(x^{2}\left( x^{4} + 36 \right) - 6\sqrt{3}\left( x^{4} + 4 \right) < 0\). \\
A2. Tenglamani yeching \((x + 1)^{5} + (x - 1)^{5} = 32x\). \\
A3. Tenglamani yeching \(\left( x^{2} - 6x \right)^{2} - 2(x - 3)^{2} = 81\). \\
B1. \(P(x + 3) = x^{2} - x + n\) bo'lsa. \(P(x - 2)\) ko'phadni \(x - 3\) ga bo'lganda qoldiq \(10\) bo'lsa, \(n\) ni toping. \\
B2. Quyidagi mulohazani ixtiyoriy natural son uchun matematik induksiya metodi yordamida isbotlang: \(2^{2} + 6^{2} + \ldots + (4n - 2)^{2} = \frac{4n(2n - 1)(2n + 1)}{3}\). \\
B3. Quyidagi mulohazani ixtiyoriy natural son uchun matematik induksiya metodi yordamida isbotlang: \(5 \cdot 2^{3n - 2} + 3^{3n - 1}\) soni 19 ga karrali \\
C1. \(5C_{n}^{3} = C_{n + 2}^{4}\) bo'lsa, \(n\) ni toping. \\
C2. Ikkita bir xil radiusli doiralar shunday joylashganki, ularning markazlari orasidagi masofa radiusga teng. Doiralar kesishgan qismi yuzining kesishgan qismiga ichki chizilgan kvadrat yuziga nisbatini toping. \\
C3. Ya. Bermlli tengsizligi. Agar \(x \geq - 1\) bo'lsa, u holda ixtiyoriy natural \(n\) soni uchun \((1 + x)^{n} \geq 1 + nx\) tengsizlik o'rinli bo'lishini isbotlang. \\

\end{tabular}
\vspace{1cm}


\begin{tabular}{m{17cm}}
\textbf{2-variant}
\newline

T1. \(P(x) = x^{6} - 3x^{5} + x^{4} - 6x^{2} + 2x - 6\) ko'phadining butun ildizlarini toping. \\
T2. Ushbu \(P(0) = 20\) va \(P(1) = 100\) shartlarini qanoatlantiruvchi \(P(x)\) ko'phadi mavjudmi? \\
A1. Tenglamani yeching \(\left( x^{2} + 10x + 10 \right)\left( x^{2} + x + 10 \right) = 10x^{2}\) . \\
A2. Tenglamani yeching. \(\sqrt{x} + \frac{2x + 1}{x + 2} = 2\). \\
A3. Tenglamani yeching. \(\frac{4x}{x^{2} + x + 3} + \frac{5x}{x^{2} - 5x + 3} = - \frac{3}{2}\). \\
B1. \(P(x) = x^{4} - 2x + 2^{n + 1}\) ko'phadni \(x - 2^{n}\) ga bo'lganda qoldiq \(2^{n - 2}\) bo'lsa, \(n\) ni toping. \\
B2. Quyidagi mulohazani ixtiyoriy natural son uchun matematik induksiya metodi yordamida isbotlang: \(\frac{1}{4 \cdot 5} + \frac{1}{5 \cdot 6} + \frac{1}{6 \cdot 7} + \ldots + \frac{1}{(n + 3) \cdot (n + 4)} = \frac{n}{4 \cdot (n + 4)}\). \\
B3. Quyidagi mulohazani ixtiyoriy natural son uchun matematik induksiya metodi yordamida isbotlang: \(n^{3} + (n + 1)^{3} + (n + 2)^{3}\) soni 9 ga karrali ; \\
C1. Ayniyatni isbotlang:\(\sum_{j = 0}^{n}C_{n}^{j}( - 1)^{j} = 0\); \\
C2. Teng yonli uchburchakning yuzi \(S\) ga teng. Yon tomonlariga tushirilgan medianalari orasidagi burchak \(\alpha\) ga teng. Uchburchak asosini toping. \\
C3. Agar \(S\) uchburchakning yuzi, \(b\) va \(c\) uning tomonlari bo'lsa, \(S \leq \frac{b^{2} + c^{2}}{4}\) bo'lishini isbotlang. \\

\end{tabular}
\vspace{1cm}


\begin{tabular}{m{17cm}}
\textbf{3-variant}
\newline

T1. Pifagor teoremasi va uning isbotlari. \\
T2. Bezu teoremasi va uning qo'llanilishi. \\
A1. Tenglamani yeching \(\left( x^{2} - 4x + 6 \right)^{2} - 4\left( x^{2} - 4x + 6 \right) + 6 = x\). \\
A2. Tenglamani yeching \((x + 4)(x + 1) - 3\sqrt{x^{2} + 5x + 2} = 6\). \\
A3. Tengsizlikni yeching: \(\frac{x^{3} + 3x^{2} - x - 3}{x^{2} + 3x - 10} < 0\). \\
B1. \(P(x + n) = (x + n)^{3} + (x - n)^{2} + x + n + 6\) ko'phadi berilgan. \(P(x)\) ko'phadi \(x - n\) ga qoldiqsiz bo'linsa, \(n\) ni toping. \\
B2. Quyidagi mulohazani ixtiyoriy natural son uchun matematik induksiya metodi yordamida isbotlang: \(\left( 1 - \frac{1}{4} \right)\left( 1 - \frac{1}{9} \right)...\left( 1 - \frac{1}{n^{2}} \right) = \frac{n + 1}{2n}\), \(n \geq 2\) \\
B3. Quyidagi mulohazani ixtiyoriy natural son uchun matematik induksiya metodi yordamida isbotlang: \(5^{2n + 1} + 3^{n + 2} \cdot 2^{n - 1}\) soni 19 ga karrali ; \\
C1. \((a + b)^{n}\) ifoda yoyilmasining barcha koeffitsiyentlari yig`indisi 4096 ga teng bo'lsa, uning eng katta koeffitsiyentin toping. \\
C2. \emph{ABC} uchburchakning \emph{AB} tomonida yotgan \(N\) nuqtadan \(NQ\| AC\) va \(NP\| BC\) to'g'ri chiziqlar o'tkazilgan. Agar \emph{BNQ} uchburchakning yuzi \(S_{1}\) ga, \emph{ANP} uchburchakning yuzi \(S_{2}\) ga tengligi ma'lum bo'lsa, \emph{ABC} uchburchakning yuzini toping. \\
C3. Muntazam uchburchakning tomoni a ga teng. Tomonini diametr deb hisoblab doira yasalgan. Uchburchakning shu doiradan tashqaridagi qismi yuzini toping. \\

\end{tabular}
\vspace{1cm}


\begin{tabular}{m{17cm}}
\textbf{4-variant}
\newline

T1. Matematik induksiya metodi va uning qo'llanilishiga misollar. \\
T2. Simmetrik ko'phadlar. \\
A1. Tenglamani yeching. \((\sqrt{x + 1} + \sqrt{x})^{3} + (\sqrt{x + 1} + \sqrt{x})^{2} = 2\). \\
A2. Tenglamani yeching. \(\sqrt[3]{x} + \sqrt[3]{x - 16} = \sqrt[3]{x - 8}\). \\
A3. Tengsizlikni yeching: \(\sqrt{x + 3} + \sqrt{x - 2} - \sqrt{2x + 4} > 0\). \\
B1. \(P(x + 1) + P(x - 3) = 2x^{2} - 10x + 16\) bo'lsa, \(P(x)\) ni toping. \\
B2. Quyidagi mulohazani ixtiyoriy natural son uchun matematik induksiya metodi yordamida isbotlang: \(1^{2} + 3^{2} + 5^{2} + ... + (2n - 1)^{2} = \frac{n\left( 4n^{2} - 1 \right)}{3}\); \\
B3. Quyidagi mulohazani ixtiyoriy natural son uchun matematik induksiya metodi yordamida isbotlang:\(6^{2n - 2} + 3^{n + 1} + 3^{n - 1}\) soni 11 karrali ; \\
C1. \(x(1 - x)^{4} + x^{2}(1 + 2x)^{8} + x^{3}(1 + 3x)^{12}\) ifodada \(x^{4}\) oldidagi koeffitsiyentni toping. \\
C2. To'g'ri burchakli uchburchakda katetlar 7 sm va 24 sm ga teng. To'g'ri burchakning bissektrisasi o'tkazilgan. Bu bissektrisa gipotenuzani qanday uzunlikdagi kesmalarga ajratadi? \\
C3. Ya. Bermlli tengsizligi. Agar \(x \geq - 1\) bo'lsa, u holda ixtiyoriy natural \(n\) soni uchun \((1 + x)^{n} \geq 1 + nx\) tengsizlik o'rinli bo'lishini isbotlang. \\

\end{tabular}
\vspace{1cm}


\begin{tabular}{m{17cm}}
\textbf{5-variant}
\newline

T1. \(n\) darajaning qanday qiymatlarida \((x + 1)^{n} + (x - 1)^{n}\) ifodasi \(x\) ifodaga qoldiqsiz bo'linadi? \\
T2. \(b\) parametrining qanday qiymatlarida \(x^{3} + 17x^{2} + bx - 17 = 0\) tenglamasining ildizlari butun sonlardan iborat bo'ladi? \\
A1. Tenglamani yeching. \((x - 4)^{3} + (x - 4)^{2} + (x - 4)(x - 3) + (x - 3)^{2} + (x - 3)^{3} = 6\). \\
A2. Tengsizlikni yeching: \(\sqrt{x^{2} - 4x} > x - 3\). \\
A3. Tenglamani yeching. \(\frac{z}{z + 1} - 2\sqrt{\frac{z + 1}{2}} = 3\). \\
B1. \(P(x)\) ko'phadni \(3x^{2} - 4x + 1\) ga bo'lganimizda qoldiq \(6x - 11\) bo'lsa, \(P(x)\) ko'phadni \(3x - 1\)ga bo'lganda qoldiqni toping. \\
B2. Quyidagi mulohazani ixtiyoriy natural son uchun matematik induksiya metodi yordamida isbotlang: \(\frac{1}{1 \cdot 4} + \frac{1}{4 \cdot 7} + \frac{1}{7 \cdot 10} + \ldots + \frac{1}{(3n - 2) \cdot (3n + 1)} = \frac{n}{(3n + 1)}\). \\
B3. Quyidagi mulohazani ixtiyoriy natural son uchun matematik induksiya metodi yordamida isbotlang: \(5^{n} - 4n + 15\) soni 16 ga karrali ; \\
C1. \(\left( x\sqrt{x} - \frac{1}{x^{4}} \right)^{n}\) binom yoyilmasida 3-had koeffitsiyenti 2-had koeffitsiyentidan 44 ga katta.Ozod hadini toping. \\
C2. Parallelogrammning tomonlari \(a\) va \(b\), ular orasidagi burchak \(\alpha\). bo'lsa, paralllelogramm ichki burchaklari bissektrisalari kesishishidan hosil bo'lgan to'rtburchak yuzini toping. \\
C3. Muntazam uchburchakning tomoni a ga teng. Tomonini diametr deb hisoblab doira yasalgan. Uchburchakning shu doiradan tashqaridagi qismi yuzini toping. \\

\end{tabular}
\vspace{1cm}


\begin{tabular}{m{17cm}}
\textbf{6-variant}
\newline

T1. \(x\) o'zgaruvchining ixtiyoriy butun qiymatida \(ax^{2} + bx + c\) uchhadining qiymati butun bo'lishi uchun \(2a,\ a + b\) va \(c\) sonlarining butun bo'lishi zarur va yetarli ekanligini isbotlang. \\
T2. \(2^{81} + 1\) soni 9 soniga qoldiqsiz bo'linishini isbotlang. \\
A1. Tenglamani yeching. \(\sqrt[3]{x - 1} + \sqrt[3]{x - 2} - \sqrt{2x - 3} = 0\). \\
A2. Tenglamani yeching. \(\sqrt{3x^{2} - 2x + 15} + \sqrt{3x^{2} - 2x + 8} = 7\). \\
A3. Tenglamani yeching \(\sqrt{\frac{18 - 7x - x^{2}}{8 - 6x + x^{2}}} + \sqrt{\frac{8 - 6x + x^{2}}{18 - 7x - x^{2}}} = \frac{13}{6}\). \\
B1. \(P(x + 3)\) ko'phadni \(x + 1\) ga bo'lganda qoldiq -3, \(Q(2x - 1)\) ko'phadni \(x - 1\)ga bo'lganda qoldiq 2 bo'lsa, \(P(x + 4) + x^{2}Q(x + 3)\) ko'phadni \(x + 2\) ga bo'lgandagi qoldiqni toping. \\
B2. Quyidagi mulohazani ixtiyoriy natural son uchun matematik induksiya metodi yordamida isbotlang: \(1 \cdot 2 + 2 \cdot 3 + 3 \cdot 4 + ... + n(n + 1) = \frac{n(n + 1)(n + 2)}{3}\); \\
B3. Quyidagi mulohazani ixtiyoriy natural son uchun matematik induksiya metodi yordamida isbotlang: \(5^{n + 2} + 26 \cdot 5^{n} + 8^{2n + 1}\) soni 59 ga karrali; \\
C1. \(\left( 2x^{\ ^{2}} - \frac{b}{2x^{3}} \right)^{10}\) binom yoyilmasining \(x\) qatnashmagan hadin toping. \\
C2. To'g'ri burchakli uchburchak o'tkir burchaklarining bisєektrisalari AD va BK \(AB^{2} = AD \cdot BK\) bo'lsa, uchburchakning burchaklarini toping. \\
C3. Agar \(S\) uchburchakning yuzi, \(b\) va \(c\) uning tomonlari bo'lsa, \(S \leq \frac{b^{2} + c^{2}}{4}\) bo'lishini isbotlang. \\

\end{tabular}
\vspace{1cm}


\begin{tabular}{m{17cm}}
\textbf{7-variant}
\newline

T1. Yig'indisi birga teng bo'lgan \(x,y,z\) musbat sonlari uchun \(\frac{1}{x} + \frac{1}{y} + \frac{1}{z} \geq 9\) tengsizligi o'rinli bo'lishini isbotlang. \\
T2. Fales teoremasi va uning qo'llanilishi. \\
A1. Tenglamani yeching. \(\sqrt{x + 8 + 2\sqrt{x + 7}} + \sqrt{x + 1 - \sqrt{x + 7}} = 4\). \\
A2. Tenglamani yeching. \(\sqrt{x^{2} + x + 4} + \sqrt{x^{2} + x + 1} = \sqrt{2x^{2} + 2x + 9}\). \\
A3. Tenglamani yeching. \(\sqrt{x^{2} + x + 4} + \sqrt{x^{2} + x + 1} = \sqrt{2x^{2} + 2x + 9}\). \\
B1. \(P(2x - 1) + P(x - 1) = 10x^{2} - 12x + 2\) bo'lsa, \(P(x)\) ni toping. \\
B2. Quyidagi mulohazani ixtiyoriy natural son uchun matematik induksiya metodi yordamida isbotlang: \(1^{2} + 2^{2} + 3^{2} + ... + n^{2} = \frac{n(n + 1)(2n + 1)}{6}\); \\
B3. Quyidagi mulohazani ixtiyoriy natural son uchun matematik induksiya metodi yordamida isbotlang: \(n\left( 2n^{2} - 3n + 1 \right)\) soni 6 ga karrali ; \\
C1. Teńlemeni sheshiń \(\frac{C_{2x}^{x + 1}}{C_{2x + 1}^{x - 1}} = \frac{2}{3}\), \(x \in N\) \\
C2. Agar teng yonli uchburchakning perimetri 32 dm , o'rta chizig'i 6 dm ga teng bo'lsa, uning tomonlari uzunliklari topilsin. \\
C3. Muntazam uchburchakning tomoni a ga teng. Tomonini diametr deb hisoblab doira yasalgan. Uchburchakning shu doiradan tashqaridagi qismi yuzini toping. \\

\end{tabular}
\vspace{1cm}


\begin{tabular}{m{17cm}}
\textbf{8-variant}
\newline

T1. \(P(x) = (x - 1)^{20}\left( x^{2} + 25 \right)\) ko'phadining koeffitsentlari yig'indisini toping. \\
T2. Haqiqiy \(a_{1},\ a_{2},\ .\ .\ .\ ,\ a_{n},\ b_{1},\ b_{2},\ .\ .\ .\ ,\ b_{n}\) sonlari uchun \(\left( a_{1}b_{1} + a_{2}b_{2} + \ .\ .\ .\  + a_{n}b_{n} \right)^{2} \leq \left( a_{1}^{2} + a_{2}^{2} + \ .\ .\ .\  + a_{n}^{2} \right)\left( b_{1}^{2} + b_{2}^{2} + \ .\ .\ .\  + b_{n}^{2} \right)\) \\
A1. Tenglamani yeching \(\left( x^{2} - 4x + 6 \right)^{2} - 4\left( x^{2} - 4x + 6 \right) + 6 = x\). \\
A2. Tenglamani yeching. \(\sqrt[3]{x} + \sqrt[3]{x - 16} = \sqrt[3]{x - 8}\). \\
A3. Tengsizlikni yeching:\(x^{2}\left( x^{4} + 36 \right) - 6\sqrt{3}\left( x^{4} + 4 \right) < 0\). \\
B1. \(P(x) = x^{33} - 2ax^{21} + x^{8} + 8\) ko'phadi berilgan. \(a\) ning qaysi qiymati uchun \(P(x)\) ko'phadi \(x + 1\) ga qoldiqsiz bo'liadi? \\
B2. Quyidagi mulohazani ixtiyoriy natural son uchun matematik induksiya metodi yordamida isbotlang: \(1^{3} + 2^{3} + 3^{3} + ... + n^{3} = \left( \frac{n(n + 1)}{2} \right)^{2}\); \\
B3. Quyidagi mulohazani ixtiyoriy natural son uchun matematik induksiya metodi yordamida isbotlang: \(7^{n} - 1\) soni 6 ga karrali; \\
C1. Ayniyatni isbotlang:\(C_{n}^{j} = C_{n}^{n - j}\); \\
C2. \(ABCD(AD\| BC)\) trapetsiya diagonallari \(O\) nuqtada kesishadi. Agar \emph{AOD} uchburchakning yuzi \(a^{2}\) ga, \emph{BOC} uchburchakning yuzi \(b^{2}\) ga tengligi ma'lum bo'lsa, trapesiya yuzini toping. \\
C3. Agar \(S\) uchburchakning yuzi, \(b\) va \(c\) uning tomonlari bo'lsa, \(S \leq \frac{b^{2} + c^{2}}{4}\) bo'lishini isbotlang. \\

\end{tabular}
\vspace{1cm}


\begin{tabular}{m{17cm}}
\textbf{9-variant}
\newline

T1. Ixtiyoriy \(a,b,c \in (0;1)\) sonlari uchun \(a(1 - b) > 1/4,\ b(1 - c) > 1/4,\ c(1 - a) > 1/4\) tengsizliklari bir vaqtda o'rinli bo'la olmasligini isbotlang. \\
T2. Ixtiyoriy \(a\) parametri va \(x\) uchun \(x(a - x) \leq a^{2}/4\) tengsizligi o'rinli bo'lishini isbotlang. \\
A1. Tenglamani yeching \(\sqrt{\frac{18 - 7x - x^{2}}{8 - 6x + x^{2}}} + \sqrt{\frac{8 - 6x + x^{2}}{18 - 7x - x^{2}}} = \frac{13}{6}\). \\
A2. Tengsizlikni yeching: \(\sqrt{x^{2} - 4x} > x - 3\). \\
A3. Tenglamani yeching. \(\frac{4x}{x^{2} + x + 3} + \frac{5x}{x^{2} - 5x + 3} = - \frac{3}{2}\). \\
B1. \(P(x + 2) + P(x - 1) = - 2x^{2} - 2x + 7\) bo'lsa, \(P(x)\) ni \(x + 4\) ga bo'lgandagi qoldiqni toping. \\
B2. Quyidagi mulohazani ixtiyoriy natural son uchun matematik induksiya metodi yordamida isbotlang: \(1 \cdot 1! + 2 \cdot 2! + 3 \cdot 3! + \ldots + n \cdot n! = (n + 1)! - 1\). \\
B3. Quyidagi mulohazani ixtiyoriy natural son uchun matematik induksiya metodi yordamida isbotlang: \(2n^{3} + 3n^{2} + 7n\) soni 6 ga karrali ; \\
C1. \(C_{n + 4}^{n + 1} - C_{n + 3}^{n} = 15(n + 2)\) bo'lsa, \(n\) ni toping. \\
C2. Muntazam uchburchakning uchlari uchta parallel to'g'ri chiziqlarda yotadi. Agar o'rtadagi to'g'ri chiziqdan chekkalardagi to'g'ri chiziklargacha bo'lgan masofa \(a\) va \(b\) ga teng bo'lsa, uchburchakning tomonini toping. \\
C3. Ya. Bermlli tengsizligi. Agar \(x \geq - 1\) bo'lsa, u holda ixtiyoriy natural \(n\) soni uchun \((1 + x)^{n} \geq 1 + nx\) tengsizlik o'rinli bo'lishini isbotlang. \\

\end{tabular}
\vspace{1cm}


\begin{tabular}{m{17cm}}
\textbf{10-variant}
\newline

T1. Koshi tengsizligini isbotlang. \\
T2. Kombinatorika elementlari va Nyuton binomi. \\
A1. Tenglamani yeching. \(\sqrt{x} + \frac{2x + 1}{x + 2} = 2\). \\
A2. Tenglamani yeching. \(\sqrt{3x^{2} - 2x + 15} + \sqrt{3x^{2} - 2x + 8} = 7\). \\
A3. Tenglamani yeching. \((x - 4)^{3} + (x - 4)^{2} + (x - 4)(x - 3) + (x - 3)^{2} + (x - 3)^{3} = 6\). \\
B1. \(P(x) = (x - 5)^{2n + 1} + (x - 1)^{2n + 3}\) ko'phadni \(x - 3\) ga bo'lganda qoldiq \(3 \cdot 2^{3n - 4}\) bo'lsa, \(n\) ni toping. \\
B2. Quyidagi mulohazani ixtiyoriy natural son uchun matematik induksiya metodi yordamida isbotlang: \(1 \cdot 2 + 2 \cdot 3 + 3 \cdot 4 + \ldots + n \cdot (n + 1) = \frac{n \cdot (n + 1) \cdot (n + 2)}{3}\). \\
B3. Quyidagi mulohazani ixtiyoriy natural son uchun matematik induksiya metodi yordamida isbotlang: \(5^{n + 2} + 26 \cdot 5^{n} + 8^{2n + 1}\) soni 59 ga karrali; \\
C1. Ayniyatni isbotlang:\(C_{n + 2}^{j + 2} = C_{n}^{j} + 2C_{n}^{j + 1} + C_{n}^{j + 2}\); \\
C2. To'g'ri burchakli uchburchakning balandligi gipotenuzani uzunliklari 18 va 32 sm ga teng bo'lgan kesmalarga ajratadi. Uchburchakning yuzi hisoblansin. \\
C3. Agar \(S\) uchburchakning yuzi, \(b\) va \(c\) uning tomonlari bo'lsa, \(S \leq \frac{b^{2} + c^{2}}{4}\) bo'lishini isbotlang. \\

\end{tabular}
\vspace{1cm}


\begin{tabular}{m{17cm}}
\textbf{11-variant}
\newline

T1. Matematik induksiya metodi va uning qo'llanilishiga misollar. \\
T2. \(P(x) = (x - 1)^{20}\left( x^{2} + 25 \right)\) ko'phadining koeffitsentlari yig'indisini toping. \\
A1. Tengsizlikni yeching: \(\sqrt{x + 3} + \sqrt{x - 2} - \sqrt{2x + 4} > 0\). \\
A2. Tengsizlikni yeching: \(\frac{x^{3} + 3x^{2} - x - 3}{x^{2} + 3x - 10} < 0\). \\
A3. Tenglamani yeching \((x + 4)(x + 1) - 3\sqrt{x^{2} + 5x + 2} = 6\). \\
B1. \(P(x) = x^{33} - 2ax^{21} + x^{8} + 8\) ko'phadi berilgan. \(a\) ning qaysi qiymati uchun \(P(x)\) ko'phadi \(x + 1\) ga qoldiqsiz bo'liadi? \\
B2. Quyidagi mulohazani ixtiyoriy natural son uchun matematik induksiya metodi yordamida isbotlang: \(\frac{1}{1 \cdot 5} + \frac{1}{5 \cdot 9} + ... + \frac{1}{(4n - 3)(4n + 1)} = \frac{n}{4n + 1}\); \\
B3. Quyidagi mulohazani ixtiyoriy natural son uchun matematik induksiya metodi yordamida isbotlang: \(n\left( 2n^{2} - 3n + 1 \right)\) soni 6 ga karrali ; \\
C1. Teńsizlikti sheshiń \(C_{13}^{x} < C_{13}^{x + 2}\), \(x \in N\) \\
C2. To'g'ri burchakli uchburchakning katetlari \(b\) va \(c\) ga teng. To'g'ri burchak bissektrisasining uzunligi topilsin. \\
C3. Muntazam uchburchakning tomoni a ga teng. Tomonini diametr deb hisoblab doira yasalgan. Uchburchakning shu doiradan tashqaridagi qismi yuzini toping. \\

\end{tabular}
\vspace{1cm}


\begin{tabular}{m{17cm}}
\textbf{12-variant}
\newline

T1. \(2^{81} + 1\) soni 9 soniga qoldiqsiz bo'linishini isbotlang. \\
T2. Kombinatorika elementlari va Nyuton binomi. \\
A1. Tenglamani yeching \(\left( x^{2} + 10x + 10 \right)\left( x^{2} + x + 10 \right) = 10x^{2}\) . \\
A2. Tenglamani yeching. \(\sqrt[3]{x - 1} + \sqrt[3]{x - 2} - \sqrt{2x - 3} = 0\). \\
A3. Tenglamani yeching. \((\sqrt{x + 1} + \sqrt{x})^{3} + (\sqrt{x + 1} + \sqrt{x})^{2} = 2\). \\
B1. \(P(x + 3) = x^{2} - x + n\) bo'lsa. \(P(x - 2)\) ko'phadni \(x - 3\) ga bo'lganda qoldiq \(10\) bo'lsa, \(n\) ni toping. \\
B2. Quyidagi mulohazani ixtiyoriy natural son uchun matematik induksiya metodi yordamida isbotlang: \(1^{3} + 2^{3} + 3^{3} + ... + n^{3} = \left( \frac{n(n + 1)}{2} \right)^{2}\); \\
B3. Quyidagi mulohazani ixtiyoriy natural son uchun matematik induksiya metodi yordamida isbotlang: \(5^{2n + 1} + 3^{n + 2} \cdot 2^{n - 1}\) soni 19 ga karrali ; \\
C1. Teńsizlikti sheshiń \(5C_{x}^{3} < C_{x + 2}^{4}\), \(x \in N\) \\
C2. Bir burchagi \(60^{{^\circ}}\) bo'lgan uchburchakka ichki chizilgan aylananing urinish nuqtasi shu burchakka qarama- qarshi tomonini \(a\) va \(b\) kesmalarga ajratadi. Uchburchak yuzini toping. \\
C3. Ya. Bermlli tengsizligi. Agar \(x \geq - 1\) bo'lsa, u holda ixtiyoriy natural \(n\) soni uchun \((1 + x)^{n} \geq 1 + nx\) tengsizlik o'rinli bo'lishini isbotlang. \\

\end{tabular}
\vspace{1cm}


\begin{tabular}{m{17cm}}
\textbf{13-variant}
\newline

T1. \(n\) darajaning qanday qiymatlarida \((x + 1)^{n} + (x - 1)^{n}\) ifodasi \(x\) ifodaga qoldiqsiz bo'linadi? \\
T2. Fales teoremasi va uning qo'llanilishi. \\
A1. Tenglamani yeching \((x + 1)^{5} + (x - 1)^{5} = 32x\). \\
A2. Tenglamani yeching. \(\sqrt{x + 8 + 2\sqrt{x + 7}} + \sqrt{x + 1 - \sqrt{x + 7}} = 4\). \\
A3. Tenglamani yeching. \(\frac{z}{z + 1} - 2\sqrt{\frac{z + 1}{2}} = 3\). \\
B1. \(P(x + 2) + P(x - 1) = - 2x^{2} - 2x + 7\) bo'lsa, \(P(x)\) ni \(x + 4\) ga bo'lgandagi qoldiqni toping. \\
B2. Quyidagi mulohazani ixtiyoriy natural son uchun matematik induksiya metodi yordamida isbotlang: \(2^{2} + 6^{2} + \ldots + (4n - 2)^{2} = \frac{4n(2n - 1)(2n + 1)}{3}\). \\
B3. Quyidagi mulohazani ixtiyoriy natural son uchun matematik induksiya metodi yordamida isbotlang: \(7^{n} - 1\) soni 6 ga karrali; \\
C1. \((x + 1)^{3} + (x + 1)^{4} + (x + 1)^{5} + ... + (x + 1)^{10}\) ifodada \(x^{3}\) oldidagi koeffitsiyentni toping \\
C2. \(R\) radiusli doiraga bitta umumiy uchga ega bo'lgan muntazam uchburchak va kvadrat ichki chizilgan. Ularning kesishgan qismining yuzini toping. \\
C3. Ya. Bermlli tengsizligi. Agar \(x \geq - 1\) bo'lsa, u holda ixtiyoriy natural \(n\) soni uchun \((1 + x)^{n} \geq 1 + nx\) tengsizlik o'rinli bo'lishini isbotlang. \\

\end{tabular}
\vspace{1cm}


\begin{tabular}{m{17cm}}
\textbf{14-variant}
\newline

T1. Haqiqiy \(a_{1},\ a_{2},\ .\ .\ .\ ,\ a_{n},\ b_{1},\ b_{2},\ .\ .\ .\ ,\ b_{n}\) sonlari uchun \(\left( a_{1}b_{1} + a_{2}b_{2} + \ .\ .\ .\  + a_{n}b_{n} \right)^{2} \leq \left( a_{1}^{2} + a_{2}^{2} + \ .\ .\ .\  + a_{n}^{2} \right)\left( b_{1}^{2} + b_{2}^{2} + \ .\ .\ .\  + b_{n}^{2} \right)\) \\
T2. Bezu teoremasi va uning qo'llanilishi. \\
A1. Tenglamani yeching \(\left( x^{2} - 6x \right)^{2} - 2(x - 3)^{2} = 81\). \\
A2. Tenglamani yeching. \(\sqrt{x} + \frac{2x + 1}{x + 2} = 2\). \\
A3. Tenglamani yeching \(\left( x^{2} - 4x + 6 \right)^{2} - 4\left( x^{2} - 4x + 6 \right) + 6 = x\). \\
B1. \(P(x)\) ko'phadni \(3x^{2} - 4x + 1\) ga bo'lganimizda qoldiq \(6x - 11\) bo'lsa, \(P(x)\) ko'phadni \(3x - 1\)ga bo'lganda qoldiqni toping. \\
B2. Quyidagi mulohazani ixtiyoriy natural son uchun matematik induksiya metodi yordamida isbotlang: \(1 \cdot 1! + 2 \cdot 2! + 3 \cdot 3! + \ldots + n \cdot n! = (n + 1)! - 1\). \\
B3. Quyidagi mulohazani ixtiyoriy natural son uchun matematik induksiya metodi yordamida isbotlang: \(n^{3} + (n + 1)^{3} + (n + 2)^{3}\) soni 9 ga karrali ; \\
C1. Ayniyatni isbotlang: \(C_{n + 1}^{j + 1} = C_{n}^{j} + C_{n}^{j + 1}\); \\
C2. Markazlari \(O_{1}\) va \(O_{2}\) nuqtalarda va radiusi \(R\) bo'lgan ikkita teng aylanalar tashqi urinadi. \(l\) to'g'ri chiziq bu aylanalarni A, B, C va \(D\) nuqtalarda shunday kesib o'tadiki, \(AB = BC = CD\) bo'ladi. \(O_{1}ADO_{2}\) to'rtburchak yuzini toping. \\
C3. Muntazam uchburchakning tomoni a ga teng. Tomonini diametr deb hisoblab doira yasalgan. Uchburchakning shu doiradan tashqaridagi qismi yuzini toping. \\

\end{tabular}
\vspace{1cm}


\begin{tabular}{m{17cm}}
\textbf{15-variant}
\newline

T1. \(b\) parametrining qanday qiymatlarida \(x^{3} + 17x^{2} + bx - 17 = 0\) tenglamasining ildizlari butun sonlardan iborat bo'ladi? \\
T2. Simmetrik ko'phadlar. \\
A1. Tenglamani yeching. \((x - 4)^{3} + (x - 4)^{2} + (x - 4)(x - 3) + (x - 3)^{2} + (x - 3)^{3} = 6\). \\
A2. Tenglamani yeching. \(\sqrt{3x^{2} - 2x + 15} + \sqrt{3x^{2} - 2x + 8} = 7\). \\
A3. Tenglamani yeching. \(\sqrt{x^{2} + x + 4} + \sqrt{x^{2} + x + 1} = \sqrt{2x^{2} + 2x + 9}\). \\
B1. \(P(2x - 1) + P(x - 1) = 10x^{2} - 12x + 2\) bo'lsa, \(P(x)\) ni toping. \\
B2. Quyidagi mulohazani ixtiyoriy natural son uchun matematik induksiya metodi yordamida isbotlang: \(1^{2} + 2^{2} + 3^{2} + ... + n^{2} = \frac{n(n + 1)(2n + 1)}{6}\); \\
B3. Quyidagi mulohazani ixtiyoriy natural son uchun matematik induksiya metodi yordamida isbotlang: \(2n^{3} + 3n^{2} + 7n\) soni 6 ga karrali ; \\
C1. Ayniyatni isbotlang: \(C_{n + k}^{j + k} = \sum_{s = 0}^{k}C_{n}^{j + s}C_{k}^{s}\); \\
C2. Asoslari \(x\) va 3 bo'lgan trapetsiyada diagonallar o'rtalari orasidagi masofani \(x\) ning funksiyasi sifatida ifodalang. \(x\) nіnng qanday qiymatida bu masofa 1 ga teng bo'ladi? \\
C3. Agar \(S\) uchburchakning yuzi, \(b\) va \(c\) uning tomonlari bo'lsa, \(S \leq \frac{b^{2} + c^{2}}{4}\) bo'lishini isbotlang. \\

\end{tabular}
\vspace{1cm}


\begin{tabular}{m{17cm}}
\textbf{16-variant}
\newline

T1. Koshi tengsizligini isbotlang. \\
T2. Pifagor teoremasi va uning isbotlari. \\
A1. Tenglamani yeching. \((\sqrt{x + 1} + \sqrt{x})^{3} + (\sqrt{x + 1} + \sqrt{x})^{2} = 2\). \\
A2. Tenglamani yeching. \(\sqrt[3]{x} + \sqrt[3]{x - 16} = \sqrt[3]{x - 8}\). \\
A3. Tengsizlikni yeching: \(\sqrt{x^{2} - 4x} > x - 3\). \\
B1. \(P(x + 3)\) ko'phadni \(x + 1\) ga bo'lganda qoldiq -3, \(Q(2x - 1)\) ko'phadni \(x - 1\)ga bo'lganda qoldiq 2 bo'lsa, \(P(x + 4) + x^{2}Q(x + 3)\) ko'phadni \(x + 2\) ga bo'lgandagi qoldiqni toping. \\
B2. Quyidagi mulohazani ixtiyoriy natural son uchun matematik induksiya metodi yordamida isbotlang: \(\frac{1}{1 \cdot 5} + \frac{1}{5 \cdot 9} + ... + \frac{1}{(4n - 3)(4n + 1)} = \frac{n}{4n + 1}\); \\
B3. Quyidagi mulohazani ixtiyoriy natural son uchun matematik induksiya metodi yordamida isbotlang: \(5^{n} - 4n + 15\) soni 16 ga karrali ; \\
C1. \(\left( \sqrt{x} + \frac{1}{\sqrt[3]{x^{2}}} \right)^{n}\) binom yoyilmasida 5-had koeffitsiyentining 3-had koeffitsiyentiga nisbati 7:2 ga teng. \(x\) ning darajasi 1 ga teng bo'lgan ahadin toping. \\
C2. Uchburchakning perimetri \(4,5dm\) ga teng, bissektrisa esa qarshi tomonni uzunliklari 6 va 9 sm ga teng bo'lgan kesmalarga ajratadi. Uchburchakning tomonlari topilsin. \\
C3. Muntazam uchburchakning tomoni a ga teng. Tomonini diametr deb hisoblab doira yasalgan. Uchburchakning shu doiradan tashqaridagi qismi yuzini toping. \\

\end{tabular}
\vspace{1cm}


\begin{tabular}{m{17cm}}
\textbf{17-variant}
\newline

T1. Yig'indisi birga teng bo'lgan \(x,y,z\) musbat sonlari uchun \(\frac{1}{x} + \frac{1}{y} + \frac{1}{z} \geq 9\) tengsizligi o'rinli bo'lishini isbotlang. \\
T2. \(P(x) = x^{6} - 3x^{5} + x^{4} - 6x^{2} + 2x - 6\) ko'phadining butun ildizlarini toping. \\
A1. Tenglamani yeching \(\sqrt{\frac{18 - 7x - x^{2}}{8 - 6x + x^{2}}} + \sqrt{\frac{8 - 6x + x^{2}}{18 - 7x - x^{2}}} = \frac{13}{6}\). \\
A2. Tengsizlikni yeching:\(x^{2}\left( x^{4} + 36 \right) - 6\sqrt{3}\left( x^{4} + 4 \right) < 0\). \\
A3. Tenglamani yeching \((x + 4)(x + 1) - 3\sqrt{x^{2} + 5x + 2} = 6\). \\
B1. \(P(x + n) = (x + n)^{3} + (x - n)^{2} + x + n + 6\) ko'phadi berilgan. \(P(x)\) ko'phadi \(x - n\) ga qoldiqsiz bo'linsa, \(n\) ni toping. \\
B2. Quyidagi mulohazani ixtiyoriy natural son uchun matematik induksiya metodi yordamida isbotlang: \(\left( 1 - \frac{1}{4} \right)\left( 1 - \frac{1}{9} \right)...\left( 1 - \frac{1}{n^{2}} \right) = \frac{n + 1}{2n}\), \(n \geq 2\) \\
B3. Quyidagi mulohazani ixtiyoriy natural son uchun matematik induksiya metodi yordamida isbotlang:\(6^{2n - 2} + 3^{n + 1} + 3^{n - 1}\) soni 11 karrali ; \\
C1. Ayniyatni isbotlang: \(\sum_{j = 0}^{n}C_{n}^{j} = 2^{n}\); \\
C2. Uchburchakning a, b va \(c\) tomonlari arifmetik progressiya tashkil qiladi. \(ac = 6Rr\) bo'lishini isbotlang. Bu yerda \(R\) va \(r\) tashqi va ichki chizilgan aylanalarning radiuslari. \\
C3. Agar \(S\) uchburchakning yuzi, \(b\) va \(c\) uning tomonlari bo'lsa, \(S \leq \frac{b^{2} + c^{2}}{4}\) bo'lishini isbotlang. \\

\end{tabular}
\vspace{1cm}


\begin{tabular}{m{17cm}}
\textbf{18-variant}
\newline

T1. Ixtiyoriy \(a\) parametri va \(x\) uchun \(x(a - x) \leq a^{2}/4\) tengsizligi o'rinli bo'lishini isbotlang. \\
T2. Ushbu \(P(x) = x^{5} + 11x^{4} + 37x^{3} + 35x^{2} - 44x - 40\) ko'phadi \(Q(x) = x^{2} + 3x + 2\) ko'phadiga qoldiqsiz bo'linadimi? \\
A1. Tenglamani yeching. \(\sqrt{x + 8 + 2\sqrt{x + 7}} + \sqrt{x + 1 - \sqrt{x + 7}} = 4\). \\
A2. Tenglamani yeching \(\left( x^{2} - 6x \right)^{2} - 2(x - 3)^{2} = 81\). \\
A3. Tenglamani yeching. \(\sqrt[3]{x - 1} + \sqrt[3]{x - 2} - \sqrt{2x - 3} = 0\). \\
B1. \(P(x) = x^{4} - 2x + 2^{n + 1}\) ko'phadni \(x - 2^{n}\) ga bo'lganda qoldiq \(2^{n - 2}\) bo'lsa, \(n\) ni toping. \\
B2. Quyidagi mulohazani ixtiyoriy natural son uchun matematik induksiya metodi yordamida isbotlang: \(1 \cdot 2 + 2 \cdot 3 + 3 \cdot 4 + ... + n(n + 1) = \frac{n(n + 1)(n + 2)}{3}\); \\
B3. Quyidagi mulohazani ixtiyoriy natural son uchun matematik induksiya metodi yordamida isbotlang: \(5 \cdot 2^{3n - 2} + 3^{3n - 1}\) soni 19 ga karrali \\
C1. \(\left( x^{3} - \frac{3}{x^{2}} \right)^{10}\) binom yoyilmasining \(x\) qatnashmagan hadin toping. \\
C2. \emph{ABCD} parallelogrammning \emph{AD} tomoni \(n\) ta teng bo'lakka bo'lingan. Birinchi bo'linish nuqtasi \(P\) va \(B\) uch bilan birlashtirilgan. \emph{BP} to'g'ri chiziq \emph{AC} dioganaldan uning \(\frac{1}{n + 1}\) qismiga teng \emph{AQ} kesma ajratishini isbotlang. \\
C3. Ya. Bermlli tengsizligi. Agar \(x \geq - 1\) bo'lsa, u holda ixtiyoriy natural \(n\) soni uchun \((1 + x)^{n} \geq 1 + nx\) tengsizlik o'rinli bo'lishini isbotlang. \\

\end{tabular}
\vspace{1cm}


\begin{tabular}{m{17cm}}
\textbf{19-variant}
\newline

T1. \(x\) o'zgaruvchining ixtiyoriy butun qiymatida \(ax^{2} + bx + c\) uchhadining qiymati butun bo'lishi uchun \(2a,\ a + b\) va \(c\) sonlarining butun bo'lishi zarur va yetarli ekanligini isbotlang. \\
T2. Ixtiyoriy \(a,b,c \in (0;1)\) sonlari uchun \(a(1 - b) > 1/4,\ b(1 - c) > 1/4,\ c(1 - a) > 1/4\) tengsizliklari bir vaqtda o'rinli bo'la olmasligini isbotlang. \\
A1. Tengsizlikni yeching: \(\sqrt{x + 3} + \sqrt{x - 2} - \sqrt{2x + 4} > 0\). \\
A2. Tengsizlikni yeching: \(\frac{x^{3} + 3x^{2} - x - 3}{x^{2} + 3x - 10} < 0\). \\
A3. Tenglamani yeching \(\left( x^{2} + 10x + 10 \right)\left( x^{2} + x + 10 \right) = 10x^{2}\) . \\
B1. \(P(x) = (x - 5)^{2n + 1} + (x - 1)^{2n + 3}\) ko'phadni \(x - 3\) ga bo'lganda qoldiq \(3 \cdot 2^{3n - 4}\) bo'lsa, \(n\) ni toping. \\
B2. Quyidagi mulohazani ixtiyoriy natural son uchun matematik induksiya metodi yordamida isbotlang: \(\frac{1}{1 \cdot 4} + \frac{1}{4 \cdot 7} + \frac{1}{7 \cdot 10} + \ldots + \frac{1}{(3n - 2) \cdot (3n + 1)} = \frac{n}{(3n + 1)}\). \\
B3. Quyidagi mulohazani ixtiyoriy natural son uchun matematik induksiya metodi yordamida isbotlang: \(7^{n} - 1\) soni 6 ga karrali; \\
C1. Tengsizlikni yeching: \(C_{10}^{x - 1} > 2C_{10}^{x}\) \\
C2. To'g'ri burchakli uchburchakning balandligi gipotenuzani uzunliklari \emph{x} va \emph{y} ga teng bo'lgan kesmalarga ajratadi. Uchburchakning yuzi hisoblansin. \\
C3. Ya. Bermlli tengsizligi. Agar \(x \geq - 1\) bo'lsa, u holda ixtiyoriy natural \(n\) soni uchun \((1 + x)^{n} \geq 1 + nx\) tengsizlik o'rinli bo'lishini isbotlang. \\

\end{tabular}
\vspace{1cm}


\begin{tabular}{m{17cm}}
\textbf{20-variant}
\newline

T1. \(a\) parametrining qanday qiymatlarida \(P(x) = x^{2017} + ax - 5\) ko'phadi \((x + 1)\) ko'phadiga qoldiqsiz bo'linadi? \\
T2. Ushbu \(P(0) = 20\) va \(P(1) = 100\) shartlarini qanoatlantiruvchi \(P(x)\) ko'phadi mavjudmi? \\
A1. Tenglamani yeching. \(\frac{z}{z + 1} - 2\sqrt{\frac{z + 1}{2}} = 3\). \\
A2. Tenglamani yeching \((x + 1)^{5} + (x - 1)^{5} = 32x\). \\
A3. Tenglamani yeching. \(\frac{4x}{x^{2} + x + 3} + \frac{5x}{x^{2} - 5x + 3} = - \frac{3}{2}\). \\
B1. \(P(x + 1) + P(x - 3) = 2x^{2} - 10x + 16\) bo'lsa, \(P(x)\) ni toping. \\
B2. Quyidagi mulohazani ixtiyoriy natural son uchun matematik induksiya metodi yordamida isbotlang: \(1^{2} + 3^{2} + 5^{2} + ... + (2n - 1)^{2} = \frac{n\left( 4n^{2} - 1 \right)}{3}\); \\
B3. Quyidagi mulohazani ixtiyoriy natural son uchun matematik induksiya metodi yordamida isbotlang: \(5 \cdot 2^{3n - 2} + 3^{3n - 1}\) soni 19 ga karrali \\
C1. \(\frac{1}{C_{4}^{n}} = \frac{1}{C_{5}^{n}} + \frac{1}{C_{6}^{n}}\) bo'lsa, \(n\) ni toping \\
C2. Teng yonli uchburchakning yon tomoni 13 sm , yon tomoniga o'tkazilgan balandlik 5 sm ga teng. Uchburchak asosining uzunligi topilsin. \\
C3. Agar \(S\) uchburchakning yuzi, \(b\) va \(c\) uning tomonlari bo'lsa, \(S \leq \frac{b^{2} + c^{2}}{4}\) bo'lishini isbotlang. \\

\end{tabular}
\vspace{1cm}


\begin{tabular}{m{17cm}}
\textbf{21-variant}
\newline

T1. Ushbu \(P(0) = 20\) va \(P(1) = 100\) shartlarini qanoatlantiruvchi \(P(x)\) ko'phadi mavjudmi? \\
T2. Koshi tengsizligini isbotlang. \\
A1. Tenglamani yeching. \(\sqrt{x} + \frac{2x + 1}{x + 2} = 2\). \\
A2. Tenglamani yeching. \(\sqrt{x^{2} + x + 4} + \sqrt{x^{2} + x + 1} = \sqrt{2x^{2} + 2x + 9}\). \\
A3. Tenglamani yeching. \(\sqrt[3]{x} + \sqrt[3]{x - 16} = \sqrt[3]{x - 8}\). \\
B1. \(P(x) = x^{33} - 2ax^{21} + x^{8} + 8\) ko'phadi berilgan. \(a\) ning qaysi qiymati uchun \(P(x)\) ko'phadi \(x + 1\) ga qoldiqsiz bo'liadi? \\
B2. Quyidagi mulohazani ixtiyoriy natural son uchun matematik induksiya metodi yordamida isbotlang: \(\frac{1}{4 \cdot 5} + \frac{1}{5 \cdot 6} + \frac{1}{6 \cdot 7} + \ldots + \frac{1}{(n + 3) \cdot (n + 4)} = \frac{n}{4 \cdot (n + 4)}\). \\
B3. Quyidagi mulohazani ixtiyoriy natural son uchun matematik induksiya metodi yordamida isbotlang: \(2n^{3} + 3n^{2} + 7n\) soni 6 ga karrali ; \\
C1. \(x(1 - x)^{4} + x^{2}(1 + 2x)^{8} + x^{3}(1 + 3x)^{12}\) ifodada \(x^{4}\) oldidagi koeffitsiyentni toping. \\
C2. \emph{ABC} uchburchakning \(B\) uchidan \emph{AC} tomoniga \emph{BD} kesma o'tkazildi. \emph{BD} kesma bu uchburchakning yuzini teng ikkiga bulladi. Agar \(AC = a\) bo'lsa, \emph{AD} va \emph{DC} kesmalarning uzunliklarini toping. \\
C3. Muntazam uchburchakning tomoni a ga teng. Tomonini diametr deb hisoblab doira yasalgan. Uchburchakning shu doiradan tashqaridagi qismi yuzini toping. \\

\end{tabular}
\vspace{1cm}


\begin{tabular}{m{17cm}}
\textbf{22-variant}
\newline

T1. Haqiqiy \(a_{1},\ a_{2},\ .\ .\ .\ ,\ a_{n},\ b_{1},\ b_{2},\ .\ .\ .\ ,\ b_{n}\) sonlari uchun \(\left( a_{1}b_{1} + a_{2}b_{2} + \ .\ .\ .\  + a_{n}b_{n} \right)^{2} \leq \left( a_{1}^{2} + a_{2}^{2} + \ .\ .\ .\  + a_{n}^{2} \right)\left( b_{1}^{2} + b_{2}^{2} + \ .\ .\ .\  + b_{n}^{2} \right)\) \\
T2. Ushbu \(P(x) = x^{5} + 11x^{4} + 37x^{3} + 35x^{2} - 44x - 40\) ko'phadi \(Q(x) = x^{2} + 3x + 2\) ko'phadiga qoldiqsiz bo'linadimi? \\
A1. Tengsizlikni yeching: \(\sqrt{x + 3} + \sqrt{x - 2} - \sqrt{2x + 4} > 0\). \\
A2. Tenglamani yeching. \(\sqrt{x + 8 + 2\sqrt{x + 7}} + \sqrt{x + 1 - \sqrt{x + 7}} = 4\). \\
A3. Tenglamani yeching. \((x - 4)^{3} + (x - 4)^{2} + (x - 4)(x - 3) + (x - 3)^{2} + (x - 3)^{3} = 6\). \\
B1. \(P(x + 2) + P(x - 1) = - 2x^{2} - 2x + 7\) bo'lsa, \(P(x)\) ni \(x + 4\) ga bo'lgandagi qoldiqni toping. \\
B2. Quyidagi mulohazani ixtiyoriy natural son uchun matematik induksiya metodi yordamida isbotlang: \(1 \cdot 2 + 2 \cdot 3 + 3 \cdot 4 + \ldots + n \cdot (n + 1) = \frac{n \cdot (n + 1) \cdot (n + 2)}{3}\). \\
B3. Quyidagi mulohazani ixtiyoriy natural son uchun matematik induksiya metodi yordamida isbotlang: \(5^{n + 2} + 26 \cdot 5^{n} + 8^{2n + 1}\) soni 59 ga karrali; \\
C1. \((a + b)^{n}\) ifoda yoyilmasining barcha koeffitsiyentlari yig`indisi 4096 ga teng bo'lsa, uning eng katta koeffitsiyentin toping. \\
C2. Uchburchakning ishida olingan nuqtadan uning tomonlariga parallel to'g'ri chiziqlar o'tkazilgan. Ular uchburchakni 6 qismga bo'ladi. Agar hosil bo'lgan uchburchaklarning yuzlari \(S_{1},S_{2}\) va \(S_{3}\) bo'lsa, berilgan uchburchak yuzini toping. \\
C3. Ya. Bermlli tengsizligi. Agar \(x \geq - 1\) bo'lsa, u holda ixtiyoriy natural \(n\) soni uchun \((1 + x)^{n} \geq 1 + nx\) tengsizlik o'rinli bo'lishini isbotlang. \\

\end{tabular}
\vspace{1cm}


\begin{tabular}{m{17cm}}
\textbf{23-variant}
\newline

T1. \(b\) parametrining qanday qiymatlarida \(x^{3} + 17x^{2} + bx - 17 = 0\) tenglamasining ildizlari butun sonlardan iborat bo'ladi? \\
T2. Matematik induksiya metodi va uning qo'llanilishiga misollar. \\
A1. Tengsizlikni yeching: \(\frac{x^{3} + 3x^{2} - x - 3}{x^{2} + 3x - 10} < 0\). \\
A2. Tengsizlikni yeching: \(\sqrt{x^{2} - 4x} > x - 3\). \\
A3. Tenglamani yeching. \(\sqrt[3]{x - 1} + \sqrt[3]{x - 2} - \sqrt{2x - 3} = 0\). \\
B1. \(P(x)\) ko'phadni \(3x^{2} - 4x + 1\) ga bo'lganimizda qoldiq \(6x - 11\) bo'lsa, \(P(x)\) ko'phadni \(3x - 1\)ga bo'lganda qoldiqni toping. \\
B2. Quyidagi mulohazani ixtiyoriy natural son uchun matematik induksiya metodi yordamida isbotlang: \(1^{2} + 3^{2} + 5^{2} + ... + (2n - 1)^{2} = \frac{n\left( 4n^{2} - 1 \right)}{3}\); \\
B3. Quyidagi mulohazani ixtiyoriy natural son uchun matematik induksiya metodi yordamida isbotlang: \(5^{n} - 4n + 15\) soni 16 ga karrali ; \\
C1. \(\left( 2x^{\ ^{2}} - \frac{b}{2x^{3}} \right)^{10}\) binom yoyilmasining \(x\) qatnashmagan hadin toping. \\
C2. \(\bigtriangleup ABC\) da \(AB = 2sm,BD\) mediana, \(BD = 1sm\), \(\angle BDA = 30^{{^\circ}}\). Uchburchakning yuzi hisoblansin. \\
C3. Muntazam uchburchakning tomoni a ga teng. Tomonini diametr deb hisoblab doira yasalgan. Uchburchakning shu doiradan tashqaridagi qismi yuzini toping. \\

\end{tabular}
\vspace{1cm}


\begin{tabular}{m{17cm}}
\textbf{24-variant}
\newline

T1. Ixtiyoriy \(a,b,c \in (0;1)\) sonlari uchun \(a(1 - b) > 1/4,\ b(1 - c) > 1/4,\ c(1 - a) > 1/4\) tengsizliklari bir vaqtda o'rinli bo'la olmasligini isbotlang. \\
T2. \(n\) darajaning qanday qiymatlarida \((x + 1)^{n} + (x - 1)^{n}\) ifodasi \(x\) ifodaga qoldiqsiz bo'linadi? \\
A1. Tenglamani yeching \(\sqrt{\frac{18 - 7x - x^{2}}{8 - 6x + x^{2}}} + \sqrt{\frac{8 - 6x + x^{2}}{18 - 7x - x^{2}}} = \frac{13}{6}\). \\
A2. Tenglamani yeching \(\left( x^{2} - 4x + 6 \right)^{2} - 4\left( x^{2} - 4x + 6 \right) + 6 = x\). \\
A3. Tenglamani yeching. \(\frac{z}{z + 1} - 2\sqrt{\frac{z + 1}{2}} = 3\). \\
B1. \(P(x) = (x - 5)^{2n + 1} + (x - 1)^{2n + 3}\) ko'phadni \(x - 3\) ga bo'lganda qoldiq \(3 \cdot 2^{3n - 4}\) bo'lsa, \(n\) ni toping. \\
B2. Quyidagi mulohazani ixtiyoriy natural son uchun matematik induksiya metodi yordamida isbotlang: \(\frac{1}{4 \cdot 5} + \frac{1}{5 \cdot 6} + \frac{1}{6 \cdot 7} + \ldots + \frac{1}{(n + 3) \cdot (n + 4)} = \frac{n}{4 \cdot (n + 4)}\). \\
B3. Quyidagi mulohazani ixtiyoriy natural son uchun matematik induksiya metodi yordamida isbotlang: \(5^{2n + 1} + 3^{n + 2} \cdot 2^{n - 1}\) soni 19 ga karrali ; \\
C1. Tengsizlikni yeching: \(C_{10}^{x - 1} > 2C_{10}^{x}\) \\
C2. Uchburchakning asosiga tushirilgan balandligi \(h\) ga teng. Uchburchakning asosiga parallel kesma uchburchakning yuzini teng ikkiga bo'ladi. Uchburchakning uchidan shu kesmagacha bo'lgan masofa topilsin. \\
C3. Agar \(S\) uchburchakning yuzi, \(b\) va \(c\) uning tomonlari bo'lsa, \(S \leq \frac{b^{2} + c^{2}}{4}\) bo'lishini isbotlang. \\

\end{tabular}
\vspace{1cm}


\begin{tabular}{m{17cm}}
\textbf{25-variant}
\newline

T1. Yig'indisi birga teng bo'lgan \(x,y,z\) musbat sonlari uchun \(\frac{1}{x} + \frac{1}{y} + \frac{1}{z} \geq 9\) tengsizligi o'rinli bo'lishini isbotlang. \\
T2. Bezu teoremasi va uning qo'llanilishi. \\
A1. Tengsizlikni yeching:\(x^{2}\left( x^{4} + 36 \right) - 6\sqrt{3}\left( x^{4} + 4 \right) < 0\). \\
A2. Tenglamani yeching. \(\frac{4x}{x^{2} + x + 3} + \frac{5x}{x^{2} - 5x + 3} = - \frac{3}{2}\). \\
A3. Tenglamani yeching \(\left( x^{2} + 10x + 10 \right)\left( x^{2} + x + 10 \right) = 10x^{2}\) . \\
B1. \(P(2x - 1) + P(x - 1) = 10x^{2} - 12x + 2\) bo'lsa, \(P(x)\) ni toping. \\
B2. Quyidagi mulohazani ixtiyoriy natural son uchun matematik induksiya metodi yordamida isbotlang: \(\frac{1}{1 \cdot 5} + \frac{1}{5 \cdot 9} + ... + \frac{1}{(4n - 3)(4n + 1)} = \frac{n}{4n + 1}\); \\
B3. Quyidagi mulohazani ixtiyoriy natural son uchun matematik induksiya metodi yordamida isbotlang:\(6^{2n - 2} + 3^{n + 1} + 3^{n - 1}\) soni 11 karrali ; \\
C1. \(\left( \sqrt{x} + \frac{1}{\sqrt[3]{x^{2}}} \right)^{n}\) binom yoyilmasida 5-had koeffitsiyentining 3-had koeffitsiyentiga nisbati 7:2 ga teng. \(x\) ning darajasi 1 ga teng bo'lgan ahadin toping. \\
C2. \emph{ABC} uchburchak berilgan. Uning medianalaridan \(\bigtriangleup A_{1}B_{1}C_{1}\) yasalgan. \(\bigtriangleup ABC\) va \(\bigtriangleup A_{1}B_{1}C_{1}\) yuzlarining nisbati topilsin. \\
C3. Muntazam uchburchakning tomoni a ga teng. Tomonini diametr deb hisoblab doira yasalgan. Uchburchakning shu doiradan tashqaridagi qismi yuzini toping. \\

\end{tabular}
\vspace{1cm}


\begin{tabular}{m{17cm}}
\textbf{26-variant}
\newline

T1. \(2^{81} + 1\) soni 9 soniga qoldiqsiz bo'linishini isbotlang. \\
T2. \(P(x) = (x - 1)^{20}\left( x^{2} + 25 \right)\) ko'phadining koeffitsentlari yig'indisini toping. \\
A1. Tenglamani yeching \((x + 4)(x + 1) - 3\sqrt{x^{2} + 5x + 2} = 6\). \\
A2. Tenglamani yeching. \((\sqrt{x + 1} + \sqrt{x})^{3} + (\sqrt{x + 1} + \sqrt{x})^{2} = 2\). \\
A3. Tenglamani yeching \(\left( x^{2} - 6x \right)^{2} - 2(x - 3)^{2} = 81\). \\
B1. \(P(x + n) = (x + n)^{3} + (x - n)^{2} + x + n + 6\) ko'phadi berilgan. \(P(x)\) ko'phadi \(x - n\) ga qoldiqsiz bo'linsa, \(n\) ni toping. \\
B2. Quyidagi mulohazani ixtiyoriy natural son uchun matematik induksiya metodi yordamida isbotlang: \(\frac{1}{1 \cdot 4} + \frac{1}{4 \cdot 7} + \frac{1}{7 \cdot 10} + \ldots + \frac{1}{(3n - 2) \cdot (3n + 1)} = \frac{n}{(3n + 1)}\). \\
B3. Quyidagi mulohazani ixtiyoriy natural son uchun matematik induksiya metodi yordamida isbotlang: \(n\left( 2n^{2} - 3n + 1 \right)\) soni 6 ga karrali ; \\
C1. Teńlemeni sheshiń \(\frac{C_{2x}^{x + 1}}{C_{2x + 1}^{x - 1}} = \frac{2}{3}\), \(x \in N\) \\
C2. \(\bigtriangleup ABC\) da \(AB = 3sm,AC = 5sm,\angle BAC = 120^{{^\circ}}.BD\) bissektrisaning uzunligi topilsin. \\
C3. Agar \(S\) uchburchakning yuzi, \(b\) va \(c\) uning tomonlari bo'lsa, \(S \leq \frac{b^{2} + c^{2}}{4}\) bo'lishini isbotlang. \\

\end{tabular}
\vspace{1cm}


\begin{tabular}{m{17cm}}
\textbf{27-variant}
\newline

T1. Simmetrik ko'phadlar. \\
T2. Pifagor teoremasi va uning isbotlari. \\
A1. Tenglamani yeching \((x + 1)^{5} + (x - 1)^{5} = 32x\). \\
A2. Tenglamani yeching. \(\sqrt{3x^{2} - 2x + 15} + \sqrt{3x^{2} - 2x + 8} = 7\). \\
A3. Tengsizlikni yeching: \(\sqrt{x + 3} + \sqrt{x - 2} - \sqrt{2x + 4} > 0\). \\
B1. \(P(x + 3) = x^{2} - x + n\) bo'lsa. \(P(x - 2)\) ko'phadni \(x - 3\) ga bo'lganda qoldiq \(10\) bo'lsa, \(n\) ni toping. \\
B2. Quyidagi mulohazani ixtiyoriy natural son uchun matematik induksiya metodi yordamida isbotlang: \(1 \cdot 1! + 2 \cdot 2! + 3 \cdot 3! + \ldots + n \cdot n! = (n + 1)! - 1\). \\
B3. Quyidagi mulohazani ixtiyoriy natural son uchun matematik induksiya metodi yordamida isbotlang: \(n^{3} + (n + 1)^{3} + (n + 2)^{3}\) soni 9 ga karrali ; \\
C1. Teńsizlikti sheshiń \(C_{13}^{x} < C_{13}^{x + 2}\), \(x \in N\) \\
C2. To'g'ri burchakli uchburchakning to'g'ri burchagi bissektrisasi shu uchdan o'tkazilgan mediana va balandlik orasidagi burchakni ham teng ikkiga bo'lishini isbotlang. \\
C3. Ya. Bermlli tengsizligi. Agar \(x \geq - 1\) bo'lsa, u holda ixtiyoriy natural \(n\) soni uchun \((1 + x)^{n} \geq 1 + nx\) tengsizlik o'rinli bo'lishini isbotlang. \\

\end{tabular}
\vspace{1cm}


\begin{tabular}{m{17cm}}
\textbf{28-variant}
\newline

T1. Ixtiyoriy \(a\) parametri va \(x\) uchun \(x(a - x) \leq a^{2}/4\) tengsizligi o'rinli bo'lishini isbotlang. \\
T2. \(x\) o'zgaruvchining ixtiyoriy butun qiymatida \(ax^{2} + bx + c\) uchhadining qiymati butun bo'lishi uchun \(2a,\ a + b\) va \(c\) sonlarining butun bo'lishi zarur va yetarli ekanligini isbotlang. \\
A1. Tenglamani yeching. \(\sqrt[3]{x} + \sqrt[3]{x - 16} = \sqrt[3]{x - 8}\). \\
A2. Tenglamani yeching. \(\frac{z}{z + 1} - 2\sqrt{\frac{z + 1}{2}} = 3\). \\
A3. Tenglamani yeching. \(\frac{4x}{x^{2} + x + 3} + \frac{5x}{x^{2} - 5x + 3} = - \frac{3}{2}\). \\
B1. \(P(x) = x^{4} - 2x + 2^{n + 1}\) ko'phadni \(x - 2^{n}\) ga bo'lganda qoldiq \(2^{n - 2}\) bo'lsa, \(n\) ni toping. \\
B2. Quyidagi mulohazani ixtiyoriy natural son uchun matematik induksiya metodi yordamida isbotlang: \(1^{2} + 2^{2} + 3^{2} + ... + n^{2} = \frac{n(n + 1)(2n + 1)}{6}\); \\
B3. Quyidagi mulohazani ixtiyoriy natural son uchun matematik induksiya metodi yordamida isbotlang: \(5^{2n + 1} + 3^{n + 2} \cdot 2^{n - 1}\) soni 19 ga karrali ; \\
C1. Ayniyatni isbotlang: \(C_{n + k}^{j + k} = \sum_{s = 0}^{k}C_{n}^{j + s}C_{k}^{s}\); \\
C2. Teng yonli \(ABC(AB = BC)\) uchburchakda \emph{AD} bissektrisa o'tkazilgan. Agar \(S_{ABD} = S_{1},S_{\bigtriangleup ADC} = S_{2}\) bo'lsa, \emph{AC} ni toping. \\
C3. Agar \(S\) uchburchakning yuzi, \(b\) va \(c\) uning tomonlari bo'lsa, \(S \leq \frac{b^{2} + c^{2}}{4}\) bo'lishini isbotlang. \\

\end{tabular}
\vspace{1cm}


\begin{tabular}{m{17cm}}
\textbf{29-variant}
\newline

T1. \(P(x) = x^{6} - 3x^{5} + x^{4} - 6x^{2} + 2x - 6\) ko'phadining butun ildizlarini toping. \\
T2. Fales teoremasi va uning qo'llanilishi. \\
A1. Tenglamani yeching. \((x - 4)^{3} + (x - 4)^{2} + (x - 4)(x - 3) + (x - 3)^{2} + (x - 3)^{3} = 6\). \\
A2. Tenglamani yeching \(\left( x^{2} - 6x \right)^{2} - 2(x - 3)^{2} = 81\). \\
A3. Tengsizlikni yeching:\(x^{2}\left( x^{4} + 36 \right) - 6\sqrt{3}\left( x^{4} + 4 \right) < 0\). \\
B1. \(P(x + 3)\) ko'phadni \(x + 1\) ga bo'lganda qoldiq -3, \(Q(2x - 1)\) ko'phadni \(x - 1\)ga bo'lganda qoldiq 2 bo'lsa, \(P(x + 4) + x^{2}Q(x + 3)\) ko'phadni \(x + 2\) ga bo'lgandagi qoldiqni toping. \\
B2. Quyidagi mulohazani ixtiyoriy natural son uchun matematik induksiya metodi yordamida isbotlang: \(\left( 1 - \frac{1}{4} \right)\left( 1 - \frac{1}{9} \right)...\left( 1 - \frac{1}{n^{2}} \right) = \frac{n + 1}{2n}\), \(n \geq 2\) \\
B3. Quyidagi mulohazani ixtiyoriy natural son uchun matematik induksiya metodi yordamida isbotlang: \(5 \cdot 2^{3n - 2} + 3^{3n - 1}\) soni 19 ga karrali \\
C1. \(\left( x^{3} - \frac{3}{x^{2}} \right)^{10}\) binom yoyilmasining \(x\) qatnashmagan hadin toping. \\
C2. \(\bigtriangleup ABC\) da \(\angle A\) burchak \(\angle B\) dan ikki marta katta bo'lib, \(AC = b,AB = c\). \emph{BC} tomonning uzunligi topilsin. \\
C3. Ya. Bermlli tengsizligi. Agar \(x \geq - 1\) bo'lsa, u holda ixtiyoriy natural \(n\) soni uchun \((1 + x)^{n} \geq 1 + nx\) tengsizlik o'rinli bo'lishini isbotlang. \\

\end{tabular}
\vspace{1cm}


\begin{tabular}{m{17cm}}
\textbf{30-variant}
\newline

T1. \(a\) parametrining qanday qiymatlarida \(P(x) = x^{2017} + ax - 5\) ko'phadi \((x + 1)\) ko'phadiga qoldiqsiz bo'linadi? \\
T2. Kombinatorika elementlari va Nyuton binomi. \\
A1. Tenglamani yeching. \(\sqrt{3x^{2} - 2x + 15} + \sqrt{3x^{2} - 2x + 8} = 7\). \\
A2. Tenglamani yeching \((x + 1)^{5} + (x - 1)^{5} = 32x\). \\
A3. Tenglamani yeching. \(\sqrt[3]{x - 1} + \sqrt[3]{x - 2} - \sqrt{2x - 3} = 0\). \\
B1. \(P(x + 1) + P(x - 3) = 2x^{2} - 10x + 16\) bo'lsa, \(P(x)\) ni toping. \\
B2. Quyidagi mulohazani ixtiyoriy natural son uchun matematik induksiya metodi yordamida isbotlang: \(2^{2} + 6^{2} + \ldots + (4n - 2)^{2} = \frac{4n(2n - 1)(2n + 1)}{3}\). \\
B3. Quyidagi mulohazani ixtiyoriy natural son uchun matematik induksiya metodi yordamida isbotlang: \(2n^{3} + 3n^{2} + 7n\) soni 6 ga karrali ; \\
C1. Teńsizlikti sheshiń \(5C_{x}^{3} < C_{x + 2}^{4}\), \(x \in N\) \\
C2. \emph{ABC} uchburchakning \emph{AC}, \emph{BC} va \emph{AB} tomonlarida \emph{CMPA}, \emph{BEFC} va \emph{ADKB} kvadratlar yasalgan. Agar \(AB = 13\), \(AC = 14,BC = 15\) ekanligi ma'lum bo'lsa, \emph{DKEFMP} oltiburchakning yuzini toping. \\
C3. Muntazam uchburchakning tomoni a ga teng. Tomonini diametr deb hisoblab doira yasalgan. Uchburchakning shu doiradan tashqaridagi qismi yuzini toping. \\

\end{tabular}
\vspace{1cm}


\begin{tabular}{m{17cm}}
\textbf{31-variant}
\newline

T1. Bezu teoremasi va uning qo'llanilishi. \\
T2. Koshi tengsizligini isbotlang. \\
A1. Tenglamani yeching \(\left( x^{2} - 4x + 6 \right)^{2} - 4\left( x^{2} - 4x + 6 \right) + 6 = x\). \\
A2. Tengsizlikni yeching: \(\sqrt{x^{2} - 4x} > x - 3\). \\
A3. Tenglamani yeching. \(\sqrt{x^{2} + x + 4} + \sqrt{x^{2} + x + 1} = \sqrt{2x^{2} + 2x + 9}\). \\
B1. \(P(x) = x^{4} - 2x + 2^{n + 1}\) ko'phadni \(x - 2^{n}\) ga bo'lganda qoldiq \(2^{n - 2}\) bo'lsa, \(n\) ni toping. \\
B2. Quyidagi mulohazani ixtiyoriy natural son uchun matematik induksiya metodi yordamida isbotlang: \(1 \cdot 2 + 2 \cdot 3 + 3 \cdot 4 + ... + n(n + 1) = \frac{n(n + 1)(n + 2)}{3}\); \\
B3. Quyidagi mulohazani ixtiyoriy natural son uchun matematik induksiya metodi yordamida isbotlang: \(5^{n} - 4n + 15\) soni 16 ga karrali ; \\
C1. \(\frac{1}{C_{4}^{n}} = \frac{1}{C_{5}^{n}} + \frac{1}{C_{6}^{n}}\) bo'lsa, \(n\) ni toping \\
C2. Teng yonli uchburchak asosidagi burchak \(\alpha\) ga teng. Shu burchak uchidan asosga \(\beta(\beta < \alpha)\) burchak ostida to'g'ri chiziq o'tkazilgan, u uchburchakni ikki qismga ajratadi. Hosil bo'lgan uchburchaklar yuzlarining nisbatini toping. \\
C3. Ya. Bermlli tengsizligi. Agar \(x \geq - 1\) bo'lsa, u holda ixtiyoriy natural \(n\) soni uchun \((1 + x)^{n} \geq 1 + nx\) tengsizlik o'rinli bo'lishini isbotlang. \\

\end{tabular}
\vspace{1cm}


\begin{tabular}{m{17cm}}
\textbf{32-variant}
\newline

T1. Simmetrik ko'phadlar. \\
T2. \(x\) o'zgaruvchining ixtiyoriy butun qiymatida \(ax^{2} + bx + c\) uchhadining qiymati butun bo'lishi uchun \(2a,\ a + b\) va \(c\) sonlarining butun bo'lishi zarur va yetarli ekanligini isbotlang. \\
A1. Tenglamani yeching. \(\sqrt{x} + \frac{2x + 1}{x + 2} = 2\). \\
A2. Tenglamani yeching. \((\sqrt{x + 1} + \sqrt{x})^{3} + (\sqrt{x + 1} + \sqrt{x})^{2} = 2\). \\
A3. Tengsizlikni yeching: \(\frac{x^{3} + 3x^{2} - x - 3}{x^{2} + 3x - 10} < 0\). \\
B1. \(P(x)\) ko'phadni \(3x^{2} - 4x + 1\) ga bo'lganimizda qoldiq \(6x - 11\) bo'lsa, \(P(x)\) ko'phadni \(3x - 1\)ga bo'lganda qoldiqni toping. \\
B2. Quyidagi mulohazani ixtiyoriy natural son uchun matematik induksiya metodi yordamida isbotlang: \(1^{3} + 2^{3} + 3^{3} + ... + n^{3} = \left( \frac{n(n + 1)}{2} \right)^{2}\); \\
B3. Quyidagi mulohazani ixtiyoriy natural son uchun matematik induksiya metodi yordamida isbotlang: \(n\left( 2n^{2} - 3n + 1 \right)\) soni 6 ga karrali ; \\
C1. \(\left( x\sqrt{x} - \frac{1}{x^{4}} \right)^{n}\) binom yoyilmasida 3-had koeffitsiyenti 2-had koeffitsiyentidan 44 ga katta.Ozod hadini toping. \\
C2. Uchburchakning tomonlari \(a\) va \(b\), bissektrisasi \(l_{c} = l\). \(l\) ni bilgan holda uning yuzini toping. \\
C3. Muntazam uchburchakning tomoni a ga teng. Tomonini diametr deb hisoblab doira yasalgan. Uchburchakning shu doiradan tashqaridagi qismi yuzini toping. \\

\end{tabular}
\vspace{1cm}


\begin{tabular}{m{17cm}}
\textbf{33-variant}
\newline

T1. \(a\) parametrining qanday qiymatlarida \(P(x) = x^{2017} + ax - 5\) ko'phadi \((x + 1)\) ko'phadiga qoldiqsiz bo'linadi? \\
T2. \(n\) darajaning qanday qiymatlarida \((x + 1)^{n} + (x - 1)^{n}\) ifodasi \(x\) ifodaga qoldiqsiz bo'linadi? \\
A1. Tenglamani yeching \((x + 4)(x + 1) - 3\sqrt{x^{2} + 5x + 2} = 6\). \\
A2. Tenglamani yeching \(\sqrt{\frac{18 - 7x - x^{2}}{8 - 6x + x^{2}}} + \sqrt{\frac{8 - 6x + x^{2}}{18 - 7x - x^{2}}} = \frac{13}{6}\). \\
A3. Tenglamani yeching. \(\sqrt{x + 8 + 2\sqrt{x + 7}} + \sqrt{x + 1 - \sqrt{x + 7}} = 4\). \\
B1. \(P(2x - 1) + P(x - 1) = 10x^{2} - 12x + 2\) bo'lsa, \(P(x)\) ni toping. \\
B2. Quyidagi mulohazani ixtiyoriy natural son uchun matematik induksiya metodi yordamida isbotlang: \(1 \cdot 2 + 2 \cdot 3 + 3 \cdot 4 + \ldots + n \cdot (n + 1) = \frac{n \cdot (n + 1) \cdot (n + 2)}{3}\). \\
B3. Quyidagi mulohazani ixtiyoriy natural son uchun matematik induksiya metodi yordamida isbotlang: \(n^{3} + (n + 1)^{3} + (n + 2)^{3}\) soni 9 ga karrali ; \\
C1. \((x + 1)^{3} + (x + 1)^{4} + (x + 1)^{5} + ... + (x + 1)^{10}\) ifodada \(x^{3}\) oldidagi koeffitsiyentni toping \\
C2. To'g'ri burchakli uchburchakda katetlarning nisbati 3:2 kabi, balandlik esa gipotenuzani shunday ikkita kesmaga ajratadiki, ulardan birining uzunligi ikkinchisidan 2 ga katta. Gipotenuzaning uzunligi topilsin. \\
C3. Agar \(S\) uchburchakning yuzi, \(b\) va \(c\) uning tomonlari bo'lsa, \(S \leq \frac{b^{2} + c^{2}}{4}\) bo'lishini isbotlang. \\

\end{tabular}
\vspace{1cm}


\begin{tabular}{m{17cm}}
\textbf{34-variant}
\newline

T1. Ixtiyoriy \(a,b,c \in (0;1)\) sonlari uchun \(a(1 - b) > 1/4,\ b(1 - c) > 1/4,\ c(1 - a) > 1/4\) tengsizliklari bir vaqtda o'rinli bo'la olmasligini isbotlang. \\
T2. Pifagor teoremasi va uning isbotlari. \\
A1. Tenglamani yeching \(\left( x^{2} + 10x + 10 \right)\left( x^{2} + x + 10 \right) = 10x^{2}\) . \\
A2. Tenglamani yeching \(\left( x^{2} - 4x + 6 \right)^{2} - 4\left( x^{2} - 4x + 6 \right) + 6 = x\). \\
A3. Tengsizlikni yeching: \(\sqrt{x^{2} - 4x} > x - 3\). \\
B1. \(P(x) = (x - 5)^{2n + 1} + (x - 1)^{2n + 3}\) ko'phadni \(x - 3\) ga bo'lganda qoldiq \(3 \cdot 2^{3n - 4}\) bo'lsa, \(n\) ni toping. \\
B2. Quyidagi mulohazani ixtiyoriy natural son uchun matematik induksiya metodi yordamida isbotlang: \(1^{3} + 2^{3} + 3^{3} + ... + n^{3} = \left( \frac{n(n + 1)}{2} \right)^{2}\); \\
B3. Quyidagi mulohazani ixtiyoriy natural son uchun matematik induksiya metodi yordamida isbotlang: \(5^{n + 2} + 26 \cdot 5^{n} + 8^{2n + 1}\) soni 59 ga karrali; \\
C1. \(C_{n + 4}^{n + 1} - C_{n + 3}^{n} = 15(n + 2)\) bo'lsa, \(n\) ni toping. \\
C2. Uchburchakning ishida olingan nuqtadan uning tomonlariga parallel to'g'ri chiziqlar o'tkazilgan. Ular uchburchakni 6 qismga bo'ladi. Agar hosil bo'lgan uchburchaklarning yuzlari \(S_{1},S_{2}\) va \(S_{3}\) bo'lsa, berilgan uchburchak yuzini toping. \\
C3. Agar \(S\) uchburchakning yuzi, \(b\) va \(c\) uning tomonlari bo'lsa, \(S \leq \frac{b^{2} + c^{2}}{4}\) bo'lishini isbotlang. \\

\end{tabular}
\vspace{1cm}


\begin{tabular}{m{17cm}}
\textbf{35-variant}
\newline

T1. \(b\) parametrining qanday qiymatlarida \(x^{3} + 17x^{2} + bx - 17 = 0\) tenglamasining ildizlari butun sonlardan iborat bo'ladi? \\
T2. Ushbu \(P(x) = x^{5} + 11x^{4} + 37x^{3} + 35x^{2} - 44x - 40\) ko'phadi \(Q(x) = x^{2} + 3x + 2\) ko'phadiga qoldiqsiz bo'linadimi? \\
A1. Tenglamani yeching \(\left( x^{2} - 6x \right)^{2} - 2(x - 3)^{2} = 81\). \\
A2. Tenglamani yeching. \(\sqrt{x} + \frac{2x + 1}{x + 2} = 2\). \\
A3. Tenglamani yeching \((x + 4)(x + 1) - 3\sqrt{x^{2} + 5x + 2} = 6\). \\
B1. \(P(x + n) = (x + n)^{3} + (x - n)^{2} + x + n + 6\) ko'phadi berilgan. \(P(x)\) ko'phadi \(x - n\) ga qoldiqsiz bo'linsa, \(n\) ni toping. \\
B2. Quyidagi mulohazani ixtiyoriy natural son uchun matematik induksiya metodi yordamida isbotlang: \(1 \cdot 2 + 2 \cdot 3 + 3 \cdot 4 + \ldots + n \cdot (n + 1) = \frac{n \cdot (n + 1) \cdot (n + 2)}{3}\). \\
B3. Quyidagi mulohazani ixtiyoriy natural son uchun matematik induksiya metodi yordamida isbotlang: \(7^{n} - 1\) soni 6 ga karrali; \\
C1. Ayniyatni isbotlang:\(\sum_{j = 0}^{n}C_{n}^{j}( - 1)^{j} = 0\); \\
C2. \(\bigtriangleup ABC\) da \(\angle A\) burchak \(\angle B\) dan ikki marta katta bo'lib, \(AC = b,AB = c\). \emph{BC} tomonning uzunligi topilsin. \\
C3. Muntazam uchburchakning tomoni a ga teng. Tomonini diametr deb hisoblab doira yasalgan. Uchburchakning shu doiradan tashqaridagi qismi yuzini toping. \\

\end{tabular}
\vspace{1cm}


\begin{tabular}{m{17cm}}
\textbf{36-variant}
\newline

T1. Yig'indisi birga teng bo'lgan \(x,y,z\) musbat sonlari uchun \(\frac{1}{x} + \frac{1}{y} + \frac{1}{z} \geq 9\) tengsizligi o'rinli bo'lishini isbotlang. \\
T2. Kombinatorika elementlari va Nyuton binomi. \\
A1. Tenglamani yeching. \(\sqrt[3]{x} + \sqrt[3]{x - 16} = \sqrt[3]{x - 8}\). \\
A2. Tenglamani yeching. \(\sqrt{x + 8 + 2\sqrt{x + 7}} + \sqrt{x + 1 - \sqrt{x + 7}} = 4\). \\
A3. Tengsizlikni yeching: \(\frac{x^{3} + 3x^{2} - x - 3}{x^{2} + 3x - 10} < 0\). \\
B1. \(P(x + 2) + P(x - 1) = - 2x^{2} - 2x + 7\) bo'lsa, \(P(x)\) ni \(x + 4\) ga bo'lgandagi qoldiqni toping. \\
B2. Quyidagi mulohazani ixtiyoriy natural son uchun matematik induksiya metodi yordamida isbotlang: \(1^{2} + 3^{2} + 5^{2} + ... + (2n - 1)^{2} = \frac{n\left( 4n^{2} - 1 \right)}{3}\); \\
B3. Quyidagi mulohazani ixtiyoriy natural son uchun matematik induksiya metodi yordamida isbotlang:\(6^{2n - 2} + 3^{n + 1} + 3^{n - 1}\) soni 11 karrali ; \\
C1. \(5C_{n}^{3} = C_{n + 2}^{4}\) bo'lsa, \(n\) ni toping. \\
C2. Teng yonli uchburchakning yuzi \(S\) ga teng. Yon tomonlariga tushirilgan medianalari orasidagi burchak \(\alpha\) ga teng. Uchburchak asosini toping. \\
C3. Ya. Bermlli tengsizligi. Agar \(x \geq - 1\) bo'lsa, u holda ixtiyoriy natural \(n\) soni uchun \((1 + x)^{n} \geq 1 + nx\) tengsizlik o'rinli bo'lishini isbotlang. \\

\end{tabular}
\vspace{1cm}


\begin{tabular}{m{17cm}}
\textbf{37-variant}
\newline

T1. Ushbu \(P(0) = 20\) va \(P(1) = 100\) shartlarini qanoatlantiruvchi \(P(x)\) ko'phadi mavjudmi? \\
T2. Ixtiyoriy \(a\) parametri va \(x\) uchun \(x(a - x) \leq a^{2}/4\) tengsizligi o'rinli bo'lishini isbotlang. \\
A1. Tengsizlikni yeching: \(\sqrt{x + 3} + \sqrt{x - 2} - \sqrt{2x + 4} > 0\). \\
A2. Tenglamani yeching. \(\frac{z}{z + 1} - 2\sqrt{\frac{z + 1}{2}} = 3\). \\
A3. Tengsizlikni yeching:\(x^{2}\left( x^{4} + 36 \right) - 6\sqrt{3}\left( x^{4} + 4 \right) < 0\). \\
B1. \(P(x + 3) = x^{2} - x + n\) bo'lsa. \(P(x - 2)\) ko'phadni \(x - 3\) ga bo'lganda qoldiq \(10\) bo'lsa, \(n\) ni toping. \\
B2. Quyidagi mulohazani ixtiyoriy natural son uchun matematik induksiya metodi yordamida isbotlang: \(1 \cdot 1! + 2 \cdot 2! + 3 \cdot 3! + \ldots + n \cdot n! = (n + 1)! - 1\). \\
B3. Quyidagi mulohazani ixtiyoriy natural son uchun matematik induksiya metodi yordamida isbotlang: \(5^{n + 2} + 26 \cdot 5^{n} + 8^{2n + 1}\) soni 59 ga karrali; \\
C1. Ayniyatni isbotlang: \(C_{n + 1}^{j + 1} = C_{n}^{j} + C_{n}^{j + 1}\); \\
C2. \emph{ABC} uchburchakning \(B\) uchidan \emph{AC} tomoniga \emph{BD} kesma o'tkazildi. \emph{BD} kesma bu uchburchakning yuzini teng ikkiga bulladi. Agar \(AC = a\) bo'lsa, \emph{AD} va \emph{DC} kesmalarning uzunliklarini toping. \\
C3. Ya. Bermlli tengsizligi. Agar \(x \geq - 1\) bo'lsa, u holda ixtiyoriy natural \(n\) soni uchun \((1 + x)^{n} \geq 1 + nx\) tengsizlik o'rinli bo'lishini isbotlang. \\

\end{tabular}
\vspace{1cm}


\begin{tabular}{m{17cm}}
\textbf{38-variant}
\newline

T1. Fales teoremasi va uning qo'llanilishi. \\
T2. \(2^{81} + 1\) soni 9 soniga qoldiqsiz bo'linishini isbotlang. \\
A1. Tenglamani yeching \(\sqrt{\frac{18 - 7x - x^{2}}{8 - 6x + x^{2}}} + \sqrt{\frac{8 - 6x + x^{2}}{18 - 7x - x^{2}}} = \frac{13}{6}\). \\
A2. Tenglamani yeching. \(\sqrt{x^{2} + x + 4} + \sqrt{x^{2} + x + 1} = \sqrt{2x^{2} + 2x + 9}\). \\
A3. Tenglamani yeching. \((\sqrt{x + 1} + \sqrt{x})^{3} + (\sqrt{x + 1} + \sqrt{x})^{2} = 2\). \\
B1. \(P(x + 3)\) ko'phadni \(x + 1\) ga bo'lganda qoldiq -3, \(Q(2x - 1)\) ko'phadni \(x - 1\)ga bo'lganda qoldiq 2 bo'lsa, \(P(x + 4) + x^{2}Q(x + 3)\) ko'phadni \(x + 2\) ga bo'lgandagi qoldiqni toping. \\
B2. Quyidagi mulohazani ixtiyoriy natural son uchun matematik induksiya metodi yordamida isbotlang: \(\left( 1 - \frac{1}{4} \right)\left( 1 - \frac{1}{9} \right)...\left( 1 - \frac{1}{n^{2}} \right) = \frac{n + 1}{2n}\), \(n \geq 2\) \\
B3. Quyidagi mulohazani ixtiyoriy natural son uchun matematik induksiya metodi yordamida isbotlang: \(7^{n} - 1\) soni 6 ga karrali; \\
C1. Ayniyatni isbotlang:\(C_{n + 2}^{j + 2} = C_{n}^{j} + 2C_{n}^{j + 1} + C_{n}^{j + 2}\); \\
C2. Teng yonli \(ABC(AB = BC)\) uchburchakda \emph{AD} bissektrisa o'tkazilgan. Agar \(S_{ABD} = S_{1},S_{\bigtriangleup ADC} = S_{2}\) bo'lsa, \emph{AC} ni toping. \\
C3. Muntazam uchburchakning tomoni a ga teng. Tomonini diametr deb hisoblab doira yasalgan. Uchburchakning shu doiradan tashqaridagi qismi yuzini toping. \\

\end{tabular}
\vspace{1cm}


\begin{tabular}{m{17cm}}
\textbf{39-variant}
\newline

T1. Haqiqiy \(a_{1},\ a_{2},\ .\ .\ .\ ,\ a_{n},\ b_{1},\ b_{2},\ .\ .\ .\ ,\ b_{n}\) sonlari uchun \(\left( a_{1}b_{1} + a_{2}b_{2} + \ .\ .\ .\  + a_{n}b_{n} \right)^{2} \leq \left( a_{1}^{2} + a_{2}^{2} + \ .\ .\ .\  + a_{n}^{2} \right)\left( b_{1}^{2} + b_{2}^{2} + \ .\ .\ .\  + b_{n}^{2} \right)\) \\
T2. \(P(x) = x^{6} - 3x^{5} + x^{4} - 6x^{2} + 2x - 6\) ko'phadining butun ildizlarini toping. \\
A1. Tenglamani yeching \(\left( x^{2} + 10x + 10 \right)\left( x^{2} + x + 10 \right) = 10x^{2}\) . \\
A2. Tenglamani yeching. \((x - 4)^{3} + (x - 4)^{2} + (x - 4)(x - 3) + (x - 3)^{2} + (x - 3)^{3} = 6\). \\
A3. Tenglamani yeching. \(\sqrt{3x^{2} - 2x + 15} + \sqrt{3x^{2} - 2x + 8} = 7\). \\
B1. \(P(x + 1) + P(x - 3) = 2x^{2} - 10x + 16\) bo'lsa, \(P(x)\) ni toping. \\
B2. Quyidagi mulohazani ixtiyoriy natural son uchun matematik induksiya metodi yordamida isbotlang: \(\frac{1}{1 \cdot 4} + \frac{1}{4 \cdot 7} + \frac{1}{7 \cdot 10} + \ldots + \frac{1}{(3n - 2) \cdot (3n + 1)} = \frac{n}{(3n + 1)}\). \\
B3. Quyidagi mulohazani ixtiyoriy natural son uchun matematik induksiya metodi yordamida isbotlang: \(n\left( 2n^{2} - 3n + 1 \right)\) soni 6 ga karrali ; \\
C1. Ayniyatni isbotlang:\(C_{n}^{j} = C_{n}^{n - j}\); \\
C2. Agar teng yonli uchburchakning perimetri 32 dm , o'rta chizig'i 6 dm ga teng bo'lsa, uning tomonlari uzunliklari topilsin. \\
C3. Agar \(S\) uchburchakning yuzi, \(b\) va \(c\) uning tomonlari bo'lsa, \(S \leq \frac{b^{2} + c^{2}}{4}\) bo'lishini isbotlang. \\

\end{tabular}
\vspace{1cm}


\begin{tabular}{m{17cm}}
\textbf{40-variant}
\newline

T1. Matematik induksiya metodi va uning qo'llanilishiga misollar. \\
T2. \(P(x) = (x - 1)^{20}\left( x^{2} + 25 \right)\) ko'phadining koeffitsentlari yig'indisini toping. \\
A1. Tenglamani yeching \((x + 1)^{5} + (x - 1)^{5} = 32x\). \\
A2. Tenglamani yeching. \(\sqrt[3]{x - 1} + \sqrt[3]{x - 2} - \sqrt{2x - 3} = 0\). \\
A3. Tenglamani yeching. \(\frac{4x}{x^{2} + x + 3} + \frac{5x}{x^{2} - 5x + 3} = - \frac{3}{2}\). \\
B1. \(P(x) = x^{33} - 2ax^{21} + x^{8} + 8\) ko'phadi berilgan. \(a\) ning qaysi qiymati uchun \(P(x)\) ko'phadi \(x + 1\) ga qoldiqsiz bo'liadi? \\
B2. Quyidagi mulohazani ixtiyoriy natural son uchun matematik induksiya metodi yordamida isbotlang: \(1^{2} + 2^{2} + 3^{2} + ... + n^{2} = \frac{n(n + 1)(2n + 1)}{6}\); \\
B3. Quyidagi mulohazani ixtiyoriy natural son uchun matematik induksiya metodi yordamida isbotlang:\(6^{2n - 2} + 3^{n + 1} + 3^{n - 1}\) soni 11 karrali ; \\
C1. Ayniyatni isbotlang: \(\sum_{j = 0}^{n}C_{n}^{j} = 2^{n}\); \\
C2. Bir burchagi \(60^{{^\circ}}\) bo'lgan uchburchakka ichki chizilgan aylananing urinish nuqtasi shu burchakka qarama- qarshi tomonini \(a\) va \(b\) kesmalarga ajratadi. Uchburchak yuzini toping. \\
C3. Ya. Bermlli tengsizligi. Agar \(x \geq - 1\) bo'lsa, u holda ixtiyoriy natural \(n\) soni uchun \((1 + x)^{n} \geq 1 + nx\) tengsizlik o'rinli bo'lishini isbotlang. \\

\end{tabular}
\vspace{1cm}


\begin{tabular}{m{17cm}}
\textbf{41-variant}
\newline

T1. \(P(x) = (x - 1)^{20}\left( x^{2} + 25 \right)\) ko'phadining koeffitsentlari yig'indisini toping. \\
T2. Ixtiyoriy \(a\) parametri va \(x\) uchun \(x(a - x) \leq a^{2}/4\) tengsizligi o'rinli bo'lishini isbotlang. \\
A1. Tenglamani yeching \(\left( x^{2} + 10x + 10 \right)\left( x^{2} + x + 10 \right) = 10x^{2}\) . \\
A2. Tenglamani yeching. \(\sqrt{x} + \frac{2x + 1}{x + 2} = 2\). \\
A3. Tenglamani yeching \(\left( x^{2} - 6x \right)^{2} - 2(x - 3)^{2} = 81\). \\
B1. \(P(x) = (x - 5)^{2n + 1} + (x - 1)^{2n + 3}\) ko'phadni \(x - 3\) ga bo'lganda qoldiq \(3 \cdot 2^{3n - 4}\) bo'lsa, \(n\) ni toping. \\
B2. Quyidagi mulohazani ixtiyoriy natural son uchun matematik induksiya metodi yordamida isbotlang: \(\frac{1}{4 \cdot 5} + \frac{1}{5 \cdot 6} + \frac{1}{6 \cdot 7} + \ldots + \frac{1}{(n + 3) \cdot (n + 4)} = \frac{n}{4 \cdot (n + 4)}\). \\
B3. Quyidagi mulohazani ixtiyoriy natural son uchun matematik induksiya metodi yordamida isbotlang: \(5 \cdot 2^{3n - 2} + 3^{3n - 1}\) soni 19 ga karrali \\
C1. \(\left( \sqrt{x} + \frac{1}{\sqrt[3]{x^{2}}} \right)^{n}\) binom yoyilmasida 5-had koeffitsiyentining 3-had koeffitsiyentiga nisbati 7:2 ga teng. \(x\) ning darajasi 1 ga teng bo'lgan ahadin toping. \\
C2. Uchburchakning a, b va \(c\) tomonlari arifmetik progressiya tashkil qiladi. \(ac = 6Rr\) bo'lishini isbotlang. Bu yerda \(R\) va \(r\) tashqi va ichki chizilgan aylanalarning radiuslari. \\
C3. Muntazam uchburchakning tomoni a ga teng. Tomonini diametr deb hisoblab doira yasalgan. Uchburchakning shu doiradan tashqaridagi qismi yuzini toping. \\

\end{tabular}
\vspace{1cm}


\begin{tabular}{m{17cm}}
\textbf{42-variant}
\newline

T1. Ixtiyoriy \(a,b,c \in (0;1)\) sonlari uchun \(a(1 - b) > 1/4,\ b(1 - c) > 1/4,\ c(1 - a) > 1/4\) tengsizliklari bir vaqtda o'rinli bo'la olmasligini isbotlang. \\
T2. Pifagor teoremasi va uning isbotlari. \\
A1. Tenglamani yeching. \((\sqrt{x + 1} + \sqrt{x})^{3} + (\sqrt{x + 1} + \sqrt{x})^{2} = 2\). \\
A2. Tenglamani yeching. \(\sqrt{x^{2} + x + 4} + \sqrt{x^{2} + x + 1} = \sqrt{2x^{2} + 2x + 9}\). \\
A3. Tenglamani yeching \(\sqrt{\frac{18 - 7x - x^{2}}{8 - 6x + x^{2}}} + \sqrt{\frac{8 - 6x + x^{2}}{18 - 7x - x^{2}}} = \frac{13}{6}\). \\
B1. \(P(x + 1) + P(x - 3) = 2x^{2} - 10x + 16\) bo'lsa, \(P(x)\) ni toping. \\
B2. Quyidagi mulohazani ixtiyoriy natural son uchun matematik induksiya metodi yordamida isbotlang: \(\frac{1}{1 \cdot 5} + \frac{1}{5 \cdot 9} + ... + \frac{1}{(4n - 3)(4n + 1)} = \frac{n}{4n + 1}\); \\
B3. Quyidagi mulohazani ixtiyoriy natural son uchun matematik induksiya metodi yordamida isbotlang: \(n^{3} + (n + 1)^{3} + (n + 2)^{3}\) soni 9 ga karrali ; \\
C1. Ayniyatni isbotlang: \(\sum_{j = 0}^{n}C_{n}^{j} = 2^{n}\); \\
C2. To'g'ri burchakli uchburchakning balandligi gipotenuzani uzunliklari 18 va 32 sm ga teng bo'lgan kesmalarga ajratadi. Uchburchakning yuzi hisoblansin. \\
C3. Agar \(S\) uchburchakning yuzi, \(b\) va \(c\) uning tomonlari bo'lsa, \(S \leq \frac{b^{2} + c^{2}}{4}\) bo'lishini isbotlang. \\

\end{tabular}
\vspace{1cm}


\begin{tabular}{m{17cm}}
\textbf{43-variant}
\newline

T1. Koshi tengsizligini isbotlang. \\
T2. Matematik induksiya metodi va uning qo'llanilishiga misollar. \\
A1. Tengsizlikni yeching: \(\sqrt{x + 3} + \sqrt{x - 2} - \sqrt{2x + 4} > 0\). \\
A2. Tenglamani yeching \(\left( x^{2} - 4x + 6 \right)^{2} - 4\left( x^{2} - 4x + 6 \right) + 6 = x\). \\
A3. Tenglamani yeching. \((x - 4)^{3} + (x - 4)^{2} + (x - 4)(x - 3) + (x - 3)^{2} + (x - 3)^{3} = 6\). \\
B1. \(P(2x - 1) + P(x - 1) = 10x^{2} - 12x + 2\) bo'lsa, \(P(x)\) ni toping. \\
B2. Quyidagi mulohazani ixtiyoriy natural son uchun matematik induksiya metodi yordamida isbotlang: \(2^{2} + 6^{2} + \ldots + (4n - 2)^{2} = \frac{4n(2n - 1)(2n + 1)}{3}\). \\
B3. Quyidagi mulohazani ixtiyoriy natural son uchun matematik induksiya metodi yordamida isbotlang: \(2n^{3} + 3n^{2} + 7n\) soni 6 ga karrali ; \\
C1. \(\frac{1}{C_{4}^{n}} = \frac{1}{C_{5}^{n}} + \frac{1}{C_{6}^{n}}\) bo'lsa, \(n\) ni toping \\
C2. \(R\) radiusli doiraga bitta umumiy uchga ega bo'lgan muntazam uchburchak va kvadrat ichki chizilgan. Ularning kesishgan qismining yuzini toping. \\
C3. Ya. Bermlli tengsizligi. Agar \(x \geq - 1\) bo'lsa, u holda ixtiyoriy natural \(n\) soni uchun \((1 + x)^{n} \geq 1 + nx\) tengsizlik o'rinli bo'lishini isbotlang. \\

\end{tabular}
\vspace{1cm}


\begin{tabular}{m{17cm}}
\textbf{44-variant}
\newline

T1. Ushbu \(P(0) = 20\) va \(P(1) = 100\) shartlarini qanoatlantiruvchi \(P(x)\) ko'phadi mavjudmi? \\
T2. Fales teoremasi va uning qo'llanilishi. \\
A1. Tengsizlikni yeching:\(x^{2}\left( x^{4} + 36 \right) - 6\sqrt{3}\left( x^{4} + 4 \right) < 0\). \\
A2. Tenglamani yeching. \(\frac{4x}{x^{2} + x + 3} + \frac{5x}{x^{2} - 5x + 3} = - \frac{3}{2}\). \\
A3. Tenglamani yeching. \(\sqrt{3x^{2} - 2x + 15} + \sqrt{3x^{2} - 2x + 8} = 7\). \\
B1. \(P(x + 3)\) ko'phadni \(x + 1\) ga bo'lganda qoldiq -3, \(Q(2x - 1)\) ko'phadni \(x - 1\)ga bo'lganda qoldiq 2 bo'lsa, \(P(x + 4) + x^{2}Q(x + 3)\) ko'phadni \(x + 2\) ga bo'lgandagi qoldiqni toping. \\
B2. Quyidagi mulohazani ixtiyoriy natural son uchun matematik induksiya metodi yordamida isbotlang: \(1 \cdot 2 + 2 \cdot 3 + 3 \cdot 4 + ... + n(n + 1) = \frac{n(n + 1)(n + 2)}{3}\); \\
B3. Quyidagi mulohazani ixtiyoriy natural son uchun matematik induksiya metodi yordamida isbotlang: \(5^{n} - 4n + 15\) soni 16 ga karrali ; \\
C1. Ayniyatni isbotlang:\(\sum_{j = 0}^{n}C_{n}^{j}( - 1)^{j} = 0\); \\
C2. \emph{ABC} uchburchak berilgan. Uning medianalaridan \(\bigtriangleup A_{1}B_{1}C_{1}\) yasalgan. \(\bigtriangleup ABC\) va \(\bigtriangleup A_{1}B_{1}C_{1}\) yuzlarining nisbati topilsin. \\
C3. Agar \(S\) uchburchakning yuzi, \(b\) va \(c\) uning tomonlari bo'lsa, \(S \leq \frac{b^{2} + c^{2}}{4}\) bo'lishini isbotlang. \\

\end{tabular}
\vspace{1cm}


\begin{tabular}{m{17cm}}
\textbf{45-variant}
\newline

T1. Simmetrik ko'phadlar. \\
T2. Yig'indisi birga teng bo'lgan \(x,y,z\) musbat sonlari uchun \(\frac{1}{x} + \frac{1}{y} + \frac{1}{z} \geq 9\) tengsizligi o'rinli bo'lishini isbotlang. \\
A1. Tenglamani yeching. \(\sqrt[3]{x - 1} + \sqrt[3]{x - 2} - \sqrt{2x - 3} = 0\). \\
A2. Tenglamani yeching. \(\sqrt[3]{x} + \sqrt[3]{x - 16} = \sqrt[3]{x - 8}\). \\
A3. Tengsizlikni yeching: \(\frac{x^{3} + 3x^{2} - x - 3}{x^{2} + 3x - 10} < 0\). \\
B1. \(P(x) = x^{4} - 2x + 2^{n + 1}\) ko'phadni \(x - 2^{n}\) ga bo'lganda qoldiq \(2^{n - 2}\) bo'lsa, \(n\) ni toping. \\
B2. Quyidagi mulohazani ixtiyoriy natural son uchun matematik induksiya metodi yordamida isbotlang: \(\frac{1}{4 \cdot 5} + \frac{1}{5 \cdot 6} + \frac{1}{6 \cdot 7} + \ldots + \frac{1}{(n + 3) \cdot (n + 4)} = \frac{n}{4 \cdot (n + 4)}\). \\
B3. Quyidagi mulohazani ixtiyoriy natural son uchun matematik induksiya metodi yordamida isbotlang: \(5^{2n + 1} + 3^{n + 2} \cdot 2^{n - 1}\) soni 19 ga karrali ; \\
C1. Ayniyatni isbotlang: \(C_{n + 1}^{j + 1} = C_{n}^{j} + C_{n}^{j + 1}\); \\
C2. Asoslari \(x\) va 3 bo'lgan trapetsiyada diagonallar o'rtalari orasidagi masofani \(x\) ning funksiyasi sifatida ifodalang. \(x\) nіnng qanday qiymatida bu masofa 1 ga teng bo'ladi? \\
C3. Muntazam uchburchakning tomoni a ga teng. Tomonini diametr deb hisoblab doira yasalgan. Uchburchakning shu doiradan tashqaridagi qismi yuzini toping. \\

\end{tabular}
\vspace{1cm}


\begin{tabular}{m{17cm}}
\textbf{46-variant}
\newline

T1. \(x\) o'zgaruvchining ixtiyoriy butun qiymatida \(ax^{2} + bx + c\) uchhadining qiymati butun bo'lishi uchun \(2a,\ a + b\) va \(c\) sonlarining butun bo'lishi zarur va yetarli ekanligini isbotlang. \\
T2. \(b\) parametrining qanday qiymatlarida \(x^{3} + 17x^{2} + bx - 17 = 0\) tenglamasining ildizlari butun sonlardan iborat bo'ladi? \\
A1. Tenglamani yeching. \(\sqrt{x + 8 + 2\sqrt{x + 7}} + \sqrt{x + 1 - \sqrt{x + 7}} = 4\). \\
A2. Tenglamani yeching \((x + 1)^{5} + (x - 1)^{5} = 32x\). \\
A3. Tenglamani yeching \((x + 4)(x + 1) - 3\sqrt{x^{2} + 5x + 2} = 6\). \\
B1. \(P(x)\) ko'phadni \(3x^{2} - 4x + 1\) ga bo'lganimizda qoldiq \(6x - 11\) bo'lsa, \(P(x)\) ko'phadni \(3x - 1\)ga bo'lganda qoldiqni toping. \\
B2. Quyidagi mulohazani ixtiyoriy natural son uchun matematik induksiya metodi yordamida isbotlang: \(1 \cdot 2 + 2 \cdot 3 + 3 \cdot 4 + ... + n(n + 1) = \frac{n(n + 1)(n + 2)}{3}\); \\
B3. Quyidagi mulohazani ixtiyoriy natural son uchun matematik induksiya metodi yordamida isbotlang: \(5 \cdot 2^{3n - 2} + 3^{3n - 1}\) soni 19 ga karrali \\
C1. Teńlemeni sheshiń \(\frac{C_{2x}^{x + 1}}{C_{2x + 1}^{x - 1}} = \frac{2}{3}\), \(x \in N\) \\
C2. \(\bigtriangleup ABC\) da \(AB = 2sm,BD\) mediana, \(BD = 1sm\), \(\angle BDA = 30^{{^\circ}}\). Uchburchakning yuzi hisoblansin. \\
C3. Agar \(S\) uchburchakning yuzi, \(b\) va \(c\) uning tomonlari bo'lsa, \(S \leq \frac{b^{2} + c^{2}}{4}\) bo'lishini isbotlang. \\

\end{tabular}
\vspace{1cm}


\begin{tabular}{m{17cm}}
\textbf{47-variant}
\newline

T1. Haqiqiy \(a_{1},\ a_{2},\ .\ .\ .\ ,\ a_{n},\ b_{1},\ b_{2},\ .\ .\ .\ ,\ b_{n}\) sonlari uchun \(\left( a_{1}b_{1} + a_{2}b_{2} + \ .\ .\ .\  + a_{n}b_{n} \right)^{2} \leq \left( a_{1}^{2} + a_{2}^{2} + \ .\ .\ .\  + a_{n}^{2} \right)\left( b_{1}^{2} + b_{2}^{2} + \ .\ .\ .\  + b_{n}^{2} \right)\) \\
T2. \(a\) parametrining qanday qiymatlarida \(P(x) = x^{2017} + ax - 5\) ko'phadi \((x + 1)\) ko'phadiga qoldiqsiz bo'linadi? \\
A1. Tengsizlikni yeching: \(\sqrt{x^{2} - 4x} > x - 3\). \\
A2. Tenglamani yeching. \(\frac{z}{z + 1} - 2\sqrt{\frac{z + 1}{2}} = 3\). \\
A3. Tenglamani yeching. \(\frac{z}{z + 1} - 2\sqrt{\frac{z + 1}{2}} = 3\). \\
B1. \(P(x) = x^{33} - 2ax^{21} + x^{8} + 8\) ko'phadi berilgan. \(a\) ning qaysi qiymati uchun \(P(x)\) ko'phadi \(x + 1\) ga qoldiqsiz bo'liadi? \\
B2. Quyidagi mulohazani ixtiyoriy natural son uchun matematik induksiya metodi yordamida isbotlang: \(2^{2} + 6^{2} + \ldots + (4n - 2)^{2} = \frac{4n(2n - 1)(2n + 1)}{3}\). \\
B3. Quyidagi mulohazani ixtiyoriy natural son uchun matematik induksiya metodi yordamida isbotlang: \(n^{3} + (n + 1)^{3} + (n + 2)^{3}\) soni 9 ga karrali ; \\
C1. \(C_{n + 4}^{n + 1} - C_{n + 3}^{n} = 15(n + 2)\) bo'lsa, \(n\) ni toping. \\
C2. Markazlari \(O_{1}\) va \(O_{2}\) nuqtalarda va radiusi \(R\) bo'lgan ikkita teng aylanalar tashqi urinadi. \(l\) to'g'ri chiziq bu aylanalarni A, B, C va \(D\) nuqtalarda shunday kesib o'tadiki, \(AB = BC = CD\) bo'ladi. \(O_{1}ADO_{2}\) to'rtburchak yuzini toping. \\
C3. Muntazam uchburchakning tomoni a ga teng. Tomonini diametr deb hisoblab doira yasalgan. Uchburchakning shu doiradan tashqaridagi qismi yuzini toping. \\

\end{tabular}
\vspace{1cm}


\begin{tabular}{m{17cm}}
\textbf{48-variant}
\newline

T1. \(n\) darajaning qanday qiymatlarida \((x + 1)^{n} + (x - 1)^{n}\) ifodasi \(x\) ifodaga qoldiqsiz bo'linadi? \\
T2. Bezu teoremasi va uning qo'llanilishi. \\
A1. Tenglamani yeching. \(\sqrt{3x^{2} - 2x + 15} + \sqrt{3x^{2} - 2x + 8} = 7\). \\
A2. Tenglamani yeching. \(\frac{4x}{x^{2} + x + 3} + \frac{5x}{x^{2} - 5x + 3} = - \frac{3}{2}\). \\
A3. Tenglamani yeching. \(\sqrt[3]{x} + \sqrt[3]{x - 16} = \sqrt[3]{x - 8}\). \\
B1. \(P(x + n) = (x + n)^{3} + (x - n)^{2} + x + n + 6\) ko'phadi berilgan. \(P(x)\) ko'phadi \(x - n\) ga qoldiqsiz bo'linsa, \(n\) ni toping. \\
B2. Quyidagi mulohazani ixtiyoriy natural son uchun matematik induksiya metodi yordamida isbotlang: \(1^{2} + 2^{2} + 3^{2} + ... + n^{2} = \frac{n(n + 1)(2n + 1)}{6}\); \\
B3. Quyidagi mulohazani ixtiyoriy natural son uchun matematik induksiya metodi yordamida isbotlang: \(5^{n + 2} + 26 \cdot 5^{n} + 8^{2n + 1}\) soni 59 ga karrali; \\
C1. Teńsizlikti sheshiń \(5C_{x}^{3} < C_{x + 2}^{4}\), \(x \in N\) \\
C2. To'g'ri burchakli uchburchakning balandligi gipotenuzani uzunliklari \emph{x} va \emph{y} ga teng bo'lgan kesmalarga ajratadi. Uchburchakning yuzi hisoblansin. \\
C3. Ya. Bermlli tengsizligi. Agar \(x \geq - 1\) bo'lsa, u holda ixtiyoriy natural \(n\) soni uchun \((1 + x)^{n} \geq 1 + nx\) tengsizlik o'rinli bo'lishini isbotlang. \\

\end{tabular}
\vspace{1cm}


\begin{tabular}{m{17cm}}
\textbf{49-variant}
\newline

T1. Kombinatorika elementlari va Nyuton binomi. \\
T2. \(P(x) = x^{6} - 3x^{5} + x^{4} - 6x^{2} + 2x - 6\) ko'phadining butun ildizlarini toping. \\
A1. Tenglamani yeching. \(\sqrt{x} + \frac{2x + 1}{x + 2} = 2\). \\
A2. Tenglamani yeching \((x + 1)^{5} + (x - 1)^{5} = 32x\). \\
A3. Tenglamani yeching. \(\sqrt{x + 8 + 2\sqrt{x + 7}} + \sqrt{x + 1 - \sqrt{x + 7}} = 4\). \\
B1. \(P(x + 3) = x^{2} - x + n\) bo'lsa. \(P(x - 2)\) ko'phadni \(x - 3\) ga bo'lganda qoldiq \(10\) bo'lsa, \(n\) ni toping. \\
B2. Quyidagi mulohazani ixtiyoriy natural son uchun matematik induksiya metodi yordamida isbotlang: \(\frac{1}{1 \cdot 5} + \frac{1}{5 \cdot 9} + ... + \frac{1}{(4n - 3)(4n + 1)} = \frac{n}{4n + 1}\); \\
B3. Quyidagi mulohazani ixtiyoriy natural son uchun matematik induksiya metodi yordamida isbotlang: \(5^{2n + 1} + 3^{n + 2} \cdot 2^{n - 1}\) soni 19 ga karrali ; \\
C1. \(x(1 - x)^{4} + x^{2}(1 + 2x)^{8} + x^{3}(1 + 3x)^{12}\) ifodada \(x^{4}\) oldidagi koeffitsiyentni toping. \\
C2. \(\bigtriangleup ABC\) da \(AB = 3sm,AC = 5sm,\angle BAC = 120^{{^\circ}}.BD\) bissektrisaning uzunligi topilsin. \\
C3. Muntazam uchburchakning tomoni a ga teng. Tomonini diametr deb hisoblab doira yasalgan. Uchburchakning shu doiradan tashqaridagi qismi yuzini toping. \\

\end{tabular}
\vspace{1cm}


\begin{tabular}{m{17cm}}
\textbf{50-variant}
\newline

T1. \(2^{81} + 1\) soni 9 soniga qoldiqsiz bo'linishini isbotlang. \\
T2. Ushbu \(P(x) = x^{5} + 11x^{4} + 37x^{3} + 35x^{2} - 44x - 40\) ko'phadi \(Q(x) = x^{2} + 3x + 2\) ko'phadiga qoldiqsiz bo'linadimi? \\
A1. Tengsizlikni yeching: \(\frac{x^{3} + 3x^{2} - x - 3}{x^{2} + 3x - 10} < 0\). \\
A2. Tenglamani yeching. \(\sqrt[3]{x - 1} + \sqrt[3]{x - 2} - \sqrt{2x - 3} = 0\). \\
A3. Tenglamani yeching. \(\sqrt{x^{2} + x + 4} + \sqrt{x^{2} + x + 1} = \sqrt{2x^{2} + 2x + 9}\). \\
B1. \(P(x + 2) + P(x - 1) = - 2x^{2} - 2x + 7\) bo'lsa, \(P(x)\) ni \(x + 4\) ga bo'lgandagi qoldiqni toping. \\
B2. Quyidagi mulohazani ixtiyoriy natural son uchun matematik induksiya metodi yordamida isbotlang: \(1^{3} + 2^{3} + 3^{3} + ... + n^{3} = \left( \frac{n(n + 1)}{2} \right)^{2}\); \\
B3. Quyidagi mulohazani ixtiyoriy natural son uchun matematik induksiya metodi yordamida isbotlang: \(7^{n} - 1\) soni 6 ga karrali; \\
C1. \(5C_{n}^{3} = C_{n + 2}^{4}\) bo'lsa, \(n\) ni toping. \\
C2. To'g'ri burchakli uchburchak o'tkir burchaklarining bisєektrisalari AD va BK \(AB^{2} = AD \cdot BK\) bo'lsa, uchburchakning burchaklarini toping. \\
C3. Ya. Bermlli tengsizligi. Agar \(x \geq - 1\) bo'lsa, u holda ixtiyoriy natural \(n\) soni uchun \((1 + x)^{n} \geq 1 + nx\) tengsizlik o'rinli bo'lishini isbotlang. \\

\end{tabular}
\vspace{1cm}


\begin{tabular}{m{17cm}}
\textbf{51-variant}
\newline

T1. \(P(x) = (x - 1)^{20}\left( x^{2} + 25 \right)\) ko'phadining koeffitsentlari yig'indisini toping. \\
T2. Ixtiyoriy \(a,b,c \in (0;1)\) sonlari uchun \(a(1 - b) > 1/4,\ b(1 - c) > 1/4,\ c(1 - a) > 1/4\) tengsizliklari bir vaqtda o'rinli bo'la olmasligini isbotlang. \\
A1. Tengsizlikni yeching:\(x^{2}\left( x^{4} + 36 \right) - 6\sqrt{3}\left( x^{4} + 4 \right) < 0\). \\
A2. Tenglamani yeching \((x + 4)(x + 1) - 3\sqrt{x^{2} + 5x + 2} = 6\). \\
A3. Tenglamani yeching. \((x - 4)^{3} + (x - 4)^{2} + (x - 4)(x - 3) + (x - 3)^{2} + (x - 3)^{3} = 6\). \\
B1. \(P(2x - 1) + P(x - 1) = 10x^{2} - 12x + 2\) bo'lsa, \(P(x)\) ni toping. \\
B2. Quyidagi mulohazani ixtiyoriy natural son uchun matematik induksiya metodi yordamida isbotlang: \(1 \cdot 2 + 2 \cdot 3 + 3 \cdot 4 + \ldots + n \cdot (n + 1) = \frac{n \cdot (n + 1) \cdot (n + 2)}{3}\). \\
B3. Quyidagi mulohazani ixtiyoriy natural son uchun matematik induksiya metodi yordamida isbotlang: \(5^{n} - 4n + 15\) soni 16 ga karrali ; \\
C1. Tengsizlikni yeching: \(C_{10}^{x - 1} > 2C_{10}^{x}\) \\
C2. To'g'ri burchakli uchburchakning to'g'ri burchagi bissektrisasi shu uchdan o'tkazilgan mediana va balandlik orasidagi burchakni ham teng ikkiga bo'lishini isbotlang. \\
C3. Agar \(S\) uchburchakning yuzi, \(b\) va \(c\) uning tomonlari bo'lsa, \(S \leq \frac{b^{2} + c^{2}}{4}\) bo'lishini isbotlang. \\

\end{tabular}
\vspace{1cm}


\begin{tabular}{m{17cm}}
\textbf{52-variant}
\newline

T1. \(2^{81} + 1\) soni 9 soniga qoldiqsiz bo'linishini isbotlang. \\
T2. Fales teoremasi va uning qo'llanilishi. \\
A1. Tenglamani yeching. \((\sqrt{x + 1} + \sqrt{x})^{3} + (\sqrt{x + 1} + \sqrt{x})^{2} = 2\). \\
A2. Tengsizlikni yeching: \(\sqrt{x + 3} + \sqrt{x - 2} - \sqrt{2x + 4} > 0\). \\
A3. Tenglamani yeching \(\left( x^{2} + 10x + 10 \right)\left( x^{2} + x + 10 \right) = 10x^{2}\) . \\
B1. \(P(x)\) ko'phadni \(3x^{2} - 4x + 1\) ga bo'lganimizda qoldiq \(6x - 11\) bo'lsa, \(P(x)\) ko'phadni \(3x - 1\)ga bo'lganda qoldiqni toping. \\
B2. Quyidagi mulohazani ixtiyoriy natural son uchun matematik induksiya metodi yordamida isbotlang: \(\left( 1 - \frac{1}{4} \right)\left( 1 - \frac{1}{9} \right)...\left( 1 - \frac{1}{n^{2}} \right) = \frac{n + 1}{2n}\), \(n \geq 2\) \\
B3. Quyidagi mulohazani ixtiyoriy natural son uchun matematik induksiya metodi yordamida isbotlang: \(n\left( 2n^{2} - 3n + 1 \right)\) soni 6 ga karrali ; \\
C1. Ayniyatni isbotlang: \(C_{n + k}^{j + k} = \sum_{s = 0}^{k}C_{n}^{j + s}C_{k}^{s}\); \\
C2. To'g'ri burchakli uchburchakda katetlarning nisbati 3:2 kabi, balandlik esa gipotenuzani shunday ikkita kesmaga ajratadiki, ulardan birining uzunligi ikkinchisidan 2 ga katta. Gipotenuzaning uzunligi topilsin. \\
C3. Muntazam uchburchakning tomoni a ga teng. Tomonini diametr deb hisoblab doira yasalgan. Uchburchakning shu doiradan tashqaridagi qismi yuzini toping. \\

\end{tabular}
\vspace{1cm}


\begin{tabular}{m{17cm}}
\textbf{53-variant}
\newline

T1. \(a\) parametrining qanday qiymatlarida \(P(x) = x^{2017} + ax - 5\) ko'phadi \((x + 1)\) ko'phadiga qoldiqsiz bo'linadi? \\
T2. Matematik induksiya metodi va uning qo'llanilishiga misollar. \\
A1. Tengsizlikni yeching: \(\sqrt{x^{2} - 4x} > x - 3\). \\
A2. Tenglamani yeching \(\left( x^{2} - 6x \right)^{2} - 2(x - 3)^{2} = 81\). \\
A3. Tenglamani yeching \(\sqrt{\frac{18 - 7x - x^{2}}{8 - 6x + x^{2}}} + \sqrt{\frac{8 - 6x + x^{2}}{18 - 7x - x^{2}}} = \frac{13}{6}\). \\
B1. \(P(x) = x^{4} - 2x + 2^{n + 1}\) ko'phadni \(x - 2^{n}\) ga bo'lganda qoldiq \(2^{n - 2}\) bo'lsa, \(n\) ni toping. \\
B2. Quyidagi mulohazani ixtiyoriy natural son uchun matematik induksiya metodi yordamida isbotlang: \(1 \cdot 1! + 2 \cdot 2! + 3 \cdot 3! + \ldots + n \cdot n! = (n + 1)! - 1\). \\
B3. Quyidagi mulohazani ixtiyoriy natural son uchun matematik induksiya metodi yordamida isbotlang:\(6^{2n - 2} + 3^{n + 1} + 3^{n - 1}\) soni 11 karrali ; \\
C1. \((x + 1)^{3} + (x + 1)^{4} + (x + 1)^{5} + ... + (x + 1)^{10}\) ifodada \(x^{3}\) oldidagi koeffitsiyentni toping \\
C2. To'g'ri burchakli uchburchakda katetlar 7 sm va 24 sm ga teng. To'g'ri burchakning bissektrisasi o'tkazilgan. Bu bissektrisa gipotenuzani qanday uzunlikdagi kesmalarga ajratadi? \\
C3. Ya. Bermlli tengsizligi. Agar \(x \geq - 1\) bo'lsa, u holda ixtiyoriy natural \(n\) soni uchun \((1 + x)^{n} \geq 1 + nx\) tengsizlik o'rinli bo'lishini isbotlang. \\

\end{tabular}
\vspace{1cm}


\begin{tabular}{m{17cm}}
\textbf{54-variant}
\newline

T1. Ushbu \(P(x) = x^{5} + 11x^{4} + 37x^{3} + 35x^{2} - 44x - 40\) ko'phadi \(Q(x) = x^{2} + 3x + 2\) ko'phadiga qoldiqsiz bo'linadimi? \\
T2. Kombinatorika elementlari va Nyuton binomi. \\
A1. Tenglamani yeching \(\left( x^{2} - 4x + 6 \right)^{2} - 4\left( x^{2} - 4x + 6 \right) + 6 = x\). \\
A2. Tengsizlikni yeching:\(x^{2}\left( x^{4} + 36 \right) - 6\sqrt{3}\left( x^{4} + 4 \right) < 0\). \\
A3. Tenglamani yeching \(\left( x^{2} - 6x \right)^{2} - 2(x - 3)^{2} = 81\). \\
B1. \(P(x) = (x - 5)^{2n + 1} + (x - 1)^{2n + 3}\) ko'phadni \(x - 3\) ga bo'lganda qoldiq \(3 \cdot 2^{3n - 4}\) bo'lsa, \(n\) ni toping. \\
B2. Quyidagi mulohazani ixtiyoriy natural son uchun matematik induksiya metodi yordamida isbotlang: \(1^{2} + 3^{2} + 5^{2} + ... + (2n - 1)^{2} = \frac{n\left( 4n^{2} - 1 \right)}{3}\); \\
B3. Quyidagi mulohazani ixtiyoriy natural son uchun matematik induksiya metodi yordamida isbotlang: \(2n^{3} + 3n^{2} + 7n\) soni 6 ga karrali ; \\
C1. Ayniyatni isbotlang:\(C_{n}^{j} = C_{n}^{n - j}\); \\
C2. \emph{ABC} uchburchakning \emph{AC}, \emph{BC} va \emph{AB} tomonlarida \emph{CMPA}, \emph{BEFC} va \emph{ADKB} kvadratlar yasalgan. Agar \(AB = 13\), \(AC = 14,BC = 15\) ekanligi ma'lum bo'lsa, \emph{DKEFMP} oltiburchakning yuzini toping. \\
C3. Agar \(S\) uchburchakning yuzi, \(b\) va \(c\) uning tomonlari bo'lsa, \(S \leq \frac{b^{2} + c^{2}}{4}\) bo'lishini isbotlang. \\

\end{tabular}
\vspace{1cm}


\begin{tabular}{m{17cm}}
\textbf{55-variant}
\newline

T1. Simmetrik ko'phadlar. \\
T2. Haqiqiy \(a_{1},\ a_{2},\ .\ .\ .\ ,\ a_{n},\ b_{1},\ b_{2},\ .\ .\ .\ ,\ b_{n}\) sonlari uchun \(\left( a_{1}b_{1} + a_{2}b_{2} + \ .\ .\ .\  + a_{n}b_{n} \right)^{2} \leq \left( a_{1}^{2} + a_{2}^{2} + \ .\ .\ .\  + a_{n}^{2} \right)\left( b_{1}^{2} + b_{2}^{2} + \ .\ .\ .\  + b_{n}^{2} \right)\) \\
A1. Tenglamani yeching \(\sqrt{\frac{18 - 7x - x^{2}}{8 - 6x + x^{2}}} + \sqrt{\frac{8 - 6x + x^{2}}{18 - 7x - x^{2}}} = \frac{13}{6}\). \\
A2. Tenglamani yeching. \(\sqrt{x} + \frac{2x + 1}{x + 2} = 2\). \\
A3. Tenglamani yeching. \((\sqrt{x + 1} + \sqrt{x})^{3} + (\sqrt{x + 1} + \sqrt{x})^{2} = 2\). \\
B1. \(P(x + n) = (x + n)^{3} + (x - n)^{2} + x + n + 6\) ko'phadi berilgan. \(P(x)\) ko'phadi \(x - n\) ga qoldiqsiz bo'linsa, \(n\) ni toping. \\
B2. Quyidagi mulohazani ixtiyoriy natural son uchun matematik induksiya metodi yordamida isbotlang: \(\frac{1}{1 \cdot 4} + \frac{1}{4 \cdot 7} + \frac{1}{7 \cdot 10} + \ldots + \frac{1}{(3n - 2) \cdot (3n + 1)} = \frac{n}{(3n + 1)}\). \\
B3. Quyidagi mulohazani ixtiyoriy natural son uchun matematik induksiya metodi yordamida isbotlang: \(5^{2n + 1} + 3^{n + 2} \cdot 2^{n - 1}\) soni 19 ga karrali ; \\
C1. \((a + b)^{n}\) ifoda yoyilmasining barcha koeffitsiyentlari yig`indisi 4096 ga teng bo'lsa, uning eng katta koeffitsiyentin toping. \\
C2. \emph{ABCD} parallelogrammning \emph{AD} tomoni \(n\) ta teng bo'lakka bo'lingan. Birinchi bo'linish nuqtasi \(P\) va \(B\) uch bilan birlashtirilgan. \emph{BP} to'g'ri chiziq \emph{AC} dioganaldan uning \(\frac{1}{n + 1}\) qismiga teng \emph{AQ} kesma ajratishini isbotlang. \\
C3. Muntazam uchburchakning tomoni a ga teng. Tomonini diametr deb hisoblab doira yasalgan. Uchburchakning shu doiradan tashqaridagi qismi yuzini toping. \\

\end{tabular}
\vspace{1cm}


\begin{tabular}{m{17cm}}
\textbf{56-variant}
\newline

T1. \(x\) o'zgaruvchining ixtiyoriy butun qiymatida \(ax^{2} + bx + c\) uchhadining qiymati butun bo'lishi uchun \(2a,\ a + b\) va \(c\) sonlarining butun bo'lishi zarur va yetarli ekanligini isbotlang. \\
T2. \(n\) darajaning qanday qiymatlarida \((x + 1)^{n} + (x - 1)^{n}\) ifodasi \(x\) ifodaga qoldiqsiz bo'linadi? \\
A1. Tenglamani yeching. \(\sqrt{x^{2} + x + 4} + \sqrt{x^{2} + x + 1} = \sqrt{2x^{2} + 2x + 9}\). \\
A2. Tenglamani yeching \((x + 1)^{5} + (x - 1)^{5} = 32x\). \\
A3. Tenglamani yeching. \(\frac{4x}{x^{2} + x + 3} + \frac{5x}{x^{2} - 5x + 3} = - \frac{3}{2}\). \\
B1. \(P(x + 3)\) ko'phadni \(x + 1\) ga bo'lganda qoldiq -3, \(Q(2x - 1)\) ko'phadni \(x - 1\)ga bo'lganda qoldiq 2 bo'lsa, \(P(x + 4) + x^{2}Q(x + 3)\) ko'phadni \(x + 2\) ga bo'lgandagi qoldiqni toping. \\
B2. Quyidagi mulohazani ixtiyoriy natural son uchun matematik induksiya metodi yordamida isbotlang: \(\frac{1}{1 \cdot 5} + \frac{1}{5 \cdot 9} + ... + \frac{1}{(4n - 3)(4n + 1)} = \frac{n}{4n + 1}\); \\
B3. Quyidagi mulohazani ixtiyoriy natural son uchun matematik induksiya metodi yordamida isbotlang: \(5 \cdot 2^{3n - 2} + 3^{3n - 1}\) soni 19 ga karrali \\
C1. Ayniyatni isbotlang:\(C_{n + 2}^{j + 2} = C_{n}^{j} + 2C_{n}^{j + 1} + C_{n}^{j + 2}\); \\
C2. Uchburchakning tomonlari \(a\) va \(b\), bissektrisasi \(l_{c} = l\). \(l\) ni bilgan holda uning yuzini toping. \\
C3. Ya. Bermlli tengsizligi. Agar \(x \geq - 1\) bo'lsa, u holda ixtiyoriy natural \(n\) soni uchun \((1 + x)^{n} \geq 1 + nx\) tengsizlik o'rinli bo'lishini isbotlang. \\

\end{tabular}
\vspace{1cm}


\begin{tabular}{m{17cm}}
\textbf{57-variant}
\newline

T1. \(b\) parametrining qanday qiymatlarida \(x^{3} + 17x^{2} + bx - 17 = 0\) tenglamasining ildizlari butun sonlardan iborat bo'ladi? \\
T2. Yig'indisi birga teng bo'lgan \(x,y,z\) musbat sonlari uchun \(\frac{1}{x} + \frac{1}{y} + \frac{1}{z} \geq 9\) tengsizligi o'rinli bo'lishini isbotlang. \\
A1. Tenglamani yeching. \(\sqrt[3]{x} + \sqrt[3]{x - 16} = \sqrt[3]{x - 8}\). \\
A2. Tenglamani yeching. \(\sqrt{3x^{2} - 2x + 15} + \sqrt{3x^{2} - 2x + 8} = 7\). \\
A3. Tengsizlikni yeching: \(\frac{x^{3} + 3x^{2} - x - 3}{x^{2} + 3x - 10} < 0\). \\
B1. \(P(x + 3) = x^{2} - x + n\) bo'lsa. \(P(x - 2)\) ko'phadni \(x - 3\) ga bo'lganda qoldiq \(10\) bo'lsa, \(n\) ni toping. \\
B2. Quyidagi mulohazani ixtiyoriy natural son uchun matematik induksiya metodi yordamida isbotlang: \(1 \cdot 2 + 2 \cdot 3 + 3 \cdot 4 + ... + n(n + 1) = \frac{n(n + 1)(n + 2)}{3}\); \\
B3. Quyidagi mulohazani ixtiyoriy natural son uchun matematik induksiya metodi yordamida isbotlang: \(5^{n + 2} + 26 \cdot 5^{n} + 8^{2n + 1}\) soni 59 ga karrali; \\
C1. \(\left( x^{3} - \frac{3}{x^{2}} \right)^{10}\) binom yoyilmasining \(x\) qatnashmagan hadin toping. \\
C2. Muntazam uchburchakning uchlari uchta parallel to'g'ri chiziqlarda yotadi. Agar o'rtadagi to'g'ri chiziqdan chekkalardagi to'g'ri chiziklargacha bo'lgan masofa \(a\) va \(b\) ga teng bo'lsa, uchburchakning tomonini toping. \\
C3. Agar \(S\) uchburchakning yuzi, \(b\) va \(c\) uning tomonlari bo'lsa, \(S \leq \frac{b^{2} + c^{2}}{4}\) bo'lishini isbotlang. \\

\end{tabular}
\vspace{1cm}


\begin{tabular}{m{17cm}}
\textbf{58-variant}
\newline

T1. \(P(x) = x^{6} - 3x^{5} + x^{4} - 6x^{2} + 2x - 6\) ko'phadining butun ildizlarini toping. \\
T2. Ixtiyoriy \(a\) parametri va \(x\) uchun \(x(a - x) \leq a^{2}/4\) tengsizligi o'rinli bo'lishini isbotlang. \\
A1. Tenglamani yeching. \(\frac{z}{z + 1} - 2\sqrt{\frac{z + 1}{2}} = 3\). \\
A2. Tengsizlikni yeching: \(\sqrt{x + 3} + \sqrt{x - 2} - \sqrt{2x + 4} > 0\). \\
A3. Tenglamani yeching \(\left( x^{2} - 4x + 6 \right)^{2} - 4\left( x^{2} - 4x + 6 \right) + 6 = x\). \\
B1. \(P(x + 1) + P(x - 3) = 2x^{2} - 10x + 16\) bo'lsa, \(P(x)\) ni toping. \\
B2. Quyidagi mulohazani ixtiyoriy natural son uchun matematik induksiya metodi yordamida isbotlang: \(1^{2} + 2^{2} + 3^{2} + ... + n^{2} = \frac{n(n + 1)(2n + 1)}{6}\); \\
B3. Quyidagi mulohazani ixtiyoriy natural son uchun matematik induksiya metodi yordamida isbotlang: \(n\left( 2n^{2} - 3n + 1 \right)\) soni 6 ga karrali ; \\
C1. Teńsizlikti sheshiń \(C_{13}^{x} < C_{13}^{x + 2}\), \(x \in N\) \\
C2. Uchburchakning perimetri \(4,5dm\) ga teng, bissektrisa esa qarshi tomonni uzunliklari 6 va 9 sm ga teng bo'lgan kesmalarga ajratadi. Uchburchakning tomonlari topilsin. \\
C3. Agar \(S\) uchburchakning yuzi, \(b\) va \(c\) uning tomonlari bo'lsa, \(S \leq \frac{b^{2} + c^{2}}{4}\) bo'lishini isbotlang. \\

\end{tabular}
\vspace{1cm}


\begin{tabular}{m{17cm}}
\textbf{59-variant}
\newline

T1. Bezu teoremasi va uning qo'llanilishi. \\
T2. Koshi tengsizligini isbotlang. \\
A1. Tengsizlikni yeching: \(\sqrt{x^{2} - 4x} > x - 3\). \\
A2. Tenglamani yeching. \((x - 4)^{3} + (x - 4)^{2} + (x - 4)(x - 3) + (x - 3)^{2} + (x - 3)^{3} = 6\). \\
A3. Tenglamani yeching. \(\sqrt{x + 8 + 2\sqrt{x + 7}} + \sqrt{x + 1 - \sqrt{x + 7}} = 4\). \\
B1. \(P(x + 2) + P(x - 1) = - 2x^{2} - 2x + 7\) bo'lsa, \(P(x)\) ni \(x + 4\) ga bo'lgandagi qoldiqni toping. \\
B2. Quyidagi mulohazani ixtiyoriy natural son uchun matematik induksiya metodi yordamida isbotlang: \(1 \cdot 2 + 2 \cdot 3 + 3 \cdot 4 + \ldots + n \cdot (n + 1) = \frac{n \cdot (n + 1) \cdot (n + 2)}{3}\). \\
B3. Quyidagi mulohazani ixtiyoriy natural son uchun matematik induksiya metodi yordamida isbotlang: \(n^{3} + (n + 1)^{3} + (n + 2)^{3}\) soni 9 ga karrali ; \\
C1. \(\left( 2x^{\ ^{2}} - \frac{b}{2x^{3}} \right)^{10}\) binom yoyilmasining \(x\) qatnashmagan hadin toping. \\
C2. \emph{ABC} uchburchakning \emph{AB} tomonida yotgan \(N\) nuqtadan \(NQ\| AC\) va \(NP\| BC\) to'g'ri chiziqlar o'tkazilgan. Agar \emph{BNQ} uchburchakning yuzi \(S_{1}\) ga, \emph{ANP} uchburchakning yuzi \(S_{2}\) ga tengligi ma'lum bo'lsa, \emph{ABC} uchburchakning yuzini toping. \\
C3. Muntazam uchburchakning tomoni a ga teng. Tomonini diametr deb hisoblab doira yasalgan. Uchburchakning shu doiradan tashqaridagi qismi yuzini toping. \\

\end{tabular}
\vspace{1cm}


\begin{tabular}{m{17cm}}
\textbf{60-variant}
\newline

T1. Pifagor teoremasi va uning isbotlari. \\
T2. Ushbu \(P(0) = 20\) va \(P(1) = 100\) shartlarini qanoatlantiruvchi \(P(x)\) ko'phadi mavjudmi? \\
A1. Tenglamani yeching \(\left( x^{2} + 10x + 10 \right)\left( x^{2} + x + 10 \right) = 10x^{2}\) . \\
A2. Tenglamani yeching. \(\sqrt[3]{x - 1} + \sqrt[3]{x - 2} - \sqrt{2x - 3} = 0\). \\
A3. Tenglamani yeching \((x + 4)(x + 1) - 3\sqrt{x^{2} + 5x + 2} = 6\). \\
B1. \(P(x) = x^{33} - 2ax^{21} + x^{8} + 8\) ko'phadi berilgan. \(a\) ning qaysi qiymati uchun \(P(x)\) ko'phadi \(x + 1\) ga qoldiqsiz bo'liadi? \\
B2. Quyidagi mulohazani ixtiyoriy natural son uchun matematik induksiya metodi yordamida isbotlang: \(\left( 1 - \frac{1}{4} \right)\left( 1 - \frac{1}{9} \right)...\left( 1 - \frac{1}{n^{2}} \right) = \frac{n + 1}{2n}\), \(n \geq 2\) \\
B3. Quyidagi mulohazani ixtiyoriy natural son uchun matematik induksiya metodi yordamida isbotlang: \(2n^{3} + 3n^{2} + 7n\) soni 6 ga karrali ; \\
C1. \(\left( x\sqrt{x} - \frac{1}{x^{4}} \right)^{n}\) binom yoyilmasida 3-had koeffitsiyenti 2-had koeffitsiyentidan 44 ga katta.Ozod hadini toping. \\
C2. To'g'ri burchakli uchburchakning katetlari \(b\) va \(c\) ga teng. To'g'ri burchak bissektrisasining uzunligi topilsin. \\
C3. Ya. Bermlli tengsizligi. Agar \(x \geq - 1\) bo'lsa, u holda ixtiyoriy natural \(n\) soni uchun \((1 + x)^{n} \geq 1 + nx\) tengsizlik o'rinli bo'lishini isbotlang. \\

\end{tabular}
\vspace{1cm}


\begin{tabular}{m{17cm}}
\textbf{61-variant}
\newline

T1. \(P(x) = x^{6} - 3x^{5} + x^{4} - 6x^{2} + 2x - 6\) ko'phadining butun ildizlarini toping. \\
T2. \(a\) parametrining qanday qiymatlarida \(P(x) = x^{2017} + ax - 5\) ko'phadi \((x + 1)\) ko'phadiga qoldiqsiz bo'linadi? \\
A1. Tenglamani yeching. \(\sqrt{x + 8 + 2\sqrt{x + 7}} + \sqrt{x + 1 - \sqrt{x + 7}} = 4\). \\
A2. Tenglamani yeching. \(\sqrt{x^{2} + x + 4} + \sqrt{x^{2} + x + 1} = \sqrt{2x^{2} + 2x + 9}\). \\
A3. Tengsizlikni yeching: \(\frac{x^{3} + 3x^{2} - x - 3}{x^{2} + 3x - 10} < 0\). \\
B1. \(P(x)\) ko'phadni \(3x^{2} - 4x + 1\) ga bo'lganimizda qoldiq \(6x - 11\) bo'lsa, \(P(x)\) ko'phadni \(3x - 1\)ga bo'lganda qoldiqni toping. \\
B2. Quyidagi mulohazani ixtiyoriy natural son uchun matematik induksiya metodi yordamida isbotlang: \(1 \cdot 1! + 2 \cdot 2! + 3 \cdot 3! + \ldots + n \cdot n! = (n + 1)! - 1\). \\
B3. Quyidagi mulohazani ixtiyoriy natural son uchun matematik induksiya metodi yordamida isbotlang: \(5^{n} - 4n + 15\) soni 16 ga karrali ; \\
C1. Ayniyatni isbotlang:\(C_{n}^{j} = C_{n}^{n - j}\); \\
C2. \(ABCD(AD\| BC)\) trapetsiya diagonallari \(O\) nuqtada kesishadi. Agar \emph{AOD} uchburchakning yuzi \(a^{2}\) ga, \emph{BOC} uchburchakning yuzi \(b^{2}\) ga tengligi ma'lum bo'lsa, trapesiya yuzini toping. \\
C3. Ya. Bermlli tengsizligi. Agar \(x \geq - 1\) bo'lsa, u holda ixtiyoriy natural \(n\) soni uchun \((1 + x)^{n} \geq 1 + nx\) tengsizlik o'rinli bo'lishini isbotlang. \\

\end{tabular}
\vspace{1cm}


\begin{tabular}{m{17cm}}
\textbf{62-variant}
\newline

T1. Haqiqiy \(a_{1},\ a_{2},\ .\ .\ .\ ,\ a_{n},\ b_{1},\ b_{2},\ .\ .\ .\ ,\ b_{n}\) sonlari uchun \(\left( a_{1}b_{1} + a_{2}b_{2} + \ .\ .\ .\  + a_{n}b_{n} \right)^{2} \leq \left( a_{1}^{2} + a_{2}^{2} + \ .\ .\ .\  + a_{n}^{2} \right)\left( b_{1}^{2} + b_{2}^{2} + \ .\ .\ .\  + b_{n}^{2} \right)\) \\
T2. Pifagor teoremasi va uning isbotlari. \\
A1. Tenglamani yeching \((x + 1)^{5} + (x - 1)^{5} = 32x\). \\
A2. Tenglamani yeching. \(\sqrt{3x^{2} - 2x + 15} + \sqrt{3x^{2} - 2x + 8} = 7\). \\
A3. Tenglamani yeching \(\left( x^{2} - 6x \right)^{2} - 2(x - 3)^{2} = 81\). \\
B1. \(P(2x - 1) + P(x - 1) = 10x^{2} - 12x + 2\) bo'lsa, \(P(x)\) ni toping. \\
B2. Quyidagi mulohazani ixtiyoriy natural son uchun matematik induksiya metodi yordamida isbotlang: \(\frac{1}{4 \cdot 5} + \frac{1}{5 \cdot 6} + \frac{1}{6 \cdot 7} + \ldots + \frac{1}{(n + 3) \cdot (n + 4)} = \frac{n}{4 \cdot (n + 4)}\). \\
B3. Quyidagi mulohazani ixtiyoriy natural son uchun matematik induksiya metodi yordamida isbotlang: \(7^{n} - 1\) soni 6 ga karrali; \\
C1. Ayniyatni isbotlang: \(\sum_{j = 0}^{n}C_{n}^{j} = 2^{n}\); \\
C2. Parallelogrammning tomonlari \(a\) va \(b\), ular orasidagi burchak \(\alpha\). bo'lsa, paralllelogramm ichki burchaklari bissektrisalari kesishishidan hosil bo'lgan to'rtburchak yuzini toping. \\
C3. Muntazam uchburchakning tomoni a ga teng. Tomonini diametr deb hisoblab doira yasalgan. Uchburchakning shu doiradan tashqaridagi qismi yuzini toping. \\

\end{tabular}
\vspace{1cm}


\begin{tabular}{m{17cm}}
\textbf{63-variant}
\newline

T1. Fales teoremasi va uning qo'llanilishi. \\
T2. Koshi tengsizligini isbotlang. \\
A1. Tenglamani yeching. \(\frac{z}{z + 1} - 2\sqrt{\frac{z + 1}{2}} = 3\). \\
A2. Tenglamani yeching. \(\sqrt[3]{x} + \sqrt[3]{x - 16} = \sqrt[3]{x - 8}\). \\
A3. Tenglamani yeching. \((x - 4)^{3} + (x - 4)^{2} + (x - 4)(x - 3) + (x - 3)^{2} + (x - 3)^{3} = 6\). \\
B1. \(P(x) = x^{33} - 2ax^{21} + x^{8} + 8\) ko'phadi berilgan. \(a\) ning qaysi qiymati uchun \(P(x)\) ko'phadi \(x + 1\) ga qoldiqsiz bo'liadi? \\
B2. Quyidagi mulohazani ixtiyoriy natural son uchun matematik induksiya metodi yordamida isbotlang: \(1^{3} + 2^{3} + 3^{3} + ... + n^{3} = \left( \frac{n(n + 1)}{2} \right)^{2}\); \\
B3. Quyidagi mulohazani ixtiyoriy natural son uchun matematik induksiya metodi yordamida isbotlang:\(6^{2n - 2} + 3^{n + 1} + 3^{n - 1}\) soni 11 karrali ; \\
C1. Ayniyatni isbotlang:\(C_{n + 2}^{j + 2} = C_{n}^{j} + 2C_{n}^{j + 1} + C_{n}^{j + 2}\); \\
C2. Ikkita bir xil radiusli doiralar shunday joylashganki, ularning markazlari orasidagi masofa radiusga teng. Doiralar kesishgan qismi yuzining kesishgan qismiga ichki chizilgan kvadrat yuziga nisbatini toping. \\
C3. Agar \(S\) uchburchakning yuzi, \(b\) va \(c\) uning tomonlari bo'lsa, \(S \leq \frac{b^{2} + c^{2}}{4}\) bo'lishini isbotlang. \\

\end{tabular}
\vspace{1cm}


\begin{tabular}{m{17cm}}
\textbf{64-variant}
\newline

T1. Ixtiyoriy \(a,b,c \in (0;1)\) sonlari uchun \(a(1 - b) > 1/4,\ b(1 - c) > 1/4,\ c(1 - a) > 1/4\) tengsizliklari bir vaqtda o'rinli bo'la olmasligini isbotlang. \\
T2. \(b\) parametrining qanday qiymatlarida \(x^{3} + 17x^{2} + bx - 17 = 0\) tenglamasining ildizlari butun sonlardan iborat bo'ladi? \\
A1. Tenglamani yeching \((x + 4)(x + 1) - 3\sqrt{x^{2} + 5x + 2} = 6\). \\
A2. Tenglamani yeching \(\left( x^{2} + 10x + 10 \right)\left( x^{2} + x + 10 \right) = 10x^{2}\) . \\
A3. Tenglamani yeching. \(\sqrt[3]{x - 1} + \sqrt[3]{x - 2} - \sqrt{2x - 3} = 0\). \\
B1. \(P(x) = (x - 5)^{2n + 1} + (x - 1)^{2n + 3}\) ko'phadni \(x - 3\) ga bo'lganda qoldiq \(3 \cdot 2^{3n - 4}\) bo'lsa, \(n\) ni toping. \\
B2. Quyidagi mulohazani ixtiyoriy natural son uchun matematik induksiya metodi yordamida isbotlang: \(1^{2} + 3^{2} + 5^{2} + ... + (2n - 1)^{2} = \frac{n\left( 4n^{2} - 1 \right)}{3}\); \\
B3. Quyidagi mulohazani ixtiyoriy natural son uchun matematik induksiya metodi yordamida isbotlang:\(6^{2n - 2} + 3^{n + 1} + 3^{n - 1}\) soni 11 karrali ; \\
C1. \(5C_{n}^{3} = C_{n + 2}^{4}\) bo'lsa, \(n\) ni toping. \\
C2. Teng yonli uchburchakning yon tomoni 13 sm , yon tomoniga o'tkazilgan balandlik 5 sm ga teng. Uchburchak asosining uzunligi topilsin. \\
C3. Agar \(S\) uchburchakning yuzi, \(b\) va \(c\) uning tomonlari bo'lsa, \(S \leq \frac{b^{2} + c^{2}}{4}\) bo'lishini isbotlang. \\

\end{tabular}
\vspace{1cm}


\begin{tabular}{m{17cm}}
\textbf{65-variant}
\newline

T1. Kombinatorika elementlari va Nyuton binomi. \\
T2. Ushbu \(P(x) = x^{5} + 11x^{4} + 37x^{3} + 35x^{2} - 44x - 40\) ko'phadi \(Q(x) = x^{2} + 3x + 2\) ko'phadiga qoldiqsiz bo'linadimi? \\
A1. Tenglamani yeching \(\left( x^{2} - 4x + 6 \right)^{2} - 4\left( x^{2} - 4x + 6 \right) + 6 = x\). \\
A2. Tenglamani yeching. \(\sqrt{x} + \frac{2x + 1}{x + 2} = 2\). \\
A3. Tengsizlikni yeching: \(\sqrt{x + 3} + \sqrt{x - 2} - \sqrt{2x + 4} > 0\). \\
B1. \(P(x + 3)\) ko'phadni \(x + 1\) ga bo'lganda qoldiq -3, \(Q(2x - 1)\) ko'phadni \(x - 1\)ga bo'lganda qoldiq 2 bo'lsa, \(P(x + 4) + x^{2}Q(x + 3)\) ko'phadni \(x + 2\) ga bo'lgandagi qoldiqni toping. \\
B2. Quyidagi mulohazani ixtiyoriy natural son uchun matematik induksiya metodi yordamida isbotlang: \(\frac{1}{1 \cdot 4} + \frac{1}{4 \cdot 7} + \frac{1}{7 \cdot 10} + \ldots + \frac{1}{(3n - 2) \cdot (3n + 1)} = \frac{n}{(3n + 1)}\). \\
B3. Quyidagi mulohazani ixtiyoriy natural son uchun matematik induksiya metodi yordamida isbotlang: \(7^{n} - 1\) soni 6 ga karrali; \\
C1. Teńlemeni sheshiń \(\frac{C_{2x}^{x + 1}}{C_{2x + 1}^{x - 1}} = \frac{2}{3}\), \(x \in N\) \\
C2. Teng yonli uchburchak asosidagi burchak \(\alpha\) ga teng. Shu burchak uchidan asosga \(\beta(\beta < \alpha)\) burchak ostida to'g'ri chiziq o'tkazilgan, u uchburchakni ikki qismga ajratadi. Hosil bo'lgan uchburchaklar yuzlarining nisbatini toping. \\
C3. Ya. Bermlli tengsizligi. Agar \(x \geq - 1\) bo'lsa, u holda ixtiyoriy natural \(n\) soni uchun \((1 + x)^{n} \geq 1 + nx\) tengsizlik o'rinli bo'lishini isbotlang. \\

\end{tabular}
\vspace{1cm}


\begin{tabular}{m{17cm}}
\textbf{66-variant}
\newline

T1. Bezu teoremasi va uning qo'llanilishi. \\
T2. \(n\) darajaning qanday qiymatlarida \((x + 1)^{n} + (x - 1)^{n}\) ifodasi \(x\) ifodaga qoldiqsiz bo'linadi? \\
A1. Tenglamani yeching \(\sqrt{\frac{18 - 7x - x^{2}}{8 - 6x + x^{2}}} + \sqrt{\frac{8 - 6x + x^{2}}{18 - 7x - x^{2}}} = \frac{13}{6}\). \\
A2. Tenglamani yeching. \((\sqrt{x + 1} + \sqrt{x})^{3} + (\sqrt{x + 1} + \sqrt{x})^{2} = 2\). \\
A3. Tenglamani yeching. \(\frac{4x}{x^{2} + x + 3} + \frac{5x}{x^{2} - 5x + 3} = - \frac{3}{2}\). \\
B1. \(P(x + n) = (x + n)^{3} + (x - n)^{2} + x + n + 6\) ko'phadi berilgan. \(P(x)\) ko'phadi \(x - n\) ga qoldiqsiz bo'linsa, \(n\) ni toping. \\
B2. Quyidagi mulohazani ixtiyoriy natural son uchun matematik induksiya metodi yordamida isbotlang: \(2^{2} + 6^{2} + \ldots + (4n - 2)^{2} = \frac{4n(2n - 1)(2n + 1)}{3}\). \\
B3. Quyidagi mulohazani ixtiyoriy natural son uchun matematik induksiya metodi yordamida isbotlang: \(5^{n} - 4n + 15\) soni 16 ga karrali ; \\
C1. \((a + b)^{n}\) ifoda yoyilmasining barcha koeffitsiyentlari yig`indisi 4096 ga teng bo'lsa, uning eng katta koeffitsiyentin toping. \\
C2. Uchburchakning asosiga tushirilgan balandligi \(h\) ga teng. Uchburchakning asosiga parallel kesma uchburchakning yuzini teng ikkiga bo'ladi. Uchburchakning uchidan shu kesmagacha bo'lgan masofa topilsin. \\
C3. Muntazam uchburchakning tomoni a ga teng. Tomonini diametr deb hisoblab doira yasalgan. Uchburchakning shu doiradan tashqaridagi qismi yuzini toping. \\

\end{tabular}
\vspace{1cm}


\begin{tabular}{m{17cm}}
\textbf{67-variant}
\newline

T1. \(2^{81} + 1\) soni 9 soniga qoldiqsiz bo'linishini isbotlang. \\
T2. Ixtiyoriy \(a\) parametri va \(x\) uchun \(x(a - x) \leq a^{2}/4\) tengsizligi o'rinli bo'lishini isbotlang. \\
A1. Tengsizlikni yeching: \(\sqrt{x^{2} - 4x} > x - 3\). \\
A2. Tengsizlikni yeching:\(x^{2}\left( x^{4} + 36 \right) - 6\sqrt{3}\left( x^{4} + 4 \right) < 0\). \\
A3. Tenglamani yeching. \(\sqrt{3x^{2} - 2x + 15} + \sqrt{3x^{2} - 2x + 8} = 7\). \\
B1. \(P(x) = x^{4} - 2x + 2^{n + 1}\) ko'phadni \(x - 2^{n}\) ga bo'lganda qoldiq \(2^{n - 2}\) bo'lsa, \(n\) ni toping. \\
B2. Quyidagi mulohazani ixtiyoriy natural son uchun matematik induksiya metodi yordamida isbotlang: \(\frac{1}{1 \cdot 5} + \frac{1}{5 \cdot 9} + ... + \frac{1}{(4n - 3)(4n + 1)} = \frac{n}{4n + 1}\); \\
B3. Quyidagi mulohazani ixtiyoriy natural son uchun matematik induksiya metodi yordamida isbotlang: \(5 \cdot 2^{3n - 2} + 3^{3n - 1}\) soni 19 ga karrali \\
C1. \(\left( x^{3} - \frac{3}{x^{2}} \right)^{10}\) binom yoyilmasining \(x\) qatnashmagan hadin toping. \\
C2. \emph{ABCD} parallelogrammning \emph{AD} tomoni \(n\) ta teng bo'lakka bo'lingan. Birinchi bo'linish nuqtasi \(P\) va \(B\) uch bilan birlashtirilgan. \emph{BP} to'g'ri chiziq \emph{AC} dioganaldan uning \(\frac{1}{n + 1}\) qismiga teng \emph{AQ} kesma ajratishini isbotlang. \\
C3. Ya. Bermlli tengsizligi. Agar \(x \geq - 1\) bo'lsa, u holda ixtiyoriy natural \(n\) soni uchun \((1 + x)^{n} \geq 1 + nx\) tengsizlik o'rinli bo'lishini isbotlang. \\

\end{tabular}
\vspace{1cm}


\begin{tabular}{m{17cm}}
\textbf{68-variant}
\newline

T1. Yig'indisi birga teng bo'lgan \(x,y,z\) musbat sonlari uchun \(\frac{1}{x} + \frac{1}{y} + \frac{1}{z} \geq 9\) tengsizligi o'rinli bo'lishini isbotlang. \\
T2. \(x\) o'zgaruvchining ixtiyoriy butun qiymatida \(ax^{2} + bx + c\) uchhadining qiymati butun bo'lishi uchun \(2a,\ a + b\) va \(c\) sonlarining butun bo'lishi zarur va yetarli ekanligini isbotlang. \\
A1. Tengsizlikni yeching: \(\sqrt{x^{2} - 4x} > x - 3\). \\
A2. Tengsizlikni yeching:\(x^{2}\left( x^{4} + 36 \right) - 6\sqrt{3}\left( x^{4} + 4 \right) < 0\). \\
A3. Tenglamani yeching. \((\sqrt{x + 1} + \sqrt{x})^{3} + (\sqrt{x + 1} + \sqrt{x})^{2} = 2\). \\
B1. \(P(x + 3) = x^{2} - x + n\) bo'lsa. \(P(x - 2)\) ko'phadni \(x - 3\) ga bo'lganda qoldiq \(10\) bo'lsa, \(n\) ni toping. \\
B2. Quyidagi mulohazani ixtiyoriy natural son uchun matematik induksiya metodi yordamida isbotlang: \(\frac{1}{1 \cdot 4} + \frac{1}{4 \cdot 7} + \frac{1}{7 \cdot 10} + \ldots + \frac{1}{(3n - 2) \cdot (3n + 1)} = \frac{n}{(3n + 1)}\). \\
B3. Quyidagi mulohazani ixtiyoriy natural son uchun matematik induksiya metodi yordamida isbotlang: \(n\left( 2n^{2} - 3n + 1 \right)\) soni 6 ga karrali ; \\
C1. \(x(1 - x)^{4} + x^{2}(1 + 2x)^{8} + x^{3}(1 + 3x)^{12}\) ifodada \(x^{4}\) oldidagi koeffitsiyentni toping. \\
C2. Agar teng yonli uchburchakning perimetri 32 dm , o'rta chizig'i 6 dm ga teng bo'lsa, uning tomonlari uzunliklari topilsin. \\
C3. Agar \(S\) uchburchakning yuzi, \(b\) va \(c\) uning tomonlari bo'lsa, \(S \leq \frac{b^{2} + c^{2}}{4}\) bo'lishini isbotlang. \\

\end{tabular}
\vspace{1cm}


\begin{tabular}{m{17cm}}
\textbf{69-variant}
\newline

T1. \(P(x) = (x - 1)^{20}\left( x^{2} + 25 \right)\) ko'phadining koeffitsentlari yig'indisini toping. \\
T2. Simmetrik ko'phadlar. \\
A1. Tenglamani yeching. \((x - 4)^{3} + (x - 4)^{2} + (x - 4)(x - 3) + (x - 3)^{2} + (x - 3)^{3} = 6\). \\
A2. Tenglamani yeching \(\left( x^{2} - 6x \right)^{2} - 2(x - 3)^{2} = 81\). \\
A3. Tenglamani yeching. \(\frac{z}{z + 1} - 2\sqrt{\frac{z + 1}{2}} = 3\). \\
B1. \(P(x + 1) + P(x - 3) = 2x^{2} - 10x + 16\) bo'lsa, \(P(x)\) ni toping. \\
B2. Quyidagi mulohazani ixtiyoriy natural son uchun matematik induksiya metodi yordamida isbotlang: \(\left( 1 - \frac{1}{4} \right)\left( 1 - \frac{1}{9} \right)...\left( 1 - \frac{1}{n^{2}} \right) = \frac{n + 1}{2n}\), \(n \geq 2\) \\
B3. Quyidagi mulohazani ixtiyoriy natural son uchun matematik induksiya metodi yordamida isbotlang: \(5^{n + 2} + 26 \cdot 5^{n} + 8^{2n + 1}\) soni 59 ga karrali; \\
C1. \((x + 1)^{3} + (x + 1)^{4} + (x + 1)^{5} + ... + (x + 1)^{10}\) ifodada \(x^{3}\) oldidagi koeffitsiyentni toping \\
C2. To'g'ri burchakli uchburchakning to'g'ri burchagi bissektrisasi shu uchdan o'tkazilgan mediana va balandlik orasidagi burchakni ham teng ikkiga bo'lishini isbotlang. \\
C3. Muntazam uchburchakning tomoni a ga teng. Tomonini diametr deb hisoblab doira yasalgan. Uchburchakning shu doiradan tashqaridagi qismi yuzini toping. \\

\end{tabular}
\vspace{1cm}


\begin{tabular}{m{17cm}}
\textbf{70-variant}
\newline

T1. Ushbu \(P(0) = 20\) va \(P(1) = 100\) shartlarini qanoatlantiruvchi \(P(x)\) ko'phadi mavjudmi? \\
T2. Matematik induksiya metodi va uning qo'llanilishiga misollar. \\
A1. Tenglamani yeching. \(\frac{4x}{x^{2} + x + 3} + \frac{5x}{x^{2} - 5x + 3} = - \frac{3}{2}\). \\
A2. Tenglamani yeching \(\sqrt{\frac{18 - 7x - x^{2}}{8 - 6x + x^{2}}} + \sqrt{\frac{8 - 6x + x^{2}}{18 - 7x - x^{2}}} = \frac{13}{6}\). \\
A3. Tenglamani yeching. \(\sqrt{x + 8 + 2\sqrt{x + 7}} + \sqrt{x + 1 - \sqrt{x + 7}} = 4\). \\
B1. \(P(x + 2) + P(x - 1) = - 2x^{2} - 2x + 7\) bo'lsa, \(P(x)\) ni \(x + 4\) ga bo'lgandagi qoldiqni toping. \\
B2. Quyidagi mulohazani ixtiyoriy natural son uchun matematik induksiya metodi yordamida isbotlang: \(1^{2} + 2^{2} + 3^{2} + ... + n^{2} = \frac{n(n + 1)(2n + 1)}{6}\); \\
B3. Quyidagi mulohazani ixtiyoriy natural son uchun matematik induksiya metodi yordamida isbotlang: \(5^{2n + 1} + 3^{n + 2} \cdot 2^{n - 1}\) soni 19 ga karrali ; \\
C1. \(C_{n + 4}^{n + 1} - C_{n + 3}^{n} = 15(n + 2)\) bo'lsa, \(n\) ni toping. \\
C2. Uchburchakning ishida olingan nuqtadan uning tomonlariga parallel to'g'ri chiziqlar o'tkazilgan. Ular uchburchakni 6 qismga bo'ladi. Agar hosil bo'lgan uchburchaklarning yuzlari \(S_{1},S_{2}\) va \(S_{3}\) bo'lsa, berilgan uchburchak yuzini toping. \\
C3. Muntazam uchburchakning tomoni a ga teng. Tomonini diametr deb hisoblab doira yasalgan. Uchburchakning shu doiradan tashqaridagi qismi yuzini toping. \\

\end{tabular}
\vspace{1cm}


\begin{tabular}{m{17cm}}
\textbf{71-variant}
\newline

T1. Matematik induksiya metodi va uning qo'llanilishiga misollar. \\
T2. Fales teoremasi va uning qo'llanilishi. \\
A1. Tengsizlikni yeching: \(\frac{x^{3} + 3x^{2} - x - 3}{x^{2} + 3x - 10} < 0\). \\
A2. Tenglamani yeching \(\left( x^{2} + 10x + 10 \right)\left( x^{2} + x + 10 \right) = 10x^{2}\) . \\
A3. Tenglamani yeching \((x + 4)(x + 1) - 3\sqrt{x^{2} + 5x + 2} = 6\). \\
B1. \(P(x)\) ko'phadni \(3x^{2} - 4x + 1\) ga bo'lganimizda qoldiq \(6x - 11\) bo'lsa, \(P(x)\) ko'phadni \(3x - 1\)ga bo'lganda qoldiqni toping. \\
B2. Quyidagi mulohazani ixtiyoriy natural son uchun matematik induksiya metodi yordamida isbotlang: \(1^{2} + 3^{2} + 5^{2} + ... + (2n - 1)^{2} = \frac{n\left( 4n^{2} - 1 \right)}{3}\); \\
B3. Quyidagi mulohazani ixtiyoriy natural son uchun matematik induksiya metodi yordamida isbotlang: \(2n^{3} + 3n^{2} + 7n\) soni 6 ga karrali ; \\
C1. Ayniyatni isbotlang: \(C_{n + k}^{j + k} = \sum_{s = 0}^{k}C_{n}^{j + s}C_{k}^{s}\); \\
C2. Muntazam uchburchakning uchlari uchta parallel to'g'ri chiziqlarda yotadi. Agar o'rtadagi to'g'ri chiziqdan chekkalardagi to'g'ri chiziklargacha bo'lgan masofa \(a\) va \(b\) ga teng bo'lsa, uchburchakning tomonini toping. \\
C3. Ya. Bermlli tengsizligi. Agar \(x \geq - 1\) bo'lsa, u holda ixtiyoriy natural \(n\) soni uchun \((1 + x)^{n} \geq 1 + nx\) tengsizlik o'rinli bo'lishini isbotlang. \\

\end{tabular}
\vspace{1cm}


\begin{tabular}{m{17cm}}
\textbf{72-variant}
\newline

T1. \(P(x) = x^{6} - 3x^{5} + x^{4} - 6x^{2} + 2x - 6\) ko'phadining butun ildizlarini toping. \\
T2. Koshi tengsizligini isbotlang. \\
A1. Tenglamani yeching. \(\sqrt[3]{x} + \sqrt[3]{x - 16} = \sqrt[3]{x - 8}\). \\
A2. Tengsizlikni yeching: \(\sqrt{x + 3} + \sqrt{x - 2} - \sqrt{2x + 4} > 0\). \\
A3. Tenglamani yeching. \(\sqrt[3]{x - 1} + \sqrt[3]{x - 2} - \sqrt{2x - 3} = 0\). \\
B1. \(P(x) = x^{33} - 2ax^{21} + x^{8} + 8\) ko'phadi berilgan. \(a\) ning qaysi qiymati uchun \(P(x)\) ko'phadi \(x + 1\) ga qoldiqsiz bo'liadi? \\
B2. Quyidagi mulohazani ixtiyoriy natural son uchun matematik induksiya metodi yordamida isbotlang: \(1 \cdot 1! + 2 \cdot 2! + 3 \cdot 3! + \ldots + n \cdot n! = (n + 1)! - 1\). \\
B3. Quyidagi mulohazani ixtiyoriy natural son uchun matematik induksiya metodi yordamida isbotlang: \(n^{3} + (n + 1)^{3} + (n + 2)^{3}\) soni 9 ga karrali ; \\
C1. \(\frac{1}{C_{4}^{n}} = \frac{1}{C_{5}^{n}} + \frac{1}{C_{6}^{n}}\) bo'lsa, \(n\) ni toping \\
C2. Bir burchagi \(60^{{^\circ}}\) bo'lgan uchburchakka ichki chizilgan aylananing urinish nuqtasi shu burchakka qarama- qarshi tomonini \(a\) va \(b\) kesmalarga ajratadi. Uchburchak yuzini toping. \\
C3. Agar \(S\) uchburchakning yuzi, \(b\) va \(c\) uning tomonlari bo'lsa, \(S \leq \frac{b^{2} + c^{2}}{4}\) bo'lishini isbotlang. \\

\end{tabular}
\vspace{1cm}


\begin{tabular}{m{17cm}}
\textbf{73-variant}
\newline

T1. \(n\) darajaning qanday qiymatlarida \((x + 1)^{n} + (x - 1)^{n}\) ifodasi \(x\) ifodaga qoldiqsiz bo'linadi? \\
T2. Ixtiyoriy \(a\) parametri va \(x\) uchun \(x(a - x) \leq a^{2}/4\) tengsizligi o'rinli bo'lishini isbotlang. \\
A1. Tenglamani yeching. \(\sqrt{x^{2} + x + 4} + \sqrt{x^{2} + x + 1} = \sqrt{2x^{2} + 2x + 9}\). \\
A2. Tenglamani yeching. \(\sqrt{x} + \frac{2x + 1}{x + 2} = 2\). \\
A3. Tenglamani yeching \(\left( x^{2} - 4x + 6 \right)^{2} - 4\left( x^{2} - 4x + 6 \right) + 6 = x\). \\
B1. \(P(x + 3) = x^{2} - x + n\) bo'lsa. \(P(x - 2)\) ko'phadni \(x - 3\) ga bo'lganda qoldiq \(10\) bo'lsa, \(n\) ni toping. \\
B2. Quyidagi mulohazani ixtiyoriy natural son uchun matematik induksiya metodi yordamida isbotlang: \(2^{2} + 6^{2} + \ldots + (4n - 2)^{2} = \frac{4n(2n - 1)(2n + 1)}{3}\). \\
B3. Quyidagi mulohazani ixtiyoriy natural son uchun matematik induksiya metodi yordamida isbotlang: \(5 \cdot 2^{3n - 2} + 3^{3n - 1}\) soni 19 ga karrali \\
C1. Teńsizlikti sheshiń \(C_{13}^{x} < C_{13}^{x + 2}\), \(x \in N\) \\
C2. To'g'ri burchakli uchburchakda katetlar 7 sm va 24 sm ga teng. To'g'ri burchakning bissektrisasi o'tkazilgan. Bu bissektrisa gipotenuzani qanday uzunlikdagi kesmalarga ajratadi? \\
C3. Ya. Bermlli tengsizligi. Agar \(x \geq - 1\) bo'lsa, u holda ixtiyoriy natural \(n\) soni uchun \((1 + x)^{n} \geq 1 + nx\) tengsizlik o'rinli bo'lishini isbotlang. \\

\end{tabular}
\vspace{1cm}


\begin{tabular}{m{17cm}}
\textbf{74-variant}
\newline

T1. \(a\) parametrining qanday qiymatlarida \(P(x) = x^{2017} + ax - 5\) ko'phadi \((x + 1)\) ko'phadiga qoldiqsiz bo'linadi? \\
T2. Haqiqiy \(a_{1},\ a_{2},\ .\ .\ .\ ,\ a_{n},\ b_{1},\ b_{2},\ .\ .\ .\ ,\ b_{n}\) sonlari uchun \(\left( a_{1}b_{1} + a_{2}b_{2} + \ .\ .\ .\  + a_{n}b_{n} \right)^{2} \leq \left( a_{1}^{2} + a_{2}^{2} + \ .\ .\ .\  + a_{n}^{2} \right)\left( b_{1}^{2} + b_{2}^{2} + \ .\ .\ .\  + b_{n}^{2} \right)\) \\
A1. Tenglamani yeching \((x + 1)^{5} + (x - 1)^{5} = 32x\). \\
A2. Tenglamani yeching \(\left( x^{2} - 6x \right)^{2} - 2(x - 3)^{2} = 81\). \\
A3. Tenglamani yeching. \(\sqrt{x} + \frac{2x + 1}{x + 2} = 2\). \\
B1. \(P(x + 3)\) ko'phadni \(x + 1\) ga bo'lganda qoldiq -3, \(Q(2x - 1)\) ko'phadni \(x - 1\)ga bo'lganda qoldiq 2 bo'lsa, \(P(x + 4) + x^{2}Q(x + 3)\) ko'phadni \(x + 2\) ga bo'lgandagi qoldiqni toping. \\
B2. Quyidagi mulohazani ixtiyoriy natural son uchun matematik induksiya metodi yordamida isbotlang: \(1^{3} + 2^{3} + 3^{3} + ... + n^{3} = \left( \frac{n(n + 1)}{2} \right)^{2}\); \\
B3. Quyidagi mulohazani ixtiyoriy natural son uchun matematik induksiya metodi yordamida isbotlang:\(6^{2n - 2} + 3^{n + 1} + 3^{n - 1}\) soni 11 karrali ; \\
C1. Tengsizlikni yeching: \(C_{10}^{x - 1} > 2C_{10}^{x}\) \\
C2. \(R\) radiusli doiraga bitta umumiy uchga ega bo'lgan muntazam uchburchak va kvadrat ichki chizilgan. Ularning kesishgan qismining yuzini toping. \\
C3. Muntazam uchburchakning tomoni a ga teng. Tomonini diametr deb hisoblab doira yasalgan. Uchburchakning shu doiradan tashqaridagi qismi yuzini toping. \\

\end{tabular}
\vspace{1cm}


\begin{tabular}{m{17cm}}
\textbf{75-variant}
\newline

T1. Pifagor teoremasi va uning isbotlari. \\
T2. \(x\) o'zgaruvchining ixtiyoriy butun qiymatida \(ax^{2} + bx + c\) uchhadining qiymati butun bo'lishi uchun \(2a,\ a + b\) va \(c\) sonlarining butun bo'lishi zarur va yetarli ekanligini isbotlang. \\
A1. Tengsizlikni yeching: \(\sqrt{x + 3} + \sqrt{x - 2} - \sqrt{2x + 4} > 0\). \\
A2. Tengsizlikni yeching: \(\frac{x^{3} + 3x^{2} - x - 3}{x^{2} + 3x - 10} < 0\). \\
A3. Tenglamani yeching. \(\frac{z}{z + 1} - 2\sqrt{\frac{z + 1}{2}} = 3\). \\
B1. \(P(2x - 1) + P(x - 1) = 10x^{2} - 12x + 2\) bo'lsa, \(P(x)\) ni toping. \\
B2. Quyidagi mulohazani ixtiyoriy natural son uchun matematik induksiya metodi yordamida isbotlang: \(1 \cdot 2 + 2 \cdot 3 + 3 \cdot 4 + \ldots + n \cdot (n + 1) = \frac{n \cdot (n + 1) \cdot (n + 2)}{3}\). \\
B3. Quyidagi mulohazani ixtiyoriy natural son uchun matematik induksiya metodi yordamida isbotlang: \(5^{2n + 1} + 3^{n + 2} \cdot 2^{n - 1}\) soni 19 ga karrali ; \\
C1. \(\left( 2x^{\ ^{2}} - \frac{b}{2x^{3}} \right)^{10}\) binom yoyilmasining \(x\) qatnashmagan hadin toping. \\
C2. \emph{ABC} uchburchak berilgan. Uning medianalaridan \(\bigtriangleup A_{1}B_{1}C_{1}\) yasalgan. \(\bigtriangleup ABC\) va \(\bigtriangleup A_{1}B_{1}C_{1}\) yuzlarining nisbati topilsin. \\
C3. Agar \(S\) uchburchakning yuzi, \(b\) va \(c\) uning tomonlari bo'lsa, \(S \leq \frac{b^{2} + c^{2}}{4}\) bo'lishini isbotlang. \\

\end{tabular}
\vspace{1cm}


\begin{tabular}{m{17cm}}
\textbf{76-variant}
\newline

T1. Ushbu \(P(x) = x^{5} + 11x^{4} + 37x^{3} + 35x^{2} - 44x - 40\) ko'phadi \(Q(x) = x^{2} + 3x + 2\) ko'phadiga qoldiqsiz bo'linadimi? \\
T2. Simmetrik ko'phadlar. \\
A1. Tenglamani yeching \((x + 4)(x + 1) - 3\sqrt{x^{2} + 5x + 2} = 6\). \\
A2. Tengsizlikni yeching:\(x^{2}\left( x^{4} + 36 \right) - 6\sqrt{3}\left( x^{4} + 4 \right) < 0\). \\
A3. Tenglamani yeching \(\left( x^{2} - 4x + 6 \right)^{2} - 4\left( x^{2} - 4x + 6 \right) + 6 = x\). \\
B1. \(P(x) = x^{4} - 2x + 2^{n + 1}\) ko'phadni \(x - 2^{n}\) ga bo'lganda qoldiq \(2^{n - 2}\) bo'lsa, \(n\) ni toping. \\
B2. Quyidagi mulohazani ixtiyoriy natural son uchun matematik induksiya metodi yordamida isbotlang: \(1 \cdot 2 + 2 \cdot 3 + 3 \cdot 4 + ... + n(n + 1) = \frac{n(n + 1)(n + 2)}{3}\); \\
B3. Quyidagi mulohazani ixtiyoriy natural son uchun matematik induksiya metodi yordamida isbotlang: \(n\left( 2n^{2} - 3n + 1 \right)\) soni 6 ga karrali ; \\
C1. \(\left( \sqrt{x} + \frac{1}{\sqrt[3]{x^{2}}} \right)^{n}\) binom yoyilmasida 5-had koeffitsiyentining 3-had koeffitsiyentiga nisbati 7:2 ga teng. \(x\) ning darajasi 1 ga teng bo'lgan ahadin toping. \\
C2. Parallelogrammning tomonlari \(a\) va \(b\), ular orasidagi burchak \(\alpha\). bo'lsa, paralllelogramm ichki burchaklari bissektrisalari kesishishidan hosil bo'lgan to'rtburchak yuzini toping. \\
C3. Ya. Bermlli tengsizligi. Agar \(x \geq - 1\) bo'lsa, u holda ixtiyoriy natural \(n\) soni uchun \((1 + x)^{n} \geq 1 + nx\) tengsizlik o'rinli bo'lishini isbotlang. \\

\end{tabular}
\vspace{1cm}


\begin{tabular}{m{17cm}}
\textbf{77-variant}
\newline

T1. \(2^{81} + 1\) soni 9 soniga qoldiqsiz bo'linishini isbotlang. \\
T2. Bezu teoremasi va uning qo'llanilishi. \\
A1. Tenglamani yeching. \((x - 4)^{3} + (x - 4)^{2} + (x - 4)(x - 3) + (x - 3)^{2} + (x - 3)^{3} = 6\). \\
A2. Tenglamani yeching. \(\sqrt{x^{2} + x + 4} + \sqrt{x^{2} + x + 1} = \sqrt{2x^{2} + 2x + 9}\). \\
A3. Tenglamani yeching. \(\sqrt{x + 8 + 2\sqrt{x + 7}} + \sqrt{x + 1 - \sqrt{x + 7}} = 4\). \\
B1. \(P(x + 1) + P(x - 3) = 2x^{2} - 10x + 16\) bo'lsa, \(P(x)\) ni toping. \\
B2. Quyidagi mulohazani ixtiyoriy natural son uchun matematik induksiya metodi yordamida isbotlang: \(\frac{1}{4 \cdot 5} + \frac{1}{5 \cdot 6} + \frac{1}{6 \cdot 7} + \ldots + \frac{1}{(n + 3) \cdot (n + 4)} = \frac{n}{4 \cdot (n + 4)}\). \\
B3. Quyidagi mulohazani ixtiyoriy natural son uchun matematik induksiya metodi yordamida isbotlang: \(n^{3} + (n + 1)^{3} + (n + 2)^{3}\) soni 9 ga karrali ; \\
C1. Ayniyatni isbotlang: \(C_{n + 1}^{j + 1} = C_{n}^{j} + C_{n}^{j + 1}\); \\
C2. Teng yonli uchburchak asosidagi burchak \(\alpha\) ga teng. Shu burchak uchidan asosga \(\beta(\beta < \alpha)\) burchak ostida to'g'ri chiziq o'tkazilgan, u uchburchakni ikki qismga ajratadi. Hosil bo'lgan uchburchaklar yuzlarining nisbatini toping. \\
C3. Agar \(S\) uchburchakning yuzi, \(b\) va \(c\) uning tomonlari bo'lsa, \(S \leq \frac{b^{2} + c^{2}}{4}\) bo'lishini isbotlang. \\

\end{tabular}
\vspace{1cm}


\begin{tabular}{m{17cm}}
\textbf{78-variant}
\newline

T1. \(b\) parametrining qanday qiymatlarida \(x^{3} + 17x^{2} + bx - 17 = 0\) tenglamasining ildizlari butun sonlardan iborat bo'ladi? \\
T2. Kombinatorika elementlari va Nyuton binomi. \\
A1. Tenglamani yeching. \(\sqrt[3]{x} + \sqrt[3]{x - 16} = \sqrt[3]{x - 8}\). \\
A2. Tenglamani yeching \(\left( x^{2} + 10x + 10 \right)\left( x^{2} + x + 10 \right) = 10x^{2}\) . \\
A3. Tenglamani yeching. \((\sqrt{x + 1} + \sqrt{x})^{3} + (\sqrt{x + 1} + \sqrt{x})^{2} = 2\). \\
B1. \(P(x + n) = (x + n)^{3} + (x - n)^{2} + x + n + 6\) ko'phadi berilgan. \(P(x)\) ko'phadi \(x - n\) ga qoldiqsiz bo'linsa, \(n\) ni toping. \\
B2. Quyidagi mulohazani ixtiyoriy natural son uchun matematik induksiya metodi yordamida isbotlang: \(\frac{1}{1 \cdot 4} + \frac{1}{4 \cdot 7} + \frac{1}{7 \cdot 10} + \ldots + \frac{1}{(3n - 2) \cdot (3n + 1)} = \frac{n}{(3n + 1)}\). \\
B3. Quyidagi mulohazani ixtiyoriy natural son uchun matematik induksiya metodi yordamida isbotlang: \(5^{n} - 4n + 15\) soni 16 ga karrali ; \\
C1. Teńsizlikti sheshiń \(5C_{x}^{3} < C_{x + 2}^{4}\), \(x \in N\) \\
C2. Teng yonli \(ABC(AB = BC)\) uchburchakda \emph{AD} bissektrisa o'tkazilgan. Agar \(S_{ABD} = S_{1},S_{\bigtriangleup ADC} = S_{2}\) bo'lsa, \emph{AC} ni toping. \\
C3. Muntazam uchburchakning tomoni a ga teng. Tomonini diametr deb hisoblab doira yasalgan. Uchburchakning shu doiradan tashqaridagi qismi yuzini toping. \\

\end{tabular}
\vspace{1cm}


\begin{tabular}{m{17cm}}
\textbf{79-variant}
\newline

T1. Yig'indisi birga teng bo'lgan \(x,y,z\) musbat sonlari uchun \(\frac{1}{x} + \frac{1}{y} + \frac{1}{z} \geq 9\) tengsizligi o'rinli bo'lishini isbotlang. \\
T2. Ushbu \(P(0) = 20\) va \(P(1) = 100\) shartlarini qanoatlantiruvchi \(P(x)\) ko'phadi mavjudmi? \\
A1. Tenglamani yeching. \(\sqrt{3x^{2} - 2x + 15} + \sqrt{3x^{2} - 2x + 8} = 7\). \\
A2. Tenglamani yeching. \(\frac{4x}{x^{2} + x + 3} + \frac{5x}{x^{2} - 5x + 3} = - \frac{3}{2}\). \\
A3. Tenglamani yeching \((x + 1)^{5} + (x - 1)^{5} = 32x\). \\
B1. \(P(x) = (x - 5)^{2n + 1} + (x - 1)^{2n + 3}\) ko'phadni \(x - 3\) ga bo'lganda qoldiq \(3 \cdot 2^{3n - 4}\) bo'lsa, \(n\) ni toping. \\
B2. Quyidagi mulohazani ixtiyoriy natural son uchun matematik induksiya metodi yordamida isbotlang: \(1 \cdot 1! + 2 \cdot 2! + 3 \cdot 3! + \ldots + n \cdot n! = (n + 1)! - 1\). \\
B3. Quyidagi mulohazani ixtiyoriy natural son uchun matematik induksiya metodi yordamida isbotlang: \(2n^{3} + 3n^{2} + 7n\) soni 6 ga karrali ; \\
C1. \(\left( x\sqrt{x} - \frac{1}{x^{4}} \right)^{n}\) binom yoyilmasida 3-had koeffitsiyenti 2-had koeffitsiyentidan 44 ga katta.Ozod hadini toping. \\
C2. Markazlari \(O_{1}\) va \(O_{2}\) nuqtalarda va radiusi \(R\) bo'lgan ikkita teng aylanalar tashqi urinadi. \(l\) to'g'ri chiziq bu aylanalarni A, B, C va \(D\) nuqtalarda shunday kesib o'tadiki, \(AB = BC = CD\) bo'ladi. \(O_{1}ADO_{2}\) to'rtburchak yuzini toping. \\
C3. Muntazam uchburchakning tomoni a ga teng. Tomonini diametr deb hisoblab doira yasalgan. Uchburchakning shu doiradan tashqaridagi qismi yuzini toping. \\

\end{tabular}
\vspace{1cm}


\begin{tabular}{m{17cm}}
\textbf{80-variant}
\newline

T1. \(P(x) = (x - 1)^{20}\left( x^{2} + 25 \right)\) ko'phadining koeffitsentlari yig'indisini toping. \\
T2. Ixtiyoriy \(a,b,c \in (0;1)\) sonlari uchun \(a(1 - b) > 1/4,\ b(1 - c) > 1/4,\ c(1 - a) > 1/4\) tengsizliklari bir vaqtda o'rinli bo'la olmasligini isbotlang. \\
A1. Tenglamani yeching. \(\sqrt[3]{x - 1} + \sqrt[3]{x - 2} - \sqrt{2x - 3} = 0\). \\
A2. Tengsizlikni yeching: \(\sqrt{x^{2} - 4x} > x - 3\). \\
A3. Tenglamani yeching \(\sqrt{\frac{18 - 7x - x^{2}}{8 - 6x + x^{2}}} + \sqrt{\frac{8 - 6x + x^{2}}{18 - 7x - x^{2}}} = \frac{13}{6}\). \\
B1. \(P(x + 2) + P(x - 1) = - 2x^{2} - 2x + 7\) bo'lsa, \(P(x)\) ni \(x + 4\) ga bo'lgandagi qoldiqni toping. \\
B2. Quyidagi mulohazani ixtiyoriy natural son uchun matematik induksiya metodi yordamida isbotlang: \(\frac{1}{4 \cdot 5} + \frac{1}{5 \cdot 6} + \frac{1}{6 \cdot 7} + \ldots + \frac{1}{(n + 3) \cdot (n + 4)} = \frac{n}{4 \cdot (n + 4)}\). \\
B3. Quyidagi mulohazani ixtiyoriy natural son uchun matematik induksiya metodi yordamida isbotlang: \(5^{n + 2} + 26 \cdot 5^{n} + 8^{2n + 1}\) soni 59 ga karrali; \\
C1. Ayniyatni isbotlang:\(\sum_{j = 0}^{n}C_{n}^{j}( - 1)^{j} = 0\); \\
C2. Teng yonli uchburchakning yuzi \(S\) ga teng. Yon tomonlariga tushirilgan medianalari orasidagi burchak \(\alpha\) ga teng. Uchburchak asosini toping. \\
C3. Agar \(S\) uchburchakning yuzi, \(b\) va \(c\) uning tomonlari bo'lsa, \(S \leq \frac{b^{2} + c^{2}}{4}\) bo'lishini isbotlang. \\

\end{tabular}
\vspace{1cm}


\begin{tabular}{m{17cm}}
\textbf{81-variant}
\newline

T1. \(P(x) = x^{6} - 3x^{5} + x^{4} - 6x^{2} + 2x - 6\) ko'phadining butun ildizlarini toping. \\
T2. Ixtiyoriy \(a\) parametri va \(x\) uchun \(x(a - x) \leq a^{2}/4\) tengsizligi o'rinli bo'lishini isbotlang. \\
A1. Tenglamani yeching \((x + 4)(x + 1) - 3\sqrt{x^{2} + 5x + 2} = 6\). \\
A2. Tenglamani yeching \(\sqrt{\frac{18 - 7x - x^{2}}{8 - 6x + x^{2}}} + \sqrt{\frac{8 - 6x + x^{2}}{18 - 7x - x^{2}}} = \frac{13}{6}\). \\
A3. Tenglamani yeching \(\left( x^{2} - 4x + 6 \right)^{2} - 4\left( x^{2} - 4x + 6 \right) + 6 = x\). \\
B1. \(P(x + n) = (x + n)^{3} + (x - n)^{2} + x + n + 6\) ko'phadi berilgan. \(P(x)\) ko'phadi \(x - n\) ga qoldiqsiz bo'linsa, \(n\) ni toping. \\
B2. Quyidagi mulohazani ixtiyoriy natural son uchun matematik induksiya metodi yordamida isbotlang: \(1 \cdot 2 + 2 \cdot 3 + 3 \cdot 4 + \ldots + n \cdot (n + 1) = \frac{n \cdot (n + 1) \cdot (n + 2)}{3}\). \\
B3. Quyidagi mulohazani ixtiyoriy natural son uchun matematik induksiya metodi yordamida isbotlang: \(7^{n} - 1\) soni 6 ga karrali; \\
C1. \((a + b)^{n}\) ifoda yoyilmasining barcha koeffitsiyentlari yig`indisi 4096 ga teng bo'lsa, uning eng katta koeffitsiyentin toping. \\
C2. To'g'ri burchakli uchburchakning balandligi gipotenuzani uzunliklari 18 va 32 sm ga teng bo'lgan kesmalarga ajratadi. Uchburchakning yuzi hisoblansin. \\
C3. Ya. Bermlli tengsizligi. Agar \(x \geq - 1\) bo'lsa, u holda ixtiyoriy natural \(n\) soni uchun \((1 + x)^{n} \geq 1 + nx\) tengsizlik o'rinli bo'lishini isbotlang. \\

\end{tabular}
\vspace{1cm}


\begin{tabular}{m{17cm}}
\textbf{82-variant}
\newline

T1. Simmetrik ko'phadlar. \\
T2. \(2^{81} + 1\) soni 9 soniga qoldiqsiz bo'linishini isbotlang. \\
A1. Tenglamani yeching \(\left( x^{2} + 10x + 10 \right)\left( x^{2} + x + 10 \right) = 10x^{2}\) . \\
A2. Tenglamani yeching. \(\sqrt[3]{x} + \sqrt[3]{x - 16} = \sqrt[3]{x - 8}\). \\
A3. Tengsizlikni yeching: \(\frac{x^{3} + 3x^{2} - x - 3}{x^{2} + 3x - 10} < 0\). \\
B1. \(P(x)\) ko'phadni \(3x^{2} - 4x + 1\) ga bo'lganimizda qoldiq \(6x - 11\) bo'lsa, \(P(x)\) ko'phadni \(3x - 1\)ga bo'lganda qoldiqni toping. \\
B2. Quyidagi mulohazani ixtiyoriy natural son uchun matematik induksiya metodi yordamida isbotlang: \(1 \cdot 2 + 2 \cdot 3 + 3 \cdot 4 + ... + n(n + 1) = \frac{n(n + 1)(n + 2)}{3}\); \\
B3. Quyidagi mulohazani ixtiyoriy natural son uchun matematik induksiya metodi yordamida isbotlang: \(n^{3} + (n + 1)^{3} + (n + 2)^{3}\) soni 9 ga karrali ; \\
C1. Tengsizlikni yeching: \(C_{10}^{x - 1} > 2C_{10}^{x}\) \\
C2. \(ABCD(AD\| BC)\) trapetsiya diagonallari \(O\) nuqtada kesishadi. Agar \emph{AOD} uchburchakning yuzi \(a^{2}\) ga, \emph{BOC} uchburchakning yuzi \(b^{2}\) ga tengligi ma'lum bo'lsa, trapesiya yuzini toping. \\
C3. Agar \(S\) uchburchakning yuzi, \(b\) va \(c\) uning tomonlari bo'lsa, \(S \leq \frac{b^{2} + c^{2}}{4}\) bo'lishini isbotlang. \\

\end{tabular}
\vspace{1cm}


\begin{tabular}{m{17cm}}
\textbf{83-variant}
\newline

T1. Matematik induksiya metodi va uning qo'llanilishiga misollar. \\
T2. Yig'indisi birga teng bo'lgan \(x,y,z\) musbat sonlari uchun \(\frac{1}{x} + \frac{1}{y} + \frac{1}{z} \geq 9\) tengsizligi o'rinli bo'lishini isbotlang. \\
A1. Tengsizlikni yeching: \(\sqrt{x^{2} - 4x} > x - 3\). \\
A2. Tenglamani yeching. \((\sqrt{x + 1} + \sqrt{x})^{3} + (\sqrt{x + 1} + \sqrt{x})^{2} = 2\). \\
A3. Tenglamani yeching. \(\sqrt{3x^{2} - 2x + 15} + \sqrt{3x^{2} - 2x + 8} = 7\). \\
B1. \(P(x + 3) = x^{2} - x + n\) bo'lsa. \(P(x - 2)\) ko'phadni \(x - 3\) ga bo'lganda qoldiq \(10\) bo'lsa, \(n\) ni toping. \\
B2. Quyidagi mulohazani ixtiyoriy natural son uchun matematik induksiya metodi yordamida isbotlang: \(1^{2} + 3^{2} + 5^{2} + ... + (2n - 1)^{2} = \frac{n\left( 4n^{2} - 1 \right)}{3}\); \\
B3. Quyidagi mulohazani ixtiyoriy natural son uchun matematik induksiya metodi yordamida isbotlang: \(5^{2n + 1} + 3^{n + 2} \cdot 2^{n - 1}\) soni 19 ga karrali ; \\
C1. \(\frac{1}{C_{4}^{n}} = \frac{1}{C_{5}^{n}} + \frac{1}{C_{6}^{n}}\) bo'lsa, \(n\) ni toping \\
C2. To'g'ri burchakli uchburchakning balandligi gipotenuzani uzunliklari \emph{x} va \emph{y} ga teng bo'lgan kesmalarga ajratadi. Uchburchakning yuzi hisoblansin. \\
C3. Muntazam uchburchakning tomoni a ga teng. Tomonini diametr deb hisoblab doira yasalgan. Uchburchakning shu doiradan tashqaridagi qismi yuzini toping. \\

\end{tabular}
\vspace{1cm}


\begin{tabular}{m{17cm}}
\textbf{84-variant}
\newline

T1. Koshi tengsizligini isbotlang. \\
T2. \(n\) darajaning qanday qiymatlarida \((x + 1)^{n} + (x - 1)^{n}\) ifodasi \(x\) ifodaga qoldiqsiz bo'linadi? \\
A1. Tenglamani yeching. \((x - 4)^{3} + (x - 4)^{2} + (x - 4)(x - 3) + (x - 3)^{2} + (x - 3)^{3} = 6\). \\
A2. Tenglamani yeching. \(\sqrt[3]{x - 1} + \sqrt[3]{x - 2} - \sqrt{2x - 3} = 0\). \\
A3. Tenglamani yeching. \(\frac{z}{z + 1} - 2\sqrt{\frac{z + 1}{2}} = 3\). \\
B1. \(P(x + 1) + P(x - 3) = 2x^{2} - 10x + 16\) bo'lsa, \(P(x)\) ni toping. \\
B2. Quyidagi mulohazani ixtiyoriy natural son uchun matematik induksiya metodi yordamida isbotlang: \(2^{2} + 6^{2} + \ldots + (4n - 2)^{2} = \frac{4n(2n - 1)(2n + 1)}{3}\). \\
B3. Quyidagi mulohazani ixtiyoriy natural son uchun matematik induksiya metodi yordamida isbotlang: \(5 \cdot 2^{3n - 2} + 3^{3n - 1}\) soni 19 ga karrali \\
C1. \(C_{n + 4}^{n + 1} - C_{n + 3}^{n} = 15(n + 2)\) bo'lsa, \(n\) ni toping. \\
C2. Uchburchakning tomonlari \(a\) va \(b\), bissektrisasi \(l_{c} = l\). \(l\) ni bilgan holda uning yuzini toping. \\
C3. Ya. Bermlli tengsizligi. Agar \(x \geq - 1\) bo'lsa, u holda ixtiyoriy natural \(n\) soni uchun \((1 + x)^{n} \geq 1 + nx\) tengsizlik o'rinli bo'lishini isbotlang. \\

\end{tabular}
\vspace{1cm}


\begin{tabular}{m{17cm}}
\textbf{85-variant}
\newline

T1. \(a\) parametrining qanday qiymatlarida \(P(x) = x^{2017} + ax - 5\) ko'phadi \((x + 1)\) ko'phadiga qoldiqsiz bo'linadi? \\
T2. \(x\) o'zgaruvchining ixtiyoriy butun qiymatida \(ax^{2} + bx + c\) uchhadining qiymati butun bo'lishi uchun \(2a,\ a + b\) va \(c\) sonlarining butun bo'lishi zarur va yetarli ekanligini isbotlang. \\
A1. Tenglamani yeching \((x + 1)^{5} + (x - 1)^{5} = 32x\). \\
A2. Tengsizlikni yeching: \(\sqrt{x + 3} + \sqrt{x - 2} - \sqrt{2x + 4} > 0\). \\
A3. Tenglamani yeching. \(\sqrt{x + 8 + 2\sqrt{x + 7}} + \sqrt{x + 1 - \sqrt{x + 7}} = 4\). \\
B1. \(P(x + 3)\) ko'phadni \(x + 1\) ga bo'lganda qoldiq -3, \(Q(2x - 1)\) ko'phadni \(x - 1\)ga bo'lganda qoldiq 2 bo'lsa, \(P(x + 4) + x^{2}Q(x + 3)\) ko'phadni \(x + 2\) ga bo'lgandagi qoldiqni toping. \\
B2. Quyidagi mulohazani ixtiyoriy natural son uchun matematik induksiya metodi yordamida isbotlang: \(\frac{1}{1 \cdot 5} + \frac{1}{5 \cdot 9} + ... + \frac{1}{(4n - 3)(4n + 1)} = \frac{n}{4n + 1}\); \\
B3. Quyidagi mulohazani ixtiyoriy natural son uchun matematik induksiya metodi yordamida isbotlang:\(6^{2n - 2} + 3^{n + 1} + 3^{n - 1}\) soni 11 karrali ; \\
C1. Ayniyatni isbotlang:\(C_{n + 2}^{j + 2} = C_{n}^{j} + 2C_{n}^{j + 1} + C_{n}^{j + 2}\); \\
C2. Asoslari \(x\) va 3 bo'lgan trapetsiyada diagonallar o'rtalari orasidagi masofani \(x\) ning funksiyasi sifatida ifodalang. \(x\) nіnng qanday qiymatida bu masofa 1 ga teng bo'ladi? \\
C3. Muntazam uchburchakning tomoni a ga teng. Tomonini diametr deb hisoblab doira yasalgan. Uchburchakning shu doiradan tashqaridagi qismi yuzini toping. \\

\end{tabular}
\vspace{1cm}


\begin{tabular}{m{17cm}}
\textbf{86-variant}
\newline

T1. Ixtiyoriy \(a,b,c \in (0;1)\) sonlari uchun \(a(1 - b) > 1/4,\ b(1 - c) > 1/4,\ c(1 - a) > 1/4\) tengsizliklari bir vaqtda o'rinli bo'la olmasligini isbotlang. \\
T2. \(b\) parametrining qanday qiymatlarida \(x^{3} + 17x^{2} + bx - 17 = 0\) tenglamasining ildizlari butun sonlardan iborat bo'ladi? \\
A1. Tenglamani yeching. \(\frac{4x}{x^{2} + x + 3} + \frac{5x}{x^{2} - 5x + 3} = - \frac{3}{2}\). \\
A2. Tenglamani yeching. \(\sqrt{x} + \frac{2x + 1}{x + 2} = 2\). \\
A3. Tenglamani yeching. \(\sqrt{x^{2} + x + 4} + \sqrt{x^{2} + x + 1} = \sqrt{2x^{2} + 2x + 9}\). \\
B1. \(P(x) = x^{4} - 2x + 2^{n + 1}\) ko'phadni \(x - 2^{n}\) ga bo'lganda qoldiq \(2^{n - 2}\) bo'lsa, \(n\) ni toping. \\
B2. Quyidagi mulohazani ixtiyoriy natural son uchun matematik induksiya metodi yordamida isbotlang: \(1^{2} + 2^{2} + 3^{2} + ... + n^{2} = \frac{n(n + 1)(2n + 1)}{6}\); \\
B3. Quyidagi mulohazani ixtiyoriy natural son uchun matematik induksiya metodi yordamida isbotlang: \(5^{n + 2} + 26 \cdot 5^{n} + 8^{2n + 1}\) soni 59 ga karrali; \\
C1. Ayniyatni isbotlang: \(\sum_{j = 0}^{n}C_{n}^{j} = 2^{n}\); \\
C2. \emph{ABC} uchburchakning \(B\) uchidan \emph{AC} tomoniga \emph{BD} kesma o'tkazildi. \emph{BD} kesma bu uchburchakning yuzini teng ikkiga bulladi. Agar \(AC = a\) bo'lsa, \emph{AD} va \emph{DC} kesmalarning uzunliklarini toping. \\
C3. Agar \(S\) uchburchakning yuzi, \(b\) va \(c\) uning tomonlari bo'lsa, \(S \leq \frac{b^{2} + c^{2}}{4}\) bo'lishini isbotlang. \\

\end{tabular}
\vspace{1cm}


\begin{tabular}{m{17cm}}
\textbf{87-variant}
\newline

T1. Fales teoremasi va uning qo'llanilishi. \\
T2. Ushbu \(P(x) = x^{5} + 11x^{4} + 37x^{3} + 35x^{2} - 44x - 40\) ko'phadi \(Q(x) = x^{2} + 3x + 2\) ko'phadiga qoldiqsiz bo'linadimi? \\
A1. Tengsizlikni yeching:\(x^{2}\left( x^{4} + 36 \right) - 6\sqrt{3}\left( x^{4} + 4 \right) < 0\). \\
A2. Tenglamani yeching \(\left( x^{2} - 6x \right)^{2} - 2(x - 3)^{2} = 81\). \\
A3. Tenglamani yeching. \(\sqrt{x} + \frac{2x + 1}{x + 2} = 2\). \\
B1. \(P(x) = x^{33} - 2ax^{21} + x^{8} + 8\) ko'phadi berilgan. \(a\) ning qaysi qiymati uchun \(P(x)\) ko'phadi \(x + 1\) ga qoldiqsiz bo'liadi? \\
B2. Quyidagi mulohazani ixtiyoriy natural son uchun matematik induksiya metodi yordamida isbotlang: \(\left( 1 - \frac{1}{4} \right)\left( 1 - \frac{1}{9} \right)...\left( 1 - \frac{1}{n^{2}} \right) = \frac{n + 1}{2n}\), \(n \geq 2\) \\
B3. Quyidagi mulohazani ixtiyoriy natural son uchun matematik induksiya metodi yordamida isbotlang: \(2n^{3} + 3n^{2} + 7n\) soni 6 ga karrali ; \\
C1. \(\left( x\sqrt{x} - \frac{1}{x^{4}} \right)^{n}\) binom yoyilmasida 3-had koeffitsiyenti 2-had koeffitsiyentidan 44 ga katta.Ozod hadini toping. \\
C2. To'g'ri burchakli uchburchakning katetlari \(b\) va \(c\) ga teng. To'g'ri burchak bissektrisasining uzunligi topilsin. \\
C3. Ya. Bermlli tengsizligi. Agar \(x \geq - 1\) bo'lsa, u holda ixtiyoriy natural \(n\) soni uchun \((1 + x)^{n} \geq 1 + nx\) tengsizlik o'rinli bo'lishini isbotlang. \\

\end{tabular}
\vspace{1cm}


\begin{tabular}{m{17cm}}
\textbf{88-variant}
\newline

T1. Ushbu \(P(0) = 20\) va \(P(1) = 100\) shartlarini qanoatlantiruvchi \(P(x)\) ko'phadi mavjudmi? \\
T2. \(P(x) = (x - 1)^{20}\left( x^{2} + 25 \right)\) ko'phadining koeffitsentlari yig'indisini toping. \\
A1. Tenglamani yeching \(\left( x^{2} + 10x + 10 \right)\left( x^{2} + x + 10 \right) = 10x^{2}\) . \\
A2. Tenglamani yeching. \(\sqrt[3]{x - 1} + \sqrt[3]{x - 2} - \sqrt{2x - 3} = 0\). \\
A3. Tenglamani yeching. \(\sqrt{x + 8 + 2\sqrt{x + 7}} + \sqrt{x + 1 - \sqrt{x + 7}} = 4\). \\
B1. \(P(2x - 1) + P(x - 1) = 10x^{2} - 12x + 2\) bo'lsa, \(P(x)\) ni toping. \\
B2. Quyidagi mulohazani ixtiyoriy natural son uchun matematik induksiya metodi yordamida isbotlang: \(1^{3} + 2^{3} + 3^{3} + ... + n^{3} = \left( \frac{n(n + 1)}{2} \right)^{2}\); \\
B3. Quyidagi mulohazani ixtiyoriy natural son uchun matematik induksiya metodi yordamida isbotlang: \(5^{n} - 4n + 15\) soni 16 ga karrali ; \\
C1. Ayniyatni isbotlang:\(C_{n}^{j} = C_{n}^{n - j}\); \\
C2. To'g'ri burchakli uchburchak o'tkir burchaklarining bisєektrisalari AD va BK \(AB^{2} = AD \cdot BK\) bo'lsa, uchburchakning burchaklarini toping. \\
C3. Muntazam uchburchakning tomoni a ga teng. Tomonini diametr deb hisoblab doira yasalgan. Uchburchakning shu doiradan tashqaridagi qismi yuzini toping. \\

\end{tabular}
\vspace{1cm}


\begin{tabular}{m{17cm}}
\textbf{89-variant}
\newline

T1. Bezu teoremasi va uning qo'llanilishi. \\
T2. Kombinatorika elementlari va Nyuton binomi. \\
A1. Tenglamani yeching. \(\frac{z}{z + 1} - 2\sqrt{\frac{z + 1}{2}} = 3\). \\
A2. Tengsizlikni yeching: \(\sqrt{x + 3} + \sqrt{x - 2} - \sqrt{2x + 4} > 0\). \\
A3. Tenglamani yeching \((x + 4)(x + 1) - 3\sqrt{x^{2} + 5x + 2} = 6\). \\
B1. \(P(x) = (x - 5)^{2n + 1} + (x - 1)^{2n + 3}\) ko'phadni \(x - 3\) ga bo'lganda qoldiq \(3 \cdot 2^{3n - 4}\) bo'lsa, \(n\) ni toping. \\
B2. Quyidagi mulohazani ixtiyoriy natural son uchun matematik induksiya metodi yordamida isbotlang: \(1^{2} + 2^{2} + 3^{2} + ... + n^{2} = \frac{n(n + 1)(2n + 1)}{6}\); \\
B3. Quyidagi mulohazani ixtiyoriy natural son uchun matematik induksiya metodi yordamida isbotlang: \(7^{n} - 1\) soni 6 ga karrali; \\
C1. \((x + 1)^{3} + (x + 1)^{4} + (x + 1)^{5} + ... + (x + 1)^{10}\) ifodada \(x^{3}\) oldidagi koeffitsiyentni toping \\
C2. To'g'ri burchakli uchburchakda katetlarning nisbati 3:2 kabi, balandlik esa gipotenuzani shunday ikkita kesmaga ajratadiki, ulardan birining uzunligi ikkinchisidan 2 ga katta. Gipotenuzaning uzunligi topilsin. \\
C3. Agar \(S\) uchburchakning yuzi, \(b\) va \(c\) uning tomonlari bo'lsa, \(S \leq \frac{b^{2} + c^{2}}{4}\) bo'lishini isbotlang. \\

\end{tabular}
\vspace{1cm}


\begin{tabular}{m{17cm}}
\textbf{90-variant}
\newline

T1. Haqiqiy \(a_{1},\ a_{2},\ .\ .\ .\ ,\ a_{n},\ b_{1},\ b_{2},\ .\ .\ .\ ,\ b_{n}\) sonlari uchun \(\left( a_{1}b_{1} + a_{2}b_{2} + \ .\ .\ .\  + a_{n}b_{n} \right)^{2} \leq \left( a_{1}^{2} + a_{2}^{2} + \ .\ .\ .\  + a_{n}^{2} \right)\left( b_{1}^{2} + b_{2}^{2} + \ .\ .\ .\  + b_{n}^{2} \right)\) \\
T2. Pifagor teoremasi va uning isbotlari. \\
A1. Tengsizlikni yeching: \(\sqrt{x^{2} - 4x} > x - 3\). \\
A2. Tenglamani yeching. \((\sqrt{x + 1} + \sqrt{x})^{3} + (\sqrt{x + 1} + \sqrt{x})^{2} = 2\). \\
A3. Tenglamani yeching \(\left( x^{2} - 4x + 6 \right)^{2} - 4\left( x^{2} - 4x + 6 \right) + 6 = x\). \\
B1. \(P(x + 2) + P(x - 1) = - 2x^{2} - 2x + 7\) bo'lsa, \(P(x)\) ni \(x + 4\) ga bo'lgandagi qoldiqni toping. \\
B2. Quyidagi mulohazani ixtiyoriy natural son uchun matematik induksiya metodi yordamida isbotlang: \(\frac{1}{1 \cdot 5} + \frac{1}{5 \cdot 9} + ... + \frac{1}{(4n - 3)(4n + 1)} = \frac{n}{4n + 1}\); \\
B3. Quyidagi mulohazani ixtiyoriy natural son uchun matematik induksiya metodi yordamida isbotlang: \(n\left( 2n^{2} - 3n + 1 \right)\) soni 6 ga karrali ; \\
C1. \(\left( x^{3} - \frac{3}{x^{2}} \right)^{10}\) binom yoyilmasining \(x\) qatnashmagan hadin toping. \\
C2. Uchburchakning asosiga tushirilgan balandligi \(h\) ga teng. Uchburchakning asosiga parallel kesma uchburchakning yuzini teng ikkiga bo'ladi. Uchburchakning uchidan shu kesmagacha bo'lgan masofa topilsin. \\
C3. Ya. Bermlli tengsizligi. Agar \(x \geq - 1\) bo'lsa, u holda ixtiyoriy natural \(n\) soni uchun \((1 + x)^{n} \geq 1 + nx\) tengsizlik o'rinli bo'lishini isbotlang. \\

\end{tabular}
\vspace{1cm}


\begin{tabular}{m{17cm}}
\textbf{91-variant}
\newline

T1. Ixtiyoriy \(a\) parametri va \(x\) uchun \(x(a - x) \leq a^{2}/4\) tengsizligi o'rinli bo'lishini isbotlang. \\
T2. Ushbu \(P(0) = 20\) va \(P(1) = 100\) shartlarini qanoatlantiruvchi \(P(x)\) ko'phadi mavjudmi? \\
A1. Tengsizlikni yeching:\(x^{2}\left( x^{4} + 36 \right) - 6\sqrt{3}\left( x^{4} + 4 \right) < 0\). \\
A2. Tengsizlikni yeching: \(\frac{x^{3} + 3x^{2} - x - 3}{x^{2} + 3x - 10} < 0\). \\
A3. Tenglamani yeching. \(\sqrt{x^{2} + x + 4} + \sqrt{x^{2} + x + 1} = \sqrt{2x^{2} + 2x + 9}\). \\
B1. \(P(x + n) = (x + n)^{3} + (x - n)^{2} + x + n + 6\) ko'phadi berilgan. \(P(x)\) ko'phadi \(x - n\) ga qoldiqsiz bo'linsa, \(n\) ni toping. \\
B2. Quyidagi mulohazani ixtiyoriy natural son uchun matematik induksiya metodi yordamida isbotlang: \(1^{3} + 2^{3} + 3^{3} + ... + n^{3} = \left( \frac{n(n + 1)}{2} \right)^{2}\); \\
B3. Quyidagi mulohazani ixtiyoriy natural son uchun matematik induksiya metodi yordamida isbotlang: \(7^{n} - 1\) soni 6 ga karrali; \\
C1. Teńlemeni sheshiń \(\frac{C_{2x}^{x + 1}}{C_{2x + 1}^{x - 1}} = \frac{2}{3}\), \(x \in N\) \\
C2. \(\bigtriangleup ABC\) da \(\angle A\) burchak \(\angle B\) dan ikki marta katta bo'lib, \(AC = b,AB = c\). \emph{BC} tomonning uzunligi topilsin. \\
C3. Ya. Bermlli tengsizligi. Agar \(x \geq - 1\) bo'lsa, u holda ixtiyoriy natural \(n\) soni uchun \((1 + x)^{n} \geq 1 + nx\) tengsizlik o'rinli bo'lishini isbotlang. \\

\end{tabular}
\vspace{1cm}


\begin{tabular}{m{17cm}}
\textbf{92-variant}
\newline

T1. Ushbu \(P(x) = x^{5} + 11x^{4} + 37x^{3} + 35x^{2} - 44x - 40\) ko'phadi \(Q(x) = x^{2} + 3x + 2\) ko'phadiga qoldiqsiz bo'linadimi? \\
T2. \(x\) o'zgaruvchining ixtiyoriy butun qiymatida \(ax^{2} + bx + c\) uchhadining qiymati butun bo'lishi uchun \(2a,\ a + b\) va \(c\) sonlarining butun bo'lishi zarur va yetarli ekanligini isbotlang. \\
A1. Tenglamani yeching. \((x - 4)^{3} + (x - 4)^{2} + (x - 4)(x - 3) + (x - 3)^{2} + (x - 3)^{3} = 6\). \\
A2. Tenglamani yeching. \(\sqrt{3x^{2} - 2x + 15} + \sqrt{3x^{2} - 2x + 8} = 7\). \\
A3. Tenglamani yeching. \(\frac{4x}{x^{2} + x + 3} + \frac{5x}{x^{2} - 5x + 3} = - \frac{3}{2}\). \\
B1. \(P(2x - 1) + P(x - 1) = 10x^{2} - 12x + 2\) bo'lsa, \(P(x)\) ni toping. \\
B2. Quyidagi mulohazani ixtiyoriy natural son uchun matematik induksiya metodi yordamida isbotlang: \(1^{2} + 3^{2} + 5^{2} + ... + (2n - 1)^{2} = \frac{n\left( 4n^{2} - 1 \right)}{3}\); \\
B3. Quyidagi mulohazani ixtiyoriy natural son uchun matematik induksiya metodi yordamida isbotlang:\(6^{2n - 2} + 3^{n + 1} + 3^{n - 1}\) soni 11 karrali ; \\
C1. \(5C_{n}^{3} = C_{n + 2}^{4}\) bo'lsa, \(n\) ni toping. \\
C2. Uchburchakning perimetri \(4,5dm\) ga teng, bissektrisa esa qarshi tomonni uzunliklari 6 va 9 sm ga teng bo'lgan kesmalarga ajratadi. Uchburchakning tomonlari topilsin. \\
C3. Agar \(S\) uchburchakning yuzi, \(b\) va \(c\) uning tomonlari bo'lsa, \(S \leq \frac{b^{2} + c^{2}}{4}\) bo'lishini isbotlang. \\

\end{tabular}
\vspace{1cm}


\begin{tabular}{m{17cm}}
\textbf{93-variant}
\newline

T1. Bezu teoremasi va uning qo'llanilishi. \\
T2. Ixtiyoriy \(a,b,c \in (0;1)\) sonlari uchun \(a(1 - b) > 1/4,\ b(1 - c) > 1/4,\ c(1 - a) > 1/4\) tengsizliklari bir vaqtda o'rinli bo'la olmasligini isbotlang. \\
A1. Tenglamani yeching \((x + 1)^{5} + (x - 1)^{5} = 32x\). \\
A2. Tenglamani yeching \(\sqrt{\frac{18 - 7x - x^{2}}{8 - 6x + x^{2}}} + \sqrt{\frac{8 - 6x + x^{2}}{18 - 7x - x^{2}}} = \frac{13}{6}\). \\
A3. Tenglamani yeching. \(\sqrt[3]{x} + \sqrt[3]{x - 16} = \sqrt[3]{x - 8}\). \\
B1. \(P(x)\) ko'phadni \(3x^{2} - 4x + 1\) ga bo'lganimizda qoldiq \(6x - 11\) bo'lsa, \(P(x)\) ko'phadni \(3x - 1\)ga bo'lganda qoldiqni toping. \\
B2. Quyidagi mulohazani ixtiyoriy natural son uchun matematik induksiya metodi yordamida isbotlang: \(1 \cdot 2 + 2 \cdot 3 + 3 \cdot 4 + ... + n(n + 1) = \frac{n(n + 1)(n + 2)}{3}\); \\
B3. Quyidagi mulohazani ixtiyoriy natural son uchun matematik induksiya metodi yordamida isbotlang: \(5 \cdot 2^{3n - 2} + 3^{3n - 1}\) soni 19 ga karrali \\
C1. Ayniyatni isbotlang:\(\sum_{j = 0}^{n}C_{n}^{j}( - 1)^{j} = 0\); \\
C2. \(\bigtriangleup ABC\) da \(AB = 2sm,BD\) mediana, \(BD = 1sm\), \(\angle BDA = 30^{{^\circ}}\). Uchburchakning yuzi hisoblansin. \\
C3. Muntazam uchburchakning tomoni a ga teng. Tomonini diametr deb hisoblab doira yasalgan. Uchburchakning shu doiradan tashqaridagi qismi yuzini toping. \\

\end{tabular}
\vspace{1cm}


\begin{tabular}{m{17cm}}
\textbf{94-variant}
\newline

T1. \(P(x) = x^{6} - 3x^{5} + x^{4} - 6x^{2} + 2x - 6\) ko'phadining butun ildizlarini toping. \\
T2. Yig'indisi birga teng bo'lgan \(x,y,z\) musbat sonlari uchun \(\frac{1}{x} + \frac{1}{y} + \frac{1}{z} \geq 9\) tengsizligi o'rinli bo'lishini isbotlang. \\
A1. Tenglamani yeching \(\left( x^{2} - 6x \right)^{2} - 2(x - 3)^{2} = 81\). \\
A2. Tenglamani yeching. \(\frac{z}{z + 1} - 2\sqrt{\frac{z + 1}{2}} = 3\). \\
A3. Tenglamani yeching. \(\sqrt{x} + \frac{2x + 1}{x + 2} = 2\). \\
B1. \(P(x + 3)\) ko'phadni \(x + 1\) ga bo'lganda qoldiq -3, \(Q(2x - 1)\) ko'phadni \(x - 1\)ga bo'lganda qoldiq 2 bo'lsa, \(P(x + 4) + x^{2}Q(x + 3)\) ko'phadni \(x + 2\) ga bo'lgandagi qoldiqni toping. \\
B2. Quyidagi mulohazani ixtiyoriy natural son uchun matematik induksiya metodi yordamida isbotlang: \(\frac{1}{1 \cdot 4} + \frac{1}{4 \cdot 7} + \frac{1}{7 \cdot 10} + \ldots + \frac{1}{(3n - 2) \cdot (3n + 1)} = \frac{n}{(3n + 1)}\). \\
B3. Quyidagi mulohazani ixtiyoriy natural son uchun matematik induksiya metodi yordamida isbotlang: \(2n^{3} + 3n^{2} + 7n\) soni 6 ga karrali ; \\
C1. Teńsizlikti sheshiń \(5C_{x}^{3} < C_{x + 2}^{4}\), \(x \in N\) \\
C2. \emph{ABC} uchburchakning \emph{AB} tomonida yotgan \(N\) nuqtadan \(NQ\| AC\) va \(NP\| BC\) to'g'ri chiziqlar o'tkazilgan. Agar \emph{BNQ} uchburchakning yuzi \(S_{1}\) ga, \emph{ANP} uchburchakning yuzi \(S_{2}\) ga tengligi ma'lum bo'lsa, \emph{ABC} uchburchakning yuzini toping. \\
C3. Ya. Bermlli tengsizligi. Agar \(x \geq - 1\) bo'lsa, u holda ixtiyoriy natural \(n\) soni uchun \((1 + x)^{n} \geq 1 + nx\) tengsizlik o'rinli bo'lishini isbotlang. \\

\end{tabular}
\vspace{1cm}


\begin{tabular}{m{17cm}}
\textbf{95-variant}
\newline

T1. Matematik induksiya metodi va uning qo'llanilishiga misollar. \\
T2. Fales teoremasi va uning qo'llanilishi. \\
A1. Tengsizlikni yeching:\(x^{2}\left( x^{4} + 36 \right) - 6\sqrt{3}\left( x^{4} + 4 \right) < 0\). \\
A2. Tenglamani yeching. \(\sqrt{3x^{2} - 2x + 15} + \sqrt{3x^{2} - 2x + 8} = 7\). \\
A3. Tenglamani yeching \((x + 4)(x + 1) - 3\sqrt{x^{2} + 5x + 2} = 6\). \\
B1. \(P(x + 1) + P(x - 3) = 2x^{2} - 10x + 16\) bo'lsa, \(P(x)\) ni toping. \\
B2. Quyidagi mulohazani ixtiyoriy natural son uchun matematik induksiya metodi yordamida isbotlang: \(\left( 1 - \frac{1}{4} \right)\left( 1 - \frac{1}{9} \right)...\left( 1 - \frac{1}{n^{2}} \right) = \frac{n + 1}{2n}\), \(n \geq 2\) \\
B3. Quyidagi mulohazani ixtiyoriy natural son uchun matematik induksiya metodi yordamida isbotlang: \(5^{n + 2} + 26 \cdot 5^{n} + 8^{2n + 1}\) soni 59 ga karrali; \\
C1. Ayniyatni isbotlang: \(C_{n + 1}^{j + 1} = C_{n}^{j} + C_{n}^{j + 1}\); \\
C2. Ikkita bir xil radiusli doiralar shunday joylashganki, ularning markazlari orasidagi masofa radiusga teng. Doiralar kesishgan qismi yuzining kesishgan qismiga ichki chizilgan kvadrat yuziga nisbatini toping. \\
C3. Muntazam uchburchakning tomoni a ga teng. Tomonini diametr deb hisoblab doira yasalgan. Uchburchakning shu doiradan tashqaridagi qismi yuzini toping. \\

\end{tabular}
\vspace{1cm}


\begin{tabular}{m{17cm}}
\textbf{96-variant}
\newline

T1. Haqiqiy \(a_{1},\ a_{2},\ .\ .\ .\ ,\ a_{n},\ b_{1},\ b_{2},\ .\ .\ .\ ,\ b_{n}\) sonlari uchun \(\left( a_{1}b_{1} + a_{2}b_{2} + \ .\ .\ .\  + a_{n}b_{n} \right)^{2} \leq \left( a_{1}^{2} + a_{2}^{2} + \ .\ .\ .\  + a_{n}^{2} \right)\left( b_{1}^{2} + b_{2}^{2} + \ .\ .\ .\  + b_{n}^{2} \right)\) \\
T2. Koshi tengsizligini isbotlang. \\
A1. Tengsizlikni yeching: \(\sqrt{x^{2} - 4x} > x - 3\). \\
A2. Tenglamani yeching. \(\sqrt{x^{2} + x + 4} + \sqrt{x^{2} + x + 1} = \sqrt{2x^{2} + 2x + 9}\). \\
A3. Tenglamani yeching \((x + 1)^{5} + (x - 1)^{5} = 32x\). \\
B1. \(P(x + 3) = x^{2} - x + n\) bo'lsa. \(P(x - 2)\) ko'phadni \(x - 3\) ga bo'lganda qoldiq \(10\) bo'lsa, \(n\) ni toping. \\
B2. Quyidagi mulohazani ixtiyoriy natural son uchun matematik induksiya metodi yordamida isbotlang: \(1 \cdot 2 + 2 \cdot 3 + 3 \cdot 4 + \ldots + n \cdot (n + 1) = \frac{n \cdot (n + 1) \cdot (n + 2)}{3}\). \\
B3. Quyidagi mulohazani ixtiyoriy natural son uchun matematik induksiya metodi yordamida isbotlang: \(n^{3} + (n + 1)^{3} + (n + 2)^{3}\) soni 9 ga karrali ; \\
C1. \(\left( \sqrt{x} + \frac{1}{\sqrt[3]{x^{2}}} \right)^{n}\) binom yoyilmasida 5-had koeffitsiyentining 3-had koeffitsiyentiga nisbati 7:2 ga teng. \(x\) ning darajasi 1 ga teng bo'lgan ahadin toping. \\
C2. Uchburchakning a, b va \(c\) tomonlari arifmetik progressiya tashkil qiladi. \(ac = 6Rr\) bo'lishini isbotlang. Bu yerda \(R\) va \(r\) tashqi va ichki chizilgan aylanalarning radiuslari. \\
C3. Agar \(S\) uchburchakning yuzi, \(b\) va \(c\) uning tomonlari bo'lsa, \(S \leq \frac{b^{2} + c^{2}}{4}\) bo'lishini isbotlang. \\

\end{tabular}
\vspace{1cm}


\begin{tabular}{m{17cm}}
\textbf{97-variant}
\newline

T1. Simmetrik ko'phadlar. \\
T2. \(2^{81} + 1\) soni 9 soniga qoldiqsiz bo'linishini isbotlang. \\
A1. Tengsizlikni yeching: \(\sqrt{x + 3} + \sqrt{x - 2} - \sqrt{2x + 4} > 0\). \\
A2. Tenglamani yeching. \(\frac{4x}{x^{2} + x + 3} + \frac{5x}{x^{2} - 5x + 3} = - \frac{3}{2}\). \\
A3. Tenglamani yeching. \((\sqrt{x + 1} + \sqrt{x})^{3} + (\sqrt{x + 1} + \sqrt{x})^{2} = 2\). \\
B1. \(P(x) = x^{33} - 2ax^{21} + x^{8} + 8\) ko'phadi berilgan. \(a\) ning qaysi qiymati uchun \(P(x)\) ko'phadi \(x + 1\) ga qoldiqsiz bo'liadi? \\
B2. Quyidagi mulohazani ixtiyoriy natural son uchun matematik induksiya metodi yordamida isbotlang: \(\frac{1}{4 \cdot 5} + \frac{1}{5 \cdot 6} + \frac{1}{6 \cdot 7} + \ldots + \frac{1}{(n + 3) \cdot (n + 4)} = \frac{n}{4 \cdot (n + 4)}\). \\
B3. Quyidagi mulohazani ixtiyoriy natural son uchun matematik induksiya metodi yordamida isbotlang: \(5^{n} - 4n + 15\) soni 16 ga karrali ; \\
C1. \(\left( 2x^{\ ^{2}} - \frac{b}{2x^{3}} \right)^{10}\) binom yoyilmasining \(x\) qatnashmagan hadin toping. \\
C2. \(\bigtriangleup ABC\) da \(AB = 3sm,AC = 5sm,\angle BAC = 120^{{^\circ}}.BD\) bissektrisaning uzunligi topilsin. \\
C3. Agar \(S\) uchburchakning yuzi, \(b\) va \(c\) uning tomonlari bo'lsa, \(S \leq \frac{b^{2} + c^{2}}{4}\) bo'lishini isbotlang. \\

\end{tabular}
\vspace{1cm}


\begin{tabular}{m{17cm}}
\textbf{98-variant}
\newline

T1. Pifagor teoremasi va uning isbotlari. \\
T2. \(n\) darajaning qanday qiymatlarida \((x + 1)^{n} + (x - 1)^{n}\) ifodasi \(x\) ifodaga qoldiqsiz bo'linadi? \\
A1. Tenglamani yeching. \(\sqrt[3]{x - 1} + \sqrt[3]{x - 2} - \sqrt{2x - 3} = 0\). \\
A2. Tenglamani yeching \(\left( x^{2} - 6x \right)^{2} - 2(x - 3)^{2} = 81\). \\
A3. Tenglamani yeching. \(\sqrt{x + 8 + 2\sqrt{x + 7}} + \sqrt{x + 1 - \sqrt{x + 7}} = 4\). \\
B1. \(P(x) = (x - 5)^{2n + 1} + (x - 1)^{2n + 3}\) ko'phadni \(x - 3\) ga bo'lganda qoldiq \(3 \cdot 2^{3n - 4}\) bo'lsa, \(n\) ni toping. \\
B2. Quyidagi mulohazani ixtiyoriy natural son uchun matematik induksiya metodi yordamida isbotlang: \(2^{2} + 6^{2} + \ldots + (4n - 2)^{2} = \frac{4n(2n - 1)(2n + 1)}{3}\). \\
B3. Quyidagi mulohazani ixtiyoriy natural son uchun matematik induksiya metodi yordamida isbotlang: \(5^{2n + 1} + 3^{n + 2} \cdot 2^{n - 1}\) soni 19 ga karrali ; \\
C1. Ayniyatni isbotlang: \(C_{n + k}^{j + k} = \sum_{s = 0}^{k}C_{n}^{j + s}C_{k}^{s}\); \\
C2. \emph{ABC} uchburchakning \emph{AC}, \emph{BC} va \emph{AB} tomonlarida \emph{CMPA}, \emph{BEFC} va \emph{ADKB} kvadratlar yasalgan. Agar \(AB = 13\), \(AC = 14,BC = 15\) ekanligi ma'lum bo'lsa, \emph{DKEFMP} oltiburchakning yuzini toping. \\
C3. Muntazam uchburchakning tomoni a ga teng. Tomonini diametr deb hisoblab doira yasalgan. Uchburchakning shu doiradan tashqaridagi qismi yuzini toping. \\

\end{tabular}
\vspace{1cm}


\begin{tabular}{m{17cm}}
\textbf{99-variant}
\newline

T1. \(b\) parametrining qanday qiymatlarida \(x^{3} + 17x^{2} + bx - 17 = 0\) tenglamasining ildizlari butun sonlardan iborat bo'ladi? \\
T2. \(P(x) = (x - 1)^{20}\left( x^{2} + 25 \right)\) ko'phadining koeffitsentlari yig'indisini toping. \\
A1. Tenglamani yeching. \(\sqrt[3]{x} + \sqrt[3]{x - 16} = \sqrt[3]{x - 8}\). \\
A2. Tengsizlikni yeching: \(\frac{x^{3} + 3x^{2} - x - 3}{x^{2} + 3x - 10} < 0\). \\
A3. Tenglamani yeching. \((x - 4)^{3} + (x - 4)^{2} + (x - 4)(x - 3) + (x - 3)^{2} + (x - 3)^{3} = 6\). \\
B1. \(P(x + 2) + P(x - 1) = - 2x^{2} - 2x + 7\) bo'lsa, \(P(x)\) ni \(x + 4\) ga bo'lgandagi qoldiqni toping. \\
B2. Quyidagi mulohazani ixtiyoriy natural son uchun matematik induksiya metodi yordamida isbotlang: \(1 \cdot 1! + 2 \cdot 2! + 3 \cdot 3! + \ldots + n \cdot n! = (n + 1)! - 1\). \\
B3. Quyidagi mulohazani ixtiyoriy natural son uchun matematik induksiya metodi yordamida isbotlang: \(n\left( 2n^{2} - 3n + 1 \right)\) soni 6 ga karrali ; \\
C1. Teńsizlikti sheshiń \(C_{13}^{x} < C_{13}^{x + 2}\), \(x \in N\) \\
C2. Teng yonli uchburchakning yon tomoni 13 sm , yon tomoniga o'tkazilgan balandlik 5 sm ga teng. Uchburchak asosining uzunligi topilsin. \\
C3. Ya. Bermlli tengsizligi. Agar \(x \geq - 1\) bo'lsa, u holda ixtiyoriy natural \(n\) soni uchun \((1 + x)^{n} \geq 1 + nx\) tengsizlik o'rinli bo'lishini isbotlang. \\

\end{tabular}
\vspace{1cm}


\begin{tabular}{m{17cm}}
\textbf{100-variant}
\newline

T1. \(a\) parametrining qanday qiymatlarida \(P(x) = x^{2017} + ax - 5\) ko'phadi \((x + 1)\) ko'phadiga qoldiqsiz bo'linadi? \\
T2. Kombinatorika elementlari va Nyuton binomi. \\
A1. Tenglamani yeching \(\left( x^{2} - 4x + 6 \right)^{2} - 4\left( x^{2} - 4x + 6 \right) + 6 = x\). \\
A2. Tenglamani yeching \(\sqrt{\frac{18 - 7x - x^{2}}{8 - 6x + x^{2}}} + \sqrt{\frac{8 - 6x + x^{2}}{18 - 7x - x^{2}}} = \frac{13}{6}\). \\
A3. Tenglamani yeching \(\left( x^{2} + 10x + 10 \right)\left( x^{2} + x + 10 \right) = 10x^{2}\) . \\
B1. \(P(x) = x^{4} - 2x + 2^{n + 1}\) ko'phadni \(x - 2^{n}\) ga bo'lganda qoldiq \(2^{n - 2}\) bo'lsa, \(n\) ni toping. \\
B2. Quyidagi mulohazani ixtiyoriy natural son uchun matematik induksiya metodi yordamida isbotlang: \(1^{3} + 2^{3} + 3^{3} + ... + n^{3} = \left( \frac{n(n + 1)}{2} \right)^{2}\); \\
B3. Quyidagi mulohazani ixtiyoriy natural son uchun matematik induksiya metodi yordamida isbotlang: \(5^{n} - 4n + 15\) soni 16 ga karrali ; \\
C1. \(x(1 - x)^{4} + x^{2}(1 + 2x)^{8} + x^{3}(1 + 3x)^{12}\) ifodada \(x^{4}\) oldidagi koeffitsiyentni toping. \\
C2. Bir burchagi \(60^{{^\circ}}\) bo'lgan uchburchakka ichki chizilgan aylananing urinish nuqtasi shu burchakka qarama- qarshi tomonini \(a\) va \(b\) kesmalarga ajratadi. Uchburchak yuzini toping. \\
C3. Muntazam uchburchakning tomoni a ga teng. Tomonini diametr deb hisoblab doira yasalgan. Uchburchakning shu doiradan tashqaridagi qismi yuzini toping. \\

\end{tabular}
\vspace{1cm}



\end{document}
