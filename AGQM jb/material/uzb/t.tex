Bezu teoremasi va uning qo'llanilishi.
Simmetrik ko'phadlar.
Matematik induksiya metodi va uning qo'llanilishiga misollar.
Kombinatorika elementlari va Nyuton binomi.
Pifagor teoremasi va uning isbotlari.
Fales teoremasi va uning qo'llanilishi.
\(2^{81} + 1\) soni 9 soniga qoldiqsiz bo'linishini isbotlang.
\(a\) parametrining qanday qiymatlarida \(P(x) = x^{2017} + ax - 5\) ko'phadi \((x + 1)\) ko'phadiga qoldiqsiz bo'linadi?
\(b\) parametrining qanday qiymatlarida \(x^{3} + 17x^{2} + bx - 17 = 0\) tenglamasining ildizlari butun sonlardan iborat bo'ladi?
\(n\) darajaning qanday qiymatlarida \((x + 1)^{n} + (x - 1)^{n}\) ifodasi \(x\) ifodaga qoldiqsiz bo'linadi?
Ushbu \(P(x) = x^{5} + 11x^{4} + 37x^{3} + 35x^{2} - 44x - 40\) ko'phadi \(Q(x) = x^{2} + 3x + 2\) ko'phadiga qoldiqsiz bo'linadimi?
Ushbu \(P(0) = 20\) va \(P(1) = 100\) shartlarini qanoatlantiruvchi \(P(x)\) ko'phadi mavjudmi?
\(P(x) = x^{6} - 3x^{5} + x^{4} - 6x^{2} + 2x - 6\) ko'phadining butun ildizlarini toping.
\(P(x) = (x - 1)^{20}\left( x^{2} + 25 \right)\) ko'phadining koeffitsentlari yig'indisini toping.
Ixtiyoriy \(a\) parametri va \(x\) uchun \(x(a - x) \leq a^{2}/4\) tengsizligi o'rinli bo'lishini isbotlang.
Ixtiyoriy \(a,b,c \in (0;1)\) sonlari uchun \(a(1 - b) > 1/4,\ b(1 - c) > 1/4,\ c(1 - a) > 1/4\) tengsizliklari bir vaqtda o'rinli bo'la olmasligini isbotlang.
Yig'indisi birga teng bo'lgan \(x,y,z\) musbat sonlari uchun \(\frac{1}{x} + \frac{1}{y} + \frac{1}{z} \geq 9\) tengsizligi o'rinli bo'lishini isbotlang.
Haqiqiy \(a_{1},\ a_{2},\ .\ .\ .\ ,\ a_{n},\ b_{1},\ b_{2},\ .\ .\ .\ ,\ b_{n}\) sonlari uchun \(\left( a_{1}b_{1} + a_{2}b_{2} + \ .\ .\ .\  + a_{n}b_{n} \right)^{2} \leq \left( a_{1}^{2} + a_{2}^{2} + \ .\ .\ .\  + a_{n}^{2} \right)\left( b_{1}^{2} + b_{2}^{2} + \ .\ .\ .\  + b_{n}^{2} \right)\)
Koshi tengsizligini isbotlang.
\(x\) o'zgaruvchining ixtiyoriy butun qiymatida \(ax^{2} + bx + c\) uchhadining qiymati butun bo'lishi uchun \(2a,\ a + b\) va \(c\) sonlarining butun bo'lishi zarur va yetarli ekanligini isbotlang.