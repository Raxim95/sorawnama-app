Ayniyatni isbotlang:\(C_{n}^{j} = C_{n}^{n - j}\);
Ayniyatni isbotlang: \(C_{n + 1}^{j + 1} = C_{n}^{j} + C_{n}^{j + 1}\);
Ayniyatni isbotlang:\(C_{n + 2}^{j + 2} = C_{n}^{j} + 2C_{n}^{j + 1} + C_{n}^{j + 2}\);
Ayniyatni isbotlang: \(\sum_{j = 0}^{n}C_{n}^{j} = 2^{n}\);
Ayniyatni isbotlang:\(\sum_{j = 0}^{n}C_{n}^{j}( - 1)^{j} = 0\);
Ayniyatni isbotlang: \(C_{n + k}^{j + k} = \sum_{s = 0}^{k}C_{n}^{j + s}C_{k}^{s}\);
\(\frac{1}{C_{4}^{n}} = \frac{1}{C_{5}^{n}} + \frac{1}{C_{6}^{n}}\) bo'lsa, \(n\) ni toping
\(5C_{n}^{3} = C_{n + 2}^{4}\) bo'lsa, \(n\) ni toping.
\(C_{n + 4}^{n + 1} - C_{n + 3}^{n} = 15(n + 2)\) bo'lsa, \(n\) ni toping.
Tengsizlikni yeching: \(C_{10}^{x - 1} > 2C_{10}^{x}\)
Teńlemeni sheshiń \(\frac{C_{2x}^{x + 1}}{C_{2x + 1}^{x - 1}} = \frac{2}{3}\), \(x \in N\)
Teńsizlikti sheshiń \(C_{13}^{x} < C_{13}^{x + 2}\), \(x \in N\)
Teńsizlikti sheshiń \(5C_{x}^{3} < C_{x + 2}^{4}\), \(x \in N\)
\(\left( \sqrt{x} + \frac{1}{\sqrt[3]{x^{2}}} \right)^{n}\) binom yoyilmasida 5-had koeffitsiyentining 3-had koeffitsiyentiga nisbati 7:2 ga teng. \(x\) ning darajasi 1 ga teng bo'lgan ahadin toping.
\(\left( x\sqrt{x} - \frac{1}{x^{4}} \right)^{n}\) binom yoyilmasida 3-had koeffitsiyenti 2-had koeffitsiyentidan 44 ga katta.Ozod hadini toping.
\((a + b)^{n}\) ifoda yoyilmasining barcha koeffitsiyentlari yig`indisi 4096 ga teng bo'lsa, uning eng katta koeffitsiyentin toping.
\(\left( x^{3} - \frac{3}{x^{2}} \right)^{10}\) binom yoyilmasining \(x\) qatnashmagan hadin toping.
\(\left( 2x^{\ ^{2}} - \frac{b}{2x^{3}} \right)^{10}\) binom yoyilmasining \(x\) qatnashmagan hadin toping.
\(x(1 - x)^{4} + x^{2}(1 + 2x)^{8} + x^{3}(1 + 3x)^{12}\) ifodada \(x^{4}\) oldidagi koeffitsiyentni toping.
\((x + 1)^{3} + (x + 1)^{4} + (x + 1)^{5} + ... + (x + 1)^{10}\) ifodada \(x^{3}\) oldidagi koeffitsiyentni toping
++++
To'g'ri burchakli uchburchakning balandligi gipotenuzani uzunliklari 18 va 32 sm ga teng bo'lgan kesmalarga ajratadi. Uchburchakning yuzi hisoblansin.
To'g'ri burchakli uchburchakning balandligi gipotenuzani uzunliklari \emph{x} va \emph{y} ga teng bo'lgan kesmalarga ajratadi. Uchburchakning yuzi hisoblansin.
Uchburchakning asosiga tushirilgan balandligi \(h\) ga teng. Uchburchakning asosiga parallel kesma uchburchakning yuzini teng ikkiga bo'ladi. Uchburchakning uchidan shu kesmagacha bo'lgan masofa topilsin.
Teng yonli uchburchakning yon tomoni 13 sm , yon tomoniga o'tkazilgan balandlik 5 sm ga teng. Uchburchak asosining uzunligi topilsin.
Agar teng yonli uchburchakning perimetri 32 dm , o'rta chizig'i 6 dm ga teng bo'lsa, uning tomonlari uzunliklari topilsin.
To'g'ri burchakli uchburchakda katetlar 7 sm va 24 sm ga teng. To'g'ri burchakning bissektrisasi o'tkazilgan. Bu bissektrisa gipotenuzani qanday uzunlikdagi kesmalarga ajratadi?
Uchburchakning perimetri \(4,5dm\) ga teng, bissektrisa esa qarshi tomonni uzunliklari 6 va 9 sm ga teng bo'lgan kesmalarga ajratadi. Uchburchakning tomonlari topilsin.
To'g'ri burchakli uchburchakda katetlarning nisbati 3:2 kabi, balandlik esa gipotenuzani shunday ikkita kesmaga ajratadiki, ulardan birining uzunligi ikkinchisidan 2 ga katta. Gipotenuzaning uzunligi topilsin.
\emph{ABC} uchburchak berilgan. Uning medianalaridan \(\bigtriangleup A_{1}B_{1}C_{1}\) yasalgan. \(\bigtriangleup ABC\) va \(\bigtriangleup A_{1}B_{1}C_{1}\) yuzlarining nisbati topilsin.
To'g'ri burchakli uchburchakning katetlari \(b\) va \(c\) ga teng. To'g'ri burchak bissektrisasining uzunligi topilsin.
\(\bigtriangleup ABC\) da \(AB = 2sm,BD\) mediana, \(BD = 1sm\), \(\angle BDA = 30^{{^\circ}}\). Uchburchakning yuzi hisoblansin.
\(\bigtriangleup ABC\) da \(AB = 3sm,AC = 5sm,\angle BAC = 120^{{^\circ}}.BD\) bissektrisaning uzunligi topilsin.
\(\bigtriangleup ABC\) da \(\angle A\) burchak \(\angle B\) dan ikki marta katta bo'lib, \(AC = b,AB = c\). \emph{BC} tomonning uzunligi topilsin.
\emph{ABC} uchburchakning \(B\) uchidan \emph{AC} tomoniga \emph{BD} kesma o'tkazildi. \emph{BD} kesma bu uchburchakning yuzini teng ikkiga bulladi. Agar \(AC = a\) bo'lsa, \emph{AD} va \emph{DC} kesmalarning uzunliklarini toping.
\emph{ABCD} parallelogrammning \emph{AD} tomoni \(n\) ta teng bo'lakka bo'lingan. Birinchi bo'linish nuqtasi \(P\) va \(B\) uch bilan birlashtirilgan. \emph{BP} to'g'ri chiziq \emph{AC} dioganaldan uning \(\frac{1}{n + 1}\) qismiga teng \emph{AQ} kesma ajratishini isbotlang.
To'g'ri burchakli uchburchakning to'g'ri burchagi bissektrisasi shu uchdan o'tkazilgan mediana va balandlik orasidagi burchakni ham teng ikkiga bo'lishini isbotlang.
Asoslari \(x\) va 3 bo'lgan trapetsiyada diagonallar o'rtalari orasidagi masofani \(x\) ning funksiyasi sifatida ifodalang. \(x\) nіnng qanday qiymatida bu masofa 1 ga teng bo'ladi?
To'g'ri burchakli uchburchak o'tkir burchaklarining bisєektrisalari AD va BK \(AB^{2} = AD \cdot BK\) bo'lsa, uchburchakning burchaklarini toping.
Muntazam uchburchakning uchlari uchta parallel to'g'ri chiziqlarda yotadi. Agar o'rtadagi to'g'ri chiziqdan chekkalardagi to'g'ri chiziklargacha bo'lgan masofa \(a\) va \(b\) ga teng bo'lsa, uchburchakning tomonini toping.
Uchburchakning tomonlari \(a\) va \(b\), bissektrisasi \(l_{c} = l\). \(l\) ni bilgan holda uning yuzini toping.
\emph{ABC} uchburchakning \emph{AB} tomonida yotgan \(N\) nuqtadan \(NQ\| AC\) va \(NP\| BC\) to'g'ri chiziqlar o'tkazilgan. Agar \emph{BNQ} uchburchakning yuzi \(S_{1}\) ga, \emph{ANP} uchburchakning yuzi \(S_{2}\) ga tengligi ma'lum bo'lsa, \emph{ABC} uchburchakning yuzini toping.
Teng yonli uchburchak asosidagi burchak \(\alpha\) ga teng. Shu burchak uchidan asosga \(\beta(\beta < \alpha)\) burchak ostida to'g'ri chiziq o'tkazilgan, u uchburchakni ikki qismga ajratadi. Hosil bo'lgan uchburchaklar yuzlarining nisbatini toping.
Bir burchagi \(60^{{^\circ}}\) bo'lgan uchburchakka ichki chizilgan aylananing urinish nuqtasi shu burchakka qarama- qarshi tomonini \(a\) va \(b\) kesmalarga ajratadi. Uchburchak yuzini toping.
Uchburchakning ishida olingan nuqtadan uning tomonlariga parallel to'g'ri chiziqlar o'tkazilgan. Ular uchburchakni 6 qismga bo'ladi. Agar hosil bo'lgan uchburchaklarning yuzlari \(S_{1},S_{2}\) va \(S_{3}\) bo'lsa, berilgan uchburchak yuzini toping.
Parallelogrammning tomonlari \(a\) va \(b\), ular orasidagi burchak \(\alpha\). bo'lsa, paralllelogramm ichki burchaklari bissektrisalari kesishishidan hosil bo'lgan to'rtburchak yuzini toping.
Markazlari \(O_{1}\) va \(O_{2}\) nuqtalarda va radiusi \(R\) bo'lgan ikkita teng aylanalar tashqi urinadi. \(l\) to'g'ri chiziq bu aylanalarni A, B, C va \(D\) nuqtalarda shunday kesib o'tadiki, \(AB = BC = CD\) bo'ladi. \(O_{1}ADO_{2}\) to'rtburchak yuzini toping.
\(ABCD(AD\| BC)\) trapetsiya diagonallari \(O\) nuqtada kesishadi. Agar \emph{AOD} uchburchakning yuzi \(a^{2}\) ga, \emph{BOC} uchburchakning yuzi \(b^{2}\) ga tengligi ma'lum bo'lsa, trapesiya yuzini toping.
\emph{ABC} uchburchakning \emph{AC}, \emph{BC} va \emph{AB} tomonlarida \emph{CMPA}, \emph{BEFC} va \emph{ADKB} kvadratlar yasalgan. Agar \(AB = 13\), \(AC = 14,BC = 15\) ekanligi ma'lum bo'lsa, \emph{DKEFMP} oltiburchakning yuzini toping.
\(R\) radiusli doiraga bitta umumiy uchga ega bo'lgan muntazam uchburchak va kvadrat ichki chizilgan. Ularning kesishgan qismining yuzini toping.
Ikkita bir xil radiusli doiralar shunday joylashganki, ularning markazlari orasidagi masofa radiusga teng. Doiralar kesishgan qismi yuzining kesishgan qismiga ichki chizilgan kvadrat yuziga nisbatini toping.
Teng yonli uchburchakning yuzi \(S\) ga teng. Yon tomonlariga tushirilgan medianalari orasidagi burchak \(\alpha\) ga teng. Uchburchak asosini toping.
Uchburchakning a, b va \(c\) tomonlari arifmetik progressiya tashkil qiladi. \(ac = 6Rr\) bo'lishini isbotlang. Bu yerda \(R\) va \(r\) tashqi va ichki chizilgan aylanalarning radiuslari.
Teng yonli \(ABC(AB = BC)\) uchburchakda \emph{AD} bissektrisa o'tkazilgan. Agar \(S_{ABD} = S_{1},S_{\bigtriangleup ADC} = S_{2}\) bo'lsa, \emph{AC} ni toping.
++++
Ya. Bermlli tengsizligi. Agar \(x \geq - 1\) bo'lsa, u holda ixtiyoriy natural \(n\) soni uchun \((1 + x)^{n} \geq 1 + nx\) tengsizlik o'rinli bo'lishini isbotlang.
Muntazam uchburchakning tomoni a ga teng. Tomonini diametr deb hisoblab doira yasalgan. Uchburchakning shu doiradan tashqaridagi qismi yuzini toping.
Agar \(S\) uchburchakning yuzi, \(b\) va \(c\) uning tomonlari bo'lsa, \(S \leq \frac{b^{2} + c^{2}}{4}\) bo'lishini isbotlang.