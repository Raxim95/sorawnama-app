\(P(x) = (x - 5)^{2n + 1} + (x - 1)^{2n + 3}\) kópaǵzalını \(x - 3\) ga bólgende qaldıq \(3 \cdot 2^{3n - 4}\) bolsa, \(n\) ni tabıń.
\(P(x) = x^{4} - 2x + 2^{n + 1}\) kópaǵzalını \(x - 2^{n}\) ga bólgende qaldıq \(2^{n - 2}\) bolsa, \(n\) ni tabıń.
\(P(x + 3) = x^{2} - x + n\) bolsa. \(P(x - 2)\) kópaǵzalını \(x - 3\) ga bólgende qaldıq \(10\) bolsa, \(n\) ni tabıń.
\(P(x + n) = (x + n)^{3} + (x - n)^{2} + x + n + 6\) kópaǵzalıi berilgan. \(P(x)\) kópaǵzalıi \(x - n\) ga qaldıqsiz bólinse, \(n\) ni tabıń.
\(P(x + 3)\) kópaǵzalını \(x + 1\) ga bólgende qaldıq -3, \(Q(2x - 1)\) kópaǵzalını \(x - 1\)ga bólgende qaldıq 2 bolsa, \(P(x + 4) + x^{2}Q(x + 3)\) kópaǵzalını \(x + 2\) ga bólgendegi qaldıqni tabıń.
\(P(x + 1) + P(x - 3) = 2x^{2} - 10x + 16\) bolsa, \(P(x)\) ni tabıń.
\(P(2x - 1) + P(x - 1) = 10x^{2} - 12x + 2\) bolsa, \(P(x)\) ni tabıń.
\(P(x + 2) + P(x - 1) = - 2x^{2} - 2x + 7\) bolsa, \(P(x)\) ni \(x + 4\) ga bólgendegi qaldıqni tabıń.
\(P(x)\) kópaǵzalını \(3x^{2} - 4x + 1\) ga bólgenimizdeqaldıq \(6x - 11\) bolsa, \(P(x)\) kópaǵzalını \(3x - 1\)ga bólgende qaldıqni tabıń.
\(P(x) = x^{33} - 2ax^{21} + x^{8} + 8\) kópaǵzalıi berilgan. \(a\) nıń qaysi qiymati ushın \(P(x)\) kópaǵzalıi \(x + 1\) ga qaldıqsiz bóliadi?
++++
Tómendegi aytımdı qálegen natural san ushın matematikalıq induksiya metodi járdeminde dálilleń: \(1 \cdot 2 + 2 \cdot 3 + 3 \cdot 4 + \ldots + n \cdot (n + 1) = \frac{n \cdot (n + 1) \cdot (n + 2)}{3}\).
Tómendegi aytımdı qálegen natural san ushın matematikalıq induksiya metodi járdeminde dálilleń: \(\frac{1}{4 \cdot 5} + \frac{1}{5 \cdot 6} + \frac{1}{6 \cdot 7} + \ldots + \frac{1}{(n + 3) \cdot (n + 4)} = \frac{n}{4 \cdot (n + 4)}\).
Tómendegi aytımdı qálegen natural san ushın matematikalıq induksiya metodi járdeminde dálilleń: \(\frac{1}{1 \cdot 4} + \frac{1}{4 \cdot 7} + \frac{1}{7 \cdot 10} + \ldots + \frac{1}{(3n - 2) \cdot (3n + 1)} = \frac{n}{(3n + 1)}\).
Tómendegi aytımdı qálegen natural san ushın matematikalıq induksiya metodi járdeminde dálilleń: \(1 \cdot 1! + 2 \cdot 2! + 3 \cdot 3! + \ldots + n \cdot n! = (n + 1)! - 1\).
Tómendegi aytımdı qálegen natural san ushın matematikalıq induksiya metodi járdeminde dálilleń: \(2^{2} + 6^{2} + \ldots + (4n - 2)^{2} = \frac{4n(2n - 1)(2n + 1)}{3}\).
Tómendegi aytımdı qálegen natural san ushın matematikalıq induksiya metodi járdeminde dálilleń: \(1^{2} + 2^{2} + 3^{2} + ... + n^{2} = \frac{n(n + 1)(2n + 1)}{6}\);
Tómendegi aytımdı qálegen natural san ushın matematikalıq induksiya metodi járdeminde dálilleń: \(1^{2} + 3^{2} + 5^{2} + ... + (2n - 1)^{2} = \frac{n\left( 4n^{2} - 1 \right)}{3}\);
Tómendegi aytımdı qálegen natural san ushın matematikalıq induksiya metodi járdeminde dálilleń: \(1^{3} + 2^{3} + 3^{3} + ... + n^{3} = \left( \frac{n(n + 1)}{2} \right)^{2}\);
Tómendegi aytımdı qálegen natural san ushın matematikalıq induksiya metodi járdeminde dálilleń: \(1 \cdot 2 + 2 \cdot 3 + 3 \cdot 4 + ... + n(n + 1) = \frac{n(n + 1)(n + 2)}{3}\);
Tómendegi aytımdı qálegen natural san ushın matematikalıq induksiya metodi járdeminde dálilleń: \(\left( 1 - \frac{1}{4} \right)\left( 1 - \frac{1}{9} \right)...\left( 1 - \frac{1}{n^{2}} \right) = \frac{n + 1}{2n}\), \(n \geq 2\)
Tómendegi aytımdı qálegen natural san ushın matematikalıq induksiya metodi járdeminde dálilleń: \(\frac{1}{1 \cdot 5} + \frac{1}{5 \cdot 9} + ... + \frac{1}{(4n - 3)(4n + 1)} = \frac{n}{4n + 1}\);
++++
Tómendegi aytımdı qálegen natural san ushın matematikalıq induksiya metodi járdeminde dálilleń: \(n^{3} + (n + 1)^{3} + (n + 2)^{3}\) sanı 9 ga eseli ;
Tómendegi aytımdı qálegen natural san ushın matematikalıq induksiya metodi járdeminde dálilleń: \(5^{2n + 1} + 3^{n + 2} \cdot 2^{n - 1}\) sanı 19 ga eseli ;
Tómendegi aytımdı qálegen natural san ushın matematikalıq induksiya metodi járdeminde dálilleń: \(5^{n + 2} + 26 \cdot 5^{n} + 8^{2n + 1}\) sanı 59 ga eseli;
Tómendegi aytımdı qálegen natural san ushın matematikalıq induksiya metodi járdeminde dálilleń:\(7^{n} - 1\) sanı 6 ga eseli;
Tómendegi aytımdı qálegen natural san ushın matematikalıq induksiya metodi járdeminde dálilleń: \(5^{n} - 4n + 15\) sanı 16 ga eseli ;
Tómendegi aytımdı qálegen natural san ushın matematikalıq induksiya metodi járdeminde dálilleń: \(n\left( 2n^{2} - 3n + 1 \right)\) sanı 6 ga eseli ;
Tómendegi aytımdı qálegen natural san ushın matematikalıq induksiya metodi járdeminde dálilleń: \(2n^{3} + 3n^{2} + 7n\) sanı 6 ga eseli ;
Tómendegi aytımdı qálegen natural san ushın matematikalıq induksiya metodi járdeminde dálilleń: \(6^{2n - 2} + 3^{n + 1} + 3^{n - 1}\) sanı 11 eseli ;
Tómendegi aytımdı qálegen natural san ushın matematikalıq induksiya metodi járdeminde dálilleń: \(5 \cdot 2^{3n - 2} + 3^{3n - 1}\) sanı 19 ga eseli