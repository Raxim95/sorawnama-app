Birdeylikti dálilleń:\(C_{n}^{j} = C_{n}^{n - j}\);
Birdeylikti dálilleń: \(C_{n + 1}^{j + 1} = C_{n}^{j} + C_{n}^{j + 1}\);
Birdeylikti dálilleń:\(C_{n + 2}^{j + 2} = C_{n}^{j} + 2C_{n}^{j + 1} + C_{n}^{j + 2}\);
Birdeylikti dálilleń: \(\sum_{j = 0}^{n}C_{n}^{j} = 2^{n}\);
Birdeylikti dálilleń:\(\sum_{j = 0}^{n}C_{n}^{j}( - 1)^{j} = 0\);
Birdeylikti dálilleń: \(C_{n + k}^{j + k} = \sum_{s = 0}^{k}C_{n}^{j + s}C_{k}^{s}\);
\(\frac{1}{C_{4}^{n}} = \frac{1}{C_{5}^{n}} + \frac{1}{C_{6}^{n}}\) bolsa, \(n\) ni tabıń
\(5C_{n}^{3} = C_{n + 2}^{4}\) bolsa, \(n\) ni tabıń.
\(C_{n + 4}^{n + 1} - C_{n + 3}^{n} = 15(n + 2)\) bolsa, \(n\) ni tabıń.
Teńsizlikti sheshiń: \(C_{10}^{x - 1} > 2C_{10}^{x}\)
Teńlemeni sheshiń \(\frac{C_{2x}^{x + 1}}{C_{2x + 1}^{x - 1}} = \frac{2}{3}\), \(x \in N\)
Teńsizlikti sheshiń \(C_{13}^{x} < C_{13}^{x + 2}\), \(x \in N\)
Teńsizlikti sheshiń \(5C_{x}^{3} < C_{x + 2}^{4}\), \(x \in N\)
\(\left( \sqrt{x} + \frac{1}{\sqrt[3]{x^{2}}} \right)^{n}\) binom jayılmasında 5-aǵza koeffitsiyentinıń 3-aǵza koeffitsiyentine qatnasi 7:2 ga teń. \(x\) nıń darajasi 1 ga teń bolǵan aǵzasın tabıń.
\(\left( x\sqrt{x} - \frac{1}{x^{4}} \right)^{n}\) binom jayılmasında 3-aǵza koeffitsiyenti 2-aǵza koeffitsiyentidan 44 ga úlken.Ozod hadini tabıń.
\((a + b)^{n}\) ańlatpa jayılmasinıń barcha koeffitsiyentlari yig`indisi 4096 ga teń bolsa, Onıń eń úlken koeffitsiyentin tabıń.
\(\left( x^{3} - \frac{3}{x^{2}} \right)^{10}\) binom jayılmasinıń \(x\) qatnashmagan aǵzasın tabıń.
\(\left( 2x^{\ ^{2}} - \frac{b}{2x^{3}} \right)^{10}\) binom jayılmasinıń \(x\) qatnashmagan aǵzasın tabıń.
\(x(1 - x)^{4} + x^{2}(1 + 2x)^{8} + x^{3}(1 + 3x)^{12}\) ańlatpada \(x^{4}\) aldıńdaǵı koeffitsiyentti tabıń.
\((x + 1)^{3} + (x + 1)^{4} + (x + 1)^{5} + ... + (x + 1)^{10}\) ańlatpada \(x^{3}\) aldıńda ǵı koeffitsiyentti tabıń
++++
Tuwrı múyeshli úshmúyeshliktiń biyikligi gipotenuzani uzınlıqları 18 hám 32 sm ga teń bolǵan kesindilerga ajıratadı. Úshmúyeshliktiń maydanı esaplansın.
Tuwrı múyeshli úshmúyeshliktiń biyikligi gipotenuzani uzınlıqları \emph{x} hám \emph{y} ga teń bolǵan kesindilerga ajıratadı. Úshmúyeshliktiń maydanı esaplansın.
Úshmúyeshliktiń ultaniga túsirilgen biyikligi \(h\) ga teń. Úshmúyeshliktiń ultaniga parallel kesindi úshmúyeshliktiń maydanıni teń ekiga bóledi. Úshmúyeshliktiń ushınan usı kesindige shekem bolǵan aralıq tabılsın.
Teń qaptallı úshmúyeshliktiń qaptal tárepi 13 sm , qaptal tárepine túsirilgen biyiklik 5 sm ga teń. Úshmúyeshlik ultaninıń uzunligi tabılsın.
Egerteń qaptallı úshmúyeshliktiń perimetri 32 dm , orta sızıǵı 6 dm ga teń bolsa, Onıń táreplari uzınlıqları tabılsın.
Tuwrı múyeshli úshmúyeshlikda katetlar 7 sm hám 24 sm ga teń. Tuwrı múyeshnıń bissektrisasi túsirilgen. Bu bissektrisa gipotenuzani qanday uzunlikdagi kesindilerga ajıratadı?
Úshmúyeshliktiń perimetri \(4,5dm\) ga teń, bissektrisa bolsa qarama-qarsı tárepni uzınlıqları 6 hám 9 sm ga teń bolǵan kesindilerga ajıratadı. Úshmúyeshliktiń táreplari tabılsın.
Tuwrı múyeshli úshmúyeshlikda katetlarnıń qatnasi 3:2 kabi, biyiklik bolsa gipotenuzani shunday eki kesindiga ajıratadı, olardan birinıń uzunligi ekinshisinen 2 ga úlken. Gipotenuzanıń uzunligi tabılsın.
\emph{ABC} úshmúyeshlik berilgan. Onıń medianalaridan \(\bigtriangleup A_{1}B_{1}C_{1}\) jasalǵan. \(\bigtriangleup ABC\) hám \(\bigtriangleup A_{1}B_{1}C_{1}\) maydanlarınıń qatnasi tabılsın.
Tuwrı múyeshli úshmúyeshliktiń katetlari \(b\) hám \(c\) ga teń. Tuwrı múyesh bissektrisasinıń uzunligi tabılsın.
\(\bigtriangleup ABC\) da \(AB = 2sm,BD\) mediana, \(BD = 1sm\), \(\angle BDA = 30^{{^\circ}}\). Úshmúyeshliktiń maydanı esaplansın.
\(\bigtriangleup ABC\) da \(AB = 3sm,AC = 5sm,\angle BAC = 120^{{^\circ}}.BD\) bissektrisanıń uzunligi tabılsın.
\(\bigtriangleup ABC\) da \(\angle A\) múyesh \(\angle B\) dan eki marta úlken bólib, \(AC = b,AB = c\). \emph{BC} tárepnıń uzunligi tabılsın.
\emph{ABC} úshmúyeshliktiń \(B\) uchidan \emph{AC} tárepiga \emph{BD} kesindi ótkazildi. \emph{BD} kesindi bu úshmúyeshliktiń maydanıni teń ekige bóledi. Eger\(AC = a\) bolsa, \emph{AD} hám \emph{DC} kesindilarnıń uzınlıqlarıni tabıń.
\emph{ABCD} parallelogrammnıń \emph{AD} tárepi \(n\) ta teń bólekke bólingen. Birinchi bólinish noqatsi \(P\) hám \(B\) uch bilan birlashtirilgan. \emph{BP} tuwrı sızıq \emph{AC} dioganaldan Onıń \(\frac{1}{n + 1}\) bólegiga teń \emph{AQ} kesindi ajratishini dálilleń.
Tuwrı múyeshli úshmúyeshliktiń tuwrı burchagi bissektrisasi shu uchdan túsirilgen mediana hám biyiklik arasındaǵı múyeshni ham teń ekige bóliniwini dálilleń.
Ultanlari \(x\) hám 3 bolǵan trapetsiyada diagonallar órtalari arasındaǵı aralıqni \(x\) nıń funksiyasi sifatida ańlatpalań. \(x\) nіnń qanday qiymatida bu aralıq 1 ga teń bóledi?
Tuwrı múyeshli úshmúyeshlik ótkir múyeshlarinıń bisєektrisalari AD hám BK \(AB^{2} = AD \cdot BK\) bolsa, úshmúyeshliktiń múyeshlarini tabıń.
Durıs úshmúyeshliktiń uchlari uchta parallel tuwrı sızıqlarda yotadi. Eger ortadagi tuwrı sızıqdan chekkalardagi tuwrı sızıqlargacha bolǵan aralıq \(a\) hám \(b\) ga teń bolsa, úshmúyeshliktiń tárepini tabıń.
Úshmúyeshliktiń táreplari \(a\) hám \(b\), bissektrisasi \(l_{c} = l\). \(l\) ni bilgan holda Onıń maydanıni tabıń.
\emph{ABC} úshmúyeshliktiń \emph{AB} tárepinda jaylasqan \(N\) noqatdan \(NQ\| AC\) hám \(NP\| BC\) tuwrı sızıqlar túsirilgen. Eger \emph{BNQ} úshmúyeshliktiń maydanı \(S_{1}\) ga, \emph{ANP} úshmúyeshliktiń maydanı \(S_{2}\) ga teńligi ma'lum bolsa, \emph{ABC} úshmúyeshliktiń maydanıni tabıń.
Teń qaptallı úshmúyeshlik ultanidagi múyesh \(\alpha\) ga teń. Shu múyesh uchidan ultanga \(\beta(\beta < \alpha)\) múyesh ostida tuwrı sızıq túsirilgen, u úshmúyeshlikni eki bólekga ajıratadı. Payda bolǵan úshmúyeshliklar maydanlarınıń qatnasini tabıń.
Bir burchagi \(60^{{^\circ}}\) bolǵan úshmúyeshlikka ishley sızılǵan sheńberdiń uriniw noqati shu múyeshke qarama- qarama-qarsı tárepini \(a\) hám \(b\) kesindilerga ajıratadı. Úshmúyeshlik maydanıni tabıń.
Úshmúyeshliktiń ishida olińan noqatdan Onıń táreplariga parallel tuwrı sızıqlar túsirilgen. Ular úshmúyeshlikni 6 bólekga bóledi. Eger payda bolǵan úshmúyeshliklarnıń maydanları \(S_{1},S_{2}\) hám \(S_{3}\) bolsa, berilgan úshmúyeshlik maydanın tabıń.
Parallelogrammnıń táreplari \(a\) hám \(b\), ular arasındaǵı múyesh \(\alpha\). bolsa, paralllelogramm ichki múyeshlari bissektrisalari kesilisiwinen payda bolǵan tórtmúyeshlik maydanın tabıń.
Orayları \(O_{1}\) hám \(O_{2}\) noqatlarda hám radiusi \(R\) bolǵan eki teń sheńberlar sirtlay urinadi. \(l\) tuwrı sızıq bu sheńberlarni A, B, C hám \(D\) noqatlarda shunday kesib ótadiki, \(AB = BC = CD\) bóledi. \(O_{1}ADO_{2}\) tórtmúyeshlik maydanıni tabıń.
\(ABCD(AD\| BC)\) trapetsiya diagonallari \(O\) noqatda kesilisedi. Eger\emph{AOD} úshmúyeshliktiń maydanı \(a^{2}\) ga, \emph{BOC} úshmúyeshliktiń maydanı \(b^{2}\) ga teńligi ma'lum bolsa, trapetsiya maydanın tabıń.
\emph{ABC} úshmúyeshliktiń \emph{AC}, \emph{BC} hám \emph{AB} táreplarida \emph{CMPA}, \emph{BEFC} hám \emph{ADKB} kvadratlar jasalǵan. Eger\(AB = 13\), \(AC = 14,BC = 15\) ekanligi ma'lum bolsa, \emph{DKEFMP} altimúyeshliktıń maydanın tabıń.
Eki birdey radiusli dóńgeleklar sonday jaylasqan, olardıń orayları arasındaǵı aralıq radiusqa teń. Dóńgeleklar kesilisken bólegi maydanınıń, kesilisken bólegine ishley sızılǵan kvadrat maydanına qatnasin tabıń.
Teń qaptallı úshmúyeshliktiń maydanı \(S\) ga teń. Qaptal táreplariga túsirilgen medianalari arasındaǵı múyesh \(\alpha\) ga teń. Úshmúyeshlik ultanini tabıń.
Úshmúyeshliktiń a, b hám \(c\) táreplari arifmetik progressiya quraydı. \(ac = 6Rr\) bolıwın dálilleń. Bu yerda \(R\) hám \(r\) sirtlay hám ishki sızılǵan sheńberlernıń radiuslari.
Teń qaptallı \(ABC(AB = BC)\) úshmúyeshlikda \emph{AD} bissektrisa túsirilgen. Eger\(S_{ABD} = S_{1},S_{\bigtriangleup ADC} = S_{2}\) bolsa, \emph{AC} ni tabıń.
++++
Ya. Bernulli teńsizligi. Eger\(x \geq - 1\) bolsa, onda qálegen natural \(n\) sanı ushın \((1 + x)^{n} \geq 1 + nx\) teńsizlik orinli bolıwın dálilleń.
Durıs úshmúyeshliktiń tárepi a ga teń. Tárepini diametr deb esaplap dóńgelek jasalǵan. Úshmúyeshliktiń usı dóńgelekten sirtindaǵi bólegi maydanın tabıń.
\(R\) radiusli dóńgelekga bitta umumiy uchga ega bolǵan durıs úshmúyeshlik hám kvadrat ishley sızılǵan. Olardıń kesilisken bóleginıń maydanıni tabıń.
Eger \(S\) úshmúyeshliktiń maydanı, \(b\) hám \(c\) onıń táreplari bolsa, \(S \leq \frac{b^{2} + c^{2}}{4}\) bolıwın dálilleń.
Eger a,b,c - oń sanlar bolsa, tómendegi teńsizlikti dálilleń: \(\sqrt{\mathbf{a}^{\mathbf{2}}\mathbf{+ ab +}\mathbf{b}^{\mathbf{2}}}\mathbf{+}\sqrt{\mathbf{b}^{\mathbf{2}}\mathbf{+ bc +}\mathbf{c}^{\mathbf{2}}}\mathbf{>}\sqrt{\mathbf{a}^{\mathbf{2}}\mathbf{+ ac +}\mathbf{c}^{\mathbf{2}}}\)
Eger a,b - oń sanlar bolsa, tómendegi teńsizlikti dálilleń: \(\sqrt[3]{\frac{a}{b}} + \sqrt[3]{\frac{b}{a}} \leq \sqrt[3]{2(a + b)\left( \frac{1}{a} + \frac{1}{b} \right)}\)