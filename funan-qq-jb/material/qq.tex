Kóplikler hám olar ústinde ámeller.
====
Kópliktiń quwatlıǵı hám onıń qásiyetleri.
====
Metrikalıq keńislik hám oǵan mısallar.
====
Metrikalıq keńisliklerde ashıq hám tuyıq kóplikler.
====
Kompakt metrikalıq keńislikler.
====
Metrikalıq keńisliklerdiń úzliksiz sáwlelendiriwleri.
++++
Egorov teoreması.
====
Tegislikte elementar kóplikler hám olardıń ólshewi.
====
Ólshewli funkciyalar hám olardıń qásiyetleri.
====
Lebeg hám Riss teoremaları.
++++
\(A = \{(x,y) \in \mathbb{R}^{2}:\ x = y\},\ B = \{(x,y) \in \mathbb{R}^{2}:\ |x| + |y| \leq 1\}\), \(A,\ B,\ A \cup B,\ A \cap B,\ A \backslash B,\ B \backslash A,\ A \bigtriangleup B\) kópliklerin anıqlań hám súwretleń.
====
\(A = \{(x,y) \in \mathbb{R}^{2}:\ y = - x\},\ B = \{(x,y) \in \mathbb{R}^{2}:\ x^{2} + y^{2} \leq 1\}\), \(A,\ B,\ A \cup B,\ A \cap B,\ A \backslash B,\ B \backslash A,\ A \bigtriangleup B\) kópliklerin anıqlań hám súwretleń.
====
\(A = \{(x,y) \in \mathbb{R}^{2}:\ y = x^{2}\},\) \(\ B = \{(x,y) \in \mathbb{R}^{2}:\ x^{2} + (y - 1)^{2} \leq 1\}\), \(A,\ B,\ A \cup B,\ A \cap B,\ A \backslash B,\ B \backslash A,\ A \bigtriangleup B\) kópliklerin anıqlań hám súwretleń.
====
\(A = \{(x,y) \in \mathbb{R}^{2}:\ xy \geq 0\},\) \(\ B = \{(x,y) \in \mathbb{R}^{2}:\ x^{2} + y^{2} \geq 1\}\), \(A,\ B,\ A \cup B,\ A \cap B,\ A \backslash B,\ B \backslash A,\ A \bigtriangleup B\) kópliklerin anıqlań hám súwretleń.
====
\(A = \{(x,y) \in \mathbb{R}^{2}:\ y = - x^{2}\},\ B = \{(x,y) \in \mathbb{R}^{2}:\ (x + 1)^{2} + (y + 1)^{2} \leq 1\}\), \(A,\ B,\ A \cup B,\ A \cap B,\ A \backslash B,\ B \backslash A,\ A \bigtriangleup B\) kópliklerin anıqlań hám súwretleń.
====
\(A = \{(x,y) \in \mathbb{R}^{2}:\ xy \leq 0\},\ B = \{(x,y) \in \mathbb{R}^{2}:\ |x| + |y| \geq 1\}\), \(A,\ B,\ A \cup B,\ A \cap B,\ A \backslash B,\ B \backslash A,\ A \bigtriangleup B\) kópliklerin anıqlań hám súwretleń.
====
\(A = \{(x,y) \in \mathbb{R}^{2}:\ x \geq y\},\) \(\ B = \{(x,y) \in \mathbb{R}^{2}:\ 9x^{2} + y^{2} \leq 36\}\),\(A,\ B,\ A \cup B,\ A \cap B,\ A \backslash B,\ B \backslash A,\ A \bigtriangleup B\) kópliklerin anıqlań hám súwretleń.
====
\(A = \{(x,y) \in \mathbb{R}^{2}:\ x \leq y\},\) \(\ B = \{(x,y) \in \mathbb{R}^{2}:\ 4x^{2} + 9y^{2} \geq 36\}\), \(A,\ B,\ A \cup B,\ A \cap B,\ A \backslash B,\ B \backslash A,\ A \bigtriangleup B\) kópliklerin anıqlań hám súwretleń.
====
\(A = \{(x,y) \in \mathbb{R}^{2}:\ \max \{|x|,|y|\} = 1\},\) \(\ B = \{(x,y) \in \mathbb{R}^{2}:\ x^{2} + y^{2} \leq 1\}\), \(A,\ B,\ A \cup B,\ A \cap B,\ A \backslash B,\ B \backslash A,\ A \bigtriangleup B\) kópliklerin anıqlań hám súwretleń.
====
\(A = \{(x,y) \in \mathbb{R}^{2}:\ \max \{|x|,|y|\} \leq 2\},\) \(\ B = \{(x,y) \in \mathbb{R}^{2}:\ y \geq x + 1\}\), \(A,\ B,\ A \cup B,\ A \cap B,\ A \backslash B,\ B \backslash A,\ A \bigtriangleup B\) kópliklerin anıqlań hám súwretleń.
====
\(A = \{(x,y) \in \mathbb{R}^{2}:\ y \geq x^{2}\},\) \(\ B = \{(x,y) \in \mathbb{R}^{2}:\ y \leq 4 - x^{2}\}\), \(A,\ B,\ A \cup B,\ A \cap B,\ A \backslash B,\ B \backslash A,\ A \bigtriangleup B\) kópliklerin anıqlań hám súwretleń.
====
\(A = \{(x,y) \in \mathbb{R}^{2}:\ x = - y\},\) \(\ B = \{(x,y) \in \mathbb{R}^{2}:\ |x| + |y| \leq 2\}\), \(A,\ B,\ A \cup B,\ A \cap B,\ A \backslash B,\ B \backslash A,\ A \bigtriangleup B\) kópliklerin anıqlań hám súwretleń.
====
\(A = \{(x,y) \in \mathbb{R}^{2}:\ |x| + |y| \geq 3\},\) \(\ B = \{(x,y) \in \mathbb{R}^{2}:\ \max \{|x|,|y|\} \leq 2\}\), \(A,\ B,\ A \cup B,\ A \cap B,\ A \backslash B,\ B \backslash A,\ A \bigtriangleup B\) kópliklerin anıqlań hám súwretleń.
====
\(A = \{(x,y) \in \mathbb{R}^{2}:\ y = - x^{2}\},\) \(\ B = \{(x,y) \in \mathbb{R}^{2}:\ (x - 1)^{2} + (y - 1)^{2} \leq 1\}\), \(A,\ B,\ A \cup B,\ A \cap B,\ A \backslash B,\ B \backslash A,\ A \bigtriangleup B\) kópliklerin anıqlań hám súwretleń.
====
\(A = \{(x,y) \in \mathbb{R}^{2}:\ xy \leq 0\},\) \(\ B = \{(x,y) \in \mathbb{R}^{2}:\ x^{2} + (y + 1)^{2} \geq 1\}\), \(A,\ B,\ A \cup B,\ A \cap B,\ A \backslash B,\ B \backslash A,\ A \bigtriangleup B\) kópliklerin anıqlań hám súwretleń.
====
\(A = \{(x,y) \in \mathbb{R}^{2}:\ xy \leq 0\},\) \(\ B = \{(x,y) \in \mathbb{R}^{2}:\ x^{2} + y^{2} \geq 4\}\), \(A,\ B,\ A \cup B,\ A \cap B,\ A \backslash B,\ B \backslash A,\ A \bigtriangleup B\) kópliklerin anıqlań hám súwretleń.
====
\(\ A = \{(x,y) \in \mathbb{R}^{2}:\ y = x^{2}\},\) \(\ B = \{(x,y) \in \mathbb{R}^{2}:\ (x - 1)^{2} + (y - 1)^{2} \leq 4\}\), \(A,\ B,\ A \cup B,\ A \cap B,\ A \backslash B,\ B \backslash A,\ A \bigtriangleup B\) kópliklerin anıqlań hám súwretleń.
====
\(A = \{(x,y) \in \mathbb{R}^{2}:\ x^{2} = y\},\) \(\ B = \{(x,y) \in \mathbb{R}^{2}:\ x^{2} + y^{2} \geq 4\}\), \(A,\ B,\ A \cup B,\ A \cap B,\ A \backslash B,\ B \backslash A,\ A \bigtriangleup B\) kópliklerin anıqlań hám súwretleń.
====
\(A = \{(x,y) \in \mathbb{R}^{2}:\ xy \geq 0\},\) \(\ B = \{(x,y) \in \mathbb{R}^{2}:\ |x| + |y - 2| \geq 1\}\), \(A,\ B,\ A \cup B,\ A \cap B,\ A \backslash B,\ B \backslash A,\ A \bigtriangleup B\) kópliklerin anıqlań hám súwretleń.
====
\(A = \{(x,y) \in \mathbb{R}^{2}:\ x = - y\},\) \(\ B = \{(x,y) \in \mathbb{R}^{2}:\ (x - 2)^{2} + (y + 3)^{2} \geq 1\}\), \(A,\ B,\ A \cup B,\ A \cap B,\ A \backslash B,\ B \backslash A,\ A \bigtriangleup B\) kópliklerin anıqlań hám súwretleń.
====
\(A = \{(x,y) \in \mathbb{R}^{2}:\ x \leq y\},\) \(\ B = \{(x,y) \in \mathbb{R}^{2}:\ 9x^{2} + y^{2} \leq 9\}\), \(A,\ B,\ A \cup B,\ A \cap B,\ A \backslash B,\ B \backslash A,\ A \bigtriangleup B\) kópliklerin anıqlań hám súwretleń.
====
\(A = \{(x,y) \in \mathbb{R}^{2}:\ x \geq y\},\) \(\ B = \{(x,y) \in \mathbb{R}^{2}:\ x^{2} + 4y^{2} \geq 4\}\), \(A,\ B,\ A \cup B,\ A \cap B,\ A \backslash B,\ B \backslash A,\ A \bigtriangleup B\) kópliklerin anıqlań hám súwretleń.
====
\(A = \{(x,y) \in \mathbb{R}^{2}:\ |x| + |y| \leq 2\},\ B = \{(x,y) \in \mathbb{R}^{2}:\ 9x^{2} + y^{2} \geq 9\}\),\(A,\ B,\ A \cup B,\ A \cap B,\ A \backslash B,\ B \backslash A,\ A \bigtriangleup B\) kópliklerin anıqlań hám súwretleń.
====
\(A = \{(x,y) \in \mathbb{R}^{2}:\ \max \{|x|,|y|\} \leq 2\},\ B = \{(x,y) \in \mathbb{R}^{2}:\ x^{2} + 1 \leq y\}\), \(A,\ B,\ A \cup B,\ A \cap B,\ A \backslash B,\ B \backslash A,\ A \bigtriangleup B\) kópliklerin anıqlań hám súwretleń.
====
\(A = \{(x,y) \in \mathbb{R}^{2}:\ \max \{|x|,|y|\} \leq 2\},\ B = \{(x,y) \in \mathbb{R}^{2}:\ 4 - x^{2} \geq y\}\), \(A,\ B,\ A \cup B,\ A \cap B,\ A \backslash B,\ B \backslash A,\ A \bigtriangleup B\) kópliklerin anıqlań hám súwretleń.
++++
\(\lbrack 0;6\rbrack\) hám \(\lbrack 0;5) \cup \lbrack 7;8\rbrack\) kóplikleri arasında óz ara bir mánisli sáykeslik ornatıń.
====
\((0;6\rbrack\) hám \((2;4) \cup \lbrack 7;11\rbrack\) kóplikleri arasında óz ara bir mánisli sáykeslik ornatıń.
====
\(\lbrack 2;6\rbrack\) hám \(\lbrack 2;4) \cup \lbrack 11;13\rbrack\) kóplikleri arasında óz ara bir mánisli sáykeslik ornatıń.
====
\(\lbrack 1;6\rbrack\) hám \(\lbrack 1;4) \cup \lbrack 7;9\rbrack\) kóplikleri arasında óz ara bir mánisli sáykeslik ornatıń.
====
\(\lbrack 3;\ 7\rbrack\) hám \(\lbrack 0;\ 2) \cup \lbrack 6;\ 8\rbrack\) kóplikleri arasında óz ara bir mánisli sáykeslik ornatıń.
====
\(\lbrack 1;7\rbrack\) hám \(\lbrack - 1;4) \cup \lbrack 6;7\rbrack\) kóplikleri arasında óz ara bir mánisli sáykeslik ornatıń.
====
\(( - 1;5\rbrack\) hám \(( - 1;\ 1\rbrack \cup (3;\ 7\rbrack\) kóplikleri arasında óz ara bir mánisli sáykeslik ornatıń.
====
\(\lbrack - 2;\ 4\rbrack\) hám \(\lbrack - 2;1) \cup \lbrack 2;5\rbrack\) kóplikleri arasında óz ara bir mánisli sáykeslik ornatıń.
====
\(\lbrack 1;\ 5\rbrack\) hám \(\lbrack 1;\ 2) \cup \lbrack 7;10\rbrack\) kóplikleri arasında óz ara bir mánisli sáykeslik ornatıń.
====
\(\lbrack 2;\ 5\rbrack\) hám \(\lbrack 0;1) \cup \lbrack 3;\ 5\rbrack\) kóplikleri arasında óz ara bir mánisli sáykeslik ornatıń.
====
\((3;6\rbrack\) hám \(( - 3; - 1) \cup \lbrack 2;3\rbrack\) kóplikleri arasında óz ara bir mánisli sáykeslik ornatıń.
====
\(\lbrack 2;6)\) hám \(\lbrack - 2;1) \cup \lbrack 4;5)\) kóplikleri arasında óz ara bir mánisli sáykeslik ornatıń.
====
\(\lbrack 0;5\rbrack\) hám \(\lbrack - 2;2) \cup \lbrack 3;4\rbrack\) kóplikleri arasında óz ara bir mánisli sáykeslik ornatıń.
====
\(\lbrack 0;4)\) hám \(\lbrack - 2;0) \cup \lbrack 7;9)\) kóplikleri arasında óz ara bir mánisli sáykeslik ornatıń.
====
\(\lbrack - 2;3)\) hám \(\lbrack - 3;1) \cup \lbrack 2;3)\) kóplikleri arasında óz ara bir mánisli sáykeslik ornatıń.
====
\(\lbrack 0;\ 3)\) hám \(\lbrack 2;4) \cup \lbrack 5;6)\) kóplikleri arasında óz ara bir mánisli sáykeslik ornatıń.
====
\(( - 2;6\rbrack\) hám \(( - 3; - 1) \cup \lbrack 1;7\rbrack\) kóplikleri arasında óz ara bir mánisli sáykeslik ornatıń.
====
\(\lbrack - 1;\ 3\rbrack\) hám \(\lbrack - 4; - 1) \cup \lbrack 2;3\rbrack\) kóplikleri arasında óz ara bir mánisli sáykeslik ornatıń.
====
\(\lbrack - 3;1\rbrack\) hám \(\lbrack - 2;1) \cup \lbrack 4;5\rbrack\) kóplikleri arasında óz ara bir mánisli sáykeslik ornatıń.
====
\(( - 4;1\rbrack\) hám \(( - 1;3) \cup \lbrack 8;9\rbrack\) kóplikleri arasında óz ara bir mánisli sáykeslik ornatıń.
====
\(\lbrack - 2;5\rbrack\) hám \(\lbrack 2;4\rbrack \cup (7;12\rbrack\) kóplikleri arasında óz ara bir mánisli sáykeslik ornatıń.
====
\((1;\ 7\rbrack\) hám \((2;4) \cup \lbrack 9;13\rbrack\) kóplikleri arasında óz ara bir mánisli sáykeslik ornatıń.
====
\(\lbrack - 3;\ 2\rbrack\) hám \(\lbrack 2;4) \cup \lbrack 5;8\rbrack\) kóplikleri arasında óz ara bir mánisli sáykeslik ornatıń.
====
\(\lbrack - 3;\ 7\rbrack\) hám \(\lbrack 2;5) \cup \lbrack 8;15\rbrack\) kóplikleri arasında óz ara bir mánisli sáykeslik ornatıń.
====
\(( - 3;\ 4\rbrack\) hám \((1;4\rbrack \cup (6;10\rbrack\) kóplikleri arasında óz ara bir mánisli sáykeslik ornatıń.
====
\(\lbrack - 3;\ 3)\) hám \(\lbrack 0;4) \cup \lbrack 7;9)\) kóplikleri arasında óz ara bir mánisli sáykeslik ornatıń.
====
\(\lbrack - 1;\ 5)\) hám \(\lbrack - 1;4) \cup \lbrack 7;8)\) kóplikleri arasında óz ara bir mánisli sáykeslik ornatıń.
====
\(\lbrack 0;5)\) hám \(\lbrack - 2;0) \cup \lbrack 1;4)\) kóplikleri arasında óz ara bir mánisli sáykeslik ornatıń.
====
\(\lbrack - 2;4)\) hám \(\lbrack 0;4) \cup \lbrack 5;7)\) kóplikleri arasında óz ara bir mánisli sáykeslik ornatıń.
====
\(\lbrack - 2;\ 2)\) hám \(\lbrack 1;3\rbrack \cup (5;7)\) kóplikleri arasında óz ara bir mánisli sáykeslik ornatıń.
====
\(\lbrack - 1;\ 7)\) hám \(\lbrack - 2;4) \cup \lbrack 7;9)\) kóplikleri arasında óz ara bir mánisli sáykeslik ornatıń.
====
\(\lbrack 2;\ 7)\) hám \(\lbrack - 2; - 1) \cup \lbrack 2;4)\) kóplikleri arasında óz ara bir mánisli sáykeslik ornatıń.
====
\(\lbrack - 4;\ 1)\) hám \(\lbrack - 3; - 1) \cup \lbrack 3;6)\) kóplikleri arasında óz ara bir mánisli sáykeslik ornatıń.
====
\(\lbrack - 2;\ 1)\) hám \(\lbrack 1;2) \cup \lbrack 3;5)\) kóplikleri arasında óz ara bir mánisli sáykeslik ornatıń.
====
\(\lbrack - 4;0)\) hám \(\lbrack 0;3) \cup \lbrack 5;6)\) kóplikleri arasında óz ara bir mánisli sáykeslik ornatıń.
++++
\(\lbrack 0,\ 1\rbrack\) kesindide jaylasqan sanlardıń onlıq bólshek jazılıwında \(1\) cifrı qatnaspaǵan barlıq sanlar kópliginiń Lebeg ólshewin tabıń.
====
\(\lbrack 0,\ 2\rbrack\) kesindide jaylasqan sanlardıń onlıq bólshek jazılıwında \(2\) cifrı qatnaspaǵan barlıq sanlar kópliginiń Lebeg ólshewin tabıń.
====
\(\lbrack 1,\ 3\rbrack\) kesindide jaylasqan sanlardıń onlıq bólshek jazılıwında \(3\) cifrı qatnaspaǵan barlıq sanlar kópliginiń Lebeg ólshewin tabıń.
====
\(\lbrack 2,\ 4\rbrack\) kesindide jaylasqan sanlardıń onlıq bólshek jazılıwında \(4\) cifrı qatnaspaǵan barlıq sanlar kópliginiń Lebeg ólshewin tabıń.
====
\(\lbrack 3,\ 5\rbrack\) kesindide jaylasqan sanlardıń onlıq bólshek jazılıwında \(5\) cifrı qatnaspaǵan barlıq sanlar kópliginiń Lebeg ólshewin tabıń.
====
\(\lbrack 4,\ 6\rbrack\) kesindide jaylasqan sanlardıń onlıq bólshek jazılıwında \(6\) cifrı qatnaspaǵan barlıq sanlar kópliginiń Lebeg ólshewin tabıń.
====
\(\lbrack 5,\ 7\rbrack\) kesindide jaylasqan sanlardıń onlıq bólshek jazılıwında \(7\) cifrı qatnaspaǵan barlıq sanlar kópliginiń Lebeg ólshewin tabıń.
====
\(\lbrack 6,\ 8\rbrack\) kesindide jaylasqan sanlardıń onlıq bólshek jazılıwında \(8\) cifrı qatnaspaǵan barlıq sanlar kópliginiń Lebeg ólshewin tabıń.
====
\(\lbrack 7,\ 9\rbrack\) kesindide jaylasqan sanlardıń onlıq bólshek jazılıwında \(9\) cifrı qatnaspaǵan barlıq sanlar kópliginiń Lebeg ólshewin tabıń.
====
\(\lbrack 8,\ 10\rbrack\) kesindide jaylasqan sanlardıń onlıq bólshek jazılıwında \(0\) cifrı qatnaspaǵan barlıq sanlar kópliginiń Lebeg ólshewin tabıń.
====
\(\lbrack 3,\ 4\rbrack\) kesindide jaylasqan sanlardıń onlıq bólshek jazılıwında \(1\) cifrı qatnaspaǵan barlıq sanlar kópliginiń Lebeg ólshewin tabıń.
====
\(\lbrack 0,\ 2\rbrack\) kesindide jaylasqan sanlardıń onlıq bólshek jazılıwında \(3\) cifrı qatnaspaǵan barlıq sanlar kópliginiń Lebeg ólshewin tabıń.
====
\(\lbrack 1,\ 3\rbrack\) kesindide jaylasqan sanlardıń onlıq bólshek jazılıwında \(4\) cifrı qatnaspaǵan barlıq sanlar kópliginiń Lebeg ólshewin tabıń.
====
\(\lbrack 2,\ 4\rbrack\) kesindide jaylasqan sanlardıń onlıq bólshek jazılıwında \(5\) cifrı qatnaspaǵan barlıq sanlar kópliginiń Lebeg ólshewin tabıń.
====
\(\lbrack 3,\ 5\rbrack\) kesindide jaylasqan sanlardıń onlıq bólshek jazılıwında \(6\) cifrı qatnaspaǵan barlıq sanlar kópliginiń Lebeg ólshewin tabıń.
====
\(\lbrack 4,\ 6\rbrack\) kesindide jaylasqan sanlardıń onlıq bólshek jazılıwında \(7\) cifrı qatnaspaǵan barlıq sanlar kópliginiń Lebeg ólshewin tabıń.
====
\(\lbrack 5,\ 7\rbrack\) kesindide jaylasqan sanlardıń onlıq bólshek jazılıwında \(8\) cifrı qatnaspaǵan barlıq sanlar kópliginiń Lebeg ólshewin tabıń.
====
\(\lbrack 6,\ 8\rbrack\) kesindide jaylasqan sanlardıń onlıq bólshek jazılıwında \(9\) cifrı qatnaspaǵan barlıq sanlar kópliginiń Lebeg ólshewin tabıń.
====
\(\lbrack 7,\ 9\rbrack\) kesindide jaylasqan sanlardıń onlıq bólshek jazılıwında \(0\) cifrı qatnaspaǵan barlıq sanlar kópliginiń Lebeg ólshewin tabıń.
====
\(\lbrack 8,\ 10\rbrack\) kesindide jaylasqan sanlardıń onlıq bólshek jazılıwında \(6\) cifrı qatnaspaǵan barlıq sanlar kópliginiń Lebeg ólshewin tabıń.
++++
\(\int_{E}^{}f(x)d\mu\) Lebeg integralın esaplań, \(E = \lbrack 0,\ 1\rbrack\), \(f(x) = \left\{ \begin{matrix}
\frac{1}{(x + 1)^{3}}\ x \in \mathbb{I} \cap \lbrack 0,\ 1\rbrack \\
7x,\ \ x\mathbb{\in Q}
\end{matrix} \right.\ \)
====
\(\int_{E}^{}f(x)d\mu\) Lebeg integralın esaplań, \(E = \lbrack 0,\ 1\rbrack\), \(f(x) = \left\{ \begin{matrix}
\frac{1}{\sqrt{x}},\ x \in \mathbb{I} \cap \lbrack 0,\ 1\rbrack \\
\sin x,\ x\mathbb{\in Q}
\end{matrix} \right.\ \)
====
\(\int_{E}^{}f(x)d\mu\) Lebeg integralın esaplań, \(f(x) = \left\{ \begin{matrix}
\frac{x^{2}}{(x + 2)(x + 4)},\ x \in \mathbb{I} \cap \lbrack 2,\ 4\rbrack \\
4x^{3},\ x\mathbb{\in Q \cap}\lbrack 2,\ 4\rbrack,\ E = \lbrack 2,\ 4\rbrack
\end{matrix} \right.\ \)
====
\(\int_{E}^{}f(x)d\mu\) Lebeg integralın esaplań, \(f(x) = \left\{ \begin{matrix}
\frac{x^{2}}{(x - 5)(x - 6)},\ x \in \mathbb{I} \cap \lbrack 0,\ 4\rbrack \\
3x^{2} - 2,\ x\mathbb{\in Q \cap}\lbrack 0,\ 4\rbrack,\ E = \lbrack 0,\ 4\rbrack
\end{matrix} \right.\ \)
====
\(\int_{E}^{}f(x)d\mu\) Lebeg integralın esaplań, \(f(x) = \left\{ \begin{matrix}
\frac{x^{2}}{(x + 2)(x + 4)},\ x \in \mathbb{I} \cap \lbrack 0,\ 4\rbrack \\
3x^{2} - 2,\ x\mathbb{\in Q \cap}\lbrack 0,\ 4\rbrack,\ E = \lbrack 0,\ 4\rbrack
\end{matrix} \right.\ \)
====
\(\int_{E}^{}f(x)d\mu\) Lebeg integralın esaplań, \(f(x) = \left\{ \begin{matrix}
\frac{x^{2}}{(x - 2)(x - 4)},\ x \in \mathbb{I} \cap \lbrack - 1;1\rbrack \\
3x^{2} - 2,\ x\mathbb{\in Q \cap}\lbrack - 1;1\rbrack,\ E = \lbrack - 1;1\rbrack
\end{matrix} \right.\ \)
====
\(\int_{E}^{}f(x)d\mu\) Lebeg integralın esaplań, \(f(x) = \left\{ \begin{matrix}
\frac{x^{2}}{(x - 2)(x - 4)},\ x \in \mathbb{I} \cap \lbrack - 4; - 1\rbrack \\
3x^{2} - 2,\ x\mathbb{\in Q \cap}\lbrack - 4; - 1\rbrack,E = \lbrack - 4; - 1\rbrack
\end{matrix} \right.\ \)
====
\(\int_{E}^{}f(x)d\mu\) Lebeg integralın esaplań, \(f(x) = \left\{ \begin{matrix}
\frac{x^{2}}{(x + 3)(x + 2)},\ x \in \mathbb{I} \cap \lbrack 2,\ 4\rbrack \\
3x^{2} - 2,\ x\mathbb{\in Q \cap}\lbrack 2,\ 4\rbrack,\ E = \lbrack 2,\ 4\rbrack
\end{matrix} \right.\ \)
====
\(\int_{E}^{}f(x)d\mu\) Lebeg integralın esaplań, \(f(x) = \left\{ \begin{matrix}
\frac{x^{2}}{(x - 5)(x - 6)},\ x \in \mathbb{I} \cap \lbrack 0,\ 4\rbrack \\
3x^{2} - 2,\ x\mathbb{\in Q \cap}\lbrack 0,\ 4\rbrack,\ E = \lbrack 0,\ 4\rbrack
\end{matrix} \right.\ \)
====
\(\int_{E}^{}f(x)d\mu\) Lebeg integralın esaplań, \(f(x) = \left\{ \begin{matrix}
\frac{x^{2}}{(x - 5)(x - 7)},\ x \in \mathbb{I} \cap \lbrack 1,\ 4\rbrack \\
3x^{2} - 2,\ x\mathbb{\in Q \cap}\lbrack 1,\ 4\rbrack,\ E = \lbrack 1,\ 4\rbrack
\end{matrix} \right.\ \)
++++
Tómende berilgenler boyınsha \(x,y \in X\) elementler arasındaǵı aralıqtı tabıń: \(X = C\lbrack 0,\pi\rbrack,\ \rho(x,y) = \max _{0 \leq t \leq \pi}|x(t) - y(t)|,x(t) = \sin2t,\ y = \cos4t\).
====
Tómende berilgenler boyınsha \(x,y \in X\) elementler arasındaǵı aralıqtı tabıń: \(X = C\left\lbrack \frac{\pi}{6};\ \frac{\pi}{3} \right\rbrack,\ \rho(x,y) = \max _{\frac{\pi}{6} \leq t \leq \frac{\pi}{3}}|x(t) - y(t)|,x(t) = ctg(t + \pi/6),\ y = tg\ t\)
====
Tómende berilgenler boyınsha \(x,y \in X\) elementler arasındaǵı aralıqtı tabıń: \(X = C\lbrack 0;\ \pi/4\rbrack,\ \rho(x,y) = \max _{0 \leq t \leq \pi/4}|x(t) - y(t)|,x(t) = \sin4t,\ y = \cos2t\)
====
Tómende berilgenler boyınsha \(x,y \in X\) elementler arasındaǵı aralıqtı tabıń: \(X = C\left\lbrack \frac{\pi}{6};\ \frac{\pi}{4} \right\rbrack,\ \rho(x,y) = \max _{\frac{\pi}{6} \leq t \leq \frac{\pi}{4}}|x(t) - y(t)|,x(t) = ctgt,\ y = tg(\ 2t - \frac{\pi}{6})\)
====
Tómende berilgenler boyınsha \(x,y \in X\) elementler arasındaǵı aralıqtı tabıń: \(X = C\lbrack 0;\ \pi/6\rbrack,\ \rho(x,y) = \max _{0 \leq t \leq \pi/6}|x(t) - y(t)|,x(t) = \sin3t,\ y = \cos t\)
====
Tómende berilgenler boyınsha \(x,y \in X\) elementler arasındaǵı aralıqtı tabıń: \(X = C\left\lbrack \frac{\pi}{6};\ \frac{\pi}{4} \right\rbrack,\ \rho(x,y) = \max _{\frac{\pi}{6} \leq t \leq \frac{\pi}{4}}|x(t) - y(t)|,x(t) = ctg(2t - \pi/6),\ y = tg(\ 2t - \pi/6)\)
====
Tómende berilgenler boyınsha \(x,y \in X\) elementler arasındaǵı aralıqtı tabıń: \(X = C\lbrack 0;\ \pi/4\rbrack,\ \rho(x,y) = \max _{0 \leq t \leq \pi/4}|x(t) - y(t)|,x(t) = \sin t,\ y = \cos3t\)
====
Tómende berilgenler boyınsha \(x,y \in X\) elementler arasındaǵı aralıqtı tabıń: \(X = C\left\lbrack \frac{\pi}{4};\ \frac{\pi}{2} \right\rbrack,\ \rho(x,y) = \max _{\frac{\pi}{4} \leq t \leq \frac{\pi}{2}}|x(t) - y(t)|,x(t) = ctg(2t - \pi/6),\ y = tg(\ t - \pi/6)\)
====
Tómende berilgenler boyınsha \(x,y \in X\) elementler arasındaǵı aralıqtı tabıń: \(X = C\lbrack 0;\ \pi/3\rbrack,\ \rho(x,y) = \max _{0 \leq t \leq \pi/3}|x(t) - y(t)|,x(t) = \sin t,\ y = \cos5t\)
====
Tómende berilgenler boyınsha \(x,y \in X\) elementler arasındaǵı aralıqtı tabıń: \(X = C\left\lbrack \frac{\pi}{4};\ \frac{\pi}{3} \right\rbrack,\ \rho(x,y) = \max _{\frac{\pi}{4} \leq t \leq \frac{\pi}{3}}|x(t) - y(t)|,x(t) = ctg(2t + \pi/6),\ y = tg(\ t - \pi/6)\)
++++
\(A\ hám\ B\) kóplikleri arasında óz ara bir mánisli sáykeslik anıqlań. \(A = ( - 1;3)\), \(B = \lbrack 0;9\rbrack\).
====
\(A\ hám\ B\) kóplikleri arasında óz ara bir mánisli sáykeslik anıqlań. \(A = ( - 1;4)\), \(B = \lbrack 2;12)\).
====
\(A\ hám\ B\) kóplikleri arasında óz ara bir mánisli sáykeslik anıqlań. \(A = ( - 2;3\rbrack\), \(B = \lbrack - 2;8\rbrack\).
====
\(A\ hám\ B\) kóplikleri arasında óz ara bir mánisli sáykeslik anıqlań. \(A = \lbrack - 1;4)\), \(B = \lbrack - 1;7\rbrack\).
====
\(A\ hám\ B\) kóplikleri arasında óz ara bir mánisli sáykeslik anıqlań. \(A = \lbrack - 2;4\rbrack\), \(B = ( - 1;9)\).
====
\(A\ hám\ B\) kóplikleri arasında óz ara bir mánisli sáykeslik anıqlań. \(A = ( - 3;3)\), \(B = \lbrack - 1;9\rbrack\).
====
\(A\ hám\ B\) kóplikleri arasında óz ara bir mánisli sáykeslik anıqlań. \(A = ( - 2;4)\), \(B = \lbrack 2;10)\).
====
\(A\ hám\ B\) kóplikleri arasında óz ara bir mánisli sáykeslik anıqlań. \(A = ( - 4;6\rbrack\), \(B = \lbrack - 2;6\rbrack\).
====
\(A\ hám\ B\) kóplikleri arasında óz ara bir mánisli sáykeslik anıqlań. \(A = \lbrack - 1;7)\), \(B = \lbrack - 3;9\rbrack\).
====
\(A\ hám\ B\) kóplikleri arasında óz ara bir mánisli sáykeslik anıqlań. \(A = \lbrack - 2;4\rbrack\), \(B = ( - 5;5)\).
====
\(A\ hám\ B\) kóplikleri arasında óz ara bir mánisli sáykeslik anıqlań. \(A = ( - 5;3)\), \(B = \lbrack - 2;8\rbrack\).
====
\(A\ hám\ B\) kóplikleri arasında óz ara bir mánisli sáykeslik anıqlań. \(A = ( - 3;4)\), \(B = \lbrack - 2;10)\).
====
\(A\ hám\ B\) kóplikleri arasında óz ara bir mánisli sáykeslik anıqlań. \(A = ( - 4;3\rbrack\), \(B = \lbrack - 4;10\rbrack\).
====
\(A\ hám\ B\) kóplikleri arasında óz ara bir mánisli sáykeslik anıqlań. \(A = \lbrack - 5;4)\), \(B = \lbrack - 3;11\rbrack\).
====
\(A\ hám\ B\) kóplikleri arasında óz ara bir mánisli sáykeslik anıqlań. \(A = \lbrack - 4;4\rbrack\), \(B = ( - 11;3)\).
====
\(A\ hám\ B\) kóplikleri arasında óz ara bir mánisli sáykeslik anıqlań. \(A = ( - 5;3)\), \(B = \lbrack - 10;3\rbrack\).
====
\(A\ hám\ B\) kóplikleri arasında óz ara bir mánisli sáykeslik anıqlań. \(A = ( - 3;5)\), \(B = \lbrack - 8;6)\).
====
\(A\ hám\ B\) kóplikleri arasında óz ara bir mánisli sáykeslik anıqlań. \(A = ( - 5;1\rbrack\), \(B = \lbrack - 4;6\rbrack\).
====
\(A\ hám\ B\) kóplikleri arasında óz ara bir mánisli sáykeslik anıqlań. \(A = \lbrack - 7;3)\), \(B = \lbrack - 5;7\rbrack\).
====
\(A\ hám\ B\) kóplikleri arasında óz ara bir mánisli sáykeslik anıqlań. \(A = \lbrack - 6;2\rbrack\), \(B = ( - 7;3)\).
++++
Kópliktiń Lebeg ólshewin tabıń: \(A = \bigcup_{k = 1}^{\infty}\left( \frac{1}{2^{k + 1}},\frac{1}{2^{k}} \right)\);
====
Kópliktiń Lebeg ólshewin tabıń: \(A = \bigcup_{k = 1}^{\infty}\left( \frac{1}{2k},\frac{1}{k} \right)\);
====
Kópliktiń Lebeg ólshewin tabıń: \(A = \bigcup_{k = 1}^{\infty}\left( \frac{1}{2k + 1},\frac{1}{2k} \right)\);
====
Kópliktiń Lebeg ólshewin tabıń: \(A = \bigcup_{k = 1}^{\infty}\left( \frac{1}{k + 1},\frac{1}{k} \right)\);
====
Kópliktiń Lebeg ólshewin tabıń: \(A = \bigcup_{k = 1}^{\infty}\left( k^{2},k^{2} + 2^{- k} \right)\);
====
Kópliktiń Lebeg ólshewin tabıń: \(A = \bigcup_{k = 1}^{\infty}\left( k^{3},k^{3} + 3^{- k} \right)\);
====
Kópliktiń Lebeg ólshewin tabıń: \(A = \bigcup_{k = 1}^{\infty}\left( k,k + \frac{1}{k!} \right)\);
====
Kópliktiń Lebeg ólshewin tabıń: \(A = \bigcup_{k = 1}^{\infty}\left( k,k + \frac{2}{k(k + 1)} \right)\);
====
Kópliktiń Lebeg ólshewin tabıń: \(A = \bigcup_{k = 1}^{\infty}\left( k,k + \frac{3}{k(k + 1)} \right)\);
====
Kópliktiń Lebeg ólshewin tabıń: \(A = \bigcup_{k = 1}^{\infty}\left( k - 2^{- k},k + \frac{1}{k!} \right)\);
====
Kópliktiń Lebeg ólshewin tabıń: \(A = \bigcup_{k = 1}^{\infty}\left( 2k - 2^{- k},2k + \frac{1}{k!} \right)\);
====
Kópliktiń Lebeg ólshewin tabıń: \(A = \bigcup_{k = 1}^{\infty}\left( \frac{1}{3^{k}},\frac{1}{3^{k - 1}} \right)\);
====
Kópliktiń Lebeg ólshewin tabıń: \(A = \bigcup_{k = 1}^{\infty}\left( \frac{1}{k + 2},\frac{1}{k} \right)\);
====
Kópliktiń Lebeg ólshewin tabıń: \(A = \bigcup_{k = 1}^{\infty}\left\lbrack e^{- 2k},e^{- 2k + 1} \right)\).
====
\(P = \{ 0 \leq x \leq 1,\ 0 \leq y \leq 1\}\ hám\ Q = \{ 0.3 \leq x \leq 0.8,\ 0 \leq y \leq 1\}\) tuwrı múyeshlikler kesilispesiniń ólshewin tabıń.
====
\(P = \{ 0 \leq x \leq 1,\ 0 \leq y \leq 1\}\ hám\ Q = \{ 0.3 \leq x \leq 0.8,\ 0 \leq y \leq 1\}\) tuwrı múyeshlikler simmetriyalıq ayırmasınıń ólshewin tabıń.
++++
Lebeg integralın (\(\int_{A}^{}{f(x)d\mu}\)) esaplań: \(f(x) = 2 - \lbrack x\rbrack\), \(A = \lbrack - 2;3)\);
====
Lebeg integralın (\(\int_{A}^{}{f(x)d\mu}\)) esaplań: \(f(x) = 2\lbrack x\rbrack\), \(A = ( - 3;3)\);
====
Lebeg integralın (\(\int_{A}^{}{f(x)d\mu}\)) esaplań: \(f(x) = \lbrack x\rbrack - 1\), \(A = \lbrack - 1;3\rbrack\);
====
Lebeg integralın (\(\int_{A}^{}{f(x)d\mu}\)) esaplań: \(f(x) = \lbrack x + 1\rbrack\), \(A = \lbrack - 2;1)\);
====
Lebeg integralın (\(\int_{A}^{}{f(x)d\mu}\)) esaplań: \(f(x) = \frac{1}{\lbrack x\rbrack}\), \(A = (1;4)\);
====
Lebeg integralın (\(\int_{A}^{}{f(x)d\mu}\)) esaplań: \(f(x) = \frac{1}{\lbrack x\rbrack - 1}\), \(A = \lbrack 2;5\rbrack\);
====
Lebeg integralın (\(\int_{A}^{}{f(x)d\mu}\)) esaplań: \(f(x) = \frac{( - 1)^{\lbrack x\rbrack}}{\lbrack x\rbrack}\), \(A = \lbrack 1;4)\);
====
Lebeg integralın (\(\int_{A}^{}{f(x)d\mu}\)) esaplań: \(f(x) = 2^{\lbrack x\rbrack}\), \(A = ( - 2;2)\);
====
Lebeg integralın (\(\int_{A}^{}{f(x)d\mu}\)) esaplań: \(f(x) = 2^{( - 1)^{\lbrack x\rbrack}}\), \(A = \lbrack 0;3)\);
====
Lebeg integralın (\(\int_{A}^{}{f(x)d\mu}\)) esaplań: \(f(x) = 2^{\lbrack 2x\rbrack}\), \(A = \lbrack 0;1)\);
====
Lebeg integralın (\(\int_{A}^{}{f(x)d\mu}\)) esaplań: \(f(x) = sign(x)\), \(A = \lbrack - 2;2)\);
====
Lebeg integralın (\(\int_{A}^{}{f(x)d\mu}\)) esaplań: \(f(x) = sign(x - 1)\), \(A = \lbrack - 1;2)\);
====
Lebeg integralın (\(\int_{A}^{}{f(x)d\mu}\)) esaplań: \(f(x) = sign(x + 1)\), \(A = \lbrack - 2;2\rbrack\);
====
Lebeg integralın (\(\int_{A}^{}{f(x)d\mu}\)) esaplań: \(f(x) = sign(2x + 1)\), \(A = ( - 1;1\rbrack\).
====
Lebeg integralın (\(\int_{A}^{}{f(x)d\mu}\)) esaplań: \(f(x) = \frac{1}{\lbrack x + 1\rbrack}\), \(A = \lbrack 1;5)\);
====
Lebeg integralın (\(\int_{A}^{}{f(x)d\mu}\)) esaplań: \(f(x) = \frac{1}{\lbrack x\rbrack!}\), \(A = \lbrack 0;4)\);
====
Lebeg integralın (\(\int_{A}^{}{f(x)d\mu}\)) esaplań: \(f(x) = \frac{1}{\lbrack x - 1\rbrack}\), \(A = (3;6)\);
====
Lebeg integralın (\(\int_{A}^{}{f(x)d\mu}\)) esaplań: \(f(x) = \frac{1}{\lbrack x\rbrack\lbrack x + 1\rbrack}\), \(A = \lbrack 1;3\rbrack\);
====
Lebeg integralın (\(\int_{A}^{}{f(x)d\mu}\)) esaplań: \(f(x) = \frac{1}{\lbrack x - 1\rbrack!}\), \(A = (1;3)\);
====
Lebeg integralın (\(\int_{A}^{}{f(x)d\mu}\)) esaplań: \(f(x) = \frac{1}{\lbrack x\rbrack\lbrack x + 1\rbrack}\), \(A = \lbrack 1;3\rbrack\).
++++
[0;3] kóplikte ólshewsiz kóplikke missal keltiriń.
====
[-4;-1] kóplikte ólshewsiz kóplikke missal keltiriń.
====
[-5;-2] kóplikte ólshewsiz kóplikke missal keltiriń.
====
[-3;0] kóplikte ólshewsiz kóplikke missal keltiriń.
====
[4;7] kóplikte ólshewsiz kóplikke missal keltiriń.
====
[3;6] kóplikte ólshewsiz kóplikke missal keltiriń.
====
[-2;1] kóplikte ólshewsiz kóplikke missal keltiriń.
====
[-6;-3] kóplikte ólshewsiz kóplikke missal keltiriń.
====
[-1;2] kóplikte ólshewsiz kóplikke missal keltiriń.
====
[1;4] kóplikte ólshewsiz kóplikke missal keltiriń.
====
[0;3] kóplikte ólshewsiz kóplikke missal keltiriń.
====
[0;3] kóplikte ólshewsiz kóplikke missal keltiriń.
====
[2;5] kóplikte ólshewsiz kóplikke missal keltiriń.
====
[5;8] kóplikte ólshewsiz kóplikke missal keltiriń.
====
[-7;-4] kóplikte ólshewsiz kóplikke missal keltiriń.
====
[6;9] kóplikte ólshewsiz kóplikke missal keltiriń.
====
[-8;-5] kóplikte ólshewsiz kóplikke missal keltiriń.
====
[7;10] kóplikte ólshewsiz kóplikke missal keltiriń.
====
[-9;-6] kóplikte ólshewsiz kóplikke missal keltiriń.
====
[8;11] kóplikte ólshewsiz kóplikke missal keltiriń.
====
[-10;-7] kóplikte ólshewsiz kóplikke missal keltiriń.
====
[9;12] kóplikte ólshewsiz kóplikke missal keltiriń.
====
[-11;-8] kóplikte ólshewsiz kóplikke missal keltiriń.
====
[10;13] kóplikte ólshewsiz kóplikke missal keltiriń.
====
[-12;-9] kóplikte ólshewsiz kóplikke missal keltiriń.
