\(n\)-darajali ko'phadning umumiy ko'rinishi
Ko'phadni \((x - a)\) ga b'lgandagi qoldiq nimaga teng
Ikki o'zgaruvchili funksiyalar qanday belgilanadi
Funksiya qanday usullarda beriladi
Funksiyaning aniqlanish sohasi qanday belgilanadi
Ikki o'zgaruvchili funksiyaning aniqlanish sohasi qayerda joylashsadi
Ikki o'zgaruvchili funksiyaning grafigi nimadan iborat
Ikki o'zgaruvchili funksiyaning birinshi tartibli xususiy hosilalari qanday belgilanadi
Ikki o'zgaruvchili funksiyaning ikkinchi tartibli xususiy hosilalari qanday belgilanadi
Ikki o'zgaruvchili funksiyaning ikkinchi tartibli aralash hosilalari qanday belgilanadi
Ikki o'zgaruvchili funksiyaning ekstremumining zarurli sharti
Ikki o'zgaruvchili funksiyaning toliq orttirmasi
++++
\((x_{0},\ y_{0})\) nuqtaning \(\varepsilon\) atrofi qanday belgilanadi
Ikki o'zgaruvchili funksiyaning \(M(x_{0},\ y_{0})\) nuqtadagi uzluksizligining ta'rifi
Bo'laklab integrallash formulasini yozing
Nyuton-Leybnis formulasini yozing
Hisoblang \(\left( \int {f(x)dx} \right)' = ?\);
Hisoblang \(d\left( \int {f(x)dx} \right) = ?\)
Funksiyaning \((x_{0},\ y_{0})\) nuqtadagi hosilasining formulasini yozing
Funksiyaning \((x_{0},\ y_{0})\) nuqtadagi uzluksizligining formulasini yozing
Funksiyaning \((x_{0},\ y_{0})\) nuqtadagi uzluksizlik shartini yozing
O'zgaruvchini almashtirib integrallash usulining formulasini yozing.
++++
Bernulli differensial tenglamasini yozing
O'zgaruvchilari ajralgan differensial tenglamaning umumiy ko'rinishini yozing
Chiziqli differensial tenglamaning umumiy ko'rinishini yozing
Chiziqli differensial tenglama ko'rinishi
Funksional qatorning umumiy ko'rinishi
Sonli qatorning umumiy ko'rinishini yozing
Chiziqli defferensial tenglamaning umumiy echimini yozing
Agar \(\sum_{n = 1}^{\infty}a_{n} = A,\sum_{n = 1}^{\infty}b_{n} = B\) bo'lsa, u holda \(\sum_{n = 1}^{\infty}\left( a_{n} + b_{n} \right) = ?\)
Agar \(\sum_{n = 1}^{\infty}a_{n} = A,\sum_{n = 1}^{\infty}b_{n} = B\) bo'lsa, u holda \(\sum_{n = 1}^{\infty}\left( a_{n} - b_{n} \right) = ?\)
Musbat hadli qatorlarning yaqinlashuvchi bo'lishning Dalamber belgisini yozing
Musbat hadli qatorlarning yaqinlashuvchi bo'lishning Koshi belgisini yozing
++++
Ehtimollikning klassik ta'rifining formulasini keltiring
O'rin almashtirish formulasini yozing
Guruhlash formulasini yozing
Ehtimollikning geometrik ta'rifining formulasini yozing
Shartli ehtimollik formulasini yozing
Ishonchli hodisaning ehtimolligi nimaga teng
Mumkin bo'magan hodisaning ehtimolligi nimaga teng?
O'rin almashtirish formulasini yozing
Bayes formulasini yozing
To'la ehtimollikning formulasini yozing
Ehtimollikning qiymatlar sohasini yozing
Chekli additivlik aksiomasini yozing
Ehtimollik fazosini yozing
++++
Aniqmas integralni hisoblang: \(\int {\left( 10x^{4} + 7x^{6} - 3 \right)dx}\).
Aniqmas integralni hisoblang: \(\int \frac{dx}{cos^{2}x}\).
Aniqmas integralni hisoblang: \(\int {\left( x^{2} + \frac{1}{x} + \sin x \right)dx}\).
Aniqmas integralni hisoblang: \(\int {e^{x}dx}\) .
Hisoblang: \(\int \left( x^{4} - \frac{1}{x} \right)dx\).
++++
Integralni hisoblang: \(\int {(x + \sin x)dx}\).
Integralni hisoblang: \(\int {2^{x}dx}\).
Integralni hisoblang:\(\int {(x - 1)^{20}}dx\).
Integralni hisoblang: \(\int {\frac{1}{\sin x}dx}\).
Ratsional funksiyani integrallang: \(\int {\frac{5}{(x - 3)(x + 2)}dx}\).
Ratsional funksiyani integrallang: \(\int {\frac{3}{(x - 1)(x + 2)}dx}\).
++++
Aniq integralni hisoblang: \(\int_{1}^{3}\frac{2}{x + 1}dx\).
Integralni hisoblang: \(\int_{1}^{\infty}{\frac{1}{x^{2}}dx}\).
Aniq integralni hisoblang: \(\int_{0}^{\pi}{\sin xdx}\).
Integralni hisoblang: \(\int_{1}^{\infty}{\frac{1}{(x + 2)^{2}}dx}\).
Aniq integralni hisoblang: \(\int_{- \pi/4}^{0}\frac{dx}{cos^{2}x}\).
Aniq integralni hisoblang: \(\int_{2}^{4}\frac{dx}{x}\).
Hisoblang: \(\int_{1}^{2}{e^{x}dx}\).
Aniq integralni hisoblang: \(\int_{0}^{1}{(3x^{2}} + 1)dx\).
Aniq integralni hisoblang: \(\int_{0}^{\frac{\pi}{2}}{\cos xdx}\).
++++
Qatorning yig'indisini hisoblang: \(\sum_{n = 1}^{\infty}\frac{1}{(2n - 1)(2n + 1)}\)
Qatorning yigindisini toping: \(\sum_{n = 1}^{\infty}\frac{1}{n(n + 3)}\)
Qatorning yig'indisini toping: \(\sum_{n = 1}^{\infty}\frac{1}{n(n + 1)}\).
Sonli qatorning dastlabki uchta a'zosini yozing: \(\sum_{n = 1}^{\infty}\frac{n!}{2^{n}}\).
Funktsional qatorning yaqinlashish sohasini toping: \(x + \frac{x^{2}}{2^{2}} + ... + \frac{x^{n}}{n^{2}} + ...\)
Funktsional qatorning yaqinlashish sohasini toping:\(1 + x + ... + x^{n} + ...\)
Funktsional qatorning yaqinlashishi sohasini yozing: \(\ln x + ln^{2}x + ... + ln^{n}x + ...\).
++++
Differensial tenglamani hisoblang: \(yy' = 4\).
Chiziqli differerntsial tenglamaning umumiy echimini toping \(y' + y = e^{- x}\).
Chiziqli differensial tenglamaning umumiy echimini toping: \(y' + y = e^{x}\).
Differensial tenglamaning umumiy echimini toping: \(xy' - 2y = 0\).
Differensial tenglamaning umumiy echimini toping: \(y' = e^{x}\).
Differensial tenglamani eching: \(y' + xy = 0\).
++++
«BIOLOGIYA» sózining harflari aloqida kartochkalarga yozilib yopib, aralashtirib qo'yilgan. Barcha kartotshkalar tasodifan ketma-ket olinib ochilib, olinish tartibida stol ustiga tizilganda yana «BIOLOGIYA» sózining kelib chiqishi ehtimolligini toping.
«MATEMATIKA» sózining harflari aloqida kartochkalarga yozilib yopib aralashtirib qo'yilgan. Barcha kartotshkalar tasodifan ketma-ket olinib ochilib, olinish tartibida stol ustiga tizilganda yana «MATEMATIKA» sózining kelib chiqishi ehtimolligini toping.
50 ta buyumdan iborat partiyada 3 buyum yaroqsiz. Tasodifan olingan 8 ta buyumning ichida 1 ta buyumi yaroqsiz bo'lish ehtimolligini toping
Doyraning ichiga kvadrat cizilgan. Doyraning ichidan tasodifan belgilangan nuqtaning kvadratning ichida yotish ehtimolligini toping.
Guruhdagi 20 talabadan nechta xil usul bilan 3 navbatchini tanlab olsa bo'ladi?
Idishda 5 oq, 8 qora shar bor. Idishdan tasodifan ketma-ket 3 shar olindi. Olingan sharlar oq, qora, qora degan ketma-ketlikda bo'lish ehtimolligini toping.
Ikki kubikti bir marta tashlaganda tushgan otshkolarning yig'indisi 4 bo'lish ehtimolligini toping.
Korobkada 15 oq, 18 qora shar bor. Tasodifan olingan bir shar oq bo'lish ehtimolligini toping.
Korobkada 3 oq, 7 qora shar bor. Tasodifan uchta shar ketma-ket olindi. Ketma-ket olingan sharlarning qora, qora, oq degan ketma-ketlikda bo'lish ehtimolligini toping.
Qutida 5 oq va 15 qora shar bor. Tasodifan olingan bitta sharning oq bo'lish ehtimolligini toping
Tangani ikki marta tashlaganda, kamida bir marta son tomoni tushish ehtimolligini toping.
Telefon raqamining oxirgi ikki tsifrasini unutib, tasodifan nomerlarni tera boshladi. Kerakli raqamni topish ehtimolligini hisoblang.
Telefon raqamining oxirgi tsifrasini unutib, tasodifan nomerlarni tera boshladi. Kerakli raqamni topish ehtimolligini hisoblang.
Uchta bir xil korobkada oq va qora sharlar bor. 1-korobkada 5 oq, 8 qora shar, 2-korobkada 3 oq, 4 qora shar, 3-korobkada 2 oq, 3 qora shar bor. Uchta korobkaning biridan tasodifan olingan bir shar oq bo'lish ehtimolligini toping.