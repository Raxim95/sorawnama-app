\(n\)-dárejeli kóp aǵzalınıń uluwma kórinisi
Kóp aǵzalını \((x - a)\) ǵa bólgendegi qaldıq nege teń
Eki ózgeriwshili funkciyalar qalay belgilenedi
Funkciya qanday usıllarda beriledi
Funkciyanıń anıqlanıw oblastı qalay belgilenedi
Eki ózgeriwshili funkciyanıń anıqlanıw oblastı qay jerde jaylasadı
Eki ózgeriwshili funkciyanıń grafigi neden ibarat
Eki ózgeriwshili funkciyanıń tolıq ósimi
Eki ózgeriwshili funkciyanıń birinshi tártipli dara tuwındıları qalay belgilenedi
Eki ózgeriwshili funkciyanıń ekinshi tártipli dara tuwındıları qalay belgilenedi
Eki ózgeriwshili funkciyanıń ekinshi tártipli aralas tuwındıları qalay belgilenedi
Eki ózgeriwshili funkciyanıń ekstremumınıń zárúrli shárti
++++
\((x_0,y_0)\) tochkanıń \(\varepsilon\) dógeregi qalay belgilenedi
Funkcianıń \((x_{0},\ y_{0})\) noqattaǵı tuwındısınıń formulasın jazıń
Funkcianıń \((x_{0},\ y_{0})\) noqattaǵı úzliksizliginiń formulasın jazıń
Eki ózgeriwshli funkciyanıń \(M(x_{0}, y_{0})\) noqattaǵı úzliksizliginiń anıqlaması
Anıq integraldı esaplawdıń Nyuton-Leybnic formulasın jazıń
Bóleklep inegrallaw formulasın jazıń
Ózgeriwshini almastırıp integrallaw usılıniń formulasın jazıń.
Esaplań \(\left( \int{f(x)dx} \right)' = ?\).
Esaplań \(d\left( \int{f(x)dx} \right) = ?\)
++++
Bernulli differenciallıq teńemesin jazıń
Ózgeriwshileri ajıralǵan differenciallıq teńlemesiniń uluwma kórinisin jazıń
Sızıqlı differenciallıq teńlemeniń uluwma kórinisin jazıń
Sızıqlı defferencial teńlemeniń uluwma sheshimin jazıń
Funkcionallıq qatardıń uluwma kórinisi
Sanlı qatardıń uluwma kórinisin jazıń
Oń aǵzalı qatarlar ushın jıynaqlılıqtıń Dalamber belgisin jazıń
Eger \(\sum_{n = 1}^{\infty}a_{n} = A,\ \sum_{n = 1}^{\infty}b_{n} = B\) bolsa, onda \(\sum_{n = 1}^{\infty}\left( a_{n} + b_{n} \right) = ?\)
Eger \(\sum_{n = 1}^{\infty}a_{n} = A,\ \sum_{n = 1}^{\infty}b_{n} = B\) bolsa, onda \(\sum_{n = 1}^{\infty}\left( a_{n} - b_{n} \right) = ?\)
Oń aǵzalı qatarlar ushın jıynaqlılıqtıń Koshi belgisin jazıń
++++
Itimallıqtıń klassikalıq anıqlamasınıń formulasın keltiriń
Itimmallıqtıń geometriyalıq anıqlamasınıń formulasın jazıń
Orın almastırıw formulasın jazıń
Gruppalaw formulasın jazıń
Shártli itimallıq formulasın jazıń
Isenimli waqıyanıń itimallıǵı nege teń
Múmkin emes waqıyanıń itimaıllıǵı nege teń
Orın awıstırıw formulasın jazıń
Bayes formulasın jazıń
Tolıq itimallıqtıń formulasın jazıń
Itimallıqtıń mánisler oblastın jazıń
Shekli additivlik aksiomasın jazıń
Itimallıq keńisligin jazıń
++++
Anıq emes integraldı esaplań: \(\int{\left( 10x^{4} + 7x^{6} - 3 \right)dx}\).
Anıq emes integraldı esaplań: \(\int\frac{dx}{cos^2 x}\).
Racional funkciyanı integrallań: \(\int{\frac{3}{(x - 1)(x + 2)}dx}\).
Racional funkciyanı integrallań: \(\int{\frac{5}{(x - 3)(x + 2)}dx}\).
Anıq emes integraldı esaplań: \(\int{\left( x^2  + \frac{1}{x} + \sin x \right)dx}\).
Anıq emes integraldı esaplań: \(\int{e^{x}dx}\) .
Integraldı esaplań: \(\int{(x + \sin x)dx}\).
Integraldı esaplań: \(\int{\frac{1}{\sin x}dx}\).
Integraldı esaplań:\(\int{(x - 1)^{20}}dx\).
++++
Anıq integraldı esaplań: \(\int_{1}^{3}\frac{2}{x + 1}dx\).
Esaplań: \(\int\left( x^{4} - \frac{1}{x} \right)dx\).
Esaplań: \(\int_{1}^2 {e^{x}dx}\).
Integraldı esaplań: \(\int_{1}^{\infty}{\frac{1}{x^2 }dx}\).
Integraldı esaplań: \(\int{2^{x}dx}\).
Integraldı esaplań: \(\int_{1}^{\infty}{\frac{1}{(x + 2)^2 }dx}\).
++++
Anıq integraldı esaplań: \(\int_{0}^{\frac{\pi}{2}}{\cos xdx}\).
Anıq integraldı esaplań: \(\int_{2}^{4}\frac{dx}{x}\).
Anıq integraldı esaplań: \(\int_{0}^{\pi}{\sin xdx}\).
Anıq integraldı esaplań: \(\int_{- \pi/4}^{0}\frac{dx}{cos^2 x}\).
Anıq integraldı esaplań: \(\int_{0}^{1}{(3x^2 } + 1)dx\).
++++
Qatardıń jıyındısın esaplań: \(\sum_{n = 1}^{\infty}\frac{1}{(2n - 1)(2n + 1)}\).
Sanlı qatardıń baslanǵısh úsh aǵzasın jazıń: \(\sum_{n = 1}^{\infty}\frac{n!}{2^{n}}\).
Qatardıń qosındısın tabıń: \(\sum_{n = 1}^{\infty}\frac{1}{n(n + 3)}\).
Qatardıń qosındısın tabıń: \(\sum_{n = 1}^{\infty}\frac{1}{n(n + 1)}\).
Funkcional qatardıń jaqınlasıw oblastın tabıń: \(x + \frac{x^2 }{2^2 } + ... + \frac{x^{n}}{n^2 } + ...\)
Funkcional qatardıń jıynaqlılıq oblastın tabıń:\(1 + x + ... + x^{n} + ...\)
Funkcional qatardıń jıynaqlılıq oblastın jazıń: \(\ln x + ln^2 x + ... + ln^{n}x + ...\).
++++
Differencial teńlemeni esaplań: \(yy' = 4\).
Differencial teńlemeniń ulıwma sheshimin tabıń: \(xy' - 2y = 0\).
Sızıqlı differerncial teńlemeniń uluwma sheshimin tabıń \(y' + y = e^{- x}\).
Sızıqlı differencial teńlemeniń ulwma sheshimin tabıń: \(y' + y = e^{x}\).
Differencial teńlemeniń ulıwma sheshimin tabıń: \(y' = e^{x}\).
Differencial teńlemeni sheshiń: \(y' + xy = 0\).
++++
Tiyindi eki márte taslaǵanda, keminde bir márte san tárepi túsiw itimallıǵın tabıń.
Qutada 5 aq hám 15 qara shar bar. Tosınnan alınǵan bir shardıń aq bolıw itimallıǵın tabıń
Gruppadaǵı 20 studentten neshe túrli usıl menen 3 náwbetshini saylap alıwǵa boladı?
Ídısta 5 aq, 8 qara shar bar. Ídıstan tosınnan izbe-iz 3 shar alındı. Alınǵan sharlar aq, qara, qara degen izbe-izlikte bolıw itimallıǵın tabıń.
Korobkada 15 aq, 18 qara shar bar. Tosınnan alınǵan bir shar aq bolıw itimallıǵın tabıń.
Dóngelektiń ishine kvadrat sızılǵan. Dóngelektiń ishinen tosınnan belgilengen noqattıń kvadrattıń ishinde jatıw itimallıǵın tabıń.
Eki kubikti bir márte taslaǵanda túsken ochkolardıń qosındısı 4 bolıw itimallıǵın tabıń.
Úsh birdey korobkada aq hám qara sharlar bar. 1-korobkada 5 aq, 8 qara shar, 2-korobkada 3 aq, 4 qara shar, 3-korobkada 2 aq, 3 qara shar bar. Úsh korobkaniń birewinen tosınnan alınǵan bir shar aq bolıw itimallıǵın tabıń.
Telefon nomerdiń aqırǵı cifrasın umıtıp, tosınnan nomerlerdi tere basladı. Kerekli nomerdi tabıw itimallıǵın esaplań.
Telefon nomerdiń aqırǵı eki cifrasın umıtıp, tosınnan nomerlerdi tere basladı. Kerekli nomerdi tabıw itimallıǵın esaplań.
Korobkada 3 aq, 7 qara shar bar. Tosınnan úsh shar izbe-iz alındı. Izbe-iz alınǵan sharlardıń qara, qara, aq degen izbe-izlikte bolıw itimallıǵın tabıń.
«BIOLOGIYA» sóziniń háripleri bólek kartochkalarǵa jazılıp jawıp, aralastırılıp qoyılǵan. Barlıq kartochkalar tosınnan izbe-iz alınıp ashılıp, alınıw tártibinde stol ústine dizilgende taǵı «BIOLOGIYA» sóziniń kelip shıǵıw itimallıǵın tabıń.
50 buyımnan ibarat partiyada 3 buyım jaramsız. Tosınnan alınǵan 8 buyımnıń ishinde 1 buyımı jaramsız bolıw itimallıǵın tabıń
«MATEMATIKA» sóziniń háripleri bólek kartochkalarǵa jazılıp jawıp aralastırılıp qoyılǵan. Barlıq kartochkalar tosınnan izbe-iz alınıp ashılıp, alınıw tártibinde stol ústine dizilgende taǵı «MATEMATIKA» sóziniń kelip shıǵıw itimallıǵın tabıń.