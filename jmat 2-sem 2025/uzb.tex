\documentclass{article}
\usepackage[fontsize=12pt]{fontsize}
\usepackage[utf8]{inputenc}
\usepackage[T2A]{fontenc}
% \usepackage{unicode-math}

\usepackage{array}
\usepackage[a4paper,
left=7mm,
right=5mm,
top=7mm,]{geometry}
\usepackage{amsmath}
% \usepackage{amssymbol}
\usepackage{amsfonts}
\usepackage{setspace}



\renewcommand{\baselinestretch}{1} 

\everymath{\displaystyle}
\everydisplay{\displaystyle}
\linespread{1.5}

\DeclareMathOperator{\sign}{sign}


\begin{document}

\pagenumbering{gobble}


  \textbf{1-variant}\\
  
  \bgroup
  \def\arraystretch{1.6} % 1 is the default, change whatever you need
  
  \begin{tabular}{|m{5.7cm}|m{9.5cm}|}
  \hline
  Shifr & \\
  \hline
  \end{tabular}
  
  \vspace{1cm}
  
  \begin{tabular}{|m{0.7cm}|m{10cm}|m{4cm}|}
  \hline
  № & Savol & Javob \\
  \hline
  1. & Ikki o'zgaruvchili funksiyaning ikkinchi tartibli aralash hosilalari qanday belgilanadi &  \\
  \hline
  2. & Musbat hadli qatorlarning yaqinlashuvchi bo'lishning Dalamber belgisini yozing &  \\
  \hline
  3. & Bayes formulasini yozing &  \\
  \hline
  4. & \(n\)-darajali ko'phadning umumiy ko'rinishi &  \\
  \hline
  5. & Aniqmas integralni hisoblang: \(\int {\left( 10x^{4} + 7x^{6} - 3 \right)dx}\). &  \\
  \hline
  6. & Aniq integralni hisoblang: \(\int_{0}^{\frac{\pi}{2}}{\cos xdx}\). &  \\
  \hline
  7. & Integralni hisoblang: \(\int_{1}^{\infty}{\frac{1}{(x + 2)^{2}}dx}\). &  \\
  \hline
  8. & Qatorning yig'indisini hisoblang: \(\sum_{n = 1}^{\infty}\frac{1}{(2n - 1)(2n + 1)}\) &  \\
  \hline
  9. & Differensial tenglamaning umumiy echimini toping: \(y' = e^{x}\). &  \\
  \hline
  10. & Korobkada 3 oq, 7 qora shar bor. Tasodifan uchta shar ketma-ket olindi. Ketma-ket olingan sharlarning qora, qora, oq degan ketma-ketlikda bo'lish ehtimolligini toping. &  \\
  \hline
  \end{tabular}
  
  \vspace{1cm}
  
  \begin{tabular}{lll}
  To'g'ri javoblar soni: \underline{\hspace{1.5cm}} & 
  Bahosi: \underline{\hspace{1.5cm}} & 
  Imtixon oluvchining imzosi: \underline{\hspace{2cm}} \\
  \end{tabular}
  
  \egroup
  
  \newpage
  
  
  \textbf{2-variant}\\
  
  \bgroup
  \def\arraystretch{1.6} % 1 is the default, change whatever you need
  
  \begin{tabular}{|m{5.7cm}|m{9.5cm}|}
  \hline
  Shifr & \\
  \hline
  \end{tabular}
  
  \vspace{1cm}
  
  \begin{tabular}{|m{0.7cm}|m{10cm}|m{4cm}|}
  \hline
  № & Savol & Javob \\
  \hline
  1. & Bo'laklab integrallash formulasini yozing &  \\
  \hline
  2. & Shartli ehtimollik formulasini yozing &  \\
  \hline
  3. & Ikki o'zgaruvchili funksiyaning toliq orttirmasi &  \\
  \hline
  4. & Ishonchli hodisaning ehtimolligi nimaga teng &  \\
  \hline
  5. & Hisoblang: \(\int \left( x^{4} - \frac{1}{x} \right)dx\). &  \\
  \hline
  6. & Aniq integralni hisoblang: \(\int_{0}^{\pi}{\sin xdx}\). &  \\
  \hline
  7. & Integralni hisoblang: \(\int_{1}^{\infty}{\frac{1}{x^{2}}dx}\). &  \\
  \hline
  8. & Funktsional qatorning yaqinlashishi sohasini yozing: \(\ln x + ln^{2}x + ... + ln^{n}x + ...\). &  \\
  \hline
  9. & Chiziqli differensial tenglamaning umumiy echimini toping: \(y' + y = e^{x}\). &  \\
  \hline
  10. & Guruhdagi 20 talabadan nechta xil usul bilan 3 navbatchini tanlab olsa bo'ladi? &  \\
  \hline
  \end{tabular}
  
  \vspace{1cm}
  
  \begin{tabular}{lll}
  To'g'ri javoblar soni: \underline{\hspace{1.5cm}} & 
  Bahosi: \underline{\hspace{1.5cm}} & 
  Imtixon oluvchining imzosi: \underline{\hspace{2cm}} \\
  \end{tabular}
  
  \egroup
  
  \newpage
  
  
  \textbf{3-variant}\\
  
  \bgroup
  \def\arraystretch{1.6} % 1 is the default, change whatever you need
  
  \begin{tabular}{|m{5.7cm}|m{9.5cm}|}
  \hline
  Shifr & \\
  \hline
  \end{tabular}
  
  \vspace{1cm}
  
  \begin{tabular}{|m{0.7cm}|m{10cm}|m{4cm}|}
  \hline
  № & Savol & Javob \\
  \hline
  1. & O'zgaruvchini almashtirib integrallash usulining formulasini yozing. &  \\
  \hline
  2. & O'rin almashtirish formulasini yozing &  \\
  \hline
  3. & Ikki o'zgaruvchili funksiyaning aniqlanish sohasi qayerda joylashsadi &  \\
  \hline
  4. & Chekli additivlik aksiomasini yozing &  \\
  \hline
  5. & Ratsional funksiyani integrallang: \(\int {\frac{3}{(x - 1)(x + 2)}dx}\). &  \\
  \hline
  6. & Aniq integralni hisoblang: \(\int_{0}^{1}{(3x^{2}} + 1)dx\). &  \\
  \hline
  7. & Aniq integralni hisoblang: \(\int_{- \pi/4}^{0}\frac{dx}{cos^{2}x}\). &  \\
  \hline
  8. & Funktsional qatorning yaqinlashish sohasini toping:\(1 + x + ... + x^{n} + ...\) &  \\
  \hline
  9. & Differensial tenglamani hisoblang: \(yy' = 4\). &  \\
  \hline
  10. & Telefon raqamining oxirgi ikki tsifrasini unutib, tasodifan nomerlarni tera boshladi. Kerakli raqamni topish ehtimolligini hisoblang. &  \\
  \hline
  \end{tabular}
  
  \vspace{1cm}
  
  \begin{tabular}{lll}
  To'g'ri javoblar soni: \underline{\hspace{1.5cm}} & 
  Bahosi: \underline{\hspace{1.5cm}} & 
  Imtixon oluvchining imzosi: \underline{\hspace{2cm}} \\
  \end{tabular}
  
  \egroup
  
  \newpage
  
  
  \textbf{4-variant}\\
  
  \bgroup
  \def\arraystretch{1.6} % 1 is the default, change whatever you need
  
  \begin{tabular}{|m{5.7cm}|m{9.5cm}|}
  \hline
  Shifr & \\
  \hline
  \end{tabular}
  
  \vspace{1cm}
  
  \begin{tabular}{|m{0.7cm}|m{10cm}|m{4cm}|}
  \hline
  № & Savol & Javob \\
  \hline
  1. & To'la ehtimollikning formulasini yozing &  \\
  \hline
  2. & Ehtimollik fazosini yozing &  \\
  \hline
  3. & Ikki o'zgaruvchili funksiyaning ekstremumining zarurli sharti &  \\
  \hline
  4. & Mumkin bo'magan hodisaning ehtimolligi nimaga teng? &  \\
  \hline
  5. & Aniqmas integralni hisoblang: \(\int {e^{x}dx}\) . &  \\
  \hline
  6. & Aniq integralni hisoblang: \(\int_{1}^{3}\frac{2}{x + 1}dx\). &  \\
  \hline
  7. & Hisoblang: \(\int_{1}^{2}{e^{x}dx}\). &  \\
  \hline
  8. & Sonli qatorning dastlabki uchta a'zosini yozing: \(\sum_{n = 1}^{\infty}\frac{n!}{2^{n}}\). &  \\
  \hline
  9. & Chiziqli differerntsial tenglamaning umumiy echimini toping \(y' + y = e^{- x}\). &  \\
  \hline
  10. & Tangani ikki marta tashlaganda, kamida bir marta son tomoni tushish ehtimolligini toping. &  \\
  \hline
  \end{tabular}
  
  \vspace{1cm}
  
  \begin{tabular}{lll}
  To'g'ri javoblar soni: \underline{\hspace{1.5cm}} & 
  Bahosi: \underline{\hspace{1.5cm}} & 
  Imtixon oluvchining imzosi: \underline{\hspace{2cm}} \\
  \end{tabular}
  
  \egroup
  
  \newpage
  
  
  \textbf{5-variant}\\
  
  \bgroup
  \def\arraystretch{1.6} % 1 is the default, change whatever you need
  
  \begin{tabular}{|m{5.7cm}|m{9.5cm}|}
  \hline
  Shifr & \\
  \hline
  \end{tabular}
  
  \vspace{1cm}
  
  \begin{tabular}{|m{0.7cm}|m{10cm}|m{4cm}|}
  \hline
  № & Savol & Javob \\
  \hline
  1. & Ikki o'zgaruvchili funksiyalar qanday belgilanadi &  \\
  \hline
  2. & Musbat hadli qatorlarning yaqinlashuvchi bo'lishning Koshi belgisini yozing &  \\
  \hline
  3. & Funksiya qanday usullarda beriladi &  \\
  \hline
  4. & Chiziqli differensial tenglamaning umumiy ko'rinishini yozing &  \\
  \hline
  5. & Integralni hisoblang:\(\int {(x - 1)^{20}}dx\). &  \\
  \hline
  6. & Aniq integralni hisoblang: \(\int_{2}^{4}\frac{dx}{x}\). &  \\
  \hline
  7. & Hisoblang: \(\int_{1}^{2}{e^{x}dx}\). &  \\
  \hline
  8. & Qatorning yig'indisini toping: \(\sum_{n = 1}^{\infty}\frac{1}{n(n + 1)}\). &  \\
  \hline
  9. & Differensial tenglamani eching: \(y' + xy = 0\). &  \\
  \hline
  10. & Telefon raqamining oxirgi tsifrasini unutib, tasodifan nomerlarni tera boshladi. Kerakli raqamni topish ehtimolligini hisoblang. &  \\
  \hline
  \end{tabular}
  
  \vspace{1cm}
  
  \begin{tabular}{lll}
  To'g'ri javoblar soni: \underline{\hspace{1.5cm}} & 
  Bahosi: \underline{\hspace{1.5cm}} & 
  Imtixon oluvchining imzosi: \underline{\hspace{2cm}} \\
  \end{tabular}
  
  \egroup
  
  \newpage
  
  
  \textbf{6-variant}\\
  
  \bgroup
  \def\arraystretch{1.6} % 1 is the default, change whatever you need
  
  \begin{tabular}{|m{5.7cm}|m{9.5cm}|}
  \hline
  Shifr & \\
  \hline
  \end{tabular}
  
  \vspace{1cm}
  
  \begin{tabular}{|m{0.7cm}|m{10cm}|m{4cm}|}
  \hline
  № & Savol & Javob \\
  \hline
  1. & Hisoblang \(d\left( \int {f(x)dx} \right) = ?\) &  \\
  \hline
  2. & Ikki o'zgaruvchili funksiyaning birinshi tartibli xususiy hosilalari qanday belgilanadi &  \\
  \hline
  3. & Ikki o'zgaruvchili funksiyaning ikkinchi tartibli xususiy hosilalari qanday belgilanadi &  \\
  \hline
  4. & Funksional qatorning umumiy ko'rinishi &  \\
  \hline
  5. & Aniqmas integralni hisoblang: \(\int \frac{dx}{cos^{2}x}\). &  \\
  \hline
  6. & Aniq integralni hisoblang: \(\int_{2}^{4}\frac{dx}{x}\). &  \\
  \hline
  7. & Aniq integralni hisoblang: \(\int_{0}^{\pi}{\sin xdx}\). &  \\
  \hline
  8. & Funktsional qatorning yaqinlashish sohasini toping: \(x + \frac{x^{2}}{2^{2}} + ... + \frac{x^{n}}{n^{2}} + ...\) &  \\
  \hline
  9. & Differensial tenglamaning umumiy echimini toping: \(xy' - 2y = 0\). &  \\
  \hline
  10. & Ikki kubikti bir marta tashlaganda tushgan otshkolarning yig'indisi 4 bo'lish ehtimolligini toping. &  \\
  \hline
  \end{tabular}
  
  \vspace{1cm}
  
  \begin{tabular}{lll}
  To'g'ri javoblar soni: \underline{\hspace{1.5cm}} & 
  Bahosi: \underline{\hspace{1.5cm}} & 
  Imtixon oluvchining imzosi: \underline{\hspace{2cm}} \\
  \end{tabular}
  
  \egroup
  
  \newpage
  
  
  \textbf{7-variant}\\
  
  \bgroup
  \def\arraystretch{1.6} % 1 is the default, change whatever you need
  
  \begin{tabular}{|m{5.7cm}|m{9.5cm}|}
  \hline
  Shifr & \\
  \hline
  \end{tabular}
  
  \vspace{1cm}
  
  \begin{tabular}{|m{0.7cm}|m{10cm}|m{4cm}|}
  \hline
  № & Savol & Javob \\
  \hline
  1. & Hisoblang \(\left( \int {f(x)dx} \right)' = ?\); &  \\
  \hline
  2. & Agar \(\sum_{n = 1}^{\infty}a_{n} = A,\sum_{n = 1}^{\infty}b_{n} = B\) bo'lsa, u holda \(\sum_{n = 1}^{\infty}\left( a_{n} - b_{n} \right) = ?\) &  \\
  \hline
  3. & O'zgaruvchilari ajralgan differensial tenglamaning umumiy ko'rinishini yozing &  \\
  \hline
  4. & Nyuton-Leybnis formulasini yozing &  \\
  \hline
  5. & Integralni hisoblang:\(\int {(x - 1)^{20}}dx\). &  \\
  \hline
  6. & Aniq integralni hisoblang: \(\int_{1}^{3}\frac{2}{x + 1}dx\). &  \\
  \hline
  7. & Aniq integralni hisoblang: \(\int_{0}^{\frac{\pi}{2}}{\cos xdx}\). &  \\
  \hline
  8. & Qatorning yigindisini toping: \(\sum_{n = 1}^{\infty}\frac{1}{n(n + 3)}\) &  \\
  \hline
  9. & Chiziqli differerntsial tenglamaning umumiy echimini toping \(y' + y = e^{- x}\). &  \\
  \hline
  10. & «MATEMATIKA» sózining harflari aloqida kartochkalarga yozilib yopib aralashtirib qo'yilgan. Barcha kartotshkalar tasodifan ketma-ket olinib ochilib, olinish tartibida stol ustiga tizilganda yana «MATEMATIKA» sózining kelib chiqishi ehtimolligini toping. &  \\
  \hline
  \end{tabular}
  
  \vspace{1cm}
  
  \begin{tabular}{lll}
  To'g'ri javoblar soni: \underline{\hspace{1.5cm}} & 
  Bahosi: \underline{\hspace{1.5cm}} & 
  Imtixon oluvchining imzosi: \underline{\hspace{2cm}} \\
  \end{tabular}
  
  \egroup
  
  \newpage
  
  
  \textbf{8-variant}\\
  
  \bgroup
  \def\arraystretch{1.6} % 1 is the default, change whatever you need
  
  \begin{tabular}{|m{5.7cm}|m{9.5cm}|}
  \hline
  Shifr & \\
  \hline
  \end{tabular}
  
  \vspace{1cm}
  
  \begin{tabular}{|m{0.7cm}|m{10cm}|m{4cm}|}
  \hline
  № & Savol & Javob \\
  \hline
  1. & Ehtimollikning klassik ta'rifining formulasini keltiring &  \\
  \hline
  2. & Chiziqli differensial tenglama ko'rinishi &  \\
  \hline
  3. & Guruhlash formulasini yozing &  \\
  \hline
  4. & Ehtimollikning qiymatlar sohasini yozing &  \\
  \hline
  5. & Integralni hisoblang: \(\int {\frac{1}{\sin x}dx}\). &  \\
  \hline
  6. & Aniq integralni hisoblang: \(\int_{0}^{1}{(3x^{2}} + 1)dx\). &  \\
  \hline
  7. & Integralni hisoblang: \(\int_{1}^{\infty}{\frac{1}{x^{2}}dx}\). &  \\
  \hline
  8. & Funktsional qatorning yaqinlashish sohasini toping:\(1 + x + ... + x^{n} + ...\) &  \\
  \hline
  9. & Differensial tenglamani eching: \(y' + xy = 0\). &  \\
  \hline
  10. & «BIOLOGIYA» sózining harflari aloqida kartochkalarga yozilib yopib, aralashtirib qo'yilgan. Barcha kartotshkalar tasodifan ketma-ket olinib ochilib, olinish tartibida stol ustiga tizilganda yana «BIOLOGIYA» sózining kelib chiqishi ehtimolligini toping. &  \\
  \hline
  \end{tabular}
  
  \vspace{1cm}
  
  \begin{tabular}{lll}
  To'g'ri javoblar soni: \underline{\hspace{1.5cm}} & 
  Bahosi: \underline{\hspace{1.5cm}} & 
  Imtixon oluvchining imzosi: \underline{\hspace{2cm}} \\
  \end{tabular}
  
  \egroup
  
  \newpage
  
  
  \textbf{9-variant}\\
  
  \bgroup
  \def\arraystretch{1.6} % 1 is the default, change whatever you need
  
  \begin{tabular}{|m{5.7cm}|m{9.5cm}|}
  \hline
  Shifr & \\
  \hline
  \end{tabular}
  
  \vspace{1cm}
  
  \begin{tabular}{|m{0.7cm}|m{10cm}|m{4cm}|}
  \hline
  № & Savol & Javob \\
  \hline
  1. & Funksiyaning aniqlanish sohasi qanday belgilanadi &  \\
  \hline
  2. & Funksiyaning \((x_{0},\ y_{0})\) nuqtadagi uzluksizlik shartini yozing &  \\
  \hline
  3. & Ehtimollikning geometrik ta'rifining formulasini yozing &  \\
  \hline
  4. & Funksiyaning \((x_{0},\ y_{0})\) nuqtadagi hosilasining formulasini yozing &  \\
  \hline
  5. & Hisoblang: \(\int \left( x^{4} - \frac{1}{x} \right)dx\). &  \\
  \hline
  6. & Integralni hisoblang: \(\int_{1}^{\infty}{\frac{1}{(x + 2)^{2}}dx}\). &  \\
  \hline
  7. & Aniq integralni hisoblang: \(\int_{- \pi/4}^{0}\frac{dx}{cos^{2}x}\). &  \\
  \hline
  8. & Funktsional qatorning yaqinlashish sohasini toping: \(x + \frac{x^{2}}{2^{2}} + ... + \frac{x^{n}}{n^{2}} + ...\) &  \\
  \hline
  9. & Chiziqli differensial tenglamaning umumiy echimini toping: \(y' + y = e^{x}\). &  \\
  \hline
  10. & 50 ta buyumdan iborat partiyada 3 buyum yaroqsiz. Tasodifan olingan 8 ta buyumning ichida 1 ta buyumi yaroqsiz bo'lish ehtimolligini toping &  \\
  \hline
  \end{tabular}
  
  \vspace{1cm}
  
  \begin{tabular}{lll}
  To'g'ri javoblar soni: \underline{\hspace{1.5cm}} & 
  Bahosi: \underline{\hspace{1.5cm}} & 
  Imtixon oluvchining imzosi: \underline{\hspace{2cm}} \\
  \end{tabular}
  
  \egroup
  
  \newpage
  
  
  \textbf{10-variant}\\
  
  \bgroup
  \def\arraystretch{1.6} % 1 is the default, change whatever you need
  
  \begin{tabular}{|m{5.7cm}|m{9.5cm}|}
  \hline
  Shifr & \\
  \hline
  \end{tabular}
  
  \vspace{1cm}
  
  \begin{tabular}{|m{0.7cm}|m{10cm}|m{4cm}|}
  \hline
  № & Savol & Javob \\
  \hline
  1. & Bernulli differensial tenglamasini yozing &  \\
  \hline
  2. & Ikki o'zgaruvchili funksiyaning grafigi nimadan iborat &  \\
  \hline
  3. & Sonli qatorning umumiy ko'rinishini yozing &  \\
  \hline
  4. & Chiziqli defferensial tenglamaning umumiy echimini yozing &  \\
  \hline
  5. & Aniqmas integralni hisoblang: \(\int \frac{dx}{cos^{2}x}\). &  \\
  \hline
  6. & Aniq integralni hisoblang: \(\int_{- \pi/4}^{0}\frac{dx}{cos^{2}x}\). &  \\
  \hline
  7. & Aniq integralni hisoblang: \(\int_{2}^{4}\frac{dx}{x}\). &  \\
  \hline
  8. & Qatorning yig'indisini toping: \(\sum_{n = 1}^{\infty}\frac{1}{n(n + 1)}\). &  \\
  \hline
  9. & Differensial tenglamaning umumiy echimini toping: \(xy' - 2y = 0\). &  \\
  \hline
  10. & Doyraning ichiga kvadrat cizilgan. Doyraning ichidan tasodifan belgilangan nuqtaning kvadratning ichida yotish ehtimolligini toping. &  \\
  \hline
  \end{tabular}
  
  \vspace{1cm}
  
  \begin{tabular}{lll}
  To'g'ri javoblar soni: \underline{\hspace{1.5cm}} & 
  Bahosi: \underline{\hspace{1.5cm}} & 
  Imtixon oluvchining imzosi: \underline{\hspace{2cm}} \\
  \end{tabular}
  
  \egroup
  
  \newpage
  
  
  \textbf{11-variant}\\
  
  \bgroup
  \def\arraystretch{1.6} % 1 is the default, change whatever you need
  
  \begin{tabular}{|m{5.7cm}|m{9.5cm}|}
  \hline
  Shifr & \\
  \hline
  \end{tabular}
  
  \vspace{1cm}
  
  \begin{tabular}{|m{0.7cm}|m{10cm}|m{4cm}|}
  \hline
  № & Savol & Javob \\
  \hline
  1. & Ikki o'zgaruvchili funksiyaning \(M(x_{0},\ y_{0})\) noqtadagi uzluksizligining ta'rifi &  \\
  \hline
  2. & Funksiyaning \((x_{0},\ y_{0})\) nuqtadagi uzluksizligining formulasini yozing &  \\
  \hline
  3. & Agar \(\sum_{n = 1}^{\infty}a_{n} = A,\sum_{n = 1}^{\infty}b_{n} = B\) bo'lsa, u holda \(\sum_{n = 1}^{\infty}\left( a_{n} + b_{n} \right) = ?\) &  \\
  \hline
  4. & \((x_{0},\ y_{0})\) nuqtaning \(\varepsilon\) atrofi qanday belgilanadi &  \\
  \hline
  5. & Ratsional funksiyani integrallang: \(\int {\frac{3}{(x - 1)(x + 2)}dx}\). &  \\
  \hline
  6. & Hisoblang: \(\int_{1}^{2}{e^{x}dx}\). &  \\
  \hline
  7. & Aniq integralni hisoblang: \(\int_{1}^{3}\frac{2}{x + 1}dx\). &  \\
  \hline
  8. & Qatorning yigindisini toping: \(\sum_{n = 1}^{\infty}\frac{1}{n(n + 3)}\) &  \\
  \hline
  9. & Differensial tenglamani hisoblang: \(yy' = 4\). &  \\
  \hline
  10. & Idishda 5 oq, 8 qora shar bor. Idishdan tasodifan ketma-ket 3 shar olindi. Olingan sharlar oq, qora, qora degan ketma-ketlikda bo'lish ehtimolligini toping. &  \\
  \hline
  \end{tabular}
  
  \vspace{1cm}
  
  \begin{tabular}{lll}
  To'g'ri javoblar soni: \underline{\hspace{1.5cm}} & 
  Bahosi: \underline{\hspace{1.5cm}} & 
  Imtixon oluvchining imzosi: \underline{\hspace{2cm}} \\
  \end{tabular}
  
  \egroup
  
  \newpage
  
  
  \textbf{12-variant}\\
  
  \bgroup
  \def\arraystretch{1.6} % 1 is the default, change whatever you need
  
  \begin{tabular}{|m{5.7cm}|m{9.5cm}|}
  \hline
  Shifr & \\
  \hline
  \end{tabular}
  
  \vspace{1cm}
  
  \begin{tabular}{|m{0.7cm}|m{10cm}|m{4cm}|}
  \hline
  № & Savol & Javob \\
  \hline
  1. & Ko'phadni \((x - a)\) ga b'lgandagi qoldiq nimaga teng &  \\
  \hline
  2. & O'rin almashtirish formulasini yozing &  \\
  \hline
  3. & Hisoblang \(\left( \int {f(x)dx} \right)' = ?\); &  \\
  \hline
  4. & Ko'phadni \((x - a)\) ga b'lgandagi qoldiq nimaga teng &  \\
  \hline
  5. & Hisoblang: \(\int \left( x^{4} - \frac{1}{x} \right)dx\). &  \\
  \hline
  6. & Integralni hisoblang: \(\int_{1}^{\infty}{\frac{1}{x^{2}}dx}\). &  \\
  \hline
  7. & Aniq integralni hisoblang: \(\int_{0}^{\pi}{\sin xdx}\). &  \\
  \hline
  8. & Sonli qatorning dastlabki uchta a'zosini yozing: \(\sum_{n = 1}^{\infty}\frac{n!}{2^{n}}\). &  \\
  \hline
  9. & Differensial tenglamaning umumiy echimini toping: \(y' = e^{x}\). &  \\
  \hline
  10. & Qutida 5 oq va 15 qora shar bor. Tasodifan olingan bitta sharning oq bo'lish ehtimolligini toping &  \\
  \hline
  \end{tabular}
  
  \vspace{1cm}
  
  \begin{tabular}{lll}
  To'g'ri javoblar soni: \underline{\hspace{1.5cm}} & 
  Bahosi: \underline{\hspace{1.5cm}} & 
  Imtixon oluvchining imzosi: \underline{\hspace{2cm}} \\
  \end{tabular}
  
  \egroup
  
  \newpage
  
  
  \textbf{13-variant}\\
  
  \bgroup
  \def\arraystretch{1.6} % 1 is the default, change whatever you need
  
  \begin{tabular}{|m{5.7cm}|m{9.5cm}|}
  \hline
  Shifr & \\
  \hline
  \end{tabular}
  
  \vspace{1cm}
  
  \begin{tabular}{|m{0.7cm}|m{10cm}|m{4cm}|}
  \hline
  № & Savol & Javob \\
  \hline
  1. & Bernulli differensial tenglamasini yozing &  \\
  \hline
  2. & Sonli qatorning umumiy ko'rinishini yozing &  \\
  \hline
  3. & To'la ehtimollikning formulasini yozing &  \\
  \hline
  4. & Chiziqli differensial tenglamaning umumiy ko'rinishini yozing &  \\
  \hline
  5. & Ratsional funksiyani integrallang: \(\int {\frac{5}{(x - 3)(x + 2)}dx}\). &  \\
  \hline
  6. & Integralni hisoblang: \(\int_{1}^{\infty}{\frac{1}{(x + 2)^{2}}dx}\). &  \\
  \hline
  7. & Aniq integralni hisoblang: \(\int_{0}^{1}{(3x^{2}} + 1)dx\). &  \\
  \hline
  8. & Funktsional qatorning yaqinlashishi sohasini yozing: \(\ln x + ln^{2}x + ... + ln^{n}x + ...\). &  \\
  \hline
  9. & Differensial tenglamani eching: \(y' + xy = 0\). &  \\
  \hline
  10. & Korobkada 15 oq, 18 qora shar bor. Tasodifan olingan bir shar oq bo'lish ehtimolligini toping. &  \\
  \hline
  \end{tabular}
  
  \vspace{1cm}
  
  \begin{tabular}{lll}
  To'g'ri javoblar soni: \underline{\hspace{1.5cm}} & 
  Bahosi: \underline{\hspace{1.5cm}} & 
  Imtixon oluvchining imzosi: \underline{\hspace{2cm}} \\
  \end{tabular}
  
  \egroup
  
  \newpage
  
  
  \textbf{14-variant}\\
  
  \bgroup
  \def\arraystretch{1.6} % 1 is the default, change whatever you need
  
  \begin{tabular}{|m{5.7cm}|m{9.5cm}|}
  \hline
  Shifr & \\
  \hline
  \end{tabular}
  
  \vspace{1cm}
  
  \begin{tabular}{|m{0.7cm}|m{10cm}|m{4cm}|}
  \hline
  № & Savol & Javob \\
  \hline
  1. & Agar \(\sum_{n = 1}^{\infty}a_{n} = A,\sum_{n = 1}^{\infty}b_{n} = B\) bo'lsa, u holda \(\sum_{n = 1}^{\infty}\left( a_{n} - b_{n} \right) = ?\) &  \\
  \hline
  2. & Funksional qatorning umumiy ko'rinishi &  \\
  \hline
  3. & Ikki o'zgaruvchili funksiyaning ikkinchi tartibli xususiy hosilalari qanday belgilanadi &  \\
  \hline
  4. & Ikki o'zgaruvchili funksiyaning ikkinchi tartibli aralash hosilalari qanday belgilanadi &  \\
  \hline
  5. & Integralni hisoblang: \(\int {2^{x}dx}\). &  \\
  \hline
  6. & Aniq integralni hisoblang: \(\int_{0}^{\frac{\pi}{2}}{\cos xdx}\). &  \\
  \hline
  7. & Aniq integralni hisoblang: \(\int_{0}^{\pi}{\sin xdx}\). &  \\
  \hline
  8. & Qatorning yig'indisini hisoblang: \(\sum_{n = 1}^{\infty}\frac{1}{(2n - 1)(2n + 1)}\) &  \\
  \hline
  9. & Chiziqli differensial tenglamaning umumiy echimini toping: \(y' + y = e^{x}\). &  \\
  \hline
  10. & Uchta bir xil korobkada oq va qora sharlar bor. 1-korobkada 5 oq, 8 qora shar, 2-korobkada 3 oq, 4 qora shar, 3-korobkada 2 oq, 3 qora shar bor. Uchta korobkaning biridan tasodifan olingan bir shar oq bo'lish ehtimolligini toping. &  \\
  \hline
  \end{tabular}
  
  \vspace{1cm}
  
  \begin{tabular}{lll}
  To'g'ri javoblar soni: \underline{\hspace{1.5cm}} & 
  Bahosi: \underline{\hspace{1.5cm}} & 
  Imtixon oluvchining imzosi: \underline{\hspace{2cm}} \\
  \end{tabular}
  
  \egroup
  
  \newpage
  
  
  \textbf{15-variant}\\
  
  \bgroup
  \def\arraystretch{1.6} % 1 is the default, change whatever you need
  
  \begin{tabular}{|m{5.7cm}|m{9.5cm}|}
  \hline
  Shifr & \\
  \hline
  \end{tabular}
  
  \vspace{1cm}
  
  \begin{tabular}{|m{0.7cm}|m{10cm}|m{4cm}|}
  \hline
  № & Savol & Javob \\
  \hline
  1. & Nyuton-Leybnis formulasini yozing &  \\
  \hline
  2. & Chiziqli defferensial tenglamaning umumiy echimini yozing &  \\
  \hline
  3. & O'rin almashtirish formulasini yozing &  \\
  \hline
  4. & Ikki o'zgaruvchili funksiyaning aniqlanish sohasi qayerda joylashsadi &  \\
  \hline
  5. & Aniqmas integralni hisoblang: \(\int {\left( x^{2} + \frac{1}{x} + \sin x \right)dx}\). &  \\
  \hline
  6. & Aniq integralni hisoblang: \(\int_{0}^{\frac{\pi}{2}}{\cos xdx}\). &  \\
  \hline
  7. & Hisoblang: \(\int_{1}^{2}{e^{x}dx}\). &  \\
  \hline
  8. & Funktsional qatorning yaqinlashish sohasini toping:\(1 + x + ... + x^{n} + ...\) &  \\
  \hline
  9. & Differensial tenglamaning umumiy echimini toping: \(xy' - 2y = 0\). &  \\
  \hline
  10. & Guruhdagi 20 talabadan nechta xil usul bilan 3 navbatchini tanlab olsa bo'ladi? &  \\
  \hline
  \end{tabular}
  
  \vspace{1cm}
  
  \begin{tabular}{lll}
  To'g'ri javoblar soni: \underline{\hspace{1.5cm}} & 
  Bahosi: \underline{\hspace{1.5cm}} & 
  Imtixon oluvchining imzosi: \underline{\hspace{2cm}} \\
  \end{tabular}
  
  \egroup
  
  \newpage
  
  
  \textbf{16-variant}\\
  
  \bgroup
  \def\arraystretch{1.6} % 1 is the default, change whatever you need
  
  \begin{tabular}{|m{5.7cm}|m{9.5cm}|}
  \hline
  Shifr & \\
  \hline
  \end{tabular}
  
  \vspace{1cm}
  
  \begin{tabular}{|m{0.7cm}|m{10cm}|m{4cm}|}
  \hline
  № & Savol & Javob \\
  \hline
  1. & Ehtimollik fazosini yozing &  \\
  \hline
  2. & Hisoblang \(d\left( \int {f(x)dx} \right) = ?\) &  \\
  \hline
  3. & Ikki o'zgaruvchili funksiyaning toliq orttirmasi &  \\
  \hline
  4. & Bayes formulasini yozing &  \\
  \hline
  5. & Integralni hisoblang: \(\int {\frac{1}{\sin x}dx}\). &  \\
  \hline
  6. & Integralni hisoblang: \(\int_{1}^{\infty}{\frac{1}{x^{2}}dx}\). &  \\
  \hline
  7. & Aniq integralni hisoblang: \(\int_{- \pi/4}^{0}\frac{dx}{cos^{2}x}\). &  \\
  \hline
  8. & Funktsional qatorning yaqinlashish sohasini toping: \(x + \frac{x^{2}}{2^{2}} + ... + \frac{x^{n}}{n^{2}} + ...\) &  \\
  \hline
  9. & Differensial tenglamaning umumiy echimini toping: \(y' = e^{x}\). &  \\
  \hline
  10. & Uchta bir xil korobkada oq va qora sharlar bor. 1-korobkada 5 oq, 8 qora shar, 2-korobkada 3 oq, 4 qora shar, 3-korobkada 2 oq, 3 qora shar bor. Uchta korobkaning biridan tasodifan olingan bir shar oq bo'lish ehtimolligini toping. &  \\
  \hline
  \end{tabular}
  
  \vspace{1cm}
  
  \begin{tabular}{lll}
  To'g'ri javoblar soni: \underline{\hspace{1.5cm}} & 
  Bahosi: \underline{\hspace{1.5cm}} & 
  Imtixon oluvchining imzosi: \underline{\hspace{2cm}} \\
  \end{tabular}
  
  \egroup
  
  \newpage
  
  
  \textbf{17-variant}\\
  
  \bgroup
  \def\arraystretch{1.6} % 1 is the default, change whatever you need
  
  \begin{tabular}{|m{5.7cm}|m{9.5cm}|}
  \hline
  Shifr & \\
  \hline
  \end{tabular}
  
  \vspace{1cm}
  
  \begin{tabular}{|m{0.7cm}|m{10cm}|m{4cm}|}
  \hline
  № & Savol & Javob \\
  \hline
  1. & Chekli additivlik aksiomasini yozing &  \\
  \hline
  2. & Ikki o'zgaruvchili funksiyaning grafigi nimadan iborat &  \\
  \hline
  3. & Musbat hadli qatorlarning yaqinlashuvchi bo'lishning Koshi belgisini yozing &  \\
  \hline
  4. & Ikki o'zgaruvchili funksiyalar qanday belgilanadi &  \\
  \hline
  5. & Aniqmas integralni hisoblang: \(\int {e^{x}dx}\) . &  \\
  \hline
  6. & Integralni hisoblang: \(\int_{1}^{\infty}{\frac{1}{(x + 2)^{2}}dx}\). &  \\
  \hline
  7. & Aniq integralni hisoblang: \(\int_{0}^{1}{(3x^{2}} + 1)dx\). &  \\
  \hline
  8. & Qatorning yigindisini toping: \(\sum_{n = 1}^{\infty}\frac{1}{n(n + 3)}\) &  \\
  \hline
  9. & Chiziqli differerntsial tenglamaning umumiy echimini toping \(y' + y = e^{- x}\). &  \\
  \hline
  10. & «BIOLOGIYA» sózining harflari aloqida kartochkalarga yozilib yopib, aralashtirib qo'yilgan. Barcha kartotshkalar tasodifan ketma-ket olinib ochilib, olinish tartibida stol ustiga tizilganda yana «BIOLOGIYA» sózining kelib chiqishi ehtimolligini toping. &  \\
  \hline
  \end{tabular}
  
  \vspace{1cm}
  
  \begin{tabular}{lll}
  To'g'ri javoblar soni: \underline{\hspace{1.5cm}} & 
  Bahosi: \underline{\hspace{1.5cm}} & 
  Imtixon oluvchining imzosi: \underline{\hspace{2cm}} \\
  \end{tabular}
  
  \egroup
  
  \newpage
  
  
  \textbf{18-variant}\\
  
  \bgroup
  \def\arraystretch{1.6} % 1 is the default, change whatever you need
  
  \begin{tabular}{|m{5.7cm}|m{9.5cm}|}
  \hline
  Shifr & \\
  \hline
  \end{tabular}
  
  \vspace{1cm}
  
  \begin{tabular}{|m{0.7cm}|m{10cm}|m{4cm}|}
  \hline
  № & Savol & Javob \\
  \hline
  1. & \((x_{0},\ y_{0})\) nuqtaning \(\varepsilon\) atrofi qanday belgilanadi &  \\
  \hline
  2. & O'rin almashtirish formulasini yozing &  \\
  \hline
  3. & Ehtimollikning klassik ta'rifining formulasini keltiring &  \\
  \hline
  4. & \(n\)-darajali ko'phadning umumiy ko'rinishi &  \\
  \hline
  5. & Integralni hisoblang: \(\int {\frac{1}{\sin x}dx}\). &  \\
  \hline
  6. & Aniq integralni hisoblang: \(\int_{2}^{4}\frac{dx}{x}\). &  \\
  \hline
  7. & Aniq integralni hisoblang: \(\int_{1}^{3}\frac{2}{x + 1}dx\). &  \\
  \hline
  8. & Sonli qatorning dastlabki uchta a'zosini yozing: \(\sum_{n = 1}^{\infty}\frac{n!}{2^{n}}\). &  \\
  \hline
  9. & Differensial tenglamani hisoblang: \(yy' = 4\). &  \\
  \hline
  10. & Telefon raqamining oxirgi ikki tsifrasini unutib, tasodifan nomerlarni tera boshladi. Kerakli raqamni topish ehtimolligini hisoblang. &  \\
  \hline
  \end{tabular}
  
  \vspace{1cm}
  
  \begin{tabular}{lll}
  To'g'ri javoblar soni: \underline{\hspace{1.5cm}} & 
  Bahosi: \underline{\hspace{1.5cm}} & 
  Imtixon oluvchining imzosi: \underline{\hspace{2cm}} \\
  \end{tabular}
  
  \egroup
  
  \newpage
  
  
  \textbf{19-variant}\\
  
  \bgroup
  \def\arraystretch{1.6} % 1 is the default, change whatever you need
  
  \begin{tabular}{|m{5.7cm}|m{9.5cm}|}
  \hline
  Shifr & \\
  \hline
  \end{tabular}
  
  \vspace{1cm}
  
  \begin{tabular}{|m{0.7cm}|m{10cm}|m{4cm}|}
  \hline
  № & Savol & Javob \\
  \hline
  1. & Shartli ehtimollik formulasini yozing &  \\
  \hline
  2. & Ikki o'zgaruvchili funksiyaning \(M(x_{0},\ y_{0})\) noqtadagi uzluksizligining ta'rifi &  \\
  \hline
  3. & Funksiyaning \((x_{0},\ y_{0})\) nuqtadagi hosilasining formulasini yozing &  \\
  \hline
  4. & Chiziqli differensial tenglama ko'rinishi &  \\
  \hline
  5. & Aniqmas integralni hisoblang: \(\int {\left( 10x^{4} + 7x^{6} - 3 \right)dx}\). &  \\
  \hline
  6. & Aniq integralni hisoblang: \(\int_{- \pi/4}^{0}\frac{dx}{cos^{2}x}\). &  \\
  \hline
  7. & Aniq integralni hisoblang: \(\int_{0}^{1}{(3x^{2}} + 1)dx\). &  \\
  \hline
  8. & Qatorning yig'indisini toping: \(\sum_{n = 1}^{\infty}\frac{1}{n(n + 1)}\). &  \\
  \hline
  9. & Chiziqli differerntsial tenglamaning umumiy echimini toping \(y' + y = e^{- x}\). &  \\
  \hline
  10. & Telefon raqamining oxirgi tsifrasini unutib, tasodifan nomerlarni tera boshladi. Kerakli raqamni topish ehtimolligini hisoblang. &  \\
  \hline
  \end{tabular}
  
  \vspace{1cm}
  
  \begin{tabular}{lll}
  To'g'ri javoblar soni: \underline{\hspace{1.5cm}} & 
  Bahosi: \underline{\hspace{1.5cm}} & 
  Imtixon oluvchining imzosi: \underline{\hspace{2cm}} \\
  \end{tabular}
  
  \egroup
  
  \newpage
  
  
  \textbf{20-variant}\\
  
  \bgroup
  \def\arraystretch{1.6} % 1 is the default, change whatever you need
  
  \begin{tabular}{|m{5.7cm}|m{9.5cm}|}
  \hline
  Shifr & \\
  \hline
  \end{tabular}
  
  \vspace{1cm}
  
  \begin{tabular}{|m{0.7cm}|m{10cm}|m{4cm}|}
  \hline
  № & Savol & Javob \\
  \hline
  1. & Bo'laklab integrallash formulasini yozing &  \\
  \hline
  2. & Ehtimollikning qiymatlar sohasini yozing &  \\
  \hline
  3. & Agar \(\sum_{n = 1}^{\infty}a_{n} = A,\sum_{n = 1}^{\infty}b_{n} = B\) bo'lsa, u holda \(\sum_{n = 1}^{\infty}\left( a_{n} + b_{n} \right) = ?\) &  \\
  \hline
  4. & Guruhlash formulasini yozing &  \\
  \hline
  5. & Integralni hisoblang: \(\int {(x + \sin x)dx}\). &  \\
  \hline
  6. & Aniq integralni hisoblang: \(\int_{0}^{\pi}{\sin xdx}\). &  \\
  \hline
  7. & Aniq integralni hisoblang: \(\int_{1}^{3}\frac{2}{x + 1}dx\). &  \\
  \hline
  8. & Qatorning yig'indisini hisoblang: \(\sum_{n = 1}^{\infty}\frac{1}{(2n - 1)(2n + 1)}\) &  \\
  \hline
  9. & Differensial tenglamani eching: \(y' + xy = 0\). &  \\
  \hline
  10. & «MATEMATIKA» sózining harflari aloqida kartochkalarga yozilib yopib aralashtirib qo'yilgan. Barcha kartotshkalar tasodifan ketma-ket olinib ochilib, olinish tartibida stol ustiga tizilganda yana «MATEMATIKA» sózining kelib chiqishi ehtimolligini toping. &  \\
  \hline
  \end{tabular}
  
  \vspace{1cm}
  
  \begin{tabular}{lll}
  To'g'ri javoblar soni: \underline{\hspace{1.5cm}} & 
  Bahosi: \underline{\hspace{1.5cm}} & 
  Imtixon oluvchining imzosi: \underline{\hspace{2cm}} \\
  \end{tabular}
  
  \egroup
  
  \newpage
  
  
  \textbf{21-variant}\\
  
  \bgroup
  \def\arraystretch{1.6} % 1 is the default, change whatever you need
  
  \begin{tabular}{|m{5.7cm}|m{9.5cm}|}
  \hline
  Shifr & \\
  \hline
  \end{tabular}
  
  \vspace{1cm}
  
  \begin{tabular}{|m{0.7cm}|m{10cm}|m{4cm}|}
  \hline
  № & Savol & Javob \\
  \hline
  1. & Mumkin bo'magan hodisaning ehtimolligi nimaga teng? &  \\
  \hline
  2. & O'zgaruvchini almashtirib integrallash usulining formulasini yozing. &  \\
  \hline
  3. & O'zgaruvchilari ajralgan differensial tenglamaning umumiy ko'rinishini yozing &  \\
  \hline
  4. & Ikki o'zgaruvchili funksiyaning birinshi tartibli xususiy hosilalari qanday belgilanadi &  \\
  \hline
  5. & Integralni hisoblang:\(\int {(x - 1)^{20}}dx\). &  \\
  \hline
  6. & Integralni hisoblang: \(\int_{1}^{\infty}{\frac{1}{x^{2}}dx}\). &  \\
  \hline
  7. & Hisoblang: \(\int_{1}^{2}{e^{x}dx}\). &  \\
  \hline
  8. & Funktsional qatorning yaqinlashishi sohasini yozing: \(\ln x + ln^{2}x + ... + ln^{n}x + ...\). &  \\
  \hline
  9. & Differensial tenglamaning umumiy echimini toping: \(xy' - 2y = 0\). &  \\
  \hline
  10. & Korobkada 15 oq, 18 qora shar bor. Tasodifan olingan bir shar oq bo'lish ehtimolligini toping. &  \\
  \hline
  \end{tabular}
  
  \vspace{1cm}
  
  \begin{tabular}{lll}
  To'g'ri javoblar soni: \underline{\hspace{1.5cm}} & 
  Bahosi: \underline{\hspace{1.5cm}} & 
  Imtixon oluvchining imzosi: \underline{\hspace{2cm}} \\
  \end{tabular}
  
  \egroup
  
  \newpage
  
  
  \textbf{22-variant}\\
  
  \bgroup
  \def\arraystretch{1.6} % 1 is the default, change whatever you need
  
  \begin{tabular}{|m{5.7cm}|m{9.5cm}|}
  \hline
  Shifr & \\
  \hline
  \end{tabular}
  
  \vspace{1cm}
  
  \begin{tabular}{|m{0.7cm}|m{10cm}|m{4cm}|}
  \hline
  № & Savol & Javob \\
  \hline
  1. & Funksiyaning \((x_{0},\ y_{0})\) nuqtadagi uzluksizlik shartini yozing &  \\
  \hline
  2. & Ikki o'zgaruvchili funksiyaning ekstremumining zarurli sharti &  \\
  \hline
  3. & Funksiyaning aniqlanish sohasi qanday belgilanadi &  \\
  \hline
  4. & Funksiyaning \((x_{0},\ y_{0})\) nuqtadagi uzluksizligining formulasini yozing &  \\
  \hline
  5. & Ratsional funksiyani integrallang: \(\int {\frac{3}{(x - 1)(x + 2)}dx}\). &  \\
  \hline
  6. & Aniq integralni hisoblang: \(\int_{0}^{\frac{\pi}{2}}{\cos xdx}\). &  \\
  \hline
  7. & Aniq integralni hisoblang: \(\int_{2}^{4}\frac{dx}{x}\). &  \\
  \hline
  8. & Qatorning yig'indisini hisoblang: \(\sum_{n = 1}^{\infty}\frac{1}{(2n - 1)(2n + 1)}\) &  \\
  \hline
  9. & Differensial tenglamani hisoblang: \(yy' = 4\). &  \\
  \hline
  10. & Doyraning ichiga kvadrat cizilgan. Doyraning ichidan tasodifan belgilangan nuqtaning kvadratning ichida yotish ehtimolligini toping. &  \\
  \hline
  \end{tabular}
  
  \vspace{1cm}
  
  \begin{tabular}{lll}
  To'g'ri javoblar soni: \underline{\hspace{1.5cm}} & 
  Bahosi: \underline{\hspace{1.5cm}} & 
  Imtixon oluvchining imzosi: \underline{\hspace{2cm}} \\
  \end{tabular}
  
  \egroup
  
  \newpage
  
  
  \textbf{23-variant}\\
  
  \bgroup
  \def\arraystretch{1.6} % 1 is the default, change whatever you need
  
  \begin{tabular}{|m{5.7cm}|m{9.5cm}|}
  \hline
  Shifr & \\
  \hline
  \end{tabular}
  
  \vspace{1cm}
  
  \begin{tabular}{|m{0.7cm}|m{10cm}|m{4cm}|}
  \hline
  № & Savol & Javob \\
  \hline
  1. & Musbat hadli qatorlarning yaqinlashuvchi bo'lishning Dalamber belgisini yozing &  \\
  \hline
  2. & Ishonchli hodisaning ehtimolligi nimaga teng &  \\
  \hline
  3. & Ehtimollikning geometrik ta'rifining formulasini yozing &  \\
  \hline
  4. & Funksiya qanday usullarda beriladi &  \\
  \hline
  5. & Integralni hisoblang: \(\int {(x + \sin x)dx}\). &  \\
  \hline
  6. & Integralni hisoblang: \(\int_{1}^{\infty}{\frac{1}{(x + 2)^{2}}dx}\). &  \\
  \hline
  7. & Aniq integralni hisoblang: \(\int_{1}^{3}\frac{2}{x + 1}dx\). &  \\
  \hline
  8. & Funktsional qatorning yaqinlashish sohasini toping:\(1 + x + ... + x^{n} + ...\) &  \\
  \hline
  9. & Chiziqli differensial tenglamaning umumiy echimini toping: \(y' + y = e^{x}\). &  \\
  \hline
  10. & Tangani ikki marta tashlaganda, kamida bir marta son tomoni tushish ehtimolligini toping. &  \\
  \hline
  \end{tabular}
  
  \vspace{1cm}
  
  \begin{tabular}{lll}
  To'g'ri javoblar soni: \underline{\hspace{1.5cm}} & 
  Bahosi: \underline{\hspace{1.5cm}} & 
  Imtixon oluvchining imzosi: \underline{\hspace{2cm}} \\
  \end{tabular}
  
  \egroup
  
  \newpage
  
  
  \textbf{24-variant}\\
  
  \bgroup
  \def\arraystretch{1.6} % 1 is the default, change whatever you need
  
  \begin{tabular}{|m{5.7cm}|m{9.5cm}|}
  \hline
  Shifr & \\
  \hline
  \end{tabular}
  
  \vspace{1cm}
  
  \begin{tabular}{|m{0.7cm}|m{10cm}|m{4cm}|}
  \hline
  № & Savol & Javob \\
  \hline
  1. & Ikki o'zgaruvchili funksiyaning grafigi nimadan iborat &  \\
  \hline
  2. & Ikki o'zgaruvchili funksiyaning birinshi tartibli xususiy hosilalari qanday belgilanadi &  \\
  \hline
  3. & Hisoblang \(\left( \int {f(x)dx} \right)' = ?\); &  \\
  \hline
  4. & Ko'phadni \((x - a)\) ga b'lgandagi qoldiq nimaga teng &  \\
  \hline
  5. & Aniqmas integralni hisoblang: \(\int {\left( 10x^{4} + 7x^{6} - 3 \right)dx}\). &  \\
  \hline
  6. & Aniq integralni hisoblang: \(\int_{2}^{4}\frac{dx}{x}\). &  \\
  \hline
  7. & Aniq integralni hisoblang: \(\int_{0}^{\frac{\pi}{2}}{\cos xdx}\). &  \\
  \hline
  8. & Sonli qatorning dastlabki uchta a'zosini yozing: \(\sum_{n = 1}^{\infty}\frac{n!}{2^{n}}\). &  \\
  \hline
  9. & Differensial tenglamaning umumiy echimini toping: \(y' = e^{x}\). &  \\
  \hline
  10. & 50 ta buyumdan iborat partiyada 3 buyum yaroqsiz. Tasodifan olingan 8 ta buyumning ichida 1 ta buyumi yaroqsiz bo'lish ehtimolligini toping &  \\
  \hline
  \end{tabular}
  
  \vspace{1cm}
  
  \begin{tabular}{lll}
  To'g'ri javoblar soni: \underline{\hspace{1.5cm}} & 
  Bahosi: \underline{\hspace{1.5cm}} & 
  Imtixon oluvchining imzosi: \underline{\hspace{2cm}} \\
  \end{tabular}
  
  \egroup
  
  \newpage
  
  
  \textbf{25-variant}\\
  
  \bgroup
  \def\arraystretch{1.6} % 1 is the default, change whatever you need
  
  \begin{tabular}{|m{5.7cm}|m{9.5cm}|}
  \hline
  Shifr & \\
  \hline
  \end{tabular}
  
  \vspace{1cm}
  
  \begin{tabular}{|m{0.7cm}|m{10cm}|m{4cm}|}
  \hline
  № & Savol & Javob \\
  \hline
  1. & Guruhlash formulasini yozing &  \\
  \hline
  2. & O'zgaruvchilari ajralgan differensial tenglamaning umumiy ko'rinishini yozing &  \\
  \hline
  3. & Chekli additivlik aksiomasini yozing &  \\
  \hline
  4. & Ikki o'zgaruvchili funksiyaning ekstremumining zarurli sharti &  \\
  \hline
  5. & Ratsional funksiyani integrallang: \(\int {\frac{5}{(x - 3)(x + 2)}dx}\). &  \\
  \hline
  6. & Aniq integralni hisoblang: \(\int_{0}^{1}{(3x^{2}} + 1)dx\). &  \\
  \hline
  7. & Hisoblang: \(\int_{1}^{2}{e^{x}dx}\). &  \\
  \hline
  8. & Funktsional qatorning yaqinlashishi sohasini yozing: \(\ln x + ln^{2}x + ... + ln^{n}x + ...\). &  \\
  \hline
  9. & Chiziqli differensial tenglamaning umumiy echimini toping: \(y' + y = e^{x}\). &  \\
  \hline
  10. & Korobkada 3 oq, 7 qora shar bor. Tasodifan uchta shar ketma-ket olindi. Ketma-ket olingan sharlarning qora, qora, oq degan ketma-ketlikda bo'lish ehtimolligini toping. &  \\
  \hline
  \end{tabular}
  
  \vspace{1cm}
  
  \begin{tabular}{lll}
  To'g'ri javoblar soni: \underline{\hspace{1.5cm}} & 
  Bahosi: \underline{\hspace{1.5cm}} & 
  Imtixon oluvchining imzosi: \underline{\hspace{2cm}} \\
  \end{tabular}
  
  \egroup
  
  \newpage
  
  
  \textbf{26-variant}\\
  
  \bgroup
  \def\arraystretch{1.6} % 1 is the default, change whatever you need
  
  \begin{tabular}{|m{5.7cm}|m{9.5cm}|}
  \hline
  Shifr & \\
  \hline
  \end{tabular}
  
  \vspace{1cm}
  
  \begin{tabular}{|m{0.7cm}|m{10cm}|m{4cm}|}
  \hline
  № & Savol & Javob \\
  \hline
  1. & Bernulli differensial tenglamasini yozing &  \\
  \hline
  2. & Nyuton-Leybnis formulasini yozing &  \\
  \hline
  3. & Ikki o'zgaruvchili funksiyaning ikkinchi tartibli aralash hosilalari qanday belgilanadi &  \\
  \hline
  4. & Sonli qatorning umumiy ko'rinishini yozing &  \\
  \hline
  5. & Aniqmas integralni hisoblang: \(\int {e^{x}dx}\) . &  \\
  \hline
  6. & Integralni hisoblang: \(\int_{1}^{\infty}{\frac{1}{x^{2}}dx}\). &  \\
  \hline
  7. & Integralni hisoblang: \(\int_{1}^{\infty}{\frac{1}{(x + 2)^{2}}dx}\). &  \\
  \hline
  8. & Qatorning yigindisini toping: \(\sum_{n = 1}^{\infty}\frac{1}{n(n + 3)}\) &  \\
  \hline
  9. & Differensial tenglamaning umumiy echimini toping: \(y' = e^{x}\). &  \\
  \hline
  10. & Qutida 5 oq va 15 qora shar bor. Tasodifan olingan bitta sharning oq bo'lish ehtimolligini toping &  \\
  \hline
  \end{tabular}
  
  \vspace{1cm}
  
  \begin{tabular}{lll}
  To'g'ri javoblar soni: \underline{\hspace{1.5cm}} & 
  Bahosi: \underline{\hspace{1.5cm}} & 
  Imtixon oluvchining imzosi: \underline{\hspace{2cm}} \\
  \end{tabular}
  
  \egroup
  
  \newpage
  
  
  \textbf{27-variant}\\
  
  \bgroup
  \def\arraystretch{1.6} % 1 is the default, change whatever you need
  
  \begin{tabular}{|m{5.7cm}|m{9.5cm}|}
  \hline
  Shifr & \\
  \hline
  \end{tabular}
  
  \vspace{1cm}
  
  \begin{tabular}{|m{0.7cm}|m{10cm}|m{4cm}|}
  \hline
  № & Savol & Javob \\
  \hline
  1. & Ishonchli hodisaning ehtimolligi nimaga teng &  \\
  \hline
  2. & Ikki o'zgaruvchili funksiyaning aniqlanish sohasi qayerda joylashsadi &  \\
  \hline
  3. & Musbat hadli qatorlarning yaqinlashuvchi bo'lishning Koshi belgisini yozing &  \\
  \hline
  4. & Funksiyaning \((x_{0},\ y_{0})\) nuqtadagi uzluksizlik shartini yozing &  \\
  \hline
  5. & Integralni hisoblang:\(\int {(x - 1)^{20}}dx\). &  \\
  \hline
  6. & Aniq integralni hisoblang: \(\int_{0}^{\pi}{\sin xdx}\). &  \\
  \hline
  7. & Aniq integralni hisoblang: \(\int_{- \pi/4}^{0}\frac{dx}{cos^{2}x}\). &  \\
  \hline
  8. & Funktsional qatorning yaqinlashish sohasini toping: \(x + \frac{x^{2}}{2^{2}} + ... + \frac{x^{n}}{n^{2}} + ...\) &  \\
  \hline
  9. & Chiziqli differerntsial tenglamaning umumiy echimini toping \(y' + y = e^{- x}\). &  \\
  \hline
  10. & Ikki kubikti bir marta tashlaganda tushgan otshkolarning yig'indisi 4 bo'lish ehtimolligini toping. &  \\
  \hline
  \end{tabular}
  
  \vspace{1cm}
  
  \begin{tabular}{lll}
  To'g'ri javoblar soni: \underline{\hspace{1.5cm}} & 
  Bahosi: \underline{\hspace{1.5cm}} & 
  Imtixon oluvchining imzosi: \underline{\hspace{2cm}} \\
  \end{tabular}
  
  \egroup
  
  \newpage
  
  
  \textbf{28-variant}\\
  
  \bgroup
  \def\arraystretch{1.6} % 1 is the default, change whatever you need
  
  \begin{tabular}{|m{5.7cm}|m{9.5cm}|}
  \hline
  Shifr & \\
  \hline
  \end{tabular}
  
  \vspace{1cm}
  
  \begin{tabular}{|m{0.7cm}|m{10cm}|m{4cm}|}
  \hline
  № & Savol & Javob \\
  \hline
  1. & O'zgaruvchini almashtirib integrallash usulining formulasini yozing. &  \\
  \hline
  2. & Ikki o'zgaruvchili funksiyalar qanday belgilanadi &  \\
  \hline
  3. & Ehtimollikning geometrik ta'rifining formulasini yozing &  \\
  \hline
  4. & To'la ehtimollikning formulasini yozing &  \\
  \hline
  5. & Hisoblang: \(\int \left( x^{4} - \frac{1}{x} \right)dx\). &  \\
  \hline
  6. & Aniq integralni hisoblang: \(\int_{2}^{4}\frac{dx}{x}\). &  \\
  \hline
  7. & Aniq integralni hisoblang: \(\int_{0}^{\pi}{\sin xdx}\). &  \\
  \hline
  8. & Qatorning yig'indisini toping: \(\sum_{n = 1}^{\infty}\frac{1}{n(n + 1)}\). &  \\
  \hline
  9. & Differensial tenglamaning umumiy echimini toping: \(xy' - 2y = 0\). &  \\
  \hline
  10. & Idishda 5 oq, 8 qora shar bor. Idishdan tasodifan ketma-ket 3 shar olindi. Olingan sharlar oq, qora, qora degan ketma-ketlikda bo'lish ehtimolligini toping. &  \\
  \hline
  \end{tabular}
  
  \vspace{1cm}
  
  \begin{tabular}{lll}
  To'g'ri javoblar soni: \underline{\hspace{1.5cm}} & 
  Bahosi: \underline{\hspace{1.5cm}} & 
  Imtixon oluvchining imzosi: \underline{\hspace{2cm}} \\
  \end{tabular}
  
  \egroup
  
  \newpage
  
  
  \textbf{29-variant}\\
  
  \bgroup
  \def\arraystretch{1.6} % 1 is the default, change whatever you need
  
  \begin{tabular}{|m{5.7cm}|m{9.5cm}|}
  \hline
  Shifr & \\
  \hline
  \end{tabular}
  
  \vspace{1cm}
  
  \begin{tabular}{|m{0.7cm}|m{10cm}|m{4cm}|}
  \hline
  № & Savol & Javob \\
  \hline
  1. & Mumkin bo'magan hodisaning ehtimolligi nimaga teng? &  \\
  \hline
  2. & Funksiyaning \((x_{0},\ y_{0})\) nuqtadagi uzluksizligining formulasini yozing &  \\
  \hline
  3. & Bo'laklab integrallash formulasini yozing &  \\
  \hline
  4. & Ikki o'zgaruvchili funksiyaning toliq orttirmasi &  \\
  \hline
  5. & Integralni hisoblang:\(\int {(x - 1)^{20}}dx\). &  \\
  \hline
  6. & Hisoblang: \(\int_{1}^{2}{e^{x}dx}\). &  \\
  \hline
  7. & Aniq integralni hisoblang: \(\int_{0}^{1}{(3x^{2}} + 1)dx\). &  \\
  \hline
  8. & Qatorning yigindisini toping: \(\sum_{n = 1}^{\infty}\frac{1}{n(n + 3)}\) &  \\
  \hline
  9. & Differensial tenglamani eching: \(y' + xy = 0\). &  \\
  \hline
  10. & Tangani ikki marta tashlaganda, kamida bir marta son tomoni tushish ehtimolligini toping. &  \\
  \hline
  \end{tabular}
  
  \vspace{1cm}
  
  \begin{tabular}{lll}
  To'g'ri javoblar soni: \underline{\hspace{1.5cm}} & 
  Bahosi: \underline{\hspace{1.5cm}} & 
  Imtixon oluvchining imzosi: \underline{\hspace{2cm}} \\
  \end{tabular}
  
  \egroup
  
  \newpage
  
  
  \textbf{30-variant}\\
  
  \bgroup
  \def\arraystretch{1.6} % 1 is the default, change whatever you need
  
  \begin{tabular}{|m{5.7cm}|m{9.5cm}|}
  \hline
  Shifr & \\
  \hline
  \end{tabular}
  
  \vspace{1cm}
  
  \begin{tabular}{|m{0.7cm}|m{10cm}|m{4cm}|}
  \hline
  № & Savol & Javob \\
  \hline
  1. & \(n\)-darajali ko'phadning umumiy ko'rinishi &  \\
  \hline
  2. & Agar \(\sum_{n = 1}^{\infty}a_{n} = A,\sum_{n = 1}^{\infty}b_{n} = B\) bo'lsa, u holda \(\sum_{n = 1}^{\infty}\left( a_{n} - b_{n} \right) = ?\) &  \\
  \hline
  3. & O'rin almashtirish formulasini yozing &  \\
  \hline
  4. & Bayes formulasini yozing &  \\
  \hline
  5. & Ratsional funksiyani integrallang: \(\int {\frac{5}{(x - 3)(x + 2)}dx}\). &  \\
  \hline
  6. & Aniq integralni hisoblang: \(\int_{1}^{3}\frac{2}{x + 1}dx\). &  \\
  \hline
  7. & Aniq integralni hisoblang: \(\int_{0}^{\frac{\pi}{2}}{\cos xdx}\). &  \\
  \hline
  8. & Funktsional qatorning yaqinlashish sohasini toping: \(x + \frac{x^{2}}{2^{2}} + ... + \frac{x^{n}}{n^{2}} + ...\) &  \\
  \hline
  9. & Differensial tenglamani hisoblang: \(yy' = 4\). &  \\
  \hline
  10. & Uchta bir xil korobkada oq va qora sharlar bor. 1-korobkada 5 oq, 8 qora shar, 2-korobkada 3 oq, 4 qora shar, 3-korobkada 2 oq, 3 qora shar bor. Uchta korobkaning biridan tasodifan olingan bir shar oq bo'lish ehtimolligini toping. &  \\
  \hline
  \end{tabular}
  
  \vspace{1cm}
  
  \begin{tabular}{lll}
  To'g'ri javoblar soni: \underline{\hspace{1.5cm}} & 
  Bahosi: \underline{\hspace{1.5cm}} & 
  Imtixon oluvchining imzosi: \underline{\hspace{2cm}} \\
  \end{tabular}
  
  \egroup
  
  \newpage
  
  
  \textbf{31-variant}\\
  
  \bgroup
  \def\arraystretch{1.6} % 1 is the default, change whatever you need
  
  \begin{tabular}{|m{5.7cm}|m{9.5cm}|}
  \hline
  Shifr & \\
  \hline
  \end{tabular}
  
  \vspace{1cm}
  
  \begin{tabular}{|m{0.7cm}|m{10cm}|m{4cm}|}
  \hline
  № & Savol & Javob \\
  \hline
  1. & Funksional qatorning umumiy ko'rinishi &  \\
  \hline
  2. & Ikki o'zgaruvchili funksiyaning \(M(x_{0},\ y_{0})\) noqtadagi uzluksizligining ta'rifi &  \\
  \hline
  3. & Funksiyaning \((x_{0},\ y_{0})\) nuqtadagi hosilasining formulasini yozing &  \\
  \hline
  4. & Ikki o'zgaruvchili funksiyaning ikkinchi tartibli xususiy hosilalari qanday belgilanadi &  \\
  \hline
  5. & Ratsional funksiyani integrallang: \(\int {\frac{3}{(x - 1)(x + 2)}dx}\). &  \\
  \hline
  6. & Aniq integralni hisoblang: \(\int_{- \pi/4}^{0}\frac{dx}{cos^{2}x}\). &  \\
  \hline
  7. & Integralni hisoblang: \(\int_{1}^{\infty}{\frac{1}{(x + 2)^{2}}dx}\). &  \\
  \hline
  8. & Funktsional qatorning yaqinlashish sohasini toping:\(1 + x + ... + x^{n} + ...\) &  \\
  \hline
  9. & Differensial tenglamaning umumiy echimini toping: \(y' = e^{x}\). &  \\
  \hline
  10. & Doyraning ichiga kvadrat cizilgan. Doyraning ichidan tasodifan belgilangan nuqtaning kvadratning ichida yotish ehtimolligini toping. &  \\
  \hline
  \end{tabular}
  
  \vspace{1cm}
  
  \begin{tabular}{lll}
  To'g'ri javoblar soni: \underline{\hspace{1.5cm}} & 
  Bahosi: \underline{\hspace{1.5cm}} & 
  Imtixon oluvchining imzosi: \underline{\hspace{2cm}} \\
  \end{tabular}
  
  \egroup
  
  \newpage
  
  
  \textbf{32-variant}\\
  
  \bgroup
  \def\arraystretch{1.6} % 1 is the default, change whatever you need
  
  \begin{tabular}{|m{5.7cm}|m{9.5cm}|}
  \hline
  Shifr & \\
  \hline
  \end{tabular}
  
  \vspace{1cm}
  
  \begin{tabular}{|m{0.7cm}|m{10cm}|m{4cm}|}
  \hline
  № & Savol & Javob \\
  \hline
  1. & Ehtimollik fazosini yozing &  \\
  \hline
  2. & Agar \(\sum_{n = 1}^{\infty}a_{n} = A,\sum_{n = 1}^{\infty}b_{n} = B\) bo'lsa, u holda \(\sum_{n = 1}^{\infty}\left( a_{n} + b_{n} \right) = ?\) &  \\
  \hline
  3. & Funksiya qanday usullarda beriladi &  \\
  \hline
  4. & Ehtimollikning klassik ta'rifining formulasini keltiring &  \\
  \hline
  5. & Aniqmas integralni hisoblang: \(\int {e^{x}dx}\) . &  \\
  \hline
  6. & Integralni hisoblang: \(\int_{1}^{\infty}{\frac{1}{x^{2}}dx}\). &  \\
  \hline
  7. & Aniq integralni hisoblang: \(\int_{0}^{1}{(3x^{2}} + 1)dx\). &  \\
  \hline
  8. & Qatorning yig'indisini hisoblang: \(\sum_{n = 1}^{\infty}\frac{1}{(2n - 1)(2n + 1)}\) &  \\
  \hline
  9. & Differensial tenglamani eching: \(y' + xy = 0\). &  \\
  \hline
  10. & 50 ta buyumdan iborat partiyada 3 buyum yaroqsiz. Tasodifan olingan 8 ta buyumning ichida 1 ta buyumi yaroqsiz bo'lish ehtimolligini toping &  \\
  \hline
  \end{tabular}
  
  \vspace{1cm}
  
  \begin{tabular}{lll}
  To'g'ri javoblar soni: \underline{\hspace{1.5cm}} & 
  Bahosi: \underline{\hspace{1.5cm}} & 
  Imtixon oluvchining imzosi: \underline{\hspace{2cm}} \\
  \end{tabular}
  
  \egroup
  
  \newpage
  
  
  \textbf{33-variant}\\
  
  \bgroup
  \def\arraystretch{1.6} % 1 is the default, change whatever you need
  
  \begin{tabular}{|m{5.7cm}|m{9.5cm}|}
  \hline
  Shifr & \\
  \hline
  \end{tabular}
  
  \vspace{1cm}
  
  \begin{tabular}{|m{0.7cm}|m{10cm}|m{4cm}|}
  \hline
  № & Savol & Javob \\
  \hline
  1. & Hisoblang \(d\left( \int {f(x)dx} \right) = ?\) &  \\
  \hline
  2. & Musbat hadli qatorlarning yaqinlashuvchi bo'lishning Dalamber belgisini yozing &  \\
  \hline
  3. & Shartli ehtimollik formulasini yozing &  \\
  \hline
  4. & \((x_{0},\ y_{0})\) nuqtaning \(\varepsilon\) atrofi qanday belgilanadi &  \\
  \hline
  5. & Aniqmas integralni hisoblang: \(\int \frac{dx}{cos^{2}x}\). &  \\
  \hline
  6. & Aniq integralni hisoblang: \(\int_{- \pi/4}^{0}\frac{dx}{cos^{2}x}\). &  \\
  \hline
  7. & Integralni hisoblang: \(\int_{1}^{\infty}{\frac{1}{(x + 2)^{2}}dx}\). &  \\
  \hline
  8. & Sonli qatorning dastlabki uchta a'zosini yozing: \(\sum_{n = 1}^{\infty}\frac{n!}{2^{n}}\). &  \\
  \hline
  9. & Differensial tenglamaning umumiy echimini toping: \(xy' - 2y = 0\). &  \\
  \hline
  10. & Telefon raqamining oxirgi ikki tsifrasini unutib, tasodifan nomerlarni tera boshladi. Kerakli raqamni topish ehtimolligini hisoblang. &  \\
  \hline
  \end{tabular}
  
  \vspace{1cm}
  
  \begin{tabular}{lll}
  To'g'ri javoblar soni: \underline{\hspace{1.5cm}} & 
  Bahosi: \underline{\hspace{1.5cm}} & 
  Imtixon oluvchining imzosi: \underline{\hspace{2cm}} \\
  \end{tabular}
  
  \egroup
  
  \newpage
  
  
  \textbf{34-variant}\\
  
  \bgroup
  \def\arraystretch{1.6} % 1 is the default, change whatever you need
  
  \begin{tabular}{|m{5.7cm}|m{9.5cm}|}
  \hline
  Shifr & \\
  \hline
  \end{tabular}
  
  \vspace{1cm}
  
  \begin{tabular}{|m{0.7cm}|m{10cm}|m{4cm}|}
  \hline
  № & Savol & Javob \\
  \hline
  1. & Funksiyaning aniqlanish sohasi qanday belgilanadi &  \\
  \hline
  2. & Chiziqli differensial tenglama ko'rinishi &  \\
  \hline
  3. & Chiziqli differensial tenglamaning umumiy ko'rinishini yozing &  \\
  \hline
  4. & O'rin almashtirish formulasini yozing &  \\
  \hline
  5. & Aniqmas integralni hisoblang: \(\int {\left( x^{2} + \frac{1}{x} + \sin x \right)dx}\). &  \\
  \hline
  6. & Integralni hisoblang: \(\int_{1}^{\infty}{\frac{1}{x^{2}}dx}\). &  \\
  \hline
  7. & Aniq integralni hisoblang: \(\int_{0}^{\pi}{\sin xdx}\). &  \\
  \hline
  8. & Qatorning yig'indisini toping: \(\sum_{n = 1}^{\infty}\frac{1}{n(n + 1)}\). &  \\
  \hline
  9. & Chiziqli differensial tenglamaning umumiy echimini toping: \(y' + y = e^{x}\). &  \\
  \hline
  10. & Ikki kubikti bir marta tashlaganda tushgan otshkolarning yig'indisi 4 bo'lish ehtimolligini toping. &  \\
  \hline
  \end{tabular}
  
  \vspace{1cm}
  
  \begin{tabular}{lll}
  To'g'ri javoblar soni: \underline{\hspace{1.5cm}} & 
  Bahosi: \underline{\hspace{1.5cm}} & 
  Imtixon oluvchining imzosi: \underline{\hspace{2cm}} \\
  \end{tabular}
  
  \egroup
  
  \newpage
  
  
  \textbf{35-variant}\\
  
  \bgroup
  \def\arraystretch{1.6} % 1 is the default, change whatever you need
  
  \begin{tabular}{|m{5.7cm}|m{9.5cm}|}
  \hline
  Shifr & \\
  \hline
  \end{tabular}
  
  \vspace{1cm}
  
  \begin{tabular}{|m{0.7cm}|m{10cm}|m{4cm}|}
  \hline
  № & Savol & Javob \\
  \hline
  1. & Chiziqli defferensial tenglamaning umumiy echimini yozing &  \\
  \hline
  2. & Ehtimollikning qiymatlar sohasini yozing &  \\
  \hline
  3. & Musbat hadli qatorlarning yaqinlashuvchi bo'lishning Dalamber belgisini yozing &  \\
  \hline
  4. & Funksiyaning aniqlanish sohasi qanday belgilanadi &  \\
  \hline
  5. & Hisoblang: \(\int \left( x^{4} - \frac{1}{x} \right)dx\). &  \\
  \hline
  6. & Aniq integralni hisoblang: \(\int_{2}^{4}\frac{dx}{x}\). &  \\
  \hline
  7. & Hisoblang: \(\int_{1}^{2}{e^{x}dx}\). &  \\
  \hline
  8. & Funktsional qatorning yaqinlashishi sohasini yozing: \(\ln x + ln^{2}x + ... + ln^{n}x + ...\). &  \\
  \hline
  9. & Chiziqli differerntsial tenglamaning umumiy echimini toping \(y' + y = e^{- x}\). &  \\
  \hline
  10. & Korobkada 3 oq, 7 qora shar bor. Tasodifan uchta shar ketma-ket olindi. Ketma-ket olingan sharlarning qora, qora, oq degan ketma-ketlikda bo'lish ehtimolligini toping. &  \\
  \hline
  \end{tabular}
  
  \vspace{1cm}
  
  \begin{tabular}{lll}
  To'g'ri javoblar soni: \underline{\hspace{1.5cm}} & 
  Bahosi: \underline{\hspace{1.5cm}} & 
  Imtixon oluvchining imzosi: \underline{\hspace{2cm}} \\
  \end{tabular}
  
  \egroup
  
  \newpage
  
  
  \textbf{36-variant}\\
  
  \bgroup
  \def\arraystretch{1.6} % 1 is the default, change whatever you need
  
  \begin{tabular}{|m{5.7cm}|m{9.5cm}|}
  \hline
  Shifr & \\
  \hline
  \end{tabular}
  
  \vspace{1cm}
  
  \begin{tabular}{|m{0.7cm}|m{10cm}|m{4cm}|}
  \hline
  № & Savol & Javob \\
  \hline
  1. & Ehtimollikning klassik ta'rifining formulasini keltiring &  \\
  \hline
  2. & Guruhlash formulasini yozing &  \\
  \hline
  3. & Mumkin bo'magan hodisaning ehtimolligi nimaga teng? &  \\
  \hline
  4. & Chiziqli defferensial tenglamaning umumiy echimini yozing &  \\
  \hline
  5. & Aniqmas integralni hisoblang: \(\int {\left( 10x^{4} + 7x^{6} - 3 \right)dx}\). &  \\
  \hline
  6. & Aniq integralni hisoblang: \(\int_{1}^{3}\frac{2}{x + 1}dx\). &  \\
  \hline
  7. & Aniq integralni hisoblang: \(\int_{0}^{\frac{\pi}{2}}{\cos xdx}\). &  \\
  \hline
  8. & Funktsional qatorning yaqinlashishi sohasini yozing: \(\ln x + ln^{2}x + ... + ln^{n}x + ...\). &  \\
  \hline
  9. & Differensial tenglamani hisoblang: \(yy' = 4\). &  \\
  \hline
  10. & «BIOLOGIYA» sózining harflari aloqida kartochkalarga yozilib yopib, aralashtirib qo'yilgan. Barcha kartotshkalar tasodifan ketma-ket olinib ochilib, olinish tartibida stol ustiga tizilganda yana «BIOLOGIYA» sózining kelib chiqishi ehtimolligini toping. &  \\
  \hline
  \end{tabular}
  
  \vspace{1cm}
  
  \begin{tabular}{lll}
  To'g'ri javoblar soni: \underline{\hspace{1.5cm}} & 
  Bahosi: \underline{\hspace{1.5cm}} & 
  Imtixon oluvchining imzosi: \underline{\hspace{2cm}} \\
  \end{tabular}
  
  \egroup
  
  \newpage
  
  
  \textbf{37-variant}\\
  
  \bgroup
  \def\arraystretch{1.6} % 1 is the default, change whatever you need
  
  \begin{tabular}{|m{5.7cm}|m{9.5cm}|}
  \hline
  Shifr & \\
  \hline
  \end{tabular}
  
  \vspace{1cm}
  
  \begin{tabular}{|m{0.7cm}|m{10cm}|m{4cm}|}
  \hline
  № & Savol & Javob \\
  \hline
  1. & Hisoblang \(d\left( \int {f(x)dx} \right) = ?\) &  \\
  \hline
  2. & Ikki o'zgaruvchili funksiyaning ikkinchi tartibli aralash hosilalari qanday belgilanadi &  \\
  \hline
  3. & Funksiyaning \((x_{0},\ y_{0})\) nuqtadagi uzluksizligining formulasini yozing &  \\
  \hline
  4. & O'rin almashtirish formulasini yozing &  \\
  \hline
  5. & Integralni hisoblang:\(\int {(x - 1)^{20}}dx\). &  \\
  \hline
  6. & Aniq integralni hisoblang: \(\int_{0}^{\pi}{\sin xdx}\). &  \\
  \hline
  7. & Aniq integralni hisoblang: \(\int_{0}^{\frac{\pi}{2}}{\cos xdx}\). &  \\
  \hline
  8. & Funktsional qatorning yaqinlashish sohasini toping: \(x + \frac{x^{2}}{2^{2}} + ... + \frac{x^{n}}{n^{2}} + ...\) &  \\
  \hline
  9. & Differensial tenglamani eching: \(y' + xy = 0\). &  \\
  \hline
  10. & «MATEMATIKA» sózining harflari aloqida kartochkalarga yozilib yopib aralashtirib qo'yilgan. Barcha kartotshkalar tasodifan ketma-ket olinib ochilib, olinish tartibida stol ustiga tizilganda yana «MATEMATIKA» sózining kelib chiqishi ehtimolligini toping. &  \\
  \hline
  \end{tabular}
  
  \vspace{1cm}
  
  \begin{tabular}{lll}
  To'g'ri javoblar soni: \underline{\hspace{1.5cm}} & 
  Bahosi: \underline{\hspace{1.5cm}} & 
  Imtixon oluvchining imzosi: \underline{\hspace{2cm}} \\
  \end{tabular}
  
  \egroup
  
  \newpage
  
  
  \textbf{38-variant}\\
  
  \bgroup
  \def\arraystretch{1.6} % 1 is the default, change whatever you need
  
  \begin{tabular}{|m{5.7cm}|m{9.5cm}|}
  \hline
  Shifr & \\
  \hline
  \end{tabular}
  
  \vspace{1cm}
  
  \begin{tabular}{|m{0.7cm}|m{10cm}|m{4cm}|}
  \hline
  № & Savol & Javob \\
  \hline
  1. & Ehtimollikning geometrik ta'rifining formulasini yozing &  \\
  \hline
  2. & Ikki o'zgaruvchili funksiyaning ikkinchi tartibli xususiy hosilalari qanday belgilanadi &  \\
  \hline
  3. & Shartli ehtimollik formulasini yozing &  \\
  \hline
  4. & Bayes formulasini yozing &  \\
  \hline
  5. & Aniqmas integralni hisoblang: \(\int \frac{dx}{cos^{2}x}\). &  \\
  \hline
  6. & Hisoblang: \(\int_{1}^{2}{e^{x}dx}\). &  \\
  \hline
  7. & Aniq integralni hisoblang: \(\int_{1}^{3}\frac{2}{x + 1}dx\). &  \\
  \hline
  8. & Qatorning yigindisini toping: \(\sum_{n = 1}^{\infty}\frac{1}{n(n + 3)}\) &  \\
  \hline
  9. & Differensial tenglamaning umumiy echimini toping: \(y' = e^{x}\). &  \\
  \hline
  10. & Korobkada 15 oq, 18 qora shar bor. Tasodifan olingan bir shar oq bo'lish ehtimolligini toping. &  \\
  \hline
  \end{tabular}
  
  \vspace{1cm}
  
  \begin{tabular}{lll}
  To'g'ri javoblar soni: \underline{\hspace{1.5cm}} & 
  Bahosi: \underline{\hspace{1.5cm}} & 
  Imtixon oluvchining imzosi: \underline{\hspace{2cm}} \\
  \end{tabular}
  
  \egroup
  
  \newpage
  
  
  \textbf{39-variant}\\
  
  \bgroup
  \def\arraystretch{1.6} % 1 is the default, change whatever you need
  
  \begin{tabular}{|m{5.7cm}|m{9.5cm}|}
  \hline
  Shifr & \\
  \hline
  \end{tabular}
  
  \vspace{1cm}
  
  \begin{tabular}{|m{0.7cm}|m{10cm}|m{4cm}|}
  \hline
  № & Savol & Javob \\
  \hline
  1. & Funksiyaning \((x_{0},\ y_{0})\) nuqtadagi uzluksizlik shartini yozing &  \\
  \hline
  2. & Ikki o'zgaruvchili funksiyalar qanday belgilanadi &  \\
  \hline
  3. & Chiziqli differensial tenglamaning umumiy ko'rinishini yozing &  \\
  \hline
  4. & O'rin almashtirish formulasini yozing &  \\
  \hline
  5. & Integralni hisoblang: \(\int {\frac{1}{\sin x}dx}\). &  \\
  \hline
  6. & Aniq integralni hisoblang: \(\int_{2}^{4}\frac{dx}{x}\). &  \\
  \hline
  7. & Aniq integralni hisoblang: \(\int_{- \pi/4}^{0}\frac{dx}{cos^{2}x}\). &  \\
  \hline
  8. & Sonli qatorning dastlabki uchta a'zosini yozing: \(\sum_{n = 1}^{\infty}\frac{n!}{2^{n}}\). &  \\
  \hline
  9. & Chiziqli differensial tenglamaning umumiy echimini toping: \(y' + y = e^{x}\). &  \\
  \hline
  10. & Guruhdagi 20 talabadan nechta xil usul bilan 3 navbatchini tanlab olsa bo'ladi? &  \\
  \hline
  \end{tabular}
  
  \vspace{1cm}
  
  \begin{tabular}{lll}
  To'g'ri javoblar soni: \underline{\hspace{1.5cm}} & 
  Bahosi: \underline{\hspace{1.5cm}} & 
  Imtixon oluvchining imzosi: \underline{\hspace{2cm}} \\
  \end{tabular}
  
  \egroup
  
  \newpage
  
  
  \textbf{40-variant}\\
  
  \bgroup
  \def\arraystretch{1.6} % 1 is the default, change whatever you need
  
  \begin{tabular}{|m{5.7cm}|m{9.5cm}|}
  \hline
  Shifr & \\
  \hline
  \end{tabular}
  
  \vspace{1cm}
  
  \begin{tabular}{|m{0.7cm}|m{10cm}|m{4cm}|}
  \hline
  № & Savol & Javob \\
  \hline
  1. & Bo'laklab integrallash formulasini yozing &  \\
  \hline
  2. & Bernulli differensial tenglamasini yozing &  \\
  \hline
  3. & O'zgaruvchilari ajralgan differensial tenglamaning umumiy ko'rinishini yozing &  \\
  \hline
  4. & Nyuton-Leybnis formulasini yozing &  \\
  \hline
  5. & Integralni hisoblang: \(\int {\frac{1}{\sin x}dx}\). &  \\
  \hline
  6. & Integralni hisoblang: \(\int_{1}^{\infty}{\frac{1}{x^{2}}dx}\). &  \\
  \hline
  7. & Aniq integralni hisoblang: \(\int_{0}^{1}{(3x^{2}} + 1)dx\). &  \\
  \hline
  8. & Funktsional qatorning yaqinlashish sohasini toping:\(1 + x + ... + x^{n} + ...\) &  \\
  \hline
  9. & Differensial tenglamani hisoblang: \(yy' = 4\). &  \\
  \hline
  10. & Telefon raqamining oxirgi tsifrasini unutib, tasodifan nomerlarni tera boshladi. Kerakli raqamni topish ehtimolligini hisoblang. &  \\
  \hline
  \end{tabular}
  
  \vspace{1cm}
  
  \begin{tabular}{lll}
  To'g'ri javoblar soni: \underline{\hspace{1.5cm}} & 
  Bahosi: \underline{\hspace{1.5cm}} & 
  Imtixon oluvchining imzosi: \underline{\hspace{2cm}} \\
  \end{tabular}
  
  \egroup
  
  \newpage
  
  
  \textbf{41-variant}\\
  
  \bgroup
  \def\arraystretch{1.6} % 1 is the default, change whatever you need
  
  \begin{tabular}{|m{5.7cm}|m{9.5cm}|}
  \hline
  Shifr & \\
  \hline
  \end{tabular}
  
  \vspace{1cm}
  
  \begin{tabular}{|m{0.7cm}|m{10cm}|m{4cm}|}
  \hline
  № & Savol & Javob \\
  \hline
  1. & Ikki o'zgaruvchili funksiyaning toliq orttirmasi &  \\
  \hline
  2. & O'zgaruvchini almashtirib integrallash usulining formulasini yozing. &  \\
  \hline
  3. & Funksiya qanday usullarda beriladi &  \\
  \hline
  4. & Ehtimollikning qiymatlar sohasini yozing &  \\
  \hline
  5. & Aniqmas integralni hisoblang: \(\int {\left( x^{2} + \frac{1}{x} + \sin x \right)dx}\). &  \\
  \hline
  6. & Integralni hisoblang: \(\int_{1}^{\infty}{\frac{1}{(x + 2)^{2}}dx}\). &  \\
  \hline
  7. & Aniq integralni hisoblang: \(\int_{0}^{\pi}{\sin xdx}\). &  \\
  \hline
  8. & Qatorning yig'indisini toping: \(\sum_{n = 1}^{\infty}\frac{1}{n(n + 1)}\). &  \\
  \hline
  9. & Differensial tenglamaning umumiy echimini toping: \(xy' - 2y = 0\). &  \\
  \hline
  10. & Qutida 5 oq va 15 qora shar bor. Tasodifan olingan bitta sharning oq bo'lish ehtimolligini toping &  \\
  \hline
  \end{tabular}
  
  \vspace{1cm}
  
  \begin{tabular}{lll}
  To'g'ri javoblar soni: \underline{\hspace{1.5cm}} & 
  Bahosi: \underline{\hspace{1.5cm}} & 
  Imtixon oluvchining imzosi: \underline{\hspace{2cm}} \\
  \end{tabular}
  
  \egroup
  
  \newpage
  
  
  \textbf{42-variant}\\
  
  \bgroup
  \def\arraystretch{1.6} % 1 is the default, change whatever you need
  
  \begin{tabular}{|m{5.7cm}|m{9.5cm}|}
  \hline
  Shifr & \\
  \hline
  \end{tabular}
  
  \vspace{1cm}
  
  \begin{tabular}{|m{0.7cm}|m{10cm}|m{4cm}|}
  \hline
  № & Savol & Javob \\
  \hline
  1. & Musbat hadli qatorlarning yaqinlashuvchi bo'lishning Koshi belgisini yozing &  \\
  \hline
  2. & Sonli qatorning umumiy ko'rinishini yozing &  \\
  \hline
  3. & \(n\)-darajali ko'phadning umumiy ko'rinishi &  \\
  \hline
  4. & Ishonchli hodisaning ehtimolligi nimaga teng &  \\
  \hline
  5. & Aniqmas integralni hisoblang: \(\int {e^{x}dx}\) . &  \\
  \hline
  6. & Aniq integralni hisoblang: \(\int_{2}^{4}\frac{dx}{x}\). &  \\
  \hline
  7. & Integralni hisoblang: \(\int_{1}^{\infty}{\frac{1}{(x + 2)^{2}}dx}\). &  \\
  \hline
  8. & Qatorning yig'indisini hisoblang: \(\sum_{n = 1}^{\infty}\frac{1}{(2n - 1)(2n + 1)}\) &  \\
  \hline
  9. & Chiziqli differerntsial tenglamaning umumiy echimini toping \(y' + y = e^{- x}\). &  \\
  \hline
  10. & Idishda 5 oq, 8 qora shar bor. Idishdan tasodifan ketma-ket 3 shar olindi. Olingan sharlar oq, qora, qora degan ketma-ketlikda bo'lish ehtimolligini toping. &  \\
  \hline
  \end{tabular}
  
  \vspace{1cm}
  
  \begin{tabular}{lll}
  To'g'ri javoblar soni: \underline{\hspace{1.5cm}} & 
  Bahosi: \underline{\hspace{1.5cm}} & 
  Imtixon oluvchining imzosi: \underline{\hspace{2cm}} \\
  \end{tabular}
  
  \egroup
  
  \newpage
  
  
  \textbf{43-variant}\\
  
  \bgroup
  \def\arraystretch{1.6} % 1 is the default, change whatever you need
  
  \begin{tabular}{|m{5.7cm}|m{9.5cm}|}
  \hline
  Shifr & \\
  \hline
  \end{tabular}
  
  \vspace{1cm}
  
  \begin{tabular}{|m{0.7cm}|m{10cm}|m{4cm}|}
  \hline
  № & Savol & Javob \\
  \hline
  1. & Funksiyaning \((x_{0},\ y_{0})\) nuqtadagi hosilasining formulasini yozing &  \\
  \hline
  2. & Ikki o'zgaruvchili funksiyaning ekstremumining zarurli sharti &  \\
  \hline
  3. & Ko'phadni \((x - a)\) ga b'lgandagi qoldiq nimaga teng &  \\
  \hline
  4. & Hisoblang \(\left( \int {f(x)dx} \right)' = ?\); &  \\
  \hline
  5. & Aniqmas integralni hisoblang: \(\int {\left( 10x^{4} + 7x^{6} - 3 \right)dx}\). &  \\
  \hline
  6. & Hisoblang: \(\int_{1}^{2}{e^{x}dx}\). &  \\
  \hline
  7. & Aniq integralni hisoblang: \(\int_{1}^{3}\frac{2}{x + 1}dx\). &  \\
  \hline
  8. & Funktsional qatorning yaqinlashish sohasini toping: \(x + \frac{x^{2}}{2^{2}} + ... + \frac{x^{n}}{n^{2}} + ...\) &  \\
  \hline
  9. & Differensial tenglamaning umumiy echimini toping: \(y' = e^{x}\). &  \\
  \hline
  10. & «MATEMATIKA» sózining harflari aloqida kartochkalarga yozilib yopib aralashtirib qo'yilgan. Barcha kartotshkalar tasodifan ketma-ket olinib ochilib, olinish tartibida stol ustiga tizilganda yana «MATEMATIKA» sózining kelib chiqishi ehtimolligini toping. &  \\
  \hline
  \end{tabular}
  
  \vspace{1cm}
  
  \begin{tabular}{lll}
  To'g'ri javoblar soni: \underline{\hspace{1.5cm}} & 
  Bahosi: \underline{\hspace{1.5cm}} & 
  Imtixon oluvchining imzosi: \underline{\hspace{2cm}} \\
  \end{tabular}
  
  \egroup
  
  \newpage
  
  
  \textbf{44-variant}\\
  
  \bgroup
  \def\arraystretch{1.6} % 1 is the default, change whatever you need
  
  \begin{tabular}{|m{5.7cm}|m{9.5cm}|}
  \hline
  Shifr & \\
  \hline
  \end{tabular}
  
  \vspace{1cm}
  
  \begin{tabular}{|m{0.7cm}|m{10cm}|m{4cm}|}
  \hline
  № & Savol & Javob \\
  \hline
  1. & \((x_{0},\ y_{0})\) nuqtaning \(\varepsilon\) atrofi qanday belgilanadi &  \\
  \hline
  2. & Ikki o'zgaruvchili funksiyaning grafigi nimadan iborat &  \\
  \hline
  3. & Ikki o'zgaruvchili funksiyaning birinshi tartibli xususiy hosilalari qanday belgilanadi &  \\
  \hline
  4. & To'la ehtimollikning formulasini yozing &  \\
  \hline
  5. & Aniqmas integralni hisoblang: \(\int \frac{dx}{cos^{2}x}\). &  \\
  \hline
  6. & Aniq integralni hisoblang: \(\int_{0}^{\frac{\pi}{2}}{\cos xdx}\). &  \\
  \hline
  7. & Aniq integralni hisoblang: \(\int_{0}^{1}{(3x^{2}} + 1)dx\). &  \\
  \hline
  8. & Funktsional qatorning yaqinlashishi sohasini yozing: \(\ln x + ln^{2}x + ... + ln^{n}x + ...\). &  \\
  \hline
  9. & Chiziqli differerntsial tenglamaning umumiy echimini toping \(y' + y = e^{- x}\). &  \\
  \hline
  10. & Ikki kubikti bir marta tashlaganda tushgan otshkolarning yig'indisi 4 bo'lish ehtimolligini toping. &  \\
  \hline
  \end{tabular}
  
  \vspace{1cm}
  
  \begin{tabular}{lll}
  To'g'ri javoblar soni: \underline{\hspace{1.5cm}} & 
  Bahosi: \underline{\hspace{1.5cm}} & 
  Imtixon oluvchining imzosi: \underline{\hspace{2cm}} \\
  \end{tabular}
  
  \egroup
  
  \newpage
  
  
  \textbf{45-variant}\\
  
  \bgroup
  \def\arraystretch{1.6} % 1 is the default, change whatever you need
  
  \begin{tabular}{|m{5.7cm}|m{9.5cm}|}
  \hline
  Shifr & \\
  \hline
  \end{tabular}
  
  \vspace{1cm}
  
  \begin{tabular}{|m{0.7cm}|m{10cm}|m{4cm}|}
  \hline
  № & Savol & Javob \\
  \hline
  1. & Agar \(\sum_{n = 1}^{\infty}a_{n} = A,\sum_{n = 1}^{\infty}b_{n} = B\) bo'lsa, u holda \(\sum_{n = 1}^{\infty}\left( a_{n} + b_{n} \right) = ?\) &  \\
  \hline
  2. & Chekli additivlik aksiomasini yozing &  \\
  \hline
  3. & Ehtimollik fazosini yozing &  \\
  \hline
  4. & Funksional qatorning umumiy ko'rinishi &  \\
  \hline
  5. & Integralni hisoblang: \(\int {\frac{1}{\sin x}dx}\). &  \\
  \hline
  6. & Aniq integralni hisoblang: \(\int_{- \pi/4}^{0}\frac{dx}{cos^{2}x}\). &  \\
  \hline
  7. & Integralni hisoblang: \(\int_{1}^{\infty}{\frac{1}{x^{2}}dx}\). &  \\
  \hline
  8. & Qatorning yigindisini toping: \(\sum_{n = 1}^{\infty}\frac{1}{n(n + 3)}\) &  \\
  \hline
  9. & Chiziqli differensial tenglamaning umumiy echimini toping: \(y' + y = e^{x}\). &  \\
  \hline
  10. & Uchta bir xil korobkada oq va qora sharlar bor. 1-korobkada 5 oq, 8 qora shar, 2-korobkada 3 oq, 4 qora shar, 3-korobkada 2 oq, 3 qora shar bor. Uchta korobkaning biridan tasodifan olingan bir shar oq bo'lish ehtimolligini toping. &  \\
  \hline
  \end{tabular}
  
  \vspace{1cm}
  
  \begin{tabular}{lll}
  To'g'ri javoblar soni: \underline{\hspace{1.5cm}} & 
  Bahosi: \underline{\hspace{1.5cm}} & 
  Imtixon oluvchining imzosi: \underline{\hspace{2cm}} \\
  \end{tabular}
  
  \egroup
  
  \newpage
  
  
  \textbf{46-variant}\\
  
  \bgroup
  \def\arraystretch{1.6} % 1 is the default, change whatever you need
  
  \begin{tabular}{|m{5.7cm}|m{9.5cm}|}
  \hline
  Shifr & \\
  \hline
  \end{tabular}
  
  \vspace{1cm}
  
  \begin{tabular}{|m{0.7cm}|m{10cm}|m{4cm}|}
  \hline
  № & Savol & Javob \\
  \hline
  1. & Agar \(\sum_{n = 1}^{\infty}a_{n} = A,\sum_{n = 1}^{\infty}b_{n} = B\) bo'lsa, u holda \(\sum_{n = 1}^{\infty}\left( a_{n} - b_{n} \right) = ?\) &  \\
  \hline
  2. & Ikki o'zgaruvchili funksiyaning aniqlanish sohasi qayerda joylashsadi &  \\
  \hline
  3. & Chiziqli differensial tenglama ko'rinishi &  \\
  \hline
  4. & Ikki o'zgaruvchili funksiyaning \(M(x_{0},\ y_{0})\) noqtadagi uzluksizligining ta'rifi &  \\
  \hline
  5. & Integralni hisoblang: \(\int {2^{x}dx}\). &  \\
  \hline
  6. & Integralni hisoblang: \(\int_{1}^{\infty}{\frac{1}{x^{2}}dx}\). &  \\
  \hline
  7. & Aniq integralni hisoblang: \(\int_{2}^{4}\frac{dx}{x}\). &  \\
  \hline
  8. & Sonli qatorning dastlabki uchta a'zosini yozing: \(\sum_{n = 1}^{\infty}\frac{n!}{2^{n}}\). &  \\
  \hline
  9. & Differensial tenglamani hisoblang: \(yy' = 4\). &  \\
  \hline
  10. & Korobkada 3 oq, 7 qora shar bor. Tasodifan uchta shar ketma-ket olindi. Ketma-ket olingan sharlarning qora, qora, oq degan ketma-ketlikda bo'lish ehtimolligini toping. &  \\
  \hline
  \end{tabular}
  
  \vspace{1cm}
  
  \begin{tabular}{lll}
  To'g'ri javoblar soni: \underline{\hspace{1.5cm}} & 
  Bahosi: \underline{\hspace{1.5cm}} & 
  Imtixon oluvchining imzosi: \underline{\hspace{2cm}} \\
  \end{tabular}
  
  \egroup
  
  \newpage
  
  
  \textbf{47-variant}\\
  
  \bgroup
  \def\arraystretch{1.6} % 1 is the default, change whatever you need
  
  \begin{tabular}{|m{5.7cm}|m{9.5cm}|}
  \hline
  Shifr & \\
  \hline
  \end{tabular}
  
  \vspace{1cm}
  
  \begin{tabular}{|m{0.7cm}|m{10cm}|m{4cm}|}
  \hline
  № & Savol & Javob \\
  \hline
  1. & To'la ehtimollikning formulasini yozing &  \\
  \hline
  2. & Funksiyaning aniqlanish sohasi qanday belgilanadi &  \\
  \hline
  3. & Ikki o'zgaruvchili funksiyaning toliq orttirmasi &  \\
  \hline
  4. & Bo'laklab integrallash formulasini yozing &  \\
  \hline
  5. & Integralni hisoblang:\(\int {(x - 1)^{20}}dx\). &  \\
  \hline
  6. & Aniq integralni hisoblang: \(\int_{- \pi/4}^{0}\frac{dx}{cos^{2}x}\). &  \\
  \hline
  7. & Integralni hisoblang: \(\int_{1}^{\infty}{\frac{1}{(x + 2)^{2}}dx}\). &  \\
  \hline
  8. & Qatorning yig'indisini toping: \(\sum_{n = 1}^{\infty}\frac{1}{n(n + 1)}\). &  \\
  \hline
  9. & Differensial tenglamaning umumiy echimini toping: \(xy' - 2y = 0\). &  \\
  \hline
  10. & 50 ta buyumdan iborat partiyada 3 buyum yaroqsiz. Tasodifan olingan 8 ta buyumning ichida 1 ta buyumi yaroqsiz bo'lish ehtimolligini toping &  \\
  \hline
  \end{tabular}
  
  \vspace{1cm}
  
  \begin{tabular}{lll}
  To'g'ri javoblar soni: \underline{\hspace{1.5cm}} & 
  Bahosi: \underline{\hspace{1.5cm}} & 
  Imtixon oluvchining imzosi: \underline{\hspace{2cm}} \\
  \end{tabular}
  
  \egroup
  
  \newpage
  
  
  \textbf{48-variant}\\
  
  \bgroup
  \def\arraystretch{1.6} % 1 is the default, change whatever you need
  
  \begin{tabular}{|m{5.7cm}|m{9.5cm}|}
  \hline
  Shifr & \\
  \hline
  \end{tabular}
  
  \vspace{1cm}
  
  \begin{tabular}{|m{0.7cm}|m{10cm}|m{4cm}|}
  \hline
  № & Savol & Javob \\
  \hline
  1. & Guruhlash formulasini yozing &  \\
  \hline
  2. & Musbat hadli qatorlarning yaqinlashuvchi bo'lishning Koshi belgisini yozing &  \\
  \hline
  3. & Nyuton-Leybnis formulasini yozing &  \\
  \hline
  4. & Shartli ehtimollik formulasini yozing &  \\
  \hline
  5. & Aniqmas integralni hisoblang: \(\int \frac{dx}{cos^{2}x}\). &  \\
  \hline
  6. & Hisoblang: \(\int_{1}^{2}{e^{x}dx}\). &  \\
  \hline
  7. & Aniq integralni hisoblang: \(\int_{1}^{3}\frac{2}{x + 1}dx\). &  \\
  \hline
  8. & Qatorning yig'indisini hisoblang: \(\sum_{n = 1}^{\infty}\frac{1}{(2n - 1)(2n + 1)}\) &  \\
  \hline
  9. & Differensial tenglamani eching: \(y' + xy = 0\). &  \\
  \hline
  10. & «BIOLOGIYA» sózining harflari aloqida kartochkalarga yozilib yopib, aralashtirib qo'yilgan. Barcha kartotshkalar tasodifan ketma-ket olinib ochilib, olinish tartibida stol ustiga tizilganda yana «BIOLOGIYA» sózining kelib chiqishi ehtimolligini toping. &  \\
  \hline
  \end{tabular}
  
  \vspace{1cm}
  
  \begin{tabular}{lll}
  To'g'ri javoblar soni: \underline{\hspace{1.5cm}} & 
  Bahosi: \underline{\hspace{1.5cm}} & 
  Imtixon oluvchining imzosi: \underline{\hspace{2cm}} \\
  \end{tabular}
  
  \egroup
  
  \newpage
  
  
  \textbf{49-variant}\\
  
  \bgroup
  \def\arraystretch{1.6} % 1 is the default, change whatever you need
  
  \begin{tabular}{|m{5.7cm}|m{9.5cm}|}
  \hline
  Shifr & \\
  \hline
  \end{tabular}
  
  \vspace{1cm}
  
  \begin{tabular}{|m{0.7cm}|m{10cm}|m{4cm}|}
  \hline
  № & Savol & Javob \\
  \hline
  1. & Ehtimollik fazosini yozing &  \\
  \hline
  2. & Ikki o'zgaruvchili funksiyaning ikkinchi tartibli aralash hosilalari qanday belgilanadi &  \\
  \hline
  3. & Hisoblang \(d\left( \int {f(x)dx} \right) = ?\) &  \\
  \hline
  4. & O'zgaruvchilari ajralgan differensial tenglamaning umumiy ko'rinishini yozing &  \\
  \hline
  5. & Aniqmas integralni hisoblang: \(\int {\left( x^{2} + \frac{1}{x} + \sin x \right)dx}\). &  \\
  \hline
  6. & Aniq integralni hisoblang: \(\int_{0}^{\pi}{\sin xdx}\). &  \\
  \hline
  7. & Aniq integralni hisoblang: \(\int_{0}^{\frac{\pi}{2}}{\cos xdx}\). &  \\
  \hline
  8. & Funktsional qatorning yaqinlashish sohasini toping:\(1 + x + ... + x^{n} + ...\) &  \\
  \hline
  9. & Differensial tenglamaning umumiy echimini toping: \(xy' - 2y = 0\). &  \\
  \hline
  10. & Doyraning ichiga kvadrat cizilgan. Doyraning ichidan tasodifan belgilangan nuqtaning kvadratning ichida yotish ehtimolligini toping. &  \\
  \hline
  \end{tabular}
  
  \vspace{1cm}
  
  \begin{tabular}{lll}
  To'g'ri javoblar soni: \underline{\hspace{1.5cm}} & 
  Bahosi: \underline{\hspace{1.5cm}} & 
  Imtixon oluvchining imzosi: \underline{\hspace{2cm}} \\
  \end{tabular}
  
  \egroup
  
  \newpage
  
  
  \textbf{50-variant}\\
  
  \bgroup
  \def\arraystretch{1.6} % 1 is the default, change whatever you need
  
  \begin{tabular}{|m{5.7cm}|m{9.5cm}|}
  \hline
  Shifr & \\
  \hline
  \end{tabular}
  
  \vspace{1cm}
  
  \begin{tabular}{|m{0.7cm}|m{10cm}|m{4cm}|}
  \hline
  № & Savol & Javob \\
  \hline
  1. & Ishonchli hodisaning ehtimolligi nimaga teng &  \\
  \hline
  2. & Sonli qatorning umumiy ko'rinishini yozing &  \\
  \hline
  3. & Ikki o'zgaruvchili funksiyaning grafigi nimadan iborat &  \\
  \hline
  4. & Funksional qatorning umumiy ko'rinishi &  \\
  \hline
  5. & Aniqmas integralni hisoblang: \(\int {\left( 10x^{4} + 7x^{6} - 3 \right)dx}\). &  \\
  \hline
  6. & Aniq integralni hisoblang: \(\int_{0}^{1}{(3x^{2}} + 1)dx\). &  \\
  \hline
  7. & Aniq integralni hisoblang: \(\int_{0}^{\frac{\pi}{2}}{\cos xdx}\). &  \\
  \hline
  8. & Qatorning yigindisini toping: \(\sum_{n = 1}^{\infty}\frac{1}{n(n + 3)}\) &  \\
  \hline
  9. & Chiziqli differensial tenglamaning umumiy echimini toping: \(y' + y = e^{x}\). &  \\
  \hline
  10. & Telefon raqamining oxirgi ikki tsifrasini unutib, tasodifan nomerlarni tera boshladi. Kerakli raqamni topish ehtimolligini hisoblang. &  \\
  \hline
  \end{tabular}
  
  \vspace{1cm}
  
  \begin{tabular}{lll}
  To'g'ri javoblar soni: \underline{\hspace{1.5cm}} & 
  Bahosi: \underline{\hspace{1.5cm}} & 
  Imtixon oluvchining imzosi: \underline{\hspace{2cm}} \\
  \end{tabular}
  
  \egroup
  
  \newpage
  
  

\end{document}
