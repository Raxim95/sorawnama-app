\documentclass{article}
\usepackage[fontsize=12pt]{fontsize}
\usepackage[utf8]{inputenc}
\usepackage[T2A]{fontenc}
% \usepackage{unicode-math}

\usepackage{array}
\usepackage[a4paper,
left=7mm,
right=5mm,
top=7mm,]{geometry}
\usepackage{amsmath}
% \usepackage{amssymbol}
\usepackage{amsfonts}
\usepackage{setspace}



\renewcommand{\baselinestretch}{1} 

\everymath{\displaystyle}
\everydisplay{\displaystyle}
\linespread{1.5}

\DeclareMathOperator{\sign}{sign}


\begin{document}

\pagenumbering{gobble}


  \textbf{1-variant}\\
  
  \bgroup
  \def\arraystretch{1.6} % 1 is the default, change whatever you need
  
  \begin{tabular}{|m{5.7cm}|m{9.5cm}|}
  \hline
  Shifr & \\
  \hline
  \end{tabular}
  
  \vspace{1cm}
  
  \begin{tabular}{|m{0.7cm}|m{10cm}|m{4cm}|}
  \hline
  № & Soraw & Juwap \\
  \hline
  1. & Eki ózgeriwshili funkciyanıń ekinshi tártipli aralas tuwındıları qalay belgilenedi &  \\
  \hline
  2. & Esaplań \(d\left( \int{f(x)dx} \right) = ?\) &  \\
  \hline
  3. & Bernulli differenciallıq teńemesin jazıń &  \\
  \hline
  4. & Shártli itimallıq formulasın jazıń &  \\
  \hline
  5. & Anıq emes integraldı esaplań: \(\int{e^{x}dx}\) . &  \\
  \hline
  6. & Esaplań: \(\int\left( x^{4} - \frac{1}{x} \right)dx\). &  \\
  \hline
  7. & Anıq integraldı esaplań: \(\int_{0}^{\pi}{\sin xdx}\). &  \\
  \hline
  8. & Funkcional qatardıń jaqınlasıw oblastın tabıń: \(x + \frac{x^2 }{2^2 } + ... + \frac{x^{n}}{n^2 } + ...\) &  \\
  \hline
  9. & Differencial teńlemeni esaplań: \(yy' = 4\). &  \\
  \hline
  10. & Korobkada 15 aq, 18 qara shar bar. Tosınnan alınǵan bir shar aq bolıw itimallıǵın tabıń. &  \\
  \hline
  \end{tabular}
  
  \vspace{1cm}
  
  \begin{tabular}{lll}
  Tuwrı juwaplar sanı: \underline{\hspace{1.5cm}} & 
  Bahası: \underline{\hspace{1.5cm}} & 
  Imtixan alıwshınıń qolı: \underline{\hspace{2cm}} \\
  \end{tabular}
  
  \egroup
  
  \newpage
  
  
  \textbf{2-variant}\\
  
  \bgroup
  \def\arraystretch{1.6} % 1 is the default, change whatever you need
  
  \begin{tabular}{|m{5.7cm}|m{9.5cm}|}
  \hline
  Shifr & \\
  \hline
  \end{tabular}
  
  \vspace{1cm}
  
  \begin{tabular}{|m{0.7cm}|m{10cm}|m{4cm}|}
  \hline
  № & Soraw & Juwap \\
  \hline
  1. & Eki ózgeriwshili funkciyanıń ekinshi tártipli dara tuwındıları qalay belgilenedi &  \\
  \hline
  2. & Bóleklep inegrallaw formulasın jazıń &  \\
  \hline
  3. & Oń aǵzalı qatarlar ushın jıynaqlılıqtıń Dalamber belgisin jazıń &  \\
  \hline
  4. & Shekli additivlik aksiomasın jazıń &  \\
  \hline
  5. & Racional funkciyanı integrallań: \(\int{\frac{3}{(x - 1)(x + 2)}dx}\). &  \\
  \hline
  6. & Integraldı esaplań: \(\int_{1}^{\infty}{\frac{1}{x^2 }dx}\). &  \\
  \hline
  7. & Anıq integraldı esaplań: \(\int_{0}^{\frac{\pi}{2}}{\cos xdx}\). &  \\
  \hline
  8. & Qatardıń jıyındısın esaplań: \(\sum_{n = 1}^{\infty}\frac{1}{(2n - 1)(2n + 1)}\). &  \\
  \hline
  9. & Differencial teńlemeni sheshiń: \(y' + xy = 0\). &  \\
  \hline
  10. & «BIOLOGIYA» sóziniń háripleri bólek kartochkalarǵa jazılıp jawıp, aralastırılıp qoyılǵan. Barlıq kartochkalar tosınnan izbe-iz alınıp ashılıp, alınıw tártibinde stol ústine dizilgende taǵı «BIOLOGIYA» sóziniń kelip shıǵıw itimallıǵın tabıń. &  \\
  \hline
  \end{tabular}
  
  \vspace{1cm}
  
  \begin{tabular}{lll}
  Tuwrı juwaplar sanı: \underline{\hspace{1.5cm}} & 
  Bahası: \underline{\hspace{1.5cm}} & 
  Imtixan alıwshınıń qolı: \underline{\hspace{2cm}} \\
  \end{tabular}
  
  \egroup
  
  \newpage
  
  
  \textbf{3-variant}\\
  
  \bgroup
  \def\arraystretch{1.6} % 1 is the default, change whatever you need
  
  \begin{tabular}{|m{5.7cm}|m{9.5cm}|}
  \hline
  Shifr & \\
  \hline
  \end{tabular}
  
  \vspace{1cm}
  
  \begin{tabular}{|m{0.7cm}|m{10cm}|m{4cm}|}
  \hline
  № & Soraw & Juwap \\
  \hline
  1. & Funkciyanıń anıqlanıw oblastı qalay belgilenedi &  \\
  \hline
  2. & Funkcianıń \((x_{0},\ y_{0})\) noqattaǵı úzliksizliginiń formulasın jazıń &  \\
  \hline
  3. & Sızıqlı differenciallıq teńlemeniń uluwma kórinisin jazıń &  \\
  \hline
  4. & Orın awıstırıw formulasın jazıń &  \\
  \hline
  5. & Integraldı esaplań: \(\int{(x + \sin x)dx}\). &  \\
  \hline
  6. & Anıq integraldı esaplań: \(\int_{1}^{3}\frac{2}{x + 1}dx\). &  \\
  \hline
  7. & Anıq integraldı esaplań: \(\int_{0}^{1}{(3x^2 } + 1)dx\). &  \\
  \hline
  8. & Funkcional qatardıń jıynaqlılıq oblastın tabıń:\(1 + x + ... + x^{n} + ...\) &  \\
  \hline
  9. & Differencial teńlemeniń ulıwma sheshimin tabıń: \(y' = e^{x}\). &  \\
  \hline
  10. & Korobkada 3 aq, 7 qara shar bar. Tosınnan úsh shar izbe-iz alındı. Izbe-iz alınǵan sharlardıń qara, qara, aq degen izbe-izlikte bolıw itimallıǵın tabıń. &  \\
  \hline
  \end{tabular}
  
  \vspace{1cm}
  
  \begin{tabular}{lll}
  Tuwrı juwaplar sanı: \underline{\hspace{1.5cm}} & 
  Bahası: \underline{\hspace{1.5cm}} & 
  Imtixan alıwshınıń qolı: \underline{\hspace{2cm}} \\
  \end{tabular}
  
  \egroup
  
  \newpage
  
  
  \textbf{4-variant}\\
  
  \bgroup
  \def\arraystretch{1.6} % 1 is the default, change whatever you need
  
  \begin{tabular}{|m{5.7cm}|m{9.5cm}|}
  \hline
  Shifr & \\
  \hline
  \end{tabular}
  
  \vspace{1cm}
  
  \begin{tabular}{|m{0.7cm}|m{10cm}|m{4cm}|}
  \hline
  № & Soraw & Juwap \\
  \hline
  1. & Funkciya qanday usıllarda beriledi &  \\
  \hline
  2. & \((x_0,y_0)\) tochkanıń \(\varepsilon\) dógeregi qalay belgilenedi &  \\
  \hline
  3. & Sanlı qatardıń uluwma kórinisin jazıń &  \\
  \hline
  4. & Isenimli waqıyanıń itimallıǵı nege teń &  \\
  \hline
  5. & Integraldı esaplań:\(\int{(x - 1)^{20}}dx\). &  \\
  \hline
  6. & Integraldı esaplań: \(\int_{1}^{\infty}{\frac{1}{(x + 2)^2 }dx}\). &  \\
  \hline
  7. & Anıq integraldı esaplań: \(\int_{2}^{4}\frac{dx}{x}\). &  \\
  \hline
  8. & Qatardıń qosındısın tabıń: \(\sum_{n = 1}^{\infty}\frac{1}{n(n + 3)}\). &  \\
  \hline
  9. & Sızıqlı differerncial teńlemeniń uluwma sheshimin tabıń \(y' + y = e^{- x}\). &  \\
  \hline
  10. & 50 buyımnan ibarat partiyada 3 buyım jaramsız. Tosınnan alınǵan 8 buyımnıń ishinde 1 buyımı jaramsız bolıw itimallıǵın tabıń &  \\
  \hline
  \end{tabular}
  
  \vspace{1cm}
  
  \begin{tabular}{lll}
  Tuwrı juwaplar sanı: \underline{\hspace{1.5cm}} & 
  Bahası: \underline{\hspace{1.5cm}} & 
  Imtixan alıwshınıń qolı: \underline{\hspace{2cm}} \\
  \end{tabular}
  
  \egroup
  
  \newpage
  
  
  \textbf{5-variant}\\
  
  \bgroup
  \def\arraystretch{1.6} % 1 is the default, change whatever you need
  
  \begin{tabular}{|m{5.7cm}|m{9.5cm}|}
  \hline
  Shifr & \\
  \hline
  \end{tabular}
  
  \vspace{1cm}
  
  \begin{tabular}{|m{0.7cm}|m{10cm}|m{4cm}|}
  \hline
  № & Soraw & Juwap \\
  \hline
  1. & Eki ózgeriwshili funkciyalar qalay belgilenedi &  \\
  \hline
  2. & Ózgeriwshini almastırıp integrallaw usılıniń formulasın jazıń. &  \\
  \hline
  3. & Oń aǵzalı qatarlar ushın jıynaqlılıqtıń Koshi belgisin jazıń &  \\
  \hline
  4. & Tolıq itimallıqtıń formulasın jazıń &  \\
  \hline
  5. & Anıq emes integraldı esaplań: \(\int\frac{dx}{cos^2 x}\). &  \\
  \hline
  6. & Esaplań: \(\int_{1}^2 {e^{x}dx}\). &  \\
  \hline
  7. & Anıq integraldı esaplań: \(\int_{- \pi/4}^{0}\frac{dx}{cos^2 x}\). &  \\
  \hline
  8. & Qatardıń qosındısın tabıń: \(\sum_{n = 1}^{\infty}\frac{1}{n(n + 1)}\). &  \\
  \hline
  9. & Differencial teńlemeniń ulıwma sheshimin tabıń: \(xy' - 2y = 0\). &  \\
  \hline
  10. & Telefon nomerdiń aqırǵı eki cifrasın umıtıp, tosınnan nomerlerdi tere basladı. Kerekli nomerdi tabıw itimallıǵın esaplań. &  \\
  \hline
  \end{tabular}
  
  \vspace{1cm}
  
  \begin{tabular}{lll}
  Tuwrı juwaplar sanı: \underline{\hspace{1.5cm}} & 
  Bahası: \underline{\hspace{1.5cm}} & 
  Imtixan alıwshınıń qolı: \underline{\hspace{2cm}} \\
  \end{tabular}
  
  \egroup
  
  \newpage
  
  
  \textbf{6-variant}\\
  
  \bgroup
  \def\arraystretch{1.6} % 1 is the default, change whatever you need
  
  \begin{tabular}{|m{5.7cm}|m{9.5cm}|}
  \hline
  Shifr & \\
  \hline
  \end{tabular}
  
  \vspace{1cm}
  
  \begin{tabular}{|m{0.7cm}|m{10cm}|m{4cm}|}
  \hline
  № & Soraw & Juwap \\
  \hline
  1. & Eki ózgeriwshili funkciyanıń grafigi neden ibarat &  \\
  \hline
  2. & Eki ózgeriwshli funkciyanıń \(M(x_{0}, y_{0})\) noqattaǵı úzliksizliginiń anıqlaması &  \\
  \hline
  3. & Eger \(\sum_{n = 1}^{\infty}a_{n} = A,\ \sum_{n = 1}^{\infty}b_{n} = B\) bolsa, onda \(\sum_{n = 1}^{\infty}\left( a_{n} - b_{n} \right) = ?\) &  \\
  \hline
  4. & Gruppalaw formulasın jazıń &  \\
  \hline
  5. & Racional funkciyanı integrallań: \(\int{\frac{5}{(x - 3)(x + 2)}dx}\). &  \\
  \hline
  6. & Integraldı esaplań: \(\int{2^{x}dx}\). &  \\
  \hline
  7. & Anıq integraldı esaplań: \(\int_{- \pi/4}^{0}\frac{dx}{cos^2 x}\). &  \\
  \hline
  8. & Funkcional qatardıń jıynaqlılıq oblastın jazıń: \(\ln x + ln^2 x + ... + ln^{n}x + ...\). &  \\
  \hline
  9. & Sızıqlı differencial teńlemeniń ulwma sheshimin tabıń: \(y' + y = e^{x}\). &  \\
  \hline
  10. & Dóngelektiń ishine kvadrat sızılǵan. Dóngelektiń ishinen tosınnan belgilengen noqattıń kvadrattıń ishinde jatıw itimallıǵın tabıń. &  \\
  \hline
  \end{tabular}
  
  \vspace{1cm}
  
  \begin{tabular}{lll}
  Tuwrı juwaplar sanı: \underline{\hspace{1.5cm}} & 
  Bahası: \underline{\hspace{1.5cm}} & 
  Imtixan alıwshınıń qolı: \underline{\hspace{2cm}} \\
  \end{tabular}
  
  \egroup
  
  \newpage
  
  
  \textbf{7-variant}\\
  
  \bgroup
  \def\arraystretch{1.6} % 1 is the default, change whatever you need
  
  \begin{tabular}{|m{5.7cm}|m{9.5cm}|}
  \hline
  Shifr & \\
  \hline
  \end{tabular}
  
  \vspace{1cm}
  
  \begin{tabular}{|m{0.7cm}|m{10cm}|m{4cm}|}
  \hline
  № & Soraw & Juwap \\
  \hline
  1. & \(n\)-dárejeli kóp aǵzalınıń uluwma kórinisi &  \\
  \hline
  2. & Anıq integraldı esaplawdıń Nyuton-Leybnic formulasın jazıń &  \\
  \hline
  3. & Sızıqlı defferencial teńlemeniń uluwma sheshimin jazıń &  \\
  \hline
  4. & Itimallıq keńisligin jazıń &  \\
  \hline
  5. & Anıq emes integraldı esaplań: \(\int{\left( 10x^{4} + 7x^{6} - 3 \right)dx}\). &  \\
  \hline
  6. & Anıq integraldı esaplań: \(\int_{1}^{3}\frac{2}{x + 1}dx\). &  \\
  \hline
  7. & Anıq integraldı esaplań: \(\int_{0}^{1}{(3x^2 } + 1)dx\). &  \\
  \hline
  8. & Sanlı qatardıń baslanǵısh úsh aǵzasın jazıń: \(\sum_{n = 1}^{\infty}\frac{n!}{2^{n}}\). &  \\
  \hline
  9. & Sızıqlı differerncial teńlemeniń uluwma sheshimin tabıń \(y' + y = e^{- x}\). &  \\
  \hline
  10. & «MATEMATIKA» sóziniń háripleri bólek kartochkalarǵa jazılıp jawıp aralastırılıp qoyılǵan. Barlıq kartochkalar tosınnan izbe-iz alınıp ashılıp, alınıw tártibinde stol ústine dizilgende taǵı «MATEMATIKA» sóziniń kelip shıǵıw itimallıǵın tabıń. &  \\
  \hline
  \end{tabular}
  
  \vspace{1cm}
  
  \begin{tabular}{lll}
  Tuwrı juwaplar sanı: \underline{\hspace{1.5cm}} & 
  Bahası: \underline{\hspace{1.5cm}} & 
  Imtixan alıwshınıń qolı: \underline{\hspace{2cm}} \\
  \end{tabular}
  
  \egroup
  
  \newpage
  
  
  \textbf{8-variant}\\
  
  \bgroup
  \def\arraystretch{1.6} % 1 is the default, change whatever you need
  
  \begin{tabular}{|m{5.7cm}|m{9.5cm}|}
  \hline
  Shifr & \\
  \hline
  \end{tabular}
  
  \vspace{1cm}
  
  \begin{tabular}{|m{0.7cm}|m{10cm}|m{4cm}|}
  \hline
  № & Soraw & Juwap \\
  \hline
  1. & Kóp aǵzalını \((x - a)\) ǵa bólgendegi qaldıq nege teń &  \\
  \hline
  2. & Esaplań \(\left( \int{f(x)dx} \right)' = ?\). &  \\
  \hline
  3. & Eger \(\sum_{n = 1}^{\infty}a_{n} = A,\ \sum_{n = 1}^{\infty}b_{n} = B\) bolsa, onda \(\sum_{n = 1}^{\infty}\left( a_{n} + b_{n} \right) = ?\) &  \\
  \hline
  4. & Orın almastırıw formulasın jazıń &  \\
  \hline
  5. & Integraldı esaplań: \(\int{\frac{1}{\sin x}dx}\). &  \\
  \hline
  6. & Esaplań: \(\int\left( x^{4} - \frac{1}{x} \right)dx\). &  \\
  \hline
  7. & Anıq integraldı esaplań: \(\int_{0}^{\pi}{\sin xdx}\). &  \\
  \hline
  8. & Sanlı qatardıń baslanǵısh úsh aǵzasın jazıń: \(\sum_{n = 1}^{\infty}\frac{n!}{2^{n}}\). &  \\
  \hline
  9. & Differencial teńlemeniń ulıwma sheshimin tabıń: \(xy' - 2y = 0\). &  \\
  \hline
  10. & Úsh birdey korobkada aq hám qara sharlar bar. 1-korobkada 5 aq, 8 qara shar, 2-korobkada 3 aq, 4 qara shar, 3-korobkada 2 aq, 3 qara shar bar. Úsh korobkaniń birewinen tosınnan alınǵan bir shar aq bolıw itimallıǵın tabıń. &  \\
  \hline
  \end{tabular}
  
  \vspace{1cm}
  
  \begin{tabular}{lll}
  Tuwrı juwaplar sanı: \underline{\hspace{1.5cm}} & 
  Bahası: \underline{\hspace{1.5cm}} & 
  Imtixan alıwshınıń qolı: \underline{\hspace{2cm}} \\
  \end{tabular}
  
  \egroup
  
  \newpage
  
  
  \textbf{9-variant}\\
  
  \bgroup
  \def\arraystretch{1.6} % 1 is the default, change whatever you need
  
  \begin{tabular}{|m{5.7cm}|m{9.5cm}|}
  \hline
  Shifr & \\
  \hline
  \end{tabular}
  
  \vspace{1cm}
  
  \begin{tabular}{|m{0.7cm}|m{10cm}|m{4cm}|}
  \hline
  № & Soraw & Juwap \\
  \hline
  1. & Eki ózgeriwshili funkciyanıń birinshi tártipli dara tuwındıları qalay belgilenedi &  \\
  \hline
  2. & Funkcianıń \((x_{0},\ y_{0})\) noqattaǵı tuwındısınıń formulasın jazıń &  \\
  \hline
  3. & Funkcionallıq qatardıń uluwma kórinisi &  \\
  \hline
  4. & Itimallıqtıń mánisler oblastın jazıń &  \\
  \hline
  5. & Anıq emes integraldı esaplań: \(\int{\left( x^2  + \frac{1}{x} + \sin x \right)dx}\). &  \\
  \hline
  6. & Integraldı esaplań: \(\int_{1}^{\infty}{\frac{1}{x^2 }dx}\). &  \\
  \hline
  7. & Anıq integraldı esaplań: \(\int_{0}^{\frac{\pi}{2}}{\cos xdx}\). &  \\
  \hline
  8. & Qatardıń qosındısın tabıń: \(\sum_{n = 1}^{\infty}\frac{1}{n(n + 3)}\). &  \\
  \hline
  9. & Differencial teńlemeni esaplań: \(yy' = 4\). &  \\
  \hline
  10. & Telefon nomerdiń aqırǵı cifrasın umıtıp, tosınnan nomerlerdi tere basladı. Kerekli nomerdi tabıw itimallıǵın esaplań. &  \\
  \hline
  \end{tabular}
  
  \vspace{1cm}
  
  \begin{tabular}{lll}
  Tuwrı juwaplar sanı: \underline{\hspace{1.5cm}} & 
  Bahası: \underline{\hspace{1.5cm}} & 
  Imtixan alıwshınıń qolı: \underline{\hspace{2cm}} \\
  \end{tabular}
  
  \egroup
  
  \newpage
  
  
  \textbf{10-variant}\\
  
  \bgroup
  \def\arraystretch{1.6} % 1 is the default, change whatever you need
  
  \begin{tabular}{|m{5.7cm}|m{9.5cm}|}
  \hline
  Shifr & \\
  \hline
  \end{tabular}
  
  \vspace{1cm}
  
  \begin{tabular}{|m{0.7cm}|m{10cm}|m{4cm}|}
  \hline
  № & Soraw & Juwap \\
  \hline
  1. & Eki ózgeriwshili funkciyanıń ekstremumınıń zárúrli shárti &  \\
  \hline
  2. & \((x_0,y_0)\) tochkanıń \(\varepsilon\) dógeregi qalay belgilenedi &  \\
  \hline
  3. & Ózgeriwshileri ajıralǵan differenciallıq teńlemesiniń uluwma kórinisin jazıń &  \\
  \hline
  4. & Múmkin emes waqıyanıń itimaıllıǵı nege teń &  \\
  \hline
  5. & Racional funkciyanı integrallań: \(\int{\frac{5}{(x - 3)(x + 2)}dx}\). &  \\
  \hline
  6. & Esaplań: \(\int_{1}^2 {e^{x}dx}\). &  \\
  \hline
  7. & Anıq integraldı esaplań: \(\int_{2}^{4}\frac{dx}{x}\). &  \\
  \hline
  8. & Funkcional qatardıń jıynaqlılıq oblastın jazıń: \(\ln x + ln^2 x + ... + ln^{n}x + ...\). &  \\
  \hline
  9. & Sızıqlı differencial teńlemeniń ulwma sheshimin tabıń: \(y' + y = e^{x}\). &  \\
  \hline
  10. & Ídısta 5 aq, 8 qara shar bar. Ídıstan tosınnan izbe-iz 3 shar alındı. Alınǵan sharlar aq, qara, qara degen izbe-izlikte bolıw itimallıǵın tabıń. &  \\
  \hline
  \end{tabular}
  
  \vspace{1cm}
  
  \begin{tabular}{lll}
  Tuwrı juwaplar sanı: \underline{\hspace{1.5cm}} & 
  Bahası: \underline{\hspace{1.5cm}} & 
  Imtixan alıwshınıń qolı: \underline{\hspace{2cm}} \\
  \end{tabular}
  
  \egroup
  
  \newpage
  
  
  \textbf{11-variant}\\
  
  \bgroup
  \def\arraystretch{1.6} % 1 is the default, change whatever you need
  
  \begin{tabular}{|m{5.7cm}|m{9.5cm}|}
  \hline
  Shifr & \\
  \hline
  \end{tabular}
  
  \vspace{1cm}
  
  \begin{tabular}{|m{0.7cm}|m{10cm}|m{4cm}|}
  \hline
  № & Soraw & Juwap \\
  \hline
  1. & Eki ózgeriwshili funkciyanıń tolıq ósimi &  \\
  \hline
  2. & Eki ózgeriwshli funkciyanıń \(M(x_{0}, y_{0})\) noqattaǵı úzliksizliginiń anıqlaması &  \\
  \hline
  3. & Oń aǵzalı qatarlar ushın jıynaqlılıqtıń Dalamber belgisin jazıń &  \\
  \hline
  4. & Itimallıqtıń klassikalıq anıqlamasınıń formulasın keltiriń &  \\
  \hline
  5. & Integraldı esaplań: \(\int{\frac{1}{\sin x}dx}\). &  \\
  \hline
  6. & Integraldı esaplań: \(\int{2^{x}dx}\). &  \\
  \hline
  7. & Anıq integraldı esaplań: \(\int_{2}^{4}\frac{dx}{x}\). &  \\
  \hline
  8. & Qatardıń jıyındısın esaplań: \(\sum_{n = 1}^{\infty}\frac{1}{(2n - 1)(2n + 1)}\). &  \\
  \hline
  9. & Differencial teńlemeni sheshiń: \(y' + xy = 0\). &  \\
  \hline
  10. & Tiyindi eki márte taslaǵanda, keminde bir márte san tárepi túsiw itimallıǵın tabıń. &  \\
  \hline
  \end{tabular}
  
  \vspace{1cm}
  
  \begin{tabular}{lll}
  Tuwrı juwaplar sanı: \underline{\hspace{1.5cm}} & 
  Bahası: \underline{\hspace{1.5cm}} & 
  Imtixan alıwshınıń qolı: \underline{\hspace{2cm}} \\
  \end{tabular}
  
  \egroup
  
  \newpage
  
  
  \textbf{12-variant}\\
  
  \bgroup
  \def\arraystretch{1.6} % 1 is the default, change whatever you need
  
  \begin{tabular}{|m{5.7cm}|m{9.5cm}|}
  \hline
  Shifr & \\
  \hline
  \end{tabular}
  
  \vspace{1cm}
  
  \begin{tabular}{|m{0.7cm}|m{10cm}|m{4cm}|}
  \hline
  № & Soraw & Juwap \\
  \hline
  1. & Eki ózgeriwshili funkciyanıń anıqlanıw oblastı qay jerde jaylasadı &  \\
  \hline
  2. & Anıq integraldı esaplawdıń Nyuton-Leybnic formulasın jazıń &  \\
  \hline
  3. & Sızıqlı differenciallıq teńlemeniń uluwma kórinisin jazıń &  \\
  \hline
  4. & Bayes formulasın jazıń &  \\
  \hline
  5. & Anıq emes integraldı esaplań: \(\int{\left( x^2  + \frac{1}{x} + \sin x \right)dx}\). &  \\
  \hline
  6. & Integraldı esaplań: \(\int_{1}^{\infty}{\frac{1}{(x + 2)^2 }dx}\). &  \\
  \hline
  7. & Anıq integraldı esaplań: \(\int_{0}^{\pi}{\sin xdx}\). &  \\
  \hline
  8. & Funkcional qatardıń jıynaqlılıq oblastın tabıń:\(1 + x + ... + x^{n} + ...\) &  \\
  \hline
  9. & Differencial teńlemeniń ulıwma sheshimin tabıń: \(y' = e^{x}\). &  \\
  \hline
  10. & Qutada 5 aq hám 15 qara shar bar. Tosınnan alınǵan bir shardıń aq bolıw itimallıǵın tabıń &  \\
  \hline
  \end{tabular}
  
  \vspace{1cm}
  
  \begin{tabular}{lll}
  Tuwrı juwaplar sanı: \underline{\hspace{1.5cm}} & 
  Bahası: \underline{\hspace{1.5cm}} & 
  Imtixan alıwshınıń qolı: \underline{\hspace{2cm}} \\
  \end{tabular}
  
  \egroup
  
  \newpage
  
  
  \textbf{13-variant}\\
  
  \bgroup
  \def\arraystretch{1.6} % 1 is the default, change whatever you need
  
  \begin{tabular}{|m{5.7cm}|m{9.5cm}|}
  \hline
  Shifr & \\
  \hline
  \end{tabular}
  
  \vspace{1cm}
  
  \begin{tabular}{|m{0.7cm}|m{10cm}|m{4cm}|}
  \hline
  № & Soraw & Juwap \\
  \hline
  1. & Eki ózgeriwshili funkciyanıń tolıq ósimi &  \\
  \hline
  2. & Esaplań \(\left( \int{f(x)dx} \right)' = ?\). &  \\
  \hline
  3. & Sanlı qatardıń uluwma kórinisin jazıń &  \\
  \hline
  4. & Itimmallıqtıń geometriyalıq anıqlamasınıń formulasın jazıń &  \\
  \hline
  5. & Integraldı esaplań:\(\int{(x - 1)^{20}}dx\). &  \\
  \hline
  6. & Integraldı esaplań: \(\int{2^{x}dx}\). &  \\
  \hline
  7. & Anıq integraldı esaplań: \(\int_{- \pi/4}^{0}\frac{dx}{cos^2 x}\). &  \\
  \hline
  8. & Funkcional qatardıń jaqınlasıw oblastın tabıń: \(x + \frac{x^2 }{2^2 } + ... + \frac{x^{n}}{n^2 } + ...\) &  \\
  \hline
  9. & Differencial teńlemeni sheshiń: \(y' + xy = 0\). &  \\
  \hline
  10. & Gruppadaǵı 20 studentten neshe túrli usıl menen 3 náwbetshini saylap alıwǵa boladı? &  \\
  \hline
  \end{tabular}
  
  \vspace{1cm}
  
  \begin{tabular}{lll}
  Tuwrı juwaplar sanı: \underline{\hspace{1.5cm}} & 
  Bahası: \underline{\hspace{1.5cm}} & 
  Imtixan alıwshınıń qolı: \underline{\hspace{2cm}} \\
  \end{tabular}
  
  \egroup
  
  \newpage
  
  
  \textbf{14-variant}\\
  
  \bgroup
  \def\arraystretch{1.6} % 1 is the default, change whatever you need
  
  \begin{tabular}{|m{5.7cm}|m{9.5cm}|}
  \hline
  Shifr & \\
  \hline
  \end{tabular}
  
  \vspace{1cm}
  
  \begin{tabular}{|m{0.7cm}|m{10cm}|m{4cm}|}
  \hline
  № & Soraw & Juwap \\
  \hline
  1. & Eki ózgeriwshili funkciyanıń ekinshi tártipli dara tuwındıları qalay belgilenedi &  \\
  \hline
  2. & Esaplań \(d\left( \int{f(x)dx} \right) = ?\) &  \\
  \hline
  3. & Sızıqlı defferencial teńlemeniń uluwma sheshimin jazıń &  \\
  \hline
  4. & Tolıq itimallıqtıń formulasın jazıń &  \\
  \hline
  5. & Anıq emes integraldı esaplań: \(\int{e^{x}dx}\) . &  \\
  \hline
  6. & Anıq integraldı esaplań: \(\int_{1}^{3}\frac{2}{x + 1}dx\). &  \\
  \hline
  7. & Anıq integraldı esaplań: \(\int_{0}^{\frac{\pi}{2}}{\cos xdx}\). &  \\
  \hline
  8. & Qatardıń qosındısın tabıń: \(\sum_{n = 1}^{\infty}\frac{1}{n(n + 1)}\). &  \\
  \hline
  9. & Sızıqlı differencial teńlemeniń ulwma sheshimin tabıń: \(y' + y = e^{x}\). &  \\
  \hline
  10. & Eki kubikti bir márte taslaǵanda túsken ochkolardıń qosındısı 4 bolıw itimallıǵın tabıń. &  \\
  \hline
  \end{tabular}
  
  \vspace{1cm}
  
  \begin{tabular}{lll}
  Tuwrı juwaplar sanı: \underline{\hspace{1.5cm}} & 
  Bahası: \underline{\hspace{1.5cm}} & 
  Imtixan alıwshınıń qolı: \underline{\hspace{2cm}} \\
  \end{tabular}
  
  \egroup
  
  \newpage
  
  
  \textbf{15-variant}\\
  
  \bgroup
  \def\arraystretch{1.6} % 1 is the default, change whatever you need
  
  \begin{tabular}{|m{5.7cm}|m{9.5cm}|}
  \hline
  Shifr & \\
  \hline
  \end{tabular}
  
  \vspace{1cm}
  
  \begin{tabular}{|m{0.7cm}|m{10cm}|m{4cm}|}
  \hline
  № & Soraw & Juwap \\
  \hline
  1. & Eki ózgeriwshili funkciyanıń ekinshi tártipli aralas tuwındıları qalay belgilenedi &  \\
  \hline
  2. & Ózgeriwshini almastırıp integrallaw usılıniń formulasın jazıń. &  \\
  \hline
  3. & Eger \(\sum_{n = 1}^{\infty}a_{n} = A,\ \sum_{n = 1}^{\infty}b_{n} = B\) bolsa, onda \(\sum_{n = 1}^{\infty}\left( a_{n} + b_{n} \right) = ?\) &  \\
  \hline
  4. & Orın awıstırıw formulasın jazıń &  \\
  \hline
  5. & Integraldı esaplań: \(\int{(x + \sin x)dx}\). &  \\
  \hline
  6. & Esaplań: \(\int_{1}^2 {e^{x}dx}\). &  \\
  \hline
  7. & Anıq integraldı esaplań: \(\int_{0}^{1}{(3x^2 } + 1)dx\). &  \\
  \hline
  8. & Sanlı qatardıń baslanǵısh úsh aǵzasın jazıń: \(\sum_{n = 1}^{\infty}\frac{n!}{2^{n}}\). &  \\
  \hline
  9. & Differencial teńlemeniń ulıwma sheshimin tabıń: \(xy' - 2y = 0\). &  \\
  \hline
  10. & Gruppadaǵı 20 studentten neshe túrli usıl menen 3 náwbetshini saylap alıwǵa boladı? &  \\
  \hline
  \end{tabular}
  
  \vspace{1cm}
  
  \begin{tabular}{lll}
  Tuwrı juwaplar sanı: \underline{\hspace{1.5cm}} & 
  Bahası: \underline{\hspace{1.5cm}} & 
  Imtixan alıwshınıń qolı: \underline{\hspace{2cm}} \\
  \end{tabular}
  
  \egroup
  
  \newpage
  
  
  \textbf{16-variant}\\
  
  \bgroup
  \def\arraystretch{1.6} % 1 is the default, change whatever you need
  
  \begin{tabular}{|m{5.7cm}|m{9.5cm}|}
  \hline
  Shifr & \\
  \hline
  \end{tabular}
  
  \vspace{1cm}
  
  \begin{tabular}{|m{0.7cm}|m{10cm}|m{4cm}|}
  \hline
  № & Soraw & Juwap \\
  \hline
  1. & \(n\)-dárejeli kóp aǵzalınıń uluwma kórinisi &  \\
  \hline
  2. & Bóleklep inegrallaw formulasın jazıń &  \\
  \hline
  3. & Ózgeriwshileri ajıralǵan differenciallıq teńlemesiniń uluwma kórinisin jazıń &  \\
  \hline
  4. & Gruppalaw formulasın jazıń &  \\
  \hline
  5. & Racional funkciyanı integrallań: \(\int{\frac{3}{(x - 1)(x + 2)}dx}\). &  \\
  \hline
  6. & Integraldı esaplań: \(\int_{1}^{\infty}{\frac{1}{(x + 2)^2 }dx}\). &  \\
  \hline
  7. & Anıq integraldı esaplań: \(\int_{- \pi/4}^{0}\frac{dx}{cos^2 x}\). &  \\
  \hline
  8. & Funkcional qatardıń jıynaqlılıq oblastın tabıń:\(1 + x + ... + x^{n} + ...\) &  \\
  \hline
  9. & Differencial teńlemeniń ulıwma sheshimin tabıń: \(y' = e^{x}\). &  \\
  \hline
  10. & Qutada 5 aq hám 15 qara shar bar. Tosınnan alınǵan bir shardıń aq bolıw itimallıǵın tabıń &  \\
  \hline
  \end{tabular}
  
  \vspace{1cm}
  
  \begin{tabular}{lll}
  Tuwrı juwaplar sanı: \underline{\hspace{1.5cm}} & 
  Bahası: \underline{\hspace{1.5cm}} & 
  Imtixan alıwshınıń qolı: \underline{\hspace{2cm}} \\
  \end{tabular}
  
  \egroup
  
  \newpage
  
  
  \textbf{17-variant}\\
  
  \bgroup
  \def\arraystretch{1.6} % 1 is the default, change whatever you need
  
  \begin{tabular}{|m{5.7cm}|m{9.5cm}|}
  \hline
  Shifr & \\
  \hline
  \end{tabular}
  
  \vspace{1cm}
  
  \begin{tabular}{|m{0.7cm}|m{10cm}|m{4cm}|}
  \hline
  № & Soraw & Juwap \\
  \hline
  1. & Eki ózgeriwshili funkciyanıń grafigi neden ibarat &  \\
  \hline
  2. & Funkcianıń \((x_{0},\ y_{0})\) noqattaǵı úzliksizliginiń formulasın jazıń &  \\
  \hline
  3. & Bernulli differenciallıq teńemesin jazıń &  \\
  \hline
  4. & Itimallıqtıń klassikalıq anıqlamasınıń formulasın keltiriń &  \\
  \hline
  5. & Anıq emes integraldı esaplań: \(\int\frac{dx}{cos^2 x}\). &  \\
  \hline
  6. & Integraldı esaplań: \(\int_{1}^{\infty}{\frac{1}{x^2 }dx}\). &  \\
  \hline
  7. & Anıq integraldı esaplań: \(\int_{2}^{4}\frac{dx}{x}\). &  \\
  \hline
  8. & Qatardıń qosındısın tabıń: \(\sum_{n = 1}^{\infty}\frac{1}{n(n + 1)}\). &  \\
  \hline
  9. & Sızıqlı differerncial teńlemeniń uluwma sheshimin tabıń \(y' + y = e^{- x}\). &  \\
  \hline
  10. & 50 buyımnan ibarat partiyada 3 buyım jaramsız. Tosınnan alınǵan 8 buyımnıń ishinde 1 buyımı jaramsız bolıw itimallıǵın tabıń &  \\
  \hline
  \end{tabular}
  
  \vspace{1cm}
  
  \begin{tabular}{lll}
  Tuwrı juwaplar sanı: \underline{\hspace{1.5cm}} & 
  Bahası: \underline{\hspace{1.5cm}} & 
  Imtixan alıwshınıń qolı: \underline{\hspace{2cm}} \\
  \end{tabular}
  
  \egroup
  
  \newpage
  
  
  \textbf{18-variant}\\
  
  \bgroup
  \def\arraystretch{1.6} % 1 is the default, change whatever you need
  
  \begin{tabular}{|m{5.7cm}|m{9.5cm}|}
  \hline
  Shifr & \\
  \hline
  \end{tabular}
  
  \vspace{1cm}
  
  \begin{tabular}{|m{0.7cm}|m{10cm}|m{4cm}|}
  \hline
  № & Soraw & Juwap \\
  \hline
  1. & Eki ózgeriwshili funkciyalar qalay belgilenedi &  \\
  \hline
  2. & Funkcianıń \((x_{0},\ y_{0})\) noqattaǵı tuwındısınıń formulasın jazıń &  \\
  \hline
  3. & Funkcionallıq qatardıń uluwma kórinisi &  \\
  \hline
  4. & Orın almastırıw formulasın jazıń &  \\
  \hline
  5. & Anıq emes integraldı esaplań: \(\int{\left( 10x^{4} + 7x^{6} - 3 \right)dx}\). &  \\
  \hline
  6. & Esaplań: \(\int\left( x^{4} - \frac{1}{x} \right)dx\). &  \\
  \hline
  7. & Anıq integraldı esaplań: \(\int_{0}^{1}{(3x^2 } + 1)dx\). &  \\
  \hline
  8. & Funkcional qatardıń jaqınlasıw oblastın tabıń: \(x + \frac{x^2 }{2^2 } + ... + \frac{x^{n}}{n^2 } + ...\) &  \\
  \hline
  9. & Differencial teńlemeni esaplań: \(yy' = 4\). &  \\
  \hline
  10. & Dóngelektiń ishine kvadrat sızılǵan. Dóngelektiń ishinen tosınnan belgilengen noqattıń kvadrattıń ishinde jatıw itimallıǵın tabıń. &  \\
  \hline
  \end{tabular}
  
  \vspace{1cm}
  
  \begin{tabular}{lll}
  Tuwrı juwaplar sanı: \underline{\hspace{1.5cm}} & 
  Bahası: \underline{\hspace{1.5cm}} & 
  Imtixan alıwshınıń qolı: \underline{\hspace{2cm}} \\
  \end{tabular}
  
  \egroup
  
  \newpage
  
  
  \textbf{19-variant}\\
  
  \bgroup
  \def\arraystretch{1.6} % 1 is the default, change whatever you need
  
  \begin{tabular}{|m{5.7cm}|m{9.5cm}|}
  \hline
  Shifr & \\
  \hline
  \end{tabular}
  
  \vspace{1cm}
  
  \begin{tabular}{|m{0.7cm}|m{10cm}|m{4cm}|}
  \hline
  № & Soraw & Juwap \\
  \hline
  1. & Eki ózgeriwshili funkciyanıń ekstremumınıń zárúrli shárti &  \\
  \hline
  2. & Esaplań \(d\left( \int{f(x)dx} \right) = ?\) &  \\
  \hline
  3. & Oń aǵzalı qatarlar ushın jıynaqlılıqtıń Koshi belgisin jazıń &  \\
  \hline
  4. & Itimmallıqtıń geometriyalıq anıqlamasınıń formulasın jazıń &  \\
  \hline
  5. & Integraldı esaplań:\(\int{(x - 1)^{20}}dx\). &  \\
  \hline
  6. & Integraldı esaplań: \(\int{2^{x}dx}\). &  \\
  \hline
  7. & Anıq integraldı esaplań: \(\int_{0}^{\pi}{\sin xdx}\). &  \\
  \hline
  8. & Funkcional qatardıń jıynaqlılıq oblastın jazıń: \(\ln x + ln^2 x + ... + ln^{n}x + ...\). &  \\
  \hline
  9. & Differencial teńlemeniń ulıwma sheshimin tabıń: \(y' = e^{x}\). &  \\
  \hline
  10. & Korobkada 3 aq, 7 qara shar bar. Tosınnan úsh shar izbe-iz alındı. Izbe-iz alınǵan sharlardıń qara, qara, aq degen izbe-izlikte bolıw itimallıǵın tabıń. &  \\
  \hline
  \end{tabular}
  
  \vspace{1cm}
  
  \begin{tabular}{lll}
  Tuwrı juwaplar sanı: \underline{\hspace{1.5cm}} & 
  Bahası: \underline{\hspace{1.5cm}} & 
  Imtixan alıwshınıń qolı: \underline{\hspace{2cm}} \\
  \end{tabular}
  
  \egroup
  
  \newpage
  
  
  \textbf{20-variant}\\
  
  \bgroup
  \def\arraystretch{1.6} % 1 is the default, change whatever you need
  
  \begin{tabular}{|m{5.7cm}|m{9.5cm}|}
  \hline
  Shifr & \\
  \hline
  \end{tabular}
  
  \vspace{1cm}
  
  \begin{tabular}{|m{0.7cm}|m{10cm}|m{4cm}|}
  \hline
  № & Soraw & Juwap \\
  \hline
  1. & Kóp aǵzalını \((x - a)\) ǵa bólgendegi qaldıq nege teń &  \\
  \hline
  2. & Funkcianıń \((x_{0},\ y_{0})\) noqattaǵı tuwındısınıń formulasın jazıń &  \\
  \hline
  3. & Eger \(\sum_{n = 1}^{\infty}a_{n} = A,\ \sum_{n = 1}^{\infty}b_{n} = B\) bolsa, onda \(\sum_{n = 1}^{\infty}\left( a_{n} - b_{n} \right) = ?\) &  \\
  \hline
  4. & Múmkin emes waqıyanıń itimaıllıǵı nege teń &  \\
  \hline
  5. & Racional funkciyanı integrallań: \(\int{\frac{5}{(x - 3)(x + 2)}dx}\). &  \\
  \hline
  6. & Integraldı esaplań: \(\int_{1}^{\infty}{\frac{1}{(x + 2)^2 }dx}\). &  \\
  \hline
  7. & Anıq integraldı esaplań: \(\int_{0}^{\frac{\pi}{2}}{\cos xdx}\). &  \\
  \hline
  8. & Qatardıń qosındısın tabıń: \(\sum_{n = 1}^{\infty}\frac{1}{n(n + 3)}\). &  \\
  \hline
  9. & Differencial teńlemeni sheshiń: \(y' + xy = 0\). &  \\
  \hline
  10. & Úsh birdey korobkada aq hám qara sharlar bar. 1-korobkada 5 aq, 8 qara shar, 2-korobkada 3 aq, 4 qara shar, 3-korobkada 2 aq, 3 qara shar bar. Úsh korobkaniń birewinen tosınnan alınǵan bir shar aq bolıw itimallıǵın tabıń. &  \\
  \hline
  \end{tabular}
  
  \vspace{1cm}
  
  \begin{tabular}{lll}
  Tuwrı juwaplar sanı: \underline{\hspace{1.5cm}} & 
  Bahası: \underline{\hspace{1.5cm}} & 
  Imtixan alıwshınıń qolı: \underline{\hspace{2cm}} \\
  \end{tabular}
  
  \egroup
  
  \newpage
  
  
  \textbf{21-variant}\\
  
  \bgroup
  \def\arraystretch{1.6} % 1 is the default, change whatever you need
  
  \begin{tabular}{|m{5.7cm}|m{9.5cm}|}
  \hline
  Shifr & \\
  \hline
  \end{tabular}
  
  \vspace{1cm}
  
  \begin{tabular}{|m{0.7cm}|m{10cm}|m{4cm}|}
  \hline
  № & Soraw & Juwap \\
  \hline
  1. & Funkciya qanday usıllarda beriledi &  \\
  \hline
  2. & Ózgeriwshini almastırıp integrallaw usılıniń formulasın jazıń. &  \\
  \hline
  3. & Sanlı qatardıń uluwma kórinisin jazıń &  \\
  \hline
  4. & Bayes formulasın jazıń &  \\
  \hline
  5. & Anıq emes integraldı esaplań: \(\int{\left( x^2  + \frac{1}{x} + \sin x \right)dx}\). &  \\
  \hline
  6. & Esaplań: \(\int\left( x^{4} - \frac{1}{x} \right)dx\). &  \\
  \hline
  7. & Anıq integraldı esaplań: \(\int_{- \pi/4}^{0}\frac{dx}{cos^2 x}\). &  \\
  \hline
  8. & Qatardıń jıyındısın esaplań: \(\sum_{n = 1}^{\infty}\frac{1}{(2n - 1)(2n + 1)}\). &  \\
  \hline
  9. & Sızıqlı differencial teńlemeniń ulwma sheshimin tabıń: \(y' + y = e^{x}\). &  \\
  \hline
  10. & «MATEMATIKA» sóziniń háripleri bólek kartochkalarǵa jazılıp jawıp aralastırılıp qoyılǵan. Barlıq kartochkalar tosınnan izbe-iz alınıp ashılıp, alınıw tártibinde stol ústine dizilgende taǵı «MATEMATIKA» sóziniń kelip shıǵıw itimallıǵın tabıń. &  \\
  \hline
  \end{tabular}
  
  \vspace{1cm}
  
  \begin{tabular}{lll}
  Tuwrı juwaplar sanı: \underline{\hspace{1.5cm}} & 
  Bahası: \underline{\hspace{1.5cm}} & 
  Imtixan alıwshınıń qolı: \underline{\hspace{2cm}} \\
  \end{tabular}
  
  \egroup
  
  \newpage
  
  
  \textbf{22-variant}\\
  
  \bgroup
  \def\arraystretch{1.6} % 1 is the default, change whatever you need
  
  \begin{tabular}{|m{5.7cm}|m{9.5cm}|}
  \hline
  Shifr & \\
  \hline
  \end{tabular}
  
  \vspace{1cm}
  
  \begin{tabular}{|m{0.7cm}|m{10cm}|m{4cm}|}
  \hline
  № & Soraw & Juwap \\
  \hline
  1. & Eki ózgeriwshili funkciyanıń birinshi tártipli dara tuwındıları qalay belgilenedi &  \\
  \hline
  2. & Esaplań \(\left( \int{f(x)dx} \right)' = ?\). &  \\
  \hline
  3. & Eger \(\sum_{n = 1}^{\infty}a_{n} = A,\ \sum_{n = 1}^{\infty}b_{n} = B\) bolsa, onda \(\sum_{n = 1}^{\infty}\left( a_{n} + b_{n} \right) = ?\) &  \\
  \hline
  4. & Shártli itimallıq formulasın jazıń &  \\
  \hline
  5. & Anıq emes integraldı esaplań: \(\int{\left( 10x^{4} + 7x^{6} - 3 \right)dx}\). &  \\
  \hline
  6. & Anıq integraldı esaplań: \(\int_{1}^{3}\frac{2}{x + 1}dx\). &  \\
  \hline
  7. & Anıq integraldı esaplań: \(\int_{0}^{\pi}{\sin xdx}\). &  \\
  \hline
  8. & Sanlı qatardıń baslanǵısh úsh aǵzasın jazıń: \(\sum_{n = 1}^{\infty}\frac{n!}{2^{n}}\). &  \\
  \hline
  9. & Differencial teńlemeni esaplań: \(yy' = 4\). &  \\
  \hline
  10. & Eki kubikti bir márte taslaǵanda túsken ochkolardıń qosındısı 4 bolıw itimallıǵın tabıń. &  \\
  \hline
  \end{tabular}
  
  \vspace{1cm}
  
  \begin{tabular}{lll}
  Tuwrı juwaplar sanı: \underline{\hspace{1.5cm}} & 
  Bahası: \underline{\hspace{1.5cm}} & 
  Imtixan alıwshınıń qolı: \underline{\hspace{2cm}} \\
  \end{tabular}
  
  \egroup
  
  \newpage
  
  
  \textbf{23-variant}\\
  
  \bgroup
  \def\arraystretch{1.6} % 1 is the default, change whatever you need
  
  \begin{tabular}{|m{5.7cm}|m{9.5cm}|}
  \hline
  Shifr & \\
  \hline
  \end{tabular}
  
  \vspace{1cm}
  
  \begin{tabular}{|m{0.7cm}|m{10cm}|m{4cm}|}
  \hline
  № & Soraw & Juwap \\
  \hline
  1. & Eki ózgeriwshili funkciyanıń anıqlanıw oblastı qay jerde jaylasadı &  \\
  \hline
  2. & \((x_0,y_0)\) tochkanıń \(\varepsilon\) dógeregi qalay belgilenedi &  \\
  \hline
  3. & Oń aǵzalı qatarlar ushın jıynaqlılıqtıń Dalamber belgisin jazıń &  \\
  \hline
  4. & Shekli additivlik aksiomasın jazıń &  \\
  \hline
  5. & Racional funkciyanı integrallań: \(\int{\frac{3}{(x - 1)(x + 2)}dx}\). &  \\
  \hline
  6. & Esaplań: \(\int_{1}^2 {e^{x}dx}\). &  \\
  \hline
  7. & Anıq integraldı esaplań: \(\int_{0}^{1}{(3x^2 } + 1)dx\). &  \\
  \hline
  8. & Qatardıń qosındısın tabıń: \(\sum_{n = 1}^{\infty}\frac{1}{n(n + 1)}\). &  \\
  \hline
  9. & Sızıqlı differerncial teńlemeniń uluwma sheshimin tabıń \(y' + y = e^{- x}\). &  \\
  \hline
  10. & Tiyindi eki márte taslaǵanda, keminde bir márte san tárepi túsiw itimallıǵın tabıń. &  \\
  \hline
  \end{tabular}
  
  \vspace{1cm}
  
  \begin{tabular}{lll}
  Tuwrı juwaplar sanı: \underline{\hspace{1.5cm}} & 
  Bahası: \underline{\hspace{1.5cm}} & 
  Imtixan alıwshınıń qolı: \underline{\hspace{2cm}} \\
  \end{tabular}
  
  \egroup
  
  \newpage
  
  
  \textbf{24-variant}\\
  
  \bgroup
  \def\arraystretch{1.6} % 1 is the default, change whatever you need
  
  \begin{tabular}{|m{5.7cm}|m{9.5cm}|}
  \hline
  Shifr & \\
  \hline
  \end{tabular}
  
  \vspace{1cm}
  
  \begin{tabular}{|m{0.7cm}|m{10cm}|m{4cm}|}
  \hline
  № & Soraw & Juwap \\
  \hline
  1. & Funkciyanıń anıqlanıw oblastı qalay belgilenedi &  \\
  \hline
  2. & Bóleklep inegrallaw formulasın jazıń &  \\
  \hline
  3. & Sızıqlı defferencial teńlemeniń uluwma sheshimin jazıń &  \\
  \hline
  4. & Itimallıqtıń mánisler oblastın jazıń &  \\
  \hline
  5. & Integraldı esaplań: \(\int{\frac{1}{\sin x}dx}\). &  \\
  \hline
  6. & Integraldı esaplań: \(\int_{1}^{\infty}{\frac{1}{x^2 }dx}\). &  \\
  \hline
  7. & Anıq integraldı esaplań: \(\int_{2}^{4}\frac{dx}{x}\). &  \\
  \hline
  8. & Funkcional qatardıń jaqınlasıw oblastın tabıń: \(x + \frac{x^2 }{2^2 } + ... + \frac{x^{n}}{n^2 } + ...\) &  \\
  \hline
  9. & Differencial teńlemeniń ulıwma sheshimin tabıń: \(xy' - 2y = 0\). &  \\
  \hline
  10. & Ídısta 5 aq, 8 qara shar bar. Ídıstan tosınnan izbe-iz 3 shar alındı. Alınǵan sharlar aq, qara, qara degen izbe-izlikte bolıw itimallıǵın tabıń. &  \\
  \hline
  \end{tabular}
  
  \vspace{1cm}
  
  \begin{tabular}{lll}
  Tuwrı juwaplar sanı: \underline{\hspace{1.5cm}} & 
  Bahası: \underline{\hspace{1.5cm}} & 
  Imtixan alıwshınıń qolı: \underline{\hspace{2cm}} \\
  \end{tabular}
  
  \egroup
  
  \newpage
  
  
  \textbf{25-variant}\\
  
  \bgroup
  \def\arraystretch{1.6} % 1 is the default, change whatever you need
  
  \begin{tabular}{|m{5.7cm}|m{9.5cm}|}
  \hline
  Shifr & \\
  \hline
  \end{tabular}
  
  \vspace{1cm}
  
  \begin{tabular}{|m{0.7cm}|m{10cm}|m{4cm}|}
  \hline
  № & Soraw & Juwap \\
  \hline
  1. & Kóp aǵzalını \((x - a)\) ǵa bólgendegi qaldıq nege teń &  \\
  \hline
  2. & Eki ózgeriwshli funkciyanıń \(M(x_{0}, y_{0})\) noqattaǵı úzliksizliginiń anıqlaması &  \\
  \hline
  3. & Sızıqlı differenciallıq teńlemeniń uluwma kórinisin jazıń &  \\
  \hline
  4. & Itimallıq keńisligin jazıń &  \\
  \hline
  5. & Anıq emes integraldı esaplań: \(\int\frac{dx}{cos^2 x}\). &  \\
  \hline
  6. & Anıq integraldı esaplań: \(\int_{1}^{3}\frac{2}{x + 1}dx\). &  \\
  \hline
  7. & Anıq integraldı esaplań: \(\int_{0}^{\frac{\pi}{2}}{\cos xdx}\). &  \\
  \hline
  8. & Qatardıń qosındısın tabıń: \(\sum_{n = 1}^{\infty}\frac{1}{n(n + 3)}\). &  \\
  \hline
  9. & Differencial teńlemeni esaplań: \(yy' = 4\). &  \\
  \hline
  10. & «BIOLOGIYA» sóziniń háripleri bólek kartochkalarǵa jazılıp jawıp, aralastırılıp qoyılǵan. Barlıq kartochkalar tosınnan izbe-iz alınıp ashılıp, alınıw tártibinde stol ústine dizilgende taǵı «BIOLOGIYA» sóziniń kelip shıǵıw itimallıǵın tabıń. &  \\
  \hline
  \end{tabular}
  
  \vspace{1cm}
  
  \begin{tabular}{lll}
  Tuwrı juwaplar sanı: \underline{\hspace{1.5cm}} & 
  Bahası: \underline{\hspace{1.5cm}} & 
  Imtixan alıwshınıń qolı: \underline{\hspace{2cm}} \\
  \end{tabular}
  
  \egroup
  
  \newpage
  
  
  \textbf{26-variant}\\
  
  \bgroup
  \def\arraystretch{1.6} % 1 is the default, change whatever you need
  
  \begin{tabular}{|m{5.7cm}|m{9.5cm}|}
  \hline
  Shifr & \\
  \hline
  \end{tabular}
  
  \vspace{1cm}
  
  \begin{tabular}{|m{0.7cm}|m{10cm}|m{4cm}|}
  \hline
  № & Soraw & Juwap \\
  \hline
  1. & Eki ózgeriwshili funkciyanıń ekinshi tártipli dara tuwındıları qalay belgilenedi &  \\
  \hline
  2. & Funkcianıń \((x_{0},\ y_{0})\) noqattaǵı úzliksizliginiń formulasın jazıń &  \\
  \hline
  3. & Funkcionallıq qatardıń uluwma kórinisi &  \\
  \hline
  4. & Isenimli waqıyanıń itimallıǵı nege teń &  \\
  \hline
  5. & Integraldı esaplań: \(\int{(x + \sin x)dx}\). &  \\
  \hline
  6. & Integraldı esaplań: \(\int{2^{x}dx}\). &  \\
  \hline
  7. & Anıq integraldı esaplań: \(\int_{0}^{1}{(3x^2 } + 1)dx\). &  \\
  \hline
  8. & Funkcional qatardıń jıynaqlılıq oblastın tabıń:\(1 + x + ... + x^{n} + ...\) &  \\
  \hline
  9. & Sızıqlı differencial teńlemeniń ulwma sheshimin tabıń: \(y' + y = e^{x}\). &  \\
  \hline
  10. & Telefon nomerdiń aqırǵı cifrasın umıtıp, tosınnan nomerlerdi tere basladı. Kerekli nomerdi tabıw itimallıǵın esaplań. &  \\
  \hline
  \end{tabular}
  
  \vspace{1cm}
  
  \begin{tabular}{lll}
  Tuwrı juwaplar sanı: \underline{\hspace{1.5cm}} & 
  Bahası: \underline{\hspace{1.5cm}} & 
  Imtixan alıwshınıń qolı: \underline{\hspace{2cm}} \\
  \end{tabular}
  
  \egroup
  
  \newpage
  
  
  \textbf{27-variant}\\
  
  \bgroup
  \def\arraystretch{1.6} % 1 is the default, change whatever you need
  
  \begin{tabular}{|m{5.7cm}|m{9.5cm}|}
  \hline
  Shifr & \\
  \hline
  \end{tabular}
  
  \vspace{1cm}
  
  \begin{tabular}{|m{0.7cm}|m{10cm}|m{4cm}|}
  \hline
  № & Soraw & Juwap \\
  \hline
  1. & Eki ózgeriwshili funkciyanıń ekstremumınıń zárúrli shárti &  \\
  \hline
  2. & Anıq integraldı esaplawdıń Nyuton-Leybnic formulasın jazıń &  \\
  \hline
  3. & Bernulli differenciallıq teńemesin jazıń &  \\
  \hline
  4. & Tolıq itimallıqtıń formulasın jazıń &  \\
  \hline
  5. & Anıq emes integraldı esaplań: \(\int{e^{x}dx}\) . &  \\
  \hline
  6. & Integraldı esaplań: \(\int_{1}^{\infty}{\frac{1}{(x + 2)^2 }dx}\). &  \\
  \hline
  7. & Anıq integraldı esaplań: \(\int_{2}^{4}\frac{dx}{x}\). &  \\
  \hline
  8. & Funkcional qatardıń jıynaqlılıq oblastın jazıń: \(\ln x + ln^2 x + ... + ln^{n}x + ...\). &  \\
  \hline
  9. & Differencial teńlemeni sheshiń: \(y' + xy = 0\). &  \\
  \hline
  10. & Korobkada 15 aq, 18 qara shar bar. Tosınnan alınǵan bir shar aq bolıw itimallıǵın tabıń. &  \\
  \hline
  \end{tabular}
  
  \vspace{1cm}
  
  \begin{tabular}{lll}
  Tuwrı juwaplar sanı: \underline{\hspace{1.5cm}} & 
  Bahası: \underline{\hspace{1.5cm}} & 
  Imtixan alıwshınıń qolı: \underline{\hspace{2cm}} \\
  \end{tabular}
  
  \egroup
  
  \newpage
  
  
  \textbf{28-variant}\\
  
  \bgroup
  \def\arraystretch{1.6} % 1 is the default, change whatever you need
  
  \begin{tabular}{|m{5.7cm}|m{9.5cm}|}
  \hline
  Shifr & \\
  \hline
  \end{tabular}
  
  \vspace{1cm}
  
  \begin{tabular}{|m{0.7cm}|m{10cm}|m{4cm}|}
  \hline
  № & Soraw & Juwap \\
  \hline
  1. & Eki ózgeriwshili funkciyalar qalay belgilenedi &  \\
  \hline
  2. & Esaplań \(d\left( \int{f(x)dx} \right) = ?\) &  \\
  \hline
  3. & Eger \(\sum_{n = 1}^{\infty}a_{n} = A,\ \sum_{n = 1}^{\infty}b_{n} = B\) bolsa, onda \(\sum_{n = 1}^{\infty}\left( a_{n} - b_{n} \right) = ?\) &  \\
  \hline
  4. & Itimallıqtıń klassikalıq anıqlamasınıń formulasın keltiriń &  \\
  \hline
  5. & Anıq emes integraldı esaplań: \(\int{e^{x}dx}\) . &  \\
  \hline
  6. & Integraldı esaplań: \(\int_{1}^{\infty}{\frac{1}{x^2 }dx}\). &  \\
  \hline
  7. & Anıq integraldı esaplań: \(\int_{0}^{\frac{\pi}{2}}{\cos xdx}\). &  \\
  \hline
  8. & Qatardıń jıyındısın esaplań: \(\sum_{n = 1}^{\infty}\frac{1}{(2n - 1)(2n + 1)}\). &  \\
  \hline
  9. & Differencial teńlemeniń ulıwma sheshimin tabıń: \(xy' - 2y = 0\). &  \\
  \hline
  10. & Telefon nomerdiń aqırǵı eki cifrasın umıtıp, tosınnan nomerlerdi tere basladı. Kerekli nomerdi tabıw itimallıǵın esaplań. &  \\
  \hline
  \end{tabular}
  
  \vspace{1cm}
  
  \begin{tabular}{lll}
  Tuwrı juwaplar sanı: \underline{\hspace{1.5cm}} & 
  Bahası: \underline{\hspace{1.5cm}} & 
  Imtixan alıwshınıń qolı: \underline{\hspace{2cm}} \\
  \end{tabular}
  
  \egroup
  
  \newpage
  
  
  \textbf{29-variant}\\
  
  \bgroup
  \def\arraystretch{1.6} % 1 is the default, change whatever you need
  
  \begin{tabular}{|m{5.7cm}|m{9.5cm}|}
  \hline
  Shifr & \\
  \hline
  \end{tabular}
  
  \vspace{1cm}
  
  \begin{tabular}{|m{0.7cm}|m{10cm}|m{4cm}|}
  \hline
  № & Soraw & Juwap \\
  \hline
  1. & Eki ózgeriwshili funkciyanıń anıqlanıw oblastı qay jerde jaylasadı &  \\
  \hline
  2. & Funkcianıń \((x_{0},\ y_{0})\) noqattaǵı tuwındısınıń formulasın jazıń &  \\
  \hline
  3. & Oń aǵzalı qatarlar ushın jıynaqlılıqtıń Koshi belgisin jazıń &  \\
  \hline
  4. & Bayes formulasın jazıń &  \\
  \hline
  5. & Anıq emes integraldı esaplań: \(\int{\left( x^2  + \frac{1}{x} + \sin x \right)dx}\). &  \\
  \hline
  6. & Esaplań: \(\int\left( x^{4} - \frac{1}{x} \right)dx\). &  \\
  \hline
  7. & Anıq integraldı esaplań: \(\int_{- \pi/4}^{0}\frac{dx}{cos^2 x}\). &  \\
  \hline
  8. & Sanlı qatardıń baslanǵısh úsh aǵzasın jazıń: \(\sum_{n = 1}^{\infty}\frac{n!}{2^{n}}\). &  \\
  \hline
  9. & Sızıqlı differerncial teńlemeniń uluwma sheshimin tabıń \(y' + y = e^{- x}\). &  \\
  \hline
  10. & Tiyindi eki márte taslaǵanda, keminde bir márte san tárepi túsiw itimallıǵın tabıń. &  \\
  \hline
  \end{tabular}
  
  \vspace{1cm}
  
  \begin{tabular}{lll}
  Tuwrı juwaplar sanı: \underline{\hspace{1.5cm}} & 
  Bahası: \underline{\hspace{1.5cm}} & 
  Imtixan alıwshınıń qolı: \underline{\hspace{2cm}} \\
  \end{tabular}
  
  \egroup
  
  \newpage
  
  
  \textbf{30-variant}\\
  
  \bgroup
  \def\arraystretch{1.6} % 1 is the default, change whatever you need
  
  \begin{tabular}{|m{5.7cm}|m{9.5cm}|}
  \hline
  Shifr & \\
  \hline
  \end{tabular}
  
  \vspace{1cm}
  
  \begin{tabular}{|m{0.7cm}|m{10cm}|m{4cm}|}
  \hline
  № & Soraw & Juwap \\
  \hline
  1. & Eki ózgeriwshili funkciyanıń ekinshi tártipli aralas tuwındıları qalay belgilenedi &  \\
  \hline
  2. & Ózgeriwshini almastırıp integrallaw usılıniń formulasın jazıń. &  \\
  \hline
  3. & Ózgeriwshileri ajıralǵan differenciallıq teńlemesiniń uluwma kórinisin jazıń &  \\
  \hline
  4. & Orın almastırıw formulasın jazıń &  \\
  \hline
  5. & Racional funkciyanı integrallań: \(\int{\frac{3}{(x - 1)(x + 2)}dx}\). &  \\
  \hline
  6. & Esaplań: \(\int_{1}^2 {e^{x}dx}\). &  \\
  \hline
  7. & Anıq integraldı esaplań: \(\int_{0}^{\pi}{\sin xdx}\). &  \\
  \hline
  8. & Funkcional qatardıń jıynaqlılıq oblastın tabıń:\(1 + x + ... + x^{n} + ...\) &  \\
  \hline
  9. & Differencial teńlemeniń ulıwma sheshimin tabıń: \(y' = e^{x}\). &  \\
  \hline
  10. & Telefon nomerdiń aqırǵı cifrasın umıtıp, tosınnan nomerlerdi tere basladı. Kerekli nomerdi tabıw itimallıǵın esaplań. &  \\
  \hline
  \end{tabular}
  
  \vspace{1cm}
  
  \begin{tabular}{lll}
  Tuwrı juwaplar sanı: \underline{\hspace{1.5cm}} & 
  Bahası: \underline{\hspace{1.5cm}} & 
  Imtixan alıwshınıń qolı: \underline{\hspace{2cm}} \\
  \end{tabular}
  
  \egroup
  
  \newpage
  
  
  \textbf{31-variant}\\
  
  \bgroup
  \def\arraystretch{1.6} % 1 is the default, change whatever you need
  
  \begin{tabular}{|m{5.7cm}|m{9.5cm}|}
  \hline
  Shifr & \\
  \hline
  \end{tabular}
  
  \vspace{1cm}
  
  \begin{tabular}{|m{0.7cm}|m{10cm}|m{4cm}|}
  \hline
  № & Soraw & Juwap \\
  \hline
  1. & Funkciya qanday usıllarda beriledi &  \\
  \hline
  2. & Anıq integraldı esaplawdıń Nyuton-Leybnic formulasın jazıń &  \\
  \hline
  3. & Oń aǵzalı qatarlar ushın jıynaqlılıqtıń Dalamber belgisin jazıń &  \\
  \hline
  4. & Shekli additivlik aksiomasın jazıń &  \\
  \hline
  5. & Integraldı esaplań: \(\int{\frac{1}{\sin x}dx}\). &  \\
  \hline
  6. & Esaplań: \(\int_{1}^2 {e^{x}dx}\). &  \\
  \hline
  7. & Anıq integraldı esaplań: \(\int_{- \pi/4}^{0}\frac{dx}{cos^2 x}\). &  \\
  \hline
  8. & Funkcional qatardıń jıynaqlılıq oblastın jazıń: \(\ln x + ln^2 x + ... + ln^{n}x + ...\). &  \\
  \hline
  9. & Differencial teńlemeni sheshiń: \(y' + xy = 0\). &  \\
  \hline
  10. & «BIOLOGIYA» sóziniń háripleri bólek kartochkalarǵa jazılıp jawıp, aralastırılıp qoyılǵan. Barlıq kartochkalar tosınnan izbe-iz alınıp ashılıp, alınıw tártibinde stol ústine dizilgende taǵı «BIOLOGIYA» sóziniń kelip shıǵıw itimallıǵın tabıń. &  \\
  \hline
  \end{tabular}
  
  \vspace{1cm}
  
  \begin{tabular}{lll}
  Tuwrı juwaplar sanı: \underline{\hspace{1.5cm}} & 
  Bahası: \underline{\hspace{1.5cm}} & 
  Imtixan alıwshınıń qolı: \underline{\hspace{2cm}} \\
  \end{tabular}
  
  \egroup
  
  \newpage
  
  
  \textbf{32-variant}\\
  
  \bgroup
  \def\arraystretch{1.6} % 1 is the default, change whatever you need
  
  \begin{tabular}{|m{5.7cm}|m{9.5cm}|}
  \hline
  Shifr & \\
  \hline
  \end{tabular}
  
  \vspace{1cm}
  
  \begin{tabular}{|m{0.7cm}|m{10cm}|m{4cm}|}
  \hline
  № & Soraw & Juwap \\
  \hline
  1. & Eki ózgeriwshili funkciyanıń tolıq ósimi &  \\
  \hline
  2. & Bóleklep inegrallaw formulasın jazıń &  \\
  \hline
  3. & Funkcionallıq qatardıń uluwma kórinisi &  \\
  \hline
  4. & Gruppalaw formulasın jazıń &  \\
  \hline
  5. & Racional funkciyanı integrallań: \(\int{\frac{5}{(x - 3)(x + 2)}dx}\). &  \\
  \hline
  6. & Anıq integraldı esaplań: \(\int_{1}^{3}\frac{2}{x + 1}dx\). &  \\
  \hline
  7. & Anıq integraldı esaplań: \(\int_{0}^{\frac{\pi}{2}}{\cos xdx}\). &  \\
  \hline
  8. & Funkcional qatardıń jaqınlasıw oblastın tabıń: \(x + \frac{x^2 }{2^2 } + ... + \frac{x^{n}}{n^2 } + ...\) &  \\
  \hline
  9. & Differencial teńlemeniń ulıwma sheshimin tabıń: \(xy' - 2y = 0\). &  \\
  \hline
  10. & Gruppadaǵı 20 studentten neshe túrli usıl menen 3 náwbetshini saylap alıwǵa boladı? &  \\
  \hline
  \end{tabular}
  
  \vspace{1cm}
  
  \begin{tabular}{lll}
  Tuwrı juwaplar sanı: \underline{\hspace{1.5cm}} & 
  Bahası: \underline{\hspace{1.5cm}} & 
  Imtixan alıwshınıń qolı: \underline{\hspace{2cm}} \\
  \end{tabular}
  
  \egroup
  
  \newpage
  
  
  \textbf{33-variant}\\
  
  \bgroup
  \def\arraystretch{1.6} % 1 is the default, change whatever you need
  
  \begin{tabular}{|m{5.7cm}|m{9.5cm}|}
  \hline
  Shifr & \\
  \hline
  \end{tabular}
  
  \vspace{1cm}
  
  \begin{tabular}{|m{0.7cm}|m{10cm}|m{4cm}|}
  \hline
  № & Soraw & Juwap \\
  \hline
  1. & Eki ózgeriwshili funkciyanıń birinshi tártipli dara tuwındıları qalay belgilenedi &  \\
  \hline
  2. & Eki ózgeriwshli funkciyanıń \(M(x_{0}, y_{0})\) noqattaǵı úzliksizliginiń anıqlaması &  \\
  \hline
  3. & Eger \(\sum_{n = 1}^{\infty}a_{n} = A,\ \sum_{n = 1}^{\infty}b_{n} = B\) bolsa, onda \(\sum_{n = 1}^{\infty}\left( a_{n} + b_{n} \right) = ?\) &  \\
  \hline
  4. & Isenimli waqıyanıń itimallıǵı nege teń &  \\
  \hline
  5. & Integraldı esaplań: \(\int{(x + \sin x)dx}\). &  \\
  \hline
  6. & Esaplań: \(\int\left( x^{4} - \frac{1}{x} \right)dx\). &  \\
  \hline
  7. & Anıq integraldı esaplań: \(\int_{2}^{4}\frac{dx}{x}\). &  \\
  \hline
  8. & Qatardıń qosındısın tabıń: \(\sum_{n = 1}^{\infty}\frac{1}{n(n + 1)}\). &  \\
  \hline
  9. & Differencial teńlemeniń ulıwma sheshimin tabıń: \(y' = e^{x}\). &  \\
  \hline
  10. & 50 buyımnan ibarat partiyada 3 buyım jaramsız. Tosınnan alınǵan 8 buyımnıń ishinde 1 buyımı jaramsız bolıw itimallıǵın tabıń &  \\
  \hline
  \end{tabular}
  
  \vspace{1cm}
  
  \begin{tabular}{lll}
  Tuwrı juwaplar sanı: \underline{\hspace{1.5cm}} & 
  Bahası: \underline{\hspace{1.5cm}} & 
  Imtixan alıwshınıń qolı: \underline{\hspace{2cm}} \\
  \end{tabular}
  
  \egroup
  
  \newpage
  
  
  \textbf{34-variant}\\
  
  \bgroup
  \def\arraystretch{1.6} % 1 is the default, change whatever you need
  
  \begin{tabular}{|m{5.7cm}|m{9.5cm}|}
  \hline
  Shifr & \\
  \hline
  \end{tabular}
  
  \vspace{1cm}
  
  \begin{tabular}{|m{0.7cm}|m{10cm}|m{4cm}|}
  \hline
  № & Soraw & Juwap \\
  \hline
  1. & \(n\)-dárejeli kóp aǵzalınıń uluwma kórinisi &  \\
  \hline
  2. & Esaplań \(\left( \int{f(x)dx} \right)' = ?\). &  \\
  \hline
  3. & Ózgeriwshileri ajıralǵan differenciallıq teńlemesiniń uluwma kórinisin jazıń &  \\
  \hline
  4. & Múmkin emes waqıyanıń itimaıllıǵı nege teń &  \\
  \hline
  5. & Integraldı esaplań:\(\int{(x - 1)^{20}}dx\). &  \\
  \hline
  6. & Integraldı esaplań: \(\int{2^{x}dx}\). &  \\
  \hline
  7. & Anıq integraldı esaplań: \(\int_{0}^{1}{(3x^2 } + 1)dx\). &  \\
  \hline
  8. & Qatardıń jıyındısın esaplań: \(\sum_{n = 1}^{\infty}\frac{1}{(2n - 1)(2n + 1)}\). &  \\
  \hline
  9. & Sızıqlı differencial teńlemeniń ulwma sheshimin tabıń: \(y' + y = e^{x}\). &  \\
  \hline
  10. & Korobkada 15 aq, 18 qara shar bar. Tosınnan alınǵan bir shar aq bolıw itimallıǵın tabıń. &  \\
  \hline
  \end{tabular}
  
  \vspace{1cm}
  
  \begin{tabular}{lll}
  Tuwrı juwaplar sanı: \underline{\hspace{1.5cm}} & 
  Bahası: \underline{\hspace{1.5cm}} & 
  Imtixan alıwshınıń qolı: \underline{\hspace{2cm}} \\
  \end{tabular}
  
  \egroup
  
  \newpage
  
  
  \textbf{35-variant}\\
  
  \bgroup
  \def\arraystretch{1.6} % 1 is the default, change whatever you need
  
  \begin{tabular}{|m{5.7cm}|m{9.5cm}|}
  \hline
  Shifr & \\
  \hline
  \end{tabular}
  
  \vspace{1cm}
  
  \begin{tabular}{|m{0.7cm}|m{10cm}|m{4cm}|}
  \hline
  № & Soraw & Juwap \\
  \hline
  1. & Eki ózgeriwshili funkciyanıń grafigi neden ibarat &  \\
  \hline
  2. & Funkcianıń \((x_{0},\ y_{0})\) noqattaǵı úzliksizliginiń formulasın jazıń &  \\
  \hline
  3. & Sızıqlı defferencial teńlemeniń uluwma sheshimin jazıń &  \\
  \hline
  4. & Orın awıstırıw formulasın jazıń &  \\
  \hline
  5. & Anıq emes integraldı esaplań: \(\int\frac{dx}{cos^2 x}\). &  \\
  \hline
  6. & Integraldı esaplań: \(\int_{1}^{\infty}{\frac{1}{(x + 2)^2 }dx}\). &  \\
  \hline
  7. & Anıq integraldı esaplań: \(\int_{0}^{\pi}{\sin xdx}\). &  \\
  \hline
  8. & Qatardıń qosındısın tabıń: \(\sum_{n = 1}^{\infty}\frac{1}{n(n + 3)}\). &  \\
  \hline
  9. & Differencial teńlemeni esaplań: \(yy' = 4\). &  \\
  \hline
  10. & Telefon nomerdiń aqırǵı eki cifrasın umıtıp, tosınnan nomerlerdi tere basladı. Kerekli nomerdi tabıw itimallıǵın esaplań. &  \\
  \hline
  \end{tabular}
  
  \vspace{1cm}
  
  \begin{tabular}{lll}
  Tuwrı juwaplar sanı: \underline{\hspace{1.5cm}} & 
  Bahası: \underline{\hspace{1.5cm}} & 
  Imtixan alıwshınıń qolı: \underline{\hspace{2cm}} \\
  \end{tabular}
  
  \egroup
  
  \newpage
  
  
  \textbf{36-variant}\\
  
  \bgroup
  \def\arraystretch{1.6} % 1 is the default, change whatever you need
  
  \begin{tabular}{|m{5.7cm}|m{9.5cm}|}
  \hline
  Shifr & \\
  \hline
  \end{tabular}
  
  \vspace{1cm}
  
  \begin{tabular}{|m{0.7cm}|m{10cm}|m{4cm}|}
  \hline
  № & Soraw & Juwap \\
  \hline
  1. & Funkciyanıń anıqlanıw oblastı qalay belgilenedi &  \\
  \hline
  2. & \((x_0,y_0)\) tochkanıń \(\varepsilon\) dógeregi qalay belgilenedi &  \\
  \hline
  3. & Bernulli differenciallıq teńemesin jazıń &  \\
  \hline
  4. & Shártli itimallıq formulasın jazıń &  \\
  \hline
  5. & Anıq emes integraldı esaplań: \(\int{\left( 10x^{4} + 7x^{6} - 3 \right)dx}\). &  \\
  \hline
  6. & Integraldı esaplań: \(\int_{1}^{\infty}{\frac{1}{x^2 }dx}\). &  \\
  \hline
  7. & Anıq integraldı esaplań: \(\int_{2}^{4}\frac{dx}{x}\). &  \\
  \hline
  8. & Sanlı qatardıń baslanǵısh úsh aǵzasın jazıń: \(\sum_{n = 1}^{\infty}\frac{n!}{2^{n}}\). &  \\
  \hline
  9. & Sızıqlı differerncial teńlemeniń uluwma sheshimin tabıń \(y' + y = e^{- x}\). &  \\
  \hline
  10. & Úsh birdey korobkada aq hám qara sharlar bar. 1-korobkada 5 aq, 8 qara shar, 2-korobkada 3 aq, 4 qara shar, 3-korobkada 2 aq, 3 qara shar bar. Úsh korobkaniń birewinen tosınnan alınǵan bir shar aq bolıw itimallıǵın tabıń. &  \\
  \hline
  \end{tabular}
  
  \vspace{1cm}
  
  \begin{tabular}{lll}
  Tuwrı juwaplar sanı: \underline{\hspace{1.5cm}} & 
  Bahası: \underline{\hspace{1.5cm}} & 
  Imtixan alıwshınıń qolı: \underline{\hspace{2cm}} \\
  \end{tabular}
  
  \egroup
  
  \newpage
  
  
  \textbf{37-variant}\\
  
  \bgroup
  \def\arraystretch{1.6} % 1 is the default, change whatever you need
  
  \begin{tabular}{|m{5.7cm}|m{9.5cm}|}
  \hline
  Shifr & \\
  \hline
  \end{tabular}
  
  \vspace{1cm}
  
  \begin{tabular}{|m{0.7cm}|m{10cm}|m{4cm}|}
  \hline
  № & Soraw & Juwap \\
  \hline
  1. & Eki ózgeriwshili funkciyanıń anıqlanıw oblastı qay jerde jaylasadı &  \\
  \hline
  2. & Esaplań \(d\left( \int{f(x)dx} \right) = ?\) &  \\
  \hline
  3. & Eger \(\sum_{n = 1}^{\infty}a_{n} = A,\ \sum_{n = 1}^{\infty}b_{n} = B\) bolsa, onda \(\sum_{n = 1}^{\infty}\left( a_{n} - b_{n} \right) = ?\) &  \\
  \hline
  4. & Itimmallıqtıń geometriyalıq anıqlamasınıń formulasın jazıń &  \\
  \hline
  5. & Anıq emes integraldı esaplań: \(\int{e^{x}dx}\) . &  \\
  \hline
  6. & Integraldı esaplań: \(\int_{1}^{\infty}{\frac{1}{x^2 }dx}\). &  \\
  \hline
  7. & Anıq integraldı esaplań: \(\int_{- \pi/4}^{0}\frac{dx}{cos^2 x}\). &  \\
  \hline
  8. & Funkcional qatardıń jaqınlasıw oblastın tabıń: \(x + \frac{x^2 }{2^2 } + ... + \frac{x^{n}}{n^2 } + ...\) &  \\
  \hline
  9. & Sızıqlı differerncial teńlemeniń uluwma sheshimin tabıń \(y' + y = e^{- x}\). &  \\
  \hline
  10. & Korobkada 3 aq, 7 qara shar bar. Tosınnan úsh shar izbe-iz alındı. Izbe-iz alınǵan sharlardıń qara, qara, aq degen izbe-izlikte bolıw itimallıǵın tabıń. &  \\
  \hline
  \end{tabular}
  
  \vspace{1cm}
  
  \begin{tabular}{lll}
  Tuwrı juwaplar sanı: \underline{\hspace{1.5cm}} & 
  Bahası: \underline{\hspace{1.5cm}} & 
  Imtixan alıwshınıń qolı: \underline{\hspace{2cm}} \\
  \end{tabular}
  
  \egroup
  
  \newpage
  
  
  \textbf{38-variant}\\
  
  \bgroup
  \def\arraystretch{1.6} % 1 is the default, change whatever you need
  
  \begin{tabular}{|m{5.7cm}|m{9.5cm}|}
  \hline
  Shifr & \\
  \hline
  \end{tabular}
  
  \vspace{1cm}
  
  \begin{tabular}{|m{0.7cm}|m{10cm}|m{4cm}|}
  \hline
  № & Soraw & Juwap \\
  \hline
  1. & Eki ózgeriwshili funkciyanıń grafigi neden ibarat &  \\
  \hline
  2. & Funkcianıń \((x_{0},\ y_{0})\) noqattaǵı úzliksizliginiń formulasın jazıń &  \\
  \hline
  3. & Oń aǵzalı qatarlar ushın jıynaqlılıqtıń Koshi belgisin jazıń &  \\
  \hline
  4. & Itimallıq keńisligin jazıń &  \\
  \hline
  5. & Integraldı esaplań:\(\int{(x - 1)^{20}}dx\). &  \\
  \hline
  6. & Anıq integraldı esaplań: \(\int_{1}^{3}\frac{2}{x + 1}dx\). &  \\
  \hline
  7. & Anıq integraldı esaplań: \(\int_{0}^{\pi}{\sin xdx}\). &  \\
  \hline
  8. & Qatardıń qosındısın tabıń: \(\sum_{n = 1}^{\infty}\frac{1}{n(n + 3)}\). &  \\
  \hline
  9. & Sızıqlı differencial teńlemeniń ulwma sheshimin tabıń: \(y' + y = e^{x}\). &  \\
  \hline
  10. & Eki kubikti bir márte taslaǵanda túsken ochkolardıń qosındısı 4 bolıw itimallıǵın tabıń. &  \\
  \hline
  \end{tabular}
  
  \vspace{1cm}
  
  \begin{tabular}{lll}
  Tuwrı juwaplar sanı: \underline{\hspace{1.5cm}} & 
  Bahası: \underline{\hspace{1.5cm}} & 
  Imtixan alıwshınıń qolı: \underline{\hspace{2cm}} \\
  \end{tabular}
  
  \egroup
  
  \newpage
  
  
  \textbf{39-variant}\\
  
  \bgroup
  \def\arraystretch{1.6} % 1 is the default, change whatever you need
  
  \begin{tabular}{|m{5.7cm}|m{9.5cm}|}
  \hline
  Shifr & \\
  \hline
  \end{tabular}
  
  \vspace{1cm}
  
  \begin{tabular}{|m{0.7cm}|m{10cm}|m{4cm}|}
  \hline
  № & Soraw & Juwap \\
  \hline
  1. & Funkciyanıń anıqlanıw oblastı qalay belgilenedi &  \\
  \hline
  2. & Anıq integraldı esaplawdıń Nyuton-Leybnic formulasın jazıń &  \\
  \hline
  3. & Sızıqlı differenciallıq teńlemeniń uluwma kórinisin jazıń &  \\
  \hline
  4. & Itimallıqtıń mánisler oblastın jazıń &  \\
  \hline
  5. & Integraldı esaplań: \(\int{\frac{1}{\sin x}dx}\). &  \\
  \hline
  6. & Esaplań: \(\int\left( x^{4} - \frac{1}{x} \right)dx\). &  \\
  \hline
  7. & Anıq integraldı esaplań: \(\int_{0}^{\frac{\pi}{2}}{\cos xdx}\). &  \\
  \hline
  8. & Qatardıń jıyındısın esaplań: \(\sum_{n = 1}^{\infty}\frac{1}{(2n - 1)(2n + 1)}\). &  \\
  \hline
  9. & Differencial teńlemeniń ulıwma sheshimin tabıń: \(xy' - 2y = 0\). &  \\
  \hline
  10. & Ídısta 5 aq, 8 qara shar bar. Ídıstan tosınnan izbe-iz 3 shar alındı. Alınǵan sharlar aq, qara, qara degen izbe-izlikte bolıw itimallıǵın tabıń. &  \\
  \hline
  \end{tabular}
  
  \vspace{1cm}
  
  \begin{tabular}{lll}
  Tuwrı juwaplar sanı: \underline{\hspace{1.5cm}} & 
  Bahası: \underline{\hspace{1.5cm}} & 
  Imtixan alıwshınıń qolı: \underline{\hspace{2cm}} \\
  \end{tabular}
  
  \egroup
  
  \newpage
  
  
  \textbf{40-variant}\\
  
  \bgroup
  \def\arraystretch{1.6} % 1 is the default, change whatever you need
  
  \begin{tabular}{|m{5.7cm}|m{9.5cm}|}
  \hline
  Shifr & \\
  \hline
  \end{tabular}
  
  \vspace{1cm}
  
  \begin{tabular}{|m{0.7cm}|m{10cm}|m{4cm}|}
  \hline
  № & Soraw & Juwap \\
  \hline
  1. & Eki ózgeriwshili funkciyanıń ekstremumınıń zárúrli shárti &  \\
  \hline
  2. & Esaplań \(\left( \int{f(x)dx} \right)' = ?\). &  \\
  \hline
  3. & Sanlı qatardıń uluwma kórinisin jazıń &  \\
  \hline
  4. & Bayes formulasın jazıń &  \\
  \hline
  5. & Anıq emes integraldı esaplań: \(\int{\left( 10x^{4} + 7x^{6} - 3 \right)dx}\). &  \\
  \hline
  6. & Esaplań: \(\int_{1}^2 {e^{x}dx}\). &  \\
  \hline
  7. & Anıq integraldı esaplań: \(\int_{0}^{1}{(3x^2 } + 1)dx\). &  \\
  \hline
  8. & Qatardıń qosındısın tabıń: \(\sum_{n = 1}^{\infty}\frac{1}{n(n + 1)}\). &  \\
  \hline
  9. & Differencial teńlemeni sheshiń: \(y' + xy = 0\). &  \\
  \hline
  10. & Dóngelektiń ishine kvadrat sızılǵan. Dóngelektiń ishinen tosınnan belgilengen noqattıń kvadrattıń ishinde jatıw itimallıǵın tabıń. &  \\
  \hline
  \end{tabular}
  
  \vspace{1cm}
  
  \begin{tabular}{lll}
  Tuwrı juwaplar sanı: \underline{\hspace{1.5cm}} & 
  Bahası: \underline{\hspace{1.5cm}} & 
  Imtixan alıwshınıń qolı: \underline{\hspace{2cm}} \\
  \end{tabular}
  
  \egroup
  
  \newpage
  
  
  \textbf{41-variant}\\
  
  \bgroup
  \def\arraystretch{1.6} % 1 is the default, change whatever you need
  
  \begin{tabular}{|m{5.7cm}|m{9.5cm}|}
  \hline
  Shifr & \\
  \hline
  \end{tabular}
  
  \vspace{1cm}
  
  \begin{tabular}{|m{0.7cm}|m{10cm}|m{4cm}|}
  \hline
  № & Soraw & Juwap \\
  \hline
  1. & Eki ózgeriwshili funkciyanıń tolıq ósimi &  \\
  \hline
  2. & Ózgeriwshini almastırıp integrallaw usılıniń formulasın jazıń. &  \\
  \hline
  3. & Sanlı qatardıń uluwma kórinisin jazıń &  \\
  \hline
  4. & Itimallıqtıń klassikalıq anıqlamasınıń formulasın keltiriń &  \\
  \hline
  5. & Racional funkciyanı integrallań: \(\int{\frac{5}{(x - 3)(x + 2)}dx}\). &  \\
  \hline
  6. & Integraldı esaplań: \(\int{2^{x}dx}\). &  \\
  \hline
  7. & Anıq integraldı esaplań: \(\int_{0}^{\pi}{\sin xdx}\). &  \\
  \hline
  8. & Funkcional qatardıń jıynaqlılıq oblastın jazıń: \(\ln x + ln^2 x + ... + ln^{n}x + ...\). &  \\
  \hline
  9. & Differencial teńlemeni esaplań: \(yy' = 4\). &  \\
  \hline
  10. & «MATEMATIKA» sóziniń háripleri bólek kartochkalarǵa jazılıp jawıp aralastırılıp qoyılǵan. Barlıq kartochkalar tosınnan izbe-iz alınıp ashılıp, alınıw tártibinde stol ústine dizilgende taǵı «MATEMATIKA» sóziniń kelip shıǵıw itimallıǵın tabıń. &  \\
  \hline
  \end{tabular}
  
  \vspace{1cm}
  
  \begin{tabular}{lll}
  Tuwrı juwaplar sanı: \underline{\hspace{1.5cm}} & 
  Bahası: \underline{\hspace{1.5cm}} & 
  Imtixan alıwshınıń qolı: \underline{\hspace{2cm}} \\
  \end{tabular}
  
  \egroup
  
  \newpage
  
  
  \textbf{42-variant}\\
  
  \bgroup
  \def\arraystretch{1.6} % 1 is the default, change whatever you need
  
  \begin{tabular}{|m{5.7cm}|m{9.5cm}|}
  \hline
  Shifr & \\
  \hline
  \end{tabular}
  
  \vspace{1cm}
  
  \begin{tabular}{|m{0.7cm}|m{10cm}|m{4cm}|}
  \hline
  № & Soraw & Juwap \\
  \hline
  1. & Eki ózgeriwshili funkciyanıń ekinshi tártipli dara tuwındıları qalay belgilenedi &  \\
  \hline
  2. & Bóleklep inegrallaw formulasın jazıń &  \\
  \hline
  3. & Eger \(\sum_{n = 1}^{\infty}a_{n} = A,\ \sum_{n = 1}^{\infty}b_{n} = B\) bolsa, onda \(\sum_{n = 1}^{\infty}\left( a_{n} - b_{n} \right) = ?\) &  \\
  \hline
  4. & Itimallıqtıń mánisler oblastın jazıń &  \\
  \hline
  5. & Integraldı esaplań: \(\int{(x + \sin x)dx}\). &  \\
  \hline
  6. & Integraldı esaplań: \(\int_{1}^{\infty}{\frac{1}{(x + 2)^2 }dx}\). &  \\
  \hline
  7. & Anıq integraldı esaplań: \(\int_{- \pi/4}^{0}\frac{dx}{cos^2 x}\). &  \\
  \hline
  8. & Funkcional qatardıń jıynaqlılıq oblastın tabıń:\(1 + x + ... + x^{n} + ...\) &  \\
  \hline
  9. & Differencial teńlemeniń ulıwma sheshimin tabıń: \(y' = e^{x}\). &  \\
  \hline
  10. & Qutada 5 aq hám 15 qara shar bar. Tosınnan alınǵan bir shardıń aq bolıw itimallıǵın tabıń &  \\
  \hline
  \end{tabular}
  
  \vspace{1cm}
  
  \begin{tabular}{lll}
  Tuwrı juwaplar sanı: \underline{\hspace{1.5cm}} & 
  Bahası: \underline{\hspace{1.5cm}} & 
  Imtixan alıwshınıń qolı: \underline{\hspace{2cm}} \\
  \end{tabular}
  
  \egroup
  
  \newpage
  
  
  \textbf{43-variant}\\
  
  \bgroup
  \def\arraystretch{1.6} % 1 is the default, change whatever you need
  
  \begin{tabular}{|m{5.7cm}|m{9.5cm}|}
  \hline
  Shifr & \\
  \hline
  \end{tabular}
  
  \vspace{1cm}
  
  \begin{tabular}{|m{0.7cm}|m{10cm}|m{4cm}|}
  \hline
  № & Soraw & Juwap \\
  \hline
  1. & Kóp aǵzalını \((x - a)\) ǵa bólgendegi qaldıq nege teń &  \\
  \hline
  2. & Funkcianıń \((x_{0},\ y_{0})\) noqattaǵı tuwındısınıń formulasın jazıń &  \\
  \hline
  3. & Bernulli differenciallıq teńemesin jazıń &  \\
  \hline
  4. & Orın almastırıw formulasın jazıń &  \\
  \hline
  5. & Anıq emes integraldı esaplań: \(\int\frac{dx}{cos^2 x}\). &  \\
  \hline
  6. & Esaplań: \(\int\left( x^{4} - \frac{1}{x} \right)dx\). &  \\
  \hline
  7. & Anıq integraldı esaplań: \(\int_{0}^{\frac{\pi}{2}}{\cos xdx}\). &  \\
  \hline
  8. & Funkcional qatardıń jıynaqlılıq oblastın tabıń:\(1 + x + ... + x^{n} + ...\) &  \\
  \hline
  9. & Differencial teńlemeni sheshiń: \(y' + xy = 0\). &  \\
  \hline
  10. & Telefon nomerdiń aqırǵı eki cifrasın umıtıp, tosınnan nomerlerdi tere basladı. Kerekli nomerdi tabıw itimallıǵın esaplań. &  \\
  \hline
  \end{tabular}
  
  \vspace{1cm}
  
  \begin{tabular}{lll}
  Tuwrı juwaplar sanı: \underline{\hspace{1.5cm}} & 
  Bahası: \underline{\hspace{1.5cm}} & 
  Imtixan alıwshınıń qolı: \underline{\hspace{2cm}} \\
  \end{tabular}
  
  \egroup
  
  \newpage
  
  
  \textbf{44-variant}\\
  
  \bgroup
  \def\arraystretch{1.6} % 1 is the default, change whatever you need
  
  \begin{tabular}{|m{5.7cm}|m{9.5cm}|}
  \hline
  Shifr & \\
  \hline
  \end{tabular}
  
  \vspace{1cm}
  
  \begin{tabular}{|m{0.7cm}|m{10cm}|m{4cm}|}
  \hline
  № & Soraw & Juwap \\
  \hline
  1. & Funkciya qanday usıllarda beriledi &  \\
  \hline
  2. & \((x_0,y_0)\) tochkanıń \(\varepsilon\) dógeregi qalay belgilenedi &  \\
  \hline
  3. & Ózgeriwshileri ajıralǵan differenciallıq teńlemesiniń uluwma kórinisin jazıń &  \\
  \hline
  4. & Itimmallıqtıń geometriyalıq anıqlamasınıń formulasın jazıń &  \\
  \hline
  5. & Anıq emes integraldı esaplań: \(\int{\left( x^2  + \frac{1}{x} + \sin x \right)dx}\). &  \\
  \hline
  6. & Integraldı esaplań: \(\int{2^{x}dx}\). &  \\
  \hline
  7. & Anıq integraldı esaplań: \(\int_{0}^{1}{(3x^2 } + 1)dx\). &  \\
  \hline
  8. & Funkcional qatardıń jıynaqlılıq oblastın jazıń: \(\ln x + ln^2 x + ... + ln^{n}x + ...\). &  \\
  \hline
  9. & Differencial teńlemeniń ulıwma sheshimin tabıń: \(y' = e^{x}\). &  \\
  \hline
  10. & 50 buyımnan ibarat partiyada 3 buyım jaramsız. Tosınnan alınǵan 8 buyımnıń ishinde 1 buyımı jaramsız bolıw itimallıǵın tabıń &  \\
  \hline
  \end{tabular}
  
  \vspace{1cm}
  
  \begin{tabular}{lll}
  Tuwrı juwaplar sanı: \underline{\hspace{1.5cm}} & 
  Bahası: \underline{\hspace{1.5cm}} & 
  Imtixan alıwshınıń qolı: \underline{\hspace{2cm}} \\
  \end{tabular}
  
  \egroup
  
  \newpage
  
  
  \textbf{45-variant}\\
  
  \bgroup
  \def\arraystretch{1.6} % 1 is the default, change whatever you need
  
  \begin{tabular}{|m{5.7cm}|m{9.5cm}|}
  \hline
  Shifr & \\
  \hline
  \end{tabular}
  
  \vspace{1cm}
  
  \begin{tabular}{|m{0.7cm}|m{10cm}|m{4cm}|}
  \hline
  № & Soraw & Juwap \\
  \hline
  1. & \(n\)-dárejeli kóp aǵzalınıń uluwma kórinisi &  \\
  \hline
  2. & Eki ózgeriwshli funkciyanıń \(M(x_{0}, y_{0})\) noqattaǵı úzliksizliginiń anıqlaması &  \\
  \hline
  3. & Funkcionallıq qatardıń uluwma kórinisi &  \\
  \hline
  4. & Shártli itimallıq formulasın jazıń &  \\
  \hline
  5. & Racional funkciyanı integrallań: \(\int{\frac{3}{(x - 1)(x + 2)}dx}\). &  \\
  \hline
  6. & Integraldı esaplań: \(\int_{1}^{\infty}{\frac{1}{x^2 }dx}\). &  \\
  \hline
  7. & Anıq integraldı esaplań: \(\int_{2}^{4}\frac{dx}{x}\). &  \\
  \hline
  8. & Sanlı qatardıń baslanǵısh úsh aǵzasın jazıń: \(\sum_{n = 1}^{\infty}\frac{n!}{2^{n}}\). &  \\
  \hline
  9. & Differencial teńlemeni esaplań: \(yy' = 4\). &  \\
  \hline
  10. & Qutada 5 aq hám 15 qara shar bar. Tosınnan alınǵan bir shardıń aq bolıw itimallıǵın tabıń &  \\
  \hline
  \end{tabular}
  
  \vspace{1cm}
  
  \begin{tabular}{lll}
  Tuwrı juwaplar sanı: \underline{\hspace{1.5cm}} & 
  Bahası: \underline{\hspace{1.5cm}} & 
  Imtixan alıwshınıń qolı: \underline{\hspace{2cm}} \\
  \end{tabular}
  
  \egroup
  
  \newpage
  
  
  \textbf{46-variant}\\
  
  \bgroup
  \def\arraystretch{1.6} % 1 is the default, change whatever you need
  
  \begin{tabular}{|m{5.7cm}|m{9.5cm}|}
  \hline
  Shifr & \\
  \hline
  \end{tabular}
  
  \vspace{1cm}
  
  \begin{tabular}{|m{0.7cm}|m{10cm}|m{4cm}|}
  \hline
  № & Soraw & Juwap \\
  \hline
  1. & Eki ózgeriwshili funkciyanıń ekinshi tártipli aralas tuwındıları qalay belgilenedi &  \\
  \hline
  2. & Anıq integraldı esaplawdıń Nyuton-Leybnic formulasın jazıń &  \\
  \hline
  3. & Oń aǵzalı qatarlar ushın jıynaqlılıqtıń Koshi belgisin jazıń &  \\
  \hline
  4. & Orın awıstırıw formulasın jazıń &  \\
  \hline
  5. & Anıq emes integraldı esaplań: \(\int\frac{dx}{cos^2 x}\). &  \\
  \hline
  6. & Integraldı esaplań: \(\int_{1}^{\infty}{\frac{1}{(x + 2)^2 }dx}\). &  \\
  \hline
  7. & Anıq integraldı esaplań: \(\int_{0}^{\frac{\pi}{2}}{\cos xdx}\). &  \\
  \hline
  8. & Funkcional qatardıń jaqınlasıw oblastın tabıń: \(x + \frac{x^2 }{2^2 } + ... + \frac{x^{n}}{n^2 } + ...\) &  \\
  \hline
  9. & Sızıqlı differencial teńlemeniń ulwma sheshimin tabıń: \(y' + y = e^{x}\). &  \\
  \hline
  10. & Úsh birdey korobkada aq hám qara sharlar bar. 1-korobkada 5 aq, 8 qara shar, 2-korobkada 3 aq, 4 qara shar, 3-korobkada 2 aq, 3 qara shar bar. Úsh korobkaniń birewinen tosınnan alınǵan bir shar aq bolıw itimallıǵın tabıń. &  \\
  \hline
  \end{tabular}
  
  \vspace{1cm}
  
  \begin{tabular}{lll}
  Tuwrı juwaplar sanı: \underline{\hspace{1.5cm}} & 
  Bahası: \underline{\hspace{1.5cm}} & 
  Imtixan alıwshınıń qolı: \underline{\hspace{2cm}} \\
  \end{tabular}
  
  \egroup
  
  \newpage
  
  
  \textbf{47-variant}\\
  
  \bgroup
  \def\arraystretch{1.6} % 1 is the default, change whatever you need
  
  \begin{tabular}{|m{5.7cm}|m{9.5cm}|}
  \hline
  Shifr & \\
  \hline
  \end{tabular}
  
  \vspace{1cm}
  
  \begin{tabular}{|m{0.7cm}|m{10cm}|m{4cm}|}
  \hline
  № & Soraw & Juwap \\
  \hline
  1. & Eki ózgeriwshili funkciyalar qalay belgilenedi &  \\
  \hline
  2. & Ózgeriwshini almastırıp integrallaw usılıniń formulasın jazıń. &  \\
  \hline
  3. & Oń aǵzalı qatarlar ushın jıynaqlılıqtıń Dalamber belgisin jazıń &  \\
  \hline
  4. & Itimallıq keńisligin jazıń &  \\
  \hline
  5. & Racional funkciyanı integrallań: \(\int{\frac{3}{(x - 1)(x + 2)}dx}\). &  \\
  \hline
  6. & Anıq integraldı esaplań: \(\int_{1}^{3}\frac{2}{x + 1}dx\). &  \\
  \hline
  7. & Anıq integraldı esaplań: \(\int_{2}^{4}\frac{dx}{x}\). &  \\
  \hline
  8. & Qatardıń jıyındısın esaplań: \(\sum_{n = 1}^{\infty}\frac{1}{(2n - 1)(2n + 1)}\). &  \\
  \hline
  9. & Sızıqlı differerncial teńlemeniń uluwma sheshimin tabıń \(y' + y = e^{- x}\). &  \\
  \hline
  10. & «BIOLOGIYA» sóziniń háripleri bólek kartochkalarǵa jazılıp jawıp, aralastırılıp qoyılǵan. Barlıq kartochkalar tosınnan izbe-iz alınıp ashılıp, alınıw tártibinde stol ústine dizilgende taǵı «BIOLOGIYA» sóziniń kelip shıǵıw itimallıǵın tabıń. &  \\
  \hline
  \end{tabular}
  
  \vspace{1cm}
  
  \begin{tabular}{lll}
  Tuwrı juwaplar sanı: \underline{\hspace{1.5cm}} & 
  Bahası: \underline{\hspace{1.5cm}} & 
  Imtixan alıwshınıń qolı: \underline{\hspace{2cm}} \\
  \end{tabular}
  
  \egroup
  
  \newpage
  
  
  \textbf{48-variant}\\
  
  \bgroup
  \def\arraystretch{1.6} % 1 is the default, change whatever you need
  
  \begin{tabular}{|m{5.7cm}|m{9.5cm}|}
  \hline
  Shifr & \\
  \hline
  \end{tabular}
  
  \vspace{1cm}
  
  \begin{tabular}{|m{0.7cm}|m{10cm}|m{4cm}|}
  \hline
  № & Soraw & Juwap \\
  \hline
  1. & Eki ózgeriwshili funkciyanıń birinshi tártipli dara tuwındıları qalay belgilenedi &  \\
  \hline
  2. & Funkcianıń \((x_{0},\ y_{0})\) noqattaǵı tuwındısınıń formulasın jazıń &  \\
  \hline
  3. & Eger \(\sum_{n = 1}^{\infty}a_{n} = A,\ \sum_{n = 1}^{\infty}b_{n} = B\) bolsa, onda \(\sum_{n = 1}^{\infty}\left( a_{n} + b_{n} \right) = ?\) &  \\
  \hline
  4. & Shekli additivlik aksiomasın jazıń &  \\
  \hline
  5. & Anıq emes integraldı esaplań: \(\int{\left( x^2  + \frac{1}{x} + \sin x \right)dx}\). &  \\
  \hline
  6. & Esaplań: \(\int_{1}^2 {e^{x}dx}\). &  \\
  \hline
  7. & Anıq integraldı esaplań: \(\int_{- \pi/4}^{0}\frac{dx}{cos^2 x}\). &  \\
  \hline
  8. & Qatardıń qosındısın tabıń: \(\sum_{n = 1}^{\infty}\frac{1}{n(n + 3)}\). &  \\
  \hline
  9. & Differencial teńlemeniń ulıwma sheshimin tabıń: \(xy' - 2y = 0\). &  \\
  \hline
  10. & Dóngelektiń ishine kvadrat sızılǵan. Dóngelektiń ishinen tosınnan belgilengen noqattıń kvadrattıń ishinde jatıw itimallıǵın tabıń. &  \\
  \hline
  \end{tabular}
  
  \vspace{1cm}
  
  \begin{tabular}{lll}
  Tuwrı juwaplar sanı: \underline{\hspace{1.5cm}} & 
  Bahası: \underline{\hspace{1.5cm}} & 
  Imtixan alıwshınıń qolı: \underline{\hspace{2cm}} \\
  \end{tabular}
  
  \egroup
  
  \newpage
  
  
  \textbf{49-variant}\\
  
  \bgroup
  \def\arraystretch{1.6} % 1 is the default, change whatever you need
  
  \begin{tabular}{|m{5.7cm}|m{9.5cm}|}
  \hline
  Shifr & \\
  \hline
  \end{tabular}
  
  \vspace{1cm}
  
  \begin{tabular}{|m{0.7cm}|m{10cm}|m{4cm}|}
  \hline
  № & Soraw & Juwap \\
  \hline
  1. & Eki ózgeriwshili funkciyanıń anıqlanıw oblastı qay jerde jaylasadı &  \\
  \hline
  2. & Eki ózgeriwshli funkciyanıń \(M(x_{0}, y_{0})\) noqattaǵı úzliksizliginiń anıqlaması &  \\
  \hline
  3. & Sızıqlı defferencial teńlemeniń uluwma sheshimin jazıń &  \\
  \hline
  4. & Tolıq itimallıqtıń formulasın jazıń &  \\
  \hline
  5. & Anıq emes integraldı esaplań: \(\int{e^{x}dx}\) . &  \\
  \hline
  6. & Esaplań: \(\int_{1}^2 {e^{x}dx}\). &  \\
  \hline
  7. & Anıq integraldı esaplań: \(\int_{0}^{\pi}{\sin xdx}\). &  \\
  \hline
  8. & Qatardıń qosındısın tabıń: \(\sum_{n = 1}^{\infty}\frac{1}{n(n + 1)}\). &  \\
  \hline
  9. & Differencial teńlemeni sheshiń: \(y' + xy = 0\). &  \\
  \hline
  10. & Tiyindi eki márte taslaǵanda, keminde bir márte san tárepi túsiw itimallıǵın tabıń. &  \\
  \hline
  \end{tabular}
  
  \vspace{1cm}
  
  \begin{tabular}{lll}
  Tuwrı juwaplar sanı: \underline{\hspace{1.5cm}} & 
  Bahası: \underline{\hspace{1.5cm}} & 
  Imtixan alıwshınıń qolı: \underline{\hspace{2cm}} \\
  \end{tabular}
  
  \egroup
  
  \newpage
  
  
  \textbf{50-variant}\\
  
  \bgroup
  \def\arraystretch{1.6} % 1 is the default, change whatever you need
  
  \begin{tabular}{|m{5.7cm}|m{9.5cm}|}
  \hline
  Shifr & \\
  \hline
  \end{tabular}
  
  \vspace{1cm}
  
  \begin{tabular}{|m{0.7cm}|m{10cm}|m{4cm}|}
  \hline
  № & Soraw & Juwap \\
  \hline
  1. & Eki ózgeriwshili funkciyanıń birinshi tártipli dara tuwındıları qalay belgilenedi &  \\
  \hline
  2. & Esaplań \(d\left( \int{f(x)dx} \right) = ?\) &  \\
  \hline
  3. & Sızıqlı differenciallıq teńlemeniń uluwma kórinisin jazıń &  \\
  \hline
  4. & Múmkin emes waqıyanıń itimaıllıǵı nege teń &  \\
  \hline
  5. & Racional funkciyanı integrallań: \(\int{\frac{5}{(x - 3)(x + 2)}dx}\). &  \\
  \hline
  6. & Integraldı esaplań: \(\int{2^{x}dx}\). &  \\
  \hline
  7. & Anıq integraldı esaplań: \(\int_{0}^{1}{(3x^2 } + 1)dx\). &  \\
  \hline
  8. & Qatardıń qosındısın tabıń: \(\sum_{n = 1}^{\infty}\frac{1}{n(n + 3)}\). &  \\
  \hline
  9. & Sızıqlı differencial teńlemeniń ulwma sheshimin tabıń: \(y' + y = e^{x}\). &  \\
  \hline
  10. & Eki kubikti bir márte taslaǵanda túsken ochkolardıń qosındısı 4 bolıw itimallıǵın tabıń. &  \\
  \hline
  \end{tabular}
  
  \vspace{1cm}
  
  \begin{tabular}{lll}
  Tuwrı juwaplar sanı: \underline{\hspace{1.5cm}} & 
  Bahası: \underline{\hspace{1.5cm}} & 
  Imtixan alıwshınıń qolı: \underline{\hspace{2cm}} \\
  \end{tabular}
  
  \egroup
  
  \newpage
  
  
  \textbf{51-variant}\\
  
  \bgroup
  \def\arraystretch{1.6} % 1 is the default, change whatever you need
  
  \begin{tabular}{|m{5.7cm}|m{9.5cm}|}
  \hline
  Shifr & \\
  \hline
  \end{tabular}
  
  \vspace{1cm}
  
  \begin{tabular}{|m{0.7cm}|m{10cm}|m{4cm}|}
  \hline
  № & Soraw & Juwap \\
  \hline
  1. & Eki ózgeriwshili funkciyalar qalay belgilenedi &  \\
  \hline
  2. & Esaplań \(\left( \int{f(x)dx} \right)' = ?\). &  \\
  \hline
  3. & Sızıqlı defferencial teńlemeniń uluwma sheshimin jazıń &  \\
  \hline
  4. & Gruppalaw formulasın jazıń &  \\
  \hline
  5. & Integraldı esaplań: \(\int{(x + \sin x)dx}\). &  \\
  \hline
  6. & Esaplań: \(\int\left( x^{4} - \frac{1}{x} \right)dx\). &  \\
  \hline
  7. & Anıq integraldı esaplań: \(\int_{2}^{4}\frac{dx}{x}\). &  \\
  \hline
  8. & Funkcional qatardıń jaqınlasıw oblastın tabıń: \(x + \frac{x^2 }{2^2 } + ... + \frac{x^{n}}{n^2 } + ...\) &  \\
  \hline
  9. & Differencial teńlemeni esaplań: \(yy' = 4\). &  \\
  \hline
  10. & «MATEMATIKA» sóziniń háripleri bólek kartochkalarǵa jazılıp jawıp aralastırılıp qoyılǵan. Barlıq kartochkalar tosınnan izbe-iz alınıp ashılıp, alınıw tártibinde stol ústine dizilgende taǵı «MATEMATIKA» sóziniń kelip shıǵıw itimallıǵın tabıń. &  \\
  \hline
  \end{tabular}
  
  \vspace{1cm}
  
  \begin{tabular}{lll}
  Tuwrı juwaplar sanı: \underline{\hspace{1.5cm}} & 
  Bahası: \underline{\hspace{1.5cm}} & 
  Imtixan alıwshınıń qolı: \underline{\hspace{2cm}} \\
  \end{tabular}
  
  \egroup
  
  \newpage
  
  
  \textbf{52-variant}\\
  
  \bgroup
  \def\arraystretch{1.6} % 1 is the default, change whatever you need
  
  \begin{tabular}{|m{5.7cm}|m{9.5cm}|}
  \hline
  Shifr & \\
  \hline
  \end{tabular}
  
  \vspace{1cm}
  
  \begin{tabular}{|m{0.7cm}|m{10cm}|m{4cm}|}
  \hline
  № & Soraw & Juwap \\
  \hline
  1. & Funkciya qanday usıllarda beriledi &  \\
  \hline
  2. & \((x_0,y_0)\) tochkanıń \(\varepsilon\) dógeregi qalay belgilenedi &  \\
  \hline
  3. & Oń aǵzalı qatarlar ushın jıynaqlılıqtıń Dalamber belgisin jazıń &  \\
  \hline
  4. & Isenimli waqıyanıń itimallıǵı nege teń &  \\
  \hline
  5. & Integraldı esaplań: \(\int{\frac{1}{\sin x}dx}\). &  \\
  \hline
  6. & Integraldı esaplań: \(\int_{1}^{\infty}{\frac{1}{x^2 }dx}\). &  \\
  \hline
  7. & Anıq integraldı esaplań: \(\int_{0}^{1}{(3x^2 } + 1)dx\). &  \\
  \hline
  8. & Qatardıń qosındısın tabıń: \(\sum_{n = 1}^{\infty}\frac{1}{n(n + 1)}\). &  \\
  \hline
  9. & Differencial teńlemeniń ulıwma sheshimin tabıń: \(xy' - 2y = 0\). &  \\
  \hline
  10. & Korobkada 15 aq, 18 qara shar bar. Tosınnan alınǵan bir shar aq bolıw itimallıǵın tabıń. &  \\
  \hline
  \end{tabular}
  
  \vspace{1cm}
  
  \begin{tabular}{lll}
  Tuwrı juwaplar sanı: \underline{\hspace{1.5cm}} & 
  Bahası: \underline{\hspace{1.5cm}} & 
  Imtixan alıwshınıń qolı: \underline{\hspace{2cm}} \\
  \end{tabular}
  
  \egroup
  
  \newpage
  
  
  \textbf{53-variant}\\
  
  \bgroup
  \def\arraystretch{1.6} % 1 is the default, change whatever you need
  
  \begin{tabular}{|m{5.7cm}|m{9.5cm}|}
  \hline
  Shifr & \\
  \hline
  \end{tabular}
  
  \vspace{1cm}
  
  \begin{tabular}{|m{0.7cm}|m{10cm}|m{4cm}|}
  \hline
  № & Soraw & Juwap \\
  \hline
  1. & Kóp aǵzalını \((x - a)\) ǵa bólgendegi qaldıq nege teń &  \\
  \hline
  2. & Bóleklep inegrallaw formulasın jazıń &  \\
  \hline
  3. & Funkcionallıq qatardıń uluwma kórinisi &  \\
  \hline
  4. & Itimallıqtıń klassikalıq anıqlamasınıń formulasın keltiriń &  \\
  \hline
  5. & Anıq emes integraldı esaplań: \(\int{\left( 10x^{4} + 7x^{6} - 3 \right)dx}\). &  \\
  \hline
  6. & Integraldı esaplań: \(\int_{1}^{\infty}{\frac{1}{(x + 2)^2 }dx}\). &  \\
  \hline
  7. & Anıq integraldı esaplań: \(\int_{0}^{\frac{\pi}{2}}{\cos xdx}\). &  \\
  \hline
  8. & Funkcional qatardıń jıynaqlılıq oblastın jazıń: \(\ln x + ln^2 x + ... + ln^{n}x + ...\). &  \\
  \hline
  9. & Sızıqlı differerncial teńlemeniń uluwma sheshimin tabıń \(y' + y = e^{- x}\). &  \\
  \hline
  10. & Gruppadaǵı 20 studentten neshe túrli usıl menen 3 náwbetshini saylap alıwǵa boladı? &  \\
  \hline
  \end{tabular}
  
  \vspace{1cm}
  
  \begin{tabular}{lll}
  Tuwrı juwaplar sanı: \underline{\hspace{1.5cm}} & 
  Bahası: \underline{\hspace{1.5cm}} & 
  Imtixan alıwshınıń qolı: \underline{\hspace{2cm}} \\
  \end{tabular}
  
  \egroup
  
  \newpage
  
  
  \textbf{54-variant}\\
  
  \bgroup
  \def\arraystretch{1.6} % 1 is the default, change whatever you need
  
  \begin{tabular}{|m{5.7cm}|m{9.5cm}|}
  \hline
  Shifr & \\
  \hline
  \end{tabular}
  
  \vspace{1cm}
  
  \begin{tabular}{|m{0.7cm}|m{10cm}|m{4cm}|}
  \hline
  № & Soraw & Juwap \\
  \hline
  1. & Eki ózgeriwshili funkciyanıń ekinshi tártipli aralas tuwındıları qalay belgilenedi &  \\
  \hline
  2. & Funkcianıń \((x_{0},\ y_{0})\) noqattaǵı úzliksizliginiń formulasın jazıń &  \\
  \hline
  3. & Sızıqlı differenciallıq teńlemeniń uluwma kórinisin jazıń &  \\
  \hline
  4. & Múmkin emes waqıyanıń itimaıllıǵı nege teń &  \\
  \hline
  5. & Integraldı esaplań:\(\int{(x - 1)^{20}}dx\). &  \\
  \hline
  6. & Anıq integraldı esaplań: \(\int_{1}^{3}\frac{2}{x + 1}dx\). &  \\
  \hline
  7. & Anıq integraldı esaplań: \(\int_{0}^{\pi}{\sin xdx}\). &  \\
  \hline
  8. & Funkcional qatardıń jıynaqlılıq oblastın tabıń:\(1 + x + ... + x^{n} + ...\) &  \\
  \hline
  9. & Differencial teńlemeniń ulıwma sheshimin tabıń: \(y' = e^{x}\). &  \\
  \hline
  10. & Telefon nomerdiń aqırǵı cifrasın umıtıp, tosınnan nomerlerdi tere basladı. Kerekli nomerdi tabıw itimallıǵın esaplań. &  \\
  \hline
  \end{tabular}
  
  \vspace{1cm}
  
  \begin{tabular}{lll}
  Tuwrı juwaplar sanı: \underline{\hspace{1.5cm}} & 
  Bahası: \underline{\hspace{1.5cm}} & 
  Imtixan alıwshınıń qolı: \underline{\hspace{2cm}} \\
  \end{tabular}
  
  \egroup
  
  \newpage
  
  
  \textbf{55-variant}\\
  
  \bgroup
  \def\arraystretch{1.6} % 1 is the default, change whatever you need
  
  \begin{tabular}{|m{5.7cm}|m{9.5cm}|}
  \hline
  Shifr & \\
  \hline
  \end{tabular}
  
  \vspace{1cm}
  
  \begin{tabular}{|m{0.7cm}|m{10cm}|m{4cm}|}
  \hline
  № & Soraw & Juwap \\
  \hline
  1. & Eki ózgeriwshili funkciyanıń ekinshi tártipli dara tuwındıları qalay belgilenedi &  \\
  \hline
  2. & \((x_0,y_0)\) tochkanıń \(\varepsilon\) dógeregi qalay belgilenedi &  \\
  \hline
  3. & Sanlı qatardıń uluwma kórinisin jazıń &  \\
  \hline
  4. & Bayes formulasın jazıń &  \\
  \hline
  5. & Integraldı esaplań:\(\int{(x - 1)^{20}}dx\). &  \\
  \hline
  6. & Integraldı esaplań: \(\int_{1}^{\infty}{\frac{1}{x^2 }dx}\). &  \\
  \hline
  7. & Anıq integraldı esaplań: \(\int_{- \pi/4}^{0}\frac{dx}{cos^2 x}\). &  \\
  \hline
  8. & Sanlı qatardıń baslanǵısh úsh aǵzasın jazıń: \(\sum_{n = 1}^{\infty}\frac{n!}{2^{n}}\). &  \\
  \hline
  9. & Differencial teńlemeniń ulıwma sheshimin tabıń: \(y' = e^{x}\). &  \\
  \hline
  10. & Ídısta 5 aq, 8 qara shar bar. Ídıstan tosınnan izbe-iz 3 shar alındı. Alınǵan sharlar aq, qara, qara degen izbe-izlikte bolıw itimallıǵın tabıń. &  \\
  \hline
  \end{tabular}
  
  \vspace{1cm}
  
  \begin{tabular}{lll}
  Tuwrı juwaplar sanı: \underline{\hspace{1.5cm}} & 
  Bahası: \underline{\hspace{1.5cm}} & 
  Imtixan alıwshınıń qolı: \underline{\hspace{2cm}} \\
  \end{tabular}
  
  \egroup
  
  \newpage
  
  
  \textbf{56-variant}\\
  
  \bgroup
  \def\arraystretch{1.6} % 1 is the default, change whatever you need
  
  \begin{tabular}{|m{5.7cm}|m{9.5cm}|}
  \hline
  Shifr & \\
  \hline
  \end{tabular}
  
  \vspace{1cm}
  
  \begin{tabular}{|m{0.7cm}|m{10cm}|m{4cm}|}
  \hline
  № & Soraw & Juwap \\
  \hline
  1. & Eki ózgeriwshili funkciyanıń tolıq ósimi &  \\
  \hline
  2. & Ózgeriwshini almastırıp integrallaw usılıniń formulasın jazıń. &  \\
  \hline
  3. & Eger \(\sum_{n = 1}^{\infty}a_{n} = A,\ \sum_{n = 1}^{\infty}b_{n} = B\) bolsa, onda \(\sum_{n = 1}^{\infty}\left( a_{n} - b_{n} \right) = ?\) &  \\
  \hline
  4. & Orın almastırıw formulasın jazıń &  \\
  \hline
  5. & Integraldı esaplań: \(\int{\frac{1}{\sin x}dx}\). &  \\
  \hline
  6. & Integraldı esaplań: \(\int{2^{x}dx}\). &  \\
  \hline
  7. & Anıq integraldı esaplań: \(\int_{0}^{\pi}{\sin xdx}\). &  \\
  \hline
  8. & Qatardıń jıyındısın esaplań: \(\sum_{n = 1}^{\infty}\frac{1}{(2n - 1)(2n + 1)}\). &  \\
  \hline
  9. & Differencial teńlemeni sheshiń: \(y' + xy = 0\). &  \\
  \hline
  10. & Korobkada 3 aq, 7 qara shar bar. Tosınnan úsh shar izbe-iz alındı. Izbe-iz alınǵan sharlardıń qara, qara, aq degen izbe-izlikte bolıw itimallıǵın tabıń. &  \\
  \hline
  \end{tabular}
  
  \vspace{1cm}
  
  \begin{tabular}{lll}
  Tuwrı juwaplar sanı: \underline{\hspace{1.5cm}} & 
  Bahası: \underline{\hspace{1.5cm}} & 
  Imtixan alıwshınıń qolı: \underline{\hspace{2cm}} \\
  \end{tabular}
  
  \egroup
  
  \newpage
  
  
  \textbf{57-variant}\\
  
  \bgroup
  \def\arraystretch{1.6} % 1 is the default, change whatever you need
  
  \begin{tabular}{|m{5.7cm}|m{9.5cm}|}
  \hline
  Shifr & \\
  \hline
  \end{tabular}
  
  \vspace{1cm}
  
  \begin{tabular}{|m{0.7cm}|m{10cm}|m{4cm}|}
  \hline
  № & Soraw & Juwap \\
  \hline
  1. & Funkciyanıń anıqlanıw oblastı qalay belgilenedi &  \\
  \hline
  2. & Funkcianıń \((x_{0},\ y_{0})\) noqattaǵı tuwındısınıń formulasın jazıń &  \\
  \hline
  3. & Bernulli differenciallıq teńemesin jazıń &  \\
  \hline
  4. & Itimallıqtıń mánisler oblastın jazıń &  \\
  \hline
  5. & Integraldı esaplań: \(\int{(x + \sin x)dx}\). &  \\
  \hline
  6. & Esaplań: \(\int_{1}^2 {e^{x}dx}\). &  \\
  \hline
  7. & Anıq integraldı esaplań: \(\int_{2}^{4}\frac{dx}{x}\). &  \\
  \hline
  8. & Qatardıń qosındısın tabıń: \(\sum_{n = 1}^{\infty}\frac{1}{n(n + 3)}\). &  \\
  \hline
  9. & Sızıqlı differencial teńlemeniń ulwma sheshimin tabıń: \(y' + y = e^{x}\). &  \\
  \hline
  10. & Ídısta 5 aq, 8 qara shar bar. Ídıstan tosınnan izbe-iz 3 shar alındı. Alınǵan sharlar aq, qara, qara degen izbe-izlikte bolıw itimallıǵın tabıń. &  \\
  \hline
  \end{tabular}
  
  \vspace{1cm}
  
  \begin{tabular}{lll}
  Tuwrı juwaplar sanı: \underline{\hspace{1.5cm}} & 
  Bahası: \underline{\hspace{1.5cm}} & 
  Imtixan alıwshınıń qolı: \underline{\hspace{2cm}} \\
  \end{tabular}
  
  \egroup
  
  \newpage
  
  
  \textbf{58-variant}\\
  
  \bgroup
  \def\arraystretch{1.6} % 1 is the default, change whatever you need
  
  \begin{tabular}{|m{5.7cm}|m{9.5cm}|}
  \hline
  Shifr & \\
  \hline
  \end{tabular}
  
  \vspace{1cm}
  
  \begin{tabular}{|m{0.7cm}|m{10cm}|m{4cm}|}
  \hline
  № & Soraw & Juwap \\
  \hline
  1. & Eki ózgeriwshili funkciyanıń ekstremumınıń zárúrli shárti &  \\
  \hline
  2. & Anıq integraldı esaplawdıń Nyuton-Leybnic formulasın jazıń &  \\
  \hline
  3. & Ózgeriwshileri ajıralǵan differenciallıq teńlemesiniń uluwma kórinisin jazıń &  \\
  \hline
  4. & Orın awıstırıw formulasın jazıń &  \\
  \hline
  5. & Racional funkciyanı integrallań: \(\int{\frac{3}{(x - 1)(x + 2)}dx}\). &  \\
  \hline
  6. & Integraldı esaplań: \(\int_{1}^{\infty}{\frac{1}{(x + 2)^2 }dx}\). &  \\
  \hline
  7. & Anıq integraldı esaplań: \(\int_{0}^{\frac{\pi}{2}}{\cos xdx}\). &  \\
  \hline
  8. & Qatardıń qosındısın tabıń: \(\sum_{n = 1}^{\infty}\frac{1}{n(n + 1)}\). &  \\
  \hline
  9. & Differencial teńlemeni esaplań: \(yy' = 4\). &  \\
  \hline
  10. & 50 buyımnan ibarat partiyada 3 buyım jaramsız. Tosınnan alınǵan 8 buyımnıń ishinde 1 buyımı jaramsız bolıw itimallıǵın tabıń &  \\
  \hline
  \end{tabular}
  
  \vspace{1cm}
  
  \begin{tabular}{lll}
  Tuwrı juwaplar sanı: \underline{\hspace{1.5cm}} & 
  Bahası: \underline{\hspace{1.5cm}} & 
  Imtixan alıwshınıń qolı: \underline{\hspace{2cm}} \\
  \end{tabular}
  
  \egroup
  
  \newpage
  
  
  \textbf{59-variant}\\
  
  \bgroup
  \def\arraystretch{1.6} % 1 is the default, change whatever you need
  
  \begin{tabular}{|m{5.7cm}|m{9.5cm}|}
  \hline
  Shifr & \\
  \hline
  \end{tabular}
  
  \vspace{1cm}
  
  \begin{tabular}{|m{0.7cm}|m{10cm}|m{4cm}|}
  \hline
  № & Soraw & Juwap \\
  \hline
  1. & Eki ózgeriwshili funkciyanıń grafigi neden ibarat &  \\
  \hline
  2. & Funkcianıń \((x_{0},\ y_{0})\) noqattaǵı úzliksizliginiń formulasın jazıń &  \\
  \hline
  3. & Oń aǵzalı qatarlar ushın jıynaqlılıqtıń Koshi belgisin jazıń &  \\
  \hline
  4. & Itimallıq keńisligin jazıń &  \\
  \hline
  5. & Anıq emes integraldı esaplań: \(\int{\left( 10x^{4} + 7x^{6} - 3 \right)dx}\). &  \\
  \hline
  6. & Anıq integraldı esaplań: \(\int_{1}^{3}\frac{2}{x + 1}dx\). &  \\
  \hline
  7. & Anıq integraldı esaplań: \(\int_{- \pi/4}^{0}\frac{dx}{cos^2 x}\). &  \\
  \hline
  8. & Funkcional qatardıń jıynaqlılıq oblastın tabıń:\(1 + x + ... + x^{n} + ...\) &  \\
  \hline
  9. & Sızıqlı differerncial teńlemeniń uluwma sheshimin tabıń \(y' + y = e^{- x}\). &  \\
  \hline
  10. & Korobkada 15 aq, 18 qara shar bar. Tosınnan alınǵan bir shar aq bolıw itimallıǵın tabıń. &  \\
  \hline
  \end{tabular}
  
  \vspace{1cm}
  
  \begin{tabular}{lll}
  Tuwrı juwaplar sanı: \underline{\hspace{1.5cm}} & 
  Bahası: \underline{\hspace{1.5cm}} & 
  Imtixan alıwshınıń qolı: \underline{\hspace{2cm}} \\
  \end{tabular}
  
  \egroup
  
  \newpage
  
  
  \textbf{60-variant}\\
  
  \bgroup
  \def\arraystretch{1.6} % 1 is the default, change whatever you need
  
  \begin{tabular}{|m{5.7cm}|m{9.5cm}|}
  \hline
  Shifr & \\
  \hline
  \end{tabular}
  
  \vspace{1cm}
  
  \begin{tabular}{|m{0.7cm}|m{10cm}|m{4cm}|}
  \hline
  № & Soraw & Juwap \\
  \hline
  1. & \(n\)-dárejeli kóp aǵzalınıń uluwma kórinisi &  \\
  \hline
  2. & Esaplań \(\left( \int{f(x)dx} \right)' = ?\). &  \\
  \hline
  3. & Eger \(\sum_{n = 1}^{\infty}a_{n} = A,\ \sum_{n = 1}^{\infty}b_{n} = B\) bolsa, onda \(\sum_{n = 1}^{\infty}\left( a_{n} + b_{n} \right) = ?\) &  \\
  \hline
  4. & Tolıq itimallıqtıń formulasın jazıń &  \\
  \hline
  5. & Anıq emes integraldı esaplań: \(\int\frac{dx}{cos^2 x}\). &  \\
  \hline
  6. & Esaplań: \(\int\left( x^{4} - \frac{1}{x} \right)dx\). &  \\
  \hline
  7. & Anıq integraldı esaplań: \(\int_{0}^{1}{(3x^2 } + 1)dx\). &  \\
  \hline
  8. & Funkcional qatardıń jaqınlasıw oblastın tabıń: \(x + \frac{x^2 }{2^2 } + ... + \frac{x^{n}}{n^2 } + ...\) &  \\
  \hline
  9. & Differencial teńlemeniń ulıwma sheshimin tabıń: \(xy' - 2y = 0\). &  \\
  \hline
  10. & Korobkada 3 aq, 7 qara shar bar. Tosınnan úsh shar izbe-iz alındı. Izbe-iz alınǵan sharlardıń qara, qara, aq degen izbe-izlikte bolıw itimallıǵın tabıń. &  \\
  \hline
  \end{tabular}
  
  \vspace{1cm}
  
  \begin{tabular}{lll}
  Tuwrı juwaplar sanı: \underline{\hspace{1.5cm}} & 
  Bahası: \underline{\hspace{1.5cm}} & 
  Imtixan alıwshınıń qolı: \underline{\hspace{2cm}} \\
  \end{tabular}
  
  \egroup
  
  \newpage
  
  
  \textbf{61-variant}\\
  
  \bgroup
  \def\arraystretch{1.6} % 1 is the default, change whatever you need
  
  \begin{tabular}{|m{5.7cm}|m{9.5cm}|}
  \hline
  Shifr & \\
  \hline
  \end{tabular}
  
  \vspace{1cm}
  
  \begin{tabular}{|m{0.7cm}|m{10cm}|m{4cm}|}
  \hline
  № & Soraw & Juwap \\
  \hline
  1. & Eki ózgeriwshili funkciyanıń anıqlanıw oblastı qay jerde jaylasadı &  \\
  \hline
  2. & Eki ózgeriwshli funkciyanıń \(M(x_{0}, y_{0})\) noqattaǵı úzliksizliginiń anıqlaması &  \\
  \hline
  3. & Sızıqlı defferencial teńlemeniń uluwma sheshimin jazıń &  \\
  \hline
  4. & Shártli itimallıq formulasın jazıń &  \\
  \hline
  5. & Racional funkciyanı integrallań: \(\int{\frac{5}{(x - 3)(x + 2)}dx}\). &  \\
  \hline
  6. & Esaplań: \(\int_{1}^2 {e^{x}dx}\). &  \\
  \hline
  7. & Anıq integraldı esaplań: \(\int_{- \pi/4}^{0}\frac{dx}{cos^2 x}\). &  \\
  \hline
  8. & Funkcional qatardıń jıynaqlılıq oblastın jazıń: \(\ln x + ln^2 x + ... + ln^{n}x + ...\). &  \\
  \hline
  9. & Differencial teńlemeni sheshiń: \(y' + xy = 0\). &  \\
  \hline
  10. & Qutada 5 aq hám 15 qara shar bar. Tosınnan alınǵan bir shardıń aq bolıw itimallıǵın tabıń &  \\
  \hline
  \end{tabular}
  
  \vspace{1cm}
  
  \begin{tabular}{lll}
  Tuwrı juwaplar sanı: \underline{\hspace{1.5cm}} & 
  Bahası: \underline{\hspace{1.5cm}} & 
  Imtixan alıwshınıń qolı: \underline{\hspace{2cm}} \\
  \end{tabular}
  
  \egroup
  
  \newpage
  
  
  \textbf{62-variant}\\
  
  \bgroup
  \def\arraystretch{1.6} % 1 is the default, change whatever you need
  
  \begin{tabular}{|m{5.7cm}|m{9.5cm}|}
  \hline
  Shifr & \\
  \hline
  \end{tabular}
  
  \vspace{1cm}
  
  \begin{tabular}{|m{0.7cm}|m{10cm}|m{4cm}|}
  \hline
  № & Soraw & Juwap \\
  \hline
  1. & Eki ózgeriwshili funkciyanıń tolıq ósimi &  \\
  \hline
  2. & Esaplań \(d\left( \int{f(x)dx} \right) = ?\) &  \\
  \hline
  3. & Ózgeriwshileri ajıralǵan differenciallıq teńlemesiniń uluwma kórinisin jazıń &  \\
  \hline
  4. & Shekli additivlik aksiomasın jazıń &  \\
  \hline
  5. & Anıq emes integraldı esaplań: \(\int{e^{x}dx}\) . &  \\
  \hline
  6. & Integraldı esaplań: \(\int_{1}^{\infty}{\frac{1}{x^2 }dx}\). &  \\
  \hline
  7. & Anıq integraldı esaplań: \(\int_{0}^{1}{(3x^2 } + 1)dx\). &  \\
  \hline
  8. & Qatardıń jıyındısın esaplań: \(\sum_{n = 1}^{\infty}\frac{1}{(2n - 1)(2n + 1)}\). &  \\
  \hline
  9. & Differencial teńlemeni esaplań: \(yy' = 4\). &  \\
  \hline
  10. & Gruppadaǵı 20 studentten neshe túrli usıl menen 3 náwbetshini saylap alıwǵa boladı? &  \\
  \hline
  \end{tabular}
  
  \vspace{1cm}
  
  \begin{tabular}{lll}
  Tuwrı juwaplar sanı: \underline{\hspace{1.5cm}} & 
  Bahası: \underline{\hspace{1.5cm}} & 
  Imtixan alıwshınıń qolı: \underline{\hspace{2cm}} \\
  \end{tabular}
  
  \egroup
  
  \newpage
  
  
  \textbf{63-variant}\\
  
  \bgroup
  \def\arraystretch{1.6} % 1 is the default, change whatever you need
  
  \begin{tabular}{|m{5.7cm}|m{9.5cm}|}
  \hline
  Shifr & \\
  \hline
  \end{tabular}
  
  \vspace{1cm}
  
  \begin{tabular}{|m{0.7cm}|m{10cm}|m{4cm}|}
  \hline
  № & Soraw & Juwap \\
  \hline
  1. & Eki ózgeriwshili funkciyanıń ekinshi tártipli dara tuwındıları qalay belgilenedi &  \\
  \hline
  2. & Bóleklep inegrallaw formulasın jazıń &  \\
  \hline
  3. & Eger \(\sum_{n = 1}^{\infty}a_{n} = A,\ \sum_{n = 1}^{\infty}b_{n} = B\) bolsa, onda \(\sum_{n = 1}^{\infty}\left( a_{n} + b_{n} \right) = ?\) &  \\
  \hline
  4. & Isenimli waqıyanıń itimallıǵı nege teń &  \\
  \hline
  5. & Anıq emes integraldı esaplań: \(\int{\left( x^2  + \frac{1}{x} + \sin x \right)dx}\). &  \\
  \hline
  6. & Esaplań: \(\int\left( x^{4} - \frac{1}{x} \right)dx\). &  \\
  \hline
  7. & Anıq integraldı esaplań: \(\int_{2}^{4}\frac{dx}{x}\). &  \\
  \hline
  8. & Sanlı qatardıń baslanǵısh úsh aǵzasın jazıń: \(\sum_{n = 1}^{\infty}\frac{n!}{2^{n}}\). &  \\
  \hline
  9. & Sızıqlı differencial teńlemeniń ulwma sheshimin tabıń: \(y' + y = e^{x}\). &  \\
  \hline
  10. & Eki kubikti bir márte taslaǵanda túsken ochkolardıń qosındısı 4 bolıw itimallıǵın tabıń. &  \\
  \hline
  \end{tabular}
  
  \vspace{1cm}
  
  \begin{tabular}{lll}
  Tuwrı juwaplar sanı: \underline{\hspace{1.5cm}} & 
  Bahası: \underline{\hspace{1.5cm}} & 
  Imtixan alıwshınıń qolı: \underline{\hspace{2cm}} \\
  \end{tabular}
  
  \egroup
  
  \newpage
  
  
  \textbf{64-variant}\\
  
  \bgroup
  \def\arraystretch{1.6} % 1 is the default, change whatever you need
  
  \begin{tabular}{|m{5.7cm}|m{9.5cm}|}
  \hline
  Shifr & \\
  \hline
  \end{tabular}
  
  \vspace{1cm}
  
  \begin{tabular}{|m{0.7cm}|m{10cm}|m{4cm}|}
  \hline
  № & Soraw & Juwap \\
  \hline
  1. & Eki ózgeriwshili funkciyalar qalay belgilenedi &  \\
  \hline
  2. & Funkcianıń \((x_{0},\ y_{0})\) noqattaǵı úzliksizliginiń formulasın jazıń &  \\
  \hline
  3. & Bernulli differenciallıq teńemesin jazıń &  \\
  \hline
  4. & Gruppalaw formulasın jazıń &  \\
  \hline
  5. & Anıq emes integraldı esaplań: \(\int{e^{x}dx}\) . &  \\
  \hline
  6. & Anıq integraldı esaplań: \(\int_{1}^{3}\frac{2}{x + 1}dx\). &  \\
  \hline
  7. & Anıq integraldı esaplań: \(\int_{0}^{\frac{\pi}{2}}{\cos xdx}\). &  \\
  \hline
  8. & Funkcional qatardıń jaqınlasıw oblastın tabıń: \(x + \frac{x^2 }{2^2 } + ... + \frac{x^{n}}{n^2 } + ...\) &  \\
  \hline
  9. & Sızıqlı differerncial teńlemeniń uluwma sheshimin tabıń \(y' + y = e^{- x}\). &  \\
  \hline
  10. & Telefon nomerdiń aqırǵı cifrasın umıtıp, tosınnan nomerlerdi tere basladı. Kerekli nomerdi tabıw itimallıǵın esaplań. &  \\
  \hline
  \end{tabular}
  
  \vspace{1cm}
  
  \begin{tabular}{lll}
  Tuwrı juwaplar sanı: \underline{\hspace{1.5cm}} & 
  Bahası: \underline{\hspace{1.5cm}} & 
  Imtixan alıwshınıń qolı: \underline{\hspace{2cm}} \\
  \end{tabular}
  
  \egroup
  
  \newpage
  
  
  \textbf{65-variant}\\
  
  \bgroup
  \def\arraystretch{1.6} % 1 is the default, change whatever you need
  
  \begin{tabular}{|m{5.7cm}|m{9.5cm}|}
  \hline
  Shifr & \\
  \hline
  \end{tabular}
  
  \vspace{1cm}
  
  \begin{tabular}{|m{0.7cm}|m{10cm}|m{4cm}|}
  \hline
  № & Soraw & Juwap \\
  \hline
  1. & Eki ózgeriwshili funkciyanıń ekinshi tártipli aralas tuwındıları qalay belgilenedi &  \\
  \hline
  2. & Esaplań \(\left( \int{f(x)dx} \right)' = ?\). &  \\
  \hline
  3. & Oń aǵzalı qatarlar ushın jıynaqlılıqtıń Koshi belgisin jazıń &  \\
  \hline
  4. & Itimmallıqtıń geometriyalıq anıqlamasınıń formulasın jazıń &  \\
  \hline
  5. & Anıq emes integraldı esaplań: \(\int\frac{dx}{cos^2 x}\). &  \\
  \hline
  6. & Integraldı esaplań: \(\int_{1}^{\infty}{\frac{1}{(x + 2)^2 }dx}\). &  \\
  \hline
  7. & Anıq integraldı esaplań: \(\int_{0}^{\pi}{\sin xdx}\). &  \\
  \hline
  8. & Funkcional qatardıń jıynaqlılıq oblastın jazıń: \(\ln x + ln^2 x + ... + ln^{n}x + ...\). &  \\
  \hline
  9. & Differencial teńlemeniń ulıwma sheshimin tabıń: \(xy' - 2y = 0\). &  \\
  \hline
  10. & Dóngelektiń ishine kvadrat sızılǵan. Dóngelektiń ishinen tosınnan belgilengen noqattıń kvadrattıń ishinde jatıw itimallıǵın tabıń. &  \\
  \hline
  \end{tabular}
  
  \vspace{1cm}
  
  \begin{tabular}{lll}
  Tuwrı juwaplar sanı: \underline{\hspace{1.5cm}} & 
  Bahası: \underline{\hspace{1.5cm}} & 
  Imtixan alıwshınıń qolı: \underline{\hspace{2cm}} \\
  \end{tabular}
  
  \egroup
  
  \newpage
  
  
  \textbf{66-variant}\\
  
  \bgroup
  \def\arraystretch{1.6} % 1 is the default, change whatever you need
  
  \begin{tabular}{|m{5.7cm}|m{9.5cm}|}
  \hline
  Shifr & \\
  \hline
  \end{tabular}
  
  \vspace{1cm}
  
  \begin{tabular}{|m{0.7cm}|m{10cm}|m{4cm}|}
  \hline
  № & Soraw & Juwap \\
  \hline
  1. & Funkciyanıń anıqlanıw oblastı qalay belgilenedi &  \\
  \hline
  2. & Funkcianıń \((x_{0},\ y_{0})\) noqattaǵı tuwındısınıń formulasın jazıń &  \\
  \hline
  3. & Sanlı qatardıń uluwma kórinisin jazıń &  \\
  \hline
  4. & Shekli additivlik aksiomasın jazıń &  \\
  \hline
  5. & Racional funkciyanı integrallań: \(\int{\frac{3}{(x - 1)(x + 2)}dx}\). &  \\
  \hline
  6. & Integraldı esaplań: \(\int{2^{x}dx}\). &  \\
  \hline
  7. & Anıq integraldı esaplań: \(\int_{0}^{\frac{\pi}{2}}{\cos xdx}\). &  \\
  \hline
  8. & Funkcional qatardıń jıynaqlılıq oblastın tabıń:\(1 + x + ... + x^{n} + ...\) &  \\
  \hline
  9. & Differencial teńlemeniń ulıwma sheshimin tabıń: \(y' = e^{x}\). &  \\
  \hline
  10. & Úsh birdey korobkada aq hám qara sharlar bar. 1-korobkada 5 aq, 8 qara shar, 2-korobkada 3 aq, 4 qara shar, 3-korobkada 2 aq, 3 qara shar bar. Úsh korobkaniń birewinen tosınnan alınǵan bir shar aq bolıw itimallıǵın tabıń. &  \\
  \hline
  \end{tabular}
  
  \vspace{1cm}
  
  \begin{tabular}{lll}
  Tuwrı juwaplar sanı: \underline{\hspace{1.5cm}} & 
  Bahası: \underline{\hspace{1.5cm}} & 
  Imtixan alıwshınıń qolı: \underline{\hspace{2cm}} \\
  \end{tabular}
  
  \egroup
  
  \newpage
  
  
  \textbf{67-variant}\\
  
  \bgroup
  \def\arraystretch{1.6} % 1 is the default, change whatever you need
  
  \begin{tabular}{|m{5.7cm}|m{9.5cm}|}
  \hline
  Shifr & \\
  \hline
  \end{tabular}
  
  \vspace{1cm}
  
  \begin{tabular}{|m{0.7cm}|m{10cm}|m{4cm}|}
  \hline
  № & Soraw & Juwap \\
  \hline
  1. & Eki ózgeriwshili funkciyanıń birinshi tártipli dara tuwındıları qalay belgilenedi &  \\
  \hline
  2. & \((x_0,y_0)\) tochkanıń \(\varepsilon\) dógeregi qalay belgilenedi &  \\
  \hline
  3. & Eger \(\sum_{n = 1}^{\infty}a_{n} = A,\ \sum_{n = 1}^{\infty}b_{n} = B\) bolsa, onda \(\sum_{n = 1}^{\infty}\left( a_{n} - b_{n} \right) = ?\) &  \\
  \hline
  4. & Itimallıq keńisligin jazıń &  \\
  \hline
  5. & Integraldı esaplań: \(\int{\frac{1}{\sin x}dx}\). &  \\
  \hline
  6. & Integraldı esaplań: \(\int{2^{x}dx}\). &  \\
  \hline
  7. & Anıq integraldı esaplań: \(\int_{0}^{\pi}{\sin xdx}\). &  \\
  \hline
  8. & Qatardıń qosındısın tabıń: \(\sum_{n = 1}^{\infty}\frac{1}{n(n + 3)}\). &  \\
  \hline
  9. & Sızıqlı differerncial teńlemeniń uluwma sheshimin tabıń \(y' + y = e^{- x}\). &  \\
  \hline
  10. & Tiyindi eki márte taslaǵanda, keminde bir márte san tárepi túsiw itimallıǵın tabıń. &  \\
  \hline
  \end{tabular}
  
  \vspace{1cm}
  
  \begin{tabular}{lll}
  Tuwrı juwaplar sanı: \underline{\hspace{1.5cm}} & 
  Bahası: \underline{\hspace{1.5cm}} & 
  Imtixan alıwshınıń qolı: \underline{\hspace{2cm}} \\
  \end{tabular}
  
  \egroup
  
  \newpage
  
  
  \textbf{68-variant}\\
  
  \bgroup
  \def\arraystretch{1.6} % 1 is the default, change whatever you need
  
  \begin{tabular}{|m{5.7cm}|m{9.5cm}|}
  \hline
  Shifr & \\
  \hline
  \end{tabular}
  
  \vspace{1cm}
  
  \begin{tabular}{|m{0.7cm}|m{10cm}|m{4cm}|}
  \hline
  № & Soraw & Juwap \\
  \hline
  1. & Kóp aǵzalını \((x - a)\) ǵa bólgendegi qaldıq nege teń &  \\
  \hline
  2. & Esaplań \(d\left( \int{f(x)dx} \right) = ?\) &  \\
  \hline
  3. & Oń aǵzalı qatarlar ushın jıynaqlılıqtıń Dalamber belgisin jazıń &  \\
  \hline
  4. & Orın awıstırıw formulasın jazıń &  \\
  \hline
  5. & Racional funkciyanı integrallań: \(\int{\frac{5}{(x - 3)(x + 2)}dx}\). &  \\
  \hline
  6. & Integraldı esaplań: \(\int_{1}^{\infty}{\frac{1}{x^2 }dx}\). &  \\
  \hline
  7. & Anıq integraldı esaplań: \(\int_{2}^{4}\frac{dx}{x}\). &  \\
  \hline
  8. & Qatardıń jıyındısın esaplań: \(\sum_{n = 1}^{\infty}\frac{1}{(2n - 1)(2n + 1)}\). &  \\
  \hline
  9. & Differencial teńlemeniń ulıwma sheshimin tabıń: \(xy' - 2y = 0\). &  \\
  \hline
  10. & «MATEMATIKA» sóziniń háripleri bólek kartochkalarǵa jazılıp jawıp aralastırılıp qoyılǵan. Barlıq kartochkalar tosınnan izbe-iz alınıp ashılıp, alınıw tártibinde stol ústine dizilgende taǵı «MATEMATIKA» sóziniń kelip shıǵıw itimallıǵın tabıń. &  \\
  \hline
  \end{tabular}
  
  \vspace{1cm}
  
  \begin{tabular}{lll}
  Tuwrı juwaplar sanı: \underline{\hspace{1.5cm}} & 
  Bahası: \underline{\hspace{1.5cm}} & 
  Imtixan alıwshınıń qolı: \underline{\hspace{2cm}} \\
  \end{tabular}
  
  \egroup
  
  \newpage
  
  
  \textbf{69-variant}\\
  
  \bgroup
  \def\arraystretch{1.6} % 1 is the default, change whatever you need
  
  \begin{tabular}{|m{5.7cm}|m{9.5cm}|}
  \hline
  Shifr & \\
  \hline
  \end{tabular}
  
  \vspace{1cm}
  
  \begin{tabular}{|m{0.7cm}|m{10cm}|m{4cm}|}
  \hline
  № & Soraw & Juwap \\
  \hline
  1. & \(n\)-dárejeli kóp aǵzalınıń uluwma kórinisi &  \\
  \hline
  2. & Eki ózgeriwshli funkciyanıń \(M(x_{0}, y_{0})\) noqattaǵı úzliksizliginiń anıqlaması &  \\
  \hline
  3. & Funkcionallıq qatardıń uluwma kórinisi &  \\
  \hline
  4. & Itimallıqtıń klassikalıq anıqlamasınıń formulasın keltiriń &  \\
  \hline
  5. & Integraldı esaplań:\(\int{(x - 1)^{20}}dx\). &  \\
  \hline
  6. & Esaplań: \(\int_{1}^2 {e^{x}dx}\). &  \\
  \hline
  7. & Anıq integraldı esaplań: \(\int_{0}^{1}{(3x^2 } + 1)dx\). &  \\
  \hline
  8. & Qatardıń qosındısın tabıń: \(\sum_{n = 1}^{\infty}\frac{1}{n(n + 1)}\). &  \\
  \hline
  9. & Differencial teńlemeniń ulıwma sheshimin tabıń: \(y' = e^{x}\). &  \\
  \hline
  10. & «BIOLOGIYA» sóziniń háripleri bólek kartochkalarǵa jazılıp jawıp, aralastırılıp qoyılǵan. Barlıq kartochkalar tosınnan izbe-iz alınıp ashılıp, alınıw tártibinde stol ústine dizilgende taǵı «BIOLOGIYA» sóziniń kelip shıǵıw itimallıǵın tabıń. &  \\
  \hline
  \end{tabular}
  
  \vspace{1cm}
  
  \begin{tabular}{lll}
  Tuwrı juwaplar sanı: \underline{\hspace{1.5cm}} & 
  Bahası: \underline{\hspace{1.5cm}} & 
  Imtixan alıwshınıń qolı: \underline{\hspace{2cm}} \\
  \end{tabular}
  
  \egroup
  
  \newpage
  
  
  \textbf{70-variant}\\
  
  \bgroup
  \def\arraystretch{1.6} % 1 is the default, change whatever you need
  
  \begin{tabular}{|m{5.7cm}|m{9.5cm}|}
  \hline
  Shifr & \\
  \hline
  \end{tabular}
  
  \vspace{1cm}
  
  \begin{tabular}{|m{0.7cm}|m{10cm}|m{4cm}|}
  \hline
  № & Soraw & Juwap \\
  \hline
  1. & Funkciya qanday usıllarda beriledi &  \\
  \hline
  2. & Anıq integraldı esaplawdıń Nyuton-Leybnic formulasın jazıń &  \\
  \hline
  3. & Sızıqlı differenciallıq teńlemeniń uluwma kórinisin jazıń &  \\
  \hline
  4. & Bayes formulasın jazıń &  \\
  \hline
  5. & Integraldı esaplań: \(\int{(x + \sin x)dx}\). &  \\
  \hline
  6. & Anıq integraldı esaplań: \(\int_{1}^{3}\frac{2}{x + 1}dx\). &  \\
  \hline
  7. & Anıq integraldı esaplań: \(\int_{- \pi/4}^{0}\frac{dx}{cos^2 x}\). &  \\
  \hline
  8. & Sanlı qatardıń baslanǵısh úsh aǵzasın jazıń: \(\sum_{n = 1}^{\infty}\frac{n!}{2^{n}}\). &  \\
  \hline
  9. & Differencial teńlemeni esaplań: \(yy' = 4\). &  \\
  \hline
  10. & Telefon nomerdiń aqırǵı eki cifrasın umıtıp, tosınnan nomerlerdi tere basladı. Kerekli nomerdi tabıw itimallıǵın esaplań. &  \\
  \hline
  \end{tabular}
  
  \vspace{1cm}
  
  \begin{tabular}{lll}
  Tuwrı juwaplar sanı: \underline{\hspace{1.5cm}} & 
  Bahası: \underline{\hspace{1.5cm}} & 
  Imtixan alıwshınıń qolı: \underline{\hspace{2cm}} \\
  \end{tabular}
  
  \egroup
  
  \newpage
  
  
  \textbf{71-variant}\\
  
  \bgroup
  \def\arraystretch{1.6} % 1 is the default, change whatever you need
  
  \begin{tabular}{|m{5.7cm}|m{9.5cm}|}
  \hline
  Shifr & \\
  \hline
  \end{tabular}
  
  \vspace{1cm}
  
  \begin{tabular}{|m{0.7cm}|m{10cm}|m{4cm}|}
  \hline
  № & Soraw & Juwap \\
  \hline
  1. & Eki ózgeriwshili funkciyanıń ekstremumınıń zárúrli shárti &  \\
  \hline
  2. & Bóleklep inegrallaw formulasın jazıń &  \\
  \hline
  3. & Eger \(\sum_{n = 1}^{\infty}a_{n} = A,\ \sum_{n = 1}^{\infty}b_{n} = B\) bolsa, onda \(\sum_{n = 1}^{\infty}\left( a_{n} + b_{n} \right) = ?\) &  \\
  \hline
  4. & Isenimli waqıyanıń itimallıǵı nege teń &  \\
  \hline
  5. & Anıq emes integraldı esaplań: \(\int{\left( x^2  + \frac{1}{x} + \sin x \right)dx}\). &  \\
  \hline
  6. & Integraldı esaplań: \(\int_{1}^{\infty}{\frac{1}{(x + 2)^2 }dx}\). &  \\
  \hline
  7. & Anıq integraldı esaplań: \(\int_{2}^{4}\frac{dx}{x}\). &  \\
  \hline
  8. & Qatardıń jıyındısın esaplań: \(\sum_{n = 1}^{\infty}\frac{1}{(2n - 1)(2n + 1)}\). &  \\
  \hline
  9. & Sızıqlı differencial teńlemeniń ulwma sheshimin tabıń: \(y' + y = e^{x}\). &  \\
  \hline
  10. & Ídısta 5 aq, 8 qara shar bar. Ídıstan tosınnan izbe-iz 3 shar alındı. Alınǵan sharlar aq, qara, qara degen izbe-izlikte bolıw itimallıǵın tabıń. &  \\
  \hline
  \end{tabular}
  
  \vspace{1cm}
  
  \begin{tabular}{lll}
  Tuwrı juwaplar sanı: \underline{\hspace{1.5cm}} & 
  Bahası: \underline{\hspace{1.5cm}} & 
  Imtixan alıwshınıń qolı: \underline{\hspace{2cm}} \\
  \end{tabular}
  
  \egroup
  
  \newpage
  
  
  \textbf{72-variant}\\
  
  \bgroup
  \def\arraystretch{1.6} % 1 is the default, change whatever you need
  
  \begin{tabular}{|m{5.7cm}|m{9.5cm}|}
  \hline
  Shifr & \\
  \hline
  \end{tabular}
  
  \vspace{1cm}
  
  \begin{tabular}{|m{0.7cm}|m{10cm}|m{4cm}|}
  \hline
  № & Soraw & Juwap \\
  \hline
  1. & Eki ózgeriwshili funkciyanıń grafigi neden ibarat &  \\
  \hline
  2. & Ózgeriwshini almastırıp integrallaw usılıniń formulasın jazıń. &  \\
  \hline
  3. & Funkcionallıq qatardıń uluwma kórinisi &  \\
  \hline
  4. & Tolıq itimallıqtıń formulasın jazıń &  \\
  \hline
  5. & Anıq emes integraldı esaplań: \(\int{\left( 10x^{4} + 7x^{6} - 3 \right)dx}\). &  \\
  \hline
  6. & Esaplań: \(\int\left( x^{4} - \frac{1}{x} \right)dx\). &  \\
  \hline
  7. & Anıq integraldı esaplań: \(\int_{- \pi/4}^{0}\frac{dx}{cos^2 x}\). &  \\
  \hline
  8. & Funkcional qatardıń jıynaqlılıq oblastın jazıń: \(\ln x + ln^2 x + ... + ln^{n}x + ...\). &  \\
  \hline
  9. & Differencial teńlemeni sheshiń: \(y' + xy = 0\). &  \\
  \hline
  10. & Korobkada 3 aq, 7 qara shar bar. Tosınnan úsh shar izbe-iz alındı. Izbe-iz alınǵan sharlardıń qara, qara, aq degen izbe-izlikte bolıw itimallıǵın tabıń. &  \\
  \hline
  \end{tabular}
  
  \vspace{1cm}
  
  \begin{tabular}{lll}
  Tuwrı juwaplar sanı: \underline{\hspace{1.5cm}} & 
  Bahası: \underline{\hspace{1.5cm}} & 
  Imtixan alıwshınıń qolı: \underline{\hspace{2cm}} \\
  \end{tabular}
  
  \egroup
  
  \newpage
  
  
  \textbf{73-variant}\\
  
  \bgroup
  \def\arraystretch{1.6} % 1 is the default, change whatever you need
  
  \begin{tabular}{|m{5.7cm}|m{9.5cm}|}
  \hline
  Shifr & \\
  \hline
  \end{tabular}
  
  \vspace{1cm}
  
  \begin{tabular}{|m{0.7cm}|m{10cm}|m{4cm}|}
  \hline
  № & Soraw & Juwap \\
  \hline
  1. & Eki ózgeriwshili funkciyanıń ekinshi tártipli dara tuwındıları qalay belgilenedi &  \\
  \hline
  2. & Eki ózgeriwshli funkciyanıń \(M(x_{0}, y_{0})\) noqattaǵı úzliksizliginiń anıqlaması &  \\
  \hline
  3. & Eger \(\sum_{n = 1}^{\infty}a_{n} = A,\ \sum_{n = 1}^{\infty}b_{n} = B\) bolsa, onda \(\sum_{n = 1}^{\infty}\left( a_{n} - b_{n} \right) = ?\) &  \\
  \hline
  4. & Itimmallıqtıń geometriyalıq anıqlamasınıń formulasın jazıń &  \\
  \hline
  5. & Anıq emes integraldı esaplań: \(\int{e^{x}dx}\) . &  \\
  \hline
  6. & Integraldı esaplań: \(\int_{1}^{\infty}{\frac{1}{x^2 }dx}\). &  \\
  \hline
  7. & Anıq integraldı esaplań: \(\int_{0}^{\frac{\pi}{2}}{\cos xdx}\). &  \\
  \hline
  8. & Funkcional qatardıń jaqınlasıw oblastın tabıń: \(x + \frac{x^2 }{2^2 } + ... + \frac{x^{n}}{n^2 } + ...\) &  \\
  \hline
  9. & Differencial teńlemeni sheshiń: \(y' + xy = 0\). &  \\
  \hline
  10. & Telefon nomerdiń aqırǵı cifrasın umıtıp, tosınnan nomerlerdi tere basladı. Kerekli nomerdi tabıw itimallıǵın esaplań. &  \\
  \hline
  \end{tabular}
  
  \vspace{1cm}
  
  \begin{tabular}{lll}
  Tuwrı juwaplar sanı: \underline{\hspace{1.5cm}} & 
  Bahası: \underline{\hspace{1.5cm}} & 
  Imtixan alıwshınıń qolı: \underline{\hspace{2cm}} \\
  \end{tabular}
  
  \egroup
  
  \newpage
  
  
  \textbf{74-variant}\\
  
  \bgroup
  \def\arraystretch{1.6} % 1 is the default, change whatever you need
  
  \begin{tabular}{|m{5.7cm}|m{9.5cm}|}
  \hline
  Shifr & \\
  \hline
  \end{tabular}
  
  \vspace{1cm}
  
  \begin{tabular}{|m{0.7cm}|m{10cm}|m{4cm}|}
  \hline
  № & Soraw & Juwap \\
  \hline
  1. & Eki ózgeriwshili funkciyanıń birinshi tártipli dara tuwındıları qalay belgilenedi &  \\
  \hline
  2. & Esaplań \(d\left( \int{f(x)dx} \right) = ?\) &  \\
  \hline
  3. & Bernulli differenciallıq teńemesin jazıń &  \\
  \hline
  4. & Gruppalaw formulasın jazıń &  \\
  \hline
  5. & Racional funkciyanı integrallań: \(\int{\frac{5}{(x - 3)(x + 2)}dx}\). &  \\
  \hline
  6. & Esaplań: \(\int_{1}^2 {e^{x}dx}\). &  \\
  \hline
  7. & Anıq integraldı esaplań: \(\int_{0}^{1}{(3x^2 } + 1)dx\). &  \\
  \hline
  8. & Sanlı qatardıń baslanǵısh úsh aǵzasın jazıń: \(\sum_{n = 1}^{\infty}\frac{n!}{2^{n}}\). &  \\
  \hline
  9. & Sızıqlı differencial teńlemeniń ulwma sheshimin tabıń: \(y' + y = e^{x}\). &  \\
  \hline
  10. & Tiyindi eki márte taslaǵanda, keminde bir márte san tárepi túsiw itimallıǵın tabıń. &  \\
  \hline
  \end{tabular}
  
  \vspace{1cm}
  
  \begin{tabular}{lll}
  Tuwrı juwaplar sanı: \underline{\hspace{1.5cm}} & 
  Bahası: \underline{\hspace{1.5cm}} & 
  Imtixan alıwshınıń qolı: \underline{\hspace{2cm}} \\
  \end{tabular}
  
  \egroup
  
  \newpage
  
  
  \textbf{75-variant}\\
  
  \bgroup
  \def\arraystretch{1.6} % 1 is the default, change whatever you need
  
  \begin{tabular}{|m{5.7cm}|m{9.5cm}|}
  \hline
  Shifr & \\
  \hline
  \end{tabular}
  
  \vspace{1cm}
  
  \begin{tabular}{|m{0.7cm}|m{10cm}|m{4cm}|}
  \hline
  № & Soraw & Juwap \\
  \hline
  1. & Eki ózgeriwshili funkciyanıń anıqlanıw oblastı qay jerde jaylasadı &  \\
  \hline
  2. & Anıq integraldı esaplawdıń Nyuton-Leybnic formulasın jazıń &  \\
  \hline
  3. & Sızıqlı defferencial teńlemeniń uluwma sheshimin jazıń &  \\
  \hline
  4. & Shártli itimallıq formulasın jazıń &  \\
  \hline
  5. & Anıq emes integraldı esaplań: \(\int{\left( 10x^{4} + 7x^{6} - 3 \right)dx}\). &  \\
  \hline
  6. & Anıq integraldı esaplań: \(\int_{1}^{3}\frac{2}{x + 1}dx\). &  \\
  \hline
  7. & Anıq integraldı esaplań: \(\int_{0}^{\pi}{\sin xdx}\). &  \\
  \hline
  8. & Qatardıń qosındısın tabıń: \(\sum_{n = 1}^{\infty}\frac{1}{n(n + 1)}\). &  \\
  \hline
  9. & Differencial teńlemeniń ulıwma sheshimin tabıń: \(xy' - 2y = 0\). &  \\
  \hline
  10. & Eki kubikti bir márte taslaǵanda túsken ochkolardıń qosındısı 4 bolıw itimallıǵın tabıń. &  \\
  \hline
  \end{tabular}
  
  \vspace{1cm}
  
  \begin{tabular}{lll}
  Tuwrı juwaplar sanı: \underline{\hspace{1.5cm}} & 
  Bahası: \underline{\hspace{1.5cm}} & 
  Imtixan alıwshınıń qolı: \underline{\hspace{2cm}} \\
  \end{tabular}
  
  \egroup
  
  \newpage
  
  
  \textbf{76-variant}\\
  
  \bgroup
  \def\arraystretch{1.6} % 1 is the default, change whatever you need
  
  \begin{tabular}{|m{5.7cm}|m{9.5cm}|}
  \hline
  Shifr & \\
  \hline
  \end{tabular}
  
  \vspace{1cm}
  
  \begin{tabular}{|m{0.7cm}|m{10cm}|m{4cm}|}
  \hline
  № & Soraw & Juwap \\
  \hline
  1. & Kóp aǵzalını \((x - a)\) ǵa bólgendegi qaldıq nege teń &  \\
  \hline
  2. & Esaplań \(\left( \int{f(x)dx} \right)' = ?\). &  \\
  \hline
  3. & Sızıqlı differenciallıq teńlemeniń uluwma kórinisin jazıń &  \\
  \hline
  4. & Múmkin emes waqıyanıń itimaıllıǵı nege teń &  \\
  \hline
  5. & Anıq emes integraldı esaplań: \(\int\frac{dx}{cos^2 x}\). &  \\
  \hline
  6. & Integraldı esaplań: \(\int_{1}^{\infty}{\frac{1}{(x + 2)^2 }dx}\). &  \\
  \hline
  7. & Anıq integraldı esaplań: \(\int_{0}^{\pi}{\sin xdx}\). &  \\
  \hline
  8. & Qatardıń qosındısın tabıń: \(\sum_{n = 1}^{\infty}\frac{1}{n(n + 3)}\). &  \\
  \hline
  9. & Sızıqlı differerncial teńlemeniń uluwma sheshimin tabıń \(y' + y = e^{- x}\). &  \\
  \hline
  10. & Gruppadaǵı 20 studentten neshe túrli usıl menen 3 náwbetshini saylap alıwǵa boladı? &  \\
  \hline
  \end{tabular}
  
  \vspace{1cm}
  
  \begin{tabular}{lll}
  Tuwrı juwaplar sanı: \underline{\hspace{1.5cm}} & 
  Bahası: \underline{\hspace{1.5cm}} & 
  Imtixan alıwshınıń qolı: \underline{\hspace{2cm}} \\
  \end{tabular}
  
  \egroup
  
  \newpage
  
  
  \textbf{77-variant}\\
  
  \bgroup
  \def\arraystretch{1.6} % 1 is the default, change whatever you need
  
  \begin{tabular}{|m{5.7cm}|m{9.5cm}|}
  \hline
  Shifr & \\
  \hline
  \end{tabular}
  
  \vspace{1cm}
  
  \begin{tabular}{|m{0.7cm}|m{10cm}|m{4cm}|}
  \hline
  № & Soraw & Juwap \\
  \hline
  1. & Eki ózgeriwshili funkciyanıń ekstremumınıń zárúrli shárti &  \\
  \hline
  2. & Bóleklep inegrallaw formulasın jazıń &  \\
  \hline
  3. & Sanlı qatardıń uluwma kórinisin jazıń &  \\
  \hline
  4. & Itimallıqtıń mánisler oblastın jazıń &  \\
  \hline
  5. & Anıq emes integraldı esaplań: \(\int{\left( x^2  + \frac{1}{x} + \sin x \right)dx}\). &  \\
  \hline
  6. & Esaplań: \(\int\left( x^{4} - \frac{1}{x} \right)dx\). &  \\
  \hline
  7. & Anıq integraldı esaplań: \(\int_{- \pi/4}^{0}\frac{dx}{cos^2 x}\). &  \\
  \hline
  8. & Funkcional qatardıń jıynaqlılıq oblastın tabıń:\(1 + x + ... + x^{n} + ...\) &  \\
  \hline
  9. & Differencial teńlemeniń ulıwma sheshimin tabıń: \(y' = e^{x}\). &  \\
  \hline
  10. & «MATEMATIKA» sóziniń háripleri bólek kartochkalarǵa jazılıp jawıp aralastırılıp qoyılǵan. Barlıq kartochkalar tosınnan izbe-iz alınıp ashılıp, alınıw tártibinde stol ústine dizilgende taǵı «MATEMATIKA» sóziniń kelip shıǵıw itimallıǵın tabıń. &  \\
  \hline
  \end{tabular}
  
  \vspace{1cm}
  
  \begin{tabular}{lll}
  Tuwrı juwaplar sanı: \underline{\hspace{1.5cm}} & 
  Bahası: \underline{\hspace{1.5cm}} & 
  Imtixan alıwshınıń qolı: \underline{\hspace{2cm}} \\
  \end{tabular}
  
  \egroup
  
  \newpage
  
  
  \textbf{78-variant}\\
  
  \bgroup
  \def\arraystretch{1.6} % 1 is the default, change whatever you need
  
  \begin{tabular}{|m{5.7cm}|m{9.5cm}|}
  \hline
  Shifr & \\
  \hline
  \end{tabular}
  
  \vspace{1cm}
  
  \begin{tabular}{|m{0.7cm}|m{10cm}|m{4cm}|}
  \hline
  № & Soraw & Juwap \\
  \hline
  1. & Eki ózgeriwshili funkciyanıń ekinshi tártipli aralas tuwındıları qalay belgilenedi &  \\
  \hline
  2. & Funkcianıń \((x_{0},\ y_{0})\) noqattaǵı tuwındısınıń formulasın jazıń &  \\
  \hline
  3. & Oń aǵzalı qatarlar ushın jıynaqlılıqtıń Dalamber belgisin jazıń &  \\
  \hline
  4. & Orın almastırıw formulasın jazıń &  \\
  \hline
  5. & Integraldı esaplań: \(\int{\frac{1}{\sin x}dx}\). &  \\
  \hline
  6. & Integraldı esaplań: \(\int{2^{x}dx}\). &  \\
  \hline
  7. & Anıq integraldı esaplań: \(\int_{0}^{1}{(3x^2 } + 1)dx\). &  \\
  \hline
  8. & Qatardıń qosındısın tabıń: \(\sum_{n = 1}^{\infty}\frac{1}{n(n + 3)}\). &  \\
  \hline
  9. & Differencial teńlemeni esaplań: \(yy' = 4\). &  \\
  \hline
  10. & Dóngelektiń ishine kvadrat sızılǵan. Dóngelektiń ishinen tosınnan belgilengen noqattıń kvadrattıń ishinde jatıw itimallıǵın tabıń. &  \\
  \hline
  \end{tabular}
  
  \vspace{1cm}
  
  \begin{tabular}{lll}
  Tuwrı juwaplar sanı: \underline{\hspace{1.5cm}} & 
  Bahası: \underline{\hspace{1.5cm}} & 
  Imtixan alıwshınıń qolı: \underline{\hspace{2cm}} \\
  \end{tabular}
  
  \egroup
  
  \newpage
  
  
  \textbf{79-variant}\\
  
  \bgroup
  \def\arraystretch{1.6} % 1 is the default, change whatever you need
  
  \begin{tabular}{|m{5.7cm}|m{9.5cm}|}
  \hline
  Shifr & \\
  \hline
  \end{tabular}
  
  \vspace{1cm}
  
  \begin{tabular}{|m{0.7cm}|m{10cm}|m{4cm}|}
  \hline
  № & Soraw & Juwap \\
  \hline
  1. & Eki ózgeriwshili funkciyanıń tolıq ósimi &  \\
  \hline
  2. & Funkcianıń \((x_{0},\ y_{0})\) noqattaǵı úzliksizliginiń formulasın jazıń &  \\
  \hline
  3. & Oń aǵzalı qatarlar ushın jıynaqlılıqtıń Koshi belgisin jazıń &  \\
  \hline
  4. & Shekli additivlik aksiomasın jazıń &  \\
  \hline
  5. & Integraldı esaplań:\(\int{(x - 1)^{20}}dx\). &  \\
  \hline
  6. & Integraldı esaplań: \(\int_{1}^{\infty}{\frac{1}{x^2 }dx}\). &  \\
  \hline
  7. & Anıq integraldı esaplań: \(\int_{0}^{\frac{\pi}{2}}{\cos xdx}\). &  \\
  \hline
  8. & Sanlı qatardıń baslanǵısh úsh aǵzasın jazıń: \(\sum_{n = 1}^{\infty}\frac{n!}{2^{n}}\). &  \\
  \hline
  9. & Differencial teńlemeni esaplań: \(yy' = 4\). &  \\
  \hline
  10. & Telefon nomerdiń aqırǵı eki cifrasın umıtıp, tosınnan nomerlerdi tere basladı. Kerekli nomerdi tabıw itimallıǵın esaplań. &  \\
  \hline
  \end{tabular}
  
  \vspace{1cm}
  
  \begin{tabular}{lll}
  Tuwrı juwaplar sanı: \underline{\hspace{1.5cm}} & 
  Bahası: \underline{\hspace{1.5cm}} & 
  Imtixan alıwshınıń qolı: \underline{\hspace{2cm}} \\
  \end{tabular}
  
  \egroup
  
  \newpage
  
  
  \textbf{80-variant}\\
  
  \bgroup
  \def\arraystretch{1.6} % 1 is the default, change whatever you need
  
  \begin{tabular}{|m{5.7cm}|m{9.5cm}|}
  \hline
  Shifr & \\
  \hline
  \end{tabular}
  
  \vspace{1cm}
  
  \begin{tabular}{|m{0.7cm}|m{10cm}|m{4cm}|}
  \hline
  № & Soraw & Juwap \\
  \hline
  1. & Eki ózgeriwshili funkciyanıń grafigi neden ibarat &  \\
  \hline
  2. & \((x_0,y_0)\) tochkanıń \(\varepsilon\) dógeregi qalay belgilenedi &  \\
  \hline
  3. & Ózgeriwshileri ajıralǵan differenciallıq teńlemesiniń uluwma kórinisin jazıń &  \\
  \hline
  4. & Tolıq itimallıqtıń formulasın jazıń &  \\
  \hline
  5. & Integraldı esaplań: \(\int{(x + \sin x)dx}\). &  \\
  \hline
  6. & Integraldı esaplań: \(\int_{1}^{\infty}{\frac{1}{(x + 2)^2 }dx}\). &  \\
  \hline
  7. & Anıq integraldı esaplań: \(\int_{2}^{4}\frac{dx}{x}\). &  \\
  \hline
  8. & Funkcional qatardıń jıynaqlılıq oblastın jazıń: \(\ln x + ln^2 x + ... + ln^{n}x + ...\). &  \\
  \hline
  9. & Differencial teńlemeniń ulıwma sheshimin tabıń: \(y' = e^{x}\). &  \\
  \hline
  10. & Korobkada 15 aq, 18 qara shar bar. Tosınnan alınǵan bir shar aq bolıw itimallıǵın tabıń. &  \\
  \hline
  \end{tabular}
  
  \vspace{1cm}
  
  \begin{tabular}{lll}
  Tuwrı juwaplar sanı: \underline{\hspace{1.5cm}} & 
  Bahası: \underline{\hspace{1.5cm}} & 
  Imtixan alıwshınıń qolı: \underline{\hspace{2cm}} \\
  \end{tabular}
  
  \egroup
  
  \newpage
  
  
  \textbf{81-variant}\\
  
  \bgroup
  \def\arraystretch{1.6} % 1 is the default, change whatever you need
  
  \begin{tabular}{|m{5.7cm}|m{9.5cm}|}
  \hline
  Shifr & \\
  \hline
  \end{tabular}
  
  \vspace{1cm}
  
  \begin{tabular}{|m{0.7cm}|m{10cm}|m{4cm}|}
  \hline
  № & Soraw & Juwap \\
  \hline
  1. & Funkciyanıń anıqlanıw oblastı qalay belgilenedi &  \\
  \hline
  2. & Ózgeriwshini almastırıp integrallaw usılıniń formulasın jazıń. &  \\
  \hline
  3. & Bernulli differenciallıq teńemesin jazıń &  \\
  \hline
  4. & Múmkin emes waqıyanıń itimaıllıǵı nege teń &  \\
  \hline
  5. & Racional funkciyanı integrallań: \(\int{\frac{3}{(x - 1)(x + 2)}dx}\). &  \\
  \hline
  6. & Esaplań: \(\int_{1}^2 {e^{x}dx}\). &  \\
  \hline
  7. & Anıq integraldı esaplań: \(\int_{2}^{4}\frac{dx}{x}\). &  \\
  \hline
  8. & Qatardıń qosındısın tabıń: \(\sum_{n = 1}^{\infty}\frac{1}{n(n + 1)}\). &  \\
  \hline
  9. & Sızıqlı differerncial teńlemeniń uluwma sheshimin tabıń \(y' + y = e^{- x}\). &  \\
  \hline
  10. & Úsh birdey korobkada aq hám qara sharlar bar. 1-korobkada 5 aq, 8 qara shar, 2-korobkada 3 aq, 4 qara shar, 3-korobkada 2 aq, 3 qara shar bar. Úsh korobkaniń birewinen tosınnan alınǵan bir shar aq bolıw itimallıǵın tabıń. &  \\
  \hline
  \end{tabular}
  
  \vspace{1cm}
  
  \begin{tabular}{lll}
  Tuwrı juwaplar sanı: \underline{\hspace{1.5cm}} & 
  Bahası: \underline{\hspace{1.5cm}} & 
  Imtixan alıwshınıń qolı: \underline{\hspace{2cm}} \\
  \end{tabular}
  
  \egroup
  
  \newpage
  
  
  \textbf{82-variant}\\
  
  \bgroup
  \def\arraystretch{1.6} % 1 is the default, change whatever you need
  
  \begin{tabular}{|m{5.7cm}|m{9.5cm}|}
  \hline
  Shifr & \\
  \hline
  \end{tabular}
  
  \vspace{1cm}
  
  \begin{tabular}{|m{0.7cm}|m{10cm}|m{4cm}|}
  \hline
  № & Soraw & Juwap \\
  \hline
  1. & Eki ózgeriwshili funkciyalar qalay belgilenedi &  \\
  \hline
  2. & Anıq integraldı esaplawdıń Nyuton-Leybnic formulasın jazıń &  \\
  \hline
  3. & Sanlı qatardıń uluwma kórinisin jazıń &  \\
  \hline
  4. & Itimallıqtıń klassikalıq anıqlamasınıń formulasın keltiriń &  \\
  \hline
  5. & Anıq emes integraldı esaplań: \(\int{\left( x^2  + \frac{1}{x} + \sin x \right)dx}\). &  \\
  \hline
  6. & Anıq integraldı esaplań: \(\int_{1}^{3}\frac{2}{x + 1}dx\). &  \\
  \hline
  7. & Anıq integraldı esaplań: \(\int_{- \pi/4}^{0}\frac{dx}{cos^2 x}\). &  \\
  \hline
  8. & Funkcional qatardıń jıynaqlılıq oblastın tabıń:\(1 + x + ... + x^{n} + ...\) &  \\
  \hline
  9. & Sızıqlı differencial teńlemeniń ulwma sheshimin tabıń: \(y' + y = e^{x}\). &  \\
  \hline
  10. & Qutada 5 aq hám 15 qara shar bar. Tosınnan alınǵan bir shardıń aq bolıw itimallıǵın tabıń &  \\
  \hline
  \end{tabular}
  
  \vspace{1cm}
  
  \begin{tabular}{lll}
  Tuwrı juwaplar sanı: \underline{\hspace{1.5cm}} & 
  Bahası: \underline{\hspace{1.5cm}} & 
  Imtixan alıwshınıń qolı: \underline{\hspace{2cm}} \\
  \end{tabular}
  
  \egroup
  
  \newpage
  
  
  \textbf{83-variant}\\
  
  \bgroup
  \def\arraystretch{1.6} % 1 is the default, change whatever you need
  
  \begin{tabular}{|m{5.7cm}|m{9.5cm}|}
  \hline
  Shifr & \\
  \hline
  \end{tabular}
  
  \vspace{1cm}
  
  \begin{tabular}{|m{0.7cm}|m{10cm}|m{4cm}|}
  \hline
  № & Soraw & Juwap \\
  \hline
  1. & Funkciya qanday usıllarda beriledi &  \\
  \hline
  2. & \((x_0,y_0)\) tochkanıń \(\varepsilon\) dógeregi qalay belgilenedi &  \\
  \hline
  3. & Funkcionallıq qatardıń uluwma kórinisi &  \\
  \hline
  4. & Itimallıq keńisligin jazıń &  \\
  \hline
  5. & Anıq emes integraldı esaplań: \(\int{\left( 10x^{4} + 7x^{6} - 3 \right)dx}\). &  \\
  \hline
  6. & Esaplań: \(\int\left( x^{4} - \frac{1}{x} \right)dx\). &  \\
  \hline
  7. & Anıq integraldı esaplań: \(\int_{0}^{\frac{\pi}{2}}{\cos xdx}\). &  \\
  \hline
  8. & Qatardıń jıyındısın esaplań: \(\sum_{n = 1}^{\infty}\frac{1}{(2n - 1)(2n + 1)}\). &  \\
  \hline
  9. & Differencial teńlemeniń ulıwma sheshimin tabıń: \(xy' - 2y = 0\). &  \\
  \hline
  10. & 50 buyımnan ibarat partiyada 3 buyım jaramsız. Tosınnan alınǵan 8 buyımnıń ishinde 1 buyımı jaramsız bolıw itimallıǵın tabıń &  \\
  \hline
  \end{tabular}
  
  \vspace{1cm}
  
  \begin{tabular}{lll}
  Tuwrı juwaplar sanı: \underline{\hspace{1.5cm}} & 
  Bahası: \underline{\hspace{1.5cm}} & 
  Imtixan alıwshınıń qolı: \underline{\hspace{2cm}} \\
  \end{tabular}
  
  \egroup
  
  \newpage
  
  
  \textbf{84-variant}\\
  
  \bgroup
  \def\arraystretch{1.6} % 1 is the default, change whatever you need
  
  \begin{tabular}{|m{5.7cm}|m{9.5cm}|}
  \hline
  Shifr & \\
  \hline
  \end{tabular}
  
  \vspace{1cm}
  
  \begin{tabular}{|m{0.7cm}|m{10cm}|m{4cm}|}
  \hline
  № & Soraw & Juwap \\
  \hline
  1. & \(n\)-dárejeli kóp aǵzalınıń uluwma kórinisi &  \\
  \hline
  2. & Esaplań \(\left( \int{f(x)dx} \right)' = ?\). &  \\
  \hline
  3. & Eger \(\sum_{n = 1}^{\infty}a_{n} = A,\ \sum_{n = 1}^{\infty}b_{n} = B\) bolsa, onda \(\sum_{n = 1}^{\infty}\left( a_{n} + b_{n} \right) = ?\) &  \\
  \hline
  4. & Bayes formulasın jazıń &  \\
  \hline
  5. & Racional funkciyanı integrallań: \(\int{\frac{3}{(x - 1)(x + 2)}dx}\). &  \\
  \hline
  6. & Integraldı esaplań: \(\int{2^{x}dx}\). &  \\
  \hline
  7. & Anıq integraldı esaplań: \(\int_{0}^{\pi}{\sin xdx}\). &  \\
  \hline
  8. & Funkcional qatardıń jaqınlasıw oblastın tabıń: \(x + \frac{x^2 }{2^2 } + ... + \frac{x^{n}}{n^2 } + ...\) &  \\
  \hline
  9. & Differencial teńlemeni sheshiń: \(y' + xy = 0\). &  \\
  \hline
  10. & «BIOLOGIYA» sóziniń háripleri bólek kartochkalarǵa jazılıp jawıp, aralastırılıp qoyılǵan. Barlıq kartochkalar tosınnan izbe-iz alınıp ashılıp, alınıw tártibinde stol ústine dizilgende taǵı «BIOLOGIYA» sóziniń kelip shıǵıw itimallıǵın tabıń. &  \\
  \hline
  \end{tabular}
  
  \vspace{1cm}
  
  \begin{tabular}{lll}
  Tuwrı juwaplar sanı: \underline{\hspace{1.5cm}} & 
  Bahası: \underline{\hspace{1.5cm}} & 
  Imtixan alıwshınıń qolı: \underline{\hspace{2cm}} \\
  \end{tabular}
  
  \egroup
  
  \newpage
  
  
  \textbf{85-variant}\\
  
  \bgroup
  \def\arraystretch{1.6} % 1 is the default, change whatever you need
  
  \begin{tabular}{|m{5.7cm}|m{9.5cm}|}
  \hline
  Shifr & \\
  \hline
  \end{tabular}
  
  \vspace{1cm}
  
  \begin{tabular}{|m{0.7cm}|m{10cm}|m{4cm}|}
  \hline
  № & Soraw & Juwap \\
  \hline
  1. & Eki ózgeriwshili funkciyanıń tolıq ósimi &  \\
  \hline
  2. & Funkcianıń \((x_{0},\ y_{0})\) noqattaǵı úzliksizliginiń formulasın jazıń &  \\
  \hline
  3. & Eger \(\sum_{n = 1}^{\infty}a_{n} = A,\ \sum_{n = 1}^{\infty}b_{n} = B\) bolsa, onda \(\sum_{n = 1}^{\infty}\left( a_{n} - b_{n} \right) = ?\) &  \\
  \hline
  4. & Itimallıqtıń mánisler oblastın jazıń &  \\
  \hline
  5. & Racional funkciyanı integrallań: \(\int{\frac{5}{(x - 3)(x + 2)}dx}\). &  \\
  \hline
  6. & Esaplań: \(\int\left( x^{4} - \frac{1}{x} \right)dx\). &  \\
  \hline
  7. & Anıq integraldı esaplań: \(\int_{0}^{1}{(3x^2 } + 1)dx\). &  \\
  \hline
  8. & Qatardıń qosındısın tabıń: \(\sum_{n = 1}^{\infty}\frac{1}{n(n + 3)}\). &  \\
  \hline
  9. & Sızıqlı differerncial teńlemeniń uluwma sheshimin tabıń \(y' + y = e^{- x}\). &  \\
  \hline
  10. & Ídısta 5 aq, 8 qara shar bar. Ídıstan tosınnan izbe-iz 3 shar alındı. Alınǵan sharlar aq, qara, qara degen izbe-izlikte bolıw itimallıǵın tabıń. &  \\
  \hline
  \end{tabular}
  
  \vspace{1cm}
  
  \begin{tabular}{lll}
  Tuwrı juwaplar sanı: \underline{\hspace{1.5cm}} & 
  Bahası: \underline{\hspace{1.5cm}} & 
  Imtixan alıwshınıń qolı: \underline{\hspace{2cm}} \\
  \end{tabular}
  
  \egroup
  
  \newpage
  
  
  \textbf{86-variant}\\
  
  \bgroup
  \def\arraystretch{1.6} % 1 is the default, change whatever you need
  
  \begin{tabular}{|m{5.7cm}|m{9.5cm}|}
  \hline
  Shifr & \\
  \hline
  \end{tabular}
  
  \vspace{1cm}
  
  \begin{tabular}{|m{0.7cm}|m{10cm}|m{4cm}|}
  \hline
  № & Soraw & Juwap \\
  \hline
  1. & Kóp aǵzalını \((x - a)\) ǵa bólgendegi qaldıq nege teń &  \\
  \hline
  2. & Esaplań \(d\left( \int{f(x)dx} \right) = ?\) &  \\
  \hline
  3. & Sızıqlı defferencial teńlemeniń uluwma sheshimin jazıń &  \\
  \hline
  4. & Orın almastırıw formulasın jazıń &  \\
  \hline
  5. & Integraldı esaplań: \(\int{(x + \sin x)dx}\). &  \\
  \hline
  6. & Integraldı esaplań: \(\int{2^{x}dx}\). &  \\
  \hline
  7. & Anıq integraldı esaplań: \(\int_{0}^{\pi}{\sin xdx}\). &  \\
  \hline
  8. & Qatardıń jıyındısın esaplań: \(\sum_{n = 1}^{\infty}\frac{1}{(2n - 1)(2n + 1)}\). &  \\
  \hline
  9. & Differencial teńlemeni sheshiń: \(y' + xy = 0\). &  \\
  \hline
  10. & Gruppadaǵı 20 studentten neshe túrli usıl menen 3 náwbetshini saylap alıwǵa boladı? &  \\
  \hline
  \end{tabular}
  
  \vspace{1cm}
  
  \begin{tabular}{lll}
  Tuwrı juwaplar sanı: \underline{\hspace{1.5cm}} & 
  Bahası: \underline{\hspace{1.5cm}} & 
  Imtixan alıwshınıń qolı: \underline{\hspace{2cm}} \\
  \end{tabular}
  
  \egroup
  
  \newpage
  
  
  \textbf{87-variant}\\
  
  \bgroup
  \def\arraystretch{1.6} % 1 is the default, change whatever you need
  
  \begin{tabular}{|m{5.7cm}|m{9.5cm}|}
  \hline
  Shifr & \\
  \hline
  \end{tabular}
  
  \vspace{1cm}
  
  \begin{tabular}{|m{0.7cm}|m{10cm}|m{4cm}|}
  \hline
  № & Soraw & Juwap \\
  \hline
  1. & Funkciya qanday usıllarda beriledi &  \\
  \hline
  2. & Ózgeriwshini almastırıp integrallaw usılıniń formulasın jazıń. &  \\
  \hline
  3. & Ózgeriwshileri ajıralǵan differenciallıq teńlemesiniń uluwma kórinisin jazıń &  \\
  \hline
  4. & Shártli itimallıq formulasın jazıń &  \\
  \hline
  5. & Anıq emes integraldı esaplań: \(\int{e^{x}dx}\) . &  \\
  \hline
  6. & Anıq integraldı esaplań: \(\int_{1}^{3}\frac{2}{x + 1}dx\). &  \\
  \hline
  7. & Anıq integraldı esaplań: \(\int_{0}^{\frac{\pi}{2}}{\cos xdx}\). &  \\
  \hline
  8. & Qatardıń qosındısın tabıń: \(\sum_{n = 1}^{\infty}\frac{1}{n(n + 1)}\). &  \\
  \hline
  9. & Sızıqlı differencial teńlemeniń ulwma sheshimin tabıń: \(y' + y = e^{x}\). &  \\
  \hline
  10. & Telefon nomerdiń aqırǵı cifrasın umıtıp, tosınnan nomerlerdi tere basladı. Kerekli nomerdi tabıw itimallıǵın esaplań. &  \\
  \hline
  \end{tabular}
  
  \vspace{1cm}
  
  \begin{tabular}{lll}
  Tuwrı juwaplar sanı: \underline{\hspace{1.5cm}} & 
  Bahası: \underline{\hspace{1.5cm}} & 
  Imtixan alıwshınıń qolı: \underline{\hspace{2cm}} \\
  \end{tabular}
  
  \egroup
  
  \newpage
  
  
  \textbf{88-variant}\\
  
  \bgroup
  \def\arraystretch{1.6} % 1 is the default, change whatever you need
  
  \begin{tabular}{|m{5.7cm}|m{9.5cm}|}
  \hline
  Shifr & \\
  \hline
  \end{tabular}
  
  \vspace{1cm}
  
  \begin{tabular}{|m{0.7cm}|m{10cm}|m{4cm}|}
  \hline
  № & Soraw & Juwap \\
  \hline
  1. & Eki ózgeriwshili funkciyanıń ekinshi tártipli aralas tuwındıları qalay belgilenedi &  \\
  \hline
  2. & Funkcianıń \((x_{0},\ y_{0})\) noqattaǵı tuwındısınıń formulasın jazıń &  \\
  \hline
  3. & Sızıqlı differenciallıq teńlemeniń uluwma kórinisin jazıń &  \\
  \hline
  4. & Gruppalaw formulasın jazıń &  \\
  \hline
  5. & Integraldı esaplań:\(\int{(x - 1)^{20}}dx\). &  \\
  \hline
  6. & Integraldı esaplań: \(\int_{1}^{\infty}{\frac{1}{(x + 2)^2 }dx}\). &  \\
  \hline
  7. & Anıq integraldı esaplań: \(\int_{2}^{4}\frac{dx}{x}\). &  \\
  \hline
  8. & Funkcional qatardıń jıynaqlılıq oblastın jazıń: \(\ln x + ln^2 x + ... + ln^{n}x + ...\). &  \\
  \hline
  9. & Differencial teńlemeniń ulıwma sheshimin tabıń: \(xy' - 2y = 0\). &  \\
  \hline
  10. & Korobkada 3 aq, 7 qara shar bar. Tosınnan úsh shar izbe-iz alındı. Izbe-iz alınǵan sharlardıń qara, qara, aq degen izbe-izlikte bolıw itimallıǵın tabıń. &  \\
  \hline
  \end{tabular}
  
  \vspace{1cm}
  
  \begin{tabular}{lll}
  Tuwrı juwaplar sanı: \underline{\hspace{1.5cm}} & 
  Bahası: \underline{\hspace{1.5cm}} & 
  Imtixan alıwshınıń qolı: \underline{\hspace{2cm}} \\
  \end{tabular}
  
  \egroup
  
  \newpage
  
  
  \textbf{89-variant}\\
  
  \bgroup
  \def\arraystretch{1.6} % 1 is the default, change whatever you need
  
  \begin{tabular}{|m{5.7cm}|m{9.5cm}|}
  \hline
  Shifr & \\
  \hline
  \end{tabular}
  
  \vspace{1cm}
  
  \begin{tabular}{|m{0.7cm}|m{10cm}|m{4cm}|}
  \hline
  № & Soraw & Juwap \\
  \hline
  1. & Eki ózgeriwshili funkciyanıń ekinshi tártipli dara tuwındıları qalay belgilenedi &  \\
  \hline
  2. & Bóleklep inegrallaw formulasın jazıń &  \\
  \hline
  3. & Oń aǵzalı qatarlar ushın jıynaqlılıqtıń Dalamber belgisin jazıń &  \\
  \hline
  4. & Isenimli waqıyanıń itimallıǵı nege teń &  \\
  \hline
  5. & Anıq emes integraldı esaplań: \(\int\frac{dx}{cos^2 x}\). &  \\
  \hline
  6. & Integraldı esaplań: \(\int_{1}^{\infty}{\frac{1}{x^2 }dx}\). &  \\
  \hline
  7. & Anıq integraldı esaplań: \(\int_{0}^{1}{(3x^2 } + 1)dx\). &  \\
  \hline
  8. & Sanlı qatardıń baslanǵısh úsh aǵzasın jazıń: \(\sum_{n = 1}^{\infty}\frac{n!}{2^{n}}\). &  \\
  \hline
  9. & Differencial teńlemeni esaplań: \(yy' = 4\). &  \\
  \hline
  10. & Dóngelektiń ishine kvadrat sızılǵan. Dóngelektiń ishinen tosınnan belgilengen noqattıń kvadrattıń ishinde jatıw itimallıǵın tabıń. &  \\
  \hline
  \end{tabular}
  
  \vspace{1cm}
  
  \begin{tabular}{lll}
  Tuwrı juwaplar sanı: \underline{\hspace{1.5cm}} & 
  Bahası: \underline{\hspace{1.5cm}} & 
  Imtixan alıwshınıń qolı: \underline{\hspace{2cm}} \\
  \end{tabular}
  
  \egroup
  
  \newpage
  
  
  \textbf{90-variant}\\
  
  \bgroup
  \def\arraystretch{1.6} % 1 is the default, change whatever you need
  
  \begin{tabular}{|m{5.7cm}|m{9.5cm}|}
  \hline
  Shifr & \\
  \hline
  \end{tabular}
  
  \vspace{1cm}
  
  \begin{tabular}{|m{0.7cm}|m{10cm}|m{4cm}|}
  \hline
  № & Soraw & Juwap \\
  \hline
  1. & Eki ózgeriwshili funkciyanıń grafigi neden ibarat &  \\
  \hline
  2. & Eki ózgeriwshli funkciyanıń \(M(x_{0}, y_{0})\) noqattaǵı úzliksizliginiń anıqlaması &  \\
  \hline
  3. & Oń aǵzalı qatarlar ushın jıynaqlılıqtıń Koshi belgisin jazıń &  \\
  \hline
  4. & Orın awıstırıw formulasın jazıń &  \\
  \hline
  5. & Integraldı esaplań: \(\int{\frac{1}{\sin x}dx}\). &  \\
  \hline
  6. & Esaplań: \(\int_{1}^2 {e^{x}dx}\). &  \\
  \hline
  7. & Anıq integraldı esaplań: \(\int_{- \pi/4}^{0}\frac{dx}{cos^2 x}\). &  \\
  \hline
  8. & Funkcional qatardıń jaqınlasıw oblastın tabıń: \(x + \frac{x^2 }{2^2 } + ... + \frac{x^{n}}{n^2 } + ...\) &  \\
  \hline
  9. & Differencial teńlemeniń ulıwma sheshimin tabıń: \(y' = e^{x}\). &  \\
  \hline
  10. & «BIOLOGIYA» sóziniń háripleri bólek kartochkalarǵa jazılıp jawıp, aralastırılıp qoyılǵan. Barlıq kartochkalar tosınnan izbe-iz alınıp ashılıp, alınıw tártibinde stol ústine dizilgende taǵı «BIOLOGIYA» sóziniń kelip shıǵıw itimallıǵın tabıń. &  \\
  \hline
  \end{tabular}
  
  \vspace{1cm}
  
  \begin{tabular}{lll}
  Tuwrı juwaplar sanı: \underline{\hspace{1.5cm}} & 
  Bahası: \underline{\hspace{1.5cm}} & 
  Imtixan alıwshınıń qolı: \underline{\hspace{2cm}} \\
  \end{tabular}
  
  \egroup
  
  \newpage
  
  
  \textbf{91-variant}\\
  
  \bgroup
  \def\arraystretch{1.6} % 1 is the default, change whatever you need
  
  \begin{tabular}{|m{5.7cm}|m{9.5cm}|}
  \hline
  Shifr & \\
  \hline
  \end{tabular}
  
  \vspace{1cm}
  
  \begin{tabular}{|m{0.7cm}|m{10cm}|m{4cm}|}
  \hline
  № & Soraw & Juwap \\
  \hline
  1. & Eki ózgeriwshili funkciyanıń birinshi tártipli dara tuwındıları qalay belgilenedi &  \\
  \hline
  2. & Bóleklep inegrallaw formulasın jazıń &  \\
  \hline
  3. & Bernulli differenciallıq teńemesin jazıń &  \\
  \hline
  4. & Itimmallıqtıń geometriyalıq anıqlamasınıń formulasın jazıń &  \\
  \hline
  5. & Anıq emes integraldı esaplań: \(\int{\left( x^2  + \frac{1}{x} + \sin x \right)dx}\). &  \\
  \hline
  6. & Esaplań: \(\int\left( x^{4} - \frac{1}{x} \right)dx\). &  \\
  \hline
  7. & Anıq integraldı esaplań: \(\int_{- \pi/4}^{0}\frac{dx}{cos^2 x}\). &  \\
  \hline
  8. & Funkcional qatardıń jıynaqlılıq oblastın tabıń:\(1 + x + ... + x^{n} + ...\) &  \\
  \hline
  9. & Differencial teńlemeni esaplań: \(yy' = 4\). &  \\
  \hline
  10. & Qutada 5 aq hám 15 qara shar bar. Tosınnan alınǵan bir shardıń aq bolıw itimallıǵın tabıń &  \\
  \hline
  \end{tabular}
  
  \vspace{1cm}
  
  \begin{tabular}{lll}
  Tuwrı juwaplar sanı: \underline{\hspace{1.5cm}} & 
  Bahası: \underline{\hspace{1.5cm}} & 
  Imtixan alıwshınıń qolı: \underline{\hspace{2cm}} \\
  \end{tabular}
  
  \egroup
  
  \newpage
  
  
  \textbf{92-variant}\\
  
  \bgroup
  \def\arraystretch{1.6} % 1 is the default, change whatever you need
  
  \begin{tabular}{|m{5.7cm}|m{9.5cm}|}
  \hline
  Shifr & \\
  \hline
  \end{tabular}
  
  \vspace{1cm}
  
  \begin{tabular}{|m{0.7cm}|m{10cm}|m{4cm}|}
  \hline
  № & Soraw & Juwap \\
  \hline
  1. & \(n\)-dárejeli kóp aǵzalınıń uluwma kórinisi &  \\
  \hline
  2. & Esaplań \(\left( \int{f(x)dx} \right)' = ?\). &  \\
  \hline
  3. & Sızıqlı defferencial teńlemeniń uluwma sheshimin jazıń &  \\
  \hline
  4. & Isenimli waqıyanıń itimallıǵı nege teń &  \\
  \hline
  5. & Racional funkciyanı integrallań: \(\int{\frac{3}{(x - 1)(x + 2)}dx}\). &  \\
  \hline
  6. & Integraldı esaplań: \(\int_{1}^{\infty}{\frac{1}{(x + 2)^2 }dx}\). &  \\
  \hline
  7. & Anıq integraldı esaplań: \(\int_{0}^{\frac{\pi}{2}}{\cos xdx}\). &  \\
  \hline
  8. & Qatardıń jıyındısın esaplań: \(\sum_{n = 1}^{\infty}\frac{1}{(2n - 1)(2n + 1)}\). &  \\
  \hline
  9. & Differencial teńlemeniń ulıwma sheshimin tabıń: \(xy' - 2y = 0\). &  \\
  \hline
  10. & Korobkada 15 aq, 18 qara shar bar. Tosınnan alınǵan bir shar aq bolıw itimallıǵın tabıń. &  \\
  \hline
  \end{tabular}
  
  \vspace{1cm}
  
  \begin{tabular}{lll}
  Tuwrı juwaplar sanı: \underline{\hspace{1.5cm}} & 
  Bahası: \underline{\hspace{1.5cm}} & 
  Imtixan alıwshınıń qolı: \underline{\hspace{2cm}} \\
  \end{tabular}
  
  \egroup
  
  \newpage
  
  
  \textbf{93-variant}\\
  
  \bgroup
  \def\arraystretch{1.6} % 1 is the default, change whatever you need
  
  \begin{tabular}{|m{5.7cm}|m{9.5cm}|}
  \hline
  Shifr & \\
  \hline
  \end{tabular}
  
  \vspace{1cm}
  
  \begin{tabular}{|m{0.7cm}|m{10cm}|m{4cm}|}
  \hline
  № & Soraw & Juwap \\
  \hline
  1. & Funkciyanıń anıqlanıw oblastı qalay belgilenedi &  \\
  \hline
  2. & Anıq integraldı esaplawdıń Nyuton-Leybnic formulasın jazıń &  \\
  \hline
  3. & Ózgeriwshileri ajıralǵan differenciallıq teńlemesiniń uluwma kórinisin jazıń &  \\
  \hline
  4. & Itimallıqtıń klassikalıq anıqlamasınıń formulasın keltiriń &  \\
  \hline
  5. & Anıq emes integraldı esaplań: \(\int{e^{x}dx}\) . &  \\
  \hline
  6. & Integraldı esaplań: \(\int{2^{x}dx}\). &  \\
  \hline
  7. & Anıq integraldı esaplań: \(\int_{2}^{4}\frac{dx}{x}\). &  \\
  \hline
  8. & Qatardıń qosındısın tabıń: \(\sum_{n = 1}^{\infty}\frac{1}{n(n + 1)}\). &  \\
  \hline
  9. & Differencial teńlemeniń ulıwma sheshimin tabıń: \(y' = e^{x}\). &  \\
  \hline
  10. & 50 buyımnan ibarat partiyada 3 buyım jaramsız. Tosınnan alınǵan 8 buyımnıń ishinde 1 buyımı jaramsız bolıw itimallıǵın tabıń &  \\
  \hline
  \end{tabular}
  
  \vspace{1cm}
  
  \begin{tabular}{lll}
  Tuwrı juwaplar sanı: \underline{\hspace{1.5cm}} & 
  Bahası: \underline{\hspace{1.5cm}} & 
  Imtixan alıwshınıń qolı: \underline{\hspace{2cm}} \\
  \end{tabular}
  
  \egroup
  
  \newpage
  
  
  \textbf{94-variant}\\
  
  \bgroup
  \def\arraystretch{1.6} % 1 is the default, change whatever you need
  
  \begin{tabular}{|m{5.7cm}|m{9.5cm}|}
  \hline
  Shifr & \\
  \hline
  \end{tabular}
  
  \vspace{1cm}
  
  \begin{tabular}{|m{0.7cm}|m{10cm}|m{4cm}|}
  \hline
  № & Soraw & Juwap \\
  \hline
  1. & Eki ózgeriwshili funkciyalar qalay belgilenedi &  \\
  \hline
  2. & \((x_0,y_0)\) tochkanıń \(\varepsilon\) dógeregi qalay belgilenedi &  \\
  \hline
  3. & Funkcionallıq qatardıń uluwma kórinisi &  \\
  \hline
  4. & Orın awıstırıw formulasın jazıń &  \\
  \hline
  5. & Integraldı esaplań: \(\int{\frac{1}{\sin x}dx}\). &  \\
  \hline
  6. & Esaplań: \(\int_{1}^2 {e^{x}dx}\). &  \\
  \hline
  7. & Anıq integraldı esaplań: \(\int_{0}^{\pi}{\sin xdx}\). &  \\
  \hline
  8. & Qatardıń qosındısın tabıń: \(\sum_{n = 1}^{\infty}\frac{1}{n(n + 3)}\). &  \\
  \hline
  9. & Sızıqlı differencial teńlemeniń ulwma sheshimin tabıń: \(y' + y = e^{x}\). &  \\
  \hline
  10. & «MATEMATIKA» sóziniń háripleri bólek kartochkalarǵa jazılıp jawıp aralastırılıp qoyılǵan. Barlıq kartochkalar tosınnan izbe-iz alınıp ashılıp, alınıw tártibinde stol ústine dizilgende taǵı «MATEMATIKA» sóziniń kelip shıǵıw itimallıǵın tabıń. &  \\
  \hline
  \end{tabular}
  
  \vspace{1cm}
  
  \begin{tabular}{lll}
  Tuwrı juwaplar sanı: \underline{\hspace{1.5cm}} & 
  Bahası: \underline{\hspace{1.5cm}} & 
  Imtixan alıwshınıń qolı: \underline{\hspace{2cm}} \\
  \end{tabular}
  
  \egroup
  
  \newpage
  
  
  \textbf{95-variant}\\
  
  \bgroup
  \def\arraystretch{1.6} % 1 is the default, change whatever you need
  
  \begin{tabular}{|m{5.7cm}|m{9.5cm}|}
  \hline
  Shifr & \\
  \hline
  \end{tabular}
  
  \vspace{1cm}
  
  \begin{tabular}{|m{0.7cm}|m{10cm}|m{4cm}|}
  \hline
  № & Soraw & Juwap \\
  \hline
  1. & Eki ózgeriwshili funkciyanıń anıqlanıw oblastı qay jerde jaylasadı &  \\
  \hline
  2. & Esaplań \(d\left( \int{f(x)dx} \right) = ?\) &  \\
  \hline
  3. & Eger \(\sum_{n = 1}^{\infty}a_{n} = A,\ \sum_{n = 1}^{\infty}b_{n} = B\) bolsa, onda \(\sum_{n = 1}^{\infty}\left( a_{n} + b_{n} \right) = ?\) &  \\
  \hline
  4. & Itimallıq keńisligin jazıń &  \\
  \hline
  5. & Anıq emes integraldı esaplań: \(\int{\left( 10x^{4} + 7x^{6} - 3 \right)dx}\). &  \\
  \hline
  6. & Anıq integraldı esaplań: \(\int_{1}^{3}\frac{2}{x + 1}dx\). &  \\
  \hline
  7. & Anıq integraldı esaplań: \(\int_{0}^{1}{(3x^2 } + 1)dx\). &  \\
  \hline
  8. & Sanlı qatardıń baslanǵısh úsh aǵzasın jazıń: \(\sum_{n = 1}^{\infty}\frac{n!}{2^{n}}\). &  \\
  \hline
  9. & Sızıqlı differerncial teńlemeniń uluwma sheshimin tabıń \(y' + y = e^{- x}\). &  \\
  \hline
  10. & Úsh birdey korobkada aq hám qara sharlar bar. 1-korobkada 5 aq, 8 qara shar, 2-korobkada 3 aq, 4 qara shar, 3-korobkada 2 aq, 3 qara shar bar. Úsh korobkaniń birewinen tosınnan alınǵan bir shar aq bolıw itimallıǵın tabıń. &  \\
  \hline
  \end{tabular}
  
  \vspace{1cm}
  
  \begin{tabular}{lll}
  Tuwrı juwaplar sanı: \underline{\hspace{1.5cm}} & 
  Bahası: \underline{\hspace{1.5cm}} & 
  Imtixan alıwshınıń qolı: \underline{\hspace{2cm}} \\
  \end{tabular}
  
  \egroup
  
  \newpage
  
  
  \textbf{96-variant}\\
  
  \bgroup
  \def\arraystretch{1.6} % 1 is the default, change whatever you need
  
  \begin{tabular}{|m{5.7cm}|m{9.5cm}|}
  \hline
  Shifr & \\
  \hline
  \end{tabular}
  
  \vspace{1cm}
  
  \begin{tabular}{|m{0.7cm}|m{10cm}|m{4cm}|}
  \hline
  № & Soraw & Juwap \\
  \hline
  1. & Eki ózgeriwshili funkciyanıń ekstremumınıń zárúrli shárti &  \\
  \hline
  2. & Funkcianıń \((x_{0},\ y_{0})\) noqattaǵı úzliksizliginiń formulasın jazıń &  \\
  \hline
  3. & Sızıqlı differenciallıq teńlemeniń uluwma kórinisin jazıń &  \\
  \hline
  4. & Gruppalaw formulasın jazıń &  \\
  \hline
  5. & Integraldı esaplań: \(\int{(x + \sin x)dx}\). &  \\
  \hline
  6. & Integraldı esaplań: \(\int_{1}^{\infty}{\frac{1}{x^2 }dx}\). &  \\
  \hline
  7. & Anıq integraldı esaplań: \(\int_{- \pi/4}^{0}\frac{dx}{cos^2 x}\). &  \\
  \hline
  8. & Funkcional qatardıń jıynaqlılıq oblastın tabıń:\(1 + x + ... + x^{n} + ...\) &  \\
  \hline
  9. & Differencial teńlemeni sheshiń: \(y' + xy = 0\). &  \\
  \hline
  10. & Telefon nomerdiń aqırǵı eki cifrasın umıtıp, tosınnan nomerlerdi tere basladı. Kerekli nomerdi tabıw itimallıǵın esaplań. &  \\
  \hline
  \end{tabular}
  
  \vspace{1cm}
  
  \begin{tabular}{lll}
  Tuwrı juwaplar sanı: \underline{\hspace{1.5cm}} & 
  Bahası: \underline{\hspace{1.5cm}} & 
  Imtixan alıwshınıń qolı: \underline{\hspace{2cm}} \\
  \end{tabular}
  
  \egroup
  
  \newpage
  
  
  \textbf{97-variant}\\
  
  \bgroup
  \def\arraystretch{1.6} % 1 is the default, change whatever you need
  
  \begin{tabular}{|m{5.7cm}|m{9.5cm}|}
  \hline
  Shifr & \\
  \hline
  \end{tabular}
  
  \vspace{1cm}
  
  \begin{tabular}{|m{0.7cm}|m{10cm}|m{4cm}|}
  \hline
  № & Soraw & Juwap \\
  \hline
  1. & Kóp aǵzalını \((x - a)\) ǵa bólgendegi qaldıq nege teń &  \\
  \hline
  2. & Eki ózgeriwshli funkciyanıń \(M(x_{0}, y_{0})\) noqattaǵı úzliksizliginiń anıqlaması &  \\
  \hline
  3. & Oń aǵzalı qatarlar ushın jıynaqlılıqtıń Koshi belgisin jazıń &  \\
  \hline
  4. & Tolıq itimallıqtıń formulasın jazıń &  \\
  \hline
  5. & Racional funkciyanı integrallań: \(\int{\frac{5}{(x - 3)(x + 2)}dx}\). &  \\
  \hline
  6. & Integraldı esaplań: \(\int_{1}^{\infty}{\frac{1}{(x + 2)^2 }dx}\). &  \\
  \hline
  7. & Anıq integraldı esaplań: \(\int_{0}^{1}{(3x^2 } + 1)dx\). &  \\
  \hline
  8. & Funkcional qatardıń jıynaqlılıq oblastın jazıń: \(\ln x + ln^2 x + ... + ln^{n}x + ...\). &  \\
  \hline
  9. & Differencial teńlemeniń ulıwma sheshimin tabıń: \(xy' - 2y = 0\). &  \\
  \hline
  10. & Eki kubikti bir márte taslaǵanda túsken ochkolardıń qosındısı 4 bolıw itimallıǵın tabıń. &  \\
  \hline
  \end{tabular}
  
  \vspace{1cm}
  
  \begin{tabular}{lll}
  Tuwrı juwaplar sanı: \underline{\hspace{1.5cm}} & 
  Bahası: \underline{\hspace{1.5cm}} & 
  Imtixan alıwshınıń qolı: \underline{\hspace{2cm}} \\
  \end{tabular}
  
  \egroup
  
  \newpage
  
  
  \textbf{98-variant}\\
  
  \bgroup
  \def\arraystretch{1.6} % 1 is the default, change whatever you need
  
  \begin{tabular}{|m{5.7cm}|m{9.5cm}|}
  \hline
  Shifr & \\
  \hline
  \end{tabular}
  
  \vspace{1cm}
  
  \begin{tabular}{|m{0.7cm}|m{10cm}|m{4cm}|}
  \hline
  № & Soraw & Juwap \\
  \hline
  1. & Eki ózgeriwshili funkciyanıń birinshi tártipli dara tuwındıları qalay belgilenedi &  \\
  \hline
  2. & Funkcianıń \((x_{0},\ y_{0})\) noqattaǵı tuwındısınıń formulasın jazıń &  \\
  \hline
  3. & Sanlı qatardıń uluwma kórinisin jazıń &  \\
  \hline
  4. & Itimallıqtıń mánisler oblastın jazıń &  \\
  \hline
  5. & Anıq emes integraldı esaplań: \(\int\frac{dx}{cos^2 x}\). &  \\
  \hline
  6. & Esaplań: \(\int\left( x^{4} - \frac{1}{x} \right)dx\). &  \\
  \hline
  7. & Anıq integraldı esaplań: \(\int_{0}^{\pi}{\sin xdx}\). &  \\
  \hline
  8. & Funkcional qatardıń jaqınlasıw oblastın tabıń: \(x + \frac{x^2 }{2^2 } + ... + \frac{x^{n}}{n^2 } + ...\) &  \\
  \hline
  9. & Differencial teńlemeniń ulıwma sheshimin tabıń: \(y' = e^{x}\). &  \\
  \hline
  10. & Tiyindi eki márte taslaǵanda, keminde bir márte san tárepi túsiw itimallıǵın tabıń. &  \\
  \hline
  \end{tabular}
  
  \vspace{1cm}
  
  \begin{tabular}{lll}
  Tuwrı juwaplar sanı: \underline{\hspace{1.5cm}} & 
  Bahası: \underline{\hspace{1.5cm}} & 
  Imtixan alıwshınıń qolı: \underline{\hspace{2cm}} \\
  \end{tabular}
  
  \egroup
  
  \newpage
  
  
  \textbf{99-variant}\\
  
  \bgroup
  \def\arraystretch{1.6} % 1 is the default, change whatever you need
  
  \begin{tabular}{|m{5.7cm}|m{9.5cm}|}
  \hline
  Shifr & \\
  \hline
  \end{tabular}
  
  \vspace{1cm}
  
  \begin{tabular}{|m{0.7cm}|m{10cm}|m{4cm}|}
  \hline
  № & Soraw & Juwap \\
  \hline
  1. & Eki ózgeriwshili funkciyanıń ekinshi tártipli aralas tuwındıları qalay belgilenedi &  \\
  \hline
  2. & Ózgeriwshini almastırıp integrallaw usılıniń formulasın jazıń. &  \\
  \hline
  3. & Oń aǵzalı qatarlar ushın jıynaqlılıqtıń Dalamber belgisin jazıń &  \\
  \hline
  4. & Orın almastırıw formulasın jazıń &  \\
  \hline
  5. & Integraldı esaplań:\(\int{(x - 1)^{20}}dx\). &  \\
  \hline
  6. & Integraldı esaplań: \(\int{2^{x}dx}\). &  \\
  \hline
  7. & Anıq integraldı esaplań: \(\int_{2}^{4}\frac{dx}{x}\). &  \\
  \hline
  8. & Funkcional qatardıń jıynaqlılıq oblastın tabıń:\(1 + x + ... + x^{n} + ...\) &  \\
  \hline
  9. & Sızıqlı differencial teńlemeniń ulwma sheshimin tabıń: \(y' + y = e^{x}\). &  \\
  \hline
  10. & 50 buyımnan ibarat partiyada 3 buyım jaramsız. Tosınnan alınǵan 8 buyımnıń ishinde 1 buyımı jaramsız bolıw itimallıǵın tabıń &  \\
  \hline
  \end{tabular}
  
  \vspace{1cm}
  
  \begin{tabular}{lll}
  Tuwrı juwaplar sanı: \underline{\hspace{1.5cm}} & 
  Bahası: \underline{\hspace{1.5cm}} & 
  Imtixan alıwshınıń qolı: \underline{\hspace{2cm}} \\
  \end{tabular}
  
  \egroup
  
  \newpage
  
  
  \textbf{100-variant}\\
  
  \bgroup
  \def\arraystretch{1.6} % 1 is the default, change whatever you need
  
  \begin{tabular}{|m{5.7cm}|m{9.5cm}|}
  \hline
  Shifr & \\
  \hline
  \end{tabular}
  
  \vspace{1cm}
  
  \begin{tabular}{|m{0.7cm}|m{10cm}|m{4cm}|}
  \hline
  № & Soraw & Juwap \\
  \hline
  1. & Eki ózgeriwshili funkciyanıń anıqlanıw oblastı qay jerde jaylasadı &  \\
  \hline
  2. & Funkcianıń \((x_{0},\ y_{0})\) noqattaǵı úzliksizliginiń formulasın jazıń &  \\
  \hline
  3. & Eger \(\sum_{n = 1}^{\infty}a_{n} = A,\ \sum_{n = 1}^{\infty}b_{n} = B\) bolsa, onda \(\sum_{n = 1}^{\infty}\left( a_{n} - b_{n} \right) = ?\) &  \\
  \hline
  4. & Shártli itimallıq formulasın jazıń &  \\
  \hline
  5. & Racional funkciyanı integrallań: \(\int{\frac{5}{(x - 3)(x + 2)}dx}\). &  \\
  \hline
  6. & Integraldı esaplań: \(\int_{1}^{\infty}{\frac{1}{x^2 }dx}\). &  \\
  \hline
  7. & Anıq integraldı esaplań: \(\int_{0}^{\frac{\pi}{2}}{\cos xdx}\). &  \\
  \hline
  8. & Funkcional qatardıń jıynaqlılıq oblastın jazıń: \(\ln x + ln^2 x + ... + ln^{n}x + ...\). &  \\
  \hline
  9. & Differencial teńlemeni esaplań: \(yy' = 4\). &  \\
  \hline
  10. & Telefon nomerdiń aqırǵı eki cifrasın umıtıp, tosınnan nomerlerdi tere basladı. Kerekli nomerdi tabıw itimallıǵın esaplań. &  \\
  \hline
  \end{tabular}
  
  \vspace{1cm}
  
  \begin{tabular}{lll}
  Tuwrı juwaplar sanı: \underline{\hspace{1.5cm}} & 
  Bahası: \underline{\hspace{1.5cm}} & 
  Imtixan alıwshınıń qolı: \underline{\hspace{2cm}} \\
  \end{tabular}
  
  \egroup
  
  \newpage
  
  

\end{document}
