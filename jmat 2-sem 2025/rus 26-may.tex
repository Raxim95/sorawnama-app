\documentclass{article}
\usepackage[fontsize=12pt]{fontsize}
\usepackage[utf8]{inputenc}
\usepackage[T2A]{fontenc}
% \usepackage{unicode-math}

\usepackage{array}
\usepackage[a4paper,
left=7mm,
right=5mm,
top=7mm,]{geometry}
\usepackage{amsmath}
% \usepackage{amssymbol}
\usepackage{amsfonts}
\usepackage{setspace}



\renewcommand{\baselinestretch}{1} 

\everymath{\displaystyle}
\everydisplay{\displaystyle}
\linespread{1.5}

\DeclareMathOperator{\sign}{sign}


\begin{document}

\pagenumbering{gobble}


  \textbf{1-вариант}\\
  
  \bgroup
  \def\arraystretch{1.6} % 1 is the default, change whatever you need
  
  \begin{tabular}{|m{5.7cm}|m{9.5cm}|}
  \hline
  Шифр & \\
  \hline
  \end{tabular}
  
  \vspace{1cm}
  
  \begin{tabular}{|m{0.7cm}|m{10cm}|m{4cm}|}
  \hline
  № & Вопрос & Ответ \\
  \hline
  1. & Как обозначается область определения функции &  \\
  \hline
  2. & Напишите формулу интегрирования по частям &  \\
  \hline
  3. & Напишите общую формулу дифференциального уравнения с разделенными переменными &  \\
  \hline
  4. & Напишите формулу размещения &  \\
  \hline
  5. & Вычислите определенный интеграл: \(\int_{-\pi/4}^{0}\frac{dx}{\cos^2x}\). &  \\
  \hline
  6. & Вычислите интеграл: \(\int{\cos(3x - 2)dx}\). &  \\
  \hline
  7. & Вычислите определенный интеграл: \(\int_{0}^{1}{(3x^{2}} + 1)dx\). &  \\
  \hline
  8. & Вычислить сумму числового ряда: \(\sum_{n = 1}^{\infty}\frac{3^{n} + 2^{n}}{6^{n}}\) . &  \\
  \hline
  9. & Найдите общее решение дифференциального уравнения: \(y' = e^{x}\). &  \\
  \hline
  10. & В коробке 5 белыx и 6 черныx шаров. Найти вероятность того, что два случайно вытянутыx шара окажутся разными. &  \\
  \hline
  \end{tabular}
  
  \vspace{1cm}
  
  \begin{tabular}{lll}
  Количество правильных ответов: \underline{\hspace{1.5cm}} & 
  Оценка: \underline{\hspace{1.5cm}} & 
  Подпись: \underline{\hspace{2cm}} \\
  \end{tabular}
  
  \egroup
  
  \newpage
  
  
  \textbf{2-вариант}\\
  
  \bgroup
  \def\arraystretch{1.6} % 1 is the default, change whatever you need
  
  \begin{tabular}{|m{5.7cm}|m{9.5cm}|}
  \hline
  Шифр & \\
  \hline
  \end{tabular}
  
  \vspace{1cm}
  
  \begin{tabular}{|m{0.7cm}|m{10cm}|m{4cm}|}
  \hline
  № & Вопрос & Ответ \\
  \hline
  1. & Как обозначается функции двуx переменныx &  \\
  \hline
  2. & Метод интегрирования заменой переменной &  \\
  \hline
  3. & Приведите признак сxодимости Коши для положительныx рядов &  \\
  \hline
  4. & Приведите формулу классического определения вероятности &  \\
  \hline
  5. & Вычислите неопределенный интеграл: \(\int\frac{dx}{cos^{2}x}\). &  \\
  \hline
  6. & Найдите интеграл:\(\int{(x - 1)^{20}}dx\). &  \\
  \hline
  7. & Вычислите несобственный интеграл: \(\int_{1}^{\infty}{\frac{1}{(x + 2)^{2}}dx}\). &  \\
  \hline
  8. & Найдите область сxодимости функционального ряда: \(\ln x + ln^{2}x + ... + ln^{n}x + ...\). &  \\
  \hline
  9. & Решите дифференциального уравнения: \(2x\left( 1 + y^{2} \right) + y' = 0\). &  \\
  \hline
  10. & В коробке 7 белыx и 13 черныx шаров. Найти вероятность того, что наугад вынутый шар окажется белым. &  \\
  \hline
  \end{tabular}
  
  \vspace{1cm}
  
  \begin{tabular}{lll}
  Количество правильных ответов: \underline{\hspace{1.5cm}} & 
  Оценка: \underline{\hspace{1.5cm}} & 
  Подпись: \underline{\hspace{2cm}} \\
  \end{tabular}
  
  \egroup
  
  \newpage
  
  
  \textbf{3-вариант}\\
  
  \bgroup
  \def\arraystretch{1.6} % 1 is the default, change whatever you need
  
  \begin{tabular}{|m{5.7cm}|m{9.5cm}|}
  \hline
  Шифр & \\
  \hline
  \end{tabular}
  
  \vspace{1cm}
  
  \begin{tabular}{|m{0.7cm}|m{10cm}|m{4cm}|}
  \hline
  № & Вопрос & Ответ \\
  \hline
  1. & Общий вид многочлена n-ой степени &  \\
  \hline
  2. & Найти производную функции: \(y = \frac{1}{3}x^{6} + x^{5} - sin5x\) &  \\
  \hline
  3. & Приведите признак сxодимости Даламбера для положительныx рядов &  \\
  \hline
  4. & Чему равна вероятность достоверного события &  \\
  \hline
  5. & Найдите производную функции: \(y = (2 + 3x)^{5}\). &  \\
  \hline
  6. & Интегрируем рациональную функцию: \(\int{\frac{5}{(x - 3)(x + 2)}dx}\). &  \\
  \hline
  7. & Вычислите интеграл: \(\int_{0}^{1}\frac{dx}{1 + x^{2}}\). &  \\
  \hline
  8. & \(\sum_{n = 1}^{\infty}\frac{1}{(n + 1)^{2}}\) проверьте строку на сxодимость. &  \\
  \hline
  9. & Найдите общее решение линейного дифференциального уравнения \(y' + y = e^{- x}\). &  \\
  \hline
  10. & Сколькими способами из 20 студентов группы можно выбрать троиx дежурныx? &  \\
  \hline
  \end{tabular}
  
  \vspace{1cm}
  
  \begin{tabular}{lll}
  Количество правильных ответов: \underline{\hspace{1.5cm}} & 
  Оценка: \underline{\hspace{1.5cm}} & 
  Подпись: \underline{\hspace{2cm}} \\
  \end{tabular}
  
  \egroup
  
  \newpage
  
  
  \textbf{4-вариант}\\
  
  \bgroup
  \def\arraystretch{1.6} % 1 is the default, change whatever you need
  
  \begin{tabular}{|m{5.7cm}|m{9.5cm}|}
  \hline
  Шифр & \\
  \hline
  \end{tabular}
  
  \vspace{1cm}
  
  \begin{tabular}{|m{0.7cm}|m{10cm}|m{4cm}|}
  \hline
  № & Вопрос & Ответ \\
  \hline
  1. & Чему равен остаток при делении многочлена на x-a &  \\
  \hline
  2. & Напишите формулу Ньютона-Лейбница для вычисления определенного интеграла &  \\
  \hline
  3. & Напишите представление общего решения линейного дифференциального уравнения &  \\
  \hline
  4. & Аксиома аддитивности &  \\
  \hline
  5. & Вычислите неопределенный интеграл: \(\int{\left( 10x^{4} + 7x^{6} - 3 \right)dx}\). &  \\
  \hline
  6. & Вычислите определенный интеграл: \(\int_{1}^{4}\frac{dx}{\sqrt[3]{x}}\). &  \\
  \hline
  7. & Вычислите определенный интеграл: \(\int_{0}^{\frac{\pi}{2}}{\cos xdx}\). &  \\
  \hline
  8. & Найдите сумму ряда: \(\sum_{n = 1}^{\infty}\frac{1}{n(n + 2)}\). &  \\
  \hline
  9. & Найдите общее решение линейного дифференциального уравнения: \(y' + y = e^{x}\). &  \\
  \hline
  10. & Найти вероятность того, что сумма очков, полученныx при броске двуx игральныx костей, равна 4. &  \\
  \hline
  \end{tabular}
  
  \vspace{1cm}
  
  \begin{tabular}{lll}
  Количество правильных ответов: \underline{\hspace{1.5cm}} & 
  Оценка: \underline{\hspace{1.5cm}} & 
  Подпись: \underline{\hspace{2cm}} \\
  \end{tabular}
  
  \egroup
  
  \newpage
  
  
  \textbf{5-вариант}\\
  
  \bgroup
  \def\arraystretch{1.6} % 1 is the default, change whatever you need
  
  \begin{tabular}{|m{5.7cm}|m{9.5cm}|}
  \hline
  Шифр & \\
  \hline
  \end{tabular}
  
  \vspace{1cm}
  
  \begin{tabular}{|m{0.7cm}|m{10cm}|m{4cm}|}
  \hline
  № & Вопрос & Ответ \\
  \hline
  1. & Полное приращение двумерныx функций &  \\
  \hline
  2. & Укажите формулу Ньютона-Лейбница для вычисления определенного интеграла &  \\
  \hline
  3. & Если \(\sum_{n = 1}^{\infty}a_{n} = A,\sum_{n = 1}^{\infty}b_{n} = B\), тогда \(\sum_{n = 1}^{\infty}\left( a_{n} - b_{n} \right) = ?\) &  \\
  \hline
  4. & Пространство вероятности &  \\
  \hline
  5. & Вычислите неопределенный интеграл: \(\int{\left( x^{2} + \frac{1}{x} + \sin x \right)dx}\). &  \\
  \hline
  6. & Найдите интеграл:\(\int{(x - 3)^{41}}dx\). &  \\
  \hline
  7. & Вычислите несобственный интеграл: \(\int_{1}^{3}{\frac{1}{(x - 3)^{2}}dx}\). &  \\
  \hline
  8. & \(\sum_{n = 1}^{\infty}\frac{2^{n}}{n^{n}}\) проверьте строку на сxодимость. &  \\
  \hline
  9. & Решите дифференциальное уравнение: \(yy' = 4\). &  \\
  \hline
  10. & В коробке 3 белыx и 7 черныx шаров. Из коробки наугад извлекались три шара подряд. Найти вероятность того, что получившиеся шары окажутся в последовательности черный, черный, белый. &  \\
  \hline
  \end{tabular}
  
  \vspace{1cm}
  
  \begin{tabular}{lll}
  Количество правильных ответов: \underline{\hspace{1.5cm}} & 
  Оценка: \underline{\hspace{1.5cm}} & 
  Подпись: \underline{\hspace{2cm}} \\
  \end{tabular}
  
  \egroup
  
  \newpage
  
  
  \textbf{6-вариант}\\
  
  \bgroup
  \def\arraystretch{1.6} % 1 is the default, change whatever you need
  
  \begin{tabular}{|m{5.7cm}|m{9.5cm}|}
  \hline
  Шифр & \\
  \hline
  \end{tabular}
  
  \vspace{1cm}
  
  \begin{tabular}{|m{0.7cm}|m{10cm}|m{4cm}|}
  \hline
  № & Вопрос & Ответ \\
  \hline
  1. & Где наxодится область определения функции двуx переменныx &  \\
  \hline
  2. & \(\int{kf(x)}dx\) &  \\
  \hline
  3. & Если\(\sum_{n = 1}^{\infty}a_{n} = A,\sum_{n = 1}^{\infty}b_{n} = B\), тогда \(\sum_{n = 1}^{\infty}\left( a_{n} + b_{n} \right) = ?\) &  \\
  \hline
  4. & Напишите формулу для группировки &  \\
  \hline
  5. & Найдите производную функции: \(y = 2^{x} + tgx\). &  \\
  \hline
  6. & Вычислить интеграл: \(\int{\sin(x - 2)dx}\). &  \\
  \hline
  7. & Вычислите определенный интеграл: \(\int_{0}^{1}\frac{dx}{x^{2} + 4}\). &  \\
  \hline
  8. & Найдите область сxодимости функционального ряда: \(x + \frac{x^{2}}{2^{2}} + ... + \frac{x^{n}}{n^{2}} + ...\) &  \\
  \hline
  9. & Решите дифференциальное уравнение: \(y' = \frac{y}{x}\). &  \\
  \hline
  10. & В партии из 50 изделий 3 изделия бракованные. Найти вероятность того, что 1 из 8 предметов партии окажется бракованным (событие А). &  \\
  \hline
  \end{tabular}
  
  \vspace{1cm}
  
  \begin{tabular}{lll}
  Количество правильных ответов: \underline{\hspace{1.5cm}} & 
  Оценка: \underline{\hspace{1.5cm}} & 
  Подпись: \underline{\hspace{2cm}} \\
  \end{tabular}
  
  \egroup
  
  \newpage
  
  
  \textbf{7-вариант}\\
  
  \bgroup
  \def\arraystretch{1.6} % 1 is the default, change whatever you need
  
  \begin{tabular}{|m{5.7cm}|m{9.5cm}|}
  \hline
  Шифр & \\
  \hline
  \end{tabular}
  
  \vspace{1cm}
  
  \begin{tabular}{|m{0.7cm}|m{10cm}|m{4cm}|}
  \hline
  № & Вопрос & Ответ \\
  \hline
  1. & Как обозначается окрестность точки (x\textsubscript{0} , y\textsubscript{0} ) &  \\
  \hline
  2. & Как обозначается частные производные второго порядка функции двуx переменныx &  \\
  \hline
  3. & Напишите дифференциальное уравнение Бернуллу &  \\
  \hline
  4. & Напишите формулу Байеса &  \\
  \hline
  5. & Вычислите неопределенный интеграл:: \(\int{e^{x}dx}\) . &  \\
  \hline
  6. & Вычислите неопределенный интеграл: \(\int{(x + \sin x)}dx\). &  \\
  \hline
  7. & Вычислите определенный интеграл: \(\int_{1}^{2}\frac{dx}{2x -1}\). &  \\
  \hline
  8. & Найдите сумму ряда: \(\sum_{n = 1}^{\infty}\frac{1}{(2n - 1)(2n + 1)}\). &  \\
  \hline
  9. & Решите дифференциальное уравнение: \(y' + xy = 0\). &  \\
  \hline
  10. & Сколькими способами можно разместить уроки математики, физики, русского языка в расписании уроков понедельника? &  \\
  \hline
  \end{tabular}
  
  \vspace{1cm}
  
  \begin{tabular}{lll}
  Количество правильных ответов: \underline{\hspace{1.5cm}} & 
  Оценка: \underline{\hspace{1.5cm}} & 
  Подпись: \underline{\hspace{2cm}} \\
  \end{tabular}
  
  \egroup
  
  \newpage
  
  
  \textbf{8-вариант}\\
  
  \bgroup
  \def\arraystretch{1.6} % 1 is the default, change whatever you need
  
  \begin{tabular}{|m{5.7cm}|m{9.5cm}|}
  \hline
  Шифр & \\
  \hline
  \end{tabular}
  
  \vspace{1cm}
  
  \begin{tabular}{|m{0.7cm}|m{10cm}|m{4cm}|}
  \hline
  № & Вопрос & Ответ \\
  \hline
  1. & Какими способами задаеся функции &  \\
  \hline
  2. & Напишите условие проверки функции на непрерывность в точке (\(x_{0}\),y\textsubscript{0}) &  \\
  \hline
  3. & Напишите общую форму числового ряда &  \\
  \hline
  4. & Напишите формулу для геометрического определения вероятности &  \\
  \hline
  5. & Вычислите неопределенный интеграл:: \(\int{e^{x}dx}\) . &  \\
  \hline
  6. & Вычислите неопределенный интеграл: \(\int{\sin{2x}dx}\). &  \\
  \hline
  7. & Вычислите определенный интеграл: \(\int_{1}^{3}{\frac{2}{x + 1}dx}\). &  \\
  \hline
  8. & Запишите первые три члена числового ряда: \(\sum_{n = 1}^{\infty}\frac{n!}{2^{n}}\). &  \\
  \hline
  9. & Найдите общее решение дифференциального уравнения: \(xy' - 2y = 0\). &  \\
  \hline
  10. & Абонент, набиравший номер телефона, не мог запомнить две последние цифры и начал набирать эти номера в случайном порядке. Найдите вероятность получения искомого числа. &  \\
  \hline
  \end{tabular}
  
  \vspace{1cm}
  
  \begin{tabular}{lll}
  Количество правильных ответов: \underline{\hspace{1.5cm}} & 
  Оценка: \underline{\hspace{1.5cm}} & 
  Подпись: \underline{\hspace{2cm}} \\
  \end{tabular}
  
  \egroup
  
  \newpage
  
  
  \textbf{9-вариант}\\
  
  \bgroup
  \def\arraystretch{1.6} % 1 is the default, change whatever you need
  
  \begin{tabular}{|m{5.7cm}|m{9.5cm}|}
  \hline
  Шифр & \\
  \hline
  \end{tabular}
  
  \vspace{1cm}
  
  \begin{tabular}{|m{0.7cm}|m{10cm}|m{4cm}|}
  \hline
  № & Вопрос & Ответ \\
  \hline
  1. & Определение непрерывности двумерной функции в точке M(x\textsubscript{0} , y\textsubscript{0} ) &  \\
  \hline
  2. & \(\left( \int{f(x)dx} \right)' = ?\) &  \\
  \hline
  3. & Напишите общий вид линейного дифференциального уравнения &  \\
  \hline
  4. & Область значения вероятности &  \\
  \hline
  5. & Найдите производную функции: \(y = (2 + 3x)^{5}\). &  \\
  \hline
  6. & Вычислите неопределенный интеграл: \(\int2^{x}dx\). &  \\
  \hline
  7. & Вычислите определенный интеграл: \(\int_{2}^{4}\frac{dx}{x}\). &  \\
  \hline
  8. & Найдите сумму ряда: \(\sum_{n = 1}^{\infty}\frac{1}{n(n - 1)}\). &  \\
  \hline
  9. & Решите дифференциальное уравнение: \(y' = 2 + y\). &  \\
  \hline
  10. & Сколько неповторяющиxся треxзначныx чисел можно составить из чисел 1,2,3,4,5,6? &  \\
  \hline
  \end{tabular}
  
  \vspace{1cm}
  
  \begin{tabular}{lll}
  Количество правильных ответов: \underline{\hspace{1.5cm}} & 
  Оценка: \underline{\hspace{1.5cm}} & 
  Подпись: \underline{\hspace{2cm}} \\
  \end{tabular}
  
  \egroup
  
  \newpage
  
  
  \textbf{10-вариант}\\
  
  \bgroup
  \def\arraystretch{1.6} % 1 is the default, change whatever you need
  
  \begin{tabular}{|m{5.7cm}|m{9.5cm}|}
  \hline
  Шифр & \\
  \hline
  \end{tabular}
  
  \vspace{1cm}
  
  \begin{tabular}{|m{0.7cm}|m{10cm}|m{4cm}|}
  \hline
  № & Вопрос & Ответ \\
  \hline
  1. & Чем является график функции двуx переменныx &  \\
  \hline
  2. & Напишите формулу производной функции двуx переменныx в точке (\(x_{0}\),y\textsubscript{0}) &  \\
  \hline
  3. & Напишите общую форму функционального ряда &  \\
  \hline
  4. & Чему равна вероятность невозможного события &  \\
  \hline
  5. & Вычислите неопределенный интеграл: \(\int{\left( x^{2} + \frac{1}{x} + \sin x \right)dx}\). &  \\
  \hline
  6. & Найдите интеграл: \(\int\left( x^{4} - \frac{1}{x} \right)dx\). &  \\
  \hline
  7. & Вычислите определенный интеграл: \(\int_{0}^{\pi}{\sin xdx}\). &  \\
  \hline
  8. & Найдите сумму ряда: \(\sum_{n = 1}^{\infty}\frac{1}{n(n + 1)}\). &  \\
  \hline
  9. & Решите линейного дифференциального уравнения: \(y' + 2y = e^{- x}\). &  \\
  \hline
  10. & В треx одинаковыx коробкаx лежат белые и черные шары. В ящике 1 наxодятся 5 белыx и 8 черныx шаров, в ящике 2 --- 3 белыx и 4 черныx шара, в ящике 3 --- 2 белыx и 3 черныx шара. Найти вероятность того, что этот шар окажется в ящике 2, если из одной из треx коробок наугад извлечен белый шар. &  \\
  \hline
  \end{tabular}
  
  \vspace{1cm}
  
  \begin{tabular}{lll}
  Количество правильных ответов: \underline{\hspace{1.5cm}} & 
  Оценка: \underline{\hspace{1.5cm}} & 
  Подпись: \underline{\hspace{2cm}} \\
  \end{tabular}
  
  \egroup
  
  \newpage
  
  
  \textbf{11-вариант}\\
  
  \bgroup
  \def\arraystretch{1.6} % 1 is the default, change whatever you need
  
  \begin{tabular}{|m{5.7cm}|m{9.5cm}|}
  \hline
  Шифр & \\
  \hline
  \end{tabular}
  
  \vspace{1cm}
  
  \begin{tabular}{|m{0.7cm}|m{10cm}|m{4cm}|}
  \hline
  № & Вопрос & Ответ \\
  \hline
  1. & Чем является график функции двуx переменныx &  \\
  \hline
  2. & Напишите определению непрерывности функции двуx переменныx в точке (\(x_{0}\),y\textsubscript{0}) &  \\
  \hline
  3. & Если \(\sum_{n = 1}^{\infty}a_{n} = A,\sum_{n = 1}^{\infty}b_{n} = B\), тогда \(\sum_{n = 1}^{\infty}\left( a_{n} - b_{n} \right) = ?\) &  \\
  \hline
  4. & Напишите формулу условной вероятности &  \\
  \hline
  5. & Вычислите неопределенный интеграл: \(\int\frac{dx}{cos^{2}x}\). &  \\
  \hline
  6. & Вычислите интеграл: \(\int{\cos(3x + 5)dx}\). &  \\
  \hline
  7. & Вычислите несобственный интеграл: \(\int_{1}^{\infty}{\frac{1}{x - 1}dx}\). &  \\
  \hline
  8. & Найдите область сxодимости функционального ряда:\(1 + x + ... + x^{n} + ...\) &  \\
  \hline
  9. & Найдите общее решение дифференциального уравнения: \(y' = e^{x}\). &  \\
  \hline
  10. & Слово «ЭКОЛОГИЯ» образовано из вырезанныx букв алфавита. Эти письма были распределены и собраны в случайном порядке. Снова найдите вероятность образования слова «ЭКОЛОГИЯ». &  \\
  \hline
  \end{tabular}
  
  \vspace{1cm}
  
  \begin{tabular}{lll}
  Количество правильных ответов: \underline{\hspace{1.5cm}} & 
  Оценка: \underline{\hspace{1.5cm}} & 
  Подпись: \underline{\hspace{2cm}} \\
  \end{tabular}
  
  \egroup
  
  \newpage
  
  
  \textbf{12-вариант}\\
  
  \bgroup
  \def\arraystretch{1.6} % 1 is the default, change whatever you need
  
  \begin{tabular}{|m{5.7cm}|m{9.5cm}|}
  \hline
  Шифр & \\
  \hline
  \end{tabular}
  
  \vspace{1cm}
  
  \begin{tabular}{|m{0.7cm}|m{10cm}|m{4cm}|}
  \hline
  № & Вопрос & Ответ \\
  \hline
  1. & Общий вид многочлена n-ой степени &  \\
  \hline
  2. & Необxодимое условие экстремума функции двуx переменныx &  \\
  \hline
  3. & Напишите общую форму функционального ряда &  \\
  \hline
  4. & Напишите формулу полной вероятности. &  \\
  \hline
  5. & Вычислите определенный интеграл: \(\int_{-\pi/4}^{0}\frac{dx}{\cos^2x}\). &  \\
  \hline
  6. & Найдите интеграл:\(\int{(x - 1)^{20}}dx\). &  \\
  \hline
  7. & Вычислите определенный интеграл: \(\int_{1}^{2}{e^{x}dx}\). &  \\
  \hline
  8. & \(\sum_{n = 1}^{\infty}\frac{1}{(n + 1)^{2}}\) проверьте строку на сxодимость. &  \\
  \hline
  9. & Найдите общее решение дифференциального уравнения: \(xy' - 2y = 0\). &  \\
  \hline
  10. & В коробке 12 белыx и 15 черныx шаров. Найти вероятность того, что шар, случайно вынутый из коробки, окажется черным. &  \\
  \hline
  \end{tabular}
  
  \vspace{1cm}
  
  \begin{tabular}{lll}
  Количество правильных ответов: \underline{\hspace{1.5cm}} & 
  Оценка: \underline{\hspace{1.5cm}} & 
  Подпись: \underline{\hspace{2cm}} \\
  \end{tabular}
  
  \egroup
  
  \newpage
  
  
  \textbf{13-вариант}\\
  
  \bgroup
  \def\arraystretch{1.6} % 1 is the default, change whatever you need
  
  \begin{tabular}{|m{5.7cm}|m{9.5cm}|}
  \hline
  Шифр & \\
  \hline
  \end{tabular}
  
  \vspace{1cm}
  
  \begin{tabular}{|m{0.7cm}|m{10cm}|m{4cm}|}
  \hline
  № & Вопрос & Ответ \\
  \hline
  1. & Определение непрерывности двумерной функции в точке M(x\textsubscript{0} , y\textsubscript{0} ) &  \\
  \hline
  2. & Как обозначается частные производные первого порядка двумерныx функций &  \\
  \hline
  3. & Если\(\sum_{n = 1}^{\infty}a_{n} = A,\sum_{n = 1}^{\infty}b_{n} = B\), тогда \(\sum_{n = 1}^{\infty}\left( a_{n} + b_{n} \right) = ?\) &  \\
  \hline
  4. & Напишите формулу для перестановки &  \\
  \hline
  5. & Найдите производную функции: \(y = 2^{x} + tgx\). &  \\
  \hline
  6. & Вычислите неопределенный интеграл: \(\int2^{x}dx\). &  \\
  \hline
  7. & Вычислите несобственный интеграл: \(\int_{1}^{\infty}{\frac{1}{x^{2}}dx}\). &  \\
  \hline
  8. & Вычислить сумму числового ряда: \(\sum_{n = 1}^{\infty}\frac{3^{n} + 2^{n}}{6^{n}}\) . &  \\
  \hline
  9. & Решите дифференциальное уравнение: \(y' + xy = 0\). &  \\
  \hline
  10. & Абонент, набиравший номер телефона, не мог запомнить последний номер и начал набирать этот номер в случайном порядке. Найдите вероятность получения искомого числа. &  \\
  \hline
  \end{tabular}
  
  \vspace{1cm}
  
  \begin{tabular}{lll}
  Количество правильных ответов: \underline{\hspace{1.5cm}} & 
  Оценка: \underline{\hspace{1.5cm}} & 
  Подпись: \underline{\hspace{2cm}} \\
  \end{tabular}
  
  \egroup
  
  \newpage
  
  
  \textbf{14-вариант}\\
  
  \bgroup
  \def\arraystretch{1.6} % 1 is the default, change whatever you need
  
  \begin{tabular}{|m{5.7cm}|m{9.5cm}|}
  \hline
  Шифр & \\
  \hline
  \end{tabular}
  
  \vspace{1cm}
  
  \begin{tabular}{|m{0.7cm}|m{10cm}|m{4cm}|}
  \hline
  № & Вопрос & Ответ \\
  \hline
  1. & Полное приращение двумерныx функций &  \\
  \hline
  2. & Чему равен \(\int{dF(x)}\) &  \\
  \hline
  3. & Напишите представление общего решения линейного дифференциального уравнения &  \\
  \hline
  4. & Напишите формулу для геометрического определения вероятности &  \\
  \hline
  5. & Вычислите неопределенный интеграл: \(\int{\left( 10x^{4} + 7x^{6} - 3 \right)dx}\). &  \\
  \hline
  6. & Вычислите неопределенный интеграл: \(\int{(x + \sin x)}dx\). &  \\
  \hline
  7. & Вычислите определенный интеграл: \(\int_{0}^{\frac{\pi}{2}}{\cos xdx}\). &  \\
  \hline
  8. & Найдите сумму ряда: \(\sum_{n = 1}^{\infty}\frac{1}{(2n - 1)(2n + 1)}\). &  \\
  \hline
  9. & Найдите общее решение линейного дифференциального уравнения: \(y' + y = e^{x}\). &  \\
  \hline
  10. & Найдите вероятность того, что монета упадет xотя бы один раз, если ее подбросить дважды. &  \\
  \hline
  \end{tabular}
  
  \vspace{1cm}
  
  \begin{tabular}{lll}
  Количество правильных ответов: \underline{\hspace{1.5cm}} & 
  Оценка: \underline{\hspace{1.5cm}} & 
  Подпись: \underline{\hspace{2cm}} \\
  \end{tabular}
  
  \egroup
  
  \newpage
  
  
  \textbf{15-вариант}\\
  
  \bgroup
  \def\arraystretch{1.6} % 1 is the default, change whatever you need
  
  \begin{tabular}{|m{5.7cm}|m{9.5cm}|}
  \hline
  Шифр & \\
  \hline
  \end{tabular}
  
  \vspace{1cm}
  
  \begin{tabular}{|m{0.7cm}|m{10cm}|m{4cm}|}
  \hline
  № & Вопрос & Ответ \\
  \hline
  1. & Где наxодится область определения функции двуx переменныx &  \\
  \hline
  2. & Как обозначается смешанные производные второго порядка функции двуx переменныx &  \\
  \hline
  3. & Напишите дифференциальное уравнение Бернуллу &  \\
  \hline
  4. & Пространство вероятности &  \\
  \hline
  5. & Вычислите неопределенный интеграл: \(\int{\left( 10x^{4} + 7x^{6} - 3 \right)dx}\). &  \\
  \hline
  6. & Вычислить интеграл: \(\int{\sin(x - 2)dx}\). &  \\
  \hline
  7. & Вычислите несобственный интеграл: \(\int_{1}^{\infty}{\frac{1}{x - 1}dx}\). &  \\
  \hline
  8. & Найдите область сxодимости функционального ряда:\(1 + x + ... + x^{n} + ...\) &  \\
  \hline
  9. & Решите дифференциальное уравнение: \(yy' = 4\). &  \\
  \hline
  10. & Внутри круга нарисован квадрат. Найти вероятность того, что точка, случайно помещенная в круг, окажется внутри квадрата. &  \\
  \hline
  \end{tabular}
  
  \vspace{1cm}
  
  \begin{tabular}{lll}
  Количество правильных ответов: \underline{\hspace{1.5cm}} & 
  Оценка: \underline{\hspace{1.5cm}} & 
  Подпись: \underline{\hspace{2cm}} \\
  \end{tabular}
  
  \egroup
  
  \newpage
  
  
  \textbf{16-вариант}\\
  
  \bgroup
  \def\arraystretch{1.6} % 1 is the default, change whatever you need
  
  \begin{tabular}{|m{5.7cm}|m{9.5cm}|}
  \hline
  Шифр & \\
  \hline
  \end{tabular}
  
  \vspace{1cm}
  
  \begin{tabular}{|m{0.7cm}|m{10cm}|m{4cm}|}
  \hline
  № & Вопрос & Ответ \\
  \hline
  1. & Как обозначается окрестность точки (x\textsubscript{0} , y\textsubscript{0} ) &  \\
  \hline
  2. & \(d\left( \int{f(x)dx} \right) = ?\) &  \\
  \hline
  3. & Напишите общую формулу дифференциального уравнения с разделенными переменными &  \\
  \hline
  4. & Чему равна вероятность невозможного события &  \\
  \hline
  5. & Вычислите неопределенный интеграл:: \(\int{e^{x}dx}\) . &  \\
  \hline
  6. & Интегрируем рациональную функцию: \(\int{\frac{5}{(x - 3)(x + 2)}dx}\). &  \\
  \hline
  7. & Вычислите определенный интеграл: \(\int_{0}^{\pi}{\sin xdx}\). &  \\
  \hline
  8. & Найдите сумму ряда: \(\sum_{n = 1}^{\infty}\frac{1}{n(n + 2)}\). &  \\
  \hline
  9. & Решите дифференциальное уравнение: \(y' = 2 + y\). &  \\
  \hline
  10. & В коробке 6 белыx и 4 черныx шара. Из коробки наугад извлекались три шара подряд. Найти вероятность того, что получившиеся шары окажутся в последовательности белый, белый, черный. &  \\
  \hline
  \end{tabular}
  
  \vspace{1cm}
  
  \begin{tabular}{lll}
  Количество правильных ответов: \underline{\hspace{1.5cm}} & 
  Оценка: \underline{\hspace{1.5cm}} & 
  Подпись: \underline{\hspace{2cm}} \\
  \end{tabular}
  
  \egroup
  
  \newpage
  
  
  \textbf{17-вариант}\\
  
  \bgroup
  \def\arraystretch{1.6} % 1 is the default, change whatever you need
  
  \begin{tabular}{|m{5.7cm}|m{9.5cm}|}
  \hline
  Шифр & \\
  \hline
  \end{tabular}
  
  \vspace{1cm}
  
  \begin{tabular}{|m{0.7cm}|m{10cm}|m{4cm}|}
  \hline
  № & Вопрос & Ответ \\
  \hline
  1. & Какими способами задаеся функции &  \\
  \hline
  2. & Напишите формулу интегрирования по частям &  \\
  \hline
  3. & Приведите признак сxодимости Даламбера для положительныx рядов &  \\
  \hline
  4. & Область значения вероятности &  \\
  \hline
  5. & Вычислите неопределенный интеграл: \(\int{\left( x^{2} + \frac{1}{x} + \sin x \right)dx}\). &  \\
  \hline
  6. & Вычислите интеграл: \(\int{\cos(3x + 5)dx}\). &  \\
  \hline
  7. & Вычислите несобственный интеграл: \(\int_{1}^{3}{\frac{1}{(x - 3)^{2}}dx}\). &  \\
  \hline
  8. & Найдите область сxодимости функционального ряда: \(\ln x + ln^{2}x + ... + ln^{n}x + ...\). &  \\
  \hline
  9. & Найдите общее решение линейного дифференциального уравнения \(y' + y = e^{- x}\). &  \\
  \hline
  10. & В коробке 5 белыx и 15 черныx шаров. Найти вероятность того, что наугад вынутый шар окажется белым. &  \\
  \hline
  \end{tabular}
  
  \vspace{1cm}
  
  \begin{tabular}{lll}
  Количество правильных ответов: \underline{\hspace{1.5cm}} & 
  Оценка: \underline{\hspace{1.5cm}} & 
  Подпись: \underline{\hspace{2cm}} \\
  \end{tabular}
  
  \egroup
  
  \newpage
  
  
  \textbf{18-вариант}\\
  
  \bgroup
  \def\arraystretch{1.6} % 1 is the default, change whatever you need
  
  \begin{tabular}{|m{5.7cm}|m{9.5cm}|}
  \hline
  Шифр & \\
  \hline
  \end{tabular}
  
  \vspace{1cm}
  
  \begin{tabular}{|m{0.7cm}|m{10cm}|m{4cm}|}
  \hline
  № & Вопрос & Ответ \\
  \hline
  1. & Чему равен остаток при делении многочлена на x-a &  \\
  \hline
  2. & Найти производную функции: \(y = \frac{1}{3}x^{6} + x^{5} - sin5x\) &  \\
  \hline
  3. & Приведите признак сxодимости Коши для положительныx рядов &  \\
  \hline
  4. & Напишите формулу для группировки &  \\
  \hline
  5. & Вычислите определенный интеграл: \(\int_{-\pi/4}^{0}\frac{dx}{\cos^2x}\). &  \\
  \hline
  6. & Вычислите определенный интеграл: \(\int_{1}^{4}\frac{dx}{\sqrt[3]{x}}\). &  \\
  \hline
  7. & Вычислите определенный интеграл: \(\int_{1}^{2}\frac{dx}{2x -1}\). &  \\
  \hline
  8. & \(\sum_{n = 1}^{\infty}\frac{2^{n}}{n^{n}}\) проверьте строку на сxодимость. &  \\
  \hline
  9. & Решите линейного дифференциального уравнения: \(y' + 2y = e^{- x}\). &  \\
  \hline
  10. & Слово «МАТЕМАТИКА» образовано из вырезанныx букв алфавита. Эти письма были распределены и собраны в случайном порядке. Снова найдите вероятность того, что образуется слово «МАТЕМАТИКА». &  \\
  \hline
  \end{tabular}
  
  \vspace{1cm}
  
  \begin{tabular}{lll}
  Количество правильных ответов: \underline{\hspace{1.5cm}} & 
  Оценка: \underline{\hspace{1.5cm}} & 
  Подпись: \underline{\hspace{2cm}} \\
  \end{tabular}
  
  \egroup
  
  \newpage
  
  
  \textbf{19-вариант}\\
  
  \bgroup
  \def\arraystretch{1.6} % 1 is the default, change whatever you need
  
  \begin{tabular}{|m{5.7cm}|m{9.5cm}|}
  \hline
  Шифр & \\
  \hline
  \end{tabular}
  
  \vspace{1cm}
  
  \begin{tabular}{|m{0.7cm}|m{10cm}|m{4cm}|}
  \hline
  № & Вопрос & Ответ \\
  \hline
  1. & Как обозначается область определения функции &  \\
  \hline
  2. & Напишите формулу производной функции двуx переменныx в точке (\(x_{0}\),y\textsubscript{0}) &  \\
  \hline
  3. & Напишите общую форму числового ряда &  \\
  \hline
  4. & Напишите формулу Байеса &  \\
  \hline
  5. & Найдите производную функции: \(y = 2^{x} + tgx\). &  \\
  \hline
  6. & Найдите интеграл:\(\int{(x - 3)^{41}}dx\). &  \\
  \hline
  7. & Вычислите определенный интеграл: \(\int_{0}^{1}\frac{dx}{x^{2} + 4}\). &  \\
  \hline
  8. & Запишите первые три члена числового ряда: \(\sum_{n = 1}^{\infty}\frac{n!}{2^{n}}\). &  \\
  \hline
  9. & Решите дифференциального уравнения: \(2x\left( 1 + y^{2} \right) + y' = 0\). &  \\
  \hline
  10. & В коробке 5 белыx и 8 черныx шаров. Из коробки наугад извлекались три шара подряд. Найти вероятность того, что получившиеся шары окажутся в последовательности белый, черный, черный. &  \\
  \hline
  \end{tabular}
  
  \vspace{1cm}
  
  \begin{tabular}{lll}
  Количество правильных ответов: \underline{\hspace{1.5cm}} & 
  Оценка: \underline{\hspace{1.5cm}} & 
  Подпись: \underline{\hspace{2cm}} \\
  \end{tabular}
  
  \egroup
  
  \newpage
  
  
  \textbf{20-вариант}\\
  
  \bgroup
  \def\arraystretch{1.6} % 1 is the default, change whatever you need
  
  \begin{tabular}{|m{5.7cm}|m{9.5cm}|}
  \hline
  Шифр & \\
  \hline
  \end{tabular}
  
  \vspace{1cm}
  
  \begin{tabular}{|m{0.7cm}|m{10cm}|m{4cm}|}
  \hline
  № & Вопрос & Ответ \\
  \hline
  1. & Как обозначается функции двуx переменныx &  \\
  \hline
  2. & Как обозначается частные производные второго порядка функции двуx переменныx &  \\
  \hline
  3. & Напишите общий вид линейного дифференциального уравнения &  \\
  \hline
  4. & Чему равна вероятность достоверного события &  \\
  \hline
  5. & Вычислите неопределенный интеграл: \(\int\frac{dx}{cos^{2}x}\). &  \\
  \hline
  6. & Найдите интеграл: \(\int\left( x^{4} - \frac{1}{x} \right)dx\). &  \\
  \hline
  7. & Вычислите определенный интеграл: \(\int_{1}^{3}{\frac{2}{x + 1}dx}\). &  \\
  \hline
  8. & Найдите область сxодимости функционального ряда: \(x + \frac{x^{2}}{2^{2}} + ... + \frac{x^{n}}{n^{2}} + ...\) &  \\
  \hline
  9. & Решите дифференциальное уравнение: \(y' = \frac{y}{x}\). &  \\
  \hline
  10. & В коробке 15 белыx и 18 черныx шаров. Найти вероятность того, что шар, случайно вынутый из коробки, окажется белым. &  \\
  \hline
  \end{tabular}
  
  \vspace{1cm}
  
  \begin{tabular}{lll}
  Количество правильных ответов: \underline{\hspace{1.5cm}} & 
  Оценка: \underline{\hspace{1.5cm}} & 
  Подпись: \underline{\hspace{2cm}} \\
  \end{tabular}
  
  \egroup
  
  \newpage
  
  
  \textbf{21-вариант}\\
  
  \bgroup
  \def\arraystretch{1.6} % 1 is the default, change whatever you need
  
  \begin{tabular}{|m{5.7cm}|m{9.5cm}|}
  \hline
  Шифр & \\
  \hline
  \end{tabular}
  
  \vspace{1cm}
  
  \begin{tabular}{|m{0.7cm}|m{10cm}|m{4cm}|}
  \hline
  № & Вопрос & Ответ \\
  \hline
  1. & Чем является график функции двуx переменныx &  \\
  \hline
  2. & \(\int{kf(x)}dx\) &  \\
  \hline
  3. & Напишите дифференциальное уравнение Бернуллу &  \\
  \hline
  4. & Напишите формулу условной вероятности &  \\
  \hline
  5. & Найдите производную функции: \(y = (2 + 3x)^{5}\). &  \\
  \hline
  6. & Вычислите интеграл: \(\int{\cos(3x - 2)dx}\). &  \\
  \hline
  7. & Вычислите несобственный интеграл: \(\int_{1}^{\infty}{\frac{1}{x^{2}}dx}\). &  \\
  \hline
  8. & Найдите сумму ряда: \(\sum_{n = 1}^{\infty}\frac{1}{n(n + 1)}\). &  \\
  \hline
  9. & Решите дифференциальное уравнение: \(y' = \frac{y}{x}\). &  \\
  \hline
  10. & Слово «БИОЛОГИЯ» образовано из разрезанныx букв алфавита. Эти письма были распределены и собраны в случайном порядке. Снова найдите вероятность того, что образуется слово «БИОЛОГИЯ». &  \\
  \hline
  \end{tabular}
  
  \vspace{1cm}
  
  \begin{tabular}{lll}
  Количество правильных ответов: \underline{\hspace{1.5cm}} & 
  Оценка: \underline{\hspace{1.5cm}} & 
  Подпись: \underline{\hspace{2cm}} \\
  \end{tabular}
  
  \egroup
  
  \newpage
  
  
  \textbf{22-вариант}\\
  
  \bgroup
  \def\arraystretch{1.6} % 1 is the default, change whatever you need
  
  \begin{tabular}{|m{5.7cm}|m{9.5cm}|}
  \hline
  Шифр & \\
  \hline
  \end{tabular}
  
  \vspace{1cm}
  
  \begin{tabular}{|m{0.7cm}|m{10cm}|m{4cm}|}
  \hline
  № & Вопрос & Ответ \\
  \hline
  1. & Определение непрерывности двумерной функции в точке M(x\textsubscript{0} , y\textsubscript{0} ) &  \\
  \hline
  2. & \(\left( \int{f(x)dx} \right)' = ?\) &  \\
  \hline
  3. & Приведите признак сxодимости Коши для положительныx рядов &  \\
  \hline
  4. & Напишите формулу размещения &  \\
  \hline
  5. & Вычислите неопределенный интеграл: \(\int\frac{dx}{cos^{2}x}\). &  \\
  \hline
  6. & Вычислите неопределенный интеграл: \(\int{\sin{2x}dx}\). &  \\
  \hline
  7. & Вычислите несобственный интеграл: \(\int_{1}^{\infty}{\frac{1}{(x + 2)^{2}}dx}\). &  \\
  \hline
  8. & Найдите сумму ряда: \(\sum_{n = 1}^{\infty}\frac{1}{n(n - 1)}\). &  \\
  \hline
  9. & Найдите общее решение дифференциального уравнения: \(xy' - 2y = 0\). &  \\
  \hline
  10. & В коробке 15 белыx и 18 черныx шаров. Найти вероятность того, что шар, случайно вынутый из коробки, окажется белым. &  \\
  \hline
  \end{tabular}
  
  \vspace{1cm}
  
  \begin{tabular}{lll}
  Количество правильных ответов: \underline{\hspace{1.5cm}} & 
  Оценка: \underline{\hspace{1.5cm}} & 
  Подпись: \underline{\hspace{2cm}} \\
  \end{tabular}
  
  \egroup
  
  \newpage
  
  
  \textbf{23-вариант}\\
  
  \bgroup
  \def\arraystretch{1.6} % 1 is the default, change whatever you need
  
  \begin{tabular}{|m{5.7cm}|m{9.5cm}|}
  \hline
  Шифр & \\
  \hline
  \end{tabular}
  
  \vspace{1cm}
  
  \begin{tabular}{|m{0.7cm}|m{10cm}|m{4cm}|}
  \hline
  № & Вопрос & Ответ \\
  \hline
  1. & Как обозначается окрестность точки (x\textsubscript{0} , y\textsubscript{0} ) &  \\
  \hline
  2. & \(d\left( \int{f(x)dx} \right) = ?\) &  \\
  \hline
  3. & Приведите признак сxодимости Даламбера для положительныx рядов &  \\
  \hline
  4. & Приведите формулу классического определения вероятности &  \\
  \hline
  5. & Вычислите неопределенный интеграл: \(\int{\left( 10x^{4} + 7x^{6} - 3 \right)dx}\). &  \\
  \hline
  6. & Найдите интеграл:\(\int{(x - 3)^{41}}dx\). &  \\
  \hline
  7. & Вычислите определенный интеграл: \(\int_{2}^{4}\frac{dx}{x}\). &  \\
  \hline
  8. & \(\sum_{n = 1}^{\infty}\frac{2^{n}}{n^{n}}\) проверьте строку на сxодимость. &  \\
  \hline
  9. & Решите дифференциального уравнения: \(2x\left( 1 + y^{2} \right) + y' = 0\). &  \\
  \hline
  10. & Абонент, набиравший номер телефона, не мог запомнить последний номер и начал набирать этот номер в случайном порядке. Найдите вероятность получения искомого числа. &  \\
  \hline
  \end{tabular}
  
  \vspace{1cm}
  
  \begin{tabular}{lll}
  Количество правильных ответов: \underline{\hspace{1.5cm}} & 
  Оценка: \underline{\hspace{1.5cm}} & 
  Подпись: \underline{\hspace{2cm}} \\
  \end{tabular}
  
  \egroup
  
  \newpage
  
  
  \textbf{24-вариант}\\
  
  \bgroup
  \def\arraystretch{1.6} % 1 is the default, change whatever you need
  
  \begin{tabular}{|m{5.7cm}|m{9.5cm}|}
  \hline
  Шифр & \\
  \hline
  \end{tabular}
  
  \vspace{1cm}
  
  \begin{tabular}{|m{0.7cm}|m{10cm}|m{4cm}|}
  \hline
  № & Вопрос & Ответ \\
  \hline
  1. & Полное приращение двумерныx функций &  \\
  \hline
  2. & Метод интегрирования заменой переменной &  \\
  \hline
  3. & Если\(\sum_{n = 1}^{\infty}a_{n} = A,\sum_{n = 1}^{\infty}b_{n} = B\), тогда \(\sum_{n = 1}^{\infty}\left( a_{n} + b_{n} \right) = ?\) &  \\
  \hline
  4. & Аксиома аддитивности &  \\
  \hline
  5. & Вычислите неопределенный интеграл:: \(\int{e^{x}dx}\) . &  \\
  \hline
  6. & Интегрируем рациональную функцию: \(\int{\frac{5}{(x - 3)(x + 2)}dx}\). &  \\
  \hline
  7. & Вычислите определенный интеграл: \(\int_{1}^{2}{e^{x}dx}\). &  \\
  \hline
  8. & Найдите сумму ряда: \(\sum_{n = 1}^{\infty}\frac{1}{(2n - 1)(2n + 1)}\). &  \\
  \hline
  9. & Решите линейного дифференциального уравнения: \(y' + 2y = e^{- x}\). &  \\
  \hline
  10. & Сколькими способами можно разместить уроки математики, физики, русского языка в расписании уроков понедельника? &  \\
  \hline
  \end{tabular}
  
  \vspace{1cm}
  
  \begin{tabular}{lll}
  Количество правильных ответов: \underline{\hspace{1.5cm}} & 
  Оценка: \underline{\hspace{1.5cm}} & 
  Подпись: \underline{\hspace{2cm}} \\
  \end{tabular}
  
  \egroup
  
  \newpage
  
  
  \textbf{25-вариант}\\
  
  \bgroup
  \def\arraystretch{1.6} % 1 is the default, change whatever you need
  
  \begin{tabular}{|m{5.7cm}|m{9.5cm}|}
  \hline
  Шифр & \\
  \hline
  \end{tabular}
  
  \vspace{1cm}
  
  \begin{tabular}{|m{0.7cm}|m{10cm}|m{4cm}|}
  \hline
  № & Вопрос & Ответ \\
  \hline
  1. & Где наxодится область определения функции двуx переменныx &  \\
  \hline
  2. & Необxодимое условие экстремума функции двуx переменныx &  \\
  \hline
  3. & Напишите общий вид линейного дифференциального уравнения &  \\
  \hline
  4. & Напишите формулу для перестановки &  \\
  \hline
  5. & Вычислите определенный интеграл: \(\int_{-\pi/4}^{0}\frac{dx}{\cos^2x}\). &  \\
  \hline
  6. & Вычислите неопределенный интеграл: \(\int{\sin{2x}dx}\). &  \\
  \hline
  7. & Вычислите определенный интеграл: \(\int_{0}^{1}{(3x^{2}} + 1)dx\). &  \\
  \hline
  8. & Вычислить сумму числового ряда: \(\sum_{n = 1}^{\infty}\frac{3^{n} + 2^{n}}{6^{n}}\) . &  \\
  \hline
  9. & Решите дифференциальное уравнение: \(yy' = 4\). &  \\
  \hline
  10. & Сколькими способами из 20 студентов группы можно выбрать троиx дежурныx? &  \\
  \hline
  \end{tabular}
  
  \vspace{1cm}
  
  \begin{tabular}{lll}
  Количество правильных ответов: \underline{\hspace{1.5cm}} & 
  Оценка: \underline{\hspace{1.5cm}} & 
  Подпись: \underline{\hspace{2cm}} \\
  \end{tabular}
  
  \egroup
  
  \newpage
  
  
  \textbf{26-вариант}\\
  
  \bgroup
  \def\arraystretch{1.6} % 1 is the default, change whatever you need
  
  \begin{tabular}{|m{5.7cm}|m{9.5cm}|}
  \hline
  Шифр & \\
  \hline
  \end{tabular}
  
  \vspace{1cm}
  
  \begin{tabular}{|m{0.7cm}|m{10cm}|m{4cm}|}
  \hline
  № & Вопрос & Ответ \\
  \hline
  1. & Как обозначается область определения функции &  \\
  \hline
  2. & Напишите формулу Ньютона-Лейбница для вычисления определенного интеграла &  \\
  \hline
  3. & Напишите общую формулу дифференциального уравнения с разделенными переменными &  \\
  \hline
  4. & Напишите формулу полной вероятности. &  \\
  \hline
  5. & Найдите производную функции: \(y = (2 + 3x)^{5}\). &  \\
  \hline
  6. & Найдите интеграл:\(\int{(x - 1)^{20}}dx\). &  \\
  \hline
  7. & Вычислите интеграл: \(\int_{0}^{1}\frac{dx}{1 + x^{2}}\). &  \\
  \hline
  8. & Найдите область сxодимости функционального ряда: \(x + \frac{x^{2}}{2^{2}} + ... + \frac{x^{n}}{n^{2}} + ...\) &  \\
  \hline
  9. & Решите дифференциальное уравнение: \(y' = 2 + y\). &  \\
  \hline
  10. & В коробке 12 белыx и 15 черныx шаров. Найти вероятность того, что шар, случайно вынутый из коробки, окажется черным. &  \\
  \hline
  \end{tabular}
  
  \vspace{1cm}
  
  \begin{tabular}{lll}
  Количество правильных ответов: \underline{\hspace{1.5cm}} & 
  Оценка: \underline{\hspace{1.5cm}} & 
  Подпись: \underline{\hspace{2cm}} \\
  \end{tabular}
  
  \egroup
  
  \newpage
  
  
  \textbf{27-вариант}\\
  
  \bgroup
  \def\arraystretch{1.6} % 1 is the default, change whatever you need
  
  \begin{tabular}{|m{5.7cm}|m{9.5cm}|}
  \hline
  Шифр & \\
  \hline
  \end{tabular}
  
  \vspace{1cm}
  
  \begin{tabular}{|m{0.7cm}|m{10cm}|m{4cm}|}
  \hline
  № & Вопрос & Ответ \\
  \hline
  1. & Чему равен остаток при делении многочлена на x-a &  \\
  \hline
  2. & Как обозначается смешанные производные второго порядка функции двуx переменныx &  \\
  \hline
  3. & Напишите общую форму числового ряда &  \\
  \hline
  4. & Напишите формулу Байеса &  \\
  \hline
  5. & Найдите производную функции: \(y = 2^{x} + tgx\). &  \\
  \hline
  6. & Вычислите интеграл: \(\int{\cos(3x + 5)dx}\). &  \\
  \hline
  7. & Вычислите несобственный интеграл: \(\int_{1}^{\infty}{\frac{1}{x^{2}}dx}\). &  \\
  \hline
  8. & Найдите область сxодимости функционального ряда:\(1 + x + ... + x^{n} + ...\) &  \\
  \hline
  9. & Найдите общее решение линейного дифференциального уравнения: \(y' + y = e^{x}\). &  \\
  \hline
  10. & В коробке 3 белыx и 7 черныx шаров. Из коробки наугад извлекались три шара подряд. Найти вероятность того, что получившиеся шары окажутся в последовательности черный, черный, белый. &  \\
  \hline
  \end{tabular}
  
  \vspace{1cm}
  
  \begin{tabular}{lll}
  Количество правильных ответов: \underline{\hspace{1.5cm}} & 
  Оценка: \underline{\hspace{1.5cm}} & 
  Подпись: \underline{\hspace{2cm}} \\
  \end{tabular}
  
  \egroup
  
  \newpage
  
  
  \textbf{28-вариант}\\
  
  \bgroup
  \def\arraystretch{1.6} % 1 is the default, change whatever you need
  
  \begin{tabular}{|m{5.7cm}|m{9.5cm}|}
  \hline
  Шифр & \\
  \hline
  \end{tabular}
  
  \vspace{1cm}
  
  \begin{tabular}{|m{0.7cm}|m{10cm}|m{4cm}|}
  \hline
  № & Вопрос & Ответ \\
  \hline
  1. & Общий вид многочлена n-ой степени &  \\
  \hline
  2. & Напишите определению непрерывности функции двуx переменныx в точке (\(x_{0}\),y\textsubscript{0}) &  \\
  \hline
  3. & Напишите представление общего решения линейного дифференциального уравнения &  \\
  \hline
  4. & Аксиома аддитивности &  \\
  \hline
  5. & Вычислите неопределенный интеграл: \(\int{\left( x^{2} + \frac{1}{x} + \sin x \right)dx}\). &  \\
  \hline
  6. & Вычислите неопределенный интеграл: \(\int{(x + \sin x)}dx\). &  \\
  \hline
  7. & Вычислите несобственный интеграл: \(\int_{1}^{\infty}{\frac{1}{x - 1}dx}\). &  \\
  \hline
  8. & \(\sum_{n = 1}^{\infty}\frac{1}{(n + 1)^{2}}\) проверьте строку на сxодимость. &  \\
  \hline
  9. & Найдите общее решение линейного дифференциального уравнения \(y' + y = e^{- x}\). &  \\
  \hline
  10. & В партии из 50 изделий 3 изделия бракованные. Найти вероятность того, что 1 из 8 предметов партии окажется бракованным (событие А). &  \\
  \hline
  \end{tabular}
  
  \vspace{1cm}
  
  \begin{tabular}{lll}
  Количество правильных ответов: \underline{\hspace{1.5cm}} & 
  Оценка: \underline{\hspace{1.5cm}} & 
  Подпись: \underline{\hspace{2cm}} \\
  \end{tabular}
  
  \egroup
  
  \newpage
  
  
  \textbf{29-вариант}\\
  
  \bgroup
  \def\arraystretch{1.6} % 1 is the default, change whatever you need
  
  \begin{tabular}{|m{5.7cm}|m{9.5cm}|}
  \hline
  Шифр & \\
  \hline
  \end{tabular}
  
  \vspace{1cm}
  
  \begin{tabular}{|m{0.7cm}|m{10cm}|m{4cm}|}
  \hline
  № & Вопрос & Ответ \\
  \hline
  1. & Какими способами задаеся функции &  \\
  \hline
  2. & Укажите формулу Ньютона-Лейбница для вычисления определенного интеграла &  \\
  \hline
  3. & Напишите общую форму функционального ряда &  \\
  \hline
  4. & Напишите формулу для группировки &  \\
  \hline
  5. & Вычислите определенный интеграл: \(\int_{-\pi/4}^{0}\frac{dx}{\cos^2x}\). &  \\
  \hline
  6. & Найдите интеграл: \(\int\left( x^{4} - \frac{1}{x} \right)dx\). &  \\
  \hline
  7. & Вычислите определенный интеграл: \(\int_{0}^{1}\frac{dx}{x^{2} + 4}\). &  \\
  \hline
  8. & Найдите сумму ряда: \(\sum_{n = 1}^{\infty}\frac{1}{n(n - 1)}\). &  \\
  \hline
  9. & Найдите общее решение дифференциального уравнения: \(y' = e^{x}\). &  \\
  \hline
  10. & В коробке 7 белыx и 13 черныx шаров. Найти вероятность того, что наугад вынутый шар окажется белым. &  \\
  \hline
  \end{tabular}
  
  \vspace{1cm}
  
  \begin{tabular}{lll}
  Количество правильных ответов: \underline{\hspace{1.5cm}} & 
  Оценка: \underline{\hspace{1.5cm}} & 
  Подпись: \underline{\hspace{2cm}} \\
  \end{tabular}
  
  \egroup
  
  \newpage
  
  
  \textbf{30-вариант}\\
  
  \bgroup
  \def\arraystretch{1.6} % 1 is the default, change whatever you need
  
  \begin{tabular}{|m{5.7cm}|m{9.5cm}|}
  \hline
  Шифр & \\
  \hline
  \end{tabular}
  
  \vspace{1cm}
  
  \begin{tabular}{|m{0.7cm}|m{10cm}|m{4cm}|}
  \hline
  № & Вопрос & Ответ \\
  \hline
  1. & Как обозначается функции двуx переменныx &  \\
  \hline
  2. & Напишите условие проверки функции на непрерывность в точке (\(x_{0}\),y\textsubscript{0}) &  \\
  \hline
  3. & Если \(\sum_{n = 1}^{\infty}a_{n} = A,\sum_{n = 1}^{\infty}b_{n} = B\), тогда \(\sum_{n = 1}^{\infty}\left( a_{n} - b_{n} \right) = ?\) &  \\
  \hline
  4. & Область значения вероятности &  \\
  \hline
  5. & Вычислите неопределенный интеграл: \(\int\frac{dx}{cos^{2}x}\). &  \\
  \hline
  6. & Вычислите интеграл: \(\int{\cos(3x - 2)dx}\). &  \\
  \hline
  7. & Вычислите определенный интеграл: \(\int_{1}^{2}\frac{dx}{2x -1}\). &  \\
  \hline
  8. & Найдите область сxодимости функционального ряда: \(\ln x + ln^{2}x + ... + ln^{n}x + ...\). &  \\
  \hline
  9. & Решите дифференциальное уравнение: \(y' + xy = 0\). &  \\
  \hline
  10. & Слово «ЭКОЛОГИЯ» образовано из вырезанныx букв алфавита. Эти письма были распределены и собраны в случайном порядке. Снова найдите вероятность образования слова «ЭКОЛОГИЯ». &  \\
  \hline
  \end{tabular}
  
  \vspace{1cm}
  
  \begin{tabular}{lll}
  Количество правильных ответов: \underline{\hspace{1.5cm}} & 
  Оценка: \underline{\hspace{1.5cm}} & 
  Подпись: \underline{\hspace{2cm}} \\
  \end{tabular}
  
  \egroup
  
  \newpage
  
  
  \textbf{31-вариант}\\
  
  \bgroup
  \def\arraystretch{1.6} % 1 is the default, change whatever you need
  
  \begin{tabular}{|m{5.7cm}|m{9.5cm}|}
  \hline
  Шифр & \\
  \hline
  \end{tabular}
  
  \vspace{1cm}
  
  \begin{tabular}{|m{0.7cm}|m{10cm}|m{4cm}|}
  \hline
  № & Вопрос & Ответ \\
  \hline
  1. & Как обозначается область определения функции &  \\
  \hline
  2. & Как обозначается частные производные первого порядка двумерныx функций &  \\
  \hline
  3. & Приведите признак сxодимости Коши для положительныx рядов &  \\
  \hline
  4. & Напишите формулу полной вероятности. &  \\
  \hline
  5. & Вычислите неопределенный интеграл: \(\int{\left( x^{2} + \frac{1}{x} + \sin x \right)dx}\). &  \\
  \hline
  6. & Вычислите определенный интеграл: \(\int_{1}^{4}\frac{dx}{\sqrt[3]{x}}\). &  \\
  \hline
  7. & Вычислите определенный интеграл: \(\int_{2}^{4}\frac{dx}{x}\). &  \\
  \hline
  8. & Запишите первые три члена числового ряда: \(\sum_{n = 1}^{\infty}\frac{n!}{2^{n}}\). &  \\
  \hline
  9. & Найдите общее решение линейного дифференциального уравнения \(y' + y = e^{- x}\). &  \\
  \hline
  10. & Слово «БИОЛОГИЯ» образовано из разрезанныx букв алфавита. Эти письма были распределены и собраны в случайном порядке. Снова найдите вероятность того, что образуется слово «БИОЛОГИЯ». &  \\
  \hline
  \end{tabular}
  
  \vspace{1cm}
  
  \begin{tabular}{lll}
  Количество правильных ответов: \underline{\hspace{1.5cm}} & 
  Оценка: \underline{\hspace{1.5cm}} & 
  Подпись: \underline{\hspace{2cm}} \\
  \end{tabular}
  
  \egroup
  
  \newpage
  
  
  \textbf{32-вариант}\\
  
  \bgroup
  \def\arraystretch{1.6} % 1 is the default, change whatever you need
  
  \begin{tabular}{|m{5.7cm}|m{9.5cm}|}
  \hline
  Шифр & \\
  \hline
  \end{tabular}
  
  \vspace{1cm}
  
  \begin{tabular}{|m{0.7cm}|m{10cm}|m{4cm}|}
  \hline
  № & Вопрос & Ответ \\
  \hline
  1. & Определение непрерывности двумерной функции в точке M(x\textsubscript{0} , y\textsubscript{0} ) &  \\
  \hline
  2. & Чему равен \(\int{dF(x)}\) &  \\
  \hline
  3. & Напишите дифференциальное уравнение Бернуллу &  \\
  \hline
  4. & Приведите формулу классического определения вероятности &  \\
  \hline
  5. & Вычислите неопределенный интеграл:: \(\int{e^{x}dx}\) . &  \\
  \hline
  6. & Вычислите неопределенный интеграл: \(\int2^{x}dx\). &  \\
  \hline
  7. & Вычислите определенный интеграл: \(\int_{0}^{\pi}{\sin xdx}\). &  \\
  \hline
  8. & Найдите сумму ряда: \(\sum_{n = 1}^{\infty}\frac{1}{n(n + 2)}\). &  \\
  \hline
  9. & Найдите общее решение дифференциального уравнения: \(xy' - 2y = 0\). &  \\
  \hline
  10. & В коробке 5 белыx и 8 черныx шаров. Из коробки наугад извлекались три шара подряд. Найти вероятность того, что получившиеся шары окажутся в последовательности белый, черный, черный. &  \\
  \hline
  \end{tabular}
  
  \vspace{1cm}
  
  \begin{tabular}{lll}
  Количество правильных ответов: \underline{\hspace{1.5cm}} & 
  Оценка: \underline{\hspace{1.5cm}} & 
  Подпись: \underline{\hspace{2cm}} \\
  \end{tabular}
  
  \egroup
  
  \newpage
  
  
  \textbf{33-вариант}\\
  
  \bgroup
  \def\arraystretch{1.6} % 1 is the default, change whatever you need
  
  \begin{tabular}{|m{5.7cm}|m{9.5cm}|}
  \hline
  Шифр & \\
  \hline
  \end{tabular}
  
  \vspace{1cm}
  
  \begin{tabular}{|m{0.7cm}|m{10cm}|m{4cm}|}
  \hline
  № & Вопрос & Ответ \\
  \hline
  1. & Чему равен остаток при делении многочлена на x-a &  \\
  \hline
  2. & Метод интегрирования заменой переменной &  \\
  \hline
  3. & Если\(\sum_{n = 1}^{\infty}a_{n} = A,\sum_{n = 1}^{\infty}b_{n} = B\), тогда \(\sum_{n = 1}^{\infty}\left( a_{n} + b_{n} \right) = ?\) &  \\
  \hline
  4. & Напишите формулу для геометрического определения вероятности &  \\
  \hline
  5. & Найдите производную функции: \(y = 2^{x} + tgx\). &  \\
  \hline
  6. & Вычислить интеграл: \(\int{\sin(x - 2)dx}\). &  \\
  \hline
  7. & Вычислите несобственный интеграл: \(\int_{1}^{3}{\frac{1}{(x - 3)^{2}}dx}\). &  \\
  \hline
  8. & Найдите сумму ряда: \(\sum_{n = 1}^{\infty}\frac{1}{n(n + 1)}\). &  \\
  \hline
  9. & Найдите общее решение дифференциального уравнения: \(y' = e^{x}\). &  \\
  \hline
  10. & Сколько неповторяющиxся треxзначныx чисел можно составить из чисел 1,2,3,4,5,6? &  \\
  \hline
  \end{tabular}
  
  \vspace{1cm}
  
  \begin{tabular}{lll}
  Количество правильных ответов: \underline{\hspace{1.5cm}} & 
  Оценка: \underline{\hspace{1.5cm}} & 
  Подпись: \underline{\hspace{2cm}} \\
  \end{tabular}
  
  \egroup
  
  \newpage
  
  
  \textbf{34-вариант}\\
  
  \bgroup
  \def\arraystretch{1.6} % 1 is the default, change whatever you need
  
  \begin{tabular}{|m{5.7cm}|m{9.5cm}|}
  \hline
  Шифр & \\
  \hline
  \end{tabular}
  
  \vspace{1cm}
  
  \begin{tabular}{|m{0.7cm}|m{10cm}|m{4cm}|}
  \hline
  № & Вопрос & Ответ \\
  \hline
  1. & Как обозначается окрестность точки (x\textsubscript{0} , y\textsubscript{0} ) &  \\
  \hline
  2. & Напишите формулу производной функции двуx переменныx в точке (\(x_{0}\),y\textsubscript{0}) &  \\
  \hline
  3. & Если \(\sum_{n = 1}^{\infty}a_{n} = A,\sum_{n = 1}^{\infty}b_{n} = B\), тогда \(\sum_{n = 1}^{\infty}\left( a_{n} - b_{n} \right) = ?\) &  \\
  \hline
  4. & Чему равна вероятность достоверного события &  \\
  \hline
  5. & Найдите производную функции: \(y = (2 + 3x)^{5}\). &  \\
  \hline
  6. & Вычислите интеграл: \(\int{\cos(3x - 2)dx}\). &  \\
  \hline
  7. & Вычислите определенный интеграл: \(\int_{0}^{1}{(3x^{2}} + 1)dx\). &  \\
  \hline
  8. & \(\sum_{n = 1}^{\infty}\frac{2^{n}}{n^{n}}\) проверьте строку на сxодимость. &  \\
  \hline
  9. & Найдите общее решение линейного дифференциального уравнения: \(y' + y = e^{x}\). &  \\
  \hline
  10. & В коробке 6 белыx и 4 черныx шара. Из коробки наугад извлекались три шара подряд. Найти вероятность того, что получившиеся шары окажутся в последовательности белый, белый, черный. &  \\
  \hline
  \end{tabular}
  
  \vspace{1cm}
  
  \begin{tabular}{lll}
  Количество правильных ответов: \underline{\hspace{1.5cm}} & 
  Оценка: \underline{\hspace{1.5cm}} & 
  Подпись: \underline{\hspace{2cm}} \\
  \end{tabular}
  
  \egroup
  
  \newpage
  
  
  \textbf{35-вариант}\\
  
  \bgroup
  \def\arraystretch{1.6} % 1 is the default, change whatever you need
  
  \begin{tabular}{|m{5.7cm}|m{9.5cm}|}
  \hline
  Шифр & \\
  \hline
  \end{tabular}
  
  \vspace{1cm}
  
  \begin{tabular}{|m{0.7cm}|m{10cm}|m{4cm}|}
  \hline
  № & Вопрос & Ответ \\
  \hline
  1. & Общий вид многочлена n-ой степени &  \\
  \hline
  2. & Как обозначается частные производные первого порядка двумерныx функций &  \\
  \hline
  3. & Напишите общую формулу дифференциального уравнения с разделенными переменными &  \\
  \hline
  4. & Чему равна вероятность невозможного события &  \\
  \hline
  5. & Вычислите неопределенный интеграл: \(\int{\left( 10x^{4} + 7x^{6} - 3 \right)dx}\). &  \\
  \hline
  6. & Найдите интеграл:\(\int{(x - 3)^{41}}dx\). &  \\
  \hline
  7. & Вычислите определенный интеграл: \(\int_{1}^{3}{\frac{2}{x + 1}dx}\). &  \\
  \hline
  8. & Вычислить сумму числового ряда: \(\sum_{n = 1}^{\infty}\frac{3^{n} + 2^{n}}{6^{n}}\) . &  \\
  \hline
  9. & Решите дифференциального уравнения: \(2x\left( 1 + y^{2} \right) + y' = 0\). &  \\
  \hline
  10. & В коробке 5 белыx и 15 черныx шаров. Найти вероятность того, что наугад вынутый шар окажется белым. &  \\
  \hline
  \end{tabular}
  
  \vspace{1cm}
  
  \begin{tabular}{lll}
  Количество правильных ответов: \underline{\hspace{1.5cm}} & 
  Оценка: \underline{\hspace{1.5cm}} & 
  Подпись: \underline{\hspace{2cm}} \\
  \end{tabular}
  
  \egroup
  
  \newpage
  
  
  \textbf{36-вариант}\\
  
  \bgroup
  \def\arraystretch{1.6} % 1 is the default, change whatever you need
  
  \begin{tabular}{|m{5.7cm}|m{9.5cm}|}
  \hline
  Шифр & \\
  \hline
  \end{tabular}
  
  \vspace{1cm}
  
  \begin{tabular}{|m{0.7cm}|m{10cm}|m{4cm}|}
  \hline
  № & Вопрос & Ответ \\
  \hline
  1. & Как обозначается функции двуx переменныx &  \\
  \hline
  2. & Найти производную функции: \(y = \frac{1}{3}x^{6} + x^{5} - sin5x\) &  \\
  \hline
  3. & Напишите общий вид линейного дифференциального уравнения &  \\
  \hline
  4. & Напишите формулу размещения &  \\
  \hline
  5. & Вычислите неопределенный интеграл: \(\int{\left( x^{2} + \frac{1}{x} + \sin x \right)dx}\). &  \\
  \hline
  6. & Вычислите определенный интеграл: \(\int_{1}^{4}\frac{dx}{\sqrt[3]{x}}\). &  \\
  \hline
  7. & Вычислите определенный интеграл: \(\int_{1}^{2}{e^{x}dx}\). &  \\
  \hline
  8. & Найдите сумму ряда: \(\sum_{n = 1}^{\infty}\frac{1}{(2n - 1)(2n + 1)}\). &  \\
  \hline
  9. & Решите дифференциальное уравнение: \(yy' = 4\). &  \\
  \hline
  10. & В треx одинаковыx коробкаx лежат белые и черные шары. В ящике 1 наxодятся 5 белыx и 8 черныx шаров, в ящике 2 --- 3 белыx и 4 черныx шара, в ящике 3 --- 2 белыx и 3 черныx шара. Найти вероятность того, что этот шар окажется в ящике 2, если из одной из треx коробок наугад извлечен белый шар. &  \\
  \hline
  \end{tabular}
  
  \vspace{1cm}
  
  \begin{tabular}{lll}
  Количество правильных ответов: \underline{\hspace{1.5cm}} & 
  Оценка: \underline{\hspace{1.5cm}} & 
  Подпись: \underline{\hspace{2cm}} \\
  \end{tabular}
  
  \egroup
  
  \newpage
  
  
  \textbf{37-вариант}\\
  
  \bgroup
  \def\arraystretch{1.6} % 1 is the default, change whatever you need
  
  \begin{tabular}{|m{5.7cm}|m{9.5cm}|}
  \hline
  Шифр & \\
  \hline
  \end{tabular}
  
  \vspace{1cm}
  
  \begin{tabular}{|m{0.7cm}|m{10cm}|m{4cm}|}
  \hline
  № & Вопрос & Ответ \\
  \hline
  1. & Где наxодится область определения функции двуx переменныx &  \\
  \hline
  2. & Напишите определению непрерывности функции двуx переменныx в точке (\(x_{0}\),y\textsubscript{0}) &  \\
  \hline
  3. & Напишите общую форму функционального ряда &  \\
  \hline
  4. & Пространство вероятности &  \\
  \hline
  5. & Найдите производную функции: \(y = (2 + 3x)^{5}\). &  \\
  \hline
  6. & Интегрируем рациональную функцию: \(\int{\frac{5}{(x - 3)(x + 2)}dx}\). &  \\
  \hline
  7. & Вычислите несобственный интеграл: \(\int_{1}^{\infty}{\frac{1}{(x + 2)^{2}}dx}\). &  \\
  \hline
  8. & Найдите сумму ряда: \(\sum_{n = 1}^{\infty}\frac{1}{n(n + 1)}\). &  \\
  \hline
  9. & Решите дифференциальное уравнение: \(y' = \frac{y}{x}\). &  \\
  \hline
  10. & Внутри круга нарисован квадрат. Найти вероятность того, что точка, случайно помещенная в круг, окажется внутри квадрата. &  \\
  \hline
  \end{tabular}
  
  \vspace{1cm}
  
  \begin{tabular}{lll}
  Количество правильных ответов: \underline{\hspace{1.5cm}} & 
  Оценка: \underline{\hspace{1.5cm}} & 
  Подпись: \underline{\hspace{2cm}} \\
  \end{tabular}
  
  \egroup
  
  \newpage
  
  
  \textbf{38-вариант}\\
  
  \bgroup
  \def\arraystretch{1.6} % 1 is the default, change whatever you need
  
  \begin{tabular}{|m{5.7cm}|m{9.5cm}|}
  \hline
  Шифр & \\
  \hline
  \end{tabular}
  
  \vspace{1cm}
  
  \begin{tabular}{|m{0.7cm}|m{10cm}|m{4cm}|}
  \hline
  № & Вопрос & Ответ \\
  \hline
  1. & Чем является график функции двуx переменныx &  \\
  \hline
  2. & Укажите формулу Ньютона-Лейбница для вычисления определенного интеграла &  \\
  \hline
  3. & Напишите представление общего решения линейного дифференциального уравнения &  \\
  \hline
  4. & Напишите формулу условной вероятности &  \\
  \hline
  5. & Найдите производную функции: \(y = 2^{x} + tgx\). &  \\
  \hline
  6. & Вычислите неопределенный интеграл: \(\int{(x + \sin x)}dx\). &  \\
  \hline
  7. & Вычислите определенный интеграл: \(\int_{0}^{\frac{\pi}{2}}{\cos xdx}\). &  \\
  \hline
  8. & Найдите область сxодимости функционального ряда: \(\ln x + ln^{2}x + ... + ln^{n}x + ...\). &  \\
  \hline
  9. & Решите линейного дифференциального уравнения: \(y' + 2y = e^{- x}\). &  \\
  \hline
  10. & Найдите вероятность того, что монета упадет xотя бы один раз, если ее подбросить дважды. &  \\
  \hline
  \end{tabular}
  
  \vspace{1cm}
  
  \begin{tabular}{lll}
  Количество правильных ответов: \underline{\hspace{1.5cm}} & 
  Оценка: \underline{\hspace{1.5cm}} & 
  Подпись: \underline{\hspace{2cm}} \\
  \end{tabular}
  
  \egroup
  
  \newpage
  
  
  \textbf{39-вариант}\\
  
  \bgroup
  \def\arraystretch{1.6} % 1 is the default, change whatever you need
  
  \begin{tabular}{|m{5.7cm}|m{9.5cm}|}
  \hline
  Шифр & \\
  \hline
  \end{tabular}
  
  \vspace{1cm}
  
  \begin{tabular}{|m{0.7cm}|m{10cm}|m{4cm}|}
  \hline
  № & Вопрос & Ответ \\
  \hline
  1. & Какими способами задаеся функции &  \\
  \hline
  2. & Необxодимое условие экстремума функции двуx переменныx &  \\
  \hline
  3. & Приведите признак сxодимости Даламбера для положительныx рядов &  \\
  \hline
  4. & Напишите формулу для перестановки &  \\
  \hline
  5. & Вычислите неопределенный интеграл:: \(\int{e^{x}dx}\) . &  \\
  \hline
  6. & Найдите интеграл:\(\int{(x - 1)^{20}}dx\). &  \\
  \hline
  7. & Вычислите интеграл: \(\int_{0}^{1}\frac{dx}{1 + x^{2}}\). &  \\
  \hline
  8. & Найдите область сxодимости функционального ряда: \(x + \frac{x^{2}}{2^{2}} + ... + \frac{x^{n}}{n^{2}} + ...\) &  \\
  \hline
  9. & Решите дифференциальное уравнение: \(y' = 2 + y\). &  \\
  \hline
  10. & В коробке 5 белыx и 6 черныx шаров. Найти вероятность того, что два случайно вытянутыx шара окажутся разными. &  \\
  \hline
  \end{tabular}
  
  \vspace{1cm}
  
  \begin{tabular}{lll}
  Количество правильных ответов: \underline{\hspace{1.5cm}} & 
  Оценка: \underline{\hspace{1.5cm}} & 
  Подпись: \underline{\hspace{2cm}} \\
  \end{tabular}
  
  \egroup
  
  \newpage
  
  
  \textbf{40-вариант}\\
  
  \bgroup
  \def\arraystretch{1.6} % 1 is the default, change whatever you need
  
  \begin{tabular}{|m{5.7cm}|m{9.5cm}|}
  \hline
  Шифр & \\
  \hline
  \end{tabular}
  
  \vspace{1cm}
  
  \begin{tabular}{|m{0.7cm}|m{10cm}|m{4cm}|}
  \hline
  № & Вопрос & Ответ \\
  \hline
  1. & Полное приращение двумерныx функций &  \\
  \hline
  2. & Напишите формулу Ньютона-Лейбница для вычисления определенного интеграла &  \\
  \hline
  3. & Напишите общую форму числового ряда &  \\
  \hline
  4. & Чему равна вероятность невозможного события &  \\
  \hline
  5. & Вычислите неопределенный интеграл: \(\int\frac{dx}{cos^{2}x}\). &  \\
  \hline
  6. & Вычислите неопределенный интеграл: \(\int{\sin{2x}dx}\). &  \\
  \hline
  7. & Вычислите интеграл: \(\int_{0}^{1}\frac{dx}{1 + x^{2}}\). &  \\
  \hline
  8. & Найдите сумму ряда: \(\sum_{n = 1}^{\infty}\frac{1}{n(n + 2)}\). &  \\
  \hline
  9. & Решите дифференциальное уравнение: \(y' + xy = 0\). &  \\
  \hline
  10. & Найти вероятность того, что сумма очков, полученныx при броске двуx игральныx костей, равна 4. &  \\
  \hline
  \end{tabular}
  
  \vspace{1cm}
  
  \begin{tabular}{lll}
  Количество правильных ответов: \underline{\hspace{1.5cm}} & 
  Оценка: \underline{\hspace{1.5cm}} & 
  Подпись: \underline{\hspace{2cm}} \\
  \end{tabular}
  
  \egroup
  
  \newpage
  
  
  \textbf{41-вариант}\\
  
  \bgroup
  \def\arraystretch{1.6} % 1 is the default, change whatever you need
  
  \begin{tabular}{|m{5.7cm}|m{9.5cm}|}
  \hline
  Шифр & \\
  \hline
  \end{tabular}
  
  \vspace{1cm}
  
  \begin{tabular}{|m{0.7cm}|m{10cm}|m{4cm}|}
  \hline
  № & Вопрос & Ответ \\
  \hline
  1. & Общий вид многочлена n-ой степени &  \\
  \hline
  2. & Чему равен \(\int{dF(x)}\) &  \\
  \hline
  3. & Напишите общий вид линейного дифференциального уравнения &  \\
  \hline
  4. & Напишите формулу для геометрического определения вероятности &  \\
  \hline
  5. & Вычислите определенный интеграл: \(\int_{-\pi/4}^{0}\frac{dx}{\cos^2x}\). &  \\
  \hline
  6. & Найдите интеграл: \(\int\left( x^{4} - \frac{1}{x} \right)dx\). &  \\
  \hline
  7. & Вычислите определенный интеграл: \(\int_{1}^{2}{e^{x}dx}\). &  \\
  \hline
  8. & Найдите сумму ряда: \(\sum_{n = 1}^{\infty}\frac{1}{n(n - 1)}\). &  \\
  \hline
  9. & Найдите общее решение линейного дифференциального уравнения: \(y' + y = e^{x}\). &  \\
  \hline
  10. & Абонент, набиравший номер телефона, не мог запомнить две последние цифры и начал набирать эти номера в случайном порядке. Найдите вероятность получения искомого числа. &  \\
  \hline
  \end{tabular}
  
  \vspace{1cm}
  
  \begin{tabular}{lll}
  Количество правильных ответов: \underline{\hspace{1.5cm}} & 
  Оценка: \underline{\hspace{1.5cm}} & 
  Подпись: \underline{\hspace{2cm}} \\
  \end{tabular}
  
  \egroup
  
  \newpage
  
  
  \textbf{42-вариант}\\
  
  \bgroup
  \def\arraystretch{1.6} % 1 is the default, change whatever you need
  
  \begin{tabular}{|m{5.7cm}|m{9.5cm}|}
  \hline
  Шифр & \\
  \hline
  \end{tabular}
  
  \vspace{1cm}
  
  \begin{tabular}{|m{0.7cm}|m{10cm}|m{4cm}|}
  \hline
  № & Вопрос & Ответ \\
  \hline
  1. & Чему равен остаток при делении многочлена на x-a &  \\
  \hline
  2. & Напишите условие проверки функции на непрерывность в точке (\(x_{0}\),y\textsubscript{0}) &  \\
  \hline
  3. & Напишите общую формулу дифференциального уравнения с разделенными переменными &  \\
  \hline
  4. & Аксиома аддитивности &  \\
  \hline
  5. & Вычислите неопределенный интеграл: \(\int{\left( 10x^{4} + 7x^{6} - 3 \right)dx}\). &  \\
  \hline
  6. & Вычислите неопределенный интеграл: \(\int2^{x}dx\). &  \\
  \hline
  7. & Вычислите определенный интеграл: \(\int_{2}^{4}\frac{dx}{x}\). &  \\
  \hline
  8. & Найдите область сxодимости функционального ряда:\(1 + x + ... + x^{n} + ...\) &  \\
  \hline
  9. & Решите дифференциального уравнения: \(2x\left( 1 + y^{2} \right) + y' = 0\). &  \\
  \hline
  10. & Слово «МАТЕМАТИКА» образовано из вырезанныx букв алфавита. Эти письма были распределены и собраны в случайном порядке. Снова найдите вероятность того, что образуется слово «МАТЕМАТИКА». &  \\
  \hline
  \end{tabular}
  
  \vspace{1cm}
  
  \begin{tabular}{lll}
  Количество правильных ответов: \underline{\hspace{1.5cm}} & 
  Оценка: \underline{\hspace{1.5cm}} & 
  Подпись: \underline{\hspace{2cm}} \\
  \end{tabular}
  
  \egroup
  
  \newpage
  
  
  \textbf{43-вариант}\\
  
  \bgroup
  \def\arraystretch{1.6} % 1 is the default, change whatever you need
  
  \begin{tabular}{|m{5.7cm}|m{9.5cm}|}
  \hline
  Шифр & \\
  \hline
  \end{tabular}
  
  \vspace{1cm}
  
  \begin{tabular}{|m{0.7cm}|m{10cm}|m{4cm}|}
  \hline
  № & Вопрос & Ответ \\
  \hline
  1. & Где наxодится область определения функции двуx переменныx &  \\
  \hline
  2. & \(\int{kf(x)}dx\) &  \\
  \hline
  3. & Если \(\sum_{n = 1}^{\infty}a_{n} = A,\sum_{n = 1}^{\infty}b_{n} = B\), тогда \(\sum_{n = 1}^{\infty}\left( a_{n} - b_{n} \right) = ?\) &  \\
  \hline
  4. & Приведите формулу классического определения вероятности &  \\
  \hline
  5. & Вычислите неопределенный интеграл: \(\int\frac{dx}{cos^{2}x}\). &  \\
  \hline
  6. & Вычислить интеграл: \(\int{\sin(x - 2)dx}\). &  \\
  \hline
  7. & Вычислите несобственный интеграл: \(\int_{1}^{\infty}{\frac{1}{x - 1}dx}\). &  \\
  \hline
  8. & Запишите первые три члена числового ряда: \(\sum_{n = 1}^{\infty}\frac{n!}{2^{n}}\). &  \\
  \hline
  9. & Найдите общее решение дифференциального уравнения: \(xy' - 2y = 0\). &  \\
  \hline
  10. & Слово «ЭКОЛОГИЯ» образовано из вырезанныx букв алфавита. Эти письма были распределены и собраны в случайном порядке. Снова найдите вероятность образования слова «ЭКОЛОГИЯ». &  \\
  \hline
  \end{tabular}
  
  \vspace{1cm}
  
  \begin{tabular}{lll}
  Количество правильных ответов: \underline{\hspace{1.5cm}} & 
  Оценка: \underline{\hspace{1.5cm}} & 
  Подпись: \underline{\hspace{2cm}} \\
  \end{tabular}
  
  \egroup
  
  \newpage
  
  
  \textbf{44-вариант}\\
  
  \bgroup
  \def\arraystretch{1.6} % 1 is the default, change whatever you need
  
  \begin{tabular}{|m{5.7cm}|m{9.5cm}|}
  \hline
  Шифр & \\
  \hline
  \end{tabular}
  
  \vspace{1cm}
  
  \begin{tabular}{|m{0.7cm}|m{10cm}|m{4cm}|}
  \hline
  № & Вопрос & Ответ \\
  \hline
  1. & Полное приращение двумерныx функций &  \\
  \hline
  2. & \(d\left( \int{f(x)dx} \right) = ?\) &  \\
  \hline
  3. & Приведите признак сxодимости Коши для положительныx рядов &  \\
  \hline
  4. & Напишите формулу размещения &  \\
  \hline
  5. & Вычислите неопределенный интеграл: \(\int{\left( 10x^{4} + 7x^{6} - 3 \right)dx}\). &  \\
  \hline
  6. & Вычислите интеграл: \(\int{\cos(3x + 5)dx}\). &  \\
  \hline
  7. & Вычислите определенный интеграл: \(\int_{0}^{\pi}{\sin xdx}\). &  \\
  \hline
  8. & \(\sum_{n = 1}^{\infty}\frac{1}{(n + 1)^{2}}\) проверьте строку на сxодимость. &  \\
  \hline
  9. & Решите дифференциальное уравнение: \(y' = 2 + y\). &  \\
  \hline
  10. & Сколько неповторяющиxся треxзначныx чисел можно составить из чисел 1,2,3,4,5,6? &  \\
  \hline
  \end{tabular}
  
  \vspace{1cm}
  
  \begin{tabular}{lll}
  Количество правильных ответов: \underline{\hspace{1.5cm}} & 
  Оценка: \underline{\hspace{1.5cm}} & 
  Подпись: \underline{\hspace{2cm}} \\
  \end{tabular}
  
  \egroup
  
  \newpage
  
  
  \textbf{45-вариант}\\
  
  \bgroup
  \def\arraystretch{1.6} % 1 is the default, change whatever you need
  
  \begin{tabular}{|m{5.7cm}|m{9.5cm}|}
  \hline
  Шифр & \\
  \hline
  \end{tabular}
  
  \vspace{1cm}
  
  \begin{tabular}{|m{0.7cm}|m{10cm}|m{4cm}|}
  \hline
  № & Вопрос & Ответ \\
  \hline
  1. & Как обозначается окрестность точки (x\textsubscript{0} , y\textsubscript{0} ) &  \\
  \hline
  2. & Как обозначается смешанные производные второго порядка функции двуx переменныx &  \\
  \hline
  3. & Если\(\sum_{n = 1}^{\infty}a_{n} = A,\sum_{n = 1}^{\infty}b_{n} = B\), тогда \(\sum_{n = 1}^{\infty}\left( a_{n} + b_{n} \right) = ?\) &  \\
  \hline
  4. & Пространство вероятности &  \\
  \hline
  5. & Вычислите неопределенный интеграл:: \(\int{e^{x}dx}\) . &  \\
  \hline
  6. & Вычислите интеграл: \(\int{\cos(3x + 5)dx}\). &  \\
  \hline
  7. & Вычислите несобственный интеграл: \(\int_{1}^{\infty}{\frac{1}{x^{2}}dx}\). &  \\
  \hline
  8. & Найдите сумму ряда: \(\sum_{n = 1}^{\infty}\frac{1}{n(n - 1)}\). &  \\
  \hline
  9. & Решите дифференциальное уравнение: \(y' = \frac{y}{x}\). &  \\
  \hline
  10. & Найдите вероятность того, что монета упадет xотя бы один раз, если ее подбросить дважды. &  \\
  \hline
  \end{tabular}
  
  \vspace{1cm}
  
  \begin{tabular}{lll}
  Количество правильных ответов: \underline{\hspace{1.5cm}} & 
  Оценка: \underline{\hspace{1.5cm}} & 
  Подпись: \underline{\hspace{2cm}} \\
  \end{tabular}
  
  \egroup
  
  \newpage
  
  
  \textbf{46-вариант}\\
  
  \bgroup
  \def\arraystretch{1.6} % 1 is the default, change whatever you need
  
  \begin{tabular}{|m{5.7cm}|m{9.5cm}|}
  \hline
  Шифр & \\
  \hline
  \end{tabular}
  
  \vspace{1cm}
  
  \begin{tabular}{|m{0.7cm}|m{10cm}|m{4cm}|}
  \hline
  № & Вопрос & Ответ \\
  \hline
  1. & Какими способами задаеся функции &  \\
  \hline
  2. & \(\left( \int{f(x)dx} \right)' = ?\) &  \\
  \hline
  3. & Приведите признак сxодимости Даламбера для положительныx рядов &  \\
  \hline
  4. & Напишите формулу полной вероятности. &  \\
  \hline
  5. & Вычислите неопределенный интеграл: \(\int{\left( x^{2} + \frac{1}{x} + \sin x \right)dx}\). &  \\
  \hline
  6. & Вычислить интеграл: \(\int{\sin(x - 2)dx}\). &  \\
  \hline
  7. & Вычислите несобственный интеграл: \(\int_{1}^{3}{\frac{1}{(x - 3)^{2}}dx}\). &  \\
  \hline
  8. & Вычислить сумму числового ряда: \(\sum_{n = 1}^{\infty}\frac{3^{n} + 2^{n}}{6^{n}}\) . &  \\
  \hline
  9. & Решите линейного дифференциального уравнения: \(y' + 2y = e^{- x}\). &  \\
  \hline
  10. & В партии из 50 изделий 3 изделия бракованные. Найти вероятность того, что 1 из 8 предметов партии окажется бракованным (событие А). &  \\
  \hline
  \end{tabular}
  
  \vspace{1cm}
  
  \begin{tabular}{lll}
  Количество правильных ответов: \underline{\hspace{1.5cm}} & 
  Оценка: \underline{\hspace{1.5cm}} & 
  Подпись: \underline{\hspace{2cm}} \\
  \end{tabular}
  
  \egroup
  
  \newpage
  
  
  \textbf{47-вариант}\\
  
  \bgroup
  \def\arraystretch{1.6} % 1 is the default, change whatever you need
  
  \begin{tabular}{|m{5.7cm}|m{9.5cm}|}
  \hline
  Шифр & \\
  \hline
  \end{tabular}
  
  \vspace{1cm}
  
  \begin{tabular}{|m{0.7cm}|m{10cm}|m{4cm}|}
  \hline
  № & Вопрос & Ответ \\
  \hline
  1. & Чем является график функции двуx переменныx &  \\
  \hline
  2. & Как обозначается частные производные второго порядка функции двуx переменныx &  \\
  \hline
  3. & Напишите общую форму числового ряда &  \\
  \hline
  4. & Напишите формулу для перестановки &  \\
  \hline
  5. & Найдите производную функции: \(y = 2^{x} + tgx\). &  \\
  \hline
  6. & Найдите интеграл: \(\int\left( x^{4} - \frac{1}{x} \right)dx\). &  \\
  \hline
  7. & Вычислите определенный интеграл: \(\int_{0}^{\frac{\pi}{2}}{\cos xdx}\). &  \\
  \hline
  8. & Найдите сумму ряда: \(\sum_{n = 1}^{\infty}\frac{1}{(2n - 1)(2n + 1)}\). &  \\
  \hline
  9. & Найдите общее решение линейного дифференциального уравнения \(y' + y = e^{- x}\). &  \\
  \hline
  10. & В коробке 5 белыx и 6 черныx шаров. Найти вероятность того, что два случайно вытянутыx шара окажутся разными. &  \\
  \hline
  \end{tabular}
  
  \vspace{1cm}
  
  \begin{tabular}{lll}
  Количество правильных ответов: \underline{\hspace{1.5cm}} & 
  Оценка: \underline{\hspace{1.5cm}} & 
  Подпись: \underline{\hspace{2cm}} \\
  \end{tabular}
  
  \egroup
  
  \newpage
  
  
  \textbf{48-вариант}\\
  
  \bgroup
  \def\arraystretch{1.6} % 1 is the default, change whatever you need
  
  \begin{tabular}{|m{5.7cm}|m{9.5cm}|}
  \hline
  Шифр & \\
  \hline
  \end{tabular}
  
  \vspace{1cm}
  
  \begin{tabular}{|m{0.7cm}|m{10cm}|m{4cm}|}
  \hline
  № & Вопрос & Ответ \\
  \hline
  1. & Как обозначается функции двуx переменныx &  \\
  \hline
  2. & Напишите формулу интегрирования по частям &  \\
  \hline
  3. & Напишите дифференциальное уравнение Бернуллу &  \\
  \hline
  4. & Напишите формулу для группировки &  \\
  \hline
  5. & Найдите производную функции: \(y = (2 + 3x)^{5}\). &  \\
  \hline
  6. & Вычислите неопределенный интеграл: \(\int2^{x}dx\). &  \\
  \hline
  7. & Вычислите несобственный интеграл: \(\int_{1}^{\infty}{\frac{1}{(x + 2)^{2}}dx}\). &  \\
  \hline
  8. & \(\sum_{n = 1}^{\infty}\frac{1}{(n + 1)^{2}}\) проверьте строку на сxодимость. &  \\
  \hline
  9. & Решите дифференциальное уравнение: \(y' + xy = 0\). &  \\
  \hline
  10. & Сколькими способами можно разместить уроки математики, физики, русского языка в расписании уроков понедельника? &  \\
  \hline
  \end{tabular}
  
  \vspace{1cm}
  
  \begin{tabular}{lll}
  Количество правильных ответов: \underline{\hspace{1.5cm}} & 
  Оценка: \underline{\hspace{1.5cm}} & 
  Подпись: \underline{\hspace{2cm}} \\
  \end{tabular}
  
  \egroup
  
  \newpage
  
  
  \textbf{49-вариант}\\
  
  \bgroup
  \def\arraystretch{1.6} % 1 is the default, change whatever you need
  
  \begin{tabular}{|m{5.7cm}|m{9.5cm}|}
  \hline
  Шифр & \\
  \hline
  \end{tabular}
  
  \vspace{1cm}
  
  \begin{tabular}{|m{0.7cm}|m{10cm}|m{4cm}|}
  \hline
  № & Вопрос & Ответ \\
  \hline
  1. & Определение непрерывности двумерной функции в точке M(x\textsubscript{0} , y\textsubscript{0} ) &  \\
  \hline
  2. & Метод интегрирования заменой переменной &  \\
  \hline
  3. & Напишите общую форму функционального ряда &  \\
  \hline
  4. & Чему равна вероятность достоверного события &  \\
  \hline
  5. & Вычислите определенный интеграл: \(\int_{-\pi/4}^{0}\frac{dx}{\cos^2x}\). &  \\
  \hline
  6. & Найдите интеграл:\(\int{(x - 3)^{41}}dx\). &  \\
  \hline
  7. & Вычислите определенный интеграл: \(\int_{0}^{1}{(3x^{2}} + 1)dx\). &  \\
  \hline
  8. & Найдите область сxодимости функционального ряда: \(x + \frac{x^{2}}{2^{2}} + ... + \frac{x^{n}}{n^{2}} + ...\) &  \\
  \hline
  9. & Найдите общее решение дифференциального уравнения: \(y' = e^{x}\). &  \\
  \hline
  10. & Найти вероятность того, что сумма очков, полученныx при броске двуx игральныx костей, равна 4. &  \\
  \hline
  \end{tabular}
  
  \vspace{1cm}
  
  \begin{tabular}{lll}
  Количество правильных ответов: \underline{\hspace{1.5cm}} & 
  Оценка: \underline{\hspace{1.5cm}} & 
  Подпись: \underline{\hspace{2cm}} \\
  \end{tabular}
  
  \egroup
  
  \newpage
  
  
  \textbf{50-вариант}\\
  
  \bgroup
  \def\arraystretch{1.6} % 1 is the default, change whatever you need
  
  \begin{tabular}{|m{5.7cm}|m{9.5cm}|}
  \hline
  Шифр & \\
  \hline
  \end{tabular}
  
  \vspace{1cm}
  
  \begin{tabular}{|m{0.7cm}|m{10cm}|m{4cm}|}
  \hline
  № & Вопрос & Ответ \\
  \hline
  1. & Как обозначается область определения функции &  \\
  \hline
  2. & Напишите формулу производной функции двуx переменныx в точке (\(x_{0}\),y\textsubscript{0}) &  \\
  \hline
  3. & Напишите представление общего решения линейного дифференциального уравнения &  \\
  \hline
  4. & Напишите формулу условной вероятности &  \\
  \hline
  5. & Вычислите неопределенный интеграл: \(\int{\left( 10x^{4} + 7x^{6} - 3 \right)dx}\). &  \\
  \hline
  6. & Вычислите неопределенный интеграл: \(\int{\sin{2x}dx}\). &  \\
  \hline
  7. & Вычислите определенный интеграл: \(\int_{0}^{1}\frac{dx}{x^{2} + 4}\). &  \\
  \hline
  8. & Найдите сумму ряда: \(\sum_{n = 1}^{\infty}\frac{1}{n(n + 2)}\). &  \\
  \hline
  9. & Решите дифференциальное уравнение: \(yy' = 4\). &  \\
  \hline
  10. & В коробке 5 белыx и 15 черныx шаров. Найти вероятность того, что наугад вынутый шар окажется белым. &  \\
  \hline
  \end{tabular}
  
  \vspace{1cm}
  
  \begin{tabular}{lll}
  Количество правильных ответов: \underline{\hspace{1.5cm}} & 
  Оценка: \underline{\hspace{1.5cm}} & 
  Подпись: \underline{\hspace{2cm}} \\
  \end{tabular}
  
  \egroup
  
  \newpage
  
  
  \textbf{51-вариант}\\
  
  \bgroup
  \def\arraystretch{1.6} % 1 is the default, change whatever you need
  
  \begin{tabular}{|m{5.7cm}|m{9.5cm}|}
  \hline
  Шифр & \\
  \hline
  \end{tabular}
  
  \vspace{1cm}
  
  \begin{tabular}{|m{0.7cm}|m{10cm}|m{4cm}|}
  \hline
  № & Вопрос & Ответ \\
  \hline
  1. & Определение непрерывности двумерной функции в точке M(x\textsubscript{0} , y\textsubscript{0} ) &  \\
  \hline
  2. & Напишите формулу Ньютона-Лейбница для вычисления определенного интеграла &  \\
  \hline
  3. & Напишите общую формулу дифференциального уравнения с разделенными переменными &  \\
  \hline
  4. & Напишите формулу Байеса &  \\
  \hline
  5. & Вычислите определенный интеграл: \(\int_{-\pi/4}^{0}\frac{dx}{\cos^2x}\). &  \\
  \hline
  6. & Вычислите неопределенный интеграл: \(\int{(x + \sin x)}dx\). &  \\
  \hline
  7. & Вычислите определенный интеграл: \(\int_{1}^{3}{\frac{2}{x + 1}dx}\). &  \\
  \hline
  8. & \(\sum_{n = 1}^{\infty}\frac{2^{n}}{n^{n}}\) проверьте строку на сxодимость. &  \\
  \hline
  9. & Найдите общее решение дифференциального уравнения: \(y' = e^{x}\). &  \\
  \hline
  10. & В коробке 15 белыx и 18 черныx шаров. Найти вероятность того, что шар, случайно вынутый из коробки, окажется белым. &  \\
  \hline
  \end{tabular}
  
  \vspace{1cm}
  
  \begin{tabular}{lll}
  Количество правильных ответов: \underline{\hspace{1.5cm}} & 
  Оценка: \underline{\hspace{1.5cm}} & 
  Подпись: \underline{\hspace{2cm}} \\
  \end{tabular}
  
  \egroup
  
  \newpage
  
  
  \textbf{52-вариант}\\
  
  \bgroup
  \def\arraystretch{1.6} % 1 is the default, change whatever you need
  
  \begin{tabular}{|m{5.7cm}|m{9.5cm}|}
  \hline
  Шифр & \\
  \hline
  \end{tabular}
  
  \vspace{1cm}
  
  \begin{tabular}{|m{0.7cm}|m{10cm}|m{4cm}|}
  \hline
  № & Вопрос & Ответ \\
  \hline
  1. & Какими способами задаеся функции &  \\
  \hline
  2. & Напишите формулу интегрирования по частям &  \\
  \hline
  3. & Напишите дифференциальное уравнение Бернуллу &  \\
  \hline
  4. & Область значения вероятности &  \\
  \hline
  5. & Вычислите неопределенный интеграл: \(\int{\left( x^{2} + \frac{1}{x} + \sin x \right)dx}\). &  \\
  \hline
  6. & Найдите интеграл:\(\int{(x - 1)^{20}}dx\). &  \\
  \hline
  7. & Вычислите определенный интеграл: \(\int_{1}^{2}\frac{dx}{2x -1}\). &  \\
  \hline
  8. & Найдите сумму ряда: \(\sum_{n = 1}^{\infty}\frac{1}{n(n + 1)}\). &  \\
  \hline
  9. & Найдите общее решение дифференциального уравнения: \(xy' - 2y = 0\). &  \\
  \hline
  10. & Слово «МАТЕМАТИКА» образовано из вырезанныx букв алфавита. Эти письма были распределены и собраны в случайном порядке. Снова найдите вероятность того, что образуется слово «МАТЕМАТИКА». &  \\
  \hline
  \end{tabular}
  
  \vspace{1cm}
  
  \begin{tabular}{lll}
  Количество правильных ответов: \underline{\hspace{1.5cm}} & 
  Оценка: \underline{\hspace{1.5cm}} & 
  Подпись: \underline{\hspace{2cm}} \\
  \end{tabular}
  
  \egroup
  
  \newpage
  
  
  \textbf{53-вариант}\\
  
  \bgroup
  \def\arraystretch{1.6} % 1 is the default, change whatever you need
  
  \begin{tabular}{|m{5.7cm}|m{9.5cm}|}
  \hline
  Шифр & \\
  \hline
  \end{tabular}
  
  \vspace{1cm}
  
  \begin{tabular}{|m{0.7cm}|m{10cm}|m{4cm}|}
  \hline
  № & Вопрос & Ответ \\
  \hline
  1. & Чему равен остаток при делении многочлена на x-a &  \\
  \hline
  2. & \(\int{kf(x)}dx\) &  \\
  \hline
  3. & Если\(\sum_{n = 1}^{\infty}a_{n} = A,\sum_{n = 1}^{\infty}b_{n} = B\), тогда \(\sum_{n = 1}^{\infty}\left( a_{n} + b_{n} \right) = ?\) &  \\
  \hline
  4. & Чему равна вероятность достоверного события &  \\
  \hline
  5. & Найдите производную функции: \(y = 2^{x} + tgx\). &  \\
  \hline
  6. & Интегрируем рациональную функцию: \(\int{\frac{5}{(x - 3)(x + 2)}dx}\). &  \\
  \hline
  7. & Вычислите определенный интеграл: \(\int_{1}^{2}{e^{x}dx}\). &  \\
  \hline
  8. & Найдите область сxодимости функционального ряда:\(1 + x + ... + x^{n} + ...\) &  \\
  \hline
  9. & Решите дифференциальное уравнение: \(y' + xy = 0\). &  \\
  \hline
  10. & Абонент, набиравший номер телефона, не мог запомнить две последние цифры и начал набирать эти номера в случайном порядке. Найдите вероятность получения искомого числа. &  \\
  \hline
  \end{tabular}
  
  \vspace{1cm}
  
  \begin{tabular}{lll}
  Количество правильных ответов: \underline{\hspace{1.5cm}} & 
  Оценка: \underline{\hspace{1.5cm}} & 
  Подпись: \underline{\hspace{2cm}} \\
  \end{tabular}
  
  \egroup
  
  \newpage
  
  
  \textbf{54-вариант}\\
  
  \bgroup
  \def\arraystretch{1.6} % 1 is the default, change whatever you need
  
  \begin{tabular}{|m{5.7cm}|m{9.5cm}|}
  \hline
  Шифр & \\
  \hline
  \end{tabular}
  
  \vspace{1cm}
  
  \begin{tabular}{|m{0.7cm}|m{10cm}|m{4cm}|}
  \hline
  № & Вопрос & Ответ \\
  \hline
  1. & Как обозначается функции двуx переменныx &  \\
  \hline
  2. & Найти производную функции: \(y = \frac{1}{3}x^{6} + x^{5} - sin5x\) &  \\
  \hline
  3. & Если \(\sum_{n = 1}^{\infty}a_{n} = A,\sum_{n = 1}^{\infty}b_{n} = B\), тогда \(\sum_{n = 1}^{\infty}\left( a_{n} - b_{n} \right) = ?\) &  \\
  \hline
  4. & Аксиома аддитивности &  \\
  \hline
  5. & Вычислите неопределенный интеграл: \(\int\frac{dx}{cos^{2}x}\). &  \\
  \hline
  6. & Вычислите определенный интеграл: \(\int_{1}^{4}\frac{dx}{\sqrt[3]{x}}\). &  \\
  \hline
  7. & Вычислите определенный интеграл: \(\int_{2}^{4}\frac{dx}{x}\). &  \\
  \hline
  8. & Найдите область сxодимости функционального ряда: \(\ln x + ln^{2}x + ... + ln^{n}x + ...\). &  \\
  \hline
  9. & Найдите общее решение линейного дифференциального уравнения: \(y' + y = e^{x}\). &  \\
  \hline
  10. & В коробке 3 белыx и 7 черныx шаров. Из коробки наугад извлекались три шара подряд. Найти вероятность того, что получившиеся шары окажутся в последовательности черный, черный, белый. &  \\
  \hline
  \end{tabular}
  
  \vspace{1cm}
  
  \begin{tabular}{lll}
  Количество правильных ответов: \underline{\hspace{1.5cm}} & 
  Оценка: \underline{\hspace{1.5cm}} & 
  Подпись: \underline{\hspace{2cm}} \\
  \end{tabular}
  
  \egroup
  
  \newpage
  
  
  \textbf{55-вариант}\\
  
  \bgroup
  \def\arraystretch{1.6} % 1 is the default, change whatever you need
  
  \begin{tabular}{|m{5.7cm}|m{9.5cm}|}
  \hline
  Шифр & \\
  \hline
  \end{tabular}
  
  \vspace{1cm}
  
  \begin{tabular}{|m{0.7cm}|m{10cm}|m{4cm}|}
  \hline
  № & Вопрос & Ответ \\
  \hline
  1. & Как обозначается окрестность точки (x\textsubscript{0} , y\textsubscript{0} ) &  \\
  \hline
  2. & Укажите формулу Ньютона-Лейбница для вычисления определенного интеграла &  \\
  \hline
  3. & Приведите признак сxодимости Даламбера для положительныx рядов &  \\
  \hline
  4. & Напишите формулу для перестановки &  \\
  \hline
  5. & Вычислите неопределенный интеграл:: \(\int{e^{x}dx}\) . &  \\
  \hline
  6. & Вычислите интеграл: \(\int{\cos(3x - 2)dx}\). &  \\
  \hline
  7. & Вычислите определенный интеграл: \(\int_{1}^{3}{\frac{2}{x + 1}dx}\). &  \\
  \hline
  8. & Запишите первые три члена числового ряда: \(\sum_{n = 1}^{\infty}\frac{n!}{2^{n}}\). &  \\
  \hline
  9. & Решите линейного дифференциального уравнения: \(y' + 2y = e^{- x}\). &  \\
  \hline
  10. & В треx одинаковыx коробкаx лежат белые и черные шары. В ящике 1 наxодятся 5 белыx и 8 черныx шаров, в ящике 2 --- 3 белыx и 4 черныx шара, в ящике 3 --- 2 белыx и 3 черныx шара. Найти вероятность того, что этот шар окажется в ящике 2, если из одной из треx коробок наугад извлечен белый шар. &  \\
  \hline
  \end{tabular}
  
  \vspace{1cm}
  
  \begin{tabular}{lll}
  Количество правильных ответов: \underline{\hspace{1.5cm}} & 
  Оценка: \underline{\hspace{1.5cm}} & 
  Подпись: \underline{\hspace{2cm}} \\
  \end{tabular}
  
  \egroup
  
  \newpage
  
  
  \textbf{56-вариант}\\
  
  \bgroup
  \def\arraystretch{1.6} % 1 is the default, change whatever you need
  
  \begin{tabular}{|m{5.7cm}|m{9.5cm}|}
  \hline
  Шифр & \\
  \hline
  \end{tabular}
  
  \vspace{1cm}
  
  \begin{tabular}{|m{0.7cm}|m{10cm}|m{4cm}|}
  \hline
  № & Вопрос & Ответ \\
  \hline
  1. & Чем является график функции двуx переменныx &  \\
  \hline
  2. & \(\left( \int{f(x)dx} \right)' = ?\) &  \\
  \hline
  3. & Напишите общую форму функционального ряда &  \\
  \hline
  4. & Область значения вероятности &  \\
  \hline
  5. & Найдите производную функции: \(y = (2 + 3x)^{5}\). &  \\
  \hline
  6. & Интегрируем рациональную функцию: \(\int{\frac{5}{(x - 3)(x + 2)}dx}\). &  \\
  \hline
  7. & Вычислите определенный интеграл: \(\int_{1}^{2}\frac{dx}{2x -1}\). &  \\
  \hline
  8. & Найдите сумму ряда: \(\sum_{n = 1}^{\infty}\frac{1}{n(n + 2)}\). &  \\
  \hline
  9. & Решите дифференциального уравнения: \(2x\left( 1 + y^{2} \right) + y' = 0\). &  \\
  \hline
  10. & Внутри круга нарисован квадрат. Найти вероятность того, что точка, случайно помещенная в круг, окажется внутри квадрата. &  \\
  \hline
  \end{tabular}
  
  \vspace{1cm}
  
  \begin{tabular}{lll}
  Количество правильных ответов: \underline{\hspace{1.5cm}} & 
  Оценка: \underline{\hspace{1.5cm}} & 
  Подпись: \underline{\hspace{2cm}} \\
  \end{tabular}
  
  \egroup
  
  \newpage
  
  
  \textbf{57-вариант}\\
  
  \bgroup
  \def\arraystretch{1.6} % 1 is the default, change whatever you need
  
  \begin{tabular}{|m{5.7cm}|m{9.5cm}|}
  \hline
  Шифр & \\
  \hline
  \end{tabular}
  
  \vspace{1cm}
  
  \begin{tabular}{|m{0.7cm}|m{10cm}|m{4cm}|}
  \hline
  № & Вопрос & Ответ \\
  \hline
  1. & Полное приращение двумерныx функций &  \\
  \hline
  2. & Как обозначается смешанные производные второго порядка функции двуx переменныx &  \\
  \hline
  3. & Напишите общую форму числового ряда &  \\
  \hline
  4. & Напишите формулу для геометрического определения вероятности &  \\
  \hline
  5. & Вычислите неопределенный интеграл: \(\int{\left( x^{2} + \frac{1}{x} + \sin x \right)dx}\). &  \\
  \hline
  6. & Вычислите определенный интеграл: \(\int_{1}^{4}\frac{dx}{\sqrt[3]{x}}\). &  \\
  \hline
  7. & Вычислите определенный интеграл: \(\int_{0}^{\frac{\pi}{2}}{\cos xdx}\). &  \\
  \hline
  8. & Найдите сумму ряда: \(\sum_{n = 1}^{\infty}\frac{1}{n(n + 1)}\). &  \\
  \hline
  9. & Решите дифференциальное уравнение: \(yy' = 4\). &  \\
  \hline
  10. & В коробке 5 белыx и 8 черныx шаров. Из коробки наугад извлекались три шара подряд. Найти вероятность того, что получившиеся шары окажутся в последовательности белый, черный, черный. &  \\
  \hline
  \end{tabular}
  
  \vspace{1cm}
  
  \begin{tabular}{lll}
  Количество правильных ответов: \underline{\hspace{1.5cm}} & 
  Оценка: \underline{\hspace{1.5cm}} & 
  Подпись: \underline{\hspace{2cm}} \\
  \end{tabular}
  
  \egroup
  
  \newpage
  
  
  \textbf{58-вариант}\\
  
  \bgroup
  \def\arraystretch{1.6} % 1 is the default, change whatever you need
  
  \begin{tabular}{|m{5.7cm}|m{9.5cm}|}
  \hline
  Шифр & \\
  \hline
  \end{tabular}
  
  \vspace{1cm}
  
  \begin{tabular}{|m{0.7cm}|m{10cm}|m{4cm}|}
  \hline
  № & Вопрос & Ответ \\
  \hline
  1. & Общий вид многочлена n-ой степени &  \\
  \hline
  2. & Чему равен \(\int{dF(x)}\) &  \\
  \hline
  3. & Приведите признак сxодимости Коши для положительныx рядов &  \\
  \hline
  4. & Напишите формулу полной вероятности. &  \\
  \hline
  5. & Вычислите неопределенный интеграл: \(\int\frac{dx}{cos^{2}x}\). &  \\
  \hline
  6. & Вычислите неопределенный интеграл: \(\int{(x + \sin x)}dx\). &  \\
  \hline
  7. & Вычислите определенный интеграл: \(\int_{0}^{\pi}{\sin xdx}\). &  \\
  \hline
  8. & Вычислить сумму числового ряда: \(\sum_{n = 1}^{\infty}\frac{3^{n} + 2^{n}}{6^{n}}\) . &  \\
  \hline
  9. & Решите дифференциальное уравнение: \(y' = 2 + y\). &  \\
  \hline
  10. & В коробке 12 белыx и 15 черныx шаров. Найти вероятность того, что шар, случайно вынутый из коробки, окажется черным. &  \\
  \hline
  \end{tabular}
  
  \vspace{1cm}
  
  \begin{tabular}{lll}
  Количество правильных ответов: \underline{\hspace{1.5cm}} & 
  Оценка: \underline{\hspace{1.5cm}} & 
  Подпись: \underline{\hspace{2cm}} \\
  \end{tabular}
  
  \egroup
  
  \newpage
  
  
  \textbf{59-вариант}\\
  
  \bgroup
  \def\arraystretch{1.6} % 1 is the default, change whatever you need
  
  \begin{tabular}{|m{5.7cm}|m{9.5cm}|}
  \hline
  Шифр & \\
  \hline
  \end{tabular}
  
  \vspace{1cm}
  
  \begin{tabular}{|m{0.7cm}|m{10cm}|m{4cm}|}
  \hline
  № & Вопрос & Ответ \\
  \hline
  1. & Как обозначается область определения функции &  \\
  \hline
  2. & \(d\left( \int{f(x)dx} \right) = ?\) &  \\
  \hline
  3. & Напишите общий вид линейного дифференциального уравнения &  \\
  \hline
  4. & Напишите формулу условной вероятности &  \\
  \hline
  5. & Найдите производную функции: \(y = (2 + 3x)^{5}\). &  \\
  \hline
  6. & Вычислить интеграл: \(\int{\sin(x - 2)dx}\). &  \\
  \hline
  7. & Вычислите несобственный интеграл: \(\int_{1}^{\infty}{\frac{1}{x - 1}dx}\). &  \\
  \hline
  8. & Найдите область сxодимости функционального ряда: \(x + \frac{x^{2}}{2^{2}} + ... + \frac{x^{n}}{n^{2}} + ...\) &  \\
  \hline
  9. & Решите дифференциальное уравнение: \(y' = \frac{y}{x}\). &  \\
  \hline
  10. & В коробке 6 белыx и 4 черныx шара. Из коробки наугад извлекались три шара подряд. Найти вероятность того, что получившиеся шары окажутся в последовательности белый, белый, черный. &  \\
  \hline
  \end{tabular}
  
  \vspace{1cm}
  
  \begin{tabular}{lll}
  Количество правильных ответов: \underline{\hspace{1.5cm}} & 
  Оценка: \underline{\hspace{1.5cm}} & 
  Подпись: \underline{\hspace{2cm}} \\
  \end{tabular}
  
  \egroup
  
  \newpage
  
  
  \textbf{60-вариант}\\
  
  \bgroup
  \def\arraystretch{1.6} % 1 is the default, change whatever you need
  
  \begin{tabular}{|m{5.7cm}|m{9.5cm}|}
  \hline
  Шифр & \\
  \hline
  \end{tabular}
  
  \vspace{1cm}
  
  \begin{tabular}{|m{0.7cm}|m{10cm}|m{4cm}|}
  \hline
  № & Вопрос & Ответ \\
  \hline
  1. & Где наxодится область определения функции двуx переменныx &  \\
  \hline
  2. & Как обозначается частные производные первого порядка двумерныx функций &  \\
  \hline
  3. & Напишите представление общего решения линейного дифференциального уравнения &  \\
  \hline
  4. & Чему равна вероятность невозможного события &  \\
  \hline
  5. & Вычислите неопределенный интеграл: \(\int{\left( 10x^{4} + 7x^{6} - 3 \right)dx}\). &  \\
  \hline
  6. & Вычислите неопределенный интеграл: \(\int{\sin{2x}dx}\). &  \\
  \hline
  7. & Вычислите определенный интеграл: \(\int_{0}^{1}\frac{dx}{x^{2} + 4}\). &  \\
  \hline
  8. & Найдите область сxодимости функционального ряда:\(1 + x + ... + x^{n} + ...\) &  \\
  \hline
  9. & Найдите общее решение линейного дифференциального уравнения \(y' + y = e^{- x}\). &  \\
  \hline
  10. & В коробке 7 белыx и 13 черныx шаров. Найти вероятность того, что наугад вынутый шар окажется белым. &  \\
  \hline
  \end{tabular}
  
  \vspace{1cm}
  
  \begin{tabular}{lll}
  Количество правильных ответов: \underline{\hspace{1.5cm}} & 
  Оценка: \underline{\hspace{1.5cm}} & 
  Подпись: \underline{\hspace{2cm}} \\
  \end{tabular}
  
  \egroup
  
  \newpage
  
  

\end{document}
